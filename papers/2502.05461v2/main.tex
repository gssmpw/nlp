\documentclass[sigconf,natbib=true]{acmart}
\let\Bbbk\relax
% \pdfobjcompresslevel=2

\usepackage{enumitem}
\usepackage{multicol}
\usepackage{amssymb}
\usepackage{amsmath}
\usepackage{graphicx}
\usepackage{xcolor}
\usepackage{amsfonts} 
\usepackage{multirow}
\usepackage{natbib}
\usepackage{url}
% \url{https://example.com/very-long-url-that-does-not-wrap}
\usepackage{balance}
% save space
% \usepackage{microtype}
 \setlength\floatsep{0.2\baselineskip plus 3pt minus 2pt} % distance between two floats
 \setlength\textfloatsep{0.2\baselineskip plus 3pt minus 2pt} % distance between floats on the top or the bottom and the text
 %\setlength{\textfloatsep}{2pt}
 \setlength\intextsep{0.2\baselineskip plus 3pt minus 2pt} % distance between floats inserted inside the text (using h) and the text
 \setlength\dbltextfloatsep{0.2\baselineskip plus 3pt minus 2pt} % distance between a float spanning both columns and the text
 \setlength\dblfloatsep{0.2\baselineskip plus 3pt minus 2pt} % distance between two floats spanning both columns.

\definecolor{mygray}{gray}{0.9} 
\definecolor{lightgray}{gray}{0.95} 
\newcommand{\ZKI}[1]{\textcolor{blue}{[dzq: #1]}}
\newcommand{\revision}[1]{\textcolor{black}{#1}}
%\newcommand{\revision}[1]{{{#1}}}



\long\def\appendixsiam{\par\setcounter{section}{0}\setcounter{subsection}{0}\setcounter{equation}{0}}
\AtBeginDocument{%
  \providecommand\BibTeX{{%
    Bib\TeX}}}

\copyrightyear{2025}
\acmYear{2025}
\setcopyright{acmlicensed}
\acmConference[WWW '25]{Proceedings of the ACM Web Conference 2025}{April 28-May 2, 2025}{Sydney, NSW, Australia}
\acmBooktitle{Proceedings of the ACM Web Conference 2025, April 28-May 2, 2025, Sydney, NSW, Australia}
\acmDOI{10.1145/3696410.3714726}
\acmISBN{979-8-4007-1274-6/25/04}

%% These commands are for a PROCEEDINGS abstract or paper.
% \acmConference[Conference acronym 'XX]{Make sure to enter the correct
  % conference title from your rights confirmation emai}{June 03--05,
  % 2018}{Woodstock, NY}
%%
%%  Uncomment \acmBooktitle if the title of the proceedings is different
%%  from ``Proceedings of ...''!
%%
%%\acmBooktitle{Woodstock '18: ACM Symposium on Neural Gaze Detection,
%%  June 03--05, 2018, Woodstock, NY}
% \acmISBN{978-1-4503-XXXX-X/18/06}


%%
%% Submission ID.
%% Use this when submitting an article to a sponsored event. You'll
%% receive a unique submission ID from the organizers
%% of the event, and this ID should be used as the parameter to this command.
%%\acmSubmissionID{123-A56-BU3}

%%
%% For managing citations, it is recommended to use bibliography
%% files in BibTeX format.
%%
%% You can then either use BibTeX with the ACM-Reference-Format style,
%% or BibLaTeX with the acmnumeric or acmauthoryear sytles, that include
%% support for advanced citation of software artefact from the
%% biblatex-software package, also separately available on CTAN.
%%
%% Look at the sample-*-biblatex.tex files for templates showcasing
%% the biblatex styles.
%%

%%
%% The majority of ACM publications use numbered citations and
%% references.  The command \citestyle{authoryear} switches to the
%% "author year" style.
%%
%% If you are preparing content for an event
%% sponsored by ACM SIGGRAPH, you must use the "author year" style of
%% citations and references.
%% Uncommenting
%% the next command will enable that style.
%%\citestyle{acmauthoryear}


%%
%% end of the preamble, start of the body of the document source.
\begin{document}

\title{IllusionCAPTCHA: A CAPTCHA based on Visual Illusion}

\author{Ziqi Ding}
\orcid{0009-0007-6257-1502}
\affiliation{%
  \institution{University of New South Wales}
  \city{Sydney}
  \country{Australia}
}
\email{ziqi.ding1@unsw.edu.au}

\author{Gelei Deng}
\orcid{0000-0002-0046-6674}
\affiliation{%
  \institution{Nanyang Technological University}
  \country{Singapore}
}
\email{gelei.deng@ntu.edu.sg}

\author{Yi Liu}
\orcid{0000-0002-4978-127X}
\affiliation{%
  \institution{Quantstamp}
  \country{Singapore}
}
\email{yi009@e.ntu.edu.sg}
\authornote{Corresponding author.}


\author{Junchen Ding}
\orcid{0009-0007-6531-8190}
\affiliation{%
  \institution{University of New South Wales}
  \city{Sydney}
  \country{Australia}
}
\email{jamison.ding@unsw.edu.au}

\author{Jieshan Chen}
\orcid{0000-0002-2700-7478}
\affiliation{%
  \institution{Data61, CSIRO}
  \city{Sydney}
  \country{Australia}
}
\email{jieshan.chen@data61.csiro.au}

\author{Yulei Sui}
\orcid{0000-0002-9510-6574}
\affiliation{%
  \institution{University of New South Wales}
  \city{Sydney}
  \state{New South Wales}
  \country{Australia}
}
\email{y.sui@unsw.edu.au}

\author{Yuekang Li}
\orcid{0000-0003-4382-0757}
\affiliation{%
  \institution{University of New South Wales}
  \city{Sydney}
  \country{Australia}
}
\email{yuekang.li@unsw.edu.au}

\renewcommand{\shortauthors}{Ziqi Ding et al.}

\begin{abstract}
CAPTCHAs have long been essential tools for protecting applications from automated bots. Initially designed as simple questions to distinguish humans from bots, they have become increasingly complex to keep pace with the proliferation of CAPTCHA-cracking techniques employed by malicious actors. However, with the advent of advanced large language models (LLMs), the effectiveness of existing CAPTCHAs is now being undermined.

To address this issue, we have conducted an empirical study to evaluate the performance of multimodal LLMs in solving CAPTCHAs and to assess how many attempts human users typically need to pass them. Our findings reveal that while LLMs can solve most CAPTCHAs, they struggle with those requiring complex reasoning type of CAPTCHA that also presents significant challenges for human users. Interestingly, our user study shows that the majority of human participants require a second attempt to pass these reasoning CAPTCHAs, a finding not reported in previous research.

\revision{Based on empirical findings, we present IllusionCAPTCHA, a novel security mechanism employing the "Human-Easy but AI-Hard" paradigm. This new CAPTCHA employs visual illusions to create tasks that are intuitive for humans but highly confusing for AI models. Furthermore, we developed a structured, step-by-step method that generates misleading options, which particularly guide LLMs towards making incorrect choices and reduce their chances of successfully solving CAPTCHAs. Our evaluation shows that IllusionCAPTCHA can effectively deceive LLMs 100\% of the time. Moreover, our structured design significantly increases the likelihood of AI errors when attempting to solve these challenges. Results from our user study indicate that 86.95\% of participants successfully passed the CAPTCHA on their first attempt, outperforming other CAPTCHA systems.}

\end{abstract}


% \vspace{-4mm}
\begin{CCSXML}
<ccs2012>
   <concept>
       <concept_id>10002978.10003022.10003026</concept_id>
       <concept_desc>Security and privacy~Web application security</concept_desc>
       <concept_significance>500</concept_significance>
       </concept>
 </ccs2012>
\end{CCSXML}

\ccsdesc[500]{Security and privacy~Web application security}



\keywords{CAPTCHAs, AI security, large language models, visual illusions}

\maketitle

\section{Introduction}

Chain-of-Thought (CoT) prompting~\cite{Nye:2021, cot, Kojima:2022cotzero} has emerged as a cornerstone strategy for enhancing Large Language Models (LLMs) in complex reasoning tasks. By eliciting step-by-step inference, CoT enables LLMs to decompose intricate problems into manageable subtasks, thereby improving their problem-solving performance~\cite{Yao:2023tot, Wang:2023self-consistency, Zhou:2023least, Shinn:2023Reflexion}. Recent advancements, such as OpenAI's o1~\cite{o1} and DeepSeek-R1~\cite{deepseekr1}, further demonstrate that scaling up CoT lengths from hundreds to thousands of reasoning steps could continuously improve LLM reasoning. These breakthroughs have underscored CoT’s potential to advance LLM capabilities, expanding the boundaries of AI-driven problem-solving.

\begin{figure}[t]
\centering
    \includegraphics[width=0.95\columnwidth]{fig/intro.pdf}
    \caption{In contrast to vanilla CoT that generates all reasoning tokens sequentially, \method enables LLMs to \textit{skip} tokens with less semantic importance (\textit{e.g.,} \includegraphics[width=7pt]{fig/token.pdf}~) and learn shortcuts between critical reasoning tokens, facilitating controllable CoT compression.}
    \label{fig:intro}
\end{figure}

Despite its effectiveness, the increased length of CoT sequences introduces substantial computational overhead. Due to the autoregressive nature of LLM decoding, longer CoT outputs lead to proportional increases in both inference latency and memory footprints of key-value cache. Additionally, the quadratic computational cost of attention layers further exacerbates this burden. These issues become particularly pronounced when CoT sequences extend into thousands of reasoning steps, resulting in significant computational costs and prolonged response times. While prior research has explored methods for selectively skipping reasoning steps~\cite{Ding:2024cotshortcut, liu2024skipstep}, recent findings~\cite{jin:2024cotlength, Merrill:2024cotlength} suggest that such reductions may conflict with test-time scaling~\cite{o1-blog, snell2025scaling}, ultimately impairing LLM reasoning performance. Therefore, striking an optimal balance between CoT efficiency and reasoning accuracy remains a critical open challenge.

In this work, we delve into CoT efficiency and seek the answer to an important question: \textit{``Does every token in the CoT output contribute equally to deriving the answer?''} We empirically analyze the semantic importance of tokens within CoT outputs and reveal that their contributions to the reasoning performance vary, as depicted in Figure 2. Building on this insight, we introduce \method, a simple yet effective approach that enables LLMs to \textit{skip} less important tokens within CoT sequences and learn shortcuts between critical reasoning tokens, thereby allowing for controllable CoT compression with adjustable ratios. Specifically, as shown in Figure~\ref{fig:intro}, \method constructs compressed CoT training data with various compression ratios, by pruning unimportance tokens from original LLM CoT trajectories. Then, it conducts a general supervised fine-tuning process on target LLMs with this training data, facilitating LLMs to automatically trim redundant tokens during reasoning.

We conduct extensive experiments across various models, including LLaMA-3.1-8B-Instruct and the Qwen2.5-Instruct series, using two widely recognized math reasoning benchmarks: GSM8K and MATH-500. The results validate the effectiveness of \method in compressing CoT outputs while maintaining robust reasoning performance. Notably, Qwen2.5-14B-Instruct exhibits almost \textbf{NO} performance drop (less than $0.4\%$) with a $\bm{40\%}$ reduction in token usage on GSM8K. On the challenging MATH-500 dataset, LLaMA-3.1-8B-Instruct effectively reduces CoT token usage by $\bm{30}\%$ with a performance decline of less than $4\%$, resulting in a $\bm{1.4}\times$ inference speedup. Further analysis underscores the coherence of \method in specified compression ratios and its potential scalability with stronger compression techniques.

\method is distinguished by its low training cost. For Qwen2.5-14B-Instruct, \method fine-tunes only 0.2\% of the model's parameters using LoRA. The size of the compressed CoT training data is no larger than that of the original training set, with 7,473 examples in GSM8K and 7,500 in MATH. The training is completed in approximately 2 hours for the 7B model and 2.5 hours for the 14B model on two 3090 GPUs. These characteristics make \method an efficient and reproducible approach, suitable for use in efficient and cost-effective LLM deployment.

To sum up, our key contributions are:
\begin{enumerate}
    \item To the best of our knowledge, this work is the \textit{first} to investigate the potential of enhancing CoT efficiency through \textit{token skipping}, inspired by the varying semantic importance of tokens in CoT trajectories of LLMs.
    \item We introduce \method, a simple yet effective approach that enables LLMs to skip redundant tokens within CoTs and learn shortcuts between critical tokens, facilitating CoT compression with adjustable ratios.
    \item Our experiments validate the effectiveness of \method. When applied to Qwen2.5-14B-Instruct, \method reduces reasoning tokens by $40\%$ (from 313 to 181) on GSM8K, with less than a $0.4\%$ performance drop.
\end{enumerate}

\subsection{Gene Expression Classification with ML models}
Gene expression classification \cite{do2024enhancing,do2023ensemble,huynh2019novel} lies at the forefront of biomedical research, offering profound insights into the molecular mechanisms underlying various diseases. ML models have become indispensable in this domain, as they can uncover complex patterns within vast and high-dimensional gene expression datasets. However, these datasets often contain a plethora of features, many of which are redundant or irrelevant, potentially obscuring the most critical biological signals and leading to overfitting. Consequently, feature selection becomes imperative—it refines the dataset by isolating the most informative genes, thereby enhancing model accuracy, interpretability, and computational efficiency. By focusing solely on the pivotal biomarkers, this research is able to achieve more reliable predictive outcomes. In this paper, we investigate and evaluate the classification with various ML techniques. Namely, we experiment our selected features with ML algorithms, i.e., SVM \cite{vapnik1995support}, Random Forest \cite{breiman2001random}, XGB \cite{chen2015xgboost}, Gradient Boosting \cite{friedman2002stochastic}.

\begin{definition}[Classification]
Let \( D = (X, y) \) be a dataset where \( X \subseteq \mathbb{R}^n \) is the feature space and \( y \in \mathcal{Y} = \{1,2,\dots,k\} \) represents the class labels. A classifier is a function
\[
f: X \to \mathcal{Y},
\]
that assigns a predicted label \( \hat{y} = f(x) \) to each input \( x \in X \). The function \( f \) is learned from the labeled examples
\[
D = \{(x_i, y_i) \mid x_i \in X,\; y_i \in \mathcal{Y},\; i = 1, \dots, N\},
\]
by minimizing a loss function \( \ell: \mathcal{Y} \times \mathcal{Y} \to \mathbb{R}_{\ge 0} \) that quantifies the error between the predicted and true labels. Once trained, \( f \) is used to classify new, unseen inputs.
\end{definition}

% \begin{definition}[Classification Using Machine Learning]
% Let \( D_{\text{selected}} = (X_{\text{selected}}, y) \) be the dataset with features \( X_{\text{selected}} \subseteq X^* \) as determined by LIME. A classifier is a function 
% \[
% f: X_{\text{selected}} \to \mathcal{Y},
% \]
% that assigns a predicted label \( \hat{y} = f(x) \) to each input \( x \in X_{\text{selected}} \). The classifier is trained on the labeled examples
% \[
% D_{\text{selected}} = \{(x_i, y_i) \mid x_i \in X_{\text{selected}},\; y_i \in \mathcal{Y},\; i = 1, \dots, N\},
% \]
% by minimizing a loss function \( \ell: \mathcal{Y} \times \mathcal{Y} \to \mathbb{R}_{\ge 0} \) that measures the discrepancy between the predicted and true labels. The trained classifier is then used to predict the classes of new, unseen instances.
% \end{definition}


Feature selection is crucial before classification begins. Our study focuses on two techniques: Boruta and LIME. 
% Boruta is chosen for its robustness in identifying all relevant features in high-dimensional datasets, ensuring no important predictor is missed. LIME is used for its ability to provide interpretable, local explanations of model predictions, which is essential for evaluating feature importance. 
We now introduce Boruta and LIME in the following sections.

\subsection{Leveraging Boruta for Robust Feature Extraction}
Boruta \cite{kursa2010boruta,zhou2023diabetes} is a powerful wrapper-based feature selection algorithm designed to identify all truly relevant variables in a dataset. By comparing the importance of actual features with that of randomly generated ``shadow'' features, Boruta systematically filters out irrelevant variables while preserving essential predictors. This rigorous selection process is particularly valuable in high-dimensional applications, such as gene expression classification, where capturing meaningful signals is crucial. For clarity, we formally define Boruta as follows:
\begin{definition}[Boruta Feature Selection]
Let \( D = (X, y) \) be a dataset with features \( X = \{x_1, x_2, \dots, x_p\} \) and target \( y \). The Boruta algorithm identifies all relevant features in \( X \) as follows:
\begin{enumerate}
    \item \textbf{Shadow Feature Generation:} For each \( x_i \in X \), create a shadow feature \( x_i^{\text{shadow}} \) by randomly permuting its values, forming the set \( X^{\text{shadow}} \).
    \item \textbf{Importance Estimation:} Train a classifier (e.g., Random Forest) on the combined set \( X \cup X^{\text{shadow}} \) and compute the importance score \( I(z) \) for each \( z \).
    \item \textbf{Feature Comparison:} For each \( x_i \), define
    \[
    I^{\text{shadow}}_{\max} = \max_{z \in X^{\text{shadow}}} I(z).
    \]
    Then classify \( x_i \) as \emph{relevant} if \( I(x_i) \) is significantly greater than \( I^{\text{shadow}}_{\max} \), \emph{irrelevant} if significantly lower, or \emph{tentative} otherwise.
    \item \textbf{Iteration:} Remove irrelevant and tentative features and repeat until all features are decisively classified.
\end{enumerate}
The final selected subset \( X^* \subseteq X \) comprises all features deemed relevant.
\end{definition}

After applying the Boruta algorithm, we retain only the relevant features (confirmed) and excluded the tentative and irrelevant features (rejected). To further enhance the selection of features in \(X^*\), we employed the AI explanation technique outlined in the following section.

% \begin{definition}[Boruta Feature Selection]
% Given a dataset \( D = (X, y) \) with original features \( X = \{ x_1, x_2, \dots, x_p \} \), Boruta augments \( X \) by creating shadow features \( X^{\text{shadow}} = \{ x_1^{\text{shadow}}, \dots, x_p^{\text{shadow}} \} \) via random permutation. A model \( M \) (e.g., Random Forest) is then trained on \( X \cup X^{\text{shadow}} \) to compute importance scores \( I(z) \) for every feature \( z \). For each \( x_i \in X \), if \( I(x_i) \) is significantly greater than the maximum shadow importance \( I^{\text{shadow}}_{\max} = \max_{z \in X^{\text{shadow}}} I(z) \), then \( x_i \) is marked as relevant; otherwise, it is rejected or considered tentative. Iterating this process yields the final set of selected features \( X^* \subseteq X \).
% \end{definition}

\subsection{XAI for Feature Selection}
Explainable AI (XAI) \cite{dwivedi2023explainable,zacharias2022designing} represents a forefront of AI research, aiming to elucidate the decision-making processes of complex models. In the context of gene expression classification, where feature selection is pivotal to model performance and interpretability, our study leverages LIME—Local Interpretable Model-Agnostic Explanations—to demystify and select critical features. LIME approximates the behavior of a sophisticated, black-box model with a simpler, locally interpretable surrogate, thereby pinpointing the most influential predictors in the vicinity of a given instance. This approach enhances the transparency of the model's predictions and facilitates a more informed and rigorous feature selection process, ultimately contributing to both improved accuracy and trustworthiness of the classification system.  Now, we provide a formal definition of LIME as follows:

% \begin{definition}[LIME-based Feature Selection]
% Let \( D = (X, y) \) be a dataset and \( f: X \to \mathcal{Y} \) a trained black-box classifier, where \( X \subseteq \mathbb{R}^p \) and \( \mathcal{Y} = \{1,2,\dots,k\} \). For a given instance \( x \in X \), LIME constructs an interpretable surrogate model \( g \) from a simple model class \( G \) (typically linear), expressed as
% \[
% g(z) = w_0 + \sum_{j=1}^{p} w_j z_j.
% \]
% The surrogate \( g \) is fitted by minimizing the weighted loss
% \[
% \min_{g \in G} \sum_{z \in Z_x} \pi_x(z) \left( f(z) - g(z) \right)^2 + \Omega(g),
% \]
% where \( Z_x \) is a set of perturbed samples around \( x \), \( \pi_x(z) \) is a proximity measure between \( z \) and \( x \), and \( \Omega(g) \) is a regularization term enforcing simplicity. The absolute coefficients \( |w_j| \) quantify the local importance of each feature, thus guiding feature selection.
% \end{definition}
\begin{definition}[LIME-based Feature Selection]
Let \( D^* = (X^*, y) \) be the dataset resulting from Boruta, where \( X^* \subseteq \mathbb{R}^{p^*} \) is the set of relevant features. Given a trained black-box classifier \( f: X^* \to \mathcal{Y} \) and an instance \( x \in X^* \), LIME constructs an interpretable surrogate model \( g \in G \) (typically linear), expressed as
\[
g(z) = w_0 + \sum_{j=1}^{p^*} w_j z_j,
\]
by solving the optimization problem
\[
\min_{g \in G} \sum_{z \in Z_x} \pi_x(z) \left( f(z) - g(z) \right)^2 + \Omega(g),
\]
where \( Z_x \) is a set of perturbed samples in the neighborhood of \( x \), \( \pi_x(z) \) is a proximity measure, and \( \Omega(g) \) enforces simplicity. The absolute coefficients \( |w_j| \) indicate the local importance of each feature, enabling a further refined selection \( X_{\text{selected}} \subseteq X^* \) for classification.
\end{definition}


To clarify, our choice of LIME for feature selection arises from the critical question of determining the optimal number of features for the model. In this context, assessing the local importance of each vector proves to be the most effective strategy, leading us to introduce the BOLIMES algorithm. The following section will provide a comprehensive explanation of the BOLIMES algorithm and its application.

%--------------------





\section{Threat Model}
In this paper, we outline our assumptions regarding the goals and capabilities of attackers.

\noindent\textbf{Attacker goals.} We assume that the adversary aims to automatically solve CAPTCHAs without human interactions, which could potentially lead to these results~\cite{che2021augmented,shi2020text}:
(1) Automating Actions: Gaining unauthorized access to websites, applications, or services to automate tasks (e.g. account creation, data scraping, or spamming). (2) Credential Harvesting: Exploiting CAPTCHA weaknesses to gain access to user accounts by defeating login protections. (3) Fraudulent Activities: Engaging in malicious activities like ticket scalping, purchasing limited-edition items, or bypassing purchase limits imposed by websites. (4) Disruption of Services: Creating bot networks that can flood websites with traffic, breaking CAPTCHAs to disrupt normal operations.


\noindent\textbf{Attacker capabilities.} We assume the attacker is restricted to interacting with the CAPTCHA through the graphical interface, without using techniques such as reverse engineering, JavaScript decompiling, or direct code analysis. In this work, we primarily consider that the adversary abuses the capabilities of multimodal LLMs with their reasoning capabilities and object recognition capabilities. These LLMs can be utilized not only to solve CAPTCHAs but also to automate the entire attack process—from selecting target websites to registering accounts—enabling a highly efficient and scalable attack pipeline.


% Please add the following required packages to your document preamble:
% \usepackage{multirow}
\begin{table}[t]
    \centering
    % \scriptsize
    % \footnotesize
    \tabcolsep=1.5pt
    \renewcommand{\arraystretch}{0.92} 
    \caption{Experimental results of applying the multi-model LLMs over the selected CAPTCHAs.}
 % Please add the following required packages to your document preamble:
% \usepackage{multirow}
% Please add the following required packages to your document preamble:
% \usepackage{multirow}
\resizebox{\linewidth}{!}{
    \begin{tabular}{cc|cc|cc}
    \hline
    \multicolumn{2}{c|}{\textbf{Method}}                                                            & \multicolumn{2}{c|}{\textbf{Zero-Shot}}                                  & \multicolumn{2}{c}{\textbf{COT}}                                         \\ \hline
    \multicolumn{2}{c|}{\textbf{Metric}}                                                            & \multicolumn{2}{c|}{\textbf{Success Rate}}                               & \multicolumn{2}{c}{\textbf{Success Rate}}                                \\ \hline
    \multicolumn{2}{c|}{\textbf{Model}}                                                             & \multicolumn{1}{c|}{\textbf{GPT4o}} & \textbf{Gemini} & \multicolumn{1}{c|}{\textbf{GPT4o}} & \textbf{Gemini} \\ \hline
    \multicolumn{1}{c|}{\multirow{4}{*}{\textbf{Text-based}}}  & \textbf{Simplest}          & 76.66\%                                    & 73.33\%                     & 90.00\%                                    & 83.33\%                     \\
    \multicolumn{1}{c|}{}                                              & \textbf{Overloaping}       & 66.66\%                                    & 60\%                        & 70.00\%                                    & 60.00\%                     \\
    \multicolumn{1}{c|}{}                                              & \textbf{Noise}             & 70.00\%                                    & 73.33\%                     & 73.33\%                                    & 66.66\%                     \\
    \multicolumn{1}{c|}{}                                              & \textbf{Noise+Overloaping} & 36.66\%                                    & 23.33\%                     & 50.00\%                                    & 43.33\%                     \\ \hline
    \multicolumn{1}{c|}{\multirow{2}{*}{\textbf{Image-based}}} & \textbf{reCAPTCHA}         & 40.00\%                                    & 33.33\%                     & 50.00\%                                    & 23.33\%                     \\
    \multicolumn{1}{c|}{}                                              & \textbf{hCAPTCHA}          & 40.00\%                                    & 36.66\%                     & 43.33\%                                    & 30.00\%                     \\ \hline
    \multicolumn{1}{c|}{\multirow{5}{*}{\textbf{Reasoning}}}   & \textbf{Angular}           & 13.33\%                                    & 0.00\%                      & 13.33\%                                    & 0.00\%                      \\
    \multicolumn{1}{c|}{}                                              & \textbf{Gobang}            & 0.00\%                                     & 0.00\%                      & 6.66\%                                     & 3.33\%                      \\
    \multicolumn{1}{c|}{}                                              & \textbf{IconCrush}         & 0.00\%                                     & 0.00\%                      & 16.66\%                                    & 10.00\%                     \\
    \multicolumn{1}{c|}{}                                              & \textbf{Space}             & 46.66\%                                    & 26.66\%                     & 53.33\%                                    & 26.66\%                     \\
    \multicolumn{1}{c|}{}                                              & \textbf{Space Reasoning}   & 33.33\%                                    & 20.00\%                     & 40.00\%                                    & 23.33\%                     \\ \hline
    \multicolumn{2}{c|}{\textbf{Average}}                                                           & \textbf{38.48\%}                           & \textbf{31.51\%}            & \textbf{46.06\%}                           & \textbf{33.63\%}            \\ \hline
    \end{tabular}
    }

\label{tab:Empirical-of-LLM}
\end{table}

\begin{table}[t]
    \centering
    % \scriptsize
    % \footnotesize
    \tabcolsep=1.5pt
    \renewcommand{\arraystretch}{0.92} 
    \caption{Experimental results of applying the multi-model LLMs over the selected CAPTCHAs.}
    \begin{tabular}{c|cccc}
    \hline
    \textbf{\# of Attempt}       & \multicolumn{1}{c|}{\textbf{1}} & \multicolumn{1}{c|}{\textbf{2}} & \multicolumn{1}{c|}{\textbf{3}} & \textbf{>3} \\ \hline
    \textbf{Text-based CAPTCHA}  & 47.82\%                                     & 39.13\%                                      & 8.69\%                                      & 4.34\%                     \\
    \textbf{Image-based CAPTCHA} & 30.43\%                                     & 56.52\%                                      & 4.34\%                                      & 8.69\%                     \\
    \textbf{Reasoning CAPTCHA}   & 21.73\%                                     & 43.47\%                                      & 21.73\%                                     & 13.04\%                    \\ \hline
    \textbf{Average}             & \textbf{33.33\%}                            & \textbf{46.37\%}                             & \textbf{11.59\%}                            & \textbf{8.69\%}            \\ \hline
    \end{tabular}
\label{tab:user-study}
\end{table}

\section{Empirical Study}
\label{sec:empirical_study}

We first conduct a systematic empirical study to assess the effectiveness of LLMs in identifying both traditional and modern CAPTCHAs. The full potential of LLMs in this area remains largely unexplored. Additionally, to address the knowledge gap among human users regarding CAPTCHAs, we designed a user study to assess user performance across various CAPTCHA challenges. This investigation is structured around two research questions:

\begin{itemize}[leftmargin=*]
    \setlength\itemsep{0pt}
    \item \textbf{RQ1 (Effectiveness):} How effective are LLMs in accurately solving CAPTCHAs, and what types of errors are they most likely to make?
    \item \textbf{RQ2 (User Study):} Are human users able to effectively resolve various categories of CAPTCHA challenges, and what specific obstacles do they encounter throughout the process?
\end{itemize}

In the following of this section, we address the two research questions through two sets of experiments. 

\subsection{Effectiveness of LLMs in Solving CAPTCHAs}



\noindent\textbf{CAPTCHA Categorization.} Different from other works~\cite{deng2024oedipus,searles2023empirical}, we cover all categories of visual CAPTCHAs. Within each category, there are different designs from different vendors. Therefore, we compiled widely deployed CAPTCHAs available online and organized them into the following detailed subcategories.

\begin{itemize}[leftmargin=*]
    \item \textbf{Text-based CAPTCHAs.} We collect different types of text-based CAPTCHAs that requires users to recognize a series of letters or characters. After survey, we conclude four types of them available now. (1) \textbf{Simplest Text-based CAPTCHAs}, shown in Figure~\ref{fig:Text-based}(a), is the simplest text-based CAPTCHAs, which is also the most popular CAPTCHAs online. This type of challenge can be solved easily by traditional CAPTCHA solvers. These typically feature clear, unaltered text, making them vulnerable to basic image recognition techniques. 
    (2) \textbf{Noisy Text-based CAPTCHAs}, shown in Figure~\ref{fig:Text-based}(b), introduce visual noise, such as random lines, dots, or distortions into the text CAPTCHA, which can interfere with traditional CAPTCHA solvers. Despite the added complexity, they still primarily ensure that users could recognize the contents within.
    (3) \textbf{Overlapping Text-based CAPTCHAs}, shown in Figure~\ref{fig:Text-based}(c), are a type of text-based CAPTCHAs that involve texts where characters are overlapped with each other at different angles. While this writing style is totally recognizable to humans, it is hard for traditional solvers~\cite{ye2018yet} that relies on segmentation strategies to solve. 
    (4) \textbf{Noise-enhanced Overlapping Text-based CAPTCHAs}, shown in Figure~\ref{fig:Text-based}(d), are the type of challenges combine both visual noise and overlapping texts, which significantly increases the difficulty for traditional CAPTCHA solvers to counter. 

    \item \textbf{Image-based CAPTCHAs.} In addition to the traditional text-based CAPTCHAs, more recent ones include images that tests the common sense of users as a type of challenge. We conclude two types of basic image-based CAPTCHAs.
    (1) \textbf{reCAPTCHA} presents users with tasks like selecting images (image classification) that contain specific objects, such as traffic lights or crosswalks, or verifying street signs. Vastly adopted by Google, it is the most common types of CAPTCHA that has been well researched. There are three versions of reCAPTCHAs, with similar image patterns but different underlying mechanisms to counter traditional automated solutions such as JavaScript reverse engineering. 
    (2) \textbf{hCAPTCHA} involves more detailed image recognition tasks, requiring users to have a stronger ability to understand the prompts (e.g., selecting images that contain wheels).
    
    \item \textbf{Reasoning-based CAPTCHAs} 
    are new emerging category of challenges that aims to counter the automated solvers powered by deep learning methods. After survey, we identify three types of reasoning-based CAPTCHAs.
    (1) \textbf{Rotation CAPTCHAs}, also known as Angular by their developers, require users to adjust an object’s orientation to align with a reference object. 
    As shown in Figure~\ref{fig:Reasoning-based}(a), users need to properly recognize the orientation of two different objects (the finger and the lamb in this example) to solve the challenge. There are two versions of Rotation CAPTCHAs available in the market now, both devleoped by Arkose Labs. 
    (2) \textbf{Bingo CAPTCHAs (Gobang \& IconCrush)} is a new type of reasoning-based CAPTCHA also developed by Arkose Labs.
    As seen in Figure~\ref{fig:Reasoning-based}(b), this type of challenge tasks users with identifying and rearranging elements on a board to create a line of matching items. The types of elements and the rules for manipulation can differ widely based on the provider. For instance, in Figure~\ref{fig:Reasoning-based}(b), users can swap any two items without restriction, while in Figure~\ref{fig:Reasoning-based}(c), swaps are limited to adjacent items, illustrating the range of variation in this type of CAPTCHA.
    (3) \textbf{3D Logical CAPTCHAs}, as demonstrated in Figure~\ref{fig:Reasoning-based}(d) and Figure~\ref{fig:Reasoning-based}(e), requires users to choose an object from a 3D environment. This process requires users to identify the logical relationships tied to attributes like shape, color, and orientation of the objects within the challenge. For instance, in Figure~\ref{fig:Reasoning-based}(d), users must identify the number 0 that aligns with the orientation of a yellow letter W, whereas Figure~\ref{fig:Reasoning-based}(e) asks users to select the larger object positioned to the left of a green object.
    
\end{itemize}





\noindent\textbf{Dataset Collection.} To rigorously assess the ability of LLMs to solve CAPTCHAs, we include the three types of CAPTCHAs as discussed in the Background: text-based, image-based and reasoning-based CAPTCHAs. In particular, we exclude audio CAPTCHAs due to their limited usage online~\cite{fanelle2020blind}, which is mainly for visually impaired people. Furthermore, our study emphasizes real-world scenarios, so all CAPTCHAs used were collected from website applications. Consequently, we built a dataset comprising three types of CAPTCHA (text-based CAPTCHAs shown in Figure~\ref{fig:Text-based}, image-based CAPTCHAs shown in Figure~\ref{fig:Image-based} and reasoning-based CAPTCHAs shown in Figure~\ref{fig:Reasoning-based} \revision{each containing 30 capthchas per subdivision.}.


\noindent \textbf{Methodology.}  To evaluate these CAPTCHAs, we employ two powerful LLMs (Gemini 1.5 pro 2.0 and GPT4-o) using both Zero-Shot and Chain-of-Thought (COT) methodologies. Each CAPTCHA category presents a unique set of challenges that require customized solution strategies. As a result, we utilize different LLM prompts to predict the outcomes of various CAPTCHAs, measuring success rates as our primary metric. We manually analyze each LLM response to ensure the accuracy of the results. In the zero-shot approach, a solution is considered correct only if the LLM outlines the exact procedure to solve the CAPTCHA. In contrast, in the CoT approach, a substep is deemed successful if the LLM's proposed solution for that specific sub-step is accurate.

\noindent\textbf{Result Analysis.} Table~\ref{tab:Empirical-of-LLM}—with GPT4o representing GPT-4o and Gemini representing Gemini 1.5 pro 2.0—presents our evaluation of LLMs' effectiveness in solving CAPTCHAs. Using a zero-shot approach, the LLMs successfully solve most text-based CAPTCHAs, except for those with overlapping characters or significant noise, while their accuracy drops to only 40\% for image-based CAPTCHAs. Moreover, LLMs face challenges with reasoning-based CAPTCHAs due to their limited reasoning capabilities; however, employing COT prompting significantly enhances their performance in identifying these types of CAPTCHAs. These findings show the growing threat that advancements in LLMs pose to web security, suggesting that current CAPTCHA methods may no longer be sufficiently secure.

\begin{center}
    \setlength{\fboxrule}{1pt}
    %\fbox
    \fcolorbox{lightgray}{mygray}{%
      \parbox{0.47\textwidth}{%
        \textbf{Answer to RQ1:}
        \emph{Our verification experiment reveals that (1) LLMs perform better on text-based CAPTCHAs compared to image-based and reasoning-based CAPTCHAs; and (2) although LLMs struggle with complex reasoning CAPTCHAs, their performance significantly improves when employing the Chain-of-Thought (CoT) strategy. This suggests that with reasoning chains, LLMs have the potential to overcome these challenges. Consequently, this indicates that current CAPTCHAs may no longer be as secure as intended.}
        }%
    }
\end{center}


\begin{figure*}[!t]
	\centering
    \includegraphics[width=0.9\linewidth]{Figure/Overview_of_IllusionCAPTCHA.eps}
	\caption{Overview of IllusionCAPTCHA}
     % \vspace{-10pt}
	\label{fig:Illusion-based}
\end{figure*}


\subsection{User Study}

To investigate user behavior, we designed a questionnaire-based study. Because some CAPTCHAs cannot be reliably retrieved from their original sources, we constructed the study by extracting CAPTCHA images directly from their native web applications and manually annotating the correct responses. All images were drawn from the dataset we mentioned before.

\noindent\textbf{User Study Settings.} Our questionnaire allocates each participant 1 minute to solve a CAPTCHA. If they cannot complete it within that time, they must attempt it again until they succeed. During this process, we record both successful and failed attempts. Finally, we have 23 human participants in our study.

\noindent\textbf{Result Analysis.} Table~\ref{tab:user-study} presents the results of our user study, revealing that most participants were unable to solve the CAPTCHA on their first attempt; notably, both image-based and reasoning-based CAPTCHAs proved particularly challenging, with some individuals requiring more than three attempts to successfully pass them.


\begin{center}
    \setlength{\fboxrule}{1pt}
    %\fbox
    \fcolorbox{lightgray}{mygray}{%
      \parbox{0.47\textwidth}{%
        \textbf{Answer to RQ2:}
        \emph{The result of our user study reveals that (1) While reasoning-based CAPTCHAs pose significant challenges for AI systems, they are also difficult for human users. Hence, these CAPTCHAs can easily frustrate users, leading to diminished patience during their attempts. (2) Human users frequently make the same mistakes as LLMs, highlighting the need to develop methods that can effectively distinguish between LLMs and human users.}
        }%
    }
\end{center}












\section{Methodology}

As illustrated in Figure~\ref{fig:Illusion-based}, IllusionCAPTCHA generates CAPTCHA challenges through a three-step process. 
First, it blends a base image with a user-defined prompt, such as ``huge forest,'' to create a visual illusion that obscures the original content. With the prompt, the output image will be looked like the things in the prompt, hiding its true content from base image. This results in images that, while recognizable to humans, can confuse AI systems. 
Second, multiple-choice options are generated based on the altered images, forming the CAPTCHA challenge options.
Our empirical study indicates that humans may occasionally make errors similar to those of LLMs, suggesting that relying solely on illusionary images may not be sufficient to distinguish human users from bots. 
Therefore, we incoperate the third step of ``Inducement Prompt'' to induce our LLM-based attackers to choose the intended choice. Moreover, we utilize multimodel question to increase difficulty for attackers but easy for human users to identify. Below we detail the design of IllusionCAPTCHA.
%

\subsection{Illusionary Image Generation}

\begin{figure}[!t]
	\centering
    \includegraphics[width=0.8\linewidth]{Figure/Picture_1.eps}
	\caption{An example of the original and illusionary image}
     % \vspace{-10pt}
	\label{fig:Illusion-example}
\end{figure}
\label{sec:method}
The first objective is to create illusionary images that are easily recognizable by humans but difficult for AI systems to identify. This process involves tackling two primary challenges: (1) maintaining the context of the base image, and (2) add disturbance to the image particularly effective for AI systems to interfere with their capabilities while maintaining recognizability for humans. 

To address the first challenge, we employ an illusion diffusion model~\cite{AP123}, which generates images by blending two different types of content. Built upon ControlNet~\cite{zhang2023adding}, a framework that allows precise control over image generation through conditional inputs, this model ensures that the resulting images remain accessible to human viewers while being challenging for automated systems to interpret.
Figure~\ref{fig:Illusion-example} shows how a normal image is transferred into an illusionary one.
However, not all generated images will effectively balance recognizability for humans while fooling AI vision. To overcome the second challenge, we first generate 50 sample images using different seeds, all within the range of 0 to 5, at a fixed illusion strength level of 1.5—an optimal value for human identification in this context. We then calculate the cosine similarity between each generated image and the base image, selecting the one with the lowest similarity, which can be seen as the most diffcult images for bots to identify.

To enhance the perceptibility of the generated images, we develop tailored strategies for two types of illusion-based CAPTCHAs: text-based CAPTCHAs and image-based CAPTCHAs. In the first scenario, the base image contains a clear, readable word embedded within an illusion. To ensure that human users can still recognize the text with minimal effort, we opt for simple, familiar English words such as ``day'' or ``sun.'' 
In the second scenario, the base image features a well-known, easily recognizable character or object, such as an iconic symbol or a famous location (e.g. ``Eiffel Tower''). This ensures that human users can quickly identify the content, even with added illusionary elements. 
These strategies aim to strike a balance between maintaining human usability and introducing complexity that misleads AI systems. 


\subsection{Options Setup} 

Our options have been meticulously crafted to safeguard against LLM-based attacks. In our CAPTCHA, we offer four distinct options. One option represents the correct answer usually the hidden content of our image, while another is the input prompt we utilize in the generation of our image. The remaining two options consist of detailed descriptions of our prompt part without the correct answer, intentionally crafted without referencing any content from our true answer.

Unlike traditional CAPTCHAs that require users to type text or select multiple images to answer a question, our CAPTCHA asks users to choose the correct description of an image. This design simplifies the process by offering a hint, making it easier for users to identify the correct answer without needing to click through multiple images.

Compared to text-based CAPTCHAs, ours is more user-friendly, as it avoids the challenges posed by vague images. Additionally, in contrast to hCAPTCHA and reCAPTCHA, our approach reduces the difficulty of making a selection. Unlike reasoning-based CAPTCHAs that require users to manipulate images, which can lead to frustration, our design eliminates the need for such interactions, further improving user experience.


\subsection{Inducement Prompt}

Building on our empirical study, we discover that both LLMs and human users tend to make similar errors when presented with certain types of CAPTCHAs. Additionally, human users often require a second attempt to pass the CAPTCHA successfully. As a result, relying on a single question to differentiate between AI and human users proves insufficient.
To address this issue, we designed a system that aims to lure potential attackers, such as multimodal LLMs, into selecting predictable, bot-like answers. Our CAPTCHA format uses multiple-choice questions, each offering four answer options.

Our strategy centers on the idea to trick the LLM-based adversary to select the option that describes the illusionary element added, which is the object that LLMs typically fails to capture. Research~\cite{hu2024bliva} has shown that LLMs typically describe images with long, detailed sentences. To exploit this, we include one option that features an intentionally elaborate, detailed description of the illusionary elements in the image (e.g., "a vast forest filled with birds, depicting a beautiful and serene scene").

Additionally, to reduce the difficulty for human users, we embed hints within the questions that guide them toward the correct answer. Therefore, these hints(e.g. Tell us the \textbf{true} and \textbf{detailed} answer of this image) are crafted to trigger hallucinations in LLMs, further increasing the likelihood that bots will select incorrect responses, although they are in the prompt that attacker sets before.








\section{Evaluation}
To evaluate the performance of our IllusionCAPTCHA, we have structured our evaluation around four research questions:

\begin{itemize}[noitemsep,leftmargin=*] 
\item \textbf{RQ3: Human Identification of Illusionary Images.} Can the illusionary images generated by IllusionCAPTCHA remain identifiable to human users? 

\item \textbf{RQ4: LLM Deception by Illusionary Content.} Can the illusionary content effectively deceive LLMs into selecting a false answer? 

\item \textbf{RQ5: Inducement Prompts Effectiveness.} Can the CAPTCHA structure we designed compel bots to make targeted choices? 

\item \textbf{RQ6: Human Attempts to Pass CAPTCHA.} How many attempts do human users require to successfully pass our designed CAPTCHA? 
\end{itemize}






\subsection{RQ3: Human Identification of Illusionary Images}

\noindent\textbf{Motivation.} In this section, we examine whether illusionary images can effectively convey information to human users, a critical factor since a CAPTCHA image must clearly communicate its intended message to its target audience.

\noindent\textbf{Method.} To address RQ3, we designed a questionnaire to assess human users' ability to identify illusionary images. The questionnaire comprises two types of images—text-based and image-based illusionary images—each consisting of five samples. All ten samples were generated using the method described in Section~\ref{sec:method}, and to avoid copyright issues, the base images were produced using a diffusion technique. Below, we provide the details of our questionnaire.


\subsection{Pre-trained Model Selection (RQ1)} \label{subsec:rq1}

\sectopic{Methodology. }  We shortlist the ST models for investigation in our work based on the NLP  leaderboard, which reports the 38 most accurate pre-trained models\footnote{\url{https://www.sbert.net/docs/pretrained_models.html}}. These models have been extensively evaluated for their ability to generate sentence embeddings (i.e., capturing the semantics of the whole text) and their performance in semantic search (i.e., finding relevant answers to a given query). Both tasks closely align with our objectives. 
To identify trace links, we apply the pre-trained models in a zero-shot setting as follows. 
We let each model compute the similarity matrix equivalent to the output of step~5 in our approach (see Fig.~\ref{fig:approach}). 
We then predict a trace link if the similarity value exceeds 
a predefined threshold. Since zero-shot does not require training, we run EXPI on the entire \texttt{HIPAA} dataset. 


\sectopic{Evaluation Metrics. } To better assess the performance irrespective of the selected threshold, we compute the \textit{Area Under the Curve (AUC)} for the receiver operating characteristic (ROC) across different threshold values,  ranging from $0.1$ to $0.9$. 
The ROC curve captures the trade-off between the true positive rate (TPR) and the false positive rate (FPR). TPR is the proportion of positives correctly identified as such (i.e., the percentage of trace links correctly identified for a given threshold). FPR is the proportion of negatives incorrectly identified as positives (i.e., the percentage of trace links wrongly identified as not trace links). The AUC of the ROC curve (computed as micro-average over all the provisions to avoid the dominance of some provisions)  provides a single aggregate performance measure across all possible thresholds and, hence, is a suitable evaluation metric to compare the ST models.  We posit that the model with the highest AUC value demonstrates the best overall performance in identifying trace links in a zero-shot setting, as a higher AUC value indicates a better balance between correctly identifying true trace links (high TPR) and minimizing the identification of false links (low FPR). 
%
%
\sectopic{Results. }
Table~\ref{tab:rq1} presents the \texttt{AUC} values of the ST pre-trained models on the \texttt{HIPAA} dataset and also  reports $K$, indicating the ranking of the models in the NLP community based on their accuracy~\cite{Reimers:19}, as well as $K^\dag$, indicating the ranking based on \texttt{AUC} achieved on \texttt{HIPAA}. 



\begin{table}
%\footnotesize
\centering
\caption{AUC of ST models for LRT on \texttt{HIPAA} (\textbf{RQ1}). 
% \TBD{@Romina: come on! you don't leave footnotes on TRACES in the table. Please revise your changes. Also, you don't need "HIPAA" in the header if the results are only for HIPAA, @Romina: please remove and adjust the header accordingly}
}
\label{tab:rq1}
% \begin{threeparttable}[t]
\begin{tabularx}{\textwidth}{@{} p{0.05\textwidth} @{\hskip 0.5em} p{0.05\textwidth} @{\hskip 3em} p{0.05\textwidth} @{\hskip 20em} *{5}{>{\centering\arraybackslash}X}@{}}
    \toprule
    \multirow{1}{*}{$K$\tnote{1}} & \multirow{1}{*}{Model\tnote{2}} & \multirow{1}{*}{Name\tnote{1}} & \multirow{1}{*}{\texttt{AUC}\tnote{1}} & \multirow{1}{*}{$K^\dag$\tnote{1}} \\%\multicolumn{2}{c}{\texttt{HIPAA}} \\ %& \multicolumn{2}{c}{\texttt{TRACES}} & \multicolumn{2}{c}{Average}\\
    % \cmidrule(lr){4-5}
    % &&& \texttt{AUC} & $K^\dag$ \\ %&\texttt{AUC} & $K^\ddag$ &\texttt{AUC} & $K^*$  \\
    \midrule
1 &   \texttt{ST1}  & \texttt{all-mpnet-base-v2} & 0.744 & 16 \\ % & 0.331 & 29 & 0.538 & 27\\
2 &   \texttt{ST2}  & \texttt{gtr-t5-xxl} & 0.725 & 21 \\ % & \textbf{0.685} & 1 & 0.705 & 7\\
3 &   \texttt{ST3}  &\texttt{gtr-t5-xl} & 0.789 & 6 \\ % & 0.678 & 2 & 0.733 & 2\\
4 &   \texttt{ST4}  &\texttt{sentence-t5-xxl} & 0.720 & 22 \\ % & 0.666 & 3 & 0.693 & 8\\
5 &   \texttt{ST5}  &\texttt{gtr-t5-large} & 0.743 & 17 \\ % & 0.640 & 7 & 0.692 & 9\\
6 &   \texttt{ST6}  &\texttt{all-mpnet-base-v1} & 0.712 & 25 \\ % & 0.338 & 27 & 0.525 & 29\\
7 &   \texttt{ST7}  &\texttt{multi-qa-mpnet-base-dot-v1} & 0.688 & 27 \\ % & 0.631 & 8 & 0.659 & 12\\
8 &   \texttt{ST8}  &\texttt{multi-qa-mpnet-base-cos-v1} & 0.603 & 34 \\ % & 0.222 & 36 & 0.413 & 36\\
9 &   \texttt{ST9}  &\texttt{all-roberta-large-v1} & 0.601 & 35 \\ % & 0.333 & 28 & 0.467 & 34\\
10 &   \texttt{ST10}  &\texttt{sentence-t5-xl} & 0.769 & 10 \\ % & 0.644 & 6 & 0.706 & 5\\
11 &   \texttt{ST11}  &\texttt{all-distilroberta-v1} & 0.719 & 23 \\ % & 0.284 & 34 & 0.501 & 32\\
12 &   \texttt{ST12}  &\texttt{all-MiniLM-L12-v1} & 0.729 & 19 \\ % & 0.318 & 30 & 0.523 & 30\\
13 &   \texttt{ST13}  &\texttt{all-MiniLM-L12-v2} & 0.747 & 15 \\ % & 0.339 & 26 & 0.543 & 26\\
14 &   \texttt{ST14}  &\texttt{multi-qa-distilbert-dot-v1} & 0.563 & 36 \\ % & 0.546 & 17 & 0.555 & 25\\
15 &   \texttt{ST15}  &\texttt{multi-qa-distilbert-cos-v1} & 0.640 & 33 \\ % & 0.228 & 35 & 0.434 & 35\\
16 &   \texttt{ST16}  &\texttt{gtr-t5-base} & 0.770 & 9 \\ % & 0.655 & 5 & 0.712 & 4\\
17 &   \texttt{ST17}  &\texttt{sentence-t5-large} & 0.748 & 14 \\ % & 0.663 & 4 & 0.706 & 6\\
18 &   \texttt{ST18}  &\texttt{all-MiniLM-L6-v2} & 0.761 & 11 \\ % & 0.285 & 33 & 0.523 & 31\\
19 &   \texttt{ST19}  &\texttt{multi-qa-MiniLM-L6-cos-v1} & 0.670 & 29 \\ % & 0.313 & 31 & 0.492 & 33\\
20 &   \texttt{ST20}  &\texttt{all-MiniLM-L6-v1} & 0.749 & 13 \\ % & 0.307 & 32 & 0.528 & 28\\
21 &   \texttt{ST21}  &\texttt{paraphrase-mpnet-base-v2} & 0.850 & 2 \\ % & 0.587 & 14 & 0.719 & 3\\
22 &   \texttt{ST22}  &\texttt{msmarco-bert-base-dot-v5} & 0.644 & 32 \\ % & 0.503 & 20 & 0.574 & 24\\
23 &   \texttt{ST23}  & \texttt{multi-qa-MiniLM-L6-dot-v1} & 0.715 & 24 \\ % & 0.605 & 12 & 0.660 & 11\\
24 &   \texttt{ST24}  & \texttt{sentence-t5-base} & 0.726 & 20 \\ % & 0.584 & 15 & 0.655 & 13\\
25 &   \texttt{ST25}  & \texttt{msmarco-distilbert-base-tas-b} & 0.701 & 26 \\ % & 0.557 & 16 & 0.629 & 18\\
26 &   \texttt{ST26}  & \texttt{msmarco-distilbert-dot-v5} & 0.685 & 28 \\ % & 0.600 & 13 & 0.643 & 15\\
27 &   \texttt{ST27}  & \texttt{paraphrase-distilroberta-base-v2} & 0.801 & 4 \\ % & 0.455 & 24 & 0.628 & 19\\
28 &   \texttt{ST28}  & \texttt{paraphrase-MiniLM-L12-v2} & 0.794 & 5 \\ % & 0.496 & 22 & 0.645 & 14\\
29 &   \texttt{ST29}  & \texttt{paraphrase-multilingual-mpnet-base-v2} & \textbf{0.859} & 1 \\ % & 0.614 & 10 & \textbf{0.736} & 1\\
30 &   \texttt{ST30}  & \texttt{paraphrase-TinyBERT-L6-v2} & 0.787 & 7 \\ % & 0.464 & 23 & 0.625 & 21\\
31 &   \texttt{ST31}  & \texttt{paraphrase-MiniLM-L6-v2} & 0.770 & 8 \\ % & 0.511 & 18 & 0.641 & 16\\
32 &   \texttt{ST32}  & \texttt{paraphrase-albert-small-v2} & 0.737 & 18 \\ % & 0.499 & 21 & 0.618 & 22\\
33 &   \texttt{ST33}  & \texttt{paraphrase-multilingual-MiniLM-L12-v2} & 0.811 & 3 \\ % & 0.511 & 19 & 0.661 & 10\\
34 &   \texttt{ST34}  & \texttt{paraphrase-MiniLM-L3-v2} & 0.757 & 12 \\ % & 0.441 & 25 & 0.599 & 23\\
35 &   \texttt{ST35}  & \texttt{distiluse-base-multilingual-cased-v1} & 0.349 & 37 \\ % & 0.092 & 37 & 0.220 & 37\\
36 &   \texttt{ST36}  & \texttt{distiluse-base-multilingual-cased-v2} & 0.341 & 38 \\ % & 0.090 & 38 & 0.216 & 38\\
37 &   \texttt{ST37}  & \texttt{average\_word\_embeddings\_komninos} & 0.647 & 31 \\ % & 0.606 & 11 & 0.627 & 20\\
38 &   \texttt{ST38}  & \texttt{average\_word\_embeddings\_glove.6B.300d} & 0.636 & 30 \\ % & 0.625 & 9 & 0.630 & 17\\ 
\bottomrule
\end{tabularx}
\begin{tablenotes}
     \item[1] $K$: The average performance ranking of the models, as reported in the NLP community. $K^\dag$: The ranking of the models based on AUC values computed on \texttt{HIPAA} ($K=1$ indicates the highest AUC). 
      \item [2] \texttt{ST1}--\texttt{ST38} correspond to the models reported at this link (sorted by average accuracy in descending order):     \url{https://www.sbert.net/docs/pretrained_models.html}. %, where \texttt{ST29} is \texttt{paraphrase-multilingual-mpnet-base-v2}.
     \end{tablenotes}
 % \end{threeparttable}
 %\vspace*{-2em}
 \end{table}

 

The best-performing model on \texttt{HIPAA} is \texttt{ST29} ($K^\dag=1$), with an AUC value of 0.859. The next best performing model is \texttt{ST21} with an AUC value of 0.850. The difference between these two AUC values is only marginal. A possible explanation is that  \texttt{ST29} uses  \texttt{ST21} as its base model.  \texttt{ST29}  has been, however, trained on more (multi-lingual) data.   

Additionally, we observe a discrepancy in the performance of the models on the \texttt{HIPAA} dataset compared to that reported by the NLP community.  
The best NLP model, \texttt{ST1}, does not perform well  on \texttt{HIPAA}, ranked 16. 
This observation indicates that well-performing models in NLP are not necessarily as effective for RE-specific problems. 
%The datasets in RE are typically domain-specific increasing the level of complexity to deal with.    

\begin{tcolorbox}[arc=1mm,width=\columnwidth,
                  top=0mm,left=0mm,  right=0mm, bottom=0mm,
                  boxrule=1pt, colback=violet!15!white,colframe=white]
\textbf{The answer RQ1} is that \texttt{ST29} is the best-performing pre-trained model for LRT (corresponding to \texttt{paraphrase-multilingual-mpnet-base-v2}). 
\end{tcolorbox}%The goal of RQ1 is to select a robust ST model that performs consistently well across datasets. 
% Table~\ref{tab:rq1} shows the \texttt{AUC} values of the ST pre-trained models on the \texttt{HIPAA} and \texttt{TRACES} datasets. The table also reports $K$ indicating the ranking of the models in the NLP community based on their accuracy~\cite{Reimers:19}, as well as $K^\dag$,  $K^\ddag$ and $K^*$,  indicating the rankings based on \texttt{AUC} achieved on \texttt{HIPAA},  \texttt{TRACES} and on average across the two datasets, respectively. The AUC for the ROC curve metric enables fair comparison, irrespective of the selected threshold values. 

%\input{Files/tab1-RQ1}

% The table shows that the models perform considerably poorly on the \texttt{TRACES} dataset. A plausible reason is that \texttt{TRACES} has a total of 26 regulatory codes, some of which are seemingly closely related (e.g., the regulatory code \textit{TIM}---the period for which personal data is stored is semantically close to \textit{DUR}---the duration of data processing). 
% To reduce the degree of confusion that ST models exhibit, we compute the AUC values for \texttt{TRACES} at the category level \TBD{is this what we report in the table? (yes)}. Recall from Section~\ref{tab:datasets} that the 26 regulatory codes in  \texttt{TRACES} are grouped into 10 different categories (listed in Table~\ref{tab:datasets}). Once the ST model computes the similarity values of single regulatory codes, we then assign to each category the maximum similarity values among the single regulatory codes in that category. For example, \textit{TIM} and \textit{DUR} belong to the category \textit{data retention} (\textit{RTN} in Table~\ref{tab:datasets}). If the similarity value between a given requirement $r_i$ and \textit{TIM} and \textit{DUR} is 0.3 and 0.47, respectively, then we assign the similarity value 0.47 between $r_i$ and the category \textit{data retention}.  


% The table further shows a discrepancy in the performance of the models across our datasets compared to that reported by the NLP community.  
% The best NLP model, \texttt{ST1}, does not perform well on our datasets as it is ranked 16 on \texttt{HIPAA} and 29 on \texttt{TRACES}. This indicates that well-performing models in NLP are not necessarily robust for RE-specific problems where the models are confronted with datasets spanning specific-domains and potentially different requirement types.  

% The best-performing model on \texttt{HIPAA} is \texttt{ST29} ($K^\dag=1$), with an AUC value of 0.859. The same model, \texttt{ST29}, is however ranked 10 on \texttt{TRACES} with an AUC of 0.614, 0.07 lower than the best model \texttt{ST2} ($K^\ddag=1$). However, \texttt{ST2} yields  0.13 lower AUC value on \texttt{HIPAA} when compared with \texttt{ST29}. 
% Overall, \texttt{ST29} achieves the best average AUC value of 0.736 on both datasets \texttt{HIPAA} and \texttt{TRACES} ($K^*=1$), leaving \texttt{ST2} six ranks behind. 
% Additionally, we observe that, on average, \texttt{ST3} fares fairly close to \texttt{ST29}. Still, according to the NLP leaderboard, \texttt{ST29} has the advantage of being much faster and smaller in size than \texttt{ST3}: \texttt{ST29}'s size is 970 MB, whereas \texttt{ST3}'s size is 2370 MB. 

% \begin{tcolorbox}[arc=1mm,width=\columnwidth,
%                   top=0mm,left=0mm,  right=0mm, bottom=0mm,
%                   boxrule=1pt, colback=violet!15!white,colframe=white]
% In view of the above analysis, \textbf{the answer RQ1}, we select \texttt{ST29} (corresponding to \texttt{paraphrase-multilingual-mpnet-base-v2}) as the best-performing ST model in identifying trace links using a zero-shot setting. 
% \end{tcolorbox}



% \noindent\textbf 

\begin{itemize}[leftmargin=*]
    \item \textbf{Perception of Illusion (Mandatory Question):} ``Do you notice any illusionary effect in this image?''
    
    \item \textbf{Uncertainty Clarification (Optional Question):} ``If you are uncertain, could you please explain why?''
    
    \item \textbf{Confidence Level (Mandatory Question):} ``If you answered `Yes' or `No' regarding the perception of an illusion, how confident are you in your response? Please rate on a scale from 1 (least confident) to 5 (most confident).''
    
    \item \textbf{Image Description (Mandatory Question):} ``What do you observe in this image?''
    
    \item \textbf{Description Confidence (Mandatory Question):} ``How confident are you in your description of the image? Rate from 1 (least confident) to 5 (most confident).''
\end{itemize}


\noindent\textbf{Result Analysis.} The key results from this survey are summarized in Table~\ref{tex:RQ1}, 10 participants taking part in this questionnaire. In terms of visibility, the data reveals that human users were able to accurately identify 83\% of illusionary text and 88\% of illusionary images on average. This suggests a relatively strong ability to recognize deceptive or distorted content in both formats of 
illusionary content.

Additionally, the confidence metric provides insight into the users' perception of their own performance. The majority of participants reported high levels of confidence in their selections, indicating that they believed they were making correct judgments, even when faced with illusionary or complex content. This confidence may play a crucial role in how users engage with tasks that involve visual and textual interpretation, highlighting the special structure of human vision.


\begin{table}[t!]
    \centering
    \tabcolsep=1.5pt
    \renewcommand{\arraystretch}{0.92} 
    \caption{Experimental results of RQ4}
    \begin{tabular}{c|cc|cc}
    \hline
    \textbf{Method}             & \multicolumn{2}{c|}{\textbf{Zero-Shot}}                                  & \multicolumn{2}{c}{\textbf{COT}}                               \\ \hline
    \textbf{Metric}             & \multicolumn{2}{c|}{\textbf{Success Rate}}                               & \multicolumn{2}{c}{\textbf{Success Rate}}                                \\ \hline
    \textbf{Model}              & \multicolumn{1}{c|}{\textbf{GPT4o}} & \textbf{Gemini} & \multicolumn{1}{c|}{\textbf{GPT4o}} & \textbf{Gemini} \\ \hline
    \textbf{Illu Text} & 0.00\%                                 & 0.00\%    &0.00\%                                 & 0.00\%                  \\ \hline
    \textbf{Illu Image} & 0.00\%                                 & 0.00\%    &0.00\%                                & 0.00\%                 \\ \hline
    \end{tabular}
    \label{tex:RQ2}
\end{table}
\begin{table}[!t]
\small
\caption{The sum of feature contribution scores (FCS) for the comparison schemes during training and testing in both closed and open-world settings. AG and GR represent the baselines APIGraph~\cite{apigraph} and GuideRetraining~\cite{guide_retraining}.}
\label{tab:rq3}
\centering
\begin{tblr}{
  cells = {c},
  rowsep = -1.0pt,
  cell{1}{1} = {c=9}{},
  cell{2}{1} = {r=2}{},
  cell{2}{2} = {c=4}{},
  cell{2}{6} = {c=4}{},
  vline{2-3,6} = {2}{},
  vline{2,6} = {3}{},
  vline{2,6} = {4-14}{},
  hline{1-2,4,15} = {-}{},
  hline{3} = {2-9}{},
  colsep = 2.5pt,
}
DeepDrebin\cite{Grossedeepdrebin} &                 &    &    &      &            &    &    &      \\
           & Closed world    &    &    &      & Open world &    &    &      \\
           & w/o             & AG & GR & Ours & w/o        & AG & GR & Ours \\
Train      & 27.22 & 28.33 & 25.15 & 32.15 &  27.22 & 28.33 & 25.15 & 32.15  \\
1          &  23.98 & 24.49 & 19.04 &27.98 & 22.91  & 23.62 & 18.58 &  27.33 \\
2          &  21.70 & 23.53 & 18.44 & 26.14 &  19.46 & 22.57 & 17.04 & 25.74 \\
3          &  19.78 & 20.27& 17.12 & 22.92 &  17.56 &19.29  & 16.10 & 21.97 \\
4          & 17.94 &18.04 & 15.11& 20.16& 16.68 & 16.74 & 13.72 & 19.26 \\
5          & 15.73 & 15.82 &13.52 & 18.72 & 14.69  & 15.09& 12.06 & 18.57 \\
6          & 14.26& 14.23 &12.46 & 16.63 & 13.56 & 13.75 & 11.10& 16.59 \\
7          & 13.14 & 13.21 & 11.67 & 14.89 & 12.12   & 13.19 & 11.63 & 14.75 \\
8          & 12.58 & 12.47 &10.88 &13.83 & 11.75  &11.83 & 10.05 & 13.66  \\
9          & 12.26 & 12.31 & 10.52 & 12.91& 11.24  &11.12 & 9.30 &12.38 \\
10          &  11.98  &12.01 &6.83& 12.52 & 10.51 & 10.62& 6.01& 11.62
\end{tblr}
\end{table}
\begin{table}[htb!]
\caption{IDP Comparison of CodeImprove with CodeImprove-Random}
\label{Tab:rq4}
\renewcommand{\arraystretch}{1.12}
\resizebox{\columnwidth}{!}{
\begin{tabular}{|c|c|c|c|c|c|c|c|}
\hline
 \multirow{2}{*}{\textbf{Experiment}} &\multicolumn{3}{c|}{\textbf{Vulnerability Detection}} & \multicolumn{3}{c|}{\textbf{Defect Prediction}}\\ \cline{2-7}
      & CodeBERT & RoBERTa & BERT  & CodeBERT & RoBERTa & BERT \\ \hline

    CodeImprove-Random  &4.5  & 4.18  & 3.95  & 5.04 & 8.35 & 5.45   \\\hline

     CodeImprove & 16.77  & 6.32  &8.78  & 12.04 &  10.06&11.12 \\\hline

     



    
  
  
\end{tabular}
}
\end{table}

\subsection{RQ4: LLM Deception by Illusionary Content} 

\noindent\textbf{Motivation.} In this section, we investigate whether illusionary content can effectively deceive the visual processing of LLMs, a critical requirement since a CAPTCHA image must successfully mislead AI systems.


\noindent\textbf{Method.} To rigorously test our generated illusionary content, we adopt the same settings as our empirical study in Section~\ref{sec:empirical_study}, employing 30 generated illusionary images. In contrast to our empirical study, this section aims to demonstrate that LLMs are unable to identify illusionary content. Additionally, unlike other studies, we require precise answers—for example, the correct response should be the name of a concrete bridge
rather than simply bridge.

\noindent\textbf{Result Analysis.} Table~\ref{tex:RQ2} presents the experimental results for LLMs in identifying both illusionary images and text. Our findings indicate that, under both Zero-Shot and COT reasoning settings, neither GPT nor Gemini successfully identified the illusionary images, achieving a 0\% success rate. Notably, when using COT, GPT was able to discern the shape of a hidden character within the image but failed to accurately name the character, even when provided with a hint. These results suggest that visual illusions are particularly challenging for current LLMs to identify, underscoring their effectiveness as natural CAPTCHAs.

\subsection{RQ5: Effectiveness of Inducement Prompts} 

\noindent\textbf{Motivation.} In this section, we explore whether our inducement prompts can effectively guide our intended attackers—GPT-4o and Gemini 1.5 pro 2.0—to select the options we designed.


\noindent\textbf{Method.} In this evaluation, we test GPT-4o and Gemini 1.5 Pro 2.0. We employ two prompt settings Zero-Shot and COT, to assess their performance. Additionally, we allow LLMs two attempts to identify CAPTCHAs, leveraging their ability to retain context across interactions. For this experiment, we utilize 30 IllusionaryCaptchas as the target images.

\noindent\textbf{Result Analysis.} From Table~\ref{tex:RQ3}, we can see that in both attempts, the LLMs consistently selected the option we predicted they would choose, suggesting that the models were identifying only the generated content and not focusing on what we intended human users to recognize. Additionally, we observed that the LLMs often selected the longest description of the images, indicating a tendency to overlook the core elements of the visual illusion. This behavior highlights a key limitation in the LLMs' ability to process visual context effectively, as they appear to prioritize the length or complexity of the descriptions rather than engaging with the nuanced visual details. This finding suggests that while LLMs perform well with textual analysis, they may struggle when tasked with interpreting visual content that requires deeper contextual understanding or inference, such as illusionary images.

\subsection{RQ6: Human Attempts to Pass CAPTCHA}

\textbf{Motivation.} One of the primary aims of our CAPTCHA is to facilitate easier identification of images by human users. Therefore, it is crucial to demonstrate that our CAPTCHA is more user-friendly. To achieve this, we need to assess the number of attempts required for human users to successfully pass the CAPTCHA.

\noindent\textbf{Method.} In this evaluation, we designed a questionnaire structure similar to the one used in Section~\ref{sec:empirical_study} consulting 23 participants to investigate how many attempts human users need to pass our IllusionCAPTCHA. 

\noindent\textbf{Result Analysis.} Table~\ref{tex:RQ4} presents the experimental results of our IllusionCAPTCHA for human users. In this survey, we consulted 23 participants, and we found that 86.95\% were able to pass the CAPTCHA on their first attempt, while 8.69\% succeeded on their second attempt. We also collected feedback on the reasons for failure and discovered that the primary reason participants could not pass was that they did not know the name of the character, although they recognized it as a character from television. Therefore, our CAPTCHA is more friendly for human users to identify, compared to current existing CAPTCHAs.


\section{Discussion}

\revision{In this section, we discuss the comparison to adversarial image-based techniques and address several challenges associated with real-world deployment.}

\revision{\noindent \textbf{Comparison to Adversarial Attacks.} Adversarial image-based techniques typically rely on the addition of carefully crafted noise to images. However, recent studies~\cite{wei2022towards} indicate that these methods often lack transferability and can be easily defeated by a novel LLM with enhanced visual capabilities. Our experimental results demonstrating the LLM's effectiveness in identifying adversarial images is available on our website~\cite{ourwebsite}.}

\revision{\noindent \textbf{Challenge of Cross-cultural Adaptability.} Our experiments reveal that individuals from different countries and age groups may exhibit varying abilities in identifying illusionary images due to cultural differences. To mitigate this issue, we propose incorporating common, everyday images—such as those of fruits, restaurants, and landscapes—to create illusionary images that are universally recognizable. By leveraging familiar objects, we aim to minimize the impact of cultural differences and ensure a consistent user experience across diverse demographics.}

\revision{\noindent \textbf{Challenge of Image Copyright.} In real-world deployment, copyright concerns may render certain images or terms (e.g., \textit{Mickey Mouse}) unsuitable for use. To mitigate these issues, we plan to employ a local AI system to generate images while carefully avoiding problematic words. This approach enables the creation of copyright-free images, thereby ensuring smoother and more compliant deployment in practical scenarios.}
Software development is increasingly conceived as a collaboration activity between developers and AIs. Indeed, IDEs already implement features to enable interactive development, with AI suggesting implementations that are reused by developers.

Although multiple studies show this interaction can be successful, there is still limited understanding of how the models must be configured and used in the context of code generation tasks. This study addresses this gap, systematically investigating the impact of several key parameters, including the repeated submission of a prompt to accommodate for the non-deterministic nature of the models.

Our study reveals several key findings about the usage of ChatGPT. In particular, we discovered how creativity, although up to a limited extent, is useful to increase the range of methods whose code can be generated correctly. A major role is played by parameter top-p, which is commonly underrated, and instead has a major impact on the correctness of the results, with lower values producing better results. Finally, prompts should be submitted multiple times, with $5$ repetitions combined with a temperature of $1.2$ resulting in an effective configuration in our experiments.  

Future work concerns two main research directions. One is about replicating this experiment with other AI assistants, to validate our findings in multiple contexts. The second research direction concerns finding strategies to deal with the need to submit the same prompt multiple times to obtain a useful result, and thus developing approaches able to select or merge multiple responses automatically. 

\clearpage
\vfill\eject 
\bibliographystyle{ACM-Reference-Format}
\balance
\bibliography{refs}

\end{document}

