% Please add the following required packages to your document preamble:
% \usepackage{multirow}
\begin{table}[t]
    \centering
    % \scriptsize
    % \footnotesize
    \tabcolsep=1.5pt
    \renewcommand{\arraystretch}{0.92} 
    \caption{Experimental results of applying the multi-model LLMs over the selected CAPTCHAs.}
 % Please add the following required packages to your document preamble:
% \usepackage{multirow}
% Please add the following required packages to your document preamble:
% \usepackage{multirow}
\resizebox{\linewidth}{!}{
    \begin{tabular}{cc|cc|cc}
    \hline
    \multicolumn{2}{c|}{\textbf{Method}}                                                            & \multicolumn{2}{c|}{\textbf{Zero-Shot}}                                  & \multicolumn{2}{c}{\textbf{COT}}                                         \\ \hline
    \multicolumn{2}{c|}{\textbf{Metric}}                                                            & \multicolumn{2}{c|}{\textbf{Success Rate}}                               & \multicolumn{2}{c}{\textbf{Success Rate}}                                \\ \hline
    \multicolumn{2}{c|}{\textbf{Model}}                                                             & \multicolumn{1}{c|}{\textbf{GPT4o}} & \textbf{Gemini} & \multicolumn{1}{c|}{\textbf{GPT4o}} & \textbf{Gemini} \\ \hline
    \multicolumn{1}{c|}{\multirow{4}{*}{\textbf{Text-based}}}  & \textbf{Simplest}          & 76.66\%                                    & 73.33\%                     & 90.00\%                                    & 83.33\%                     \\
    \multicolumn{1}{c|}{}                                              & \textbf{Overloaping}       & 66.66\%                                    & 60\%                        & 70.00\%                                    & 60.00\%                     \\
    \multicolumn{1}{c|}{}                                              & \textbf{Noise}             & 70.00\%                                    & 73.33\%                     & 73.33\%                                    & 66.66\%                     \\
    \multicolumn{1}{c|}{}                                              & \textbf{Noise+Overloaping} & 36.66\%                                    & 23.33\%                     & 50.00\%                                    & 43.33\%                     \\ \hline
    \multicolumn{1}{c|}{\multirow{2}{*}{\textbf{Image-based}}} & \textbf{reCAPTCHA}         & 40.00\%                                    & 33.33\%                     & 50.00\%                                    & 23.33\%                     \\
    \multicolumn{1}{c|}{}                                              & \textbf{hCAPTCHA}          & 40.00\%                                    & 36.66\%                     & 43.33\%                                    & 30.00\%                     \\ \hline
    \multicolumn{1}{c|}{\multirow{5}{*}{\textbf{Reasoning}}}   & \textbf{Angular}           & 13.33\%                                    & 0.00\%                      & 13.33\%                                    & 0.00\%                      \\
    \multicolumn{1}{c|}{}                                              & \textbf{Gobang}            & 0.00\%                                     & 0.00\%                      & 6.66\%                                     & 3.33\%                      \\
    \multicolumn{1}{c|}{}                                              & \textbf{IconCrush}         & 0.00\%                                     & 0.00\%                      & 16.66\%                                    & 10.00\%                     \\
    \multicolumn{1}{c|}{}                                              & \textbf{Space}             & 46.66\%                                    & 26.66\%                     & 53.33\%                                    & 26.66\%                     \\
    \multicolumn{1}{c|}{}                                              & \textbf{Space Reasoning}   & 33.33\%                                    & 20.00\%                     & 40.00\%                                    & 23.33\%                     \\ \hline
    \multicolumn{2}{c|}{\textbf{Average}}                                                           & \textbf{38.48\%}                           & \textbf{31.51\%}            & \textbf{46.06\%}                           & \textbf{33.63\%}            \\ \hline
    \end{tabular}
    }

\label{tab:Empirical-of-LLM}
\end{table}

\begin{table}[t]
    \centering
    % \scriptsize
    % \footnotesize
    \tabcolsep=1.5pt
    \renewcommand{\arraystretch}{0.92} 
    \caption{Experimental results of applying the multi-model LLMs over the selected CAPTCHAs.}
    \begin{tabular}{c|cccc}
    \hline
    \textbf{\# of Attempt}       & \multicolumn{1}{c|}{\textbf{1}} & \multicolumn{1}{c|}{\textbf{2}} & \multicolumn{1}{c|}{\textbf{3}} & \textbf{>3} \\ \hline
    \textbf{Text-based CAPTCHA}  & 47.82\%                                     & 39.13\%                                      & 8.69\%                                      & 4.34\%                     \\
    \textbf{Image-based CAPTCHA} & 30.43\%                                     & 56.52\%                                      & 4.34\%                                      & 8.69\%                     \\
    \textbf{Reasoning CAPTCHA}   & 21.73\%                                     & 43.47\%                                      & 21.73\%                                     & 13.04\%                    \\ \hline
    \textbf{Average}             & \textbf{33.33\%}                            & \textbf{46.37\%}                             & \textbf{11.59\%}                            & \textbf{8.69\%}            \\ \hline
    \end{tabular}
\label{tab:user-study}
\end{table}

\section{Empirical Study}
\label{sec:empirical_study}

We first conduct a systematic empirical study to assess the effectiveness of LLMs in identifying both traditional and modern CAPTCHAs. The full potential of LLMs in this area remains largely unexplored. Additionally, to address the knowledge gap among human users regarding CAPTCHAs, we designed a user study to assess user performance across various CAPTCHA challenges. This investigation is structured around two research questions:

\begin{itemize}[leftmargin=*]
    \setlength\itemsep{0pt}
    \item \textbf{RQ1 (Effectiveness):} How effective are LLMs in accurately solving CAPTCHAs, and what types of errors are they most likely to make?
    \item \textbf{RQ2 (User Study):} Are human users able to effectively resolve various categories of CAPTCHA challenges, and what specific obstacles do they encounter throughout the process?
\end{itemize}

In the following of this section, we address the two research questions through two sets of experiments. 

\subsection{Effectiveness of LLMs in Solving CAPTCHAs}



\noindent\textbf{CAPTCHA Categorization.} Different from other works~\cite{deng2024oedipus,searles2023empirical}, we cover all categories of visual CAPTCHAs. Within each category, there are different designs from different vendors. Therefore, we compiled widely deployed CAPTCHAs available online and organized them into the following detailed subcategories.

\begin{itemize}[leftmargin=*]
    \item \textbf{Text-based CAPTCHAs.} We collect different types of text-based CAPTCHAs that requires users to recognize a series of letters or characters. After survey, we conclude four types of them available now. (1) \textbf{Simplest Text-based CAPTCHAs}, shown in Figure~\ref{fig:Text-based}(a), is the simplest text-based CAPTCHAs, which is also the most popular CAPTCHAs online. This type of challenge can be solved easily by traditional CAPTCHA solvers. These typically feature clear, unaltered text, making them vulnerable to basic image recognition techniques. 
    (2) \textbf{Noisy Text-based CAPTCHAs}, shown in Figure~\ref{fig:Text-based}(b), introduce visual noise, such as random lines, dots, or distortions into the text CAPTCHA, which can interfere with traditional CAPTCHA solvers. Despite the added complexity, they still primarily ensure that users could recognize the contents within.
    (3) \textbf{Overlapping Text-based CAPTCHAs}, shown in Figure~\ref{fig:Text-based}(c), are a type of text-based CAPTCHAs that involve texts where characters are overlapped with each other at different angles. While this writing style is totally recognizable to humans, it is hard for traditional solvers~\cite{ye2018yet} that relies on segmentation strategies to solve. 
    (4) \textbf{Noise-enhanced Overlapping Text-based CAPTCHAs}, shown in Figure~\ref{fig:Text-based}(d), are the type of challenges combine both visual noise and overlapping texts, which significantly increases the difficulty for traditional CAPTCHA solvers to counter. 

    \item \textbf{Image-based CAPTCHAs.} In addition to the traditional text-based CAPTCHAs, more recent ones include images that tests the common sense of users as a type of challenge. We conclude two types of basic image-based CAPTCHAs.
    (1) \textbf{reCAPTCHA} presents users with tasks like selecting images (image classification) that contain specific objects, such as traffic lights or crosswalks, or verifying street signs. Vastly adopted by Google, it is the most common types of CAPTCHA that has been well researched. There are three versions of reCAPTCHAs, with similar image patterns but different underlying mechanisms to counter traditional automated solutions such as JavaScript reverse engineering. 
    (2) \textbf{hCAPTCHA} involves more detailed image recognition tasks, requiring users to have a stronger ability to understand the prompts (e.g., selecting images that contain wheels).
    
    \item \textbf{Reasoning-based CAPTCHAs} 
    are new emerging category of challenges that aims to counter the automated solvers powered by deep learning methods. After survey, we identify three types of reasoning-based CAPTCHAs.
    (1) \textbf{Rotation CAPTCHAs}, also known as Angular by their developers, require users to adjust an object’s orientation to align with a reference object. 
    As shown in Figure~\ref{fig:Reasoning-based}(a), users need to properly recognize the orientation of two different objects (the finger and the lamb in this example) to solve the challenge. There are two versions of Rotation CAPTCHAs available in the market now, both devleoped by Arkose Labs. 
    (2) \textbf{Bingo CAPTCHAs (Gobang \& IconCrush)} is a new type of reasoning-based CAPTCHA also developed by Arkose Labs.
    As seen in Figure~\ref{fig:Reasoning-based}(b), this type of challenge tasks users with identifying and rearranging elements on a board to create a line of matching items. The types of elements and the rules for manipulation can differ widely based on the provider. For instance, in Figure~\ref{fig:Reasoning-based}(b), users can swap any two items without restriction, while in Figure~\ref{fig:Reasoning-based}(c), swaps are limited to adjacent items, illustrating the range of variation in this type of CAPTCHA.
    (3) \textbf{3D Logical CAPTCHAs}, as demonstrated in Figure~\ref{fig:Reasoning-based}(d) and Figure~\ref{fig:Reasoning-based}(e), requires users to choose an object from a 3D environment. This process requires users to identify the logical relationships tied to attributes like shape, color, and orientation of the objects within the challenge. For instance, in Figure~\ref{fig:Reasoning-based}(d), users must identify the number 0 that aligns with the orientation of a yellow letter W, whereas Figure~\ref{fig:Reasoning-based}(e) asks users to select the larger object positioned to the left of a green object.
    
\end{itemize}





\noindent\textbf{Dataset Collection.} To rigorously assess the ability of LLMs to solve CAPTCHAs, we include the three types of CAPTCHAs as discussed in the Background: text-based, image-based and reasoning-based CAPTCHAs. In particular, we exclude audio CAPTCHAs due to their limited usage online~\cite{fanelle2020blind}, which is mainly for visually impaired people. Furthermore, our study emphasizes real-world scenarios, so all CAPTCHAs used were collected from website applications. Consequently, we built a dataset comprising three types of CAPTCHA (text-based CAPTCHAs shown in Figure~\ref{fig:Text-based}, image-based CAPTCHAs shown in Figure~\ref{fig:Image-based} and reasoning-based CAPTCHAs shown in Figure~\ref{fig:Reasoning-based} \revision{each containing 30 capthchas per subdivision.}.


\noindent \textbf{Methodology.}  To evaluate these CAPTCHAs, we employ two powerful LLMs (Gemini 1.5 pro 2.0 and GPT4-o) using both Zero-Shot and Chain-of-Thought (COT) methodologies. Each CAPTCHA category presents a unique set of challenges that require customized solution strategies. As a result, we utilize different LLM prompts to predict the outcomes of various CAPTCHAs, measuring success rates as our primary metric. We manually analyze each LLM response to ensure the accuracy of the results. In the zero-shot approach, a solution is considered correct only if the LLM outlines the exact procedure to solve the CAPTCHA. In contrast, in the CoT approach, a substep is deemed successful if the LLM's proposed solution for that specific sub-step is accurate.

\noindent\textbf{Result Analysis.} Table~\ref{tab:Empirical-of-LLM}—with GPT4o representing GPT-4o and Gemini representing Gemini 1.5 pro 2.0—presents our evaluation of LLMs' effectiveness in solving CAPTCHAs. Using a zero-shot approach, the LLMs successfully solve most text-based CAPTCHAs, except for those with overlapping characters or significant noise, while their accuracy drops to only 40\% for image-based CAPTCHAs. Moreover, LLMs face challenges with reasoning-based CAPTCHAs due to their limited reasoning capabilities; however, employing COT prompting significantly enhances their performance in identifying these types of CAPTCHAs. These findings show the growing threat that advancements in LLMs pose to web security, suggesting that current CAPTCHA methods may no longer be sufficiently secure.

\begin{center}
    \setlength{\fboxrule}{1pt}
    %\fbox
    \fcolorbox{lightgray}{mygray}{%
      \parbox{0.47\textwidth}{%
        \textbf{Answer to RQ1:}
        \emph{Our verification experiment reveals that (1) LLMs perform better on text-based CAPTCHAs compared to image-based and reasoning-based CAPTCHAs; and (2) although LLMs struggle with complex reasoning CAPTCHAs, their performance significantly improves when employing the Chain-of-Thought (CoT) strategy. This suggests that with reasoning chains, LLMs have the potential to overcome these challenges. Consequently, this indicates that current CAPTCHAs may no longer be as secure as intended.}
        }%
    }
\end{center}


\begin{figure*}[!t]
	\centering
    \includegraphics[width=0.9\linewidth]{Figure/Overview_of_IllusionCAPTCHA.eps}
	\caption{Overview of IllusionCAPTCHA}
     % \vspace{-10pt}
	\label{fig:Illusion-based}
\end{figure*}


\subsection{User Study}

To investigate user behavior, we designed a questionnaire-based study. Because some CAPTCHAs cannot be reliably retrieved from their original sources, we constructed the study by extracting CAPTCHA images directly from their native web applications and manually annotating the correct responses. All images were drawn from the dataset we mentioned before.

\noindent\textbf{User Study Settings.} Our questionnaire allocates each participant 1 minute to solve a CAPTCHA. If they cannot complete it within that time, they must attempt it again until they succeed. During this process, we record both successful and failed attempts. Finally, we have 23 human participants in our study.

\noindent\textbf{Result Analysis.} Table~\ref{tab:user-study} presents the results of our user study, revealing that most participants were unable to solve the CAPTCHA on their first attempt; notably, both image-based and reasoning-based CAPTCHAs proved particularly challenging, with some individuals requiring more than three attempts to successfully pass them.


\begin{center}
    \setlength{\fboxrule}{1pt}
    %\fbox
    \fcolorbox{lightgray}{mygray}{%
      \parbox{0.47\textwidth}{%
        \textbf{Answer to RQ2:}
        \emph{The result of our user study reveals that (1) While reasoning-based CAPTCHAs pose significant challenges for AI systems, they are also difficult for human users. Hence, these CAPTCHAs can easily frustrate users, leading to diminished patience during their attempts. (2) Human users frequently make the same mistakes as LLMs, highlighting the need to develop methods that can effectively distinguish between LLMs and human users.}
        }%
    }
\end{center}










