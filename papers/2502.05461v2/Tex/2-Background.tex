\section{Background}

\subsection{CAPTCHAs and CAPTCHA Solver}

CAPTCHAs~\cite{deng2024oedipus,gao2021research} have evolved from simple text recognition to complex reasoning challenges to distinguish between human users and bots. This ongoing development mirrors the ``cat-and-mouse'' dynamic in cybersecurity, where both CAPTCHAs and CAPTCHA solvers become increasingly innovative in response to one another. This transformation has accelerated the shift from traditional CAPTCHA-solving (e.g. OCR\cite{ye2018yet}) methods to modern AI technology, posing a significant threat to the effectiveness of CAPTCHAs.



\noindent\textbf{Text-based CAPTCHAs.}
Text-based CAPTCHAs are the earliest form of CAPTCHA, designed to leverage text recognition tasks that are easy for humans but challenging for machines. As shown in Figure~\ref{fig:Text-based}(a), the simplest text-based CAPTCHAs consist of a string of English characters with no added noise. However, as machine learning techniques have advanced, text-based CAPTCHAs have become increasingly complex, incorporating more than just English characters and moving beyond simple backgrounds~\cite{searles2023empirical}. However, the complexity of CAPTCHA also makes human users hard to identify.

\begin{figure}[t!]
	\centering         \includegraphics[width=0.6\linewidth]{Figure/Text_based_CAPTCHA.eps}
	\caption{Text-based CAPTCHA}
     % \vspace{-10pt}
	\label{fig:Text-based}
\end{figure}

\noindent\textbf{Image-based CAPTCHAs.} 
Image-based CAPTCHAs are the most popular CAPTCHAs used online. Compared to text-based CAPTCHAs, image-based CAPTCHAs requires more vision capture ability, with more abundant image categories in image content. Based on the particular workloads embeded in the image-based CAPTCHAs, we categorize them into two groups. 

\begin{figure}[t!]
	\centering
    \includegraphics[width=\linewidth]{Figure/Image-based_CAPTCHA.eps}
	\caption{Image-based CAPTCHA}
     % \vspace{-10pt}
	\label{fig:Image-based}
\end{figure}

\begin{itemize}[leftmargin=*]

\item \textbf{Object Classification. }
This group (e.g., reCAPTCHA~\cite{reCaptcha}, as shown in Figure~\ref{fig:Image-based}(a)) typically presents a set of images and asks users to identify specific ones from various given categories. Early image-based CAPTCHAs relying on object recognition used relatively simple images. However, to combat increasingly sophisticated automated bots, modern image-based CAPTCHAs now incorporate noise and other distortions into the images, making it more challenging for AI systems to accurately recognize the objects. This added complexity aims to disrupt the efficiency of automated classification while still allowing human users to complete the task with ease.
    
\item \textbf{Object Recognition.} Compared to object classification, object recognition demands a deeper level of visual understanding. For instance, hCAPTCHA~\cite{hCaptcha} requires users to click on the correct images based on a given description, as shown in Figure~\ref{fig:Image-based}(b). This task involves not only identifying objects but also understanding the context of the question and selecting images that match the description. Unlike simple object classification, which may only involve labeling objects in an image, object recognition in CAPTCHAs requires users to interpret complex scenarios or differentiate between visually similar objects.
    
\end{itemize}
\noindent\textbf{Reasoning-based CAPTCHAs.}
The evolution of reasoning-based CAPTCHAs~\cite{wang2018captcha} signifies a shift from traditional visual recognition tasks to cognitive challenges that demand more advanced logical reasoning and image comprehension. As shown in Figure~\ref{fig:Reasoning-based}, reasoning-based CAPTCHAs usually need human users to click move some icons to pass the check. This development highlights the limitations of conventional CAPTCHA solvers(e.g. OCR) in handling these more complex tasks. However, reasoning-based CAPTCHAs also require users to engage in higher-level reasoning, which can lead to increased frustration and impatience among human users.

\begin{figure}[t!]
	\centering
    \includegraphics[width=0.9\linewidth]{Figure/reasoning-based.drawio.eps}
	\caption{Reasoning-based CAPTCHA}
     % \vspace{-10pt}
	\label{fig:Reasoning-based}
\end{figure}

\subsection{Large Language Models}
The evolution of Large Language Models (LLMs) has transformed traditional AI learning method~\cite{achiam2023gpt}. By increasing the scale of training data, model can significantly improve their ability to understand, generate, and process human language with greater accuracy and contextual relevance. Notably, recent advancements in multimodal LLMs~\cite{GPT4-o,bai2023qwen} have facilitated the integration of text and images, enabling AI systems to analyze complex visuals and describe them using natural language. While the reasoning capabilities of LLMs are still being evaluated, their potential to address reasoning-based tasks is both promising and continuously expanding~\cite{sun2024determlr}. Consequently, the capabilities demonstrated by LLMs pose a substantial threat to the security of traditional CAPTCHA systems~\cite{deng2024oedipus}.

\subsection{Visual Illusion}
Visual illusions~\cite{gurnsey1992parallel,von1989mechanisms,krizhevsky2009learning} illustrate the complexities of human visual reasoning, demonstrating that our brain interprets the world in ways far more intricate than what we directly perceive. These illusions provide valuable insights into how cognitive processes, shaped by perception and context, influence our understanding of reality. While existing research shows that modern LLMs can identify objects similarly to humans, their imaginative capabilities remain limited~\cite{chakrabarty2024art}, making it difficult for them to match human-level reasoning. 









