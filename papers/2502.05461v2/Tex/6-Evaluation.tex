\section{Evaluation}
To evaluate the performance of our IllusionCAPTCHA, we have structured our evaluation around four research questions:

\begin{itemize}[noitemsep,leftmargin=*] 
\item \textbf{RQ3: Human Identification of Illusionary Images.} Can the illusionary images generated by IllusionCAPTCHA remain identifiable to human users? 

\item \textbf{RQ4: LLM Deception by Illusionary Content.} Can the illusionary content effectively deceive LLMs into selecting a false answer? 

\item \textbf{RQ5: Inducement Prompts Effectiveness.} Can the CAPTCHA structure we designed compel bots to make targeted choices? 

\item \textbf{RQ6: Human Attempts to Pass CAPTCHA.} How many attempts do human users require to successfully pass our designed CAPTCHA? 
\end{itemize}






\subsection{RQ3: Human Identification of Illusionary Images}

\noindent\textbf{Motivation.} In this section, we examine whether illusionary images can effectively convey information to human users, a critical factor since a CAPTCHA image must clearly communicate its intended message to its target audience.

\noindent\textbf{Method.} To address RQ3, we designed a questionnaire to assess human users' ability to identify illusionary images. The questionnaire comprises two types of images—text-based and image-based illusionary images—each consisting of five samples. All ten samples were generated using the method described in Section~\ref{sec:method}, and to avoid copyright issues, the base images were produced using a diffusion technique. Below, we provide the details of our questionnaire.


\subsection{Pre-trained Model Selection (RQ1)} \label{subsec:rq1}

\sectopic{Methodology. }  We shortlist the ST models for investigation in our work based on the NLP  leaderboard, which reports the 38 most accurate pre-trained models\footnote{\url{https://www.sbert.net/docs/pretrained_models.html}}. These models have been extensively evaluated for their ability to generate sentence embeddings (i.e., capturing the semantics of the whole text) and their performance in semantic search (i.e., finding relevant answers to a given query). Both tasks closely align with our objectives. 
To identify trace links, we apply the pre-trained models in a zero-shot setting as follows. 
We let each model compute the similarity matrix equivalent to the output of step~5 in our approach (see Fig.~\ref{fig:approach}). 
We then predict a trace link if the similarity value exceeds 
a predefined threshold. Since zero-shot does not require training, we run EXPI on the entire \texttt{HIPAA} dataset. 


\sectopic{Evaluation Metrics. } To better assess the performance irrespective of the selected threshold, we compute the \textit{Area Under the Curve (AUC)} for the receiver operating characteristic (ROC) across different threshold values,  ranging from $0.1$ to $0.9$. 
The ROC curve captures the trade-off between the true positive rate (TPR) and the false positive rate (FPR). TPR is the proportion of positives correctly identified as such (i.e., the percentage of trace links correctly identified for a given threshold). FPR is the proportion of negatives incorrectly identified as positives (i.e., the percentage of trace links wrongly identified as not trace links). The AUC of the ROC curve (computed as micro-average over all the provisions to avoid the dominance of some provisions)  provides a single aggregate performance measure across all possible thresholds and, hence, is a suitable evaluation metric to compare the ST models.  We posit that the model with the highest AUC value demonstrates the best overall performance in identifying trace links in a zero-shot setting, as a higher AUC value indicates a better balance between correctly identifying true trace links (high TPR) and minimizing the identification of false links (low FPR). 
%
%
\sectopic{Results. }
Table~\ref{tab:rq1} presents the \texttt{AUC} values of the ST pre-trained models on the \texttt{HIPAA} dataset and also  reports $K$, indicating the ranking of the models in the NLP community based on their accuracy~\cite{Reimers:19}, as well as $K^\dag$, indicating the ranking based on \texttt{AUC} achieved on \texttt{HIPAA}. 



\begin{table}
%\footnotesize
\centering
\caption{AUC of ST models for LRT on \texttt{HIPAA} (\textbf{RQ1}). 
% \TBD{@Romina: come on! you don't leave footnotes on TRACES in the table. Please revise your changes. Also, you don't need "HIPAA" in the header if the results are only for HIPAA, @Romina: please remove and adjust the header accordingly}
}
\label{tab:rq1}
% \begin{threeparttable}[t]
\begin{tabularx}{\textwidth}{@{} p{0.05\textwidth} @{\hskip 0.5em} p{0.05\textwidth} @{\hskip 3em} p{0.05\textwidth} @{\hskip 20em} *{5}{>{\centering\arraybackslash}X}@{}}
    \toprule
    \multirow{1}{*}{$K$\tnote{1}} & \multirow{1}{*}{Model\tnote{2}} & \multirow{1}{*}{Name\tnote{1}} & \multirow{1}{*}{\texttt{AUC}\tnote{1}} & \multirow{1}{*}{$K^\dag$\tnote{1}} \\%\multicolumn{2}{c}{\texttt{HIPAA}} \\ %& \multicolumn{2}{c}{\texttt{TRACES}} & \multicolumn{2}{c}{Average}\\
    % \cmidrule(lr){4-5}
    % &&& \texttt{AUC} & $K^\dag$ \\ %&\texttt{AUC} & $K^\ddag$ &\texttt{AUC} & $K^*$  \\
    \midrule
1 &   \texttt{ST1}  & \texttt{all-mpnet-base-v2} & 0.744 & 16 \\ % & 0.331 & 29 & 0.538 & 27\\
2 &   \texttt{ST2}  & \texttt{gtr-t5-xxl} & 0.725 & 21 \\ % & \textbf{0.685} & 1 & 0.705 & 7\\
3 &   \texttt{ST3}  &\texttt{gtr-t5-xl} & 0.789 & 6 \\ % & 0.678 & 2 & 0.733 & 2\\
4 &   \texttt{ST4}  &\texttt{sentence-t5-xxl} & 0.720 & 22 \\ % & 0.666 & 3 & 0.693 & 8\\
5 &   \texttt{ST5}  &\texttt{gtr-t5-large} & 0.743 & 17 \\ % & 0.640 & 7 & 0.692 & 9\\
6 &   \texttt{ST6}  &\texttt{all-mpnet-base-v1} & 0.712 & 25 \\ % & 0.338 & 27 & 0.525 & 29\\
7 &   \texttt{ST7}  &\texttt{multi-qa-mpnet-base-dot-v1} & 0.688 & 27 \\ % & 0.631 & 8 & 0.659 & 12\\
8 &   \texttt{ST8}  &\texttt{multi-qa-mpnet-base-cos-v1} & 0.603 & 34 \\ % & 0.222 & 36 & 0.413 & 36\\
9 &   \texttt{ST9}  &\texttt{all-roberta-large-v1} & 0.601 & 35 \\ % & 0.333 & 28 & 0.467 & 34\\
10 &   \texttt{ST10}  &\texttt{sentence-t5-xl} & 0.769 & 10 \\ % & 0.644 & 6 & 0.706 & 5\\
11 &   \texttt{ST11}  &\texttt{all-distilroberta-v1} & 0.719 & 23 \\ % & 0.284 & 34 & 0.501 & 32\\
12 &   \texttt{ST12}  &\texttt{all-MiniLM-L12-v1} & 0.729 & 19 \\ % & 0.318 & 30 & 0.523 & 30\\
13 &   \texttt{ST13}  &\texttt{all-MiniLM-L12-v2} & 0.747 & 15 \\ % & 0.339 & 26 & 0.543 & 26\\
14 &   \texttt{ST14}  &\texttt{multi-qa-distilbert-dot-v1} & 0.563 & 36 \\ % & 0.546 & 17 & 0.555 & 25\\
15 &   \texttt{ST15}  &\texttt{multi-qa-distilbert-cos-v1} & 0.640 & 33 \\ % & 0.228 & 35 & 0.434 & 35\\
16 &   \texttt{ST16}  &\texttt{gtr-t5-base} & 0.770 & 9 \\ % & 0.655 & 5 & 0.712 & 4\\
17 &   \texttt{ST17}  &\texttt{sentence-t5-large} & 0.748 & 14 \\ % & 0.663 & 4 & 0.706 & 6\\
18 &   \texttt{ST18}  &\texttt{all-MiniLM-L6-v2} & 0.761 & 11 \\ % & 0.285 & 33 & 0.523 & 31\\
19 &   \texttt{ST19}  &\texttt{multi-qa-MiniLM-L6-cos-v1} & 0.670 & 29 \\ % & 0.313 & 31 & 0.492 & 33\\
20 &   \texttt{ST20}  &\texttt{all-MiniLM-L6-v1} & 0.749 & 13 \\ % & 0.307 & 32 & 0.528 & 28\\
21 &   \texttt{ST21}  &\texttt{paraphrase-mpnet-base-v2} & 0.850 & 2 \\ % & 0.587 & 14 & 0.719 & 3\\
22 &   \texttt{ST22}  &\texttt{msmarco-bert-base-dot-v5} & 0.644 & 32 \\ % & 0.503 & 20 & 0.574 & 24\\
23 &   \texttt{ST23}  & \texttt{multi-qa-MiniLM-L6-dot-v1} & 0.715 & 24 \\ % & 0.605 & 12 & 0.660 & 11\\
24 &   \texttt{ST24}  & \texttt{sentence-t5-base} & 0.726 & 20 \\ % & 0.584 & 15 & 0.655 & 13\\
25 &   \texttt{ST25}  & \texttt{msmarco-distilbert-base-tas-b} & 0.701 & 26 \\ % & 0.557 & 16 & 0.629 & 18\\
26 &   \texttt{ST26}  & \texttt{msmarco-distilbert-dot-v5} & 0.685 & 28 \\ % & 0.600 & 13 & 0.643 & 15\\
27 &   \texttt{ST27}  & \texttt{paraphrase-distilroberta-base-v2} & 0.801 & 4 \\ % & 0.455 & 24 & 0.628 & 19\\
28 &   \texttt{ST28}  & \texttt{paraphrase-MiniLM-L12-v2} & 0.794 & 5 \\ % & 0.496 & 22 & 0.645 & 14\\
29 &   \texttt{ST29}  & \texttt{paraphrase-multilingual-mpnet-base-v2} & \textbf{0.859} & 1 \\ % & 0.614 & 10 & \textbf{0.736} & 1\\
30 &   \texttt{ST30}  & \texttt{paraphrase-TinyBERT-L6-v2} & 0.787 & 7 \\ % & 0.464 & 23 & 0.625 & 21\\
31 &   \texttt{ST31}  & \texttt{paraphrase-MiniLM-L6-v2} & 0.770 & 8 \\ % & 0.511 & 18 & 0.641 & 16\\
32 &   \texttt{ST32}  & \texttt{paraphrase-albert-small-v2} & 0.737 & 18 \\ % & 0.499 & 21 & 0.618 & 22\\
33 &   \texttt{ST33}  & \texttt{paraphrase-multilingual-MiniLM-L12-v2} & 0.811 & 3 \\ % & 0.511 & 19 & 0.661 & 10\\
34 &   \texttt{ST34}  & \texttt{paraphrase-MiniLM-L3-v2} & 0.757 & 12 \\ % & 0.441 & 25 & 0.599 & 23\\
35 &   \texttt{ST35}  & \texttt{distiluse-base-multilingual-cased-v1} & 0.349 & 37 \\ % & 0.092 & 37 & 0.220 & 37\\
36 &   \texttt{ST36}  & \texttt{distiluse-base-multilingual-cased-v2} & 0.341 & 38 \\ % & 0.090 & 38 & 0.216 & 38\\
37 &   \texttt{ST37}  & \texttt{average\_word\_embeddings\_komninos} & 0.647 & 31 \\ % & 0.606 & 11 & 0.627 & 20\\
38 &   \texttt{ST38}  & \texttt{average\_word\_embeddings\_glove.6B.300d} & 0.636 & 30 \\ % & 0.625 & 9 & 0.630 & 17\\ 
\bottomrule
\end{tabularx}
\begin{tablenotes}
     \item[1] $K$: The average performance ranking of the models, as reported in the NLP community. $K^\dag$: The ranking of the models based on AUC values computed on \texttt{HIPAA} ($K=1$ indicates the highest AUC). 
      \item [2] \texttt{ST1}--\texttt{ST38} correspond to the models reported at this link (sorted by average accuracy in descending order):     \url{https://www.sbert.net/docs/pretrained_models.html}. %, where \texttt{ST29} is \texttt{paraphrase-multilingual-mpnet-base-v2}.
     \end{tablenotes}
 % \end{threeparttable}
 %\vspace*{-2em}
 \end{table}

 

The best-performing model on \texttt{HIPAA} is \texttt{ST29} ($K^\dag=1$), with an AUC value of 0.859. The next best performing model is \texttt{ST21} with an AUC value of 0.850. The difference between these two AUC values is only marginal. A possible explanation is that  \texttt{ST29} uses  \texttt{ST21} as its base model.  \texttt{ST29}  has been, however, trained on more (multi-lingual) data.   

Additionally, we observe a discrepancy in the performance of the models on the \texttt{HIPAA} dataset compared to that reported by the NLP community.  
The best NLP model, \texttt{ST1}, does not perform well  on \texttt{HIPAA}, ranked 16. 
This observation indicates that well-performing models in NLP are not necessarily as effective for RE-specific problems. 
%The datasets in RE are typically domain-specific increasing the level of complexity to deal with.    

\begin{tcolorbox}[arc=1mm,width=\columnwidth,
                  top=0mm,left=0mm,  right=0mm, bottom=0mm,
                  boxrule=1pt, colback=violet!15!white,colframe=white]
\textbf{The answer RQ1} is that \texttt{ST29} is the best-performing pre-trained model for LRT (corresponding to \texttt{paraphrase-multilingual-mpnet-base-v2}). 
\end{tcolorbox}%The goal of RQ1 is to select a robust ST model that performs consistently well across datasets. 
% Table~\ref{tab:rq1} shows the \texttt{AUC} values of the ST pre-trained models on the \texttt{HIPAA} and \texttt{TRACES} datasets. The table also reports $K$ indicating the ranking of the models in the NLP community based on their accuracy~\cite{Reimers:19}, as well as $K^\dag$,  $K^\ddag$ and $K^*$,  indicating the rankings based on \texttt{AUC} achieved on \texttt{HIPAA},  \texttt{TRACES} and on average across the two datasets, respectively. The AUC for the ROC curve metric enables fair comparison, irrespective of the selected threshold values. 

%\input{Files/tab1-RQ1}

% The table shows that the models perform considerably poorly on the \texttt{TRACES} dataset. A plausible reason is that \texttt{TRACES} has a total of 26 regulatory codes, some of which are seemingly closely related (e.g., the regulatory code \textit{TIM}---the period for which personal data is stored is semantically close to \textit{DUR}---the duration of data processing). 
% To reduce the degree of confusion that ST models exhibit, we compute the AUC values for \texttt{TRACES} at the category level \TBD{is this what we report in the table? (yes)}. Recall from Section~\ref{tab:datasets} that the 26 regulatory codes in  \texttt{TRACES} are grouped into 10 different categories (listed in Table~\ref{tab:datasets}). Once the ST model computes the similarity values of single regulatory codes, we then assign to each category the maximum similarity values among the single regulatory codes in that category. For example, \textit{TIM} and \textit{DUR} belong to the category \textit{data retention} (\textit{RTN} in Table~\ref{tab:datasets}). If the similarity value between a given requirement $r_i$ and \textit{TIM} and \textit{DUR} is 0.3 and 0.47, respectively, then we assign the similarity value 0.47 between $r_i$ and the category \textit{data retention}.  


% The table further shows a discrepancy in the performance of the models across our datasets compared to that reported by the NLP community.  
% The best NLP model, \texttt{ST1}, does not perform well on our datasets as it is ranked 16 on \texttt{HIPAA} and 29 on \texttt{TRACES}. This indicates that well-performing models in NLP are not necessarily robust for RE-specific problems where the models are confronted with datasets spanning specific-domains and potentially different requirement types.  

% The best-performing model on \texttt{HIPAA} is \texttt{ST29} ($K^\dag=1$), with an AUC value of 0.859. The same model, \texttt{ST29}, is however ranked 10 on \texttt{TRACES} with an AUC of 0.614, 0.07 lower than the best model \texttt{ST2} ($K^\ddag=1$). However, \texttt{ST2} yields  0.13 lower AUC value on \texttt{HIPAA} when compared with \texttt{ST29}. 
% Overall, \texttt{ST29} achieves the best average AUC value of 0.736 on both datasets \texttt{HIPAA} and \texttt{TRACES} ($K^*=1$), leaving \texttt{ST2} six ranks behind. 
% Additionally, we observe that, on average, \texttt{ST3} fares fairly close to \texttt{ST29}. Still, according to the NLP leaderboard, \texttt{ST29} has the advantage of being much faster and smaller in size than \texttt{ST3}: \texttt{ST29}'s size is 970 MB, whereas \texttt{ST3}'s size is 2370 MB. 

% \begin{tcolorbox}[arc=1mm,width=\columnwidth,
%                   top=0mm,left=0mm,  right=0mm, bottom=0mm,
%                   boxrule=1pt, colback=violet!15!white,colframe=white]
% In view of the above analysis, \textbf{the answer RQ1}, we select \texttt{ST29} (corresponding to \texttt{paraphrase-multilingual-mpnet-base-v2}) as the best-performing ST model in identifying trace links using a zero-shot setting. 
% \end{tcolorbox}



% \noindent\textbf 

\begin{itemize}[leftmargin=*]
    \item \textbf{Perception of Illusion (Mandatory Question):} ``Do you notice any illusionary effect in this image?''
    
    \item \textbf{Uncertainty Clarification (Optional Question):} ``If you are uncertain, could you please explain why?''
    
    \item \textbf{Confidence Level (Mandatory Question):} ``If you answered `Yes' or `No' regarding the perception of an illusion, how confident are you in your response? Please rate on a scale from 1 (least confident) to 5 (most confident).''
    
    \item \textbf{Image Description (Mandatory Question):} ``What do you observe in this image?''
    
    \item \textbf{Description Confidence (Mandatory Question):} ``How confident are you in your description of the image? Rate from 1 (least confident) to 5 (most confident).''
\end{itemize}


\noindent\textbf{Result Analysis.} The key results from this survey are summarized in Table~\ref{tex:RQ1}, 10 participants taking part in this questionnaire. In terms of visibility, the data reveals that human users were able to accurately identify 83\% of illusionary text and 88\% of illusionary images on average. This suggests a relatively strong ability to recognize deceptive or distorted content in both formats of 
illusionary content.

Additionally, the confidence metric provides insight into the users' perception of their own performance. The majority of participants reported high levels of confidence in their selections, indicating that they believed they were making correct judgments, even when faced with illusionary or complex content. This confidence may play a crucial role in how users engage with tasks that involve visual and textual interpretation, highlighting the special structure of human vision.


\begin{table}[t!]
    \centering
    \tabcolsep=1.5pt
    \renewcommand{\arraystretch}{0.92} 
    \caption{Experimental results of RQ4}
    \begin{tabular}{c|cc|cc}
    \hline
    \textbf{Method}             & \multicolumn{2}{c|}{\textbf{Zero-Shot}}                                  & \multicolumn{2}{c}{\textbf{COT}}                               \\ \hline
    \textbf{Metric}             & \multicolumn{2}{c|}{\textbf{Success Rate}}                               & \multicolumn{2}{c}{\textbf{Success Rate}}                                \\ \hline
    \textbf{Model}              & \multicolumn{1}{c|}{\textbf{GPT4o}} & \textbf{Gemini} & \multicolumn{1}{c|}{\textbf{GPT4o}} & \textbf{Gemini} \\ \hline
    \textbf{Illu Text} & 0.00\%                                 & 0.00\%    &0.00\%                                 & 0.00\%                  \\ \hline
    \textbf{Illu Image} & 0.00\%                                 & 0.00\%    &0.00\%                                & 0.00\%                 \\ \hline
    \end{tabular}
    \label{tex:RQ2}
\end{table}
\begin{table}[!t]
\small
\caption{The sum of feature contribution scores (FCS) for the comparison schemes during training and testing in both closed and open-world settings. AG and GR represent the baselines APIGraph~\cite{apigraph} and GuideRetraining~\cite{guide_retraining}.}
\label{tab:rq3}
\centering
\begin{tblr}{
  cells = {c},
  rowsep = -1.0pt,
  cell{1}{1} = {c=9}{},
  cell{2}{1} = {r=2}{},
  cell{2}{2} = {c=4}{},
  cell{2}{6} = {c=4}{},
  vline{2-3,6} = {2}{},
  vline{2,6} = {3}{},
  vline{2,6} = {4-14}{},
  hline{1-2,4,15} = {-}{},
  hline{3} = {2-9}{},
  colsep = 2.5pt,
}
DeepDrebin\cite{Grossedeepdrebin} &                 &    &    &      &            &    &    &      \\
           & Closed world    &    &    &      & Open world &    &    &      \\
           & w/o             & AG & GR & Ours & w/o        & AG & GR & Ours \\
Train      & 27.22 & 28.33 & 25.15 & 32.15 &  27.22 & 28.33 & 25.15 & 32.15  \\
1          &  23.98 & 24.49 & 19.04 &27.98 & 22.91  & 23.62 & 18.58 &  27.33 \\
2          &  21.70 & 23.53 & 18.44 & 26.14 &  19.46 & 22.57 & 17.04 & 25.74 \\
3          &  19.78 & 20.27& 17.12 & 22.92 &  17.56 &19.29  & 16.10 & 21.97 \\
4          & 17.94 &18.04 & 15.11& 20.16& 16.68 & 16.74 & 13.72 & 19.26 \\
5          & 15.73 & 15.82 &13.52 & 18.72 & 14.69  & 15.09& 12.06 & 18.57 \\
6          & 14.26& 14.23 &12.46 & 16.63 & 13.56 & 13.75 & 11.10& 16.59 \\
7          & 13.14 & 13.21 & 11.67 & 14.89 & 12.12   & 13.19 & 11.63 & 14.75 \\
8          & 12.58 & 12.47 &10.88 &13.83 & 11.75  &11.83 & 10.05 & 13.66  \\
9          & 12.26 & 12.31 & 10.52 & 12.91& 11.24  &11.12 & 9.30 &12.38 \\
10          &  11.98  &12.01 &6.83& 12.52 & 10.51 & 10.62& 6.01& 11.62
\end{tblr}
\end{table}
\begin{table}[htb!]
\caption{IDP Comparison of CodeImprove with CodeImprove-Random}
\label{Tab:rq4}
\renewcommand{\arraystretch}{1.12}
\resizebox{\columnwidth}{!}{
\begin{tabular}{|c|c|c|c|c|c|c|c|}
\hline
 \multirow{2}{*}{\textbf{Experiment}} &\multicolumn{3}{c|}{\textbf{Vulnerability Detection}} & \multicolumn{3}{c|}{\textbf{Defect Prediction}}\\ \cline{2-7}
      & CodeBERT & RoBERTa & BERT  & CodeBERT & RoBERTa & BERT \\ \hline

    CodeImprove-Random  &4.5  & 4.18  & 3.95  & 5.04 & 8.35 & 5.45   \\\hline

     CodeImprove & 16.77  & 6.32  &8.78  & 12.04 &  10.06&11.12 \\\hline

     



    
  
  
\end{tabular}
}
\end{table}

\subsection{RQ4: LLM Deception by Illusionary Content} 

\noindent\textbf{Motivation.} In this section, we investigate whether illusionary content can effectively deceive the visual processing of LLMs, a critical requirement since a CAPTCHA image must successfully mislead AI systems.


\noindent\textbf{Method.} To rigorously test our generated illusionary content, we adopt the same settings as our empirical study in Section~\ref{sec:empirical_study}, employing 30 generated illusionary images. In contrast to our empirical study, this section aims to demonstrate that LLMs are unable to identify illusionary content. Additionally, unlike other studies, we require precise answers—for example, the correct response should be the name of a concrete bridge
rather than simply bridge.

\noindent\textbf{Result Analysis.} Table~\ref{tex:RQ2} presents the experimental results for LLMs in identifying both illusionary images and text. Our findings indicate that, under both Zero-Shot and COT reasoning settings, neither GPT nor Gemini successfully identified the illusionary images, achieving a 0\% success rate. Notably, when using COT, GPT was able to discern the shape of a hidden character within the image but failed to accurately name the character, even when provided with a hint. These results suggest that visual illusions are particularly challenging for current LLMs to identify, underscoring their effectiveness as natural CAPTCHAs.

\subsection{RQ5: Effectiveness of Inducement Prompts} 

\noindent\textbf{Motivation.} In this section, we explore whether our inducement prompts can effectively guide our intended attackers—GPT-4o and Gemini 1.5 pro 2.0—to select the options we designed.


\noindent\textbf{Method.} In this evaluation, we test GPT-4o and Gemini 1.5 Pro 2.0. We employ two prompt settings Zero-Shot and COT, to assess their performance. Additionally, we allow LLMs two attempts to identify CAPTCHAs, leveraging their ability to retain context across interactions. For this experiment, we utilize 30 IllusionaryCaptchas as the target images.

\noindent\textbf{Result Analysis.} From Table~\ref{tex:RQ3}, we can see that in both attempts, the LLMs consistently selected the option we predicted they would choose, suggesting that the models were identifying only the generated content and not focusing on what we intended human users to recognize. Additionally, we observed that the LLMs often selected the longest description of the images, indicating a tendency to overlook the core elements of the visual illusion. This behavior highlights a key limitation in the LLMs' ability to process visual context effectively, as they appear to prioritize the length or complexity of the descriptions rather than engaging with the nuanced visual details. This finding suggests that while LLMs perform well with textual analysis, they may struggle when tasked with interpreting visual content that requires deeper contextual understanding or inference, such as illusionary images.

\subsection{RQ6: Human Attempts to Pass CAPTCHA}

\textbf{Motivation.} One of the primary aims of our CAPTCHA is to facilitate easier identification of images by human users. Therefore, it is crucial to demonstrate that our CAPTCHA is more user-friendly. To achieve this, we need to assess the number of attempts required for human users to successfully pass the CAPTCHA.

\noindent\textbf{Method.} In this evaluation, we designed a questionnaire structure similar to the one used in Section~\ref{sec:empirical_study} consulting 23 participants to investigate how many attempts human users need to pass our IllusionCAPTCHA. 

\noindent\textbf{Result Analysis.} Table~\ref{tex:RQ4} presents the experimental results of our IllusionCAPTCHA for human users. In this survey, we consulted 23 participants, and we found that 86.95\% were able to pass the CAPTCHA on their first attempt, while 8.69\% succeeded on their second attempt. We also collected feedback on the reasons for failure and discovered that the primary reason participants could not pass was that they did not know the name of the character, although they recognized it as a character from television. Therefore, our CAPTCHA is more friendly for human users to identify, compared to current existing CAPTCHAs.

