\section{Limitations}


\paragraph{Probing Tasks} While the probing tasks we have proposed provide valuable insights into the visual arithmetic capabilities of VLMs, it is important to acknowledge that they may not encompass all possible dimensions of visual reasoning. Our choice to limit the scope of these tasks was intentional, as they serve as initial, simple tests to determine whether VLMs exhibit failure in fundamental aspects of visual arithmetic. These tasks allow us to iterate different experiments in a controlled and efficient manner, providing clear, actionable insights without the complexity that more comprehensive tasks might introduce. However, there is potential to explore additional tasks that involve more complex interactions of basic geometric properties. For instance, tasks requiring the model to simultaneously assess both length and angle, or combinations of length and area, could be valuable for understanding the compositionality of these atomic tasks. \looseness=-1

\paragraph{Training Data Synthesis}The training data synthesis method of \method~ is not only scalable but also effectively enhances the visual arithmetic capabilities of VLMs. Our approach serves as a proof-of-concept, demonstrating the potential of automated data generation for improving models' understanding of basic geometric properties. To further enrich the training data, we could consider utilizing additional configurations for each task. For instance, in generating positive and negative responses, we could leverage LLMs to produce rationales based on the specific configuration of each figure. By including explanations or justifications for why a particular geometric property holds or does not hold, we could foster deeper understanding within the VLMs. \looseness=-1

