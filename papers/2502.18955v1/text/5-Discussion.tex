\section{Conclusion}
In this work, we demonstrate a critical problem in offline RL -- identifying the reduced dataset to improve offline algorithm performance with low computational complexity.
We cast the issue as the gradient approximation problem.
By transforming the common actor-critic framework into the submodular objective, we apply the orthogonal matching pursuit method to construct the reduced dataset.
Further, we propose multiple key modifications to stabilize the learning process.
We validate the effectiveness of our proposed data selection method through theoretical analysis and extensive experiments.
For future work, we attempt to apply our method to robot tasks in the real world.

\section*{Acknowledgments}
This work was supported by Strategic Priority Research Program of the Chinese Academy of Sciences under Grant No.XDA27040200, in part by the National Key R\&D Program of China under Grant No.2022ZD0116405.

% The data subset selecting method might have broad applications in RL.
% For instance, in continual learning~\cite{parisi2019continual}, we can select critical subsets across different tasks, thereby alleviating forgetting of past knowledge when learning new tasks with a large volume of new data.
