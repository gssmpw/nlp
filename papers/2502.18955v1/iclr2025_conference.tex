
\documentclass{article} % For LaTeX2e
\usepackage{iclr2025_conference,times}

% Optional math commands from https://github.com/goodfeli/dlbook_notation.
%%%%% NEW MATH DEFINITIONS %%%%%

% \usepackage{amsmath,amsfonts,bm}
\usepackage{amsmath,amsfonts}

\usepackage{pifont}


\newcommand{\R}{\mathbb{R}}


\def\va{{\mathbf{a}}}
\def\vg{{\mathbf{g}}}

% Sets
\def\sR{\mathbb{R}}
\def\sC{\mathbb{C}}
\def\sZ{\mathbb{Z}}
\def\sN{\mathbb{N}}
\def\sQ{\mathbb{Q}}

\def\sS{\mathcal{S}}



% Vectors
\def\vzero{{\mathbf{0}}}
\def\vone{{\mathbf{1}}}
\def\vmu{{\mathbf{\mu}}}
\def\vtheta{{\mathbf{\theta}}}
\def\va{{\mathbf{a}}}
\def\vb{{\mathbf{b}}}
\def\vc{{\mathbf{c}}}
\def\vd{{\mathbf{d}}}
\def\ve{{\mathbf{e}}}
\def\vf{{\mathbf{f}}}
\def\vg{{\mathbf{g}}}
\def\vh{{\mathbf{h}}}
\def\vi{{\mathbf{i}}}
\def\vj{{\mathbf{j}}}
\def\vk{{\mathbf{k}}}
\def\vl{{\mathbf{l}}}
\def\vm{{\mathbf{m}}}
\def\vn{{\mathbf{n}}}
\def\vo{{\mathbf{o}}}
\def\vp{{\mathbf{p}}}
\def\vq{{\mathbf{q}}}
\def\vr{{\mathbf{r}}}
\def\vs{{\mathbf{s}}}
\def\vt{{\mathbf{t}}}
\def\vu{{\mathbf{u}}}
\def\vv{{\mathbf{v}}}
\def\vw{{\mathbf{w}}}
\def\vx{{\mathbf{x}}}
\def\vy{{\mathbf{y}}}
\def\vz{{\mathbf{z}}}
\def\vzeta{{\mathbf{\zeta}}}

% Matrix
\def\mA{{\mathbf{A}}}
\def\mB{{\mathbf{B}}}
\def\mC{{\mathbf{C}}}
\def\mD{{\mathbf{D}}}
\def\mE{{\mathbf{E}}}
\def\mF{{\mathbf{F}}}
\def\mG{{\mathbf{G}}}
\def\mH{{\mathbf{H}}}
\def\mI{{\mathbf{I}}}
\def\mJ{{\mathbf{J}}}
\def\mK{{\mathbf{K}}}
\def\mL{{\mathbf{L}}}
\def\mM{{\mathbf{M}}}
\def\mN{{\mathbf{N}}}
\def\mO{{\mathbf{O}}}
\def\mP{{\mathbf{P}}}
\def\mQ{{\mathbf{Q}}}
\def\mR{{\mathbf{R}}}
\def\mS{{\mathbf{S}}}
\def\mT{{\mathbf{T}}}
\def\mU{{\mathbf{U}}}
\def\mV{{\mathbf{V}}}
\def\mW{{\mathbf{W}}}
\def\mX{{\mathbf{X}}}
\def\mY{{\mathbf{Y}}}
\def\mZ{{\mathbf{Z}}}
\def\mBeta{{\mathbf{\beta}}}
\def\mPhi{{\mathbf{\Phi}}}
\def\mLambda{{\mathbf{\Lambda}}}
\def\mSigma{{\mathbf{\Sigma}}}


% Expectation
% \def\eE{\mathop{\mathbb{E}}\limits}
\def\eE{\mathbb{E}}

% Probability
\def\pP{\mathbb{P}}

% Tilde
\def\tf{\tilde{f}}
\def\tS{\tilde{S}}
\def\wtF{\widetilde{\mathcal{F}}}
\def\whR{\widehat{R}}
\def\tvx{\tilde{\mathbf{x}}}
\def\ty{\tilde{y}}


\def\defeq{\overset{\textup{def}}{=}}
% \def\defeq{\overset{.}{=}}
\def\defone{\overset{\text{\ding{172}}}{=}}
\def\deftwo{\overset{\text{\ding{173}}}{=}}
\def\leqone{\overset{\text{\ding{172}}}{\leq}}
\def\leqtwo{\overset{\text{\ding{173}}}{\leq}}
\def\leqthree{\overset{\text{\ding{174}}}{\leq}}
\def\leqfour{\overset{\text{\ding{175}}}{\leq}}
\def\eqone{\overset{\text{\ding{172}}}{=}}
\def\eqtwo{\overset{\text{\ding{173}}}{=}}
\def\eqthree{\overset{\text{\ding{174}}}{=}}
\def\eqfour{\overset{\text{\ding{175}}}{=}}
\def\geqfive{\overset{\text{\ding{176}}}{\geq}}

\usepackage{hyperref}
\usepackage{url}

\usepackage{amsmath}
\usepackage{amssymb}
\usepackage{mathtools}
\usepackage{amsthm}
\usepackage{xcolor}
\usepackage{microtype}
\usepackage{graphicx}
\usepackage{subfigure}
\usepackage{booktabs} % for professional tables


\usepackage{algorithm}
\usepackage{algorithmic}

\usepackage[utf8]{inputenc} % allow utf-8 input
\usepackage[T1]{fontenc}    % use 8-bit T1 fonts
\usepackage{hyperref}       % hyperlinks
\usepackage{url}            % simple URL typesetting
\usepackage{booktabs}       % professional-quality tables
\usepackage{amsfonts}       % blackboard math symbols
\usepackage{nicefrac}       % compact symbols for 1/2, etc.
\usepackage{microtype}      % microtypography
\usepackage{xcolor}         % colors

% \theoremstyle{plain}
\newtheorem{theorem}{Theorem}[section]
\newtheorem{proposition}[theorem]{Proposition}
\newtheorem{lemma}[theorem]{Lemma}
\newtheorem{corollary}[theorem]{Corollary}
\newtheorem{definition}[theorem]{Definition}
\newtheorem{assumption}[theorem]{Assumption}
\newtheorem{remark}[theorem]{Remark}

\usepackage[textsize=tiny]{todonotes}

\newcount\Comments  % 0 suppresses notes to selves in text
\Comments = 1
\newcommand{\kibitz}[2]{\ifnum\Comments=1{\color{#1}{#2}}\fi}
\newcommand{\tw}[1]{\kibitz{red}{#1}}

\usepackage{xcolor}
\newcommand\TODO[1]{\textcolor{red}{[TODO: #1]}}
\newcommand\CHANGE[1]{\textcolor{blue}{#1}}
%%%%% NEW MATH DEFINITIONS %%%%%

% \usepackage{amsmath,amsfonts,bm}
\usepackage{amsmath,amsfonts}

\usepackage{pifont}


\newcommand{\R}{\mathbb{R}}


\def\va{{\mathbf{a}}}
\def\vg{{\mathbf{g}}}

% Sets
\def\sR{\mathbb{R}}
\def\sC{\mathbb{C}}
\def\sZ{\mathbb{Z}}
\def\sN{\mathbb{N}}
\def\sQ{\mathbb{Q}}

\def\sS{\mathcal{S}}



% Vectors
\def\vzero{{\mathbf{0}}}
\def\vone{{\mathbf{1}}}
\def\vmu{{\mathbf{\mu}}}
\def\vtheta{{\mathbf{\theta}}}
\def\va{{\mathbf{a}}}
\def\vb{{\mathbf{b}}}
\def\vc{{\mathbf{c}}}
\def\vd{{\mathbf{d}}}
\def\ve{{\mathbf{e}}}
\def\vf{{\mathbf{f}}}
\def\vg{{\mathbf{g}}}
\def\vh{{\mathbf{h}}}
\def\vi{{\mathbf{i}}}
\def\vj{{\mathbf{j}}}
\def\vk{{\mathbf{k}}}
\def\vl{{\mathbf{l}}}
\def\vm{{\mathbf{m}}}
\def\vn{{\mathbf{n}}}
\def\vo{{\mathbf{o}}}
\def\vp{{\mathbf{p}}}
\def\vq{{\mathbf{q}}}
\def\vr{{\mathbf{r}}}
\def\vs{{\mathbf{s}}}
\def\vt{{\mathbf{t}}}
\def\vu{{\mathbf{u}}}
\def\vv{{\mathbf{v}}}
\def\vw{{\mathbf{w}}}
\def\vx{{\mathbf{x}}}
\def\vy{{\mathbf{y}}}
\def\vz{{\mathbf{z}}}
\def\vzeta{{\mathbf{\zeta}}}

% Matrix
\def\mA{{\mathbf{A}}}
\def\mB{{\mathbf{B}}}
\def\mC{{\mathbf{C}}}
\def\mD{{\mathbf{D}}}
\def\mE{{\mathbf{E}}}
\def\mF{{\mathbf{F}}}
\def\mG{{\mathbf{G}}}
\def\mH{{\mathbf{H}}}
\def\mI{{\mathbf{I}}}
\def\mJ{{\mathbf{J}}}
\def\mK{{\mathbf{K}}}
\def\mL{{\mathbf{L}}}
\def\mM{{\mathbf{M}}}
\def\mN{{\mathbf{N}}}
\def\mO{{\mathbf{O}}}
\def\mP{{\mathbf{P}}}
\def\mQ{{\mathbf{Q}}}
\def\mR{{\mathbf{R}}}
\def\mS{{\mathbf{S}}}
\def\mT{{\mathbf{T}}}
\def\mU{{\mathbf{U}}}
\def\mV{{\mathbf{V}}}
\def\mW{{\mathbf{W}}}
\def\mX{{\mathbf{X}}}
\def\mY{{\mathbf{Y}}}
\def\mZ{{\mathbf{Z}}}
\def\mBeta{{\mathbf{\beta}}}
\def\mPhi{{\mathbf{\Phi}}}
\def\mLambda{{\mathbf{\Lambda}}}
\def\mSigma{{\mathbf{\Sigma}}}


% Expectation
% \def\eE{\mathop{\mathbb{E}}\limits}
\def\eE{\mathbb{E}}

% Probability
\def\pP{\mathbb{P}}

% Tilde
\def\tf{\tilde{f}}
\def\tS{\tilde{S}}
\def\wtF{\widetilde{\mathcal{F}}}
\def\whR{\widehat{R}}
\def\tvx{\tilde{\mathbf{x}}}
\def\ty{\tilde{y}}


\def\defeq{\overset{\textup{def}}{=}}
% \def\defeq{\overset{.}{=}}
\def\defone{\overset{\text{\ding{172}}}{=}}
\def\deftwo{\overset{\text{\ding{173}}}{=}}
\def\leqone{\overset{\text{\ding{172}}}{\leq}}
\def\leqtwo{\overset{\text{\ding{173}}}{\leq}}
\def\leqthree{\overset{\text{\ding{174}}}{\leq}}
\def\leqfour{\overset{\text{\ding{175}}}{\leq}}
\def\eqone{\overset{\text{\ding{172}}}{=}}
\def\eqtwo{\overset{\text{\ding{173}}}{=}}
\def\eqthree{\overset{\text{\ding{174}}}{=}}
\def\eqfour{\overset{\text{\ding{175}}}{=}}
\def\geqfive{\overset{\text{\ding{176}}}{\geq}}
\newcommand{\shortn}{\textup{\texttt{-}}}
\newcommand{\shorte}{\textup{\texttt{=}}}
\newcommand{\shortp}{\textup{\texttt{+}}}
\newcommand{\shortl}{\textup{\texttt{<}}}
\newcommand{\shortg}{\textup{\texttt{>}}}
\newcommand{\ie}{\textit{i}.\textit{e}.}
\newcommand{\eg}{\textit{e}.\textit{g}.}
\newcommand{\etal}{\textit{et al}.}
\newcommand{\etc}{\textit{etc}.}
\newcommand{\Tau}{\mathrm{T}}
\newcommand{\name}{\textsc{ReDOR}}
\newcommand{\rdcshort}{\mathtt{rdc}}
\newcommand{\namep}{$\mathtt{Prioritized}$}
\newcommand{\nameo}{$\mathtt{Complete\ Dataset}$}
\newcommand{\nameh}{$\mathtt{Top}\ x\%\ \mathtt{BC}$ }
\newcommand{\namer}{$\mathtt{Random}$}
\newcommand{\namec}{$\mathtt{No\ Cluster}$}
\newcommand{\namei}{$\mathtt{Single\ Round}$}
\newcommand{\nameq}{$\mathtt{Q\ Target}$}
\newcommand{\tact}{Transform2Act}

\newcommand{\originalant}{Handcrafted Robot}
\newcommand{\locomotionft}{Locomotion on Flat Terrain}
\newcommand{\locomotionvt}{Locomotion on Variable Terrain}
\newcommand{\escape}{Escape Bowl}
\newcommand{\pointnav}{Point Navigation}
\newcommand{\manipulationbox}{Manipulate Box}
\newcommand{\patrol}{Patrol}

\usepackage{thm-restate}

\title{Fewer May Be Better: Enhancing Offline Reinforcement Learning with Reduced Dataset}

% \title{ReDOR: Reduced Dataset for Offline Reinforcement Learning}

% Authors must not appear in the submitted version. They should be hidden
% as long as the \iclrfinalcopy macro remains commented out below.
% Non-anonymous submissions will be rejected without review.

\author{
	Yiqin Yang$^1$, Quanwei Wang$^2$, Chenghao Li$^2$, Hao Hu$^2$, Chengjie Wu$^2$, Yuhua Jiang$^2$, \\ 
	\textbf{Dianyu Zhong$^2$, Ziyou Zhang$^2$, Qianchuan Zhao$^2$, Chongjie Zhang$^3$, Bo Xu$^1$\footnotemark[2]} \\
    $^1$The Key Laboratory of Cognition and Decision Intelligence for Complex Systems, \\ 
    \ \ Institute of Automation, Chinese Academy of Sciences \\
    $^2$Tsinghua University \\ 
    $^3$Washington University in St. Louis \\
    \texttt{yiqin.yang@ia.ac.cn}
}

% The \author macro works with any number of authors. There are two commands
% used to separate the names and addresses of multiple authors: \And and \AND.
%
% Using \And between authors leaves it to \LaTeX{} to determine where to break
% the lines. Using \AND forces a linebreak at that point. So, if \LaTeX{}
% puts 3 of 4 authors names on the first line, and the last on the second
% line, try using \AND instead of \And before the third author name.

\newcommand{\fix}{\marginpar{FIX}}
\newcommand{\new}{\marginpar{NEW}}

\iclrfinalcopy % Uncomment for camera-ready version, but NOT for submission.
\begin{document}


\maketitle

\renewcommand{\thefootnote}{\fnsymbol{footnote}}
\footnotetext[2]{Corresponding Author}

\begin{abstract}
% Research in offline reinforcement learning (RL) marks a paradigm shift in RL.
% However, a critical yet under-investigated aspect of offline RL is determining the subset of the offline dataset, which is used to improve algorithm performance while accelerating algorithm training. Moreover, the size of reduced datasets can uncover the requisite offline data volume essential for addressing analogous challenges.
% Based on the above considerations, we propose identifying Reduced Datasets for Offline RL (\name) by formulating it as a gradient approximation optimization problem. 
% We prove that the common actor-critic framework in reinforcement learning can be transformed into a submodular objective.
% This insight enables us to construct a subset by adopting the orthogonal matching pursuit (OMP).
% Specifically, we have made several critical modifications to OMP to enable successful adaptation with Offline RL algorithms.
% The experimental results indicate that the data subsets constructed by the ReDOR can significantly improve algorithm performance with low computational complexity.
Offline reinforcement learning (RL) represents a significant shift in RL research, allowing agents to learn from pre-collected datasets without further interaction with the environment. A key, yet underexplored, challenge in offline RL is selecting an optimal subset of the offline dataset that enhances both algorithm performance and training efficiency. Reducing dataset size can also reveal the minimal data requirements necessary for solving similar problems.
In response to this challenge, we introduce ReDOR (Reduced Datasets for Offline RL), a method that frames dataset selection as a gradient approximation optimization problem. We demonstrate that the widely used actor-critic framework in RL can be reformulated as a submodular optimization objective, enabling efficient subset selection. To achieve this, we adapt orthogonal matching pursuit (OMP), incorporating several novel modifications tailored for offline RL.
Our experimental results show that the data subsets identified by ReDOR not only boost algorithm performance but also do so with significantly lower computational complexity.
\end{abstract}

\section{Introduction}

Chain-of-Thought (CoT) prompting~\cite{Nye:2021, cot, Kojima:2022cotzero} has emerged as a cornerstone strategy for enhancing Large Language Models (LLMs) in complex reasoning tasks. By eliciting step-by-step inference, CoT enables LLMs to decompose intricate problems into manageable subtasks, thereby improving their problem-solving performance~\cite{Yao:2023tot, Wang:2023self-consistency, Zhou:2023least, Shinn:2023Reflexion}. Recent advancements, such as OpenAI's o1~\cite{o1} and DeepSeek-R1~\cite{deepseekr1}, further demonstrate that scaling up CoT lengths from hundreds to thousands of reasoning steps could continuously improve LLM reasoning. These breakthroughs have underscored CoT’s potential to advance LLM capabilities, expanding the boundaries of AI-driven problem-solving.

\begin{figure}[t]
\centering
    \includegraphics[width=0.95\columnwidth]{fig/intro.pdf}
    \caption{In contrast to vanilla CoT that generates all reasoning tokens sequentially, \method enables LLMs to \textit{skip} tokens with less semantic importance (\textit{e.g.,} \includegraphics[width=7pt]{fig/token.pdf}~) and learn shortcuts between critical reasoning tokens, facilitating controllable CoT compression.}
    \label{fig:intro}
\end{figure}

Despite its effectiveness, the increased length of CoT sequences introduces substantial computational overhead. Due to the autoregressive nature of LLM decoding, longer CoT outputs lead to proportional increases in both inference latency and memory footprints of key-value cache. Additionally, the quadratic computational cost of attention layers further exacerbates this burden. These issues become particularly pronounced when CoT sequences extend into thousands of reasoning steps, resulting in significant computational costs and prolonged response times. While prior research has explored methods for selectively skipping reasoning steps~\cite{Ding:2024cotshortcut, liu2024skipstep}, recent findings~\cite{jin:2024cotlength, Merrill:2024cotlength} suggest that such reductions may conflict with test-time scaling~\cite{o1-blog, snell2025scaling}, ultimately impairing LLM reasoning performance. Therefore, striking an optimal balance between CoT efficiency and reasoning accuracy remains a critical open challenge.

In this work, we delve into CoT efficiency and seek the answer to an important question: \textit{``Does every token in the CoT output contribute equally to deriving the answer?''} We empirically analyze the semantic importance of tokens within CoT outputs and reveal that their contributions to the reasoning performance vary, as depicted in Figure 2. Building on this insight, we introduce \method, a simple yet effective approach that enables LLMs to \textit{skip} less important tokens within CoT sequences and learn shortcuts between critical reasoning tokens, thereby allowing for controllable CoT compression with adjustable ratios. Specifically, as shown in Figure~\ref{fig:intro}, \method constructs compressed CoT training data with various compression ratios, by pruning unimportance tokens from original LLM CoT trajectories. Then, it conducts a general supervised fine-tuning process on target LLMs with this training data, facilitating LLMs to automatically trim redundant tokens during reasoning.

We conduct extensive experiments across various models, including LLaMA-3.1-8B-Instruct and the Qwen2.5-Instruct series, using two widely recognized math reasoning benchmarks: GSM8K and MATH-500. The results validate the effectiveness of \method in compressing CoT outputs while maintaining robust reasoning performance. Notably, Qwen2.5-14B-Instruct exhibits almost \textbf{NO} performance drop (less than $0.4\%$) with a $\bm{40\%}$ reduction in token usage on GSM8K. On the challenging MATH-500 dataset, LLaMA-3.1-8B-Instruct effectively reduces CoT token usage by $\bm{30}\%$ with a performance decline of less than $4\%$, resulting in a $\bm{1.4}\times$ inference speedup. Further analysis underscores the coherence of \method in specified compression ratios and its potential scalability with stronger compression techniques.

\method is distinguished by its low training cost. For Qwen2.5-14B-Instruct, \method fine-tunes only 0.2\% of the model's parameters using LoRA. The size of the compressed CoT training data is no larger than that of the original training set, with 7,473 examples in GSM8K and 7,500 in MATH. The training is completed in approximately 2 hours for the 7B model and 2.5 hours for the 14B model on two 3090 GPUs. These characteristics make \method an efficient and reproducible approach, suitable for use in efficient and cost-effective LLM deployment.

To sum up, our key contributions are:
\begin{enumerate}
    \item To the best of our knowledge, this work is the \textit{first} to investigate the potential of enhancing CoT efficiency through \textit{token skipping}, inspired by the varying semantic importance of tokens in CoT trajectories of LLMs.
    \item We introduce \method, a simple yet effective approach that enables LLMs to skip redundant tokens within CoTs and learn shortcuts between critical tokens, facilitating CoT compression with adjustable ratios.
    \item Our experiments validate the effectiveness of \method. When applied to Qwen2.5-14B-Instruct, \method reduces reasoning tokens by $40\%$ (from 313 to 181) on GSM8K, with less than a $0.4\%$ performance drop.
\end{enumerate}


\section{Preliminary}
We briefly review related research on flow matching and LMs, providing foundations for introducing our method.

\subsection{Rectified Flow}\label{sec:rw_rf}

Rectified flow~\cite{liu2022flow,albergo2023building} emerges as a robust and powerful generative model and has recently served as the basis for popular commercial tools like Stable Diffusion 3~\cite{stabilityAI2023}. It is based on flow matching~\cite{chen2018neural,lipman2022flow}, which models the generative process as an Ordinary Differential Equation (ODE).  Formally, a \emph{continuous normalizing flow} transports an input $\vz_0\in \mathbb{R}^d$ to $\vz_t=\phi(t, \vz_0)$ at time $t\in[0,1]$ via the ODE:
\begin{align}
    \frac{d}{dt}\phi(t, \vz_0) = \varphi\left(t, \phi(t, \vz_0)\right).
\end{align}
Here, $\phi:[0,1]\times \mathbb{R}^d\rightarrow\mathbb{R}^d$ is the \emph{flow}, and the \emph{vector field} $\varphi: [0,1]\times \mathbb{R}^d\rightarrow \mathbb{R}^d$ specifies the change rate of the state $\vz_t$. \citet{chen2018neural} suggests representing the vector field $\varphi$ with a neural network.

The flow $\phi$ transforms an initial random variable $Z_0\sim p_0(\vz_0)$ to $Z_1\sim p_1(\vz_1)$ at final time 1. Rectified flow tries to drive the flow to follow the linear path in the direction $(Z_1-Z_0)$ as much as possible:
\begin{align}
    \min_\varphi \int_0^1 \mathbb{E}\left[\|(Z_1-Z_0)-\varphi(t, Z_t)\|^2\right]dt,\label{equ:rf}
\end{align}
where $Z_t=t\cdot Z_1 + (1-t)\cdot Z_0$ is the linear interpolation of $Z_0$ and $Z_1$. Typically, the vector field network $\varphi$ is implemented as a U-Net~\citep{ronneberger2015unet} for image inputs or an MLP for vector inputs~\cite{wang2024diffusion}.

\subsection{Transformer} 
The Transformer architecture~\citep{vaswani2017attention} is foundational to recent progress in large language models (LLMs)~\citep{liu2024deepseek, zeng2024skywork,yang2024qwen2,team2023gemini}. For an input sequence of tokens $x = (\idx[\vx][][1], \dots, \idx[\vx][][N])$, let \( \idx[E][][n][] = [e(\idx[\vx][][1]), \dots, e(\idx[\vx][][n])] \) denote the sequence of token embeddings up to position $n$, where $e(\cdot)$ is the token embedding function. A standard LLM generates its output by
\begin{align}
&\idx[\mH][][n]=\textsc{Transformer}\left(\idx[E][][n]\right),\nonumber \\
&M\left(\idx[\vy][][n+1] \mid \idx[\vx][][\leq n]\right)=\mW \idx[\vh][][n],\label{equ:token_logits}
\end{align}
where $\idx[\mH][][n] \in \mathbb{R}^{n \times d}$ is the last hidden state for the first $n$ tokens, with $d$ representing the hidden dimension. $\idx[\vh][][n]$ is the last hidden state at position $n$, i.e., $\idx[\vh][][n] = \idx[\mH][][n][] [n, :]$. $\mW$ is the output projection matrix, $M$ is the model's generation logits, and $\vy$ is the output token.

We consider an LLM with $L$ layers, the hidden state after $l$ layers, $\idx[\mH][][n,l]$, is projected by three weight matrices \(\mW_Q \), \(\mW_K \), and \(\mW_V \) to the query, key, and value embeddings $\idx[\mQ][][n,l]$, $\idx[\mK][][n,l]$, and $\idx[\mV][][n,l]$, respectively. The self-attention is calculated as:
\begin{align}
&(\idx[\mQ][][n,l], \idx[\mK][][n,l], \idx[\mV][][n,l]) = \idx[\mH][][n,l] (\mW_Q, \mW_K, \mW_V) \nonumber\\
&\idx[\mA][][n,l] = \frac{\idx[\mQ][][n,l] {\idx[\mK][][n,l]}^\top}{\sqrt{d_K}}, \text{Attn}(\idx[\mH][][n,l]) = \sigma(\idx[\mA][][n,l]) \idx[\mV][][n,l],    \nonumber
\end{align}
where $\sigma(\cdot)$ is SoftMax, and $\mA$ is the self-attention matrix. We omit the multi-head attention for simplicity.


\subsection{RLHF}
Reinforcement learning from human feedback~($\mathtt{RLHF}$)~\citep{bai2022training, wang2023helpsteer, ouyang2022training, dong2024rlhf} is critical to aligning LLM behavior with human preferences such as helpfulness, harmlessness, and honesty~\citep{ganguli2022red, achiam2023gpt, team2023gemini}. An RL-based method trains a reward model~\citep{liu2024skywork} to approximate human preferences. Given a preference dataset $\mathcal{D} = (x, y_w, y_l)$, where $x$ is the input, $y_w$ is the preferred output, and $y_l$ is the less preferred output, a reward model $r_\theta$ can be trained using the standard Bradley-Terry model~\citep{bradley1952rank} with a pairwise ranking loss. With $r_\theta$, the policy model (LLM) is trained via $\mathtt{PPO}$~\citep{schulman2017proximal}.
% with the following objective:
% \begin{equation}
%     \max_{\pi_\theta} \mathbb{E}_{x\sim\mathcal{D},y \sim \pi_\theta(\cdot|x)} \left[r_{\theta}(x, y)\right] - \beta \mathbb{D}_\text{KL}\left[\pi_\theta(\cdot|x) || \pi_{\text{ref}}(\cdot|x)\right].\nonumber
% \end{equation}
% Here, the KL divergence penalty is applied to prevent excessive deviation from the reference model $\pi_{\text{ref}}$, \ie, the initial supervised fine-tuned (SFT) model.
However, training a reward model can be costly. Direct preference learning ($\mathtt{DPO}$) \cite{rafailov2024direct} enables direct training with preference data, which can be adapted to different human utility models ($\mathtt{KTO}$,~\citet{ethayarajh2024kto}).
% ~(\eg, $\mathtt{KTO}$~\cite{ethayarajh2024kto})
%shows that it is possible to directly train with the preference data

% DPO optimizes the following loss function to train the policy model (LLM) $\pi_\theta$:
% \begin{align}
%     \mathcal{L}_{\text{DPO}}(\theta) = \mathbb{E}_{\mathcal{D}}
%     \left[\shortn\log \sigma\left(\beta \log \frac{\pi_\theta(y_w|x)}{\pi_{\text{ref}}(y_w|x)} \shortn \beta \log \frac{\pi_\theta(y_l|x)}{\pi_{\text{ref}}(y_l|x)}\right)\right],\nonumber
% \end{align}
% where $\beta$ is a parameter controlling the deviation from the supervised fine-tuned model $\pi_{\text{ref}}$.




% At the ODE time $t$, the hidden state after the $m$-th layer corresponding to the input $(t, \vz_t)$ is denoted by $\idx[\vh][f][t,m](\vz_t)$.
% \begin{equation}
%     \mathcal{L}_{\text{reward}}(\theta) = -\log \left(\sigma\left(r_\theta\left(x, y_w\right) - r_\theta\left(x, y_l\right)\right)\right),
% \end{equation}
%where $\sigma$ is the logistic function.
% to guide LLMs toward desired behaviors
% leverage recent advancements in reinforcement learning (e.g., PPO~\citep{schulman2017proximal}) to enhance the alignment of LLMs~\citep{achiam2023gpt}. 
% A key component of these methods is the development of , which learn a reward function based on 
% Instead, DPO derives a reward signal directly from the currently optimized model and an initial supervised fine-tuned model \citep{rafailov2024r}, effectively reparameterizing preference learning within the model itself.
\section{\name: Modeling Task-driven Eye Movement on Charts}
\label{sec:model}

This section introduces the problem formulation and presents the computational model of eye movement control on charts in settings of analytical tasks.

\subsection{Problem Formulation}

Given a chart image $C$ and an associated analytical task $x$ stated as text, the model is expected to generate a sequence of fixation positions $\{ p_1, p_2, \dots , p_t\}$.
The objective of the output sequence is to closely match the scanpath from humans reading the chart. 
Specifically, the sequence of fixations represents the visual reasoning process, and the information in the patches of pixels fixated upon should be able to support $x$.
We consider general analytical tasks in information visualization~\cite{amar2005low}, and
select three of them used in a human eye-tracking data collection~\cite{polatsek2018exploring}: % \textit{RV}, \textit{F}, and \textit{FE} tasks
\rv{
\begin{itemize}
    \item[1)] \textit{Retrieve value (\textit{RV})}: Given a specific target, find the data value of the target (e.g., what is the value for a certain category?)
    \item[2)] \textit{Filter (\textit{F})}: Given a concrete condition, find which data point satisfies it (e.g., which category has the specific value stated?)
    \item[3)] \textit{Find extreme (\textit{FE})}: Find the data point showing an extreme value for a given attribute within the set of data (e.g., which category shows the highest/lowest value?)
\end{itemize}
}

\subsection{Modeling Overview}

Our goal was to develop the model \name to handle tasks articulated as free-form text and be able to perform gaze movement at a detailed pixel level.
We conceptualize the design of the hierarchical gaze control model in Figure~\ref{fig:model}, where the high-level (cognitive) controller is responsible for reasoning while the low-level (oculomotor) controller determines details of gaze movement. 
The idea behind this is hierarchical supervisory control~\cite{eppe2022intelligent}, which refers to a tiered control system in which the superior controller set goals for its subordinates. The actions from subordinates are integrated into an overall pattern for high-level control~\cite{pew1966acquisition}.
The concept also follows the modeling principle of computational rationality, where we assume that the controllers optimize their policy to maximize expected utility within relevant cognitive bounds~\cite{oulasvirta2022computational,chandramouli2024workflow}.
Specifically, the high-level controller handles abstract information processing, comprehension, and memory storage. 
It sets subtasks to the low-level controller, which then moves the gaze to gather information for task completion. Subsequently, the high-level controller utilizes the amassed information to answer the question.

\begin{figure*}[!t]
\centering
  \includegraphics[width=\textwidth]{Images/h-gaze-control.png}
  \caption{\textbf{An overview of the hierarchical eye-movement control architecture.} When presented with a chart and a task, a cognitive controller, powered by large language models, makes decisions on what to look at next and judges whether it is confident enough to provide an answer to the task's question. It relies on internal memory, which summarizes the information gathered from the chart through eye movements. Once cognitive control has determined the next action, the oculomotor controller is responsible for moving the gaze and observing the chart through a limited vision field. The model's objective is to accurately address the task as quickly as possible within set cognitive and physical constraints.}
  \Description{An overview of the hierarchical eye movement control architecture.}
  \label{fig:model}
\end{figure*}

\subsection{Cognitive Control}

The high-level controller provides cognitive control over the mental processes for a chart, control that performs reasoning in working memory~\cite{liu2010mental}. When performing vision tasks, one observes and analyzes visual information interactively~\cite{chen2020air}. Throughout this process, people analyze the information in their memory and try to gather more useful information to reduce uncertainty in solving the task.
To represent this decision problem accurately, we formulate it as a bounded optimality problem in a partially observable Markov decision process (POMDP). Instead of having access to a full state ($\mathcal{S}$) with pixels of the chart associated with the given task, the POMDP expresses a subset of ($\mathcal{S}$) as the observation of the model:
\begin{itemize}
    \item Observation $O$ refers to the information in memory that is captured from eye movements over the chart.
    \item Action $A$ includes subtasks that the model gives to oculomotor control for performing eye movements.
    \item Reward $R$ is the correctness of the answer for the task from the chart question answering.
\end{itemize}
To solve this POMDP, our model uses LLMs for the policy. The rationale behind this choice is that LLMs are well suited to processing higher-level information, as they have been pre-trained on human text data encompassing a wealth of logic related to planning, reasoning, and interaction~\cite{huang2022language, vemprala2024chatgpt, li2023interactive}.
Although LLMs are limited in their ability to control low-level motor functions in a precise manner~\cite{dalal2024psl}, they are proficient at planning and reasoning, with LLaMA~\cite{touvron2023llama} and GPT~\cite{achiam2023gpt} showing impressive language interpretation and reasoning capabilities.
Also, recent work has shown that utilizing LLMs in the high-level controllers in hierarchical architecture can produce promising results~\cite{huang2022language, brohan2023can, liang2023code}.
For our setting, we used GPT-4o~\cite{achiam2023gpt} for the policy, which takes the information accumulated in the memory as the observation and sets subtasks to guide eye movements in order to obtain information needed for solving the task efficiently.

We consider two human limitations when constructing the model's observation: a limited field of vision~\cite{duchowski2018gaze} and memory capacity~\cite{loftus2019human}. 
The model gets information from the gaze position purely by mimicking the human vision system. 
An optical character recognition technique~\cite{singh2010optical} is used to extract text from the pixels of the chart, and the text in the gaze area, with the position, is passed to the memory.
As a result, the observation consists of image patches (in a limited number) from the full set of chart pixels. The reliability of items in memory is determined by their visit history~\cite{li2023modeling}, with overall memory capacity being restricted too. When new information is added to the memory, a previously added item is removed on the basis of a forgetting probability. The probability of forgetting an item in the memory is calculated by means of the formula $\text{Softmax}(\rho \cdot (t-t_i)) $, where $t$ is the current fixation index, $t_i$ is the index of the $i$th item in the memory, and $\rho$ is the weight parameter (set to 0.1 here). The observation is designed as a prompt that summarizes the memory in line with the memory model and explains the model's goal. 

Given the summary of the memory information, the LLM policy selects predefined operations for task solving~\cite{brohan2023can, liang2023code}. The operations here are based on a sequence of cognitive stages for charts~\cite{goldberg2011eye} -- 1) \textit{search for text label}: visually searching for a text label or value label related to the task, 2) \textit{find associated mark}: visually searching for a graphical mark of the data point when given a reference label, 3) \textit{read associated value}: visually searching to read the given mark's associated value or textual label.
All these actions are allowed to be reused in the process, which enables the model to revisit previous positions for confirmation of the information.
Ultimately, if the information in the memory is sufficient to address the task, the gaze movement can stop and an answer can be given. Operations other than answering the question will be performed by the oculomotor controller for detailed gaze movement. 

The examples in Figure~\ref{fig:memory} demonstrate how utilizing memory information and predefined operations aids in scanpath prediction. Model memory uses the summarization capability of LLMs to convert the text and positions gathered to a paragraph as the observation (as shown in the green boxes). The LLM policy then makes decisions and issues subtasks as actions (in red boxes) for the oculomotor control, which performs pixel-level gaze movements.

\begin{figure*}[!t]
\centering
  \includegraphics[width=\textwidth]{Images/scanpath-memory.png}
  \caption{The figure gives examples of how the internal memory helps the cognitive controller to remember what has been read and then select actions for detailed gaze movement. A green box indicates the information held in memory, a red box represents the action selected by cognitive control, and the blue lines in the images reflect the eye movement scanpaths.}
  \label{fig:memory}
\end{figure*}

\subsection{Oculomotor Control}

The oculomotor controller acts as the interface between the cognitive controller and the actual chart-pixel images. Its main function is to control the movement of the gaze over the pixels in order to gather information related to the task at hand.
Generating oculomotor behavior at pixel level is another sequential decision-making problem that can be formulated as a POMDP:
\begin{itemize}
    \item Observation $o$ comprises vision information obtained from the external environment, which is jointly represented by the human vision system and visual short-term memory (VSTM).
    \item Action $a$  involves specifying the coordinates $(x, y)$ of a particular position to move to.
    \item Reward $r$ is designed to encourage the gaze to reach the target with less cost. It takes into account the number of target hits as well as the cost associated with the distance of the gaze movement.
\end{itemize}

Our modeling of a chart reader's observation follows an idea similar to that in visual search~\cite{yang2020predicting}. Utilizing a representation for accumulating information through fixations, this employs four components: 
1) The foveal and peripheral view come from the human vision system, which receives high-resolution visual input only from the region of the image around the fixation location. It includes two pixel-based modules to read the chart: foveal and peripheral vision~\cite{duchowski2018gaze}). 
2) Visual saliency provides a bottom-up signal to a chart reader for the given task. The saliency of the chart affects gaze behavior. We use a task-driven saliency model to represent this feature ~\cite{wang2024salchartqa}.
3)  Visit history represents VSTM, which stores visual information for a few seconds, thereby allowing its use in ongoing cognitive tasks~\cite{alvarez2004capacity}. We represent this history through a matrix where each point is marked as visited or not.
4) A goal-related reference position serves as the initial starting point of gaze movement. For example, the reader might begin at the position of a text label for locating the associated graphical mark, where the position of the text label serves as the reference for the sub-goal. 
\rv{We use a one-hot matrix to represent the reference, in which all cell values are 0 apart from the single 1 that identifies the target.}
All these components are encoded together via the deep convolutional neural network, followed by a fully connected network.

We train reinforcement learning policies to solve the POMDP for the oculomotor control, because it has been proven to effectively address decision-making challenges in prediction of details of gaze movement~\cite{yang2020predicting, jiang2024eyeformer, shi2024crtypist, bai2024heads}.
In our detail-level implementation, we resize the input chart images to be $320 \times 320$ and discretize the fixation position into a $20 \times 20$ map. Consequently, each fixation becomes a $16 \times 16$ image patch, and the gaze position is randomly sampled from within that patch. In this setup, the maximum approximation error resulting from this discretization process is less than one degree of the visual angle~\cite{yang2020predicting}.
\rv{
Ultimately, both the scanpath and the image will be converted back to the original chart size from $320 \times 320$ pixels.
}

\subsection{\rv{Workflow}}
\label{sec:workflow}

\rv{
Our implementation of \name is trained and tested on a collection of tasks and charts. There are four steps, illustrated in Figure~\ref{fig:pipeline}.
In Step 1, real-world charts are manually collected and labeled for areas of interest (AOIs), while synthetic charts are automatically generated and labeled in a manner powered by Vega-Lite~\cite{satyanarayan2016vega}. The inclusion of synthetic charts helps increase the diversity of the chart collection and addresses the challenge of obtaining numerous annotated charts.
In Step 2, tasks are automatically generated in line with specific rules for the \textit{RV}, \textit{F}, and \textit{FE} tasks. These tasks and labeled charts constitute a data collection for the training environment.t
With Step 3, the policies for oculomotor control are trained through reinforcement learning (using proximal policy pptimization, PPO~\cite{schulman2017proximal}) to optimize gaze movements, enabling the system to reach task-relevant positions as quickly as possible while adhering to vision constraints. Importantly, no eye tracking data are required for PPO training.
In the last phase, prediction, the hierarchical architecture combines pre-trained LLMs (GPT-4o) for cognitive control with RL policies for oculomotor control to generate the scanpath prediction.
}

\begin{figure*}[!h]
\centering
  \includegraphics[width=\textwidth]{Images/pipeline.png}
  \caption{\rv{An overview of the training workflow: 1) chart collection and labeling, wherein diverse real-world and synthetic charts are gathered, involving manual and automatic annotation of AOIs; 2) task generation, utilizing a rule-based approach to create tasks based on labeled charts to construct a data collection for training; 3) policy training, in which policy models are trained via RL from chart images with tasks; and 4) scanpath prediction, wherein pre-trained LLMs and RL policies are coordinated hierarchically to predict task-driven gaze movements over charts.}}
  \label{fig:pipeline}
  % \vspace{-10mm}
\end{figure*}
\section{Theoretical Analysis}\label{sec:theory}
% Recall that the proposed \name~learning framework aims to find Q-function parameters $\theta$ that can closely approximate the gradients of Q-function evaluated on 

In this section, we study the convergence property of our method and the error bounds of the solutions it finds. We work with mild assumptions that the gradient of the TD loss is Lipschitz smooth with constant $L$: $\|\nabla \mathcal L(\theta') - \nabla \mathcal L(\theta)\| \leq L\|\theta' - \theta\|$, and that the gradient is bounded by $\sigma$: $\| \nabla \mathcal L(\theta) \| \leq \sigma$.

% Let $L_{T}(\theta_{t}; \mathcal{D})$ denote the training 
% \begin{align}
%     L_{T}(\theta_{t}; \mathcal{D}) = 
%     \mathcal L(\theta)=\sum_{i \in \mathcal{D}}\mathcal L^i(\theta) = \sum_{i \in \mathcal{D}}\mathcal L_{\mathtt{TD}}(s_{i}, a_{i}, r_i, s'_{i}, \theta)
% \end{align}
% loss on the original dataset $\mathcal{D}$.
% Then, we suppose the training loss $L_T$ and 
%  and 
% Similarly, we let $L_{T}(\theta_{t}; \mathcal{S})$ denote the training loss on the reduced subset $\mathcal{S}$ with the same smooth assumption.
% \begin{align}
% \mathcal L_{\rdcshort}(\vw,\theta) = \sum\nolimits_{i \in \mathcal{S}} w_i\mathcal L^i(\theta)
% \end{align}
% Let $\Theta_t$ be the angle between $\nabla_{\theta} L_{T}(\theta_t; \mathcal{D})$ and $\nabla_{\theta} L_{T}(\theta_{t}; \mathcal{S})$.
% The cosine similarity of $\Theta_t$ is $\cos\Theta_t = \frac{\nabla_{\theta} L_{T}(\theta_t; \mathcal{D})^T\nabla_{\theta} L_{T}(\theta_{t}; \mathcal{S})}{\|\nabla_{\theta} L_{T}(\theta_t; \mathcal{S})\|\|\nabla_{\theta} L_{T}(\theta_{t}; \mathcal{S})\|}$.
Firstly, we show that the TD loss of the offline Q function $Q^{\pi_\mathcal{S}}$ trained on the reduced dataset $\mathcal{S}$ can converge.
\begin{restatable}{theorem}{convergence}\label{thm:convergence}
    \label{thm:convergence}
    Let $\theta^*$ denote the optimal $Q^{\pi_\mathcal{S}}$ parameters, $\theta_t$ the parameters after $t$ training steps. We have
    \begin{align}
        \min_{t=1:G}\mathcal{L}(\theta_t)\leq \mathcal{L}(\theta^*) + \frac{D\sigma}{\sqrt{G}} + \frac{D}{G}\sum_{t=1}^{G-1}\varepsilon.
    \end{align}
    Here 
    $\mathcal{L}(\theta)=\sum_{i \in \mathcal{D}}\mathcal L_{\mathtt{TD}}(s_{i}, a_{i}, r_i, s'_{i}, \theta)$ is the TD loss, $G$ is the number of total training steps, $D=\|\theta^*-\theta_t\|$, and $\varepsilon=\operatorname{Err}\left(\vw, \mathcal{S}, \mathcal L, \theta_t\right)$ is the gradient approximation errors.
\end{restatable}
\begin{proof}
    Please refer to Appendix~\ref{appendix: convergence} for detailed proof. 
\end{proof}

We assume the gradients of selected data are diverse and they can be divided into $K$ clusters $\{\mathcal{C}_1,\cdots,\mathcal{C}_K\}$ with the cluster centers set $\mathcal C=\{c_1,\cdots,c_K\}$.
Then, we prove the residual error $\operatorname{Err}\left(\vw, \mathcal{S}, \mathcal L, \theta\right)$ can be upper bounded:

% The cornerstone of this approach, i.e., the relationship between the dataset selection problem and the clustering problem, is shown in the following theorem.
\begin{restatable}{theorem}{cluster}\label{thm:cluster}
% \begin{theorem}\label{thm:cluster}
The residual error $\operatorname{Err}\left(\vw, \mathcal{S}, \mathcal L, \theta\right)$ is upper bounded according to the sample's gradient of TD loss:
\begin{align}
    \min_{\mathcal C}\sum_{i\in\mathcal D} \min_{c\in \mathcal C}\|\nabla_{\theta} \mathcal L^i\left(\theta\right) - \nabla_{\theta} \mathcal L^c\left(\theta\right) \|_2. 
\end{align}
% \end{theorem}
\end{restatable}
\begin{proof}
Please refer to Appendix~\ref{appendix: cluster theory} for detailed proof.
\end{proof}

We then prove that the reduced dataset selected by our method can achieve a good approximation for the gradient calculated on the complete dataset, which also means $\varepsilon=\operatorname{Err}\left(\vw, \mathcal{S}, \mathcal L, \theta_t\right)$ in Theorem~\ref{thm:convergence} is bounded.

\begin{corollary}[Approximation Error Bound of the Reduced Dataset]\label{thm:c_bound}
    The expected gradient approximation error achieved by our method is at most $5(\ln K+2)$ times the error of the optimal solution $\mathcal{S}^*$:
    \begin{align}
        \operatorname{Err}\left(\vw, \mathcal{S}, \mathcal L, \theta\right) \le 5(\ln K+2)\operatorname{Err}\left(\vw, \mathcal{S}^*, \mathcal L, \theta\right).
    \end{align}
\end{corollary}
\begin{proof}
The proof is derived by applying Theorem~\ref{thm:cluster} along with Theorem 4.3 from~\citep{makarychev2020improved}, by observing that cluster centers are included in the reduced dataset. 
% This gradient approximation error bound ensures that our reduced dataset will not result in significant performance degradation.
\end{proof}
% \vspace{-1cm}

\paragraph{Discussion}
{The aforementioned theoretical analysis has the following limitations: 
First, we assume that the gradients are uniformly bounded.
Therefore, if the gradients of the algorithm diverge in practice, it would contradict our assumptions, and the selected data subset would no longer be valuable. 
Current offline RL methods can only ensure that the Q-values do not diverge~\cite{kumar2020conservative, fujimoto2021minimalist}.
Although this can, to some extent, reflect the gradients of the Q-network that have not diverged, there is no rigorous proof that the bounds of the gradients can be guaranteed. 
Second, the above theoretical analysis is based on the classic TD loss.
However, to provide a consistent learning signal and mitigate instability caused by changing target value, the techniques in Section~\ref{sec:method:outer} adopt a fixed target rather than TD loss.}

\section{Experiment}\label{sec: exp}
In this section, we assess the efficacy of our algorithm by addressing the following key questions. 
(1) Can offline RL algorithms achieve stronger performance on the reduced datasets selected by~\name?
(2) How does \name~perform compare to other offline data selection methods? 
(3) What are the factors that contribute to \name's effectiveness?

\begin{figure}[t]
    \centering
    \subfigure{\includegraphics[scale=0.24]{d4rl-hard/walker2d-medium-v0-hard.pdf}}
    \hspace{0.2cm}
    \subfigure{\includegraphics[scale=0.24]{d4rl-hard/walker2d-expert-v0-hard.pdf}}
    \hspace{0.2cm}
    \subfigure{\includegraphics[scale=0.24]{d4rl-hard/walker2d-medium-replay-v0-hard.pdf}}
    % \subfigure{\includegraphics[scale=0.20]{d4rl-hard/walker2d-medium-expert-v0-hard.pdf}}
    \subfigure{\includegraphics[scale=0.24]{d4rl-hard/hopper-medium-v0-hard.pdf}}
    \hspace{0.2cm}
    \subfigure{\includegraphics[scale=0.24]{d4rl-hard/hopper-expert-v0-hard.pdf}}
    \hspace{0.2cm}
    \subfigure{\includegraphics[scale=0.24]{d4rl-hard/hopper-medium-replay-v0-hard.pdf}}
    % \subfigure{\includegraphics[scale=0.20]{d4rl-hard/hopper-medium-expert-v0-hard.pdf}}
    \subfigure{\includegraphics[scale=0.24]{d4rl-hard/halfcheetah-medium-expert-v0-hard.pdf}}
    \hspace{0.2cm}
    \subfigure{\includegraphics[scale=0.24]{d4rl-hard/halfcheetah-expert-v0-hard.pdf}}
    \hspace{0.2cm}
    \subfigure{\includegraphics[scale=0.24]{d4rl-hard/halfcheetah-medium-replay-v0-hard.pdf}}
    % \subfigure{\includegraphics[scale=0.20]{d4rl-hard/halfcheetah-medium-v0-hard.pdf}}
    \caption{Experimental results on the D4RL (Hard) offline datasets. All experiment results were averaged over five random seeds. Our method achieves better or
    comparable results than the baselines with lower computational complexity.}
    \label{fig: d4rl hard}
    \vspace{-0.5cm}
\end{figure}

% \begin{figure*}[t]
%     \centering
%     \includegraphics[width=\linewidth]{mujoco/fig1.pdf}
%     \vspace{-2em}
%     \caption{Sample-based selection performance of several baselines and \name~with different selected subset sizes~($x\%$).
%     The horizontal line is the performance of TD3+BC trained with the original dataset.}
%     \label{fig: d4rl minimal ratio}
%     \vspace{-1em}
% \end{figure*}

% \begin{figure}[t]
%     \centering
%     \includegraphics[width=\linewidth]{mujoco/traj.pdf}
%     \caption{In trajectory-based selection, \name~outperforms behavior cloning (\nameh) using trajectories with the highest accumulative returns, presenting a robust method for selecting the most useful data from training sets of compromised quality.}
%     \label{fig: d4rl topbc}
%     \vspace{-1em}
% \end{figure}

\subsection{Setup}
We evaluate algorithms on the offline RL benchmark D4RL~\citep{fu2020d4rl} to answer the aforementioned questions.
In addition, we consider a more challenging scenario where we add additional low-quality data to the dataset to simulate noise in real-world tasks, named D4RL~(hard).
The evaluation process commences with the selection of offline data, followed by the training of a widely recognized offline RL algorithm, TD3+BC~\citep{fujimoto2021minimalist}, on this reduced dataset for 1 million time steps.
To ensure a fair comparison, we apply the same offline RL algorithm to data subsets obtained by different algorithms. 
Evaluation points are set at every 5,000 training time steps and involve calculating the return of 10 episodes per point.
The results, comprising averages and standard deviations, are computed with five independent random seeds.
On the other hand, we can also incorporate our method into offline model-based approaches, such as MOPO~\citep{yu2020mopo} and MoERL~\citep{kidambi2020morel}.
Similarly, we only need to replace the current offline loss with the corresponding policy and model loss.

\textbf{Baselines}. 
We compare \name~with data selection methods in RL.
Specifically, previous work on prioritized experience replay for online RL~\citep{schaul2015prioritized} aligns closely with our objective. 
We make this a baseline \namep~where samples with the highest TD losses form the reduced dataset. 
Baseline \nameo~presents the performance by training TD3+BC with the original, complete dataset. 
Baseline \namer~randomly selects subsets from the D4RL dataset that are of the same size as \name.
We also compare our method with general dataset reduction techniques from supervised learning.
Specifically, we adopt the coherence criterion from Kernel recursive least squares~($\mathtt{KRLS}$)~\citep{engel2004kernel}, the log det criterion by forward selection in informative vector machines~($\mathtt{LogDet}$)~\citep{seeger2004greedy} and the adapting kernel representation~($\mathtt{BlockGreedy}$)~\citep{schlegel2017adapting} as our baselines.

%Specifically, we consider randomly selecting offline coreset as our baseline algorithms.
% In addition, we consider separately selecting high-reward offline datasets and low-reward offline datasets as our baseline algorithms.

\subsection{Experimental Results}
\label{sec:exp_perf}
% To compare the performance of different algorithms, we adopt two data selection schemes: sample-based selection and trajectory-based selection. They differ in the smallest unit of selection: the first selects samples in each iteration, while the second selects trajectories.

% As for the trajectory-based selection, prioritized sampling is no loner applicable. As an alternative, we compare with \nameh, which selects trajectories with the highest accumulative reward from the complete dataset. We again compare with the \nameo~as the reference to an upper limit of performance.

\begin{table*}[t]
    \centering
    \begin{tabular}{c|cccc}
    \toprule
    & KRLS & Log-Det & BlockGreedy & \name \\
    \midrule
    Hopper-medium-v0 & 69.4$\pm$2.5 & 58.4$\pm$3.6 & 83.7$\pm$2.2 & \textbf{94.3$\pm$4.6}\\
    Hopper-expert-v0 & 91.0$\pm$1.1 & 90.7$\pm$1.3 & 98.7$\pm$0.5 & \textbf{110.0$\pm$0.5}\\
    Hopper-medium-replay-v0 & 28.5$\pm$3.2 & 29.4$\pm$1.2 & 30.5$\pm$2.4 & \textbf{35.3$\pm$3.2}\\
    Walker2d-medium-v0 & 49.1$\pm$2.8 & 47.5$\pm$3.4 & 53.3$\pm$3.6 & \textbf{80.5$\pm$2.9}\\
    Walker2d-expert-v0 & 68.4$\pm$3.2 & 67.5$\pm$5.6 & 74.8$\pm$3.4 & \textbf{104.6$\pm$2.5}\\
    Walker2d-medium-replay-v0 & 14.3$\pm$1.2 & 15.2$\pm$2.2 & 16.7$\pm$1.3 & \textbf{21.1$\pm$1.8}\\
    Halfcheetah-medium-v0 & 23.4$\pm$0.5 & 21.9$\pm$0.9 & 27.5$\pm$0.7 & \textbf{41.0$\pm$0.2}\\
    Halfcheetah-expert-v0 & 73.9$\pm$1.4 & 72.1$\pm$2.2 & 79.2$\pm$1.8 & \textbf{88.5$\pm$2.4}\\
    Halfcheetah-medium-replay-v0 & 39.5$\pm$0.3 &39.9$\pm$0.5 & 40.5$\pm$1.0 & \textbf{41.1$\pm$0.4}\\
    \bottomrule
    \end{tabular}
    \caption{Experimental results on the D4RL~(Hard) offline datasets. All experiment results were averaged over five random seeds. Our method performs better than the dataset reduction baselines.}
    \label{tab: varied performance}
\end{table*}

\begin{figure}[t]
    \centering
    \subfigure{\includegraphics[scale=0.20]{d4rl/halfcheetah-medium-expert-v0.pdf}}
    \subfigure{\includegraphics[scale=0.20]{d4rl/hopper-medium-v0.pdf}}
    \subfigure{\includegraphics[scale=0.20]{d4rl/hopper-medium-expert-v0.pdf}}
    \subfigure{\includegraphics[scale=0.20]{d4rl/walker2d-medium-expert-v0.pdf}}
    \caption{Experimental results on the D4RL offline datasets. All experiment results were averaged over five random seeds. Our method achieves better or comparable results than the baselines consistently.}
    \label{fig: d4rl original}
\end{figure}

\paragraph{Answer of Question 1:}
To show that \name~can improve offline RL algorithms, we compare \name~with Complete Dataset, Prioritized, and Random in the Mujoco domain.
The experimental results in Figure~\ref{fig: d4rl hard} show that our method achieves superior performance than baselines.
By leveraging the reduced dataset generated from \name, the agent can learn much faster than learning from the complete dataset.
Further, the results in Figure~\ref{fig: d4rl original} show that \name~also performs better than the complete dataset and data selection RL baselines in the standard D4RL datasets. 
This is because prior methods select data in a random or loss-priority manner, which lacks guidance for subset selection and leads to degraded performance for downstream tasks.

In addition, to test \name's generality across various offline RL algorithms on various domains, we also conduct experiments on Antmaze tasks.
We use IQL~\citep{kostrikov2021offline} as the backbone of offline RL algorithms.
The experimental results in Table~\ref{tab: other domain2} show that our method achieves stronger performance than baselines.
In the antmaze tasks, the agent is required to stitch together various trajectories to reach the target location.
In this scenario, randomly removing data could result in the loss of critical data, thereby preventing complete the task.
Differently, \name~extracts valuable subset by balancing data quantity with performance, achieving a stronger performance than the complete dataset.

% In Figure~\ref{fig: d4rl minimal ratio}, we show the performance of different algorithms with the sample-based selection scheme. The experimental results show that \name~can achieve performance close to \nameo~with a small amount of data. For example, we use only $3\%$ of the original dataset in the Hopper tasks. \namer~and \namep, on the other hand, present a stark contrast, even not showing a stable learning trend with the same amount of training data. 
% In addition, we also evaluate the performance on the trajectory-based selection setting. Please refer to Appendix~\ref{appendix: trajectory} for the detailed experimental results.
% For the trajectory-based selection, experimental results in Figure~\ref{fig: d4rl topbc} show that \name~maintains its superiority in this setting with suboptimal (e.g., \texttt{medium}) datasets. This evidence suggests that \name~provides a valuable strategy for selecting data conducive to effective training under conditions of compromised data quality.

\paragraph{Answer of Question 2:}
To test whether \name~can select more valuable data than the data selection algorithms in supervised learning, we compare our method with KRLS~\citep{engel2004kernel}, Log-Det~\citep{seeger2004greedy} and BlockGreedy~\citep{schlegel2017adapting} in the D4RL~(Hard) datasets.
The experimental results in Table~\ref{tab: varied performance} show that our method generally outperforms baselines.
We hypothesize that supervised learning is static with fixed learning objectives, while offline RL's dynamic nature makes the target values evolve with policy updates, complicating the data selection process.
Therefore, the data selection methods in supervised learning cannot be directly applied to offline RL scenarios.

% Additionally, we observe that  $\texttt{Random}$ performs better than $\texttt{Q-diff}$.
% We attribute this phenomenon to the broader data coverage of $\texttt{Random}$, while the data coverage of $\texttt{Q-diff}$ is narrow.
% However, we also note that in some tasks, such as $\texttt{Hopper-medium-expert-v0}$, $\texttt{Hopper-expert-v0}$ and $\texttt{Walker2d-expert-v0}$, $\texttt{Random}$ initially performs well, but as training progresses, its performance starts to decline.
% We find that this coincides with unstable Q-values, which can be attributed to the increased extrapolation error caused by the reduced training dataset.
% In contrast, \name~performs better since it closely approximates the original gradients, thus preventing Q-values from diverging.


% For this reason, when the dataset quality is high~(e.g., \texttt{medium-expert} dataset), TopBC performs comparably to \name.

% \begin{table*}[t]
%     \centering
%     \caption{\name~with varying dataset sizes~($x\%$). Highlighted is the performance comparable to training TD3+BC with the complete dataset. \name~typically achieves good results with 5\%-15\% data, indicating that existing offline RL datasets contain a high degree of redundancy.
%     We adopt the normalized score metric proposed by the D4RL benchmark. Scores roughly range from 0 to 100, where 0 corresponds to the performance of a random policy and 100 indicates the performance of an expert.} 
%     \label{tab: varied performance}
%     \begin{tabular}{c|cccc}
%     \toprule
%         & 5\% & 10\% & 15\% & 20\% \\
%         \midrule
%         Hopper-medium-v0 & 91.8$\pm$3.6 & 92.6$\pm$3.0 & 94.0$\pm$4.8 & 95.2$\pm$1.6\\
%         Walker2d-medium-v0 & 14.8$\pm$7.3 & 57.9$\pm$3.6 & 69.3$\pm$4.0 & 71.7$\pm$1.9 \\
%         Halfcheetah-medium-v0 & 40.5$\pm$0.0 & 40.9$\pm$0.1 & 41.3$\pm$0.1 & 41.2$\pm$0.5 \\
%         Hopper-expert-v0 & 111.6$\pm$0.9 & 110.6$\pm$1.9 & 112.7$\pm$0.1 & 112.4$\pm$0.1 \\
%         Walker2d-expert-v0 & 74.5$\pm$6.4 & 84.4$\pm$5.0 & 97.6$\pm$3.1 & 100.2$\pm$1.0 \\
%         Halfcheetah-expert-v0 & 57.5$\pm$6.4 & 84.3$\pm$2.7 & 97.8$\pm$0.8 & 100.1$\pm$3.0 \\
%         Hopper-medium-expert-v0 & 108.1$\pm$1.1 & 112.4$\pm$0.3 & 112.3$\pm$0.05 & 112.8$\pm$0.1\\
%         Walker2d-medium-expert-v0 & 79.3$\pm$2.1 & 85.4$\pm$5.3 & 96.2$\pm$6.7 & 101.4$\pm$3.6 \\
%         Halfcheetah-medium-expert-v0 & 67.5$\pm$0.5 & 86.2$\pm$5.0 & 85.8$\pm$1.5 & 92.4$\pm$1.3\\
%     \bottomrule
%     \end{tabular}
% \end{table*}


% \subsection{Ablation Study}\label{sec:exp_ab}
% \textbf{Varying dataset size}.\ \ In Table~\ref{tab: varied performance}, we show the performance of \name~with varying dataset sizes ranging from $5\%$ to $20\%$.
% The results demonstrate that \name~requires only $5\%$ or $10\%$ of the original dataset to obtain good performance.
% Further, \name~can achieve similar performance with \nameo~with $20\%$ data of the original dataset.
% This indicates that existing offline RL datasets are characterized by a high degree of redundancy.

\begin{figure}[t]
    \centering
    \includegraphics[width=0.97\linewidth]{visual.jpg}
    \caption{Visualization of the \textcolor{blue}{complete dataset} and the \textcolor{orange}{reduced dataset} in \texttt{halfcheetah} task. The higher opacity of a point represents a large time step towards the end of an episode. The dataset embedding is characterized by its division into different components. 
    % In \texttt{walker2d} (upper), components vary with time steps.
     Samples selected by \name~connect different components by focusing on the data related to the task.}
    \label{fig: t-sne}
\end{figure}

\begin{table}[t]
    \centering 
    \begin{tabular}{c|cccc}
    \toprule
        Env & Random & Prioritized & Complete Dataset & \name\\
        \midrule
        Antmaze-umaze-v0 & 75.1$\pm$2.5 & 70.2$\pm$3.6 & 87.5$\pm$1.3 & \textbf{90.7$\pm$3.3}\\
        Antmaze-umaze-diverse-v0 & 46.3$\pm$1.9 & 44.7$\pm$2.7 & 62.2$\pm$2.0 & \textbf{76.7$\pm$2.2} \\
        Antmaze-medium-play-v0 & 59.3$\pm$1.6 & 60.3$\pm$2.9 & 71.2$\pm$2.2 & \textbf{80.3$\pm$2.9}\\
        Antmaze-medium-diverse-v0 & 43.6$\pm$2.7 & 46.9$\pm$3.8 & 70.0$\pm$1.6 & \textbf{84.9$\pm$3.8}\\
        Antmaze-large-play-v0 &	3.7$\pm$0.7 & 15.0$\pm$3.5 & 39.6$\pm$3.6 & \textbf{46.0$\pm$3.5}\\
        Antmaze-large-diverse-v0 & 16.0$\pm$3.6 & 20.5$\pm$3.7 & 47.5$\pm$1.1 & \textbf{52.0$\pm$3.7}\\
    \bottomrule
    \end{tabular}
    \caption{Experimental results on the Antmaze offline datasets. All experiment results were averaged over five random seeds. Our method performs better than baselines. }
    \label{tab: other domain2}
\end{table}

% \begin{figure*}[t]
%     \centering
%     \subfigure{\includegraphics[scale=0.27]{ablation_moduler1.pdf}}
%     \hspace{0.3cm}\subfigure{\includegraphics[scale=0.27]{ablation_moduler2.pdf}}
%     \caption{Ablation results on D4RL~(Hard) tasks with the normalized score metric.}
%     \label{fig: modular ablation}
% \end{figure*}

% In this subsection, we conduct ablation studies to study the effect of different modules and import hyper-parameters.


\paragraph{Answer of Question 3:}
To study the contribution of each component in our learning framework, we conduct the following ablation study. 
\nameq: We replace the empirical returns used to update Q functions with the standard target Q function in the TD loss function. 
\namei: We set the number of data selection rounds to 1 and study the function of multi-round data selection.
The experimental results in Figure~\ref{fig: modular ablation} in Appendix~\ref{sec: ablation} show that removing any of these two modules will worsen the performance of \name. In case like $\texttt{walker2d-medium}$, ablation \namei~even decrease the performance by over 80\%, and ablation \nameq~results in a 95\% performance drop in $\texttt{walker2d-expert}$. Furthermore, we also find that in the $\texttt{halfcheetah}$ tasks, the impact of removing the two modules is relatively small. This result can be attributable to the fact that this task has a limited state space, and we can directly apply OMP to the entire dataset and identify important and diverse data.

We visualize the selected data by \name~to better understand how it works. 
Figure~\ref{fig: t-sne} displays the t-SNE low-dimensional embeddings, with the complete dataset in blue and the selected data in orange. 
The higher opacity of a point indicates a larger time step. The dataset's structure is revealed by its segmentation into diverse components: 
In \texttt{halfcheetah}, each component reflects a distinct skill of the agent.
For example, from 1 to 7, they represent falling, leg lifting, jumping, landing, leg swapping, stepping, and starting, respectively.
We can observe that the selected samples by \name~ not only cover each component of the dataset but also effectively bridge the gaps between them, enhancing the dataset's versatility and coherence. 
Moreover, we find that \name~is less concerned with the falling data and instead focuses on the data related to the task.
This observation can explain the improved performance of \name. For additional visualizations, please refer to Appendix~\ref{appendix: visual}.

% \textbf{Generalizability of \name}. \ \
% We evaluate the generalizability of \name~from two perspectives.
% First, we add IQL~\cite{kostrikov2021offline} as a baseline and apply \name~to IQL by using the gradient of the training loss of the V-function in IQL as the criterion.
% On the other hand, we evaluate \name~on the other domains, such as robotic manipulation (Adroit) and sparse reward (Antmaze) tasks.
% The experiments in Appendix~\ref{appendix: other domain} and Appendix~\ref{appendix: other algorithm} show that \name~is not only applicable to other algorithms, such as IQL~\cite{kostrikov2021offline}, but also to other domains.

% \textbf{Generalizability of subset}. \ \
% To test the generalizability of the dataset selected by~\name, we select subset by applying~\name~to TD3+BC.
% Then we evaluate the performance of IQL on the selected subset. 
% The experimental results in Table~\ref{tab: td3bc2iql} in Appendix~\ref{appendix: tb3bc2iql} demonstrate that the selected subset based on TD3+BC is effectively applicable to IQL.

% \textbf{Sensitivity for hyperparameter}. \ \
% We evaluate the performance of \name~with various cluster numbers~(from 1 to 50) and approximation bounds~(from 0.0001 to 0.05).
% The experimental results in Appendix~\ref{appendix: cluster number} and Appendix~\ref{appendix: approx bound} show that the suitable cluster number is between 25 and 50.
% Too few clusters (e.g., less than 5) are detrimental to the algorithm.
% In addition, a smaller approximation bound represents a larger reduced dataset.
% Similar to the ablation of the size of the reduced dataset in Table~\ref{tab: varied performance}, \name~requires only a 0.01 approximation bound to obtain good performance.

\subsection{Computational complexity}
We report the computational overhead of \name~on various datasets. 
All experiments are conducted on the same computational device (GeForce RTX 3090 GPU). 
The results in Appendix~\ref{appendix: computation complexity} indicate that even on datasets containing millions of data points, the computational overhead of our method remains low~(e.g., several minutes).
This low computational complexity can be attributed to the trajectory-based selection technique in Sec.~\ref{sec: offline omp}~(II) and the regularized constraint technique in Sec.~\ref{sec:method:outer}, making our method easily scalable to large-scale datasets. 

% This low computational complexity can be attributed to the batch mechanism designed in section 3.2 (IV), which reduces the computational complexity from $O(MN)$ to $O(|\mathcal{B}|N)$, making our method easily scalable to large-scale datasets. $M, N, |\mathcal{B}|$ are the size of the full dataset, reduced dataset, and batch respectively.

% We conduct t-SNE based dimensionality reduction to the cluster centroids and these five trajectories.
% The experimental results are shown in the , where darker colors indicate moving towards the end of the trajectory.

% From the experimental results, we find that in the walker2d task, \name~ tends to select more low-reward but more diverse data points ~(upper right) while selecting a few high-reward data points~(left and bottom).
% We attribute this phenomenon to the narrow distribution of the high-reward points, allowing us to approximate the original gradients with only a few points. 
% In the halfcheetah task, \name~ connects useful information while ignoring low-quality data~(e.g., data point \texttt{1}).

% \clearpage
\section{Related works}
\label{appx:related_work}
% As deep learning usually trains on abundant data, a considerable number of works have focused on identifying important training samples and figuring out the ideal size of the dataset. However, the employment in policy training tasks is under-explored. Our work is closely related to offline RL and data subset selection.

\textbf{Offline Reinforcement Learning.}\ \
Offline RL can execute policy training entirely based on static datasets without further interaction with the environment~\cite{levine2020offline}.
Therefore, it faces challenges such as distribution shift and value overestimation.
To address this issue, some prior works attempted to constrain the learned policy and behavior policy by limiting the action difference~\cite{fujimoto2019off}, adding KL-divergence~\cite{nair2020awac,peng2019advantage,wu2019behavior}, or regularization~\cite{kumar2019stabilizing}.
Other works consider employing conservative estimates of future values~\cite{kumar2020conservative,ma2021conservative} or penalizing uncertain actions~\cite{janner2019trust,yu2021combo,kidambi2020morel} by uncertainty.
There are also some new attempts, such as lightweight implementation~\cite{fujimoto2021minimalist} or avoiding distribution shift by single-step policy iteration~\cite{kostrikov2021offline}.
These studies provide a solid foundation for implementing and transferring reinforcement learning to real-world tasks.
However, there has been limited research addressing considerations related to the dataset.
Some works attempted to explore which dataset characteristics dominate in offline RL algorithms~\cite{schweighofer2021understanding, swazinna2021measuring} or investigate the data generation~\cite{yarats2022don}.
Recently, some researchers attempted to solve the sub-optimal trajectories issue by constraining policy to good data rather than all actions in the dataset~\cite{hong2023beyond} or re-weighting policy~\cite{hong2023harnessing}.
Differently, our work aims to figure out the ideal size of the dataset needed for effective policy training.

\textbf{Data subset selection.}\ \ The research on identifying crucial samples within datasets is concentrated in the field of computer vision.
Some prior works use uncertainty of samples~\cite{coleman2019selection,paul2021deep} or the frequency of being forgotten~\cite{toneva2018empirical} as the proxy function to prune the dataset.
Another research line focuses on constructing weighted data subsets to approximate the full dataset~\cite{feldman2020core}, which often transforms the subset selecting to the submodular set cover problem~\cite{wei2015submodularity,kaushal2019learning}.
Specifically, several works adopted loss functions as the optimization target~\cite{lucic2018training,campbell2018bayesian}, while recent research finds that approximating full gradient is more efficient~\cite{mirzasoleiman2020coresets, killamsetty2021grad,killamsetty2021glister,killamsetty2021retrieve}.
These studies establish the critical importance of selecting key samples from datasets for effective training. However, the different learning objectives and training methodologies between reinforcement learning and computer vision mean that these techniques cannot be directly applied to Offline RL.
\section{Discussion}

\paragraph{Quadratic Programming vs. Logistic Regression.}  
Our formulation estimates the attribute weights $\mathbf{p}$ by transforming the Bradley-Terry loss into a quadratic program. An alternative approach based on logistic regression—which assigns absolute labels of 1 and 0 to win/lose responses—can also be used, as demonstrated by \citep{go2023compositional}. 
We compared these two formulations using Drift attributes in Table~\ref{fig:discussion}. The logistic regression approach proves highly unstable and shows lower performance when training examples are limited. We interpret this instability as follows: preference judgments are inherently relative—what constitutes a winning response in one context might be considered a losing response when compared to an even better alternative. Thus, imposing absolute labels through regression can lead to overfitting, particularly when data are scarce. Our results suggest that approaching preference problems from a relative perspective is crucial for effective preference modeling.
\begin{figure}[ht]
\centering
\includegraphics[trim=7 8 2 2, clip, width=0.65\columnwidth]{figs/discussion.pdf}
\caption{Few-shot preference modeling results for \texttt{user1008} in the PRISM with quadratic programming (QP) and logistic regression (LQ).}
\label{fig:discussion}
\vspace{-5mm}
\end{figure}


\paragraph{Compatible with samplers.}
\label{sec:practical-2}
Autoregressive sampling in LLMs has various decoding strategies at the token-level distribution. Drift steers distributions at the logit level—applying its computations before the softmax—making it compatible with a wide range of sampling methods tailored to different objectives~\citep{vijayakumar2016diverse, fan2018hierarchical, holtzman2019curious}. our analysis indicates that the backbone LLM exhibits an average next-token entropy of about 0.27 bits, which increases to approximately 0.63 bits after applying Drift. While this boost in entropy can substantially enhance generation diversity, it may also increase the likelihood of selecting unreliable tokens. Therefore, we recommend combining Drift with top-p or top-k sampling strategies to control an optimal balance between diversity and reliability.

\paragraph{Practical Implications.}
While traditional RLHF methods may eventually surpass Drift when user data becomes abundant, Drift offers several advantages in practical settings. 
First, conventional reward models struggle with \textit{continual learning}; retraining on an ever-expanding user dataset is impractical. In contrast, Drift can be updated instantly by simply appending new instances to the $W-L$—no retraining required. 
Second, personal preferences often \textit{change more rapidly than general preferences}. Drift’s interpretability allows real-time tracking of preference shifts, enabling dynamic adjustments for improved personalization. 
Third, when collecting additional user annotations, the variance observed in each attribute can inform an \textit{active learning} strategy~\citep{miller2020active} for efficient data collection. These benefits make Drift an attractive complement to existing RLHF pipelines in personalized applications.


\bibliography{iclr2025_conference}
\bibliographystyle{iclr2025_conference}

\newpage
\appendix
\onecolumn
% % \section{Algorithm}
% \label{appendix: alg}

% \begin{algorithm}[H]
%     \caption{Reduce Dataset for Offline RL~(\name)}
%     \label{alg: offline data selection}
%     \begin{algorithmic}[1]
%         \STATE {\bf Require}: Complete offline dataset $\mathcal{D}$
%         \STATE Initialize parameters of the offline agent for data selection $Q_{\theta}, \pi_{\phi}$ and update interval $T^s$
%         \STATE Run a clustering algorithm on the original dataset $\mathcal D$ and obtain $K$ clusters $\{\mathcal{C}_1,\cdots,\mathcal{C}_K\}$
%         \STATE Cluster centers set $\mathcal C=\{c_1,\cdots,c_K\}$
%         \FOR{$k=1, \cdots, K$}
%         \FOR{$j=1, \cdots, J$}
%         \STATE Sample mini-batch $\mathcal{B}_{j}^{k}$ from the cluster $\mathcal{C}_k$
%         \STATE Calculate $\nabla_{\theta}\mathcal{L}^B, \nabla_{\theta}\mathcal L_{\rdcshort}^B$ under $\mathcal{B}_{j}^{k}$
%         \STATE $\mathcal{S}_{j}^{k}, \vw_{j}^k$ = OMP$(\nabla_{\theta}\mathcal{L}^B, \nabla_{\theta}\mathcal L_{\rdcshort}^B, Q_{\theta}, c_{k})$
%         \IF{$j$ mod $T^s$ = 0}
%         \STATE Train $Q_{\theta}, \pi_{\phi}$ based on Equation~\ref{eq: q-target value}
%         \ENDIF
%         \ENDFOR
%         \ENDFOR
%     \STATE Reduced offline dataset $\mathcal S\leftarrow\cup_{j\in[J],k\in[K]}\mathcal S_j^k$
%     \STATE Initialize parameters of the offline agent for training on the reduced offline dataset $Q_{\vartheta},\pi_{\varphi}$
%     \STATE Train $Q_{\vartheta},\pi_{\varphi}$ based on $\mathcal{S}$ and $\vw$
%     \end{algorithmic}
% \end{algorithm}

% \begin{algorithm}[H]
%     \caption{OMP algorithm}
%     \label{alg: omp}
%     \begin{algorithmic}[1]
%         \STATE {\bf Require}: $\nabla_{\theta}\mathcal{L}^B$, $\nabla_{\theta}\mathcal L_{\rdcshort}^B$, parameter $\theta$ of $Q_{\theta}$, cluster center $c_k$, regularization coefficient $\lambda$, subset size $\frac{|\mathcal{B}|N}{M}$
%         \STATE $\mathcal{S}_j^k\leftarrow\{c_k\}$
%         \STATE $r\leftarrow\operatorname{Err}_{\lambda}^B\left(\vw_j^k, \mathcal{S}_j^k, \mathcal L^B, \theta\right)$
%         \WHILE{$|\mathcal{S}_j^k|\leq\frac{|\mathcal{B}|N}{M}$}
%         \STATE $e=\arg\max_{i\notin\mathcal{S}_j^k}|\langle\nabla_{\theta}^i\mathcal L^{B}_{\rdcshort}\left(\theta\right), r\rangle|$
%         \STATE $\mathcal{S}_j^k\leftarrow\mathcal{S}_j^k\cup\{e\}$
%         \STATE $\vw_j^k\leftarrow\arg\min_{\vw_j^k}\operatorname{Err}_{\lambda}^B\left(\vw_j^k, \mathcal{S}_j^k, \mathcal L^B, \theta\right)$
%         \STATE $r\leftarrow\operatorname{Err}_{\lambda}^B\left(\vw_j^k, \mathcal{S}_j^k, \mathcal L^B, \theta\right)$
%         \ENDWHILE\\
%     \STATE \textbf{Return} $\mathcal{S}_j^k$ and $\vw_j^k$
%     \end{algorithmic}
% \end{algorithm}
% \clearpage
\section{Related works}
\label{appx:related_work}
% As deep learning usually trains on abundant data, a considerable number of works have focused on identifying important training samples and figuring out the ideal size of the dataset. However, the employment in policy training tasks is under-explored. Our work is closely related to offline RL and data subset selection.

\textbf{Offline Reinforcement Learning.}\ \
Offline RL can execute policy training entirely based on static datasets without further interaction with the environment~\cite{levine2020offline}.
Therefore, it faces challenges such as distribution shift and value overestimation.
To address this issue, some prior works attempted to constrain the learned policy and behavior policy by limiting the action difference~\cite{fujimoto2019off}, adding KL-divergence~\cite{nair2020awac,peng2019advantage,wu2019behavior}, or regularization~\cite{kumar2019stabilizing}.
Other works consider employing conservative estimates of future values~\cite{kumar2020conservative,ma2021conservative} or penalizing uncertain actions~\cite{janner2019trust,yu2021combo,kidambi2020morel} by uncertainty.
There are also some new attempts, such as lightweight implementation~\cite{fujimoto2021minimalist} or avoiding distribution shift by single-step policy iteration~\cite{kostrikov2021offline}.
These studies provide a solid foundation for implementing and transferring reinforcement learning to real-world tasks.
However, there has been limited research addressing considerations related to the dataset.
Some works attempted to explore which dataset characteristics dominate in offline RL algorithms~\cite{schweighofer2021understanding, swazinna2021measuring} or investigate the data generation~\cite{yarats2022don}.
Recently, some researchers attempted to solve the sub-optimal trajectories issue by constraining policy to good data rather than all actions in the dataset~\cite{hong2023beyond} or re-weighting policy~\cite{hong2023harnessing}.
Differently, our work aims to figure out the ideal size of the dataset needed for effective policy training.

\textbf{Data subset selection.}\ \ The research on identifying crucial samples within datasets is concentrated in the field of computer vision.
Some prior works use uncertainty of samples~\cite{coleman2019selection,paul2021deep} or the frequency of being forgotten~\cite{toneva2018empirical} as the proxy function to prune the dataset.
Another research line focuses on constructing weighted data subsets to approximate the full dataset~\cite{feldman2020core}, which often transforms the subset selecting to the submodular set cover problem~\cite{wei2015submodularity,kaushal2019learning}.
Specifically, several works adopted loss functions as the optimization target~\cite{lucic2018training,campbell2018bayesian}, while recent research finds that approximating full gradient is more efficient~\cite{mirzasoleiman2020coresets, killamsetty2021grad,killamsetty2021glister,killamsetty2021retrieve}.
These studies establish the critical importance of selecting key samples from datasets for effective training. However, the different learning objectives and training methodologies between reinforcement learning and computer vision mean that these techniques cannot be directly applied to Offline RL.
\clearpage
\section{Proofs of theoretical analysis}
{\subsection{Notations}}

\begin{table}[ht]
    \centering
    {\begin{tabular}{ll}
    \toprule
    Notation & Explanation \\
    \midrule
    \hspace{0.3cm} $U_\mathtt{TD}$ & Bound of TD Loss \\
    \hspace{0.3cm} $U_{\nabla Q}$ & Bound of Gradient \\
    \hspace{0.3cm} $U_{\nabla a}$ & Bound of Gradient \\
    \hspace{0.3cm} $U_a$ & Bound of Action Difference \\
    \hspace{0.3cm} $U_\pi$ & Bound of Action \\
    \hspace{0.3cm} $\mathcal{D}$ & Complete Dataset \\
    \hspace{0.3cm} $\mathcal{S}$ & Reduced Dataset \\
    \hspace{0.3cm} $N$ & Size of Reduced Dataset \\
    \hspace{0.3cm} $\lambda$ & Minimum Eigenvalues \\
    \hspace{0.3cm} $\mathcal{C}$ & Cluster \\
    \hspace{0.3cm} $G$ & Total Training Steps \\
    \hspace{0.3cm} $\epsilon$ & Gradient Approximation Errors \\
    \hspace{0.3cm} $\theta_t$ & Updated parameter at the $t^{th}$ epoch \\
    \hspace{0.3cm} $\theta_t^*$ & Optimal model parameter \\
    \bottomrule
    \end{tabular}}
    {\caption{Organization of the notations used througout this paper}}
    \label{tab: notation}
\end{table}

\subsection{Submodular}
\label{appendix: submodular}

% First, we restate Theorem \ref{thm: submodular}.
% \begin{theorem}
%     For any $\mathcal S$ with $|\mathcal S| \leq N$ and sample $(s_i,a_i,r_i,s'_i)\in \mathcal{D}$, suppose that the TD loss and gradients are bounded: $|\mathcal{L}^i(\theta)| \leq U_\mathtt{TD}$, $ \|\nabla_\theta Q_\theta(s_i,a_i)\|_2 \leq U_{\nabla Q}$, $\|\nabla_{\pi_{\phi}(s_i)}Q_\theta(s_i,\pi_{\phi}(s_i))\|_2 \leq U_{\nabla a}$, $\|\pi_{\phi}(s_i)-a_i\|_2 \leq U_a$, $\|\pi_{\phi}(s_i)\|_2\leq U_\pi$, and $\|\nabla_\phi \pi_{\phi}(s_i)\|_2 \leq U_{\nabla\pi}$, then $F_\lambda^Q(\mathcal{S})$ is $\delta$-weakly submodular, with
%     \begin{align}
%         \delta \geq \frac{\lambda}{\lambda+4 N (U_\mathtt{TD}U_{\nabla Q})^2},\nonumber
%     \end{align}
%     and $F_\lambda^\pi(\mathcal{S})$ is $\delta$-weakly submodular, with 
%     \begin{align}
%         \delta \geq \frac{\lambda}{\lambda + N(U_{\nabla a}/\alpha+2U_a U_\pi)^2 U_{\nabla\pi}^2}.\nonumber
%     \end{align}
%     \label{thm: submodular}
% \end{theorem}
\submodular*

\begin{proof}
As mentioned in Section \ref{sec: preliminary}, we use the TD3+BC algorithm as the basic offline RL algorithm. 
TD3+BC follows the actor-critic framework, which trains policy and value networks separately. 
For a single sample $(s_i,a_i,r_i,s'_i)$, the loss of the value network is also named as TD error, which is defined by:
\begin{align}
    & \mathcal L_{Q}^i(\theta) = (y_i - Q_\theta(s_i,a_i))^2   \\
    & \text{where}\quad y_i = r_i + \gamma Q_{\theta'}(s'_i,\pi_{\phi'}(s'_i)+\epsilon)  \\
\end{align}

The gradient is:

\begin{align}
    -\frac{1}{2} \nabla_{\theta} \mathcal L^i_Q(\theta)=(y_i- Q_\theta(s_i,a_i))\nabla_\theta Q_\theta(s_i,a_i)
    \label{eq: td_gradient}
\end{align}

Offline RL algorithms attempt to minimize the TD error and compute the Q-value through a neural network.
Therefore, we assume the upper bound of the TD error is $\max_i\|y_i- Q_\theta(s_i,a_i)\|_2\leq U_\mathtt{TD}$.
The upper bound of the gradient of the value network is $\max_i \|\nabla_\theta Q_\theta(s_i,a_i)\|_2\leq U_{\nabla Q}$.
Then, Equation~\ref{eq: td_gradient} can be transformed into:
\begin{equation}
    \|\nabla_\theta \mathcal L^i_Q(\theta)\|_2 \leq 2U_\mathtt{TD} U_{\nabla Q}
\end{equation}

Similarly, for a single sample$(s_i,a_i,r_i,s'_i)$, the loss of the policy network is
\begin{align}
    \mathcal L_{\pi}^i(\phi) &= -\frac{1}{\alpha} Q_\theta(s_i, \pi_{\phi}(s_i))+\|\pi_{\phi}(s_i)-a_i\|_2^2 \\
\end{align}

The gradient is:

\begin{align}
    \nabla_{\phi} \mathcal L_{\pi}^i(\phi) &= \frac{\partial \mathcal L_{\pi}^i(\phi)}{\partial \pi_{\phi}(s_i)}\times \frac{\partial \pi_{\phi}(s_i)}{\partial \phi}   \\
    &= [-\frac{1}{\alpha} \nabla_{\pi_{\phi}(s_i)}Q_\theta(s_i,\pi_{\phi}(s_i))+2(\pi_{\phi}(s_i)-a_i)^\top \pi_{\phi}(s_i)] \times \nabla_\phi \pi_{\phi}(s_i)
    \label{eq: policy_gradient}
\end{align}

Here $\alpha$ is used to balance the conservatism and generalization in Offline RL, which is defined by:

\begin{align}
    \alpha= \frac{\mathbb{E}_{(s_i,a_i)}[|Q(s_i,a_i)|]}{\kappa}
\end{align}

where $\kappa$ is a hyper-parameter in TD3+BC.
Note that although $\alpha$ includes $Q$, it is not differentiated over. 

Offline RL algorithms attempt to limit the deviation of the current learned policy from the behavior policy while maximizing the Q-value of the optimized policy.
Therefore, we assume the upper bound of the gradient of the value network is $\max_i\|\nabla_{\pi_{\phi}(s_i)}Q_\theta(s_i,\pi_{\phi}(s_i))\|_2 \leq U_{\nabla a}$.
The upper bound of the action error is $\max_i\|\pi_{\phi}(s_i)-a_i\|_2\leq U_a$.
The upper bound of the output of the policy is $\max_i\|\pi_{\phi}(s_i)\|_2 \leq U_\pi$.
The upper bound of the gradient of the policy network is
$\max_i\|\nabla_\phi \pi_{\phi}(s_i)\|_2\leq U_{\nabla \pi}$.

Then, Equation~\ref{eq: policy_gradient} can be bound:
\begin{equation}
    \|\nabla_\phi \mathcal L_{\pi}^i(\phi)\|_2 \leq (U_{\nabla a}/\alpha+2U_a U_\pi)U_{\nabla \pi}
\end{equation}

We can define two functions $l_Q(\mathbf{\beta}), l_\pi(\mathbf{\beta}): \mathbb{R}^{|\mathcal{D}|} \rightarrow \mathbb{R}$
\begin{equation}
\begin{aligned}
    l_Q(\mathbf{\beta}) &= -\|\sum_{i=1}^{{|\mathcal{D}|}} \beta_i\nabla_\theta \mathcal L_Q^i(\theta)-\nabla_\theta \mathcal L(\theta)\|_2 - \lambda\|\beta\|_2^2 \\
    l_\pi(\mathbf{\beta}) &= -\|\sum_{i=1}^{{|\mathcal{D}|}} \beta_i\nabla_\phi \mathcal L^i_\pi(\phi)-\nabla_\phi \mathcal L(\phi)\|_2 - \lambda\|\beta\|_2^2
\end{aligned}
\end{equation}

We assume $\beta$ is a $N$-sparse vector that is 0 on all but $N$ indices.
Then we can transform maximizing $F^Q_\lambda(\mathcal{S}), F^\pi_\lambda(\mathcal{S})$ into maximizing $l(\beta)-l(\mathbf{0})$:
\begin{equation}
\begin{aligned}
    \max_{\mathcal{S}:|\mathcal{S}| \leq N} F^Q_\lambda(\mathcal{S}) &\xleftrightarrow{} \max_{\substack{\beta:\beta_{S^c=0} \\|\mathcal{S}|\leq N}} l_Q(\mathbf{\beta})-l_Q(\mathbf{0}) \\
    \max_{\mathcal{S}:|\mathcal{S}| \leq N} F^\pi_\lambda(\mathcal{S}) &\xleftrightarrow{} \max_{\substack{\beta:\beta_{S^c=0} \\|\mathcal{S}|\leq N}} l_\pi(\mathbf{\beta})-l_\pi(\mathbf{0})
\end{aligned}
\end{equation}
where ${S^c}$ means the complementary set of $S$, and $\beta_{S^c}=0$ means $\beta$ is 0 on all but indices $i$ that $i \in S$. 
$l(\mathbf{0})$ means the value of $l(\cdot)$ when input is zero vector $\mathbf{0}$, it serves as a basic value.
Since $l_Q(\beta)\leq 0, l_\pi(\beta) \leq 0$,  we can easily find that the minimum eigenvalues of $-l_Q(\beta)$ and $-l_\pi(\beta)$ are both at least $\lambda$. 

Next, the maximum eigenvalues of $-l_Q(\beta)$ and $-l_\pi(\beta)$ are
\begin{equation}
\begin{aligned}
\Lambda_{\max}(-l_Q(\beta))&=
\lambda+\operatorname{Trace}\left(\left[\begin{array}{c}
\beta_1\nabla_\theta \mathcal L_Q^{1 \top}\left(\theta\right) \\
\beta_2\nabla_\theta \mathcal L_Q^{2 \top}\left(\theta\right) \\
\ldots\\
\beta_{|\mathcal{D}|}\nabla_\theta \mathcal L_Q^{|\mathcal{D}| \top}\left(\theta_t\right)
\end{array}\right]\left[\begin{array}{c}
\beta_1\nabla_\theta \mathcal L_Q^{1 \top}\left(\theta\right) \\
\beta_2\nabla_\theta \mathcal L_Q^{2 \top}\left(\theta\right) \\
\ldots \\
\beta_{|\mathcal{D}|}\nabla_\theta \mathcal L_Q^{|\mathcal{D}| \top}\left(\theta\right)
\end{array}\right]^{\top}\right)    \\
&=\lambda + \sum_{i=1}^{{|\mathcal{D}|}} \beta_i^2 \| \nabla_\theta \mathcal L_Q^{i}(\theta) \|^2\\ &\leq \lambda+4 N (U_\mathtt{TD}U_{\nabla Q})^2 \\
\Lambda_{\max}(-l_\pi(\beta))&\leq \lambda + N(U_{\nabla a}/\alpha+2U_a U_\pi)^2 U_{\nabla\pi}^2
\end{aligned}
\end{equation}

Following the Theorem~1 in \cite{elenberg2018restricted}, we can derive that $F_\lambda^Q(\mathcal{S})$ is $\delta$-weakly submodular with $\delta \geq \frac{\lambda}{\lambda+4 N (U_\mathtt{TD}U_{\nabla Q})^2}$. 
And $F_\lambda^\pi(\mathcal{S})$ is $\delta$-weakly submodular with $\delta \geq \frac{\lambda}{\lambda + N(U_{\nabla a}/\alpha+2U_a U_\pi)^2 U_{\nabla\pi}^2}$.
\end{proof}

\subsection{Upper Bound of Residual Error}
\label{appendix: cluster theory}

\cluster*

\begin{proof}
The residual error is no larger than the special case where all $w_i$ are $|\mathcal{D}|/|\mathcal{S}|$:
\begin{align}
\operatorname{Err}\left(\vw, \mathcal{S}, \mathcal L, \theta\right)\le\|\frac{|\mathcal D|}{|\mathcal S|}\sum_{i\in \mathcal S}\nabla_{\theta} \mathcal L^i\left(\theta\right) - \sum_{i\in \mathcal D}\nabla_{\theta} \mathcal L^i\left(\theta\right) \|_2. \nonumber
\end{align}
Using Jensen's inequality, we have
\begin{align}
\operatorname{Err}\left(\vw, \mathcal{S}, \mathcal L, \theta\right)\le\sum_{i\in \mathcal D} \|\nabla_{\theta} \mathcal L^i\left(\theta\right) - \frac{1}{|\mathcal S|}\sum_{s\in\mathcal S}\nabla_{\theta}\mathcal L^s\left(\theta\right) \|_2. \nonumber
\end{align}
% In our formulation, samples are selected in mini-batches, and the gradient of sample $i$ from mini-batch $\mathcal B_{j_i}^{k_i}$ is approximated by those of $\mathcal S_{j_i}^{k_i}$ (Eq.~\ref{eq: batch gradient approx}). Therefore, we have
% \begin{align}
%     \operatorname{Err}\left(\vw, \mathcal{S}, \mathcal L, \theta\right)\le\sum_{i\in \mathcal D}\|\nabla_{\theta} \mathcal L^i\left(\theta\right) - \frac{1}{|\mathcal S_{j_i}^{k_i}|}\sum_{s\in\mathcal S_{j_i}^{k_i}}\nabla_{\theta}\mathcal L^s\left(\theta\right) \|_2.\nonumber
% \end{align}
According to the monotone property of submodular functions, adding more samples to $S^k$ reduces the residual error. We assume $S^k$ starts with the cluster center $\{c_k\}$, it follows that
    \begin{align}
    \operatorname{Err}&\left(\vw, \mathcal{S}, \mathcal L, \theta\right) \le \sum_{i\in \mathcal D}\|\nabla_{\theta} \mathcal L^i\left(\theta\right) - \nabla_{\theta}\mathcal L^{c_k}\left(\theta\right) \|_2\nonumber\\
    =& \sum_{i\in\mathcal D} \min_{c\in \mathcal C}\|\nabla_{\theta} \mathcal L^i\left(\theta\right) - \nabla_{\theta} \mathcal L^c\left(\theta\right) \|_2.\label{equ:cluster_obj}
\end{align}
Eq.~\ref{equ:cluster_obj} is exactly the optimization objective typical of the clustering problem.
\end{proof}

\subsection{Convergence Analysis}
\label{appendix: convergence}

% \begin{theorem}
%     Let $\theta^*$ denote the optimal $Q^{\pi_\mathcal{S}}$ parameters, $\theta_t$ the parameters after $t$ training steps. The TD loss satisfies 
%     \begin{align}
%         \min_{t=1:G}\mathcal{L}(\theta_t)\leq \mathcal{L}(\theta^*) + \frac{D\sigma}{\sqrt{G}} + \frac{D}{G}\sum_{t=1}^{G-1}\varepsilon.\nonumber
%     \end{align}
%     Here 
%     $\mathcal{L}(\theta)=\sum_{i \in \mathcal{D}}\mathcal L_{\mathtt{TD}}(s_{i}, a_{i}, r_i, s'_{i}, \theta)$, $G$ is the total training steps, $\varepsilon=\operatorname{Err}\left(\vw, \mathcal{S}, \mathcal L, \theta_t\right)$ is the gradient approximation errors that are bounded in Corollary~\ref{thm:c_bound}.
% \end{theorem}
\convergence*

\begin{proof}
    From the definition of Gradient Descent, we have:

    \begin{align}
    \nabla_{\theta} \mathcal L_{\rdcshort}(\theta_t)^T(\theta_t - \theta^*) &= \frac{1}{\alpha_t}(\theta_t-\theta_{t+1})^T(\theta_t-\theta^*) \\
    \nabla_{\theta} \mathcal L_{\rdcshort}(\theta_t)^T(\theta_t - \theta^*) &= \frac{1}{2\alpha_t}\left(\|\theta_t-\theta_{t+1}\|^2 + \|\theta_t-\theta^*\|^2 - \|\theta_{t+1}-\theta^*\|^2\right) \\
    \nabla_{\theta} \mathcal L_{\rdcshort}(\theta_t)^T(\theta_t - \theta^*) &= \frac{1}{2\alpha_t}\left(\|\alpha_t \nabla_{\theta} \mathcal L_{\rdcshort}(\theta_t)\|^2 + \|\theta_t-\theta^*\|^2 - \|\theta_{t+1}-\theta^*\|^2\right)
    \end{align}

    Then, we rewrite the function $\nabla_{\theta} \mathcal L_{\rdcshort}(\theta_t)^T(\theta_t - \theta^*)$ as follows:

    \begin{align}
        \nabla_{\theta} \mathcal L_{\rdcshort}(\theta_t)^T(\theta_t - \theta^*) = \nabla_{\theta} \mathcal L_{\rdcshort}(\theta_t)^T(\theta_t - \theta^*) - \nabla_{\theta} \mathcal{L}(\theta_t)^T(\theta_t - \theta^*) + \nabla_{\theta} \mathcal{L}(\theta_t)^T(\theta_t - \theta^*)
    \end{align}

    Combining the above equations we have:

    \begin{align}
    \nabla_{\theta} \mathcal L_{\rdcshort}(\theta_t)^T(\theta_t - \theta^*) - \nabla_{\theta} \mathcal{L}(\theta_t)^T(\theta_t - \theta^*) + \nabla_{\theta} \mathcal{L}(\theta_t)^T(\theta_t - \theta^*) = \\
    \frac{1}{2\alpha_t}\left(\|\alpha_t\nabla_{\theta} \mathcal L_{\rdcshort}(\theta_t)\|^2 + \|\theta_t - \theta^*\|^2 - \|\theta_{t+1} - \theta^*\|^2\right)
    \end{align}

    \begin{align}
    \nabla_{\theta} \mathcal{L}(\theta_t)^T(\theta_t - \theta^*) =
    \frac{1}{2\alpha_t}\left(\|\alpha_t\nabla_{\theta} \mathcal L_{\rdcshort}(\theta_t)\|^2 + \|\theta_t - \theta^*\|^2 - \|\theta_{t+1} - \theta^*\|^2\right) - \\ (\nabla_{\theta} \mathcal L_{\rdcshort}(\theta_t) - \nabla_{\theta} \mathcal{L}(\theta_t))^T(\theta_t - \theta^*)
    \end{align}

    Summing up the above equation for different value of $t\in [0,G-1]$ and the learning rate $\alpha_t$ is a constant $\alpha$, then we have:

    \begin{align}
    \sum_{t=0}^{G-1} \nabla_{\theta} \mathcal{L}(\theta_t)^T(\theta_t - \theta^*) = \frac{1}{2\alpha} \|\theta_0 - \theta^*\|^2 - \|\theta_G - \theta^*\|^2 + \sum_{t=0}^{G-1}\left(\frac{1}{2\alpha}\|\alpha\nabla_{\theta} \mathcal L_{\rdcshort}(\theta_t)\|^2\right) \\
    + \sum_{t=0}^{G-1}\left((\nabla_{\theta} \mathcal L_{\rdcshort}(\theta_t) - \nabla_{\theta} \mathcal{L}(\theta_t) )^T(\theta_t - \theta^*)\right)
    \end{align}

    Since $\|\theta_G - \theta^*\|^2 \geq 0$, we have:

    \begin{align}
        \sum_{t=0}^{G-1} \nabla_{\theta} \mathcal{L}(\theta_t)^T(\theta_t - \theta^*) \leq \frac{1}{2\alpha} \|\theta_0 - \theta^*\|^2 + \sum_{t=0}^{G-1}\left(\frac{1}{2\alpha}\|\alpha\nabla_{\theta} \mathcal L_{\rdcshort}(\theta_t)\|^2\right) \\
        + \sum_{t=0}^{G-1}\left((\nabla_{\theta} \mathcal L_{\rdcshort}(\theta_t) - \nabla_{\theta} \mathcal{L}(\theta_t) )^T(\theta_t - \theta^*)\right)
        \label{eq: sum}
    \end{align}

From the convexity of function $\mathcal{L}(\theta)$, we have:

\begin{align}
    \mathcal{L}(\theta_t) - \mathcal{L}(\theta^*) \leq \nabla_{\theta} \mathcal{L}(\theta_t)^T(\theta_t - \theta^*)
    \label{eq: convexity}
\end{align}

Combining the Equation~\ref{eq: sum} and Equation~\ref{eq: convexity}, we have:

\begin{align}
    \sum_{t=0}^{G-1}\mathcal{L}(\theta_t) - \mathcal{L}(\theta^*) \leq \frac{1}{2\alpha} \|\theta_0 - \theta^*\|^2 + \sum_{t=0}^{G-1}\left(\frac{1}{2\alpha}\|\alpha\nabla_{\theta} \mathcal L_{\rdcshort}(\theta_t)\|^2\right) \\
    + \sum_{t=0}^{G-1}\left((\nabla_{\theta} \mathcal L_{\rdcshort}(\theta_t) - \nabla_{\theta} \mathcal{L}(\theta_t))^T(\theta_t - \theta^*)\right)
\end{align}

We assume that $\|\theta - \theta^*\|\leq D$.
Since $\| \nabla \mathcal L(\theta) \| \leq \sigma$, we have:

\begin{align}
    \sum_{t=0}^{G-1}\mathcal{L}(\theta_t) - \mathcal{L}(\theta^*) \leq \frac{D^2}{2\alpha} + \frac{G\alpha\sigma^2}{2}
    + \sum_{t=0}^{G-1}D(\|\nabla_{\theta} \mathcal L_{\rdcshort}(\theta_t) - \nabla_{\theta} \mathcal{L}(\theta_t)\|)
\end{align}

Then:

\begin{align}
    \frac{\sum_{t=0}^{G-1}\mathcal{L}(\theta_t) - \mathcal{L}(\theta^*)}{G} \leq \frac{D^2}{2\alpha G} + \frac{\alpha\sigma^2}{2}
    + \sum_{t=0}^{G-1}\frac{D}{G}(\|\nabla_{\theta} \mathcal L_{\rdcshort}(\theta_t) - \nabla_{\theta} \mathcal{L}(\theta_t)\|)
\end{align}

Since $\min(\mathcal{L}(\theta_t) - \mathcal{L}(\theta^*))\leq \frac{\sum_{t=0}^{G-1}\mathcal{L}(\theta_t) - \mathcal{L}(\theta^*)}{G}$, we have:

\begin{align}
    \min(\mathcal{L}(\theta_t) - \mathcal{L}(\theta^*))\leq \frac{D^2}{2\alpha G} + \frac{\alpha\sigma^2}{2}
    + \sum_{t=0}^{G-1}\frac{D}{G}(\|\nabla_{\theta} \mathcal L_{\rdcshort}(\theta_t) - \nabla_{\theta} \mathcal{L}(\theta_t)\|)
\end{align}

We adopt $\varepsilon$ to denote $\|\nabla_{\theta} \mathcal L_{\rdcshort}(\theta_t) - \nabla_{\theta} \mathcal{L}(\theta_t)\|$, then we have:

\begin{align}
    \min(\mathcal{L}(\theta_t) - \mathcal{L}(\theta^*))\leq \frac{D^2}{2\alpha G} + \frac{\alpha\sigma^2}{2}
    + \sum_{t=0}^{G-1}\frac{D}{G}\varepsilon
\end{align}
     
\end{proof}


\begin{theorem}\label{thm:monotone}
    The training loss on original dataset always monotonically decreases with every training epoch $t$, $\mathcal{L}(\theta_{t+1}) \leq \mathcal{L}(\theta_t)$ if it satisfies the condition that $\nabla_{\theta} \mathcal{L}(\theta_t)^T\nabla_{\theta} \mathcal L_{\rdcshort}(\theta_t) \geq 0$ for $0\leq t \leq G$ and the learning rate $\alpha \leq \min_{t} \frac{2}{L}\frac{\nabla_{\theta} \mathcal{L}(\theta_t)^T\nabla_{\theta} \mathcal L_{\rdcshort}(\theta_t)}{\nabla_{\theta} \mathcal L_{\rdcshort}(\theta_t)^T\nabla_{\theta} \mathcal L_{\rdcshort}(\theta_t)}$.
\end{theorem}

% Let $L_{T}(\theta_{t}; \mathcal{D})$ denote the training loss on the original dataset $\mathcal{D}$.
% Then, we suppose the training loss $L_T$ is Lipschitz smooth with constant $\mathcal{L}$, and the gradient is bounded by $\sigma_T$:
% $\|\nabla L_{T}(\theta_{t+1}) - \nabla L_{T}(\theta_{t})\| \leq \mathcal{L}\|\theta_{t+1} - \theta_{t}\|$ and $\| \nabla L_{T}(\theta_{t}) \| \leq \sigma_T$.
% Similarly, we let $L_{T}(\theta_{t}; \mathcal{S})$ denote the training loss on the reduced subset $\mathcal{S}$ with the same smooth assumption.

\begin{proof}
    Since the training loss $\mathcal{L}(\theta)$ is lipschitz smooth, we have:
\begin{align}
    \mathcal{L}(\theta_{t+1}) & \leq \mathcal{L}(\theta_t) + \nabla_{\theta} \mathcal{L}(\theta_t)^T\Delta \theta + \frac{L}{2}\|\Delta\theta\|^2, \\
    &\text{where} \qquad \Delta\theta=\theta_{t+1} - \theta_{t}.
\end{align}

Since, we are using SGD to optimize the reduced subset training loss $\mathcal L_{\rdcshort}(\theta_t)$ model parameters.
The update equation is:

\begin{align}
    \theta_{t+1} = \theta_{t} - \alpha \nabla_{\theta} \mathcal L_{\rdcshort}(\theta_t)
\end{align}

Combining the above two equations, we have:
\begin{align}
    \mathcal{L}(\theta_{t+1}) \leq \mathcal{L}(\theta_t) + \nabla_{\theta} \mathcal{L}(\theta_t)^T(- \alpha \nabla_{\theta} \mathcal L_{\rdcshort}(\theta_t)) + \frac{L}{2}\|- \alpha \nabla_{\theta} \mathcal L_{\rdcshort}(\theta_t)\|^2
\end{align}

Next, we have:

\begin{align}
    \mathcal{L}(\theta_{t+1}) - \mathcal{L}(\theta_t) \leq \nabla_{\theta} \mathcal{L}(\theta_t)^T(- \alpha \nabla_{\theta} \mathcal L_{\rdcshort}(\theta_t)) + \frac{L}{2}\|- \alpha \nabla_{\theta} \mathcal L_{\rdcshort}(\theta_t)\|^2
\end{align}

From the above equation, we have:

\begin{align}
    \mathcal{L}(\theta_{t+1}) \leq \mathcal{L}(\theta_{t}), \quad  \text{if} \quad \nabla_{\theta} \mathcal{L}(\theta_{t})^T\nabla_{\theta} \mathcal L_{\rdcshort}(\theta_t) 
    - \frac{\alpha L}{2}\|\nabla_{\theta} \mathcal L_{\rdcshort}(\theta_t)\|^2 \geq 0
\end{align}

Since $\|\nabla_{\theta} \mathcal L_{\rdcshort}(\theta_t)\|^2\geq 0$, we will have the necessary condition $\nabla_{\theta} \mathcal{L}(\theta_{t})^T\nabla_{\theta} \mathcal L_{\rdcshort}(\theta_t) \geq 0$.
Next, we rewrite the above condition as follows:

\begin{align}
    \nabla_{\theta} \mathcal{L}(\theta_{t})^T\nabla_{\theta} \mathcal L_{\rdcshort}(\theta_t) \geq \frac{\alpha L}{2}\|\nabla_{\theta} \mathcal L_{\rdcshort}(\theta_t)\|^2
\end{align}

Therefore, the necessary condition for the learning rate $\alpha$ is:

\begin{align}
    \alpha \leq \frac{2}{L}\frac{\nabla_{\theta} \mathcal{L}(\theta_t)^T\nabla_{\theta} \mathcal L_{\rdcshort}(\theta_t)}{\nabla_{\theta} \mathcal L_{\rdcshort}(\theta_t)^T\nabla_{\theta} \mathcal L_{\rdcshort}(\theta_t)}
\end{align}

Since the above condition needs to be true for all values for $t$, we have the following conditions for the learning rate:

\begin{align}
    \alpha \leq \min_{t} \frac{2}{L}\frac{\nabla_{\theta} \mathcal{L}(\theta_t)^T\nabla_{\theta} \mathcal L_{\rdcshort}(\theta_t)}{\nabla_{\theta} \mathcal L_{\rdcshort}(\theta_t)^T\nabla_{\theta} \mathcal L_{\rdcshort}(\theta_t)}
\end{align}

\end{proof}
\clearpage
\section{Ablation Study}
\label{sec: ablation}
To study the contribution of each component in our learning framework, we conduct the following ablation study. 
\nameq: We replace the empirical returns used to update Q functions with the standard target Q function in the TD loss function. 
\namei: We set the number of data selection rounds to 1 and study the function of multi-round data selection.
The experimental results in Figure~\ref{fig: modular ablation}show that removing any of these two modules will worsen the performance of \name. In case like $\texttt{walker2d-medium}$, ablation \namei~even decrease the performance by over 80\%, and ablation \nameq~results in a 95\% performance drop in $\texttt{walker2d-expert}$. Furthermore, we also find that in the $\texttt{halfcheetah}$ tasks, the impact of removing the two modules is relatively small. This result can be attributable to the fact that this task has a limited state space, and we can directly apply OMP to the entire dataset and identify important and diverse data.

\begin{figure}[H]
    \centering
    \subfigure{\includegraphics[scale=0.27]{ablation_moduler1.pdf}}
    \hspace{0.3cm}\subfigure{\includegraphics[scale=0.27]{ablation_moduler2.pdf}}
    \caption{Ablation results on D4RL~(Hard) tasks with the normalized score metric.}
    \label{fig: modular ablation}
\end{figure}


\section{Computational Complexity}
\label{appendix: computation complexity}
We report the computational overhead of \name~on various datasets. 
All experiments are conducted on the same computational device (GeForce RTX 3090 GPU). 
The results in the following Table indicate that even on datasets containing millions of data points, the computational overhead remains low. 
This low computational complexity can be attributed to the trajectory-based selection technique in Sec.~\ref{sec: offline omp}~(II) and the regularized constraint technique in Sec.~\ref{sec:method:outer}, making our method easily scalable to large-scale datasets. 

\begin{table*}[h]
    \centering
    \begin{tabular}{c|cc}
    \toprule
    Env & Data Number & \name \\
    \midrule
    Hopper-medium-v0 & 999981 & 8m \\
    Walker2d-medium-v0 & 999874 & 8m \\
    Halfcheetah-medium-v0 & 998999 & 8m \\
    Hopper-expert-v0 & 999034 & 8m \\
    Walker2d-expert-v0 & 999304 & 8m \\
    Halfcheetah-expert-v0 & 998999 & 8m\\
    Hopper-medium-expert-v0 & 1199953 & 8m\\
    Walker2d-medium-expert-v0 & 1999179  & 13m\\
    Halfcheetah-medium-expert-v0 & 1997998 & 14m\\
    Hopper-medium-replay-v0 & 200918 & 3m\\
    Walker2d-medium-replay-v0 & 100929  & 3m\\
    Halfcheetah-medium-replay-v0 & 100899 & 3m\\
    \bottomrule
    \end{tabular}
    \label{tab: cc}
    \caption{The computational complexity associated with \name~in various datasets. $m$ represents minutes.} 
\end{table*}

% \subsection{Trajectory-based selection}
% \label{appendix: trajectory}

% Experimental results in Figure~\ref{fig: d4rl topbc} show that \name~maintains its superiority in this setting with suboptimal (e.g., \texttt{medium}) datasets. This evidence suggests that \name~provides a valuable strategy for selecting data conducive to effective training under conditions of compromised data quality.

% \subsection{Generalizability of \name~ to other domains}
% \label{appendix: other domain}
% We evaluate our algorithm on robotic manipulation (Adroit) and sparse reward (Antmaze) tasks. 
% The experimental results in Table~\ref{tab: other domain} indicate that in sparse reward tasks, \name~ achieves comparable performance close to that on the full dataset with only 20\% of the data. In the robotic manipulation tasks, \name~ requires even less data.

% \begin{table}[h]
%     \centering
%     \caption{Experimental results of \name~ with various dataset sizes ($x\%$) in Antmaze and Adroit tasks. 
%     Highlighted is the performance comparable to training TD3+BC with the complete dataset. } 
%     \label{tab: other domain}
%     \begin{tabular}{c|cccc}
%     \toprule
%         Env & 10\% & 20\% & 30\% & All Data \\
%         \midrule
%         Antmaze-umaze-v0 & 70.2$\pm$3.6 & 75.1$\pm$2.5 & 84.7$\pm$3.3 & 87.5$\pm$1.3 \\
%         Antmaze-umaze-diverse-v0 & 44.7$\pm$2.7 & 46.3$\pm$1.9 & 47.7$\pm$2.2 & 62.2$\pm$2.0 \\
%         Antmaze-medium-play-v0 & 2.1$\pm$1.3 & 59.3$\pm$1.6 & 60.3$\pm$2.9 &	71.2$\pm$2.2 \\
%         Antmaze-medium-diverse-v0 &	7.3$\pm$3.1 & 43.6$\pm$2.7 & 64.9$\pm$3.8 & 70.0$\pm$1.6 \\
%         % Antmaze-large-play-v0 &	2.0$\pm$0.5 & 3.7$\pm$0.7 & 16.0$\pm$3.5 & 39.6$\pm$3.6 \\
%         % Antmaze-large-diverse-v0 & 2.4$\pm$1.0 & 16.0$\pm$3.6 & 20.5$\pm$3.7 & 47.5$\pm$1.1 \\
%         Pen-expert-v0 & 121.8$\pm$1.6 & 121.5$\pm$1.0 & 119.3$\pm$2.2 & 136.7$\pm$2.5 \\
%         Hammer-expert-v0 & 127.0$\pm$1.3 & 126.9$\pm$1.6 & 119.4$\pm$2.1 & 121.5$\pm$1.1 \\
%         Relocate-expert-v0 & 103.5$\pm$3.7 & 106.8$\pm$2.7 & 106.3$\pm$2.3 & 108.8$\pm$3.5\\ 
%         Door-expert-v0 & 105.1$\pm$2.7 & 105.2$\pm$1.9 & 105.4$\pm$3.6 & 106.3$\pm$2.9\\
%     \bottomrule
%     \end{tabular}
% \end{table}

% \clearpage
% \subsection{Generalizability of \name~ to other algorithms}
% \label{appendix: other algorithm}

% We add IQL~\cite{kostrikov2021offline} as a baseline and apply \name~to IQL by using the gradient of the training loss of the V-function in IQL as the criterion. 
% The experimental results in Table~\ref{tab: other algorithm} demonstrate both \name$_{\rm IQL}$ and \name$_{\rm TD3+BC}$ can achieve performance close to Complete Dataset (Best) with a small amount of data.

% \begin{table}[h]
%     \centering
%     \caption{Experimental results of applying \name~ to IQL.} 
%     \label{tab: other algorithm}
%     \begin{tabular}{c|cccc}
%     \toprule
%     Env & \name$_{\rm IQL}$ & \name$_{\rm TD3+BC}$ & IQL~(All Data) & TD3+BC~(All Data) \\
%     \midrule
%     Hopper-medium-v0 & 91.7$\pm$1.3 & 93.3$\pm$2.5 & 98.7$\pm$1.2 & 99.5$\pm$1.0\\
%     Walker2d-medium-v0 & 63.2$\pm$2.3 & 64.3$\pm$2.2 & 70.5$\pm$1.7 & 79.7$\pm$1.8\\
%     Halfcheetah-medium-v0 & 32.5$\pm$0.7 & 33.0$\pm$0.8 & 40.2$\pm$0.5 & 42.8$\pm$0.3\\
%     Hopper-medium-expert-v0 & 106.7$\pm$0.3 & 103.0$\pm$0.5 & 112.0$\pm$1.0 & 112.2$\pm$0.2\\
%     Walker2d-medium-expert-v0 & 84.0$\pm$8.1 & 77.0$\pm$8.6 & 105.0$\pm$4.7 & 101.1$\pm$9.3\\
%     Halfcheetah-medium-expert-v0 & 78.6$\pm$3.2 & 80.5$\pm$6.0 & 92.1$\pm$4.6 & 97.9$\pm$4.4\\
%     Hopper-expert-v0 & 112.2$\pm$0.3 & 108.6$\pm$0.8 & 112.2$\pm$0.6 & 112.2$\pm$0.2\\
%     Walker2d-expert-v0 & 83.0$\pm$4.5 & 83.8$\pm$4.2 & 106.8$\pm$2.6 & 105.7$\pm$2.7\\
%     Halfcheetah-expert-v0 & 78.9$\pm$1.7 & 85.6$\pm$1.2 & 107.0$\pm$2.1 & 105.7$\pm$1.9\\
%     \bottomrule
%     \end{tabular}
% \end{table}

% \clearpage

% \subsection{Generalizability of subset selecting by~\name}
% \label{appendix: tb3bc2iql}
% To test the generalizability of the dataset selected by~\name, we select subset by applying~\name~to TD3+BC.
% Then we evaluate the performance of IQL on the selected subset. 
% The experimental results in Table~\ref{tab: td3bc2iql} demonstrate that the selected subset based on TD3+BC is effectively applicable to IQL.
%  Across all tasks, the subset size is 10\% of the entire dataset.

% \begin{table}[h]
%     \centering
%     \caption{The performance of IQL on the subset selected based on TD3+BC.}
%     \label{tab: td3bc2iql}
%     \begin{tabular}{c|cc}
%     \toprule
%     Env & \name$_{\rm TD3+BC\rightarrow IQL}$ & IQL (All Data) \\
%     \midrule
%     Hopper-medium-v0 & 88.9$\pm$8.7 & 98.7$\pm$1.2\\
%     Walker2d-medium-v0 & 59.1$\pm$6.9 & 70.5$\pm$1.7\\
%     Halfcheetah-medium-v0 & 37.0$\pm$0.1 & 40.2$\pm$0.5 \\
%     Hopper-medium-expert-v0 & 100.5$\pm$1.6 & 112.0$\pm$1.0\\
%     Walker2d-medium-expert-v0 & 82.1$\pm$4.8 & 105.0$\pm$4.7 \\
%     Halfcheetah-medium-expert-v0 & 50.4$\pm$0.1 & 92.1$\pm$4.6 \\
%     Hopper-expert-v0 & 110.9$\pm$0.6 & 112.2$\pm$0.6\\
%     Walker2d-expert-v0 & 83.5$\pm$4.2 & 106.8$\pm$2.6\\
%     Halfcheetah-expert-v0 & 92.5$\pm$1.4 & 107.0$\pm$2.1\\
%     \bottomrule
%     \end{tabular}
% \end{table}

% \clearpage
% \subsection{Ablation study for cluster number}
% \label{appendix: cluster number}
% We evaluate the performance of \name~with various cluster numbers. The experimental results in Table~\ref{tab: cluster number} show that the suitable cluster number is between 25 and 50. Too few clusters (e.g., less than 5) are detrimental to the algorithm.

% \begin{table}[h]
%     \centering
%     \caption{Ablation study with the cluster number.} 
%     \label{tab: cluster number}
%     \begin{tabular}{c|cccccc}
%     \toprule
%     Cluster Number & 1 & 5 & 15 & 25 & 50\\
%     \midrule
%     Hopper-medium-v0 & 47.6$\pm$1.6 & 81.7$\pm$3.0 & 96.2$\pm$2.0 & 99.1$\pm$3.3 & 92.6$\pm$3.0 \\
%     Walker2d-medium-v0 & 9.5$\pm$1.1 & 5.9$\pm$3.6 & 32.2$\pm$2.4 & 64.1$\pm$1.9 & 57.9$\pm$3.6 \\
%     Halfcheetah-medium-v0 & 40.4$\pm$0.2 & 41.2$\pm$0.7 & 41.3$\pm$0.4 & 41.4$\pm$0.2 & 40.9$\pm$0.1\\
%     Hopper-expert-v0 & 97.5$\pm$1.9 & 112.2$\pm$1.4 & 111.3$\pm$2.1 & 111.5$\pm$1.6 & 110.6$\pm$1.9 \\
%     Walker2d-expert-v0 & 76.9$\pm$3.2 & 80.8$\pm$5.2 & 81.7$\pm$3.4 & 80.8$\pm$2.8 & 84.4$\pm$5.0 \\
%     Halfcheetah-expert-v0 & 84.3$\pm$2.7 & 82.9$\pm$2.8 & 83.0$\pm$3.2 & 82.3$\pm$1.9 & 84.3$\pm$3.5 \\
%     Hopper-medium-expert-v0 & 112.0$\pm$0.7 & 112.1$\pm$0.2 & 112.1$\pm$0.6 & 112.3$\pm$0.3 & 112.4$\pm$0.3 \\
%     Walker2d-medium-expert-v0 & 78.6$\pm$3.6 & 82.5$\pm$3.2 & 85.0$\pm$2.8 & 84.6$\pm$2.9 & 85.4$\pm$5.3 \\
%     Halfcheetah-medium-expert-v0 & 63.5$\pm$3.3 & 66.7$\pm$3.9 & 84.1$\pm$4.2 & 85.0$\pm$5.2 & 86.2$\pm$5.0 \\
%     \bottomrule
%     \end{tabular}
% \end{table}

% \subsection{Ablation study for approximation bounds}
% \label{appendix: approx bound} 
% We evaluate the performance of \name~with various approximation bounds (from 0.0001 to 0.05). 
% A smaller approximation bound represents a larger reduced dataset. The experimental results in Table~\ref{tab: approx bound} show that similar to the ablation of the size of the reduced dataset, \name~requires only a 0.01 approximation bound to obtain good performance.

% \begin{table}[h]
%     \centering
%     \caption{Ablation study with the approximation bounds.} 
%     \label{tab: approx bound}
%     \begin{tabular}{c|ccccc}
%     \toprule
%     Approximation Bounds & 0.0001 & 0.001 & 0.01 & 0.05 & All Data\\
%     \midrule
%     Hopper-medium-v0 & 97.9$\pm$1.3 & 94.6$\pm$0.8 & 92.2$\pm$1.1 & 31.9$\pm$1.9 & 99.5$\pm$1.0 \\
%     Walker2d-medium-v0 & 75.7$\pm$0.9 & 70.5$\pm$2.2 & 36.3$\pm$1.6 & 1.3$\pm$0.5 & 79.7$\pm$1.8\\
%     Halfcheetah-medium-v0 & 42.0$\pm$0.8 & 41.2$\pm$0.7 & 40.7$\pm$0.5 & 30.6$\pm$0.9 & 42.8$\pm$0.3 \\
%     Hopper-expert-v0 & 112.3$\pm$0.1 & 112.5$\pm$0.3 & 111.1$\pm$0.2 & 26.7$\pm$0.5 & 112.2$\pm$0.2\\
%     Walker2d-expert-v0 & 102.9$\pm$2.5 & 98.9$\pm$2.2 & 79.4$\pm$1.6 & 0.6$\pm$0.1 & 105.7$\pm$2.7\\
%     Halfcheetah-expert-v0 & 102.9$\pm$1.1 & 98.9$\pm$1.4 & 70.9$\pm$1.9 & 1.2$\pm$0.2 & 105.7$\pm$1.9\\
%     Hopper-medium-expert-v0 & 112.5$\pm$0.3 & 112.5$\pm$0.8 & 110.2$\pm$0.5 & 6.4$\pm$0.3 & 112.2$\pm$0.2\\
%     Walker2d-medium-expert-v0 & 101.2$\pm$5.7 & 98.8$\pm$7.4 & 82.3$\pm$6.9 & 1.2$\pm$0.6 & 101.1$\pm$9.3\\
%     Halfcheetah-medium-expert-v0 & 95.1$\pm$5.6 &	89.1$\pm$3.4 & 76.8$\pm$4.7 & 1.3$\pm$0.3 & 97.9$\pm$4.4\\
%     \bottomrule
%     \end{tabular}
% \end{table}
\clearpage
\subsection{Visualization Results}
\label{appendix: visual}
We visualize the selected data of ReDOR on various tasks based on the same method in Section~\ref{sec: exp}.


\begin{figure*}[ht]
    \centering
    \subfigure{\includegraphics[scale=0.4]{visual/hopper-medium-v0-on-iteration-0.pdf}}
    \caption{Visualization of selected data on hopper-medium-v0.}
\end{figure*}

\begin{figure*}[ht]
    \centering
    \subfigure{\includegraphics[scale=0.4]{visual/hopper-medium-expert-v0-on-iteration-0.pdf}}
    \caption{Visualization of selected data on hopper-medium-expert-v0.}
\end{figure*}

\begin{figure*}[ht]
    \centering
    \subfigure{\includegraphics[scale=0.4]{visual/hopper-expert-v0-on-iteration-0.pdf}}
    \caption{Visualization of selected data on hopper-expert-v0.}
\end{figure*}

\begin{figure*}[ht]
    \centering
    \subfigure{\includegraphics[scale=0.4]{visual/walker2d-medium-expert-v0-on-iteration-0.pdf}}
    \caption{Visualization of selected data on walker2d-medium-expert-v0.}
\end{figure*}

\begin{figure*}[ht]
    \centering
    \subfigure{\includegraphics[scale=0.4]{visual/walker2d-expert-v0-on-iteration-0.pdf}}
    \caption{Visualization of selected data on walker2d-expert-v0.}
\end{figure*}

\begin{figure*}[ht]
    \centering
    \subfigure{\includegraphics[scale=0.4]{visual/halfcheetah-medium-v0-on-iteration-0.pdf}}
    \caption{Visualization of selected data on halfcheetah-medium-v0.}
\end{figure*}

\begin{figure*}[ht]
    \centering
    \subfigure{\includegraphics[scale=0.4]{visual/halfcheetah-medium-expert-v0-on-iteration-0.pdf}}
    \caption{Visualization of selected data on halfcheetah-medium-expert-v0.}
\end{figure*}

\begin{figure*}[ht]
    \centering
    \subfigure{\includegraphics[scale=0.4]{visual/halfcheetah-expert-v0-on-iteration-0.pdf}}
    \caption{Visualization of selected data on halfcheetah-expert-v0.}
\end{figure*}
% \clearpage
% \section{ELBOW Experiments}
% \label{appendix: elbow}

% Determining the cluster number is crucial as it is used to solve the outer optimization issue in Equation~\ref{eq: gradient approx}.
% For this reason, we adopt the simple yet efficient elbow method to solve this issue.
% As shown in Figure~\ref{fig: elbow}, the appropriate number of clusters for these tasks tends to concentrate between 75 and 100.

% \begin{figure*}[h]
%     \centering
%     \subfigure{\includegraphics[scale=0.5]{elbow/hopper-medium-v0.pdf}}\subfigure{\includegraphics[scale=0.5]{elbow/hopper-medium-expert-v0.pdf}}\subfigure{\includegraphics[scale=0.5]{elbow/hopper-expert-v0.pdf}}\\\subfigure{\includegraphics[scale=0.5]{elbow/halfcheetah-medium-v0.pdf}}\subfigure{\includegraphics[scale=0.5]{elbow/halfcheetah-medium-expert-v0.pdf}}\subfigure{\includegraphics[scale=0.5]{elbow/halfcheetah-expert-v0.pdf}}
%     \\\subfigure{\includegraphics[scale=0.5]{elbow/walker2d-medium-v0.pdf}}\subfigure{\includegraphics[scale=0.5]{elbow/walker2d-expert-v0.pdf}}
%     \caption{The sum of squared errors of the data points~(named distance) with various cluster numbers.}
%     \label{fig: elbow}
% \end{figure*}

\clearpage
\section{Experimental Details}
\label{appendix: exp details}

\paragraph{Hyper-parameters.}
For the Mujoco tasks, we adopt the TD3+BC as the backbone of the offline algorithms.
For the Antmaze tasks, we adopt the IQL as the backbone of the offline algorithms.
We outline the hyper-parameters used by \name~ in Table~\ref{tab: parameters mujoco}.

\begin{table}[ht]
    \centering
    \begin{tabular}{ll}
    \toprule
    Hyperparameter & Value \\
    \midrule
    \hspace{0.3cm} Optimizer & Adam \\
    \hspace{0.3cm} Critic learning rate & 3e-4 \\
    \hspace{0.3cm} Actor learning rate & 3e-4 \\
    \hspace{0.3cm} Mini-batch size & 256 \\
    \hspace{0.3cm} Discount factor & 0.99 \\
    \hspace{0.3cm} Target update rate & 5e-3 \\
    \hspace{0.3cm} Policy noise & 0.2 \\
    \hspace{0.3cm} Policy noise clipping & (-0.5, 0.5) \\
    \hspace{0.3cm} TD3+BC regularized parameter & 2.5 \\
    \midrule
    Architecture & Value \\
    \midrule
    \hspace{0.3cm} Critic hidden dim & 256 \\
    \hspace{0.3cm} Critic hidden layers & 2 \\
    \hspace{0.3cm} Critic activation function & ReLU \\
    \hspace{0.3cm} Actor hidden dim & 256 \\
    \hspace{0.3cm} Actor hidden layers & 2 \\
    \hspace{0.3cm} Actor activation function & ReLU \\
    \midrule
    \name~Parameters & Value \\
    \midrule
    \hspace{0.3cm} Training rounds $T$ & 50 \\
    \hspace{0.3cm} $m$ & 50 \\
    \hspace{0.3cm} $\epsilon$ & 0.01 \\
    \bottomrule
    \end{tabular}
    \caption{Hyper-parameters sheet of ~\name}
    \label{tab: parameters mujoco}
\end{table}

% \section{Discussion of Limitations}
% \label{appendix: limitation}
% In this work, we consider using the TD loss gradient as the data subset selection criterion. 
% This is because if the gradients of the loss function used to train the $Q$ function are similar, the differences between $Q$ functions are also relatively small, thus making the policy on the data subset closer to that on the full dataset. 
% However, the theoretical framework does not directly present the relationship between the solution in the subset and the optimal solution for the original empirical Markov Decision Process (MDP).
% Nonetheless, our experimental results demonstrate the effectiveness of the proposed approach.

% \section{Broader Impacts}
% \label{appendix: impacts}
% This paper introduces a new perspective and pioneers a new path in the research of offline reinforcement learning. 
% This paper not only offers a reliable method for reducing dataset size, substantiated by sufficient proof, but also delineates the thresholds between adequate and inadequate dataset sizes through experiments, which provides considerable societal importance. 
% Our method can significantly reduce the burdens of training and storage by identifying a more compact subset of data. 
% % Conversely, in more societal domains where accumulating vast amounts of data is impractical, our approach offers guidance on the sufficient amount of data required. 
% This has the potential to expand the current boundaries of application in the field of offline reinforcement learning, making it more accessible and applicable in a broader range of societal contexts.

% % As for ethical aspects, to the best of our knowledge, the research presented in this paper does not directly engage with them. 
% % However, we acknowledge the importance of ethical considerations in machine learning research and strive to ensure that our work aligns with general ethical standards.
% \input{text/E-checklist}

\end{document}
