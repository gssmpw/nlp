
\documentclass{article} % For LaTeX2e
\usepackage{iclr2025_conference,times}

% Optional math commands from https://github.com/goodfeli/dlbook_notation.
%%%%% NEW MATH DEFINITIONS %%%%%

\usepackage{amsmath,amsfonts,bm}
\usepackage{derivative}
% Mark sections of captions for referring to divisions of figures
\newcommand{\figleft}{{\em (Left)}}
\newcommand{\figcenter}{{\em (Center)}}
\newcommand{\figright}{{\em (Right)}}
\newcommand{\figtop}{{\em (Top)}}
\newcommand{\figbottom}{{\em (Bottom)}}
\newcommand{\captiona}{{\em (a)}}
\newcommand{\captionb}{{\em (b)}}
\newcommand{\captionc}{{\em (c)}}
\newcommand{\captiond}{{\em (d)}}

% Highlight a newly defined term
\newcommand{\newterm}[1]{{\bf #1}}

% Derivative d 
\newcommand{\deriv}{{\mathrm{d}}}

% Figure reference, lower-case.
\def\figref#1{figure~\ref{#1}}
% Figure reference, capital. For start of sentence
\def\Figref#1{Figure~\ref{#1}}
\def\twofigref#1#2{figures \ref{#1} and \ref{#2}}
\def\quadfigref#1#2#3#4{figures \ref{#1}, \ref{#2}, \ref{#3} and \ref{#4}}
% Section reference, lower-case.
\def\secref#1{section~\ref{#1}}
% Section reference, capital.
\def\Secref#1{Section~\ref{#1}}
% Reference to two sections.
\def\twosecrefs#1#2{sections \ref{#1} and \ref{#2}}
% Reference to three sections.
\def\secrefs#1#2#3{sections \ref{#1}, \ref{#2} and \ref{#3}}
% Reference to an equation, lower-case.
\def\eqref#1{equation~\ref{#1}}
% Reference to an equation, upper case
\def\Eqref#1{Equation~\ref{#1}}
% A raw reference to an equation---avoid using if possible
\def\plaineqref#1{\ref{#1}}
% Reference to a chapter, lower-case.
\def\chapref#1{chapter~\ref{#1}}
% Reference to an equation, upper case.
\def\Chapref#1{Chapter~\ref{#1}}
% Reference to a range of chapters
\def\rangechapref#1#2{chapters\ref{#1}--\ref{#2}}
% Reference to an algorithm, lower-case.
\def\algref#1{algorithm~\ref{#1}}
% Reference to an algorithm, upper case.
\def\Algref#1{Algorithm~\ref{#1}}
\def\twoalgref#1#2{algorithms \ref{#1} and \ref{#2}}
\def\Twoalgref#1#2{Algorithms \ref{#1} and \ref{#2}}
% Reference to a part, lower case
\def\partref#1{part~\ref{#1}}
% Reference to a part, upper case
\def\Partref#1{Part~\ref{#1}}
\def\twopartref#1#2{parts \ref{#1} and \ref{#2}}

\def\ceil#1{\lceil #1 \rceil}
\def\floor#1{\lfloor #1 \rfloor}
\def\1{\bm{1}}
\newcommand{\train}{\mathcal{D}}
\newcommand{\valid}{\mathcal{D_{\mathrm{valid}}}}
\newcommand{\test}{\mathcal{D_{\mathrm{test}}}}

\def\eps{{\epsilon}}


% Random variables
\def\reta{{\textnormal{$\eta$}}}
\def\ra{{\textnormal{a}}}
\def\rb{{\textnormal{b}}}
\def\rc{{\textnormal{c}}}
\def\rd{{\textnormal{d}}}
\def\re{{\textnormal{e}}}
\def\rf{{\textnormal{f}}}
\def\rg{{\textnormal{g}}}
\def\rh{{\textnormal{h}}}
\def\ri{{\textnormal{i}}}
\def\rj{{\textnormal{j}}}
\def\rk{{\textnormal{k}}}
\def\rl{{\textnormal{l}}}
% rm is already a command, just don't name any random variables m
\def\rn{{\textnormal{n}}}
\def\ro{{\textnormal{o}}}
\def\rp{{\textnormal{p}}}
\def\rq{{\textnormal{q}}}
\def\rr{{\textnormal{r}}}
\def\rs{{\textnormal{s}}}
\def\rt{{\textnormal{t}}}
\def\ru{{\textnormal{u}}}
\def\rv{{\textnormal{v}}}
\def\rw{{\textnormal{w}}}
\def\rx{{\textnormal{x}}}
\def\ry{{\textnormal{y}}}
\def\rz{{\textnormal{z}}}

% Random vectors
\def\rvepsilon{{\mathbf{\epsilon}}}
\def\rvphi{{\mathbf{\phi}}}
\def\rvtheta{{\mathbf{\theta}}}
\def\rva{{\mathbf{a}}}
\def\rvb{{\mathbf{b}}}
\def\rvc{{\mathbf{c}}}
\def\rvd{{\mathbf{d}}}
\def\rve{{\mathbf{e}}}
\def\rvf{{\mathbf{f}}}
\def\rvg{{\mathbf{g}}}
\def\rvh{{\mathbf{h}}}
\def\rvu{{\mathbf{i}}}
\def\rvj{{\mathbf{j}}}
\def\rvk{{\mathbf{k}}}
\def\rvl{{\mathbf{l}}}
\def\rvm{{\mathbf{m}}}
\def\rvn{{\mathbf{n}}}
\def\rvo{{\mathbf{o}}}
\def\rvp{{\mathbf{p}}}
\def\rvq{{\mathbf{q}}}
\def\rvr{{\mathbf{r}}}
\def\rvs{{\mathbf{s}}}
\def\rvt{{\mathbf{t}}}
\def\rvu{{\mathbf{u}}}
\def\rvv{{\mathbf{v}}}
\def\rvw{{\mathbf{w}}}
\def\rvx{{\mathbf{x}}}
\def\rvy{{\mathbf{y}}}
\def\rvz{{\mathbf{z}}}

% Elements of random vectors
\def\erva{{\textnormal{a}}}
\def\ervb{{\textnormal{b}}}
\def\ervc{{\textnormal{c}}}
\def\ervd{{\textnormal{d}}}
\def\erve{{\textnormal{e}}}
\def\ervf{{\textnormal{f}}}
\def\ervg{{\textnormal{g}}}
\def\ervh{{\textnormal{h}}}
\def\ervi{{\textnormal{i}}}
\def\ervj{{\textnormal{j}}}
\def\ervk{{\textnormal{k}}}
\def\ervl{{\textnormal{l}}}
\def\ervm{{\textnormal{m}}}
\def\ervn{{\textnormal{n}}}
\def\ervo{{\textnormal{o}}}
\def\ervp{{\textnormal{p}}}
\def\ervq{{\textnormal{q}}}
\def\ervr{{\textnormal{r}}}
\def\ervs{{\textnormal{s}}}
\def\ervt{{\textnormal{t}}}
\def\ervu{{\textnormal{u}}}
\def\ervv{{\textnormal{v}}}
\def\ervw{{\textnormal{w}}}
\def\ervx{{\textnormal{x}}}
\def\ervy{{\textnormal{y}}}
\def\ervz{{\textnormal{z}}}

% Random matrices
\def\rmA{{\mathbf{A}}}
\def\rmB{{\mathbf{B}}}
\def\rmC{{\mathbf{C}}}
\def\rmD{{\mathbf{D}}}
\def\rmE{{\mathbf{E}}}
\def\rmF{{\mathbf{F}}}
\def\rmG{{\mathbf{G}}}
\def\rmH{{\mathbf{H}}}
\def\rmI{{\mathbf{I}}}
\def\rmJ{{\mathbf{J}}}
\def\rmK{{\mathbf{K}}}
\def\rmL{{\mathbf{L}}}
\def\rmM{{\mathbf{M}}}
\def\rmN{{\mathbf{N}}}
\def\rmO{{\mathbf{O}}}
\def\rmP{{\mathbf{P}}}
\def\rmQ{{\mathbf{Q}}}
\def\rmR{{\mathbf{R}}}
\def\rmS{{\mathbf{S}}}
\def\rmT{{\mathbf{T}}}
\def\rmU{{\mathbf{U}}}
\def\rmV{{\mathbf{V}}}
\def\rmW{{\mathbf{W}}}
\def\rmX{{\mathbf{X}}}
\def\rmY{{\mathbf{Y}}}
\def\rmZ{{\mathbf{Z}}}

% Elements of random matrices
\def\ermA{{\textnormal{A}}}
\def\ermB{{\textnormal{B}}}
\def\ermC{{\textnormal{C}}}
\def\ermD{{\textnormal{D}}}
\def\ermE{{\textnormal{E}}}
\def\ermF{{\textnormal{F}}}
\def\ermG{{\textnormal{G}}}
\def\ermH{{\textnormal{H}}}
\def\ermI{{\textnormal{I}}}
\def\ermJ{{\textnormal{J}}}
\def\ermK{{\textnormal{K}}}
\def\ermL{{\textnormal{L}}}
\def\ermM{{\textnormal{M}}}
\def\ermN{{\textnormal{N}}}
\def\ermO{{\textnormal{O}}}
\def\ermP{{\textnormal{P}}}
\def\ermQ{{\textnormal{Q}}}
\def\ermR{{\textnormal{R}}}
\def\ermS{{\textnormal{S}}}
\def\ermT{{\textnormal{T}}}
\def\ermU{{\textnormal{U}}}
\def\ermV{{\textnormal{V}}}
\def\ermW{{\textnormal{W}}}
\def\ermX{{\textnormal{X}}}
\def\ermY{{\textnormal{Y}}}
\def\ermZ{{\textnormal{Z}}}

% Vectors
\def\vzero{{\bm{0}}}
\def\vone{{\bm{1}}}
\def\vmu{{\bm{\mu}}}
\def\vtheta{{\bm{\theta}}}
\def\vphi{{\bm{\phi}}}
\def\va{{\bm{a}}}
\def\vb{{\bm{b}}}
\def\vc{{\bm{c}}}
\def\vd{{\bm{d}}}
\def\ve{{\bm{e}}}
\def\vf{{\bm{f}}}
\def\vg{{\bm{g}}}
\def\vh{{\bm{h}}}
\def\vi{{\bm{i}}}
\def\vj{{\bm{j}}}
\def\vk{{\bm{k}}}
\def\vl{{\bm{l}}}
\def\vm{{\bm{m}}}
\def\vn{{\bm{n}}}
\def\vo{{\bm{o}}}
\def\vp{{\bm{p}}}
\def\vq{{\bm{q}}}
\def\vr{{\bm{r}}}
\def\vs{{\bm{s}}}
\def\vt{{\bm{t}}}
\def\vu{{\bm{u}}}
\def\vv{{\bm{v}}}
\def\vw{{\bm{w}}}
\def\vx{{\bm{x}}}
\def\vy{{\bm{y}}}
\def\vz{{\bm{z}}}

% Elements of vectors
\def\evalpha{{\alpha}}
\def\evbeta{{\beta}}
\def\evepsilon{{\epsilon}}
\def\evlambda{{\lambda}}
\def\evomega{{\omega}}
\def\evmu{{\mu}}
\def\evpsi{{\psi}}
\def\evsigma{{\sigma}}
\def\evtheta{{\theta}}
\def\eva{{a}}
\def\evb{{b}}
\def\evc{{c}}
\def\evd{{d}}
\def\eve{{e}}
\def\evf{{f}}
\def\evg{{g}}
\def\evh{{h}}
\def\evi{{i}}
\def\evj{{j}}
\def\evk{{k}}
\def\evl{{l}}
\def\evm{{m}}
\def\evn{{n}}
\def\evo{{o}}
\def\evp{{p}}
\def\evq{{q}}
\def\evr{{r}}
\def\evs{{s}}
\def\evt{{t}}
\def\evu{{u}}
\def\evv{{v}}
\def\evw{{w}}
\def\evx{{x}}
\def\evy{{y}}
\def\evz{{z}}

% Matrix
\def\mA{{\bm{A}}}
\def\mB{{\bm{B}}}
\def\mC{{\bm{C}}}
\def\mD{{\bm{D}}}
\def\mE{{\bm{E}}}
\def\mF{{\bm{F}}}
\def\mG{{\bm{G}}}
\def\mH{{\bm{H}}}
\def\mI{{\bm{I}}}
\def\mJ{{\bm{J}}}
\def\mK{{\bm{K}}}
\def\mL{{\bm{L}}}
\def\mM{{\bm{M}}}
\def\mN{{\bm{N}}}
\def\mO{{\bm{O}}}
\def\mP{{\bm{P}}}
\def\mQ{{\bm{Q}}}
\def\mR{{\bm{R}}}
\def\mS{{\bm{S}}}
\def\mT{{\bm{T}}}
\def\mU{{\bm{U}}}
\def\mV{{\bm{V}}}
\def\mW{{\bm{W}}}
\def\mX{{\bm{X}}}
\def\mY{{\bm{Y}}}
\def\mZ{{\bm{Z}}}
\def\mBeta{{\bm{\beta}}}
\def\mPhi{{\bm{\Phi}}}
\def\mLambda{{\bm{\Lambda}}}
\def\mSigma{{\bm{\Sigma}}}

% Tensor
\DeclareMathAlphabet{\mathsfit}{\encodingdefault}{\sfdefault}{m}{sl}
\SetMathAlphabet{\mathsfit}{bold}{\encodingdefault}{\sfdefault}{bx}{n}
\newcommand{\tens}[1]{\bm{\mathsfit{#1}}}
\def\tA{{\tens{A}}}
\def\tB{{\tens{B}}}
\def\tC{{\tens{C}}}
\def\tD{{\tens{D}}}
\def\tE{{\tens{E}}}
\def\tF{{\tens{F}}}
\def\tG{{\tens{G}}}
\def\tH{{\tens{H}}}
\def\tI{{\tens{I}}}
\def\tJ{{\tens{J}}}
\def\tK{{\tens{K}}}
\def\tL{{\tens{L}}}
\def\tM{{\tens{M}}}
\def\tN{{\tens{N}}}
\def\tO{{\tens{O}}}
\def\tP{{\tens{P}}}
\def\tQ{{\tens{Q}}}
\def\tR{{\tens{R}}}
\def\tS{{\tens{S}}}
\def\tT{{\tens{T}}}
\def\tU{{\tens{U}}}
\def\tV{{\tens{V}}}
\def\tW{{\tens{W}}}
\def\tX{{\tens{X}}}
\def\tY{{\tens{Y}}}
\def\tZ{{\tens{Z}}}


% Graph
\def\gA{{\mathcal{A}}}
\def\gB{{\mathcal{B}}}
\def\gC{{\mathcal{C}}}
\def\gD{{\mathcal{D}}}
\def\gE{{\mathcal{E}}}
\def\gF{{\mathcal{F}}}
\def\gG{{\mathcal{G}}}
\def\gH{{\mathcal{H}}}
\def\gI{{\mathcal{I}}}
\def\gJ{{\mathcal{J}}}
\def\gK{{\mathcal{K}}}
\def\gL{{\mathcal{L}}}
\def\gM{{\mathcal{M}}}
\def\gN{{\mathcal{N}}}
\def\gO{{\mathcal{O}}}
\def\gP{{\mathcal{P}}}
\def\gQ{{\mathcal{Q}}}
\def\gR{{\mathcal{R}}}
\def\gS{{\mathcal{S}}}
\def\gT{{\mathcal{T}}}
\def\gU{{\mathcal{U}}}
\def\gV{{\mathcal{V}}}
\def\gW{{\mathcal{W}}}
\def\gX{{\mathcal{X}}}
\def\gY{{\mathcal{Y}}}
\def\gZ{{\mathcal{Z}}}

% Sets
\def\sA{{\mathbb{A}}}
\def\sB{{\mathbb{B}}}
\def\sC{{\mathbb{C}}}
\def\sD{{\mathbb{D}}}
% Don't use a set called E, because this would be the same as our symbol
% for expectation.
\def\sF{{\mathbb{F}}}
\def\sG{{\mathbb{G}}}
\def\sH{{\mathbb{H}}}
\def\sI{{\mathbb{I}}}
\def\sJ{{\mathbb{J}}}
\def\sK{{\mathbb{K}}}
\def\sL{{\mathbb{L}}}
\def\sM{{\mathbb{M}}}
\def\sN{{\mathbb{N}}}
\def\sO{{\mathbb{O}}}
\def\sP{{\mathbb{P}}}
\def\sQ{{\mathbb{Q}}}
\def\sR{{\mathbb{R}}}
\def\sS{{\mathbb{S}}}
\def\sT{{\mathbb{T}}}
\def\sU{{\mathbb{U}}}
\def\sV{{\mathbb{V}}}
\def\sW{{\mathbb{W}}}
\def\sX{{\mathbb{X}}}
\def\sY{{\mathbb{Y}}}
\def\sZ{{\mathbb{Z}}}

% Entries of a matrix
\def\emLambda{{\Lambda}}
\def\emA{{A}}
\def\emB{{B}}
\def\emC{{C}}
\def\emD{{D}}
\def\emE{{E}}
\def\emF{{F}}
\def\emG{{G}}
\def\emH{{H}}
\def\emI{{I}}
\def\emJ{{J}}
\def\emK{{K}}
\def\emL{{L}}
\def\emM{{M}}
\def\emN{{N}}
\def\emO{{O}}
\def\emP{{P}}
\def\emQ{{Q}}
\def\emR{{R}}
\def\emS{{S}}
\def\emT{{T}}
\def\emU{{U}}
\def\emV{{V}}
\def\emW{{W}}
\def\emX{{X}}
\def\emY{{Y}}
\def\emZ{{Z}}
\def\emSigma{{\Sigma}}

% entries of a tensor
% Same font as tensor, without \bm wrapper
\newcommand{\etens}[1]{\mathsfit{#1}}
\def\etLambda{{\etens{\Lambda}}}
\def\etA{{\etens{A}}}
\def\etB{{\etens{B}}}
\def\etC{{\etens{C}}}
\def\etD{{\etens{D}}}
\def\etE{{\etens{E}}}
\def\etF{{\etens{F}}}
\def\etG{{\etens{G}}}
\def\etH{{\etens{H}}}
\def\etI{{\etens{I}}}
\def\etJ{{\etens{J}}}
\def\etK{{\etens{K}}}
\def\etL{{\etens{L}}}
\def\etM{{\etens{M}}}
\def\etN{{\etens{N}}}
\def\etO{{\etens{O}}}
\def\etP{{\etens{P}}}
\def\etQ{{\etens{Q}}}
\def\etR{{\etens{R}}}
\def\etS{{\etens{S}}}
\def\etT{{\etens{T}}}
\def\etU{{\etens{U}}}
\def\etV{{\etens{V}}}
\def\etW{{\etens{W}}}
\def\etX{{\etens{X}}}
\def\etY{{\etens{Y}}}
\def\etZ{{\etens{Z}}}

% The true underlying data generating distribution
\newcommand{\pdata}{p_{\rm{data}}}
\newcommand{\ptarget}{p_{\rm{target}}}
\newcommand{\pprior}{p_{\rm{prior}}}
\newcommand{\pbase}{p_{\rm{base}}}
\newcommand{\pref}{p_{\rm{ref}}}

% The empirical distribution defined by the training set
\newcommand{\ptrain}{\hat{p}_{\rm{data}}}
\newcommand{\Ptrain}{\hat{P}_{\rm{data}}}
% The model distribution
\newcommand{\pmodel}{p_{\rm{model}}}
\newcommand{\Pmodel}{P_{\rm{model}}}
\newcommand{\ptildemodel}{\tilde{p}_{\rm{model}}}
% Stochastic autoencoder distributions
\newcommand{\pencode}{p_{\rm{encoder}}}
\newcommand{\pdecode}{p_{\rm{decoder}}}
\newcommand{\precons}{p_{\rm{reconstruct}}}

\newcommand{\laplace}{\mathrm{Laplace}} % Laplace distribution

\newcommand{\E}{\mathbb{E}}
\newcommand{\Ls}{\mathcal{L}}
\newcommand{\R}{\mathbb{R}}
\newcommand{\emp}{\tilde{p}}
\newcommand{\lr}{\alpha}
\newcommand{\reg}{\lambda}
\newcommand{\rect}{\mathrm{rectifier}}
\newcommand{\softmax}{\mathrm{softmax}}
\newcommand{\sigmoid}{\sigma}
\newcommand{\softplus}{\zeta}
\newcommand{\KL}{D_{\mathrm{KL}}}
\newcommand{\Var}{\mathrm{Var}}
\newcommand{\standarderror}{\mathrm{SE}}
\newcommand{\Cov}{\mathrm{Cov}}
% Wolfram Mathworld says $L^2$ is for function spaces and $\ell^2$ is for vectors
% But then they seem to use $L^2$ for vectors throughout the site, and so does
% wikipedia.
\newcommand{\normlzero}{L^0}
\newcommand{\normlone}{L^1}
\newcommand{\normltwo}{L^2}
\newcommand{\normlp}{L^p}
\newcommand{\normmax}{L^\infty}

\newcommand{\parents}{Pa} % See usage in notation.tex. Chosen to match Daphne's book.

\DeclareMathOperator*{\argmax}{arg\,max}
\DeclareMathOperator*{\argmin}{arg\,min}

\DeclareMathOperator{\sign}{sign}
\DeclareMathOperator{\Tr}{Tr}
\let\ab\allowbreak


\usepackage{hyperref}
\usepackage{url}

\usepackage{amsmath}
\usepackage{amssymb}
\usepackage{mathtools}
\usepackage{amsthm}
\usepackage{xcolor}
\usepackage{microtype}
\usepackage{graphicx}
\usepackage{subfigure}
\usepackage{booktabs} % for professional tables


\usepackage{algorithm}
\usepackage{algorithmic}

\usepackage[utf8]{inputenc} % allow utf-8 input
\usepackage[T1]{fontenc}    % use 8-bit T1 fonts
\usepackage{hyperref}       % hyperlinks
\usepackage{url}            % simple URL typesetting
\usepackage{booktabs}       % professional-quality tables
\usepackage{amsfonts}       % blackboard math symbols
\usepackage{nicefrac}       % compact symbols for 1/2, etc.
\usepackage{microtype}      % microtypography
\usepackage{xcolor}         % colors

% \theoremstyle{plain}
\newtheorem{theorem}{Theorem}[section]
\newtheorem{proposition}[theorem]{Proposition}
\newtheorem{lemma}[theorem]{Lemma}
\newtheorem{corollary}[theorem]{Corollary}
\newtheorem{definition}[theorem]{Definition}
\newtheorem{assumption}[theorem]{Assumption}
\newtheorem{remark}[theorem]{Remark}

\usepackage[textsize=tiny]{todonotes}

\newcount\Comments  % 0 suppresses notes to selves in text
\Comments = 1
\newcommand{\kibitz}[2]{\ifnum\Comments=1{\color{#1}{#2}}\fi}
\newcommand{\tw}[1]{\kibitz{red}{#1}}

\usepackage{xcolor}
\newcommand\TODO[1]{\textcolor{red}{[TODO: #1]}}
\newcommand\CHANGE[1]{\textcolor{blue}{#1}}
%%%%% NEW MATH DEFINITIONS %%%%%

\usepackage{amsmath,amsfonts,bm}
\usepackage{derivative}
% Mark sections of captions for referring to divisions of figures
\newcommand{\figleft}{{\em (Left)}}
\newcommand{\figcenter}{{\em (Center)}}
\newcommand{\figright}{{\em (Right)}}
\newcommand{\figtop}{{\em (Top)}}
\newcommand{\figbottom}{{\em (Bottom)}}
\newcommand{\captiona}{{\em (a)}}
\newcommand{\captionb}{{\em (b)}}
\newcommand{\captionc}{{\em (c)}}
\newcommand{\captiond}{{\em (d)}}

% Highlight a newly defined term
\newcommand{\newterm}[1]{{\bf #1}}

% Derivative d 
\newcommand{\deriv}{{\mathrm{d}}}

% Figure reference, lower-case.
\def\figref#1{figure~\ref{#1}}
% Figure reference, capital. For start of sentence
\def\Figref#1{Figure~\ref{#1}}
\def\twofigref#1#2{figures \ref{#1} and \ref{#2}}
\def\quadfigref#1#2#3#4{figures \ref{#1}, \ref{#2}, \ref{#3} and \ref{#4}}
% Section reference, lower-case.
\def\secref#1{section~\ref{#1}}
% Section reference, capital.
\def\Secref#1{Section~\ref{#1}}
% Reference to two sections.
\def\twosecrefs#1#2{sections \ref{#1} and \ref{#2}}
% Reference to three sections.
\def\secrefs#1#2#3{sections \ref{#1}, \ref{#2} and \ref{#3}}
% Reference to an equation, lower-case.
\def\eqref#1{equation~\ref{#1}}
% Reference to an equation, upper case
\def\Eqref#1{Equation~\ref{#1}}
% A raw reference to an equation---avoid using if possible
\def\plaineqref#1{\ref{#1}}
% Reference to a chapter, lower-case.
\def\chapref#1{chapter~\ref{#1}}
% Reference to an equation, upper case.
\def\Chapref#1{Chapter~\ref{#1}}
% Reference to a range of chapters
\def\rangechapref#1#2{chapters\ref{#1}--\ref{#2}}
% Reference to an algorithm, lower-case.
\def\algref#1{algorithm~\ref{#1}}
% Reference to an algorithm, upper case.
\def\Algref#1{Algorithm~\ref{#1}}
\def\twoalgref#1#2{algorithms \ref{#1} and \ref{#2}}
\def\Twoalgref#1#2{Algorithms \ref{#1} and \ref{#2}}
% Reference to a part, lower case
\def\partref#1{part~\ref{#1}}
% Reference to a part, upper case
\def\Partref#1{Part~\ref{#1}}
\def\twopartref#1#2{parts \ref{#1} and \ref{#2}}

\def\ceil#1{\lceil #1 \rceil}
\def\floor#1{\lfloor #1 \rfloor}
\def\1{\bm{1}}
\newcommand{\train}{\mathcal{D}}
\newcommand{\valid}{\mathcal{D_{\mathrm{valid}}}}
\newcommand{\test}{\mathcal{D_{\mathrm{test}}}}

\def\eps{{\epsilon}}


% Random variables
\def\reta{{\textnormal{$\eta$}}}
\def\ra{{\textnormal{a}}}
\def\rb{{\textnormal{b}}}
\def\rc{{\textnormal{c}}}
\def\rd{{\textnormal{d}}}
\def\re{{\textnormal{e}}}
\def\rf{{\textnormal{f}}}
\def\rg{{\textnormal{g}}}
\def\rh{{\textnormal{h}}}
\def\ri{{\textnormal{i}}}
\def\rj{{\textnormal{j}}}
\def\rk{{\textnormal{k}}}
\def\rl{{\textnormal{l}}}
% rm is already a command, just don't name any random variables m
\def\rn{{\textnormal{n}}}
\def\ro{{\textnormal{o}}}
\def\rp{{\textnormal{p}}}
\def\rq{{\textnormal{q}}}
\def\rr{{\textnormal{r}}}
\def\rs{{\textnormal{s}}}
\def\rt{{\textnormal{t}}}
\def\ru{{\textnormal{u}}}
\def\rv{{\textnormal{v}}}
\def\rw{{\textnormal{w}}}
\def\rx{{\textnormal{x}}}
\def\ry{{\textnormal{y}}}
\def\rz{{\textnormal{z}}}

% Random vectors
\def\rvepsilon{{\mathbf{\epsilon}}}
\def\rvphi{{\mathbf{\phi}}}
\def\rvtheta{{\mathbf{\theta}}}
\def\rva{{\mathbf{a}}}
\def\rvb{{\mathbf{b}}}
\def\rvc{{\mathbf{c}}}
\def\rvd{{\mathbf{d}}}
\def\rve{{\mathbf{e}}}
\def\rvf{{\mathbf{f}}}
\def\rvg{{\mathbf{g}}}
\def\rvh{{\mathbf{h}}}
\def\rvu{{\mathbf{i}}}
\def\rvj{{\mathbf{j}}}
\def\rvk{{\mathbf{k}}}
\def\rvl{{\mathbf{l}}}
\def\rvm{{\mathbf{m}}}
\def\rvn{{\mathbf{n}}}
\def\rvo{{\mathbf{o}}}
\def\rvp{{\mathbf{p}}}
\def\rvq{{\mathbf{q}}}
\def\rvr{{\mathbf{r}}}
\def\rvs{{\mathbf{s}}}
\def\rvt{{\mathbf{t}}}
\def\rvu{{\mathbf{u}}}
\def\rvv{{\mathbf{v}}}
\def\rvw{{\mathbf{w}}}
\def\rvx{{\mathbf{x}}}
\def\rvy{{\mathbf{y}}}
\def\rvz{{\mathbf{z}}}

% Elements of random vectors
\def\erva{{\textnormal{a}}}
\def\ervb{{\textnormal{b}}}
\def\ervc{{\textnormal{c}}}
\def\ervd{{\textnormal{d}}}
\def\erve{{\textnormal{e}}}
\def\ervf{{\textnormal{f}}}
\def\ervg{{\textnormal{g}}}
\def\ervh{{\textnormal{h}}}
\def\ervi{{\textnormal{i}}}
\def\ervj{{\textnormal{j}}}
\def\ervk{{\textnormal{k}}}
\def\ervl{{\textnormal{l}}}
\def\ervm{{\textnormal{m}}}
\def\ervn{{\textnormal{n}}}
\def\ervo{{\textnormal{o}}}
\def\ervp{{\textnormal{p}}}
\def\ervq{{\textnormal{q}}}
\def\ervr{{\textnormal{r}}}
\def\ervs{{\textnormal{s}}}
\def\ervt{{\textnormal{t}}}
\def\ervu{{\textnormal{u}}}
\def\ervv{{\textnormal{v}}}
\def\ervw{{\textnormal{w}}}
\def\ervx{{\textnormal{x}}}
\def\ervy{{\textnormal{y}}}
\def\ervz{{\textnormal{z}}}

% Random matrices
\def\rmA{{\mathbf{A}}}
\def\rmB{{\mathbf{B}}}
\def\rmC{{\mathbf{C}}}
\def\rmD{{\mathbf{D}}}
\def\rmE{{\mathbf{E}}}
\def\rmF{{\mathbf{F}}}
\def\rmG{{\mathbf{G}}}
\def\rmH{{\mathbf{H}}}
\def\rmI{{\mathbf{I}}}
\def\rmJ{{\mathbf{J}}}
\def\rmK{{\mathbf{K}}}
\def\rmL{{\mathbf{L}}}
\def\rmM{{\mathbf{M}}}
\def\rmN{{\mathbf{N}}}
\def\rmO{{\mathbf{O}}}
\def\rmP{{\mathbf{P}}}
\def\rmQ{{\mathbf{Q}}}
\def\rmR{{\mathbf{R}}}
\def\rmS{{\mathbf{S}}}
\def\rmT{{\mathbf{T}}}
\def\rmU{{\mathbf{U}}}
\def\rmV{{\mathbf{V}}}
\def\rmW{{\mathbf{W}}}
\def\rmX{{\mathbf{X}}}
\def\rmY{{\mathbf{Y}}}
\def\rmZ{{\mathbf{Z}}}

% Elements of random matrices
\def\ermA{{\textnormal{A}}}
\def\ermB{{\textnormal{B}}}
\def\ermC{{\textnormal{C}}}
\def\ermD{{\textnormal{D}}}
\def\ermE{{\textnormal{E}}}
\def\ermF{{\textnormal{F}}}
\def\ermG{{\textnormal{G}}}
\def\ermH{{\textnormal{H}}}
\def\ermI{{\textnormal{I}}}
\def\ermJ{{\textnormal{J}}}
\def\ermK{{\textnormal{K}}}
\def\ermL{{\textnormal{L}}}
\def\ermM{{\textnormal{M}}}
\def\ermN{{\textnormal{N}}}
\def\ermO{{\textnormal{O}}}
\def\ermP{{\textnormal{P}}}
\def\ermQ{{\textnormal{Q}}}
\def\ermR{{\textnormal{R}}}
\def\ermS{{\textnormal{S}}}
\def\ermT{{\textnormal{T}}}
\def\ermU{{\textnormal{U}}}
\def\ermV{{\textnormal{V}}}
\def\ermW{{\textnormal{W}}}
\def\ermX{{\textnormal{X}}}
\def\ermY{{\textnormal{Y}}}
\def\ermZ{{\textnormal{Z}}}

% Vectors
\def\vzero{{\bm{0}}}
\def\vone{{\bm{1}}}
\def\vmu{{\bm{\mu}}}
\def\vtheta{{\bm{\theta}}}
\def\vphi{{\bm{\phi}}}
\def\va{{\bm{a}}}
\def\vb{{\bm{b}}}
\def\vc{{\bm{c}}}
\def\vd{{\bm{d}}}
\def\ve{{\bm{e}}}
\def\vf{{\bm{f}}}
\def\vg{{\bm{g}}}
\def\vh{{\bm{h}}}
\def\vi{{\bm{i}}}
\def\vj{{\bm{j}}}
\def\vk{{\bm{k}}}
\def\vl{{\bm{l}}}
\def\vm{{\bm{m}}}
\def\vn{{\bm{n}}}
\def\vo{{\bm{o}}}
\def\vp{{\bm{p}}}
\def\vq{{\bm{q}}}
\def\vr{{\bm{r}}}
\def\vs{{\bm{s}}}
\def\vt{{\bm{t}}}
\def\vu{{\bm{u}}}
\def\vv{{\bm{v}}}
\def\vw{{\bm{w}}}
\def\vx{{\bm{x}}}
\def\vy{{\bm{y}}}
\def\vz{{\bm{z}}}

% Elements of vectors
\def\evalpha{{\alpha}}
\def\evbeta{{\beta}}
\def\evepsilon{{\epsilon}}
\def\evlambda{{\lambda}}
\def\evomega{{\omega}}
\def\evmu{{\mu}}
\def\evpsi{{\psi}}
\def\evsigma{{\sigma}}
\def\evtheta{{\theta}}
\def\eva{{a}}
\def\evb{{b}}
\def\evc{{c}}
\def\evd{{d}}
\def\eve{{e}}
\def\evf{{f}}
\def\evg{{g}}
\def\evh{{h}}
\def\evi{{i}}
\def\evj{{j}}
\def\evk{{k}}
\def\evl{{l}}
\def\evm{{m}}
\def\evn{{n}}
\def\evo{{o}}
\def\evp{{p}}
\def\evq{{q}}
\def\evr{{r}}
\def\evs{{s}}
\def\evt{{t}}
\def\evu{{u}}
\def\evv{{v}}
\def\evw{{w}}
\def\evx{{x}}
\def\evy{{y}}
\def\evz{{z}}

% Matrix
\def\mA{{\bm{A}}}
\def\mB{{\bm{B}}}
\def\mC{{\bm{C}}}
\def\mD{{\bm{D}}}
\def\mE{{\bm{E}}}
\def\mF{{\bm{F}}}
\def\mG{{\bm{G}}}
\def\mH{{\bm{H}}}
\def\mI{{\bm{I}}}
\def\mJ{{\bm{J}}}
\def\mK{{\bm{K}}}
\def\mL{{\bm{L}}}
\def\mM{{\bm{M}}}
\def\mN{{\bm{N}}}
\def\mO{{\bm{O}}}
\def\mP{{\bm{P}}}
\def\mQ{{\bm{Q}}}
\def\mR{{\bm{R}}}
\def\mS{{\bm{S}}}
\def\mT{{\bm{T}}}
\def\mU{{\bm{U}}}
\def\mV{{\bm{V}}}
\def\mW{{\bm{W}}}
\def\mX{{\bm{X}}}
\def\mY{{\bm{Y}}}
\def\mZ{{\bm{Z}}}
\def\mBeta{{\bm{\beta}}}
\def\mPhi{{\bm{\Phi}}}
\def\mLambda{{\bm{\Lambda}}}
\def\mSigma{{\bm{\Sigma}}}

% Tensor
\DeclareMathAlphabet{\mathsfit}{\encodingdefault}{\sfdefault}{m}{sl}
\SetMathAlphabet{\mathsfit}{bold}{\encodingdefault}{\sfdefault}{bx}{n}
\newcommand{\tens}[1]{\bm{\mathsfit{#1}}}
\def\tA{{\tens{A}}}
\def\tB{{\tens{B}}}
\def\tC{{\tens{C}}}
\def\tD{{\tens{D}}}
\def\tE{{\tens{E}}}
\def\tF{{\tens{F}}}
\def\tG{{\tens{G}}}
\def\tH{{\tens{H}}}
\def\tI{{\tens{I}}}
\def\tJ{{\tens{J}}}
\def\tK{{\tens{K}}}
\def\tL{{\tens{L}}}
\def\tM{{\tens{M}}}
\def\tN{{\tens{N}}}
\def\tO{{\tens{O}}}
\def\tP{{\tens{P}}}
\def\tQ{{\tens{Q}}}
\def\tR{{\tens{R}}}
\def\tS{{\tens{S}}}
\def\tT{{\tens{T}}}
\def\tU{{\tens{U}}}
\def\tV{{\tens{V}}}
\def\tW{{\tens{W}}}
\def\tX{{\tens{X}}}
\def\tY{{\tens{Y}}}
\def\tZ{{\tens{Z}}}


% Graph
\def\gA{{\mathcal{A}}}
\def\gB{{\mathcal{B}}}
\def\gC{{\mathcal{C}}}
\def\gD{{\mathcal{D}}}
\def\gE{{\mathcal{E}}}
\def\gF{{\mathcal{F}}}
\def\gG{{\mathcal{G}}}
\def\gH{{\mathcal{H}}}
\def\gI{{\mathcal{I}}}
\def\gJ{{\mathcal{J}}}
\def\gK{{\mathcal{K}}}
\def\gL{{\mathcal{L}}}
\def\gM{{\mathcal{M}}}
\def\gN{{\mathcal{N}}}
\def\gO{{\mathcal{O}}}
\def\gP{{\mathcal{P}}}
\def\gQ{{\mathcal{Q}}}
\def\gR{{\mathcal{R}}}
\def\gS{{\mathcal{S}}}
\def\gT{{\mathcal{T}}}
\def\gU{{\mathcal{U}}}
\def\gV{{\mathcal{V}}}
\def\gW{{\mathcal{W}}}
\def\gX{{\mathcal{X}}}
\def\gY{{\mathcal{Y}}}
\def\gZ{{\mathcal{Z}}}

% Sets
\def\sA{{\mathbb{A}}}
\def\sB{{\mathbb{B}}}
\def\sC{{\mathbb{C}}}
\def\sD{{\mathbb{D}}}
% Don't use a set called E, because this would be the same as our symbol
% for expectation.
\def\sF{{\mathbb{F}}}
\def\sG{{\mathbb{G}}}
\def\sH{{\mathbb{H}}}
\def\sI{{\mathbb{I}}}
\def\sJ{{\mathbb{J}}}
\def\sK{{\mathbb{K}}}
\def\sL{{\mathbb{L}}}
\def\sM{{\mathbb{M}}}
\def\sN{{\mathbb{N}}}
\def\sO{{\mathbb{O}}}
\def\sP{{\mathbb{P}}}
\def\sQ{{\mathbb{Q}}}
\def\sR{{\mathbb{R}}}
\def\sS{{\mathbb{S}}}
\def\sT{{\mathbb{T}}}
\def\sU{{\mathbb{U}}}
\def\sV{{\mathbb{V}}}
\def\sW{{\mathbb{W}}}
\def\sX{{\mathbb{X}}}
\def\sY{{\mathbb{Y}}}
\def\sZ{{\mathbb{Z}}}

% Entries of a matrix
\def\emLambda{{\Lambda}}
\def\emA{{A}}
\def\emB{{B}}
\def\emC{{C}}
\def\emD{{D}}
\def\emE{{E}}
\def\emF{{F}}
\def\emG{{G}}
\def\emH{{H}}
\def\emI{{I}}
\def\emJ{{J}}
\def\emK{{K}}
\def\emL{{L}}
\def\emM{{M}}
\def\emN{{N}}
\def\emO{{O}}
\def\emP{{P}}
\def\emQ{{Q}}
\def\emR{{R}}
\def\emS{{S}}
\def\emT{{T}}
\def\emU{{U}}
\def\emV{{V}}
\def\emW{{W}}
\def\emX{{X}}
\def\emY{{Y}}
\def\emZ{{Z}}
\def\emSigma{{\Sigma}}

% entries of a tensor
% Same font as tensor, without \bm wrapper
\newcommand{\etens}[1]{\mathsfit{#1}}
\def\etLambda{{\etens{\Lambda}}}
\def\etA{{\etens{A}}}
\def\etB{{\etens{B}}}
\def\etC{{\etens{C}}}
\def\etD{{\etens{D}}}
\def\etE{{\etens{E}}}
\def\etF{{\etens{F}}}
\def\etG{{\etens{G}}}
\def\etH{{\etens{H}}}
\def\etI{{\etens{I}}}
\def\etJ{{\etens{J}}}
\def\etK{{\etens{K}}}
\def\etL{{\etens{L}}}
\def\etM{{\etens{M}}}
\def\etN{{\etens{N}}}
\def\etO{{\etens{O}}}
\def\etP{{\etens{P}}}
\def\etQ{{\etens{Q}}}
\def\etR{{\etens{R}}}
\def\etS{{\etens{S}}}
\def\etT{{\etens{T}}}
\def\etU{{\etens{U}}}
\def\etV{{\etens{V}}}
\def\etW{{\etens{W}}}
\def\etX{{\etens{X}}}
\def\etY{{\etens{Y}}}
\def\etZ{{\etens{Z}}}

% The true underlying data generating distribution
\newcommand{\pdata}{p_{\rm{data}}}
\newcommand{\ptarget}{p_{\rm{target}}}
\newcommand{\pprior}{p_{\rm{prior}}}
\newcommand{\pbase}{p_{\rm{base}}}
\newcommand{\pref}{p_{\rm{ref}}}

% The empirical distribution defined by the training set
\newcommand{\ptrain}{\hat{p}_{\rm{data}}}
\newcommand{\Ptrain}{\hat{P}_{\rm{data}}}
% The model distribution
\newcommand{\pmodel}{p_{\rm{model}}}
\newcommand{\Pmodel}{P_{\rm{model}}}
\newcommand{\ptildemodel}{\tilde{p}_{\rm{model}}}
% Stochastic autoencoder distributions
\newcommand{\pencode}{p_{\rm{encoder}}}
\newcommand{\pdecode}{p_{\rm{decoder}}}
\newcommand{\precons}{p_{\rm{reconstruct}}}

\newcommand{\laplace}{\mathrm{Laplace}} % Laplace distribution

\newcommand{\E}{\mathbb{E}}
\newcommand{\Ls}{\mathcal{L}}
\newcommand{\R}{\mathbb{R}}
\newcommand{\emp}{\tilde{p}}
\newcommand{\lr}{\alpha}
\newcommand{\reg}{\lambda}
\newcommand{\rect}{\mathrm{rectifier}}
\newcommand{\softmax}{\mathrm{softmax}}
\newcommand{\sigmoid}{\sigma}
\newcommand{\softplus}{\zeta}
\newcommand{\KL}{D_{\mathrm{KL}}}
\newcommand{\Var}{\mathrm{Var}}
\newcommand{\standarderror}{\mathrm{SE}}
\newcommand{\Cov}{\mathrm{Cov}}
% Wolfram Mathworld says $L^2$ is for function spaces and $\ell^2$ is for vectors
% But then they seem to use $L^2$ for vectors throughout the site, and so does
% wikipedia.
\newcommand{\normlzero}{L^0}
\newcommand{\normlone}{L^1}
\newcommand{\normltwo}{L^2}
\newcommand{\normlp}{L^p}
\newcommand{\normmax}{L^\infty}

\newcommand{\parents}{Pa} % See usage in notation.tex. Chosen to match Daphne's book.

\DeclareMathOperator*{\argmax}{arg\,max}
\DeclareMathOperator*{\argmin}{arg\,min}

\DeclareMathOperator{\sign}{sign}
\DeclareMathOperator{\Tr}{Tr}
\let\ab\allowbreak

\newcommand{\shortn}{\textup{\texttt{-}}}
\newcommand{\shorte}{\textup{\texttt{=}}}
\newcommand{\shortp}{\textup{\texttt{+}}}
\newcommand{\shortl}{\textup{\texttt{<}}}
\newcommand{\shortg}{\textup{\texttt{>}}}
\newcommand{\ie}{\textit{i}.\textit{e}.}
\newcommand{\eg}{\textit{e}.\textit{g}.}
\newcommand{\etal}{\textit{et al}.}
\newcommand{\etc}{\textit{etc}.}
\newcommand{\Tau}{\mathrm{T}}
\newcommand{\name}{\textsc{ReDOR}}
\newcommand{\rdcshort}{\mathtt{rdc}}
\newcommand{\namep}{$\mathtt{Prioritized}$}
\newcommand{\nameo}{$\mathtt{Complete\ Dataset}$}
\newcommand{\nameh}{$\mathtt{Top}\ x\%\ \mathtt{BC}$ }
\newcommand{\namer}{$\mathtt{Random}$}
\newcommand{\namec}{$\mathtt{No\ Cluster}$}
\newcommand{\namei}{$\mathtt{Single\ Round}$}
\newcommand{\nameq}{$\mathtt{Q\ Target}$}
\newcommand{\tact}{Transform2Act}

\newcommand{\originalant}{Handcrafted Robot}
\newcommand{\locomotionft}{Locomotion on Flat Terrain}
\newcommand{\locomotionvt}{Locomotion on Variable Terrain}
\newcommand{\escape}{Escape Bowl}
\newcommand{\pointnav}{Point Navigation}
\newcommand{\manipulationbox}{Manipulate Box}
\newcommand{\patrol}{Patrol}

\usepackage{thm-restate}

\title{Fewer May Be Better: Enhancing Offline Reinforcement Learning with Reduced Dataset}

% \title{ReDOR: Reduced Dataset for Offline Reinforcement Learning}

% Authors must not appear in the submitted version. They should be hidden
% as long as the \iclrfinalcopy macro remains commented out below.
% Non-anonymous submissions will be rejected without review.

\author{
	Yiqin Yang$^1$, Quanwei Wang$^2$, Chenghao Li$^2$, Hao Hu$^2$, Chengjie Wu$^2$, Yuhua Jiang$^2$, \\ 
	\textbf{Dianyu Zhong$^2$, Ziyou Zhang$^2$, Qianchuan Zhao$^2$, Chongjie Zhang$^3$, Bo Xu$^1$\footnotemark[2]} \\
    $^1$The Key Laboratory of Cognition and Decision Intelligence for Complex Systems, \\ 
    \ \ Institute of Automation, Chinese Academy of Sciences \\
    $^2$Tsinghua University \\ 
    $^3$Washington University in St. Louis \\
    \texttt{yiqin.yang@ia.ac.cn}
}

% The \author macro works with any number of authors. There are two commands
% used to separate the names and addresses of multiple authors: \And and \AND.
%
% Using \And between authors leaves it to \LaTeX{} to determine where to break
% the lines. Using \AND forces a linebreak at that point. So, if \LaTeX{}
% puts 3 of 4 authors names on the first line, and the last on the second
% line, try using \AND instead of \And before the third author name.

\newcommand{\fix}{\marginpar{FIX}}
\newcommand{\new}{\marginpar{NEW}}

\iclrfinalcopy % Uncomment for camera-ready version, but NOT for submission.
\begin{document}


\maketitle

\renewcommand{\thefootnote}{\fnsymbol{footnote}}
\footnotetext[2]{Corresponding Author}

\begin{abstract}
% Research in offline reinforcement learning (RL) marks a paradigm shift in RL.
% However, a critical yet under-investigated aspect of offline RL is determining the subset of the offline dataset, which is used to improve algorithm performance while accelerating algorithm training. Moreover, the size of reduced datasets can uncover the requisite offline data volume essential for addressing analogous challenges.
% Based on the above considerations, we propose identifying Reduced Datasets for Offline RL (\name) by formulating it as a gradient approximation optimization problem. 
% We prove that the common actor-critic framework in reinforcement learning can be transformed into a submodular objective.
% This insight enables us to construct a subset by adopting the orthogonal matching pursuit (OMP).
% Specifically, we have made several critical modifications to OMP to enable successful adaptation with Offline RL algorithms.
% The experimental results indicate that the data subsets constructed by the ReDOR can significantly improve algorithm performance with low computational complexity.
Offline reinforcement learning (RL) represents a significant shift in RL research, allowing agents to learn from pre-collected datasets without further interaction with the environment. A key, yet underexplored, challenge in offline RL is selecting an optimal subset of the offline dataset that enhances both algorithm performance and training efficiency. Reducing dataset size can also reveal the minimal data requirements necessary for solving similar problems.
In response to this challenge, we introduce ReDOR (Reduced Datasets for Offline RL), a method that frames dataset selection as a gradient approximation optimization problem. We demonstrate that the widely used actor-critic framework in RL can be reformulated as a submodular optimization objective, enabling efficient subset selection. To achieve this, we adapt orthogonal matching pursuit (OMP), incorporating several novel modifications tailored for offline RL.
Our experimental results show that the data subsets identified by ReDOR not only boost algorithm performance but also do so with significantly lower computational complexity.
\end{abstract}

\section{Introduction}

Deep Reinforcement Learning (DRL) has emerged as a transformative paradigm for solving complex sequential decision-making problems. By enabling autonomous agents to interact with an environment, receive feedback in the form of rewards, and iteratively refine their policies, DRL has demonstrated remarkable success across a diverse range of domains including games (\eg Atari~\citep{mnih2013playing,kaiser2020model}, Go~\citep{silver2018general,silver2017mastering}, and StarCraft II~\citep{vinyals2019grandmaster,vinyals2017starcraft}), robotics~\citep{kalashnikov2018scalable}, communication networks~\citep{feriani2021single}, and finance~\citep{liu2024dynamic}. These successes underscore DRL's capability to surpass traditional rule-based systems, particularly in high-dimensional and dynamically evolving environments.

Despite these advances, a fundamental challenge remains: DRL agents typically rely on deep neural networks, which operate as black-box models, obscuring the rationale behind their decision-making processes. This opacity poses significant barriers to adoption in safety-critical and high-stakes applications, where interpretability is crucial for trust, compliance, and debugging. The lack of transparency in DRL can lead to unreliable decision-making, rendering it unsuitable for domains where explainability is a prerequisite, such as healthcare, autonomous driving, and financial risk assessment.

To address these concerns, the field of Explainable Deep Reinforcement Learning (XRL) has emerged, aiming to develop techniques that enhance the interpretability of DRL policies. XRL seeks to provide insights into an agent’s decision-making process, enabling researchers, practitioners, and end-users to understand, validate, and refine learned policies. By facilitating greater transparency, XRL contributes to the development of safer, more robust, and ethically aligned AI systems.

Furthermore, the increasing integration of Reinforcement Learning (RL) with Large Language Models (LLMs) has placed RL at the forefront of natural language processing (NLP) advancements. Methods such as Reinforcement Learning from Human Feedback (RLHF)~\citep{bai2022training,ouyang2022training} have become essential for aligning LLM outputs with human preferences and ethical guidelines. By treating language generation as a sequential decision-making process, RL-based fine-tuning enables LLMs to optimize for attributes such as factual accuracy, coherence, and user satisfaction, surpassing conventional supervised learning techniques. However, the application of RL in LLM alignment further amplifies the explainability challenge, as the complex interactions between RL updates and neural representations remain poorly understood.

This survey provides a systematic review of explainability methods in DRL, with a particular focus on their integration with LLMs and human-in-the-loop systems. We first introduce fundamental RL concepts and highlight key advances in DRL. We then categorize and analyze existing explanation techniques, encompassing feature-level, state-level, dataset-level, and model-level approaches. Additionally, we discuss methods for evaluating XRL techniques, considering both qualitative and quantitative assessment criteria. Finally, we explore real-world applications of XRL, including policy refinement, adversarial attack mitigation, and emerging challenges in ensuring interpretability in modern AI systems. Through this survey, we aim to provide a comprehensive perspective on the current state of XRL and outline future research directions to advance the development of interpretable and trustworthy DRL models.
\section{Preliminaries}

\textbf{Sealed-Bid Combinatorial Auction}. We consider sealed-bid CAs with a single bidder and $m$ items, $M=\{1,\ldots,m\}$.
The bidder has a {\em valuation function}, $v: 2^M \rightarrow \mathbb{R}_{\ge 0}$. Valuation $v$ is drawn independently from a distribution $F$ defined on the space of possible valuation functions $V$, determining how valuable each bundle $S\in 2^M$ is for the bidder. We consider bounded valuation functions: $v(S)\in[0, v_{\max}]$, $S\subset 2^M$, with $v_{\max}>0$, and they are normalized so that $v(\varnothing)=0$.
%
%\forall v_i\in \supp(F_i)$} \dcp{i just realized that we need a bounded domain $V_i$ for a grid to be well defined, right? comment on this.}
% We use $\vv$ to denote the value of the bidder for each of the $2^m$ bundles. 
The auctioneer  knows distribution $F$ but not the  valuation  $v$. The bidder reports their valuation function, perhaps untruthfully, as their {\em bid (function)}, $b\in V$. 

In CAs, a suitable {\em bidding language} is critical to allow a bidder to report their
bid without needing to enumerate a value for every possible bundle.  There
are many ways to do this, but a  common approach is to use the {\em XOR bidding language}, which allows bidders to submit bid prices for each of multiple bundles under an exclusive-or condition; in effect, only one bid price on a bundle can be accepted. Popular CA testbeds such as CATS~\citep{leyton2000towards} and SATS~\citep{weiss2017sats} employ this bidding language extensively.\footnote{When representing the values of multiple bidders these testbeds often also introduce so-called dummy items for distinguishing the bids of different 
bidders. Still, the semantics for a single bidder is, in effect, that of the XOR language.}
The semantics of the XOR bidding language is that the value on a bundle $S$ is the maximum bid price on
any bundle $S'$, submitted as part of the XOR bid, and for which $S'\subseteq S$. XOR bids are 
succinct for valuation functions in which the bidder is only interested in a bounded number
of possible bundles.

We seek an auction $(g,p)$ that maximizes expected revenue. Here, $g: V\rightarrow \mathcal{X}$ is the {\em allocation rule},  where $\mathcal{X}$ is the space of feasible allocations (i.e., no item allocated more than once), so that $g(b)\subseteq M$ denotes the set of items (perhaps empty) allocated to the bidder at bid $b$.
%
Also, $p: V\rightarrow \mathbb{R}_{\ge 0}$ is the {\em payment rule},
specifying the price associated with allocation $g(b)$. 
 %
The utility to the bidder with valuation function $v$ at bid  $b$ is $u(v;b)=v(g(b))-p(b)$, which is
the standard model of quasi-linearity so that values are in effect quantified in monetary units, say dollars.
%
 In full generality, the allocation and payment rules may be \emph{randomized}, with
 the bidder assumed to be risk neutral  and seeking
to maximize their 
expected utility.


In a \emph{dominant-strategy incentive compatible} (DSIC) auction, or {\em strategy-proof (SP)} auction, the bidder's utility is  maximized by bidding their true valuation $v$, whatever this valuation is; i.e., $u(v; v)\ge u(v;b)$, for $\forall v\in V, \forall b\in V$.
An auction is \emph{individually rational} (IR) if the bidder receives a non-negative utility when participating and truthfully reporting: $u(v;v)\ge 0$, for $\forall v\in V$.
Following the revelation principle, it is without loss of generality to focus on  
SP
auctions, as any auction that achieves a particular expected revenue in a dominant-strategy equilibrium
 can be transformed into an SP auction with the same expected revenue.
 %
 Optimal auction design therefore seeks to identify an SP and IR auction that maximizes the expected revenue, i.e., $\mathbb{E}_{v\sim \bm F}[p(v)]$. 

\textbf{Menu-Based CAs}. In a {\em menu-based auction}, allocation and payment rules  are
represented through a menu, $B$, consisting of
$K\ge 1$  {\em menu elements}.
%
We write $B=(B^{(1)},\ldots,B^{(K)})$, 
and the $k$th \emph{menu element}, $B^{(k)}$,
 specifies allocation probabilities on bundles,
 $\alpha^{(k)}: 2^M\rightarrow [0,1]$, and a {\em price}, $\beta^{(k)}\in \mathbb{R}$.
%
Here, we allow randomization, where  $\alpha^{(k)}(S)\in[0,1]$ denotes the
  probability that bundle $S\in 2^M$ is assigned to the bidder in menu element $k$. 
  % this assignment made independently of the 
  % assignment of some other item $j'\neq j$ (conditioned on the choice of
  % menu element $k$).
  %
   % \dcp{i'm wondering if this independent distribution is wlog for additive and unit-demand valuations---you can ask Sai and Michael if they know about anything.}
   %
  %
We refer to the menu $B$ as corresponding to a {\em menu-based representation 
of an auction.} The bidder with bid $b$ is assigned the element from menu $B$ that maximizes their utility according to the reported valuation: $k^*\in \arg\max_k \sum_{S\in 2^M}\alpha^{(k)}(S)b(S)-\beta^{(k)}$. We denote this optimal element by $(\alpha^*(b), \beta^*(b))$. 
The use of menu-based representations for auction design 
is without loss of generality and DSIC~\cite{hammond1979straightforward}.
%\dcp{need to explain how the allocation and payment rule is defined, via arg max given bid b and the menu} \dcp{need a bit more here, defining $\beta^*$ as the optimal element from menu $B$}
%
%In the context of menu-based auction design, the 
The optimal auction design problem is to find a menu-based representation that 
maximizes  expected revenue, i.e., $\mathbb{E}_{v\sim F}[ \beta^{*}(v)]$. Deep menu-based methods~\cite{dutting2024optimal,shen2019automated} in the differentiable economics literature~\cite{zheng2022ai,finocchiaro2021bridging,wang2023deep,ivanov2024principal,zhang2024position,hossain2024multi,rahme2020auction,ivanov2022optimal,curry2022differentiable,duan2023scalable} learn to generate such menus by neural networks.
%dcp cut among all  menus of a certain size.

% \tw{which means the number of elements in a menu?} \dcp{yes; now I say `among all menus` not `among all menu representations`. ok?}.  


\textbf{Diffusion Models and Continuous Normalizing Flow}. Diffusion models have  emerged as a powerful class of generative AI methods, spurring notable advances in a wide range of tasks such as image generation~\cite{rombach2022high,esser2024scaling}, video generation~\cite{ho2022video,ceylan2023pix2video,ho2022imagen}, molecular design~\cite{gruver2024protein}, text generation~\cite{lou2024discrete}, and multi-agent learning~\cite{wang2024diffusion}. At their core, these models perform a {\em forward noising process} in which noise is incrementally added to training data over multiple steps, gradually corrupting the original sample.
%\dcp{is `signal` right? or `sample`?} \tw{My opinion is that these works are rooted in the information theory, so they use "signal" a lot.} it's only used here in our paper, so I have dropped `signal`
%
A {\em reverse diffusion process} is then learned to iteratively remove noise, thereby reconstructing data from near-random initial states. In our setting, instead of reconstructing data, we extend the diffusion process to develop a tractable and differentiable method that optimizes a high-dimensional distribution.
%\tw{Do we need this here?}
%dcp -- yes, I like it

In particular, {\em score-based diffusion models} enjoy strong mathematical and physical underpinnings. The forward noising process is an {\em Itô stochastic differential equation} (SDE),
\begin{align}
    d\vx = \vf(\vx,t)dt + h(t)d\vw,
\end{align}
%
where $\vx(t) \in\mathbb{R}^\ell$ is the {\em state} at time $t$, for some $\ell\in \mathbb{Z}_{>0}$, $\vf(\cdot,t):\mathbb{R}^\ell\rightarrow\mathbb{R}^\ell$ is the {\em drift coefficient}, $h(\cdot):\mathbb{R}\rightarrow\mathbb{R}$ is the {\em diffusion coefficient}, and $\vw$ is the {\em standard Wiener process} (Brownian motion). Different forward processes are designed by specifying functional forms for $\vf(\cdot,t)$ and $h(\cdot)$. The generation of data is then based on the reverse process, which is  a diffusion process  given by the {\em reverse-time SDE}~\cite{anderson1982reverse},
%
\begin{align}
    d\vx = [\vf(\vx,t)-h(t)^2\nabla_\vx\log q_t(\vx)]dt + h(t)d\bar{\vw},\label{equ:r-sde}
\end{align}
%
where $dt$ is an infinitesimal negative timestep,  $\bar{\vw}$ is the {\em standard Brownian motion with reversed time flow},  and {\em $q_t(\vx)$ is the distribution of state  $\vx(t)$
at time $t$}.
The principal task in diffusion models is to learn the {\em score function}, $\nabla_\vx\log q_t(\vx)$, which has been effectively achieved using neural networks in recent work. This
enables solving the reverse-time SDE and  generating new data samples.  Notably, in the diffusion model  (and more broadly, generative AI) literature, $q_0(\vx)$ is typically a known target distribution over data samples from a pre-training dateset.
%\dcp{as $\vx_T$ given samples $\vs_0$?}

The reverse-time SDE (Eq.~\ref{equ:r-sde}) can be mathematically intricate, motivating the study of an equivalent, \emph{deterministic reverse process} modeled by an ordinary differential equation (ODE),
%
\begin{align}
    d\vx = [\vf(\vx,t)-\frac{1}{2}h(t)^2\nabla_\vx\log q_t(\vx)]dt,\label{equ:r-ode}
\end{align}
%
which  preserves the same marginal probability densities $\{q_t(\vx)\}_{t=0}^T$ as the SDE in Eq.~\ref{equ:r-sde}~\cite{song2021score}.
%
% Drawbacks of the normalizing flow include that we require the function $f$ to be invertible, and working its Jacobian might be expensive. We can consider a continuous version of normalizing flow where the function $f$ is applied an infinite time. 
%
Eq.~\ref{equ:r-ode} also highlights the connection between diffusion models and \emph{continuous normalizing flow}: each of them learns to transform and manipulate distributions by an ODE. Intuitively, a {\em continuous normalizing flow} transports an input $\vx_0\in \mathbb{R}^\ell$ to $\vx_t=\phi(t, \vx_0)$ at timestep $t\in[0,T]$.
%
Here, $\phi(t, \cdot):\mathbb{R}^\ell\rightarrow\mathbb{R}^\ell$ is the  \emph{flow}, and is governed by the ODE,
%
\begin{align}
    \frac{d}{dt}\vx_t = \varphi\left(t, \vx_t\right),\label{equ:f-ode}
\end{align}
%
where the vector field $\varphi: [0,T]\times \mathbb{R}^\ell\rightarrow \mathbb{R}^\ell$ specifies the  rate of
change of the state $\vx$.
%
Continuous normalizing flow \citep{chen2018neural} suggests to represent vector field $\varphi$ with a neural network. The flow $\phi$ transforms an initial distribution $p_0(\vx)$ to a final distribution $p_T(\vx)$ an time $T$.

% and the final distribution $p(z)=p_1(z_1)$ can be obtained by solving .

\textbf{Rectified Flow}. A bottleneck that restricts the use of continuous normalizing flow in large-scale problems is that the ODE (Eq.~\ref{equ:f-ode})
is hard to solve when the vector field $\varphi$ is complex.  The {\em rectified flow}~\cite{liu2022flow} addresses this by encouraging the flow to follow the linear path:
\begin{align}
    \min_\varphi \int_0^T \mathbb{E}_{\vx_0\sim p_0(\vx),\vx_T\sim p_T(\vx)}\left[\|(\vx_T-\vx_0)-\varphi(t, \vx_t)\|^2\right]dt, \ \ \ \vx_t = t\vx_T + (1-t)\vx_0.\label{equ:rf}
\end{align}

Here, the target distribution $p_T(\vx)$ (from which $\vx_T$ are sampled) and the initial distribution $p_0(\vx)$ (from which $\vx_0$ are sampled) are known. $\vx_t,t\in[0,T]$ is the interpolated point between $\vx_T$ and $\vx_0$, and the rectified flow encourages the vector field to align as closely as possible with the straight line $\vx_T-\vx_0$.

As discussed in the introduction, the application of diffusion models or continuous normalizing flow in generative AI tasks relies on access to a known target distribution $p_T(\vx)$, but in our optimal CA design task, $p_T(\vx)$ is unknown and needs to be optimized. 


% \dcp{same comment, do you want int from $0$ to $T$, and $Z_T$ not $Z_1$?}

% A rectified flow converting $X_0\sim p_0(\vx)$ to $X_T\sim p_1(z_1)$ \dcp{do you want $Z_T, p_T, z_T$ here?} is an ODE
% %
% \begin{align}
%     dZ_t = \varphi(t, Z_t) dt,
% \end{align}
%
% where $\varphi(t, Z_t)$ \dcp{earlier $\varphi(t, z_t)$}
% is trained to drive  $(Z_1-Z_0)$:

\section{\underline{V}ision \underline{L}anguage \underline{D}isinformation Detection \underline{Bench}mark}  
\label{method}  
\textsf{\textbf{\textsc{VLDBench}}} (Figure~\ref{fig:vlbias}) is a comprehensive classification multimodal benchmark for disinformation detection in news articles. It comprises 31,339 articles and visual samples curated from 58 news sources ranging from the Financial Times, CNN, and New York Times to Axios and Wall Street Journal as shown in Figure \ref{fig:news_sources_distribution}. \textsf{\textbf{\textsc{VLDBench}}} spans 13 unique categories (Figure \ref{fig:news_categories}) : \textit{National, Business and Finance, International, Entertainment, Local/Regional, Opinion/Editorial, Health, Sports, Politics, Weather and Environment, Technology, Science, and Other} —adding depth to the disinformation domains. We present the further statistical details in Appendix \ref{app:data-analysis}.
 
\subsection{Task Definition}
\textbf{Disinformation Detection:}  
The core task is to determine whether a text–image news article contains disinformation. We adopt the following definition:  

\emph{“False, misleading, or manipulated information—textual or visual—intentionally created or disseminated to deceive, harm, or influence individuals, groups, or public opinion.”}  

This definition aligns with established social science research \cite{benkler2018network} and governance frameworks \cite{unesco2023journalism}. We specifically focus on the `intent' behind disinformation, which remains relevant over time but has broader effects beyond just being factually incorrect.
\begin{figure*}
    \centering
    \includegraphics[width=0.95\textwidth]{figures/qualitative_figure.pdf}
    \vspace{-1em}
\caption{Disinformation Trends Across News Categories: We analyze the likelihood of disinformation across different categories, based on disinformation narratives and confidence levels generated by GPT-4o.}
    \vspace{-1em}
    \label{fig:disinfo-analysis}
\end{figure*}
\subsection{Data Pipeline}  
\textbf{Dataset Collection:}  
From May 6, 2023, to September 6, 2023, we aggregated data via Google RSS feeds from diverse news sources (Table~\ref{tab:sources}), adhering to Google’s terms of service \cite{google_tos}. We carefully curated  high-quality visual samples from these news sources to ensure a diverse representation of topics. All data collection complied with ethical guidelines \cite{uwaterloo_ethics_review}, regarding intellectual property and privacy protection. 

\textbf{Quality Assurance.}  
Collected articles underwent a rigorous human review and pre-processing phase. First, we removed entries with incomplete text, low-resolution or missing images, duplicates, and media-focused URLs (e.g., \texttt{/video}, \texttt{/gallery}). Articles with fewer than 20 sentences were discarded to ensure textual depth. For each article, the first image was selected to represent the visual context. We periodically reviewed the quality of the curated data to ensure the API returned valid and consistent results. These steps yielded over 31k curated text-image news articles that are moved to the annotation pipeline.

\subsection{Annotation Pipeline}  
To the best of our knowledge, \textsf{\textbf{\textsc{VLDBench}}} is the largest and most comprehensive humanly verified disinformation detection benchmark with over 300 hours of human verification. Figure \ref{fig:vlbias} shows our semi-annotated, data collection and annotation pipeline. After quality assurance, each article was prompted and categorized by GPT-4o as either \texttt{Likely} or \texttt{Unlikely} to contain disinformation, a binary choice designed to balance nuance with manageability. To ensure reliability, GPT-4o assessed text-image alignment three times per sample—first, to minimize random variance in its responses, and second, to resolve potential ties in classification, with an odd number of evaluations ensuring a definitive outcome. GPT-4o was chosen for this task due to its demonstrated effectiveness in both textual \cite{kim2024meganno+} and visual reasoning tasks \cite{shahriar2024putting}. An example is shown in Figure \ref{fig:disinfo-analysis}.

We categorized our data into 13 unique news categories (Figure \ref{fig:news_categories}) by providing image-text pairs to GPT-4o, drawing inspiration from AllSlides \cite{allsides_mediabiaschart}  and frameworks like Media Cloud \cite{media_cloud}. The dataset statistics are given in Table~\ref{tab:dataset_statistics}.

\begin{figure}[ht]
    \centering
    \includegraphics[width=0.48\textwidth]{figures/news_categories_distribution.pdf}
    \caption{Category distribution with overlaps. Total unique articles = 31,339. Percentages sum to \(>100\%\) due to multi-category articles.}
    \label{fig:news_categories}
    \vspace{-1em}
\end{figure}

To ensure high-quality benchmarking, a team of 22 domain experts (Appendix~\ref{app:team}) systematically reviewed the GPT-4o labels and rationales, assessing their accuracy, consistency, and alignment with human judgment. This process included a rigorous structured reconciliation phase, refining annotation guidelines and finalizing the labels. The evaluation resulted in a Cohen’s $\kappa$ of 0.78, indicating strong inter-annotator agreement.


\paragraph{Stability of Automatic Annotations:} To assess the reliability of automated annotations, we conducted a controlled experiment comparing GPT-4o labels with those of human annotators. We randomly selected 1,000 GPT-4o-annotated samples from the previous step, provided annotation guidelines, and asked domain experts (without showing the GPT-4o labels) to manually annotate them. Comparing both sets of labels, GPT-4o achieved an F1 score of 0.89 and an MCC of 0.77, while human annotators scored F1 = 0.92 and MCC = 0.81 (Figure~\ref{fig:alignment_metrics}). These results demonstrate the effectiveness of our semi-annotated pipeline, aligning well with human judgment and ensuring reliable automated labeling.

\begin{table*}[!t]
    \centering
    \resizebox{0.8\textwidth}{!}{
    \begin{tabular}{@{}l|c|c|c|c|c||c@{}}
        \toprule
        & \makecell{MATRES} & \makecell{TB-Dense} & \makecell{TCR} & \makecell{TDD-Manual} & \makecell{NarrativeTime} & \makecell{\textbf{\App{}}} \\
        \midrule
        \multicolumn{7}{c}{\textbf{Datasets Statistics}} \\
        \midrule
        Documents & 275 & 36 & 25 & 34 & 36 & 30 \\
        Events & 6,099 & 1,498 & 1,134 & 1,101 & 1,715 & 470 \\
        \midrule
        \textit{before} & 6,852 (50) & 1,361 (21) & 1,780 (67) & 1,561 (25) & 17,011 (22) & 1,540 (44) \\
        \textit{after} & 4,752 (35) & 1,182 (19) & 862 (33) & 1,054 (17) & 18,366 (23) & 1,347 (39) \\
        \textit{equal} & 448 (4) & 237 (4) & 4 (0) & 140 (2) & 5,298 (7) & 150 (4) \\
        \textit{vague} & 1,525 (11) & 2,837 (45) & -- & -- & 25,679 (33) & 446 (13) \\
        \textit{includes} & -- & 305 (5) & -- & 2,008 (33) & 5,781 (7) & -- \\
        \textit{is-included} & -- & 383 (6) & -- & 1,387 (23) & 6,639 (8) & -- \\
        \textit{overlaps} & -- & -- & -- & -- & 227 (0) & -- \\
        \midrule
        Total Relations & 13,577 & 6,305 & 2,646 & 6,150 & 79,001 & 3,483 \\
        \midrule
        \multicolumn{7}{c}{\textbf{Per Document Average Annotation Sparsity}} \\
        \midrule
        Events & 22.2 & 41.6 & 45.4 & 32.4 & 47.6 & 15.6 \\
        Actual Relations & 49.4 & 183.7 & 105.8 & 180.9 & 1,110.1 & 114.9 \\
        Expected Relations & 234.8 & 844.5 & 1,006.1 & 508.1 & 1,110.1 & 114.9 \\
        \midrule
        Missing Relations & 79\% & 78.3\% & 89.5\% & 64.4\% & 0\% & 0\% \\
        \bottomrule
    \end{tabular}}
    \caption{The upper part of the table presents the statistics of notable datasets for the temporal relation extraction task alongside \App{}. In parentheses, the values indicate the percentage of each relation type relative to the total relations in the dataset. The bottom part of the table summarizes the average percentage of missing relations per document, calculated as the ratio of actual annotated relations to a complete relation coverage, referred to as \textit{Expected Relations}.}
    \label{tab:stats_all}
\end{table*}


% \begin{table*}[!t]
%     \centering
%     \resizebox{0.8\textwidth}{!}{
%     \begin{tabular}{@{}l|c|c|c|c|c|c@{}}
%         \toprule
%         & \makecell{MATRES} & \makecell{TBD} & \makecell{TCR} & \makecell{TDD-Man} & \makecell{NarrativeTime} & \makecell{\App{}} \\
%         \midrule
%         Docs & 275 & 36 & 25 & 34 & 36 & 30 \\
%         Events & 6,099 & 1,498 & 1,134 & 1,101 & 1,715 & 470 \\
%         \midrule
%         Before (\%) & 6,852 (50) & 1,361 (21) & 1,780 (67) & 1,561 (25) & 17,011 (22) & 1,540 (44) \\
%         After (\%) & 4,752 (35) & 1,182 (19) & 862 (33) & 1,054 (17) & 18,366 (23) & 1,347 (39) \\
%         Equal (\%) & 448 (4) & 237 (4) & 4 (0) & 140 (2) & 5,298 (7) & 150 (4) \\
%         Vague (\%) & 1,525 (11) & 2,837 (45) & -- & -- & 25,679 (33) & 446 (13) \\
%         Includes (\%) & -- & 305 (5) & -- & 2,008 (33) & 5,781 (7) & -- \\
%         IsIncluded (\%) & -- & 383 (6) & -- & 1,387 (23) & 6,639 (8) & -- \\
%         Overlaps (\%) & -- & -- & -- & -- & 227 (0) & -- \\
%         \midrule
%         Total Rels & 13,577 & 6,305 & 2,646 & 6,150 & 79,001 & 3,483 \\
%         \bottomrule
%     \end{tabular}}
%     \caption{Statistics of notable datasets for the temporal relation extraction task.}
%     \label{tab:stats}
% \end{table*}



\section{Theoretical Analysis}\label{sec:theory}
% Recall that the proposed \name~learning framework aims to find Q-function parameters $\theta$ that can closely approximate the gradients of Q-function evaluated on 

In this section, we study the convergence property of our method and the error bounds of the solutions it finds. We work with mild assumptions that the gradient of the TD loss is Lipschitz smooth with constant $L$: $\|\nabla \mathcal L(\theta') - \nabla \mathcal L(\theta)\| \leq L\|\theta' - \theta\|$, and that the gradient is bounded by $\sigma$: $\| \nabla \mathcal L(\theta) \| \leq \sigma$.

% Let $L_{T}(\theta_{t}; \mathcal{D})$ denote the training 
% \begin{align}
%     L_{T}(\theta_{t}; \mathcal{D}) = 
%     \mathcal L(\theta)=\sum_{i \in \mathcal{D}}\mathcal L^i(\theta) = \sum_{i \in \mathcal{D}}\mathcal L_{\mathtt{TD}}(s_{i}, a_{i}, r_i, s'_{i}, \theta)
% \end{align}
% loss on the original dataset $\mathcal{D}$.
% Then, we suppose the training loss $L_T$ and 
%  and 
% Similarly, we let $L_{T}(\theta_{t}; \mathcal{S})$ denote the training loss on the reduced subset $\mathcal{S}$ with the same smooth assumption.
% \begin{align}
% \mathcal L_{\rdcshort}(\vw,\theta) = \sum\nolimits_{i \in \mathcal{S}} w_i\mathcal L^i(\theta)
% \end{align}
% Let $\Theta_t$ be the angle between $\nabla_{\theta} L_{T}(\theta_t; \mathcal{D})$ and $\nabla_{\theta} L_{T}(\theta_{t}; \mathcal{S})$.
% The cosine similarity of $\Theta_t$ is $\cos\Theta_t = \frac{\nabla_{\theta} L_{T}(\theta_t; \mathcal{D})^T\nabla_{\theta} L_{T}(\theta_{t}; \mathcal{S})}{\|\nabla_{\theta} L_{T}(\theta_t; \mathcal{S})\|\|\nabla_{\theta} L_{T}(\theta_{t}; \mathcal{S})\|}$.
Firstly, we show that the TD loss of the offline Q function $Q^{\pi_\mathcal{S}}$ trained on the reduced dataset $\mathcal{S}$ can converge.
\begin{restatable}{theorem}{convergence}\label{thm:convergence}
    \label{thm:convergence}
    Let $\theta^*$ denote the optimal $Q^{\pi_\mathcal{S}}$ parameters, $\theta_t$ the parameters after $t$ training steps. We have
    \begin{align}
        \min_{t=1:G}\mathcal{L}(\theta_t)\leq \mathcal{L}(\theta^*) + \frac{D\sigma}{\sqrt{G}} + \frac{D}{G}\sum_{t=1}^{G-1}\varepsilon.
    \end{align}
    Here 
    $\mathcal{L}(\theta)=\sum_{i \in \mathcal{D}}\mathcal L_{\mathtt{TD}}(s_{i}, a_{i}, r_i, s'_{i}, \theta)$ is the TD loss, $G$ is the number of total training steps, $D=\|\theta^*-\theta_t\|$, and $\varepsilon=\operatorname{Err}\left(\vw, \mathcal{S}, \mathcal L, \theta_t\right)$ is the gradient approximation errors.
\end{restatable}
\begin{proof}
    Please refer to Appendix~\ref{appendix: convergence} for detailed proof. 
\end{proof}

We assume the gradients of selected data are diverse and they can be divided into $K$ clusters $\{\mathcal{C}_1,\cdots,\mathcal{C}_K\}$ with the cluster centers set $\mathcal C=\{c_1,\cdots,c_K\}$.
Then, we prove the residual error $\operatorname{Err}\left(\vw, \mathcal{S}, \mathcal L, \theta\right)$ can be upper bounded:

% The cornerstone of this approach, i.e., the relationship between the dataset selection problem and the clustering problem, is shown in the following theorem.
\begin{restatable}{theorem}{cluster}\label{thm:cluster}
% \begin{theorem}\label{thm:cluster}
The residual error $\operatorname{Err}\left(\vw, \mathcal{S}, \mathcal L, \theta\right)$ is upper bounded according to the sample's gradient of TD loss:
\begin{align}
    \min_{\mathcal C}\sum_{i\in\mathcal D} \min_{c\in \mathcal C}\|\nabla_{\theta} \mathcal L^i\left(\theta\right) - \nabla_{\theta} \mathcal L^c\left(\theta\right) \|_2. 
\end{align}
% \end{theorem}
\end{restatable}
\begin{proof}
Please refer to Appendix~\ref{appendix: cluster theory} for detailed proof.
\end{proof}

We then prove that the reduced dataset selected by our method can achieve a good approximation for the gradient calculated on the complete dataset, which also means $\varepsilon=\operatorname{Err}\left(\vw, \mathcal{S}, \mathcal L, \theta_t\right)$ in Theorem~\ref{thm:convergence} is bounded.

\begin{corollary}[Approximation Error Bound of the Reduced Dataset]\label{thm:c_bound}
    The expected gradient approximation error achieved by our method is at most $5(\ln K+2)$ times the error of the optimal solution $\mathcal{S}^*$:
    \begin{align}
        \operatorname{Err}\left(\vw, \mathcal{S}, \mathcal L, \theta\right) \le 5(\ln K+2)\operatorname{Err}\left(\vw, \mathcal{S}^*, \mathcal L, \theta\right).
    \end{align}
\end{corollary}
\begin{proof}
The proof is derived by applying Theorem~\ref{thm:cluster} along with Theorem 4.3 from~\citep{makarychev2020improved}, by observing that cluster centers are included in the reduced dataset. 
% This gradient approximation error bound ensures that our reduced dataset will not result in significant performance degradation.
\end{proof}
% \vspace{-1cm}

\paragraph{Discussion}
{The aforementioned theoretical analysis has the following limitations: 
First, we assume that the gradients are uniformly bounded.
Therefore, if the gradients of the algorithm diverge in practice, it would contradict our assumptions, and the selected data subset would no longer be valuable. 
Current offline RL methods can only ensure that the Q-values do not diverge~\cite{kumar2020conservative, fujimoto2021minimalist}.
Although this can, to some extent, reflect the gradients of the Q-network that have not diverged, there is no rigorous proof that the bounds of the gradients can be guaranteed. 
Second, the above theoretical analysis is based on the classic TD loss.
However, to provide a consistent learning signal and mitigate instability caused by changing target value, the techniques in Section~\ref{sec:method:outer} adopt a fixed target rather than TD loss.}

\section{Experiments}
\label{sec:experiments}

\subsection{Next K-mer Prediction}
\label{sec:kmer_predition}
\begin{figure}[t]
    \centering
    \includegraphics[width=0.5\textwidth]{figures/pdf/kmer_prediction_main_text.pdf}
    \caption{Evaluation of next K-mer prediction. (A) Accuracy of the next K-mer prediction task across various tokenizers and input token lengths. (B) Comparison of the \textbf{Gener}\textit{ator} against baseline models on a dataset comprised exclusively mammalian DNA.}
    \label{fig:kmer_main}
\end{figure}

As mentioned in \textit{Sec.} \ref{sec:tokenization}, we conducted extensive experiments to explore the most suitable tokenizer for training causal DNA language models. This was achieved by training multiple models on identical datasets, each employing a different tokenizer. All models share the same architecture as the \textbf{Gener}\textit{ator} and are uniformly compared at 32,000 training steps. We employed the accuracy of the next K-mer prediction task as our evaluation metric. This zero-shot task facilitates a direct assessment of the pre-trained model quality, ensuring equitable comparisons across various tokenizers. As depicted in \textit{Fig.} \ref{fig:kmer_main}A, the tested tokenizers include BPE tokenizers with vocabulary sizes ranging from 512 to 8192, and K-mer tokenizers with K values from 1 to 8 (noting that the single nucleotide tokenizer corresponds to a K-mer tokenizer with K=1). Overall, K-mer tokenizers demonstrate superior performance compared to BPE tokenizers. Among the K-mer tokenizers, the 6-mer tokenizer is selected for its robust performance with limited input tokens and its ability to maintain top-tier performance as the number of input tokens increases.

Moreover, we evaluated the performance of Mamba \cite{Mamba,Mamba-2}, recognized for its capacity in handling long-context pre-training. To adequately assess its capabilities, we configured a Mamba model utilizing the single nucleotide tokenizer with 1.2B parameters and a context length of 98k bp. The Mamba model is compared to the 1-mer and 6-mer models under varied configurations. The comparison with the 1-mer model is straightforward; the Mamba model (denoted as Mamba\texttimes1 in \textit{Fig.} \ref{fig:kmer_main}A) exhibits slightly better performance with fewer input tokens but underperforms as the token count increases. Despite Mamba's context length being six times that of the 1-mer model, this feature does not translate into improved performance. This might suggest that Mamba's renowned ability to handle long-context pre-training primarily refers to cost-effective training rather than enhanced model performance \cite{Empirical, DeciMamba}. To compare against the 6-mer model, we adjust the input token count for the Mamba model by a factor of six (denoted as Mamba\texttimes6) to compare the models on the same base-pair basis. In this context, Mamba\texttimes6 shows slightly better performance with fewer input tokens; however, it rapidly lags as the token count increases. These findings collectively indicate that a transformer decoder architecture paired with a 6-mer tokenizer provides the most effective approach for training causal DNA language models, aligning with the configuration of the \textbf{Gener}\textit{ator}.

We further compared the \textbf{Gener}\textit{ator} model with other baseline models to evaluate their generative capabilities. As illustrated in \textit{Fig.} \ref{fig:kmer_main}B, we assess model performance using a dataset composed exclusively of mammalian DNA, given that HyenaDNA and GROVER are trained solely on human genomes. The \textbf{Gener}\textit{ator} significantly outperforms other baseline models, including its variant, \textbf{Gener}\textit{ator}-All, which incorporates pre-training on non-gene regions. This suggests that the gene sequence training strategy, which emphasizes semantically rich regions, provides a more effective training scheme compared to the conventional whole sequence training. This effectiveness is likely due to the sparsity of gene segments in the whole genome (less than 10\%) and the disproportionate importance of these segments. Among the other baseline models, NT-multi demonstrates the best performance, likely attributable to its extensive model scale (2.5B parameters), yet it still lags significantly behind the \textbf{Gener}\textit{ator}. This result aligns with expectations, as the MLM training paradigm is recognized for its limitations in generative capabilities. Meanwhile, HyenaDNA, despite utilizing the NTP training paradigm, does not show improved performance compared to other masked language models, likely due to its overly small model size (55M parameters), insufficient for exhibiting robust generative abilities. This comparison underscores the critical role of the \textbf{Gener}\textit{ator} in bridging the gap for large-scale generative DNA language models within the eukaryotic domain.

Due to space constraints, we have chosen only to demonstrate specific examples with mammalian DNA data and a fixed K-mer prediction length of 16 bp in \textit{Fig.} \ref{fig:kmer_main}. A more comprehensive analysis across various taxonomic groups and K-mer lengths is provided in the appendix.

\subsection{Benchmark Evaluations}
In this section, we compare the \textbf{Gener}\textit{ator} with state-of-the-art genomic foundation models: Enformer~\cite{enformer}, DNABERT-2, HyenaDNA, Nucleotide Transformer, Caduceus, and GROVER, across various benchmark tasks. To ensure a fair comparison, we uniformly fine-tune each model and perform a 10-fold cross-validation on all datasets. For each model on each dataset, we conduct a hyperparameter search, exhaustively tuning learning rates in $\{1e^{-5}, 2e^{-5}, 5e^{-5}, \ldots, 1e^{-3}, 2e^{-3}, 5e^{-3}\}$ and batch sizes in $\{64, 128, 256, 512\}$. Detailed hyperparameter settings and implementation specifics are provided in the appendix.

\paragraph{Nucleotide Transformer Tasks}
Since the NT task dataset was revised recently~\cite{nucleotide-transformer}, we conducted experiments on both the original and revised datasets. The results for the revised NT tasks are provided in Table~\ref{tab:nucleotide_transformer_tasks_revised}, and the results for the original NT tasks are provided in Table~\ref{tab:nucleotide_transformer_tasks}. Overall, the \textbf{Gener}\textit{ator} outperforms other baseline models. However, the \textbf{Gener}\textit{ator}-All variant shows some performance decline. Notably, despite its earlier release, Enformer continues to deliver competitive results in chromatin profile and regulatory element tasks. This performance could be attributed to its original training in a supervised manner specifically for chromatin and gene expression tasks. The latest release of Nucleotide Transformer, NT-v2, although smaller in size (500M), demonstrates enhanced performance compared to NT-multi (2.5B). In contrast, DNABERT-2 and GROVER, which utilize BPE tokenizers, along with HyenaDNA and Caduceus, which employ the finer-grained single nucleotide tokenizer, do not show distinct performance advantages, likely due to the limited model scope and data scale.

\paragraph{Genomic Benchmarks}
We also conducted a comparative analysis on the Genomic Benchmarks~\cite{genomic-benchmarks}, which primarily focus on the human genome. The evaluation results are provided in Table~\ref{tab:genomic_benchmarks}. Overall, the \textbf{Gener}\textit{ator} still outperforms other models. However, it is worth noting that the Caduceus models also exhibit comparable performance while being significantly smaller (8M). This is likely due to the fact that Caduceus models are trained exclusively on the human genome, making them efficient and compact. Nevertheless, this exclusivity may limit their generalizability to other genomic contexts.

\paragraph{Gener Tasks} 
Lastly, we evaluated the newly proposed Gener tasks, which focus on assessing genomic context comprehension across various sequence lengths and organisms. As shown in Table~\ref{tab:gener_tasks}, the \textbf{Gener}\textit{ator} achieves the best performance on both gene and taxonomic classification tasks, with NT-v2 also demonstrating similar results. Further details on the evaluation of Gener tasks, including visualizations of confusion matrices, are provided in the appendix. The superior performance of the \textbf{Gener}\textit{ator} and NT-v2 is likely due to their pre-training on multispecies datasets. In contrast, despite also being trained on multispecies data, DNABERT-2 exhibits noticeable performance degradation. This may be attributed to its limited model size (117M for DNABERT-2, 500M for NT-v2, and 1.2B for \textbf{Gener}\textit{ator}) and shorter context length (3k for DNABERT-2, 12k for NT-v2, and 98k for \textbf{Gener}\textit{ator}). Other models, such as HyenaDNA and Caduceus, although trained exclusively on the human genome, still exhibit relevant generalizability on both tasks after fine-tuning, attributable to their long-context capacity (\textgreater 100k). GROVER, on the other hand, significantly lags behind in taxonomic classification due to its limited context length (3k).

\begin{table*}[!htb]
\small
\renewcommand{\arraystretch}{1}
\centering
\caption{Evaluation of the revised Nucleotide Transformer tasks. The reported values represent the Matthews correlation coefficient (MCC) averaged over 10-fold cross-validation, with the standard error in parentheses.}
\resizebox{\textwidth}{!}{
\begin{tabular}{lcccccccccc}
\toprule
& Enformer & DNABERT-2 & HyenaDNA & NT-multi & NT-v2 & Caduceus-Ph & Caduceus-PS & GROVER & \textbf{Gener}\textit{ator} & \textbf{Gener}\textit{ator}-All \\
& (252M) & (117M) & (55M) & (2.5B) & (500M) & (8M) & (8M) & (87M) & (1.2B) & (1.2B) \\
\midrule
H2AFZ          & 0.522 (0.019) & 0.490 (0.013) & 0.455 (0.015) & 0.503 (0.010) & \underline{0.524 (0.008)} & 0.417 (0.016) & 0.501 (0.013) & 0.509 (0.013) & \textbf{0.529 (0.009)} & 0.506 (0.019) \\
H3K27ac        & \underline{0.520 (0.015)} & 0.491 (0.010) & 0.423 (0.017) & 0.481 (0.020) & 0.488 (0.013) & 0.464 (0.018) & 0.464 (0.022) & 0.489 (0.023) & \textbf{0.546 (0.015)} & 0.496 (0.014) \\
H3K27me3       & 0.552 (0.007) & 0.599 (0.010) & 0.541 (0.018) & 0.593 (0.016) & \underline{0.610 (0.006)} & 0.547 (0.010) & 0.561 (0.036) & 0.600 (0.008) & \textbf{0.619 (0.008)} & 0.590 (0.014) \\
H3K36me3       & 0.567 (0.017) & \underline{0.637 (0.007)} & 0.543 (0.010) & 0.635 (0.016) & 0.633 (0.015) & 0.543 (0.009) & 0.602 (0.008) & 0.585 (0.008) & \textbf{0.650 (0.006)} & 0.621 (0.013) \\
H3K4me1        & \textbf{0.504 (0.021)} & \underline{0.490 (0.008)} & 0.430 (0.014) & 0.481 (0.012) & \underline{0.490 (0.017)} & 0.411 (0.012) & 0.434 (0.030) & 0.468 (0.011) & \textbf{0.504 (0.010)} & \underline{0.490 (0.016)} \\
H3K4me2        & \textbf{0.626 (0.015)} & 0.558 (0.013) & 0.521 (0.024) & 0.552 (0.022) & 0.552 (0.013) & 0.480 (0.013) & 0.526 (0.035) & 0.558 (0.012) & \underline{0.607 (0.010)} & 0.569 (0.012) \\
H3K4me3        & 0.635 (0.019) & \underline{0.646 (0.008)} & 0.596 (0.015) & 0.618 (0.015) & 0.627 (0.020) & 0.588 (0.020) & 0.611 (0.015) & 0.634 (0.011) & \textbf{0.653 (0.008)} & 0.628 (0.018) \\
H3K9ac         & \textbf{0.593 (0.020)} & 0.564 (0.013) & 0.484 (0.022) & 0.527 (0.017) & 0.551 (0.016) & 0.514 (0.014) & 0.518 (0.018) & 0.531 (0.014) & \underline{0.570 (0.017)} & 0.556 (0.018) \\
H3K9me3        & 0.453 (0.016) & 0.443 (0.025) & 0.375 (0.026) & 0.447 (0.018) & 0.467 (0.044) & 0.435 (0.019) & 0.455 (0.019) & 0.441 (0.017) & \textbf{0.509 (0.013)} & \underline{0.480 (0.037)} \\
H4K20me1       & 0.606 (0.016) & \underline{0.655 (0.011)} & 0.580 (0.009) & 0.650 (0.014) & 0.654 (0.011) & 0.572 (0.012) & 0.590 (0.020) & 0.634 (0.006) & \textbf{0.670 (0.006)} & 0.652 (0.010) \\
Enhancer       & \textbf{0.614 (0.010)} & 0.517 (0.011) & 0.475 (0.006) & 0.527 (0.012) & 0.575 (0.023) & 0.480 (0.008) & 0.490 (0.009) & 0.519 (0.009) & \underline{0.594 (0.013)} & 0.553 (0.020) \\
Enhancer type & \textbf{0.573 (0.013)} & 0.476 (0.009) & 0.441 (0.010) & 0.484 (0.012) & 0.541 (0.013) & 0.461 (0.009) & 0.459 (0.011) & 0.481 (0.009) & \underline{0.547 (0.017)} & 0.510 (0.022) \\
Promoter all   & 0.745 (0.012) & 0.754 (0.009) & 0.693 (0.016) & 0.761 (0.009) & \underline{0.780 (0.012)} & 0.707 (0.017) & 0.722 (0.014) & 0.721 (0.011) & \textbf{0.795 (0.005)} & 0.765 (0.009) \\
Promoter non-TATA & 0.763 (0.012) & 0.769 (0.009) & 0.723 (0.013) & 0.773 (0.010) & 0.785 (0.009) & 0.740 (0.012) & 0.746 (0.009) & 0.739 (0.018) & \textbf{0.801 (0.005)} & \underline{0.786 (0.007)} \\
Promoter TATA  & 0.793 (0.026) & 0.784 (0.036) & 0.648 (0.044) & \underline{0.944 (0.016)} & 0.919 (0.028) & 0.868 (0.023) & 0.853 (0.034) & 0.891 (0.041) & \textbf{0.950 (0.009)} & 0.862 (0.024) \\
Splice acceptor & 0.749 (0.007) & 0.837 (0.006) & 0.815 (0.049) & 0.958 (0.003) & \textbf{0.965 (0.004)} & 0.906 (0.015) & 0.939 (0.012) & 0.812 (0.012) & \underline{0.964 (0.003)} & 0.951 (0.006) \\
Splice site all & 0.739 (0.011) & 0.855 (0.005) & 0.854 (0.053) & 0.964 (0.003) & \textbf{0.968 (0.003)} & 0.941 (0.006) & 0.942 (0.012) & 0.849 (0.015) & \underline{0.966 (0.003)} & 0.959 (0.003) \\
Splice donor   & 0.780 (0.007) & 0.861 (0.004) & 0.943 (0.024) & 0.970 (0.002) & \underline{0.976 (0.003)} & 0.944 (0.026) & 0.964 (0.010) & 0.842 (0.009) & \textbf{0.977 (0.002)} & 0.971 (0.002) \\
\bottomrule
\end{tabular}
}
\label{tab:nucleotide_transformer_tasks_revised}
\end{table*}
\begin{table*}[!htb]
\small
\renewcommand{\arraystretch}{1.2}
\centering
\caption{Evaluation of the original Nucleotide Transformer tasks. The reported values represent the Matthews correlation coefficient (MCC) averaged over 10-fold cross-validation, with the standard error in parentheses.}
\resizebox{\textwidth}{!}{%
\begin{tabular}{lcccccccccc}
\toprule
& Enformer & DNABERT-2 & HyenaDNA & NT-multi & NT-v2 & Caduceus-Ph & Caduceus-PS & GROVER & \textbf{Gener}\textit{ator} & \textbf{Gener}\textit{ator}-All \\
& (252M) & (117M) & (55M) & (2.5B) & (500M) & (8M) & (8M) & (87M) & (1.2B) & (1.2B) \\
\midrule
H3 & 0.724 (0.018) & 0.785 (0.012) & 0.781 (0.015) & 0.793 (0.013) & 0.788 (0.010) & 0.794 (0.012) & 0.772 (0.022) & 0.768 (0.008) & \textbf{0.806 (0.005)} & \underline{0.803 (0.007)} \\
H3K14ac & 0.284 (0.024) & 0.515 (0.009) & \textbf{0.608 (0.020)} & 0.538 (0.009) & 0.538 (0.015) & 0.564 (0.033) & 0.596 (0.038) & 0.548 (0.020) & \underline{0.605 (0.008)} & 0.580 (0.038) \\
H3K36me3 & 0.345 (0.019) & 0.591 (0.005) & 0.614 (0.014) & 0.618 (0.011) & 0.618 (0.015) & 0.590 (0.018) & 0.611 (0.048) & 0.563 (0.017) & \textbf{0.657 (0.007)} & \underline{0.631 (0.013)} \\
H3K4me1 & 0.291 (0.016) & 0.512 (0.008) & 0.512 (0.008) & 0.541 (0.005) & 0.544 (0.009) & 0.468 (0.015) & 0.487 (0.029) & 0.461 (0.018) & \textbf{0.553 (0.009)} & \underline{0.549 (0.018)} \\
H3K4me2 & 0.207 (0.021) & 0.333 (0.013) & \textbf{0.455 (0.028)} & 0.324 (0.014) & 0.302 (0.020) & 0.332 (0.034) & \underline{0.431 (0.016)} & 0.403 (0.042) & 0.424 (0.013) & 0.400 (0.015) \\
H3K4me3 & 0.156 (0.022) & 0.353 (0.021) & \textbf{0.550 (0.015)} & 0.408 (0.011) & 0.437 (0.028) & 0.490 (0.042) & \underline{0.528 (0.033)} & 0.458 (0.022) & 0.512 (0.009) & 0.473 (0.047) \\
H3K79me3 & 0.498 (0.013) & 0.615 (0.010) & 0.669 (0.014) & 0.623 (0.010) & 0.621 (0.012) & 0.641 (0.028) & \textbf{0.682 (0.018)} & 0.626 (0.026) & \underline{0.670 (0.011)} & 0.631 (0.021) \\
H3K9ac & 0.415 (0.020) & 0.545 (0.009) & 0.586 (0.021) & 0.547 (0.011) & 0.567 (0.020) & 0.575 (0.024) & 0.564 (0.018) & 0.581 (0.015) & \textbf{0.612 (0.006)} & \underline{0.603 (0.019)} \\
H4 & 0.735 (0.023) & 0.797 (0.008) & 0.763 (0.012) & \underline{0.808 (0.007)} & 0.795 (0.008) & 0.788 (0.010) & 0.799 (0.010) & 0.769 (0.017) & \textbf{0.815 (0.008)} & \underline{0.808 (0.010)} \\
H4ac & 0.275 (0.022) & 0.465 (0.013) & 0.564 (0.011) & 0.492 (0.014) & 0.502 (0.025) & 0.548 (0.027) & \underline{0.585 (0.018)} & 0.530 (0.017) & \textbf{0.592 (0.015)} & 0.565 (0.035) \\
Enhancer & 0.454 (0.029) & 0.525 (0.026) & 0.520 (0.031) & 0.545 (0.028) & \underline{0.561 (0.029)} & 0.522 (0.024) & 0.511 (0.026) & 0.516 (0.018) & \textbf{0.580 (0.015)} & 0.540 (0.026) \\
Enhancer type & 0.312 (0.043) & 0.423 (0.018) & 0.403 (0.056) & 0.444 (0.022) & 0.444 (0.036) & 0.403 (0.028) & 0.410 (0.026) & 0.433 (0.029) & \textbf{0.477 (0.017)} & \underline{0.463 (0.023)} \\
Promoter all & 0.910 (0.004) & 0.945 (0.003) & 0.919 (0.003) & 0.951 (0.004) & 0.952 (0.002) & 0.937 (0.002) & 0.941 (0.003) & 0.926 (0.004) & \textbf{0.962 (0.002)} & \underline{0.955 (0.002)} \\
Promoter non-TATA & 0.910 (0.006) & 0.944 (0.003) & 0.919 (0.004) & \underline{0.955 (0.003)} & 0.952 (0.003) & 0.935 (0.007) & 0.940 (0.002) & 0.925 (0.006) & \textbf{0.962 (0.001)} & \underline{0.955 (0.002)} \\
Promoter TATA & 0.920 (0.012) & 0.911 (0.011) & 0.881 (0.020) & 0.919 (0.008) & \underline{0.933 (0.009)} & 0.895 (0.010) & 0.903 (0.010) & 0.891 (0.009) & \textbf{0.948 (0.008)} & 0.931 (0.007) \\
Splice acceptor & 0.772 (0.007) & 0.909 (0.004) & 0.935 (0.005) & \underline{0.973 (0.002)} & \underline{0.973 (0.004)} & 0.918 (0.017) & 0.907 (0.015) & 0.912 (0.010) & \textbf{0.981 (0.002)} & 0.957 (0.009) \\
Splice site all & 0.831 (0.012) & 0.950 (0.003) & 0.917 (0.006) & 0.974 (0.004) & \underline{0.975 (0.002)} & 0.935 (0.011) & 0.953 (0.005) & 0.919 (0.005) & \textbf{0.978 (0.001)} & 0.973 (0.002) \\
Splice donor & 0.813 (0.015) & 0.927 (0.003) & 0.894 (0.013) & 0.974 (0.002) & \underline{0.977 (0.007)} & 0.912 (0.009) & 0.930 (0.010) & 0.888 (0.012) & \textbf{0.978 (0.002)} & 0.967 (0.005) \\
\bottomrule
\end{tabular}
}
\label{tab:nucleotide_transformer_tasks}
\end{table*}
\begin{table*}[!htb]
\small
\renewcommand{\arraystretch}{1.2}
\centering
\caption{Evaluation of the Genomic Benchmarks. The reported values represent the accuracy averaged over 10-fold cross-validation, with the standard error in parentheses.}
\resizebox{\textwidth}{!}{
\begin{tabular}{lcccccccc}
\toprule
& DNABERT-2 & HyenaDNA & NT-v2 & Caduceus-Ph & Caduceus-PS & GROVER & \textbf{Gener}\textit{ator} & \textbf{Gener}\textit{ator}-All \\
& (117M) & (55M) & (500M) & (8M) & (8M) & (87M) & (1.2B) & (1.2B) \\
\midrule
Coding vs. Intergenomic & 0.951 (0.002) & 0.902 (0.004) & 0.955 (0.001) & 0.933 (0.001) & 0.944 (0.002) & 0.919 (0.002) & \textbf{0.963 (0.000)} & \underline{0.959 (0.001)} \\
Drosophila Enhancers Stark & 0.774 (0.011) & 0.770 (0.016) & 0.797 (0.009) & \textbf{0.827 (0.010)} & 0.816 (0.015) & 0.761 (0.011) & \underline{0.821 (0.005)} & 0.768 (0.015) \\
Human Enhancers Cohn & \underline{0.758 (0.005)} & 0.725 (0.009) & 0.756 (0.006) & 0.747 (0.003) & 0.749 (0.003) & 0.738 (0.003) & \textbf{0.763 (0.002)} & 0.754 (0.006) \\
Human Enhancers Ensembl & 0.918 (0.003) & 0.901 (0.003) & 0.921 (0.004) & \textbf{0.924 (0.002)} & \underline{0.923 (0.002)} & 0.911 (0.004) & 0.917 (0.002) & 0.912 (0.002) \\
Human Ensembl Regulatory & 0.874 (0.007) & 0.932 (0.001) & \textbf{0.941 (0.001)} & \underline{0.938 (0.004)} & \textbf{0.941 (0.002)} & 0.897 (0.001) & 0.928 (0.001) & 0.926 (0.001) \\
Human non-TATA Promoters & 0.957 (0.008) & 0.894 (0.023) & 0.932 (0.006) & \textbf{0.961 (0.003)} & \textbf{0.961 (0.002)} & 0.950 (0.005) & \underline{0.958 (0.001)} & 0.955 (0.005) \\
Human OCR Ensembl & 0.806 (0.003) & 0.774 (0.004) & 0.813 (0.001) & \underline{0.825 (0.004)} & \textbf{0.826 (0.003)} & 0.789 (0.002) & 0.823 (0.002) & 0.812 (0.003) \\
Human vs. Worm & 0.977 (0.001) & 0.958 (0.004) & 0.976 (0.001) & 0.975 (0.001) & 0.976 (0.001) & 0.966 (0.001) & \textbf{0.980 (0.000)} & \underline{0.978 (0.001)} \\
Mouse Enhancers Ensembl & \underline{0.865 (0.014)} & 0.756 (0.030) & 0.855 (0.018) & 0.788 (0.028) & 0.826 (0.021) & 0.742 (0.025) & \textbf{0.871 (0.015)} & 0.784 (0.027) \\
\bottomrule
\end{tabular}
}
\label{tab:genomic_benchmarks}
\end{table*}
\begin{table*}[!htb]
\small
\renewcommand{\arraystretch}{1}
\centering
\caption{Evaluation of the Gener tasks. The reported values represent the weighted F1 score averaged over 10-fold cross-validation, with the standard error in parentheses.}
\resizebox{\textwidth}{!}{
\begin{tabular}{lcccccccc}
\toprule
& DNABERT-2 & HyenaDNA & NT-v2 & Caduceus-Ph & Caduceus-PS & GROVER & \textbf{Gener}\textit{ator} & \textbf{Gener}\textit{ator}-All \\
& (117M) & (55M) & (500M) & (8M) & (8M) & (87M) & (1.2B) & (1.2B) \\
\midrule
Gene & 0.660 (0.002) & 0.610 (0.007) & \underline{0.692 (0.005)} & 0.629 (0.005) & 0.644 (0.007) & 0.630 (0.003) & \textbf{0.700 (0.002)} & 0.687 (0.003) \\
Taxonomic & 0.922 (0.003) & 0.970 (0.024) & 0.981 (0.001) & 0.958 (0.021) & 0.968 (0.006) & 0.843 (0.006) & \textbf{0.999 (0.000)} & \underline{0.998 (0.001)} \\
\bottomrule
\end{tabular}
}
\label{tab:gener_tasks}
\end{table*}

\subsection{Central Dogma}

In our experimental setup, we selected two target protein families from the UniProt~\cite{UniProt} database: the Histone and Cytochrome P450 families. By cross-referencing gene IDs and protein IDs, we extracted the corresponding protein-coding DNA sequences from RefSeq~\cite{RefSeq}. These sequences served as training data for fine-tuning the \textbf{Gener}\textit{ator} model, directing it to generate analogous protein-coding DNA sequences.

To assess the quality of generation, we compared several summary statistics. The results for the Histone family are provided in \textit{Fig.} \ref{fig:histone_generation}, while the evaluation results for the Cytochrome P450 family are provided in \textit{Fig.} \ref{fig:cytochrome_generation}.  After deduplication, the lengths of the generated DNA sequences and their translated protein sequences, using a codon table, closely resemble the distributions observed in the target families. This preliminary validation suggests that our generated DNA sequences maintain stable codon structures that are translatable into proteins. We conducted a more in-depth structural and functional analysis of these translated protein sequences. First, we assessed whether protein language models `recognize' these generated protein sequences by calculating their perplexity (PPL) using Progen2~\cite{progen2}. The results show that the PPL distribution of generated sequences closely matches that of the natural families and significantly differs from the shuffled sequences.

Furthermore, we used AlphaFold3 to predict the folded structures of the generated protein sequences and employed Foldseek~\cite{Foldseek} to find analogous proteins in the Protein Data Bank (PDB)~\cite{RCSBPDB}. Remarkably, we identified numerous instances where the conformations of the generated sequences exhibited high similarity to established structures in the PDB ($\text{TM-score}>0.8$). This structural congruence is observed despite substantial divergence in sequence composition, as indicated by sequence identities less than $0.3$. This low sequence identity positively suggests that the model is not merely replicating existing protein sequences but has learned the underlying principles to design new molecules with similar structures. This highlights the capability of the \textbf{Gener}\textit{ator} in generating biologically relevant sequences. 

\subsection{Enhancer Design}
We employed the enhancer activity data from DeepSTARR~\cite{DeepSTARR}, following the dataset split initially proposed by DeepSTARR and later adopted by NT. Using this data, we developed an enhancer activity predictor by fine-tuning the \textbf{Gener}\textit{ator}. This predictor surpasses the accuracy of DeepSTARR and NT-multi (Table \ref{tab:enhancer_benchmark}), establishing itself as the current state-of-the-art predictor. By applying our refined SFT approach as outlined in \textit{Sec.} \ref{sec:sequence_design}, we generated a collection of candidate enhancer sequences with specific activity profiles. As illustrated in \textit{Fig.} \ref{fig:enhancer_design}, the predicted activities of these candidates exhibit significant differentiation between the generated high/low-activity enhancers and natural samples.

To our knowledge, this represents one of the first attempts to use LLMs for prompt-guided design of DNA sequences, highlighting the capability of the \textbf{Gener}\textit{ator} in this domain. These generated sequences, and more broadly, this sequence design paradigm using the \textbf{Gener}\textit{ator}, merit further exploration. Our approach underscores the potential of the \textbf{Gener}\textit{ator} model to transform DNA sequence design methodologies, providing a novel pathway for the conditional design of DNA sequences using LLMs. In our subsequent research, we plan to extend our evaluations through further validations in wet lab conditions to explore the real-world applicability of these designed sequences.

\begin{figure}[!htb]
    \centering
    \includegraphics[width=0.6\textwidth]{figures/pdf/histone_generation.pdf}
    \caption{Histone generation. (A) Distribution densities of the protein sequence lengths for generated and natural samples. (B) Distribution densities of Progen2 PPL for generated and natural samples, along with randomly shuffled sequences. (C) Scatter plot of TM-score and AlphaFold3 prediction confidence (pLDDT) with marginal distributions. (D) Two folded structures of generated samples displaying structural congruence with natural samples.}
    \label{fig:histone_generation}
\end{figure}
\begin{figure}[!htb]
    \centering
    \includegraphics[width=0.6\textwidth]{figures/pdf/cytochrome_generation.pdf}
    \caption{Cytochrome P450 generation. (A) Distribution densities of the protein sequence lengths for generated and natural samples. (B) Distribution densities of Progen2 PPL for generated and natural samples, along with randomly shuffled sequences. (C) Scatter plot of TM-score and AlphaFold3 prediction confidence (pLDDT) with marginal distributions. (D) Two folded structures of generated samples displaying structural congruence with natural samples.}
    \label{fig:cytochrome_generation}
\end{figure}

\begin{table}[!htb]
\small
\renewcommand{\arraystretch}{1}
\centering
\caption{Evaluation of the DeepSTARR dataset. The reported values represent the Pearson correlation coefficient.}
\begin{tabular}{lccc}
\toprule
 & DeepSTARR & NT-multi & \textbf{Gener}\textit{ator} \\
\midrule
Developmental & \underline{0.68} & 0.64 & \textbf{0.70} \\
Housekeeping & 0.74 & \underline{0.75} & \textbf{0.79} \\
\bottomrule
\end{tabular}
\label{tab:enhancer_benchmark}
\end{table}
\begin{figure}[!htb]
    \centering
    \includegraphics[width=0.6\textwidth]{figures/pdf/enhancer_design.pdf}
    \caption{Enhancer design. (A-B) Pearson correlation between the predicted enhancer activity and the measured activity. (C-D) Distribution densities of the predicted activity of generated enhancer sequences with distinct activity profiles.}
    \label{fig:enhancer_design}
\end{figure}

% \clearpage
\section{Related works}
\label{appx:related_work}
% As deep learning usually trains on abundant data, a considerable number of works have focused on identifying important training samples and figuring out the ideal size of the dataset. However, the employment in policy training tasks is under-explored. Our work is closely related to offline RL and data subset selection.

\textbf{Offline Reinforcement Learning.}\ \
Offline RL can execute policy training entirely based on static datasets without further interaction with the environment~\cite{levine2020offline}.
Therefore, it faces challenges such as distribution shift and value overestimation.
To address this issue, some prior works attempted to constrain the learned policy and behavior policy by limiting the action difference~\cite{fujimoto2019off}, adding KL-divergence~\cite{nair2020awac,peng2019advantage,wu2019behavior}, or regularization~\cite{kumar2019stabilizing}.
Other works consider employing conservative estimates of future values~\cite{kumar2020conservative,ma2021conservative} or penalizing uncertain actions~\cite{janner2019trust,yu2021combo,kidambi2020morel} by uncertainty.
There are also some new attempts, such as lightweight implementation~\cite{fujimoto2021minimalist} or avoiding distribution shift by single-step policy iteration~\cite{kostrikov2021offline}.
These studies provide a solid foundation for implementing and transferring reinforcement learning to real-world tasks.
However, there has been limited research addressing considerations related to the dataset.
Some works attempted to explore which dataset characteristics dominate in offline RL algorithms~\cite{schweighofer2021understanding, swazinna2021measuring} or investigate the data generation~\cite{yarats2022don}.
Recently, some researchers attempted to solve the sub-optimal trajectories issue by constraining policy to good data rather than all actions in the dataset~\cite{hong2023beyond} or re-weighting policy~\cite{hong2023harnessing}.
Differently, our work aims to figure out the ideal size of the dataset needed for effective policy training.

\textbf{Data subset selection.}\ \ The research on identifying crucial samples within datasets is concentrated in the field of computer vision.
Some prior works use uncertainty of samples~\cite{coleman2019selection,paul2021deep} or the frequency of being forgotten~\cite{toneva2018empirical} as the proxy function to prune the dataset.
Another research line focuses on constructing weighted data subsets to approximate the full dataset~\cite{feldman2020core}, which often transforms the subset selecting to the submodular set cover problem~\cite{wei2015submodularity,kaushal2019learning}.
Specifically, several works adopted loss functions as the optimization target~\cite{lucic2018training,campbell2018bayesian}, while recent research finds that approximating full gradient is more efficient~\cite{mirzasoleiman2020coresets, killamsetty2021grad,killamsetty2021glister,killamsetty2021retrieve}.
These studies establish the critical importance of selecting key samples from datasets for effective training. However, the different learning objectives and training methodologies between reinforcement learning and computer vision mean that these techniques cannot be directly applied to Offline RL.
\section{Discussion}
\label{sec:discussion}

\rv{
While the results show that \name is able to simulate human-like eye movements when performing analytical tasks, there is a need to expand on our discussion of the model's practical implications, the generalizability of the modeling approach, and the limitations and potential for supporting sophisticated chart-based question answering.
}

\subsection{\rv{Applications}}

\rv{
\paragraph{Visualization design evaluation}
\name can assist in evaluation of chart design.
With well-controlled experiment conditions, eye tracking data afford valuable insight into chart designs, especially relative to alternative designs.
For example, \citet{goldberg2011eye} showcased eye tracking's value in comparing line and radial graphs for reading of values, by allowing researchers to understand the viewing order of AOIs and the task completion time.
\name holds potential to replace human input to evaluation based on eye tracking.
With the simulated scanpaths from \name, chart designers can obtain quick and cost-effective feedback that yields the benefits from eye tracking without requiring an expensive empirical study.
}
\rv{
\paragraph{Visualization design optimization}
Beyond evaluation, another potential usage application of \name is to help optimize visualization design~\cite{shin2023perceptual}. 
Like other fields of design, visualization design requires user feedback for continual iteration. When visualization designers create charts for specific tasks, they may wonder if the design is suitable for delivery.
With the predicted scanpaths from the model, they can easily access quick and affordable feedback before deeming a candidate design ready for expensive evaluation in a user study.
Predictive models could offer feedback to designers or even provide optimization goals in automated visualization design frameworks.
The ultimate goal is grounding for recommendations for visualizations that support specific tasks~\cite{albers2014task} and even automation of visualization design in real time.
Today's human-in-the-loop design optimization paradigm~\cite{kadner2021adaptifont} could shift to a user-agent-in-the-loop approach, wherein a computational agent that simulates human feedback enables scalable and efficient design evaluation.
}

\rv{
\paragraph{Explainable AI in chart question answering}
Systems for answering questions via charts~\cite{masry2022chartqa} are typically viewed as black boxes that generate answers directly from a given chart and natural-language question. 
In contrast, \name introduces a glass-box approach that answers questions through a step-by-step reasoning process. This method enhances the alignment between human and machine attention~\cite{sood2023multimodal}.
We anticipate that this approach could lead to significant improvements in chart question answering~\cite{masry2022chartqa} and greater compatibility with explainable AI systems.
}

\subsection{\rv{Extending the Model beyond Bar Charts}}

\rv{
Our modeling approach can be extended to many visualization types besides bar charts.
We analyzed the visualization taxonomy outlined in prior work~\cite{borkin2013makes, borkin2015beyond}, including area, circle, diagram, distribution, grid, line, map, point, table, text, tree, and network, then categorize these visualization techniques into two groups: those that are feasible to extend with minor changes and those that are out of reach, requiring additional features.
}

\begin{figure}[!t]
    \centering
    \subfigure[\rv{An \textit{RV} task with a line chart: ``What was the revenue from newspaper advertising in 1980?''}]{\label{fig:a}\includegraphics[width=0.48\textwidth]{Images/line-case.png}}
    \hspace{0.02\columnwidth}
    \subfigure[\rv{An \textit{F} task with a scatterplot: ``In which countries do people anticipate spending about \$700 for personal Christmas gifts?''}]{\label{fig:b}\includegraphics[width=0.48\textwidth]{Images/point-case.png}}
    \caption{\rv{Two cases that illustrate the generalizability of the modeling approach, showing the extension of \name to a line chart and a scatterplot. The model's predictions are spatially similar to human ground-truth scanpaths.}}
    \label{fig:case}
    % \vspace{-5mm}
\end{figure}

\rv{
Our modeling approach can be applied to most statistical charts either directly or upon rectification of minor issues. For instance, extending the model to interpret \textit{line charts} and \textit{area charts} is feasible when the axis labels are clearly defined. The trend patterns of lines and areas can be perceived by the peripheral vision as visual guidance.
For \textit{point charts}, such as scatterplots, the model performs well in conditions of sparse data points. However, individual points may be obscured in dense scatterplots, making it difficult to label data when points are cluttered or overlapping. 
\textit{Distribution charts}, such as histograms, and \textit{circle charts}, such as pie charts, are similar to bar charts in that they use the area of marks to represent values. Retrieving exact values from these two presentation types can be imprecise on account of the ranges of the bins and inaccuracies in estimating angles or arc lengths.
Reading \textit{grid charts} (e.g., heatmaps) too is feasible; however, identifying the values necessitates understanding color intensity, a factor that can sometimes lead to ambiguity.
Modeling scanpaths on \textit{tables} or \textit{text} for retrieval tasks is tractable under the current modeling approach, but a lack of visual pattern recognition may render the results poor.
To further examine the generalizability of this category, we considered two additional cases, using a line chart and a scatterplot. We manually labeled the charts, trained the model, and made predictions. As Figure~\ref{fig:case} attests, the trained model performs well for these two chart types when compared to human ground-truth scanpaths.
}

\rv{
Other, sophisticated visualization types are out of reach because they require additional features, particularly prior knowledge and advanced reasoning abilities. For instance, reading \textit{maps} involves associating spatial regions with colors, sizes, or symbols to retrieve related values. Also, when interpreting maps, people rely heavily on preexisting geographical knowledge as a basis for efficient visual searches. Complex designs with intricate structures, such as \textit{diagrams}, \textit{trees}, and \textit{network graphs}, typically require advanced reasoning based on connections. All these skill requirements point to a need for further study in this area.
}

\subsection{\rv{Paths toward Sophisticated Tasks in Chart Question Answering}}

\rv{
Although the model focuses primarily on gaze prediction, it is worth exploring potential improvements for enriching its sophisticated question answering capabilities. We also discuss its limitations.
}

\rv{
Our current model does not achieve the same level of accuracy as the state-of-the-art models represented by the ChartQA benchmark~\cite{masry2022chartqa}. 
Unlike other models that can access the full chart image, \name is limited by its foveal vision and restricted spatial reasoning abilities. For instance, if a bar's height falls between two labeled values, such as 10 and 15, the model might choose either 10 or 15 as its answer when interpreting the axis, failing to provide a more precise value. 
This limitation stems from the constrained spatial perception capabilities of LLMs, which are central to cognitive control.
One possible solution is integrating multi-modal LLMs~\cite{cuarbune2024chart}, for which recent research has demonstrated an accuracy rate of 81.3\%.
}

\rv{
The sense-making process for complex visualizations may be inherently challenging. Even humans often struggle with understanding how the data are encoded, recognizing a given chart's purpose, tackling readability issues, performing numerical calculations, identifying relationships among data points, and navigating the spatial arrangement of graphical elements~\cite{rezaie2024struggles}.
Our model is designed to be straightforward and objective, focusing on analysis tasks related to statistical charts, but it does not fully capture the complexities of visualizations.
A possible enhancement in this respect would be to integrate the model with human sense-making practices~\cite{rezaie2024struggles} or to incorporate a framework of human understanding~\cite{albers2014task}. Such integration could facilitate better simulation of a human-like problem-solving process.
}


\bibliography{iclr2025_conference}
\bibliographystyle{iclr2025_conference}

\newpage
\appendix
\onecolumn
% % \section{Algorithm}
% \label{appendix: alg}

% \begin{algorithm}[H]
%     \caption{Reduce Dataset for Offline RL~(\name)}
%     \label{alg: offline data selection}
%     \begin{algorithmic}[1]
%         \STATE {\bf Require}: Complete offline dataset $\mathcal{D}$
%         \STATE Initialize parameters of the offline agent for data selection $Q_{\theta}, \pi_{\phi}$ and update interval $T^s$
%         \STATE Run a clustering algorithm on the original dataset $\mathcal D$ and obtain $K$ clusters $\{\mathcal{C}_1,\cdots,\mathcal{C}_K\}$
%         \STATE Cluster centers set $\mathcal C=\{c_1,\cdots,c_K\}$
%         \FOR{$k=1, \cdots, K$}
%         \FOR{$j=1, \cdots, J$}
%         \STATE Sample mini-batch $\mathcal{B}_{j}^{k}$ from the cluster $\mathcal{C}_k$
%         \STATE Calculate $\nabla_{\theta}\mathcal{L}^B, \nabla_{\theta}\mathcal L_{\rdcshort}^B$ under $\mathcal{B}_{j}^{k}$
%         \STATE $\mathcal{S}_{j}^{k}, \vw_{j}^k$ = OMP$(\nabla_{\theta}\mathcal{L}^B, \nabla_{\theta}\mathcal L_{\rdcshort}^B, Q_{\theta}, c_{k})$
%         \IF{$j$ mod $T^s$ = 0}
%         \STATE Train $Q_{\theta}, \pi_{\phi}$ based on Equation~\ref{eq: q-target value}
%         \ENDIF
%         \ENDFOR
%         \ENDFOR
%     \STATE Reduced offline dataset $\mathcal S\leftarrow\cup_{j\in[J],k\in[K]}\mathcal S_j^k$
%     \STATE Initialize parameters of the offline agent for training on the reduced offline dataset $Q_{\vartheta},\pi_{\varphi}$
%     \STATE Train $Q_{\vartheta},\pi_{\varphi}$ based on $\mathcal{S}$ and $\vw$
%     \end{algorithmic}
% \end{algorithm}

% \begin{algorithm}[H]
%     \caption{OMP algorithm}
%     \label{alg: omp}
%     \begin{algorithmic}[1]
%         \STATE {\bf Require}: $\nabla_{\theta}\mathcal{L}^B$, $\nabla_{\theta}\mathcal L_{\rdcshort}^B$, parameter $\theta$ of $Q_{\theta}$, cluster center $c_k$, regularization coefficient $\lambda$, subset size $\frac{|\mathcal{B}|N}{M}$
%         \STATE $\mathcal{S}_j^k\leftarrow\{c_k\}$
%         \STATE $r\leftarrow\operatorname{Err}_{\lambda}^B\left(\vw_j^k, \mathcal{S}_j^k, \mathcal L^B, \theta\right)$
%         \WHILE{$|\mathcal{S}_j^k|\leq\frac{|\mathcal{B}|N}{M}$}
%         \STATE $e=\arg\max_{i\notin\mathcal{S}_j^k}|\langle\nabla_{\theta}^i\mathcal L^{B}_{\rdcshort}\left(\theta\right), r\rangle|$
%         \STATE $\mathcal{S}_j^k\leftarrow\mathcal{S}_j^k\cup\{e\}$
%         \STATE $\vw_j^k\leftarrow\arg\min_{\vw_j^k}\operatorname{Err}_{\lambda}^B\left(\vw_j^k, \mathcal{S}_j^k, \mathcal L^B, \theta\right)$
%         \STATE $r\leftarrow\operatorname{Err}_{\lambda}^B\left(\vw_j^k, \mathcal{S}_j^k, \mathcal L^B, \theta\right)$
%         \ENDWHILE\\
%     \STATE \textbf{Return} $\mathcal{S}_j^k$ and $\vw_j^k$
%     \end{algorithmic}
% \end{algorithm}
% \clearpage
\section{Related works}
\label{appx:related_work}
% As deep learning usually trains on abundant data, a considerable number of works have focused on identifying important training samples and figuring out the ideal size of the dataset. However, the employment in policy training tasks is under-explored. Our work is closely related to offline RL and data subset selection.

\textbf{Offline Reinforcement Learning.}\ \
Offline RL can execute policy training entirely based on static datasets without further interaction with the environment~\cite{levine2020offline}.
Therefore, it faces challenges such as distribution shift and value overestimation.
To address this issue, some prior works attempted to constrain the learned policy and behavior policy by limiting the action difference~\cite{fujimoto2019off}, adding KL-divergence~\cite{nair2020awac,peng2019advantage,wu2019behavior}, or regularization~\cite{kumar2019stabilizing}.
Other works consider employing conservative estimates of future values~\cite{kumar2020conservative,ma2021conservative} or penalizing uncertain actions~\cite{janner2019trust,yu2021combo,kidambi2020morel} by uncertainty.
There are also some new attempts, such as lightweight implementation~\cite{fujimoto2021minimalist} or avoiding distribution shift by single-step policy iteration~\cite{kostrikov2021offline}.
These studies provide a solid foundation for implementing and transferring reinforcement learning to real-world tasks.
However, there has been limited research addressing considerations related to the dataset.
Some works attempted to explore which dataset characteristics dominate in offline RL algorithms~\cite{schweighofer2021understanding, swazinna2021measuring} or investigate the data generation~\cite{yarats2022don}.
Recently, some researchers attempted to solve the sub-optimal trajectories issue by constraining policy to good data rather than all actions in the dataset~\cite{hong2023beyond} or re-weighting policy~\cite{hong2023harnessing}.
Differently, our work aims to figure out the ideal size of the dataset needed for effective policy training.

\textbf{Data subset selection.}\ \ The research on identifying crucial samples within datasets is concentrated in the field of computer vision.
Some prior works use uncertainty of samples~\cite{coleman2019selection,paul2021deep} or the frequency of being forgotten~\cite{toneva2018empirical} as the proxy function to prune the dataset.
Another research line focuses on constructing weighted data subsets to approximate the full dataset~\cite{feldman2020core}, which often transforms the subset selecting to the submodular set cover problem~\cite{wei2015submodularity,kaushal2019learning}.
Specifically, several works adopted loss functions as the optimization target~\cite{lucic2018training,campbell2018bayesian}, while recent research finds that approximating full gradient is more efficient~\cite{mirzasoleiman2020coresets, killamsetty2021grad,killamsetty2021glister,killamsetty2021retrieve}.
These studies establish the critical importance of selecting key samples from datasets for effective training. However, the different learning objectives and training methodologies between reinforcement learning and computer vision mean that these techniques cannot be directly applied to Offline RL.
\clearpage
\section{Proofs of theoretical analysis}
{\subsection{Notations}}

\begin{table}[ht]
    \centering
    {\begin{tabular}{ll}
    \toprule
    Notation & Explanation \\
    \midrule
    \hspace{0.3cm} $U_\mathtt{TD}$ & Bound of TD Loss \\
    \hspace{0.3cm} $U_{\nabla Q}$ & Bound of Gradient \\
    \hspace{0.3cm} $U_{\nabla a}$ & Bound of Gradient \\
    \hspace{0.3cm} $U_a$ & Bound of Action Difference \\
    \hspace{0.3cm} $U_\pi$ & Bound of Action \\
    \hspace{0.3cm} $\mathcal{D}$ & Complete Dataset \\
    \hspace{0.3cm} $\mathcal{S}$ & Reduced Dataset \\
    \hspace{0.3cm} $N$ & Size of Reduced Dataset \\
    \hspace{0.3cm} $\lambda$ & Minimum Eigenvalues \\
    \hspace{0.3cm} $\mathcal{C}$ & Cluster \\
    \hspace{0.3cm} $G$ & Total Training Steps \\
    \hspace{0.3cm} $\epsilon$ & Gradient Approximation Errors \\
    \hspace{0.3cm} $\theta_t$ & Updated parameter at the $t^{th}$ epoch \\
    \hspace{0.3cm} $\theta_t^*$ & Optimal model parameter \\
    \bottomrule
    \end{tabular}}
    {\caption{Organization of the notations used througout this paper}}
    \label{tab: notation}
\end{table}

\subsection{Submodular}
\label{appendix: submodular}

% First, we restate Theorem \ref{thm: submodular}.
% \begin{theorem}
%     For any $\mathcal S$ with $|\mathcal S| \leq N$ and sample $(s_i,a_i,r_i,s'_i)\in \mathcal{D}$, suppose that the TD loss and gradients are bounded: $|\mathcal{L}^i(\theta)| \leq U_\mathtt{TD}$, $ \|\nabla_\theta Q_\theta(s_i,a_i)\|_2 \leq U_{\nabla Q}$, $\|\nabla_{\pi_{\phi}(s_i)}Q_\theta(s_i,\pi_{\phi}(s_i))\|_2 \leq U_{\nabla a}$, $\|\pi_{\phi}(s_i)-a_i\|_2 \leq U_a$, $\|\pi_{\phi}(s_i)\|_2\leq U_\pi$, and $\|\nabla_\phi \pi_{\phi}(s_i)\|_2 \leq U_{\nabla\pi}$, then $F_\lambda^Q(\mathcal{S})$ is $\delta$-weakly submodular, with
%     \begin{align}
%         \delta \geq \frac{\lambda}{\lambda+4 N (U_\mathtt{TD}U_{\nabla Q})^2},\nonumber
%     \end{align}
%     and $F_\lambda^\pi(\mathcal{S})$ is $\delta$-weakly submodular, with 
%     \begin{align}
%         \delta \geq \frac{\lambda}{\lambda + N(U_{\nabla a}/\alpha+2U_a U_\pi)^2 U_{\nabla\pi}^2}.\nonumber
%     \end{align}
%     \label{thm: submodular}
% \end{theorem}
\submodular*

\begin{proof}
As mentioned in Section \ref{sec: preliminary}, we use the TD3+BC algorithm as the basic offline RL algorithm. 
TD3+BC follows the actor-critic framework, which trains policy and value networks separately. 
For a single sample $(s_i,a_i,r_i,s'_i)$, the loss of the value network is also named as TD error, which is defined by:
\begin{align}
    & \mathcal L_{Q}^i(\theta) = (y_i - Q_\theta(s_i,a_i))^2   \\
    & \text{where}\quad y_i = r_i + \gamma Q_{\theta'}(s'_i,\pi_{\phi'}(s'_i)+\epsilon)  \\
\end{align}

The gradient is:

\begin{align}
    -\frac{1}{2} \nabla_{\theta} \mathcal L^i_Q(\theta)=(y_i- Q_\theta(s_i,a_i))\nabla_\theta Q_\theta(s_i,a_i)
    \label{eq: td_gradient}
\end{align}

Offline RL algorithms attempt to minimize the TD error and compute the Q-value through a neural network.
Therefore, we assume the upper bound of the TD error is $\max_i\|y_i- Q_\theta(s_i,a_i)\|_2\leq U_\mathtt{TD}$.
The upper bound of the gradient of the value network is $\max_i \|\nabla_\theta Q_\theta(s_i,a_i)\|_2\leq U_{\nabla Q}$.
Then, Equation~\ref{eq: td_gradient} can be transformed into:
\begin{equation}
    \|\nabla_\theta \mathcal L^i_Q(\theta)\|_2 \leq 2U_\mathtt{TD} U_{\nabla Q}
\end{equation}

Similarly, for a single sample$(s_i,a_i,r_i,s'_i)$, the loss of the policy network is
\begin{align}
    \mathcal L_{\pi}^i(\phi) &= -\frac{1}{\alpha} Q_\theta(s_i, \pi_{\phi}(s_i))+\|\pi_{\phi}(s_i)-a_i\|_2^2 \\
\end{align}

The gradient is:

\begin{align}
    \nabla_{\phi} \mathcal L_{\pi}^i(\phi) &= \frac{\partial \mathcal L_{\pi}^i(\phi)}{\partial \pi_{\phi}(s_i)}\times \frac{\partial \pi_{\phi}(s_i)}{\partial \phi}   \\
    &= [-\frac{1}{\alpha} \nabla_{\pi_{\phi}(s_i)}Q_\theta(s_i,\pi_{\phi}(s_i))+2(\pi_{\phi}(s_i)-a_i)^\top \pi_{\phi}(s_i)] \times \nabla_\phi \pi_{\phi}(s_i)
    \label{eq: policy_gradient}
\end{align}

Here $\alpha$ is used to balance the conservatism and generalization in Offline RL, which is defined by:

\begin{align}
    \alpha= \frac{\mathbb{E}_{(s_i,a_i)}[|Q(s_i,a_i)|]}{\kappa}
\end{align}

where $\kappa$ is a hyper-parameter in TD3+BC.
Note that although $\alpha$ includes $Q$, it is not differentiated over. 

Offline RL algorithms attempt to limit the deviation of the current learned policy from the behavior policy while maximizing the Q-value of the optimized policy.
Therefore, we assume the upper bound of the gradient of the value network is $\max_i\|\nabla_{\pi_{\phi}(s_i)}Q_\theta(s_i,\pi_{\phi}(s_i))\|_2 \leq U_{\nabla a}$.
The upper bound of the action error is $\max_i\|\pi_{\phi}(s_i)-a_i\|_2\leq U_a$.
The upper bound of the output of the policy is $\max_i\|\pi_{\phi}(s_i)\|_2 \leq U_\pi$.
The upper bound of the gradient of the policy network is
$\max_i\|\nabla_\phi \pi_{\phi}(s_i)\|_2\leq U_{\nabla \pi}$.

Then, Equation~\ref{eq: policy_gradient} can be bound:
\begin{equation}
    \|\nabla_\phi \mathcal L_{\pi}^i(\phi)\|_2 \leq (U_{\nabla a}/\alpha+2U_a U_\pi)U_{\nabla \pi}
\end{equation}

We can define two functions $l_Q(\mathbf{\beta}), l_\pi(\mathbf{\beta}): \mathbb{R}^{|\mathcal{D}|} \rightarrow \mathbb{R}$
\begin{equation}
\begin{aligned}
    l_Q(\mathbf{\beta}) &= -\|\sum_{i=1}^{{|\mathcal{D}|}} \beta_i\nabla_\theta \mathcal L_Q^i(\theta)-\nabla_\theta \mathcal L(\theta)\|_2 - \lambda\|\beta\|_2^2 \\
    l_\pi(\mathbf{\beta}) &= -\|\sum_{i=1}^{{|\mathcal{D}|}} \beta_i\nabla_\phi \mathcal L^i_\pi(\phi)-\nabla_\phi \mathcal L(\phi)\|_2 - \lambda\|\beta\|_2^2
\end{aligned}
\end{equation}

We assume $\beta$ is a $N$-sparse vector that is 0 on all but $N$ indices.
Then we can transform maximizing $F^Q_\lambda(\mathcal{S}), F^\pi_\lambda(\mathcal{S})$ into maximizing $l(\beta)-l(\mathbf{0})$:
\begin{equation}
\begin{aligned}
    \max_{\mathcal{S}:|\mathcal{S}| \leq N} F^Q_\lambda(\mathcal{S}) &\xleftrightarrow{} \max_{\substack{\beta:\beta_{S^c=0} \\|\mathcal{S}|\leq N}} l_Q(\mathbf{\beta})-l_Q(\mathbf{0}) \\
    \max_{\mathcal{S}:|\mathcal{S}| \leq N} F^\pi_\lambda(\mathcal{S}) &\xleftrightarrow{} \max_{\substack{\beta:\beta_{S^c=0} \\|\mathcal{S}|\leq N}} l_\pi(\mathbf{\beta})-l_\pi(\mathbf{0})
\end{aligned}
\end{equation}
where ${S^c}$ means the complementary set of $S$, and $\beta_{S^c}=0$ means $\beta$ is 0 on all but indices $i$ that $i \in S$. 
$l(\mathbf{0})$ means the value of $l(\cdot)$ when input is zero vector $\mathbf{0}$, it serves as a basic value.
Since $l_Q(\beta)\leq 0, l_\pi(\beta) \leq 0$,  we can easily find that the minimum eigenvalues of $-l_Q(\beta)$ and $-l_\pi(\beta)$ are both at least $\lambda$. 

Next, the maximum eigenvalues of $-l_Q(\beta)$ and $-l_\pi(\beta)$ are
\begin{equation}
\begin{aligned}
\Lambda_{\max}(-l_Q(\beta))&=
\lambda+\operatorname{Trace}\left(\left[\begin{array}{c}
\beta_1\nabla_\theta \mathcal L_Q^{1 \top}\left(\theta\right) \\
\beta_2\nabla_\theta \mathcal L_Q^{2 \top}\left(\theta\right) \\
\ldots\\
\beta_{|\mathcal{D}|}\nabla_\theta \mathcal L_Q^{|\mathcal{D}| \top}\left(\theta_t\right)
\end{array}\right]\left[\begin{array}{c}
\beta_1\nabla_\theta \mathcal L_Q^{1 \top}\left(\theta\right) \\
\beta_2\nabla_\theta \mathcal L_Q^{2 \top}\left(\theta\right) \\
\ldots \\
\beta_{|\mathcal{D}|}\nabla_\theta \mathcal L_Q^{|\mathcal{D}| \top}\left(\theta\right)
\end{array}\right]^{\top}\right)    \\
&=\lambda + \sum_{i=1}^{{|\mathcal{D}|}} \beta_i^2 \| \nabla_\theta \mathcal L_Q^{i}(\theta) \|^2\\ &\leq \lambda+4 N (U_\mathtt{TD}U_{\nabla Q})^2 \\
\Lambda_{\max}(-l_\pi(\beta))&\leq \lambda + N(U_{\nabla a}/\alpha+2U_a U_\pi)^2 U_{\nabla\pi}^2
\end{aligned}
\end{equation}

Following the Theorem~1 in \cite{elenberg2018restricted}, we can derive that $F_\lambda^Q(\mathcal{S})$ is $\delta$-weakly submodular with $\delta \geq \frac{\lambda}{\lambda+4 N (U_\mathtt{TD}U_{\nabla Q})^2}$. 
And $F_\lambda^\pi(\mathcal{S})$ is $\delta$-weakly submodular with $\delta \geq \frac{\lambda}{\lambda + N(U_{\nabla a}/\alpha+2U_a U_\pi)^2 U_{\nabla\pi}^2}$.
\end{proof}

\subsection{Upper Bound of Residual Error}
\label{appendix: cluster theory}

\cluster*

\begin{proof}
The residual error is no larger than the special case where all $w_i$ are $|\mathcal{D}|/|\mathcal{S}|$:
\begin{align}
\operatorname{Err}\left(\vw, \mathcal{S}, \mathcal L, \theta\right)\le\|\frac{|\mathcal D|}{|\mathcal S|}\sum_{i\in \mathcal S}\nabla_{\theta} \mathcal L^i\left(\theta\right) - \sum_{i\in \mathcal D}\nabla_{\theta} \mathcal L^i\left(\theta\right) \|_2. \nonumber
\end{align}
Using Jensen's inequality, we have
\begin{align}
\operatorname{Err}\left(\vw, \mathcal{S}, \mathcal L, \theta\right)\le\sum_{i\in \mathcal D} \|\nabla_{\theta} \mathcal L^i\left(\theta\right) - \frac{1}{|\mathcal S|}\sum_{s\in\mathcal S}\nabla_{\theta}\mathcal L^s\left(\theta\right) \|_2. \nonumber
\end{align}
% In our formulation, samples are selected in mini-batches, and the gradient of sample $i$ from mini-batch $\mathcal B_{j_i}^{k_i}$ is approximated by those of $\mathcal S_{j_i}^{k_i}$ (Eq.~\ref{eq: batch gradient approx}). Therefore, we have
% \begin{align}
%     \operatorname{Err}\left(\vw, \mathcal{S}, \mathcal L, \theta\right)\le\sum_{i\in \mathcal D}\|\nabla_{\theta} \mathcal L^i\left(\theta\right) - \frac{1}{|\mathcal S_{j_i}^{k_i}|}\sum_{s\in\mathcal S_{j_i}^{k_i}}\nabla_{\theta}\mathcal L^s\left(\theta\right) \|_2.\nonumber
% \end{align}
According to the monotone property of submodular functions, adding more samples to $S^k$ reduces the residual error. We assume $S^k$ starts with the cluster center $\{c_k\}$, it follows that
    \begin{align}
    \operatorname{Err}&\left(\vw, \mathcal{S}, \mathcal L, \theta\right) \le \sum_{i\in \mathcal D}\|\nabla_{\theta} \mathcal L^i\left(\theta\right) - \nabla_{\theta}\mathcal L^{c_k}\left(\theta\right) \|_2\nonumber\\
    =& \sum_{i\in\mathcal D} \min_{c\in \mathcal C}\|\nabla_{\theta} \mathcal L^i\left(\theta\right) - \nabla_{\theta} \mathcal L^c\left(\theta\right) \|_2.\label{equ:cluster_obj}
\end{align}
Eq.~\ref{equ:cluster_obj} is exactly the optimization objective typical of the clustering problem.
\end{proof}

\subsection{Convergence Analysis}
\label{appendix: convergence}

% \begin{theorem}
%     Let $\theta^*$ denote the optimal $Q^{\pi_\mathcal{S}}$ parameters, $\theta_t$ the parameters after $t$ training steps. The TD loss satisfies 
%     \begin{align}
%         \min_{t=1:G}\mathcal{L}(\theta_t)\leq \mathcal{L}(\theta^*) + \frac{D\sigma}{\sqrt{G}} + \frac{D}{G}\sum_{t=1}^{G-1}\varepsilon.\nonumber
%     \end{align}
%     Here 
%     $\mathcal{L}(\theta)=\sum_{i \in \mathcal{D}}\mathcal L_{\mathtt{TD}}(s_{i}, a_{i}, r_i, s'_{i}, \theta)$, $G$ is the total training steps, $\varepsilon=\operatorname{Err}\left(\vw, \mathcal{S}, \mathcal L, \theta_t\right)$ is the gradient approximation errors that are bounded in Corollary~\ref{thm:c_bound}.
% \end{theorem}
\convergence*

\begin{proof}
    From the definition of Gradient Descent, we have:

    \begin{align}
    \nabla_{\theta} \mathcal L_{\rdcshort}(\theta_t)^T(\theta_t - \theta^*) &= \frac{1}{\alpha_t}(\theta_t-\theta_{t+1})^T(\theta_t-\theta^*) \\
    \nabla_{\theta} \mathcal L_{\rdcshort}(\theta_t)^T(\theta_t - \theta^*) &= \frac{1}{2\alpha_t}\left(\|\theta_t-\theta_{t+1}\|^2 + \|\theta_t-\theta^*\|^2 - \|\theta_{t+1}-\theta^*\|^2\right) \\
    \nabla_{\theta} \mathcal L_{\rdcshort}(\theta_t)^T(\theta_t - \theta^*) &= \frac{1}{2\alpha_t}\left(\|\alpha_t \nabla_{\theta} \mathcal L_{\rdcshort}(\theta_t)\|^2 + \|\theta_t-\theta^*\|^2 - \|\theta_{t+1}-\theta^*\|^2\right)
    \end{align}

    Then, we rewrite the function $\nabla_{\theta} \mathcal L_{\rdcshort}(\theta_t)^T(\theta_t - \theta^*)$ as follows:

    \begin{align}
        \nabla_{\theta} \mathcal L_{\rdcshort}(\theta_t)^T(\theta_t - \theta^*) = \nabla_{\theta} \mathcal L_{\rdcshort}(\theta_t)^T(\theta_t - \theta^*) - \nabla_{\theta} \mathcal{L}(\theta_t)^T(\theta_t - \theta^*) + \nabla_{\theta} \mathcal{L}(\theta_t)^T(\theta_t - \theta^*)
    \end{align}

    Combining the above equations we have:

    \begin{align}
    \nabla_{\theta} \mathcal L_{\rdcshort}(\theta_t)^T(\theta_t - \theta^*) - \nabla_{\theta} \mathcal{L}(\theta_t)^T(\theta_t - \theta^*) + \nabla_{\theta} \mathcal{L}(\theta_t)^T(\theta_t - \theta^*) = \\
    \frac{1}{2\alpha_t}\left(\|\alpha_t\nabla_{\theta} \mathcal L_{\rdcshort}(\theta_t)\|^2 + \|\theta_t - \theta^*\|^2 - \|\theta_{t+1} - \theta^*\|^2\right)
    \end{align}

    \begin{align}
    \nabla_{\theta} \mathcal{L}(\theta_t)^T(\theta_t - \theta^*) =
    \frac{1}{2\alpha_t}\left(\|\alpha_t\nabla_{\theta} \mathcal L_{\rdcshort}(\theta_t)\|^2 + \|\theta_t - \theta^*\|^2 - \|\theta_{t+1} - \theta^*\|^2\right) - \\ (\nabla_{\theta} \mathcal L_{\rdcshort}(\theta_t) - \nabla_{\theta} \mathcal{L}(\theta_t))^T(\theta_t - \theta^*)
    \end{align}

    Summing up the above equation for different value of $t\in [0,G-1]$ and the learning rate $\alpha_t$ is a constant $\alpha$, then we have:

    \begin{align}
    \sum_{t=0}^{G-1} \nabla_{\theta} \mathcal{L}(\theta_t)^T(\theta_t - \theta^*) = \frac{1}{2\alpha} \|\theta_0 - \theta^*\|^2 - \|\theta_G - \theta^*\|^2 + \sum_{t=0}^{G-1}\left(\frac{1}{2\alpha}\|\alpha\nabla_{\theta} \mathcal L_{\rdcshort}(\theta_t)\|^2\right) \\
    + \sum_{t=0}^{G-1}\left((\nabla_{\theta} \mathcal L_{\rdcshort}(\theta_t) - \nabla_{\theta} \mathcal{L}(\theta_t) )^T(\theta_t - \theta^*)\right)
    \end{align}

    Since $\|\theta_G - \theta^*\|^2 \geq 0$, we have:

    \begin{align}
        \sum_{t=0}^{G-1} \nabla_{\theta} \mathcal{L}(\theta_t)^T(\theta_t - \theta^*) \leq \frac{1}{2\alpha} \|\theta_0 - \theta^*\|^2 + \sum_{t=0}^{G-1}\left(\frac{1}{2\alpha}\|\alpha\nabla_{\theta} \mathcal L_{\rdcshort}(\theta_t)\|^2\right) \\
        + \sum_{t=0}^{G-1}\left((\nabla_{\theta} \mathcal L_{\rdcshort}(\theta_t) - \nabla_{\theta} \mathcal{L}(\theta_t) )^T(\theta_t - \theta^*)\right)
        \label{eq: sum}
    \end{align}

From the convexity of function $\mathcal{L}(\theta)$, we have:

\begin{align}
    \mathcal{L}(\theta_t) - \mathcal{L}(\theta^*) \leq \nabla_{\theta} \mathcal{L}(\theta_t)^T(\theta_t - \theta^*)
    \label{eq: convexity}
\end{align}

Combining the Equation~\ref{eq: sum} and Equation~\ref{eq: convexity}, we have:

\begin{align}
    \sum_{t=0}^{G-1}\mathcal{L}(\theta_t) - \mathcal{L}(\theta^*) \leq \frac{1}{2\alpha} \|\theta_0 - \theta^*\|^2 + \sum_{t=0}^{G-1}\left(\frac{1}{2\alpha}\|\alpha\nabla_{\theta} \mathcal L_{\rdcshort}(\theta_t)\|^2\right) \\
    + \sum_{t=0}^{G-1}\left((\nabla_{\theta} \mathcal L_{\rdcshort}(\theta_t) - \nabla_{\theta} \mathcal{L}(\theta_t))^T(\theta_t - \theta^*)\right)
\end{align}

We assume that $\|\theta - \theta^*\|\leq D$.
Since $\| \nabla \mathcal L(\theta) \| \leq \sigma$, we have:

\begin{align}
    \sum_{t=0}^{G-1}\mathcal{L}(\theta_t) - \mathcal{L}(\theta^*) \leq \frac{D^2}{2\alpha} + \frac{G\alpha\sigma^2}{2}
    + \sum_{t=0}^{G-1}D(\|\nabla_{\theta} \mathcal L_{\rdcshort}(\theta_t) - \nabla_{\theta} \mathcal{L}(\theta_t)\|)
\end{align}

Then:

\begin{align}
    \frac{\sum_{t=0}^{G-1}\mathcal{L}(\theta_t) - \mathcal{L}(\theta^*)}{G} \leq \frac{D^2}{2\alpha G} + \frac{\alpha\sigma^2}{2}
    + \sum_{t=0}^{G-1}\frac{D}{G}(\|\nabla_{\theta} \mathcal L_{\rdcshort}(\theta_t) - \nabla_{\theta} \mathcal{L}(\theta_t)\|)
\end{align}

Since $\min(\mathcal{L}(\theta_t) - \mathcal{L}(\theta^*))\leq \frac{\sum_{t=0}^{G-1}\mathcal{L}(\theta_t) - \mathcal{L}(\theta^*)}{G}$, we have:

\begin{align}
    \min(\mathcal{L}(\theta_t) - \mathcal{L}(\theta^*))\leq \frac{D^2}{2\alpha G} + \frac{\alpha\sigma^2}{2}
    + \sum_{t=0}^{G-1}\frac{D}{G}(\|\nabla_{\theta} \mathcal L_{\rdcshort}(\theta_t) - \nabla_{\theta} \mathcal{L}(\theta_t)\|)
\end{align}

We adopt $\varepsilon$ to denote $\|\nabla_{\theta} \mathcal L_{\rdcshort}(\theta_t) - \nabla_{\theta} \mathcal{L}(\theta_t)\|$, then we have:

\begin{align}
    \min(\mathcal{L}(\theta_t) - \mathcal{L}(\theta^*))\leq \frac{D^2}{2\alpha G} + \frac{\alpha\sigma^2}{2}
    + \sum_{t=0}^{G-1}\frac{D}{G}\varepsilon
\end{align}
     
\end{proof}


\begin{theorem}\label{thm:monotone}
    The training loss on original dataset always monotonically decreases with every training epoch $t$, $\mathcal{L}(\theta_{t+1}) \leq \mathcal{L}(\theta_t)$ if it satisfies the condition that $\nabla_{\theta} \mathcal{L}(\theta_t)^T\nabla_{\theta} \mathcal L_{\rdcshort}(\theta_t) \geq 0$ for $0\leq t \leq G$ and the learning rate $\alpha \leq \min_{t} \frac{2}{L}\frac{\nabla_{\theta} \mathcal{L}(\theta_t)^T\nabla_{\theta} \mathcal L_{\rdcshort}(\theta_t)}{\nabla_{\theta} \mathcal L_{\rdcshort}(\theta_t)^T\nabla_{\theta} \mathcal L_{\rdcshort}(\theta_t)}$.
\end{theorem}

% Let $L_{T}(\theta_{t}; \mathcal{D})$ denote the training loss on the original dataset $\mathcal{D}$.
% Then, we suppose the training loss $L_T$ is Lipschitz smooth with constant $\mathcal{L}$, and the gradient is bounded by $\sigma_T$:
% $\|\nabla L_{T}(\theta_{t+1}) - \nabla L_{T}(\theta_{t})\| \leq \mathcal{L}\|\theta_{t+1} - \theta_{t}\|$ and $\| \nabla L_{T}(\theta_{t}) \| \leq \sigma_T$.
% Similarly, we let $L_{T}(\theta_{t}; \mathcal{S})$ denote the training loss on the reduced subset $\mathcal{S}$ with the same smooth assumption.

\begin{proof}
    Since the training loss $\mathcal{L}(\theta)$ is lipschitz smooth, we have:
\begin{align}
    \mathcal{L}(\theta_{t+1}) & \leq \mathcal{L}(\theta_t) + \nabla_{\theta} \mathcal{L}(\theta_t)^T\Delta \theta + \frac{L}{2}\|\Delta\theta\|^2, \\
    &\text{where} \qquad \Delta\theta=\theta_{t+1} - \theta_{t}.
\end{align}

Since, we are using SGD to optimize the reduced subset training loss $\mathcal L_{\rdcshort}(\theta_t)$ model parameters.
The update equation is:

\begin{align}
    \theta_{t+1} = \theta_{t} - \alpha \nabla_{\theta} \mathcal L_{\rdcshort}(\theta_t)
\end{align}

Combining the above two equations, we have:
\begin{align}
    \mathcal{L}(\theta_{t+1}) \leq \mathcal{L}(\theta_t) + \nabla_{\theta} \mathcal{L}(\theta_t)^T(- \alpha \nabla_{\theta} \mathcal L_{\rdcshort}(\theta_t)) + \frac{L}{2}\|- \alpha \nabla_{\theta} \mathcal L_{\rdcshort}(\theta_t)\|^2
\end{align}

Next, we have:

\begin{align}
    \mathcal{L}(\theta_{t+1}) - \mathcal{L}(\theta_t) \leq \nabla_{\theta} \mathcal{L}(\theta_t)^T(- \alpha \nabla_{\theta} \mathcal L_{\rdcshort}(\theta_t)) + \frac{L}{2}\|- \alpha \nabla_{\theta} \mathcal L_{\rdcshort}(\theta_t)\|^2
\end{align}

From the above equation, we have:

\begin{align}
    \mathcal{L}(\theta_{t+1}) \leq \mathcal{L}(\theta_{t}), \quad  \text{if} \quad \nabla_{\theta} \mathcal{L}(\theta_{t})^T\nabla_{\theta} \mathcal L_{\rdcshort}(\theta_t) 
    - \frac{\alpha L}{2}\|\nabla_{\theta} \mathcal L_{\rdcshort}(\theta_t)\|^2 \geq 0
\end{align}

Since $\|\nabla_{\theta} \mathcal L_{\rdcshort}(\theta_t)\|^2\geq 0$, we will have the necessary condition $\nabla_{\theta} \mathcal{L}(\theta_{t})^T\nabla_{\theta} \mathcal L_{\rdcshort}(\theta_t) \geq 0$.
Next, we rewrite the above condition as follows:

\begin{align}
    \nabla_{\theta} \mathcal{L}(\theta_{t})^T\nabla_{\theta} \mathcal L_{\rdcshort}(\theta_t) \geq \frac{\alpha L}{2}\|\nabla_{\theta} \mathcal L_{\rdcshort}(\theta_t)\|^2
\end{align}

Therefore, the necessary condition for the learning rate $\alpha$ is:

\begin{align}
    \alpha \leq \frac{2}{L}\frac{\nabla_{\theta} \mathcal{L}(\theta_t)^T\nabla_{\theta} \mathcal L_{\rdcshort}(\theta_t)}{\nabla_{\theta} \mathcal L_{\rdcshort}(\theta_t)^T\nabla_{\theta} \mathcal L_{\rdcshort}(\theta_t)}
\end{align}

Since the above condition needs to be true for all values for $t$, we have the following conditions for the learning rate:

\begin{align}
    \alpha \leq \min_{t} \frac{2}{L}\frac{\nabla_{\theta} \mathcal{L}(\theta_t)^T\nabla_{\theta} \mathcal L_{\rdcshort}(\theta_t)}{\nabla_{\theta} \mathcal L_{\rdcshort}(\theta_t)^T\nabla_{\theta} \mathcal L_{\rdcshort}(\theta_t)}
\end{align}

\end{proof}
\clearpage
\section{Ablation Study}
\label{sec: ablation}
To study the contribution of each component in our learning framework, we conduct the following ablation study. 
\nameq: We replace the empirical returns used to update Q functions with the standard target Q function in the TD loss function. 
\namei: We set the number of data selection rounds to 1 and study the function of multi-round data selection.
The experimental results in Figure~\ref{fig: modular ablation}show that removing any of these two modules will worsen the performance of \name. In case like $\texttt{walker2d-medium}$, ablation \namei~even decrease the performance by over 80\%, and ablation \nameq~results in a 95\% performance drop in $\texttt{walker2d-expert}$. Furthermore, we also find that in the $\texttt{halfcheetah}$ tasks, the impact of removing the two modules is relatively small. This result can be attributable to the fact that this task has a limited state space, and we can directly apply OMP to the entire dataset and identify important and diverse data.

\begin{figure}[H]
    \centering
    \subfigure{\includegraphics[scale=0.27]{ablation_moduler1.pdf}}
    \hspace{0.3cm}\subfigure{\includegraphics[scale=0.27]{ablation_moduler2.pdf}}
    \caption{Ablation results on D4RL~(Hard) tasks with the normalized score metric.}
    \label{fig: modular ablation}
\end{figure}


\section{Computational Complexity}
\label{appendix: computation complexity}
We report the computational overhead of \name~on various datasets. 
All experiments are conducted on the same computational device (GeForce RTX 3090 GPU). 
The results in the following Table indicate that even on datasets containing millions of data points, the computational overhead remains low. 
This low computational complexity can be attributed to the trajectory-based selection technique in Sec.~\ref{sec: offline omp}~(II) and the regularized constraint technique in Sec.~\ref{sec:method:outer}, making our method easily scalable to large-scale datasets. 

\begin{table*}[h]
    \centering
    \begin{tabular}{c|cc}
    \toprule
    Env & Data Number & \name \\
    \midrule
    Hopper-medium-v0 & 999981 & 8m \\
    Walker2d-medium-v0 & 999874 & 8m \\
    Halfcheetah-medium-v0 & 998999 & 8m \\
    Hopper-expert-v0 & 999034 & 8m \\
    Walker2d-expert-v0 & 999304 & 8m \\
    Halfcheetah-expert-v0 & 998999 & 8m\\
    Hopper-medium-expert-v0 & 1199953 & 8m\\
    Walker2d-medium-expert-v0 & 1999179  & 13m\\
    Halfcheetah-medium-expert-v0 & 1997998 & 14m\\
    Hopper-medium-replay-v0 & 200918 & 3m\\
    Walker2d-medium-replay-v0 & 100929  & 3m\\
    Halfcheetah-medium-replay-v0 & 100899 & 3m\\
    \bottomrule
    \end{tabular}
    \label{tab: cc}
    \caption{The computational complexity associated with \name~in various datasets. $m$ represents minutes.} 
\end{table*}

% \subsection{Trajectory-based selection}
% \label{appendix: trajectory}

% Experimental results in Figure~\ref{fig: d4rl topbc} show that \name~maintains its superiority in this setting with suboptimal (e.g., \texttt{medium}) datasets. This evidence suggests that \name~provides a valuable strategy for selecting data conducive to effective training under conditions of compromised data quality.

% \subsection{Generalizability of \name~ to other domains}
% \label{appendix: other domain}
% We evaluate our algorithm on robotic manipulation (Adroit) and sparse reward (Antmaze) tasks. 
% The experimental results in Table~\ref{tab: other domain} indicate that in sparse reward tasks, \name~ achieves comparable performance close to that on the full dataset with only 20\% of the data. In the robotic manipulation tasks, \name~ requires even less data.

% \begin{table}[h]
%     \centering
%     \caption{Experimental results of \name~ with various dataset sizes ($x\%$) in Antmaze and Adroit tasks. 
%     Highlighted is the performance comparable to training TD3+BC with the complete dataset. } 
%     \label{tab: other domain}
%     \begin{tabular}{c|cccc}
%     \toprule
%         Env & 10\% & 20\% & 30\% & All Data \\
%         \midrule
%         Antmaze-umaze-v0 & 70.2$\pm$3.6 & 75.1$\pm$2.5 & 84.7$\pm$3.3 & 87.5$\pm$1.3 \\
%         Antmaze-umaze-diverse-v0 & 44.7$\pm$2.7 & 46.3$\pm$1.9 & 47.7$\pm$2.2 & 62.2$\pm$2.0 \\
%         Antmaze-medium-play-v0 & 2.1$\pm$1.3 & 59.3$\pm$1.6 & 60.3$\pm$2.9 &	71.2$\pm$2.2 \\
%         Antmaze-medium-diverse-v0 &	7.3$\pm$3.1 & 43.6$\pm$2.7 & 64.9$\pm$3.8 & 70.0$\pm$1.6 \\
%         % Antmaze-large-play-v0 &	2.0$\pm$0.5 & 3.7$\pm$0.7 & 16.0$\pm$3.5 & 39.6$\pm$3.6 \\
%         % Antmaze-large-diverse-v0 & 2.4$\pm$1.0 & 16.0$\pm$3.6 & 20.5$\pm$3.7 & 47.5$\pm$1.1 \\
%         Pen-expert-v0 & 121.8$\pm$1.6 & 121.5$\pm$1.0 & 119.3$\pm$2.2 & 136.7$\pm$2.5 \\
%         Hammer-expert-v0 & 127.0$\pm$1.3 & 126.9$\pm$1.6 & 119.4$\pm$2.1 & 121.5$\pm$1.1 \\
%         Relocate-expert-v0 & 103.5$\pm$3.7 & 106.8$\pm$2.7 & 106.3$\pm$2.3 & 108.8$\pm$3.5\\ 
%         Door-expert-v0 & 105.1$\pm$2.7 & 105.2$\pm$1.9 & 105.4$\pm$3.6 & 106.3$\pm$2.9\\
%     \bottomrule
%     \end{tabular}
% \end{table}

% \clearpage
% \subsection{Generalizability of \name~ to other algorithms}
% \label{appendix: other algorithm}

% We add IQL~\cite{kostrikov2021offline} as a baseline and apply \name~to IQL by using the gradient of the training loss of the V-function in IQL as the criterion. 
% The experimental results in Table~\ref{tab: other algorithm} demonstrate both \name$_{\rm IQL}$ and \name$_{\rm TD3+BC}$ can achieve performance close to Complete Dataset (Best) with a small amount of data.

% \begin{table}[h]
%     \centering
%     \caption{Experimental results of applying \name~ to IQL.} 
%     \label{tab: other algorithm}
%     \begin{tabular}{c|cccc}
%     \toprule
%     Env & \name$_{\rm IQL}$ & \name$_{\rm TD3+BC}$ & IQL~(All Data) & TD3+BC~(All Data) \\
%     \midrule
%     Hopper-medium-v0 & 91.7$\pm$1.3 & 93.3$\pm$2.5 & 98.7$\pm$1.2 & 99.5$\pm$1.0\\
%     Walker2d-medium-v0 & 63.2$\pm$2.3 & 64.3$\pm$2.2 & 70.5$\pm$1.7 & 79.7$\pm$1.8\\
%     Halfcheetah-medium-v0 & 32.5$\pm$0.7 & 33.0$\pm$0.8 & 40.2$\pm$0.5 & 42.8$\pm$0.3\\
%     Hopper-medium-expert-v0 & 106.7$\pm$0.3 & 103.0$\pm$0.5 & 112.0$\pm$1.0 & 112.2$\pm$0.2\\
%     Walker2d-medium-expert-v0 & 84.0$\pm$8.1 & 77.0$\pm$8.6 & 105.0$\pm$4.7 & 101.1$\pm$9.3\\
%     Halfcheetah-medium-expert-v0 & 78.6$\pm$3.2 & 80.5$\pm$6.0 & 92.1$\pm$4.6 & 97.9$\pm$4.4\\
%     Hopper-expert-v0 & 112.2$\pm$0.3 & 108.6$\pm$0.8 & 112.2$\pm$0.6 & 112.2$\pm$0.2\\
%     Walker2d-expert-v0 & 83.0$\pm$4.5 & 83.8$\pm$4.2 & 106.8$\pm$2.6 & 105.7$\pm$2.7\\
%     Halfcheetah-expert-v0 & 78.9$\pm$1.7 & 85.6$\pm$1.2 & 107.0$\pm$2.1 & 105.7$\pm$1.9\\
%     \bottomrule
%     \end{tabular}
% \end{table}

% \clearpage

% \subsection{Generalizability of subset selecting by~\name}
% \label{appendix: tb3bc2iql}
% To test the generalizability of the dataset selected by~\name, we select subset by applying~\name~to TD3+BC.
% Then we evaluate the performance of IQL on the selected subset. 
% The experimental results in Table~\ref{tab: td3bc2iql} demonstrate that the selected subset based on TD3+BC is effectively applicable to IQL.
%  Across all tasks, the subset size is 10\% of the entire dataset.

% \begin{table}[h]
%     \centering
%     \caption{The performance of IQL on the subset selected based on TD3+BC.}
%     \label{tab: td3bc2iql}
%     \begin{tabular}{c|cc}
%     \toprule
%     Env & \name$_{\rm TD3+BC\rightarrow IQL}$ & IQL (All Data) \\
%     \midrule
%     Hopper-medium-v0 & 88.9$\pm$8.7 & 98.7$\pm$1.2\\
%     Walker2d-medium-v0 & 59.1$\pm$6.9 & 70.5$\pm$1.7\\
%     Halfcheetah-medium-v0 & 37.0$\pm$0.1 & 40.2$\pm$0.5 \\
%     Hopper-medium-expert-v0 & 100.5$\pm$1.6 & 112.0$\pm$1.0\\
%     Walker2d-medium-expert-v0 & 82.1$\pm$4.8 & 105.0$\pm$4.7 \\
%     Halfcheetah-medium-expert-v0 & 50.4$\pm$0.1 & 92.1$\pm$4.6 \\
%     Hopper-expert-v0 & 110.9$\pm$0.6 & 112.2$\pm$0.6\\
%     Walker2d-expert-v0 & 83.5$\pm$4.2 & 106.8$\pm$2.6\\
%     Halfcheetah-expert-v0 & 92.5$\pm$1.4 & 107.0$\pm$2.1\\
%     \bottomrule
%     \end{tabular}
% \end{table}

% \clearpage
% \subsection{Ablation study for cluster number}
% \label{appendix: cluster number}
% We evaluate the performance of \name~with various cluster numbers. The experimental results in Table~\ref{tab: cluster number} show that the suitable cluster number is between 25 and 50. Too few clusters (e.g., less than 5) are detrimental to the algorithm.

% \begin{table}[h]
%     \centering
%     \caption{Ablation study with the cluster number.} 
%     \label{tab: cluster number}
%     \begin{tabular}{c|cccccc}
%     \toprule
%     Cluster Number & 1 & 5 & 15 & 25 & 50\\
%     \midrule
%     Hopper-medium-v0 & 47.6$\pm$1.6 & 81.7$\pm$3.0 & 96.2$\pm$2.0 & 99.1$\pm$3.3 & 92.6$\pm$3.0 \\
%     Walker2d-medium-v0 & 9.5$\pm$1.1 & 5.9$\pm$3.6 & 32.2$\pm$2.4 & 64.1$\pm$1.9 & 57.9$\pm$3.6 \\
%     Halfcheetah-medium-v0 & 40.4$\pm$0.2 & 41.2$\pm$0.7 & 41.3$\pm$0.4 & 41.4$\pm$0.2 & 40.9$\pm$0.1\\
%     Hopper-expert-v0 & 97.5$\pm$1.9 & 112.2$\pm$1.4 & 111.3$\pm$2.1 & 111.5$\pm$1.6 & 110.6$\pm$1.9 \\
%     Walker2d-expert-v0 & 76.9$\pm$3.2 & 80.8$\pm$5.2 & 81.7$\pm$3.4 & 80.8$\pm$2.8 & 84.4$\pm$5.0 \\
%     Halfcheetah-expert-v0 & 84.3$\pm$2.7 & 82.9$\pm$2.8 & 83.0$\pm$3.2 & 82.3$\pm$1.9 & 84.3$\pm$3.5 \\
%     Hopper-medium-expert-v0 & 112.0$\pm$0.7 & 112.1$\pm$0.2 & 112.1$\pm$0.6 & 112.3$\pm$0.3 & 112.4$\pm$0.3 \\
%     Walker2d-medium-expert-v0 & 78.6$\pm$3.6 & 82.5$\pm$3.2 & 85.0$\pm$2.8 & 84.6$\pm$2.9 & 85.4$\pm$5.3 \\
%     Halfcheetah-medium-expert-v0 & 63.5$\pm$3.3 & 66.7$\pm$3.9 & 84.1$\pm$4.2 & 85.0$\pm$5.2 & 86.2$\pm$5.0 \\
%     \bottomrule
%     \end{tabular}
% \end{table}

% \subsection{Ablation study for approximation bounds}
% \label{appendix: approx bound} 
% We evaluate the performance of \name~with various approximation bounds (from 0.0001 to 0.05). 
% A smaller approximation bound represents a larger reduced dataset. The experimental results in Table~\ref{tab: approx bound} show that similar to the ablation of the size of the reduced dataset, \name~requires only a 0.01 approximation bound to obtain good performance.

% \begin{table}[h]
%     \centering
%     \caption{Ablation study with the approximation bounds.} 
%     \label{tab: approx bound}
%     \begin{tabular}{c|ccccc}
%     \toprule
%     Approximation Bounds & 0.0001 & 0.001 & 0.01 & 0.05 & All Data\\
%     \midrule
%     Hopper-medium-v0 & 97.9$\pm$1.3 & 94.6$\pm$0.8 & 92.2$\pm$1.1 & 31.9$\pm$1.9 & 99.5$\pm$1.0 \\
%     Walker2d-medium-v0 & 75.7$\pm$0.9 & 70.5$\pm$2.2 & 36.3$\pm$1.6 & 1.3$\pm$0.5 & 79.7$\pm$1.8\\
%     Halfcheetah-medium-v0 & 42.0$\pm$0.8 & 41.2$\pm$0.7 & 40.7$\pm$0.5 & 30.6$\pm$0.9 & 42.8$\pm$0.3 \\
%     Hopper-expert-v0 & 112.3$\pm$0.1 & 112.5$\pm$0.3 & 111.1$\pm$0.2 & 26.7$\pm$0.5 & 112.2$\pm$0.2\\
%     Walker2d-expert-v0 & 102.9$\pm$2.5 & 98.9$\pm$2.2 & 79.4$\pm$1.6 & 0.6$\pm$0.1 & 105.7$\pm$2.7\\
%     Halfcheetah-expert-v0 & 102.9$\pm$1.1 & 98.9$\pm$1.4 & 70.9$\pm$1.9 & 1.2$\pm$0.2 & 105.7$\pm$1.9\\
%     Hopper-medium-expert-v0 & 112.5$\pm$0.3 & 112.5$\pm$0.8 & 110.2$\pm$0.5 & 6.4$\pm$0.3 & 112.2$\pm$0.2\\
%     Walker2d-medium-expert-v0 & 101.2$\pm$5.7 & 98.8$\pm$7.4 & 82.3$\pm$6.9 & 1.2$\pm$0.6 & 101.1$\pm$9.3\\
%     Halfcheetah-medium-expert-v0 & 95.1$\pm$5.6 &	89.1$\pm$3.4 & 76.8$\pm$4.7 & 1.3$\pm$0.3 & 97.9$\pm$4.4\\
%     \bottomrule
%     \end{tabular}
% \end{table}
\clearpage
\subsection{Visualization Results}
\label{appendix: visual}
We visualize the selected data of ReDOR on various tasks based on the same method in Section~\ref{sec: exp}.


\begin{figure*}[ht]
    \centering
    \subfigure{\includegraphics[scale=0.4]{visual/hopper-medium-v0-on-iteration-0.pdf}}
    \caption{Visualization of selected data on hopper-medium-v0.}
\end{figure*}

\begin{figure*}[ht]
    \centering
    \subfigure{\includegraphics[scale=0.4]{visual/hopper-medium-expert-v0-on-iteration-0.pdf}}
    \caption{Visualization of selected data on hopper-medium-expert-v0.}
\end{figure*}

\begin{figure*}[ht]
    \centering
    \subfigure{\includegraphics[scale=0.4]{visual/hopper-expert-v0-on-iteration-0.pdf}}
    \caption{Visualization of selected data on hopper-expert-v0.}
\end{figure*}

\begin{figure*}[ht]
    \centering
    \subfigure{\includegraphics[scale=0.4]{visual/walker2d-medium-expert-v0-on-iteration-0.pdf}}
    \caption{Visualization of selected data on walker2d-medium-expert-v0.}
\end{figure*}

\begin{figure*}[ht]
    \centering
    \subfigure{\includegraphics[scale=0.4]{visual/walker2d-expert-v0-on-iteration-0.pdf}}
    \caption{Visualization of selected data on walker2d-expert-v0.}
\end{figure*}

\begin{figure*}[ht]
    \centering
    \subfigure{\includegraphics[scale=0.4]{visual/halfcheetah-medium-v0-on-iteration-0.pdf}}
    \caption{Visualization of selected data on halfcheetah-medium-v0.}
\end{figure*}

\begin{figure*}[ht]
    \centering
    \subfigure{\includegraphics[scale=0.4]{visual/halfcheetah-medium-expert-v0-on-iteration-0.pdf}}
    \caption{Visualization of selected data on halfcheetah-medium-expert-v0.}
\end{figure*}

\begin{figure*}[ht]
    \centering
    \subfigure{\includegraphics[scale=0.4]{visual/halfcheetah-expert-v0-on-iteration-0.pdf}}
    \caption{Visualization of selected data on halfcheetah-expert-v0.}
\end{figure*}
% \clearpage
% \section{ELBOW Experiments}
% \label{appendix: elbow}

% Determining the cluster number is crucial as it is used to solve the outer optimization issue in Equation~\ref{eq: gradient approx}.
% For this reason, we adopt the simple yet efficient elbow method to solve this issue.
% As shown in Figure~\ref{fig: elbow}, the appropriate number of clusters for these tasks tends to concentrate between 75 and 100.

% \begin{figure*}[h]
%     \centering
%     \subfigure{\includegraphics[scale=0.5]{elbow/hopper-medium-v0.pdf}}\subfigure{\includegraphics[scale=0.5]{elbow/hopper-medium-expert-v0.pdf}}\subfigure{\includegraphics[scale=0.5]{elbow/hopper-expert-v0.pdf}}\\\subfigure{\includegraphics[scale=0.5]{elbow/halfcheetah-medium-v0.pdf}}\subfigure{\includegraphics[scale=0.5]{elbow/halfcheetah-medium-expert-v0.pdf}}\subfigure{\includegraphics[scale=0.5]{elbow/halfcheetah-expert-v0.pdf}}
%     \\\subfigure{\includegraphics[scale=0.5]{elbow/walker2d-medium-v0.pdf}}\subfigure{\includegraphics[scale=0.5]{elbow/walker2d-expert-v0.pdf}}
%     \caption{The sum of squared errors of the data points~(named distance) with various cluster numbers.}
%     \label{fig: elbow}
% \end{figure*}

\clearpage
\section{Experimental Details}
\label{appendix: exp details}

\paragraph{Hyper-parameters.}
For the Mujoco tasks, we adopt the TD3+BC as the backbone of the offline algorithms.
For the Antmaze tasks, we adopt the IQL as the backbone of the offline algorithms.
We outline the hyper-parameters used by \name~ in Table~\ref{tab: parameters mujoco}.

\begin{table}[ht]
    \centering
    \begin{tabular}{ll}
    \toprule
    Hyperparameter & Value \\
    \midrule
    \hspace{0.3cm} Optimizer & Adam \\
    \hspace{0.3cm} Critic learning rate & 3e-4 \\
    \hspace{0.3cm} Actor learning rate & 3e-4 \\
    \hspace{0.3cm} Mini-batch size & 256 \\
    \hspace{0.3cm} Discount factor & 0.99 \\
    \hspace{0.3cm} Target update rate & 5e-3 \\
    \hspace{0.3cm} Policy noise & 0.2 \\
    \hspace{0.3cm} Policy noise clipping & (-0.5, 0.5) \\
    \hspace{0.3cm} TD3+BC regularized parameter & 2.5 \\
    \midrule
    Architecture & Value \\
    \midrule
    \hspace{0.3cm} Critic hidden dim & 256 \\
    \hspace{0.3cm} Critic hidden layers & 2 \\
    \hspace{0.3cm} Critic activation function & ReLU \\
    \hspace{0.3cm} Actor hidden dim & 256 \\
    \hspace{0.3cm} Actor hidden layers & 2 \\
    \hspace{0.3cm} Actor activation function & ReLU \\
    \midrule
    \name~Parameters & Value \\
    \midrule
    \hspace{0.3cm} Training rounds $T$ & 50 \\
    \hspace{0.3cm} $m$ & 50 \\
    \hspace{0.3cm} $\epsilon$ & 0.01 \\
    \bottomrule
    \end{tabular}
    \caption{Hyper-parameters sheet of ~\name}
    \label{tab: parameters mujoco}
\end{table}

% \section{Discussion of Limitations}
% \label{appendix: limitation}
% In this work, we consider using the TD loss gradient as the data subset selection criterion. 
% This is because if the gradients of the loss function used to train the $Q$ function are similar, the differences between $Q$ functions are also relatively small, thus making the policy on the data subset closer to that on the full dataset. 
% However, the theoretical framework does not directly present the relationship between the solution in the subset and the optimal solution for the original empirical Markov Decision Process (MDP).
% Nonetheless, our experimental results demonstrate the effectiveness of the proposed approach.

% \section{Broader Impacts}
% \label{appendix: impacts}
% This paper introduces a new perspective and pioneers a new path in the research of offline reinforcement learning. 
% This paper not only offers a reliable method for reducing dataset size, substantiated by sufficient proof, but also delineates the thresholds between adequate and inadequate dataset sizes through experiments, which provides considerable societal importance. 
% Our method can significantly reduce the burdens of training and storage by identifying a more compact subset of data. 
% % Conversely, in more societal domains where accumulating vast amounts of data is impractical, our approach offers guidance on the sufficient amount of data required. 
% This has the potential to expand the current boundaries of application in the field of offline reinforcement learning, making it more accessible and applicable in a broader range of societal contexts.

% % As for ethical aspects, to the best of our knowledge, the research presented in this paper does not directly engage with them. 
% % However, we acknowledge the importance of ethical considerations in machine learning research and strive to ensure that our work aligns with general ethical standards.
% \newpage
\section*{NeurIPS Paper Checklist}

% %%% BEGIN INSTRUCTIONS %%%
% The checklist is designed to encourage best practices for responsible machine learning research, addressing issues of reproducibility, transparency, research ethics, and societal impact. Do not remove the checklist: {\bf The papers not including the checklist will be desk rejected.} The checklist should follow the references and follow the (optional) supplemental material.  The checklist does NOT count towards the page
% limit. 

% Please read the checklist guidelines carefully for information on how to answer these questions. For each question in the checklist:
% \begin{itemize}
%     \item You should answer \answerYes{}, \answerNo{}, or \answerNA{}.
%     \item \answerNA{} means either that the question is Not Applicable for that particular paper or the relevant information is Not Available.
%     \item Please provide a short (1–2 sentence) justification right after your answer (even for NA). 
%    % \item {\bf The papers not including the checklist will be desk rejected.}
% \end{itemize}

% {\bf The checklist answers are an integral part of your paper submission.} They are visible to the reviewers, area chairs, senior area chairs, and ethics reviewers. You will be asked to also include it (after eventual revisions) with the final version of your paper, and its final version will be published with the paper.

% The reviewers of your paper will be asked to use the checklist as one of the factors in their evaluation. While "\answerYes{}" is generally preferable to "\answerNo{}", it is perfectly acceptable to answer "\answerNo{}" provided a proper justification is given (e.g., "error bars are not reported because it would be too computationally expensive" or "we were unable to find the license for the dataset we used"). In general, answering "\answerNo{}" or "\answerNA{}" is not grounds for rejection. While the questions are phrased in a binary way, we acknowledge that the true answer is often more nuanced, so please just use your best judgment and write a justification to elaborate. All supporting evidence can appear either in the main paper or the supplemental material, provided in appendix. If you answer \answerYes{} to a question, in the justification please point to the section(s) where related material for the question can be found.

% IMPORTANT, please:
% \begin{itemize}
%     \item {\bf Delete this instruction block, but keep the section heading ``NeurIPS paper checklist"},
%     \item  {\bf Keep the checklist subsection headings, questions/answers and guidelines below.}
%     \item {\bf Do not modify the questions and only use the provided macros for your answers}.
% \end{itemize} 
 

% %%% END INSTRUCTIONS %%%


\begin{enumerate}

\item {\bf Claims}
    \item[] Question: Do the main claims made in the abstract and introduction accurately reflect the paper's contributions and scope?
    \item[] Answer: \answerYes{} % Replace by \answerYes{}, \answerNo{}, or \answerNA{}.
    \item[] Justification: Our paper aims to solve a critical yet under-investigated issue in offline RL: determining the minimum subset of the offline dataset while ensuring superior performance for offline RL.
    This is presented in abstract and introduction.
    % \justificationTODO{}
    \item[] Guidelines:
    \begin{itemize}
        \item The answer NA means that the abstract and introduction do not include the claims made in the paper.
        \item The abstract and/or introduction should clearly state the claims made, including the contributions made in the paper and important assumptions and limitations. A No or NA answer to this question will not be perceived well by the reviewers. 
        \item The claims made should match theoretical and experimental results, and reflect how much the results can be expected to generalize to other settings. 
        \item It is fine to include aspirational goals as motivation as long as it is clear that these goals are not attained by the paper. 
    \end{itemize}

\item {\bf Limitations}
    \item[] Question: Does the paper discuss the limitations of the work performed by the authors?
    \item[] Answer: \answerYes{} % Replace by \answerYes{}, \answerNo{}, or \answerNA{}.
    \item[] Justification: We provide discussion of limitations in Appendix~\ref{appendix: limitation} due to the limited space in the main text.
    % \justificationTODO{}
    \item[] Guidelines:
    \begin{itemize}
        \item The answer NA means that the paper has no limitation while the answer No means that the paper has limitations, but those are not discussed in the paper. 
        \item The authors are encouraged to create a separate "Limitations" section in their paper.
        \item The paper should point out any strong assumptions and how robust the results are to violations of these assumptions (e.g., independence assumptions, noiseless settings, model well-specification, asymptotic approximations only holding locally). The authors should reflect on how these assumptions might be violated in practice and what the implications would be.
        \item The authors should reflect on the scope of the claims made, e.g., if the approach was only tested on a few datasets or with a few runs. In general, empirical results often depend on implicit assumptions, which should be articulated.
        \item The authors should reflect on the factors that influence the performance of the approach. For example, a facial recognition algorithm may perform poorly when image resolution is low or images are taken in low lighting. Or a speech-to-text system might not be used reliably to provide closed captions for online lectures because it fails to handle technical jargon.
        \item The authors should discuss the computational efficiency of the proposed algorithms and how they scale with dataset size.
        \item If applicable, the authors should discuss possible limitations of their approach to address problems of privacy and fairness.
        \item While the authors might fear that complete honesty about limitations might be used by reviewers as grounds for rejection, a worse outcome might be that reviewers discover limitations that aren't acknowledged in the paper. The authors should use their best judgment and recognize that individual actions in favor of transparency play an important role in developing norms that preserve the integrity of the community. Reviewers will be specifically instructed to not penalize honesty concerning limitations.
    \end{itemize}

\item {\bf Theory Assumptions and Proofs}
    \item[] Question: For each theoretical result, does the paper provide the full set of assumptions and a complete (and correct) proof?
    \item[] Answer: \answerYes{} % Replace by \answerYes{}, \answerNo{}, or \answerNA{}.
    \item[] Justification: For each theoretical result, we have provided the full set of assumptions and a complete (and correct) proof in the Appendix.
    % \justificationTODO{}
    \item[] Guidelines:
    \begin{itemize}
        \item The answer NA means that the paper does not include theoretical results. 
        \item All the theorems, formulas, and proofs in the paper should be numbered and cross-referenced.
        \item All assumptions should be clearly stated or referenced in the statement of any theorems.
        \item The proofs can either appear in the main paper or the supplemental material, but if they appear in the supplemental material, the authors are encouraged to provide a short proof sketch to provide intuition. 
        \item Inversely, any informal proof provided in the core of the paper should be complemented by formal proofs provided in appendix or supplemental material.
        \item Theorems and Lemmas that the proof relies upon should be properly referenced. 
    \end{itemize}

    \item {\bf Experimental Result Reproducibility}
    \item[] Question: Does the paper fully disclose all the information needed to reproduce the main experimental results of the paper to the extent that it affects the main claims and/or conclusions of the paper (regardless of whether the code and data are provided or not)?
    \item[] Answer: \answerYes{} % Replace by \answerYes{}, \answerNo{}, or \answerNA{}.
    \item[] Justification: We have provided needed hyper-parameter in Appendix~\ref{appendix: exp details}.
    We also provided code in the supplemental material.
    % \justificationTODO{}
    \item[] Guidelines:
    \begin{itemize}
        \item The answer NA means that the paper does not include experiments.
        \item If the paper includes experiments, a No answer to this question will not be perceived well by the reviewers: Making the paper reproducible is important, regardless of whether the code and data are provided or not.
        \item If the contribution is a dataset and/or model, the authors should describe the steps taken to make their results reproducible or verifiable. 
        \item Depending on the contribution, reproducibility can be accomplished in various ways. For example, if the contribution is a novel architecture, describing the architecture fully might suffice, or if the contribution is a specific model and empirical evaluation, it may be necessary to either make it possible for others to replicate the model with the same dataset, or provide access to the model. In general. releasing code and data is often one good way to accomplish this, but reproducibility can also be provided via detailed instructions for how to replicate the results, access to a hosted model (e.g., in the case of a large language model), releasing of a model checkpoint, or other means that are appropriate to the research performed.
        \item While NeurIPS does not require releasing code, the conference does require all submissions to provide some reasonable avenue for reproducibility, which may depend on the nature of the contribution. For example
        \begin{enumerate}
            \item If the contribution is primarily a new algorithm, the paper should make it clear how to reproduce that algorithm.
            \item If the contribution is primarily a new model architecture, the paper should describe the architecture clearly and fully.
            \item If the contribution is a new model (e.g., a large language model), then there should either be a way to access this model for reproducing the results or a way to reproduce the model (e.g., with an open-source dataset or instructions for how to construct the dataset).
            \item We recognize that reproducibility may be tricky in some cases, in which case authors are welcome to describe the particular way they provide for reproducibility. In the case of closed-source models, it may be that access to the model is limited in some way (e.g., to registered users), but it should be possible for other researchers to have some path to reproducing or verifying the results.
        \end{enumerate}
    \end{itemize}


\item {\bf Open access to data and code}
    \item[] Question: Does the paper provide open access to the data and code, with sufficient instructions to faithfully reproduce the main experimental results, as described in supplemental material?
    \item[] Answer: \answerYes{} % Replace by \answerYes{}, \answerNo{}, or \answerNA{}.
    \item[] Justification: We also provided code in the supplemental material.
    % \justificationTODO{}
    \item[] Guidelines:
    \begin{itemize}
        \item The answer NA means that paper does not include experiments requiring code.
        \item Please see the NeurIPS code and data submission guidelines (\url{https://nips.cc/public/guides/CodeSubmissionPolicy}) for more details.
        \item While we encourage the release of code and data, we understand that this might not be possible, so “No” is an acceptable answer. Papers cannot be rejected simply for not including code, unless this is central to the contribution (e.g., for a new open-source benchmark).
        \item The instructions should contain the exact command and environment needed to run to reproduce the results. See the NeurIPS code and data submission guidelines (\url{https://nips.cc/public/guides/CodeSubmissionPolicy}) for more details.
        \item The authors should provide instructions on data access and preparation, including how to access the raw data, preprocessed data, intermediate data, and generated data, etc.
        \item The authors should provide scripts to reproduce all experimental results for the new proposed method and baselines. If only a subset of experiments are reproducible, they should state which ones are omitted from the script and why.
        \item At submission time, to preserve anonymity, the authors should release anonymized versions (if applicable).
        \item Providing as much information as possible in supplemental material (appended to the paper) is recommended, but including URLs to data and code is permitted.
    \end{itemize}


\item {\bf Experimental Setting/Details}
    \item[] Question: Does the paper specify all the training and test details (e.g., data splits, hyperparameters, how they were chosen, type of optimizer, etc.) necessary to understand the results?
    \item[] Answer: \answerYes{} % Replace by \answerYes{}, \answerNo{}, or \answerNA{}.
    \item[] Justification: We have provided training details and hyper-parameters in Algorithm~\ref{alg: offline data selection} and Appendix~\ref{appendix: exp details}.
    % \justificationTODO{}
    \item[] Guidelines:
    \begin{itemize}
        \item The answer NA means that the paper does not include experiments.
        \item The experimental setting should be presented in the core of the paper to a level of detail that is necessary to appreciate the results and make sense of them.
        \item The full details can be provided either with the code, in appendix, or as supplemental material.
    \end{itemize}

\item {\bf Experiment Statistical Significance}
    \item[] Question: Does the paper report error bars suitably and correctly defined or other appropriate information about the statistical significance of the experiments?
    \item[] Answer: \answerYes{} % Replace by \answerYes{}, \answerNo{}, or \answerNA{}.
    \item[] Justification: For each set of experiments, we ran 5 seeds and reported error bars.
    % \justificationTODO{}
    \item[] Guidelines:
    \begin{itemize}
        \item The answer NA means that the paper does not include experiments.
        \item The authors should answer "Yes" if the results are accompanied by error bars, confidence intervals, or statistical significance tests, at least for the experiments that support the main claims of the paper.
        \item The factors of variability that the error bars are capturing should be clearly stated (for example, train/test split, initialization, random drawing of some parameter, or overall run with given experimental conditions).
        \item The method for calculating the error bars should be explained (closed form formula, call to a library function, bootstrap, etc.)
        \item The assumptions made should be given (e.g., Normally distributed errors).
        \item It should be clear whether the error bar is the standard deviation or the standard error of the mean.
        \item It is OK to report 1-sigma error bars, but one should state it. The authors should preferably report a 2-sigma error bar than state that they have a 96\% CI, if the hypothesis of Normality of errors is not verified.
        \item For asymmetric distributions, the authors should be careful not to show in tables or figures symmetric error bars that would yield results that are out of range (e.g. negative error rates).
        \item If error bars are reported in tables or plots, The authors should explain in the text how they were calculated and reference the corresponding figures or tables in the text.
    \end{itemize}

\item {\bf Experiments Compute Resources}
    \item[] Question: For each experiment, does the paper provide sufficient information on the computer resources (type of compute workers, memory, time of execution) needed to reproduce the experiments?
    \item[] Answer: \answerYes{} % Replace by \answerYes{}, \answerNo{}, or \answerNA{}.
    \item[] Justification: We provided the compute resources in Appendix~\ref{appendix: computation complexity}.
    % \justificationTODO{}
    \item[] Guidelines:
    \begin{itemize}
        \item The answer NA means that the paper does not include experiments.
        \item The paper should indicate the type of compute workers CPU or GPU, internal cluster, or cloud provider, including relevant memory and storage.
        \item The paper should provide the amount of compute required for each of the individual experimental runs as well as estimate the total compute. 
        \item The paper should disclose whether the full research project required more compute than the experiments reported in the paper (e.g., preliminary or failed experiments that didn't make it into the paper). 
    \end{itemize}
    
\item {\bf Code Of Ethics}
    \item[] Question: Does the research conducted in the paper conform, in every respect, with the NeurIPS Code of Ethics \url{https://neurips.cc/public/EthicsGuidelines}?
    \item[] Answer: \answerYes{} % Replace by \answerYes{}, \answerNo{}, or \answerNA{}.
    \item[] Justification: Our paper conform, in every respect, with the NeurIPS Code of Ethics.
    % \justificationTODO{}
    \item[] Guidelines:
    \begin{itemize}
        \item The answer NA means that the authors have not reviewed the NeurIPS Code of Ethics.
        \item If the authors answer No, they should explain the special circumstances that require a deviation from the Code of Ethics.
        \item The authors should make sure to preserve anonymity (e.g., if there is a special consideration due to laws or regulations in their jurisdiction).
    \end{itemize}


\item {\bf Broader Impacts}
    \item[] Question: Does the paper discuss both potential positive societal impacts and negative societal impacts of the work performed?
    \item[] Answer: \answerYes{} % Replace by \answerYes{}, \answerNo{}, or \answerNA{}.
    \item[] Justification: We provide discussion of broader impacts in Appendix~\ref{appendix: impacts} due to the limited space in the main text.
    % \justificationTODO{}
    \item[] Guidelines:
    \begin{itemize}
        \item The answer NA means that there is no societal impact of the work performed.
        \item If the authors answer NA or No, they should explain why their work has no societal impact or why the paper does not address societal impact.
        \item Examples of negative societal impacts include potential malicious or unintended uses (e.g., disinformation, generating fake profiles, surveillance), fairness considerations (e.g., deployment of technologies that could make decisions that unfairly impact specific groups), privacy considerations, and security considerations.
        \item The conference expects that many papers will be foundational research and not tied to particular applications, let alone deployments. However, if there is a direct path to any negative applications, the authors should point it out. For example, it is legitimate to point out that an improvement in the quality of generative models could be used to generate deepfakes for disinformation. On the other hand, it is not needed to point out that a generic algorithm for optimizing neural networks could enable people to train models that generate Deepfakes faster.
        \item The authors should consider possible harms that could arise when the technology is being used as intended and functioning correctly, harms that could arise when the technology is being used as intended but gives incorrect results, and harms following from (intentional or unintentional) misuse of the technology.
        \item If there are negative societal impacts, the authors could also discuss possible mitigation strategies (e.g., gated release of models, providing defenses in addition to attacks, mechanisms for monitoring misuse, mechanisms to monitor how a system learns from feedback over time, improving the efficiency and accessibility of ML).
    \end{itemize}
    
\item {\bf Safeguards}
    \item[] Question: Does the paper describe safeguards that have been put in place for responsible release of data or models that have a high risk for misuse (e.g., pretrained language models, image generators, or scraped datasets)?
    \item[] Answer: \answerNA{} % Replace by \answerYes{}, \answerNo{}, or \answerNA{}.
    \item[] Justification: Our experiments on standard offline reinforcement learning datasets do not involve this risk.
    % \justificationTODO{}
    \item[] Guidelines:
    \begin{itemize}
        \item The answer NA means that the paper poses no such risks.
        \item Released models that have a high risk for misuse or dual-use should be released with necessary safeguards to allow for controlled use of the model, for example by requiring that users adhere to usage guidelines or restrictions to access the model or implementing safety filters. 
        \item Datasets that have been scraped from the Internet could pose safety risks. The authors should describe how they avoided releasing unsafe images.
        \item We recognize that providing effective safeguards is challenging, and many papers do not require this, but we encourage authors to take this into account and make a best faith effort.
    \end{itemize}

\item {\bf Licenses for existing assets}
    \item[] Question: Are the creators or original owners of assets (e.g., code, data, models), used in the paper, properly credited and are the license and terms of use explicitly mentioned and properly respected?
    \item[] Answer: \answerYes{} % Replace by \answerYes{}, \answerNo{}, or \answerNA{}.
    \item[] Justification: We use standard offline reinforcement learning datasets and are properly cited in the paper.
    % \justificationTODO{}
    \item[] Guidelines:
    \begin{itemize}
        \item The answer NA means that the paper does not use existing assets.
        \item The authors should cite the original paper that produced the code package or dataset.
        \item The authors should state which version of the asset is used and, if possible, include a URL.
        \item The name of the license (e.g., CC-BY 4.0) should be included for each asset.
        \item For scraped data from a particular source (e.g., website), the copyright and terms of service of that source should be provided.
        \item If assets are released, the license, copyright information, and terms of use in the package should be provided. For popular datasets, \url{paperswithcode.com/datasets} has curated licenses for some datasets. Their licensing guide can help determine the license of a dataset.
        \item For existing datasets that are re-packaged, both the original license and the license of the derived asset (if it has changed) should be provided.
        \item If this information is not available online, the authors are encouraged to reach out to the asset's creators.
    \end{itemize}

\item {\bf New Assets}
    \item[] Question: Are new assets introduced in the paper well documented and is the documentation provided alongside the assets?
    \item[] Answer: \answerNA{} % Replace by \answerYes{}, \answerNo{}, or \answerNA{}.
    \item[] Justification: The paper does not release new assets.
    % \justificationTODO{}
    \item[] Guidelines:
    \begin{itemize}
        \item The answer NA means that the paper does not release new assets.
        \item Researchers should communicate the details of the dataset/code/model as part of their submissions via structured templates. This includes details about training, license, limitations, etc. 
        \item The paper should discuss whether and how consent was obtained from people whose asset is used.
        \item At submission time, remember to anonymize your assets (if applicable). You can either create an anonymized URL or include an anonymized zip file.
    \end{itemize}

\item {\bf Crowdsourcing and Research with Human Subjects}
    \item[] Question: For crowdsourcing experiments and research with human subjects, does the paper include the full text of instructions given to participants and screenshots, if applicable, as well as details about compensation (if any)? 
    \item[] Answer: \answerNA{} % Replace by \answerYes{}, \answerNo{}, or \answerNA{}.
    \item[] Justification: The paper does not involve crowdsourcing nor research with human subjects.
    % \justificationTODO{}
    \item[] Guidelines:
    \begin{itemize}
        \item The answer NA means that the paper does not involve crowdsourcing nor research with human subjects.
        \item Including this information in the supplemental material is fine, but if the main contribution of the paper involves human subjects, then as much detail as possible should be included in the main paper. 
        \item According to the NeurIPS Code of Ethics, workers involved in data collection, curation, or other labor should be paid at least the minimum wage in the country of the data collector. 
    \end{itemize}

\item {\bf Institutional Review Board (IRB) Approvals or Equivalent for Research with Human Subjects}
    \item[] Question: Does the paper describe potential risks incurred by study participants, whether such risks were disclosed to the subjects, and whether Institutional Review Board (IRB) approvals (or an equivalent approval/review based on the requirements of your country or institution) were obtained?
    \item[] Answer: \answerNA{} % Replace by \answerYes{}, \answerNo{}, or \answerNA{}.
    \item[] Justification: The paper does not involve crowdsourcing nor research with human subjects.
    % \justificationTODO{}
    \item[] Guidelines:
    \begin{itemize}
        \item The answer NA means that the paper does not involve crowdsourcing nor research with human subjects.
        \item Depending on the country in which research is conducted, IRB approval (or equivalent) may be required for any human subjects research. If you obtained IRB approval, you should clearly state this in the paper. 
        \item We recognize that the procedures for this may vary significantly between institutions and locations, and we expect authors to adhere to the NeurIPS Code of Ethics and the guidelines for their institution. 
        \item For initial submissions, do not include any information that would break anonymity (if applicable), such as the institution conducting the review.
    \end{itemize}

\end{enumerate}

\end{document}
