Biological knowledge bases provide systemically functional pathways of cells or organisms in terms of molecular interaction.
However, recognizing more targeted pathways, particularly when incorporating wet-lab experimental data, remains challenging and typically requires downstream biological analyses and expertise.
In this paper, we frame this challenge as a solvable graph learning and explaining task and propose a novel pathway inference framework, \method, that explicitly integrates experimental data, specifically amino acid sequences (AA-seqs), to classify various graphs (bio-networks) in biological databases.
The links (representing pathways) that contribute more to classification can be considered as targeted pathways.
Technically, \method comprises three components: (1) a large protein language model (pLM) that encodes and embeds AA-seqs into graph, overcoming traditional obstacles in processing AA-seq data, such as BLAST; (2) \classifier, a hybrid architecture combining graph neural networks (GNNs) with state-space sequence modeling (Mamba) to capture both local interactions and global pathway-level dependencies; 
and (3) \explainer, a subgraph learning module that identifies functionally critical nodes and edges through trainable pathway masks.  
We also propose ML-oriented biological evaluations and a new metric.
The experiments involving 301 bio-networks evaluations demonstrate that pathways inferred by \method maintain biological meaningfulness.
We will publicly release curated 301 bio-network data soon.
The source code is available at:
\url{https://anonymous.4open.science/r/ExPath}
% {\color{red}
% Rikuto: need to revise according our revised paper
% }
% We further provide theoretical analysis for the power of \method.
% Experiments on a newly curated dataset of 301 pathway networks spanning four biological function categories demonstrate that our framework outperforms existing methods by XX\% on key graph metrics. Biological validations further confirm that the extracted subgraphs preserve XX\% of known functional interactions.


% formulate biological networks to graphs, learning and classifying graphs in a data-driven manner, explicitly integrate experimental data, identifying more contributable links, i,e, pathways as targeted 

% to explicitly integrate experimental data, specifically amino acid sequences (AA-seq), with knowledge bases for identifying targeted pathways in a data-driven manner.

% We formulate 
% Our framework comprises three main components: (1) a pLM that encodes amino acid (AA) sequences into functional embeddings, overcoming traditional obstacles in processing AA sequence data; (2) \classifier, a hybrid architecture combining graph neural networks (GNNs) with state-space sequence modeling (Mamba) to capture both local interactions and global pathway-level dependencies; and (3) \explainer, a subgraph learning module that identifies functionally critical nodes and edges through trainable pathway masks. Experiments on a newly curated dataset of 301 pathway networks spanning four biological function categories demonstrate that our framework outperforms existing methods by XX\% on key graph metrics. Biological validations further confirm that the extracted subgraphs preserve XX\% of known functional interactions.


% % Knowledge databases that compile various biological networks or functional pathways provide foundational references for modeling cellular processes and disease mechanisms. 
% % However, these networks often exhibit functional redundancy and fail to incorporate critical wet-lab data—essential for explaining targeted interactions, hindering the identification of biological function-specific pathways and their underlying mechanisms. 
% To address this challenge, we propose a novel framework, PathNet, for mining function-specific pathway networks by integrating large protein language models (pLMs) with graph-based explainable learning.
% Our framework comprises three main components: (1) a pLM that encodes amino acid (AA) sequences into functional embeddings, overcoming traditional obstacles in processing AA sequence data; (2) \classifier, a hybrid architecture combining graph neural networks (GNNs) with state-space sequence modeling (Mamba) to capture both local interactions and global pathway-level dependencies; and (3) \explainer, a subgraph learning module that identifies functionally critical nodes and edges through trainable pathway masks. Experiments on a newly curated dataset of 301 pathway networks spanning four biological function categories demonstrate that our framework outperforms existing methods by XX\% on key graph metrics. Biological validations further confirm that the extracted subgraphs preserve XX\% of known functional interactions.
