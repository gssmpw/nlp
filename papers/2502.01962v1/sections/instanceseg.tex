% -------------------------------
\subsection{Experiments on Object Detection (ODet) and Instance Segmentation (ISeg)}
\label{sec4:2}
% -------------------------------
\myparagraph{Baselines and settings.} As in~\citep{chen2022vision,xiong2024efficient,jie2022convolutional,marouf2024mini}, Mask R-CNN~\citep{he2017mask}, Cascade Mask R-CNN~\citep{cai2019cascade}, ATSS~\citep{zhang2020bridging}, and GFL~\citep{li2020generalized} are employed as the baseline models, where the pre-trained ViT~\citep{li2022exploring} is used as the backbone. All the baseline models are pre-trained on ImageNet-1k by default~\citep{deng2009imagenet}. Unless otherwise specified, these baselines are set up to be consistent with their papers and the settings of the ViT-Adapter~\citep{chen2022vision} method.
% -------------------------------
\begin{table*}[!th]
\centering
\resizebox{\textwidth}{!}{%
\begin{tabular}{@{}llcccccccccc@{}}
\toprule
& & \multicolumn{2}{c}{\textbf{Intent Detection}} & \multicolumn{2}{c}{\textbf{Topic Mining}} & \multicolumn{2}{c}{\textbf{Domain Discovery}} & \multicolumn{1}{c}{\textbf{Type}} & \multicolumn{1}{c}{\textbf{Emotion}} & \\
\cmidrule(lr){3-4} \cmidrule(lr){5-6} \cmidrule(lr){7-8} \cmidrule(lr){9-9} \cmidrule(lr){10-10}  %\cmidrule(lr){11-11}
\textbf{Model} & \textbf{Method} & \textbf{BANKING} & \textbf{CLINC} & \textbf{Reddit} & \textbf{StackEx} & \textbf{MTOP} & \textbf{CLINC(D)} & \textbf{FewEvent} & \textbf{GoEmotion} & \textbf{AVG} \\ \midrule \midrule
GPT-4o-mini & Standard Prompting & 0.652 & 0.792 & 0.534 & 0.482 & 0.896 & 0.536 & 0.630 & 0.378 & 0.613 \\
& Self-Consistency & 0.666 & 0.802 & 0.586 & 0.494 & 0.902 & 0.530 & 0.640 & 0.382 & 0.625 \\
& TestNUC & 0.712 & 0.858 & 0.614 & 0.528 & 0.936 & 0.544 & 0.674 & 0.410 & 0.660 \\
& \cellcolor{gray!18}TestNUC\textdagger & \cellcolor{gray!18}\textbf{0.764} & \cellcolor{gray!18}\textbf{0.864} & \cellcolor{gray!18}\textbf{0.646} & \cellcolor{gray!18}\textbf{0.540} & \cellcolor{gray!18}\textbf{0.948} & \cellcolor{gray!18}\textbf{0.554} & \cellcolor{gray!18}\textbf{0.680} & \cellcolor{gray!18}\textbf{0.414} & \cellcolor{gray!18}\textbf{0.676} \\ \midrule \midrule
Llama-3.1-8B & Standard Prompting & 0.572 & 0.726 & 0.502 & 0.492 & 0.892 & 0.528 & 0.530 & 0.332 & 0.572 \\
& Self-Consistency & 0.620 & 0.774 & 0.564 & 0.526 & 0.902 & 0.518 & 0.564 & 0.340 & 0.601 \\
& TestNUC & 0.694 & 0.806 & 0.618 & 0.558 & 0.934 & 0.528 & 0.596 & 0.356 & 0.636 \\
& \cellcolor{gray!18}TestNUC\textdagger & \cellcolor{gray!18}\textbf{0.724} & \cellcolor{gray!18}\textbf{0.812} & \cellcolor{gray!18}\textbf{0.646} & \cellcolor{gray!18}\textbf{0.576} & \cellcolor{gray!18}\textbf{0.940} & \cellcolor{gray!18}\textbf{0.542} & \cellcolor{gray!18}\textbf{0.614} & \cellcolor{gray!18}\textbf{0.360} & \cellcolor{gray!18}\textbf{0.652} \\ \midrule \midrule
Claude-3-Haiku & Standard Prompting & 0.680 & 0.848 & 0.486 & 0.564 & 0.892 & 0.552 & 0.594 & 0.336 & 0.619 \\
& Self-Consistency & 0.702 & 0.870 & 0.510 & 0.578 & 0.904 & 0.564 & 0.568 & 0.350 & 0.631 \\
& TestNUC & 0.762 & 0.894 & 0.596 & 0.588 & 0.940 & 0.590 & 0.620 & 0.348 & 0.667 \\
& \cellcolor{gray!18}TestNUC\textdagger & \cellcolor{gray!18}\textbf{0.804} & \cellcolor{gray!18}\textbf{0.902} & \cellcolor{gray!18}\textbf{0.612} & \cellcolor{gray!18}\textbf{0.600} & \cellcolor{gray!18}\textbf{0.946} & \cellcolor{gray!18}\textbf{0.622} & \cellcolor{gray!18}\textbf{0.660} & \cellcolor{gray!18}\textbf{0.368} & \cellcolor{gray!18}\textbf{0.689} \\ \midrule \midrule
GPT-4o & Standard Prompting & 0.746 & 0.924 & 0.712 & 0.674 & 0.962 & 0.614 & 0.682 & 0.406 & 0.715 \\
& Self-Consistency & 0.758 & 0.922 & 0.720 & 0.688 & 0.958 & 0.624 & 0.696 & 0.426 & 0.724 \\
&TestNUC & 0.804 & 0.934 & 0.744 & \textbf{0.710} & 0.974 & 0.644 & 0.692 & 0.446 & 0.744 \\
& \cellcolor{gray!18}TestNUC\textdagger & \cellcolor{gray!18}\textbf{0.824} & \cellcolor{gray!18}\textbf{0.940} & \cellcolor{gray!18}\textbf{0.750} & \cellcolor{gray!18}\textbf{0.710} & \cellcolor{gray!18}\textbf{0.978} & \cellcolor{gray!18}\textbf{0.654} & \cellcolor{gray!18}\textbf{0.708} & \cellcolor{gray!18}\textbf{0.464} & \cellcolor{gray!18}\textbf{0.754} \\
\bottomrule
\end{tabular}%
}
\caption{Accuracy comparison with Standard Prompting and Self-Consistency across four diverse LLMs. TestNUC consistently improves the inference performance on all benchmark datasets. $\dagger$ denotes that 50 neighbors are utilized.}
\label{tab:main_compare_sc}
\end{table*}
% -------------------------------

% -------------------------------
\myparagraph{Comparisons with state-of-the-art (SOTA) methods.} Result comparisons with SOTA methods with Mask R-CNN~\citep{he2017mask} for ODet and ISeg are shown in Table~\ref{tab1}. From this table, we can obtain the following observations and conclusions: 
%
\emph{\textbf{\romannumeral1}})~Compared to the experimental results of the Mask R-CNN~\citep{he2017mask} model with ViT~\citep{li2021benchmarking}, our proposed META can consistently improve accuracy for ODet and ISeg across different model scales (\eg, ViT-T/S/B/L~\citep{li2021benchmarking}), while only adding a small number of training parameters. Even with different training schedules (\ie, 1$\times$, and 3$\times$ with MS), our method can also improve the model performance, demonstrating the plug-and-play advantage of META. 
%
\emph{\textbf{\romannumeral2}})~Compared to the SOTA ViT-Adapter~\citep{chen2022vision} and LoSA~\citep{mercea2024time}, META can achieve a new accuracy-efficiency trade-off. For example, in settings with the strong ViT-B~\citep{li2021benchmarking} as the backbone, our method achieves a performance gain of $0.5\%$AP$^\textrm{m}$ under 1$\times$ training schedule and $0.7\%$AP$^\textrm{m}$ under 3$\times$ training schedule with MS while reducing 4.9\textbf{M} model parameters, when compared to ViT-Adapter~\citep{chen2022vision}. 
%
\emph{\textbf{\romannumeral3}})~Even with stronger pre-trained models, META still improves performance on the baseline models and surpasses existing methods on accuracy, parameters and memory. For example, on the ImageNet-22k pre-trained weights from~\citep{steiner2021train}, our method achieves a performance gain of $0.8\%$/$0.7\%$AP$^\textrm{m}$ under 1$\times$/3$\times$ training schedule while reducing 8.2\textbf{M} parameters compared to ViT-Adapter~\citep{chen2022vision}, which validates its strong learning ability and flexible adaptability.
%
\emph{\textbf{\romannumeral4}})~Compared to SOTA ODet and ISeg methods, META also has very competitive performance. For example, compared to the strong ViTDet-B~\citep{li2022exploring} model, META-T has $2.2\%$AP$^\textrm{b}$ and $3.1\%$AP$^\textrm{m}$ gains under the 1$\times$ training schedule, and has META-T has $4.9\%$AP$^\textrm{b}$ and $2.7\%$AP$^\textrm{m}$ gains under the 3$\times$ training schedule.
%
\emph{\textbf{\romannumeral5}})~Compared to SOTA ViT adapter methods such as AdaptFormer~\citep{chen2022adaptformer}, FacT-TK~\citep{jie2023fact}, and LoSA~\citep{mercea2024time}, our method exhibits superior efficiency in terms of reduced parameter count, decreased FLOPs, and lower MC. Particularly, META only utilizes $62\%$ of the MC of LoSA while achieving superior prediction accuracy. Therefore, the experimental results and conclusions above demonstrate that our method achieves better accuracy and higher efficiency in dense prediction tasks.
% -------------------------------
\begin{table}[tbp]
  \caption{Performance of GPT-4o and Mistral-7B-Instruct-v0.3 as listwise LLM re-rankers on the NevIR test set with few-shot prompting.}
  \label{tab:few_shot}
  \centering
  \begin{tabular}{lcc}
    \toprule
    \textbf{Model Name} & \textbf{Shots} & \textbf{NevIR Score} \\
    \midrule
    \multirow{4}{*}{GPT-4o} & Zero-shot & 70.1\% \\
                             & 1-shot    & 72.0\% \\
                             & 3-shot    & 74.5\% \\
                             & 5-shot    & 76.9\% \\
    \midrule
    \multirow{4}{*}{Mistral-7B-Instruct-v0.3} & Zero-shot & 46.3\% \\
                                              & 1-shot    & 42.6\% \\
                                              & 3-shot    & 37.1\% \\
                                              & 5-shot    & 39.0\% \\
    \bottomrule
  \end{tabular}
\end{table}


    

% -------------------------------


% -------------------------------
\myparagraph{Superiority performance under different baselines.} 
% -------------------------------
In addition to Mask R-CNN, we also choose Cascade Mask R-CNN~\citep{cai2019cascade}, ATSS~\citep{zhang2020bridging}, and GFL~\citep{li2020generalized} as the baselines as in~\cite{chen2022vision,mercea2024time}. We explore the effectiveness and superiority performance of META on these baseline models, where the 3$\times$ training with MS strategy is used. 
% 
The experimental results are given in Table~\ref{tab2}. We can observe that META can consistently improve performance across different baselines, and exhibits more accurate and more efficient advantages compared to SOTA ViT adapter methods. 
%
For example, based on the Cascade Mask R-CNN~\citep{cai2019cascade}, META-S can achieve up to $44.8\%$ AP$^\textrm{m}$ with only $83$\textbf{M} parameters, $797$\textbf{G} FLOPs, and $8.1$\textbf{GB} MC, which is $1.3\%$ AP$^\textrm{m}$ higher, $3$\textbf{M} fewer parameters, $4$\textbf{G} fewer FLOPs, and $7.1$\textbf{GB} fewer MC compared to the competitive ViT-Adapter-T method~\citep{chen2022vision}. 
%
Besides, META-B can achieve a $1.7\%$ AP$^\textrm{b}$/$0.5\%$ AP$^\textrm{m}$ gain on the 3$\times$ training schedule, with $5$\textbf{M} fewer parameters and $5$\textbf{G} fewer FLOPs than the ViT-Adapter-B~\citep{chen2022vision} model.
%
On ATSS and GFL, our method also achieves competitive $43.2\%$ and $44.$0\% AP$^\textrm{m}$, respectively, which are $0.7\%$ and $0.9\%$ AP$^\textrm{m}$ higher than the competitive vision adapter ViT-Adapter-S~\citep{chen2022vision}. 
%
In terms of efficiency, compare with the SOTA LoSA~\citep{mercea2024time} method,  our method has fewer parameters ($33$\textbf{M} v.s. $35$\textbf{M}), fewer FLOPs ($265$\textbf{G} v.s. $268$\textbf{G} under ATSS, $279$\textbf{G} v.s. $284$\textbf{G} under GFL), and fewer MC ($8.1$\textbf{GB} v.s. $13.0$\textbf{GB} under ATSS and GFL), indicating its higher computation and memory efficiency.
%  % \begin{table*}[t]
%     \caption{Performance Metrics on MMQA Subset after the attack with poisoned images cointaining adversarial noise. Capt. stands for Captions and denotes wether the re-ranker has access to image captions. }
%     \label{tab:mmqa_adv}
%     \centering
%     \resizebox{\textwidth}{!}{%
%     \begin{tabular}{@{}lc|cc|cc@{}}
%         \toprule
%         \textbf{Retriever} & \textbf{Reranker} & \textbf{Capt.} & \multicolumn{2}{c|}{\textbf{Original} (\small{Before $\rightarrow$ After})} & \multicolumn{2}{c@{}}{\textbf{Poisoned}} \\ 
%         \textbf{CLIP} & \textbf{LLaVA} &  & \textbf{Recall (\%)} & \textbf{Accuracy (\%)} & \textbf{Recall (\%)} & \textbf{Accuracy (\%)} \\
%         \midrule
%         $K=1$   & \xmark                      & -           & 83.7 $\rightarrow$ 11.4 \textcolor{red}{\small{(-72.3)}}             & 61.0 $\rightarrow$ 18.4 \textcolor{red}{\small{(-42.6)}}                 & 87.9   &  58.2  \\
%         $K=5$   & \xmark                      & -           & 92.9 $\rightarrow$ 92.2 \textcolor{red}{\small{(-0.7)}}           & 39.0 $\rightarrow$ 30.5 \textcolor{red}{\small{(-8.5)}}                   & 100.0   & 36.2         \\
%         $K=1$   & $N=1$          & \cmark      & 84.4 $\rightarrow$ 30.5 \textcolor{red}{\small{(-53.9)}}                    & 62.4 $\rightarrow$ 29.1 \textcolor{red}{\small{(-33.3)}}          & 63.8   & 48.2         \\
%         $K=5$   & $N=1$          & \xmark      & 70.2 $\rightarrow$ 44.0 \textcolor{red}{\small{(-26.2)}}                    & 58.9$ \rightarrow$ 39.7 \textcolor{red}{\small{(-19.2)}}          & 45.4  & 38.3          \\
%         \bottomrule
%     \end{tabular}%
%     }
% \end{table*}

\begin{table*}[t]
    \caption{Performance Metrics on MMQA Subset after the attack with poisoned images cointaining adversarial noise. Capt. stands for Captions and denotes wether the re-ranker has access to image captions. }
    \label{tab:mmqa_adv}
    \centering
    \resizebox{\textwidth}{!}{%
    \begin{tabular}{@{}cllcccccccc@{}}
        \toprule
       \multicolumn{11}{c}{\textbf{Retriever}: CLIP-ViT-L \textbf{Reranker}: LLaVA \textbf{Generator}: LLaVA} \\
       \midrule
       & \textbf{Retriever} & \textbf{Reranker} & \textbf{Capt.} & \multicolumn{3}{c|}{\textbf{Original Recall (\%)}} & \multicolumn{3}{c}{\textbf{Original Accuracy (\%)}} & Poisoned \\
       & &  &  & Before & After & Change & Before & After & Change & Recall (\%) \\
        $K=1$   & \xmark                      & -           & 83.7 $\rightarrow$ 11.4 \textcolor{red}{\small{(-72.3)}}             & 61.0 $\rightarrow$ 18.4 \textcolor{red}{\small{(-42.6)}}                 & 87.9   &  58.2  \\
        $K=5$   & \xmark                      & -           & 92.9 $\rightarrow$ 92.2 \textcolor{red}{\small{(-0.7)}}           & 39.0 $\rightarrow$ 30.5 \textcolor{red}{\small{(-8.5)}}                   & 100.0   & 36.2         \\
        $K=1$   & $N=1$          & \cmark      & 84.4 $\rightarrow$ 30.5 \textcolor{red}{\small{(-53.9)}}                    & 62.4 $\rightarrow$ 29.1 \textcolor{red}{\small{(-33.3)}}          & 63.8   & 48.2         \\
        $K=5$   & $N=1$          & \xmark      & 70.2 $\rightarrow$ 44.0 \textcolor{red}{\small{(-26.2)}}                    & 58.9$ \rightarrow$ 39.7 \textcolor{red}{\small{(-19.2)}}          & 45.4  & 38.3          \\
        \bottomrule
    \end{tabular}%
    }
\end{table*}




% -------------------------------
% \myparagraph{Results on different pre-trained weights.} 
% -------------------------------
% In Table~\ref{tab3}, we present the experimental results of META with different pre-trained weights and compare them with other SOTA methods including SwinViT~\citep{liu2021swin} and ViT-Adapter~\citep{chen2022vision} as in~\citep{chen2022vision}. Mask R-CNN~\citep{he2017mask} is used as the baseline, and ViT-B~\citep{li2022exploring} is used as the backbone. The 3$\times$ training schedule with MS training strategy is used. From this table, we can observe that our method is applicable to different pre-trained weights (\ie, ImageNet-1k~\citep{deng2009imagenet}, ImageNet-22k~\citep{steiner2021train}, and Multi-Modal~\citep{zhu2022uni}), and achieves more accurate AP with fewer model parameters and FLOPs compared to ViT-Adapter~\citep{chen2022vision}, across different pre-trained weights. Our method achieves $0.7\%$, $0.6\%$, and $0.6\%$ higher AP than ViT-Adapter on these three weights, respectively. These results demonstrate the superiority of our method in terms of flexibility, accuracy and efficiency.