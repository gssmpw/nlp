% -------------------------------
\subsection{Experiments on Semantic Segmentation (SSeg)}
\label{sec4:3}
% -------------------------------
\myparagraph{Baselines and Settings.} 
% Experiments on SSeg are conducted using the MMSegmentation framework~\citep{mmseg2020}. 
Following~\citep{chen2022vision,jie2023fact,jie2022convolutional}, we select Semantic FPN~\citep{kirillov2019panoptic} and UperNet~\citep{xiao2018unified} as baseline models, where the Semantic FPN is trained for $80$k iterations and the UperNet is trained for $160$k iterations as in~\citep{wang2021pyramid,liu2021swin}. 
%The input images are cropped to a fix size of 512 $\times$ 512 pixels as in~\citep{xiong2024efficient,chen2022vision}. The training batch size is set to $16$, and AdamW~\citep{loshchilov2017decoupled} is used as the optimizer with the initial learning rate of $1 \times 10^{-5}$ and the weight decay of $0.05$. Following~\citep{li2022exploring,liu2021swin}, the layer-wise learning rate decay is set to $0.9$ and the drop path rate is set to $0.4$. We report the experimental results on both single scale training and MS training strategies. 
Unless otherwise specified, the training and inference settings are set up to be consistent with the ViT-Adapter~\citep{chen2022vision}.
% -------------------------------
% \begin{table}[!th]
% \centering
% \resizebox{\columnwidth}{!}{
% \begin{tabular}{@{}lcccccc@{}}
% \toprule
% \textbf{Embedder} & \textbf{BANKING} & \textbf{CLINC} & \textbf{Reddit} & \textbf{MTOP} & \textbf{CLINC(D)} & \textbf{AVG} \\ \midrule \midrule
% Standard Prompting & 0.652 & 0.792 & 0.534 & 0.896 & 0.536 & 0.682 \\ \midrule \midrule
% all-MiniLM-L12-v2-120M & 0.706 & 0.832 & 0.584 & 0.928 & 0.550 & 0.720 \\
% all-distilroberta-v1-290M & 0.690 & 0.840 & 0.586 & 0.938 & 0.566 & 0.724 \\
% all-mpnet-base-v2-420M & 0.712 & 0.852 & 0.586 & 0.942 & 0.560 & 0.730 \\ \midrule \midrule
% gte-Qwen2-1.5B-instruct & 0.694 & 0.844 & 0.614 & 0.946 & 0.562 & 0.732 \\
% stella-en-400M-v5 & 0.728 & 0.834 & 0.618 & 0.946 & 0.564 & 0.738 \\
% NV-Embed-v2-7B & 0.764 & 0.864 & 0.646 & 0.948 & 0.554 & 0.755 \\
% \bottomrule
% \end{tabular}}
% \caption{Embedders.}
% \label{tab:embedder_comparison}
% \end{table}


\begin{table}[!t]
\centering
\resizebox{\columnwidth}{!}{
\begin{tabular}{@{}lccccc@{}}
\toprule
\textbf{Embedder} & \textbf{BANKING} & \textbf{CLINC} & \textbf{Reddit} & \textbf{MTOP} & \textbf{AVG} \\ \midrule \midrule
Standard Prompting & 0.652 & 0.792 & 0.534 & 0.896 & 0.719 \\ \midrule \midrule
all-MiniLM-L12-v2-120M & 0.706 & 0.832 & 0.584 & 0.928 & 0.763 \\
all-distilroberta-v1-290M & 0.690 & 0.840 & 0.586 & 0.938 & 0.764 \\
all-mpnet-base-v2-420M & 0.712 & 0.852 & 0.586 & 0.942 & 0.773 \\ \midrule \midrule
gte-Qwen2-1.5B-instruct & 0.694 & 0.844 & 0.614 & 0.946 & 0.775 \\
stella-en-400M-v5 & 0.728 & 0.834 & 0.618 & 0.946 & 0.782 \\
NV-Embed-v2-7B & 0.764 & 0.864 & 0.646 & 0.948 & 0.806 \\
\bottomrule
\end{tabular}}
\caption{Results with varying embedding models. TestNUC is effective on diverse embedding models from different companies and of different sizes.}
\label{tab:embedder_comparison}
\end{table}

% -------------------------------

% -------------------------------
\myparagraph{Comparisons with state-of-the-art methods.} 
We show the SSeg experimental results of META under different settings and compare them with SOTA SSeg methods in Table~\ref{tab4}. We can observe that \emph{\textbf{\romannumeral1}}) Our method can consistently improve performance across different baselines, model scales, training strategies, and pre-training weights (including IN-1K, IN-22K, and MM), while having fewer model parameters and memory consumption compared to the advanced ViT-Adapter~\citep{chen2022vision} and LoSA~\citep{mercea2024time} methods. This demonstrates the strong applicability and learning ability of META, which can solve the problem of limited applicability in certain scenarios of downstream models in the traditional pre-training and fine-tuning strategy at the application level. \emph{\textbf{\romannumeral2}}) Compared to SOTA methods, META also can achieve a new SOTA accuracy-cost trade-off. The parameter from META is only about $0.2\%$ of the overall model parameter count, indicating a minimal impact on the total parameter count. Furthermore, the memory consumption of our method accounts for only 55.14\% of the SOTA LoSA~\citep{mercea2024time}.
\emph{\textbf{\romannumeral3}}) On larger model scales, META achieves higher efficiency with fewer parameters (\eg, -$1.4$\textbf{M} on {META-T} and -$4.2$\textbf{M} on {META-B}). This indicates that our method is suitable for fine-tuning large ViT models.