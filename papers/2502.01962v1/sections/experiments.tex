% -------------------------------
\section{Experiments}
% -------------------------------
\subsection{Datasets and Evaluation Metrics}
\label{sec4:1}
% -------------------------------
\myparagraph{Datasets.} To facilitate a fair result comparison with existing methods, we conduct experiments, including the ablation analysis, on two commonly used datasets: MS-COCO~\citep{caesar2018coco} for ODet and ISeg, and ADE20K~\citep{zhou2017scene} for SSeg. Due to page limitations, the implementation details of these datasets will be given in the supplementary materials. 
% Below are the details of the used datasets:

% -------------------------------
% \begin{itemize}
% -------------------------------
% \item MS-COCO~\citep{caesar2018coco} is a representative yet challenging dataset for common scene IS and object detection, which consists of $118$k, $5$k and $20$k images for the \emph{training} set, the \emph{val} set and the \emph{test} set, respectively. In our experiments, the model is trained on the \emph{training} set and evaluated on the \emph{val} set.
% -------------------------------
% \item ADE20K~\citep{zhou2017scene} is a scene parsing dataset with $20$k images and $150$ object categories. Each image has pixel-level annotations for SS of objects and regions within the scene. The dataset is divided into $20$k, $2$k, and $3$k images for \emph{training}, \emph{val} and \emph{test}, respectively. Our model is trained on the \emph{training} set and evaluated on the \emph{val} set.
% -------------------------------
% \end{itemize}
% -------------------------------
% For data augmentation, random horizontal flip, brightness jittering and random scaling within the range of $[0.5, 2]$ are used in training as in~\citep{chen2022vision,luo2023forgery,zhang2023cae}. By default, the inference results are obtained at a single scale, unless explicitly specified otherwise.    

\myparagraph{Evaluation metrics.} The commonly adopted average precision (AP) and mean intersection-over-union (mIoU) are used to assess the model accuracy for ODet (AP$^\textrm{b}$)/ISeg (AP$^\textrm{m}$) and SSeg, respectively. Besides, to evaluate the efficiency, model Parameters (\#P), floating point operations (FLOPs), memory consumption (MC) of the adapter, and frames per second (FPS) are also adopted. The reported inference results are measured by A100 GPUs with per-GPU batch size 2.
%
% \emph{We acknowledge that using memory access cost can provide an intuitive reflection of a model's memory efficiency. However, since this metric is seldom reported in publicly available papers, we refrain from comparing this metric to ensure a fair comparison. Instead, we use FPS as the evaluation metric (Ref Sec.~\ref{sec4:4}), which has a strong correlation with the model's memory access cost}~\citep{liu2023efficientvit,dao2022flashattention,gu2021towards}\footnote{Due to page limitations, we only provide the key quantitative results. Qualitative results and more ablation studies will be presented in the supplementary material.}.  
% -------------------------------
% -------------------------------
\subsection{Experiments on Object Detection (ODet) and Instance Segmentation (ISeg)}
\label{sec4:2}
% -------------------------------
\myparagraph{Baselines and settings.} As in~\citep{chen2022vision,xiong2024efficient,jie2022convolutional,marouf2024mini}, Mask R-CNN~\citep{he2017mask}, Cascade Mask R-CNN~\citep{cai2019cascade}, ATSS~\citep{zhang2020bridging}, and GFL~\citep{li2020generalized} are employed as the baseline models, where the pre-trained ViT~\citep{li2022exploring} is used as the backbone. All the baseline models are pre-trained on ImageNet-1k by default~\citep{deng2009imagenet}. Unless otherwise specified, these baselines are set up to be consistent with their papers and the settings of the ViT-Adapter~\citep{chen2022vision} method.
% -------------------------------
\begin{table*}
\centering
\renewcommand{\dashlinedash}{0.5pt} 
\renewcommand{\dashlinegap}{2pt}    
\begin{tabular}{lllll}
\hline\hline
 & \shortstack[l]{\space \\ \space \\ \textbf{Methods} \\ \space } & 
   \shortstack[l]{\space \\ \space \\ \textbf{Venue} \\ \space } & 
   \shortstack[l]{\space \\ \space \\ \textbf{Scheme} \\ \space } & 
   \shortstack[l]{\space \\ \space \\ \textbf{Cross-stage fusion} \\ \space} \\
\bottomrule   \\
Single-stage
& Zhang et al.~\cite{zhang2018single} & CVPR 2018 & $I \rightarrow [T, R]$ & - \\ 
& ERRNet~\cite{wei2019single} & CVPR 2019 & $I \rightarrow T$ & - \\ 
& RobustSIRR~\cite{song2023robust} & CVPR 2023 & $I_{multiscale} \rightarrow T$ & - \\
& YTMT~\cite{hu2021trash} & NeurIPS 2021 & $I \rightarrow [T, R]$ & - \\   \bottomrule   \\
Two-stage
& CoRRN~\cite{wan2019corrn} & TPAMI 2019 & \shortstack[l]{\space \\$I \rightarrow E_{T}$ \\ $[I, E_{T}] \rightarrow T$} & Convolutional Fusion \\ \cline{2-5}
& DMGN~\cite{feng2021deep} & TIP 2021 & \shortstack[l]{\space \\ $I \rightarrow [T_{1}, R]$ \\ $[I, T_{1}, R] \rightarrow T$} & Convolutional Fusion \\ \cline{2-5}
& RAGNet~\cite{li2023two} & Appl. Intell. 2023 & \shortstack[l]{\space \\ $I \rightarrow R$ \\ $ [I, R] \rightarrow T$} & Convolutional Fusion \\ \cline{2-5}
& CEILNet~\cite{fan2017generic} & ICCV 2017 & \shortstack[l]{\space \\ $[I, E_{I}] \rightarrow E_{T}$ \\ $[I, E_{T}] \rightarrow T$ } & Concat \\ \cline{2-5}
& DSRNet~\cite{hu2023single} & ICCV 2023 & \shortstack[l]{\space \\ $I \rightarrow (T_{1}, R_{1})$ \\ $(R_{1}, T_{1}) \rightarrow (R, T, residue)$} & N/A \\ \cline{2-5}
& SP-net BT-net~\cite{kim2020single} & CVPR 2020 & \shortstack[l]{\space \\ $I \rightarrow [T_{1}, R_{1}]$ \\ $R_{1} \rightarrow R$} & N/A \\ \cline{2-5}
& Wan et al.~\cite{wan2020reflection} & CVPR 2020  & \shortstack[l]{\space \\ $[I, E_{I}] \rightarrow R_{1}$ \\ $R_{1} \rightarrow R$} & N/A \\ \cline{2-5}
& Zheng et al.~\cite{zheng2021single} & CVPR 2021 & \shortstack[l]{\space \\ $I \rightarrow e$ \\ $[I, e] \rightarrow T$} & Concat \\ \cline{2-5}
& Zhu et al.~\cite{zhu2024revisiting} & CVPR 2024 & \shortstack[l]{\space \\ $I \rightarrow E_{R}$ \\ $ [I, E_{R}] \rightarrow T$} & Concat \\ \cline{2-5}
& Language-Guided~\cite{zhong2024language} & CVPR 2024 & \shortstack[l]{\space \\ $[I, Texts] \rightarrow R\ or\ T$ \\ $[I, R\ or\ T] \rightarrow T\ or\ R$} & Feature-Level Concat
\\   \bottomrule   \\
Multi-stage
& BDN~\cite{yang2018seeing} & ECCV 2018 & \shortstack[l]{\space \\ $I \rightarrow T_{1}$ \\ $[I, T_{1}] \rightarrow R$ \\ $[I, R] \rightarrow T$ } & Concat \\ \cline{2-5}
& IBCLN~\cite{li2020single} & CVPR 2020 & \shortstack[l]{\space \\ $[I, R_{0}, T_{0}] \rightarrow [R_{1}, T_{1}]$ \\ $[I, R_{1}, T_{1}] \rightarrow [R_{2}, T_{2}]$ \\ \ldots} & \shortstack[l]{Concat \\ Recurrent} \\ \cline{2-5}
& Chang et al.~\cite{chang2021single} & WACV 2021 & \shortstack[l]{\space \\ $I \rightarrow E_{T}$ \\ $[I, E_{T}] \rightarrow T_{1} \rightarrow R_{1} \rightarrow T_{2}$ \\ $[I, E_{T}, T_{2}] \rightarrow R \rightarrow T$} & \shortstack[l]{Concat \\ Recurrent} \\ \cline{2-5}
& LANet~\cite{dong2021location} & ICCV 2021 & \shortstack[l]{\space \\ $[I, T_{0}] \rightarrow R_{1} \rightarrow T_{1}$ \\ $[I, T_{1}] \rightarrow R_{2} \rightarrow T_{2}$ \\ \ldots} & \shortstack[l]{Concat \\ Recurrent} \\ \cline{2-5}
& V-DESIRR~\cite{prasad2021v} & ICCV 2021 & \shortstack[l]{\space \\ $I_{1} \rightarrow T_{1}$ \\ $[I_{1}, T_{1}, I_{2}] \rightarrow T_{2}$ \\ \ldots \\ $[I_{n-1}, T_{n-1}, I_{n}] \rightarrow T$ } & \shortstack[l]{Convolutional Fusion \\ Recurrent} \\
\hline\hline
\end{tabular}
\caption{\textbf{I}, \textbf{R}, \textbf{T}, and \textbf{E} represent the \textbf{I}nput, \textbf{R}eflection, \textbf{T}ransmission, and \textbf{E}dge map, respectively. The subscripts of \textbf{T} and \textbf{R} represent intermediate process outputs. The Absorption Effect $e$ is introduced in~\cite{zheng2021single} to describe light attenuation as it passes through the glass. The output $residue$ term, proposed in~\cite{hu2023single}, is used to correct errors in the additive reconstruction of the reflection and transmission layers. Language descriptions in~\cite{zhong2024language} provide contextual information about the image layers, assisting in addressing the ill-posed nature of the reflection separation problem.}
\label{tab:1}
\end{table*}

% -------------------------------

% -------------------------------
\myparagraph{Comparisons with state-of-the-art (SOTA) methods.} Result comparisons with SOTA methods with Mask R-CNN~\citep{he2017mask} for ODet and ISeg are shown in Table~\ref{tab1}. From this table, we can obtain the following observations and conclusions: 
%
\emph{\textbf{\romannumeral1}})~Compared to the experimental results of the Mask R-CNN~\citep{he2017mask} model with ViT~\citep{li2021benchmarking}, our proposed META can consistently improve accuracy for ODet and ISeg across different model scales (\eg, ViT-T/S/B/L~\citep{li2021benchmarking}), while only adding a small number of training parameters. Even with different training schedules (\ie, 1$\times$, and 3$\times$ with MS), our method can also improve the model performance, demonstrating the plug-and-play advantage of META. 
%
\emph{\textbf{\romannumeral2}})~Compared to the SOTA ViT-Adapter~\citep{chen2022vision} and LoSA~\citep{mercea2024time}, META can achieve a new accuracy-efficiency trade-off. For example, in settings with the strong ViT-B~\citep{li2021benchmarking} as the backbone, our method achieves a performance gain of $0.5\%$AP$^\textrm{m}$ under 1$\times$ training schedule and $0.7\%$AP$^\textrm{m}$ under 3$\times$ training schedule with MS while reducing 4.9\textbf{M} model parameters, when compared to ViT-Adapter~\citep{chen2022vision}. 
%
\emph{\textbf{\romannumeral3}})~Even with stronger pre-trained models, META still improves performance on the baseline models and surpasses existing methods on accuracy, parameters and memory. For example, on the ImageNet-22k pre-trained weights from~\citep{steiner2021train}, our method achieves a performance gain of $0.8\%$/$0.7\%$AP$^\textrm{m}$ under 1$\times$/3$\times$ training schedule while reducing 8.2\textbf{M} parameters compared to ViT-Adapter~\citep{chen2022vision}, which validates its strong learning ability and flexible adaptability.
%
\emph{\textbf{\romannumeral4}})~Compared to SOTA ODet and ISeg methods, META also has very competitive performance. For example, compared to the strong ViTDet-B~\citep{li2022exploring} model, META-T has $2.2\%$AP$^\textrm{b}$ and $3.1\%$AP$^\textrm{m}$ gains under the 1$\times$ training schedule, and has META-T has $4.9\%$AP$^\textrm{b}$ and $2.7\%$AP$^\textrm{m}$ gains under the 3$\times$ training schedule.
%
\emph{\textbf{\romannumeral5}})~Compared to SOTA ViT adapter methods such as AdaptFormer~\citep{chen2022adaptformer}, FacT-TK~\citep{jie2023fact}, and LoSA~\citep{mercea2024time}, our method exhibits superior efficiency in terms of reduced parameter count, decreased FLOPs, and lower MC. Particularly, META only utilizes $62\%$ of the MC of LoSA while achieving superior prediction accuracy. Therefore, the experimental results and conclusions above demonstrate that our method achieves better accuracy and higher efficiency in dense prediction tasks.
% -------------------------------
\begin{table*}[t]
  \centering
  
\begin{subtable}[h]{0.62\textwidth}
\vspace{-3mm}

\resizebox{1\columnwidth}{!}{
\begin{tabular}{llllll}
\toprule
\textbf{Methods} & Photo & Art & Cartoon & Sketch & \textit{Mean} \\
 \midrule
ERM & 41.2  &   40.9 & 53.7 &46.2  &45.5  \\
Wang \etal \cite{wang2021tent} & 42.0  &  41.6 & 56.5 & 53.8 & 48.5 \\
Zhou \etal \cite{zhou2021domain} & 56.3 & 44.5 & 55.8 & 46.7 & 50.8 \\
Li \etal \cite{li2022uncertainty} & \textbf{60.2} & 45.0 & 54.4 & 49.2 & 52.2\\
Zhang \etal \cite{zhang2022exact} & 57.9 & \textbf{46.0} & 55.3 & 50.0 & 52.3 \\
 \rowcolor{lightorange}
{\textit{\textbf{This paper}}} & 50.0 \scriptsize{$\pm$0.2} & 45.7 \scriptsize{$\pm$0.3} & \textbf{61.2} \scriptsize{$\pm$0.28} & \textbf{55.5} \scriptsize{$\pm$0.5} & \textbf{53.1} \scriptsize{$\pm$0.31} \\
\bottomrule
\end{tabular}}
\caption{Output-level shift.}
\label{table:dist_shift}
\end{subtable}
\begin{subtable}[h]{0.293\textwidth}
\vspace{-3mm}

\resizebox{1\columnwidth}{!}{
\begin{tabular}{ll}
\toprule
\textbf{Methods} & \textbf{Accuracy} \\ 
\midrule
ERM & 69.3\\
Wang \etal \cite{wang2021tent} & 68.8 \\
Long \etal \cite{long2018conditional} & 70.7 \\
Ganin \etal \cite{ganin2016domain} & 72.1 \\
\rowcolor{lightorange}
\textit{\textbf{{\textit{\textbf{This paper}}}}} & \textbf{73.7}  \scriptsize{$\pm$0.7} \\ 
\bottomrule
\end{tabular}}
\caption{Feature-level shift}
\label{tab:feautre_level}
\end{subtable}
\vspace{-1em}
\caption{\textbf{Comparisons on output-level and feature-level shifts} for ResNet-18 on PACS. Our method achieves the best overall performance.}
\vspace{-4mm}
\end{table*}

% -------------------------------


% -------------------------------
\myparagraph{Superiority performance under different baselines.} 
% -------------------------------
In addition to Mask R-CNN, we also choose Cascade Mask R-CNN~\citep{cai2019cascade}, ATSS~\citep{zhang2020bridging}, and GFL~\citep{li2020generalized} as the baselines as in~\cite{chen2022vision,mercea2024time}. We explore the effectiveness and superiority performance of META on these baseline models, where the 3$\times$ training with MS strategy is used. 
% 
The experimental results are given in Table~\ref{tab2}. We can observe that META can consistently improve performance across different baselines, and exhibits more accurate and more efficient advantages compared to SOTA ViT adapter methods. 
%
For example, based on the Cascade Mask R-CNN~\citep{cai2019cascade}, META-S can achieve up to $44.8\%$ AP$^\textrm{m}$ with only $83$\textbf{M} parameters, $797$\textbf{G} FLOPs, and $8.1$\textbf{GB} MC, which is $1.3\%$ AP$^\textrm{m}$ higher, $3$\textbf{M} fewer parameters, $4$\textbf{G} fewer FLOPs, and $7.1$\textbf{GB} fewer MC compared to the competitive ViT-Adapter-T method~\citep{chen2022vision}. 
%
Besides, META-B can achieve a $1.7\%$ AP$^\textrm{b}$/$0.5\%$ AP$^\textrm{m}$ gain on the 3$\times$ training schedule, with $5$\textbf{M} fewer parameters and $5$\textbf{G} fewer FLOPs than the ViT-Adapter-B~\citep{chen2022vision} model.
%
On ATSS and GFL, our method also achieves competitive $43.2\%$ and $44.$0\% AP$^\textrm{m}$, respectively, which are $0.7\%$ and $0.9\%$ AP$^\textrm{m}$ higher than the competitive vision adapter ViT-Adapter-S~\citep{chen2022vision}. 
%
In terms of efficiency, compare with the SOTA LoSA~\citep{mercea2024time} method,  our method has fewer parameters ($33$\textbf{M} v.s. $35$\textbf{M}), fewer FLOPs ($265$\textbf{G} v.s. $268$\textbf{G} under ATSS, $279$\textbf{G} v.s. $284$\textbf{G} under GFL), and fewer MC ($8.1$\textbf{GB} v.s. $13.0$\textbf{GB} under ATSS and GFL), indicating its higher computation and memory efficiency.
%  \begin{table*}
\small
\centering
\begin{tabular}{lccccccccc}
\toprule
\multicolumn{1}{l}{\multirow{2}{*}{Model}} & \multicolumn{4}{c}{Anomaly Detection} & \multicolumn{3}{c}{Low-level Reasoning} &  \multicolumn{2}{c}{Complex Reasoning}\\ 
\cmidrule(lr){2-5}\cmidrule(lr){6-8}\cmidrule(lr){9-10}
                 & Accuracy           & Precision & Recall & F1-score & ROUGE-L               & SBERT & GPT-Score & SBERT                & GPT-Score \\
\hline
GPT-4V \cite{gpt-api-4vision}          & 0.68              & 0.90      & 0.49   & 0.55     &        0.16             &   0.65    &      3.31     &  0.77                    &     5.64      \\
GPT-4o \cite{gpt-api-4o}          & 0.70              & 0.83      & 0.71   & 0.68     &           0.24          &    0.71   &    \textbf{4.84}       &  0.81                    &      \textbf{6.89}     \\
Qwen2-VL-2B \cite{qwen2vl}      & 0.65              & 0.87      & 0.55   & 0.59     & 0.22                    &   0.55    &  1.94         & 0.74                     &   4.26        \\
Qwen2-VL-7B \cite{qwen2vl}     & 0.76              & \underline{0.91}      & 0.69   & 0.75     & 0.25                    &  0.61     &  3.09         & 0.68                     &  4.62         \\
InternVL-2-8B \cite{internvl}   & 0.74              & 0.78      & 0.81   & 0.76     & 0.23                    & 0.73      &  3.69         & 0.80                     & 5.08          \\
InternVL-2-26B \cite{internvl}  & 0.73              & 0.86      & 0.66   & 0.68     & 0.21                    & \textbf{0.74}      &  4.13         & 0.80                     &   5.49        \\
IXC-2.5-7B \cite{ixc25}      & 0.72              & 0.88      & 0.63   & 0.67     & 0.21                    &  0.58     &  2.45         & 0.77                     &   5.14        \\
LLaVA-OV-0.5B \cite{llavaonevision}   & 0.54              & 0.70      & 0.19   & 0.28     & 0.20                    & 0.63      &  2.54         & 0.81                     &   4.34        \\
LLaVA-OV-7B \cite{llavaonevision}     & 0.71              & \textbf{0.95}      & 0.56   & 0.63     & 0.24                    & 0.66      &  3.57         &   0.79                   &  5.44         \\
\hline
LLaVA-OV-0.5B*   & 0.71              & 0.77      & \underline{0.84}   & 0.76     & 0.31                    &  0.70     & 3.69          & 0.82                     &  5.31         \\
Anomaly-OV-0.5B & \textbf{0.79}              & 0.86      & 0.83   & \underline{0.82}     & \underline{0.33}                    & 0.72      & 3.87          &  \underline{0.83}                    &  5.67         \\
Anomaly-OV-7B    & \textbf{0.79}                  & 0.83          & \textbf{0.86}      & \textbf{0.83}         &  \textbf{0.34}                   & \underline{0.73}      & \underline{4.26}          & \textbf{0.84}                     &  \underline{6.34}        \\
\bottomrule
\end{tabular}
\caption{Quantitative comparison of text-based anomaly detection and reasoning for MLLMs. Notably, the Accuracy and F1-score for the anomaly expert of Anomaly-OV can be calculated as $\{0.78, 0.77\}$ with threshold $0.5$. * indicates the model is fine-tuned on our dataset.}
\label{Tab:3}
 \vspace{-3mm}
\end{table*}

% -------------------------------
% \myparagraph{Results on different pre-trained weights.} 
% -------------------------------
% In Table~\ref{tab3}, we present the experimental results of META with different pre-trained weights and compare them with other SOTA methods including SwinViT~\citep{liu2021swin} and ViT-Adapter~\citep{chen2022vision} as in~\citep{chen2022vision}. Mask R-CNN~\citep{he2017mask} is used as the baseline, and ViT-B~\citep{li2022exploring} is used as the backbone. The 3$\times$ training schedule with MS training strategy is used. From this table, we can observe that our method is applicable to different pre-trained weights (\ie, ImageNet-1k~\citep{deng2009imagenet}, ImageNet-22k~\citep{steiner2021train}, and Multi-Modal~\citep{zhu2022uni}), and achieves more accurate AP with fewer model parameters and FLOPs compared to ViT-Adapter~\citep{chen2022vision}, across different pre-trained weights. Our method achieves $0.7\%$, $0.6\%$, and $0.6\%$ higher AP than ViT-Adapter on these three weights, respectively. These results demonstrate the superiority of our method in terms of flexibility, accuracy and efficiency.
% -------------------------------
% -------------------------------
\subsection{Experiments on Semantic Segmentation (SSeg)}
\label{sec4:3}
% -------------------------------
\myparagraph{Baselines and Settings.} 
% Experiments on SSeg are conducted using the MMSegmentation framework~\citep{mmseg2020}. 
Following~\citep{chen2022vision,jie2023fact,jie2022convolutional}, we select Semantic FPN~\citep{kirillov2019panoptic} and UperNet~\citep{xiao2018unified} as baseline models, where the Semantic FPN is trained for $80$k iterations and the UperNet is trained for $160$k iterations as in~\citep{wang2021pyramid,liu2021swin}. 
%The input images are cropped to a fix size of 512 $\times$ 512 pixels as in~\citep{xiong2024efficient,chen2022vision}. The training batch size is set to $16$, and AdamW~\citep{loshchilov2017decoupled} is used as the optimizer with the initial learning rate of $1 \times 10^{-5}$ and the weight decay of $0.05$. Following~\citep{li2022exploring,liu2021swin}, the layer-wise learning rate decay is set to $0.9$ and the drop path rate is set to $0.4$. We report the experimental results on both single scale training and MS training strategies. 
Unless otherwise specified, the training and inference settings are set up to be consistent with the ViT-Adapter~\citep{chen2022vision}.
% -------------------------------
% \begin{table}[!th]
% \centering
% \resizebox{\columnwidth}{!}{
% \begin{tabular}{@{}lcccccc@{}}
% \toprule
% \textbf{Embedder} & \textbf{BANKING} & \textbf{CLINC} & \textbf{Reddit} & \textbf{MTOP} & \textbf{CLINC(D)} & \textbf{AVG} \\ \midrule \midrule
% Standard Prompting & 0.652 & 0.792 & 0.534 & 0.896 & 0.536 & 0.682 \\ \midrule \midrule
% all-MiniLM-L12-v2-120M & 0.706 & 0.832 & 0.584 & 0.928 & 0.550 & 0.720 \\
% all-distilroberta-v1-290M & 0.690 & 0.840 & 0.586 & 0.938 & 0.566 & 0.724 \\
% all-mpnet-base-v2-420M & 0.712 & 0.852 & 0.586 & 0.942 & 0.560 & 0.730 \\ \midrule \midrule
% gte-Qwen2-1.5B-instruct & 0.694 & 0.844 & 0.614 & 0.946 & 0.562 & 0.732 \\
% stella-en-400M-v5 & 0.728 & 0.834 & 0.618 & 0.946 & 0.564 & 0.738 \\
% NV-Embed-v2-7B & 0.764 & 0.864 & 0.646 & 0.948 & 0.554 & 0.755 \\
% \bottomrule
% \end{tabular}}
% \caption{Embedders.}
% \label{tab:embedder_comparison}
% \end{table}


\begin{table}[!t]
\centering
\resizebox{\columnwidth}{!}{
\begin{tabular}{@{}lccccc@{}}
\toprule
\textbf{Embedder} & \textbf{BANKING} & \textbf{CLINC} & \textbf{Reddit} & \textbf{MTOP} & \textbf{AVG} \\ \midrule \midrule
Standard Prompting & 0.652 & 0.792 & 0.534 & 0.896 & 0.719 \\ \midrule \midrule
all-MiniLM-L12-v2-120M & 0.706 & 0.832 & 0.584 & 0.928 & 0.763 \\
all-distilroberta-v1-290M & 0.690 & 0.840 & 0.586 & 0.938 & 0.764 \\
all-mpnet-base-v2-420M & 0.712 & 0.852 & 0.586 & 0.942 & 0.773 \\ \midrule \midrule
gte-Qwen2-1.5B-instruct & 0.694 & 0.844 & 0.614 & 0.946 & 0.775 \\
stella-en-400M-v5 & 0.728 & 0.834 & 0.618 & 0.946 & 0.782 \\
NV-Embed-v2-7B & 0.764 & 0.864 & 0.646 & 0.948 & 0.806 \\
\bottomrule
\end{tabular}}
\caption{Results with varying embedding models. TestNUC is effective on diverse embedding models from different companies and of different sizes.}
\label{tab:embedder_comparison}
\end{table}

% -------------------------------

% -------------------------------
\myparagraph{Comparisons with state-of-the-art methods.} 
We show the SSeg experimental results of META under different settings and compare them with SOTA SSeg methods in Table~\ref{tab4}. We can observe that \emph{\textbf{\romannumeral1}}) Our method can consistently improve performance across different baselines, model scales, training strategies, and pre-training weights (including IN-1K, IN-22K, and MM), while having fewer model parameters and memory consumption compared to the advanced ViT-Adapter~\citep{chen2022vision} and LoSA~\citep{mercea2024time} methods. This demonstrates the strong applicability and learning ability of META, which can solve the problem of limited applicability in certain scenarios of downstream models in the traditional pre-training and fine-tuning strategy at the application level. \emph{\textbf{\romannumeral2}}) Compared to SOTA methods, META also can achieve a new SOTA accuracy-cost trade-off. The parameter from META is only about $0.2\%$ of the overall model parameter count, indicating a minimal impact on the total parameter count. Furthermore, the memory consumption of our method accounts for only 55.14\% of the SOTA LoSA~\citep{mercea2024time}.
\emph{\textbf{\romannumeral3}}) On larger model scales, META achieves higher efficiency with fewer parameters (\eg, -$1.4$\textbf{M} on {META-T} and -$4.2$\textbf{M} on {META-B}). This indicates that our method is suitable for fine-tuning large ViT models.
% -------------------------------
\section{Ablation Study}
\label{sec:ablation}
\subsection{Public Dataset Size}
The server in PDA-FD can control communication overhead by adjusting the size of the public dataset used in each collaborative training round.
We investigate its impact on the performance of LDIA and MIA. 
As public data does not affect co-op LiRA, our evaluations of MIA mainly focus on distillation-based LiRA. 
In our experiments, we use the DS-FL framework on the CIFAR-10 dataset with $\alpha=1$.
\begin{table}[h]
    \caption{Impact of Public Data Quantity on Label Distribution and Membership Information Leakage in PDA-FD.}
    \centering
    \scriptsize
    \resizebox{0.9\linewidth}{!}{%
    \begin{tabular}{c|c|c}
        \toprule
        Datasets size&  MIA (TPR at 1\%FPR) & LDIA (KL divergence)\\
        \midrule
        5000 & 29.28\% &  0.10  \\
        7500 & 31.84\% &  0.09 \\
        10000& 32.01\% &  0.07 \\
        \bottomrule
    \end{tabular}%
    }
    \label{tab:public_dataset_size}
\end{table}
Table \ref{tab:public_dataset_size} illustrates the degree of label distribution information and membership information leakage from clients when the quantity of the public data samples is set to 5000, 7500, and 10000, respectively.
The results indicate that larger public datasets contribute to increased privacy leakage risks for clients. 
We attribute this trend to two factors. For distillation-based LiRA, a larger public dataset provides a more extensive distillation dataset, enabling the attacker to obtain more robust reference models.
In the case of LDIA, a larger public dataset serving as the inference dataset allows the attacker to mitigate the impact of outliers or atypical data, thereby improving attack accuracy.


\subsection{Number of Epochs in Local Updates Phase}
Prior to the communication phase, clients train their local models on their private datasets during the local updates phase. 
This process enhances the local model's memorization of private data, facilitating knowledge transfer between clients but also potentially increasing privacy leakage. 
We measure the impact of the number of training epochs in the local updates phase on the leakage of label information and membership information from clients.
\begin{table}[h]
    \caption{Impact of Number of Training Epochs on Label Distribution and Membership Information Leakage in PDA-FD.}
    \centering
    \scriptsize
    \resizebox{0.9\linewidth}{!}{%
    \begin{tabular}{c|c|c}
        \toprule
        Number of Epochs&  MIA (TPR at 1\%FPR) & LDIA (KL divergence)\\
        \midrule
        2 & 8.10\% &  0.15  \\
        4 & 14.64\% &  0.10 \\
        6 & 15.43\% &  0.09 \\
        \bottomrule
    \end{tabular}% 
    }
    \label{tab:epochs_local_updates}
\end{table}
As shown in Table \ref{tab:epochs_local_updates}, there is an increase in label distribution and membership information leakage from clients in DS-FL as the number of the local update training rounds increases from 2 to 6 on the CIFAR-10 dataset ($\alpha$=1).


\subsection{Number of Reference Models}
In LiRA, the attacker can form a more accurate Gaussian distribution by utilizing a larger number of reference models, thereby enhancing the precision of determining whether a target sample belongs to the target model's training data. 
We evaluate the performance of distillation-based LiRA with varying numbers of reference models. 
Figure \ref{fig:distillation_lira_num_models} shows results from experiments using the Cronus framework on CIFAR-10 with $\alpha=0.1$. 
The data reveals that the performance of the distillation-based LiRA's improves as the number of distilled reference models increases.
\begin{figure}[h]
    \centering
    \includegraphics[width=0.9\linewidth]{figures/mia_diff_models.png}
    \caption{The performance of distillation-based LiRA vs. number of the distilled reference model.}
    \label{fig:distillation_lira_num_models}
\end{figure}

\subsection{Resilience Against DP-SGD}
To evaluate the robustness of our proposed LDIA and MIA methods, we assess their effectiveness when the target client employs DP-SGD\cite{abadi2016deep} during the local updates phase. 
DP-SGD is a state-of-the-art privacy-preserving model training technique.
Our experimental setup includes 10 clients participating in DS-FL training on the CIFAR-10 dataset ($\alpha=10$). We conduct LDIA and Co-op LiRA attacks against the clients during the second round of training.
In DP-SGD, it introduces noise to gradients during training, governed by three key parameters. 
The clipping bound ($C$) limits the influence of individual data points on model parameters. 
The noise multiplier ($\sigma$) determines the amount of noise added to gradients. 
The privacy budget ($\varepsilon$) balances privacy guarantees and model utility, with smaller values providing stronger privacy at the cost of potentially noisier updates.
In our experiments, we set $C$ to be 10 and vary $\sigma$ to adjust $\varepsilon$. 
This setup allows us to evaluate our proposed attack under different privacy protection levels.
\begin{table}[h]
    \caption{Performance of MIA and LDIA against DP-SGD for DS-FL trained on CIFAR-10.}
    \scriptsize
    \resizebox{0.9\linewidth}{!}{%
    \centering
    \begin{tabular}{c|c|c|c|c}
    \toprule
         $\sigma$&$\varepsilon$&Average acc&LDIA(KL divergence)&MIA(TPR at 1\%FPR)\\
    \midrule
         0  &$\infty$ & 59.09\% & 0.03 & 15.76\% \\
         0.1& >10000  & 48.68\% & 0.07 & 2.86\% \\
         0.3& >5000   & 41.29\% & 0.08 & 2.11\% \\
         0.5& >2000   & 28.53\% & 0.09 & 1.54\% \\
         1.0& 231     & 21.34\% & 0.10 & 1.29\% \\
    \bottomrule
    \end{tabular}%
    }
    \label{tab:DP}
\end{table}



\subsection{Resilience Against Evasive Clients}
To proactively protect their privacy,  cautious clients may choose to, in each communication round, avoid sending to the server the logits of some samples in the public dataset, particularly the ones that are also in its training dataset. 
% , even though the FD protocol requests them. 
% Particularly, avoid the logits of the ones that 
To counter such defense, we propose two countermeasures as follows:
% (1) We can select the target sample as public data for the communication phase across multiple rounds.
(1) In co-op LiRA, a shadow target model can be distilled using the logits of the samples in the public dataset provided by the target model, and then obtain the logits of the target sample from this shadow target model as an approximation to the one from the target clients' model. 
The intuition is that although knowledge distillation reduces the distilled student model's membership information of the teacher model~\cite{jagielski2024students}, it still preserves statistically significant enough membership information for a percentage of the members in teacher's training data, thus allowing some success in the MIA attack to the teacher model.
(2) We can also leverage a technique called indirect queries~\cite{wen2022canary, long2020pragmatic}, which is to first obtain logits of samples in the target sample's neighborhood from the target model and subsequently perform MIA using information encoded in these neighborhood logits. Neighbor samples are generated by adding noises to the target sample.

We conduct experiments to evaluate the effectiveness, and the experiments are on the CIFAR-10 dataset with a Dirichlet distribution parameter $\alpha$=10, with co-op LiRA as the MIA method.
Equipped with the first countermeasure, the attack achieves a TPR of 4.53\% at 1\% FPR.
Implementing a simplified version of the second countermeasure gives the attack a TPR of 4.23\% at 1\% FPR.
Note that in implementing countermeasure two, we add random Gaussian noise to the target samples to generate neighbor samples, with the noise clipped to the [-0.7, 0.7] range.
Studies in~\cite{wen2022canary, long2020pragmatic} implement more advanced schemes to learn from the neighbor logits, leading to better attacks. We leave studying such schemes as future work.
