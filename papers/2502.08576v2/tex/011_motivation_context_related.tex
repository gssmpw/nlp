

In this section, we examine the increasing interest from both public and private stakeholders in using \gls{genai} to support \gls{nmm} processes (Sec.~\ref{subsec:genai_context}). Next, we discuss related surveys that analyze the impact of \gls{genai} methods in the networking domain \ref{subsec:related}. Finally, we outline the positioning and scope of this survey
(Sec.~\ref{subsec:positioning}).

\subsection{GenAI in Networking: Context}
\label{subsec:genai_context}
The huge and general interest in \gls{genai} solutions also maps to the networking domain, where recent initiatives reflect the endeavors of several private and public stakeholders.
Table~\ref{tab:stake} provides an overview of this interest, reporting the efforts of different stakeholders in the context of \gls{genai} for \gls{nmm}.

For instance, the current interest in \gls{genai} is witnessed by the recent establishment of IEEE ComSoc Emerging Technology Initiative on Large Generative \gls{ai} Models in Telecom (GenAINet)~\cite{genainet}.
%
ACM 
SIGCOMM
has already featured several online talks
in which experts have discussed the huge interest in the application of \gls{genai} to \gls{nmm} (and, in general, to networking)~\cite{networkingchannel}.
%
Similarly, the International Telecommunication Union (ITU) via its initiative ``AI for Good'' is showcasing both industry- and academic-oriented viewpoints, as well as first \gls{llm}-based challenges~\cite{aiforgood}.
Trending interest is also observed at the IETF, with a first side meeting
entirely dedicated to the use of \glspl{llm} in networking~\cite{ietf}.
It is further witnessed by the latest academic networking conferences and workshops that stably include the application of \gls{genai} among the topics of their call, consistently seeking contributions in this direction (\eg IEEE GLOBECOM 2024 will feature both dedicated workshops and symposia centered on \gls{genai}).%
\footnote{\url{https://globecom2024.ieee-globecom.org/call-papers}}

At the governmental level, the EU has launched a strategy for developing \gls{genai} models over the past two years, highlighted by the EIC Accelerator funding program under the Horizon Europe framework aimed at supporting start-ups and small-medium enterprises~\cite{eic}.
Specifically, one of the $2024$ challenges,
``Human Centric Generative \gls{ai} made in Europe'' ($50$ million EUR budget), aims to promote a European human-centric approach to \gls{genai}, addressing issues like transparency and trust, and seeking to 
($i$)~advance foundation language and multimodal frontier models, while also focusing on 
($ii$)~smaller foundation models with high performance in \emph{specific domains}---%
like the case of this paper.




















Network providers have also attempted to capitalize on the benefits of \gls{genai}.
For instance, Bell Labs acknowledges the benefits of \gls{genai}, 
classifies several use cases
(in areas such as customer care operations, network design, network performance and optimization, and testing), 
and envisions their expected role in shaping the future of organizations and functions of Telecom service providers~\cite{nokia}.
Huawei has recently launched \emph{Net Master}~\cite{huawei},
an innovative network large model
powered by \gls{genai} 
that aims to enhance the efficiency of network operations and maintenance.
This solution is trained using Huawei's Pangu models
(\ie different foundation models tailored to different domains or specific use cases) 
and it is based on a $50$-billion-level corpus and the experience of more than $10$k networking experts. 
According to Ericsson,
in the Telecom domain, the integration of \gls{genai} capability to convert natural language to SQL and to execute complex SQL queries enables seamless interaction between users and data,
replacing the traditional search process with a more intuitive and conversational experience~\cite{ericsson}.
This capability allows Telecom companies to empower users to effortlessly access and analyze data, easily supporting data-driven decisions.
Telefonica has partnered with Microsoft to integrate Azure 
\gls{genai} in its digital ecosystem, enhancing its capabilities for key workflows, 
such as customer identity management or access to network \glspl{api}~\cite{telefonica}.
Similarly, AT\&T has launched \emph{Ask AT\&T} a \gls{genai} tool based on OpenAI’s ChatGPT, integrated within a secure AT\&T-dedicated Azure environment~\cite{att}.
This tool aims to enhance employees' productivity by translating documents, optimizing network operations, updating legacy software, and improving customer support. 
On the same line,
Cisco proposed its \gls{ai} Assistant for accessing data at large scale 
to guide and inform human decision-making and enhance productivity while guaranteeing data protection and privacy~\cite{cisco}.
Lastly, TIM
is exploring the integration of \gls{genai} across various sectors (\eg marketing, customer care, network operations).
%
The aim is to support customer service and technical operations through conversational interfaces,
enhance document search and summarization, assist in code generation for IT tasks, 
and improve data analysis with natural language queries~\cite{tim}.
%











\subsection{Related Surveys and Overviews on GenAI in the Networking/Telco Domain}
\label{subsec:related}


Given the enormous hype 
surrounding \gls{genai} techniques, 
a large number of recent surveys and tutorial-style studies aim to analyze and discuss their impact within the \emph{wide domain of networking}.
These works contribute to defining a rich but equally fragmented picture.
%
Indeed, the available studies are characterized by different focuses, scopes, and depths in the provided pictures of the state of the art. 
Thus, they result in identifying 
($i$)~different vertical application fields, use cases, and networking tasks that can benefit from the (rapid) progress in \gls{genai}, 
as well as ($ii$)~different families of \gls{ai} tools.
Such studies and related aspects are summarized in Tab.~\ref{tab:other_surveys} and briefly discussed in the following.
%

The majority of the works aim to analyze the role of \gls{genai} in the fields of the \textbf{\gls{iot}} and/or \textbf{cybersecurity}~\cite{sai2024empowering, ferrag2024generative, hassanin2024comprehensive, halvorsen2024applying, alwahedi2024ml}.
For instance, \citet{sai2024empowering}
explore the potential of combining \gls{genai} with \emph{\gls{iot}}, which enables the generation of synthetic data that can be used to train \gls{dl} models to overcome data insufficiency or incompleteness in \gls{iot} systems.
\citet{ferrag2024generative} provide a comprehensive survey of \glspl{llm} for \emph{cybersecurity} identifying $9$ application fields:
threat detection and analysis, 
phishing detection and response, 
incident response, 
security automation, 
cyber forensics, 
chatbots, 
penetration testing, 
security protocol verification, and 
security training and awareness.
Although the authors offer an in-depth analysis of the potentiality of \glspl{llm} for cybersecurity,
the potential application fields they identify
are not fully centered on networking
and do not consider several promising \gls{llm} applications in this domain.
\citet{hassanin2024comprehensive}
overview the recent progress of \glspl{llm} in \emph{cyber defense},
considering verticals that include threat intelligence, vulnerability assessment, network security, privacy preservation, and operations automation.
Moreover, \citet{halvorsen2024applying} explore the application of \gls{genai} for \emph{intrusion detection} and
discuss how \gls{genai} can support 
penetration testing, supplementing datasets, or developing 
detection models.
They claim that both the training and test phases of intrusion systems benefit from \gls{genai}.
\citet{alwahedi2024ml} aim at providing a comprehensive overview of applying \gls{ml} techniques for \emph{\gls{iot} security}.
In their future vision, the authors 
introduce the contribution of
\gls{genai} and \glspl{llm} to enhance \gls{iot} security---\eg optimization of cyber threat detection, lightweight encryption, access control, vulnerability identification, and automated penetration testing.
\emph{Unfortunately, we underline that the \gls{genai} applications and use cases discussed in the above surveys usually are not corroborated by existing state-of-the-art works.}


To the best of our knowledge, only a limited number of works~\cite{huang2023large, celik2024dawn, liu2024large, zhou2024large} aim at providing a \textbf{broader perspective} of \gls{genai} in the \textbf{networking/telecommunication field}.
In detail,
\citet{huang2023large} 
propose \emph{ChatNet}, a domain-adapted network \gls{llm} framework with access to various external network tools.
The authors discuss how \glspl{llm} promise to unify network intelligence through \emph{natural language interfaces}. 
Specifically, they remark that domain adaptation of \glspl{llm} is paramount to fill the gap between natural language and network language
and identify pre-training, fine-tuning, inference, and prompt engineering as the main enabling techniques.
\citet{liu2024large}
provide a more condensed overview of the recent advances of \glspl{llm} in networking and 
present an abstract workflow to describe the fundamental process involved in applying \gls{llm} in such a domain, including
task definition, data representation, prompt engineering, model evolution, tool integration, and validation.
Interestingly, they 
remark that \emph{network-specific \glspl{llm}} are expected to be more effective
than using \glspl{llm} originally designed for general domains to perform network-related tasks.
The works in~\cite{huang2023large} and~\cite{liu2024large} both identify
network design, 
diagnosis,
configuration, and
security
as the main vertical fields in networking  
impacted by \glspl{llm}.
On the other hand,
\citet{zhou2024large}
survey fundamentals, key techniques, and applications of \gls{llm}-enabled telecommunication networks.
Specifically, they focus on four telecommunication scenarios:
\begin{enumerate*}[label=(\emph{\roman*})]
    \item generation problems, \ie answering telecommunication-domain questions and generating troubleshooting reports, project coding, and network configuration;
    \item classification problems, \ie network-attack, telecommunication-text, image, and traffic classification;
    \item network-performance optimization, \ie automated reward function design to improve reinforcement learning applications; and 
    \item prediction problems, \ie prediction of channel state information and traffic load, and prediction-based beamforming.
\end{enumerate*}
\citet{karapantelakis2024generative} 
focus on \gls{genai} for \emph{mobile telecommunication networks} 
and consider applications lying in verticals, 
such as optimizations in \glspl{ran}, network management, and requirements engineering.
From the perspective of 
telco operations, 
\citet{chaccour2024telecom} 
identify \gls{genai} as a key to improving network operations such as predictive maintenance and real-time optimization.
They discuss use cases of \glspl{llm} and \gls{genai} for telco,
including: 
($i$) customer incident and trouble report management, proactive network management and repair, digital twin for network management, and intelligent network alert correlation---associated with \glspl{llm}; 
($ii$) generating customized network configurations, creating dynamic service descriptions, and proactive fault prediction and resolution---associated with \gls{genai} solutions beyond \glspl{llm}, \ie those performing content creation.
Finally, \citet{celik2024dawn} focus on applying \gls{genai} models within the domain of \emph{wireless communications}. The authors provide a tutorial on \gls{genai} models and a survey on their application across various wireless research areas, including: 
\begin{enumerate*}[label=(\emph{\roman*})]
    \item physical layer design,
    \item network organization and management,
    \item network traffic analytics,
    \item cross-layer network security, and
    \item localization and positioning.
\end{enumerate*}
Specifically, in the domain of network traffic analytics, the authors focus on use cases such as network traffic generation, encrypted traffic classification, traffic prediction, and traffic morphing. 
In contrast, the exploration of \gls{genai} networks security models is limited, with only a small portion addressing the enhancement of \glspl{nids}, particularly in terms of improving their robustness.
\emph{As a final remark, we note that all the works surveyed in the areas of
networking and telecommunications
mainly utilize \glspl{gan}, hence they only marginally cover the latest advancements involving more sophisticated techniques such as \glspl{llm}, Diffusion Models, and \glspl{ssm}.}

For the sake of completeness, we mention that some works~\cite{xu2024unleashing, wang2024toward} deepen \emph{how the network is expected to support \gls{genai} applications}, \eg with focus on cloud-edge-mobile infrastructure and security \& privacy concerns.
We do not consider such research paths in our study but rather consider the opposite point of view, investigating \emph{how \gls{genai} can support network-related tasks}.

\subsection{Positioning and Survey Scope}
\label{subsec:positioning}

In light of the rich but scattered literature scenario,
we position the present work against the existing
surveys and overviews in terms of the scope of the applications and tools considered,
as well as the provided outcomes of the analyses.

To the best of our knowledge, none of the considered studies surveying the impact of recent advancements in \gls{genai} primarily focuses on network monitoring and management.
In fact, the studies that are primarily centered on networking~\cite{huang2023large, huang2024digital, liu2024large} share a focus that is slightly close to ours.
However, while envisioning the great potential of \gls{genai} in networking, they lack a detailed survey and taxonomization of the current landscape, being aimed at \emph{providing only a general overview based on the analysis of a very limited number of works} (indeed, these studies reference $15$ papers each in their bibliography).
On the other hand, the studies that provide a more systematic and in-depth analysis of the literature~\cite{zhou2024large, karapantelakis2024generative} emphasize different facets of the communication networks, being oriented at capturing telecommunication aspects placed at lower layers in the communication stack---\eg \gls{ran} improvement, mobile-network management, channel state information prediction, prediction-based beamforming.
Hence, we believe \emph{they provide a view that is complementary to ours}.

In this survey,
we explore $5$ use cases:
($i$) \emph{network traffic generation}, ($ii$) \emph{network traffic classification}, ($iii$) \emph{network intrusion detection}, ($iv$) \emph{networked system log analysis}, and ($v$) \emph{network digital assistance}, which are crucial for network monitoring and management and are mostly overlooked in other such surveys.




Unlike all the related surveys, we perform an in-depth analysis of each mentioned use case aimed at identifying and providing taxonomies of the solutions proposed in the networking domain.
Specifically, we report for each task the adopted \gls{genai} architecture, its public availability, the input fed to the model, and the dataset leveraged for its pre-train or fine-tuning. 
Indeed, our study is intended for researchers and practitioners interested in capitalizing on the benefits of \gls{genai} for network monitoring and management.
Hence, we place a strong emphasis on the reproducibility of the proposals.
Therefore, we also contribute to the taxonomization of the models used for each networking application we identify.
While centered on the impact of the latest \gls{llm} wave,
our study does not simply focus on \gls{llm}-based generative solutions---such as the majority of similar surveys~\cite{hassanin2024comprehensive, zhou2024large, huang2023large, huang2024digital, liu2024large, chaccour2024telecom}. %
%
Instead, we analyze contributions that include the latest achievements based on Diffusion Models and \glspl{ssm}, which are often overlooked in related surveys.
On the other hand, we purposely exclude in our analysis generative algorithms such as
\glspl{gan}, \glspl{vae}, and normalizing-flows.
These methods, while significant in past years, are considered less relevant compared to the latest advancements in \gls{genai}.
%














