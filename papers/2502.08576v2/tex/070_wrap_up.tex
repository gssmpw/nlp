








The advent of \gls{genai} presents a transformative potential for network monitoring and management tasks. 
By leveraging advanced generative models, 
network systems can 
address the growing complexity and dynamic nature of modern networks with a more predictive and proactive stance, moving beyond the traditional reactive measures.

%
Despite the positive aspects of this breakthrough approach, \gls{genai} also has several \emph{limitations} summarized in the left side of Figure~\ref{fig: genai_limitations}. 
One of the major concerns is 
(\textit{i}) \emph{the lack of trustworthiness and robustness}~\cite{jacobs2022ai}.
The issue of trustworthiness is 
due to ``closed box'' nature of the \gls{genai}-model architectures.
Hence, users often find it difficult to 
understand the decision-making processes of a \gls{genai} model, thus complicating the debugging, refinement, and efforts to improve model performance and address biases.
Furthermore,
the evaluation of \gls{genai} models focuses only on a few publicly-available datasets, thus not reflecting dynamic real-world scenarios. 
Additionally, network attacks or crafted inputs to evade malware identification can \emph{intentionally} modify the nature of network traffic.
These points raise doubts about the robustness of \gls{genai} solutions and their applicability to network monitoring and management in practice.


%
Furthermore, (\textit{ii}) \emph{these models require substantial computational resources for pre-training and fine-tuning}. 
Achieving high performance typically involves processing extensive datasets, which demands significant computational power and time. 
%
Additionally, current \gls{genai} models consist of trillions of parameters and require extensive training periods and significant resources that may not be available in smaller computational environments.
In fact, some studies~\cite{piovesan2024telecom, erak2024leveraging} suggest using more compact models (\eg \gls{slm}) for network monitoring and management tasks, which, with a significantly smaller number of parameters, can deliver performance comparable to larger models.
%
This high resource requirement represents a major hindrance to the widespread adoption of \gls{genai} and poses challenges for its online application in network environments. 
As a result, integrating such models into operational networks remains complex and costly.
Additionally, the complexity of these models also introduces (\textit{iii}) \emph{potential vulnerabilities to adversarial attacks}, posing security risks. 
Attackers can exploit their architectures, manipulating their behavior or compromising data integrity.
Expressly, adversarial attacks may use subtle perturbations to trick the model into making incorrect predictions, risking data security and system reliability.
Lastly, the use of \gls{genai} raises (\textit{iv}) \emph{ethical concerns} related to data privacy and misuse of generated content.
Personal data are used in training models, potentially causing
significant privacy issues.
Then, the ability to generate misleading content (such as deepfakes) can lead to misinformation and manipulation.

In the following, we identify possible \textbf{future directions} to improve and overcome the drawbacks of \gls{genai} in network monitoring and management. We graphically summarize these perspectives on the right side of Figure~\ref{fig: genai_limitations}.

\noindent
\textbf{Real-time suitability and efficiency:}
%
for network monitoring and management tasks, timely operation is a strict requirement.
%
Nevertheless, \gls{genai} models demand substantial computational power and extensive datasets for both training and fine-tuning. 
%
\emph{Federated Learning}~\cite{zhang2021survey, li2020review} can help overcome these challenges by enabling decentralized training across multiple devices, thereby minimizing the need for centralized data storage and processing~\cite{huang2024federated}.
%
Additionally, large model sizes can be a significant bottleneck, resulting in longer inference times.
A key future direction is to develop efficient \gls{genai} models that can be effectively trained on smaller datasets.
Furthermore, model compression, efficient architecture design, and hardware acceleration can improve the processing speed and reduce computational overhead.
In this direction, integrating \emph{Green AI}~\cite{schwartz2020green} principles in \gls{genai} development allows
an environmentally-sustainable progress,
reducing the energy consumption and carbon footprint associated with model training and deployment.
Specifically, 
\emph{TinyML}~\cite{ray2022review, dutta2021tinyml} can offer a powerful combination of efficiency and intelligence. 
TinyML enables real-time, low-power data processing on edge devices, while \gls{genai} provides advanced predictive modeling and simulation.
This synergy allows for proactive network management, offering localized insights and responses that enhance network efficiency, even in resource-constrained environments (\eg when running \gls{genai} models directly on handheld devices).
Additionally, 
quick model adaptation to network shifts and anomalies
is crucial for maintaining efficient and responsive network operations.
In this context, \emph{Quantum ML}~\cite{ciliberto2018quantum, dunjko2018machine} may significantly boost this capability in the long term, speeding up model training and enhancing predictive precision in intricate and evolving scenarios.

\vspace{5pt}
\noindent
\textbf{Handle network data complexity:}
integrating \emph{multimodal \gls{genai}} promises to revolutionize network monitoring and management. 
By leveraging the capability of \gls{ai} to process and analyze diverse data types---ranging from textual logs and metrics to visual network topologies---network operators can achieve unprecedented levels of insight and automation.
This holistic approach allows for more accurate anomaly detection, predictive maintenance, and dynamic resource allocation, ultimately leading to more resilient and efficient network infrastructures.
%
The ability of multimodal \gls{ai} to synthesize information from multiple sources enhances ``on-the-fly'' decision-making while paving the way for adaptive, self-healing networks, aligning with the vision of a fully automated, intelligent network management paradigm.
Furthermore, more advanced techniques,
such as Reinforcement Learning~\cite{du2023beyond, du2024enhancing}, enhance the adaptivity of \glspl{llm}, enabling them to evolve during their operational mode.
Additionally, \textit{data distillation} procedures can be enforced to manage the large scale and redundancy of datasets.
This approach helps in reducing the dataset size by retaining only the most essential samples and discarding unnecessary ones. 
Such a procedure can significantly decrease both the training time and the resources required.
%

%
\vspace{5pt}
\noindent
\textbf{Interpretability and robustness:}
%
the convergence of \emph{\gls{xai}}~\cite{xu2019explainable, dwivedi2023explainable} and \emph{\gls{genai}} heralds a new era in network monitoring and management, where transparency and innovation go hand in hand. 
%
\gls{xai} provides much-needed clarity in the decision-making processes of \gls{ai} systems, enabling network operators to trust and understand the actions taken by their automated tools~\cite{nascita2024survey}.
%
When coupled with the \gls{genai}'s ability to simulate and predict network behaviors, this synergy offers a robust framework for proactive management. For instance, network anomalies can be not only detected but also explained leveraging \gls{xai} tools and addressed with \gls{ai}-generated tailored responses. 
This fusion ensures actionable and transparent \gls{ai}-driven insights,
fostering a deeper integration of \gls{ai} in network operations. 
As a result, network management becomes more intelligent, reliable, and user-centric, 
%
enabling networks to be
%
both self-optimizing and comprehensible.
%
Additionally, to improve model robustness, we can leverage diverse and comprehensive datasets to improve the \textit{generalizability of GenAI models across different contexts}, optionally including multi-task learning techniques, and their \textit{resistance to adversarial attacks}, exploiting adversarial training or data augmentation strategies. 

%

\vspace{5pt}
\noindent
\textbf{Integration with other systems:}
integrating \gls{genai} models with existing network systems enhances their utility in network monitoring and management by ensuring interoperability with the current infrastructure, developing robust \glspl{api} for seamless embedding, and enabling data fusion from multiple sources. This integration also ensures scalability and leverages automation to handle routine tasks, providing comprehensive and intelligent network management solutions.
Future directions could focus on integrating \emph{Causal \gls{ai}}~\cite{kaddour2022causal}, which emphasizes understanding cause-and-effect relationships, and \emph{Neuro-symbolic \gls{ai}}~\cite{sarker2021neuro, hitzler2022neuro}, 
which combines the learning capabilities of neural networks with the logical reasoning of symbolic \gls{ai} into \gls{genai} models. 
%
This integration could improve the ability of these models to handle complex long-term tasks and multi-step decision-making, empowering them to accomplish intelligent planning.
%

\vspace{5pt}
\noindent
\textbf{Security and privacy:}
as \gls{genai} models are increasingly used in network monitoring and management, ensuring their security and privacy is essential. 
Future directions should focus on
integrating \emph{Blockchain}~\cite{zheng2018blockchain, pilkington2016blockchain} technology can provide a decentralized and immutable framework for secure data sharing and model updates, further enhancing the overall security and transparency of \gls{genai} applications.
Moreover, implementing privacy-preserving techniques, such as \emph{differential
privacy}, can safeguard user data, and ensure secure deployment with robust access control and secure communication channels.
%
In addition, addressing ethical considerations by ensuring transparent data usage, unbiased model training, and accountability in \gls{ai} decisions will be crucial in building trust and reliability in \gls{genai} applications for network management.
%

\vspace{5pt}
%
In conclusion, integrating \gls{genai} into network monitoring and management holds significant promise. However, it is crucial to carefully manage expectations and avoid overly optimistic assumptions about its capabilities. Tackling the associated limitations is essential to ensure a realistic and effective implementation.
%
Collaboration between academia and industry is vital to ensure that the generative models developed are not only theoretically sound but also practical and scalable in real-world applications. Establishing benchmarks and standardized datasets to evaluate the performance of \gls{genai} in network monitoring and management can provide a foundation for continuous improvement and innovation.
By driving advancements in predictive analytics, anomaly detection, and automation, future research can pave the way for more intelligent, efficient, and secure network systems. As we continue to explore this intersection of \gls{genai} and networking, it is imperative to address the associated challenges and ethical considerations to fully harness the potential of \gls{genai}.







