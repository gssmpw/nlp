


This section provides an overview of the current status of \gls{genai} in the \gls{nmm} context. Accordingly, Sec.~\ref{subsec:dowstream_tasks} outlines the five \gls{nmm} use cases where \gls{genai} is currently being utilized, along with a description of the benefits \gls{genai} offers for each. Then, Secs.~\ref{sec:traffic_generation}-\ref{sec:network_management} dissect the works that employ \gls{genai} for each use case.




\subsection{Use Cases for GenAI in Network Monitoring and Management}
\label{subsec:dowstream_tasks}



\gls{genai} is actively used to address \gls{nmm} use cases in various networking domains. We have identified five key use cases where it is currently employed: ($i$) \emph{Network Traffic Generation}, ($ii$) \emph{Network Traffic Classification}, ($iii$) \emph{Network Intrusion Detection}, ($iv$) \emph{Networked System Log Analysis}, and ($v$) \emph{Network Digital Assistance for Documentation \& Configuration}.
Such use cases are summarized in Figure~\ref{fig: downstream} along with the acronyms we use in this work. 
Below, we provide detailed descriptions for each use case.





\vspace{5pt}
\noindent
\textbf{\gls{traffgen}}
refers to the process of creating synthetic network data, ranging from
the generation of
\mbox{(bi-)flow} statistical features
or sequence of features extracted from the packets within a \mbox{(bi-)flow} (e.g., packet size, inter-arrival time, and packet direction), to the generation of
the entire PCAP trace.%
\footnote{In this survey, we do not cover sensor data generation, such as temperature or pressure measurements, since this task involves modeling physical phenomena rather than actual network traffic.}

\gls{traffgen} is crucial for network traffic analysis from various applicative perspectives. It enables the simulation of different scenarios to (stress) test network infrastructure and services, validate security measures (e.g., for automated penetration tests), and 
augment training data for improving \gls{ml} models 
performance and generalization capabilities.
The key challenge of \gls{traffgen} is producing high-fidelity synthetic network samples that closely resemble real traffic.
Hence, a critical aspect 
is the validation of the synthetic traffic generated, since both \emph{effectiveness}
(in enhancing \gls{ml} or \gls{dl} models performance) 
and \emph{validity} 
(in simulating with high fidelity the real traffic) 
are desiderata of the synthetic network traffic generation task~\cite{adeleke2022network}.

\noindent
\emph{How \gls{traffgen} 
can benefit from \gls{genai}: }
\gls{genai} can significantly enhance the \gls{traffgen} task by leveraging its ability to understand and mimic natural language patterns, thus modeling network traffic as the ``language of the Internet''. Accordingly, it helps in generating realistic protocol sequences and user interactions, producing high-quality synthetic traffic that mirrors the diverse and complex traffic patterns of real environments~\cite{meng2023netgpt, jiang2024netdiffusion}.

\vspace{5pt}
\noindent
\textbf{\gls{traffclass}}
aims to categorize network traffic represented by various 
\emph{\glspl{to}}
such as packets, bursts, flows, biflows, or sessions. This process may include identifying the protocol (especially when nonstandard transport ports are used), the name of the application (\eg YouTube, Netflix, Facebook), or the type of service (\eg streaming, web browsing, VoIP) that generated the traffic.
Generally, \gls{traffclass} involves modeling target network traffic classes (\eg by using \gls{ml} or \gls{dl} algorithms) and differentiating the traffic into one of these target classes.

Since \gls{traffclass} can identify user behaviors and predict traffic categories, it is crucial in enhancing network management operations. By applying rules based on \gls{traffclass} results, network management can be adapted to address the specific needs of the network, optimizing the handling of different types of traffic. 
The main challenges affecting \gls{traffclass} include the limited availability of high-quality data to train effective models and the poor generalization capabilities shown by the state-of-the-art \gls{traffclass} techniques~\cite{pacheco2018towards, azab2022network,aceto2023aipowered}.


\noindent
\emph{How \gls{traffclass} 
can benefit from \gls{genai}: }
\gls{genai} can significantly enhance \gls{traffclass} through advanced contextual awareness and pattern recognition capabilities of pre-trained models.
These models leverage large unlabeled datasets to learn unbiased data representations, which can be easily transferred to various downstream tasks by fine-tuning on limited labeled data.
The killer idea can be the modeling of network traffic as a language, namely the language of machine-$2$-machine communication.
Thus, \gls{genai} can produce highly versatile pre-trained models that, due to their high generalizability, can be adapted to solve different \gls{traffclass} tasks with minimal effort, eliminating the need to train new models from scratch for each task~\cite{lin2022,wang2024netmamba}. 

\vspace{5pt}
\noindent
\textbf{\gls{nid}} aims to identify anomalous or malicious traffic traversing the network.
Specifically, \gls{nid} focuses on monitoring the traffic exchanged between connected entities (e.g., mobile devices, computers, servers) to secure them. 
Its main objective is to detect anomalous behaviors that may be related to security threats or intrusions by analyzing the exchanged traffic. 

\gls{nid} is crucial in identifying malicious activities by distinguishing legitimate (\viz~benign) traffic from potentially harmful (\viz~malicious) traffic. Moreover, it can be used even to identify specific attack traffic~\cite{chou2021survey}. These operations enable prompt response to threats and minimize potential damage to network infrastructure and its users. 

\gls{nid} should be seen as a specialization of \gls{traffclass} when dealing with supervised multiclass or binary classification (\viz misuse detection). In this context, \gls{nid} leverages techniques common to \gls{traffclass} to identify types of attacks based on knowledge extracted from labeled training data. However, \gls{nid} also encompasses \gls{ad}, which involves identifying outliers or abnormal behaviors that deviate from the norm. This process is typically addressed via out-of-distribution detection or one-class classification methodologies. 
For these reasons, we treated it as a separate use case in this survey also due to its importance, dedicated modeling solutions, and extensive related literature.



\noindent
\emph{How \gls{nid} 
can benefit from \gls{genai}: }
Similarly to \gls{traffclass}, \gls{genai} can enhance \gls{nid} through contextual awareness, enabling the detection of anomalies by understanding the context of network events over time. Its adaptability by means of transfer learning allows rapid adaptation to new threats, while semantic analysis identifies unusual command sequences. In addition, complex pattern recognition and unsupervised learning help detect subtle deviations and unknown threats~\cite{manocchio2024flowtransformer, ferrag2024}.


\vspace{5pt}
\noindent
\textbf{\gls{nla}} refers to solutions that automate the extraction of knowledge from network or system logs to summarize them (\ie to identify key elements) or to detect anomalies (\viz~log anomaly detection).
Network and system logs typically consist of semi-structured text/records of data that collect network or system events.
Specifically, the logs considered in this work pertain to network-related applications, such as web or email servers, or related to network entities, such as network managers.

\gls{nla} is crucial for enhancing security by identifying unauthorized access and potential security breaches. It ensures system reliability by providing hints to identify performance bottlenecks or diagnosing the root cause of the fault. Moreover, it improves software quality through log debugging or ensuring software robustness. It optimizes operations by analyzing user behaviors or auditing activities and helps maintain compliance across various domains (\eg supporting predictive maintenance)~\cite{he2021survey}.

\noindent
\emph{How \gls{nla}  
can benefit from \gls{genai}: }
Leveraging modern and advanced \glspl{llm}, \gls{genai} can efficiently parse, interpret, and summarize log data written in natural language. It extracts significant events and patterns through semantic analysis, enhancing the understanding of log data~\cite{han2023loggpt,boffa2024logprecis}.


\vspace{5pt}
\noindent
\textbf{\gls{nda}} (briefly, Network Digital Assistance) 
focuses on monitoring and controlling network operations to ensure efficient and reliable performance.
Specifically, \gls{nda} aims to maintain network reliability and availability, optimize network performance, ensure security, and enable efficient resource scheduling.
%
Moreover, it is fundamental for reducing downtime, preventing data leaks, and ensuring uninterrupted service delivery.
%
Hence, \gls{nda} is crucial for various applications. 
It facilitates interoperability issues in heterogeneous network environments characterized by multi-layer and multi-vendor infrastructures. \gls{nda} also supports advanced networking frameworks like \gls{sdn} %
and simplifies the management of ever-growing \gls{iot} environments. 
In this context, resource provisioning, device configuration, network monitoring, and software update management are essential for reducing energy consumption and strengthening the security of 
\gls{iot} devices that are resource-constrained and insecure-by-design~\cite{martinez2014network, aarikka2017network, aboubakar2022review}.

\noindent
\emph{How \gls{nda} 
can benefit from \gls{genai}: }
\gls{genai} allows extensive automation for network operations.
Specifically, \glspl{llm} provide a natural language interface that simplifies the retrieval of complex information in networking standards and documents crucial for \gls{nda}. 
This interface also facilitates the management of various network software and hardware,
enhancing operational efficiency~\cite{wang2023network}.

\vspace{5pt}
\noindent
\textbf{Relations among NMM use cases}:
In general, these five use cases are strongly related. 
Together, they improve the monitoring and control of network operations, leading to improved network performance, reliability, and security.
From a broad perspective, \gls{nmm} acts as the system's actuator,
leveraging the outputs of other tasks to make informed decisions and optimize network performance.
A graphical representation of this interaction is shown in Figure~\ref{fig: downstream_interaction}, highlighting the links between the various use cases.

Specifically, \gls{traffgen} creates synthetic traffic that can be used to evaluate potential network configurations before deployment---%
\ie \gls{nda}. 
By using reliable and accurate traffic generators, which can produce traffic that closely resembles real traffic, \gls{traffgen} can provide valuable insights into the behavior of network equipment in pseudo-real operating environments.
Additionally, by generating various types of synthetic traffic, both benign and malicious, \gls{traffgen} can be utilized to assess, in different contexts, the response of classification and intrusion detection systems---%
\ie
\gls{traffclass} and \gls{nid}, respectively.
Moreover, it can enhance their performance and generalizability by augmenting training data for \gls{ml} and \gls{dl} models.

Conversely, \gls{traffclass} and \gls{nid} offer insights into the (real) traffic traversing the network, enabling performance improvements (through specific traffic prioritization or routing rules) and enhancing security (by applying filtering rules to block anomalous or malicious traffic). Thus, \gls{traffclass} and \gls{nid} can facilitate online (re)configurations through \gls{nda}.
Additionally, data-driven \gls{traffclass} and \gls{nid} systems trained on real traffic data can provide valuable feedback on the quality of the synthetic traffic generated---%
\ie validating \gls{traffgen}.


Finally, while \gls{nla} shares similarities with \gls{traffclass} and \gls{nid} mechanisms, it
focuses on analyzing log data related to network equipment (\eg servers and routers) and the services provided (\eg web pages) rather than network traffic.
Therefore, \gls{nla} supports network management operations by providing summaries and insights from logs and by identifying anomalies in the operation of network equipment and services. This information can then be used to adjust the behavior of these network equipment and services (\ie \gls{nda}).








\vspace{5pt}
\noindent\textbf{From network traffic to GenAI:} 
Figure~\ref{fig: usecase_detail} details the generic pipeline for the use cases.
Specifically, \gls{traffgen}, \gls{traffclass}, and \gls{nid} leverage the same kind of input (\ie network traffic) to pre-train and fine-tune the \gls{genai} model.
Conversely, \gls{nla} and \gls{nda} typically employ pre-trained models, which are then fine-tuned with specific log data or different network documents to adapt to the task at hand.

\gls{genai} approaches are not designed to ingest network traffic directly. Instead, approaches based on \glspl{llm} or Diffusion Models are typically designed to process data in a text-based or image-like format. Therefore, for use cases involving direct processing of network traffic (\ie \gls{traffgen}, \gls{traffclass}, and \gls{nid}), traffic data need to be transformed into a text-based or image-like representation before using it as input to the \gls{genai} model. This transformation is performed via \emph{Datagram-to-Token} and \emph{Datagram-to-Image} operations, respectively.




\vspace{5pt}
\noindent
\emph{Datagram-to-Token}: Approaches based on \glspl{llm} typically employ a \emph{Datagram-to-Token} method to convert encrypted traffic into pattern-preserved token units for pre-training~\cite{lin2022}. 
%
This method involves segmenting traffic into packets and representing their characteristics as word-like tokens, similar to natural language processing. 
When packets are grouped into traffic objects, such as (bi)flows or bursts, special tokens are required to mark the packet boundaries within these traffic objects (\eg common values for these special tokens are \texttt{[SEP]}, \texttt{[MSK]}, \texttt{[PAD]}, and \texttt{[PKT]}). 
Additionally, the type of features extracted from the traffic object (or from each packet belonging to it) may require further preprocessing before being converted to tokens (\eg conversion into raw bytes, anonymization, or quantization).
%


\vspace{5pt}
\noindent
\emph{Datagram-to-Image}: Approaches based on Diffusion Models typically employ a \emph{Datagram-to-Image} method to convert encrypted traffic into image representations~\cite{jiang2024netdiffusion}. 
%
Two main Datagram-to-Image variants are commonly exploited based on the features to be used: 
($i$) raw-bytes-to-image 
and ($ii$) features-to-image. 
The former involves translating network traffic into standardized bits, where each bit corresponds to a packet header field bit. The encoded sequence of packets is then formed into a matrix, which is interpreted as an image (\eg \emph{nPrint} format~\cite{holland2021new}).
Conversely, when the model input is a time series of packet features (\eg packet sizes or inter-packet times), different transformations can be applied to encode the time series as an image, such as \emph{FlowPic}~\cite{shapira2021flowpic} or \emph{Gramian Angular Summation Field (GASF)}~\cite{sivaroopan2023netdiffus}.
%


\vspace{5pt}
Hereinafter, we provide a detailed overview of the existing literature for each \gls{nmm} use case.
