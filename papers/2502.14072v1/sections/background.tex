\section{Background}
\label{sec:background}

Simulation emerges as a crucial pedagogical tool, allowing students a practical and dynamic approach to understanding complex concepts and real-world scenarios \cite{chwif2006modelagem}. In this context, the topic explores the fundamentals of simulation and simulation teaching.
\\

Simulation is a controlled technique that replicates essential aspects of the real world in simulated environments. It involves the creation of representative models, which can be physical, mathematical, statistical or computational, depending on the phenomenon to be simulated. This approach offers the ability to control variables and experimental conditions \cite{reichard1992computer}, and specifically in computing courses, simulation is considered one of the most widely used research methods to conduct exploratory empirical investigations \cite{Molleri:2016:SGS:2961111.2962619}. Amid rapid technological evolution, it emerges as a crucial resource for training professionals capable of facing complex challenges. \\

To illustrate the simulation, we can demonstrate the use of one of the most widely used simulation tools in current research, which is the Matlab tool. Matlab is widely recognized for its ability to simulate complex systems accurately and efficiently. By using Matlab to simulate the operation of a wind turbine, we can model all of its physical components and analyze its performance under different conditions.

Simulating a wind turbine in Matlab allows us to explore a wide range of variables, from the different scenarios and environmental conditions that the turbine may face to the specific characteristics of its physical components, such as the blades, the generator and the control system. In addition, it is possible to simulate variations in voltage, electric current and other electrical characteristics of the turbine, enabling a detailed analysis of its behavior in different situations, as shown in the image below. \cite{souza2022simulador}.


Modeling allows researchers to isolate specific factors and observe their impact on the system as a whole, and in environments where direct experimentation is dangerous, impractical or too expensive, simulation offers a safe and affordable alternative. For example, in aviation, flight simulators allow pilots to be trained in emergency situations without real risks, and this is a case where "not simulating" can cause considerable costs, such as wrong assumptions or expectations or even risk to life \cite{DBLP:conf/ispw/Birkholzer12}. In addition, simulation generates data that can be analyzed to better understand the behavior of the system under study, leading to process improvements, resource optimization or more informed decision-making.

To complement the theoretical foundation, simulation education refers to the use of simulation as an educational tool. This approach provides students with the opportunity to apply theories and concepts learned in the classroom in practical and realistic contexts \cite {bradley2014review}. Some characteristics of simulation education include:
\noindent \textbf{Hands-on Experience:} Simulation provides a hands-on experience that goes beyond theory, allowing students to develop practical skills and acquire contextualized knowledge \cite{weis1998computer};
\noindent \textbf{Decision Making:} Simulated environments often present challenges that require quick and effective decision-making \cite {garrett2001value}. This helps develop critical thinking and problem-solving skills; \noindent \textbf{Immediate Feedback:} Simulation allows for immediate feedback on actions and decisions \cite{tena2017training}, enabling error correction and continuous improvement of performance;
\noindent \textbf{Controlled Environments:} Educators can create controlled environments to simulate specific situations, ensuring that students face challenges relevant to the field of study;
\noindent \textbf{Interdisciplinary Application:} Simulation is an approach that can be applied in various disciplines, from healthcare (medical simulation) to engineering, management and military training \cite{harder2010use}.