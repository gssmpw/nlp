% As a general rule, do not put math, special symbols or citations
% in the abstract
\begin{abstract} 
This paper reports results of an investigation about the level of knowledge among undergraduate computer science students in Brazil regarding the topic of simulation. Amid rapid technological evolution, simulation emerges as a crucial resource for training professionals capable of facing complex challenges. The research seeks to analyze the presence and effectiveness of simulation education, exploring students' perceptions, the tools used, the challenges faced, and the prospects for deeper study. This report highlights the importance of academic training in a dynamic technological environment, emphasizing the crucial role of simulation education in undergraduate computer science, while exploring the foundations of the methodologies and educational strategies associated with the topic. A survey research approach is adopted. 108 answers were received from 10 Brazilian states. 19 respondents from 15 different institutions said they had some contact with simulation during their studies. Results reveal that MATLAB/Simulink is the most popular formalism/tool used to teach simulation in Brazil.
\end{abstract}

\begin{IEEEkeywords}
modeling and simulation, teaching, learning, knowledge, simulating.
\end{IEEEkeywords}