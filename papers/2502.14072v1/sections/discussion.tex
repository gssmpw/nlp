\section{Threats to Validity and Limitations of the Research}
\label{sec:discussion}

By exploring the results, challenges and critical points that may influence the reliability and generalizability of the conclusions were identified, aiming to provide a critical reflection on the research conducted, highlighting areas for improvement and important considerations for future research.

The study examined the perceptions of undergraduate computer science students about simulation teaching in Brazil, aiming to provide \textit{insights} to improve educational strategies. However, it is crucial to consider the potential threats to validity and limitations that may impact the interpretation of the results.

Data collection was based on questionnaire responses, which are subject to individual interpretation and possible response bias. In addition, the exclusion of participants who answered negatively to the "Elimination Question" may have influenced the representativeness of the sample.

The generalizability of the results to the entire country may be affected by the concentration of responses in certain regions and institutions. Geographic diversity may not have been fully addressed, compromising external validity.

The definition of research questions and the choice of variables may have limited the scope of the study, influencing construct validity. Differences in the interpretation of key terms by participants may have impacted the consistency of the data.

The sample size, especially after excluding participants, may have impacted the statistical validity of the analyses. Significant results may be influenced by the heterogeneity of the sample.

In addition, the questionnaire used has limitations that should be considered. The approach adopted may have restricted the depth of responses, failing to capture important nuances in the students' experiences. The nature of the closed questions may not have provided a complete understanding of the complexity of the topic in question.

Another limitation identified refers to the concentration of responses in certain states, which may not adequately represent the geographic diversity of Brazil. The inclusion of more regions could enrich the analysis and provide a more comprehensive view of the educational landscape in relation to teaching simulation in undergraduate computing courses.

These limitations highlight the importance of interpreting the results with caution and recognizing the restrictions inherent in the methodology adopted. Future research could explore more comprehensive approaches, considering the inclusion of participants from different regions and using methods that allow for a deeper understanding of students’ experiences in simulation education.