\section{Final Remarks and Future Work} 
\label{sec:FR}
This study provided a comprehensive analysis of simulation teaching in undergraduate computer science courses in Brazil, highlighting the importance of this topic in students’ academic training. Throughout the research, several aspects were addressed, from the methodology used to the results obtained, with the aim of offering insights to improve teaching and understand students’ perceptions.

The research highlighted the growing importance of simulation teaching in students’ training, recognizing it as a fundamental tool for teaching real-world market processes, especially when direct experience is challenging. The diversity of tools used by students, with emphasis on MATLAB Simulink, highlights the variety of approaches and technologies available in this field, although the lack of exposure to some tools suggests possible gaps in the integration of these platforms into educational curricula.

The challenges identified, such as lack of updated content, financial inaccessibility of paid tools, and technical difficulties, highlight the complexity of simulation studies. The scarcity of educational materials and tutorials is highlighted as a significant barrier to effective learning. Difficulties and challenges are encountered in the teaching of information systems in Brazil, including from the perspective of teachers \cite{Neves2023}.

The diversity in the levels of interest of students in deepening their studies in simulation after graduation highlights opportunities and challenges in teaching this practice. The assessment of practical knowledge reveals heterogeneity among participants, pointing to the need for flexible educational approaches.

The analysis of the incidence of simulation teaching in institutions revealed a variety of perceptions, with some students indicating low or no incidence. This diversity highlights the importance of a more in-depth analysis on the integration of simulation teaching in the curricula of higher education institutions.

The recognition of simulation as an essential tool to improve learning, offering practical and applied benefits in several disciplines, stood out in the conclusions. Recommendations for improving simulation teaching include updating content, expanding the variety of tools taught, and developing strategies to overcome identified challenges.

The study contributed to the development of more effective educational strategies, addressing specific challenges and promoting a more enriching learning environment aligned with contemporary demands. Despite the limitations identified, the work provided valuable information about the presence of simulation in educational institutions, the tools used, the difficulties faced by students, and the interest in further studies in this area after graduation. These conclusions can serve as a basis for improvements in teaching and future research directions.