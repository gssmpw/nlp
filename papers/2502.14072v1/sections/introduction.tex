\section{Introduction}
\label{sec:introduction}

In the dynamic and fast-paced context of contemporary technologies, academic training plays a crucial role in preparing professionals capable of facing the complex challenges of the real world. In the context of undergraduate computing courses in Brazil, the growing use of simulation tools has emerged as an essential component to enhance students' educational experience, providing a practical and innovative approach to learning.

The survey was motivated by the significant importance of simulation knowledge for undergraduate computing students in Brazil, which is a fundamental technique in several areas, providing the practical application of theories and concepts learned in the classroom. A growing concern is centered on the way simulation has been implemented in undergraduate computing courses in the country.

This monograph sought to explore the presence and effectiveness of simulation teaching in higher education institutions in Brazil, with a focus on undergraduate computing areas. The research aimed to provide a comprehensive understanding of the current state of simulation teaching, identifying both the opportunities and the challenges faced by students in this context.

The effectiveness of teaching through simulation is vital to the quality of training of future professionals in the area. The research sought to understand students' perceptions and level of knowledge regarding simulation, identifying gaps and challenges in its implementation. The objective is to outline guidelines that contribute to improving the integration of simulation in teaching, aligning academic training with the practical demands of the sector.

The research protocol was developed based on the guidelines proposed by \cite{kasunic:survey} Kasunic, \cite{Linaker:survey:guideline} Linaker and \cite{Molleri:2016:SGS:2961111.2962619} Molleri, by establishing a method structured into six stages: Introduction; Theoretical Foundation; Planning; Execution; Results and Conclusion.

Throughout this document, detailed analyses of students' perceptions and experiences regarding simulation teaching were presented, highlighting the most used tools, the challenges faced, and the prospects for further studies. The conclusions derived from these analyses will contribute to the development of recommendations aimed at improving the quality of simulation teaching in higher education institutions, thus consolidating the importance of this practice as a vital component in the training of computing professionals.

In addressing the challenges and opportunities in integrating simulation into teaching, the study reveals the heterogeneity in student responses, pointing to significant challenges such as the lack of updated content, financial inaccessibility of specific tools, and technical difficulties. These difficulties offer opportunities for improvement, highlighting the need for differentiated strategies to stimulate student engagement.

The research plan is detailed, outlining objectives, identification of the target audience, questionnaire design, definition of topics, evaluation scale, representative sampling, and informed consent. This plan aims to investigate the level of knowledge of undergraduate computing students in Brazil on the topic of simulation.

The questionnaire was presented, divided into sections such as Demographic Questions, Specific Questions, and Research Questions, each contributing to an in-depth understanding of the current state of simulation teaching. The questionnaire structure ranges from demographic questions to specific aspects of knowledge, practical application, tools used, learning sources, challenges faced and prospects for improving teaching.

By addressing specific issues such as basic simulation knowledge, tools and methods used, practical applications, perceived importance in training and learning sources, the questionnaire aims to capture a comprehensive view of students’ experience of simulation education.

In summary, this study not only provides an in-depth understanding of the current state of simulation education, but also offers valuable contributions to the development of more effective educational strategies. The conclusions and recommendations derived from the analysis of the data provide a guide for improving the quality of teaching, aligning it with contemporary demands and preparing students more effectively for the practical and professional challenges in their fields.

The remainder of the paper is organized as follows: Section 2 presents a brief background, Section 3 details the planning, conduction, and results of our study. Section 4 discusses the results. Finally, Section 5 concludes the paper.



