\section{Related work}
We explore two primary areas of research relevant to our work: comparable advertising mechanisms and fairness objectives in advertising.

\textbf{Comparable advertising mechanisms.} In this study, we propose a budget-based advertising approach that ensures full budget utilization. In marketplaces, auction-based advertising systems typically employ budget pacing to manage budgets ____. Budget pacing is a class of solutions designed to satisfy daily budget goals and ensure smooth spending throughout the time period.

Although various pacing algorithms exist, they differ in their working principles. Hard pacing (or throttling) ____ controls the probability of participating in auctions and requires bid values as input. Bid modification (or soft pacing) in most realizations expects bids or target action values as well (____). Some implementations also incorporate return-on-investment (ROI) constraints ____.

Beyond traditional auctions, our research is closely tied to the field of online allocation. A comprehensive review of this area can be found in ____. Online allocation presents an optimization problem to be solved in real-time upon new search requests, with a focus on maximizing reward functions under budget constraints. The primary distinction between our work and online allocation lies in the need for a predefined resource consumption function in online allocation. This requires dynamic bids (essentially creating an auction) or fixed target action values. In contrast, our problem formulation does not require such a function but instead exploiting fairness objectives to balance traffic between ads.

An advertising product design that shares similarities with our approach is Guaranteed Display (GD) advertising (____). In GD, advertisers pay a fixed budget for a specified number of target actions (typically impressions). If the advertiser's goal is not met, they typically receive a refund. Some pacing approaches have been developed for GD, such as ____, which uses Dual Mirror Descent (____).

However, GD differs from our problem formulation in that impression goals serve as the budget constraints. This means that fairness is managed automatically when signing fair contracts with advertisers. However, ensuring good platform efficiency in terms of overall paid traffic becomes more challenging due to idle sponsored spots when advertiser goals are satisfied and lack of consideration for ad quality (CTRs).

\textbf{Fairness objectives} in optimization problems have been defined and implemented in various ways. A critical survey of different schemes for formulating fairness criteria in optimization models can be found in ____. One widely used approach for combining efficiency and fairness is the $\alpha$-fairness scheme, introduced in ____. This scheme generalizes some special cases of fairness, including max-min and proportional fairness. The trade-off between fairness and efficiency based on $\alpha$-fairness was studied in ____, where theoretical estimates of the price of fairness were obtained. ____ proposed an approach that employs proportional fairness to allocate resources in scenarios with Gaussian-distributed input traffic.

However, it is worth noting that the $\alpha$-fairness scheme defines both efficiency and fairness using a single utility function, which may be a limitation in practice. In contrast, our work defines efficiency as the overall number of paid clicks and fairness as a function of impressions per unit budget. Other approaches, such as regularized online allocation problem introduced in ____, can support various fairness constraints. This approach is based on relative consumption and requires a consumption function.

The Gini Mean Difference (GMD) is a well-known measure of inequality, first proposed by ____. The GMD has been extensively applied in various fields, with numerous practical applications and modifications reported in the literature ____. Geometrically, the GMD can be interpreted as proportional to the area between the Lorenz curve and a diagonal line representing perfect equality. GMD has also been applied to advertising-related problems, such as ad inventory allocation. ____ presented an optimization model with a Gini-based objective function to allocate ad inventories efficiently. Their work demonstrated the effectiveness of using GMD in this context and provided an efficient algorithm for solving the resulting optimization problem. Notably, ad inventory allocation differs from online ad allocation in several key aspects: it operates offline, whereas online ad allocation occurs in real-time; and it focuses on audience segments rather than individual users or search queries. Besides its applications in advertising, the GMD has also been applied in ranking problems, particularly in ____.

Other notable approaches to ensuring fairness in advertisement systems include the two-stage algorithm for online allocation introduced in ____. This approach involves an offline stage where the desired amount of clicks on each item from each user segment is estimated, followed by an online optimization problem that chooses an optimal policy to display ads.