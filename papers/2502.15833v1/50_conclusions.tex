\section{Conclusions}

This paper introduces a novel approach to OOD detection using KANs, capitalizing on their unique local neuroplasticity property. 
Our method effectively differentiates between InD and OOD samples by comparing the activation patterns of a trained KAN against its untrained counterpart. 
The experimental results show that our KAN-based detector reaches SOTA performance across seven benchmarks from two different domains. 
Importantly, our experiments show that the previous methods suffer from a non-optimal InD dataset size, while our method is unaffected by these perturbations.
This makes the KAN detector a robust and versatile method that can maintain high performance across diverse and unpredictable data distributions. 
Future work will further explore the effect of different training tasks on detection performance.