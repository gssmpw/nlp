%%%%%%%%%%%%%%%%%
\FloatBarrier
\subsection{Optimization of a clamping--locator fixture system} \label{sec.ClampingLocator}
%%%%%%%%%%%%%%%%%

\noindent So-called ``clamping-locator'' fixture systems are crucial ingredients in various engineering  processes, where they allow to precisely position and hold objects during operations.  
For instance, such devices are widespread components of biological experiments, where delicate samples are observed by microscopes and must be handled without damage \cite{szolga2021robotic}, or in  medical operations, during which surgical tools need to be positioned tightly in place.
They are also ubiquitous in industrial manufacturing processes, such as CNC milling \cite{vichare2011unified} or other workflows in robotics.

As their name suggests, clamping-locator systems are made of two parts: locators hamper the motion of the target component in one or several directions, while clamps apply a very strong pressing force against locators, see \cref{fig.ClampLocator}.  
Clamping-locator systems have numerous advantages in machining; however, the application of an excessive clamping force can deform delicate workpieces, leading to higher scrap rates and higher material costs. Therefore, it is essential to optimize the geometry and placement of the clamping and locator regions to minimize the deformation of the mechanical part incurred by the system. In this section, we address this issue by using our boundary optimization technology.

%%%%%%%%%%%%
\subsubsection{The optimization problem}
%%%%%%%%%%%%

\noindent The considered mechanical structure is represented by a bounded domain $\Omega \subset \mathbb{R}^3$, whose boundary $\partial \Omega$ is made of five disjoint pieces:
\begin{equation}\label{eq.decompDOmelas}
 \partial \Omega = \overline{\Gamma_D} \cup \overline{\Gamma_N} \cup  \overline{\Gamma_T} \cup \overline{\Gamma_F} \cup \overline{\Gamma}.
 \end{equation}
In this decomposition:
\begin{itemize}
    \item The region $\Gamma_D$ is that where locators are operating, i.e. the displacement of the piece is prevented. 
    \item A strong force $fn$, where $f : \Gamma_N\to \R$ is applied in the normal direction on the clamping region $\Gamma_N$.
    \item The tool applies a force $g : \Gamma_T \to \R^3$ on the region $\Gamma_T$, which is fixed and not subject to optimization.
    \item No efforts are applied on the region $\Gamma_F$, which is not subject to optimization.
    \item The remaining region $\Gamma$ is also traction-free, but it is subject to optimization.
\end{itemize}
We also set $\Sigma_D := \partial \Gamma_D$, $\Sigma_N := \partial \Gamma_N$, and we assume that the sets $\Gamma_D$ and $\Gamma_N$ always stay well-separated by $\Gamma$, i.e. $\overline{\Gamma_D} \cap \overline{\Gamma_N} = \emptyset$, see \cref{fig.ClampLocator} for an illustration. 

\begin{figure}[ht]
    \centering
    \includegraphics[width=0.5\textwidth]{figures/figclamp.pdf}
    \caption{\it Schematics of a clamping-locator system as considered in \cref{sec.ClampingLocator}.}
    \label{fig.ClampLocator}
\end{figure}

Neglecting body forces, the displacement of the structure in this situation is the unique solution $u \in H^1(\Omega)^3$ to the following version of the linearized elasticity system:
\begin{equation*} \label{eq.BoundaryOptimization.ClampLocator.Elasticity}
\left\{
\begin{array}{cl}
    -\dv (Ae(u_{\Gamma_D,\Gamma_N})) = 0 & \text{in } \Omega, \\
    Ae(u_{\Gamma_D,\Gamma_N}) n = f n & \text{on } \Gamma_N,\\
    Ae(u_{\Gamma_D,\Gamma_N}) n = g & \text{on } \Gamma_T,\\
    Ae(u_{\Gamma_D,\Gamma_N}) n= 0 & \text{on } \Gamma_F \cup \Gamma,\\
    u_{\Gamma_D,\Gamma_N} = 0 & \text{on } \Gamma_D,
\end{array}
\right.
\end{equation*}
for given functions $f: \R^3 \to \R$ and $g : \R^3 \to \R^3$. We seek to minimize the total displacement of the structure, 
while utilizing a reasonably low amount of clamps and locators. This problem is formulated as:
\begin{equation} \label{eq.BoundaryOptimization.ClampLocator.Problem}
    \min_{\Gamma_D, \: \Gamma_N \subset \partial \Omega} \: J(\Gamma_D,\Gamma_N) + \ell_N \Area(\Gamma_N) + \ell_D \Area(\Gamma_D), \text{ where } J(\Gamma_D,\Gamma_N) :=  \dfrac{1}{2 \Vol(\Omega)} \int_{\Omega} |u_{\Gamma_D,\Gamma_N}|^2 \: \mathrm{d}x 
\end{equation}
where $\ell_D, \ell_N > 0$ are penalization parameters for the areas of the clamping and locator regions, respectively.

%%%%%%%%%%%%
\subsubsection{Approximate version of the functional $J(\Gamma_D, \Gamma_N)$ and its shape derivative}
%%%%%%%%%%%%

\noindent We follow once again the procedure of \cref{sec.hepsconduc} for deriving a smooth approximation of the shape functional $J(\Gamma_D,\Gamma_N)$.
More precisely, we trade the exact shape and topology optimization problem \cref{eq.BoundaryOptimization.ClampLocator.Problem} for an approximate counterpart, where the transition between homogeneous Dirichlet and Neumann boundary conditions around $\Sigma_D$  is smoothed by a Robin boundary condition; the transition between homogeneous and inhomogeneous Neumann conditions near $\Sigma_N$ is unaltered, as it does not pose any particular difficulty, see \cref{sec.extsdconduc}. Precisely, we consider the replacement of $J(\Gamma_D,\Gamma_N)$ in \cref{eq.BoundaryOptimization.ClampLocator.Problem} by the function: 
\begin{equation*} \label{eq.BoundaryOptimization.ClampLocator.ApproxProblem}
 J_\e(\Gamma_D,\Gamma_N) := \dfrac{1}{2\Vol(\Omega)} \int_{\Omega} \lvert u_{\Gamma_D,\Gamma_N,\e}\lvert^2 \: \d x,
\end{equation*}
where $u_{\Gamma_D,\Gamma_N,\e} \in H^1(\Omega)^3$ is characterized by the following boundary value problem:
\begin{equation*}
    \left\{
    \begin{array}{cl}
        -\dv (Ae(u_{\Gamma_D,\Gamma_N,\e}))= 0 & \text{in } \Omega, \\
        Ae(u_{\Gamma_D,\Gamma_N,\e}) n = f n& \text{on } \Gamma_N,\\
        Ae(u_{\Gamma_D,\Gamma_N,\e}) n = g & \text{on } \Gamma_T,\\
        Ae(u_{\Gamma_D,\Gamma_N,\e}) n = 0 & \text{on } \Gamma_F,\\
        Ae(u_{\Gamma_D,\Gamma_N,\e}) n + h_{\Gamma_D,\e} u_{\Gamma_D,\Gamma_N,\e} = 0 & \text{on } \Gamma_D \cup \Gamma.
    \end{array}
    \right.
    \end{equation*}
The function $h_{\Gamma_D, \e} : \partial \Omega \rightarrow \mathbb{R}$ is defined by:
\begin{equation*}
    \forall x \in \partial \Omega, \quad  h_{\Gamma_D, \e}(x) = h\left( \dfrac{d^{\partial \Omega}_{\Gamma_D} (x) }{\e} \right),
\end{equation*} 
where the geodesic signed distance function $d^{\partial \Omega}_{\Gamma_D}$ to $\Gamma_D$ is defined in \cref{sec.distmanifold}
and the transition profile $h \in C^\infty(\mathbb{R})$ satisfies \cref{eq.ShapeDerivatives.BumpFunction}.

The shape derivative of $J_\e(\Gamma_D,\Gamma_N)$ is the subject of the following proposition, whose proof is omitted for brevity. 

\begin{proposition} \label{theorem.BoundaryOptimization.ClampLocator.ShapeDerivative}
    The functional $J_\e(\Gamma_D, \Gamma_N)$ is shape differentiable and its derivative reads, for an arbitrary tangential deformation $\theta $ (i.e. such that $\theta \cdot n = 0$):
    \begin{multline*}
    J_\e'(\Gamma_D, \Gamma_N)(\theta)
    =  -\frac{1}{\e^2} \int_{\partial \Omega} h'\left(\frac{d^{\partial \Omega}_{\Gamma_D}(x)}{\e}\right) \: \theta(\pi_{\Sigma_D}(x)) \cdot n_{\Sigma_D} (\pi_{\Sigma}(x)) \: u_{\Gamma_D,\Gamma_N,\e}(x) \cdot p_{\Gamma_D,\Gamma_N,\e}(x) \: \d s(x) \\
    - \int_{\Sigma_N} f (x) (p_{\Gamma_D,\Gamma_N,\e}(x) \cdot n(x))\: (\theta \cdot n_{\Sigma_N} )(x) \: d \ell(x)\\
    \end{multline*}
    where the adjoint state $p_{\Gamma_D,\Gamma_N,\e} \in H^1(\Omega)^3$ is the solution to the following boundary value problem:
    \begin{equation*}
    \begin{aligned}
        \left\{ 
        \begin{array}{cl}
        -\dv (Ae(p_{\Gamma_D,\Gamma_N,\e})) = -\frac{u_{\Gamma_D,\Gamma_N,\e}}{\Vol(\Omega)} & \mathrm{in} \ \Omega,\\
        Ae(p_{\Gamma_D,\Gamma_N,\e})n + h_{\Gamma_D, \e} p_{\Gamma_D,\Gamma_N,\e} = 0 & \mathrm{on} \ \partial \Omega .
        \end{array}
        \right.
    \end{aligned}
    \end{equation*}
\end{proposition}\par\medskip

%%%%%%%%%%%%
\subsubsection{The topological derivative}
%%%%%%%%%%%%

\noindent The sensitivities of the shape functional $J(\Gamma_D,\Gamma_N)$ with respect to the addition of a small surfacic disk $\omega_{x_0,\e}$ centered at $x_0 \in \Gamma$ to $\Gamma_D$ or $\Gamma_N$ are provided in the next result; its proof is similar to those of the results of \cref{sec.repelas}, and it is omitted for brevity.

\begin{proposition}
Let $\Gamma_D$, $\Gamma_N$ be disjoint regions of the smooth boundary $\partial \Omega$ as in \cref{eq.decompDOmelas}, and let $x_0 \in \Gamma$ be given. Then,
 \begin{enumerate}[(i)] 
\item The perturbed criterion $J(({\Gamma_D})_{x_0, \e},\Gamma_N)$, accounting for the addition of the surfacic disk $\omega_{x_0,\e} \subset \Gamma$ to $\Gamma_D$, has the following asymptotic expansion:
    \begin{equation*}
        J(({\Gamma_D})_{x_0, \e},\Gamma_N) = 
            J(\Gamma_D,\Gamma_N)  +   \frac{1}{\lvert \log \e \lvert }\frac{\pi \mu}{1-\overline\nu} u_{\Gamma_D,\Gamma_N} (x_0) \cdot p_{\Gamma_D,\Gamma_N}(x_0)   + \o\left(\dfrac{1}{|\log \e|}\right)  \text{ if }  d = 2.
    \end{equation*}
    and 
        \begin{equation*}
        J(({\Gamma_D})_{x_0, \e},\Gamma_N) = 
            J(\Gamma_D,\Gamma_N)  +  \e \: Mu_{\Gamma_D,\Gamma_N}(x_0) \cdot p_{\Gamma_D,\Gamma_N}(x_0)   + \o(\e) \text{ if } d = 3.
    \end{equation*}
   
\item The perturbed criterion $J(\Gamma_D,({\Gamma_N})_{x_0, \e})$, accounting for the addition of $\omega_{x_0,\e} \subset \Gamma$ to $\Gamma_N$, has the following asymptotic expansion:
    \begin{equation*}
        J(\Gamma_D,({\Gamma_N})_{x_0, \e}) = 
            J(\Gamma_D,\Gamma_N) - 2\e f(x_0) n(x_0)\cdot p_{\Gamma_D,\Gamma_N}(x_0)+ \o(\e) \text{ if } d = 2.
    \end{equation*}
 and
     \begin{equation*}
        J(\Gamma_D,({\Gamma_N})_{x_0, \e}) =
            J(\Gamma_D,\Gamma_N) - \pi \e^2 f(x_0) n(x_0)\cdot p_{\Gamma_D,\Gamma_N}(x_0) + \o(\e^2) \text{ if } d = 3.
    \end{equation*}
    \end{enumerate}
 In the above formulas, the polarization tensor $M$ is defined by \cref{eq.defMelas} and the adjoint state $p_{\Gamma_D,\Gamma_N} \in H^1(\Omega)^3$ is the solution to the following boundary value problem:
\begin{equation*}
\left\{
\begin{array}{cl}
    -\dv (Ae(p_0)) = -\frac{u_0}{\Vol(\Omega)} & \text{in } \Omega, \\
    Ae(p_0) n= 0 & \text{on } \Gamma \cup \Gamma_N \cup \Gamma_T \cup \Gamma_F,\\
    p_0 = 0 & \text{on } \Gamma_D.
\end{array}
\right.
\end{equation*}
 \end{proposition}

%%%%%%%%%%%%
\subsubsection{Numerical results}
%%%%%%%%%%%%

\noindent The initial configuration is depicted on \cref{fig.BoundaryOptimization.ClampLocator.Setup}; it is supplied by a tetrahedral mesh $\mathcal{T}^0$ of $\Omega$ comprising $28,000$ vertices and $146,000$ tetrahedra, see . 
The top and bottom regions (highlighted in pink) of $\partial \Omega$ correspond to the zones $\Gamma_F$ which are free of effort and are not subject to optimization. The region in blue indicates the initial locator area, while the zone in orange is that \(\Gamma_T\) where the tool applies force. The numerical values of the parameters of the computation are supplied in \cref{tab.ClampLocator.Params}.
Like in \cref{sec.CathodeAnode}, we solve \cref{eq.BoundaryOptimization.ClampLocator.Problem} by alternating between the individual optimizations of the clamping and locator regions.
We apply \cref{alg.CouplingMethods.SurfaceOptimization} with the parameter $n_{\text{top}} =10$, i.e. every $10$ iterations, the geometric update process is replaced by a topological update step, during which a small surfacic disk is attached to either $\Gamma_D$ or $\Gamma_N$. This process continues until 100 iterations are completed, after which only geometric optimization updates are carried out. 

 This problem is challenging because many configurations could potentially stabilize the system and minimize the average displacement \cref{eq.BoundaryOptimization.ClampLocator.Problem}, making it difficult to identify an optimal configuration. Further complicating the issue, our experiments reveal that adding a new locator region could destabilize the system, leading to an increase in the objective functional, as the creation of a new clamping region might push the piece in a direction that is not yet adequately located. 

\begin{table}[!ht]
    \centering
    \begin{tabular}{|c|c|c|c|c|c|c|c|}
        \hline
        Parameter &  $\e$ &  $\ell_D$ & $\ell_N$ & $f$ & $g$ & $\hmax$ & $\hmin$\\
        \hline
        Value &  $0.00001$ & $0.0001$ & $0.0001$ & $-\frac{1}{10}$ & $(1, 0, 0)$ & $0.2$ & $0.02$\\
        \hline
    \end{tabular}
    \caption{\it Values of the parameters used in the optimal design example of the clampings and locators on the boundary of a mechanical part considered in \cref{sec.ClampingLocator}.}
    \label{tab.ClampLocator.Params}
\end{table}

\begin{figure}[ht] 
    \centering 
    \begin{tabular}{cc}
\begin{minipage}{0.4\textwidth}
\begin{overpic}[width=0.8\textwidth]{figures/ClampLocator_Setup}
\put(2,5){\fcolorbox{black}{white}{a}}
\end{overpic}
\end{minipage} & 
\begin{minipage}{0.5\textwidth}
\begin{overpic}[width=0.8\textwidth]{figures/ClampLocator_Setup_2}
\put(2,5){\fcolorbox{black}{white}{b}}
\end{overpic}
\end{minipage}
\end{tabular}
\caption{\it (a) Side views of the initial mesh $\mathcal{T}^0$ of the mechanical structure $\Omega$ considered in the optimization example of clamping-locator regions of \cref{sec.ClampingLocator}; (b) Front and back views. The pink regions are the zones $\Gamma_F$ which are not subject to optimization; the blue region is the initial locator region $\Gamma_D^0$ and the orange region in the back of the structure is that $\Gamma_T$ where the tool is operating.}
\label{fig.BoundaryOptimization.ClampLocator.Setup}
\end{figure}
\begin{figure}[ht]
    \centering
\begin{tabular}{cc}
\begin{minipage}{0.49\textwidth}
\begin{overpic}[width=1.0\textwidth]{figures/ClampLocator_1_20}
\put(2,5){\fcolorbox{black}{white}{$n=20$}}
\end{overpic}
\end{minipage} & 
\begin{minipage}{0.49\textwidth}
\begin{overpic}[width=1.0\textwidth]{figures/ClampLocator_1_40}
\put(2,5){\fcolorbox{black}{white}{$n=40$}}
\end{overpic}
\end{minipage}
\end{tabular}
\caption{\it A few intermediate shapes obtained in the optimization process of the clamping and locator regions of a mechanical structure considered in \cref{sec.ClampingLocator}. The clamping and locator regions $\Gamma_N$, $\Gamma_D$ are depicted in green and blue, respectively.}
    \label{fig.ClampLocator.Results_1}
\end{figure}
\begin{figure}[ht]
    \ContinuedFloat
    \centering
\begin{tabular}{cc}
\begin{minipage}{0.49\textwidth}
\begin{overpic}[width=1.0\textwidth]{figures/ClampLocator_1_80}
\put(2,5){\fcolorbox{black}{white}{$n=80$}}
\end{overpic}
\end{minipage} & 
\begin{minipage}{0.49\textwidth}
\begin{overpic}[width=1.0\textwidth]{figures/ClampLocator_1_120}
\put(2,5){\fcolorbox{black}{white}{$n=120$}}
\end{overpic}
\end{minipage}
\end{tabular}
\par\bigskip

\begin{tabular}{cc}
\begin{minipage}{0.49\textwidth}
\begin{overpic}[width=1.0\textwidth]{figures/ClampLocator_1_160}
\put(2,5){\fcolorbox{black}{white}{$n=160$}}
\end{overpic}
\end{minipage} & 
\begin{minipage}{0.49\textwidth}
\begin{overpic}[width=1.0\textwidth]{figures/ClampLocator_1_120}
\put(2,5){\fcolorbox{black}{white}{$n=200$}}
\end{overpic}
\end{minipage}
\end{tabular} 
    \caption{(cont.) \it A few intermediate shapes obtained in the optimization process of the clamping and locator regions of a mechanical structure considered in \cref{sec.ClampingLocator}. The clamping and locator regions $\Gamma_N$, $\Gamma_D$ are depicted in green and blue, respectively.}
\end{figure}
\begin{figure}[ht]
    \centering
\begin{overpic}[width=0.5\textwidth]{figures/ClampLocator_1_245}
\put(2,5){\fcolorbox{black}{white}{a}}
\end{overpic}
\begin{overpic}[width=0.5\textwidth]{figures/ClampLocator_Obj_1.jpg}
\put(2,5){\fcolorbox{black}{white}{b}}
\end{overpic}
    \caption{\it (a) Optimized design ($n=245$) in the clamping-locator example of \cref{sec.ClampingLocator}; (b) Associated convergence history.}
    \label{fig.ClampLocator.Final}
\end{figure}

Our algorithm successfully minimizes the mean displacement $J(\Gamma_D,\Gamma_N)$ of the mechanical part $\Omega$, reducing the value of this objective from \(46.1951\) to approximately \(0.08\).
 A few snapshots of the optimization process are shown in \cref{fig.ClampLocator.Results_1}; the final design and the convergence history are reported in \cref{fig.ClampLocator.Final}. The total simulation takes around 5 hours.
Interestingly, the optimized design predominantly relies on clamping regions to achieve this minimization:
only small locator regions are present at the front and back of the structure, positioned opposite the direction of the load applied by the tool.