%%%%%%%%%%%%%%%%%%%%%%%%%%%%%%%%%%%%%%%%%%%%%%%%%%%%%%%
\section{Introduction}
%%%%%%%%%%%%%%%%%%%%%%%%%%%%%%%%%%%%%%%%%%%%%%%%%%%%%%%

\noindent The need for energy savings and the looming concerns about material scarcity have raised a considerable interest in shape and topology optimization in the academic and industrial communities.
These techniques find applications in a large variety of areas, such as structural mechanics \cite{bendsoe2013topology,sigmund2013topology}, civil engineering and architecture \cite{adriaenssens2014shell,beghini2014connecting}, fluid mechanics \cite{alexandersen2023detailed,borrvall2003topology}, electromagnetism \cite{gangl2016sensitivity,jensen2011topology,lebbe2019robust,nishanth2022topology}, and biomedical engineering \cite{huiskes1989mathematical,quarteroni2003optimal,zhong2006finite}. 

In the classical instances of such problems, the design under scrutiny is a ``bulk'' domain $\Omega$ in $\mathbb{R}^d$ ($d = 2, 3$), which is optimized with respect to a performance criterion $J(\Omega)$, under constraints about e.g. its volume or perimeter. In most applications, $J(\Omega)$ depends on a physical ``state'' function $u$, characterized as the solution to a partial differential equation posed on $\Omega$. 
Most often, the regions of $\partial \Omega$ supporting specific boundary conditions attached to the latter are imposed by the context, and are not subject to optimization.

The present article investigates optimal design problems where the design variable is precisely one of those regions of $\partial \Omega$ supporting a particular type of boundary conditions in the formulation of the underlying physical problem. Here are a few examples:
\begin{itemize}
\item In electrostatics, $\Omega$ represents a conductor and the state is the voltage potential $u: \Omega \to \R$, solution to the conductivity equation. 
It is grounded on a subset $\Gamma_D$ of $\partial \Omega$ and a flux $g: \Gamma_N \to \R$ is imposed on a disjoint region $\Gamma_N \subset \partial \Omega$: these effects are modeled by a homogeneous Dirichlet condition on $\Gamma_D$ and an inhomogeneous Neumann condition on $\Gamma_N$. 
The remaining part $\Gamma$ of $\partial\Omega$, which is insulated from the outside, is subject to a homogeneous Neumann boundary condition. 
Although $\Gamma_D$ and $\Gamma_N$ are usually fixed, one may wish to minimize (or maximize) the amplitude of the electric field in $\Omega$ with respect to their placement on $\partial \Omega$.
\item In acoustics, $u: \Omega \to \R$ is the sound pressure in a room $\Omega$, solution to the Helmholtz equation. 
The boundary $\partial \Omega$ is decomposed into two regions $\Gamma_N$ and $\Gamma_R$: Neumann boundary conditions are imposed on $\Gamma_N$, where an incoming wave undergoes perfect reflection, 
while $\Gamma_R$ bears Robin boundary conditions, which account for a partial absorption. One may then seek to arrange $\Gamma_N$ and $\Gamma_R$ within $\partial \Omega$ so as to minimize the sound pressure in $\Omega$.
\item In structural mechanics, $\Omega$ is a mechanical part, attached on a subset $\Gamma_D$ of its boundary $\partial \Omega$, and submitted to surface loads $g: \Gamma_N \to \R^d$ on a disjoint region $\Gamma_N \subset \partial \Omega$. The vector field $u :\Omega \to \R^d$, represents the displacement of the structure. It is the solution to the linear elasticity system. Usually, $\Gamma_D$ and $\Gamma_N$ are given by the context, and only the remaining, traction-free boundary $\Gamma$ is optimized. However, it may be relevant to optimize the placement of the fixation region $\Gamma_D$ to minimize the displacement of the structure. 
\end{itemize}

These questions fit in the general shape optimization framework of a subset $G$ of a fixed ambient surface $S \subset \R^d$. 
Early studies in this context are devoted to the simulation of geometric flows within $S$, notably the mean curvature flow. 
In \cite{macdonald2008level}, this task is investigated thanks to the level set method, in a situation where $S$ is equipped with a triangular mesh; it is also considered in \cite{cheng2002motion}, where $S$ itself is defined in an implicit way. 
The perhaps most natural instance of a physical optimal design problem posed on a surface $S \subset \R^d$ concerns the optimal reinforcement of a shell structure: $S$ then plays the role of the midsurface, in which the optimized region $G$ is that made of a stiffer material. 
Popular numerical strategies feature a fixed mesh of $S$ which serves as the support of density-based topology optimization techniques \cite{pan2024density,traff2021topology}. The level set method is also employed in such setting in \cite{townsend2019level}, and it is coupled with a geometric optimization procedure for the midsurface $S$ itself in \cite{ho2022efficient}.
On a different note, in \cite{ye2019topology}, the level set method is combined with a conformal mapping strategy, reducing the surface $S$, and thereby the whole shape optimization problem, to a more classical planar situation.

In spite of their natural character and ubiquity in concrete applications, optimization problems of regions supporting the boundary conditions of a physical problem have been rarely considered in the literature. 
Without anticipating too much on the more complete overview of related contributions in the particular application contexts of \cref{sec.Numerical}, let us mention that density-based topology optimization methods are prevailing in this context also, see e.g. \cite{calabrese2017optimization,ma2011compliant} in the context of the optimal design of a fixture system.
In \cite{xia2016topology,xia2014level}, the level set method is used on a fixed mesh of a computational domain for the concurrent optimization of the 2d shape and of the region of the boundary where Dirichlet boundary conditions are imposed; see also \cite{zhang2015shape} for similar ideas.
Closer to the framework of the present article, the article \cite{desai2018topology} leverages the level set method on a fixed mesh of a box-shaped room $\Omega$ to optimize the distribution of absorbent and sound-hard materials on its boundary. 

Optimizing a function $J(G)$ depending on a region $G$ supporting boundary conditions raises challenging issues from various perspectives. 
From the theoretical viewpoint, the realization of this task requires the calculation of the derivative of $J(G)$. 
This information is indeed the basic expression of the optimality conditions of the optimization problem. It is also the main ingredient of iterative optimization algorithms, starting from the simple, unconstrained gradient descent method to more advanced constrained optimization algorithms, such as those proposed in \cite{dunning2015introducing,feppon2020null,svanberg1987method}. 
Interestingly, these derivatives can also be used to enforce the robustness of the optimization problem with respect to small perturbations of the geometry of $G$,  in the spirit of e.g. \cite{allaire2014linearized,allaire2015deterministic,martinez2019structural}.
From the numerical viewpoint, optimal design problems of regions supporting boundary conditions raise in particular the need to track the possibly dramatic evolution of a region within an ambient surface -- a task which is already notoriously difficult when the ambient medium is (a bounded domain of) the Euclidean space $\R^d$. 

The present article is the natural continuation of our previous contributions \cite{bonnetier2022small,brito2023body,dapogny2020optimization}. The article \cite{dapogny2020optimization} deals with the shape sensitivity of a function $J(G)$ depending on a region $G$ of the boundary of a domain $\Omega$ bearing the boundary conditions of a physical problem, in the spirit of the method of Hadamard: the derivative of $J(G)$ with respect to ``small'', diffeomorphic perturbations of $G$ is considered. The situation where $G$ is the support of Dirichlet conditions and the complement $\partial \Omega \setminus \overline G$ is equipped with Neumann conditions is of particular interest in that work. Indeed, the weakly singular behavior of the state function $u$ in that case renders the treatment of shape derivatives particularly difficult -- a fact which was previously acknowledged in \cite{fremiot1999shape}. To alleviate this issue, an approximation of the state problem is proposed, which lends itself to simpler calculations and numerical treatment. These developments pave the way to a numerical algorithm for the shape optimization of the region $G$, see \cref{sec.hepsconduc} below for a brief presentation. 
The article \cite{bonnetier2022small} deals with singular perturbations of $G$, at the theoretical level: asymptotic formulas are derived for the solution $u$ to the conductivity equation in the case where homogeneous Neumann boundary conditions are replaced by homogeneous Dirichlet equations (and vice-versa) in a ``vanishing'' zone $\omega_{\e}\subset \partial \Omega$. 
A preliminary application of these results to the device of a notion of topological derivative for functions $J(G)$ depending on regions $G$ bearing boundary conditions was described in the recent article \cite{brito2023body}.  
The latter stands at the numerical level; elaborating on the ideas of \cite{allaire2014shape}, it introduces a body-fitted mesh evolution method to track the (possibly dramatic) motion of a region within a fixed ambient surface, 
which efficiently combines the level set method with remeshing algorithms. 
Note that this algorithm was used in the very recent work \cite{martinet2024numerical} in view of optimizing Neumann eigenvalues on the unit 3d sphere.

In the present article, we leverage these ideas to propose a general shape and topology optimization framework for a region $G \subset \partial \Omega$ supporting the boundary conditions attached to a partial differential equation set on the fixed ambient domain $\Omega$.
Our strategy combines (adapted versions of) the notions of shape and topological derivatives to appraise the sensitivity of a function $J(G)$ with respect to small, diffeomorphic perturbations of $\partial G$ and to singular perturbations, via the addition of a small surface disk $\omega_{x_0,\e}$ around a point $x_0 \in \partial\Omega$, respectively.
From the theoretical vantage, one of our contributions is to provide formal calculation methods for both types of derivatives. These are detailed in the simple setting of electrostatics, and they can be adapted to treat several novel situations, in the more intricate contexts of acoustics and structural mechanics. Notably, the calculation of a topological derivative for a function $J(G)$ depending on a region $G$ bearing boundary conditions is rather subtle, and we describe how our methods can be adapted to achieve this purpose in each situation.
Interestingly, this article illustrates two different forays of asymptotic analysis in the realm of shape and topology optimization: on the one hand, these concepts are used to smoothen a singular transition between two zones bearing different boundary conditions, leading to a simplified calculation of shape derivatives. On the other hand, they allow to investigate singular perturbations of a smooth background problem, via the insertion of a ``small'' zone where boundary conditions are altered.

This article is organized as follows. In \cref{sec.Preliminaries}, we define the versions of shape and topological derivatives adapted to our purpose of evaluating the sensitivity of a function $J(G)$ depending on a boundary region $G$, and we sketch a generic optimal design algorithm based on these ingredients. The next \cref{sec.optbcconduc,sec.TopologicalSensitivity} concern the model physical context of electrostatics, which conveniently allows us to expose the salient features of the proposed methodology with a minimum amount of technicality. In \cref{sec.optbcconduc}, we describe the calculation of the shape derivative of a functional $J(G)$ which depends on the region supporting the homogeneous Dirichlet boundary conditions attached to the conductivity equation. We also outline the treatment of the simpler problem where the optimized region $G$ is that supporting inhomogeneous Neumann boundary conditions, and we address other easy extensions of this analysis. In the same physical situation, \cref{sec.TopologicalSensitivity} deals with the delicate calculation of the topological derivative of $J(G)$. 
In \cref{sec.Helmholtz} and \cref{sec.Elasticity}, we generalize this material to acoustics and structural mechanics, accounted for by the Helmholtz and linear elasticity equations, respectively: the calculations of approximate or exact shape derivatives for the typical functionals $J(G)$ of interest follow exactly the same trail as in the electrostatics setting. We explicitly compute the topological derivatives in these cases. 
We then turn to numerical applications in \cref{sec.Numerical}, where several examples are analyzed in the settings of electrostatics, acoustics and structural mechanics, in order to appraise the main features of our methods. 
Finally, a few conclusive remarks and perspectives for future research are drawn in \cref{sec.concl}.