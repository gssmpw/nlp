%%%%%%%%%%%%%%%%%%%%%%%%%%%%%%%%%%%%%%%%%%%%%%%%%%
%%%%%%%%%%%%%%%%%%%%%%%%%%%%%%%%%%%%%%%%%%%%%%%%%%
\section{Optimization of regions bearing boundary conditions of the Helmholtz equation} \label{sec.Helmholtz}
%%%%%%%%%%%%%%%%%%%%%%%%%%%%%%%%%%%%%%%%%%%%%%%%%%
%%%%%%%%%%%%%%%%%

\noindent In this section, we slip into the physical context of acoustics, where the state function $u$ is the solution to the Helmholtz equation, 
and we adapt the previous material to this setting. 
As we have mentioned in the introduction, exact or approximate shape derivatives of functionals $J(G)$ depending on a region $G \subset \partial\Omega$ bearing  boundary conditions of this physical problem can be calculated in the same spirit as in the context of electrostatics, tackled in \cref{sec.optbcconduc}; we refer to \cref{sec.Acoustics} for an application example. We therefore focus on the calculation of the topological derivative of such a function $J(G)$, mostly presenting the differences with the analysis conducted in \cref{sec.TopologicalSensitivity}. To emphasize the parallel between both situations, 
we retain the same notations as in there, insofar as possible. 

%%%%%%%%%%%%%%%%%%%
\subsection{Presentation of the model}\label{sec.modHelmholtz}
%%%%%%%%%%%%%%%%%%

\noindent For simplicity, we focus on a model interior Helmholtz problem, excerpted from \cite{bangtsson2003shape,desai2018topology,wadbro2006topology}.
The arguments exposed in here can be readily adapted to the situation of an infinite propagation medium, see \cref{sec.Acoustics} for an example in this context. 
Let $\Omega$ be a smooth bounded domain in $\R^d$, whose boundary $\partial \Omega$ is composed of two disjoint parts, namely:
$$ \partial \Omega = \overline{\Gamma_N} \cup \overline{\Gamma_R},$$
where:
\begin{itemize}
\item The region $\Gamma_N$ is the support of homogeneous Neumann boundary conditions, 
\item The region $\Gamma_R$ is the support of a homogeneous impedance (i.e. Robin) boundary condition. 
\end{itemize}
In this situation, the state function describes the pressure of an acoustic wave. In the ``background'' situation, it is the solution $u_0$ to the following equation:
\begin{equation}\label{eq.helmbg}
\left\{
\begin{array}{cl}
-\dv(\gamma \nabla u_0) - k^2 u_0 = f & \text{in } \Omega, \\
\gamma  \frac{\partial u_0}{\partial n} = 0 & \text{on } \Gamma_N,\\
\gamma \frac{\partial u_0}{\partial n} + i k u_0 = 0 & \text{on } \Gamma_R.
\end{array}
\right.
\end{equation}
Here, we have denoted the wave number by $k>0$; the coefficient $\gamma$ encodes  the physical properties of the medium (it is the inverse of its admittance), and it is uniformly bounded away from $0$ and $\infty$, see \cref{eq.bdgamma}. 
The smooth right-hand side $f : \R^d \to \R$ represents a source acting in the medium.
From the physical viewpoint, the homogeneous Neumann boundary condition on $\Gamma_N$ accounts for a ``hard wall'', where perfect reflection of the wave occurs, while the Robin condition encodes a partial absorption of the energy. 

\begin{remark}\label{rem.wellposedHelmholtz}
The well-posedness of \cref{eq.helmbg} is not a consequence of the usual Lax-Milgram 
theory, as the involved operator is not coercive. This property is a consequence of the fact that the coefficient of the impedance boundary condition has a non zero imaginary part; mathematically, it follows from the combination of a unique continuation principle with the so-called Banach–Necas–Babu\v{s}ka theorem, see for instance Chap. 25 and 35 in \cite{ern2021finite2}. 
\end{remark}

Let us recall that the fundamental solution $\Gamma(x,y)$ of the Helmholtz operator 
$ u \mapsto  -\Delta u - k^2 u $ in the free space $\R^d$ ($d = 2,3$) is given by the following formulas:
\begin{equation}\label{eq.GreenHelmholtz} 
\Gamma(x,y) = \left\{ 
\begin{array}{cl}
\frac{-1}{4i}H_0^{(1)}(k\lvert x - y \lvert) & \text{if } d = 2, \\
\frac{e^{ik\lvert x - y \lvert}}{4\pi \lvert x - y \lvert} & \text{if } d =3,
\end{array}
\right.
\end{equation}
where $H^{(1)}_0$ is the Hankel function of the first kind and of order $0$, which is the solution to the ordinary differential equation:
$$ \frac{1}{r} \frac{\d}{\d r}\left(r \frac{\d H}{\d r}(r)\right) + k^2 H(r) = 0, \text{ for } r >0, $$
see for instance \cite{abramowitz1965handbook}. 

The Green's function $N(x,y)$ for the background problem \cref{eq.helmbg} can be constructed from this datum by standard means, as the solution to: 
\begin{equation}\label{eq.helmGreen}
\left\{
\begin{array}{cl}
-\dv_y(\gamma \nabla_y N(x,y)) - k^2 N(x,y) = \delta_{y=x} & \text{in } \Omega, \\
\gamma \frac{\partial N}{\partial n_y}(x,y) = 0 & \text{for } y \in \Gamma_N, \\
\gamma \frac{\partial N}{\partial n_y}(x,y)  + ik N(x,y) = 0 & \text{for } y \in \Gamma_R.
\end{array}
\right.
\end{equation}


\begin{remark}\label{rem.Nfunchelm}
In this context also, a Green's function $L_\gamma(x,y)$ for the version of the boundary value problem \cref{eq.helmbg} posed on the lower half-space $H$, featuring a constant coefficient $\gamma >0$ and equipped with homogeneous Neumann boundary conditions on $\partial H$ can be constructed by the method of images: 
$$L_\gamma(x,y) = \frac{1}{\gamma} (\Gamma(x,y) + \Gamma(x,-y)).$$
\end{remark}\par\smallskip

%%%%%%%%%%%%%%%%%%%
\subsection{Sensitivity of the solution to the Helmholtz equation with respect to a singular perturbation of the Neumann boundary condition}\label{sec.pertHelmholtz}
%%%%%%%%%%%%%%%%%%

\noindent To set ideas, we consider a perturbed version of the problem \cref{eq.helmbg} where the homogeneous Neumann boundary condition is replaced by an impedance boundary condition on a ``small'' surface disk $\omega_{x_0,\e} \subset \Gamma_N$ around the point $x_0 \in \Gamma_N$, that is:
\begin{equation}\label{eq.helmpert}
\left\{
\begin{array}{cl}
-\dv(\gamma\nabla u_\e) - k^2 u_\e = f & \text{in } \Omega, \\
\gamma \frac{\partial u_\e}{\partial n} = 0 & \text{on } \Gamma_N \setminus \overline{\omega_{x_0,\e}}, \\
\gamma \frac{\partial u_\e}{\partial n} + ik u_\e = 0 & \text{on }\Gamma_R \cup \omega_{x_0,\e}, \\
\end{array}
\right.
\end{equation}
Our main result concerning the asymptotic behavior of $u_\e$ when $\e \to 0$ is the following: 

\begin{theorem}\label{sec.asymHelmh}
Let $x_0 \in \Gamma_N$ be given. The following asymptotic expansions hold true, at any point $x \in \overline\Omega \setminus ((\overline{\Gamma_N} \cap \overline{\Gamma_R}) \cup \left\{x_0\right\})$:
$$ u_\e(x) = u_0(x) - 2\e ik u_0(x_0) N(x,x_0) + \o(\e)  \text{ if } d=2, $$
and 
$$ u_\e(x) = u_0(x) - \pi\e^2 ik u_0(x_0) N(x,x_0) + \o(\e^2)  \text{ if } d=3.$$
\end{theorem} 
\begin{proof}[Sketch of the proof]
We rely on the formal argument employed in our treatment of  \cref{th.Conductivity.HNHD.Expansion}, and we only sketch the main frame for brevity. 
Without loss of generality, we set $x_0 = 0$ and $\omega_\e := \omega_{x_0,\e}$; we furthermore rely on the simplifying assumption that $\gamma$ is constant, and $\partial \Omega$ is completely flat in a neighborhood of $0$, see \cref{eq.flat}.
The error $r_\e = u_\e - u_0$ is the unique solution in the space $H^1(\Omega;\C)$ of complex-valued $H^1$ functions to the following boundary value problem:
\begin{equation}\label{eq.helmre}
\left\{
\begin{array}{cl}
-\dv(\gamma\nabla r_\e) - k^2 r_\e = 0 & \text{in } \Omega, \\
\gamma \frac{\partial r_\e}{\partial n} = 0 & \text{on } \Gamma_N \setminus \overline{\omega_\e}, \\
\gamma \frac{\partial r_\e}{\partial n} + ik r_\e = 0 & \text{on } \Gamma_R, \\
\gamma \frac{\partial r_\e}{\partial n} + ik r_\e = -iku_0 & \text{on } \omega_\e.
\end{array}
\right.
\end{equation}
Like in the proof of \cref{th.Conductivity.HNHD.Expansion}, a priori estimates for \cref{eq.helmre} allow to prove that 
\begin{equation}\label{eq.re0helm}
r_\e \xrightarrow{\e\to0} 0 \text{ in } H^1(\Omega; \C).
\end{equation}
We now proceed in four steps. \par\medskip
 
\noindent \textit{Step 1: We construct a representation formula for the values of $r_\e$ ``far'' from $0$ in terms of its values inside the region $\omega_\e$.}

To this end, we rely on the Green's function $N(x,y)$ of the background problem \cref{eq.helmbg}; it holds:
$$ \begin{array}{>{\displaystyle}cc>{\displaystyle}l}
r_\e(x) & = &-\int_{\Omega} \left( \dv_y(\gamma \nabla_y N(x,y) ) + k^2 N(x,y)\right) r_\e(y) \:\d y\\[1em]
&=& - \int_{\partial \Omega} \gamma \frac{\partial N}{\partial n_y} (x,y) r_\e(y) \:\d s(y) + \int_\Omega \left(\gamma \nabla_y N(x,y) \cdot \nabla r_\e(y) -k^2 N(x,y) r_\e(y)\right) \:\d y \\[1em]
&=& - \int_{\partial \Omega} \gamma \frac{\partial N}{\partial n_y} (x,y) r_\e(y) \:\d s(y) + \int_{\partial \Omega} \gamma \frac{\partial r_\e}{\partial n_y} (y) N(x,y) \:\d s(y),
\end{array}
$$
where the second and third lines follow from integration by parts.
Recalling the boundary conditions satisfied by $r_\e$ and $y\mapsto N(x,y)$ in \cref{eq.helmbg} and \cref{eq.helmGreen}, the above expression simplifies to:
$$ r_\e(x) =  \int_{\omega_\e} \gamma \frac{\partial r_\e}{\partial n_y} (y) N(x,y) \:\d s(y),$$
and so, after rescaling:
\begin{equation}\label{eq.repformhelm}
 r_\e(x) = \int_{\D_1} \varphi_\e(z) N(x,\e z) \:\d s(z), \text{ where } \varphi_\e(z) := \e^{d-1} \left(\gamma \frac{\partial r_\e}{\partial n}\right)(\e z)\in \widetilde{H}^{-1/2}(\D_1).
 \end{equation}
\par\medskip

\noindent \textit{Step 2: We characterize the function $\varphi_\e$ by an integral equation.}

To achieve this, we essentially repeat the calculations from the first step, except that we use the explicit Green's function for the half-space $L_\gamma(x,y)$ discussed in \cref{rem.Nfunchelm} in place of the more abstract Green's function $N(x,y)$ for the background equation \cref{eq.helmbg}. We first obtain: 
$$r_\e(x) = - \int_{\partial \Omega} \gamma \frac{\partial L_\gamma}{\partial n_y} (x,y) r_\e(y) \:\d s(y) + \int_{\partial \Omega} \gamma \frac{\partial r_\e}{\partial n_y} (y) L_\gamma(x,y) \:\d s(y).
$$
Invoking the boundary conditions satisfied by the functions $r_\e$ and $y \mapsto L_\gamma(x,y)$, it follows that: 
$$ r_\e(x) =  \int_{ \omega_\e}\gamma \frac{\partial r_\e}{\partial n_y} L_\gamma(x,y) \:\d s(y) + K r_\e(x),$$
where the term $K r_\e(x)$ gathers integrals of $r_\e$ and its derivatives whose supports are ``far'' from $0$.

We now let $x$ tend to $\omega_\e$ in this formula, and insert the resulting expression in the impedance boundary condition in \cref{eq.helmre}; this yields:
$$ \forall x \in \omega_\e, \quad \gamma \frac{\partial r_\e}{\partial n}(x) + ik\int_{ \omega_\e}\gamma \frac{\partial r_\e}{\partial n_y} L_\gamma(x,y) \:\d s(y) + K r_\e(x) = -ik u_0(x). $$
Finally, the convergence \cref{eq.re0helm} and a change of variables in the above integral yield the desired integral equation for the function $\varphi_\e$ in \cref{eq.repformhelm}:
\begin{equation}\label{eq.inteqHelmholtz}
 \forall x \in \D_1, \quad \frac{1}{\e^{d-1}}\varphi_\e(x) + ik\int_{\D_1} \varphi_\e(z) L(\e x , \e z) \:\d s(z) = -ik u_0(0) + \o(1).
 \end{equation}

\noindent \textit{Step 3: We use the integral equation \cref{eq.inteqHelmholtz} to glean information about the asymptotic behavior of $\varphi_\e$.}

To this end, we observe that, because of the homogeneity of $L(\cdot,\cdot)$, the integral term on the left-hand side of \cref{eq.inteqHelmholtz}
 is of order $\lvert \log \e \lvert$ if $d=2$, and of order $\e^{-(d-2)}$ if $d \geq 3$; it is thus negligible with respect to the first term in the left-hand side of this equation. Hence, by taking the mean value in \cref{eq.inteqHelmholtz}, we immediately obtain the following relations:
$$ \int_{\D_1} \varphi_\e(z) \:\d s(z) = \left\{
\begin{array}{cl}
- 2ik\e u_0(0) + \o(\e) & \text{if } d=2,\\
-\pi \e^{2} ik u_0(0) \pi + \o (\e^2) & \text{if } d =3,
\end{array}\right.
$$
which encompass the needed information for our purpose. 
\par\medskip
\noindent \textit{Step 4: We pass to the limit in the representation formula \cref{eq.repformhelm}.}

The application of the Lebesgue dominated convergence theorem to the representation formula \cref{eq.repformhelm} yields: 
$$ r_\e(x) = \left(\int_{\D_1} \varphi_\e(z) \:\d s(z) \right) \Big(N(x,0) + \o(1) \Big),$$
and the desired result follows immediately from the formulas derived in Step 3.
\end{proof}

%%%%%%%%%%%%%%%%%%%
\subsection{Calculation of the topological derivative of a functional depending on the region $\Gamma_R$}
%%%%%%%%%%%%%%%%%%

\noindent In this section, we cast the previous analysis in the context of shape and topology optimization, 
along the lines of \cref{sec.topderbc}. 
We consider a function $J(\Gamma_R)$ of the region $\Gamma_R \subset \partial \Omega$ bearing the impedance boundary condition of the Helmholtz equation \cref{eq.helmbg}, of the form:
$$ J(\Gamma_R) = \int_\Omega j(u_{\Gamma_R}) \:\d x,$$
where we have denoted by $u_{\Gamma_R}$ the solution to \cref{eq.helmbg}.
Here, $j : \C \to \R$ is smooth when it is seen as a function defined on the real vector space $\R^2$, and it satisfies the growth conditions \cref{eq.jgrowth};
with a small abuse of notation, we let
$$ \forall u = (u_1,u_2) \in \C \approx \R^2, \quad  j^\prime(u) := \frac{\partial j}{\partial u_1}(u) + i \frac{\partial j}{\partial u_2}(u) \in \C.$$
The sensitivity of the function $J(\Gamma_R)$ with respect to the addition of a small surface disk $\omega_{x_0,\e}$ to $\Gamma_R$ is then given by the next result. 

\begin{corollary}
Let $x_0$ be a given point on $\Gamma_N$. The mapping $\e \mapsto J((\Gamma_R)_{x_0,\e})$ has the following asymptotic expansion:
$$J((\Gamma_R)_{x_0,\e}) = J(\Gamma_R) + 2\e k \: \Im\left( \overline{u_0(x_0)}p_0(x_0) \right) + \o(\e) \text{ if } d = 2,$$
and 
$$J((\Gamma_R)_{x_0,\e}) = J(\Gamma_R) + \pi \e^2 k \: \Im\left( \overline{u_0(x_0)}p_0(x_0) \right)  + \o(\e^2) \text{ if } d = 3,$$
where $\Im(u)$ is the imaginary part $u_2$ of a complex number $u=(u_1,u_2) \in \C$ and the adjoint state $p_0 \in H^1(\Omega;\C)$ is the unique solution to the following boundary value problem:
\begin{equation}\label{eq.helmadj}
\left\{
\begin{array}{cl}
-\dv(\gamma \nabla p_0) - k^2 p_0 = -j^\prime(u_0) & \text{in } \Omega, \\
\gamma \frac{\partial p_0}{\partial n} = 0 & \text{on } \Gamma_N,\\
\gamma \frac{\partial p_0}{\partial n} - ikp_0 = 0 & \text{on } \Gamma_R.
\end{array}
\right.
\end{equation}
\end{corollary}
\begin{proof}[Hint of the proof]
The proof is similar to that of \cref{cor.Conductivity.HNHD.Jp}. Considering the case $d=3$ to set ideas,  
the Lebesgue dominated convergence theorem shows that:
$$ J((\Gamma_R)_{x_0,\e})  = J(\Gamma_R) - \pi \e^2 k\Re\left( \int_\Omega j^\prime(u_0(x)) \overline{\Big( i u_0(x_0) N(x,x_0) \Big)}\:\d x \right)  + \o(\e^2). $$
To obtain this formula, we have used the basic calculus rule:
$$\forall u,h \in \C, \quad j(u+h) = j(u) + \Re\big(j^\prime(u) \overline{h}\big) + \o(h), \text{ where } \frac{\o(h)}{\lvert h \lvert} \xrightarrow{h\to 0} 0,$$
and $\Re(u)= u_1$ is the real part of a complex number $u=(u_1,u_2)\in \C$.
Since the complex conjugate $\overline{N(x,y)}$ is the Green's function for the boundary value problem \cref{eq.helmadj} (where the sign in the Robin boundary condition is changed compared to \cref{eq.helmbg}), the representation formula \cref{eq.u0N} for $p_0$ reads:
$$ p_0(x_0) = -\int_\Omega j^\prime(u_0(x)) \overline{N(x,x_0)} \:\d x,$$
and so: 
$$ \begin{array}{>{\displaystyle}cc>{\displaystyle}l}
J((\Gamma_R)_{x_0,\e})  &=& J(\Gamma_R) - \pi \e^2 k \: \Re\left( \overline{i u_0(x_0)} \int_\Omega j^\prime(u_0(x)) \overline{N(x,x_0)}\:\d x \right)  + \o(\e^2) \\
&=& J(\Gamma_R) + \pi \e^2 k \: \Im\left( \overline{u_0(x_0)}p_0(x_0) \right) + \o(\e^2),
\end{array}
$$
which is the desired formula.
\end{proof}