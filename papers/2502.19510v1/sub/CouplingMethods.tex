%%%%%%%%%%%%%%%%%%%%
\subsection{Description of the numerical framework} \label{subsec.Numerical.Framework}
%%%%%%%%%%%%%%%%%%%%

\noindent The practical realization of the general shape and topology optimization \cref{alg.sketchoptbc} for the resolution of \cref{eq.sopb} deserves a few comments, see \cref{alg.CouplingMethods.SurfaceOptimization} below for a more specific sketch of our numerical workflow.
% As in there, we indicate with an $^n$ superscript all the instances of the objects at play at the $n^{\text{th}}$ iteration of the process, $n=0,\ldots$

Optimizing a region $G \subset \partial \Omega$ bearing boundary conditions raises antagonistic needs about the numerical representation of $G$.
On the one hand, at each stage of the optimization process, one needs to solve one or several boundary value problems on $\Omega$, whose boundary conditions are supported by the region $G$. 
Ideally, this operation ought to be conducted on a mesh $\calT$ of $\Omega$ whose associated surface mesh of $\partial\Omega$ encloses an explicit discretization of $G$, see \cref{fig.bfls} (a).
On the other hand, the numerical representation of $G$ should allow to account for possibly dramatic deformations of this region from one optimization iteration to the next in a robust manner, including changes in its topology. This task is notoriously difficult to carry out under a meshed representation.

To reconcile these requirements, we use the level set based mesh evolution strategy introduced in \cite{allaire2011topology,allaire2013mesh,allaire2014shape},
recently adapted to the case of regions on surfaces in \cite{brito2023body}. Briefly, two complementary representations of $G$ are available at each optimization iteration:
\begin{itemize}
\item (Meshed representation) The domain $\Omega$ is equipped with a high-quality simplicial mesh $\calT$, i.e. made of triangles in 2d, tetrahedra in 3d.
Its boundary $\calS$, (a line mesh if $d=2$, a surface triangulation if $d=3$), encloses an explicit discretization of the region $G$ as a submesh $\calS_{\text{int}}$, see \cref{fig.bfls} (a).
\item (Level set representation) The region $G$ is described as the negative subset of a ``level set'' function $\phi : \partial \Omega \to \R$, that is:
$$ \forall x \in \partial \Omega, \quad \left\{
\begin{array}{cl}
\phi(x) < 0 & \text{if } x \in G, \\
\phi(x) = 0 & \text{if } x \in \Sigma, \\
\phi(x) > 0 & \text{otherwise}.
\end{array}
\right.$$
In practice $\phi$ is discretized at the vertices of a surface mesh $\calS$ of $\partial \Omega$,
see \cite{osher1988fronts} for the seminal reference about the level set method, the articles \cite{allaire2004structural,osher2001level,sethian2000structural,wang2003level} about its introduction in the field of shape and topology optimization, and the books \cite{fedkiw2002level,sethian1999level} for further details, and \cref{fig.bfls} (b) for an illustration of this representation.
\end{itemize}


\begin{figure}[ht]
\begin{tabular}{cc}
\begin{minipage}{0.48\textwidth}
\begin{overpic}[width=1.0\textwidth]{figures/bfls1}
\put(2,5){\fcolorbox{black}{white}{a}}
\end{overpic}
\end{minipage}&
\begin{minipage}{0.52\textwidth}
\begin{overpic}[width=1.0\textwidth]{figures/bfls2}
\put(2,5){\fcolorbox{black}{white}{b}}
\end{overpic}
\end{minipage}
\end{tabular}
 \caption{\it (a) Meshed representation of a region $G$ (in blue) of the boundary of a 3d domain $\Omega$; (b) Corresponding representation of $G$ as the negative subdomain of a level set function defined at the vertices of the surface mesh $\calS$ of $\partial \Omega$ induced by the mesh $\calT$ of $\Omega$.}
  \label{fig.bfls}
\end{figure}

Every numerical operation in the process is performed by using the most suitable of these two representations of $G$. Notably, the resolution of boundary value problems on $\Omega$ whose boundary conditions are supported on $G$ and $\partial \Omega \setminus \overline{G}$ takes advantage of the meshed representation of $G$: the Finite Element Method can be conveniently applied on the mesh $\calT$ of $\Omega$, and in principle, any numerical solver could be used to this end, in a black-box fashion. 
On the other hand, the deformation of $G$ according to a tangential velocity field $V : \partial \Omega \to \R^d$ (which is, in our particular context, a descent direction for the considered optimization problem \cref{eq.sopb}) for a (pseudo-) time period $(0,\tau)$ is better accounted for under a level set representation $\phi$ of $G$: it is captured by solving the following advection-like equation on the surface mesh $\calS$ of $\partial\Omega$ \cite{fedkiw2002level,sethian1999level}: 
\begin{equation} \label{eq.CouplingMethods.Advection}
\left\{
\begin{array}{cl}
\frac{\partial \psi}{\partial t}(t,x) + V(t,x) \cdot \nabla_{\partial \Omega} \psi(t,x) = 0  & \text{for } t \in (0,\tau) , \: \: x \in \partial \Omega,\\
\psi(0,x) = \phi(x) &\text{for } x \in \partial \Omega.
\end{array}\right.
\end{equation}
This representation makes it even simpler to account for the addition of a tiny surfacic disk $\omega_{x_0,\e} \subset \partial\Omega$ to $G$: one replaces $\phi$ with the level set function
\begin{equation}\label{eq.LSmin}
 \min(\phi(x),\lvert x - x_0 \lvert-\e), \quad x \in \partial\Omega.
 \end{equation}

Our level set based mesh evolution strategy crucially hinges on a set of efficient numerical algorithms allowing to switch from one of these representations to the other. Notably, a level set function $\phi$ for a region $G \subset \partial \Omega$ available under meshed representation is generated by calculating the signed distance function $d^{\partial \Omega}_G$ to $G$ (see \cref{def.SignedDistance}). Conversely, the creation of a meshed representation of $G$ from the datum of a level set function $\phi : \partial\Omega \to \R$ relies on suitable mesh modification techniques, described in e.g. \cite{balarac2021tetrahedral,dapogny2014three,frey2007mesh}. 

\begin{algorithm}[!ht]
 \caption{Mesh evolution algorithm for the optimal design of a region $G \subset \partial \Omega$.}
    \label{alg.CouplingMethods.SurfaceOptimization}
\begin{algorithmic}[0]
\STATE \textbf{Initialization:}   Simplicial mesh $\mathcal{T}^0$ of $\Omega$ whose associated surface triangulation $\calS^0$ features a submesh  of the initial region $G^0 \subset \partial \Omega$.

\FOR{$n=0,...,$ until convergence}
 \STATE \begin{enumerate}
\item Calculate the state function $u_{G^n}$ on the mesh $\calT^n$.
\item Calculate the adjoint state $p_{G^n}$ on $\calT^n$.
\item Calculate the signed distance function $\phi^n$ to $G^n$ on the surface mesh $\calS^n$. 
\end{enumerate}
\IF{$n \text{ mod. } \ntop = 0$}
\STATE  \begin{enumerate}
  \setcounter{enumi}{3}
                \item Calculate the topological derivative $\d_TJ(G^n)(x)$ at the vertices $x$ of $\calS^n$.
                \item Find the point $x_0 \in \partial \Omega \setminus \overline{G^n}$ where $\d_T J(G^n)$ takes the largest negative value.
                \item Obtain the new level set function $\phi^{n+1}$ from $\phi^n$ by \cref{eq.LSmin}, for a ``small'' parameter $\e>0$.
   \end{enumerate}
\ELSE 
\STATE   \begin{enumerate}
  \setcounter{enumi}{3}
               \item Calculate the shape derivative $J^\prime(G^n)(\theta)$ of $J$, or that $J^\prime_\e(G^n)(\theta)$ of an approximate version, in the spirit of \cref{sec.hepsconduc}.
               \item Find a descent direction $\theta^n$.
               \item Select a time step $\tau^n>0$ and calculate the solution $\psi(t,x)$ to the advection equation \cref{eq.CouplingMethods.Advection} with velocity $V = \theta^n$ on the surface mesh $\calS^n$ over the interval $(0,\tau^n)$; set $\phi^{n+1} = \psi(\tau^n,\cdot)$.
\end{enumerate}
\ENDIF
\STATE   \begin{enumerate}
  \setcounter{enumi}{6}
                \item Modify $\calT^n$ into a mesh $\calT^{n+1}$ of $\Omega^{n+1}$ whose surface $\calS^{n+1}$ contains a submesh $\calS_{\text{int}}^{n+1}$ of $G^{n+1}$.
\end{enumerate}
\ENDFOR
\RETURN Mesh $\calT^n$ of $\Omega$, whose surface part $\calS^n$ contains a submesh $\calS_{\text{int}}^n$ of the optimized region $G^n$.
\end{algorithmic}
\end{algorithm}

The operations involved in this framework are carried out by black-box uses of several open-source libraries developed in our previous works. 
The boundary value problems at stake are assembled and solved with the \texttt{Rodin} open-source library \cite{Brito-Pacheco_Rodin_2023}.
 The signed distance function $d_G$ to a region $G \subset \partial \Omega$ is computed using the Fast Marching Method \cite{sethian1996fast,kimmel1998computing}, implemented in the \texttt{ISCD Mshdist} software \cite{dapogny2012computation}; the solution of the level set evolution equation \cref{eq.CouplingMethods.Advection} on the mesh $\calS$ of $\partial \Omega$ relies on the method of characteristics, implemented in the \texttt{ISCD Advection} software \cite{bui2012accurate}. Eventually, the mesh modification operations of our framework leverage the \texttt{mmg} library \cite{balarac2021tetrahedral,dapogny2014three}. We refer to \cite{dapogny2023shape} for an educational implementation of this framework, in the context of the optimization of a ``bulk'' shape $\Omega \subset \R^d$. 
All our numerical experiments are conducted on a regular laptop \texttt{Apple MacBookPro} 18,3 (M1 Pro chip) with 10 cores and 16 GB of memory.


%%%%%%%%%
\subsection{Numerical resolution of boundary integral equations}\label{sec.BEM}
%%%%%%%%%

\noindent According to \cref{sec.TopologicalSensitivity,sec.Helmholtz,sec.Elasticity}, the determination of the asymptotic sensitivity of the solution to a boundary value problem with respect to the addition of a ``small'' disk $\omega_{x_0,\e}$ to a region $G \subset \partial\Omega$ supporting its boundary conditions raises an integral equation, of the form:
\begin{equation}  \label{eq.BoundaryOptimization.IntegralEquation.Strong}
\tag{\textcolor{gray}{${\mathcal B}$}}
 \text{Search for } \varphi \in \widetilde{H}^{-1/2}(\D_1) \:\text{ s.t. }  \quad T_{L} \varphi (x) = f(x), \quad x\in \D_1,
\end{equation}
where the unknown $\varphi$ is scalar-valued for simplicity of the presentation, and the integral operator  $T_{L} : \widetilde{H}^{-1/2}(\D_1) \rightarrow H^{1/2}(\D_1)$ is defined by:
\begin{equation}
  T_{L} \varphi (x) = \int_{\D_1} L(x, z) \varphi(z) \: \d s(z).
\end{equation}
It involves a homogeneous kernel $L$ of the form of those in \cref{eq.conduc.Gamma,eq.GreenHelmholtz,eq.kelvin,eq.Mindlin1,eq.Mindlin2}, and a known source term $f \in H^{1/2}(\D_1)$.
In fortunate cases, as in electrostatics, the solution to this equation can be computed analytically, leading to a fully explicit asymptotic formula for the state $u_\e$, as in electrostatics, see \cref{th.Conductivity.HNHD.Expansion}. 
However, in several realistic situations, no closed form expression of the solution is available and the latter has to be computed numerically, see notably the case of the linear elasticity system in \cref{th.expelas}.
 
 The numerical resolution of integral equations is extensively discussed in the numerical analysis literature \cite{jerri1999introduction,hackbusch2012integral,kythe2020introduction,katsikadelis2016boundary}. 
 However, the particular situation of an equation set on a ``screen'', i.e. an open subregion of the total boundary of a domain, is much less classical, see \S 4.1.11 in \cite{sauter2011boundary} for a related discussion.
 For completeness, we briefly discuss the use of the Boundary Element Method implemented in our work to solve such equations in the case where the space dimension $d$ equals $3$, referring to \cite{gwinner2018advanced,sauter2011boundary} for details.

%%%%%
\subsubsection{The discrete Galerkin approximation of \cref{eq.BoundaryOptimization.IntegralEquation.Strong}}\label{sec.GalerkinBEM}
%%%%%

\noindent Like the ``classical'' Finite Element Method \cite{ciarlet2002finite,ern2021finite}, the Boundary Element Method is based on a variational formulation of the considered equation \cref{eq.BoundaryOptimization.IntegralEquation.Strong}. 
The latter is obtained by multiplying \cref{eq.BoundaryOptimization.IntegralEquation.Strong} by an arbitrary test function $\psi \in \widetilde{H}^{-1/2}(\D_1)$ and integrating:
\begin{multline} \label{eq.BoundaryOptimization.BIE} \tag{\textcolor{gray}{${\mathcal G}$}}
 \text{Search for } \varphi \in  \widetilde{H}^{-1/2}(\D_1)\text{ s.t. }   \forall \psi \in  \widetilde{H}^{-1/2}(\D_1), \quad a(\varphi, \psi) = \ell(\psi),\\
  \text{ where } a(\varphi,\psi):= \int_{\D_1} T_{L} \varphi (x) \psi(x) \: \d s(x), \text{ and } \ell(\psi) := \int_{\D_1} f(x) \psi(x) \: \d s(x).
\end{multline}
A discrete approximation of the solution $\varphi$ to this problem is then sought in a finite-dimensional subspace of $\widetilde{H}^{-1/2}(\D_1)$.
As regards the latter, we choose the space $V_{\calS}$ of $\mathbb{P}_1$ Lagrange elements on a given surface mesh $\calS$ of $\D_1$ made of $K$ non overlapping and conforming triangles $T_1, \ldots, T_{K}$, 
and $N$ vertices $a_1,\ldots,a_{N}$:
$$ V_{\calS} = \Big\{ v : \D_1 \to \R, \text{ is continuous and } v\lvert_{T_k} \text{ is affine }, \: k=1,\ldots,K \Big\}.$$
As usual, a basis of $V_{\calS}$ is given by the collection $\left\{\varphi_n \right\}_{n=1,\ldots,N}$ of functions defined by the relations:
$$ \text{For all } n=1,\ldots,N, \:\: \varphi_n \text{ is the unique function in } V_{\calS} \text{ s.t. } \: \varphi_n(a_m) = \left\{
\begin{array}{cl}
1 & \text{if } n = m, \\
0 & \text{otherwise}, 
\end{array}
\quad m=1,\ldots,N.
\right.$$
 For any $\varphi, \psi \in V_{\calS}$, we decompose the integral $a(\varphi,\psi)$ as: 
\begin{equation}\label{eq.decompaBEM}
 a(\varphi,\psi) = \sum\limits_{k=1}^K \sum\limits_{l=1}^K a_{kl}(\varphi,\psi), \text{ where }  a_{kl}(\varphi,\psi) := \int_{T_k} \varphi(x)  \:\left( \text{p.v.} \int_{T_l} L(x,z) \psi(z) \:\d s(z) \right) \d s(x),
 \end{equation}
where $\text{p.v.}$ stands for the Cauchy principal value, and all the integrals in the above sum can be proved to be well-defined. 

%%%%%
\subsubsection{Decomposition of singular integrals through relative coordinates} \label{sec.IntegralEquation.RelativeCoordinates}
%%%%%

\noindent The evaluation of $a(\varphi,\psi)$ for given functions $\varphi,\psi \in V_{\calS}$ via the decomposition \cref{eq.decompaBEM} leaves us with the task of approximating the integrals $a_{kl}(\varphi,\psi)$, $k,l=1,\ldots,K$.
To achieve this, we express each triangle $T_k \in \calS$ as the image $T_k = A_k (\widehat{T})$ of the reference simplex 
$$\widehat{T} := \left\{ (x_1,x_2) \in \R^2\: \text{ s.t. } 0\leq x_1 \leq 1 \text{ and } 0 \leq x_2 \leq 1-x_1\right\} \subset \R^2$$ 
via a suitable affine mapping $A_k : \R^2 \to \R^3$. 
For each triangle , a local basis $\left\{b_{k, i}\right\}_{1 \leq i \leq 3}$ of affine functions on $T_k$ is given by:
$$
 b_{k,i} = \widehat{b}_i \circ A_k, $$
  where  the functions $ \widehat{b}_i$ constitute the basis of the first-order polynomial functions on $\widehat{T}$, namely:
$$\forall x = (x_1,x_2) \in \widehat{T}, \quad \widehat{b_1}(x) = x_1, \:\: \widehat{b_2}(x) = x_2, \:\: \widehat{b_3}(x) = 1-x_1-x_2.$$
 Let $\varphi, \psi$ be given functions in $V_{\calS}$, and let us decompose $\varphi$ and $\psi$ on the basis $\left\{ \varphi_n \right\}_{n=1,\ldots,N}$ as:
\begin{equation}
 \varphi (x) = \sum_{n = 1}^N \alpha_{n} \: \varphi_n (x) , \quad  \psi (x) = \sum_{n = 1}^N \beta_{n} \: \varphi_n (x)  \quad x \in \D_1.
\end{equation}
Pulling back the integrals in $a_{kl}(\varphi,\psi) $ onto the reference triangle $\widehat{T}$, we obtain, for $k,l=1,\ldots,K$:
\begin{multline} \label{eq.IntegralEquation.Pullbacks}
    a_{kl}(\varphi,\psi) = \sum^3_{i, \: j = 1} \alpha_{n(k,i)}\beta_{n(l,j)}\hat{I}_{klij}, \text{ where }
    \widehat{I}_{klij} := \int_{\widehat{T}} \mathrm{p.v.} \int_{\widehat{T}} \widehat{L}_{klij}(x, y) \: \d y \:  \: \d x \text{ and }\\
 \widehat{L}_{klij}(x, y)  := \widehat{b}_{j} (x)  \widehat{b}_i(y) \: \Jac(A_k)\: \Jac(A_l) \: L\left( A_{k}(x), A_l(y) \right).
\end{multline}
Here, for $k=1,\ldots,K$ and $i=1,2,3$, we have denoted by $n(k,i) \in \left\{1,\ldots,N \right\}$ the global index of the $i^{\text{th}}$ vertex in the $k^{\text{th}}$ triangle $T_k$. 
The Jacobian determinants of the mappings $A_k :\widehat{T} \subset \R^2 \to \R^3$ are defined by $\Jac(A_k) := \det(\nabla A_k^T \nabla A_k)^{1/2}$, $k=1,\ldots,K$.

Suitable quadrature formulas are then used to approximate these integrals while taking into account the weak singularity of the kernel $L(x,y)$. More precisely, we rely on the regularized coordinate versions of the quantities $\hat{I}_{klij}$ proposed in \S 5.2.4 of \cite{sauter2011boundary} which can be approximated via standard Gaussian quadrature methods, see also \cite{hackbusch1993efficient,theocaris1977numerical,guiggiani1987direct,lachat1976effective} about this issue.

%%%%%
\subsubsection{Regularization of the variational problem}\label{sec.regularizedBEM}
%%%%%

\noindent Unfortunately, the direct resolution of \cref{eq.BoundaryOptimization.BIE} with the above strategy
is plagued with numerical artifacts. These are mainly due to the nature of the considered surface $\D_1$, which is open, contrary to the common setting of integral equations where it is a closed boundary. In such a situation, it is indeed well-known that the solution $\varphi$  blows up near the boundary $\partial \D_1$, 
see \cite{stephan1986boundaryelas,stephan1986boundary,stephan1987boundary} about this general behavior, and the numerical experiments in 
\cite{brito2024shape} about the induced numerical instabilities.

To remedy this issue, we resort to a simple, heuristic procedure. We approximate \cref{eq.BoundaryOptimization.BIE} by the following regularized problem:
\begin{multline}\label{eq.approxBEM}
\text{Search for } \varphi_\eta \in H^1(\D_1) \text{ s.t. } 
    \forall \psi \in  H^1(\D_1), \\
    \eta \int_{\D_1} \nabla_{\D_1} \varphi_\eta(x) \cdot \nabla_{\D_1} \psi(x) \: \d s(x) + \int_{\D_1} T_{L} \varphi_\eta (x) \psi(x) \: \d s(x) = \int_{\D_1} f(x) \psi(x) \: \d s(x).
\end{multline}
where we have slightly perturbed the main bilinear form $a(\cdot,\cdot)$ in \cref{eq.BoundaryOptimization.BIE} with the addition of a regularizing elliptic term, penalized by a ``small'' parameter $\eta > 0$.

In the next section, we test this regularization method on an example.

%%%%%
\subsubsection{An example in the setting of the conductivity equation}\label{sec.BEMconducexample}
%%%%%

\noindent Let us consider the resolution of \cref{eq.BoundaryOptimization.IntegralEquation.Strong} in the context of three-dimensional electrostatics discussed in \cref{sec.asymueconduc}: $L(x,y)$ is the fundamental solution \cref{eq.conduc.Gamma} of the Laplace equation in free space. We set the right-hand side to $f=1$, in which case the solution $\varphi$ has the analytical expression \cref{eq.eqdistlap}, see \cref{fig.BoundaryOptimization.RegPhi} (a). 

\begin{figure}[!ht]
    \centering
    \begin{overpic}[width=0.47\textwidth]{figures/eqdist_exact_sharpened}
    \put(2,5){\fcolorbox{black}{white}{a}}
    \end{overpic}
    \begin{overpic}[width=0.47\textwidth]{figures/eqdist_approx_sharpened}
    \put(2,5){\fcolorbox{black}{white}{b}}
    \end{overpic}
    \caption{\it (a) Exact, and (b) Approximate solution (using $\eta = 1e-5$, $\hsiz = 0.039$) to the boundary integral equation \cref{eq.BoundaryOptimization.IntegralEquation.Strong} in the case of the conductivity equation considered in \cref{sec.BEMconducexample}; a threshold is applied to both functions near $\partial \D_1$ for vizualization, because of the blow up of the exact solution there.}
    \label{fig.BoundaryOptimization.RegPhi}
\end{figure}

We solve the approximate problem \cref{eq.approxBEM} for several values of the mesh size $\hsiz$ (i.e. the average length of an edge in the mesh) and of the regularization parameter $\eta$, see \cref{fig.BoundaryOptimization.RegPhi} (b) for an illustration.
The accuracy of the resolution is evaluated in terms of the following three measures of error:

\begin{enumerate}
    \item The squared residual
    \begin{equation}\label{eq.Rheta}
    \mathcal{R}(\hsiz, \eta) := \int_{\mathcal{T}_h} \left( \frac{1}{4\pi} \int_{\mathcal{T}_h} \frac{1}{|x - y|} \varphi_\eta(y) \, \d s(y) - 1 \right)^2 \, \d s(x) 
    \end{equation}
    evaluates how far the approximate solution $\varphi_\eta$ is from satisfying the original equation \cref{eq.BoundaryOptimization.IntegralEquation.Strong}.
    \item The squared error over the mean value
    \[
    \mathcal{A}(\hsiz, \eta) := \left(\int_{\mathcal{T}_h} \varphi_\eta(x) \, \d s(x) - 8 \right)^2 
    \]
    measures the discrepancy between the mean values of the approximate and exact solutions.
    \item The squared error
    \[
    \mathcal{E}(\hsiz, \eta) := \int_{ \left\{x \in \D_1, \:\: |x| < 0.9 \right\}} \lvert\varphi_\eta(x) - \varphi(x) \lvert^2 \, \d s(x)
    \]
    quantifies the error between the approximate and exact solutions. Note that this error is measured on the subset $ \left\{x \in \D_1, \:\: |x| < 0.9 \right\}$ lying at a distance from the boundary $\partial \D_1$ 
    where the exact solution $\varphi$ blows up. 
\end{enumerate}

We solve \cref{eq.approxBEM} and evaluate these three quantities on 50 uniformly spaced values in the intervals $0.00001 < \eta < 0.1$ and $0.1 < \hsiz < 1$, respectively. 
The results are depicted in terms of both parameters in \cref{fig.errBEMconduc}. As expected, these measures of the error tend to $0$ as $\eta$ and $\hsiz$ go to zero. We also observe that the regularization parameter $\eta$ should not be taken too small when compared to the mesh size $\hsiz$, see notably \cref{fig.errBEMconduc} (b,c). This effect is quite understandable, as the regularizing term in \cref{eq.approxBEM} expresses a diffusion over a zone with radius $\eta^{1/2}$ -- an operation which is numerically unstable when the mesh is too coarse with respect to this length scale.

\begin{figure}[H]
    \centering
\fbox{\begin{overpic}[width=0.33\textwidth]{figures/Rerr.png}
\put(2,5){\fcolorbox{black}{white}{a}}
\end{overpic}}
\fbox{\begin{overpic}[width=0.33\textwidth]{figures/Aerr.png}
\put(2,5){\fcolorbox{black}{white}{b}}
\end{overpic}}
\vspace{1mm}

\fbox{\begin{overpic}[width=0.33\textwidth]{figures/Eerr.png}
\put(2,5){\fcolorbox{black}{white}{c}}
\end{overpic}}
\caption{\it (a) Plot of the numerical resolution error $\mathcal{R}(\hsiz, \eta)$; (b) Plot of the numerical error of the average $\mathcal{A}(\hsiz, \eta)$; (c) Plot of the numerical squared approximation error $\mathcal{E}(\hsiz, \eta)$.}
 \label{fig.errBEMconduc}
\end{figure}

%%%%%
\subsubsection{Extension to vector-valued boundary integral equations}\label{sec.elasBEM}
%%%%%

\noindent The technique presented in the previous sections can be extended to deal with vector-valued integral equations.
To set ideas, let us now consider an integral equation of the form \cref{eq.BoundaryOptimization.IntegralEquation.Strong}, involving the operator $T_L: \widetilde{H}^{-1/2}(\D_1)^3 \to H^{1/2}(\D_1)^3$ defined by:
\begin{equation}
 \forall \varphi \in \widetilde{H}^{-1/2}(\D_1)^3, \quad    T_{L} \varphi (x) = \int_{\D_1} L(x, z) \varphi(z) \: \d s(z), \quad x \in \D_1,
\end{equation}
where the kernel $L(x,z)$ is now a $3\times 3$ matrix-valued function and the right-hand side $f$ belongs to $H^{1/2}(\D_1)^3$. 

An identical strategy to that outlined in \cref{sec.GalerkinBEM,sec.IntegralEquation.RelativeCoordinates,sec.regularizedBEM} allows to treat the present case: a variational formulation is introduced for \cref{eq.BoundaryOptimization.IntegralEquation.Strong}, and the left-hand side of the latter is decomposed onto the triangles $T_k$ of the surface triangulation $\calS$ of $\D_1$, see \cref{eq.IntegralEquation.Pullbacks}. 
In this case the function $\widehat{L}_{klij} : \D_1 \times \D_1 \rightarrow \mathbb{R}^d$ defined in \cref{eq.IntegralEquation.Pullbacks} is written in terms of the matrix product:
\begin{equation}
     \widehat{L}_{klij} (x, y) =  \Jac(A_k) \: \Jac(A_l) \:\left[ L\left( A_k(x), A_l(y) \right) \hat{b}_i(y) \right] \cdot \hat{b}_{j} (x) ,
\end{equation}
and the same quadrature rules as in the scalar case can be used for the calculation of the integrals $\widehat{I}_{klij}$. In this situation also, the numerical resolution is riddled with numerical artifacts, caused by the blow up of the solution near the boundary $\partial \D_1$: a similar regularization strategy as that proposed in \cref{sec.regularizedBEM} is used to guarantee its stability. \par\medskip

Let us discuss an example of this methodology in the linear elasticity context of \cref{sec.Elasticity}: we aim to calculate the entries of the polarization tensor $M$ in \cref{eq.defMelas}, featured in the asymptotic expansion of the perturbed displacement by the addition of a ``small'' surface disk to the region of $\partial \Omega$ supporting Dirichlet boundary conditions, see \cref{th.expelas}. We consider the equation \cref{eq.BoundaryOptimization.IntegralEquation.Strong}
 where the operator $T_L$ is associated to the 3d Mindlin kernel $L(x,y)$ with physical parameters $\mu = 67.5676$ and $\nu = 0.48$, see \cref{eq.Mindlin1,eq.Mindlin2}. The right-hand side $f$ is one of the three coordinate vectors $e_i \in \R^3$, $i=1,2,3$.
The regularizing bilinear form chosen in this situation is that associated to the linearized elasticity system, i.e. we solve the following variational problem:
\begin{multline}\label{eq.IEElas}
\text{Search for } \varphi_\eta \in H^1(\D_1)^3 \text{ s.t. } 
    \forall \psi \in  H^1(\D_1)^3, \\
    \eta \int_{\D_1} Ae_{\D_1}(\varphi_\eta) : e_{\D_1}(\psi) \: \d s(x) + \int_{\D_1} T_{L} \varphi_\eta (x) \cdot \psi(x) \: \d s(x) = \int_{\D_1} f(x) \cdot \psi(x) \: \d s(x).
\end{multline}

We solve this equation for $i=1,2,3$ and various values of the mesh size $\hsiz$ and regularization coefficient $\eta$. For each of these parameters, 
we take 50 uniformly spaced values within the intervals $0.5 < \hsiz < 1$ and $0.00001 < \eta < 0.1$,
and we evaluate the accuracy of the solution by computing the resolution error $\mathcal{R}(\hsiz,\eta)$ in \cref{eq.Rheta}. The error plots are presented in \cref{fig.IntegralEquation.ErrorElas}, see also \cref{fig.IntegralEquation.ElasSolutions} for an illustration of the solution.

The error behaves as in the previous \cref{sec.BEMconducexample}. As both parameters $\hsiz$ and $\eta$ decrease, so does the quantity $\mathcal{R}(\hsiz,\eta)$. Also, as expected due to the symmetry of the physical configuration, the solutions associated to the right-hand sides $e_1$ and $e_2$ are identical up to a $90^{\circ}$ rotation. On the other hand, the solution associated to the right-hand side $e_3$ differs from the previous two: it is totally symmetric, see \cref{fig.BoundaryOptimization.RegPhi}.

\begin{figure}[H]
    \centering
    \fbox{\begin{overpic}[width=0.33\textwidth]{figures/Rerr_e1.png}
    \put(2,5){\fcolorbox{black}{white}{a}}
    \end{overpic}}
    \fbox{\begin{overpic}[width=0.33\textwidth]{figures/Rerr_e2.png}
    \put(2,5){\fcolorbox{black}{white}{b}}
    \end{overpic}}
    \vspace{1mm}
    
    \fbox{\begin{overpic}[width=0.33\textwidth]{figures/Rerr_e3.png}
    \put(2,5){\fcolorbox{black}{white}{c}}
    \end{overpic}} 
 \caption{\it Plot of the error $\mathcal{R}(\hsiz, \eta)$ in \cref{eq.Rheta} committed in the solution of the integral equation \cref{eq.IEElas} when the right-hand side $f$ equals (a) $e_1$; (b) $e_2$; (c) $e_3$; the coordinates axes for $\hsiz$, $\eta$ and  $\mathcal{R}(\hsiz, \eta)$ are expressed in logarithmic scale.}
     \label{fig.IntegralEquation.ErrorElas}
\end{figure}
\begin{figure}[H]
    \centering
    \fbox{\begin{overpic}[width=0.33\textwidth]{figures/Elas_e1.jpg}
    \put(2,5){\fcolorbox{black}{white}{a}}
    \end{overpic}}
    \fbox{\begin{overpic}[width=0.33\textwidth]{figures/Elas_e2.jpg}
    \put(2,5){\fcolorbox{black}{white}{b}}
    \end{overpic}}
    \vspace{1mm}

    \fbox{\begin{overpic}[width=0.33\textwidth]{figures/Elas_e3.jpg}
    \put(2,5){\fcolorbox{black}{white}{c}}
    \end{overpic}}
    \caption{\it Plot of the solution to the integral equation \cref{eq.IEElas} in the linear elasticity setting of \cref{sec.elasBEM} when the right-hand side $f$ equals (a) $e_1$; (b) $e_2$; (c) $e_3$; the numerical parameters of the computation are $\hsiz = 0.039$, $\eta = 0.00001$.}
    \label{fig.IntegralEquation.ElasSolutions}
\end{figure}



