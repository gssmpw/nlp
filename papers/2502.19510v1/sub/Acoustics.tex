%%%%%%%%%%%%%%%%%
\FloatBarrier
\subsection{Optimization of absorbent regions on the surface of a sound-hard obstacle for acoustic cloaking} \label{sec.Acoustics}
%%%%%%%%%%%%%%%%%


\noindent In acoustics, scattering occurs when an incident wave hits an obstacle $\Omega$ or a discontinuity of the material parameters in the propagation medium.
 This interaction induces a scattered wave by deflection, absorption, or transmission in various directions, which is testament to the presence of the obstacle. In medical applications, one aims to exacerbate this effect in order to probe a patient's body, see e.g. \cite{adler2021electrical} about tomography imaging methods used for medical diagnosis. 
On the contrary, scattering is highly undesirable in military applications, where one would rather try to hide the obstacle $\Omega$, which represents for instance a stealth submarine, or an aircraft. 

In the latter type of situations, acoustic cloaking techniques have been developed to make an obstacle invisible to an incident wave, 
i.e. to attenuate or suppress the wave scattered by the obstacle. Among them, passive techniques consist in surrounding the obstacle with a shell made of a metamaterial with very specific properties, see e.g. \cite{bryan2010impedance,greenleaf2009cloaking,greenleaf2009invisibility,kohn2008cloaking,griesmaier2014enhanced,popa2009cloaking,xi2009route}, or \cite{norris2008acoustic} for a review.
Recent contributions such as \cite{yang2022acoustic, fujii2021acoustic, ma2019design} have proposed density-based topology optimization strategies for the distribution of such a material around $\Omega$.
In this section, we revisit the design of a cloaking mechanism for an obstacle $\Omega$ from a slightly different perspective: we aim to make $\Omega$ invisible to detection by optimizing the arrangement of constituent materials on its boundary rather than the properties of the surrounding region.

%%%%%%%%%%%%
\subsubsection{Presentation of the scattering problem and of the optimization problem}
%%%%%%%%%%%%

\noindent Our physical model is based on that proposed in \cite{cominelli2022design,alves2016frequency}, see also \cref{fig.BoundaryOptimization.Acoustics.Acoustics_Setup} for an illustration. 
Let $\Omega \subset \mathbb{R}^3$ be a smooth obstacle immersed in a fluid with specific volume $\gamma \in \calC^\infty(\R^3)$; this function is uniformly bounded away from $0$ and $\infty$, as in \cref{eq.bdgamma}.
 In the time-harmonic regime, we assume that an incident wave with wave number $k$ and complex-valued amplitude $\uin(x)$ passes through the medium, 
 satisfying the Helmholtz equation (postulating a time dependency of the form $e^{-ik t}$): 
\begin{equation*}
    -\dv \left( \gamma(x) \nabla \uin(x) \right) - k^2 \uin(x) =0, \quad x \in \mathbb{R}^3 .
\end{equation*}
Denoting by $u_{\Gamma_R}(x)$ and $\utot(x) = \uin(x) + u_{\Gamma_R}(x)$ the scattered and total amplitudes of the sound pressure field, respectively, 
it holds: 
\begin{equation*} \label{eq.BoundaryOptimization.Acoustics.HelmholtzFreeSpace}
    -\dv \left( \gamma(x) \nabla \utot(x) \right) - k^2 \utot(x) = 0, \quad x \in \mathbb{R}^3 \setminus \overline{\Omega},
\end{equation*}
whence the following equation for the scattered wave: 
\begin{equation*}
    -\dv \left( \gamma(x) \nabla u_{\Gamma_R}(x) \right) - k^2 u_{\Gamma_R}(x) = 0, \quad  x \in \mathbb{R}^3 \setminus \overline{\Omega}.
\end{equation*}
The latter is complemented with the Sommerfeld radiation condition at infinity, expressing that the scattered field is outgoing:
\begin{equation} \label{eq.BoundaryOptimization.Acoustics.Sommerfield}
    \lim_{r \rightarrow \infty} r \left( \frac{\partial u_{\Gamma_R}}{\partial r}(x) - i k u_{\Gamma_R}(x) \right) = 0,
\end{equation}
where $r = \vert  x \lvert $ is the radial spherical coordinate. 

In practice, it is necessary to replace the infinite, free space where the wave is propagating by a bounded computational box $D$,
and to somehow impose suitable conditions near its boundary $\partial D$ which mimick the outgoing behavior \cref{eq.BoundaryOptimization.Acoustics.Sommerfield}.
To achieve this, we rely on the simple method proposed in \cite{shirron1998comparison}: we consider an artificial boundary $\Gamma_E \subset D$ around the obstacle $\Omega$; an approximation for \cref{eq.BoundaryOptimization.Acoustics.Sommerfield} is then obtained from a first-order approximation of the scattered field $u(x)$ ``far away'' from $\Omega$:
\begin{equation} \label{eq.BoundaryOptimization.Acoustics.SommerfieldApprox}
    \left[\gamma \dfrac{\partial u_{\Gamma_R}}{\partial n}\right] - \left( i k - \dfrac{1}{R} \right) u_{\Gamma_R} = 0 \text{ on } \Gamma_E,
\end{equation}
where
$$[\alpha](x) := \lim\limits_{t \to 0 \atop t>0} \alpha(x+tn(x)) - \lim\limits_{t \to 0 \atop t>0} \alpha(x-tn(x)), \quad x \in \Gamma_E,$$ 
denotes the jump of a quantity $\alpha$ which is discontinuous across $\Gamma_E$, and $R$ is the distance between $\Gamma_E$ and $\partial D$. 
Admittedly, more advanced numerical methods are available to impose the outgoing behavior \cref{eq.BoundaryOptimization.Acoustics.Sommerfield} to $u_{\Gamma_R}$, such as the well-known Perfectly Matched Layer method \cite{berenger1994perfectly}. \par\medskip
 
 The boundary of the obstacle $\partial \Omega$ is decomposed into two disjoint parts,
$$\partial \Omega = \overline{\Gamma_R} \cup \overline{\Gamma_N},$$
where:
\begin{itemize}
    \item The region $\Gamma_N$ is covered with a ``sound-hard'' material, i.e. the incident wave $\uin$ is entirely reflected on this region, 
    which translates as an inhomogeneous Neumann condition for the scattered field $u_{\Gamma_R}$:
    \begin{equation*}
        \gamma \dfrac{\partial u_{\Gamma_R}}{\partial n} = - \gamma \dfrac{\partial \uin}{\partial n}, \quad x \in \Gamma_N,
    \end{equation*}
    where the unit normal vector $n$ to $\partial \Omega$ is pointing outward the obstacle $\Omega$. 
    
    \item The region $\Gamma_R$ is made of an absorbent material. The incoming wave is partially absorbed in there, and the following impedance boundary condition is satisfied:
    \begin{equation*}
        \gamma \dfrac{\partial u_{\Gamma_R}}{\partial n} + \dfrac{ik}{Z} u_{\Gamma_R} = - \gamma \dfrac{\partial \uin}{\partial n} - \dfrac{ik}{Z} \uin, \quad x \in \Gamma_N,
    \end{equation*}
    where $Z > 0$ is the acoustic impedance of the material.
\end{itemize}

\begin{figure}[ht]
    \centering
    \includegraphics[width=0.8\textwidth]{figures/aircraftprincip}
    \caption{\it Illustration of the acoustic cloaking setting considered in \cref{sec.Acoustics}.}
    \label{fig.BoundaryOptimization.Acoustics.Acoustics_Setup}
\end{figure}


Gathering the above information, the scattered field $u_{\Gamma_R} \in H^1(\Omega; \C)$ is the unique complex-valued solution to the following boundary value problem:
\begin{equation} \label{eq.BoundaryOptimization.Acoustics.Scattering}
    \left\{
    \begin{array}{cl}
        - \dv (\gamma \nabla u_{\Gamma_R}) - k^2 u_{\Gamma_R} = 0 & \text{in } D \setminus \overline{\Omega}\\
        \gamma \frac{\partial u_{\Gamma_R}}{\partial n} = - \gamma \frac{\partial \uin}{\partial n} & \text{on } \Gamma_N,\\
        \gamma \frac{\partial u_{\Gamma_R}}{\partial n} + \frac{ik}{Z} u_{\Gamma_R} = - \gamma \frac{\partial \uin}{\partial n} - \frac{ik}{Z} \uin & \text{on } \Gamma_R,\\
        \left[\gamma \frac{\partial u_{\Gamma_R}}{\partial n}\right] - \left( i k - \frac{1}{R} \right) u_{\Gamma_R} = 0 & \text{on } \Gamma_E ,\\
        \gamma \frac{\partial u_{\Gamma_R}}{\partial n} = 0 & \text{on } \partial D.
    \end{array}
    \right.
\end{equation}
The attached variational formulation reads: 
\begin{multline*} 
\text{Search for } u_{\Gamma_R} \in H^1(\Omega;\C) \text{ s.t. } \forall v \in H^1(\Omega;\C), \\
    \int_{D \setminus \overline{\Omega}} \gamma \nabla u_{\Gamma_R} \cdot \overline{\nabla v} \: \d x
    - k^2 \int_{D \setminus \overline{\Omega}} u_{\Gamma_R} \overline{v} \: \d x
    - \left( i k - \dfrac{1}{R} \right) \int_{\Gamma_E} u_{\Gamma_R} \overline{v} \: \d s
    - \dfrac{i k}{Z} \int_{\Gamma_R} u_{\Gamma_R} \overline{v} \: \d s
    =
    \dfrac{i k}{Z} \int_{\Gamma_R} \uin \overline{v} \: \d s + \int_{\partial \Omega} \gamma \dfrac{\partial \uin}{\partial n} \overline{v} \: \d s,
\end{multline*}
see \cref{rem.wellposedHelmholtz} about the well-posedness of this problem. 

 In this setting, we aim to optimize the repartition of the absorbent and sound-hard regions $\Gamma_R$, $\Gamma_N$ within $\partial \Omega$ so as to minimize the amplitude of the scattered wave $u_{\Gamma_R}$ while keeping the amount of used absorbent material reasonably low;
we thus consider the following shape and topology optimization problem: 
\begin{equation} \label{eq.BoundaryOptimization.Acoustics.Criterion}
    \min_{\Gamma_R \subset \partial \Omega} J(\Gamma_R) + \ell \: \Area(\Gamma_R) , \text{ where } J(\Gamma_R) := 
    \dfrac{1}{2 \Vol(D\setminus \overline{\Omega})}\int_{D \setminus \overline\Omega} \lvert u_{\Gamma_R} \lvert ^2 \: \d x  ,
\end{equation}
and $\ell > 0$ is a penalization parameter.

%%%%%%%%%%%%
\subsubsection{Shape derivative of the objective function $J(\Gamma_R)$}
%%%%%%%%%%%%

\noindent The solution $u_{\Gamma_R}$ to the present boundary value problem \cref{eq.BoundaryOptimization.Acoustics.Scattering} does not show such a weakly singular behavior as the function $u_{\Gamma_C,\Gamma_A}$ considered in \cref{sec.CathodeAnode}, where a transition between homogeneous Dirichlet and Neumann boundary conditions is at play. The shape derivative of the objective function $J(\Gamma_R)$ in \cref{eq.BoundaryOptimization.Acoustics.Criterion} can be calculated by standard means, and we omit the details for brevity, see \cite{brito2024shape} about this point.

\begin{proposition} \label{prop.Acoustics.ShapeDerivative}
    The criterion $J(\Gamma_R)$ is shape differentiable at $\theta = 0$ and its shape derivative reads, for any tangential deformation $\theta$:
    \begin{equation*}
        J'(\Gamma_R)(\theta) = -\Im \left( \int_{\Sigma_R} \dfrac{k}{Z} \overline{(u_{\Gamma_R} + \uin)} p_{\Gamma_R} \: \theta \cdot n_{\Sigma_R} \: \d \ell \right),
    \end{equation*}
    where the adjoint state $p_{\Gamma_R} \in H^1(\Omega;\C)$ is the unique (complex-valued) solution to the following boundary value problem:
    \begin{equation}\label{eq.adjHelmholtz}
    \left\{
    \begin{array}{cl}
        - \dv (\gamma \nabla p_{\Gamma_R}) - k^2 p_{\Gamma_R} = -\frac{u_{\Gamma_R}}{\Vol(D\setminus \overline\Omega)} & \text{in } D \setminus \overline{\Omega} \\
        \gamma \frac{\partial p_{\Gamma_R}}{\partial n} = 0 & \text{on } \Gamma_N,\\
        \gamma \frac{\partial p_{\Gamma_R}}{\partial n} - \frac{ik}{Z} p_{\Gamma_R} = 0 & \text{on } \Gamma_R,\\
        \left[\gamma \frac{\partial p_{\Gamma_R}}{\partial n}\right] - \left( -i k - \frac{1}{R} \right) p_{\Gamma_R} = 0 & \text{on } \Gamma_E ,\\
        \gamma \frac{\partial p_{\Gamma_R}}{\partial n} =0 & \text{on } \partial D.
    \end{array}
    \right.
\end{equation}
\end{proposition}
\par\medskip
%%%%%%%%%%%%
\subsubsection{Topological derivative of the functional $J(\Gamma_R)$}
%%%%%%%%%%%%

\noindent The setting considered here is slightly different from the model of \cref{sec.Helmholtz}, insofar as the boundary-value problem \cref{eq.BoundaryOptimization.Acoustics.Scattering} is posed on the exterior of $\Omega$, and does not involve a homogeneous Dirichlet region. Nevertheless, a simple variation of the proof of \cref{sec.asymHelmh} allows to calculate the topological derivative of $J(\Gamma_R)$.

\begin{proposition} \label{prop.Acoustics.TopoDerivative}
    Let $x_0 \in \Gamma_N$; the perturbed criterion $J((\Gamma_R)_{x_0,\e})$ has the following asymptotic expansion:
    $$
    J((\Gamma_R)_{x_0,\e}) = J(\Gamma_R) + \pi \e^2 k \: \Im\left( \overline{(u_{\Gamma_R}(x_0) + \uin(x_0))}p_{\Gamma_R}(x_0) \right)  + \o(\e^2) \text{ if } d = 3,
    $$
    where the adjoint state $p_{\Gamma_R} \in H^1(\Omega;\C)$ is the complex-valued solution to \cref{eq.adjHelmholtz}.
\end{proposition}\par\medskip

%%%%%%%%%%%%
\subsubsection{Setup and analysis of the numerical results}
%%%%%%%%%%%%

\noindent The obstacle $\Omega$ is an aircraft embedded in a large computational box $D$. 
The specific volume of the surrounding air is $\gamma \equiv 1$; the wave number is $k = 2\pi / \lambda$, where the wavelength $\lambda$ equals $20$.
The considered incoming wave $\uin$ is a plane wave, travelling vertically in the direction $\xi = (0, 0, 1) \in \mathbb{R}^3$, that is:
\begin{equation*}
    \forall x \in \mathbb{R}^3, \quad \uin(x) = e^{i k \xi \cdot x}.
\end{equation*}
The domain $D$ is equipped with a tetrahedral mesh $\calT$, which consistently encloses a submesh $\mathcal{K}$ of the obstacle, see \cref{fig.BoundaryOptimization.Acoustics.Aircraft}. 
At every iteration of the optimization process, the lengths of the edges of $\calK$ range from $\hmax = \lambda/3$ to $\hmin = \lambda/32$, thus ensuring a sufficient resolution of the acoustic phenomenon at play \cite{marburg2008discretization}. For instance, the initial mesh $\calT^0$ consists of about $334,000$ vertices, with $1,970,000$ tetrahedra, and the submesh $\calK^0$ of the aircraft consists of $93,000$ vertices and $464,000$ tetrahedra, see \cref{tab.Acoustics.Parameters} for a summary of the parameters of the experiment. 

\begin{figure}[H]
    \centering
        \begin{tabular}{cc}
\begin{minipage}{0.45\textwidth}
\begin{overpic}[width=1.0\textwidth]{figures/Aircraft_Setup}
\put(0,0){\fcolorbox{black}{white}{a}}
\end{overpic}
\end{minipage} & 
\begin{minipage}{0.45\textwidth}
\begin{overpic}[width=1.0\textwidth]{figures/PML.jpg}
\put(0,0){\fcolorbox{black}{white}{b}}
\end{overpic}
\end{minipage}
\end{tabular}   
    \caption{\it (a) Mesh ${\mathcal K}$ of the initial design of the aircraft in \cref{sec.Acoustics}, whose boundary is made only of sound-hard material; (b) Computational mesh $\calT$ of the computational domain $D$, enclosing the fictitious surface $\Gamma_E$ bearing the approximate radiation conditions \cref{eq.BoundaryOptimization.Acoustics.SommerfieldApprox} and the mesh $\calK$ of the aircraft.}
    \label{fig.BoundaryOptimization.Acoustics.Aircraft}
\end{figure}

We solve the problem \cref{eq.BoundaryOptimization.Acoustics.Criterion} with \cref{alg.CouplingMethods.SurfaceOptimization}, starting from a situation where the boundary $\partial \Omega$ is only made of sound-hard material: $\Gamma_R^0 = \emptyset$.
The optimization process starts with 6 iterations during which new connected components are added to $\Gamma_R$ according to the information contained in the topological derivative of \cref{prop.Acoustics.TopoDerivative}. These are followed by 4 iterations of motion via boundary variation, based on the result of \cref{prop.Acoustics.ShapeDerivative}. After the 10$^{\text{th}}$ iteration, we carry out one topological update of $\Gamma_R$ every 10 iterations, intertwined with updates of its geometry, up to the stage $n= 100$, after which only geometrical updates are considered. 

\begin{table}[ht]
    \centering
    \begin{tabular}{|c|c|}
        \hline
        Parameter & Value\\
        \hline
        $\ell$ & $1e-7$\\
        $\lambda$ & $20$\\
        $k$ & $2 \pi / \lambda$\\
        $\hmax$ & $\lambda / 3$\\
        $\hmin$ & $\lambda / 32$\\
        \hline
    \end{tabular}
    \caption{\it Numerical parameters used in the optimization process of \cref{sec.Acoustics}.}
    \label{tab.Acoustics.Parameters}
\end{table}

A few snapshots of the evolution process are shown in \cref{fig.BoundaryOptimization.Acoustics.Aircraft}, and the convergence history is reported in \cref{fig.Acoustics.Final_Perspective} (d); the total computational time is about 19 hours. The first snapshot, associated to the iteration $n = 6$, displays the design after introducing 6 absorbent regions, which are located on the wings and the center of the aircraft. By iteration $n=10$, the two central zones have vanished. As the optimization proceeds, the components added during the topological updates are systematically removed during the geometric optimization updates. The design gradually develops homogenization patterns, similar to those observed in \cref{sec.CathodeAnode}; the absorbent region develops new connected components by ``stretching and cutting off'' some of its regions, rather than by the addition of new surface disks during the topological update steps.

The optimized design, attained at iteration $n=142$, is illustrated in \cref{fig.Acoustics.Final_Perspective} (a,b,c): it suggests that the absorbent material should preferably be concentrated near the wings of the aircraft. 
This trend is quite intuitive, since these regions are perpendicular to the direction $\xi = (0, 0, 1)$ of the incident wave, which causes them to reflect the latter strongly. This fact is confirmed by examination of the magnitude of the scattered pressure field, which is displayed in \cref{fig.Acoustics.CrossSection_Front}. Obviously, in the optimized design, the pressure field $u$ has been significantly reduced near the wings, making them less visible. The field has now the largest magnitude near the aircraft's rear wings, its value being twice smaller compared to that in \cref{fig.Acoustics.CrossSection_Front} (b). However, we believe that no absorbent material was added there due to the imposed penalization on the area of $\Gamma_R$, which urges the algorithm to concentrate the addition of absorbent material on the wings.

\begin{figure}[ht]
    \centering

\begin{tabular}{cc}
\begin{minipage}{0.5\textwidth}
\begin{overpic}[width=1.0\textwidth]{figures/Acoustics_6}
\put(0,0){\fcolorbox{black}{white}{$n=6$}}
\end{overpic}
\end{minipage} & 
\begin{minipage}{0.5\textwidth}
\begin{overpic}[width=1.0\textwidth]{figures/Acoustics_10}
\put(0,0){\fcolorbox{black}{white}{$n=10$}}
\end{overpic}
\end{minipage}
\end{tabular}   \par\medskip 

        \begin{tabular}{cc}
\begin{minipage}{0.5\textwidth}
\begin{overpic}[width=1.0\textwidth]{figures/Acoustics_20}
\put(0,0){\fcolorbox{black}{white}{$n=20$}}
\end{overpic}
\end{minipage} & 
\begin{minipage}{0.5\textwidth}
\begin{overpic}[width=1.0\textwidth]{figures/Acoustics_40}
\put(0,0){\fcolorbox{black}{white}{$n=40$}}
\end{overpic}
\end{minipage}
\end{tabular}
\caption{\it Snapshots of the optimization process of the distribution of absorbent (in pink) and ``sound-hard'' (in white) materials on the boundary of an aircraft (\cref{sec.Acoustics}).}
\label{fig.BoundaryOptimization.Acoustics.Aircraft}
\end{figure}
\begin{figure}[ht]
\ContinuedFloat
\centering
\begin{tabular}{cc}
\begin{minipage}{0.5\textwidth}
\begin{overpic}[width=1.0\textwidth]{figures/Acoustics_80}
\put(0,0){\fcolorbox{black}{white}{$n=80$}}
\end{overpic}
\end{minipage} & 
\begin{minipage}{0.5\textwidth}
\begin{overpic}[width=1.0\textwidth]{figures/Acoustics_120}
\put(0,0){\fcolorbox{black}{white}{$n=120$}}
\end{overpic}
\end{minipage}
\end{tabular}
\caption{(cont.) \it Snapshots of the optimization process of the distribution of ``sound-soft'' (in pink) and ``sound-hard'' (in white) materials on the boundary of an aircraft (\cref{sec.Acoustics}).}
\end{figure}
\begin{figure}[H]
    \centering
             \begin{tabular}{cc}
\begin{minipage}{0.5\textwidth}
\centering
\begin{overpic}[width=0.8\textwidth]{figures/Acoustics_142_TopBottom.jpg}
\put(-17,5){\fcolorbox{black}{white}{a}}
\end{overpic}
\end{minipage}
 & 
\begin{minipage}{0.5\textwidth}
\begin{overpic}[width=0.9\textwidth]{figures/Acoustics_142_Perspective.jpg}
\put(2,5){\fcolorbox{black}{white}{b}}
\end{overpic}
\end{minipage}
\end{tabular}
\caption{\it (a,b,c) Several  views of the optimized distribution of absorbent and ``sound-hard'' materials on the boundary of the aircraft considered in \cref{sec.Acoustics}; (d) History of the values of $J(\Gamma_R)$ during the computation.}
\label{fig.Acoustics.Final_Perspective}
\end{figure}
\begin{figure}[ht]
\ContinuedFloat
\centering
 \begin{tabular}{cc}
\begin{minipage}{0.5\textwidth}
\begin{overpic}[width=1.0\textwidth]{figures/Acoustics_142_Sides.jpg}
\put(2,5){\fcolorbox{black}{white}{c}}
\end{overpic}
\end{minipage} & 
\begin{minipage}{0.45\textwidth}
\begin{overpic}[width=1.0\textwidth]{figures/Acoustics_Obj}
\put(2,5){\fcolorbox{black}{white}{d}}
\end{overpic}
\end{minipage}
\end{tabular}
\caption{(cont.) \it (a,b,c) Several  views of the optimized distribution of absorbent and ``sound-hard'' materials on the boundary of the aircraft considered in \cref{sec.Acoustics}; (d) History of the values of $J(\Gamma_R)$ during the computation.}
\end{figure}

\begin{figure}[ht]
    \centering
            \begin{tabular}{cc}
\begin{minipage}{0.5\textwidth}
\begin{overpic}[width=1.0\textwidth]{figures/CrossSection_1}
\put(4,9){\fcolorbox{black}{white}{a}}
\end{overpic}
\end{minipage} & 
\begin{minipage}{0.5\textwidth}
\begin{overpic}[width=1.0\textwidth]{figures/CrossSection_3}
\put(4,9){\fcolorbox{black}{white}{b}}
\end{overpic}
\end{minipage}
\end{tabular}   \par\medskip 
            \begin{tabular}{cc}
\begin{minipage}{0.5\textwidth}
\begin{overpic}[width=1.0\textwidth]{figures/CrossSection_2}
\put(4,13){\fcolorbox{black}{white}{c}}
\end{overpic}
\end{minipage} & 
\begin{minipage}{0.5\textwidth}
\begin{overpic}[width=1.0\textwidth]{figures/CrossSection_4}
\put(4,13){\fcolorbox{black}{white}{d}}
\end{overpic}
\end{minipage}
\end{tabular}  
    \caption{\it Magnitude of the scattered pressure field, with (a,b) Only a ``sound-hard'' boundary; (c,d) The optimized repartition of absorbent and ``sound-hard'' materials on the boundary of the aircraft in the example of \cref{sec.Acoustics}.}
    \label{fig.Acoustics.CrossSection_Front}
\end{figure}