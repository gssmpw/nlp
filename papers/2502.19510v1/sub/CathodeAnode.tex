%%%%%%%%%%%%%%%%%
\FloatBarrier
\subsection{Optimization of the repartition of the cathode and anode regions on the boundary of a direct current electroosmotic mixer} \label{sec.CathodeAnode}
%%%%%%%%%%%%%%%%%

\noindent Our first example of optimal design problems featuring regions supporting boundary condition regions arises in the field of microfluidics, 
which deals with the handling of very small volumes of fluid, ranging between $10^{-18}$ and $10^{-9}$ L. 
This field has been notably active over the past decade. It notably heralds decisive advances in analytical chemistry, molecular biology and biomedical engineering (with the design of biochips and DNA micro-arrays and the possibility to realize electrophoresis and liquid chromatography for proteins and DNA), in optics, where it opens the way to the design of microlens arrays, etc. We refer to e.g. \cite{nguyen2019fundamentals,stone2001microfluidics} for comprehensive introductions to this subject and its challenges.

One basic operation of particular interest in microfluidics consists in mixing two fluids by electroosmosis \cite{chen2003numerical,lai2024review,qian2002chaotic,sasaki2006ac,zhang2006electro}.
This process typically takes place in a so-called electroosmotic mixer: in broad outline, two fluids are injected through an entrance channel, and an electric field is imposed inside the ring-shaped mixing chamber 
thanks to a suitable placement of anodes (positively charged electrodes) and cathodes (negatively charged electrodes) on its boundary. The induced Coulomb force electrically actuates the charged particles and ions and triggers the pumping of electrolytic fluid through drag forces. The more intense the electric field inside the chamber, the more efficient the mixing process. The resulting mixture finally leaves the device through an exit channel, see \cref{fig.CathodeAnode.EMM_Schematic}. 

\begin{figure}[H] 
    \centering    
    \includegraphics[width=0.6\textwidth]{figures/EMM_Schematic}
    \caption{\it Illustration of the electrosmootic mixer considered in \cref{sec.CathodeAnode}: two fluids enter via the inflow channel, get mixed in the central chamber under the action of the difference between the voltage potentials applied on the anode and cathode regions; the resulting mixture exits through the outflow channel.}
    \label{fig.CathodeAnode.EMM_Schematic}
\end{figure}

In this section, we seek to optimize the design of such an electroosmotic mixer.
Our study relies on a simplified version of the physical model presented in \cite{deng2018topology}. We aim to optimize the distribution of anodes and cathodes on the boundary of the mixing chamber so as to generate an electric field with maximum amplitude inside the device, and thereby improve the efficiency of the mixer. 

%%%%%%%%
\subsubsection{Presentation of the optimization problem}\label{sec.CathodeAnodeOptPb}
%%%%%%%

\noindent The optimal design problem of this section arises in the mathematical setting of the conductivity equation, see \cref{sec.setconduc}. The electroosmotic mixer is represented by a bounded domain $\Omega \subset \mathbb{R}^3$, whose boundary is decomposed into three disjoint parts: 
$$\dOmega = \overline{\Gamma_A} \cup \overline{\Gamma_C} \cup \overline{\Gamma}.$$ 
In this representation,
\begin{itemize}
    \item The region $\Gamma_C$ is the cathode, where the voltage potential is set to $0$;
    \item The region $\Gamma_A$ is the anode, where a value $\uin > 0$ is imposed for the potential;
    \item The device is perfectly insulated from the outside in the remaining region $\Gamma$.
\end{itemize}
We also let $\Sigma_C := \partial \Gamma_C$, $\Sigma_A := \partial \Gamma_A$, and  
we assume that $\Gamma_C$ and $\Gamma_A$ are ``well-separated'' from each other, in the sense that:
\begin{equation*}
 \mathrm{dist}(\Gamma_C,\Gamma_A) > \delta \text{ for some fixed value } \delta >0.
\end{equation*}
In this setting, we consider the following optimal design problem:
\begin{multline} \label{eq.CathodeAnode.Criterion}
    \min_{\Gamma_C, \Gamma_A \subset \partial \Omega} J(\Gamma_C, \Gamma_A) + \ell_A \Area(\Gamma_A) + \ell_C \Area(\Gamma_C) + m_A \Cont(\Gamma_A) + m_C \Cont(\Gamma_C), \\ 
    \text{ where } J(\Gamma_C, \Gamma_A) = \int_\Omega j(\nabla u_{\Gamma_C, \Gamma_A}) \: \d x .
\end{multline}
Here, $\ell_C, \: \ell_A > 0$ are penalization parameters for the surface areas of the cathode $\Gamma_C$ and anode $\Gamma_A$, and $m_C > 0$ and $m_A > 0$ are parameters that penalize their contours, see the definitions in \cref{eq.volper}. The function $j: \R^3 \to \R$ satisfies the growth conditions \cref{eq.jgrowth}, and $u_{\Gamma_C, \Gamma_A} \in H^1(\Omega)$ is the solution to: 
\begin{equation} \label{eq.CathodeAnode.State}
\left\{ 
\begin{array}{cl}
    -\dv  \left( \gamma \nabla u_{\Gamma_C, \Gamma_A} \right) = 0 & \text{in } \Omega,\\
u_{\Gamma_C, \Gamma_A} = 0 & \text{on } \Gamma_C,\\
u_{\Gamma_C, \Gamma_A} = \uin & \text{on } \Gamma_A,\\
\gamma\frac{\partial u_{\Gamma_C, \Gamma_A}}{\partial n} = 0 & \text{on } \Gamma.
\end{array}
\right.
\end{equation}

%%%%%%
\subsubsection{Approximation of the functional $J(\Gamma_C, \Gamma_A)$ and of its shape derivative}
%%%%%% 

\noindent As noted in \cref{sec.optbcconduc}, the solution $u_{\Gamma_C,\Gamma_A}$ to the boundary value problem \cref{eq.CathodeAnode.State} is weakly singular near the contours $\Sigma_C$ and $\Sigma_A$, where the boundary conditions change types. To overcome the difficulties induced by this behavior in the calculation of the functional $J(\Gamma_C, \Gamma_A)$ and its shape derivative, we replace it with the following counterpart:
\begin{equation} \label{eq.CathodeAnode.CriterionApprox}
 J_\e(\Gamma_C, \Gamma_A) = \int_\Omega j(\nabla u_{\Gamma_C,\Gamma_A,\e}) \: \d x,
\end{equation}
defined in terms of the solution $u_{\Gamma_C,\Gamma_A,\e} \in H^1(\Omega)$ to an approximate version of \cref{eq.CathodeAnode.State} where the sharp transitions between Dirichlet and Neumann boundary conditions are smoothed into a Robin boundary condition:
\begin{equation} \label{eq.CathodeAnode.StateApprox}
\left\{ 
\begin{array}{cl}
-\dv  \left( \gamma \nabla u_{\Gamma_C,\Gamma_A,\e} \right) = 0 & \text{in } \Omega,\\
\gamma\frac{\partial u_{\Gamma_C,\Gamma_A,\e}}{\partial n} + (h_{\Gamma_C, \e} + h_{\Gamma_A, \e}) u_{\Gamma_C,\Gamma_A,\e} - h_{\Gamma_A, \e} \uin = 0 & \text{on } \partial \Omega.
\end{array}
\right.
\end{equation}
In this formulation, the coefficient $h_{\Gamma_C,\e} : \partial \Omega \rightarrow \mathbb{R}$ is defined by:
\begin{equation}
    \forall x \in \partial \Omega, \quad  h_{\Gamma_C,\e} (x) = \dfrac{1}{\e} h\left(\dfrac{d^{\partial \Omega}_{\Gamma_C}(x)}{\e}\right),\\
\end{equation}
where $d^{\partial \Omega}_{\Gamma_C}$ is the geodesic signed distance function to $\Gamma_C$ (see \cref{sec.distmanifold}) 
and the transition profile $h \in C^\infty(\mathbb{R})$ satisfies \cref{eq.ShapeDerivatives.BumpFunction}; the function $h_{\Gamma_A,\e}$ is defined analogously. 

The shape derivative of the functional $J_\e(\Gamma_C,\Gamma_A)$ is the subject of the next proposition. The proof, which is fairly similar to that of \cref{prop.SDDirtoNeu}, is omitted for brevity. 
We refer to \cite{brito2024shape} for a detailed calculation.  

\begin{proposition} \label{theorem.BoundaryOptimization.CathodeAnode.ShapeDerivative}
    The criterion $J_\e(\Gamma_C, \Gamma_A)$ is shape differentiable, and its shape derivative reads, for any tangential deformation $\theta$ (i.e. $\theta\cdot n \equiv 0 \text{ on } \partial\Omega$):
    \begin{equation*}
    \begin{aligned}
    J_\e'(\Gamma_C, \Gamma_A)(\theta)
    =  \frac{1}{\e^2} \int_{\partial \Omega} h'\left(\frac{d^{\partial \Omega}_{\Gamma_A}(x)}{\e}\right) \theta(\pi_{\Sigma_A}(x)) \cdot n_{\Sigma_A} (\pi_{\Sigma_A}(x)) (\uin - u_{\Gamma_C,\Gamma_A,\e}(x)) \: p_{\Gamma_C,\Gamma_A,\e}(x) \: \d s(x) \\
     - \frac{1}{\e^2} \int_{\partial \Omega} h'\left(\frac{d^{\partial \Omega}_{\Gamma_C}(x)}{\e}\right) \: \theta(\pi_{\Sigma_C}(x)) \cdot n_{\Sigma_C} (\pi_{\Sigma_C}(x)) \: u_{\Gamma_C,\Gamma_A,\e}(x) \: p_{\Gamma_C,\Gamma_A,\e}(x) \: \d s(x), 
    \end{aligned}
    \end{equation*}
    where the adjoint state $p_{\Gamma_C,\Gamma_A,\e} \in H^1(\Omega) $ is the solution to:
    \begin{equation*}
    \begin{aligned}
        \left\{ 
        \begin{array}{cl}
        -\dv \left( \gamma \nabla p_{\Gamma_C,\Gamma_A,\e} \right) = \dv (j'(\nabla u_{\Gamma_C,\Gamma_A,\e})) & \text{in}\:  \Omega,\\
        \gamma \frac{\partial p_{\Gamma_C,\Gamma_A,\e}}{\partial n} + (h_{\Gamma_C, \e} + h_{\Gamma_A, \e}) p_{\Gamma_C,\Gamma_A,\e} = -j'(\nabla u_{\Gamma_C,\Gamma_A,\e}) \cdot n & \text{on}\:  \partial \Omega .
        \end{array}
        \right.
    \end{aligned}
    \end{equation*}
\end{proposition}
%\begin{proof} 
%\textit{Step 1. Lagrangian derivative.}
%For $\theta \in \Theta$ with norm $|| \theta ||_{W^{1,\infty}(\mathbb{R}^d, \mathbb{R}^d)} < 1$, the function $u_{\Omega_\theta, \e} \in H^1(\Omega_\theta)$ is the unique solution to the variational problem:
%\begin{align*}
%\forall v \in H^1(\Omega_\theta), \quad
%    \int_{\Omega_\theta} \gamma \nabla u_{\Omega_\theta, \e} \cdot \nabla v \: \d x
%    &+ \int_{\partial \Omega_\theta} \dfrac{1}{\e} \left(h\left(\dfrac{d^{\partial \Omega_\theta}_{{\Gamma_A}_\theta}}{\e}\right) + h\left(\dfrac{d^{\partial \Omega_\theta}_{{\Gamma_C}_\theta}}{\e}\right)\right) u_{\Omega_\theta, \e} v \: \d s
%    =\\
%    &\int_{\partial \Omega_\theta} \dfrac{1}{\e} h\left(\dfrac{d_{{\Gamma_A}_\theta}}{\e}\right) \uin v \: \d s ,
%\end{align*}
%Let us introduce the transported mapping:
%\begin{equation*}
%\begin{aligned}
%    W^{1, \infty}(\mathbb{R}^d, \mathbb{R}^d) &\rightarrow H^1(\Omega)\\
%    \theta &\mapsto \overline{u_\e}(\theta) := u_{\Omega_\theta, \e} \circ (\Id + \theta).
%\end{aligned}
%\end{equation*}
%Performing a change of variables (see \cref{theorem.Preliminaries.StandardChangeVariables,theorem.Preliminaries.SurfaceChangeVariables}) and choosing test functions of the form $v \circ (\Id + \theta)^{-1}$, $v \in H^1(\Omega)$, we obtain the following variational characterization for $\overline{u_\e}(\theta)$. For all $v \in H^1(\Omega)$:
%\begin{equation*}
%\begin{aligned}
%    &\int_\Omega A(\theta) \nabla \overline{u_\e}(\theta) \cdot \nabla v \: \d x\\
%    &+ \int_{\partial \Omega} \dfrac{1}{\e} \left(h\left(\dfrac{d^{\partial \Omega_\theta}_{{\Gamma_A}_\theta} \circ (\Id + \theta)}{\e}\right) + h\left(\dfrac{d^{\partial \Omega_\theta}_{{\Gamma_C}_\theta}\circ (\Id + \theta)}{\e}\right)\right) \det(I + \nabla \theta) |(I + \nabla \theta)^{-T}n| \overline{u_\e}(\theta) v \: \d s\\
%    &= \int_{\partial \Omega} \dfrac{1}{\e} h\left(\dfrac{d^{\partial \Omega_\theta}_{{\Gamma_A}_\theta} \circ (\Id + \theta)}{\e}\right) \det(I + \nabla \theta) |(I + \nabla \theta)^{-T}n| \uin \circ(\Id + \theta) v \: \d s,
%\end{aligned}
%\end{equation*}
%where:
%\begin{equation*}
%    A(\theta) := \gamma \circ (\Id + \theta) | \det (\I + \nabla \theta) | (\I + \nabla \theta)^{-1} (\I + \nabla \theta)^{-T}.
%\end{equation*}
%The Fréchet differentiability of the map $\theta \mapsto \overline{u_\e}(\theta)$ is typically established via the implicit function theorem (see \cite{allaire2020shape,henrot2006variation}) but for the sake of simplicity we omit this part of the proof.
%Differentiating both sides with respect to $\theta$ at $0$ yields the characterization for the ``Lagrangian'' derivative $\mathring{u}_{\Omega,\e}(\theta)$ of $u_{\Omega,\e}$, i.e. the derivative of the mapping $\theta \mapsto \overline{u_\e}(\theta)$, for all $v \in H^{1}(\Omega)$,
%\begin{equation} \label{eq.CathodeAnode.LagrangianDerivative}
%\begin{aligned}
%    \int_\Omega &\gamma \nabla \mathring{u}_{\Omega,\e}(\theta) \cdot \nabla v \: \d x + \int_{\partial \Omega} (h_{C, \e} + h_{A, \e}) \mathring{u}_{\e}(\theta) v \: \d s =\\
%    & - \int_{\Omega} (\nabla u_\e \cdot \nabla v) (\nabla \gamma \cdot \theta) \: \d x
%    - \int_\Omega (\nabla \cdot \theta \I - \nabla \theta - {\nabla \theta}^T) \gamma \nabla u_\e \cdot \nabla v \: \d x\\
%    &- \int_{\partial \Omega} (\nabla_{\partial \Omega} \cdot \theta) (\hCe + \hAe) u_\e v \: \d s\\
%    &- \frac{1}{\e^2}\int_{\partial \Omega}\left(h'\left(\frac{d^{\partial \Omega}_{\Gamma_C}}{\e}\right) D_C'(0)(\theta) +  h'\left(\frac{d^{\partial \Omega}_{\Gamma_A}}{\e}\right)D_A'(0)(\theta) \right) u_\e v \: \d s\\
%    &+ \int_{\partial \Omega} ((\nabla_{\partial \Omega} \cdot \theta) \uin + \nabla \uin \cdot \theta ) h_{A, \e} v \: \d s\\
%    &+ \frac{1}{\e^2} \int_{\partial \Omega} h'\left( \frac{d^{\partial \Omega}_{\Gamma_A}}{\e} \right) D_A'(0)(\theta) \uin v \: \d s,
%\end{aligned}
%\end{equation}
%where we have used the following facts (see \cite{henrot2006variation} for their statement and proof):
%\begin{align*}
%    \left. \frac{\d}{\d\theta} \right|_{\theta = 0} &\det(I + \nabla \theta) = \nabla \cdot \theta,\\
%    \left. \frac{\d}{\d\theta} \right|_{\theta = 0} &\det(I + \nabla \theta) |(I + \nabla \theta)^{-T}n_{\partial \Omega}| = \nabla_{\partial \Omega} \cdot \theta .
%\end{align*}
%
%\noindent \textit{Step 2. Shape derivative of $J_\e(\Omega)$.} We now calculate the derivative of the objective function $J_\e(\Omega)$; for sufficiently small $\theta \in \Winfty$, it holds:
%\begin{align*}
%J_\e(\Omega_\theta) &= \int_\Omega \lvert \det{(\I + \nabla \theta)}\lvert \: j((\nabla u_{\Omega_\theta, \e}) \circ (\Id + \theta))  \: \d x \\
%&= \int_\Omega \lvert\det{(\I + \nabla \theta)}\lvert \: j((\I + \nabla \theta)^{-T} \nabla \overline{u_\e} (\theta))\:\d x. \\
%\end{align*}
%Hence, taking derivatives in this formula, we obtain:
%$$
%J_\e^\prime(\Omega)(\theta)
%= \int_\Omega j'(\nabla u_\e) \cdot \nabla \mathring{u}_{\Omega,\e}(\theta) \: \d x + \int_\Omega j(\nabla u_\e) \nabla \cdot \theta \: \d x - \int_\Omega \nabla \theta^T \nabla u_\e \cdot j'(\nabla u_\e) \: \d x.
%$$
%
%\noindent \textit{Step 3. We reformulate this derivative by using the adjoint state}. Consider the weak formulation for the adjoint solution $p_\e \in H^1(\Omega)$, given by:\begin{equation*}
%\forall v \in H^1(\Omega), \quad \int_{\Omega} \gamma \nabla p_\e \cdot \nabla v \: \d x + \int_{\partial \Omega} (h_{C, \e} + h_{A, \e}) p_\e v \: \d s = -\int_\Omega j'(\nabla u_\e) \cdot \nabla v \: \d x .
%\end{equation*}
%Selecting $v = \mathring{u}_{\e}(\theta)$ as a test function results in the following expression:
%\begin{equation*}
%    \int_{\Omega} \gamma \nabla p_\e \cdot \nabla \mathring{u}_{\e}(\theta) \: \d x + \int_{\partial \Omega} (h_{C, \e} + h_{A, \e}) p_\e \mathring{u}_{\e}(\theta) \: \d s = -\int_\Omega j'(\nabla u_\e) \cdot \nabla \mathring{u}_{\e}(\theta) \: \d x .
%\end{equation*}
%On the other hand, opting for $v = p_\e$ as a test function in \cref{eq.CathodeAnode.LagrangianDerivative} and combining it with the previous equation leads to the volumetric expression of the shape derivative:
%
%\begin{equation} \label{eq.CathodeAnode.ShapeDerivativeVolumetric}
%\begin{aligned}
%    J'(\Omega)(\theta) = &\int_\Omega j(\nabla u_{\Omega, \e}) \nabla \cdot \theta ~ \d x  - \int_\Omega \nabla \theta^T \nabla u_\e \cdot j'(\nabla u_\e) ~ \d x + \int_{\Omega} (\nabla u_\e \cdot \nabla p) (\nabla \gamma \cdot \theta) ~ \d x\\
%    &+ \int_\Omega (\nabla \cdot \theta I - \nabla \theta - {\nabla \theta}^T) \gamma \nabla u_\e \cdot \nabla p_\e ~ \d x\\
%    &- \int_{\partial \Omega} ((\nabla_{\partial \Omega} \cdot \theta) \uin + \nabla \uin \cdot \theta ) h_{A, \e} p_\e ~ \d s\\
%    &+ \int_{\partial \Omega} (\nabla_{\partial \Omega} \cdot \theta) (h_{C, \e} + h_{A, \e}) u_\e p_\e ~ \d s\\
%    &+ \frac{1}{\e^2}\int_{\partial \Omega}\left(h'\left(\frac{d_{\Gamma_C}}{\e}\right) D_C'(0)(\theta) +  h'\left(\frac{d_{\Gamma_A}}{\e}\right)D_A'(0)(\theta) \right) u_\e p_\e ~ \d s\\
%    &- \frac{1}{\e^2} \int_{\partial \Omega} h'\left( \frac{d_{\Gamma_A}}{\e} \right) D_A'(0)(\theta) \uin p_\e ~ \d s,
%\end{aligned}
%\end{equation}
%
%\noindent \textit{Step 4. Derivation of the surface expression}.
%Following multiple applications of integration by parts (refer to \cref{sec.Appendix.SupportMaterial.Identities}), one can verify the cancellation of the volumetric terms, resulting in the following expression for the shape derivative:
%\begin{equation*}
%\begin{aligned}
%    J_\e'(\Omega)(\theta)
%    &=\int_{\partial \Omega} \left( j(\nabla u_\e) - \gamma \nabla u_\e \cdot \nabla p_\e \right) \: \theta \cdot n_{\partial \Omega}\: \d s - (I^1_{\e} - I^2_{\e} + I^3_{\e})\\
%    &+ \frac{1}{\e^2}\int_{\partial \Omega}\left(h'\left(\frac{d^{\partial \Omega}_{\Gamma_C}}{\e}\right) D_C'(0)(\theta) +  h'\left(\frac{d^{\partial \Omega}_{\Gamma_A}}{\e}\right)D_A'(0)(\theta) \right) u_\e p_\e \: \d s\\
%    &- \frac{1}{\e^2} \int_{\partial \Omega} h'\left( \frac{d^{\partial \Omega}_{\Gamma_A}}{\e} \right) D_A'(0)(\theta) \: \uin p_\e \: \d s,
%\end{aligned}
%\end{equation*}
%where
%\begin{align*}
%    I^1_\e &:= \int_{\partial \Omega} ((\nabla_{\partial \Omega} \cdot \theta) \uin + \nabla \uin \cdot \theta ) h_{A, \e} p_\e \: \d s,\\
%    I^2_\e &:= \int_{\partial \Omega} (\nabla_{\partial \Omega} \cdot \theta) (h_{C, \e} + h_{A, \e}) u_\e p_\e \: \d s,\\
%    I^3_{\e} &:= \int_{\partial \Omega} \left( \theta \cdot \nabla p_\e \gamma \dfrac{\partial u_\e}{\partial n_{\partial \Omega}} \: \d s + \theta \cdot \nabla u_\e \left( \gamma \dfrac{\partial p_\e}{\partial n_{\partial \Omega}} - j'(\nabla u_\e ) \cdot n_{\partial \Omega} \right) \right) \: \d s.
%\end{align*}
%The integration of this expression towards tangential terms can be accomplished by employing \cref{prop.TangentialIntByParts} and decomposing:
%\begin{equation}
%\begin{aligned}
%\nabla_{\partial \Omega} \uin &= \nabla \uin - \dfrac{\partial \uin}{\partial n_{\partial \Omega}} n_{\partial \Omega},\\
%\nabla_{\partial \Omega} u_\e &= \nabla u_\e - \dfrac{\partial u_\e}{\partial n_{\partial \Omega}} n_{\partial \Omega},\\
%\nabla_{\partial \Omega} p_\e &= \nabla p_\e - \dfrac{\partial p_\e}{\partial n_{\partial \Omega}} n_{\partial \Omega}.
%\end{aligned}
%\end{equation}
%Firstly, tangentially integrating the first term of $I^1_\e$ by parts gives:
%\begin{equation*}
%\begin{aligned}
%    I^1_\e 
%    &= \int_{\partial \Omega} \kappa \hAe \uin p_\e \: \theta \cdot n_{\partial \Omega}\: \d s + \int_{\partial \Omega} \hAe p_\e \dfrac{\partial \uin}{\partial n_{\partial \Omega}} \theta \cdot n_{\partial \Omega} \: \d s\\
%    &- \int_{\partial \Omega} \uin p_\e \nabla_{\partial \Omega} \hAe \cdot \theta \: \d s
%    - \int_{\partial \Omega} \hAe \uin \nabla_{\partial \Omega} p_\e \cdot \theta \: \d s .
%\end{aligned}
%\end{equation*}
%Next, integrating $I^2_\e$ yields:
%\begin{equation*}
%\begin{aligned}
%    I^2_\e &= \int_{\partial \Omega} \kappa (\hAe + \hCe) u_\e p_\e \: \theta \cdot n_{\partial \Omega}\: \d s\\
%    &- \int_{\partial \Omega} (\hAe + \hCe) p_\e \nabla_{\partial \Omega} u_\e \cdot \theta \: \d s
%    - \int_{\partial \Omega} (\hAe + \hCe) u_\e \nabla_{\partial \Omega} p_\e \cdot \theta \: \d s\\
%    &- \int_{\partial \Omega} \nabla_{\partial \Omega} (\hAe + \hCe) \cdot \theta \: u_\e p_\e \: \d s .
%\end{aligned}
%\end{equation*}
%Additionally, we can substitute the definitions of $\gamma \frac{\partial u_\e}{\partial n_{\partial \Omega}}$ and $\gamma \frac{\partial p_\e}{\partial n_{\partial \Omega}}$ into $I^3_\e$ and decompose them tangentially:
%\begin{equation*}
%\begin{aligned}
%    I^3_\e &= \int_{\partial \Omega} \hAe \uin \left( \nabla_{\partial \Omega} p_\e + \frac{\partial p_\e}{\partial n_{\partial \Omega}}n_{\partial \Omega}\right) \cdot \theta \: \d s\\
%    &-\int_{\partial \Omega} (\hAe + \hCe) u_\e \left( \nabla_{\partial \Omega} p_\e + \frac{\partial p_\e}{\partial n_{\partial \Omega}}n_{\partial \Omega}\right) \cdot \theta \: \d s
%    - \int_{\partial \Omega} (\hAe + \hCe ) p_\e \left(\nabla_{\partial \Omega} u_\e + \frac{\partial u_\e}{\partial n_{\partial \Omega}}n_{\partial \Omega}\right) \cdot \theta \: \d s
%\end{aligned}
%\end{equation*}
%Computing now $I^1_\e - I^2_\e + I^3_\e$ yields:
%\begin{equation*}
%\begin{aligned}
%I^1_\e - I^2_\e + I^3_\e &=
%    \int_{\partial \Omega} \kappa \hAe \uin p_\e \: \theta \cdot n_{\partial \Omega}\: \d s - \int_{\partial \Omega} \kappa (\hAe + \hCe) u_\e p_\e \: \theta \cdot n_{\partial \Omega}\: \d s\\
%    &+ \int_{\partial \Omega} \hAe p_\e \dfrac{\partial \uin}{\partial n_{\partial \Omega}} \theta \cdot n_{\partial \Omega} \: \d s - \int_{\partial \Omega} (\hAe + \hCe ) p_\e \frac{\partial u_\e}{\partial n_{\partial \Omega}} \theta \cdot n_{\partial \Omega}\: \d s\\
%    &+ \int_{\partial \Omega} \hAe \uin \frac{\partial p_\e}{\partial n_{\partial \Omega}} \theta \cdot n_{\partial \Omega}\: \d s - \int_{\partial \Omega} (\hAe + \hCe) u_\e \frac{\partial p_\e}{\partial n_{\partial \Omega}} \theta \cdot n_{\partial \Omega}\: \d s\\
%    &- \int_{\partial \Omega} \uin p_\e \nabla_{\partial \Omega} \hAe \cdot \theta \: \d s + \int_{\partial \Omega} u_\e p_\e \nabla_{\partial \Omega} (\hAe + \hCe) \cdot \theta \: \d s .
%\end{aligned}
%\end{equation*}
%Recall that by \cref{eq.CathodeAnode.StateApprox} we have:
%\begin{equation*}
%    \gamma\frac{\partial u_\e}{\partial n_{\partial \Omega}} = h_{A, \e} \uin - (h_{C, \e} + h_{A, \e}) u_\e \quad \text{on } \partial \Omega ,
%\end{equation*}
%hence we can identify $\gamma\frac{\partial u_\e}{\partial n_{\partial \Omega}}$ in the sum to see:
%\begin{equation*}
%\begin{aligned}
%I^1_\e - I^2_\e + I^3_\e &=
%    \int_{\partial \Omega} \kappa \gamma\frac{\partial u_\e}{\partial n_{\partial \Omega}} \hAe p_\e \: \theta \cdot n_{\partial \Omega}\: \d s\\
%    &+ \int_{\partial \Omega} \hAe p_\e \dfrac{\partial \uin}{\partial n_{\partial \Omega}} \: \theta \cdot n_{\partial \Omega} \: \d s - \int_{\partial \Omega} (\hAe + \hCe ) p_\e \frac{\partial u_\e}{\partial n_{\partial \Omega}} \: \theta \cdot n_{\partial \Omega}\: \d s\\
%    &+ \int_{\partial \Omega} \gamma \frac{\partial u_\e}{\partial n_{\partial \Omega}} \frac{\partial p_\e}{\partial n_{\partial \Omega}} \: \theta \cdot n_{\partial \Omega}\: \d s\\
%    &- \int_{\partial \Omega} \uin p_\e \nabla_{\partial \Omega} \hAe \cdot \theta \: \d s + \int_{\partial \Omega} u_\e p_\e \: \nabla_{\partial \Omega} (\hAe + \hCe) \cdot \theta \: \d s .
%\end{aligned}
%\end{equation*}
%Substituting back in results in the expression taking the form:
%\begin{equation*}
%    \begin{aligned}
%    J'(\Omega)(\theta)
%    &= 
%    \int_{\partial \Omega} \left( j(\nabla u_\e) - \gamma \nabla u_\e \cdot \nabla p_\e \right) \: \theta \cdot n_{\partial \Omega}\: \d s - \int_{\partial \Omega} \kappa \gamma\frac{\partial u_\e}{\partial n_{\partial \Omega}} \hAe p_\e \: \theta \cdot n_{\partial \Omega}\: \d s\\
%    &- \int_{\partial \Omega} \hAe p_\e \dfrac{\partial \uin}{\partial n_{\partial \Omega}} \: \theta \cdot n_{\partial \Omega} \: \d s + \int_{\partial \Omega} (\hAe + \hCe ) p_\e \frac{\partial u_\e}{\partial n_{\partial \Omega}} \: \theta \cdot n_{\partial \Omega}\: \d s
%    - \int_{\partial \Omega} \gamma \frac{\partial u_\e}{\partial n_{\partial \Omega}} \frac{\partial p_\e}{\partial n_{\partial \Omega}} \: \theta \cdot n_{\partial \Omega}\: \d s\\
%    &+ \int_{\partial \Omega} \uin p_\e \nabla_{\partial \Omega} \hAe \cdot \theta \: \d s - \int_{\partial \Omega} u_\e p_\e \: \nabla_{\partial \Omega} (\hAe + \hCe) \cdot \theta \: \d s\\
%    &+ \frac{1}{\e^2}\int_{\partial \Omega}\left(h'\left(\frac{d^{\partial \Omega}_{\Gamma_C}}{\e}\right) D_C'(0)(\theta) +  h'\left(\frac{d^{\partial \Omega}_{\Gamma_A}}{\e}\right)D_A'(0)(\theta) \right) u_\e p_\e \: \d s\\
%    &- \frac{1}{\e^2} \int_{\partial \Omega} h'\left( \frac{d^{\partial \Omega}_{\Gamma_A}}{\e} \right) D_A'(0)(\theta) \: \uin p_\e \: \d s .
%    \end{aligned}
%\end{equation*}
%We can further simplify the last three lines of this new expression.
%Via uses of \cref{lemma.RiemannianOptimization.PropertieSignedDistanceFunction} and \cref{theorem.RiemannianOptimization.ShapeDerivativeSignedDistance} for the region $\Gamma_A$ (and, respectively, for the region $\Gamma_C$) we can see that:
%\begin{equation*}
%\begin{aligned}
%    \nabla_{\partial \Omega} \hAe \cdot \theta &- \frac{1}{\e^2} h'\left(\frac{d^{\partial \Omega}_{\Gamma_A}}{\e}\right) D_A'(0)(\theta)\\
%    &= -\frac{1}{\e^2} h'\left(\frac{d^{\partial \Omega}_{\Gamma_A}}{\e} \right) \dfrac{\log_x (p_{\Sigma_A}(x))}{d^{\partial \Omega}_{\Gamma_A}(x)} \cdot \theta - \frac{1}{\e^2} h'\left(\frac{d^{\partial \Omega}_{\Gamma_A}}{\e}\right) D_A'(0)(\theta) \\
%    &=\frac{1}{\e^2} h'\left(\frac{d^{\partial \Omega}_{\Gamma_A}}{\e}\right) \left( \theta(p_{\Sigma_A}) \cdot n_{\Sigma_A} (p_{\Sigma_A}) - \int_0^{d^{\partial \Omega}_{\Gamma_A}} \Pi^{\partial \Omega}_{\sigma(t)} (\sigma'(t), \sigma'(t)) (\theta \cdot n) (\sigma(t)) \: \d t \right) ,
% \end{aligned}
%\end{equation*}
%which, upon substitution,  and after imposing $\theta \cdot n_{\partial \Omega}= 0$ leads to the desired expression.
%\end{proof}
An approximate counterpart for this formula which conveniently simplifies the numerical implementation is obtained by arguing along the lines of \cref{sec.approxSD}:
\begin{equation}
    J_\e'(\Gamma_C, \Gamma_A)(\theta)
    = \frac{1}{\e} \int_{\Sigma_A} (\uin - u_{\Gamma_C,\Gamma_A,\e}) \: p_{\Gamma_C,\Gamma_A,\e} \: \theta \cdot n_{\Sigma_A} \: \d \ell  - \frac{1}{\e} \int_{\Sigma_C} u_{\Gamma_C,\Gamma_A,\e} \: p_{\Gamma_C,\Gamma_A,\e} \: \theta \cdot n_{\Sigma_C} \d\ell .
\end{equation}

%%%%%%%
\subsubsection{The topological derivative of $J(\Gamma_C,\Gamma_A)$}
%%%%%%%

\noindent Let us now deal with the topological sensitivity of the criterion $J(\Gamma_C,\Gamma_A)$ when the homogeneous Neumann boundary condition is replaced by a (homogeneous or inhomogeneous) Dirichlet boundary condition on a ``small'' surface disk $\omega_{x_0,\e} \subset \Gamma$. An elementary adaptation of the proofs of \cref{th.Conductivity.HNHD.Expansion,cor.Conductivity.HNHD.Jp} (see also \cref{rem.Conductivity.HNHD.Inhomogeneous}) yields the following result.

\begin{proposition}
Let $\Gamma_C$, $\Gamma_A$ be disjoint regions of the smooth boundary $\partial \Omega$ as in \cref{sec.CathodeAnodeOptPb}, and let $x_0 \in \Gamma$. Then,
 \begin{enumerate}[(i)] 
\item The perturbed criterion $J(({\Gamma_C})_{x_0, \e},\Gamma_A)$, that accounts for the addition of the surface disk $\omega_{x_0,\e} \subset \Gamma$ to $\Gamma_C$, has the following asymptotic expansion:
    \begin{equation*}
        J(({\Gamma_C})_{x_0, \e},\Gamma_A) =  J(\Gamma_C,\Gamma_A) - \frac{\pi}{|\log \e|} \: \gamma(x_0) \: u_{\Gamma_C,\Gamma_A}(x_0) \: p_{\Gamma_C,\Gamma_A}(x_0) + \o\left(\dfrac{1}{|\log \e|}\right)  \text{ if }  d = 2,
    \end{equation*}
    and 
        \begin{equation*}
        J(({\Gamma_C})_{x_0, \e},\Gamma_A) =  J(\Gamma_C,\Gamma_A) - 4 \e \: \gamma(x_0) \: u_{\Gamma_C,\Gamma_A}(x_0) \: p_{\Gamma_C,\Gamma_A}(x_0) + \o(\e)  \text{ if }  d = 3.
    \end{equation*}
    
\item The perturbed criterion $J(\Gamma_C,({\Gamma_A})_{x_0, \e})$, that accounts for the addition of $\omega_{x_0,\e} \subset \Gamma$ to $\Gamma_A$, has the following asymptotic expansion:
    \begin{equation*}
        J(\Gamma_C,({\Gamma_A})_{x_0, \e}) = J(\Gamma_C,\Gamma_A) + \frac{\pi}{|\log \e|} \: \gamma(x_0) \: (u_{\mathrm{in}} -  u_{\Gamma_C,\Gamma_A}(x_0)) \:  p_{\Gamma_C,\Gamma_A}(x_0) + \o\left(\dfrac{1}{|\log \e|}\right) \text{ if }  d = 2,
    \end{equation*}
    and 
        \begin{equation*}
        J(\Gamma_C,({\Gamma_A})_{x_0, \e}) =  J(\Gamma_C,\Gamma_A) + 4 \e \: \gamma(x_0) \: (u_{\mathrm{in}} -  u_{\Gamma_C,\Gamma_A}(x_0)) \:  p_{\Gamma_C,\Gamma_A}(x_0) + \o(\e) \text{ if } d = 3.
    \end{equation*}
    \end{enumerate}
 In all the above formulas, the adjoint state $ p_{\Gamma_C,\Gamma_A} \in H^1(\Omega)$ is the solution to:
    \begin{equation*}
        \left\{ 
        \begin{array}{cl}
        -\dv \left( \gamma \nabla  p_{\Gamma_C,\Gamma_A} \right) = \dv( j^\prime(\nabla  u_{\Gamma_C,\Gamma_A})) & \text{in }  \Omega,\\[0.3em]
        \gamma \frac{\partial  p_{\Gamma_C,\Gamma_A}}{\partial n} =  -j^\prime(\nabla  u_{\Gamma_C,\Gamma_A}) \cdot n & \text{on } \Gamma,\\[0.3em]
         p_{\Gamma_C,\Gamma_A} = 0 &\text{on } \Gamma_C \cup \Gamma_A.
        \end{array}
        \right.
    \end{equation*}
\end{proposition}

%%%%%%%
\subsubsection{Numerical results}
%%%%%%%

\noindent We optimize \cref{eq.CathodeAnode.Criterion} when $\Omega$ is the ring-shaped electroosmotic mixer depicted on \cref{fig.CathodeAnode.EMM_Setup} \cite{chen2003numerical,zhang2004soi,jalili2020numerical}. In the perspective of maximizing the power of the electric field inside $\Omega$, the function
$ j:\R^3 \to \R$ is defined by $j(V) = -\gamma \lvert V \lvert^2$.

\begin{figure}[!ht] 
    \centering    
    \fbox{\includegraphics[width=0.55\textwidth]{figures/EMM_Setup}}
    \caption{\it Different views of the initial mesh $\mathcal{T}^0$ of the electroosmotic mixer considered in \cref{sec.CathodeAnode}. The red and pink regions correspond to the non optimizable components of the anode and cathode regions, and the green region represents the inflow region.
    }
    \label{fig.CathodeAnode.EMM_Setup}
\end{figure}

The initial mesh $\calT^0$ of the domain $\Omega$ is composed of approximately $29,000$ vertices and $135,000$ tetrahedra. 
Throughout the optimization process, we impose that the top (resp. bottom) of the inflow and outflow channels are part of the anode (resp. cathode). At each iteration of \cref{alg.sketchoptbc}, we alternately optimize either $\Gamma_C$ or $\Gamma_A$: we rely on the use of topological derivatives to attach new connected components to these regions every 10 iterations (i.e. $n_{\text{top}}=10$ in \cref{alg.CouplingMethods.SurfaceOptimization}) until 100 iterations are reached, after which only geometric optimization steps are performed, based on the use of shape derivatives.

Three numerical experiments are conducted. In all cases, the minimum (resp. maximum) size of an element in the mesh is $\hmin=0.05$ (resp. $\hmax=0.5$), and the parameter $\e$ in the regularization process of shape derivatives equals $\e =0.001$. Different values are used for the weights $\ell_C, \ell_A, \lambda_C, \lambda_A$ in \cref{alg.CouplingMethods.SurfaceOptimization}, as reported in \cref{tab.CathodeAnode.Params}.

\begin{table}[H]
\begin{tabular}{|c|c|c|c|c|}
    \hline
     & $\ell_C$ & $\ell_A$ & $\lambda_C$ & $\lambda_A$\\
    \hline
    Experiment 1 & $0.0001$ & $0.0001$ & $0$ & $0$ \\
    Experiment 2 & $0.001$ & $0.001$ & $0$ & $0$ \\
    Experiment 3 & $0.001$ & $0.001$ & $0.001$ & $0.001$ \\
    \hline
\end{tabular}
\caption{\it Values of the parameters used in the (left) first, (center) second, and (right) third experiments of optimal design of the anode and cathode of an electroosmotic mixer considered in \cref{sec.CathodeAnode}.}
\label{tab.CathodeAnode.Params}
\end{table}

Our first experiment deals with the maximization of the electric field inside $\Omega$ under a penalization of the areas of $\Gamma_C$ and $\Gamma_A$, without penalization of their contours. We perform $250$ iterations of our optimization \cref{alg.CouplingMethods.SurfaceOptimization} and a few snapshots of the optimization process are given in \cref{fig.CathodeAnode.Results_1}, see \cref{fig.CathodeAnode.Results_1_Cont} (e) for the convergence history. The method tends to assign the cathode and anode to opposite sides of the mixer $\Omega$. Initially, an anode is formed at the middle of the upper side and a corresponding cathode appears symmetrically on its lower side. The geometric optimization process then results in the expansion of the existing regions, which gradually attempt to cover the entire upper and lower sides of the mixer. 
Finally, the areas of both regions decrease because of the penalization of their values in the objective \cref{eq.CathodeAnode.Criterion}, revealing a pattern which is strongly reminiscent of the homogenization effect whereby shapes get close to optimal by developing very thin features, at the microscopic level, see e.g. \cite{allaire2002shape,bucur2002variational,henrot2018shape}. The final mesh $\calT$ of $\Omega$ contains approximately 5 million tetrahedrons, and the total simulation lasted around 3 hours.

In the second experiment, we use larger penalization parameters $\ell_A$, $\ell_C$ for the areas of $\Gamma_A$ and $\Gamma_C$. 
The results are presented in \cref{fig.CathodeAnode.Results_2}. As in the previous experiment, both regions independently develop multiple branches, while occupying a lower area of the boundary $\partial \Omega$ of the mixer. The final mesh consists of 1.8 million tetrahedra. The total computational time is approximately 2 hours.

\begin{figure}[H]
    \centering
\begin{tabular}{cc}
\begin{minipage}{0.49\textwidth}
\begin{overpic}[width=1.0\textwidth]{figures/EMM_1_40}
\put(0,0){\fcolorbox{black}{white}{$n=40$}}
\end{overpic}
\end{minipage} & 
\begin{minipage}{0.49\textwidth}
\begin{overpic}[width=1.0\textwidth]{figures/EMM_1_80}
\put(0,0){\fcolorbox{black}{white}{$n=80$}}
\end{overpic}
\end{minipage}
\end{tabular}     \par\bigskip 
    \begin{tabular}{cc}
\begin{minipage}{0.49\textwidth}
\begin{overpic}[width=1.0\textwidth]{figures/EMM_1_120}
\put(0,0){\fcolorbox{black}{white}{$n=120$}}
\end{overpic}
\end{minipage} & 
\begin{minipage}{0.49\textwidth}
\begin{overpic}[width=1.0\textwidth]{figures/EMM_1_160}
\put(0,0){\fcolorbox{black}{white}{$n=160$}}
\end{overpic}
\end{minipage}
\end{tabular}      
 \par\bigskip 
\begin{minipage}{0.49\textwidth}
\begin{overpic}[width=1.0\textwidth]{figures/EMM_1_200}
\put(0,0){\fcolorbox{black}{white}{$n=200$}}
\end{overpic}
\end{minipage}
    \caption{\it Snapshots of the optimization process of the distribution of the anode (in orange) and cathode (in blue) on the boundary of the electroosmotic mixer, in the first experiment of \cref{sec.CathodeAnode}.}
    \label{fig.CathodeAnode.Results_1}
\end{figure}

In our third experiment, we add a penalization on the lengths of the contours of $\Gamma_A$ and $\Gamma_C$, while keeping the same penalization parameters for their areas as in the second experiment. 
The results are reported on \cref{fig.CathodeAnode.Results_2} (b,d). As expected, the optimized regions have simpler shapes, containing fewer branches. The final mesh also consists of 1.8 million tetrahedra, for a total simulation time of about 2 hours.

\begin{figure}[H]
    \centering
    \begin{tabular}{cc}
\begin{minipage}{0.49\textwidth}
\begin{overpic}[width=1.0\textwidth]{figures/EMM_1_245}
\put(0,0){\fcolorbox{black}{white}{a}}
\end{overpic}
\end{minipage} & 
\begin{minipage}{0.49\textwidth}
\begin{overpic}[width=1.0\textwidth]{figures/EMM_1_245_NOLINES}
\put(0,0){\fcolorbox{black}{white}{b}}
\end{overpic}
\end{minipage}
\end{tabular}     \par\bigskip 
    \begin{tabular}{cc}
\begin{minipage}{0.49\textwidth}
\begin{overpic}[width=1.0\textwidth]{figures/EMM_1_245_Perspective_Anode_NOLINES}
\put(0,0){\fcolorbox{black}{white}{c}}
\end{overpic}
\end{minipage} & 
\begin{minipage}{0.49\textwidth}
\begin{overpic}[width=1.0\textwidth]{figures/EMM_1_245_Perspective_Cathode_NOLINES}
\put(0,0){\fcolorbox{black}{white}{d}}
\end{overpic}
\end{minipage}
\end{tabular}      
 \par\bigskip 
\begin{minipage}{0.49\textwidth}
\begin{overpic}[width=1.0\textwidth]{figures/EMM_1_Obj_Negative}
\put(0,0){\fcolorbox{black}{white}{e}}
\end{overpic}
\end{minipage}    
    \caption{\it (a,b) Optimized designs of the anode and cathode obtained in the first experiment of \cref{sec.CathodeAnode}; (c,d) perspective views of the top and bottom parts of the mixer; (e) Convergence history.}
    \label{fig.CathodeAnode.Results_1_Cont}
\end{figure}

\begin{figure}[H]
    \centering
        \begin{tabular}{cc}
\begin{minipage}{0.49\textwidth}
\begin{overpic}[width=1.0\textwidth]{figures/EMM_2}
\put(0,0){\fcolorbox{black}{white}{a}}
\end{overpic}
\end{minipage} & 
\begin{minipage}{0.49\textwidth}
\begin{overpic}[width=1.0\textwidth]{figures/EMM_3}
\put(0,0){\fcolorbox{black}{white}{b}}
\end{overpic}
\end{minipage}
\end{tabular}     \par\bigskip 
          \begin{tabular}{cc}
\begin{minipage}{0.49\textwidth}
\begin{overpic}[width=1.0\textwidth]{figures/EMM_Obj_2}
\put(0,0){\fcolorbox{black}{white}{c}}
\end{overpic}
\end{minipage} & 
\begin{minipage}{0.49\textwidth}
\begin{overpic}[width=1.0\textwidth]{figures/EMM_Obj_3}
\put(0,0){\fcolorbox{black}{white}{d}}
\end{overpic}
\end{minipage}
\end{tabular}    
    \caption{\it (a) Optimized design obtained in the second experiment of \cref{sec.CathodeAnode}; (b) Optimized design obtained in the third experiment; (c) Convergence history for the second experiment; (d) Convergence history for the third experiment.}
    \label{fig.CathodeAnode.Results_2}
\end{figure}
