\documentclass[10pt]{amsart}
\usepackage{geometry}                % See geometry.pdf to learn the layout options. There are lots.
\geometry{letterpaper}                   % ... or a4paper or a5paper or ... 
%\geometry{landscape}                % Activate for for rotated page geometry
%\usepackage[parfill]{parskip}    % Activate to begin paragraphs with an empty line rather than an indent
\usepackage{graphicx}
\usepackage{caption}
\usepackage{subcaption}
\usepackage{dsfont}
\usepackage{amssymb}
\usepackage{amsmath}
\usepackage{epstopdf}
\usepackage{fullpage}
\usepackage{enumerate}
\usepackage{color}
\usepackage{amsthm}
\usepackage{placeins}
\usepackage{array}
\usepackage{blkarray}
\usepackage{multirow}
\usepackage{float}
\usepackage{morefloats}
\usepackage{pgfplots}
\usepackage[ruled]{algorithm2e}
\usepackage{algorithmic}
\usepackage[abs]{overpic}
\usepackage{newclude}
\usepackage[textwidth=0.5\marginparwidth]{todonotes}


\newcommand{\thought}[1]{{\color[rgb]{0.2,0.39,0.66}(#1)}}
\newcommand{\todo}[1]{{\color[rgb]{1.0,0.0,0.0}(#1)}}
\newcommand{\hsh}[1]{{\color{green!50!black} Henrik: #1}}
\newcommand{\st}[1]{{\color{red!50!black} Sebastian: #1}}

\newcommand{\ulm}[1]{_{\scaleto{\mathrm{#1}}{3pt}}}
\newcommand\at[2]{\left.#1\right|_{#2}}











\newtheorem{assumption}{Assumption}

\DeclareMathOperator*{\argmax}{arg\,max}
\DeclareMathOperator*{\argmin}{arg\,min}

\newcommand{\swname}[1]{\texttt{#1}}
\newcommand{\ie}{i\/.\/e\/.,\/~}
\newcommand{\eg}{e\/.\/g\/.,\/~}
\newcommand{\cf}{cf\/.\/~}

\newcommand{\fig}{Fig\/.\/~}
\newcommand{\defn}{Def\/.\/~}
\newcommand{\sect}{Sec\/.\/~}
\newcommand{\tabl}{Tab\/.\/~}
\newcommand{\algo}{Algorithm~}
\newcommand{\theo}{Theorem~}

\newcommand{\bnnl}{3 hidden layers}
\newcommand{\bnnn}{50 neurons}
\newcommand{\bnna}{tanh activations}

\newcommand{\capt}[1]{\mdseries{\emph{#1}}}

\newcommand{\videolink}{at \url{https://youtu.be/_d7AqTRjz6g}}
\newcommand{\codelink}{\url{https://github.com/wheelbot/mini-wheelbot}}

\newcommand{\fakepar}[1]{\vspace{0mm}\noindent\textbf{#1.}}

\newcommand{\needref}{\textcolor{red}{[REF]}}

\newcommand{\plotfontsize}{9pt}

%\usepackage{subfigure}

\usepackage{amsopn}
\DeclareGraphicsRule{.tif}{png}{.png}{`convert #1 `dirname #1`/`basename #1 .tif`.png}

%%%% To get table of contents %%%%%%%%
\setcounter{tocdepth}{2}% to get subsubsections in toc

\let\oldtocsection=\tocsection

\let\oldtocsubsection=\tocsubsection

\let\oldtocsubsubsection=\tocsubsubsection

\renewcommand{\tocsection}[2]{\hspace{0em}\oldtocsection{#1}{#2}\textbf}
\renewcommand{\tocsubsection}[2]{\hspace{1em}\oldtocsubsection{#1}{#2}}
\renewcommand{\tocsubsubsection}[2]{\hspace{2em}\oldtocsubsubsection{#1}{#2}}

\newenvironment{resume}
  {\renewcommand\abstractname{R\'esum\'e}\begin{abstract}}
  {\end{abstract}}
  


%%%%%%%%%%%%%%%%%%% Macros %%%%%%%%%%%%%%%%%%%%%%
\newcommand{\dv}{\text{\rm div}}
\newcommand{\dmin}{d_\text{\rm min}}
\newcommand{\rinj}{r_\text{\rm inj}}
\renewcommand{\o}{\text{\rm o}}
\renewcommand{\O}{{\mathcal O}}
\renewcommand{\d}{\text{\rm d}}
\newcommand{\capa}{\text{\rm cap}}
\newcommand{\com}{\text{\rm com}}
\newcommand{\incle}{\omega_{\sigma,\varepsilon}}
\newcommand{\tr}{\text{\rm tr}}
\newcommand{\e}{\varepsilon}
\renewcommand{\exp}{\text{\rm exp}}
\newcommand{\I}{\text{\rm I}}
\newcommand{\Id}{\text{\rm Id}}
\newcommand{\jac}{\text{\rm Jac}}
\newcommand{\Ker}{\text{\rm Ker}}
\newcommand{\calS}{{\mathcal S}}
\newcommand{\calD}{{\mathcal D}}
\newcommand{\calL}{{\mathcal L}}
\newcommand{\calT}{{\mathcal T}}
\newcommand{\calO}{{\mathcal O}}
\newcommand{\calC}{{\mathcal C}}
\newcommand{\calK}{{\mathcal K}}
\newcommand{\Ran}{\text{\rm Ran}}
\newcommand{\Vol}{\text{\rm Vol}}
\newcommand{\Winfty}{W^{1,\infty}(\mathbb{R}^d,\mathbb{R}^d)}
\newcommand{\R}{{\mathbb R}}
\newcommand{\dist}{\text{\rm dist}}
\renewcommand{\det}{\text{\rm det}}
\newcommand{\Jac}{\text{\rm Jac}}
\newcommand{\CD}[1]{{\color{blue}{Charles: #1}}}
\definecolor{darkgreen}{RGB}{1,150,32}
\newcommand{\CB}[1]{{\color{darkgreen}{Carlos:  #1}}}
\newcommand{\D}{{\mathbb D}}
\newcommand{\C}{{\mathbb C}}
\newcommand{\N}{{\mathbb N}}
\newcommand{\calU}{{U}}
\newcommand{\dOmega}{\partial \Omega}
\newcommand{\Per}{\mathrm{Per}}
\newcommand{\uOe}{u_{\Omega,\e}}
\newcommand{\pOe}{p_{\Omega,\e}}
\newcommand{\hAe}{h_{A,\e}}
\newcommand{\hCe}{h_{C,\e}}
\newcommand{\uin}{u_{\mathrm{in}}}
\newcommand{\utot}{u_{\mathrm{tot}}}
\newcommand{\ntop}{n_{\mathrm{top}}}
\newcommand{\Area}{\mathrm{Area}}
\newcommand{\Cont}{\mathrm{Cont}}
\renewcommand{\Re}{\mathrm{Re}}
\renewcommand{\Im}{\mathrm{Im}}
\newcommand{\hsiz}{\texttt{h}}
\newcommand{\hmax}{\texttt{h}_{\texttt{max}}}
\newcommand{\hmin}{\texttt{h}_{\texttt{min}}}
\newcommand{\Diff}[1]{{\color{blue}{#1}}}

%%%%%%%%%%%%%%%%%%%%%%%%%%%%%%%%%%%%%%%%%%%%%

\begin{document}
\newtheorem{theorem}{Theorem}[section]
\newtheorem{remark}{Remark}[section]
\newtheorem{definition}{Definition}[section]
\newtheorem{lemma}{Lemma}[section]
\newtheorem{corollary}{Corollary}[section]
\newtheorem{proposition}{Proposition}[section]
\newtheorem{propdef}{Definition-Proposition}[section]
\newtheorem{example}{Example}[section]
\numberwithin{equation}{section}

\title{Numerical shape and topology optimization of regions supporting the boundary conditions of a physical problem}
\author{
E. Bonnetier\textsuperscript{1}, C. Brito-Pacheco\textsuperscript{2}, C. Dapogny \textsuperscript{2} and R. Estevez\textsuperscript{3}
}
\maketitle


%%%%%%%%%%%%%%%%%%%%%%%%%%%%%%%%%%%%%%%%%%%%%%%%%%%%%%%%%%%%%%%%%%%%%%%%%%%%%%%%%%%%%%%%%%%%%%%%
\begin{center}
\emph{\textsuperscript{1} Institut Fourier, Universit\'e Grenoble-Alpes, BP 74, 38402 Saint-Martin-d'H\`eres Cedex, France}.\\
\emph{\textsuperscript{2} Univ. Grenoble Alpes, CNRS, Grenoble INP, LJK, 38000 Grenoble, France}.\\
\emph{\textsuperscript{3} Univ. Grenoble-Alpes - CNRS UMR 5266, SIMaP, F-38000 Grenoble, France.}
\end{center}
%%%%%%%%%%%%%%%%%%%%%%%%%%%%%%%%%%%%%%%%%%%%%%%%%%%%%%%%%%%%%%%%%%%%%%%%%%%%%%%%%%%%%%%%%%%%%%%%

%%%%%%%%%%%%%%%%%%%%%%%%%%%%%%%%%%%%%%%%%%%%%%%%%%%%%%%%%%%%%%%%%%%%%%%%%%%%%%%%%%%%%%%%%%%%%%%%
 
\begin{abstract}
This article deals with a particular class of shape and topology optimization problems: 
the optimized design is a region $G$ of the boundary $\partial \Omega$ of a given domain $\Omega$,
which supports a particular type of boundary conditions in the state problem characterizing the physical situation.
In our analyses, we develop adapted versions of the notions of shape and topological derivatives, which are classically tailored to functions of a ``bulk'' domain. 
This leads to two complementary notions of derivatives for a quantity of interest $J(G)$ depending on a region $G \subset \partial \Omega$:
on the one hand, we elaborate on the boundary variation method of Hadamard for evaluating the sensitivity of $J(G)$ with respect to ``small'' perturbations of the boundary of $G$ within $\partial \Omega$. 
On the other hand, we use techniques from asymptotic analysis to appraise the sensitivity of $J(G)$ with respect to the addition of a new connected component to the region $G$, shaped as a ``small'' surface disk. 
The calculation of both types of derivatives raises original difficulties, which are closely related to the weakly singular behavior of the solution to a boundary value problem at the points of $\partial\Omega$ where the boundary conditions change types. These aspects are carefully detailed in a simple mathematical setting based on the conductivity equation. 
We notably propose formal arguments to calculate our derivatives with a minimum amount of technicality, and we show how they can be generalized to handle more intricate problems, arising for instance in the physical contexts of acoustics and structural mechanics, respectively governed by the Helmholtz equation and the linear elasticity system. 
In numerical applications, our derivatives are incorporated into a recent algorithmic framework for tracking arbitrarily dramatic motions of a region $G$ within a fixed ambient surface, which combines the level set method with remeshing techniques to offer a clear, body-fitted discretization of the evolving region.
Finally, various 3d numerical examples are presented to illustrate the salient features of our analysis. 
\end{abstract} \par\medskip

%%%%%%%%%%%%%%%%%%%%%%%%%%%%%%%%%%%%%%%%%%%%%%%%%%%%%%%%%%%%%%%%%%%%%%%%%%%%%%%%%%%%%%%%%%%%%%%%


%%%%%%%%%%%%%%%%%%%%%%%%%%%%%%%%%%%%%%%%%%%%%%%%%%%%%%%%%%%%%%%%%%%%%%%%%%%%%%%%%%%%%%%%%%%%%%%%
\bigskip
\hrule
\tableofcontents
\vspace{-0.5cm}
\hrule
\bigskip
\bigskip
%%%%%%%%%%%%%%%%%%%%%%%%%%%%%%%%%%%%%%%%%%%%%%%%%%%%%%%


\include*{Introduction}
\include*{SensitivityDomain}
\include*{ShapeDerivatives}
\include*{TopologicalSensitivity}
\include*{Helmholtz}
\include*{Elasticity}

%%%%%%%%%%%%%%%%%%%%%%%%%%%%%%%%%%%%%%%%%%%%%%%%%%%%%%%
%%%%%%%%%%%%%%%%%%%%%%%%%%%%%%%%%%%%%%%%%%%%%%%%%%%%%%%
\section{Numerical illustrations} \label{sec.Numerical}
%%%%%%%%%%%%%%%%%%%%%%%%%%%%%%%%%%%%%%%%%%%%%%%%%%%%%%%
%%%%%%%%%%%%%%%%%%%%%%%%%%%%%%%%%%%%%%%%%%%%%%%%%%%%%%%

\noindent In this section, we apply our optimization method to several examples of shape and topology optimization problems related to regions bearing the boundary conditions of a physical problem. 
After providing a few general details about our numerical implementation in \cref{subsec.Numerical.Framework}, we describe our treatment of integral equations in \cref{sec.BEM}.
Turning to physical applications, we first address in \cref{sec.CathodeAnode} the optimal design of an electroosmotic mixer, whose behavior is governed by the conductivity equation. 
An instance of such optimal design problems is then considered in the realm of acoustics  in \cref{sec.Acoustics}.
Eventually, two instances of our boundary optimization framework are tackled in the physical setting of mechanical structures in \cref{sec.StructureSupport,sec.ClampingLocator}.

\include*{CouplingMethods}
\include*{CathodeAnode}
\include*{Acoustics}
\include*{StructureSupport}
\include*{ClampingLocator}
\include*{Conclusion}

\include*{Appendix}


%%%%%%%%%%%%%%%%%%%%%%%%%%%%%%%
%
\bibliographystyle{siam}
\bibliography{repbcnum.bbl}
%%%%%%%%%%%%%%%%%%%%%%%%%%%%%%%

\end{document}
