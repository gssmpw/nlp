%%%%%%%%%%%%%%%%%%%%%%%%%%%%%%%%%%%%%%%%%%%%%%%%%%
%%%%%%%%%%%%%%%%%%%%%%%%%%%%%%%%%%%%%%%%%%%%%%%%%%
\section{Optimization of the support of boundary conditions for the linear elasticity system} \label{sec.Elasticity}
%%%%%%%%%%%%%%%%%%%%%%%%%%%%%%%%%%%%%%%%%%%%%%%%%%
%%%%%%%%%%%%%%%%%%%%%%%%%%%%%%%%%%%%%%%%%%%%%%%%%%

\noindent In this section, we adapt the findings of the previous \cref{sec.optbcconduc,sec.TopologicalSensitivity,sec.Helmholtz} to the realm of structural mechanics, 
governed by the system of linear elasticity. The calculation of the (exact or approximate) shape derivative of a functional depending on a region supporting the boundary conditions of this system follows exactly the trail conducted in \cref{sec.optbcconduc}, up to an increased level of tediousness. In contrast, the asymptotic analysis of the effet of singular changes in such regions -- and thereby, the calculation of topological derivatives -- involves non trivial specificities. 
For this reason, in this section, we focus on the calculation of topological derivatives of functionals of boundary regions featured in the linear elasticity equations, and we refer to the applications of \cref{sec.StructureSupport,sec.ClampingLocator} for typical shape derivative formulas in this context and illustrations of their practical use. 
After presenting the mathematical model at stake in \cref{sec.linelas} and a few technical preliminaries in \cref{sec.linelasGreen}, 
we discuss in \cref{sec.repelas,sec.qoielas} the calculation of the sensitivity of the elastic displacement and that of a related quantity of interest with respect to the addition of a vanishingly small Dirichlet region. 
Multiple variations of this model are available, which can be treated by simple adaptations of this material, see \cref{rem.othermodelselas}. 
In order to emphasize the parallel between the present discussions and those in \cref{sec.TopologicalSensitivity,sec.Helmholtz}, we retain our previous notations whenever possible. 

%%%%%
\subsection{Presentation of the linear elasticity setting}\label{sec.linelas}
%%%%%

\noindent In this section, $\Omega$ stands for a mechanical structure in $\R^d$ ($d=2,3$), whose boundary $\partial \Omega$ is composed of three disjoint pieces:
$$ \partial \Omega = \overline{\Gamma_D} \cup \overline{\Gamma_N} \cup \overline \Gamma, $$
where:
\begin{itemize}
\item The region $\Gamma_D$ is held fixed;
\item The region $\Gamma_N$ is subjected to surface loads $g: \R^d \to \R^d$; 
\item The remaining part $\Gamma$ is traction-free.
\end{itemize}
Assuming smooth body forces $f : \R^d \to \R^d$, the displacement $u$ of $\Omega$ is the unique solution in the space $H^1(\Omega)^d$ to the system of linear elasticity:
\begin{equation}\label{eq.elasbg}
\left\{
\begin{array}{cl}
-\dv (Ae(u)) = f & \text{in } \Omega, \\
Ae(u) n = g & \text{on } \Gamma_N,\\
Ae(u) n = 0 & \text{on } \Gamma,\\
u = 0 & \text{on } \Gamma_D.
\end{array}
\right.
\end{equation}
Here, $e(u) := \frac12 (\nabla u + \nabla u ^T)$ is the strain tensor associated to a displacement field $u : \Omega \to \R^d$, and $A$ is the Hooke's tensor, defined by:
$$ \text{For all symmetric } d \times d \text{ matrix } e, \quad Ae = 2\mu e + \lambda \tr(e) \I,$$
where $\lambda$ and $\mu$ are the Lam\'e coefficients of the elastic material. Note that these coefficients are often better expressed in terms of the more physical Young's modulus $E$ and Poisson's ratio $\nu$: 
$$ \mu = \frac{E}{2(1+\nu)}, \text{ and } \lambda = \left\{
\begin{array}{cl}
\frac{E\nu}{(1+\nu)(1-2\nu)} & \text{if } d =3, \\
\frac{E\nu}{1-\nu^2} & \text{if } d =2 \text{ (plane stress)}.
\end{array}
\right.$$

We refer to classical treaties such as \cite{gould1994introduction} for more exhaustive introductions to the physical context of linear elasticity. 

%%%%%
\subsection{The fundamental solution to the linear elasticity system}\label{sec.linelasGreen}
%%%%%

\noindent The fundamental solution to the linear elasticity operator $u \mapsto -\dv(Ae(u))$ in the free space $\R^d$ is the Kelvin matrix, defined by 
\begin{equation}\label{eq.kelvin}
\forall i,j = 1,\ldots,d, \quad \Gamma_{ij}(x,y) = \left\{
\begin{array}{cl}
\frac{\alpha}{4\pi} \frac{\delta_{ij}}{|x-y|} + \frac{\beta}{4\pi} \frac{(x_i-y_i)(x_j-y_j)}{|x-y|^3} & \text{if } d =3, \\
-\frac{\alpha}{2\pi} \delta_{ij} \log|x-y|+ \frac{\beta}{2\pi} \frac{(x_i-y_i)(x_j-y_j)}{|x-y|^2} & \text{if } d =2, \\
\end{array}
\right.
\end{equation}
where the constants $\alpha$ and $\beta$ are given by:
$$ \alpha= \frac{1}{2}\left( \frac{1}{\mu} + \frac{1}{2\mu+\lambda} \right), \text{ and } \beta = \frac{1}{2}\left( \frac{1}{\mu} - \frac{1}{2\mu+\lambda} \right),$$
see e.g. \cite{ammari2007polarization,kupradze2012three}. 
This means that, for each point $x \in \R^d$ and each index $j=1,\ldots,d$, the $j^{\text{th}}$ column vector $\Gamma_j(x,\cdot)$ of $\Gamma(x,\cdot)$ is the solution to the equation:
$$ -\dv_y(Ae_y(\Gamma_j(x,y))) = \delta_{y=x} e_j  \text{ in the sense of distributions in } \R^d,$$
where $e_j$ is the $j^{\text{th}}$ vector of the canonical basis of $\R^d$. 
From the physical point of view, if $a \in \R^d$ is a vector, $y \mapsto \Gamma(x,y) a$ is the displacement resulting from a point load $a$ applied at $x$.  

The variational counterpart of \cref{eq.kelvin} reads as follows: for a sufficiently smooth function $\varphi \in {\mathcal C}_c^\infty(\R^d)^d$, it holds:
$$\forall x \in \R^d, \quad \varphi_j(x) = \int_{\R^d} Ae_y(\Gamma_j(x,y)) : e(\varphi)(y) \:\d y.$$

Using the fundamental solution \cref{eq.kelvin}, it is possible to construct the Green's function $N(x,y)$ for the problem \cref{eq.elasbg}: 
for $j=1,\ldots,d$, the $j^{\text{th}}$ column $N_j$ of $N$ satisfies the equation
\begin{equation}\label{eq.Greenfuncelas}
\left\{ 
\begin{array}{cl}
-\dv_y (Ae_y(N_j(x,y)))= \delta_{y=x} e_j & \text{in } \Omega, \\ 
N_j(x,y) = 0 & \text{for } y\in \Gamma_D, \\
Ae_y(N_j(x,y)) = 0 & \text{for } y\in \Gamma \cup \Gamma_N.
\end{array}
\right.
\end{equation}
This system has the following variational characterization: 
\begin{equation}\label{eq.varfuncGreen}
 \text{For all } \varphi \in {\mathcal C}^\infty_c(\R^d)^d \text{ s.t. } \varphi =0 \text{ on } \Gamma_D, \quad \varphi_j(x) = \int_\Omega Ae_y(N_j(x,y)) : e(\varphi)(y) \:\d y.
 \end{equation}
 
 \begin{remark}\label{rem.mindlin}
 As in \cref{sec.TopologicalSensitivity,sec.Helmholtz}, we shall need another Green's function $L(x,y)$ for the linear elasticity system \cref{eq.elasbg}, tuned to the lower half-space $H$ (see \cref{eq.LHS}), satisfying homogeneous Neumann conditions on $\partial H$. Unfortunately, due to the vector-valued nature of the elasticity system, the expression of $L(x,y)$ cannot 
 be obtained from that of $\Gamma(x,y)$ by the method of images. 
This function -- called the Mindlin function in the literature -- has the following expression in 3d, for two points $x \neq y$ on $\partial H$: 
\begin{equation}\label{eq.Mindlin1}
 L_{ij}(x,y) = \frac{1-\nu}{2\pi \mu \lvert x - y \lvert} \delta_{ij}  + \frac{\nu}{2\pi \mu} \frac{(x_i - y_i)(x_j - y_j)}{\lvert x - y \lvert^3}, \:\: i,j = 1,2,
 \end{equation}
\begin{equation}\label{eq.Mindlin2}
 L_{3j}(x,y) = -\frac{1-2\nu}{4\pi\mu} \frac{x_j - y_j}{\lvert x- y \lvert^2}, \: j=1,2, \text{ and }
L_{33}(x,y) = \frac{1-\nu}{2\pi \mu\lvert x - y \lvert}.
 \end{equation}
We refer to \cite{mindlin1936force} for the original derivation of this formula, see also \cite{balavs2013stress,mura2013micromechanics}.
Note that, in these references, the formulas look different, as the Green's function is provided for the upper half-space.

In 2d, setting $\overline\nu = \frac{\nu}{1+\nu}$, the components of this Green's function read, according to \cite{balavs2013stress} \S 2.8.2:
%%% Formules dans le Balas
%$$ L_{11}(x,y) = -\frac{1-\overline \nu }{\pi \mu} \log\lvert x- y \lvert$$
%$$ L_{12}(x,y) = - \frac{1}{2\pi \mu}(1-2\overline \nu) \Theta, \text{ where } \Theta = \left\{ 
%\begin{array}{cl}
%\frac{\pi}{2} & \text{if } x_2 > y_2 , \\
%-\frac{\pi}{2} & \text{otherwise},
%\end{array}
%\right. \text{ and } L_{22}(x,y) =  -\frac{1-\overline \nu }{\pi \mu} \log\lvert x- y \lvert + \frac{3-4\overline \nu}{8\pi \mu(1-\overline \nu)}.$$
%% Après changement de variables
\begin{equation}\label{eq.Mindlin2d1}
 L_{11}(x,y) =  -\frac{1-\overline \nu }{\pi \mu} \log\lvert x- y \lvert + \frac{3-4\overline \nu}{8\pi \mu(1-\overline \nu)},
 \end{equation}
\begin{equation}\label{eq.Mindlin2d2}
 L_{12}(x,y) = - \frac{1}{2\pi \mu}(1-2\overline \nu) \Theta, \text{ where } \Theta = \left\{ 
\begin{array}{cl}
\frac{\pi}{2} & \text{if } x_1 > y_1 , \\
-\frac{\pi}{2} & \text{otherwise},
\end{array}
\right. \text{ and } L_{22}(x,y) =  -\frac{1-\overline \nu }{\pi \mu} \log\lvert x- y \lvert.
\end{equation}
 \end{remark}

%%%%%%%%%%%%%%%%%%%%%%%%%%%%%%%%%%%%%%%%%%%%%%%%%%%%%%%
\subsection{Asymptotic expansion of the elastic displacement under singular perturbations of the Dirichlet region}\label{sec.repelas}
%%%%%%%%%%%%%%%%%%%%%%%%%%%%%%%%%%%%%%%%%%%%%%%%%%%%%%%

\noindent In this section, we consider the situation where the reference ``background'' problem \cref{eq.elasbg}, whose solution is as usual denoted by $u_0$,  
is perturbed by the replacement of  the homogeneous Neumann boundary conditions by homogeneous Dirichlet boundary conditions on a small surface disk $\omega_{x_0,\e} \subset \Gamma_N$ centered at a given point $x_0 \in \Gamma$. The displacement $u_\e$ of $\Omega$ in this perturbed situation is the unique solution to the following boundary value problem:
\begin{equation}\label{eq.pertelas}
\left\{ 
\begin{array}{cl}
-\dv (Ae(u_\e)) = f & \text{in } \Omega, \\ 
u_\e = 0 & \text{on } \Gamma_D \cup \omega_{x_0,\e}, \\
Ae(u_\e)n = g & \text{on } \Gamma_N,\\
Ae(u_\e)n = 0 & \text{on } \Gamma \setminus \overline{\omega_{x_0,\e}}.
\end{array}
\right.
\end{equation}

The main result is the following. 

\begin{theorem}\label{th.expelas}
Let $x_0$ be a given point in $\Gamma$. Then the following asymptotic expansions hold true about the perturbed displacement $u_\e$ in \cref{eq.pertelas}:
$$ u_{\e,j}(x) = u_{0,j}(x) -  \frac{1}{\lvert \log \e \lvert }\frac{\pi \mu}{1-\overline\nu} u_0(x_0) \cdot N_j(x,x_0) + \o\left( \frac{1}{\lvert\log \e \lvert}\right), \quad j=1,2, \text{ if }d=2,$$
and 
$$ u_{\e,j}(x) = u_{0,j}(x) -  \e M u_0(x_0) \cdot N_j(x,x_0) + \o(\e), \quad j=1,2,3, \text{ if } d =3.$$
In the above formula, $M \in \R^{3\times 3}$ is a polarization tensor, whose entries read:
\begin{equation}\label{eq.defMelas}
 M_{ij} = \int_{\D_1} T_L^{-1} e_j \cdot e_i \:\d s, \quad i,j=1,2,3.
 \end{equation}
This definition involves the integral operator
\begin{equation}\label{eq.defTLelas}
T_L : \widetilde{H}^{-1/2}(\D_1)^d \to H^{1/2}(\D_1)^d, \:\:  T_L \varphi(x) = \int_{\D_1} L(x,z) \varphi(z) \:\d s(z), \quad x \in \D_1, 
 \end{equation}
whose kernel is the Mindlin function $L(x,y)$ given in \cref{eq.Mindlin1,eq.Mindlin2}. 
\end{theorem}\par\smallskip

\begin{remark}
\noindent \begin{itemize}
\item The above definition of the polarization tensor $M$ in \cref{eq.defMelas} hinges on the invertibility of the operator $T_L$ in \cref{eq.defTLelas}. 
To the best of our knowledge, this fact is not known in the literature, and it is not a straightforward adaptation of the counterpart result in the setting of the conductivity equation (see \cref{prop.SLP}). 
The derivation presented below proceeds formally under the assumption that this fact holds true.
\item The operator $T_L$ has an enlightening physical interpretation: $T_L \varphi : \D_1 \to \R^d$ is the displacement induced by a force $\varphi : \D_1 \to \R^d$ applied to the unit disk $\D_1$. Hence, for $j=1,2,3$, the $j^{th}$ column of the polarization tensor $M$ is the total magnitude of the force that should be applied on $\D_1$ to realize a uniform, unit displacement in the direction $e_j$. Interestingly, ``classical'' calculations in contact mechanics allow to identify some of the entries of this tensor, see for instance the so-called ``flat punch'' or ``indentation'' problems in \cite{barber2018contact,johnson1987contact,krenk1979circular}. However, to the best of our knowledge, the complete analytical calculation of $M$ is not available in the literature, and this task has to be realized numerically: see \cref{sec.BEM} about the numerical method employed to solve the integral equation associated to $T_L$. 
\end{itemize}
\end{remark}

\begin{proof}[Sketch of the proof]
Again, we rely on a formal calculation along the trail of \cref{th.Conductivity.HNHD.Expansion}, under the assumption \cref{eq.flat} that $x_0 = 0$, $n(0) = e_d$, and that the boundary $\partial \Omega$ is flat in a neighborhood of $0$. 
Let us introduce the error $r_\e = u_\e - u_0 \in H^1(\Omega)^d$, which is the unique solution to the following boundary value problem: 
\begin{equation}\label{eq.reelas}
\left\{ 
\begin{array}{cl}
-\dv (Ae(r_\e)) = 0 & \text{in } \Omega, \\ 
r_\e = 0 & \text{on } \Gamma_D, \\
r_\e = -u_0 & \text{on } \omega_\e, \\
Ae(r_\e)n = 0 & \text{on } (\Gamma_N \cup \Gamma) \setminus \overline{\omega_\e}.
\end{array}
\right.
\end{equation}
 \par\medskip 

\noindent \textit{Step 1: We construct a representation formula for the value $r_\e(x)$ of the error at an arbitrary point $x\in \overline\Omega \setminus \left\{ 0 \right\}$ in terms of its values inside $\omega_\e$.}

To this end, we use the definition \cref{eq.Greenfuncelas} of the Green's function $N(x,y)$ for the background elasticity problem \cref{eq.elasbg}, which yields, for each component $j=1,\ldots,d$: 
$$\begin{array}{>{\displaystyle}cc>{\displaystyle}l}
 r_{\e,j}(x) &=& - \int_\Omega{\dv_y (Ae_y(N_j(x,y))) \cdot r_\e(y) \:\d y}  \\[1em]
 &=& -\int_{\partial\Omega} Ae_y(N_j(x,y)) n(y) \cdot r_\e(y) \:\d s(y) + \int_\Omega Ae_y(N_j(x,y)) : e(r_{\e})(y) \:\d y.
 \end{array}
 $$
Using the boundary conditions in \cref{eq.Greenfuncelas,eq.reelas} for the functions $r_\e$ and $y \mapsto N(x,y)$, as well as another integration by parts, we obtain:
 $$ r_{\e,j}(x) =- \int_\Omega {\dv (Ae(r_{\e})) (y)\cdot N_j(x,y) \:\d y} + \int_{\partial \Omega}{Ae(r_{\e})(y)n(y) \cdot N_j(x,y) \:\d s(y)}.$$
Invoking again the problem satisfied by $r_\e$ in \cref{eq.reelas}, we end up with
 $$  r_{\e,j}(x) =  \int_{\omega_\e}{Ae(r_{\e})(y)n(y) \cdot N_j(x,y) \:\d s(y)}.$$
By a change of variables in the above integral, we obtain the desired representation formula:
 \begin{equation}\label{eq.repreelas}
   r_{\e,j}(x) =  \int_{\D_1}{\varphi_\e(z) \cdot N_j(x,\e z) \:\d s(z)},
   \end{equation}
 where we have introduced the rescaled quantity:
 $$ \varphi_\e (z) = \e^{d-1} \Big( Ae(r_{\e})n\Big)(\e z) \in \widetilde{H}^{-1/2}(\D_1)^d.$$
  \par\medskip 

\noindent \textit{Step 2: We construct an integral equation characterizing the function $\varphi_\e(z)$.}

As in the proof of \cref{th.Conductivity.HNHD.Expansion}, to achieve this, we repeat the above calculation, this time with the explicit Green's function $L(x,y)$ for the lower half space $H$, equipped with homogeneous Neumann boundary conditions on $\partial H$ (see \cref{rem.mindlin}), in place of the ``difficult'' Green's function $N(x,y)$ for the background equation \cref{eq.elasbg}. This yields the integral equation:
\begin{equation}\label{eq.repphieelas}
 \forall x \in \D_1, \quad    \int_{\D_1}{L(\e x,\e z) \:\varphi_\e(z) \:\d s(z)} = -u_0(\e x) + \o(1).
   \end{equation}

\par\medskip
\noindent \textit{Step 3: We use this integral equation to glean information about the function $\varphi_\e(z)$.}

This task proceeds differently, depending on the space dimension:

\begin{itemize}
\item In 2d, using the expression \cref{eq.Mindlin2d1,eq.Mindlin2d2} of $L(x,y)$, \cref{eq.repphieelas} rewrites: 
$$ \frac{1-\overline\nu}{\pi\mu} (\log\e) \int_{\D_1}\varphi_\e(y)\:\d s(y) + K \varphi_\e(y) = -u_0(\e x) + \o(1),$$
where $K : \widetilde{H}^{-1/2}(\D_1)^d \to H^{1/2}(\D_1)^d$ is a bounded operator.
We directly obtain from this equation that: 
\begin{equation}\label{eq.intvphie2delas}
\int_{\D_1}\varphi_\e(y)\:\d s(y) =  \frac{1}{\lvert \log \e \lvert }\frac{\pi \mu}{1-\overline\nu} u_0(0) + \o\left( \frac{1}{\lvert\log \e \lvert}\right),
\end{equation}
which is the needed information for our purpose. 
\item In 3d, using the homogeneity of the kernel $L(x,y)$ in \cref{eq.Mindlin1,eq.Mindlin2}, \cref{eq.repphieelas} becomes: 
$$ T_L \varphi_\e = - \e u_0(0) + \o(\e),$$
where the integral operator $T_L$ is defined in \cref{eq.defTLelas}. This immediately yields: 
$$ \varphi_\e = \sum\limits_{j=1}^3 u_{0,j}(0) T_L^{-1} e_j,$$
and so:
\begin{equation}\label{eq.intvphie3delas}
 \int_{\D_1} \varphi_\e \:\d s = M u_0(0),
 \end{equation}
involving the polarization tensor $M$ defined in \cref{eq.defMelas}. 
\end{itemize}
\par\medskip
\noindent \textit{Step 4: We pass to the limit in the representation formula \cref{eq.repreelas}.}

Applying the Lebesgue dominated convergence theorem to the representation formula \cref{eq.repreelas}, we obtain:
$$ r_{\e,j}(x) = \left( \int_{\D_1} \varphi_\e(z) \:\d s(z) \right) \cdot \Big(N_j(x,0) + \o(1) \Big),$$
and the desired result follows from the combination of this identity with \cref{eq.intvphie2delas,eq.intvphie3delas}.
\end{proof}


%%%%%%%%%%%%%%%%%%%%%%%%%%%%%%%%%%%%%%%%%%%%%%%%%%%%%%%
\subsection{Sensitivity of a quantity of interest with respect to the addition of a Dirichlet region}\label{sec.qoielas}
%%%%%%%%%%%%%%%%%%%%%%%%%%%%%%%%%%%%%%%%%%%%%%%%%%%%%%%

\noindent Let us consider the following quantity of interest, depending on the region $\Gamma_D$ of $\partial \Omega$ supporting the homogeneous Dirichlet boundary conditions of the problem \cref{eq.elasbg}: 
$$ J(\Gamma_D) = \int_\Omega j(u_{\Gamma_D}) \:\d x,$$
for a smooth function $j : \R^d \to \R$, satisfying suitable growth conditions, see \cref{eq.jgrowth}. 
Here, $u_{\Gamma_D}$ denotes the solution to the boundary value problem \cref{eq.elasbg}.

\begin{corollary}\label{cor.ElasJp}
The perturbed value $J((\Gamma_D)_{x_0,\e})$ of $J(\Gamma_D)$, accounting for the replacement of the homogeneous Neumann boundary condition on $\omega_{x_0,\e} \subset \Gamma$ by a homogeneous Dirichlet boundary condition, has the following asymptotic expansion:
$$
    J((\Gamma_D)_{x_0,\e}) =  J(\Gamma_D) +   \frac{1}{\lvert \log \e \lvert }\frac{\pi \mu}{1-\overline\nu} u_0(x_0) \cdot p_0(x_0) +  \o\left(\frac{1}{\lvert\log\e\lvert} \right)\text{ if } d = 2,
$$
and 
 $$
   J((\Gamma_D)_{x_0,\e}) =    J(\Gamma_D) +   \e \: Mu_0(x_0) \cdot p_0(x_0)  + \o(\e) \text{ if }  d = 3.
 $$
Here, the polarization tensor $M$ is defined by \cref{eq.defMelas} and the adjoint state $p_0$ is the unique solution in $H^1(\Omega)^d$ to the boundary value problem:
\begin{equation}\label{eq.ElasAdjoint}
\left\{ 
\begin{array}{cl}
-\dv(Ae(p_0)) = -j^\prime(u_0)& \text{in } \Omega, \\ 
p_0 = 0 & \text{on } \Gamma_D , \\
Ae(p_0)n = 0 & \text{on } \Gamma_N \cup \Gamma.
\end{array}
\right.
\end{equation}
\end{corollary}

\begin{remark}\label{rem.othermodelselas}
As in the case of electrostatics addressed in \cref{sec.TopologicalSensitivity}, multiple variations of the present study could be considered. For instance, one may be interested in accounting for the effect of the replacement of homogeneous Neumann boundary conditions by inhomogeneous Dirichlet or Neumann conditions, etc.
Numerical examples associated to such variations are presented in \cref{sec.ClampingLocator}.
\end{remark}