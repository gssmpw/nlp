%%%%%%%%%%%%%%%%%%%%%%%%%%%%%%%%%%%%%%%%%%%%%%%%%%%%%%%
\subsection{The case of the conductivity equation} \label{subsec.Conductivity}
%%%%%%%%%%%%%%%%%%%%%%%%%%%%%%%%%%%%%%%%%%%%%%%%%%%%%%%


  In this section, we address the case of the conductivity equation, which is arguably the simplest case to analyze and serves as a baseline for analyzing more complex equations. Here, $n = 1$, $V = H^1(\Omega)$, and $S = \varnothing$. We assume that the domain $\Omega$ is occupied by a medium with smooth isotropic conductivity $\gamma \in C^\infty(\overline{\Omega})$, satisfying the following bounds:
\begin{equation}
    \forall x \in \Omega, \: \alpha \leq \gamma(x) \leq \beta,
\end{equation}
for some fixed constants $0 < \alpha \leq \beta$.
In the background scenario, the voltage potential $u_0 \in H^1(\Omega)$, in response to a smooth external source $f \in C^\infty(\overline{\Omega})$, is the unique solution to the following boundary value problem:
\begin{equation}\label{eq.Conductivity.Background}
\left\{
\begin{aligned}
-\dv(\gamma\nabla u_0) &= f && \text{in } \Omega, \\
u_0 &= 0 && \text{on } \Gamma_D, \\
\gamma \frac{\partial u_0}{\partial n} &= 0 && \text{on } \Gamma_N.
\end{aligned}
\right.
\end{equation}
In accordance to the classical elliptic regularity theory, the voltage potential $u_0$ is smooth except maybe at the transition zone $\Sigma = \overline{\Gamma_D} \cap \overline{\Gamma_N}$, where the boundary condition changes.

%%%%%
\subsubsection{Green's function of the background equation}
%%%%%

 The primary tool for analyzing $r_\epsilon$ is the concept of Green's function. We begin by recalling the expression for the \textit{fundamental solution} $\Gamma(x)$ of the operator $-\Delta$ in free space:
\begin{equation*}
    F(x) = \left\{
    \begin{array}{cl}
    -\frac{1}{2\pi} \log |x| & \text{if } d = 2, \\[0.2em]
    \frac{1}{(d-2) \alpha_d} |x|^{2-d} & \text{if } d \geq 3,
    \end{array}
    \right.
\end{equation*}
where $\alpha_d$ is the area of the unit sphere $\mathbb{S}^{d-1} \subset \mathbb{R}^d$. The function $F$ satisfies:
\begin{equation*}
    -\Delta F = \delta_0 \text{ in the sense of distributions on } \mathbb{R}^d.
\end{equation*}
We define $F(x, y) := F(|x - y|)$ as refer to it as the \textit{Green's function for the Laplace equation in the free space}. From this, one can construct the Green's function $N(x,y)$ for the background equation \cref{eq.Conductivity.Background}. This function satisfies, for all $x \in \Omega$:
\begin{equation}\label{eq.Greenconduc}
\left\{
\begin{aligned}
    -\text{div}_y(\gamma(y)\nabla_y N(x,y)) &= \delta_{y=x} && \text{in } \Omega, \\
    N(x,y) &= 0 && \text{for } y \in \Gamma_D, \\
    \gamma(y) \frac{\partial N}{\partial n_y}(x,y) &= 0 && \text{for } y \in \Gamma_N.
\end{aligned}
\right.
\end{equation}
\begin{remark}
    The function $N(x,y)$ is symmetric in its arguments (see \cite{folland1995introduction} for a proof) and its related to the fundamental solution $\Gamma(x, y)$ via the relation:
    \begin{equation}
        N(x, y) = \dfrac{1}{\gamma(x)} F(x, y) + R(x, y),
    \end{equation}
    where $R(x, y)$ is referred to as the corrector term, solution to:
    \begin{equation}
    \left\{
        \begin{aligned}
        -\dv_y(\gamma(y)\nabla_y R(x,y)) &= \dfrac{1}{\gamma(y)} \nabla \gamma(y) \cdot \nabla_y F(x, y) && \text{in } \Omega, \\
        R(x, y) &= - \dfrac{1}{\gamma(y)} F(x, y) && \text{for } y \in \Gamma_D, \\
        \gamma(y) \frac{\partial R}{\partial n_y}(x,y) &= \dfrac{\gamma(y)}{\gamma(x)} \frac{\partial G}{\partial n_y}(x, y) && \text{for } y \in \Gamma_N.
        \end{aligned}
    \right.
    \end{equation}
    The functional characterization of the corrector term $R(x, y)$ depends on the singularity of $G(x, y)$. However, it is known that $y \mapsto R(x, y)$ belongs at least to $H^1(\Omega)$. Additionally, for every open subset $U$ compactly contained in $\mathbb{R}^d \setminus (\Sigma \cup \{ x \})$, it is of class $C^\infty$ on $\overline{\Omega} \cap U$. For more details on the characterization of the corrector term and the Green's function, refer to the standard texts \cite{brezis2010functional, gilbarg2015elliptic}.
\end{remark}
The key property of $N(x, y)$ that we will utilize is that, for any function $\varphi \in C^1(\overline{\Omega})$ such that $\varphi = 0$ on $\Gamma_D$, the following holds:
\begin{equation} \label{eq.Conductivity.PhiN}
    \varphi (x) = \int_\Omega \gamma(y) \nabla_y N(x, y) \cdot \nabla \varphi (y) \:\d y, \quad x \in \Omega.
\end{equation}
In particular, one may integrate by parts to express the solution to \cref{eq.Conductivity.Background} in terms of $N(x, y)$ as:
\begin{equation}\label{eq.u0N}
u_0(x) = \int_\Omega f(y) N(x,y) \:\mathrm{d} y.
\end{equation}

%%%%%
\subsubsection{Replacement of homogeneous Neumann boundary conditions by homogeneous Dirichlet conditions}
%%%%%

We assume that $\Gamma = \Gamma_N$ and $G = \Gamma_D$, meaning the homogeneous Neumann boundary condition on the small disk $\omega_\epsilon \subset \Gamma$ is replaced by the homogeneous Dirichlet condition defined on $G$. In this case, the perturbed potential $u_\e \in H^1_{\Gamma_D}(\Omega)$ is then the unique solution to:
\begin{equation}\label{eq.Conductivity.HNHD.Ue}
\left\{
\begin{aligned}
-\dv(\gamma\nabla u_\e) &= f && \text{in } \Omega, \\
u_\e &= 0 && \text{on } \Gamma_D \cup \omega_\e, \\
\gamma \frac{\partial u_\e}{\partial n} &= 0 && \text{on } \Gamma_N \setminus \overline{\omega_\e}.
\end{aligned}
\right.
\end{equation}
Furthermore, we have that $r_\e = u_\e - u_0$ satisfies the boundary value problem:
\begin{equation}\label{eq.Conductivity.HNHD.Re}
\left\{
\begin{array}{cl}
-\dv(\gamma\nabla r_\e) = 0 & \text{in } \Omega, \\
r_\e = 0 & \text{on } \Gamma_D, \\
r_\e = -u_0 & \text{on } \omega_\e,\\
\gamma \frac{\partial r_\e}{\partial n} = 0 & \text{on } \Gamma_N \setminus \overline{\omega_\e}.\\
\end{array}
\right.
\end{equation}
We begin the analysis by introducing the Green's function $L(x, y)$ for the version of \cref{eq.Conductivity.Background} posed on the lower half-space $H$ with homogeneous Neumann boundary conditions imposed on $\partial H$. For all $x \in H$, $y \mapsto L(x, y)$ satisfies:
\begin{equation}\label{eq.Conductivity.HNHD.L}
    \left\{
    \begin{array}{cl}
    -\dv_y(\gamma(y) \nabla_y L(x,y)) = \delta_{y=x} & \text{in } H, \\[0.2em]
    \gamma(y) \frac{\partial L}{\partial n_y}(x,y) = 0 & \text{for } y \in \partial H.
    \end{array}
    \right.
\end{equation}
The existence of this function can be established via the so called method of images, which yields the following construction:
\begin{equation} \label{eq.Conductivity.HDHD.LI}
    L(x, y) = \dfrac{1}{\gamma} \left( F(x - y) + F(x + y) \right).
\end{equation}
It is straightforward to see that the definition above satisfies \cref{eq.Conductivity.HNHD.L}. Similarly to $N(x, y)$, for $\varphi \in C^1(\overline{\Omega})$, the following formula holds:
\begin{equation}
    \varphi (x) = \int_\Omega \gamma(y) \nabla_y L(x, y) \cdot \nabla \varphi (y) \:\mathrm{d} y, \quad x \in \Omega.
\end{equation}

\begin{lemma}
    The following properties hold true:
    \begin{enumerate}
        \item The function $L(x, y)$ is a homogeneous kernel of class $-1$.
        \item The operator $T_{\e, L} : H^{-1/2}(\D_1) \rightarrow H^{1/2}(\D_1)$ defined by
        \begin{equation}
            T_{\e, L} \varphi (x) := \int_{\D_1} L(\e x, \e y) \varphi(y) \:\d s(y), \quad \forall x \in \D_1,
        \end{equation}
        is bounded and is invertible.
    \end{enumerate}
    
\end{lemma}

The main result in this setting is the following asymptotic expansion, which was rigorously proved in \cite{bonnetier2022small} under weaker hypotheses. As mentioned in \cref{subsec.Conductivity.FlatBoundary}, using a flat boundary does not affect the final result. For convenience, we will provide a formal sketch of the proof here.
\begin{theorem} \label{th.Conductivity.HNHD.Expansion}
For any point $x \in \overline{\Omega} \setminus (\Sigma \cup \left\{ 0 \right\})$, the following asymptotic expansion holds:
\begin{equation}
    r_\e (x) = \left\{
    \begin{aligned}
        &- \frac{\pi}{\lvert \log \e \lvert} \: \gamma(0) \: u_0(0) \: N(x,0) + \o\left(\frac{1}{\lvert\log\e\lvert} \right) && \mathrm{if} \ d = 2,\\
        &- 4 \e \: \gamma(0) \: u_0(0) \: N(x,0) + \o (\e) && \mathrm{if} \ d = 3.
    \end{aligned}
    \right.
\end{equation}
\end{theorem}
\begin{proof}
We now proceed in four steps to derive the asymptotic behavior of $r_\e$.\\

\textit{Step 1}: We construct a representation formula for the values of $r_\e$ ``far'' from $0$ in terms of its values inside the region $\omega_\e$.
This task starts from the integral representation of $r_\e$ with the help of the Green's function $N(x,y)$ defined in \cref{eq.Greenconduc}. For any $x \in \Omega$, it holds: 
$$ r_\e(x) = -\int_\Omega \dv_y(\gamma(y) \nabla_y N(x,y)) r_\e(y) \:\d y.$$ 
By integrating by parts twice in the above expression, we successively obtain:
\begin{align}
    r_\e(x)
    &= - \int_{\partial \Omega} \gamma(y) \frac{\partial N}{\partial n_y}(x,y) r_\e(y) \:\d s(y) + \int_\Omega \gamma(y) \nabla_y N(x,y) \cdot \nabla r_\e(y) \:\d y\\
    &= - \int_{\partial \Omega} \gamma(y) \frac{\partial N}{\partial n_y}(x,y) r_\e(y) \:\d s(y) + \int_{\partial \Omega} \gamma(y) \frac{\partial r_\e}{\partial n}(y) N(x,y) \:\d s(y).
\end{align}
Now, considering the boundary conditions satisfied by $r_\epsilon$ and $y \mapsto N(x, y)$, the first integral on the right-hand side vanishes, as does the contribution of the second integral on $\Gamma_D$ and $\Gamma_N \setminus \overline{\omega_\epsilon}$. Thus, we obtain:
\begin{equation}
    r_\e(x) = \int_{\omega_\e} \gamma(y) \frac{\partial r_\e}{\partial n}(y) N(x,y)  \:\d s (y). 
\end{equation}
Applying a change of variables in the above integral, we obtain:
\begin{equation} \label{eq.Conductivity.HNHD.Rep}
 r_\e(x) = \int_{\D_1} \varphi_\e(z) N(x,\e z)  \:\d s (z), 
\end{equation}
where we have introduced the function $\varphi_\epsilon(z) := \epsilon^{d-1}\left(\gamma\frac{\partial r_\epsilon}{\partial n}\right)(\epsilon z) \in \widetilde{H}^{-1/2}(\mathbb{D}_1)$. This provides the desired representation formula.\\

 \textit{Step 2}: We characterize the function $\varphi_\e$ by an integral equation.
To achieve this, we utilize the Green's function $L(x, y)$ defined on the lower half-space, which proves useful for obtaining a precise characterization due to its explicit expression. Again, integrating by parts, we obtain that for any point $x \in \Omega$ that is far from 0: 
\begin{align}
r_\e(x)
&= -\int_\Omega \dv_y (\gamma(y) \nabla_y L(x,y)) r_\e(y) \:\d y\\
&= -\int_{\partial \Omega} \gamma(y) \frac{\partial L}{\partial n_y}(x,y) r_\e(y) \:\d s(y) + \int_\Omega \gamma(y) \nabla_y L(x,y) \cdot \nabla r_\e(y) \:\d y \\
&= -\int_{\partial \Omega} \gamma(y) \frac{\partial L}{\partial n_y}(x,y) r_\e(y) \:\d s(y) + \int_{\partial \Omega} \gamma(y) \frac{\partial r_\e}{\partial n}(y) L(x,y)  \:\d s (y).
\end{align}
Now using the fact that $\partial \Omega$ coincides with $\partial H$ in a neighborhood $\calO$ of $0$, and the boundary conditions satisfied by $r_\e$, we obtain: 
\begin{multline*}
 r_\e(x) =   \int_{\omega_\e} \gamma(y) \frac{\partial r_\e}{\partial n}(y) L(x,y)  \:\d s (y) + K r_\e(x), \text{ where }\\
 K r_\e(x) :=  -\int_{\Gamma_N \setminus \overline{\calO}} \gamma(y)\frac{\partial L}{\partial n_y}(x,y) r_\e(y) \:\d s(y)  + \int_{\Gamma_D}  \gamma(y) \frac{\partial r_\e}{\partial n}(y) L(x,y)  \:\d s (y).
\end{multline*}
Changing variables in the integral above, and letting $x$ tend to $\omega_\e$ (with the continuity of the single layer potential), we obtain
$$ \forall x \in \D_1, \:\: -u_0(\e x) =  \int_{\omega_\e} \e^{d-1} \gamma \frac{\partial r_\e}{\partial n}(\e z) L(\e x,\e z)  \:\d s (z) + K r_\e(\e x),$$
which eventually leads to the following integral equation for the function $\varphi_\e \in \widetilde{H}^{-1/2}(\D_1)$:
\begin{equation}\label{eq.Conductivity.HNHD.IntEq}
    T_{\e, L} \varphi_\e = -u_0(0) + \o(1).
\end{equation}
\par\medskip

 \textit{Step 3: We analyze the integral equation \cref{eq.Conductivity.HNHD.IntEq} for $\varphi_\e$.}
This leverages the explicit expression \cref{eq.Conductivity.HDHD.LI} for $L(x, y)$. When $d = 2$, we obtain the following equation:
$$- \int_{\D_1} \log \lvert \e x - \e z \lvert \varphi_\e(z) \:\d s(z) = -\pi\gamma u_0(0) + \o(1),$$
which rewrites: 
$$
\lvert \log\e\lvert  \left( \int_{\D_1} \varphi_\e(z) \:\d s(z) \right) - S_1 \varphi_\e(z) = -\pi\gamma u_0(0) + \o(1).
$$
This relation yields immediately: 
$$
\int_{\D_1} \varphi_\e(z) \:\d s (z) = -\frac{1}{|\log \e|}\pi\gamma u_0(0) + \o\left( \frac{1}{\lvert\log\e\lvert}\right).
$$
For the case $d = 3$ the integral equations \cref{eq.Conductivity.HNHD.IntEq} reads:  
$$
S_1 \varphi_\e (z) = -\frac{ \e}{2} \gamma u_0(0), \text{ where } S_1 \varphi =  \frac{1}{4\pi} \int_{\D_1} \frac{1}{ \lvert x - z \lvert } \varphi(z) \:\d s(z).
$$
Using the explicit knowledge of the solution to this equation (see \cite{copson1947problem}), we obtain, in particular: 
$$ \varphi_\e(z) = -\frac{\e}{2} \gamma u_0(0) S_1^{-1} 1 + \o(\e), \text{ and so }  \int_{\D_1} \varphi_\e(z) \:\d s (z)  = -4\e\gamma u_0(0) + \o(\e).
$$

\textit{Step 4: We pass to the limit in the representation formula \cref{eq.Conductivity.HNHD.Rep}.}
\CB{Exactly same step for the other equations. Only thing wee need is the fact that:}
\begin{equation}
    || \varphi_\e || \leq C
\end{equation}
By performing a Taylor expansion for the function $y \mapsto N(x, y)$ in a neighborhood of 0, and considering that $x \notin \Sigma \cup \{ 0 \}$, we obtain:
 \begin{align*}
     \left| \int_{\D_1} \varphi_\e (z) N(x, \e z) - N(x, 0) \: \d s(z) \right| \leq || \phi_\e ||_{H^{-1/2} (\D_1)} || N(x, \e \cdot) - N(x, 0) ||_{H^{1/2}(\D_1)} \leq C \e,
 \end{align*}
which leads to
 \begin{equation*}
      r_\e(x) = \left( \int_{\D_1} \varphi_\e(z) \:\d s(z) \right) N(x,0) + o(\e),
 \end{equation*}
yielding the desired formulas.
\end{proof}

\begin{corollary}[Topological sensitivity] \label{cor.Conductivity.HNHD.Jp}
The topological sensitivity of replacing the boundary conditions on $\omega_\e \subset \Gamma_N$ by those on $\Gamma_D$ writes:
\begin{equation} \label{eq.TopologicalSensitivity.SensitivityDef}
    J_\e'(G)(0) = \left\{ \
    \begin{aligned}
        \frac{\pi}{|\log \e|} \: \gamma(0) \: u_0(0) \: p_0(0) && \mathrm{if} \ d = 2,\\
        4 \e \: \gamma(0) \: u_0(0) \: p_0(0) && \mathrm{if} \: d = 3,
    \end{aligned}
    \right.
\end{equation}
where $p_0$ is the unique solution $H^1(\Omega)$ to the boundary value problem:
\begin{equation}\label{eq.Conductivity.HNHD.Adjoint}
\left\{
    \begin{aligned}
    -\dv(\gamma\nabla p_0) &= -j^\prime(u_0) && \mathrm{in} \ \Omega,\\
    p_0 &= 0 && \mathrm{on} \ \Gamma_D,\\
    \gamma \frac{\partial p_0}{\partial n} &= 0 && \mathrm{on} \ \Gamma_N.
    \end{aligned}
\right.
\end{equation}
\end{corollary}
\begin{proof}
We show the proof for the case $d = 3$, the case $d = 2$ being exactly analogous. By \cref{th.Conductivity.HNHD.Expansion}, we have:
\begin{equation*}
    J_\e (G) = J_0(G) + I_\e + o(\e), \quad \text{where } I_\e := \int_\Omega j'(u_0(x)) \left( -4 \e \gamma(0) \: u_0(0) \: N(x,0)\right) \: \d x.
\end{equation*}
Analyzing the integral $I_\e$, we have:
\begin{align*}
   I_\e &= 4 \e \gamma(0) \: u_0(0) \int_\Omega \dv (\gamma(x) \: \nabla p_0 (x)) N(x, 0) \: \d x + o(\e)\\
    &= -4 \e \gamma(0) \: u_0(0) \int_\Omega \gamma(x) \: \nabla p_0(x) \cdot \nabla N(x, 0) \: \d x + o(\e)\\
    &= 4 \e \gamma(0) \: u_0(0) \int_\Omega p_0(x) \: \dv (\gamma(x) \nabla N(x, 0) ) \: \d x + o(\e)\\
    &= 4 \e \gamma(0) \: u_0(0) \: p_0(0) + o(\e).
\end{align*}
The first line is obtained from the interior condition in \cref{eq.Conductivity.HNHD.Adjoint}, the second is an integration by parts, the third is another integration by parts, and the last line is obtained by \cref{eq.Conductivity.PhiN}.
\end{proof}

\begin{remark}\label{rem.Conductivity.HNHD.Inhomogeneous}
It is possible to replace the homogeneous Neumann boundary condition in \cref{eq.Conductivity.Background} with an inhomogeneous Dirichlet boundary condition on $\omega_\epsilon$, where $u_\e = u_{\text{\rm in}}$ for some smooth function $u_{\text{\rm in}} \in \mathcal{C}^\infty(\mathbb{R}^d)$. The above calculations can be straightforwardly adapted to this case, and the asymptotic expansion of $r_\e$ becomes:
\begin{equation}
    r_\e(x) =
    \left\{
    \begin{aligned}
        \frac{\pi}{\lvert \log \e \lvert} \gamma(0) (u_{\text{\rm in}}(0) -u_0(0) )N(x,0) + \o\left(\frac{1}{\lvert\log\e\lvert} \right) && \mathrm{if} \ d = 2,\\
        4 \e \gamma (0) \: (u_{\text{\rm in}}(0) - u_0(0) ) N(x,0) + \o (\e) && \mathrm{if} \ d = 3.
    \end{aligned}
    \right.
\end{equation}
In this case, the sensitivity of replacing the boundary conditions on $\omega_\e \subset \Gamma_N$ by the inhomgeneous conditions on $\Gamma_D$ is:
\begin{equation}
    J_\e'(G)(0) = \left\{ \
    \begin{aligned}
        \frac{\pi}{|\log \e|} \: \gamma(0) \: (u_{\mathrm{in}} - u_0(0)) \: p_0(0) && \mathrm{if} \ d = 2,\\
        4 \e \: \gamma(0) \: (u_{\mathrm{in}} - u_0(0)) \: p_0(0) && \mathrm{if} \: d = 3.
    \end{aligned}
    \right.
\end{equation}
\end{remark}

\CB{This is where I got to. -----------------------------}

%%%%%
\subsubsection{Replacement of the homogeneous Neumann boundary condition by an inhomogeneous Neumann condition on a small subset}
%%%%%

 We presently turn to the situation where the homogeneous Neumann boundary condition on $\Gamma_N$ is replaced by an inhomogeneous Neumann condition on a small region $\omega_\e \subset \Gamma_N$. The voltage potential $u_\e$ in this perturbed situation is the solution to: 
\begin{equation}\label{eq.conducepsneu}
\left\{
\begin{array}{cl}
-\dv(\gamma\nabla u_\e) = f & \text{in } \Omega, \\
u_\e = 0 & \text{on } \Gamma_D, \\
\gamma \frac{\partial u_\e}{\partial n} = 0 & \text{on } \Gamma_N \setminus \overline{\omega_\e}, \\
\gamma \frac{\partial u_\e}{\partial n} = g & \text{on } \omega_\e. \\
\end{array}
\right.
\end{equation}
\begin{theorem}
The following asymptotic expansions hold:
$$ u_\e(x) = u_0(x) + 2 \e g(0) N(x,0) + \o(\e) \text{ if } d=2, $$
and
$$ u_\e(x) = u_0(x) + \e^{2} \pi g(0) N(x,0) + \o(\e^{2}) \text{ if } d=3. $$
\end{theorem}
\begin{proof}[Sketch of proof]
The derivation of these formulas essentially follows the trail of the proof of \cref{th.dirconduc}, in a simplified fashion. 
Again, let $r_\e := u_\e - u_0$ be the error between the perturbed and background potentials. 
This function satisfies the following boundary value problem:
\begin{equation}\label{eq.conducneure}
\left\{
\begin{array}{cl}
-\dv(\gamma\nabla r_\e) = 0 & \text{in } \Omega, \\
r_\e = 0 & \text{on } \Gamma_D, \\
\gamma \frac{\partial r_\e}{\partial n} = 0 & \text{on } \Gamma_N \setminus \overline{\omega_\e}, \\[0.2em]
\gamma \frac{\partial r_\e}{\partial n} = g & \text{on } \omega_\e. \\
\end{array}
\right.
\end{equation}
From the definition of the Green's function $N(x,y)$ in \cref{eq.Greenconduc}, it holds:
$$ r_\e(x) = - \int_\Omega \dv_y( \gamma(y) \nabla_Y N(x,y)) r_\e(y) \:\d y,$$
and so, after integration by parts:
$$ r_\e(x) = -\int_{\partial \Omega} \gamma(y) \frac{\partial N}{\partial n_y}(x,y) r_\e(y) \:\d s(y) + \int_{\partial \Omega} \gamma(y) \frac{\partial r_\e}{\partial n}(y) N(x,y) \:\d s(y).$$
Now using the boundary conditions satisfied by the functions $r_\e$ and $y \mapsto N(x,y)$, it follows:
$$ \begin{array}{>{\displaystyle}cc>{\displaystyle}l}
r_\e(x) &=& \int_{\omega_\e} g(y) N(x,y) \:\d s(y) \\
&=& \e^{d-1} \int_{\D_1} g(\e z) N(x,\e z) \:\d s(z) \\
\end{array} $$
and so
$$ r_\e(x) = \e^{d-1} g(0) N(x,0) \left(\int_{\D_1} \:\d s(z)\right) + \o(\e^{d-1}),$$
which yields the desired result.
\end{proof}

We now turn to the computation of the sensitivity of a quantity of interest
$$ J(\e) := \int_\Omega j(u_\e) \:\d x,$$ 
depending on $u_\e$, with respect to the onset of a region $\omega_\e$ within $\Gamma_N$ bearing inhomogeneous Neumann conditions. The result is the following, whose proof is omitted, as it is completely similar to that of \cref{cor.Conductivity.HNHD.Jp}. 

\begin{corollary}
The functional $J(\e)$ has the following expansion:
$$ J(\e) = J(0) + \e^{d-1} g(0)p_0(0) + \o(\e^{d-1}),$$
where $p_0 \in H^1(\Omega)$ is the unique solution to the boundary value problem \cref{eq.Conductivity.HNHD.Adjoint}. 
\end{corollary}

%\begin{proof}
%At first, a use of the Lebesgue dominated convergence theorem yields:
%$$\lim\limits_{\e \to 0} \frac{J(\e) - J(0)}{\e^{d-1}} =  \int_\Omega j^\prime(u_0) u_1(x) \:\d x, \text{ where } u_1(x) = \pi g(0) N(x,0).$$
%Hence, it follows: 
%$$ \begin{array}{>{\displaystyle}cc>{\displaystyle}l}
%J^\prime(0) &=& \pi g(0) \int_\Omega j^\prime(u_0(x)) N(x,0) \:\d x \\
%&=& -\pi g(0) \: p_0(0),
%\end{array} $$
%by using the properties of the Green's function.
%\end{proof}