%%%%%%%%%%%%%%%%%%%%%%%%%%%%%%%%%%%%%%%%%%%%%%%%%%
%%%%%%%%%%%%%%%%%%%%%%%%%%%%%%%%%%%%%%%%%%%%%%%%%%
%\FloatBarrier
\section{Conclusion and perspectives} \label{sec.concl}
%%%%%%%%%%%%%%%%%%%%%%%%%%%%%%%%%%%%%%%%%%%%%%%%%%
%%%%%%%%%%%%%%%%%%%%%%%%%%%%%%%%%%%%%%%%%%%%%%%%%%

\noindent In this article, we have explored the relatively confidential issue of optimizing the shape and topology of a region $G$ on the boundary of a given domain $\Omega$ which supports the boundary conditions attached to a physical partial differential equation. 
Elaborating on the ``classical'' notions of shape and topological sensitivities involved in the optimal design of a bulk domain, we have introduced suitable notions of shape and topological derivatives, accounting for the sensitivity of a functional with respect to either ``regular'' perturbations of the boundary of $G$, or ``singular'' perturbations, by the addition to $G$ of a new, tiny connected component. 
We have proposed practical calculation methods for our derivatives; these were thoroughly presented in the model mathematical setting of the conductivity equation, and adapted to more realistic physical situations, described by the Helmholtz equation and the linear elasticity system. Several 3d numerical examples have been addressed thanks to a shape and topology optimization algorithm combining both types of sensitivity with an efficient body-fitted mesh evolution method for tracking the motion of a region on a surface. 

This work paves the way to extensions of various natures. 
From the methodological viewpoint, our topological derivatives could be exploited with a different numerical strategy than that outlined in \cref{alg.sketchoptbc}, where they allow to occasionally add a tiny connected component to the optimized region $G$ in the course of a workflow driven by a boundary variation method, based on shape derivatives. For instance, variants of the fixed point algorithm proposed in \cite{amstutz2011analysis,amstutz2006new} or of the reaction-diffusion method in \cite{yamada2010topology} would rely on these topological derivatives in a standalone fashion.

From the perspective of applications, the ambition to optimize regions bearing the boundary conditions of a physical problem arises in multiple physical situations beyond those exemplified in this article. One of these, still concerned with mechanical structures, is surface texturing \cite{lu2020tribological}. This technique creates small-scale surface patterns (e.g., dimples, grooves) to improve physical properties of the structure $\Omega$, such as the stress concentration within, or the load bearing capabilities. For instance, in sliding bearings, surface texturing influences the tribological properties of the structure, such as friction, wear and lubrication \cite{zhang2023optimization,cui2020optimization}.
In the language of the present study, the optimal design of a surface texturing pattern could be realized by optimizing the repartition of a traction-free region and a textured region (bearing specific boundary conditions, e.g. related to its behavior with respect to friction) on the boundary of the structure $\Omega$.
Also, such optimal design issues arise in fluid mechanics, whose mathematical description involves the Stokes or the Navier-Stokes equations. One could for instance optimize the placement of the inlet and outlet regions \(\Gamma_{\text{in}}\) and \(\Gamma_{\text{out}}\) on the boundary of a duct $\Omega$ conveying a fluid, respectively bearing inhomogeneous Dirichlet and Neumann conditions, in order to reduce energy dissipation by managing viscous forces more effectively.

%The clamp-locator problem can be reframed to design an “intelligent fixing system” that adapts to different workpiece shapes and conditions. Rather than simply securing a workpiece, this system would optimize clamp placement to ensure stable force distribution and minimal deformation. By integrating sensors, the system could adjust in real-time to maintain optimal stability, as in \cite{leopold2008clamping}, which explores adaptive clamping for workpiece geometry and machining needs.

%Finally, the numerical optimization approach introduced in \cite{amstutz2006new} may be employed as an alternative method to evolve the level-set function. The main difference with our method is that it avoids the need to create new regions during optimization. For a fictitious time \( t > 0 \), \( G(t) \) evolves through the level-set function \( \phi(x, t) \), governed by:
%$$
%\begin{aligned}
%\dfrac{\partial \phi}{\partial t} &= P_{\phi^\perp}(g), \quad t \geq 0, \\
%P_{\phi^\perp}(g) &= g - \frac{(g, \phi)}{\|\phi\|_{L^2}^2} \phi,
%\end{aligned}
%$$
%where \( g(x) = -\d_TJ(G)(x) \) if \( x \in G \), and \( g(x) = \d_T J(G)(x) \) otherwise. This approach, similar to the SIMP method, performs a global update of the level-set function. The method is based on the insight that, when \( \phi \) converges to a stationary point, a nonzero topological gradient \( g \) signals a local optimum. The condition \( P_{\phi^\perp}(g) = 0 \) implies that \( g = s\phi \) for some scalar \( s \): if \( s > 0 \), this represents a local minimum, and if \( s < 0 \), a local maximum. This approach integrates smoothly with our framework, allowing straightforward transitions between meshed and level-set representations and offers the benefit of binary (black-and-white) designs, as opposed to intermediate densities found in SIMP.}

\par\bigskip

\noindent \textbf{Acknowledgements.} The work of E.B., C. B.-P. and C.D is partially supported by the projects ANR-18-CE40-0013 SHAPO and ANR-22-CE46-0006 StableProxies, financed by the French Agence Nationale de la Recherche (ANR). Part of this work was conducted while C.D. was visiting the Laboratoire Jacques-Louis Lions from Sorbonne Universit\'es, whose hospitality is gratefully acknowledged.

