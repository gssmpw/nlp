%%%%%%%%%%%%%%%%%%%%%%%%%%%%%%%%%%%%%%%%%%%%
%%%%%%%%%%%%%%%%%%%%%%%%%%%%%%%%%%%%%%%%%%%%
\section{Topological perturbations of regions supporting boundary conditions in the setting of the conductivity equation} \label{sec.TopologicalSensitivity}
%%%%%%%%%%%%%%%%%%%%%%%%%%%%%%%%%%%%%%%%%%%%
%%%%%%%%%%%%%%%%%%%%%%%%%%%%%%%%%%%%%%%%%%%%

\noindent As in \cref{sec.setconduc}, we consider the state equation \cref{eq.conduc} related to the conductivity equation. Again, in the main part of this section, 
the region $G \subset \partial \Omega$ to be optimized is $\Gamma_D$, that which supports Dirichlet boundary conditions.
We then consider the calculation of the topological derivative of the model functional \cref{eq.defJ}, in the sense of \cref{def.SD}. \par\medskip

As we have mentioned, this task leverages techniques from asymptotic analysis. 
For short, and out of consistency with the classical notation and jargon of that field, the unperturbed, ``background'' potential is denoted by $u_0$ throughout this section. It is characterized by the boundary value problem \cref{eq.conduc}, that we recall for convenience:
\begin{equation}\label{eq.conducasym}
\left\{
\begin{array}{cl}
-\dv(\gamma \nabla u_0) = f & \text{in } \Omega, \\
u_0 = 0 & \text{on } \Gamma_D, \\
\gamma \frac{\partial u_0}{\partial n} = g & \text{on }\Gamma_N, \\[0.2em]
\gamma \frac{\partial u_0}{\partial n} = 0 & \text{on } \Gamma.
\end{array}
\right.
\end{equation} 
We fix $x_0 \in \Gamma$, and consider the ``perturbed'' potential $u_\e$, which is the unique $H^1(\Omega)$ solution to:
\begin{equation}\label{eq.conduceps}
\left\{
\begin{array}{cl}
-\dv(\gamma \nabla u_\e) = f & \text{in } \Omega, \\
u_\e = 0 & \text{on } \Gamma_D \cup \omega_{x_0,\e}, \\
\gamma \frac{\partial u_\e}{\partial n} = g & \text{on }\Gamma_N, \\[0.2em]
\gamma \frac{\partial u_\e}{\partial n} = 0 & \text{on } \Gamma \setminus \overline{\omega_{x_0,\e}},
\end{array}
\right.
\end{equation} 
where $\omega_{x_0,\e}$ is the surface disk defined in \cref{eq.defomeps}. 
Our goal is to determine the asymptotic behavior of the potential $u_\e$ as $\e \to 0$, as well as that of the quantity of interest 
\begin{equation}\label{eq.JGeconduc}
J((\Gamma_D)_{x_0,\e}) = \int_\Omega j(u_\e) \:\d x .
\end{equation}

After a brief reminder about the suitable functional spaces involved in the treatment of this problem and the Green's function of the background problem \cref{eq.conducasym} in \cref{sec.prelconduc}, we recall in \cref{sec.asymueconduc} the result from \cite{bonnetier2022small} giving the asymptotic expansion of $u_\e$. We provide a formal sketch of the (difficult) proof in a representative particular case, which will be adapted to address other situations, as well as more intricate physical contexts. We then take advantage of these results in \cref{sec.asymJeconduc} to calculate the asymptotic expansion of $J((\Gamma_D)_{x_0,\e})$ and therefore identify the topological derivative of the function $J(\Gamma_D)$. 
We eventually discuss several extensions of these results to different types of changes of boundary conditions in \cref{sec.exttopoconduc}.

%%%%%%%%%%%%%%%%%
\subsection{Preliminaries}\label{sec.prelconduc}
%%%%%%%%%%%%%%%%%

\noindent In this section, we introduce the necessary background material from functional analysis and potential theory. Throughout, $\Omega$ stands for a smooth bounded domain in $\R^d$.

%%%%%%%%%%%%%%%%%
\subsubsection{Sobolev spaces on a subregion of the boundary of a domain}\label{sec.sobolev}
%%%%%%%%%%%%%%%%%

\noindent To a large extent, the calculations of the topological derivatives at stake in this article bring into play functions defined on (a region of) the boundary $\partial \Omega$ of the ambient domain $\Omega$, 
as the traces of solutions to boundary value problems posed on $\Omega$. 
The definitions of the suitable energy spaces for these are recalled in the present section; we refer for instance to \cite{grisvard2011elliptic,mclean2000strongly} about these issues.\par\medskip

Let us first consider functions defined on the whole boundary $\partial \Omega$. 
For any real number $0 < s < 1$, the fractional Sobolev space $H^s(\partial \Omega)$ is defined by:
\begin{multline*}
 H^s(\partial \Omega) = \left\{ u \in L^2(\partial \Omega) \text{ s.t. } \lvert\lvert u \lvert\lvert_{H^s(\partial \Omega)} < \infty \right\}, \\
\text{ where } \lvert\lvert u \lvert\lvert^2_{H^s(\partial \Omega)} := \lvert\lvert u \lvert\lvert^2_{L^2(\partial\Omega)} + \int_{\partial\Omega} \int_{\partial \Omega}\frac{\lvert u(x) - u(y)\lvert^2}{\lvert x-y\lvert^{d-1+2s}} \:\d s(x)\d s(y) .
\end{multline*}
By convention, $H^0(\partial\Omega) = L^2(\partial\Omega)$, and for $0<-s<1$, $H^{-s}(\partial\Omega)$ is the topological dual of $H^s(\partial\Omega)$.

Let now $G$ be a Lipschitz open subset of $\partial\Omega$; we shall use two types of fractional Sobolev spaces of functions on $G$:
\begin{itemize}
\item For any $0<s<1$, we denote by $H^s(G)$ the space of restrictions to $G$ of functions in $H^s(\partial\Omega)$:
\begin{multline*}
H^s(G) = \Big\{ U\lvert_G \text{ for some } U \in H^s(\partial\Omega) \Big\}, \text{ equipped with the quotient norm } \\
\lvert\lvert u \lvert\lvert_{H^s(G)} := \inf\Big\{ \lvert\lvert U \lvert\lvert_{H^s(\partial\Omega)},\:\: U \in H^s(\partial\Omega), \:\: U\lvert_G = u\Big\}.
\end{multline*}
\item For any $0<s<1$, we denote by $\widetilde{H}^s(G)$ the subspace of $H^s(\partial \Omega)$ made of functions with compact support in $\overline G$. It is equipped with the norm $\lvert\lvert u \lvert\lvert_{\widetilde{H}^s(G)} = \lvert\lvert u \lvert\lvert_{H^s(\partial\Omega)}$ induced by that of $H^s(\partial\Omega)$. Equivalently, $H^s(G)$ is the space of functions $u\in L^2(G)$ whose extension $\widetilde{u}$ by $0$ to $\partial\Omega$ belongs to $H^s(\partial\Omega)$.
\end{itemize}
As for the negative versions of these spaces, 
\begin{itemize}
\item For $0<s<1$, $H^{-s}(G)$ is still defined as the space of distributions on $G$ obtained by restriction of a distribution in $H^{-s}(\partial\Omega)$. The space $H^{-s}(G)$ is naturally identified with the dual space of $\widetilde{H}^s(G)$ via the following duality: 
$$\forall u \in H^{-s}(G), \:\: v \in \widetilde{H}^s(G),\quad \langle u,v \rangle_{H^{-s}(G),\widetilde{H}^s(G)} := \langle U, \widetilde{v} \rangle_{H^{-s}(\partial \Omega),H^s(\partial\Omega)}, $$
where $U$ is any element in $H^{-s}(\partial \Omega)$ such that $U\lvert_G = u$ and $\widetilde{v}$ is the extension of $v$ by $0$ to the whole hypersurface $\partial\Omega$.
\item For $0<s<1$, $\widetilde{H}^{-s}(G)$ is again the subspace of $H^{-s}(\partial\Omega)$ made of elements with compact support in $\overline G$. This space is naturally identified with the dual of $H^s(G)$ via the following duality:
$$\forall u \in \widetilde{H}^{-s}(G),\:v \in H^s(G),\quad \langle u,v \rangle_{\widetilde{H}^{-s}(G),H^s(G)} := \langle \widetilde{u}, V \rangle_{H^{-s}(\partial \Omega),H^s(\partial\Omega)}, $$
where $\widetilde{u}$ is the extension of $u$ by $0$ to $\partial\Omega$ and $V$ is any element in $H^{s}(\partial \Omega)$ such that $V\lvert_G = v$.
\end{itemize} 

%%%%%%%%%%%%%%%%%
\subsubsection{The Green's function for the background problem}
%%%%%%%%%%%%%%%%%


\noindent Let us recall the expression of the fundamental solution $\Gamma(x,y)$ of the operator $-\Delta$ in free space:
\begin{equation}\label{eq.conduc.Gamma}
\forall x,y \in \R^d, \:\: x \neq y, \quad     \Gamma(x,y) = \left\{
    \begin{array}{cl}
    -\frac{1}{2\pi} \log |x-y| & \text{if } d = 2, \\[0.2em]
    \frac{1}{(d-2) \alpha_d} |x-y|^{2-d} & \text{if } d \geq 3,
    \end{array}
    \right.
\end{equation}
where $\alpha_d$ is the area of the unit sphere $\mathbb{S}^{d-1} \subset \mathbb{R}^d$. For each $x \in \R^d$, the function $\Gamma(x,\cdot)$ satisfies:
\begin{equation*}
    -\Delta_y \Gamma(x,y) = \delta_{y=x} \text{ in the sense of distributions on } \mathbb{R}^d,
\end{equation*}
where $\delta_{y,x}$ is the Dirac distribution at $x$. Here and throughout the article, the subscript $_y$ indicates that the differential operator applies to the $y$ variable only. We set: 
$$\forall x, y \in \R^d, \: x \neq y, \quad \Gamma(x, y) := \Gamma(x - y).$$ 

 The Green's function $N(x,y)$ for the background equation \cref{eq.conducasym} is next defined by the following relation:
\begin{equation}\label{eq.Greenconduc}
\text{For all } x \in \Omega, \:\: y \mapsto N(x,y) \text{ satisfies } \left\{
\begin{array}{cl}
    -\dv_y(\gamma(y)\nabla_y N(x,y)) = \delta_{y=x} & \text{in } \Omega, \\
    N(x,y) = 0 & \text{for } y \in \Gamma_D, \\
    \gamma(y) \frac{\partial N}{\partial n_y}(x,y) = 0 & \text{for } y \in \Gamma \cup \Gamma_N.
\end{array}
\right.
\end{equation}

\begin{remark}
    The function $N(x,y)$ is symmetric in its arguments (see \cite{folland1995introduction} for a proof) and it is related to the fundamental solution $\Gamma(x, y)$ of the operator $-\Delta$ as follows:
    \begin{equation}
        N(x, y) = \dfrac{1}{\gamma(x)} \Gamma(x, y) + R(x, y),
    \end{equation}
    where the correction term $R(x, y)$ is the solution to:
    \begin{equation}
    \left\{
        \begin{array}{cl}
        -\dv_y(\gamma(y)\nabla_y R(x,y)) = \frac{1}{\gamma(y)} \nabla \gamma(y) \cdot \nabla_y \Gamma(x, y) & \text{in } \Omega, \\[0.5em]
        R(x, y) = - \frac{1}{\gamma(y)} \Gamma(x, y) & \text{for } y \in \Gamma_D, \\[0.5em]
        \gamma(y) \frac{\partial R}{\partial n_y}(x,y) = \frac{\gamma(y)}{\gamma(x)} \frac{\partial \Gamma}{\partial n_y}(x, y) & \text{for } y \in \Gamma \cup \Gamma_N.
        \end{array}
    \right.
    \end{equation}
    Since $\gamma$ and $\Omega$ are smooth, for every open subset $U \Subset \mathbb{R}^d \setminus (\Sigma \cup \{ x \})$, the function $R(x,\cdot)$ is of class $C^\infty$ on $\overline{\Omega} \cap U$, see \cite{brezis2010functional, gilbarg2015elliptic}. We refer to e.g. \cite{ammari2009layer,friedman1989identification} for further details about the construction of $N(x,y)$.
\end{remark}
The key property of the Green's function $N(x, y)$ is the following relation, which holds (at least) for functions $\varphi \in \calC^1(\overline{\Omega})$ such that $\varphi = 0$ on $\Gamma_D$:
\begin{equation} \label{eq.Conductivity.PhiN}
    \varphi (x) = \int_\Omega \gamma(y) \nabla_y N(x, y) \cdot \nabla \varphi (y) \:\d y, \quad x \in \Omega.
\end{equation}
In particular, one may integrate by parts to express the solution $u_0$ to \cref{eq.conducasym} in terms of $N(x, y)$ as:
\begin{equation}\label{eq.u0N}
u_0(x) = \int_\Omega f(y) N(x,y) \:\mathrm{d} y.
\end{equation}

%%%%%%%%%%%%%%%%%
\subsubsection{The Neumann function of the lower half-space}\label{sec.Neumannconduc}
%%%%%%%%%%%%%%%%%

\noindent In the following, we shall need another Green's function $L_\gamma(x, y)$, associated to the version of \cref{eq.conducasym} featuring a constant conductivity $\gamma >0$, posed on the lower half-space $H$ (see \cref{eq.LHS}), and equipped with homogeneous Neumann boundary conditions on $\partial H$. For $x \in H$, $y \mapsto L_\gamma(x, y)$ satisfies:
\begin{equation}\label{eq.Conductivity.HNHD.L}
    \left\{
    \begin{array}{cl}
    -\dv_y(\gamma \nabla_y L_\gamma(x,y)) = \delta_{y=x} & \text{in } H, \\[0.2em]
    \gamma \frac{\partial L_\gamma}{\partial n_y}(x,y) = 0 & \text{for } y \in \partial H.
    \end{array}
    \right.
\end{equation}
Such a function can be constructed thanks to the so-called method of images, see e.g. \cite{jackson2007classical}; precisely:
\begin{equation} \label{eq.Conductivity.HDHD.LI}
    L_\gamma(x, y) = \frac{1}{\gamma} \left( \Gamma(x - y) + \Gamma(x + y) \right).
\end{equation}
It is indeed straightforward to see that the above function satisfies \cref{eq.Conductivity.HNHD.L}. Similarly to the formula \cref{eq.Conductivity.PhiN} involving $N(x, y)$, for $\varphi \in \calC^1_c(\overline{H})$, the following relation holds:
\begin{equation}
    \varphi (x) = \int_H \gamma \nabla_y L(x, y) \cdot \nabla \varphi (y) \:\mathrm{d} y, \quad x \in H.
\end{equation}

%%%%%%%%%%%%%%%%%
\subsubsection{The single layer potential operator}\label{sec.SLP}
%%%%%%%%%%%%%%%%%

\noindent Let us recall the definition of the single layer potential operator $\calS_\Omega$, which represents the potential generated in $\R^d$ by a density of charges $\varphi : \partial \Omega \to \R$ on $\partial\Omega$.

\begin{definition}
The single layer potential associated to a smooth density $\varphi \in \calC^\infty(\partial \Omega)$ is the function defined by: 
$$ \calS_\Omega \varphi(x) = \int_{\partial\Omega} \Gamma(x,y) \varphi(y) \:\d s(y) , \quad x \in \R^d \setminus \partial \Omega,$$
where $\Gamma(x,y)$ is the fundamental solution of the operator $-\Delta$ in free space, see \cref{eq.conduc.Gamma}. 
\end{definition}

It is well-known that the single layer potential is continuous across $\partial \Omega$, and that it induces an operator $S_{\partial\Omega} : \calC^\infty(\partial \Omega) \to \calC^\infty(\partial \Omega)$ defined by:
$$ S_{\partial\Omega} \varphi(x) = \int_{\partial\Omega} \Gamma(x,y) \varphi(y) \:\d s(y) , \quad x \in \partial \Omega.$$
The next proposition gathers a few properties of this mapping. 

\begin{proposition}\label{prop.SLP}
The following facts hold true:
\begin{enumerate}[(i)]
\item 
The mapping $S_{\partial\Omega}$ has an extension as a bounded mapping from $H^{-1/2}(\partial \Omega)$ into $H^{1/2}(\partial \Omega)$. 
\item Let $G$ be a Lipschitz open subset of $\partial \Omega$; then $S_{\partial\Omega}$ induces a bounded operator $S_G : \widetilde{H}^{-1/2}(G) \to H^{1/2}(G)$ via the formula:
$$ \forall \varphi \in \widetilde{H}^{-1/2}(G), \quad S_G \varphi = (S_{\partial\Omega} \widetilde\varphi)\lvert_G,$$
where $\widetilde\varphi \in H^{-1/2}(\partial \Omega)$ is the extension by $0$ to $\partial \Omega$ of an element $\varphi \in \widetilde{H}^{-1/2}(G)$.
\item If $d \geq 3$, the mapping $S_G: \widetilde{H}^{-1/2}(G) \to H^{1/2}(G)$ is invertible.
\end{enumerate}
\end{proposition}

The proof of $(i)$ is found in \cite{mclean2000strongly}, while $(ii)$ follows almost immediately from the definitions of the functional spaces $\widetilde{H}^{-1/2}(G)$ and $H^{1/2}(G)$. The last point  $(iii)$ is more subtle, and it is proved in \cite{bonnetier2022small}. Note that when $G$ is the unit disk $\D_1$, this result also holds when $d=2$. This fact is needed in the rigorous proof of the forthcoming \cref{th.Conductivity.HNHD.Expansion}, which is conducted in \cite{bonnetier2022small}, but not in the formal calculation method proposed in the next \cref{sec.asymueconduc}. 

%%%%%%%%%%%%%%%%%
\subsection{Asymptotic expansion of the voltage potential $u_\e$}\label{sec.asymueconduc}
%%%%%%%%%%%%%%%%%

%%%% CD: The materiza in here is now contained in Section 2 and Section 4.3

%\subsubsection{Characterizing the sensitivity of topological changes in the support of boundary conditions}
%
%In general, the geometry of $\omega_\e$ is not specified We shall assume that $\omega_\e$ is a geodesic ball contained entirely in $\Gamma$ and centered at a fixed point $x_0 \in \Gamma$. We work towards the characterization of the topological sensitivity. A simple application of the fundamental theorem of calculus yields:
%\begin{equation}
%    j(u_\e) = j(u_0) + \int_0^1 j'(u_0 + t (u_\e - u_0)) \cdot (u_\e - u_0) \: \d t .
%\end{equation}The twice continuous differentiability of $j$ allows us to expand the function $t \mapsto j'(u_0 + t (u_\e - u_0))$ at $t = 0$:
%\begin{equation}
%    j'(u_0 + t (u_\e - u_0)) = j'(u_0) + t j''(u_0) \cdot (u_\e - u_0) + o(t | r_\e |) .
%\end{equation}
%Now let $r_\e := u_\e - u_0$. Utilizing the growth conditions \cref{eq.TopologicalSensitivity.GrowthConditions}, we can rewrite the perturbed criterion \cref{eq.TopologicalSensitivity.PerturbedCriterion} as:
%\begin{align}
%    J_\e (G) = J_0(G) + \int_\Omega \int_0^1 j'(u_0 + t r_\e) \cdot r_\e  \: \d t \: \d x = J_0(G) + \int_\Omega  j'(u_0) \cdot r_\e \: \d x + o(|| r_\e ||_V) .
%\end{align}
%Hence, for a point $x_0 \in \Gamma$, if $|| r_\e ||_V \rightarrow 0$ as $\e \rightarrow 0$, we denote $J_\e'(G)(x_0)$ as the \textit{topological sensitivity of replacing the boundary conditions on} $\Gamma$ \textit{around} $x_0$ \textit{by to those on} $G$, and define it by:
%\begin{equation} \label{eq.TopologicalSensitivity.SensitivityDef}
%    J_\e'(G)(x_0) := \int_\Omega j'(u_0) \cdot r_\e \: \d x ,
%\end{equation}
%for small enough $\e > 0$. From a theoretical perspective, expression \cref{eq.TopologicalSensitivity.SensitivityDef} allows us to measure the change in the criterion $J_0(G)$ when altering boundary conditions. This requires computing the difference $r_\epsilon$ and the quantity $j'(u_0)$ for various points $x_0 \in \Gamma$. However, expression \cref{eq.TopologicalSensitivity.SensitivityDef} is too general to be practical. For instance, to numerically evaluate it, one would need to solve the perturbed equation for numerous points over $\Gamma$, making it infeasible for practical use. In the following sections, this infeasibility will be addressed by establishing a representation formula for $r_\epsilon$, characterizing it with an integral equation, and rewriting $J_\epsilon'(G)(x_0)$ in terms of an adjoint state. This approach will allow us to numerically compute the topological sensitivity efficiently and use it in a numerical shape optimization algorithm.

\noindent This section is devoted to the asymptotic expansion of the perturbed potential $u_\e$, solution to \cref{eq.conduceps}, in the limit $\e \to 0$, where the support $\omega_{x_0,\e}$ of the replacement of homogeneous Neumann boundary conditions by homogeneous Dirichlet boundary conditions vanishes.
The main result in this setting is the following asymptotic expansion, whose proof is rigorously detailed in \cite{bonnetier2022small}. We provide a formal sketch of the latter, under additional technical assumptions, which lends itself to generalizations. 

\begin{theorem} \label{th.Conductivity.HNHD.Expansion}
Let $x_0$ be a given point in $\Gamma$; then for any point $x \in \overline{\Omega} \setminus (\Sigma_D \cup \left\{ x_0 \right\})$, the following asymptotic expansion holds:
\begin{equation}\label{eq.ueconduc2d}
    u_\e (x) =  u_0(x)  - \frac{\pi}{\lvert \log \e \lvert} \: \gamma(x_0) \: u_0(x_0) \: N(x,x_0) + \o\left(\frac{1}{\lvert\log\e\lvert} \right) \quad \text{if} \ d = 2,
\end{equation}
and     
\begin{equation}\label{eq.ueconduc3d}
    u_\e (x) =  u_0(x)  - 4 \e \: \gamma(x_0) \: u_0(x_0) \: N(x,x_0) + \o (\e) \quad \text{if} \ d = 3.
\end{equation}
\end{theorem}
\begin{proof}[Sketch of proof]
Without loss of generality, we assume that the center $x_0$ of the surface disk where the substitution of boundary conditions occurs is the origin $0$ and that the unit normal vector $n(0)$ to $\partial \Omega$ at $0$ coincides with the $d^{\text{th}}$ coordinate vector $e_d$; we denote $\omega_\e := \omega_{x_0,\e}$ throughout the proof. Our formal argument hinges on the simplifying assumptions that $\gamma$ is constant, and that the boundary $\partial \Omega$ is completely flat around $0$, that is:
\begin{equation}\label{eq.flat}
\partial \Omega \text{ coincides with } \partial H \text{ in a neighborhood } \calO \text{ of }0 \text{ in }\R^d.
\end{equation}
This implies in particular that $\omega_{\e}$ coincides with $\D_\e$, see \cref{fig.assumsimpl}.

\begin{figure}[!ht]
\centering
\includegraphics[width=0.75\textwidth]{figures/assumflat}
\caption{\it Illustration of the flatness \cref{eq.flat} of $\partial \Omega$ near $x_0 = 0$ assumed in the proof of \cref{th.Conductivity.HNHD.Expansion}.}
\label{fig.assumsimpl}
\end{figure}

Let us introduce the error $r_\e = u_\e - u_0 \in H^1(\Omega)$ between the perturbed and background potentials; this function satisfies the boundary value problem:
\begin{equation}\label{eq.Conductivity.HNHD.Re}
\left\{
\begin{array}{cl}
-\dv(\gamma\nabla r_\e) = 0 & \text{in } \Omega, \\
r_\e = 0 & \text{on } \Gamma_D, \\
r_\e = -u_0 & \text{on } \omega_\e,\\
\gamma \frac{\partial r_\e}{\partial n} = 0 & \text{on } \Gamma_N \cup (\Gamma \setminus \overline{\omega_\e}).
\end{array}
\right.
\end{equation}
Standard a priori estimates show this error tends to $0$: 
\begin{equation}\label{eq.cvre0conduc}
r_\e \xrightarrow{\e \to 0} 0 \text{ in } H^1(\Omega),
\end{equation}
see again \cite{bonnetier2022small}.
We now proceed in four steps to calculate an expansion of $r_\e(x)$ as $\e \to 0$.\\

\noindent 
\textit{Step 1: We construct a representation formula for the values of $r_\e$ ``far'' from $0$ in terms of its values inside the vanishing region $\omega_\e$ and the Green's function $N(x,y)$.}

This task starts from the integral expression of $r_\e$ based on the Green's function $N(x,y)$ in \cref{eq.Greenconduc}. For any point $x \in \Omega$, it holds: 
$$ r_\e(x) = -\int_\Omega \dv_y(\gamma \nabla_y N(x,y)) r_\e(y) \:\d y.$$ 
Two successive integrations by parts in the above expression yield:
\begin{align}
    r_\e(x)
    &= - \int_{\partial \Omega} \gamma \frac{\partial N}{\partial n_y}(x,y) r_\e(y) \:\d s(y) + \int_\Omega \gamma \nabla_y N(x,y) \cdot \nabla r_\e(y) \:\d y\\
    &= - \int_{\partial \Omega} \gamma \frac{\partial N}{\partial n_y}(x,y) r_\e(y) \:\d s(y) + \int_{\partial \Omega} \gamma \frac{\partial r_\e}{\partial n}(y) N(x,y) \:\d s(y),
\end{align}
where we have used the equation \cref{eq.Conductivity.HNHD.Re} satisfied by $r_\e$ for passing from the first line to the second one.
Now, taking into account the boundary conditions satisfied by $r_\epsilon$ and $y \mapsto N(x, y)$, the first integral in the above right-hand side vanishes.
Likewise, the boundary conditions in \cref{eq.Greenconduc} and \cref{eq.Conductivity.HNHD.Re} imply that the integrand of the second term vanishes on $\Gamma_D$, $\Gamma_N$ and $\Gamma \setminus \overline{\omega_\e}$, which leaves us with the expression:
\begin{equation*}
    r_\e(x) = \int_{\omega_\e} \gamma \frac{\partial r_\e}{\partial n}(y) N(x,y)  \:\d s (y). 
\end{equation*}
A change of variables in the above integral now yields:
\begin{equation} \label{eq.Conductivity.HNHD.Rep}
 r_\e(x) = \int_{\D_1} \varphi_\e(z) N(x,\e z)  \:\d s (z), 
\end{equation}
where we have introduced the function $\varphi_\e(z) := \e^{d-1}\left(\gamma\frac{\partial r_\e}{\partial n}\right)(\e z) \in \widetilde{H}^{-1/2}(\mathbb{D}_1)$. 
This is the desired representation formula for $r_\e(x)$.\\

\noindent \textit{Step 2: We characterize $\varphi_\e$ as the solution to an integral equation.}

To achieve this, we repeat the derivation of the first step, except that we replace the true (non explicit) Green's function $N(x,y)$ of the problem \cref{eq.conducasym} by the approximate, but explicit, function $L_\gamma(x,y)$ introduced in \cref{sec.Neumannconduc}, which captures the behavior of \cref{eq.conduc.Gamma} near $0$. 
Again, for any point $x \in \Omega$, two consecutive integrations by parts yield: 
\begin{align}
r_\e(x)
&= -\int_\Omega \dv_y (\gamma \nabla_y L_\gamma(x,y)) r_\e(y) \:\d y\\
&= -\int_{\partial \Omega} \gamma \frac{\partial L_\gamma}{\partial n_y}(x,y) r_\e(y) \:\d s(y) + \int_\Omega \gamma \nabla_y L_\gamma(x,y) \cdot \nabla r_\e(y) \:\d y \\
&= -\int_{\partial \Omega} \gamma \frac{\partial L_\gamma}{\partial n_y}(x,y) r_\e(y) \:\d s(y) + \int_{\partial \Omega} \gamma \frac{\partial r_\e}{\partial n}(y) L_\gamma(x,y)  \:\d s (y).
\end{align}
Now using the fact that $\partial \Omega$ coincides with $\partial H$ in a neighborhood $\calO$ of $0$ in $\R^d$, and considering the boundary conditions satisfied by $r_\e$, we obtain that: 
\begin{multline*}
 r_\e(x) =   \int_{\omega_\e} \gamma \frac{\partial r_\e}{\partial n}(y) L_\gamma(x,y)  \:\d s (y) + K r_\e(x), \text{ where }\\
 K r_\e(x) :=  -\int_{(\Gamma_N  \cup \Gamma) \setminus \overline{\calO}} \gamma\frac{\partial L_\gamma}{\partial n_y}(x,y) r_\e(y) \:\d s(y)  + \int_{\Gamma_D}  \gamma \frac{\partial r_\e}{\partial n}(y) L_\gamma(x,y)  \:\d s (y).
\end{multline*}
We now change variables in the above integrals and let $x$ approach $\omega_\e$.
Since the single layer potential is continuous across $\partial \Omega$, as expressed in \cref{prop.SLP}, we obtain that:

$$ \forall x \in \D_1, \:\: -u_0(\e x) =  \int_{\D_1} \varphi_\e(z) L_\gamma(\e x,\e z)  \:\d s (z) + K r_\e(\e x).$$

Now, since $\e x$ lies ``far'' from the support of the integrals featured in $Kr_\e$, the convergence \cref{eq.cvre0conduc} of the error $r_\e$ implies that the last term in the above right-hand side tends to $0$ as $\e \to 0$.
Taking advantage of the explicit form \cref{eq.Conductivity.HDHD.LI} of the Green's function $L_\gamma(x,y)$, this eventually leads to the following integral equation for the function $\varphi_\e \in \widetilde{H}^{-1/2}(\D_1)$:
\begin{equation}\label{eq.Conductivity.HNHD.IntEq}
   \frac{2}{\gamma}\int_{\D_1} \varphi_\e(z) \Gamma(\e x,\e z) \:\d s(z) = -u_0(0) + \o(1),
\end{equation}
where $\o(1)$ is a remainder that converges to $0$ in $H^{1/2}(\D_1)$.\par\medskip

\noindent \textit{Step 3: We analyze the integral equation \cref{eq.Conductivity.HNHD.IntEq} to glean information about $\varphi_\e$.}

Let us denote by $S_1 := S_{\D_1}$ the single layer potential attached to $\D_1$, see \cref{prop.SLP}. 
Our analysis of \cref{eq.Conductivity.HNHD.IntEq} relies on the explicit expression \cref{eq.conduc.Gamma} of $\Gamma(x, y)$, which depends on the space dimension $d$. 
\begin{itemize}
\item When $d=2$, \cref{eq.Conductivity.HNHD.IntEq} rewrites:
$$- \int_{\D_1} \log \lvert \e x - \e z \lvert \varphi_\e(z) \:\d s(z) = -\pi\gamma u_0(0) + \o(1),$$
and so: 
$$
\lvert \log\e\lvert  \left( \int_{\D_1} \varphi_\e(z) \:\d s(z) \right) - S_1 \varphi_\e(z) = -\pi\gamma u_0(0) + \o(1).
$$
This relation immediately yields: 
\begin{equation}\label{eq.expmeanvphie2d}
\int_{\D_1} \varphi_\e(z) \:\d s (z) = -\frac{1}{|\log \e|}\pi\gamma u_0(0) + \o\left( \frac{1}{\lvert\log\e\lvert}\right).
\end{equation}
\item In the case $d = 3$, \cref{eq.Conductivity.HNHD.IntEq} reads:  
$$
S_1 \varphi_\e (z) = -\frac{ \e}{2} \gamma u_0(0).
$$
Using the invertibility of $S_1: \widetilde{H}^{-1/2}(\D_1) \to H^{1/2}(\D_1)$ asserted in \cref{prop.SLP}, we obtain:
\begin{equation*}
\varphi_\e(z) = -\frac{\e}{2} \gamma u_0(0) S_1^{-1} 1(z) + \o(\e),
\end{equation*}
where the function $S_1^{-1} 1 \in \widetilde{H}^{-1/2}(\D_1)$ is the so-called equilibrium distribution of the single layer potential $S_1$.
From the physical viewpoint, it represents the density of charges on $\D_1$ that induces a constant, unit voltage potential on $\D_1$. 
Its explicit expression is derived in e.g. \cite{copson1947problem,jackson2007classical}:
\begin{equation}\label{eq.eqdistlap}
 S_1^{-1}1 (z) = \frac{4}{\pi\sqrt{1-|z|^2}}, \text{ and in particular:} \int_{\D_1} S_1^{-1} 1(z) \:\d s(z) = 8.
 \end{equation}
It thus follows that:
\begin{equation}\label{eq.expmeanvphie3d}
 \int_{\D_1} \varphi_\e(z) \:\d s (z)  = -4\e\gamma u_0(0) + \o(\e).
\end{equation}
\end{itemize}

\noindent \textit{Step 4: We pass to the limit in the representation formula \cref{eq.Conductivity.HNHD.Rep}.}

We return to this end to the representation formula \cref{eq.Conductivity.HNHD.Rep} for the value $r_\e(x)$ of the error at a point $x \notin \Sigma_D \cup \{ 0 \}$. A Taylor expansion of the smooth function $y \mapsto N(x, y)$ around $y=0$ together with the Lebesgue dominated convergence theorem yield:
 \begin{equation*}
 \begin{array}{>{\displaystyle}cc>{\displaystyle}l}
   r_\e(x) &=& \int_{\D_1} \varphi_\e(z) N(x,\e z) \:\d s(z) \\[1em]
           &=& \left(\int_{\D_1} \varphi_\e(z) \:\d s(z) \right) \left( N(x,0) + \o(1) \right),
     \end{array}
 \end{equation*}
and the desired formulas \cref{eq.ueconduc2d,eq.ueconduc3d} follow from the combination of this result with \cref{eq.expmeanvphie2d,eq.expmeanvphie3d}.
\end{proof}


%%%%%
\subsection{Asymptotic expansion of a quantity of interest}\label{sec.asymJeconduc} 
%%%%%

\noindent The asymptotic expansion of the voltage potential $u_\e$ derived in the previous section makes it possible to calculate the sensitivity of $J(\Gamma_D)$ with respect to the addition of a surface disk $\omega_{x_0,\e}$ to the Dirichlet zone $\Gamma_D$. 

\begin{corollary}\label{cor.Conductivity.HNHD.Jp}
Let $x_0 \in \Gamma$ be given. The perturbed value $J((\Gamma_D)_{x_0,\e})$ in \cref{eq.JGeconduc}, accounting for the replacement of the homogeneous Neumann boundary condition on $\omega_{x_0,\e} \subset \Gamma$ by a homogeneous Dirichlet boundary condition, has the following asymptotic expansion:
$$
    J((\Gamma_D)_{x_0,\e}) =  J(\Gamma_D) +  \frac{\pi}{|\log \e|} \: \gamma(x_0) \: u_0(x_0) \: p_0(x_0)  +  \o\left(\frac{1}{\lvert\log\e\lvert} \right)\text{ if } d = 2,
$$
and 
 $$
   J((\Gamma_D)_{x_0,\e}) =    J(\Gamma_D) +   4 \e \: \gamma(x_0) \: u_0(x_0) \: p_0(x_0)  + \o(\e) \text{ if }  d = 3.
 $$
where $p_0$ is the unique $H^1(\Omega)$ solution to the boundary value problem:
\begin{equation}\label{eq.Conductivity.HNHD.Adjoint}
\left\{
\begin{array}{cl}
    -\dv(\gamma\nabla p_0) = -j^\prime(u_0) & \text{in }  \Omega,\\
    p_0 = 0 & \text{on } \Gamma_D,\\
    \gamma \frac{\partial p_0}{\partial n} = 0 & \text{on } \Gamma \cup \Gamma_N.
 \end{array}
 \right.
\end{equation}
\end{corollary}
\begin{proof}[Sketch of the proof]
We assume for simplicity that $d = 3$, the case $d = 2$ being similar. 
At first, using the asymptotic expansion of \cref{th.Conductivity.HNHD.Expansion}, we have:
\begin{equation*}
    J ((\Gamma_D)_{x_0,\e}) = J(\Gamma_D) +  \int_\Omega j'(u_0(x)) \left( -4 \e \gamma(x_0) \: u_0(x_0) \: N(x,x_0)\right) \: \d x + \o(\e).
\end{equation*}
This follows from an application of the Lebesgue dominated convergence theorem whose rigorous justification is detailed in \cite{dapogny2020topolig}.
Now using the representation formula \cref{eq.u0N} for the adjoint state $p_0$ in \cref{eq.Conductivity.HNHD.Adjoint}, we immediately see that:
\begin{align*}
      J ((\Gamma_D)_{x_0,\e})  &= J(\Gamma_D) - 4 \e \gamma(x_0) \: u_0(x_0) \int_\Omega j^\prime(u_0(x)) N(x, x_0) \: \d x + \o(\e)\\
    &= J(\Gamma_D) + 4 \e \gamma(x_0) \: u_0(x_0) p_0(x_0) + \o(\e),\\
\end{align*}
which is the desired result.
\end{proof}

\begin{remark}\label{rem.Conductivity.HNHD.Inhomogeneous}
It is possible to replace the homogeneous Dirichlet boundary condition on $\Gamma_D$ in \cref{eq.conducasym} by an inhomogeneous Dirichlet boundary condition, i.e.  $u_\e = u_{\text{\rm in}}$, for some given, smooth function $u_{\text{\rm in}} \in \mathcal{C}^\infty(\mathbb{R}^d)$. The above calculations can be straightforwardly adapted to this case, and the asymptotic expansion of $u_\e$ becomes:
\begin{equation*}
    u_\e(x) = u_0(x)  + \frac{\pi}{\lvert \log \e \lvert} \gamma(x_0) (u_{\text{\rm in}}(x_0) -u_0(x_0) )N(x,x_0) + \o\left(\frac{1}{\lvert\log\e\lvert} \right) \text{ if } d = 2,
\end{equation*}
and 
\begin{equation*}
    u_\e(x) = u_0(x) +   4 \e \gamma (x_0)  (u_{\text{\rm in}}(x_0) - u_0(x_0) ) N(x,x_0) + \o (\e)  \text{ if } d = 3.
\end{equation*}
In this case, the sensitivity of replacing the homogeneous Neumann boundary condition on $\omega_{x_0,\e} \subset \Gamma$ by the inhomogeneous Dirichlet conditions $u=u_{\text{\rm in}}$ equals:
\begin{equation*}
    J((\Gamma_D)_{x_0,\e})= J(\Gamma_D) + \frac{\pi}{|\log \e|} \: \gamma(x_0) \: (u_{\mathrm{in}}(x_0) - u_0(x_0)) \: p_0(x_0)  + \o\left(\frac{1}{\lvert\log\e\lvert} \right) \text{ if }  d = 2,
\end{equation*}
and
\begin{equation*}
    J((\Gamma_D)_{x_0,\e})= J(\Gamma_D) +   4 \e \: \gamma(x_0) \: (u_{\mathrm{in}}(x_0) - u_0(x_0)) \: p_0(x_0)  + \o(\e) \text{ if }  d = 3,
\end{equation*}
where the adjoint state $p_0$ is again the solution to \cref{eq.Conductivity.HNHD.Adjoint}.
\end{remark}

\begin{remark}\label{rem.repDirbyNeu}
A counterpart of the arguments and results of this section holds in the case where homogeneous Dirichlet boundary conditions are replaced by homogeneous Neumann boundary conditions on a small surface disk, i.e. in the case where $G = \Gamma$ and $x_0 \in \Gamma_D$, see \cite{bonnetier2022small}. 
\end{remark}


%%%%%%%%%%%%%%%%%%%%%%%%%
\subsection{Extension: replacement of the homogeneous Neumann boundary condition by an inhomogeneous Neumann condition on a small subset}\label{sec.exttopoconduc}
%%%%%%%%%%%%%%%%%%%%%%%%%

\noindent This section considers the situation where the homogeneous Neumann boundary condition is replaced by an inhomogeneous Neumann condition on a small surface disk $\omega_{x_0,\e} \subset \Gamma$. The voltage potential $u_\e$ in this perturbed situation is the solution to the boundary value problem:
\begin{equation}\label{eq.conducepsneu}
\left\{
\begin{array}{cl}
-\dv(\gamma\nabla u_\e) = f & \text{in } \Omega, \\
u_\e = 0 & \text{on } \Gamma_D, \\
\gamma \frac{\partial u_\e}{\partial n} = 0 & \text{on } \Gamma \setminus \overline{\omega_\e}, \\[0.2em]
\gamma \frac{\partial u_\e}{\partial n} = g & \text{on } \Gamma_N \cup \omega_\e, \\
\end{array}
\right.
\end{equation}
where we recall that $g: \R^d \to \R$ is a given smooth function.

\begin{theorem}
The following asymptotic expansions hold, at any point $x \in \overline\Omega \setminus (\Sigma_D \cup \left\{ x_0 \right\})$:
$$ u_\e(x) = u_0(x) + 2 \e g(x_0) N(x,x_0) + \o(\e) \text{ if } d=2, $$
and
$$ u_\e(x) = u_0(x) + \e^{2} \pi g(x_0) N(x,x_0) + \o(\e^{2}) \text{ if } d=3. $$
\end{theorem}
\begin{proof}[Sketch of the proof]
The derivation of these formulas essentially follows the trail of the proof of \cref{th.Conductivity.HNHD.Expansion}, in a much simpler version. 
Again, for simplicity, we assume that $x_0 = 0$, $n(0) = e_d$, and that $\partial \Omega$ is flat around $0$, see \cref{eq.flat}. Let $r_\e := u_\e - u_0$ be the error between the perturbed and background potentials, defined by \cref{eq.conducepsneu} and \cref{eq.conduc}, respectively. 
This function satisfies the following boundary value problem:
\begin{equation}\label{eq.conducneure}
\left\{
\begin{array}{cl}
-\dv(\gamma\nabla r_\e) = 0 & \text{in } \Omega, \\
r_\e = 0 & \text{on } \Gamma_D, \\
\gamma \frac{\partial r_\e}{\partial n} = 0 & \text{on } (\Gamma_N \cup \Gamma) \setminus \overline{\omega_\e}, \\[0.2em]
\gamma \frac{\partial r_\e}{\partial n} = g & \text{on } \omega_\e. \\
\end{array}
\right.
\end{equation}
From the definition of the Green's function $N(x,y)$ in \cref{eq.Greenconduc}, it holds:
$$ r_\e(x) = - \int_\Omega \dv_y( \gamma(y) \nabla_y N(x,y)) r_\e(y) \:\d y,$$
and so, after integration by parts:
$$ r_\e(x) = -\int_{\partial \Omega} \gamma(y) \frac{\partial N}{\partial n_y}(x,y) r_\e(y) \:\d s(y) + \int_{\partial \Omega} \gamma(y) \frac{\partial r_\e}{\partial n}(y) N(x,y) \:\d s(y).$$
Now using the boundary conditions satisfied by the functions $r_\e$ and $y \mapsto N(x,y)$, it follows:
$$ \begin{array}{>{\displaystyle}cc>{\displaystyle}l}
r_\e(x) &=& \int_{\omega_\e} g(y) N(x,y) \:\d s(y) \\
&=& \e^{d-1} \int_{\D_1} g(\e z) N(x,\e z) \:\d s(z) \\
\end{array} $$
and so
$$ r_\e(x) = \e^{d-1} g(0) N(x,0) \left(\int_{\D_1} \:\d s(z)\right) + \o(\e^{d-1}),$$
which yields the desired result.
\end{proof}

We now turn to the computation of the sensitivity of the objective function $J(\Gamma_N)$ in \cref{eq.defJ} in the present context, i.e. we consider the asymptotic expansion of the quantity of interest:
$$ J((\Gamma_N)_{x_0,\e}) := \int_\Omega j(u_\e) \:\d x.$$ 
 The proof of the following result is omitted, as it is completely similar to that of \cref{cor.Conductivity.HNHD.Jp}. 

\begin{corollary}
Let $x_0$ be a given point in $\Gamma$; the perturbed value $J((\Gamma_N)_{x_0,\e})$ of the objective function $J(\Gamma_N)$, accounting for the replacement of the homogeneous Neumann boundary condition by the inhomogeneous Neumann boundary condition $\gamma\frac{\partial u}{\partial n} = g$ on a ``small'' surface disk $\omega_{x_0,\e}$, has the following expansion:
$$ J((\Gamma_N)_{x_0,\e}) = J(\Gamma_N) - 2\e g(x_0)p_0(x_0) + \o(\e) \text{ if } d=2,$$
and
$$ J((\Gamma_N)_{x_0,\e}) = J(\Gamma_N) - \pi\e^2 g(x_0)p_0(x_0) + \o(\e^2) \text{ if } d=3,$$
where $p_0 \in H^1(\Omega)$ is the solution to \cref{eq.Conductivity.HNHD.Adjoint}. 
\end{corollary}