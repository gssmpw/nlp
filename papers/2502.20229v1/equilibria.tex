

\section{Implications for Equilibrium Computation}\label{sec:equilibria}


We finally return to the question of equilibrium computation. Indeed, the original motivation for introducing swap regret was to design learning dynamics that converge to the notion of correlated equilibria in normal-form games (e.g., see \cite{foster1997calibrated}). It seems natural then, that when defining swap regret for polytope games, we should choose a quantity whose minimization guarantees convergence to the set of correlated equilibria.

The problem here is that, just as it is not exactly clear what the definition of swap regret should be in polytope games, it is also not exactly clear what the definition of correlated equilibrium should be in polytope games. Even for the restricted set of Bayesian games, existing correlated equilibrium notions include linear correlated equilibria, normal-form correlated equilibria, agent-normal-form correlated equilibria, and communication equilibria \citep{fujii2023bayes}. This raises the question of why we might want to compute correlated equilibria in the first place -- what properties might we desire from its definition?

One motivation for studying correlated equilibria in normal-form games, arising from the original definition in \cite{aumann1974subjectivity}, is that we can view a correlated equilibrium as an outcome that can be implemented by a third-party \emph{mediator} who privately recommends an action to each player. If no one can gain by deviating from these suggestions, the resulting strategy profile constitutes a correlated equilibrium. This gives a definition of correlated equilibrium that is particularly amenable to mechanism design (the classic example here is that of a traffic light coordinating the actions of many cars at an intersection). 

This mediator-based definition is relatively straightforward to extend to the setting of a polytope game $G$ -- we are looking for distributions over joint recommendations $(x, y) \in \learnset \otimes \optset$ with the property that neither player has an incentive to deviate after seeing their recommendation. It can be shown (see Appendix \ref{app:mediator-nf}) that these recommendations can, without loss of generality, be supported on the vertices of $\learnset$ and $\optset$, and so these mediator-based correlated equilibria are exactly the \emph{normal-form correlated equilibria (NFCE)} of $G$ (equivalently, these are the correlated equilibria of the normal-form vertex game $G^{V}$). Formally, an NFCE of $G$ is a vertex game CSP $\cspV \in \Delta(\learnvert \times \optvert)$ with the property that\footnote{Since in this section we are concerned with settings where both players are learning, instead of referring to the two players as the ``learner'' and ``optimizer'', we will refer to them as the ``$\cX$-player'' and ``$\cY$-player'', adjusting subscripts accordingly.} $\NormSwap_{X}(\cspV) = \NormSwap_{Y}(\cspV) = 0$.  Since this game can be thought of as a normal-form game over the action set simplices $\Delta(\learnvert)$ and $\Delta(\optvert)$,  $\NormSwap_{X}(\cspV)$ is simply defined as the swap-regret of a CSP $\cspV$ in this game.

But a second, learning-theoretic, motivation for studying correlated equilibria is that they directly represent possible summaries of the outcomes of learning dynamics in repeated games. These are useful for understanding what possible outcomes we might expect from repeated multi-agent interactions in strategic settings -- after all, it is now generally accepted that it is intractable (computationally and otherwise) for players to play according to the Nash equilibrium of an arbitrary general-sum game, and instead it is more reasonable to model players as performing some form of learning to decide their actions over time.

From this perspective, the definition of correlated equilibrium should follow from the definition of swap regret (specifically, whichever definition of swap regret we choose to model rational behavior in repeated games). For profile swap regret, this gives rise to the notion of \emph{profile correlated equilibria (PCE)}. In particular, a CSP $\csp$ is a PCE of a polytope game $G$ if $\CorrSwap_{X}(\csp) = \CorrSwap_{Y}(\csp) = 0$.

In normal-form games, these two motivations give rise to exactly the same notion of correlated equilibrium (and hence this distinction often goes unmade). In polytope games these two notions are not immediately comparable -- a normal-form CE $\cspV$ and a profile CE $\csp$ belong to different sets and have different dimensions. However, every normal-form CE $\cspV \in \Delta(\learnvert \times \optvert)$ naturally corresponds to a CSP $\Proj(\cspV)$ obtained by sending $\cspV$ to the element $\sum_{v \in \learnvert, w \in \optvert} \cspV(v, w) (v \otimes w)$. This raises the question: are profile correlated equilibria exactly the CSPs corresponding to valid normal-form correlated equilibria?

To see why we might expect this to be the case for profile swap regret in particular, we point out that a one-sided version of this question has a positive answer. In particular, every CSP $\csp$ that incurs zero profile swap regret for the $\cX$-player (i.e., satisfies $\CorrSwap_{\cX}(\csp) = 0$) can be instantiated as $\Proj(\cspV)$ for some vertex-game CSP $\cspV$ that incurs zero normal-form swap regret for the $\cX$-player (see Appendix \ref{app:one-sided}). From a mediator point-of-view, this means that any CSP $\csp$ the $\cX$-player encounters by playing a no-profile-swap-regret learning algorithm can also be induced by a ``one-sided mediator'' that only needs to consider the incentives of the $\cX$-player (i.e., one that the $\cY$-player will blindly trust). We note that this property follows from the fact that the no-profile-swap-regret menu is minimal (Theorem~\ref{thm:poly_minimal}), and does not hold for weaker forms of regret like linear swap regret.

However, we prove that the answer to the original question is no. Specifically, we show that although every CSP $\Proj(\cspV)$ corresponding to a normal-form CE $\cspV$ is in fact a profile CE, there exist profile CE that cannot be written in this form. In fact, we go further and show that there are profile CE with utility profiles that no normal-form CE can generate.

\begin{theorem}\label{thm:equilibrium-gap}
    If $\cspV$ is a normal-form CE in a polytope game $G$, then $\Proj(\cspV)$ is a profile CE in $G$. On the other hand, there exists a polytope game $G$ and a profile CE $\csp$ such that there does not exist a normal-form CE $\cspV$ satisfying $\Proj(\cspV) = \csp$ (in fact, there does not even exist a normal-form CE $\cspV$ where $u_X(\cspV) = u_X(\csp)$ and $u_Y(\cspV) = u_Y(\csp)$).
    % \begin{itemize}
    %     \item $\phi$ is a PCE
    %     \item There does not exist a NFCE $\sigma$ such that $u_L(\phi) = u_L(\sigma)$ and $u_O(\phi) = u_O(\sigma)$.
    %     \item $\learnset$ and $\optset$ are affine subspaces.
    % \end{itemize}
\end{theorem}

In particular, Theorem~\ref{thm:equilibrium-gap} means that there are outcomes of repeated polytope games (i.e., CSPs $\csp$) for which both players can imagine as implementable by a one-sided mediator, but for which no (two-sided) mediator protocol can actually induce. This is a fundamental property of polytope games that extends beyond the specific definition of profile swap regret, and distinguishes general polytope games from normal-form games.

Of course, we can also ask for which polytope games $G$ does a separation exist like that in Theorem~\ref{thm:equilibrium-gap} -- after all, there is no separation for the case of normal-form games, which are special cases of polytope games. Fully characterizing this is an interesting open question, but below we show that this gap disappears whenever \emph{either} player's action set is a simplex. Notably, this covers all standard Bayesian games where only one of the two players has any private information (e.g., some auction/pricing games). 

\begin{theorem}\label{thm:equiv-simplex}
Let $G$ be a polytope game where $\cX$ is a simplex (but where $\cY$ may be any polytope). Then, if $\csp$ is a profile CE in $G$, there exists a normal-form CE $\cspV$ for which $\Proj(\cspV) = \csp$.
\end{theorem}
\begin{proof}
Since $\CorrSwap_{X}(\csp) = 0$, we can write

\begin{equation}\label{eq:equiv-cspx}
\csp = \sum_{v \in \learnvert} \lambda_{v} (v \otimes y_{v}).
\end{equation}

\noindent
for some choice of $y_{v} \in \optset$ and nonnegative $\lambda_{v}$ summing to one. Similarly, since $\CorrSwap_{Y}(\csp) = 0$, we can write

\begin{equation}\label{eq:equiv-cspy}
\csp = \sum_{w \in \optvert} \mu_{w} (x_{w} \otimes w).
\end{equation}

We claim that, when $\cX$ is a simplex, it is possible to decompose each $x_{w}$ into a combination of elements in $\learnvert$ and each $y_{v}$ into a combination of elements in $\optvert$ in such a way that these two decompositions agree. That is, we can write $x_{w} = \sum_{v \in \learnvert} \gamma_{v, w}v$ and $y_{v} = \sum_{w \in \optvert} \gamma'_{v, w}w$ so that, for each $(v, w) \in \learnvert \times\optvert$,

\begin{equation}\label{eq:equiv-simplex-1}
    \lambda_{v}\gamma'_{v, w} = \mu_{w}\gamma_{v, w},
\end{equation}

\noindent
and we can therefore write

$$\csp = \sum_{v \in \learnvert} \sum_{w\in\optvert} \zeta_{v, w}(v \otimes w)$$

\noindent
where $\zeta_{v, w}$ is the common value of the two sides of \eqref{eq:equiv-simplex-1}. But now, note that this gives rise to a normal-form CE $\cspV \in \Delta(\learnvert\times\optvert)$ via $\cspV(v, w) = \zeta_{v,w}$. In particular, by \eqref{eq:equiv-cspx}, we have that $\NormSwap_{X}(\cspV) = 0$, and by \eqref{eq:equiv-cspy}, we have that $\NormSwap_{Y}(\cspV) = 0$.

It remains to show that a common decomposition (i.e., of the form in \eqref{eq:equiv-simplex-1}) exists. To see this, note that since $\cX$ is a simplex, there is a unique way to write each $x_w$ as a convex combination $\sum_{v \in \learnvert} \gamma_{v, w}v$. We claim that if we then define $\gamma'_{v, w} = (\mu_{w}\gamma_{v, w})/\lambda_{v}$ (so to satisfy \eqref{eq:equiv-simplex-1}), it must be the case that $y_{v} = \sum_{w \in \optvert} \gamma'_{v, w}w$. 

To see this, for any $v \in \learnvert$, consider the bilinear function $\rho_{v}: \learnset \otimes \optset \rightarrow \Rset\optset$ defined via $\rho_{v}(v \otimes w) = w$ and $\rho_{v}(v' \otimes w) = 0$ for $v' \neq w$ (note that this function only exists since $\cX$ is a simplex). Applying $\rho_v$ to \eqref{eq:equiv-cspx}, we have that $\rho_{v}(\phi) = \lambda_{v}y_{v}$. On the other hand, applying $\rho_v$ to \eqref{eq:equiv-cspy} (after substituting in our above decomposition for $x$), we have that $\rho_{v}(\phi) = \sum_{w}\mu_{w}\gamma_{v,w}w$. Equating these two expressions for $\rho_{v}(\phi)$, it follows that $y_{v} = \sum_{w} \gamma'_{v, w}w$, as desired.
\end{proof}

Finally, we conclude with a couple of remarks on the problem of actually computing equilibria. Note that our efficient no-profile-swap-regret learning algorithm (Theorem~\ref{thm:upper-semi-separation}) immediately gives us an efficient algorithm to compute \emph{some} profile CE in any (efficiently representable) polytope game $G$, by making both players run these low-profile-swap-regret dynamics. In particular, define an \emph{$\eps$-approximate profile CE} $\csp$ to be a CSP satisfying $\CorrDist(\csp) \le \varepsilon$ for both players. We have the following theorem.

\begin{theorem}\label{thm:ce-computation}
    Given efficient separation oracles for the sets $\learnset$ and $\optset$, there exists an algorithm that runs in polynomial time in $d_L, d_O, \frac{1}{\varepsilon}$ and computes an $\eps$-approximate profile CE.
\end{theorem}

\begin{proof}
    Our algorithm simulates repeated game play between the two players, with both players employing Algorithm~\ref{algo:semisep-approach} for $T = \Theta(\eps^{-2})$ rounds. The algorithm runs in polynomial time and by Theorem~\ref{thm:upper-semi-separation}, directly guarantees that $\CorrDist(\csp) = O(\sqrt{T}) \leq 1/\eps$ (for an appropriate choice of the constant in $\Theta(\eps^{-2})$) for both players.
\end{proof}

One interesting consequence of Theorems~\ref{thm:equiv-simplex} and \ref{thm:ce-computation} is that it leads to decentralized dynamics for efficiently computing a CSP corresponding to a normal-form CE in any polytope game where the action set of one of the two players is a simplex (e.g., the class of Bayesian games mentioned previously). However, it is also possible to efficiently compute a succinct representation of such a normal-form CE directly, by running learning dynamics where the simplex player plays a no-swap-regret algorithm and the other player best responds every round \citep{zhang_personal_communication}.

Theorem~\ref{thm:ce-computation} allows us to compute a single profile CE. On the other hand, if we want to \emph{optimize} over the set of profile CE, we need to contend with the hardness result in Theorem~\ref{thm:hardness} -- in fact, by setting the utilities of one of the players to zero (so that they are indifferent between all outcomes), the set of profile CE becomes exactly the no-profile-swap-regret menu, which is provably hard to optimize over. This situation is reminiscent of the situation in \cite{papadimitriou2008computing} for computing correlated equilibria in succinct multiplayer games (although note that here this phenomenon occurs for games with only two players).

% To gain some intuition for Theorem~\ref{thm:equilibrium-gap}, it is helpful to introduce a bit of additional notation. Let $\cM_{X}$ and $\cM_{Y}$ be the sets of CSPs $\csp \in \learnset \otimes \optset$ satisfying $\CorrSwap_{X}(\csp) = 0$ and $\CorrSwap_{Y}(\csp) = 0$, respectively (in particular, $\cM_{X}$ and $\cM_{Y}$ are the no-profile-swap-regret menus for the two players). Similarly, let $\cM_{X}^{V}$ and $\cM_{Y}^{V}$ be the sets of vertex game CSPs $\cspV \in \Delta(\learnvert \times \optvert)$ satisfying $\NormSwap_{X}(\cspV) = 0$ and $\NormSwap_{Y}(\cspV) = 0$, respectively. From these definitions, it follows that $\cM_{X} \cap \cM_{Y}$ represents the set of profile CE, and $\cM^{V}_X \cap \cM^{V}_{Y}$ represents the set of normal-form CE. The negative result in Theorem~\ref{thm:equilibrium-gap} can be thought of as exhibiting a game where $\Proj(\cM^{V}_{X} \cap \cM^{V}_{Y}) \neq \cM_{X} \cap \cM_{Y}$. 

% However, from the facts that normal-form swap regret upper bounds profile swap regret (Theorem~\ref{thm:comp-prof-swap}) and that the no-profile-swap-regret menu is minimal, it is possible to show that $\Proj(\cM_{X}^{V}) = \cM_{X}$ and $\Proj(\cM_{Y}^{V}) = \cM_{Y}$ (see \textbf{INSERT}) -- in particular, any CSP that attains . 

% \jon{separator}

% We now turn to the final consequence of no-swap regret algorithms in normal-form games: equilibrium computation. When both players deploy some no-swap regret algorithm in a normal-form game, the set of possible time-averaged CSPs is exactly the set of all possible Correlated Equilibria (CE). Does this relationship hold in all polytope games? That is, when both players deploy a no-swap regret algorithm in a polytope game, is the set of possible time-averaged CSPs exactly the set of all possible CE of the game? 

% The problem with this question, of course, is that neither swap regret nor CE have agreed-upon definitions beyond normal-form games. So far, we have found that \emph{Profile Swap Regret} is the necessary and sufficient condition to lift the properties of no-swap regret in normal-form games (non-manipulability, minimality, Pareto-optimality) to polytope games. Thus, we might hope to answer this question about Profile Swap Regret. But we still do not have a clear definition of CE. Let's examine the definition of CE to better understand how to generalize it. CE in normal-form games are defined as follows:

% \nat{add def}

% The standard interpretation of CE, including by their originator Aumann \nat{cite the 87 paper}, is that they arise from a game in which a \emph{mediator} privately recommends actions to each player based on a specified probability distribution. Each player then chooses their action contingent on the mediator’s recommendation. If no one can gain by deviating from these suggestions, the resulting strategy profile constitutes a correlated equilibrium. Under this interpretation, a CE is defined by the existence of a signal scheme which can incentivize it, and the fact that all No-Swap Regret dynamics converge to CE is an interesting byproduct of this definition. 

% More recently, \nat{stuff that Jon said about how Papadimitriou et. al think about CE as summaries of a learning process}. This work encourages thinking about summaries of play of learning dynamics as themselves fundamental equilibrium notions of a game. Under this interpretation, CE are not the result of a mediator game, but the result of learning dynamics, and a CE is defined by the existence of two No-Swap Regret algorithms that converge to it. 

% In normal-form games, this distinction is purely aesthetic, as the two perspectives lead to the exact same definition of CE. However, in polytope games, this is no longer the case. We show that there exist CSPs resulting from no-profile swap regret dynamics that cannot result from players rationally responding to a centralized signaling scheme.  

% We can define the CSPs resulting from no-profile swap regret dynamics as \emph{profile correlated equilibria} (PCE), and the CSPs resulting from no-normal form swap regret dynamics as \emph{normal-form correlated equilibria} (NFCE). To do this, we require some additional notation. Up until this point we have focused on the sequential learning setting, and thus we modeled correlations in the Learner's and Optimizer's actions as emerging from repeated rounds of non-correlated play. This action profile can be seen as a distribution over vertex pairs, and in fact this distribution representation is sufficient to compute normal-form swap regret. When thinking about NFCE we do not have any notion of sequential interaction, and instead we understand correlations as emerging via a central mediating device. We will therefore skip over the sequential perspective in this section and allow our notion of Normal-Form Swap Regret to operate directly on the joint distribution over vertex pairs. Futhermore, given that there is no natural asymmetry in this setting, we will heretoforth refer to the two players as Player 1 and Player 2.   

% \begin{definition}[Profile Correlated Equilibria (PCE)]
%     Given a polytope game $(\learnset, \optset, u_L, u_O)$, a $d_L \cdot d_O$-dimensional vector $\phi$ is a PCE iff $\CorrSwap_{L}(\phi) = 0$ and $\CorrSwap_{O}(\phi) = 0$.
% \end{definition}

% PCE are defined in terms of the dimensions of $\learnset$ and $\optset$, and thus are convex sets in $d_L \cdot d_O$ dimensions. 

% Now we must consider how to define CSPs resulting from mediators. CSPs are defined over pairs of $(\learnset, \optset)$ dimensions. But the mediator interpretation involves recommending extreme points of each player's action set $\learnvert$ and $\optvert$ to them. In normal-form games, where $\learnset = \Delta_{m}$ and $\optset = \Delta_{n}$, these objects are equivalent. Not so in polytope games. To define the space of CSPs generated by mediators, we must first considering the higher-dimensional space that naturally generates them--Normal Form Correlated Equilibria (NFCE). NFCE are distributions over pairs of \emph{extreme points} of $\learnset$ and $\optset$, denoted as $\learnvert$ and $\optvert$. Thus, NFCE are $|\learnvert| \cdot |\optvert|$-dimensional.  
% \begin{definition}[Normal-Form Correlated Equilibria (NFCE)]
%     Given a polytope game $(\learnset, \optset, u_L, u_O)$, a $|\learnvert| \cdot |\optvert|$-dimensional vector $\sigma$ is a NFCE iff $\NormSwap_{L}(\sigma) = 0$ and $\NormSwap_{O}(\sigma) = 0$.
% \end{definition}

% Clearly, these two notions are not equivalent--NFCEs are distributions over pairs of extreme points in $\learnset$ and $\optset$, while each index of a PCE $\phi$ is associated with a dimension of $\learnset$ and a dimension of $\optset$. However, every extreme point in $\learnvert$ and $\optvert$ can itself be written in terms of its coordinates in $\learnset$ and $\optset$, respectively, and therefore there is a natural projection from the space of NFCEs to the CSP space. 

% \begin{definition}[NFCE Projection $\pi$]
%     Given a NFCE $\sigma$, the projection into CSP space is defined as $\pi(\sigma) = \sum_{i, j} \sigma_{ij} \cdot (v_{i} \otimes w_{j})$
% \end{definition}

% Thus, every NFCE $\sigma$ has a corresponding CSP $\pi(\sigma)$, which is a PCE. But does every PCE $\phi$ have a corresponding NFCE? That is, for every CSP $\phi$ which is a Profile Correlated Equilibria, does there exist a NFCE $\sigma$ such that $\pi(\sigma) = \phi$? If this were the case, it would show that every PCE can be generated via a mediating device. However, we show that there exist PCE that are not the projection of any NFCE, and therefore cannot be generated via any mediating device. In fact, we go further and show a PCE that is not only not a projection of any NFCE, but generates a utility profile that no NFCE can generate. 

% Why does such a separation exist? Intuitively, a PCE must be the projection of some distribution over vertex pairs that ``look like" a NFCE from the perspective of the learner. It also must be the projection of some distribution over vertex pairs that ``look like" a NFCE from the perspective of the optimizer. But importantly, these need not be the \emph{same} distribution, and there does not always exist a single distribution over vertex pairs which satisfies both conditions simultaneously. Thus, even though an algorithm minimizing Profile Swap Regret satisfies numerous conditions that are typically associated with Swap Regret in normal-form games, there is not always a way to generate it via a signaling scheme over polytope vertices which is simultaneously rational for both players. We given an example of this below.


% \begin{comment}
% \begin{theorem}
% Let X have extreme points V(X) = $\{v_1, v_2, ..., v_m\}$, and let Y have extreme points V(Y) = $\{w_1, w_2, ..., w_n\}$. Consider some CSP $\phi \in X \otimes Y$. There exist $x_1, ..., x_n$ in X and $y_1, ... y_m$ in Y such that $\phi = \sum_i \lambda_i * (v_i \otimes y_i)
%       = \sum_j \mu_j * (x_j \otimes w_j)$. Is there a way to decompose $y_i = \sum_{j} \gamma_{ij}w_j$ and $x_j = \sum_{i} \gamma’_{ij}v_i$ such that $\lambda_i*\gamma_{ij} = \mu_j*\gamma’_{ij}$ for all i and j?
% \end{theorem}
% \begin{proof}
% Note that as $x_{j} \in X$, for any $j$, there exists $\gamma_{1j},\ldots,\gamma_{mj}$ such that $\sum_{\forall i \in [m]}\gamma_{ij}v_{i} = x_{j}$. Similarly, as $y_{i} \in Y$, for any $i$, there exists $\gamma'_{i1},\ldots,\gamma'_{in}$ such that $\sum_{\forall j \in [n]}\gamma'_{ij}w_{j} = y_{i}$. Now, by assumption, we have that

% \begin{align*}
%     & \phi = \sum_{\forall i \in [m]} \lambda_i (v_i \otimes y_i)
%       = \sum_{\forall j \in [n]} \mu_j (w_j \otimes x_j) \\
%       & \implies  \phi = \sum_{\forall i \in [m]} \lambda_i (v_i \otimes \sum_{\forall j \in [n]}\gamma'_{ij}w_{j})
%       = \sum_{\forall j \in [n]} \mu_j (w_j \otimes \sum_{\forall i \in [m]}\gamma_{ij}v_{i}) \\
%        & \implies  \phi = \sum_{\forall i \in [m], j \in [n]} \lambda_i\gamma'_{ij} (v_i \otimes w_{j})
%       = \sum_{\forall i \in [m], j \in [n]} \mu_j\gamma_{ij} (w_j \otimes v_{i}) \\
% \end{align*}

% \end{proof}
% \end{comment}