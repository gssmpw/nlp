\section{Related Work}
% Almost all of these results are constrained to the discrete setting for low dimension. Very recently  ____ and ____ presented algorithms that achieve full swap regret of $O(T/\log T)$, independent of $d$. This work improves on these results in the regime $d = o(\log(T)/\log\log(T))$. ____ and ____ study the problem of designing forecasts such that any downstream agent incurs low swap regret; the problem faced by a single downstream agent (with potentially many actions but where the payoff only depends on a low-dimensional outcome) can be interpreted as a swap-regret minimization problem in a structured game. 

\paragraph{Swap Regret and $\Phi$-regret} Swap regret has long been an object of interest in normal-form games ____. Early efficient methods to achieve bounded swap regret were established by____, who proposed efficient algorithms that guarantee low internal regret. More recently, concurrent work by ____ and ____ have shown that it is possible to minimize swap regret in any online learning setting where it is possible to minimize external regret. 

____ introduced a generalization of swap regret to convex games called $\Phi$-regret, where one competes with a family $\Phi$ of functions mapping the action set into itself. This generalization captures many other swap regret notions of interest. Restricting $\Phi$ to only contain linear functions, we obtain linear swap regret. Linear swap regret has recently been studied extensively in both Bayesian games ____ and extensive form games ____, with ____ providing efficient algorithms for minimizing linear swap regret in general polytope games. Recently ____ studied $\Phi$-regret minimization for classes of functions $\Phi$ specified by low degree polynomials. Finally, ____ study the notion of \emph{full swap regret}, allowing $\Phi$ to be the set of all (non-linear) functions mapping the action set into itself. There are some other variants of $\Phi$-regret minimization studied towards the goal of computing specific variants of correlated equilibria in Bayesian or extensive-form games; we survey those below. 

\paragraph{Strategizing in games} While no-swap regret algorithms were first conceptualized via their connection to correlated equilibrium, a more recent line of work has investigated the strategic properties of these algorithms in their own right.____ initiated the study of non-myopic responses to learning algorithms in the context of single buyer auctions, demonstrating that when bidders run standard learning algorithms to choose their bids, they can be fully manipulated by a seller (who can extract the full surplus of the auction, leaving the buyer with zero utility). Since then, there has been a large line of recent work focused on understanding which learning algorithms provide provable game-theoretic guarantees in settings such as auctions ____, principal-agent problems ____, general normal-form games ____, and Bayesian games ____. 


\paragraph{Correlated equilibria in polytope games} The concept of correlated equilibria originates from____ as a generalization of the notion of a Nash equilibrium for players who can correlate their play. Correlated equilibria also have the nice property that unlike Nash equilibria, they are computable in polynomial time ____, at least in normal-form games. One of the main motivations for designing no-swap-regret learning algorithms is to construct decentralized learning dynamics that provably converge to correlated equilibria at fast rates (e.g, ____).

On the other hand, the simplest generalization of correlated equilibria to general polytope games -- that is, to \emph{normal-form correlated equilibria (NFCE)}, formed by each extremal strategy as a pure strategy in the corresponding game -- might blow up the size of the game exponentially, and there is therefore no known efficient algorithm for computing NFCE. Moreover, there is no clear way to ``correctly'' generalize the original definition of ____ to these settings -- ____ introduces ``five legitimate definitions of correlated equilibrium'' for games with sequential imperfect information. In Bayesian games, ____ introduce a notion of Bayes correlated equilibrium, but did not discuss computational aspects; more recently, ____ studies three refinements of this notion (agent-normal-form CE, communication equilibria, and strategic-form CE), and shows how to compute communication equilibria by minimizing ``untruthful swap regret'' (a variant of linear swap regret). In extensive-form games, ____ introduce the concept of \emph{extensive-form correlated equilibria}, which can be computed in polynomial-time in the representation of the game either directly ____ or by decentralized learning dynamics ____ (minimizing ``trigger swap regret''). 

\paragraph{Blackwell approachability} The main technique we use to design efficient learning algorithms for minimizing profile swap regret is an application of the semi-separation framework of ____ to the general problem of Blackwell approachability ____. ____ demonstrated a reduction from Blackwell approachability to  regret-minimization that we use in this application. The orthant-approachability form of Blackwell approachability that we introduce in Section~\ref{sec:algorithms} appears implicitly in many follow-up works that focus on improving the rates of approachability algorithms ____.

%  In this setting,____ showed that no-swap regret algorithms are \emph{non-manipulable}, and thus optimizing against a no-swap regret algorithm involves optimizing only over the class of static strategies that do not vary over time. 

% ____ introduced the study of non-myopic responses to learning algorithms in \emph{Bayesian} games and Polytope games more generally, identifying Polytope Swap Regret as a sufficient notion for non-manipulability in these games. ____ prove that Polytope Swap Regret is also \emph{necessary} in general Bayesian games, but in the game-agnostic setting. By contrast, we show that when the structure of the game is fixed, the strictly weaker notion of~\emph{profile} swap regret is both necessary and sufficient for all Polytope games. 

 %In contrast, we show that a significant version of correlated equilibria in more general classes of games (polytope games) are achieved