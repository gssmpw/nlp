\section{Related Work}
Due to space limitations, we provide a more comprehensive discussion in the Appendix~\ref{app:related_work}.

\subsection{LLM-based Human Simulation}
LLMs have been increasingly used to simulate human behaviors across various domains, including social interactions \citep{gao2023s, qian2024chatdev}, economic decision-making \citep{horton2023large, zhao2023competeai}, and physical mobility \citep{jin2023surrealdriver, zou2023wireless}. Additionally, studies have explored hybrid models \citep{park2023choicemates, williams2023epidemic} and medical simulations \citep{du2024llms}. However, little work has been done on simulating students in academic advising.

\subsection{Student Simulation in Education}
LLMs have been leveraged to simulate students for educational purposes, such as training teaching assistants \citep{markel2023gpteach}, modeling student profiles \citep{lu2024generative}, and generating personality-driven student agents \citep{liu2024personality, jin2024teachtune, ma2024students}. However, simulating students with discrepancies in knowledge or learning skills remains unexplored.



\begin{figure}[t]
  \includegraphics[width=\columnwidth]{figs/student_profile.pdf}
  \caption{Student profiles include demographic information and other questionnaires related to studying. Detailed profile is provided in Appendix~\ref{app:profile}}
  \label{fig:student_profile}
  \vspace{-5mm}
\end{figure}