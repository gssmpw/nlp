\section{Related work\label{Section related work}
}

The Gibbs algorithm traces its origin to the work of \cite%
{boltzmann1877beziehung} and \cite{gibbs1902elementary} on statistical
mechanics, and its relevance to machine learning was recognized by \cite%
{levin1990statistical} and \cite{opper1991calculation}. \cite%
{mcallester1999pac} realized that the minimizers of the PAC-Bayesian bound
are Gibbs distributions. The fact that they are limiting distributions of
stochastic gradient Langevin dynamics (\cite{raginsky2017non}), raises the
question about the generalization properties of individual hypotheses as
addressed in this paper. Average generalization of the Gibbs posterior was
further studied notably by \cite{aminian2021exact} and \cite%
{aminian2023information}, where there are also investigations into the
limiting behavior as $\beta \rightarrow \infty $.

Theorem \textbf{\ref{Theorem Main}} is part of the circle of information
theoretic ideas in machine learning, ranging from the PAC-Bayesian theorem (%
\cite{shawe1997pac}, \cite{mcallester1999pac}, \cite{mcallester2003pac}, 
\cite{catoni2003pac}) to generalization bounds in terms of mutual
information (\cite{russo2016controlling} and \cite{xu2017information}). It
is inspired by and indebted to the disintegrated PAC-Bayesian bounds as in 
\cite{blanchard2007occam}, \cite{rivasplata2020pac} and \cite%
{viallard2024general}.

The benefit of wide minima was noted by \cite{hochreiter1997flat}, where
also a variant of the Gibbs algorithm was discussed. The idea was promoted
by \cite{keskar2016large} and others \cite{zhang2018theory}, \cite%
{iyer2023wide}. It was soon objected by \cite{dinh2017sharp} that there are
narrow reparametrizations of wide minima which compute the same function.
Several authors then searched for reparametrization-invariant measures of
"width" (\cite{andriushchenko2023modern}, \cite{kristiadi2024geometry}).
Nevertheless it was early conjectured (\cite{neyshabur2017exploring}), that
the relevant property is average width, which is also the position of the
paper at hand.