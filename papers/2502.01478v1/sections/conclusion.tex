\section{Concluding Discussion}
\label{sec:conclusion}
We present \name -- a new connectivity model for agricultural applications. We highlight the challenge of under-canopy connectivity for both CBRS and satellite signals. We present a height-variable base station design to alleviate performance degradation due to crops. Finally, we demonstrate under-canopy robot teleoperation. We conclude with some remarks:

\para{Deployment Models: }We envision two deployment models for \name. In a community-driven model, a community may share a \name\ setup and use it to serve multiple farms. Typically, both farming and harvest seasons are spread out over a few weeks in a given community. So, such sharing models may be feasible. Second, some robots/equipment companies work on a rental model, where farm equipment or even a planting/harvesting/cover-cropping service is leased to a farmer. In such cases, the equipment/service provider can bring their own \name\ infrastructure. 

\para{Self-Calibration for Different Crops/Weather: } In this work, we considered CBRS signal propagation through dry corn. To adapt \name\ to different crops and weather conditions, it is only necessary to change the attenuation coefficient $\alpha$ accordingly. Note that \name\ knows the signal reading and location of clients, location and height of its base station, and the crop height. Thus, it is possible to adapt the attenuation coefficient $\alpha$ online by inverting the signal model. Thus, \name\ can continually adapt itself accordingly for different crops and weather conditions throughout the season. We leave such exploration to future work.



\definecolor{green}{RGB}{0, 128, 0}
\definecolor{red}{RGB}{255, 0, 0}
\definecolor{mediumyellow}{RGB}{220, 204, 0}

\newcommand{\greencheck}{\textcolor{green}{\checkmark}}
\newcommand{\xmark}{\textcolor{red}{\ding{55}}}

