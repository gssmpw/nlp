
\section{Benefits of Horizontal Motion}
\label{sec:motion}

\begin{figure*}[t!] %
    \centering
    \subfloat[CBRS network coverage with one sector antenna. White: without crops. Green: with crops.]{%
        \includegraphics[width=0.28\textwidth]{figs/coverage_area_v2.pdf}
        \label{fig:coverage_area}
    }%
    \hfill
    \subfloat[Covering a $4.8 \times 4.8$ km farm by replicating deployment in \ref{fig:coverage_area}. Each bubble has radius 1 km.]{%
        \includegraphics[width=0.30\textwidth]{figs/mesh_network} 
        \label{fig:large_farm}
    }%
    \hfill
    \subfloat[Covering the same farm w/ \name. Any point can be covered by moving \name\ appropriately.] {%
        \includegraphics[width=0.30\textwidth]{figs/byon_cover.pdf}
        \label{fig:horizontal_motion}
    }%
\vspace{-0.15in}
\caption{Extending wireless coverage across a large-scale farmland. By observing that digital agriculture applications are localized in space and time, BYON can achieve large decreases in infrastructure cost when compared with conventional static infrastructure. }\vspace{-0.15in}
\label{fig:coverage}
\end{figure*}
\noindent\textbf{{Seasonal Rhythms of Agriculture:}} Our key observation is that the connectivity requirements in agriculture are restricted in space and time. For example,~\cite{iowaextension} lists some activities that use equipment and how long they take. In an 800 acre field planting corn, pre-planting activity such as nitrogen application would take about 8 days with the tractor covering about 96 acres per day (0.4 $Km^2$). Similarly, planting takes about 3 days, covering approx. 280 acres per day. At harvest time, harvesting 800 acres takes about 11 days, i.e., 70 acres per day. The average farm size in the United States is 441 acres~\cite{agcensus}. Similarly, a farm robot covers about 20-50 acres per day~\cite{laserweeds,airobotweeds}. Therefore,  for an average farm, farm activity, especially those involving equipment is limited to small parts of the farm on a given day. Moreover, this part of the farm shifts across time in a season.

\para{Analysis: }We quantify the cost benefits of horizontal motion by considering a large Midwestern farm area in United States as shown in Fig.~\ref{fig:large_farm}. This farm area measures $4.8 \times 4.8$ km, which would correspond to a large 5000-acre farm in United states. According to our measurement study, the maximum extent of coverage through crops during the peak season is roughly 1 km as shown in Fig.~\ref{fig:coverage_area} given a fixed antenna height of 5m (Fig.~\ref{fig:deployment}). Hence, to cover this farm throughout the season by replicating this setup, we would require doing so in a $3 \times 3$ grid-like arrangement as shown in Fig.~\ref{fig:large_farm}. As mentioned previously, this solution is over-provisioned as only a small number of fields within the farm will need high-bandwidth connectivity at any given moment. On the other hand, by horizontally moving \name, we can selectively choose to extend coverage to different areas of the farm as needed  (Fig.~\ref{fig:horizontal_motion}). Hence, by leveraging horizontally moving base stations, we can achieve drastic cost savings when compared to conventional static infrastructure.

Although horizontal motion allows us to move bubbles of connectivity, it does not allow us to optimize the connectivity within the bubble (i.e. deal with crop-induced attenuation). To achieve this, \name\ leverages vertical motion in addition to horizontal motion. We describe details of our variable-height base station in the following section (Sec.~\ref{sec:design}).
