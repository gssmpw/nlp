\section{Our Approach: Bring your Own Networks}
\label{sec:approach}

\begin{figure*}[!ht] %
  \centering
  \subfloat[Under-canopy robot with CBRS dongle.]{%
    \includegraphics[width=0.30\textwidth, valign=c]{figs/robot0_Large.jpeg} %
    \label{fig:robot}
  }\hfill
  \subfloat[In crops.]{%
    \includegraphics[width=0.17\textwidth, valign=c]{figs/robot_crop0_Large.jpeg} %
    \label{fig:crop-canopy}
  }\hfill
  \subfloat[Alternate view.]{%
    \includegraphics[width=0.17\textwidth, valign=c]{figs/robot_crop1_Large.jpeg} %
  }\hfill
  \subfloat[Plant phenotyping route (yellow). Crop rows in green.]{%
    \includegraphics[width=0.18\textwidth, valign=c]{figs/phenotyping_route.pdf} %
    \label{fig:phenotyping_route}
  }
  \vspace{-0.15in}
  \caption{\textbf{Under-canopy robots.} Robots are 40cm tall. In peak season, crop heights can reach 2m or more. }
  \label{fig:under-canopy}
  \vspace{-0.2in}
\end{figure*}


We propose a BYON~(Bring Your Own Network) connectivity model for digital agriculture applications. In BYON, farmers or farming communities do not need to sustain a permanent always-on network deployment that incurs large deployment costs. BYON relies on a portable deployment on a farm tractor or truck with a CBRS~(Citizens Broadband Radio Service) base stations with a satellite networking service (e.g., Starlink) as a backhaul. We choose CBRS because it allows users to deploy a private cellular network that offers mile-level range. CBRS spectrum has been recently opened up by regulators to target rural areas, where much of this spectrum is available. 80 MHz of this spectrum is always available to general users. This can go up to 150 MHz if priority users aren't using it at a given location~\cite{cbrs}. CBRS is compatible with off-the-shelf devices using a cellular SIM card. In devices without cellular capability, it can be attached using a cheap USB dongle. CBRS also offers a mile-level range.

We considered multiple choices for the backhaul -- fiber, cellular backhaul, satellite backhaul, and microwave backhaul. Fiber and cellular backhauls may not always be available in rural areas. Microwave backhauls require careful alignment every time the base station is moved. We decided to choose satellite backhauls because their performance has steadily improved and generally surpasses the requirements of our CBRS network. Finally, there has been recent push to incorporate satellite connectivity into tractors~\cite{deerespacex} which naturally aligns with the BYON model. We note that we cannot place satellite transceivers directly on end devices because  %
satellite receivers are typically bulky for drones and robots. More importantly, satellite transceivers use high frequencies which cannot penetrate cover-canopy for under-canopy applications.

\para{Target Applications:} Through this paper, we focus on video streaming from farm workers or farm equipment such as robots, tractors, or drones as the target application. Video is the highest bandwidth requirement for such equipment. Increasingly, such equipment is commonly used in agriculture. These applications regularly need to stream videos to a farm worker to seek feedback, decide on inputs (e.g., whether to kill a weed), and for help when they get stuck. Farm workers may also need to share images of what they observe on the field with other workers or the farm manager. Such applications can't be satisfied by low bandwidth technologies like LoRa. An example of such robots is shown in Fig.~\ref{fig:robot}. %

We note that there are some applications in digital agriculture like sensor-based monitoring which require permanent connectivity. However, such applications are usually low bandwidth and can be supported by low power wide area connectivity solutions (LPWANs) like LoRa or NB-IoT and are not the focus of our work. We instead focus on high-bandwidth dynamic applications like farm robots, drones, tractors, etc. 








