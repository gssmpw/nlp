\section{Experimental Evaluation}
We describe our evaluation of \name's model and height-variable base station below.

\subsection{Experimental Setup}\label{sec:eval_setup}
We need to measure variation of the CBRS signal for different crop heights, different values of distance, and different heights of the base station. Moving the base station requires heavy equipment like a tractor or a truck. Such equipment can only move along edges of the field and cannot provide extensive experiments. Therefore, we perform these experiments in a flipped manner. The client is placed on a drone (Fig.~\ref{fig:experimental_setup}(C)) that dynamically varies its height and distance. The base station is placed under the crops on a set of poles (Fig.~\ref{fig:experimental_setup}(B)). The base station's height can be adjusted to observe different effective values of $h_c$. For example, to reduce $h_c$, we can simply lift the base station slightly from the ground. The schematic is shown in Fig.~\ref{fig:experimental_setup}(A). Note that, this schematic is analogous to Fig.~\ref{fig:scenario}, except that the position of the client and base station has been flipped. Given reciprocity of wireless systems, both setups yield the same measurements. The drone is equipped with an altimeter and GPS module. The altimeter and the GPS module allow us to track the drone's 3D location during flight, allowing us to measure $h_{bs}$, $d$, and by extension $r$. 

\para{Limitations of our Experimental Setup: }We note two limitations of our experiments. First, due to the lower height of the base station, we are required by regulation to reduce the power on the base station to 1W (as opposed to 50W maximum power, i.e., 17 dB lower). Therefore, as opposed to the few Km range in Sec.~\ref{sec:measurements}, our range is reduced to hundreds of meters. In practice, \name\ will be deployed using commercially available computer-controlled height-adjustable telescoping antenna masts~\cite{willburt_mast_stilleto, willburt_mast_pneumatic, aluma_smarttower} on tractors and trucks (unfortunately, we do not have access to large farm equipment since farm tractors/trucks cost upwards of 50k USD) and transmit at the full power to support larger-range operations. 
Second, we collect signal strength measurements in our experiments, not throughput because throughput measurements react slowly to the motion of the drone and do not capture instantaneous effects. However, as we see in Fig.~\ref{fig:vs_signal}, throughput and signal strength are closely related.


\subsection{Data Collection}\label{sec:dataset}

We collect an extensive dataset for our through-crop measurements using the drone setup described above. This setup allows us to emulate varying crop heights by changing the height of the base station antennas with respect to the ground, while collecting data for multiple values of $h_{bs}$ and distances. 

\para{Drone Vertical Motion Dataset: }First, we fix a crop height $h_c \in \{0.0, 0.5, 1.0, 1.5\}$m by moving the antenna vertically along the sign post. Then, we place the UAV in the field at a distance $d > 10$m away from the base station. To collect data, we fly the UAV up to a height around 30m and then back down to land. Throughout the flight, we log the GPS coordinates, altitude, and RSRP values reported by the CBRS dongle at a rate of 2 Hz. We limit the vertical speed of the UAV to 1m/s to densely sample the RSRP variation with height. After each flight, we increase $d$ by 10 to 20 meters by finding another location further away and repeat the data collection. We repeat this process until the dongle is too far away from the base station to receive a signal or we exceed the boundaries of our allotted test field. The maximum value of $d$ depends on the extent of crop blockage $h_c$. For example, with $h_c=1.5$m the maximum $d$ is around 50m. Each flight consists of roughly 1 minute of capture time and 120 data points. Our dataset consists of data from 24 such flights. After removing invalid data points due to sensor errors, our dataset comprises 2216 usable data points in total.
\begin{figure*}
    \subfloat[CDF of test error across entire test dataset.]{
        \includegraphics[width=0.35\textwidth,valign=c]{figs/model_test_set_cdf.pdf}
          \vspace{-0.2in}
    } \hfill
    \subfloat[Predicted vs measured signal for a UAV test trace.]{
        \includegraphics[width=0.35\textwidth,valign=c]{figs/test_prediction.pdf}
          \vspace{-0.2in}
    } \hfill
    \subfloat[Test trace flight path. Darker implies higher RSRPs. ]{
        \includegraphics[width=0.17\textwidth,valign=c]{figs/test_flight_path.pdf}
          \vspace{-0.2in}
    }
    \vspace{-0.2in}
    \caption{\textbf{Evaluating the signal model.} Our model achieves $5.27$ dBm RMSE and $3.65$ dBm median absolute error. }
    \label{fig:signal_model_eval}
    \vspace{-0.2in}
\end{figure*}

\para{Drone Horizontal Motion Dataset: }We collect a new dataset in the following week. We move the drone horizontally for this set of experiments to sample densely in the horizontal plane for different drone heights. Specifically, we fix an altitude $h_{bs} \in \{5, 10, 15\}$m and have the UAV fly a zig-zag survey pattern (see Fig.~\ref{fig:signal_model_eval}(c)) over the field at this altitude. The zig-zag pattern is intended to mimic the path that a real under-crop canopy robot takes when surveying successive crop rows in a field (see Fig.~\ref{fig:phenotyping_route}). In addition to orienting the zig-zag pattern in a north-south direction (as pictured), we also orient it in an east-west direction to capture more spatial variation. We aggregate data points from 9 such flights, resulting in a dataset comprising 2656 usable data points. 

\subsection{Evaluation: Signal Model}\label{sec:eval_model}


We derive the optimal parameters for our signal model (Eqn.~\ref{eqn:signal-model}) given calibration data derived from the vertical motion dataset defined above. The resulting optimal parameters are shown in Table~\ref{tab:model_parameters}. Then, we evaluate our model's accuracy using the horizontal motion dataset described above. Note that the evaluation data was collected one week after the calibration data, so they have no overlap. The results of testing our model on this dataset are shown in Fig.~\ref{fig:signal_model_eval}. As shown, the model achieves a median error of 3.65 dBm. We show an example trajectory of the drone in Fig.~\ref{fig:signal_model_eval}(B-C). The example trajectory demonstrates that our model is able to capture the overall trend very well. There is minor temporal variation, likely due to multipath reflections and noise, that the model doesn't capture which leads to the error.

\begin{table}[h!]
\centering
\vspace{-0.1in}
\caption{Model parameters for peak-season corn.}
\vspace{-0.15in}
\begin{tabular}{cccc}
\toprule
$\bm{\alpha}$ & $\bm{\beta}$ & $\bm{\Gamma}$ & $\bm{G}$ \\
\midrule
 0.501 & 0.185 & 3.741 & -55.420 \\
\bottomrule
\end{tabular}
\label{tab:model_parameters}
\vspace{-0.15in}
\end{table}



\subsection{Evaluation: Predicting Ideal Height}
\begin{figure*}[!ht]
    \subfloat[Comparing RSRP CDF of fixed baseline vs \name.]{
        \includegraphics[width=0.31\linewidth,valign=c]{figs/fixed_vs_variable.pdf}
        \vspace{-0.1in}
        \label{fig:fixed_vs_variable}
    } \hfill
    \subfloat[\name's dynamic base station improves RSRP across our test field by an average of $4.80$ dBm.]{
        \includegraphics[width=0.31\linewidth,valign=c]{figs/fixed_vs_variable_delta.pdf}
        \vspace{-0.1in}
        \label{fig:fixed_vs_variable_delta}
    } \hfill
    \subfloat[Multi-client results for 2, 5, 10 devices.]{
        \includegraphics[width=.31\linewidth,valign=c]{figs/multiple_client.pdf}
        \vspace{-0.1in}
        \label{fig:multiple_client}
    }
    \vspace{-0.15in}
    \caption{\textbf{Comparing variable height vs fixed base station. } We compare the RSRP values achieved by the dynamic height base station vs those of fixed height base station at $h_{bs}=5$ m. }
\label{fig:height_eval}
  \vspace{-0.2in}
\end{figure*}

Next, we quantify the benefits of \name's dynamic base station versus a fixed height base station for clients in the field. To do this, we uniformly sample a $20 \times 20$ 2D grid of points in our test field. For ground truth, we use the data collected in Sec.~\ref{sec:dataset}. Specifically, for a given setting of $h_{bs}$, $r$, and $h_{c}$, we need to find the data point that corresponds to this setting and use its value as the ground truth measurement. We interpolate the measurements linearly to fill in missing values. 


For each location in our grid, we compare the RSRP (signal strength) obtained using a fixed base station height of $h_{bs}=5m$ and using \name's variable height base station. Our results are shown in Fig.~\ref{fig:fixed_vs_variable} and Fig.~\ref{fig:fixed_vs_variable_delta}.
The baseline achieves a median RSRP of -89.11dBm, our variable-height design achieves a median RSRP of -81.65dBm, which is an improvement of 7.46dBm. Using the signal strength-throughput information shown in Fig.~\ref{fig:vs_signal}, we can see that this improvement would translate to a 28.5\% higher downlink rate for the median device. From Fig.~\ref{fig:fixed_vs_variable_delta}, we also see that \name's average improvement across all devices is 4.8 dBm, with 10\% of the devices getting over 12 dBm signal strength improvement. This experiment demonstrates that \name\ can significantly improve throughput and signal quality for under-canopy applications.


\subsection{Evaluation: Multiple Clients}

Next, we quantify \name's gains under a realistic multiple client scenario. During deployment, several robots can be active at the same time working on different crop rows in parallel, e.g., along the different crop rows in Fig.~\ref{fig:phenotyping_route}. These crop rows are at different distances from the base station. In this scenario, \name\ has the choice of optimizing the throughput for some subset of clients or that of all clients simultaneously. We consider the scenario where \name\ pursues an unbiased policy i.e. optimizes average throughput to all clients. To simulate this scenario, we fix a set of $n \in \{2,5,10\}$ clients. We randomly sample a different crop row within our test field for each client and also randomize the placement of the client within the row. Then for these client locations, we compute the gains for the average RSRP from varying the base station height vs. fixing $h_{BS}=5m$. We repeat this process for $k = 100$ trials and plot the gains as a CDF. We show the results in Fig.~\ref{fig:multiple_client}. We can see that in the scenario of 2,5 and 10 clients, \name\ can achieve an average RSRP improvement of 4.54 dBm, 4.49 dBm, and 4.26 dBm respectively. While under all occasions \name\ was able to greatly improve the signal quality, we do see that such improvement slightly decreases as more devices are added to the field. This agrees with the intuition that when there are more devices, their optimal $h_{BS}$ would be different and it would be harder to optimize the signal quality for all of them.
