\section{Introduction}
\label{sec:intro}

Digital agriculture technologies, such as precision agriculture, agricultural robotics (AgBots), and the Internet of Things (IoT), can improve agricultural outcomes by tens of billions of dollars annually in the United States \cite{usda_report, fcc_report, mckinsey_agriculture_2020} and around the globe \cite{mehrabi_global_2021, 30_percent_no_internet, govuk_ruraldrive_2023}. These technologies are also key to improving sustainability outcomes in agriculture. For example, agricultural robots can plant cover crops before harvest and prevent soil erosion, reduce fertilizer use, and replace chemical herbicides with mechanical weeding \cite{digital_transformation_madhu, farmprogress_robot, mckinsey_agriculture_2020}. In practice, the adoption of such technologies is severely limited by the lack of broadband connectivity \cite{30_percent_no_internet, usda_report, comparing_on-farm_connectivity, mehrabi_global_2021}.

High-bandwidth connectivity is essential to stream data from farm equipment, to teleoperate robots, and to operate data-intensive farm equipment like autonomous tractors and drones~\cite{johndeere,johndeere2,mckinsey_agriculture_2020}. However, existing connectivity solutions fail to meet these needs. Wi-Fi operates in short ranges from fifty to hundred meters, thereby failing to cover large farmlands spanning several Kilometers. Narrowband connectivity solutions like LoRa enable few Kbps of connectivity for sensors, but cannot support cameras, robots, or tractors. Commercial cellular operators often don't cover farm areas due to their sparse population and high deployment costs~\cite{usb_report, 30_percent_no_internet}. Different estimates argue that 30-50\% of the farmers in the US have limited or no access to broadband connectivity on their farms~\cite{usb_report, 30_percent_no_internet}. Therefore, there is a pressing need for high-bandwidth, low-cost, and long-range connectivity solutions for digital agriculture applications. %



\begin{figure}[t]
    \centering
    \includegraphics[width=0.65\linewidth]{figs/intro-v2.pdf}
    \vspace{-0.1in}
    \caption{BYON gateway consisting of CBRS base station, telescoping antenna mast, and satellite terminal.}%
    \label{fig:byon}
    \vspace{-0.20in}
\end{figure}

In this paper, we argue that the goal of providing always-on connectivity that connects all of the farmland is an over-estimation of the farm connectivity problem. Instead, we observe that the demand for agricultural connectivity is restricted in time and space, i.e., agricultural equipment typically needs coverage in small parts of the farm and this requirement varies with time much like the seasonal rhythm of agricultural activity. Agricultural equipment (increasingly autonomous and video-based) sequentially plants crops at different locations in a farm across few days. Agricultural robots for mechanical weeding operate in a small window before and after that. Other robots plant cover crops in a short window of time before crop harvest. Therefore, such applications do not require permanent and full-coverage high-bandwidth connectivity. In fact, efforts to provide such connectivity, e.g., using cellular networks with dense backhauls, end up being cost-prohibitive and severely over-provisioned.



As opposed to permanent and full-coverage solutions, we seek to enable a new networking model where the farm network moves across space and time, along with the applications themselves. Specifically, \textit{we propose BYON (Bring Your Own Network) -- a new connectivity model} where high-bandwidth applications bring their own network to different parts of the farm as needed. %
The design of BYON leverages two emerging technologies. (a) We use Citizens Broadband Radio Service (CBRS)~\cite{cbrs}, which promises to enable citizen-deployed private cellular networks using shared spectrum, to provide last-mile connectivity across various edge devices. (b) BYON leverages recent satellite connectivity solutions \cite{starlink} to provide a globally accessible backhaul to the Internet. 

BYON offers multiple advantages: (a) BYON solutions are  portable and mobile. For example, it can be deployed on a tractor (see Fig.~\ref{fig:byon}) that parks at the edge of the field being worked on in a given day and moves to different fields in a farm over time. (b) BYON is compatible with off-the-shelf cellular devices, i.e., it simply requires a new SIM card to be installed in cellular devices. A USB dongle can provide cellular capabilities to devices like robots. So, BYON can provide connectivity to both farm devices and farm workers who work the field. (c) CBRS offers long range connectivity as opposed to Wi-Fi and can support operations over a large part of the farm without requiring additional movement. (d) BYON is highly configurable and can meet the need for both over-canopy and under-canopy applications.

In this paper, we focus on three key aspects of BYON:

\para{Profiling CBRS Connectivity: } CBRS is an emerging form of private cellular networks, wherein anyone can utilize the shared spectrum (up to 150 MHz) recently opened by FCC to deploy citizen-driven cellular networks~\cite{cbrs}. This is particularly targeted towards rural areas, where existing cellular networks do not have sufficient coverage. We deploy an off-the-shelf CBRS network on a production agricultural farm and profile its performance. In a 20 MHz band, we find that CBRS achieves downlink bandwidth up to 100 Mbps, and uplink bandwidths up to 20 Mbps across a distance of up to 4 Kilometers. However, due to its high frequencies around 3.5 GHz, such links suffer additional attenuation up to 30 dB for under-canopy applications. The additional attenuation leads to significantly reduced datarates under crop-canopy. %

\para{Variable Height Base Station Deployments: }As mentioned above, providing connectivity to sensors and robots covered by crops is a significant challenge. For digital agriculture applications, under-canopy robots and sensors play a crucial role. However, unlike traditional cellular infrastructure, BYON serves a small set of applications and can be dynamically configured to meet the needs of these applications. We propose a new base station design that can adapt its own height  based on crop-levels and application demand. 

In free space, the signal quality between the base station and a client device depends on the distance, $d$, between them (typically as $\frac{1}{d^2}$). However, as crops grow, the signal quality also degrades with the distance travelled through crops. Unlike free space, the attenuation  through crops is exponential with distance $e^{-\alpha d}$, where $\alpha$ depends on the electrical permittivity of the crops. This attenuation leads to an interesting tradeoff. If we set the base station height too low, the signal from the base station to the client travels a near-horizontal path which minimizes the distance and hence, the free-space attenuation. However, it travels a large distance through crops and maximizes the exponential attenuation through crops. On the other hand, at higher heights, the distance between the base station and the client increases, but the distance through crops decreases. We build a model to balance these competing factors and compute the optimal height. Our model computes the ideal height based on the base station location, throughput requirements, and crop heights. The height of the base station can be varied dynamically using computer-controlled telescoping antenna masts \cite{aluma_smarttower, willburt_mast_stilleto}, and can be controlled to serve different applications or different locations at different times of the day.




\para{Application Analysis -- Under-Canopy Robot Teleoperation: }We demonstrate the benefits of our connectivity solutions in a robot teleoperation application. Under-canopy robots are increasingly used in agriculture for applications like plant phenotyping~\cite{phenotyping1,sivakumar_learned_2021,manish_agbug_2021,kim_p-agbot_2022}, cover crop planting~\cite{icover_usda, farmprogress_robot, du_deep-cnn_2022} and mechanical weeding \cite{naio_oz, mcallister_agbots_2020, reiser_development_2019}. They hold major advantages over robots featuring elements extending above the crop canopy (e.g. \cite{mineral_rover,xiang_fieldbased_2023,xu_modular_2022}) as they (a) do not make contact with the canopy while following crop rows, which damages crop leaves and contributes unwanted drag forces onto the robot, and (b) can easily access key parts of the crop underneath the canopy including their stems and the soil, where weeds, pests and disease are likely to reside. %
In practice, the lack of connectivity due to crop canopy blockage is a key bottleneck for widespread adoption of these robots. For example, state-of-the-art under-canopy robots require manual intervention when they get stuck~\cite{sivakumar_learned_2021, velasquez_multi-sensor_2022,gasparino_cropnav_2023}. Due to the lack of connectivity, a human manually needs to walk to these robots in the field and maneuver them when they get stuck and cannot decide. Second, such robots cannot exchange information with other robots when they work in groups. We demonstrate long range teleoperation for such robots, which would not be possible with existing techniques like Wi-Fi or TV White Spaces. In a BYON setup, a human sees a video feed while being away from the robot when the robot gets stuck and can remotely teleoperate the robot through thick crop canopy cover without requiring physical intervention.

We perform measurements over farmland at distances of upto 3.6 Km. We deployed our solutions and evaluated them on a production farm. Our experiments establish the feasibility of CBRS for high-bandwidth agricultural applications, but demonstrate the challenges related to under-canopy coverage. We demonstrate that our height-variable base station design can increase median signal quality by 7.5 dBm and median throughput by 28\% for under-canopy applications. Finally, we demonstrate BYON's feasibility for under-canopy teleoperation. A demo video is available at \href{https://byon-v1.github.io/}{\color{blue}https://byon-v1.github.io/}.

Our work is novel in the underlying technologies (CBRS and satellite networks), new system design, and the application. Specifically, we make the following contributions:
\squishlist
    \item We present the first analysis of a CBRS network in a digital agriculture setup and quantify the impact of crops on the network performance. 
    \item We propose an agile variable height base station design for optimizing under-canopy coverage.
    \item We demonstrate the first teleoperation operation for an under-canopy operation over CBRS.
\squishend
