\section{Variable-Height Base Station}
\label{sec:design}

\begin{figure}[!t]
    \centering
    \includegraphics[width=.9\linewidth]{figs/propagation_model.pdf}
    \vspace{-0.15in}
    \caption{Client under crop canopy of height $h_c$ at horizontal distance $d$ from base station at height $h_{bs}$.}
    \label{fig:scenario}
    \vspace{-0.15in}
\end{figure}

In this section, we aim to optimize throughput for under-canopy applications by designing a variable height base station. Note this does not incur additional hardware cost: standard vehicle-mounted antenna masts (e.g. on CoWs) are already telescoping as they need to be stowed during transit or storage and quickly retractable during high wind conditions \cite{aluma_smarttower}. We discuss the intuition and the formulation for the variable-height base station below.

\subsection{Exploiting Height Variability} 
First, we explain the advantage of dynamic height variability. Consider the basic scenario in Fig.~\ref{fig:scenario}. Suppose there are no crops i.e. $h_c=0$. Then the optimal base station height should be close to the ground i.e. $h_{bs} \approx 0$ as it minimizes the physical distance $r$ between the base station and the client. Specifically, 
\begin{equation}
r = \sqrt{d^2 + h_{bs}^2}
\end{equation}

On the other hand, suppose there are crops i.e. $h_c > 0$. If we set $h_{bs} \approx 0$, i.e., very close to the ground, the direct path to the client will be horizontal and cover a large distance $r\approx d$ through the crops. This will lead to large attenuation for the signal and is sub-optimal.

To counter this problem, one may be inclined to increase $h_{bs}$. When $h_{bs}$ is very large, the distance travelled through crops gets increasingly closer to $h_c$, which is the minimum distance the signal can cover through crops. Therefore, increasing the height of the base station has the advantage of reducing through-crop attenuation.

However, as we increase $h_{bs}$, two other effects occur. First, the total propagation distance $r$ increases. Second, the angular deviation from the base station antenna's main lobe to the client, denoted by $\theta$, increases. $\theta$ is related to $h_{bs}$ and $d$  by \vspace{-0.20in}
\begin{equation}
\theta = \arctan\biggl(\frac{h_{bs}}{d}\biggr)
\end{equation}%
Both factors negatively affect received power at the client. In other words, there is tradeoff regarding $h_{bs}$. If it is too low, the connection will suffer from severe crop blockage. If it is too high, the connection will suffer from path loss and antenna-related losses i.e. $r$ and $\theta$ are too large. Hence, in the presence of crops, there should be an optimal base station height which balances these two competing factors.

\para{Why Not Raise the Client Antenna?} It is tempting to modify the under-canopy client to have antennas extending above the crop canopy. While the robot can travel through the space at the crop base, the canopy is dense as shown in Fig.~\ref{fig:crop-canopy}. Therefore, an antenna extending above the crop canopy will drive/cut through the crops, causing damage to the crops and contributing large drag forces onto the robot.

\begin{figure*}[!t] %
  \centering
  \subfloat[Schematic. Note the correspondence with Fig.~\ref{fig:scenario}.]{%
    \includegraphics[width=0.4\textwidth]{figs/experimental_setup.pdf} %
    \vspace{-0.3in}
  }%
  \hfill
  \subfloat[Antenna post used to vary $h_c$.]{%
    \includegraphics[width=0.28\textwidth]{figs/base_station_3_Large.jpeg} %
    \vspace{-0.3in}
  }%
  \hfill
  \subfloat[UAV with CBRS dongle used to vary $h_{\text{bs}}$.]{%
    \includegraphics[width=0.28\textwidth]{figs/drone2_Large.jpeg} %
    \vspace{-0.3in}
  }
  \vspace{-0.1in}
  \caption{\textbf{Experimental Setup.} We swap the positions of the base station and client using channel reciprocity. Our client mounted on a UAV acts as a surrogate base station, while our base station acts as a surrogate client.}
  \label{fig:experimental_setup}
\vspace{-0.25in}
\end{figure*}

\begin{figure*}[!t] %
  \centering
  \subfloat[Without crops (i.e. $h_c=0$), the signal strength monotonically decreases with height.]{%
    \includegraphics[width=0.32\textwidth]{figs/altitude_vs_rsrp_0.0m_39m.pdf} %
    \vspace{-0.3in}
  }%
  \hfill
  \subfloat[With crops (e.g. $h_c=1.0$m), the signal strength increases to an optimum, then decreases.]{%
    \includegraphics[width=0.32\textwidth]{figs/altitude_vs_rsrp_1.0m_39m.pdf} %
    \vspace{-0.3in}
  } 
  \hfill
  \subfloat[Optimal height increases with client distance $d$.]{%
    \includegraphics[width=0.32\textwidth]{figs/altitude_vs_rsrp_1.0m_74m.pdf} %
    \vspace{-0.3in}
  }
  \vspace{-0.1in}
  \caption{\textbf{Intuition for the optimal height.} We place a client at a horizontal distance $d$ from the base station and vary the height of the base station $h_{\text{bs}}$ by flying the drone vertically up and down (see Fig.~\ref{fig:experimental_setup}). We scatter plot the observed RSRP at the client vs the base station height. The best-fit line is shown in grey. The optimal heights are marked as dotted red lines.}
  \label{fig:intuition}
\vspace{-0.2in}
\end{figure*}

\subsection{Experimental Validation}


To validate these ideas, we perform an experiment using the setup in Sec.~\ref{sec:eval_setup} and  Fig.~\ref{fig:experimental_setup}. We emulate a fixed client at $d=39$ m and vary the value of $h_{bs}$ for different crop heights, $h_c$. We measure the RSRP values on the client and plot the results in Fig.~\ref{fig:intuition}(a)(b). The experiment supports our intuition: (a) The RSRP degrades continuously with distance in the absence of crops, i.e., the optimal $h_{bs} \approx 0$ without crops. (b) In the presence of crops, the signal is weak when $h_{bs}$ is low and gets stronger with increasing heights. However, beyond an optimal value of $h_{bs} > 0$, the signal gets weaker again. Note that changing the height of the base station can significantly impact the RSRP of the received signal (up to 15 dBm variation as shown in Fig.~\ref{fig:experimental_setup}(b)). In Fig.~\ref{fig:vs_signal}, we show how such RSRP variations map to throughput variations. %

Finally, we consider what happens when varying the client distance $d$. Note that with increasingly large $d$, the influence of $h_{bs}$ on both $r$ and $\theta$ becomes increasingly marginal. Hence, we are incentivized to raise the base station height further as this does not incur much of an increase on $r$ and $\theta$. To summarize, increasing client distances favor increasing base station heights. This is validated by our experiment in Fig.~\ref{fig:intuition}(c). As our clients are mobile devices, this further substantiates the need for a dynamic height-varying base station.


\subsection{Modeling the Dynamic Base Station} 

Given the performance variation due to height of the base station, we need to define a mechanism to identify the optimal height of the base station. One option is to have the base station probe different heights. In practice, such a system is bound to be slow in reacting to the robot motion under crops as it involves mechanical motion. Our approach is to derive an explicit physics-based model that predicts the client RSRP given a base station configuration. Once we have this model, we can use it to directly predict the optimal height of the base station, given crop height.


Our model incorporates 3 major factors that account for the variation of the CBRS signal strength: 


\para{(i) Path Loss:} We know that the signal becomes weaker when the receiver is further away from the base station. In vacuum and air, the energy of the signal $E$ and the distance between the sender and receiver $r$ generally follows $E\propto r^{-2}$~\cite{electrodynamics_textbook}. Therefore, the path loss is%
\begin{equation}
    \label{eqn:pass-loss}
    L_P=-20\log_{10}r%
\end{equation}
\para{(ii) Crop Attenuation:} When the wireless signal travels through crops, it suffers additional attenuation because part of the signal is deflected and absorbed by the crops. When the signal is not excessively strong, the absorption can be regarded as linear, i.e. the energy absorption is proportional to the current energy%
\begin{equation}
    \label{eqn:energy-absorption}
    \frac{dE}{dr}=-\alpha E%
\end{equation}
The energy will therefore experience an exponential decay, i.e. $E\propto e^{-\alpha r_c}$ where $r_c$ is the distance in crops. Therefore, the loss due to crop absorption is%
\begin{equation}
    \label{eqn:absorption-loss}
    L_{A}=-\alpha r_c%
\end{equation}
$\alpha$ is an absorption coefficient that depends on the crop type, and density. Using trigonometry, we have: $r_c = \frac{h_c}{h_{bs}} r$. 

\para{(iii) Antenna Directivity:} The antenna of the base station are designed to be directional, i.e. the radiation energy does not distribute evenly on all angles. In fact, the CBRS base station we are using in the evaluation has a radiation pattern similar to the one depicted in Figure~\ref{fig:scenario}. It has a relatively flat radiation pattern in the azimuthal dimension, but a constrained radiation pattern in the elevation dimension. Therefore, the signal strength will vary based on the relative angle of elevation ($\theta$) between the base station and the device. Generally, base stations use an antenna array to perform beam forming~\cite{shepard2012argos} to concentrate its energy on a narrow beam. This would yield a beam pattern of the $\sinc$ function. We thus model the loss~(gain) due to angle as%
\begin{equation}
    \label{eqn:angle-loss}
    L_{D}=10\log_{10}\left(\max\left(|\sinc\left(\Gamma\theta\right)|,\beta\right)\right)%
\end{equation}
which is a $\sinc$ function capped at a minimum value $\log\beta$ for numerical reasons.


Combining the factors, the total signal strength~(RSRP) is%
\begin{equation}
    P_{\alpha, \beta, \Gamma, G}(r, \theta, r_c) = L_{\alpha, \beta, \Gamma}(r, \theta, r_c) + G%
\end{equation}
where $G$ is the lumped gain~(constant) that encapsulates all the gains in system (i.e. from the amplifier and the antenna). Moreover, $L_{\alpha, \beta, \Gamma}(r, \theta, r_c)$ represents all the losses in the system and is defined as:%
\begin{equation}
    \label{eqn:signal-model}
    L_{\alpha, \beta, \Gamma}(r, \theta, r_c) = \underbrace{-20 \log_{10} r}_{\text{Path Loss}} \underbrace{- \alpha r_c}_{\text{Crop Attenuation}} + \underbrace{10 \log_{10} \Phi_{\beta,\Gamma}(\theta)}_{\text{Directivity}}\vspace{-0.01in}
\end{equation}
Finally, $\Phi_{\beta,\Gamma}(\theta)$ is the unit-normalized antenna radiation pattern defined as:%
\begin{equation}
    \Phi_{\beta,\Gamma}(\theta) = \max(|\sinc(\Gamma\theta)|, \beta)%
\end{equation}

Note that, our model does not capture multipath effects. This is because farm areas are open spaces and typically do not have large reflectors. We empirically validate our model in Sec.~\ref{sec:eval_model}.

\para{Estimating Model Parameters: } The power loss model defined above depends on four parameters:  $\alpha,\beta,\Gamma,G$. These are physical parameters that represent properties of the hardware and the physical environment. They can be estimated using a small number of measurements. For example, a robot can do a quick maneuver and report observed signal strength measurements along with its own positions for us to estimate these parameters. Alternatively, getting a signal from a set of already deployed sensors will provide enough data to estimate these parameters. Given a small amount of calibration data, we formulate this as a constrained non-linear least squares problem. Our objective function is to minimize the square of the error residual and our constraints are that $\alpha, \beta, \Gamma \geq 0$. We leverage trust region methods from a well-tested numerical library \cite{2020SciPy-NMeth} to perform this task. %

\para{Identifying Ideal Height of the Base Station: }We begin by considering a single target device operating in a farm, e.g., a farm robot. In this case, our model can predict the signal strength at the location of the robot for any given height. Therefore, to choose the optimal height of the base station, we simply need to identify a height that maximizes the required signal strength. To obtain the horizontal distance of the robot, the robot can send its estimated location to the base station. A robot can use GPS to self-localize whenever it is outside a crop row (i.e. when travelling between rows in Fig.~\ref{fig:phenotyping_route}). It can then continuously update its own location estimate when entering a crop row by leveraging on-board sensor readings (e.g. cameras, lidars, IMUs) \cite{sivakumar_learned_2021, velasquez_multi-sensor_2022, gasparino_cropnav_2023}.

For multiple devices with different requirements, \name\ can optimize the aggregate throughput by predicting throughput at different locations in the farm for each height. It can also optimize for other metrics of interest, e.g., throughput for a subset of devices, etc. 
