\section{Case Study: Robot Teleoperation}
\label{sec:application}

\begin{figure*}[!ht] %
  \centering
  \subfloat[Teleoperation has asymmetric uplink and downlink requirements.]{%
    \includegraphics[width=0.22\textwidth, valign=c]{figs/teleop.pdf} %
    \vspace{-0.2in}
    \label{fig:teleop}
  }
  \subfloat[Video data rate at each compression level.]{%
    \includegraphics[width=0.26\textwidth, valign=c]{figs/teleop_bw.pdf} %
    \vspace{-0.2in}
    \label{fig:teleop_bw}
  }
  \hfill
  \subfloat[Video latency at each compression level.]{%
    \includegraphics[width=0.26\textwidth, valign=c]{figs/teleop_delay.pdf} %
    \vspace{-0.2in}
    \label{fig:teleop_delay}
  }
  \subfloat[Frame drop rate versus distance.]{%
    \includegraphics[width=0.22\textwidth, valign=c]{figs/frame_drop_alt.png}
    \vspace{-0.2in}
    \label{fig:teleop_range}
  }
  \vspace{-0.15in}
  \caption{\textbf{Robot teleoperation.} The operator obtains a video feed from the stuck robot and issues corrective actions. }
  \label{fig:teleoperation}
  \vspace{-0.2in}
\end{figure*}


In this section, we explore the use of BYON in teloperating under-canopy robots. As outlined previously, these small robots fit underneath the crop canopy and are capable of autonomous row following to perform tasks such as plant phenotyping~\cite{phenotyping1,sivakumar_learned_2021,manish_agbug_2021,kim_p-agbot_2022}, cover crop planting~\cite{icover_usda, farmprogress_robot, du_deep-cnn_2022}, and mechanical weeding~\cite{ mcallister_agbots_2020, naio_oz, reiser_development_2019}. However, even with state-of-the-art row-following systems \cite{sivakumar_learned_2021,velasquez_multi-sensor_2022,gasparino_cropnav_2023}, such robots require frequent intervention (e.g., once per 250 m~\cite{sivakumar_learned_2021}). When a robot gets stuck, a human operator needs to get to the robot and manually maneuver it, e.g., using a controller operating over WiFi link to the robot. Therefore, successful operation of these robots is labor-intensive. 


We examine the use of BYON to solve this problem. With BYON, the robots and the operator are connected to the same CBRS network. This allows an operator to manage a fleet of robots over long distances from a terminal connected to the base station (see Fig.~\ref{fig:teleop}). When a robot requires intervention, the operator can teleoperate the robot from their tractor or office. This involves remotely issuing actions to the robot while monitoring its video feed.

\para{Throughput Requirements:}
First, we analyze throughput requirements for teleoperation. As depicted in Fig.~\ref{fig:teleop}, we note that teleoperation has asymmetric uplink and downlink throughput requirements. We require a high throughput uplink in order to stream video from the robot. By contrast, the downlink consists of small command messages from the teleoperator. Hence, we are bottle-necked by the limited uplink of the CBRS network. We explore several means of reducing the data rate of the video on the robot. We explore the effect of varying color channel, resolution, and compression levels. Our results are shown in Fig.~\ref{fig:teleop_bw}. The greatest effect stems from resolution and compression levels. We find the network supports 640x480 video under most compression levels.

\para{Latency Requirements:} Next, we analyze latency requirements for teleoperation. Teleoperation favors lower latency because it is easier when feedback is more immediate. To find the relationship between compression levels and latency, we conduct an experiment by varying video compression levels on a uplink that averages 12.4 Mbps. Our results are shown in Fig.~\ref{fig:teleop_delay}. We find increasing compression levels decreases latency, with delay and jitter increasing sharply below a compression threshold. 

\para{Choosing Compression Level:} The preceding experiments suggest we should increase the video compression as much as possible to decrease data rate and latency. However, increasing the compression limits the ability of the operator to understand the video. Hence, there is a tradeoff, and we are interested in finding the minimum video quality that an operator needs to effectively teleoperate a robot. To do this, we have a human attempt to drive the robot through crop rows at various compression levels. We find that human operators can tolerate compression levels as high as 90\%.

\para{Limits of Teleoperability: } Next, we ascertain the range limits of teleoperation. We fix the video compression level to 90\% (3.6 Mbps), adjust the base station (Fig.~\ref{fig:base_station}) power to normal levels (50W) and attempt to teleoperate the robot at distances of up to 2.5 km away (Fig.~\ref{fig:teleop_range}). We observe that as the distance increases, the rate of frame drops in the video increases. Empirically, we find drop rates of up to 30\% to be tolerable. In summary, we are able to teleoperate robots up to 2.0 km away without crops, up to 700 m with early season crops, and up to 200 m with peak season crops. Note that these are all in excess of Wi-Fi's range in open spaces ($\sim 100$ m). Finally, we note that some techniques (e.g. delay compensation \cite{chakraborty_towards_2024}) could further increase the range; we delegate such investigations to future work.

\para{Scalability: } How well does \name\ scale to multiple robots? Our application is bottlenecked by the uplink to the base station \cite{celona_ap11}, which supports up to 50 Mbps of uplink capacity from all clients. If we assume a conservative 20\% JPEG quality video stream consuming 4.2 Mbps (refer to Fig.~\ref{fig:teleop_bw}), we can support up to 10 robots concurrently, depending on their relative location. %
