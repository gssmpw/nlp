\section{Measurement Study}
\label{sec:measurements}

\begin{figure}[ht!] %
  \centering
  \subfloat[Cell tower.]{%
    \includegraphics[width=0.21\linewidth]{figs/cbrs_1.jpg} 
    \label{fig:base_station}
  }
  \subfloat[Overlooking crops.]{%
    \includegraphics[width=0.375\linewidth]{figs/cbrs_0.jpg} 
  }
  \subfloat[BYON deployment on a movable shipping container.]{%
    \includegraphics[width=0.375\linewidth]{figs/byontainer_0.jpg} 
  }
  \vspace{-0.15in}
  \caption{\textbf{CBRS deployments on the farm.} (A,B) Conventional deployment on static infrastructure. (C) BYON is a containerized solution that can be moved on demand. } 
  \label{fig:deployment}
  \vspace{-0.25in}
\end{figure}

In this section, we characterize the performance of a production-grade CBRS network and a satellite terminal deployed in a corn-field. We present an overview of the range, throughput, and latency that client devices on the farm experience under real world conditions.

\para{Deployment:} Our CBRS network deployment is pictured in Fig.~\ref{fig:deployment}. We mount a Celona AP11 Outdoor Access Point (AP) \cite{celona_ap11} onto a small  tower. The AP is connected to two 90-degree sectorized slant dual-polarization antennas (CN-ANT-90D \cite{celona_antenna}). The access point is set to operate in the CBRS bands in United States around 3.5 GHz, with a 20 MHz bandwidth. The antennas are mounted onto the tower at a height of 5m and directed to point southward. For maximum coverage, we use the maximum transmit power setting of 50W. We connect clients to the CBRS network using the Multi-Tech microCell CBRS dongle \cite{multitech_multiconnect} pictured in Fig.~\ref{fig:robot}. The dongle is lightweight and connects to robots, drones, sensors, etc. via USB.

For satellite links, we deploy a Starlink RV satellite terminal. We choose this terminal because it can be moved to different locations. The terminal weighs 4.2 Kilograms and cannot be carried by our under-canopy robots or drones, highlighting the need for CBRS as the on-farm link.

\para{Coverage Area:} First, we ascertain the maximum range of the CBRS network. We move a CBRS dongle away from the base station in several directions and record the locations where the signal gets lost. All experiments for coverage are done above the crop canopy to ascertain the maximum range of the system. This allows us to determine the boundaries of the network. The results are shown in Fig.~\ref{fig:coverage}. We find that clients can successfully connect to the base station at a distance of at most 3.6 Km away. Note that the area corresponds to over 2000 acres. Therefore, without crops, a CBRS base station can cover large parts of the farm without movement. %

\para{Connectivity Without Crops:} Next, we sample eight locations at varying distances within the coverage area that have a direct line of sight with the base station. In these experiments, the client device is placed above the crop canopy to avoid any crop-related losses. At these locations, we measure the throughput, latency, and reference signal received power (RSRP) reported by the dongle. The results are shown in Fig.~\ref{fig:vs_distance}. The signal strength degrades with distance as expected. The overall variation is around 40 dB across 3.6 Km. The downlink and uplink speeds degrade accordingly. We observe irregular patterns at a subset of the locations. We suspect that this is due to multipath reflections. Finally, note that the downlink speeds are higher than uplink speeds although they follow similar trends. This is largely due to power asymmetry between the base station and clients i.e. the base station transmits much higher power than the clients.



\begin{figure*}[!t] %
    \centering
    \subfloat[]{%
        \includegraphics[width=0.245\textwidth]{figs/outside_download_vs_dist.pdf} %
        \vspace{-0.25in}
    }%
    \subfloat[]{%
        \includegraphics[width=0.245\textwidth]{figs/outside_upload_vs_dist.pdf} %
       \vspace{-0.25in}
    }%
    \subfloat[]{%
        \includegraphics[width=0.245\textwidth]{figs/outside_latency_vs_dist.pdf} %
        \vspace{-0.25in}
    }%
    \subfloat[]{%
        \includegraphics[width=0.245\textwidth]{figs/outside_rsrp_vs_dist.pdf} %
        \vspace{-0.25in}
    }%
    \vspace{-0.15in}
\caption{Profiling the relationship between downlink, uplink, latency, and RSRP versus distance \textbf{without} crops.}

\label{fig:vs_distance}
    \vspace{-0.25in}
\end{figure*}

\begin{figure*}[!t] %
    \centering
    \subfloat[]{%
        \includegraphics[width=0.24\textwidth]{figs/inside_download_vs_dist.pdf} %
        \vspace{-0.15in}
    }%
    \hfill
    \subfloat[]{%
        \includegraphics[width=0.24\textwidth]{figs/inside_upload_vs_dist.pdf} %
        \vspace{-0.15in}
    }%
    \hfill
    \subfloat[]{%
        \includegraphics[width=0.24\textwidth]{figs/inside_latency_vs_dist.pdf} %
        \vspace{-0.15in}
    }%
    \hfill
    \subfloat[]{%
        \includegraphics[width=0.24\textwidth]{figs/inside_rsrp_vs_dist.pdf} %
        \vspace{-0.15in}
    }%
\vspace{-0.15in}
\caption{Profiling the relationship between downlink, uplink, latency, and RSRP versus distance \textbf{through} crops.}
\vspace{-0.2in}
\label{fig:vs_distance_crops}
\end{figure*}

\begin{figure*}[!t] %
    \centering
    \subfloat[]{%
        \includegraphics[width=0.32\textwidth]{figs/download_vs_signal.pdf} %
        \vspace{-0.15in}
    }%
    \hfill
    \subfloat[]{%
        \includegraphics[width=0.32\textwidth]{figs/upload_vs_signal.pdf} %
        \vspace{-0.15in}
    }%
    \hfill
    \subfloat[]{%
        \includegraphics[width=0.32\textwidth]{figs/latency_vs_signal.pdf} %
        \vspace{-0.15in}
    }%

\vspace{-0.15in}
\caption{Profiling the relationship between the downlink, uplink, and latency versus RSRP.}
\label{fig:vs_signal}
\vspace{-0.15in}
\end{figure*}

\para{Connectivity Through Crops:} We repeat the previous experiment with under-canopy client devices. We perform this experiment twice in the season, once with light crop cover and another later in the season with dense crop cover and plot this in Fig.~\ref{fig:vs_distance_crops}. We make three observations:
\squishlist
\item\textbf{Crops Degrade Throughput: } Under-canopy throughput, both uplink and downlink, is significantly lower than over-canopy throughput. This is because crops degrade CBRS signals. Free-space attenuation depends on the distance, $d$, between the sender and receiver and varies as $\frac{1}{d^2}$. Crops cause exponential attenuation due to their electrical permittivity and lead to the degraded throughput.

\item\textbf{Reduced Coverage: }As a corollary of the above, under-canopy coverage is reduced to around 0.2 Km and 0.75 Km in dense and light crop cover respectively. Despite the low coverage, it should be sufficient to only require infrequent movement of BYON. For example, a 0.2 Km coverage range corresponds to nearly ten acres and 0.75Km coverage corresponds to  nearly 130 acres. As we discuss in Sec.~\ref{sec:approach}, these measurements imply that a single CBRS base station is insufficient to cover a large farm. However, a BYON setup can be moved to different parts of the farm every few hours as the farming activity shifts.

\item\textbf{Seasonal Variation: }As the crops get denser, the signal obstruction due to them increases. We also expect the crop-induced variation to change with rainfall, irrigation, etc. due to variation in moisture content. 
\squishend

\para{Connectivity and RSRP:} Finally, we study the relationship between the RSRP values reported by our cellular dongle and the overall connectivity. We aggregate data points by randomly sampling points within the coverage area and plot the relationship of throughput and latency vs RSRP. The results are shown in Fig.~\ref{fig:vs_signal}. Overall, RSRP values are a good predictor of the overall connection quality. Downlink and uplink throughputs are positively correlated with RSRP and have diminishing returns as RSRP gets lower. Note that RSRP is a better predictor of downlink throughput than uplink throughput as RSRP is only measured by the client on the downlink channel. Both latency and jitter tends to decrease with increasing RSRP.

\para{Satellite Link: }We measure the uplink and downlink throughput for our Starlink terminal in outdoor setting with clear access to the sky. We measure an uplink throughput up to 50-60 Mbps, and a downlink throughput up to 200 Mbps, with temporal variations caused by satellite orbits and weather. Both uplink and downlink measurements supersede the CBRS throughput. This indicates that Starlink is a capable backhaul for the CBRS base station. In the rest of the paper, we focus on CBRS as the bottleneck link.

Finally, we place the satellite terminal in a location covered by crops. The terminal cannot connect to the satellite because satellite-ground link use high frequencies (over 10 GHz) which are easily blocked by crops. This further validates the intuition that we cannot directly equip under-canopy devices like robots with satellite links. We need a CBRS base station to provide on-farm connectivity.


