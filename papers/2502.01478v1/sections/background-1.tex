\section{Background and Related Work}\label{sec:background}

Digital agriculture spans a range of applications aimed at enhancing productivity, reducing input costs for farmers, and reducing environmental harms. Increasingly, such techniques rely on a combination of sensors, drones, robots, and farm equipment~\cite{farmbeats}. For example, agriculture has relied on herbicide-based weeding for a long time due to the efficiency of herbicide application. However, herbicide application leads to chemically-resistant weeds and have other well-documented side-effects. Recent research develops small robots that can identify and remove weeds mechanically \cite{weeding_bots, weeding_bots2, naio_oz, reiser_development_2019}. Such robots reduce chemical use, and also reduce soil compaction caused by the weight of large equipment~\cite{uyeh2021evolutionary,van2022glyphosate}. Even traditional equipment like smart tractors and combines are increasingly equipped with sensors and cameras to collect extensive data about a farm for precision agriculture tasks. Finally, farm workers increasingly rely on connectivity for sharing farm images with each other or to access farm productivity software/services.

To serve most digital agriculture applications, we must consider two layers of connectivity: (a) connectivity \textit{to} a farm (e.g, Internet fiber) that allows farmers and equipment to download information from the Internet, upload data, update software, etc.; and (b) connectivity within the farm (e.g., Wi-Fi, LoRa, TV White Spaces) that enables farmers to provide feedback to autonomous equipment (e.g., does this picture contain a weed), perform teleoperation (e.g., when a robot gets stuck), and collect data for centralized processing. Today applications requiring broadband connectivity suffer from lack of both types of connectivity.

\para{Connectivity to the Farm: }Most broadband monitoring and deployment efforts, both government and commercial, focus on connecting people. Farms are sparsely populated. Therefore, companies have little incentive to deploy traditional connectivity infrastructure like fiber or cellular connectivity. For example, laying fiber can cost up to 10,000 US dollars per mile~\cite{ifn_cost}. This challenge is further exacerbated by the large span of farmlands (e.g., one-third of United States is farmlands~\cite{agcensus}). Therefore, it is infeasible to provide always-on and complete coverage to farms using traditional infrastructure-heavy solutions~\cite{ifn_cost,usb_report}. 

\para{Connectivity within the Farm: }Connecting to robots, smart tractors, and sensors on the field requires us to extend connectivity from the edge of the field (e.g., fiber connectivity that goes to a farm shed or farmer's home) to these devices. This is challenging because of: (a) Range: farms span a few miles. (b) Terrain: Local hills and valleys cause problems for signal propagation. (c) Crops: Crops block radio signals due to their high water content~\cite{wu2017propagation}. Technologies like Wi-Fi do not meet the range and coverage, while LoRa fails to achieve high bandwidth. 



\subsection{Related Work}
\noindent\textbf{Digital Agriculture: }There has been much recent work in digital agriculture in the networking community spanning new connectivity solutions such as Whisper~\cite{whisper} and FarmBeats~\cite{farmbeats}, and new sensing solutions such as Strobe~\cite{strobe}, smol~\cite{smol}, GreenTag~\cite{greentag}, and Comet~\cite{comet}. FarmBeats focuses on leveraging a combination of sensors and drone imagery to extract insights about the farm. FarmBeats relies on TV White Spaces (TVWS) for to-farm connectivity and Wi-Fi for within farm connectivity. Wi-Fi has limited range and therefore, FarmBeats proposes deploying multiple gateways on the farm. In contrast, \name\ utilizes CBRS for on-farm connectivity and does not require multiple gateways. Instead, a single gateway moves to different parts of the farm to provide connectivity. Unlike FarmBeats, our work also profiles (and counters) the effect of crops on CBRS signals.

Similarly, Whisper uses TVWS for narrow-band transmissions for on-farm connectivity, but does not support high bandwidth applications. In general, TV White Spaces (TVWS)~\cite{whitefi,whisper} is a valuable connectivity platform for rural applications due to the availability of empty TV spectrum. TVWS can offer tens of Mbps of data rate and provide long range. However, TVWS require custom hardware and large antennas (due to their low frequencies) that can't be easily carried by drones or under-canopy robots. BYON models proposed in this paper can incorporate TVWS-based hardware where such hardware is available and feasible. 

Our work is also orthogonal to radio-based sensing ~\cite{strobe,smol,greentag,comet} as we do not seek to sense soil health based on radio signal propagation. Instead, we adapt the base station deployment to respond to crop-induced throughput degradations.

Finally,~\cite{wu2017propagation} has studied crop-induced attenuation for Wi-Fi signals, but we are the first to study the impact on CBRS spectrum. In addition, unlike~\cite{wu2017propagation}, we identify mechanisms to deal with such attenuation.

\para{Mobile Base Station Deployments: }There has been recent work on deploying cellular base stations on mobile platforms, often referred to as Cells-on-Wheels (CoWs). CoWs are designed to respond to increased traffic demands (e.g., in stadiums), disaster situations, or public safety use cases~\cite{cows1,911now}. Our work, \name, belongs to this category of research, but expands this along three axes: (a) we focus our work on the agriculture context where crops and the seasonal variation of crops plays an important role in signal quality; (b) while CoWs are generally agnostic to short-term traffic variation, \name\ explicitly adapts to application requirements by adapting the height and orientation of the base station; (c) we conduct an end to end study of a teleoperation use case in under-canopy robots. 

\name\ is also related to work in drone-based cellular deployments~\cite{drone_lte_1,dronelte2,dronelte3, kalantari2016number}. Such networks can vary the base station position depending on application demands. We believe drone-based deployments are harder to mount due to the battery constraints or the cost of deployment in tethered drone operation. However, where such deployments are feasible, they can benefit from insights in \name\ about how to adapt to different crop patterns, heights, etc. 


Finally, past work such as ~\cite{antennaradiation,antennatilt,kovsmerl2014base} has extensively studied optimal placement of cellular base stations, setting the right orientation of antennas, and improving coverage. However, in such cases, the base station placement and configuration optimizes for blanket coverage. In contrast, \name\ optimizes for targeted coverage in agricultural scenarios and can benefit from knowledge of the application. For example, a \name\ setup can use the knowledge of a robot's path to have a coverage bubble follow the robot by changing its height and orientation, while a general base station placement algorithm does not have access to such application-level information.








