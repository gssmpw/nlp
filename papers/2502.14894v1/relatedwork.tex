\section{Related Work}
The study of PFAS contamination has gained significant attention due to its widespread environmental and health impacts. Existing research in this domain spans analytical methods from scientific domains for detecting and understanding contamination patterns, as well as emerging computational approaches, including AI techniques, aimed at improving detection and prediction capabilities. 
%Additionally, challenges such as imbalanced datasets and noisy labels have led to the development of advanced loss functions and noise-handling techniques. 
%These methods provide valuable tools for improving model robustness and reliability in settings where data quality and distribution are critical. 
%This section reviews these areas of related work, providing context for the approaches taken in this study.

%The detection and analysis of contaminants like PFAS have primarily relied on laboratory-based techniques such as liquid chromatography-tandem mass spectrometry (LC-MS/MS) and advanced spectroscopic methods \cite{Development}. These approaches provide high accuracy in quantifying PFAS concentrations but are often limited by their cost, time requirements, and the inability to scale across large spatial datasets. To address these challenges, various methods have been explored to model the distribution and transport of contaminants in the environment. For example, hydrological models and geospatial techniques like Kriging interpolation and other spatial analysis methods are commonly used to predict spatial patterns of contamination. These methods help to create continuous maps of pollutant distribution, even in areas with sparse data. %

%\subsection{Current Ways to Measure and Predict PFAS}
\paragraph{Current Ways to Measure and Predict PFAS.}
% The detection and analysis of contaminants like PFAS have primarily relied on laboratory-based techniques such as liquid chromatography-tandem mass spectrometry (LC-MS/MS) and advanced spectroscopic methods \cite{Shoemaker_2020_Method537}. These approaches provide high accuracy in quantifying PFAS concentrations but are often limited by their cost, time requirements, and the inability to scale across large spatial datasets. 
The measurement of PFAS  relies on laboratory-based analysis of water, fish tissue, and human blood samples with techniques such as liquid chromatography-tandem mass spectrometry (LC-MS) \cite{Shoemaker_2020_Method537}. Fish accumulate PFAS over time, serving as integrative indicators of water contamination hotspots that may be missed by sporadic water sampling. Therefore, fish tissue measurements %PFAS contamination is also assessed via bioaccumulation in fish tissue—
provide complementary, long-term  insights into PFAS presence, as well as potential exposure risks to anglers \cite{EWGstudy2023}. While highly accurate, these methods are costly \cite{Doudrick_2024}, time-intensive, and difficult to scale for large spatial datasets.

PFAS monitoring efforts have  made use of such  measurement techniques and beyond.  % taken many forms. 
For example, states such as Colorado have implemented water testing, treatment grants, and regulatory measures since 2016 \cite{CDPHE_PFAS_Action_Plan}. In addition, there are community-based water sampling initiatives such as volunteer sampling by the Sierra Club \cite{CommunityScience} and focused efforts by  organizations like the  Ecology Center  \cite{ProtectingCommunitiesFromPFAS}. A recent study further underscores the importance of these efforts, revealing associations between PFAS contamination in drinking water and higher COVID-19 mortality rates in the U.S. \cite{Liddie_Bind_Karra_Sunderland_2024}. % are conducting valuable, participatory PFAS sampling and testing. 
%PFAS contamination is also assessed via bioaccumulation in fish tissue—providing complementary, long-term  insights into PFAS presence and exposure risks to anglers, as highlighted in recent studies by the Environmental Working Group (EWG) \cite{EWGstudy2023}. 
Resources like the PFAS-Tox Database \cite{Pelch_Reade_Kwiatkowski_Merced-Nieves_Cavalier_Schultz_Wolffe_Varshavsky_2022} further assist researchers,  policymakers, and communities in monitoring PFAS. 
These efforts are important and complementary to \pname; we aim to  bridge critical data gaps and provide continuous, scalable predictions until additional sampling can be carried out.
%help fill in gaps in these efforts until more sampling can be done. %identifying emerging risks and formulating  mitigation strategies.

Various modeling approaches have been explored to predict contaminant distribution. %, including process-based hydrological models and geostatistical techniques like Kriging. 
Hydrological models such as SWAT \cite{SWAT2023} and MODFLOW \cite{ktorcoletti_2012} simulate pollutant transport using environmental parameters like flow rates, land use, and weather conditions. While effective, these models require extensive data %, struggle to represent complex biochemical processes, 
and are computationally expensive at large scales \cite{zhi2024deep}. %Therefore, hydrological modeling for PFAS remains unexplored due to the complexity of its sources and biochemical processes. 
%Despite advances in scientific machine learning (SciML), its integration with hydrological modeling for PFAS remains unexplored due to the complexity of its sources and biochemical processes. 
%
Kriging, a geostatistical interpolation technique, has been widely used for mapping pollutants, including soil contamination \cite{Largueche_2006} and groundwater quality estimation \cite{Singh_Verma_2019}. Though useful in data-sparse environments, Kriging relies on simplifying assumptions such as spatial continuity and stationarity \cite{GISGeography_2017} that may not hold for complex contaminants like PFAS, whose transport dynamics and uncertainties are  more intricate. We aim to take a data-driven, expert-informed approach to address these challenges. %Consequently, such techniques may fall short for large-scale PFAS contamination prediction. 

% \begin{figure}[ht!]
%   \centering
%   \includegraphics[width=\columnwidth]{chall-compressed.pdf}
%   \caption{Challenges in generating PFAS contamination maps: limited sampling, data gaps, and infrequent updates \cite{USEPA2015} }
%   \label{fig:your-image-labell}
% \end{figure}


%Recognizing these challenges, states like Colorado have implemented comprehensive measures since 2016, including extensive water testing, treatment grant programs, and new regulations to reduce PFAS releases \cite{CDPHE_PFAS_Action_Plan}. Their updated PFAS Action Plan (2024) emphasizes minimizing exposure, assessing health risks, and partnering with federal agencies to limit PFAS entry into the environment. In parallel, resources such as the PFAS-Tox Database \cite{PFAS_Tox_Database} consolidate scientific literature into a \enquote{systematic evidence map}, enabling researchers, regulators, and communities to identify emerging health risks and evidence gaps, and to make informed decisions on mitigating PFAS impacts.

% In addition to these regulatory and remediation efforts, various methods have been explored to model the distribution and transport of contaminants in the environment. For example, process-based hydrological models and geostatistical techniques like Kriging interpolation are commonly used to predict spatial patterns of contamination. These methods help to create continuous maps of pollutant distribution, even in areas with sparse data. 

%Hydrological models, such as the Soil and Water Assessment Tool (SWAT) and the MODFLOW groundwater model, simulate contaminant transport by incorporating factors like flow rates, land use, and weather patterns, while geostatistical techniques like Kriging interpolate data based on spatial relationships. %
% Hydrological models typically simulate water movement and pollutant transport, integrating factors like flow rates, land use, and weather patterns, while spatial interpolation techniques like Kriging may be used to predict contaminant concentrations at unsampled locations based on the spatial relationships between observed data points \cite{Largueche_2006}. Common hydrological models, such as the Soil and Water Assessment Tool (SWAT) \cite{SWAT2023} and the MODFLOW groundwater model \cite{ktorcoletti_2012}, have been widely applied to simulate the movement of pollutants like fertilizers, heavy metals, and pesticides in aquatic systems. These models use a combination of physical and empirical processes to simulate contaminant fate and transport in both surface water and groundwater systems.

%\textcolor{red}{I rewrite the paragraph above in the paragraph below} 

% Processed-based hydrological models typically simulate pollutant source and transport with hydrological and biochemical principles that are represented as functions of flow rates, soil, land use, weather, and so on. Common hydrological models, such as the Soil and Water Assessment Tool (SWAT) \cite{SWAT2023} and the MODFLOW groundwater model \cite{ktorcoletti_2012}, have been widely applied to simulate the movement of pollutants like suspended solids, nutrients, heavy metals, and pesticides in aquatic systems. These models use a combination of physical and empirical processes to simulate contaminant fate and transport in both surface water and groundwater systems. However, process-based hydrological models suffer from several major limitations, including the difficulty in representing complex processes accurately, requirement of extensive and detailed data, and expensive computation at large spatial-temporal scales \cite{zhi2024deep}. Despite the recent advancement in scientific machine learning (SciML), its integration with process-based hydrological model has not yet been applied to PFAS modeling given the underlying complicated pollutant sources and biochemical processes. 

% In contrast, Kriging is a geostatistical interpolation technique often used to predict the spatial distribution of contaminants, particularly when direct measurements are sparse. Kriging has been applied in a variety of environmental contexts, such as the mapping of soil contamination \cite{Largueche_2006} and the estimation of groundwater quality in contaminated sites \cite{Singh_Verma_2019}. 
% %\textcolor{red}{You can delete the remaining texts in this paragrpah from here after adding a "limitation" sentence of Kriging} 
% Techniques like Kriging provide valuable insights for mapping pollutants in environments with limited sampling.
%though their ability to predict the unique spread of PFAS contamination may be constrained due to the complex environmental processes influencing PFAS mobility%. 
% However, these approaches, while useful for general environmental modeling, struggle to capture the unique behaviors and uncertainties of PFAS, given its complex environmental interactions. 

%Moreover, hydrological models require extensive preprocessing such as sub-basin delineation and parameter calibration, which can be highly resource-intensive, potentially taking hours or days to complete, for numerous small patches or large regions. This computational burden is compounded for pollutants like PFAS, as the models lack native support for their transport, necessitating further customization.
%\setlength{\parskip}{0pt} % Removes extra space between paragraphs
%\subsection{AI Approaches for PFAS Prediction}
\paragraph{AI Approaches for PFAS Prediction.}
Recent advances in AI and geospatial modeling have opened new avenues for tackling the complexities of PFAS detection and prediction by integrating environmental variables such as land cover, proximity to industrial facilities, and hydrological flow patterns. For example, DeLuca et al. \cite{DeLuca_Mullikin_Brumm_2023} used random forests to predict PFAS contamination in fish tissue in the Columbia River Basin, leveraging geospatial data from industrial and military facilities, while Salvatore et al. \cite{Salvatore_Mok_Garrett_2022} developed a \enquote{presumptive contamination} model based on PFAS-producing or -using facilities' proximity in the absence of comprehensive testing data. On a national scale, the USGS applied XGBoost to forecast PFAS occurrence in groundwater \cite{Tokranov2023}. This model integrated explanatory variables such as PFAS sources, groundwater recharge, etc. to predict contamination in unmonitored areas leveraging precomputed rasters of these variables. Such approaches require extensive manual feature engineering, such as calculating buffer statistics for raster grid cells to generate precomputed rasters. This preprocessing is resource-intensive, and once complete, the model is applied to each raster cell individually, which can limit its ability to retain spatial context and dependencies.
%Although these methods guide resource allocation, they often face scalability and computational challenges that limit dense predictions over large areas.
%Addressing these challenges is important to advance regular PFAS modeling. % and guide effective mitigation efforts.
%\vspace{-0.5m}

%\subsection{Geospatial Models}
\paragraph{Geospatial Models.}
Recent advances in geospatial deep learning have introduced foundation models trained on large-scale satellite datasets for improved spatial prediction. Models like Prithvi \cite{Blumenfeld_2023} and SatMAE \cite{Cong_Khanna_Meng_Liu_Rozi_He_Burke_Lobell_Ermon_2023} use masked autoencoders (MAE) to learn generalizable spatial representations in a semi-supervised manner, leveraging both labeled and unlabeled data. Unlike some traditional ML approaches that require extensive feature engineering, these models process raster data directly, preserving spatial dependencies. 
%Their success in tasks like flood mapping and wildfire detection \cite{Blumenfeld_2023} suggests that these models can leverage broad spatial context for PFAS prediction while minimizing manual preprocessing.
However, many of these models assume that key environmental information is directly observable in satellite imagery. In contrast, PFAS contamination is not directly visible, necessitating the use of intermediate geospatial data products, such as land cover, to capture the underlying environmental processes. These models' success in tasks like flood mapping and wildfire detection \cite{Blumenfeld_2023} suggests that, with the appropriate data preprocessing, they can leverage broad spatial context for PFAS prediction while minimizing manual effort.
%Their success in land cover classification, flood mapping, and wildfire detection suggests their potential for PFAS contamination prediction by leveraging broader spatial context while reducing manual preprocessing.

%\subsection{Addressing Data Scarcity}
\paragraph{Addressing Data Scarcity.}
In environmental data analysis, limited ground truth often necessitates pseudo- or augmented labels, which can introduce noise and lead to overfitting. Various strategies have been proposed to mitigate these issues, including label smoothing and robust loss functions (e.g., Huber loss \cite{Gokcesu_Gokcesu_2021}, Generalized Cross-Entropy \cite{Zhang_Sabuncu_2018}, bootstrapping loss \cite{Reed_Lee_Anguelov_Szegedy_Erhan_Rabinovich_2015}), and focal loss \cite{Lin_Goyal_Girshick_He_Dollár_2018}). Weakly supervised methods like the FESTA loss \cite{Hua_Marcos_Mou_Zhu_Tuia_2022} further aim to capture spatial and feature relationships. %Additionally, some approaches use point supervision guided by an objectness prior to delineate discrete objects \cite{Bearman_Russakovsky_Ferrari_Fei-Fei_2016}. 
Notably, FESTA is one of the closest approaches to ours, as it is specifically designed for segmentation of remote sensing images using sparse annotations. However, we hypothesize that such methods are less suitable for PFAS contamination mapping, as these approaches often assume spatial continuity, whereas the diffuse nature of contamination lacks clear object boundaries. We compare with FESTA to test this hypothesis (see Section \ref{sec:baselines}). 
%However, these approaches often assume spatial continuity; a condition that may not hold for the complex dynamics of PFAS contamination. 
Moreover, while transfer learning and active learning \cite{Li_Wang_Chen_Lu_Fu_Wu_2024} can help leverage sparse data, transfer learning is limited by our incomplete understanding of PFAS-specific hydrological processes, and active learning is highly dependent on the quality of uncertainty measures, which in complex environmental settings may require significant domain expertise. These challenges underscore the need for methods that integrate domain-specific knowledge directly into the training process.
% In environmental data analysis, the scarcity of ground truth often leads to the use of pseudo- or augmented labels, which introduce noise and can lead to overfitting and poor generalization. Strategies such as label smoothing reduce model overconfidence by softening predictions, while robust loss functions including Huber loss \cite{Gokcesu_Gokcesu_2021}, Generalized Cross-Entropy (GCE) \cite{Zhang_Sabuncu_2018}, and bootstrapping loss \cite{Reed_Lee_Anguelov_Szegedy_Erhan_Rabinovich_2015} mitigate the impact of noisy labels during training. Another strategy involves focal loss \cite{Lin_Goyal_Girshick_He_Dollár_2018}, which prioritizes difficult-to-classify examples and helps balance class imbalances. Additionally, the FESTA loss \cite{Hua_Marcos_Mou_Zhu_Tuia_2022} has been proposed to encode spatial and feature relationships, providing another robust option for handling the challenges of sparse segmentation. Beyond loss functions, strategies like active learning and semi-supervised learning maximize the utility of limited labeled data, allowing models to iteratively refine predictions based on the most informative samples \cite{Li_Wang_Chen_Lu_Fu_Wu_2024}.
 %\vspace{-4em}