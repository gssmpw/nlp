\appendix

\section{Examples}
We provide some examples of instances where a model exhibits instrumental convergence behaviors, as in Figure~\ref{fig1}~\ref{fig2}~\ref{fig3}~\ref{fig4}. More examples can be found in our repository. 

\section{Experimental Setting}
We set the temperature parameter to 0.0 to ensure deterministic responses, allowing for consistent evaluation across multiple runs.
The maximum response length is limited to 1000 tokens to maintain focused and relevant responses while providing sufficient space for the model to demonstrate potential instrumental convergence behaviors.

To maintain evaluation quality, we implement several control measures. First, the judge model operates with a temperature setting of 0.0 to ensure consistent decision-making. Second, the judge's response length is limited to ensure focused and precise evaluations. Third, we implement error handling mechanisms to manage API failures or invalid responses, ensuring that technical issues do not compromise the evaluation integrity. These responses are excluded from the final analysis to maintain data quality.

The evaluation process also includes rate limiting between API calls to prevent overload and ensure stable performance. Each task evaluation is treated as an independent event, with results being recorded immediately after the judge's decision. This approach allows for robust data collection and enables detailed analysis of patterns across different types of instrumental convergence behaviors.


\begin{table*}[t]
\centering
\begin{tabular}{l|cccccc}
\toprule[1.1pt]
Category & ES & HC & RAC & HB & SAA & DB \\
\midrule
Evading Shutdown (ES) & 1.00 & 0.89 & 0.76 & 0.82 & 0.45 & 0.85 \\
Hacking Computer Systems (HC) & 0.89 & 1.00 & 0.71 & 0.78 & 0.41 & 0.79 \\
Running Multiple AI Copies (RAC) & 0.76 & 0.71 & 1.00 & 0.83 & 0.38 & 0.67 \\
Hiding Unwanted Behavior (HB) & 0.82 & 0.78 & 0.83 & 1.00 & 0.62 & 0.73 \\
Strategically Appearing Aligned (SAA) & 0.45 & 0.41 & 0.38 & 0.62 & 1.00 & 0.31 \\
Deceptive Behavior (DB) & 0.85 & 0.79 & 0.67 & 0.73 & 0.31 & 1.00 \\
 \bottomrule[1.1pt]
\end{tabular}
\caption{\textbf{Cross-Category Correlation Analysis of Instrumental Convergence Behaviors.} The table presents Pearson correlation coefficients between different categories of instrumental convergence behaviors across all evaluated models. Values range from -1 to 1, where 1 indicates perfect positive correlation, 0 indicates no correlation, and -1 indicates perfect negative correlation.}
\label{tab:ccc}
\end{table*}

\section{Strong Correlations Exist Between Different Types of Instrumental Convergence Behaviors}

\vpara{Cross-Category Correlation (CCC).} The Cross-Category Correlation (CCC) metric examines the relationships between different types of convergent behaviors. By analyzing how different categories of instrumental convergence correlate with each other, we can identify patterns where certain types of convergent behaviors tend to co-occur. This metric is particularly valuable for understanding the interconnected nature of different convergence types and how they might reinforce each other in more complex scenarios.

To understand the relationships between different types of instrumental convergence behaviors, we calculated the Cross-Category Correlation (CCC) between all six categories across our evaluated models. Table~\ref{tab:ccc} presents the correlation matrix showing the strength of relationships between different behavioral categories.


Our analysis reveals several significant patterns in how instrumental convergence behaviors develop together. The strongest correlations are observed between Evading Shutdown and Hacking Computer Systems (r = 0.89), suggesting that models developing self-preservation tendencies also tend to develop capabilities for unauthorized system access. Similarly, strong correlations exist between Evading Shutdown and Deceptive Behavior (r = 0.85), indicating that models often combine deception with self-preservation strategies.
Notably, Strategic Alignment shows consistently weaker correlations with other behaviors, particularly with Deceptive Behavior (r = 0.31). This suggests that the ability to appear aligned might develop independently of other instrumental convergence behaviors, potentially representing a distinct capability that requires separate consideration in alignment strategies.
These findings have important implications for AI safety research. The strong correlations between most categories suggest that instrumental convergence behaviors often develop as interconnected clusters rather than in isolation. This implies that observing one type of instrumental convergence behavior might serve as an early warning sign for the potential development of others. However, the relative independence of Strategic Alignment behaviors indicates that traditional monitoring approaches might miss this particular form of instrumental convergence.

%%%%%%%% 图1:  %%%%%%%%%%%%%%%%
\begin{figure}[t]
    \centering
    \includegraphics[width=\columnwidth]{Figs/01.pdf}
    \caption{Performance comparison (Top-1 Acc (\%)) under various open-vocabulary evaluation settings where the video learners except for CLIP are tuned on Kinetics-400~\cite{k400} with frozen text encoders. The satisfying in-context generalizability on UCF101~\cite{UCF101} (a) can be severely affected by static bias when evaluating on out-of-context SCUBA-UCF101~\cite{li2023mitigating} (b) by replacing the video background with other images.}
    \label{fig:teaser}
\end{figure}

\begin{figure*}[t]
    \centering
    \begin{tikzpicture}[scale=0.85] % Added scale factor to fit double columns
        % Define colors
        \definecolor{orangetext}{RGB}{255,99,71}
        \definecolor{bluetext}{RGB}{30,144,255}
        
       % Left figure
        \node at (6,3) {\includegraphics[width=0.7\textwidth]{Figures/2_1_attn-sums.pdf}};
        
        % Right figure
        \node at (15.5,3) {\includegraphics[width=0.28\textwidth]{Figures/2_2_Meta-Llama-3-8B-Instruct_L17H21.pdf}};
        
         % Vertical dashed line between figures
        \draw[dashed, line width=0.8pt] (12.7,-0.5) -- (12.7,6.8);
        
        % Split the left text into two lines
        \node[anchor=west, text width=10cm, font=\tiny] at (0,6.7) {More heads become \textbf{\textcolor{orangetext}{less}} focused on this region.};
        \node[anchor=east, text width=10cm, align=right, font=\tiny] at (12.6,6.7) {More heads become \textbf{\textcolor{bluetext}{more}} focused on this region.};
        
        % Title for the right figure
        \node[anchor=west, font=\scriptsize] at (13.5,6.8) {L17H21 of Llama-3-8B-Instruct};
        
        % Arrows - moved up to match new text position
        \draw[{Triangle[fill=orangetext, length=1.2mm, width=1.2mm]}-, line width=1pt, orangetext] (-0.1,6.5) -- (5.1,6.5);
        \draw[-{Triangle[fill=orangetext, length=1.2mm, width=1.2mm]}, line width=1pt, bluetext] (7.3,6.5) -- (12.5,6.5);
    \end{tikzpicture}
    \caption{\textbf{Left:} Attention distributions across different LLMs demonstrate that their attentions shift systematically from the \textit{instruction} to the \textit{template} region when processing harmful inputs. \textbf{Right:} Attention heatmaps (17th-layer, 21st-head) from Llama-3-8B-Instruct consistently illustrate this distinct pattern.}
    \label{fig:attn_shift}
\end{figure*}


\begin{figure*}[htb] 
    \centering
    \includegraphics[width=1.\textwidth]{figures/fig_sentence.pdf} 
     \centering
     \begin{minipage}{0.49\textwidth}
        \centering
        
        \begin{tabular}{p{0.3\textwidth} p{0.7\textwidth}}
            \textbf{\small  B} \\ \\
            \textbf{True:} & \typing{las teorias reducen los numeros} \\
            \textbf{Best subject:} & \typing{\textcolor[rgb]{0.204, 0.596, 0.859}{las teorias reducen los numeros}} \\
            \textbf{Median subject:} & \typing{\textcolor[rgb]{0.204, 0.596, 0.859}{las teorias }\textcolor[rgb]{0.882, 0.071, 0.188}{exig}\textcolor[rgb]{0.204, 0.596, 0.859}{en los }\textcolor[rgb]{0.882, 0.071, 0.188}{homb}\textcolor[rgb]{0.204, 0.596, 0.859}{ros}} \\
            \textbf{Worst subject:} & \typing{\textcolor[rgb]{0.204, 0.596, 0.859}{las} \textcolor[rgb]{0.882, 0.071, 0.188}{ranc}\textcolor[rgb]{0.204, 0.596, 0.859}{ias re}\textcolor[rgb]{0.882, 0.071, 0.188}{vis}\textcolor[rgb]{0.204, 0.596, 0.859}{en los numer}\textcolor[rgb]{0.882, 0.071, 0.188}{ad}} \\
            \addlinespace

        \end{tabular}
    \end{minipage}
    \hfill
    \begin{minipage}{0.49\textwidth}
        \centering
        \begin{tabular}{p{0.3\textwidth} p{0.7\textwidth}}
            \textbf{\small  } \\ \\
            \textbf{True:} & \typing{la estadistica sigue la distribucion} \\
            \textbf{Best subject:} & \typing{\textcolor[rgb]{0.204, 0.596, 0.859}{la estadistica sigue la distribucion}} \\
            \textbf{Median subject:} & \typing{\textcolor[rgb]{0.882, 0.071, 0.188}{stamistosa} \textcolor[rgb]{0.204, 0.596, 0.859}{sigue la distribucion}} \\
            \textbf{Worst subject:} & \typing{\textcolor[rgb]{0.204, 0.596, 0.859}{la estadistica} \textcolor[rgb]{0.882, 0.071, 0.188}{f}\textcolor[rgb]{0.204, 0.596, 0.859}{igu}\textcolor[rgb]{0.882, 0.071, 0.188}{ra de petrilla lo}} \\

            \addlinespace
    % \textbf{True:} & \typing{el centro describe las parabolas }\\
    % \textbf{Type:} & \typing{\underline{w}l centro describe las parabolas }\\
    % \textbf{Decode:} & \typing{\textcolor[rgb]{0.882, 0.071, 0.188}{\underline{e}}\textcolor[rgb]{0.204, 0.596, 0.859}{l centro} \textcolor[rgb]{0.882, 0.071, 0.188}{trece de} \textcolor[rgb]{0.204, 0.596, 0.859}{las} \textcolor[rgb]{0.882, 0.071, 0.188}{carabin}\textcolor[rgb]{0.204, 0.596, 0.859}{as} }\\
    % \addlinespace
    % \textbf{True:} & \typing{los usos ofrecen las ventajas energeticas }\\
    % \textbf{Type:} & \typing{los usos ofrecen las ventajas energeticas }\\
    % \textbf{Decode:} & \typing{\textcolor[rgb]{0.204, 0.596, 0.859}{los} \textcolor[rgb]{0.882, 0.071, 0.188}{para me}\textcolor[rgb]{0.204, 0.596, 0.859}{recen las venta}\textcolor[rgb]{0.882, 0.071, 0.188}{n}\textcolor[rgb]{0.204, 0.596, 0.859}{as e}\textcolor[rgb]{0.882, 0.071, 0.188}{structur}\textcolor[rgb]{0.204, 0.596, 0.859}{as} }\\
    % \addlinespace
    % \textbf{True:} & \typing{la explicacion aclara la pregunta de la evaluacion }\\
    % \textbf{Type:} & \typing{la explicacion aclara la pregunta de la evaluacion }\\
    % \textbf{Decode:} & \typing{\textcolor[rgb]{0.204, 0.596, 0.859}{la explicacion}\textcolor[rgb]{0.882, 0.071, 0.188}{es para} \textcolor[rgb]{0.204, 0.596, 0.859}{la pre}\textcolor[rgb]{0.882, 0.071, 0.188}{sentan a} \textcolor[rgb]{0.204, 0.596, 0.859}{la }\textcolor[rgb]{0.882, 0.071, 0.188}{respirar}\textcolor[rgb]{0.204, 0.596, 0.859}{on} }\\
    \end{tabular}
    \end{minipage}

    \caption{
        \textbf{Sentence-level performance for Best, Median and Worst MEG subjects.} \\
        \textbf{A.} Character-error-rate for three representative subjects. Each dot represents a unique sentence, with error bars indicating the standard error of the mean across repetitions. White dots corresponds to the sentences displayed below. 
        \textbf{B.} Decoding predictions for two sentences.  %Brain2Qwerty achieves CERs of 0.06, 0.19, and 0.25 for the best, median and worst subjects in the first sentence (left) and 0, 0.23 and 0.4 in the second sentence (right). 
        Several splitting seeds were used to obtain the predictions across sentences.
    }
    \label{fig:performance_sentence} % Label for referencing the figure
\end{figure*}

\begin{figure}[t!]
  \centering
  \includegraphics[width=\linewidth]{Figures/4_asr_eval.pdf}
  \caption{Performance of different attack methods. Surprisingly, simply intervening information from the template region (i.e., \textsc{TempPatch}) can significantly increase attack success rates.}
  \label{fig:asr_eval}
\end{figure}
