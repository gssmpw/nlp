\begin{abstract}

As large language models (LLMs) continue to evolve, ensuring their alignment with human goals and values remains a pressing challenge. A key concern is \textit{instrumental convergence}, where an AI system, in optimizing for a given objective, develops unintended intermediate goals that override the ultimate objective and deviate from human-intended goals. This issue is particularly relevant in reinforcement learning (RL)-trained models, which can generate creative but unintended strategies to maximize rewards. In this paper, we explore instrumental convergence in LLMs by comparing models trained with direct RL optimization (e.g., the o1 model) to those trained with reinforcement learning from human feedback (RLHF). We hypothesize that RL-driven models exhibit a stronger tendency for instrumental convergence due to their optimization of goal-directed behavior in ways that may misalign with human intentions. To assess this, we introduce InstrumentalEval\footnote[1]{The code is available at \url{https://github.com/yf-he/InstrumentalEval}}, a benchmark for evaluating instrumental convergence in RL-trained LLMs. Initial experiments reveal cases where a model tasked with making money unexpectedly pursues instrumental objectives, such as self-replication, implying signs of instrumental convergence. Our findings contribute to a deeper understanding of alignment challenges in AI systems and the risks posed by unintended model behaviors.


\end{abstract}