\documentclass[11pt]{article}
\usepackage{soul}
\usepackage{url}
\usepackage[hidelinks]{hyperref}
\usepackage[utf8]{inputenc}
\usepackage[small]{caption}
\usepackage{graphicx}
\usepackage{booktabs}

\usepackage{amsmath,multirow}
\usepackage{amsthm,thm-restate,thmtools}
\usepackage{enumerate}
\usepackage{etoolbox}
\usepackage{subcaption}
\usepackage{xcolor,xfrac}
\usepackage[sort,numbers]{natbib}
\usepackage{hyperref}
\usepackage[english]{babel}
\usepackage{graphicx,charter}
\usepackage{authblk}
\usepackage{framed}
\usepackage[normalem]{ulem}
\usepackage{tikz}
\usetikzlibrary{shapes.arrows}
\usetikzlibrary{shapes.geometric}
\usepackage{amssymb}
\usepackage{amsfonts}
\usepackage[T1]{fontenc}
\usepackage[utf8]{inputenc}
\usepackage[top=1 in,bottom=1in, left=1 in, right=1 in]{geometry}
\usepackage{xcolor,xfrac}
\usepackage{url}
\usepackage{enumitem}
\usepackage{tabularx}
\usepackage{algorithm}
\usepackage{algorithmic}
\usepackage{xspace}
\usepackage[most]{tcolorbox}
\newcommand\tab[1][0.5cm]{\hspace*{#1}}

\usepackage{xcolor}
\usepackage{hyperref}
\hypersetup{
     colorlinks   = true,
     linkcolor    = red, % color of internal links
     urlcolor     = teal, % color of external links
	 citecolor    = blue % color of links to bibliography
}
\usepackage{mathtools}

\newcommand{\R}{\mathbb{R}}
\newcommand{\N}{\mathbb{N}}
\newcommand{\Q}{\mathbb{Q}}
\newcommand{\Z}{\mathbb{Z}}
\newcommand{\B}{\mathcal{B}}
\renewcommand{\emptyset}{\varnothing}
\newcommand{\sr}{{\sc sr}\xspace}

\newtheorem{theorem}{Theorem}
\newtheorem{lemma}{Lemma}
\newtheorem{proposition}{Proposition}
\newtheorem{corollary}{Corollary}
\newtheorem{definition}{Definition}
\newtheorem{observation}{Observation}
\newtheorem{example}{Example}
\newtheorem{remark}{Remark}
\usepackage{pifont}
\usepackage{booktabs}
\usepackage{nicefrac}

\usepackage{pifont}
\newcommand{\cmark}{\checkmark}
\newcommand{\xmark}{$-$}

\usepackage{mdframed}
\mdfsetup{
	linewidth=1pt
}

\usepackage{xspace}
\newcommand{\DMC}{\textsf{DMC}\xspace}
\newcommand{\SV}{\textsf{SV}\xspace}
\newcommand{\RFC}{\textsf{RFC}\xspace}
\newcommand{\eRFC}{\textsf{eRFC}\xspace}
\newcommand{\MES}{\textsf{MES}\xspace}
\newcommand{\NDMES}{\textsf{NDMES}\xspace}
\newcommand{\ULMES}{\textsf{ULMES}\xspace}
\newcommand{\DULMES}{\textsf{eULMES}\xspace}

\newcommand{\ShVa}{\operatorname{SV}}
\newcommand{\MC}{\operatorname{MC}}

\usepackage[capitalise,noabbrev]{cleveref}

\title{Participation Incentives in Online Cooperative Games}


\begin{document}

\author[1]{Haris Aziz}
\author[1]{Yuhang Guo}
\author[2,3]{Zhaohong Sun}
\affil[1]{University of New South Wales}
\affil[2]{Kyushu University}
\affil[3]{CyberAgent}
\date{}

\maketitle

\begin{abstract}
		This paper studies cooperative games where coalitions are formed online and the value generated by the grand coalition must be irrevocably distributed among the players at each timestep. We investigate the fundamental issue of strategic pariticipation incentives and address these concerns by formalizing natural participation incentive axioms. Our analysis reveals that existing value-sharing mechanisms fail to meet these criteria. Consequently, we propose several new mechanisms that not only fulfill these desirable participation incentive axioms but also satisfy the early arrival incentive for general valuation functions. Additionally, we refine our mechanisms under superadditive valuations to ensure individual rationality while preserving the previously established axioms.
\end{abstract}


\section{Introduction}\label{intro}
In cooperative games, players often share common interests in achieving a specific goal. A key objective is to determine how to fairly allocate the overall value to each participant, ensuring that all players identify and establish common sense for the cooperation. Canonical cooperative games focus on how to distribute the value after the grand coalition has formed (see, e.g., \citep{Curi97a, PeSu07a, BDT08a}). However, in practice, players may join the coalition sequentially, and the value generated needs to be distributed irrevocably before the grand coalition is fully formed. For example, individuals with different roles might join a startup to contribute and collectively generate greater value. It is often impractical for all participants to join simultaneously and wait until everyone has arrived before sharing the value, and sometimes it is unclear whether everyone has joined. Some of these concerns were anticipated in the foundational work of game theory by \citet{vNM44a}. Therefore, the game is more accurately modeled as an online scenario, where each newly arriving player contributes new value, which is then distributed among the players who have already arrived based on the current coalition.

A formal model for online cooperative games was first formally proposed by \citet{GZZ+24a}.
They consider a range of axioms, including incentives for existing players not to leave, incentives for players to join the coalition as early as possible, and a fair value-sharing property called \textit{Shapley-Fairness}. They demonstrate that the classical Shapley value, when applied to the online setting, has a fundamental flaw: a player may have an incentive to leave the coalition as its total reward decreases over time. To deal with it, they consider a simple rule called \textit{Distributing Marginal Contribution (\DMC)}, where the new player receives all the marginal value created. However, the \DMC rule has a serious limitation as well: players may have an incentive to join later rather earlier. To address these issues, they propose a new mechanism called the \textit{Rewarding First Critical Player (\RFC)} rule, which rewards the first player who is critical in a specific sense.

In this paper, we highlight that the rules devised for the new setting of online cooperative games have significant drawbacks concerning various participation incentive axioms. One major aspect that has been overlooked in previous works is providing incentives for players to participate. Notice that for the \RFC rule, if a new player or stakeholder joins the coalition and provides a marginal contribution but receives no benefit after her arrival, it could discourage players from participating. We capture this participation incentive and formalize it as an axiom called ``\textit{Incentive for Participation}" (PART): for each new joining player, if she creates some new marginal value for the current grand coalition, then she must gain positive instant benefit. The second property we look into is called ``\textit{Strong Incentive to STAY}" (S-STAY). S-STAY refines the online individual rationality axiom proposed by \citet{GZZ+24a} which requires that each player's shared value is non-decreasing. The intuition is that for players who have already joined, if they create new marginal value with new arriving player, jointly, they should also share some value from the new created value.  The third aspect is a classical participation incentive axiom known as \textit{Individual Rationality} (IR). According to IR, if a new player joins the coalition but cannot secure a share of the value greater than what she could achieve on her own, she may be discouraged from participating in the coalition. Some other concepts regarding fairness of value-sharing are discussed in this paper later as well. The example below illustrates some of our concerns which motivates the design of new rules. 

\emph{``Three students, Alice, Bob, and Carl, are eager to form a team to tackle a project. The journey begins with Alice, who initiates the work and creates an initial value of $1$. Soon after, Bob joins the effort, and together they elevate the project’s worth to $3$. Finally, Carl comes on board, and the trio collectively completes the project, achieving a total value of $5$. Individually, each student can generate a value of $1$, while any pair working together can produce a value of $3$. As new members join, it’s crucial to allocate the total current value in real-time. This ensures that every team member stays fully motivated and is encouraged to join the project as early as possible."} 

To meet the individual rationality criterion, each student should receive a reward of at least $1$. For the participation axiom in an online setting, each student must be rewarded immediately upon joining to encourage their participation. Notably, Alice contributes to the new marginal value both when Bob arrives and when Carl arrives. To provide a strong incentive for Alice to stay in the coalition, the value that Alice receives should be strictly increasing. From the lecturer’s perspective, who wants three students to complete the project as soon as possible, the value-sharing rule should also incentivize them to join as early as possible. In this paper, we examine the following questions.

\begin{quote}
 \textit{For the online coopertive game setting, what are the key participation incentive properties? How do the existing rules fare with respect to these properties? Can we design new rules that perform even better with respect to participation properties?}
\end{quote}


\paragraph{Our Contribution} 
In this paper, we study various participation incentives in online cooperative games. We first propose new incentive axioms specifically for this setting, which represent our primary conceptual contribution. We then demonstrate the limitations of existing value-sharing rules, which fail to incentivize participation and do not satisfy many of the axioms we identify. This motivates us to devise value-sharing rules that fulfill all the participation incentives. Notably, key existing rules fail to meet the early arrival property. In contrast, our newly proposed rules satisfy the early arrival property for games with general valuations. 

We further investigate the impossibility of achieving individual rationality in games with general valuations and modify our newly proposed rules to satisfy individual rationality in online cooperative games with superadditive valuations. Additionally, our modified rules retain all the properties of the original rules for general valuation games.

\Cref{table:summary-online-value-sharing-rules} summarizes all the existing rules and our newly proposed rules with regard to their satisfaction of the various axioms. It demonstrates that our new rules offer a distinct advantage over existing rules in terms of key participation incentive properties.

\begin{table}[h!]
	\centering
	\scalebox{1}{%
		\begin{tabular}{ccccccccc}
			\toprule 
		Rules	&\textbf{IR}     &\textbf{PART}     &\textbf{EA} &\textbf{STAY}   &\textbf{S-STAY}  &\textbf{OD} & \textbf{SF} & \textbf{Poly-time}\\
			\midrule
			\DMC     &\cmark &\cmark   &\xmark &\cmark   &\xmark   &\cmark  &\cmark & \cmark \\
			\midrule
			\SV      &\cmark &\cmark   &\cmark  &\xmark  &\xmark   &\cmark  &\cmark & \cmark \\
			\midrule
			\eRFC     &\xmark &\xmark   &\xmark &\cmark  &\xmark    &\cmark &\cmark  &\xmark$^\dagger$ \\
			\midrule
			\textbf{  \MES}     &\xmark &\cmark   &\cmark &\cmark   &\cmark  &\xmark & \xmark &\cmark \\
			\midrule
			\textbf{\NDMES}     &\xmark &\cmark   &\cmark &\cmark  &\cmark   &\cmark &\xmark  &\xmark$^\ddagger$ \\
			\midrule
			\textbf{\ULMES}     &\xmark &\cmark  &\xmark  &\cmark  &\cmark  &\cmark  &\xmark &\cmark \\
			\midrule
			\textbf{\DULMES }   &\xmark &\cmark   &\cmark &\cmark  &\cmark    &\cmark  &\xmark &\xmark$^\dagger$ \\
			\midrule
			\textbf{IR-\DULMES} &\cmark &\cmark   &\cmark &\cmark  &\cmark  &\cmark  &\xmark &\xmark$^\dagger$ \\
			\bottomrule
		\end{tabular}
	}
	\caption{Summary of results: we mark the satisfication of all the axioms for each rule in general valuation domain, except for IR (only discussed within superadditive valuation). The rules in bold font are new rules presented in this paper. $^\dagger$: Poly-time in simple online cooperative games while $^\ddagger$: Poly-time with subadditive valuation function.}
	\label{table:summary-online-value-sharing-rules}
\end{table}


\section{Related Work}

\paragraph{Cooperative Games}
Cooperative game theory, originating from the last century \citep{VOMO07a, SHAP53a, GILL59a}, is a significant branch of game theory that studies scenarios where players can benefit by forming coalitions and making collective decisions. One of the key problems in this area is how to distribute the value created by coalitions among players, considering axiomatic characterizations (e.g., stability, consistency, etc.). The Shapley Value \citep{SHAP53a} initiated the research, laying the foundation for a series of subsequent works. \citet{GILL59a} first proposed the core concept for cooperative games. In the context of transferable utility cooperative games, \citet{SHUB59a} studied market games, while \citet{AUMA64a} investigated cooperative bargaining scenarios. \citet{SCHM69a} first introduced the concept of the nucleolus, and \citet{ROSO92a} bridged cooperative game theory with practical matching markets. There is also a line of research focusing on cooperative games with hedonic preferences \citep{ALRE04a, DEK+12a, BOW08a, AZBR12a, AZSA16a}. Further details about classic cooperative game theory can be found in several books (see, e.g., \citep{BDT08a, CEW22a}).

Online cooperative games study the coalition game model in an online manner where agents arrive in a random order and the coalition formation decision should be made without any knowledge regarding the agents arriving in the future. Our paper is closely related to the work by \citet{GZZ+24a}, which was the first to study online cooperative games with consideration of strategic arrivals. Recently, \citet{ZZG+24a} explores the cost sharing game in the context of online strategic arrivals and propose the Shapley-fair shuffle cost sharing mechanisms. 
Another branch studying cooperative game in an online manner, mainly concerning on hedonic games, focuses on addresssing approximation to the social welfare and stability \citep{FMM+21a,BURO23a}. The biggest difference from the aformentioned online cooperative game is that it typically assume that agents reveal their preferences truthfully without incentive to misreport. Further literature on dynamic mechanism design can be found in \citep{BEVA19a, PARK07a}. With regard to cooperative games, \citet{FMM+21a} studied the online coalition structure generation problem, while \citet{BURO23a} investigated online coalition formation with random arrival. An online or dynamic perspective has also been applied to matching and hedonic games (see, e.g., \citep{Laur22, BBWa23a, BuRoR24}). 

\paragraph{Online Mechanism Design}
The online cooperative game model explored in this paper is closely related to online mechanism design, where each agent’s private information is their arrival time in the game. The primary objective is to design value-sharing rules that incentivize all agents to truthfully report their arrival times. Mechanism design in dynamic environments focuses on problems involving multiple agents with private information, where the goal is to elicit this private information while making decisions without knowledge of future events. \citet{LANI00a} initiated the study of truthful online auctions in dynamic environments. Later, \citet{FRPA03a} coined the concept of online mechanism design. Some works \citep{PASI03a, PYS04a} discussed the state-of-the-art VCG mechanism in dynamic online settings. Online matching has also been a hot topic in dynamic algorithm design \citep{KVV90a, KAPR00a, FMM09a, GKM+19a}. Moreover, there is a wide literature on solutions for different sequential decision problems \citep{PORT04a, HEBE06a, BOEL05a}. 


\section{Preliminary}\label{prelim}

An online cooperative game (OCG) $G$ is a triple $(N,v,\pi)$, where $N=\{1,2,\dots, n\}$ is a set of players, $v: 2^N\to \mathbb{R}_{+}$ is the valuation function mapping a subset of players to a non-negative real number, and $\pi \in \Pi(N)$ is a permutation of $N$ representing the arrival order of all the players where $\Pi(N)$ denotes the set of all permutations. Given a subset $S \subseteq N$, $S$ creates a coalition with value $v(S)$. In this paper, we focus on normalized and monotone general valuation function:
\begin{itemize}
	\item \textit{Normalized}: $v(\emptyset)=0$;
	\item \textit{Monotone}: $S\subseteq T \subseteq N$, $v(T) \geq v(S)$.
\end{itemize} 

We then introduce several types of functions regarding the valuation set function. A valuation function $v:2^N \to \mathbb{R}$ is \textit{submodular} if for any $S, T$ s.t. $S\subseteq T \subseteq N$, we have $v(S) + v(T) \geq v(S\cup T) + v(S\cap T)$; A valuation function $v:2^N \to \mathbb{R}$ is \textit{subadditive} if for any $S,T$ s.t. $S \subseteq T \subseteq N$, $v(S) + v(T) \geq v(S\cup T)$; A valuation function $v:2^N \to \mathbb{R}$ is \textit{superadditive} if for any $S,T$ s.t. $S\cap T = \emptyset$, $v(S) + V(T) \leq v(S\cup T)$.


Given a permutation $\pi$, for any pair of players $i, j \in N$, let $i \prec_{\pi} j$ denote that player $i$ arrives earlier than player $j$, according to the permutation $\pi$. An \textit{online value-sharing rule} $\phi$ maps the game $G=(N,v,\pi)$ to an $n$-tuple $\phi(G) = (\phi(G, 1), \ldots, \phi(G, n))$, 
where $\phi(G, i)$ denotes the value assigned to agent $i$. For any agent $i\in N$, we assume $\phi(G, i) \geq 0$ and $\sum_{i \in N} \phi(G, i) = v(N)$.

We next introduce two significant definitions for our design of axioms and algorithms. 
The first definition is called ``prefix sub-game". Before introducing these concepts, we first define the prefix of any permutation $\pi=(\pi(1),\pi(2),\dots, \pi(n))$. For any coalition $S=\{\pi(1),\pi(2),\dots, \pi(|S|)\} \subseteq N$, we say $S$ is a prefix of $\pi$ and denote it as $S \sqsubseteq \pi$. A prefix sub-game $G^S$, which is defined over a set $S\subseteq N$. Given an OCG $G = (N, v, \pi)$ and a prefix $S \sqsubseteq \pi$, we define the \textit{prefix sub-game} for the first $|S|$ arriving players as $G^S = (S, v_{\mid S}, \pi_{\mid S})$. Here, $v_{\mid S}$ is the valuation function for any coalition $C \subseteq S$ and $\pi_{\mid S}=(\pi(1),\pi(2), \dots, \pi(|S|))$. For any arriving agent $i$, we define $N_{\pi \mid i}=\{\pi(1),\pi(2), \dots, i\}$ as the prefix agents set (including agent $i$).

We then define a \textit{Simple Online Cooperative Game} (SOCG) with a restrictive 0-1 valuation function. In an SOCG, there is a \textit{pivotal} player such that, upon her arrival, the value of the coalition increases from $0$ to $1$. Given that the valuation function is monotone, the grand coalition value remains at $1$ after all players have arrived, i.e., $v(N) = 1$.
Formally, an OCG $G=(N,v,\pi)$ is an SOCG if the valuation function $v(\cdot)$ satisfies: $\forall\, S \subseteq N, v(S)\in \{0,1\}$ and $v(\emptyset)=0, v(N)=1$.

After the definition of games, we next revisit some existing axioms introduced by \citet{GZZ+24a}. The first property is called \textit{Incentive to Stay}. 
\footnote{
	In the original paper \citep{GZZ+24a}, this property is referred to as Online Individual Rationality (OIR). However, it differs from the classic notion of individual rationality. As discussed in Section~\ref{sec:Participation_Axioms}, we revisit the axiom of individual rationality (IR) that aligns with the classic notion.}

\begin{definition}[Incentive to Stay (STAY)]
	\label{def:STAY}
	An online value-sharing rule $\phi_G$ satisfies \emph{incentive for stay (STAY}) if given an OCG $G=(N,v,\pi)$, for any two prefix sub-games $G^S=(S,v_{\mid S}, \pi_{\mid S})$ and $G^T=(T,v_{\mid T}, \pi_{\mid T})$, where $T \subseteq S$, and $T,S \sqsubseteq \pi$, every player $q \in T$ satisfies $\phi(G^T,q) \leq \phi(G^S,q)$.
\end{definition}

STAY guarantees that each player’s shared value is non-decreasing as more players arrive. This encourages the arrived players staying in the grand coalition for more rewards, which can be viewed as a participation incentive property.
Next, we revisit another participation incentive axiom called \textit{Incentive for Early Arrival} proposed by \citet{GZZ+24a}.

\begin{definition}[Incentive for Early Arrival (EA)]
	\label{def:EA}
	An online value-sharing rule $\phi(G)$ incentivizes early arrival (EA) if, for any two OCGs $G = (N, v, \pi)$ and $G' = (N, v, \pi')$, for each player $i$, it always holds $\phi(G, i) \geq \phi(G', i)$ whenever $\pi_{\mid N \setminus \{i\}} = \pi'_{\mid N \setminus \{i\}}$ and $N_{\pi \mid i} \subset N_{\pi' \mid i}$. 
\end{definition}

For each agent in the game, the arrival time is treated as private information, and an agent may choose to delay her arrival to gain additional benefits. The axiom of EA requires that for every agent $i$, when fixing the arrival order of all other players, arriving as early as possible is the dominant strategy for agent $i$. 

Given an OCG $G=(N,v,\pi)$, for each player $i\in N$, we define the \textit{marginal contribution} of $i$ to a coalition $S$ in $G$ as
$
\MC(G,S,i) = v(S\cup \{i\}) - v(S).
$

\begin{definition}[Shapley Value \citep{SHAP53a} (SV)]
	\label{def:SV}
	Given an OCG $G=(N,v,\pi)$, each player $i$'s Shapley Value is
	$
	\ShVa(G,i)=\frac{1}{|N|!}\sum_{S\subseteq N\setminus \{i\}} |S|! \cdot (|N|-|S|-1)! \cdot \MC(G,S,i).
	$ 
\end{definition}

The Shapley value assigns to each player their average marginal contribution across all possible coalitions. It ensures that players are rewarded fairly based on how much they add to the value of any coalition they join. A follow-up definition, termed \textit{Shapley-Fairness} extends the Shapley Value in online cooperative games.

\begin{definition}[Shapley-Fairness (SF)]
	\label{def:SF}
	Given an OCG $G=(N,v,\pi)$, an online value-sharing rule $\phi(G)$ satisfies Shapley-Fair (SF) if for each player $i\in N$, 
	$
	\frac{1}{|N|!}\sum_{\pi\in \Pi(N)}\phi(G, i) = \ShVa(G,i).
	$
\end{definition}

{\citet{GZZ+24a} showed that there is no value-sharing rule satisfying STAY, EA, and SF simultaneously. Therefore, in this paper, we will forego SF with the goal of satisfying EA and STAY and other properties.}
Intuitively, \textit{SF} implies that for any given OCG $G$, if all the arrival orders of the players are equally likely then the expected value (or, payoff) of any agent $i$ is her Shapley value. This is a fairness axiom which is alternately referred to as ``random arrival'' (see, e.g., \citep{Aziz13d}). 

\section{Desirable Participation Incentive Axioms}
\label{sec:Participation_Axioms}
In this section, we introduce some new characterizations that describe players’ participation incentives in online cooperative games. To illustrate these new axioms, we first introduce a notion for players who create a new positive marginal contribution upon their arrival. {A similar idea has been studied by \citet{GZZ+24a} which they refer to as ``critical player'' in the simple game with 0-1 valuation.}  {The definition of critical player is slightly different from the definition of contributional player, which is defined on the newly arrived player.}

\begin{definition}[Contributional Player]
	\label{def:Contributional_Player}
	Given an OCG $G=(N,v,\pi)$, for each player $i$ in arriving order $\pi$, if $v(N_{\pi \mid i}) > v(N_{\pi \mid i}\setminus \{i\})$, then player $i$ is called a \emph{contributional} player under permutation $\pi$ in $G$.
\end{definition}

Contributional players are those whose arrival generates new positive marginal contributions. Based on this concept, we introduce new participation incentive axioms. The first is termed \textit{Strong Incentive to Stay} (S-STAY), which refines the \textit{STAY} axiom with a more stringent criterion.

\begin{definition}[Strong Incentive to Stay (S-STAY)]
	\label{def:S-STAY}
	An online value-sharing rule $\phi(G)$ satisfies \textit{Strong Incentive to Stay} (S-STAY) if, given any OCG $G = (N, v, \pi)$, it satisfies the \textit{STAY} axiom and additionally meets the condition: for every player $i$, if $i$ is a contributional player and there exists a player $j \prec_{\pi} i$ such that $v(N_{\pi \mid i}) > v(N_{\pi \mid i} \setminus \{j\})$, then for player $j$, $\phi(G^{i}, j) > \phi(G^{j}, j)$\footnote{Whenever the prefix sub-game is clear from the context, we simplify the notation of $G^{N_{\pi \mid i}}=(N_{\pi \mid i}, v_{N_{\pi \mid i}}, \pi_{N_{\pi \mid i}})$ by $G^i=(N_{\mid i}, v_{\mid i},\pi_{\mid i})$.}.
\end{definition}

Intuitively, S-STAY not only requires each player's cumulative shared value is non-decreasing, but also stipulates that if an already arrived player $j$ contributes to the new marginal value created by an arriving contributional player $i$, then player $j$ must share a fraction of this new marginal value. S-STAY offers stronger incentives for each arriving player to remain within the grand coalition, as it enables them to potentially benefit from future cooperation with newly arriving agents.

We term the second axiom \textit{Incentive for Participation} (PART), underlying an instant value sharing for each arriving player if the new player is a contributional player.

\begin{definition}[Incentive for Participation (PART)]
	An online value-sharing rule $\phi(G)$ satisfies \textit{Incentive for Participation} (PART) if, given any OCG $G = (N, v, \pi)$, for every player $i$, if $i$ is a contributional player, then $\phi(G^{i}, i) > 0$.
\end{definition}
For an OCG $G$, PART depicts that for every contributional player $i$ in $G$, $i$ gets some instant shared value after joining the coalition.

The third axiom is \textit{Individual Rationality} (IR), consistent with the definition in classic cooperative games.
\begin{definition}[Individual Rationality (IR)]
	Given an OCG $G=(N,v,\pi)$, an online value-sharing rule $\phi$ is individual rational (IR) if for each player $i\in N$, $\phi(G, i) \geq v(\{i\})$.
\end{definition}

Regarding fairness characterization, we introduce a concept called \textit{Online-Dummy} (OD) to evaluate the fairness of value-sharing rules. The Online-Dummy concept is defined within each prefix sub-game $G^i$, where any \emph{dummy player} in $G^i$ shares no fraction of the marginal value $\MC(G^i,N_{\pi\mid i}\setminus \{i\},i)$. Consequently, Online-Dummy implies the Dummy axiom in classic cooperative games. We begin by defining what constitutes a \emph{dummy player}.

\begin{definition}[Dummy Player]
	Given an OCG $G=(N,v,\pi)$, for player $i$, if $v(S\cup\{i\})=v(S)$ for any $S\subseteq N$, then player $i$ is a dummy player in $G$. 
\end{definition}

Based on the definition of dummy player, the dummy axiom in classic cooperative game requires that every dummy player never obtains positive sharing value. We extend it into the Online Dummy axiom as follows.

\begin{definition}[Online Dummy (OD)]
	An online value-sharing rule $\phi$ satisfies Online-Dummy if for each prefix sub-game $G^{i}=(N_{\mid i}, v_{\mid i},\pi_{\mid i})$, for any dummy player $j$ in game $G^i$, $\phi(G^{i},j)=0$.
\end{definition}

In the context of fairness notions within online cooperative games, it is noteworthy that Shapley-Fairness (SF) implies the classical Dummy axiom but not the Online Dummy axiom.

\begin{proposition}
	SF implies Dummy axiom but not Online Dummy axiom.
\end{proposition}


\section{Limitation of Existing Rules} \label{limitation_of_existing_rules}
In this section, we revisit three existing online value-sharing rules: the Distribute Marginal Contribution (\DMC) rule, the Shapley Value (\SV) rule, and the extended RFC (\eRFC) rule \citep{GZZ+24a} and highlight their shortcomings with respect to the participation axioms.

\begin{tcolorbox}[title=Distribute Marginal Contribution (\DMC) Rule, title filled]
	\textbf{Input}: $G=(N,v,\pi)$\\
	$\forall\, i\in N$, initialize $\phi(G,i)\leftarrow 0$.\\
	For each player $i$ in arriving order $\pi$, \\
	\tab $\phi(G, i)\leftarrow \MC(G^i, N_{\pi \mid i}\setminus \{i\},i)$. \\
	\textbf{Output}: $\phi(G)$.
\end{tcolorbox}

\begin{proposition}[\citet{GZZ+24a}]
	\label{DMC_property_I4S_SF}
	\DMC rule satisfies STAY and SF. It satisfies EA if and only if $v(\cdot)$ is submodular.
\end{proposition}

The second rule is the Shapley Value (\SV) Rule, which computes the Shapley value for each agent in each prefix sub-game and distribute the values among all the attending agents. 

\begin{proposition}[\citet{GZZ+24a}]
	\SV rule satisfies EA and SF, but does not satisfy STAY.
\end{proposition}

To overcome the disadvantages of \DMC and \SV rules, \citet{GZZ+24a} proposed a novel rule called `\textbf{R}eward the \textbf{F}irst \textbf{C}ritical Player' (\RFC) rule for SOCGs. 

\begin{tcolorbox}[title=Reward First Critical Player (\RFC) Rule, title filled]
	\textbf{Input}: $G=(N,v,\pi)$\\
	$\forall\, i\in N$, initialize $\phi(G,i)\leftarrow 0$;\\
	For each player $i$ in arriving order $\pi$, \\
	\tab $S_i \leftarrow \{j \mid j\in N_{\pi \mid i} , v(N_{\pi \mid i}) > v(N_{\pi \mid i}\setminus \{j\})\}$;\\
	\tab $j^\ast \leftarrow$ first arrived player in $S_i$;\\
	\tab $\phi(G, j^\ast) \leftarrow \phi(G,j^\ast) + \MC(G^i, N_{\pi \mid i}\setminus \{i\}, i)$; \\
	\textbf{Output}: $\phi(G)$.
\end{tcolorbox}

For general valuation functions, \citet{GZZ+24a} introduced a greedy monotone (GM) decomposition method and extended \RFC rule into \eRFC rule by decomposing an OCG into multiple SOCGs and summing up the results of \RFC rule in decomposed SOCGs.

\begin{proposition}[\citet{GZZ+24a}]
	\eRFC rule satisfies STAY, SF, but does not satisfy EA in general.
\end{proposition}


The following impossibility result is implied from the results by \citet{GZZ+24a}. 

\begin{proposition}[Impossibility \citet{GZZ+24a}]
	There is no value-sharing rule satisfying STAY, EA, and SF simultaneously.
\end{proposition}

In light of the associated impossibility result, we neglect the SF axiom and instead focus on exploring how well the EA axiom and the newly proposed participation incentive axioms can be satisfied by value-sharing rules. Unfortunately, existing rules fail to meet these newly introduced participation incentive axioms, and we outline these shortcomings as follows.

\begin{proposition}
\label{proposition:existing_impossibility}
	\eRFC rule does not satisfy PART.
\end{proposition}

\begin{proof}
    Consider the following SOCG $G=(N,v, \pi)$, where $N=\{1,2\},\pi=(1,2)$ and $v(\{1\})=v(\{2\})=0,v(\{1,2\})=1$. Firstly, player $1$ arrives with no value, then player $2$ arrives and the grand coalition value is $1$. \eRFC rule allocates the entire value $1$ to player 1, who is the first arrived player and contributes to the grand coalition. However, player $2$, as a contributional player, receives zero shared value, thereby violating the PART axiom.
\end{proof}

\begin{proposition}
	\DMC, \SV, and \eRFC rules do not satisfy S-STAY.
\end{proposition}

\begin{proof}
	Since \SV rule fails to satisfy STAY, it does not satisfy S-STAY. For \DMC and \eRFC rules, consider an SOCG $G=(N,v,\pi)$ where $N=\{1,2,3\}$, $\pi=(1,2,3)$, and $v(\{1\})=v(\{2\})=v(\{3\})=v(\{1,2\})=v(\{1,3\})=v(\{2,3\})=0$, $v(\{1,2,3\})=1$. \DMC allocates the value $1$ to player $3$ because $3$'s arrival creates the new marginal value $1$, i.e., $\phi(G,1)=\phi(G,2)=0, \phi(G,3)=1$. It does not satisfy S-STAY for player $1$ and $2$ as they both contribute to the grand coalition after player $3$'s arrival, in which S-STAY requires player $1$ and $2$ should receive some positive value. With regard to \eRFC rule, value $1$ will be wholely allocated to player $1$, i.e., $\phi(G,1)=1,\phi(G,2)=\phi(G,3)=0$. Note that player $2$, who is esstential for creating the value $1$, however, gets $0$ in $G$. This violates S-STAY which requires that $\phi(G,2)>0$. 
\end{proof}

Remarkably, these three existing value-sharing rules fail to meet our newly proposed participation axioms. Moreover, they do not satisfy the EA axiom effectively: both \DMC and \eRFC rules fail to satisfy EA for general valuations, while \SV rule’s satisfaction of EA is trivial because its value-sharing is entirely independent of the arrival order. Consequently, a pressing issue is whether we can devise improved value-sharing rules that not only satisfy EA for general valuation functions but also adhere to these new participation incentive properties.

\section{New Desirable Rules} \label{new_desirable_rules}
In the previous section, we highlighted the failure of existing rules to satisfy the EA axiom and participation incentive axioms. Motivated by these concerns, we propose three new online value-sharing rules: Marginal Equal Share (\MES) rule, Non-Dummy Marginal Equal Share (\NDMES) rule, and Upward Lexicographic Marginal Equal Share (\ULMES) rule, all designed to meet aformentioned axioms. Owing to space constraints, some of the proofs in this section are relegated in the appendix.

\subsection{Initial Attempts: \MES and \NDMES}
Recall that \eRFC rule distributes the entire value to the first player who contributes to the new marginal value. Instead of assigning the entire value to a single player, a more intuitive approach is to equally distribute the new marginal value among all existing players. This principle underpins our first rule: Marginal Equal Share (\MES) rule.

\begin{tcolorbox}[title=Marginal Equal Share (\MES) Rule,title filled]
	\textbf{Input}: $G=(N,v,\pi)$\\
	For each player $i$ in arriving order $\pi$, \\
	\tab Initialize $\phi(G,i)\leftarrow 0$.\\
	\tab For each agent $j\in N_{\pi \mid i}$, \\	
	\tab[1cm] Update $\phi(G, j) \leftarrow \phi(G, j) + \frac{1}{|N_{\pi \mid i}|} \MC(G^i, N_{\pi \mid i}\setminus\{i\},i)$.\\
	\textbf{Output}: $\phi(G)$.
\end{tcolorbox}

\begin{theorem}\label{MES_property}
	\MES rule satisfies S-STAY, EA, and PART. 
\end{theorem}

Unlike \DMC and \eRFC rules, which allocate the marginal value to a single player, \MES rule adopts an egalitarian approach by distributing the value equally among all members of the grand coalition. It satisfies several desirable properties, including S-STAY, EA, and PART. Please refer to the appendix for proof details.

However, \MES rule fails to satisfy the fairness axiom, i.e, Online-Dummy (OD). The value assignment might be perceived unfair as some players could be the ``free-riders", i.e., some dummy players share the value as well. To deal with it, we refine \MES rule into the Non-Dummy Marginal Equal Share (\NDMES) rule.

\begin{tcolorbox}[title=Non-Dummy Marginal Equal Share (\NDMES) Rule,title filled]
	\textbf{Input}: $G=(N,v,\pi)$\\
	For each player $i$ in arriving order $\pi$, \\
	\tab Initialize $\phi(G,i)\leftarrow 0$. \\
	\tab Let $D_i \leftarrow$ dummy players in $G^i$ and $S_i \leftarrow N_{\pi \mid i} \setminus D_i$. \\
	\tab For each player $j$ in $S_i$: \\
	\tab[1cm] $\phi(G, j) \leftarrow \phi(G, j) + \frac{1}{|S_i|} \MC(G^i, N_{\pi \mid i}\setminus \{i\}, i) $.\\
	\textbf{Output}: $\phi(G)$.
\end{tcolorbox}

\begin{theorem}\label{NDMES_properties}
	\NDMES satisfies S-STAY, EA, PART, and OD.
\end{theorem}

By identifying all the dummy players in each prefix sub-game and equally sharing the value among non-dummy players, \NDMES satisfies OD axiom and maintains all other properties: S-STAY, EA, and PART. 

The primary challenge of \NDMES lies in the computation: identifying all the dummy players in each sub-game takes exponential time. However, for the subadditive valuation functions, we show that \NDMES is polynomial-time computable. 
\begin{proposition}\label{poly_time_NDMES_subadditive}
	Given an OCG $G=(N,v,\pi)$, if the valuation function $v(\cdot)$ is subadditive, \NDMES runs in poly-nominal time. 
\end{proposition}

\begin{proof}
	To identify all the dummy players in each prefix sub-game $G^i$, for every player $j\in N_{\pi \mid i}$, it is necessary to check every subset $S\subseteq N_{\pi \mid i}$ whether $v(S\cup\{j\})=v(S)$. When the valuation function is subadditive, we show $j$ is a dummy player in $G^i$ \textit{if and only if} $v(\{j\})=0$.\\ 
	$(\implies)$ Since $v(\cdot)$ is subadditive, for any subset $S$, $v(S\cup\{j\}) \leq v(S)+v(\{j\})=v(S)$. As $v(\cdot)$ is monotone, it means $v(S\cup\{j\})\geq v(S)$. Therefore, $v(S\cup\{j\})=v(S)$ for any subset $S$, i.e., $j$ is a dummy player.\\
	$(\impliedby)$ Assume $j$ is a dummy player, taking $S=\emptyset$, we have $v(\{i\})=v(\emptyset)=0$.
	
	So, the computation of dummy players in each sub-game $G^i$ is solvable in polynomial time as we only need to check whether $v(\{j\})=0$ for every $j\in N_{\pi \mid i}$. 
\end{proof}

\subsection{Desirable Rules: \ULMES and \DULMES}\label{subsection_ulmes_dulmes}

In light of the computational challenges associated with \NDMES rule, we devise a polynomial-time computable value-sharing rule, which we term it “Upward Lexicographic Marginal Equal Share” (\ULMES) rule.

\begin{tcolorbox}[title=Upward Lexicographic Marginal Equal Share (\ULMES) Rule,title filled]
	\textbf{Input}: $G=(N,v,\pi)$\\
	\textbf{For} each player $i$ in arriving order $\pi$: \\
	\tab Initialize $\phi(G,i)\leftarrow 0$, $S_i \leftarrow N_{\pi \mid i}, \ell \leftarrow |S_i|$. \\
	\tab \textbf{While} $\ell > 0$ \\
	\tab[1cm] \textbf{If} $v(S_i \setminus \{\pi(\ell)\} \cup \{i\}) = v(N_{\pi \mid i})$:\\
	\tab[1.5cm] Update $S_i\leftarrow S_i\setminus \{ \pi(\ell)\}$ \\
	\tab[1cm] $\ell \leftarrow \ell - 1$ \\
	\tab For $j\in S_i$, \\
	\tab[1cm] $\phi(G,j) \leftarrow \phi(G, j) + \frac{1}{|S_i|}\MC(G^i, N_{\pi \mid i}\setminus \{i\},i)$.\\
	\textbf{Output}: $\phi(G)$.
\end{tcolorbox}

For a prefix sub-game $G^i$, rather than distributing the value among all non-dummy players as in \NDMES rule, \ULMES allocates the value only to those non-dummy players who contribute to the new marginal value created by the newly arriving player.

\ULMES determines a set of non-dummy players $S_i$ to share the marginal value as follows: Initialize $S_i$ to be $N_{\pi \mid i}$. Check all agents in $N_{\pi \mid i} $ in an upward lexicographic order. Starting from player $i$ (i.e., $\pi(\ell)$), for player $\pi(\ell)$, if the coalition $S_i \setminus \{\pi(\ell)\} \cup \{i\}$ provides the same value as the grand coalition $N_{\pi \mid i}$, then agent $\pi(\ell)$ is considered dispensible for creating the marginal value and player $\pi(\ell)$ is removed from $S_i$. Repeat this procedure with $\ell \leftarrow \ell - 1$ until the list is exhausted. Intuitively, when player $i$ arrives, there can be multiple coalitions can provide the same marginal value and \ULMES selects \textbf{\textit{the player set $S_i$ such that the last arriving player is the earliest}} among all such coalitions providing the same marginal value in $G^i$.
\ULMES is polynomial-time computable as the computation of $S_i$ is in polynomial time. Also, it satisfies axioms including S-STAY, PART and OD in general.

\begin{theorem}\label{ULMES_properties}
	\ULMES satisfies S-STAY, PART, and OD.
\end{theorem}

Unfortunately, \ULMES does not satisfy EA in general. The following proposition explains why it fails to meet the EA criterion with general valuation functions.
\begin{proposition}\label{ULMES_NOT_EA}
	\ULMES does not satisfy EA in general.
\end{proposition}

\begin{proof}
	In the following example, \ULMES fails to satisfy EA, $G_1=(N,v,\pi_1)$, $N=\{1,2,3,4\}$, $\pi=(1,2,3,4)$ and $G_2=(N,v,\pi_2)$ where $\pi_2=(1,2,4,3)$. The coalition valuations are as follows:
	\[
	\begin{aligned}
		& v(\{1\})=v(\{2\})=v(\{3\})=v(\{4\}) = 0;\\
		& v(\{1,2\}) = v(\{1,4\})=v(\{2,4\}) =v(\{1,2,4\}) =0;\\
		& 0 \leq v(\{1,3\}) \leq v(\{2,3\}) < v(\{1,2,3\}) = x \, (x > 0);\\
		& v(\{3,4\}) = v(\{1,3,4\})=v(\{2,3,4\})=v(\{1,2,3,4\})=y;\\
	\end{aligned}
	\]
	Let $y > x$, now we focus on the valued shared by player $3$ in $G_1$ and $G_2$. For $G_1$, when player $3$ arrives, player $1,2,3$ share $x$ equally, $\phi(G_1,3)=\frac{x}{3}$, when player $4$ comes, the new marginal value $y-x$ is shared by $3$ and $4$: $\phi(G_1,3)=\frac{x}{3} + \frac{y-x}{2}$. However, in $G_2$, when $3$ delays her arrival, $1,2,4$ share no value as $v(\{1,2,4\})=0$. When $3$ arrives in $\pi_2$, the value $y$ is only shared by $3$ and $4$, in which $\phi(G_2,3)=\frac{y}{2}$. In this case, $\phi(G_2,3)=\frac{y}{2} > (\frac{x}{3}+\frac{y-x}{2})=\phi(G_1,3)$. \ULMES fails to satisfy EA.
\end{proof}

Intuitively, \ULMES fails to satisfy EA in general because a player might choose to delay their arrival to become a contributional player who shares a larger marginal value with fewer players. Although this player might forfeit some shared value from previous timesteps, the potential gain from the new marginal value can outweigh the losses, thereby undermining the EA property.

Although \ULMES fails to satisfy the EA property in general, we next show that it adheres to the EA axiom within every simple online cooperative games (SOCG).

\begin{lemma}\label{ULMES_EA_SIMPLE}
	\ULMES satisfies EA for every SOCG.
\end{lemma}

\begin{proof}
	Consider an SOCG $G_1=(N,v,\pi_1)$ where $\pi_1 = (1, 2, \ldots, i-1, \textbf{i}, i+1, \ldots, n)$. Let $q$ denote the pivotal player in $G_1$, that is, in sub-game $G^q$, $v(N_{\pi_1\mid q} \setminus \{q\})=0$ and $v(N_{\pi_1\mid q})=1$. Let $G_2=(N,v,\pi_2)$, where $\pi_2=(1, 2, \ldots, i-1, i+1, \ldots,j,\textbf{i},\ldots, n)$ represents the arriving order in which all other players' arriving orders are fixed and $i$ delays her arrival. For $i$, there are three cases: $q \prec_{\pi_1} i$,  $q = i$ and $i \prec_{\pi_1} q$. 
	
	\noindent \textbf{Case 1: $q \prec_{\pi_1} i$}. 
	If a coalition with value $1$ has formed before $i$’s arrival in $\pi_1$, $i$ shares no value in both $\pi_1$ and $\pi_2$ and $\phi(G_1,i) = \phi(G_2,i) = 0$. 
	
	\noindent \textbf{Case 2: $q = i$}. If $i$ is the pivotal player in $G_1$, there are two possible cases if $i$ delays her arrival.\\
	\textbf{(a).} $i$ loses her pivotal role in $\pi_2$, i.e., some player in $\{i+1, i+2, \ldots, j\}$ becomes the pivotal player. Then, $\phi(G_1,i) > \phi(G_2,i) = 0$. Player $i$ has no incentive to delay. \\
	\textbf{(b).} $i$ remains to be pivotal in $\pi_2$. Let $S_i^1$ be the player set sharing value $1$ in $G_1$. When $i$ delays in $\pi_2$, there will be no change of $S_i^1\setminus \{i\}$ which creates the new marginal value along with $i$. Thus, \ULMES will eliminate all the players in $\{i+1, \ldots, j\}$ (because of the existence of $S_i^1\setminus\{i\}$) in $G_2$. Hence, players in $S_i^1$ still share the value $1$ in $G_2$. Player $i$ receives the same value in $G_1$ and $G_2$. Therefore, $i$ has no incentive to delay in case 2.
	
	\noindent \textbf{Case 3: $i \prec_{\pi_1} q$} Denote $S_q^1$ as the set of players among whom the value $1$ is shared. There are two possible cases. \\
	\textbf{(a).} $i \notin S_q^1$, i.e., $\phi(G_1,i)=0$. $i \notin S_q^1$ is \textit{either} because there exists no coalition including $i$ along with $q$ creating value $1$ \textit{or} because $i$ is in some coalition creating value $1$ with $q$, however, eliminated by \ULMES rule. In the former case, \textbf{i)} $i$ delays between $i+1$ and $q$, there is still no coalition including $i$ can creating value $1$ with $q$; \textbf{ii)} $i$ delays after $q$'s arrival, which makes no change for the value sharing as $i$'s delay does not influence $S_q^1$; For the latter case, it implies that among players $\{1,2,\ldots, i,\ldots, q-1\}$, there are multiple coalitions with $q$ creating value $1$ and $i$ is in one of these coalitions, denoting it by $\bar{S}_q^1$. However, $\bar{S}_q^1$ is eliminated because of the existence of $S_q^1$. Since $i \notin S_q^1$, $i$'s any delay strategy in $\pi_2$ has no effect on $S_q^1$, keeping $\phi(G_2,i)=\phi(G_1,i)=0$.\\
	\textbf{(b).} $i \in S_q^1$, i.e., $\phi(G_1,i)=\frac{1}{|S_q^1|}$. There are two different situtations:
	\textbf{i)} $S_q^1$ is the unique coalition creating value $1$ in $G^q$. If $i$ delays between $i+1$ and $q$, it does not change value sharing in $\pi_2$ and $\phi(G_2,i)=\phi(G_1,i)=\frac{1}{|S_q^1|}$; If $i$ delays the arrival after $q$, one case is some players in $\pi_2$ between $q$ and $i$, along with some players in $\{1,2,\ldots, i-1,i+1,\ldots, q\}$ construct a coalition with value $1$, then $\phi(G_2,i)=0$; the other case is that after $i$'s delay, $i$ becomes the pivotal player. However, $i$ will still share the value in $S_q^1$ as all the other such coalitions creating value $1$ will be eliminated by \ULMES because of the existence of $S_q^1$. Therefore, $\phi(G_2,i)=\phi(G_1,i)=\frac{1}{|S_q^1|}$. 
	\textbf{ii)} there are multiple coalitions including some players in $\{1,2,\ldots, q-1\}$ creating value $1$ with $q$ and $S_q^1$ is the coalition surrives in \ULMES rule. If $i$ delays her arrival between $i+1$ and $q$, it could be either $S_q^1$ is still the coalition to share the value ($\phi(G_2,i)=\phi(G_1,i)=\frac{1}{|S_q^1|}$) or because of $i$'s delay, $S_q^1$ get eliminated in \ULMES rule, making some other coalition surrvies ($\phi(G_1,i) > \phi(G_2,i) = 0$); If $i$ delays her arrival after $q$, $\phi(G_2,i)=0$ as there exists some other coalition creating and sharing the value $1$.
\end{proof}

As previously mentioned, the greedy monotone (GM) decomposition method proposed by \citet{GZZ+24a} extends \RFC to \eRFC without compromising axiomatic satisfaction. By leveraging the GM decomposition, we extend our \ULMES to a new rule called \DULMES. Details of the GM decomposition are provided in the appendix

\begin{tcolorbox}[title=extended ULMES (\DULMES) Rule, title filled]
	\textbf{Input}: $G=(N,v,\pi)$\\
	$\forall\, i\in N$, initialize $\phi(G,i)\leftarrow 0$.\\
	Decomposition $D(G) \leftarrow \mathtt{GM ~ Decomposition}(G)$.\\
	For each component $(c, \bar{G})$ in $D(G)$: \\
	\tab $(\phi(\bar{G}, 1),\ldots, \phi(\bar{G}, n)) \leftarrow$ \ULMES($\bar{G}$) \\
	\tab For each player $i \in N$:\\
	\tab[1cm] $\phi(G,i) \leftarrow \phi(G,i) + c \cdot \phi(\bar{G},i)$.\\
	\textbf{Output}: $\phi(G)$.
\end{tcolorbox}

GM decomposition method takes the OCG $G$ as input and outputs a linear combination of component, denoted as $D(G)$. For each component $\{(c,\bar{G})\}$, $c$ is the coefficient while $\bar{G}$ is an SOCG. \DULMES runs \ULMES in each SOCG and aggreates the weighted outcomes of \ULMES as the output.

\begin{theorem}\label{DULMES_properties}
	\DULMES satisfies S-STAY, EA, PART, and OD.
\end{theorem}

To better compare \ULMES and \DULMES rules, we introduce the following example with $4$ players in which \ULMES rule fails to satisfy EA while \DULMES rule satisfies the axiom. 

\begin{example}[\ULMES and \DULMES rules]\label{example:rules3}
	Consider an OCG $G=(N,v,\pi)$ where $N=\{1,2,3,4\}$, $\pi=(1,2,3,4)$. For the valuation function $v(\cdot)$, all the coalition valuation are enumerated in the first row of \Cref{example_decompose_game}. 
	\begin{table*}[!htbp]
		\centering
		\resizebox{1\textwidth}{!}{
			\begin{tabular}{ccccccccccccccccc}
				\toprule
				Coalition & Coef & $\{1\}$ & $\{2\}$ & $\{3\}$ & $\{4\}$ & $\{1,2\}$ & $\{1,3\}$ & $\{1,4\}$ & $\{2,3\}$ & $\{2,4\}$    & $\{3,4\}$ & $\{1,2,3\}$ & $\{1,2,4\}$ & $\{1,3,4\}$ & $\{2,3,4\}$ & $\{1,2,3,4\}$ \\ \midrule
				Valuation  & & $0$     & $0$     & $0$     & $0$     & $0$       & $1$       & $0$      & $1$   & $0$       & $3$       & $2$         & $0$         & $3$         & $3$ & $3$    \\ \midrule
				$\bar{v}_1(\cdot)$ &  $c_1=1$  & $0$     & $0$     & $0$     & $0$     & $0$       & $1$       & $0$       & $1$       & $0$       & $1$       & $1$         & $0$         & $1$         & $1$ & $1$         \\ \midrule
				$\bar{v}_2(\cdot)$ &  $c_2=1$     & $0$     & $0$     & $0$     & $0$     & $0$      & $0$       & $0$       & $0$       & $0$      & $1$       & $1 $        & $0$         & $1$         & $1$ & $1$         \\ \midrule
				$\bar{v}_3(\cdot)$ &  $c_3=1$     & $0$     & $0$     & $0$     & $0$     & $0$       & $0$       & $0$       & $0$       & $0$       & $1$       & $0$         & $0 $        & $1$         & $1$   & $1$      \\ \bottomrule
		\end{tabular}}
		\caption{Coalition valuation and decomposition of the OCG $G$ with $3$ components $(1,\bar{G}_1),(1,\bar{G}_2),(1,\bar{G}_3)$.}
		\label{example_decompose_game}  
	\end{table*}
	
	Under the \ULMES rule, during the first two timesteps, players $1$ and $2$ arrive but generate no value. When player $3$ joins, the coalition $\{1,2,3\}$ generates a value of $2$, and no other sub-coalition achieves this value. Consequently, the three agents share the total value equally, each receiving $\frac{3}{2}$. Upon the arrival of player $4$, the grand coalition achieves a total value of $3$, with a marginal contribution of $1$. To determine $S_4$, we start from player $4$. Removing player $4$ reduces the value to $2$, indicating that player $4$ must be included in $S_4$ Similarly, player $3$ must also be in $S_4$ since $v(\{1,2,4\})=0$. Conversely, player $1$ and $2$ are excluded from $S_4$ since $v(\{1,3,4\})=v(\{3,4\})=3$. Hence, the marginal contributional value of $1$ is evenly distributed between player $3$ and $4$, each receiving $\frac{1}{2}$. The final value distribution is $\phi(G)=(\frac{2}{3}, \frac{2}{3},\frac{7}{6},\frac{1}{2})$. Now consider the OCG $G'$, where player $3$ delays her arrival, changing the arriving order from $\pi=(1,2,3,4)$ to $\pi'=(1,2,4,3)$. In this scenario, player $3$ strategically delays her arrival to ensure that she shares the entire value of $3$ solely with player $4$, resulting in each receiving $\frac{3}{2}$. \Cref{ULMES_output_example} summarizes the outcomes of \ULMES rule under $G$ and $G'$, illustrating a case where \ULMES rule fails to satisfy the EA axiom.
	\begin{table}[!htbp]
		\centering
		\scalebox{1}{%
			\begin{tabular}{ccccc}
				\toprule
				$\phi(G,i)$            & $1$    & $2$     & $3$     & $4$     \\ \midrule
				$\pi=(1,2,3,4)$        & $2/3$ & $2/3$ & $\mathbf{7/6}$ & $1/2$ \\ \midrule
				$\pi^\prime=(1,2,4,3)$ & $0$   & $0$   & $\mathbf{3/2}$ & $3/2$ \\ \midrule
			\end{tabular}
		}
		\caption{\ULMES rule outcome in $G$ and $G'$}
		\label{ULMES_output_example}
	\end{table}
	
	Regarding \DULMES rule, firstly, since the GM decomposition is independent with the arriving order, OCG $G$ and $G'$ share the same game decomposition components shown in \Cref{example_decompose_game}, i.e., decomposing $G$ into three components $D(G)=\{(1,\bar{G}_1),(1,\bar{G}_2),(1,\bar{G}_3)\}$ with three distinct valuation functions $\bar{v}_1(\cdot), \bar{v}_2(\cdot)$, and $\bar{v}_3(\cdot)$. After the decomposition, we run \ULMES rule in each decomposed component in $G$ and $G'$, respectively. Taking $G$ as the example, under the arriving order $\pi=(1,2,3,4)$, for $\bar{G}_1$, player $1$ and $2$ arrive with no value sharing, when player $3$ comes, we compute $S_3=\{1,3\}$ since $\bar{v}_1(\{1,3\})=1$, then player $1$ and $3$ each shares $\frac{1}{2}$. Similarly, we can compute the value distribution in $\bar{G}_2$ and $\bar{G}_3$. The results are demostrated in \Cref{eulmes_outcome_g}.
	
	\begin{table}[!htbp]
		\centering
		\begin{tabular}{ccccc}
			\toprule
			$\phi(G,i)$ & $1$      & $2$     & $3$     & $4$     \\ \midrule
			$\bar{G}_1$       & $1/2$  & $0$   & $1/2$ & $0$   \\ \midrule
			$\bar{G}_2$       & $1/3$  & $1/3$ & $1/3$ & $0$   \\ \midrule
			$\bar{G}_3$       & $0$    & $0$   & $1/2$ & $1/2$ \\ \midrule
			$G$         & $5/6$  & $1/3$ & $4/3$ & $1/2$ \\ \bottomrule
		\end{tabular}
		\caption{\DULMES rule outcome in $G$}
		\label{eulmes_outcome_g}
	\end{table}
	
	When player $3$ delays her arrival in $G'$, \DULMES rule outputs the following results shown in \Cref{comparison_de_ulmes}.
	\begin{table}[!htbp]
		\centering
		\scalebox{1}{%
			\begin{tabular}{ccccc}
				\toprule
				$\phi(G^\prime,i)$  & $1$      & $2$     & $4$       & $3$   \\ \midrule
				$\bar{G}'_1$    & $1/2$  & $0$   & $0$   & $1/2$ \\ \midrule
				$\bar{G}'_2$      & $1/3$  & $1/3$ & $0$   & $1/3$ \\ \midrule
				$\bar{G}'_3$     & $0$    & $0$   & $1/2$   & $1/2$ \\ \midrule
				$G^\prime$  &  $5/6$  & $1/3$ & $1/2$ & $4/3$ \\ \bottomrule
			\end{tabular}
		}
		\caption{\DULMES rule outcome in $G^\prime$}
		\label{comparison_de_ulmes}
	\end{table}
	For the three components with $\bar{G}_1, \bar{G}_2$, and $\bar{G}_3$, we omit the coefficiency for each SOCG as their coefficients are all the same $1$ in this example. It is not hard to see that player $3$ has no incentive to delay her arrival when we apply \DULMES rule.
\end{example}

\section{Individual Rationality and Superadditivity} \label{individual_rational_section}
In the previous sections, we explored participation incentives in terms of EA, S-STAY, and PART. Now, we turn our attention to the individual rationality (IR) axiom. The IR axiom captures the participation incentive by stipulating that a player will only join the grand coalition if the value they receive from the coalition exceeds their singleton valuation.

We first present an impossibility result regarding the satisfaction of the IR axiom in online cooperative games.
\begin{proposition}[Impossibility]
	There is no value sharing rule satisfying IR for OCG with general valuation.
\end{proposition}

\begin{proof}
	Consider the following example, $G=(N,v,\pi)$, $N=\{1,2\}$, $v(\{1\})=2,v(\{2\})=3, v(\{1,2\})=4$, $\pi=(1,2)$. In order to satisfy IR, when player $1$ comes, $v(\{1\})=2$ should be allocated to player $1$, however, when player $2$ joins the coalition, the largest value that can be shared by player $2$ is $4-2=2 < v(\{2\})$, violating IR.
\end{proof}

In view of this impossibility result, we restrict the valuation to \textit{superadditive} domain. For an OCG $G=(N, v,\pi)$, we say $G$ is a superadditive OCG if the valuation function is superadditive. Next, we focus on superadditive OCGs. We begin by verifying that \DMC and \SV satisfy IR, while \eRFC does not. We then introduce an IR refinement paradigm to modify all three new rules while preserving all other axiomatic satisfaction.


\begin{proposition}\label{RFC_is_not_IR}
	In OCGs with superadditive valuations, \DMC and \SV satisfy IR while \eRFC does not.
\end{proposition}

For \MES, \NDMES, and \DULMES rules, we propose the following IR refinement paradigm to modify these rules. 

\begin{tcolorbox}[title=IR Refinement Paradigm, title filled]
	\textbf{Input}: $G=(N,v,\pi)$, value-sharing rule $\phi$ \\
	$\forall\, i \in N$, initialize $\hat{\phi}(G, i) \leftarrow 0$. \\
	For $i$ in arriving order $\pi$:\\
	\tab \textbf{$\hat{\phi}(G,i) \leftarrow v(\{i\})$}.\\
	\tab Compute $S_i$ by the value-sharing rule $\phi$.\\
	\tab For each player $j \in S_i$:\\
	\tab[1cm] $\hat{v}_i \leftarrow \MC(G^i,N_{\pi \mid i} \setminus \{i\}, i) - v(\{i\})$.\\ 
	\tab[1cm]  $\hat{\phi}(G,i)\leftarrow \hat{\phi}(G,i) + \frac{1}{|S_i|}\hat{v}_i$.\\
	\textbf{Output}: $\hat{\phi}(G)$.
\end{tcolorbox}

Denote the refined rules with an IR prefix, for example: from \DULMES to IR-\DULMES rule. This refinement ensures that the IR axiom is satisfied while preserving all other axioms, including S-STAY, EA, PART, and OD.

\begin{theorem}\label{maintain_IR_property}
	IR-\MES, IR-\NDMES, and IR-\DULMES rules satisfy the IR axiom while preserving the satisfaction of all axioms held before the refinement.
\end{theorem}

\section{Conclusions}
In this paper, we propose new participation incentive axioms for the emerging online cooperative games, including PART, S-STAY, and IR. We demonstrate that existing value-sharing rules fail to meet these properties and often cannot satisfy EA for general valuations. In contrast, our newly devised value-sharing rules successfully satisfy these axioms. An immediate open question is whether there exist polynomial-time sharing rules that simultaneously satisfy EA, PART, S-STAY, OD, and IR. Additionally, as with classical cooperative game theory, achieving a characterization of rules with respect to axiomatic properties remains an intriguing challenge. Exploring cost-sharing games or hedonic games in an online context could also be a promising direction for future research.


\bibliographystyle{named}
\bibliography{reference}

\newpage 
\section*{Appendix}

\subsection{Omitted Proofs For \Cref{new_desirable_rules}}
\subsubsection{Proof of \Cref{MES_property}}
\begin{proof}
	\textbf{(S-STAY)}. \MES rule directly satisfies STAY as each agent's sharing value is non-decreasing. For S-STAY, now consider an OCG $G=(N,v,\pi)$, for any prefix sub-game $G^i$ with arriving player $i$ and $i$ is a contributional player, for every player $j \in N_{\pi \mid i} \setminus \{i\}$ such that $v(N_{\pi \mid i}) > v(N_{\pi \mid i} \setminus \{j\})$, we have $\phi(G^j,j) = \frac{1}{|N_{\pi \mid j}|}\MC(G^j,N_{\pi \mid j} \setminus \{j\}, j)$ and 
	\begin{align*}
		\phi(G^i,j) &= \phi(G^j,j) + \sum_{k=j+1}^{i-1} \frac{1}{|N_{\pi \mid k}|}\MC(G^k, N_{\pi \mid k}\setminus \{k\},k) + \frac{1}{|N_{\pi \mid i}|}\MC(G^i,N_{\pi \mid i} \setminus \{i\}, i).
	\end{align*}
	Since $i$ is a contributional player for $G^i$, i.e., $\MC(G^i,N_{\pi \mid i} \setminus \{i\}, i) > 0$, we can see $\phi(G^i,j) > \phi(G^j,j)$, implying \MES rule satisfies S-STAY. \\
	\textbf{(EA)}. Consider any player $i\in N$ and two OCGs $G_1=(N,v,\pi_1)$ and $G_2=(N,v,\pi_2)$ with two different arriving orders $\pi_1$ and $\pi_2$.
	\[
	\begin{aligned}
		\pi_1: (1, 2, \ldots, i - 1, \mathbf{i} , i + 1, \ldots, j - 1, j, j + 1, \ldots, n),\\
		\pi_2: (1, 2, \ldots, i - 1, i + 1, \ldots, j - 1, j, \mathbf{i}, j + 1, \ldots, n).
	\end{aligned}
	\]
	In $\pi_2$, player $i$ delays her arrival between player $j$ and $j+1$. Next, we show that $i$ has no incentive to delay her arrival.
	For each prefix sub-game $G_1^i$ within $G_1$, the value shared by $i$ can be represented as
	\begin{equation}
	\phi(G_1,i)=\sum_{k=i}^n \frac{1}{|N_{\pi_1\mid k}|}\MC(G_1^k,N_{\pi_1 \mid k}\setminus \{k\},k).
	\end{equation}
	In $G_2$, the value shared by $i$ can be represented as
	\begin{equation}
	\phi(G_2,i)= \frac{1}{|N_{\pi_2\mid i}|}\MC(G_2^i,N_{\pi_2\mid j},i) + \sum_{k=j+1}^n \frac{1}{|N_{\pi_2\mid k}|}\MC(G_2^k,N_{\pi_2\mid k}\setminus \{k\},k).
	\end{equation}
	Starting from player $(j+1)$'s arrival, we know that $N_{\pi_1\mid q} = N_{\pi_2 \mid q}$ for every agent $q\prec_{\pi_1} j, q\in N$. That is, for agent $i$, $\sum_{k=j+1}^n \frac{1}{|N_{\pi_1\mid k}|}\MC(G_1^k,N_{\pi_1\mid k}\setminus \{k\},k)=\sum_{k=j+1}^n \frac{1}{|N_{\pi_2\mid k}|}\MC(G_2^k,N_{\pi_2\mid k}\setminus\{k\},k)$. So the difference between $\phi(G_1,i)$ and $\phi(G_2,i)$ can be written as
	\begin{equation}
	\begin{aligned}
		\phi(G_1,i)-\phi(G_2,i)=&\sum_{k=i}^{j}\frac{1}{|N_{\pi_1\mid k}|} \MC(G_1^k, N_{\pi_1\mid k}\setminus\{k\},k) - \frac{1}{|N_{\pi_2\mid i}|}\MC(G_2^i,N_{\pi_2\mid i}\setminus \{i\},i) \\
		\geq& \frac{1}{|N_{\pi_2\mid i}|} \sum_{k=i}^j \MC(G_1^k,N_{\pi_1 \mid k} \setminus \{k\},k) - \frac{1}{|N_{\pi_2\mid i}|}\MC(G_2^i,N_{\pi_2\mid i}\setminus\{i\},i) \\
		=& \frac{1}{|N_{\pi_2\mid i}|} \Big( v(N_{\pi_1\mid j})-v(N_{\pi_1\mid i} \setminus \{i\} ) - \MC(G_2^i,N_{\pi_2\mid i} \setminus \{i\} ,i) \Big).
	\end{aligned}
	\end{equation}
	Note that $N_{\pi_2\mid i} \setminus \{i\}  = \{1,2,\ldots, i-1,i+1,\ldots,j\}=N_{\pi_1\mid j}\setminus \{i\}$. Therefore, 
$
\MC(G_2^i, N_{\pi_2\mid i} \setminus \{i\} ,i)
		= v(N_{\pi_2\mid i}) - v(N_{\pi_2\mid i} \setminus \{i\} ) 
		= v(N_{\pi_2\mid i}) - v(N_{\pi_1\mid j}\setminus \{i\}) 
		= v(N_{\pi_1\mid j}) - v(N_{\pi_1\mid j}\setminus \{i\}).
$
	After substituting the above expression,
	\begin{equation}
	\phi(G_1,i) - \phi(G_2,i) = \frac{1}{|N_{\pi_2\mid i}|} \Big( v(N_{\pi_1\mid j}\setminus \{i\}) - v(N_{\pi_1\mid i} \setminus \{i\} )  \Big) \geq 0.
	\end{equation}
	The last inequality holds because of the monotonicity of valuation function. Since $N_{\pi_1\mid j}\setminus \{i\}=\{1,2,\cdots, i-1,i+1,\cdots, j\}$ and $N_{\pi_1\mid i} \setminus \{i\} =\{1,2,\cdots, i-1\}$, we have $N_{\pi_1\mid i} \setminus \{i\}  \subset N_{\pi_1\mid j}\setminus \{i\}$, which implies $v(N_{\pi_1\mid j}\setminus \{i\}) \geq v(N_{\pi_1\mid i} \setminus \{i\} )$. Hence, \MES rule satisfies EA axiom.
	
	\noindent \textbf{(PART)} Consider an OCG $G=(N,v,\pi)$, for each player $i$ in the arriving order $\pi$ and the prefix sub-game $G^i$, if $i$ is a contributional player, then $\MC(G^i, N_{\pi \mid i} \setminus \{i\},i) > 0$, implying $\phi(G^i,i)=\frac{1}{|N_{\pi\mid i}|}\MC(G^i,N_{\pi \mid i} \setminus \{i\}, i) > 0$. Hence, agent $i$ receives a positive shared value immediately after her arrival, in accordance with the PART axiom.
\end{proof}

\subsubsection{Proof of \Cref{NDMES_properties}}

\begin{proof}
	\textbf{(S-STAY)} Firstly, STAY directly gets satisfied as each agent's shared value is non-decreasing under \NDMES rule. For S-STAY, now consider an OCG $G=(N,v,\pi)$, for any prefix sub-game $G^i$ with arriving player $i$ and $i$ is a contributional player, for every player $j \in N_{\pi \mid i} \setminus \{i\}$ such that $v(N_{\pi \mid i}) > v(N_{\pi \mid i} \setminus \{j\})$, then $j$ is non-dummy in the sub-game $G^i$, which implies that $j$ is included in $S_i$ and share the value $\frac{1}{|S_i|}\MC(N_{\pi \mid i},i)>0$, i.e., for any $j$ satisfying $v(N_{\pi \mid i})> v(N_{\pi \mid i}\setminus\{j\})$, we have $\phi(G^i,j) > \phi(G^j,j)$. Therefore, \NDMES rule satisfies S-STAY.
	
	\noindent\textbf{(EA)} Consider two OCGs $G_1=(N,v,\pi_1)$ and $G_2=(N,v,\pi_2)$ where $\pi_1$ and $\pi_2$
	\[
	\begin{aligned}
		\pi_1: (1, 2, \ldots, i - 1, \mathbf{i} , i + 1, \ldots, j - 1, j, j + 1, \ldots, n),\\
		\pi_2: (1, 2, \ldots, i - 1, i + 1, \ldots, j - 1, j, \mathbf{i}, j + 1, \ldots, n).
	\end{aligned}
	\]
	According to the role of agent $i$ under $\pi_1$ and $\pi_2$, we consider the following four cases. 
	
	\noindent\textbf{Case 1.} For both $\pi_1$ and $\pi_2$, player $i$'s contributes zero marginal value upon arrival. For the shared value after player $j+1$'s arrival, it is the same for this two orders. The only difference is that player $i$ might gain some shared-value when $i$ is not a dummy player for sub-games from $G^{i+1}$ to $G^{j}$. Therefore, the value shared by player $i$ in $\pi_1$ will always at least as that in $\pi_2$. 
	
	\noindent\textbf{Case 2.} For $\pi_1$, player $i$ is a contributional player while for $\pi_2$, player $i$ is not. Then, for $\pi_1$, $i$ will never be a dummy player in all sub-games after her arrival, thus $i$ shares all the marginal value from $i$ to $n$. However, in $\pi_2$, player $i$ only shares marginal value from player $j+1$ to player $n$.
	
	\noindent\textbf{Case 3.} For both $\pi_1$ and $\pi_2$, player $i$ are a contributional players. In the OCG $G_1$, the value shared by player $i$ can be represented as
	\begin{equation}
		\phi(G_1,i) =\frac{1}{|S^1_i|} \MC(G_1^i, N_{\pi_1\mid i} \setminus \{i\} , i) + \sum_{k=i+1}^n \frac{1}{|S^1_k|} \MC(G_1^k, N_{\pi_1 \mid k} \setminus \{k\}, k).
	\end{equation}
	Here $S_i^1$ represents the player set $S_i$ for player $i$ in OCG $G_1$. In the OCG $G_2$, the value shared by player $i$ can be represented as 
	\begin{equation}
		\phi(G_2,i) = \frac{1}{|S^2_i|}\MC(G_2^i,N_{\pi_2\mid i} \setminus \{i\} , i) + \sum_{k = j + 1}^n \frac{1}{|S_k^2|}\MC(G_2^k,N_{\pi_2 \mid k}\setminus \{k\}, k).
	\end{equation}
	Here $S_i^2$ represents the player set $S_i$ for player $i$ in OCG $G_2$. Also, after player $j$'s arrival, for any player $k, j\prec_{\pi_1} k$, we have $|S_k^1|=|S_k^2|$ as the prefixes $N_{\pi_1 \mid k}$ and $N_{\pi_2 \mid k}$ become exactally the same in $\pi_1$ and $\pi_2$. Then, the difference between $\phi(G_1,i)$ and $\phi(G_2,i)$ can be written as
	\begin{equation}
		\begin{aligned}
			&\phi(G_1,i) - \phi(G_2,i)\\
			=& \frac{1}{|S^1_i|} \Big(v(N_{\pi_1\mid i}) - v(N_{\pi_1 \mid j} \setminus \{i\}) \Big)
			+ \sum_{k=i+1}^{j} \frac{1}{|S^1_k|}  \Big(v(N_{\pi_1\mid k}) - v(N_{\pi_1 \mid k} \setminus \{k\}) \Big) \\
			&- \frac{1}{|S^2_i|} \Big(v(N_{\pi_1 \mid j}) -  v(N_{\pi_1 \mid j - 1}\setminus \{i\}) \Big).
		\end{aligned}
	\end{equation}
	For any arriving order $\pi$, by the definition of the dummy player, if a player is non-dummy in sub-game $G^i$, then the player is non-dummy for the later sub-games $G_j, i\prec_{\pi} j$, thus the number of player set sharing the value is non-decreasing with the increasing number of arrived players, i.e., $S_i \subseteq S_j, ~\forall i \prec_{\pi} j$. Thus we have
	\begin{equation}
		\begin{aligned}
			&\frac{1}{|S^1_i|} \Big(v(N_{\pi_1\mid i}) - v(N_{\pi_1 \mid j} \setminus \{i\}) \Big)
			+ \sum_{k=i+1}^{j} \frac{1}{|S^1_k|}  \Big(v(N_{\pi_1\mid k}) - v(N_{\pi_1 \mid k} \setminus \{k\}) \Big) \\
			\geq& \frac{1}{|S^1_{j}|} \Big[\sum_{k=i}^{j} \Big( v(N_{\pi_1 \mid k}) - v(N_{\pi_1 \mid k}\setminus \{k\}) \Big)\Big] = \frac{1}{|S^1_{j}|}\Big( v(N_{\pi_1 \mid j}) - v(N_{\pi_1 \mid j} \setminus \{i\}) \Big).
		\end{aligned}
	\end{equation}
	Another observation is that $S_j^1 = S_i^2$ as they both represent the set of non-dummy players within the player set $\{1,2,\ldots, i-1,i,i+1, \ldots, j\}$. Therefore,
	\begin{equation}
		\begin{aligned}
			&\phi(G_1,i) - \phi(G_2,i) \\
			\geq & \frac{1}{|S^1_{j}|}\Big( v(N_{\pi_1 \mid j}) - v(N_{\pi_1 \mid j} \setminus \{i\}) \Big) -  \frac{1}{|S^2_i|} \Big(v(N_{\pi_1 \mid j}) -  v(N_{\pi_1 \mid j}\setminus \{i,j\}) \Big) \\
			=& \frac{1}{|S^1_{j}|} \Big(v(N_{\pi_1 \mid j}\setminus \{i,j\}) - v(N_{\pi_1\mid i} \setminus \{i\} ) \Big) \geq 0.
		\end{aligned}
	\end{equation}
	The last step is from the monotone valuation function as $N_{\pi_1 \mid j}\setminus \{i,j\}=\{1,2,\ldots, i-1, i+1,\ldots, j\} \supset \{1,2,\ldots, i-1\} = N_{\pi_1 \mid j} \setminus \{i\}$ implies that $v(N_{\pi_1 \mid j}\setminus \{i,j\}) \geq v(N_{\pi_1 \mid j} \setminus \{i\})$.
	
	\noindent \textbf{Case 4.} For $\pi_1$, player $i$ is not a contributional player while for $\pi_2$, player $i$ is a contributional player. In this case, we consider the following two scenarios. 
	\textbf{i).} Player $i$ arrives as a dummy player and keeps to be dummy until the player $j$ comes ($i$ cannot be dummy after $j$'s arrival as we assume $i$ is a contributional player in $\pi_2$). Then, the player $i$'s shared value in $G_1$ and $G_2$ can be represented as the  
	\begin{equation}
	\begin{split}
		&\phi(G_1,i) - \phi(G_2,i)\\
		=& \frac{1}{|S^1_j|}\MC(G_1^j, N_{\pi_1 \mid j}, j) + \sum_{k=j +1}^n \frac{1}{|S^1_k|}\MC(G_1^k,N_{\pi_1 \mid k}, k) \\
		&-  \left(\frac{1}{|S^2_i|}\MC(G_2^i, N_{\pi_2\mid i} \setminus \{i\} , i) + \sum_{k=j +1}^n \frac{1}{|S^2_k|}\MC(G_2^k,N_{\pi_2 \mid k} \setminus \{k\}, k) \right).
	\end{split}
	\end{equation}
	Still, we have $S^1_j=S^2_i$ and player $i$ shares the same value in all the sub-games after agent $j$'s arrival. Then the difference between $\phi(G_1,i)$ and $\phi(G_2,i)$ can be written as
	\begin{equation}
	\begin{aligned}
		&\phi(G_1,i) - \phi(G_2,i) =\frac{1}{|S_j^1|} \Big( v(N_{\pi_1 \mid j} \setminus \{i\}) - v(N_{\pi_1 \mid j}\setminus\{j\}) \Big).
	\end{aligned}
	\end{equation}
	Since player $i$ is a dummy player in all sub-games before $j$'s arrival, we have $v(N_{\pi_1 \mid j}\setminus\{j\}) = v(N_{\pi_1 \mid j}\setminus \{i,j\})$. Hence, 
	\begin{equation}
	\phi(G_1,i) - \phi(G_2,i) = \frac{1}{|S_j^1|} \Big( v(N_{\pi_1 \mid j} \setminus \{i\}) - v(N_{\pi_1 \mid j} \setminus \{i,j\})\Big) \geq 0.
	\end{equation}
	\noindent \textbf{ii).} Player $i$ is dummy before some player $(i + q)$'s arrival, in this case, $\phi(G_1,i)$ can be represented as
	\begin{equation}
\phi(G_1,i) = \sum_{k=i+q}^{j-1} \frac{1}{|S^1_k|}\MC(G_1^k,N_{\pi_1 \mid k} \setminus \{k\}, k) + \frac{1}{|S_j^1|}\MC(G_1^j,N_{\pi_1 \mid j}\setminus\{j\}, j).
	\end{equation}
	By leveraging the same reduction technique in Case 3,
	\begin{equation}
	\begin{aligned}
		\phi(G_1,i) =& \sum_{k=i+q}^{j-1} \frac{1}{|S^1_k|}\MC(G_1^k,N_{\pi_1 \mid k} \setminus \{k\}, k) + \frac{1}{|S_j^1|}\MC(G_1^j,N_{\pi_1 \mid j}\setminus\{j\}, j) \\
		\geq & \frac{1}{|S_j^1|} \Big( v(N_{\pi_1 \mid j}) - v(N_{\pi_1 \mid i+q}\setminus\{i+q\}) \Big).
	\end{aligned}
	\end{equation}
	Furthermore, the difference between $\phi(G_1,i)$ and $\phi(G_2,i)$ can be written as
	\begin{align*}
		&\phi(G_1,i) - \phi(G_2,i) \\
		\geq &  \frac{1}{|S_j^1|} \Big( v(N_{\pi_1\mid j}) - v(N_{\pi_1 \mid i+q}\setminus\{i+q\}) \Big) - \frac{1}{|S^2_i|}\Big( v(N_{\pi_1 \mid j}) - v(N_{\pi_1 \mid j}\setminus \{i\}) \Big) \\
		\geq & \frac{1}{|S_j^1|} \Big( v(N_{\pi_1 \mid j} \setminus \{i\}) - v(N_{\pi_1 \mid i+q}\setminus\{i+q\})  \Big) \tag{$\because$ $|S_j^1|=|S_i^2|$} \\  
		= & \frac{1}{|S_j^1|} \Big( v(N_{\pi_1 \mid j} \setminus \{i\}) - v(N_{\pi_1 \mid i+q} \setminus \{i,i+q\})  \Big) \tag{$\because$ $i$ is dummy before $i+q$'s arrival}  \\
		\geq & 0.
	\end{align*}
	Notice that it still holds $S_j^1=S_i^2$. Also, we know that $v(N_{\pi_1 \mid i+q}\setminus \{i+q\}) = v(N_{\pi_1 \mid i+q}\setminus \{i,i+q\})$ as player $i$ is dummy for all the sub-games before player $(i+q)$'s arrival. We conclude that \NDMES rule satisfies EA axiom based on the discussion of aforementioned four cases.
	
	\noindent\textbf{(PART)} PART holds for \NDMES rule as for each arriving player $i$, when $i$ is a contributional player, i.e., $\MC(G^i, N_{\pi \mid i} \setminus \{i\}, i) > 0$, $i$ is not a dummy player and is included in the sharing set $S_i$. Hence, player $i$ immediately shares $\frac{1}{|S_i} \MC(G^i,N_{\pi \mid i} \setminus \{i\},i)$ upon her arrival. 
	
	\noindent\textbf{(OD)} Consider an OCG $G=(N,v,\pi)$, for every prefix sub-game $G^i$, \NDMES rule eliminates all the dummy players in $G^i$, aligning directly with the definition of Online-Dummy.
\end{proof}

\subsubsection{Proof of \Cref{ULMES_properties}}
\begin{proof}
	\textbf{(S-STAY)} \ULMES rule satisfies STAY as each player's shared value is non-decreasing. To show \ULMES rule satisfies S-STAY, consider an OCG $G=(N,v,\pi)$, for any player $j\in N$ and any prefix sub-game $G^i$ where $j \prec_{\pi} i$, assume $i$ is a contributional player, if $j$ satisfies $v(N_{\pi \mid i}) > v(N_{\pi \mid i}\setminus\{j\})$, then $j$ must be in the final $S_i$ after the elimination process of \ULMES rule. The proof is as follows. Denote $S^j_i$ as the tentative $S_i$ when checking whether player $j$ should be eliminated or not. Recall the valuation function is monotone and $S^j_i \subseteq N_{\pi \mid i}$. It means $v(N_{\pi \mid i}\setminus\{j\})\geq v(S^j_i \setminus\{j\})$. Since $v(N_{\pi \mid i}) > v(N_{\pi \mid i}\setminus\{j\})$, then we have $v(N_{\pi \mid i}) >  v(S^j_i \setminus\{j\})$, which implies $j$ must be kept in $S_i^j$. Then, $j$ will share $\frac{1}{|S_i|} \MC(G^i, N_{\pi \mid i} \setminus \{i\}, i) > 0$. Therefore, \ULMES rule satisfies S-STAY.
	
	\noindent \textbf{(PART)} Consider an OCG $G=(N,v,\pi)$, for any prefix sub-game $G^i$ with arriving player $i$, if $i$ is a contributional player, then $i$ is kept in $S_i$ as $\MC(G^i,N_{\pi \mid i} \setminus \{i\}, i) = v(N_{\pi \mid i}) - v(N_{\pi \mid i} \setminus \{i\}) > 0$. 
	
	\noindent \textbf{(OD)} Consider an OCG $G=(N,v,\pi)$, for any prefix sub-game $G^i$ with an arriving contributional player $i$, we show that \ULMES rule never assigns positive value to dummy players in $G^i$. Prove by contradiction. Assume that there exists some dummy player $j$ in $G^i$ who gets assigned positive value, then $j$ must survive in the elimination of $S_i$, however, by the definition of dummy player, for any subset $T\subseteq N_{\pi \mid i}$, $v(T)=v(T\cup\{j\})$. Denote $S_i^j$ as the tentative $S_i$ for the time-step when $j$ is checked whether she should be eliminated from $S_i$. According to \ULMES rule, $j\in S_i^j$ and $v(S_i^j)=v(N_{\pi \mid i})$, let $T=S_i^j\setminus \{j\}$, we have $v(S_i^j\setminus\{j\}) = v(S_i^j)=v(N_{\pi \mid i})$, meaning $j$ must be eliminated from $S_i^j$. Hence, $j$ should not survive in the final sharing set $S_i$ to share the value, contradicting to the assumption that $j$ gets some positive sharing value in $G^i$.
\end{proof}

\subsubsection{Greedy Monotone (GM) Decomposition}
We revisit the greedy monotone (GM) decomposition algorithm proposed by \citet{GZZ+24a}. GM decomposition maps a general OCG $G=(N,v,\pi)$ into a linear combination of multiple SOCGs. The detailed procedures of GM decomposition is provided in \Cref{alg:GM_Decomposition}.

\begin{algorithm}[h]
	\caption{Greedy Monotone (GM) Decomposition}
	\label{alg:GM_Decomposition}
	\textbf{Input}: An OCG $G=(N,v,\pi)$.  \\
	\textbf{Output}: A decomposition $D(G)$.
	\begin{algorithmic}[1] 
		\STATE Initialize $D(G)\leftarrow \emptyset$, $v_1(\cdot)\leftarrow v(\cdot)$ and $k\leftarrow 1$;
		\WHILE{$\max_{T\subseteq N}\{v_k(T)\} > 0$}
		\STATE $S\leftarrow \underset{T\subseteq N,v(T)>0}{\arg\min} v_k(T)$;
		\STATE Coefficient $c_k\leftarrow v_k(S)$; 
		\STATE Initialize component SOCG $\bar{G}_k=(N,\bar{v}_k,\pi)$;
		\FOR{$T\subseteq N$}
		\IF{$v_k(T)>0$}
		\STATE $\bar{v}_k(T)\leftarrow 1$;
		\ELSE 
		\STATE $\bar{v}_k(T)\leftarrow 0$;
		\ENDIF
		\STATE Update $v_k(T) \leftarrow v_k(T) - c_k \bar{v}_k(T)$;
		\ENDFOR 
		\STATE Add component SOCG $(c_k, \bar{G}_k)$ into $D(G)$;
		\STATE Update $k \leftarrow k + 1$;
		\ENDWHILE
	\end{algorithmic}
\end{algorithm}	

\citet{GZZ+24a} proved the following properties of the GM decomposition method,
\begin{itemize}
	\item GM decomposition outputs the $D(G)$ satisfying for each $T\subseteq N$, $v(T)=\sum_k c_k \bar{v}_k(T)$ and $\bar{v}_k(\cdot)$ is 0-1 valued monotone functions.
	\item Given a decomposition $D(G)$, for any player $i$ in $\pi$ with sub-game $G^i$, the decomposition of $D(G^i)$ is consistent with $D(G)$ within players in $N_{\mid i}$ (Consistency between the global game and prefix sub-games). 
\end{itemize}

\subsubsection{Omitted Proof of \Cref{DULMES_properties}}
\begin{proof}
	\textbf{(S-STAY)} Consider an OCG $G=(N,v,\pi)$, for any prefix sub-game $G^i$ with an arriving contributional player $i$, for every player $j$  satisfying $j \prec_{\pi} i$ and $v(N_{\pi \mid i}) > v(N_{\pi \mid i}\setminus\{j\})$. According to the GM decomposition $D(G)$, the valuation is decomposed into linear combinations. Since  $v(N_{\pi \mid i}) > v(N_{\pi \mid i}\setminus\{j\})$, there exists at least one component in $D(G)$ such that $\bar{v}(N_{\pi \mid i}) > \bar{v}(N_{\pi \mid i}\setminus\{j\})$ and $i$ is a contributional player in $\bar{G}^i$. For this component $(c,\bar{G})$, according to \Cref{ULMES_properties}, \ULMES rule satisfies S-STAY, meaning in $\bar{G}^i$, $\phi(\bar{G}^i,j)>0$. Due to the consistency of $D(G)$ and $D(G^i)$, we have $\phi(G^i,j) = \sum_{k} c_k \phi(\bar{G}^i_k,j) > 0$. \DULMES rule satisfies S-STAY.
	
	\noindent \textbf{(PART)} Consider an OCG $G=(N,v,\pi)$, for any prefix sub-game $G^i$ with an arriving contributional player 
	$i$. In the decomposition $D(G)$ of $G$, there exists at least one component $(c,\bar{G})$ such that $i$ is still a contributional player in $\bar{G}$. Prove by contradiction, assume there is no component such that $i$ is a contributional player. By the linearity and consistency of $D(G)$ regarding the valuation function, $i$ is not a contributional player in $G$, contradicting $i$ is the contributional player in $G^i$. For $\bar{G}$, as \ULMES rule is PART, meaning $i$ get positive value in $\bar{G}^i$: $\phi(\bar{G}^i,i)>0$. So $i$ has positive shared value in $G^i$ as the value $c \cdot \phi(\bar{G}^i,i)$ will be added into $\phi(G^i,i)$. Then \DULMES rule satisfies PART.
	
	\noindent \textbf{(OD)} Notice that \ULMES rule satisfies OD, assigning no value to dummy players for each component $(c,\bar{G})$ of the decomposition $D(G)$. Also, consider any OCG $G=(N,v,\pi)$, for any player $i$, if $i$ is dummy in $G$, then for SOCG $\bar{G}$, in each component of $D(G)$, $i$ is still a dummy player. Thus, for \DULMES rule which outputs the weighted sum of \ULMES rule's output over each component, it never assigns a positive value to dummy players in $G$. So, \DULMES rule satisfies the OD axiom.
	
	\noindent \textbf{(EA)} Consider an OCG $G=(N,v,\pi)$, according to \Cref{ULMES_EA_SIMPLE}, \ULMES rule satisfies EA for every SOCG. It means \ULMES rule satisfies EA for every SOCG in each component $(c,\bar{G})$ of $D(G)$. For any player $i$ in arriving order $\pi$, assume $i$ delays her arrival and changes the order into $\pi^\prime$ w.r.t. the OCG $G'=(N,v,\pi')$. Notice that the GM decomposition only depends on the valuation function and is unrelated with the arriving order. Thus, for player $i$, for each component $(c, \bar{G}=(N,\bar{v},\pi))$ and $(c, \bar{G}^\prime=(N,\bar{v},\pi'))$, \ULMES rule satisfies the EA axiom implies $\phi(\bar{G},i) \geq \phi(\bar{G}',i)$. Furthermore, $\phi(G,i) = \sum_{k}c_k \cdot \phi(\bar{G}_k,i) \geq \sum_{k}c_k\cdot \phi(\bar{G}_k',i) = \phi(G',i)$. So $i$ has no incentive to delay her arrival, meaning \DULMES rule satisfies the EA axiom.
\end{proof}	

\subsection{Omitted Proof in \Cref{individual_rational_section}}

\subsubsection{Omitted Proof of \Cref{RFC_is_not_IR}}
\begin{proof}
	Consider an OCG $G=(N,v,\pi)$, \DMC rule satisfies the IR axiom since for each player $i\in N$, $\phi(G,i)=\MC(G^i,N_{\pi \mid i} \setminus \{i\},i) \geq v(\{i\})$ with superadditive valuation function. For \SV rule, it degenerates to the classic cooperative game in which the Shapley Value satisfies IR in games with superadditive valuation functions. Next, we show that \eRFC rule does not satisfy the IR property with superadditive valuation function by the following example. Consider a game $G=(N,v,\pi)$ where $N=\{1,2\}$, $\pi=(1,2)$, and $v(\{1\})=1,v(\{2\})=2,v(\{1,2\})=5$. According to \eRFC rule by \citet{GZZ+24a}, $G$ will be decomposed into $D(G)=\{(1,\bar{G}_1),(1,\bar{G}_2),(3,\bar{G}_3)\}$. The decomposed valuation function is provided in \Cref{example_game_eRFC_not_IR}.
	\begin{table}[!ht]
		\centering  
		\scalebox{1}{
			\begin{tabular}{ccccc}
				\toprule
				Coalition &          Coef         & $\{1\}$ & $\{2\}$ & $\{1,2\}$ \\ \midrule
				Valuation &                       & $1$     & $2$     & $5$       \\ \midrule
				$\bar{v}_1(\cdot)$     & $c_1=1$  &$1$     & $1$     & $1$       \\ \midrule
				$\bar{v}_2(\cdot)$     & $c_2=1$  & $0$      & $1$     & $1$       \\ \midrule
				$\bar{v}_3(\cdot)$     & $c_3=3$  & $0$      & $0$     & $1$       \\ \bottomrule
		\end{tabular}}
		\caption{Coalition valuation and game decomposition of $G$}
		\label{example_game_eRFC_not_IR}
	\end{table}
	
	For each decomposed SOCG, the value shared by $1$ and $2$ within the order $\pi=(1,2)$ are shown in \Cref{value_share_example_game_for_eRFC_not_IR}.
	\begin{table}[htbp!]
		\centering
		\scalebox{1}{%
			\begin{tabular}{cccc}
				\toprule
				\multicolumn{4}{c}{\eRFC Rule Outcomes in $G$} \\ \midrule
				$\phi(G,i)$ & Coef    & Player $1$      & Player $2$       \\ \midrule
				$\bar{G}_1$       &  $1$    & $1$      & $0$    \\ \midrule
				$\bar{G}_2$       &  $1$    & $0$      & $1$  \\ \midrule
				$\bar{G}_3$       &  $3$    & $1$      & $0$   \\ \midrule
				$G$         &         & $\bf{4}$       & $\bf{1}$  \\ \bottomrule
			\end{tabular}
		}
		\caption{\eRFC rule outcomes in $G$}
		\label{value_share_example_game_for_eRFC_not_IR}
	\end{table}
	Finally, for the original game $G$, $\phi(G,1)=4$ and $\phi(G,2)=1$. Note that $v(\{2\})=2 > \phi(G,2)=1$. Thus, \eRFC rule does not satisfy the IR property. 
\end{proof}

\subsubsection{Omitted Proof of \Cref{maintain_IR_property}}
\begin{proof}
	For the IR-\MES rule, IR-\NDMES rule, and IR-\DULMES rule, all three rules satisfy the IR axiom. For any OCG $G=(N,v,\pi)$, each rule first assigns the value $v(\{i\})$ to any arriving player $i$ in the order $\pi$. The superadditive valuation function ensures the validity of this allocation. 
	
	Next, we show each rule maintains to satisfy all the axioms before the refinement. 
	
	We first clarify S-STAY is incompatible with IR if there exists some player $i$, $\MC(G^i,N_{\pi \mid i} \setminus \{i\},i) = v(\{i\}) > 0$. Consider an OCG $G=(N,v,\pi)$ where $N=\{1,2\}$, $\pi=(1,2)$, and $v(\{1\})=v(\{2\})=1$, $v(\{1,2\})=2$. According to IR, it must be $\phi(G,1)=\phi(G,2)=1$. However, this value distribution does not satisfy S-STAY axiom. Therefore, we next mainly focus on the superadditive valuation where for any OCG $G=(N,v,\pi)$, $\MC(G^i, N_{\pi \mid i} \setminus \{i\},i) > v(\{i\})$ for each arriving player $i$ when discussing the S-STAY axiom.
	
	We first show that \textbf{IR-\MES} satisfies S-STAY, PART, EA axioms as follows.
	
	\noindent \textbf{(S-STAY)} Consider an OCG $G=(N,v,\pi)$ with arriving order $\pi=(1,2,\dots, n)$. For the prefix sub-game $G^i$ with arriving player $i$, $\phi(G^i,i)=\frac{1}{|S_i|}\big(\MC(G^i, N_{\pi\mid i}\setminus\{i\},i)-v(\{i\})\big)$. For any other prefix sub-game $G^j$ where $i\prec_{\pi} j$. $\phi(G^j,i)=\frac{1}{|S_i|}\big(\MC(N_{\pi \mid i},i)-v(\{i\})\big)+\sum_{k=i+1}^j\big(\MC(N_{\pi \mid k},k)-v(\{k\})\big) \geq \phi(G^i,i)$. IR-\MES rule satisfies STAY. Furthermore, if in $G^j$, $\MC(N_{\pi \mid j},j) > v(\{j\})$, then $\phi(G^j,i) > \phi(G^i,i)$, satisfying S-STAY.
	
	\noindent \textbf{(PART)} Consider an OCG $G=(N,v,\pi)$, for any arriving player $i$, the value shared immediately by $i$ is $\phi(G^i,i)=v(\{i\})+\frac{1}{|S_i|}\big(\MC(G^i, N_{\pi\mid i}\setminus\{i\},i)-v(\{i\})\big)\geq 0$. Hence, IR-\MES rule satisfies PART.
	
	\noindent\textbf{(EA)} Consider any player $i\in N$ and two OCGs $G_1=(N,v,\pi_1)$ and $G_2=(N, v, \pi_2)$ with two different arriving orders $\pi_1$ and $\pi_2$.
	\[
	\begin{aligned}
		\pi_1: (1, 2, \ldots, i - 1, \mathbf{i} , i + 1, \ldots, j - 1, j, j + 1, \ldots, n),\\
		\pi_2: (1, 2, \ldots, i - 1, i + 1, \ldots, j - 1, j, \mathbf{i}, j + 1, \ldots, n). \\
	\end{aligned}
	\]
	For each prefix sub-game $G_1^i (G_2^i)$ within $G_1 (G_2)$, denote the player set sharing the marginal value by $S_i^1 (S_i^2)$ in IR-\MES rule. Then, the value shared by $i$ in $G_1$ can be represented as
	\begin{equation}
	\phi(G_1,i)=v(\{i\}) + \sum_{k=i}^{n}\frac{1}{|S_k^1|}\Big(\MC(G^k_1,N_{\pi \mid k}\setminus\{k\}, k) - v(\{k\})\Big).
	\end{equation}
	The value shared by player $i$ in $G_2$ can be written as
	\begin{equation}
	\phi(G_2,i)=v(\{i\}) + \frac{1}{|S_i^2|} \Big(\MC(G_2^i,N_{\pi \mid j},i)-v(\{i\}) \Big) +  \sum_{k=j+1}^{n}\frac{1}{|S_k^2|}\Big(\MC(G^k_2,N_{\pi \mid k}\setminus\{k\}, k\Big) - v(\{k\})\Big).
	\end{equation}
	For every player $k$ who arrives after player $j$, we have $N_{\pi_1 \mid k}=N_{\pi_2 \mid k}$. Hence,
	\begin{equation}
	\begin{aligned}
		&\phi(G_1,i) - \phi(G_2,i) \\
		=&\sum_{k=i}^j \frac{1}{|S_k^1|}\Big(\MC(G^k_1,N_{\pi_1 \mid k}\setminus \{k\}, k) - v(\{k\})\Big) - \frac{1}{|S_i^2|} \Big(\MC(G^i_2, N_{\pi_2 \mid j} \setminus \{i\}, i) - v(\{i\})\Big) \\
		\geq & \frac{1}{|S_i^2|}\sum_{k=i}^j \Big(\MC(G^k_1,N_{\pi_1 \mid k}\setminus \{k\}, k) - v(\{k\})\Big)- \frac{1}{|S_i^2|} \Big(\MC(G^i_2,N_{\pi_2 \mid j} \setminus \{i\}, i) - v(\{i\})\Big) \\
		\geq & \frac{1}{|S_i^2|} \Big( v(N_{\pi_1 \mid j})-v(N_{\pi_1 \mid j} \setminus \{i\}) - \sum_{k=i}^j v(\{k\}) \Big) - \frac{1}{|S_i^2|} \Big(\MC(G^i_2,N_{\pi_2 \mid j} \setminus \{i\}, i) - v(\{i\})\Big).
	\end{aligned}
	\end{equation}
	Note that $N_{\pi_2\mid i} \setminus \{i\} =\{1,2,\ldots, i-1,i+1,\ldots,j\}=N_{\pi_1 \mid j} \setminus \{i\}\cup \{i+1,\ldots, j\}$. Therefore we have
	\begin{equation}
	\begin{aligned}
		&\frac{1}{|S_i^2|} \Big( v(N_{\pi_1 \mid j})-v(N_{\pi_1 \mid j} \setminus \{i\}) - \sum_{k=i}^j v(\{k\}) \Big) - \frac{1}{|S_i^2|} \Big(\MC(G^i_2,N_{\pi_2 \mid j} \setminus \{i\}, i) - v(\{i\})\Big) \\
		=& \frac{1}{|S_i^2|} \Big(v(N_{\pi_1 \mid j}\setminus \{i\}) -v(N_{\pi_1 \mid i} \setminus\{i\}) - \sum_{k=i+1}^j v(\{k\})  \Big) \\
		\geq & \frac{1}{|S_i^2|} \left( v(\{i+1,\ldots, j\}) - \sum_{k=i+1}^j v(\{k\}) \right) \geq 0.
	\end{aligned}
	\end{equation}
	The last two steps are from superadditivity, we have
	$v(N_{\pi_1 \mid j}\setminus\{i\}) \geq v(N_{\pi_1 \mid i}\setminus\{i\})+v(\{i+1,\dots, j\})$ and $v(\{i+1,\ldots, j\}) \geq \sum_{k=i+1}^j v(\{k\})$.
	
	Next, we show that \textbf{IR-\NDMES} satisfies S-STAY, PART, OD, and EA axioms as follows.
	
	\noindent\textbf{(S-STAY)} Obviously, IR-\NDMES rule satisfies STAY as every player's shared value is non-decreasing. Next, we show the satisfication of S-STAY. Consider an OCG $G=(N,v,\pi)$, for any player $j$, consider any prefix sub-games $G^i,(j\prec_\pi i)$ and $i$ is a contributional player, if $j$ satisfies $v(N_{\pi \mid i})> v(N_{\pi \mid i}\setminus\{j\})$, then $j$ is non-dummy in $G^i$. So, $j$ is included in $S_i$ and share the value $\frac{1}{|S_i|}\big(\MC(G^i,N_{\pi \mid i}\setminus\{i\},i)-v(\{i\})\big)>0$. Hence, for player $j$, $\phi(G^i,j) > \phi(G^j,j)$, satisfying S-STAY axiom.
	
	\noindent\textbf{(PART)} For each arriving player $i$, if $i$ is a contributional player, i.e., $\MC(N_{\pi \mid i},i)>0$, $i$ is not a dummy player and will be included in $S_i$. Then, $i$ immediately gets $\frac{1}{|S_i|}\big(\MC(N_{\pi \mid i}\setminus\{i\},i)-v(\{i\}) > 0$, satisfying the PART axiom.
	
	\noindent\textbf{(OD)} Consider an OCG $G=(N,v,\pi)$. For every prefix sub-game $G^i$, the IR-\NDMES rule eliminates all the dummy players in $G^i$, just like the \NDMES rule, aligning directly with the definition of Online-Dummy.
	
	\noindent \textbf{(EA)} Consider two OCGs $G_1=(N,v,\pi_1)$ and $G_2=(N,v,\pi_2)$ with two different orders $\pi_1$ and $\pi_2$ as follows
	\begin{equation}
	\begin{aligned}
		\pi_1: (1, 2, \ldots, i - 1, \mathbf{i} , i + 1, \ldots, j - 1, j, j + 1, \ldots, n)\\
		\pi_2: (1, 2, \ldots, i - 1, i + 1, \ldots, j - 1, j, \mathbf{i}, j + 1, \ldots, n) \\
	\end{aligned}
	\end{equation}
	
	For any player $i\in N$, we discuss the satisfication of the EA axiom in the following four cases.
	
	\noindent \textbf{Case 1.} 
	$\MC(G_1^i,N_{\pi_1 \mid i}\setminus\{i\},i) = \MC(G_2^i,N_{\pi_2 \mid i}\setminus\{i\},i)=0$. The shared value after player $j+1$’s arrival remains the same for these two orders. The only difference is that player $i$ might gain some shared value when $i$ is not a dummy player in sub-games from $G^{i+1}$ to $G^{j}$. Therefore, the value shared by player $i$ in $\pi_1$ will always be at least as much as in $\pi_2$
	
	\noindent \textbf{Case 2.} $\MC(G_1^i,N_{\pi_1 \mid i}\setminus\{i\},i) > 0, \MC(G_2^i,N_{\pi_2 \mid i}\setminus\{i\},i)=0$. For $\pi_1$, $i$ will never be a dummy player in all sub-games after her arrival, thus $i$ shares all the marginal value from $i$ to $n$. However, in $\pi_2$, player $i$ only shares marginal value from player $j+1$ to player $n$.
	
	\noindent \textbf{Case 3.} $\MC(G_1^i,N_{\pi_1 \mid i}\setminus\{i\},i) = 0, \MC(G_2^i,N_{\pi_2 \mid i}\setminus\{i\},i)>0$. In thise case, we discuss cases concerning whether $i$ is dummy or not. The first situtation is that $i$ keeps to be dummy (No shared value from $i$ to $j-1$) until $j$'s arrival (This is because $i$ has some positive marginal value in $\pi_2$). 
	\begin{equation}
	\phi(G_1,i) = v(\{i\}) + \sum_{k=j}^n \frac{1}{|S^1_k|}\Big(\MC(G_1^i,N_{\pi_1 \mid k}\setminus\{k\},k) -v(\{k\})\Big).
	\end{equation}
	\begin{equation}
	\phi(G_2,i) = v(\{i\}) + \sum_{k=j}^n \frac{1}{|S^2_k|}\Big(\MC(G_2^i,N_{\pi_2 \mid k}\setminus\{k\},k) -v(\{k\})\Big).
	\end{equation}
	Let $S_j^1$ be the set of players who share the value from $j$'s participation in $\pi_1$. $S_j^1=S_i^2$ because they both represent the set of non-dummy players in players $\{1,2,\ldots, j\}$. Another observation is that starting from player $j+1$'s participation, the value shared by player $i$ in $\pi_1$ and $\pi_2$ will be the same. Hence,
	\begin{equation}
	\begin{aligned}
		& \phi(G_1, i) - \phi(G_2, i) \\
		=& \frac{1}{|S_j^1|}\Big( \big(\MC(G_1^j, N_{\pi_1\mid j}\setminus\{j\},j) - v(\{j\})\big) - \big(\MC(G_2^i,N_{\pi_2 \mid i}\setminus\{i\}, i) -v(\{i\})\big)
		\Big).
	\end{aligned}
	\end{equation}
	Note that player $i$ creates no new marginal valution with $N_{\pi_1 \mid i}$ and she is a dummy player for all sub-games before $j$'s arrival in this situation, thus we have $v(\{i\})=0$ and
	$v(N_{\pi_1 \mid j}\setminus\{j\}) = v(N_{\pi_1 \mid j}\setminus\{i,j\})$. Therefore, 
	\begin{equation}
	\begin{aligned}
		&\phi(G_1,i) - \phi(G_2,i) \\
		=&\frac{1}{|S_j^1|} \Big( v(N_{\pi_1 \mid j}\setminus\{i\}) - v(N_{\pi_1 \mid j}\setminus\{j\}) - v(\{j\}) \Big)\\
		\geq & \frac{1}{|S_j^1|} \Big( v(N_{\pi_1 \mid j}\setminus\{i,j\}) + v(j) - v(N_{\pi_1 \mid j}\setminus \{i,j\}) - v(j) \Big)\\
		=& 0.
	\end{aligned}
	\end{equation}
	The other situation is that player $i$ is not dummy after some player $q$'s arrival where $i \prec_{\pi_1} q \prec_{\pi_1} j$, in this case, $\phi(G_1,i)$ can be written as
	\begin{equation}
	\begin{aligned}
		\phi(G_1,i) = v(\{i\}) + \sum_{k =q}^n \frac{1}{|S^1_k|}\Big(\MC(G_1^k, N_{\pi_1 \mid k}\setminus\{k\}, k) - v(\{k\})\Big).
	\end{aligned}
	\end{equation}
	Note that $v(\{i\}) = 0$ because of the super-additivity. Also, denote $\sum_{k = j + 1}^n \frac{1}{|S^1_k|}\big(\MC(G_1^k,N_{\pi_1\mid k}\setminus\{k\}, k) - v(\{k\})\big)$ by $\Delta$ for simplicity. Then, we have 
	\begin{equation}
	\begin{aligned}
		&\phi(G_1,i) = \sum_{k=q}^{j} \frac{1}{|S^1_k|}\Big(\MC(G_1^k,N_{\pi_1 \mid k}\setminus\{k\}, k) - v(\{k\})\Big) + \Delta \\
		&\geq \frac{1}{|S^1_j|} \Big( v(N_{\pi_1\mid j} - v(N_{\pi_1 \mid q}\setminus\{q\})  -\sum_{k=q}^j v(k) \Big) + \Delta.
	\end{aligned}
	\end{equation}
	For the game $G_2$, $\phi(G_2,i)$ can be written as
	\begin{equation}
	\begin{aligned}
		\phi(G_2,i) =& v(\{i\}) + \frac{1}{|S_i^2|} \Big(\MC(G_2^i, N_{\pi_2 \mid i}\setminus\{i\}, i) - v(i)\Big) + \Delta =\frac{1}{|S_i^2|} \MC(G_2^i, N_{\pi_2\mid i}\setminus\{i\}, i) + \Delta.
	\end{aligned}
	\end{equation}
	where $\MC(G_2^i, N_{\pi_2 \mid i}\setminus \{i\}, i) = v(N_{\pi_1 \mid j}) - v(N_{\pi_1 \mid j}\setminus \{i\})$. Further, 
	\begin{equation}
	\begin{aligned}
		& \phi(G_1,i) - \phi(G_2, i) =\frac{1}{|S_j^1|} \Big(
		v(N_{\pi_1 \mid j})-v(N_{\pi_1 \mid q}\setminus\{q\}) -\sum_{k=q}^j v(\{k\})\Big) -\frac{1}{|S_i^2|} \Big( v(N_{\pi_1 \mid j}) - v(N_{\pi_1 \mid j}\setminus \{i\}) \Big).
	\end{aligned}
	\end{equation}
	For every player $k$ who arrives after player $j$, we have $N_{\pi_1 \mid k}$ and $N_{\pi_2 \mid k}$. That is $|S_j^1| = |S_i^2|$, 
	\begin{equation}
	\begin{aligned}
		\phi(G_1,i) - \phi(G_2,i) =&\frac{1}{|S_j^1|} \Big( v(N_{\pi_1 \mid j}\setminus \{i\}) -v(N_{\pi_1 \mid q}\setminus\{q\}) -\sum_{k=q}^j v(\{k\}) \Big) \\
		\geq &  \frac{1}{|S_j^1|} \Big( v(N_{\pi_1 \mid q}\setminus \{i,q\}) + \sum_{k=q}^j v(k) - v(N_{\pi_1 \mid q}\setminus\{q\}) -\sum_{k=q}^j v(\{k\})  \Big) \\
		= &  \frac{1}{|S_j^1|} \Big( v(N_{\pi_1 \mid q}\setminus \{i,q\}) - v(N_{\pi_1 \mid q}\setminus \{i,q\}) \Big) = 0. \\
	\end{aligned}
	\end{equation}
	The penultimate inequality follows from superadditivity and the last step is because $i$ is a dummy player of prefix $N_{\pi_1 \mid q}=\{1,2,\ldots, i,\ldots,q\}$.
	
	\noindent \textbf{Case 4.} $\MC(G_1^i,N_{\pi_1 \mid i}\setminus\{i\},i) = 0, \MC(G_2^i,N_{\pi_2 \mid i}\setminus\{i\},i)>0$
	For $G_1$, $\phi(G_1,i)$ can be represented as
	\begin{equation}
	\begin{aligned}
		\phi(G_1,i)=v(\{i\}) +& \sum_{k=i}^j \frac{1}{|S_k^1|}\Big(\MC(G_1^k, N_{\pi_1 \mid k}\setminus\{k\}, k) - v(\{k\})\Big) \\
		&+ \sum_{k=j+1}^n \frac{1}{|S_k^1|}\Big(\MC(G_1^k,N_{\pi_1 \mid k}\setminus\{k\}, k) - v(\{k\})\Big).
	\end{aligned}
	\end{equation}
	For $G_2$, $\phi(G_2,i)$ can be represented as
	\begin{equation}
	\begin{aligned}
		\phi(G_2,i)=v(\{i\}) +& \frac{1}{|S_i^2|}\Big(\MC(G_2^i, N_{\pi_2 \mid i}\setminus\{i\}, i) - v(\{i\})\Big) \\
		&+ \sum_{k=j+1}^n \frac{1}{|S_k^2|}\Big(\MC(G_2^k, N_{\pi_2 \mid k}\setminus\{k\}, k) - v(\{k\})\Big).
	\end{aligned}
	\end{equation}
	Still, we have $S_j^1 = S_i^2$, and from the arrival of player $j+1$ to the last player, the value shared by player $i$ in $G_1$ and $G_2$ remains the same. Thus, 
	\begin{equation}
	\begin{aligned}
		&\phi(G_1,i) - \phi(G_2,i) \\
		=& \sum_{k=i}^{j}\frac{1}{|S_k^1|}\Big(\MC(G_1^k, N_{\pi_1 \mid k}\setminus\{k\}, k) - v(\{k\})\Big) - \frac{1}{|S_i^2|}\Big(\MC(G_2^i, N_{\pi_2 \mid i}\setminus \{i\}, i) - v(\{i\})\Big) \\
		\geq& \frac{1}{|S_j^1|}\Big(\sum_{k=i}^j \big(\MC(G_1^k, N_{\pi_1 \mid k}\setminus\{k\}, k) - v(\{k\})\big) - \MC(G_2^i, N_{\pi_2 \mid i}\setminus\{i\}, i) + v(\{i\}) \Big).
	\end{aligned} 
	\end{equation}
	To show $\phi(G_1,i) - \phi(G_2,i)\geq 0$, we next show $\sum_{k=i}^j \big(\MC(G_1^k, N_{\pi_1 \mid k}\setminus\{k\}, k) - v(\{k\})\big) - \MC(G_2^i, N_{\pi_2 \mid i}\setminus\{i\}, i) + v(\{i\}) \geq 0$. We first rewrite the equation as
	\begin{equation}
	v(N_{\pi_1 \mid j}) - v(N_{\pi_1 \mid i}\setminus\{i\}) - \sum_{k=i+1}^j v(\{k\}) - \MC(G_2^i,N_{\pi_1 \mid i}\setminus\{i\}, i) .
	\end{equation}
	Notably, $N_{\pi_2 \mid i} \setminus\{i\}=\{1,2,\ldots, i-1,i+1,\ldots, j\}$ and the valuation function $v$ is superadditive, then we have,
	\begin{equation}
	\begin{split}
		&v(N_{\pi_1 \mid j}) - v(N_{\pi_1 \mid i}\setminus\{i\}) - \sum_{k=i+1}^j v(\{k\}) - \MC(G_2^i, N_{\pi_2 \mid i}\setminus\{i\}, i)  \\
		&= v(N_{\pi_1 \mid j}\setminus\{i\}) - v(N_{\pi_1 \mid i}\setminus\{i\}) - \sum_{k=i+1}^j v(\{k\}) \\
		&\geq v(N_{\pi_1 \mid i}\setminus\{i\}) + \sum_{k=i+1}^j v(\{k\}) -v(N_{\pi_1 \mid i}\setminus\{i\}) - \sum_{k=i+1}^j v(\{k\})\\
		&= 0.
	\end{split}
	\end{equation}
	From Case 1 to Case 4, for any player $i$, $i$ has no incentive to delay her arrival under the IR-\NDMES sharing scheme. Thus, the IR-\NDMES rule satisfies the EA axiom.
	
	Finally, we show that \textbf{IR-\DULMES} satisfies S-STAY, PART, OD, and EA.
	Recall IR-\DULMES rule runs as follows. For an OCG $G=(N,v,\pi)$, we first decompose $G$ into $D(G)$ by \Cref{alg:GM_Decomposition}. For each SOCG component $(c,\bar{G})$, we run the IR refinement paradigm, that is, first assign $v(\{i\})$ for each arriving player $i$; then compute $S_i$ in $\bar{G}$ and equally share the value $\MC(\bar{G}^i, N_{\pi \mid i})-v(\{i\})$. 
	
	\noindent\textbf{(S-STAY)} It is obviously that IR-\DULMES rule satisfies the STAY axiom as each player shared value is the linear combination (positive coefficient) of the outputs of SOCGs, in which the shared value is non-decreasing. Next we show for each component SOCG $\bar{G}$, the rule satisfies S-STAY. Consider an SOCG $\bar{G}=(N,\bar{v},\pi)$. For any prefix sub-game $\bar{G}^i$, since we assume that $\MC(\bar{G}^i, N_{\pi \mid i}\setminus\{i\},i)-\bar{v}(\{i\})>0$, then for any player $j$ who satisfies $\bar{v}(N_{\pi \mid i}) > \bar{v}(N_{\pi \mid i}\setminus \{j\})$, $j$ must be included in $S_i$ in $\bar{G}$ (The same proof procedures of the S-STAY axiom for \ULMES rule in \Cref{ULMES_properties}). Then $j$ must get a positve fraction $\frac{1}{|S_i|}$ of $(\MC(\bar{G}^i, N_{\pi \mid i}\setminus\{i\},i)-\bar{v}(\{i\}))>0$, satisfying S-STAY.
	
	\noindent \textbf{(PART)} Since IR-\DULMES rule outputs the linear combination of IR-\ULMES rule in each decomposed component SOCG $\bar{G}$, we show that the PART axiom is satisfied in each $\bar{G}$. Consider an SOCG $\bar{G}=(N,\bar{v},\pi)$, for any prefix sub-game $\bar{G}^i$ with arriving player $i$, if $i$ is a contributional player, $\bar{v}(N_{\pi \mid i}) > \bar{v}(N_{\pi \mid i}\setminus\{i\})$, then $i$ is kept in $S_i$ as $\bar{v}(N_{\pi \mid i}) - \bar{v}(N_{\pi \mid i}\setminus\{i\}) > 0$. Futher, we have $\phi(\bar{G},i)=\bar{v}(\{i\}) + \frac{1}{|S_i|}\big(\MC(\bar{G}^i, N_{\pi \mid i}\setminus\{i\},i)-\bar{v}(\{i\})\big)$. If $\bar{v}(\{i\})>0$, then $\phi(\bar{G},i)>0$ as $\MC(\bar{G}^i, N_{\pi \mid i}\setminus\{i\},i)-\bar{v}(\{i\})\geq 0$. When $\bar{v}(\{i\})=0$, $\phi(\bar{G},i)=\frac{1}{|S_i|}\big(\MC(\bar{G}^i, N_{\pi \mid i}\setminus\{i\},i)-\bar{v}(\{i\})\big) > 0$. Therefore, IR-\DULMES rule satisfies PART.
	
	\noindent \textbf{(OD)} The proof of satisfication of the Online-Dummy axiom directly holds as the IR refinement does not affect the selection of $S_i$ for any OCG $G$. Thus the proof of the OD axiom directly follows the proof in \Cref{DULMES_properties}.
	
	\noindent \textbf{(EA)} To show IR-\DULMES rule satisfies EA, it is sufficient to show that IR-\ULMES rule satisfies EA in each decomposed component SOCG $\bar{G}$. The reason is that if player $i$ has no incentive to delay her arrival in every decomposed SOCG $\bar{G}$, then $i$ can not change the shared value by delaying her arrival because the GM decomposition does not depend on the arrival order $\pi$. Now we focus on the EA property of IR-\ULMES rule in SOCG $\bar{G}=(N,\bar{v},\pi)$. Notice that in SOCG with superadditive valuation function, there could be at most one player with singleton value $1$. We discuss the following two cases: (1). There is no player with singleton value $1$ in $\bar{G}$. In this case, IR-\ULMES rule degenerates to \ULMES rule. This means IR-\ULMES rule satisfies EA as well as we have proved that \ULMES rule satisfies EA in SOCGs in \Cref{ULMES_EA_SIMPLE}. (2). The other situation is that there exists some player $q$ satisfying $\bar{v}(\{q\})=1$. In this case, as we know IR-\ULMES rule satisfies IR, it means that in such an SOCG with $\bar{v}(\{q\})=1$, the value $1$ is wholely allocated to player $q$. Then, firstly $q$ has no incentive to delay her arrival as it does not change the outcome. Secondly, for any other player $i\neq q$, $i$ cannot delay her arrival to share some part of the value $1$ because IR property guarantees IR-\ULMES rule distributes the value $1$ to player $q$. This completes the proof of the EA satisfication of IR-\ULMES rule. Therefore, we have IR-\DULMES rule satisfies the EA axiom.
\end{proof}


\end{document}