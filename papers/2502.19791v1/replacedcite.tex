\section{Related Work}
\paragraph{Cooperative Games}
Cooperative game theory, originating from the last century ____, is a significant branch of game theory that studies scenarios where players can benefit by forming coalitions and making collective decisions. One of the key problems in this area is how to distribute the value created by coalitions among players, considering axiomatic characterizations (e.g., stability, consistency, etc.). The Shapley Value ____ initiated the research, laying the foundation for a series of subsequent works. ____ first proposed the core concept for cooperative games. In the context of transferable utility cooperative games, ____ studied market games, while ____ investigated cooperative bargaining scenarios. ____ first introduced the concept of the nucleolus, and ____ bridged cooperative game theory with practical matching markets. There is also a line of research focusing on cooperative games with hedonic preferences ____. Further details about classic cooperative game theory can be found in several books (see, e.g., ____).

Online cooperative games study the coalition game model in an online manner where agents arrive in a random order and the coalition formation decision should be made without any knowledge regarding the agents arriving in the future. Our paper is closely related to the work by ____, which was the first to study online cooperative games with consideration of strategic arrivals. Recently, ____ explores the cost sharing game in the context of online strategic arrivals and propose the Shapley-fair shuffle cost sharing mechanisms. 
Another branch studying cooperative game in an online manner, mainly concerning on hedonic games, focuses on addresssing approximation to the social welfare and stability ____. The biggest difference from the aformentioned online cooperative game is that it typically assume that agents reveal their preferences truthfully without incentive to misreport. Further literature on dynamic mechanism design can be found in ____. With regard to cooperative games, ____ studied the online coalition structure generation problem, while ____ investigated online coalition formation with random arrival. An online or dynamic perspective has also been applied to matching and hedonic games (see, e.g., ____). 

\paragraph{Online Mechanism Design}
The online cooperative game model explored in this paper is closely related to online mechanism design, where each agent’s private information is their arrival time in the game. The primary objective is to design value-sharing rules that incentivize all agents to truthfully report their arrival times. Mechanism design in dynamic environments focuses on problems involving multiple agents with private information, where the goal is to elicit this private information while making decisions without knowledge of future events. ____ initiated the study of truthful online auctions in dynamic environments. Later, ____ coined the concept of online mechanism design. Some works ____ discussed the state-of-the-art VCG mechanism in dynamic online settings. Online matching has also been a hot topic in dynamic algorithm design ____. Moreover, there is a wide literature on solutions for different sequential decision problems ____.