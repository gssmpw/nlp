
\documentclass[journal]{IEEEtran}

  \usepackage[nocompress]{cite}
  \usepackage{tcolorbox}
  \usepackage{tikz}
\usepackage{amsmath}

\usetikzlibrary{positioning}

% inlined bib file
\usepackage{filecontents}
\usepackage{enumitem}


% functionality
\usepackage{framed}%文字带边框

\usepackage{amssymb}

\usepackage{amsthm}%proof 必须放到theorem上面否则报错
\newtheorem{theorem}{Theorem}%theorem
\newtheorem{lemma}{Lemma}%lemma
\newtheorem{definition}{Definition}

% table
\usepackage{multirow}
\newcommand{\tabincell}[2]{\begin{tabular}{@{}#1@{}}#2\end{tabular}} 
\usepackage{caption}
\usepackage{makecell}
% add color in cell
\usepackage{colortbl}
\usepackage{pifont}
\usepackage{multicol}
\usepackage{booktabs}
\usepackage{algorithmic}
\usepackage[ruled,vlined]{algorithm2e} 

% subfigure


\usepackage{xspace}

\newcommand{\blue}[1]{\textcolor{blue}{#1}}

\newcommand{\meng}[1]{{\color{blue}[Meng: {#1}]}}

\usepackage{pifont}

\usepackage{float}

\graphicspath{{./figure/}}

\usepackage[colorlinks=true,bookmarks=false,citecolor=blue,urlcolor=blue]{hyperref}

\usepackage{cleveref}

\usepackage{tcolorbox}

\newcommand{\xingshuo}[1]{\textbf{\textcolor{red}{[#1]}}}

\usepackage{graphicx}

\usepackage{subcaption}

\usepackage{cite}




% Some very useful LaTeX packages include:
% (uncomment the ones you want to load)


% *** MISC UTILITY PACKAGES ***
%
%\usepackage{ifpdf}
% Heiko Oberdiek's ifpdf.sty is very useful if you need conditional
% compilation based on whether the output is pdf or dvi.
% usage:
% \ifpdf
%   % pdf code
% \else
%   % dvi code
% \fi
% The latest version of ifpdf.sty can be obtained from:
% http://www.ctan.org/pkg/ifpdf
% Also, note that IEEEtran.cls V1.7 and later provides a builtin
% \ifCLASSINFOpdf conditional that works the same way.
% When switching from latex to pdflatex and vice-versa, the compiler may
% have to be run twice to clear warning/error messages.






% *** CITATION PACKAGES ***
%
%\usepackage{cite}
% cite.sty was written by Donald Arseneau
% V1.6 and later of IEEEtran pre-defines the format of the cite.sty package
% \cite{} output to follow that of the IEEE. Loading the cite package will
% result in citation numbers being automatically sorted and properly
% "compressed/ranged". e.g., [1], [9], [2], [7], [5], [6] without using
% cite.sty will become [1], [2], [5]--[7], [9] using cite.sty. cite.sty's
% \cite will automatically add leading space, if needed. Use cite.sty's
% noadjust option (cite.sty V3.8 and later) if you want to turn this off
% such as if a citation ever needs to be enclosed in parenthesis.
% cite.sty is already installed on most LaTeX systems. Be sure and use
% version 5.0 (2009-03-20) and later if using hyperref.sty.
% The latest version can be obtained at:
% http://www.ctan.org/pkg/cite
% The documentation is contained in the cite.sty file itself.






% *** GRAPHICS RELATED PACKAGES ***
%
\ifCLASSINFOpdf
  % \usepackage[pdftex]{graphicx}
  % declare the path(s) where your graphic files are
  % \graphicspath{{../pdf/}{../jpeg/}}
  % and their extensions so you won't have to specify these with
  % every instance of \includegraphics
  % \DeclareGraphicsExtensions{.pdf,.jpeg,.png}
\else
  % or other class option (dvipsone, dvipdf, if not using dvips). graphicx
  % will default to the driver specified in the system graphics.cfg if no
  % driver is specified.
  % \usepackage[dvips]{graphicx}
  % declare the path(s) where your graphic files are
  % \graphicspath{{../eps/}}
  % and their extensions so you won't have to specify these with
  % every instance of \includegraphics
  % \DeclareGraphicsExtensions{.eps}
\fi
% graphicx was written by David Carlisle and Sebastian Rahtz. It is
% required if you want graphics, photos, etc. graphicx.sty is already
% installed on most LaTeX systems. The latest version and documentation
% can be obtained at: 
% http://www.ctan.org/pkg/graphicx
% Another good source of documentation is "Using Imported Graphics in
% LaTeX2e" by Keith Reckdahl which can be found at:
% http://www.ctan.org/pkg/epslatex
%
% latex, and pdflatex in dvi mode, support graphics in encapsulated
% postscript (.eps) format. pdflatex in pdf mode supports graphics
% in .pdf, .jpeg, .png and .mps (metapost) formats. Users should ensure
% that all non-photo figures use a vector format (.eps, .pdf, .mps) and
% not a bitmapped formats (.jpeg, .png). The IEEE frowns on bitmapped formats
% which can result in "jaggedy"/blurry rendering of lines and letters as
% well as large increases in file sizes.
%
% You can find documentation about the pdfTeX application at:
% http://www.tug.org/applications/pdftex





% *** MATH PACKAGES ***
%
%\usepackage{amsmath}
% A popular package from the American Mathematical Society that provides
% many useful and powerful commands for dealing with mathematics.
%
% Note that the amsmath package sets \interdisplaylinepenalty to 10000
% thus preventing page breaks from occurring within multiline equations. Use:
%\interdisplaylinepenalty=2500
% after loading amsmath to restore such page breaks as IEEEtran.cls normally
% does. amsmath.sty is already installed on most LaTeX systems. The latest
% version and documentation can be obtained at:
% http://www.ctan.org/pkg/amsmath





% *** SPECIALIZED LIST PACKAGES ***
%
%\usepackage{algorithmic}
% algorithmic.sty was written by Peter Williams and Rogerio Brito.
% This package provides an algorithmic environment fo describing algorithms.
% You can use the algorithmic environment in-text or within a figure
% environment to provide for a floating algorithm. Do NOT use the algorithm
% floating environment provided by algorithm.sty (by the same authors) or
% algorithm2e.sty (by Christophe Fiorio) as the IEEE does not use dedicated
% algorithm float types and packages that provide these will not provide
% correct IEEE style captions. The latest version and documentation of
% algorithmic.sty can be obtained at:
% http://www.ctan.org/pkg/algorithms
% Also of interest may be the (relatively newer and more customizable)
% algorithmicx.sty package by Szasz Janos:




% *** ALIGNMENT PACKAGES ***
%
%\usepackage{array}
% Frank Mittelbach's and David Carlisle's array.sty patches and improves
% the standard LaTeX2e array and tabular environments to provide better
% appearance and additional user controls. As the default LaTeX2e table
% generation code is lacking to the point of almost being broken with
% respect to the quality of the end results, all users are strongly
% advised to use an enhanced (at the very least that provided by array.sty)
% set of table tools. array.sty is already installed on most systems. The
% latest version and documentation can be obtained at:
% http://www.ctan.org/pkg/array


% IEEEtran contains the IEEEeqnarray family of commands that can be used to
% generate multiline equations as well as matrices, tables, etc., of high
% quality.




% *** SUBFIGURE PACKAGES ***
%\ifCLASSOPTIONcompsoc
%  \usepackage[caption=false,font=normalsize,labelfont=sf,textfont=sf]{subfig}
%\else
%  \usepackage[caption=false,font=footnotesize]{subfig}
%\fi
% subfig.sty, written by Steven Douglas Cochran, is the modern replacement
% for subfigure.sty, the latter of which is no longer maintained and is
% incompatible with some LaTeX packages including fixltx2e. However,
% subfig.sty requires and automatically loads Axel Sommerfeldt's caption.sty
% which will override IEEEtran.cls' handling of captions and this will result
% in non-IEEE style figure/table captions. To prevent this problem, be sure
% and invoke subfig.sty's "caption=false" package option (available since
% subfig.sty version 1.3, 2005/06/28) as this is will preserve IEEEtran.cls
% handling of captions.
% Note that the Computer Society format requires a larger sans serif font
% than the serif footnote size font used in traditional IEEE formatting
% and thus the need to invoke different subfig.sty package options depending
% on whether compsoc mode has been enabled.
%
% The latest version and documentation of subfig.sty can be obtained at:
% http://www.ctan.org/pkg/subfig




% *** FLOAT PACKAGES ***
%
%\usepackage{fixltx2e}
% fixltx2e, the successor to the earlier fix2col.sty, was written by
% Frank Mittelbach and David Carlisle. This package corrects a few problems
% in the LaTeX2e kernel, the most notable of which is that in current
% LaTeX2e releases, the ordering of single and double column floats is not
% guaranteed to be preserved. Thus, an unpatched LaTeX2e can allow a
% single column figure to be placed prior to an earlier double column
% figure.
% Be aware that LaTeX2e kernels dated 2015 and later have fixltx2e.sty's
% corrections already built into the system in which case a warning will
% be issued if an attempt is made to load fixltx2e.sty as it is no longer
% needed.
% The latest version and documentation can be found at:
% http://www.ctan.org/pkg/fixltx2e


%\usepackage{stfloats}
% stfloats.sty was written by Sigitas Tolusis. This package gives LaTeX2e
% the ability to do double column floats at the bottom of the page as well
% as the top. (e.g., "\begin{figure*}[!b]" is not normally possible in
% LaTeX2e). It also provides a command:
%\fnbelowfloat
% to enable the placement of footnotes below bottom floats (the standard
% LaTeX2e kernel puts them above bottom floats). This is an invasive package
% which rewrites many portions of the LaTeX2e float routines. It may not work
% with other packages that modify the LaTeX2e float routines. The latest
% version and documentation can be obtained at:
% http://www.ctan.org/pkg/stfloats
% Do not use the stfloats baselinefloat ability as the IEEE does not allow
% \baselineskip to stretch. Authors submitting work to the IEEE should note
% that the IEEE rarely uses double column equations and that authors should try
% to avoid such use. Do not be tempted to use the cuted.sty or midfloat.sty
% packages (also by Sigitas Tolusis) as the IEEE does not format its papers in
% such ways.
% Do not attempt to use stfloats with fixltx2e as they are incompatible.
% Instead, use Morten Hogholm'a dblfloatfix which combines the features
% of both fixltx2e and stfloats:
%
% \usepackage{dblfloatfix}
% The latest version can be found at:
% http://www.ctan.org/pkg/dblfloatfix




%\ifCLASSOPTIONcaptionsoff
%  \usepackage[nomarkers]{endfloat}
% \let\MYoriglatexcaption\caption
% \renewcommand{\caption}[2][\relax]{\MYoriglatexcaption[#2]{#2}}
%\fi
% endfloat.sty was written by James Darrell McCauley, Jeff Goldberg and 
% Axel Sommerfeldt. This package may be useful when used in conjunction with 
% IEEEtran.cls'  captionsoff option. Some IEEE journals/societies require that
% submissions have lists of figures/tables at the end of the paper and that
% figures/tables without any captions are placed on a page by themselves at
% the end of the document. If needed, the draftcls IEEEtran class option or
% \CLASSINPUTbaselinestretch interface can be used to increase the line
% spacing as well. Be sure and use the nomarkers option of endfloat to
% prevent endfloat from "marking" where the figures would have been placed
% in the text. The two hack lines of code above are a slight modification of
% that suggested by in the endfloat docs (section 8.4.1) to ensure that
% the full captions always appear in the list of figures/tables - even if
% the user used the short optional argument of \caption[]{}.
% IEEE papers do not typically make use of \caption[]'s optional argument,
% so this should not be an issue. A similar trick can be used to disable
% captions of packages such as subfig.sty that lack options to turn off
% the subcaptions:
% For subfig.sty:
% \let\MYorigsubfloat\subfloat
% \renewcommand{\subfloat}[2][\relax]{\MYorigsubfloat[]{#2}}
% However, the above trick will not work if both optional arguments of
% the \subfloat command are used. Furthermore, there needs to be a
% description of each subfigure *somewhere* and endfloat does not add
% subfigure captions to its list of figures. Thus, the best approach is to
% avoid the use of subfigure captions (many IEEE journals avoid them anyway)
% and instead reference/explain all the subfigures within the main caption.
% The latest version of endfloat.sty and its documentation can obtained at:
% http://www.ctan.org/pkg/endfloat
%
% The IEEEtran \ifCLASSOPTIONcaptionsoff conditional can also be used
% later in the document, say, to conditionally put the References on a 
% page by themselves.




% *** Do not adjust lengths that control margins, column widths, etc. ***
% *** Do not use packages that alter fonts (such as pslatex).         ***
% There should be no need to do such things with IEEEtran.cls V1.6 and later.
% (Unless specifically asked to do so by the journal or conference you plan
% to submit to, of course. )


% correct bad hyphenation here
\hyphenation{op-tical net-works semi-conduc-tor}


\begin{document}

\title{SSD: A State-based Stealthy Backdoor Attack For IMU/GNSS Navigation System in
UAV Route Planning}
%
%
% author names and IEEE memberships
% note positions of commas and nonbreaking spaces ( ~ ) LaTeX will not break
% a structure at a ~ so this keeps an author's name from being broken across


\author{Zhaoxuan~Wang,~\IEEEmembership{}
Yang ~Li,~\IEEEmembership{Member,~IEEE,}
Jie ~Zhang,~\IEEEmembership{}
Xingshuo  ~Han,~\IEEEmembership{}
Kangbo  ~Liu,~\IEEEmembership{}
Yang  ~Lyu,~\IEEEmembership{}
Yuan  ~Zhou,~\IEEEmembership{}
Tianwei  ~Zhang,~\IEEEmembership{Member,~IEEE,}
and~Quan ~Pan,~\IEEEmembership{Member,~IEEE.}% <-this % stops a space

\thanks{Zhaoxuan Wang is with School of Cybersecurity, Northwestern Polytechnical University, Xi'an 710129, China (e-mail: zxwang@mail.nwpu.edu.cn).}% <-this % stops a space

\thanks{Yang Li, Yang Lyu, Kangbo Liu and Quan Pan are with School of Automation, Northwestern Polytechnical University, Xi'an 710129, China (e-mail:liyangnpu@nwpu.edu.cn; liukangbo@mail.nwpu.edu.cn;  lyu.yang@nwpu.edu.cn quanpan@nwpu.edu.cn).}

\thanks{Jie Zhang is with CFAR and IHPC, Agency for Science, Technology and Research, Singapore. e-mail: (zhang\_jie@cfar.a-star.edu.sg).}

\thanks{Xingshuo Han and Tianwei Zhang are with College of Computing and Data Science, Nanyang Technological University, Singapore 639798 (e-mail: xingshuo001@e.ntu.edu.sg; tianwei.zhang@ntu.edu.sg).}% <-this % stops a space

%\thanks{Manuscript received April 19, 2005; revised August 26, 2015.}

\thanks{Yuan Zhou is with School of Computer Science and Technology, Zhejiang Sci-Tech University, Zhejiang 310018, China (email: yuanzhou@zstu.edu.cn).}

\thanks{Corresponding author: Yang Li.}

}



\markboth{}
%IEEE Transaction on Information Forensics and Security
{Shell \MakeLowercase{\textit{et al.}}: Bare Demo of IEEEtran.cls for IEEE Journals}
% The only time the second header will appear is for the odd numbered pages
% after the title page when using the twoside option.
% 
% *** Note that you probably will NOT want to include the author's ***
% *** name in the headers of peer review papers.                   ***
% You can use \ifCLASSOPTIONpeerreview for conditional compilation here if
% you desire.




% If you want to put a publisher's ID mark on the page you can do it like
% this:
%\IEEEpubid{0000--0000/00\$00.00~\copyright~2015 IEEE}
% Remember, if you use this you must call \IEEEpubidadjcol in the second
% column for its text to clear the IEEEpubid mark.



\maketitle

% As a general rule, do not put math, special symbols or citations
% in the abstract or keywords.
\begin{abstract}
Unmanned aerial vehicles (UAVs) are increasingly employed to perform high-risk tasks that require minimal human intervention. However, they face escalating cybersecurity threats, particularly from GNSS spoofing attacks. While previous studies have extensively investigated the impacts of GNSS spoofing on UAVs, few have focused on its effects on specific tasks. Moreover, the influence of UAV motion states on the assessment of cybersecurity risks is often overlooked. To address these gaps, we first provide a detailed evaluation of how motion states affect the effectiveness of network attacks. We demonstrate that nonlinear motion states not only enhance the effectiveness of position spoofing in GNSS spoofing attacks but also reduce the probability of detecting speed-related attacks. Building upon this, we propose a state-triggered backdoor attack method (SSD) to deceive GNSS systems and assess its risk to trajectory planning tasks. Extensive validation of SSD's effectiveness and stealthiness is conducted. Experimental results show that, with appropriately tuned hyperparameters, SSD significantly increases positioning errors and the risk of task failure, while maintaining high stealthy rates across three state-of-the-art detectors.
\end{abstract}

% Note that keywords are not normally used for peerreview papers.
\begin{IEEEkeywords}
Unmanned aerial vehicles, Cyber security, Backdoor attacks, GNSS spoofing.
\end{IEEEkeywords}






% For peer review papers, you can put extra information on the cover
% page as needed:
% \ifCLASSOPTIONpeerreview
% \begin{center} \bfseries EDICS Category: 3-BBND \end{center}
% \fi
%
% For peerreview papers, this IEEEtran command inserts a page break and
% creates the second title. It will be ignored for other modes.
\IEEEpeerreviewmaketitle


\section{Introduction}
% no \IEEEPARstart
Unmanned Aerial Vehicles (UAVs) are revolutionizing our understanding of low-altitude flight patterns.
Currently, UAVs are widely used in various military and civilian fields, such as courier delivery, agricultural plant protection, power patrol, firefighting, rescue, and battlefield reconnaissance. 
These increasingly complex application scenarios have necessitated stringent requirements for the autonomous flight capabilities of UAVs. 
As UAVs operate autonomously to execute their missions, the system must precisely determine its global position at a centimeter level.
As illustrated in Fig.~\ref{fig:route}, 
\begin{figure}[htp]
	\centering
 \centerline{\includegraphics[width=230pt]{./figure/planning.png}} 
		\caption{The Role of Localization and Route Planing in UAV Autonomous Flight}
		\label{fig:route}
\end{figure}
the localization capability is critical to route planning. It ensures the safety of flight and the ability to fulfill its mission, as positioning errors can directly cause the flight to deviate from course or fail to perform its mission. 



The Integrated Navigation System (INS) serves as the cornerstone of UAVs, enabling precise positioning. It accomplishes accurate position estimates by integrating data from various sensors. Integrating Inertial Measurement Units (IMUs) and Global Navigation Satellite Systems (GNSS) forms the fundamental and core navigation system in INS. Building on this foundation, researchers usually incorporated additional sensors, such as cameras, lidar, and radar to further improve positioning accuracy in different scenarios or platforms. However, direct reliance on sensor data and communication channels' noise makes INS vulnerable to cyber-attacks \cite{wang2023survey,wei2024survey}. Research has revealed that adversaries can attack INS by using adversarial examples \cite{liu2023rpau} or wireless signal injection \cite{kim2018low} to spoof sensors. Notably, GNSS is a particularly prevalent threat since it forms the foundation of INS. The attacker can leverage low-cost devices to manipulate the position and velocity measurement captured by GNSS.

Previous attacks can be classified into the following categories:(1) \textbf{Direct Attacks}\cite{tang2023gan}
: The adversary directly injects a false signal into the GNSS sensor. (2) \textbf{Stealthy Attacks} \cite{khazraei2024black,ma2024novel}: The adversary takes the detector and corresponding threshold as the constraint and computes an optimization-based payload to bypass the detectors. 

However, these attacks have the following limitations.\textbf{(1) Easily detectable} 
Sensor data fluctuations during UAV attacks are typically significant~\cite{wu2023highly}, prompting residual-based detection methods. However, our experiments show these detectors can almost always detect direct attacks. Moreover, during UAV maneuvers, such as waypoint course changes or formation shifts, the increased fluctuations make attacks even more easily identifiable.
% Sensor data fluctuates significantly when a UAV is attacked. Based on this phenomenon, researchers usually design detection variables based on residual variables and use their consistency to detect attacks. However, our experiments demonstrate that such detection methods can almost completely detect existing attacks. In addition, UAV flights are often accompanied by a large number of maneuvers, such as changing course at waypoints and switching positions in formations. In these scenarios, the fluctuations of the detection variables increase further, making the attacks even easier to detect. 
\textbf{(2) Computation efficiency} While stealthy attacks can bypass detectors through constrained optimization models, they typically require extensive matrix operations to solve high-dimensional optimization problems and derive the optimal attack payload in time. These methods introduce significant computational delays in environments with limited resources, especially affecting the attack's real-time performance.
\textbf{(3) Inadequate analysis of dynamic vulnerabilities} 
Research~\cite{shen2020drift} has shown that navigation algorithms in autonomous vehicles are vulnerable to uncertainty during specific periods. UAVs, operating with six degrees of freedom, experience frequent motion changes that affect system stability, especially during GNSS spoofing. Despite this, few studies evaluate how these dynamic motion states influence attack effectiveness.
% Previous research has demonstrated that navigation algorithms experience heightened uncertainty during specific periods, which can create significant vulnerabilities for autonomous driving vehicles. UAVs operate with six degrees of freedom, and their motion can change frequently during various missions. These dynamic changes impact system stability and increase the uncertainty in navigation algorithms. When an attacker performs GNSS spoofing, the frequent changes in motion states, particularly acceleration, may become vulnerabilities that malicious attackers exploit. However, few scholars have conducted detailed evaluations of the effectiveness of such attacks under changing motion states, especially in the context of UAVs.  
% evaluating the attack's effectiveness without considering the dynamic motion state is insufficient.
\textbf{(4) Incomplete assessment methodology} 
Prior studies \cite{kerns2014unmanned,vervisch2017influence} often focus on immediate attack outcomes, such as crashes or path deviations, but fail to assess how attacks affect UAV mission performance and overall efficacy. This leaves a gap in understanding the broader impacts of such attacks on UAV operations.
% Previous studies have primarily focused on visualizing the immediate outcomes of an attack, such as whether the UAV crashed, whether security mechanisms were effectively triggered, or whether it deviated from its intended flight path. However, UAVs are essentially utilized to perform particular tasks effectively. These studies do not thoroughly evaluate how an attack impacts the UAV's mission performance and overall efficacy, neglecting to completely reveal the more profound consequences of these attacks.

To overcome these limitations, and gain a deep understanding of the INS vulnerability posed by GNSS attacks, we first provided a detailed interpretable analysis of the relationship between motion states and GNSS spoofing attacks. Specifically, we assess the effects of linear and nonlinear motion states on attack effectiveness. Our findings reveal that changes in motion states amplify the effectiveness of positional attacks and increase the stealthiness of velocity attacks.
% We examine the impact of linear and nonlinear motion states on attacks individually and observe that changes in motion states enhance the effectiveness of positional attacks and improve the stealthiness of velocity attacks. 
We then proposed SSD, a novel stated-based stealthy backdoor attack for GNSS.
Backdoor attacks \cite{chen2022clean,chenbadpre} are a common threat in deep neural networks, where an adversary implants a latent backdoor that remains inactive under normal conditions but is triggered by specific inputs or scenarios, leading to incorrect model predictions. 
Inspired by this concept, we design backdoors for GNSS by using motion state changes as a trigger to initiate staged velocity and positional attacks. In contrast to the stealthy attacks \cite{khazraei2024black,ma2024novel}, this attack mode doesn't need prior knowledge for the detectors and eliminates the need for complex computations. Instead, it is a direct attack that leverages carefully configured parameters and straightforward function calculations to achieve an optimal balance between effectiveness and stealthiness. This makes SSD highly valuable for engineering applications.
Lastly, we selected three representative mission trajectories to assess the effectiveness of SSD. Its performance was compared against existing attack methods. The experimental results demonstrate that SSD maintains detection variables consistently within the threshold range in three classical detectors. 
Furthermore, we introduced evaluation metrics to measure the attack's impact on mission success rates and effectiveness. The experiments also reveal that SSD significantly increases the localization error, thereby effectively disrupting mission completion.

In summary, our contributions are summarized as follows:

\begin{itemize}
    \item We present an interpretable mathematical security study of how motion states influence attack outcomes and demonstrate that UAVs are more vulnerable during maneuvers than in uniform linear flights. We further experimentally prove it.
    
    \item We design SSD, a novel stated-based backdoor attack that utilizes motion state as a trigger to spoof GNSS data. It can execute velocity and positional attacks in stages to simultaneously and covertly attack both states. 
    
    \item We conduct experiments in classic specific mission trajectories and find that SSD can significantly reduce mission completion rates and maintain constant stable stealthiness under attack detection.
    
\end{itemize}

\section{Related Works and Background}
\subsection{UAV Route Planning and Integrated Navigation System}
\label{ins}
UAV route planning involves designing a feasible route from the start point to the destination while meeting all constraints and performance requirements \cite{abdel2024multiobjective}. Effective route planning is crucial for UAVs to complete their missions, and it depends on accurate position estimation. Since UAVs often operate in dynamic and complex environments, they rely on multi-sensor fusion to enhance their ability to perceive the environment. This approach integrates data from various sensors with different modalities and attributes, increasing redundancy and improving reliability in challenging conditions. The integrated navigation system of GNSS and IMU is a typical representative example. Its fusion strategy uses the GNSS data for quantitative updating, IMU data for state prediction, and an optimal estimation framework to achieve accurate positioning in the global coordinate system \cite{qi2002direct,carvalho1997optimal}. 
In GNSS-denied environments, simultaneous localization and mapping (SLAM) that rely on camera~\cite{davison2007monoslam,forster2016svo} and lidar~\cite{zhang2014loam,nguyen2021miliom} are considered more reliable solutions. However, a single sensor alone cannot fully meet the demands for positioning accuracy and response speed. Therefore, combining these sensors with an inertial measurement unit to create IMU/Camera~\cite{qin2018vins,eckenhoff2021mimc} and IMU/Lidar~\cite{xu2022fast} fusion form enables more precise position estimation and improved performance in dynamic environments. This paper focuses on the security analysis of IMU/GNSS INS since it plays a central role in UAVs. It is of generic meaning to study its security.


\subsection{GNSS Spoofing attack}
\label{attack}
Since IMUs are more difficult to manipulate in real-world scenarios, we only discuss GNSS spoofing attacks for IMU/GNSS INS on UAVs. In such attacks, the adversary transmits false location coordinates to the GNSS receiver, thereby concealing the UAV's true location. As a result, unknowingly accepting these false inputs, the navigation system calculates incorrect position information. Specifically, GNSS spoofing includes direct attacks and stealthy attacks. The direct attacks \cite{tang2023gan} can be classified into these categories. (1) \textbf{
Biased Signal Attack}
: These attacks involve adding a bias to the GNSS sensor signals, typically following a uniform distribution. (2) \textbf{Multiplicative Attacks}: In these attacks, the GNSS signals are multiplied by a constant factor, effectively scaling the original signal values. (3) \textbf{Replacement Attacks}: These attacks involve directly replacing the GNSS signals with false or manipulated data. Direct attacks are easy to implement and may lead to disastrous consequences, such as the UAV crashing into obstacles \cite{kerns2014unmanned,vervisch2017influence}. However, most detectors can detect and respond in time. The stealthy attacks~\cite{khazraei2024black,ma2024novel} are diverse, with attackers often aiming to maximize navigation residuals to determine the optimal attack sequence. The design of these attacks may cause the UAV to fall into the malicious attackers’ control~\cite{he2018friendly}. However, it is largely influenced by the detection mechanism and often needs complex computing. SSD integrates the strengths of both approaches. It achieves the same effectiveness as stealthy attacks while requiring low computational resources.

To mitigate this threat, robust countermeasures, such as software analysis \cite{ceccato2021generalized}, cryptography-based authentication \cite{bonior2017implementation} and machine learning-based detections \cite{liang2019detection,han2022ads}, have been implemented to safeguard against GNSS spoofing and ensure the integrity of UAV INS. One important method is multi-sensor fusion \cite{yang2021secure}. As described in section \ref{ins}, it not only provides more accurate estimates for perception and localization. but also enhances the data's trustworthiness, providing greater redundancy for detection and defense in the event of spoofing attacks. For example, Shen et al. \cite{shen2020drift} demonstrated that the effectiveness of constant offset GNSS spoofing attacks is greatly reduced in GNSS/INS/LiDAR fused navigation systems in autonomous driving. However, such multi-sensor fusion strategies also face a period of vulnerability and can not defend against constructed GNSS spoofing against the uncertainty that exists in the fusion algorithm itself. In this paper, we demonstrate a similar phenomenon in UAVs, where changes in motion states significantly amplify the uncertainty of the INS, making it more vulnerable to GNSS attacks. Therefore, we provide an in-depth analysis of the uncertainty and vulnerability and exploit it to design SSD.

\subsection{Threat Model}

\textbf{Attack Goal} As shown in Figure \ref{threat}, the adversary aims to make the drone deviate significantly from its pre-planned route without triggering the stealth detection threshold. This objective can be formalized as the following optimization problem:
% (stealthiness defination)

\begin{equation}
    \begin{aligned} \label{P}
&\text{argmax}_{\delta} \quad \sum_{i=0}^{T_a} \mathbf{D^t_i-D^a_i} \\
&\begin{array}{r@{\quad}r@{}l@{\quad}l}
s.t. &\chi_k&\leq \tau \quad \forall k \in \{1,...,T_{a}-1\} \\
\end{array}
\end{aligned}
\end{equation}
where $\mathbf{D^t_i}$ and $\mathbf{D^a_i}$ denote the normal trajectory and the attacked trajectory, respectively. $\delta$ is the attack payload. $\chi_k$ is the in detection statistics and $\tau$ is the threshold of detectors.

\begin{figure}[htp]
	\centering
 \centerline{\includegraphics[width=230pt]{./figure/att.png}} 
		\caption{Threat Model}
		\label{threat}
\end{figure}


\textbf{Attack Scenario} As shown in Fig. \ref{threat}, an attacker can launch an attack in two ways: \ding{172} The attacker can use a UAV to fly alongside the victim's UAV. He can transmit legitimate GNSS signals completely using wireless attack devices such as software-defined radios (SDR). \ding{173} The attacker could also inject specific backdoors \cite{IRAHUL} or viruses~\cite{dji} into the UAV by supply chain attacks, which could be used to monitor its dynamics and induce a false GNSS position \cite{xu2021novel}.

\textbf{Attacker's Capability}
1) The attackers need white-box access to obtain the victim's navigation algorithms and corresponding parameters. They can get this knowledge through open-source channels since most UAVs use standardized open-source navigation algorithms~\cite{ardupilot,px4}. Also, the adversarial can use reverse engineering to access the victim's knowledge. 2) The attacker can obtain the victim's motion state, such as position and velocity. This can be achieved by monitoring UAVs using an additional GPS module or auxiliary object detection and tracking devices. 3) As IMU data is less likely to be accessed and used, an attacker can only modify the position and velocity since GNSS measurements only provide position and velocity to the UAV.

\section{Preliminary}
\subsection{IMU/GNSS Integrated Navigation System}
\label{cns}
 In UAVs, IMU and GNSS are often combined for highly accurate and robust navigation and positioning. IMU provides data from accelerometers and gyroscopes, which measure the acceleration and angular velocity of the UAVs. At the same time, GNSS determines the UAV's position, velocity, and time information by receiving satellite signals. This type of navigation system is known as a combined IMU/GNSS navigation system.

% Please add the following required packages to your document preamble:
% \usepackage{multirow}


IMU/GNSS INS uses a 22-axis Extended Kalman Filter (EKF) structure to estimate pose in the NED reference frame. The state is defined as $\hat{X}_k=\{\hat{x}_1, \ldots, \hat{x}_n| n \in (1,22)  \}$, where the definition of each axis $\hat{x}_i$ is illustrated in Table~\ref{axis}.

\begin{table}[htb]
 \centering \caption{Element and Meaning of EKF Vector}
 \label{axis}
\begin{tabular}{c|c|c}
    \toprule[1.5pt]
    \textbf{Element} & \textbf{Label} & \textbf{Meaning}  \\
    \midrule[1pt] 
    
   $\hat{x}_1$& $\mathbf{q_0}$ &  \multirow{4}{*}{\makecell[c]{Orientation quaternion. }}\\\cline{1-2} 
   	
   $\hat{x}_2$&$\mathbf{q_1}$ & \\\cline{1-2} 
    
   $\hat{x}_3$& $\mathbf{q_2}$& \\\cline{1-2} 

   $\hat{x}_4$& $\mathbf{q_3}$&   \\\hline
   $\hat{x}_5$&$\mathbf{P_N}$ & \multirow{3}{*}{\makecell[c]{ UAV Position in local \\ NED coordinate system.} }  \\\cline{1-2} 
   $\hat{x}_6$&$\mathbf{P_E}$&   \\\cline{1-2} 
   $\hat{x}_7$&$\mathbf{P_D}$&   \\\hline
   $\hat{x}_8$& $\mathbf{V_N}$& \multirow{3}{*}{\makecell[c]{ UAV Velocity in local\\ NED coordinate system.}  }\\\cline{1-2}
   $\hat{x}_9$&$\mathbf{V_E}$ &   \\\cline{1-2}
   $\hat{x}_{10}$&$\mathbf{V_D}$ &   \\\hline
   $\hat{x}_{11}$& $\mathbf{\Delta \theta bias_X}$ &  \multirow{3}{*}{ \makecell[c]{Bias in integrated \\ gyroscope reading.}} \\\cline{1-2}
    $\hat{x}_{12}$& $\mathbf{\Delta \theta bias_Y}$ &   \\\cline{1-2}
    $\hat{x}_{13}$& $\mathbf{\Delta \theta bias_Z}$&   \\\hline
    $\hat{x}_{14}$& $\mathbf{\Delta v bias_X}$& \multirow{3}{*}{\makecell[c]{Bias in integrated \\ accelerometer reading.}}  \\\cline{1-2}
    $\hat{x}_{15}$& $\Delta v bias_Y$&   \\\cline{1-2}
    $\hat{x}_{16}$& $\mathbf{\Delta v bias_Z}$&   \\\hline
    $\hat{x}_{17}$& $\mathbf{geomagneticField_N}$ &  \multirow{3}{*}{ \makecell[c]{Estimate of geomagnetic \\field vector at the\\ reference location.}} \\\cline{1-2}
    $\hat{x}_{18}$& $\mathbf{geomagneticField_E}$&   \\\cline{1-2}
    $\hat{x}_{19}$& $\mathbf{geomagneticField_D}$&   \\\hline
    $\hat{x}_{20}$& $\mathbf{magbias_X}$ & \multirow{3}{*}{\makecell[c]{Bias in the \\magnetometer readings.}}  \\\cline{1-2}
    $\hat{x}_{21}$& $\mathbf{magbias_Y}$ &   \\\cline{1-2}
    $\hat{x}_{22}$& $\mathbf{magbias_Z}$ &   \\
   \bottomrule[1.5pt]
     \end{tabular}
\end{table}

Firstly, The system model $f(\cdot)$ uses the estimated previous state $\hat{X}_{k-1}$ and a control input $u_{k-1}$, to predict the current state $\hat{X}_k^-$.
\begin{equation}
\label{f}
\hat{X}_k^- = f(\hat{X}_{k-1}, u_k)
\end{equation}
where $u_k$ are control inputs, typically angular velocity and acceleration data from IMU. 

After prediction, INS predicts the measurement $z_k$ at time $k$ for updating the current state. we define as: 
\begin{equation}
\label{z}
    z_k=h(\hat{X}_k)+v_k
\end{equation}
where $z_k$ comprises magnetic field data from magnetometers, gravitational acceleration data from accelerometers, and position and velocity from GNSS. $h(\cdot)$ is the measurement prediction model and $v_k$ is the observational noise.

Due to hardware arithmetic limitations, INS often handles nonlinear functions by truncating their Taylor expansions with first-order linearization and neglecting the higher-order terms. This approach transforms the nonlinear problem into a linear one, as exemplified by Eq. \ref{nolinerf1} and \ref{nolinerf2}:
\begin{equation}
\label{nolinerf1}
f(X_{k-1}, u_k) \approx f(\hat{X}_{k-1}, u_k) + \frac{\partial f}{\partial x}\bigg|_{\hat{X}_{k-1}, u_k} (X_{k-1} - \hat{X}_{k-1})
\end{equation}

\begin{equation}
\label{nolinerf2}
h(X_k) \approx h(\hat{X}_k^-) + \frac{\partial h}{\partial x}\bigg|_{\hat{X}_k^-} (X_k - \hat{X}_k^-)
\end{equation}
INS defines the state transfer matrix $F_K$ and the Jacobi matrix of the measurement $H_k$ respectively:
\begin{equation}
    F_k = \frac{\partial f}{\partial x}\bigg|_{\hat{X}_{k-1}, u_k}
\end{equation}

\begin{equation}
     H_k = \frac{\partial h}{\partial x}\bigg|_{\hat{X}_k^-} 
\end{equation}

Based on Eq. \ref{f} and \ref{z}, we can derive the priori estimated covariance matrix  $P_k^-$ at time $k$.
\begin{equation}
\label{e}
    e_k=\hat{X}_k - \hat{X}_k^-
\end{equation}
\begin{equation}
\label{p}
\begin{split}
   P_k^- &=E(e_k e_k^T)\\
   & = F_k P_{k-1} F_k^T + Q_k
\end{split}
\end{equation}
where $Q_k$ refers to the process noise covariance matrix at time step $k$.

When obtaining the measurement $z_k$, the system will calculate the Kalman gain $K_k$ and update the estimated state to obtain an accurate estimation of the state information in the following way:
\begin{equation}
\label{K}
K_k = P_k^- H_k^T (H_k P_k^- H_k^T + R_k)^{-1}
\end{equation}

\begin{equation}
\hat{X}_k = \hat{X}_k^- + K_k(z_k - h(\hat{X}_k^-))
\end{equation}

\begin{equation}
P_k = (I - K_k H_k) P_k^-
\end{equation}
where $R_k$ refers to the measurement noise covariance matrix at time step $k$.


\subsection{Detector}
In dynamic system state estimation, the EKF optimizes the estimation of the system state successively through prediction and update steps. It computes the residual $r(k)=z_k - h(\hat{x}_k^-)$ in each step to reflect the difference between the actual measured value and the predicted value. With no attacks or anomalies, $r(k)$ will be presented as a zero-mean Gaussian distribution with a covariance matrix $Pr:=H_k P_k H_k^T+R_k$.

However, the system may generate outliers due to attacks, noise, faults, and other factors, all of which contribute to the measurements deviating from the true values. To prevent the EKF state from these disruptive outliers, implementing an outlier detection mechanism becomes crucial. 
The chi-square statistical test serves as an efficient tool for determining outliers~\cite{schreiber2016vehicle,piche2016online}. It evaluates the current measured value by calculating the chi-square statistic $\chi^2_k$, comparing it to a predefined statistical significance threshold. When $\chi^2_k$ surpasses this threshold $\tau$, the measurement is considered an outlier, and suitable measures are undertaken, including discarding the measurement or executing a partial update. The chi-square statistic $\chi^2_k$ is defined as:

\begin{equation}
\begin{aligned}
\chi^2_k = r(k)^T S_k r(k)
\\
S_k=(H_k P_k^- H_k^T + R_k)
\end{aligned}
\end{equation}

\section{Security Analysis}
\subsection{Attack Formulation}
Viewed from the perspective of navigation equations, the process of a spoofing attack on GNSS signals by an attacker can be described as follows: the attacker injects $n$ spoofing signal $\{\delta _k|k=1,\dots,n\}$ into the measurement data, resulting in a modification of the measurement $h$ as follows.
\begin{equation}
    z_k=h(\hat{X}_k)+\delta _k+v_k
\end{equation}

Due to the higher occurrence of data errors in the GNSS \textit{z-axis} and the availability of alternative altitude data sources, only the \textit{NE} (North-East) directional updates are applied to the position vector. As for the attacker, they can only modify the position and speed provided by GNSS, i.e., dimensions 5-6 and 8-10 in Table \ref{axis}.
%
%\begin{equation}
 %   \dot{\mathbf{x}} = f(\mathbf{x}, \mathbf{u}) + \mathbf{w}, \quad \mathbf{w} \sim \mathcal{N}(0, Q)
%\mathbf{z}_k = h(\mathbf{x}_k) + \mathbf{v}_k, \quad \mathbf{v}_k \sim \mathcal{N}(0, R)
%\end{equation}
%\begin{equation}
%V(e) = e^T P^{-1} e
%\dot{V}(e) = e^T (F^T P^{-1} + P^{-1} F) e + 2e^T P^{-1} K \delta
%\end{equation}
%\begin{equation}
%\exists \alpha > 0 \quad s.t. \dot{V}(e) \leq -\alpha V(e) + \beta \|\delta\|
%\end{equation}
%
\subsection{Study of Attack}
\label{sec:study}
The GNSS observation matrix $H_{GNSS}$ are as follows.
\begin{equation}
 H_{GNSS}= 
     \begin{bmatrix}
        0_{1\times4}&1 & 0& 0&0_{1\times3} & 0_{1\times14}\\ 
        0_{1\times4}& 0& 1& 0& 0_{1\times3} & 0_{1\times14} \\
        0_{1\times4}& 0 & 0 & 0&0_{1\times3} &0_{1\times14} \\
        0_{3\times4}&0_{3\times1} &0_{3\times1} & 0_{3\times1}&I_{3\times3} & 0_{3\times14} \\
     \end{bmatrix}
\end{equation}

Since $H_{GNSS}$ is a sparse matrix, when updating position and velocity, $K_k$ can be simplified to the following form:
\begin{equation}
\label{k1}
\begin{split}
    K_{k} &= P_k^- H_{GPS}^T (H_{GPS} P_k^- H_{GPS}^T + R_k)^{-1} \\
    &=(P_{k-1}+Q_k)(P_{k-1}+Q_k+R_k)^{-1} \\
    &=I-R_k(P_{k-1}+Q_k+R_k)^{-1}
\end{split}
\end{equation}

 From Eq.\ref{k1}, we can see that the state transfer error and the measurement error affect the magnitude of the gain $K_k$ simultaneously. \textbf{$Q_k$ and $R_k$ reflect the ability to cover systematic uncertainty and measurement uncertainty, respectively}. Therefore, inappropriate $Q_k$ and $R_k$ can lead to filter divergence or biased estimation. However, in INS, $Q_k$ and $R_k$ are generally determined based on a priori knowledge by pre-running the filter calculations offline and remain constant during the filtering process online. As a result, both the process estimation error covariance $R_k$ and the Kalman gain $K_k$ converge quickly and remain constant during the process, demonstrating that the value of $K_k$ is determined by the ratio of $Q_k$ and $R_k$.
 
We assume the adversarial adds an  $\delta_i$ at time $i$. The prediction equation for the EKF becomes
\begin{equation}
\begin{split}
    \hat{x}_i^a &= \hat{x}_i^- + K_i(z_i + \delta _i - h(\hat{x}_i^-, 0))\\
    &=\hat{x_i}+K_i \delta _i
\end{split}
\end{equation}

\begin{equation}
P_k = (I - K_k H_k) P_k^-
\end{equation}
Therefore, when the UAV performs maneuvers, the impact of spoofing on localization results can be described in the following two ways: 
\begin{itemize}
    \item \textbf{Q uncertainty:} EKF uses Euler integrals to update the positional status, i.e:
    \begin{equation}
       \begin{bmatrix}  
    P_{N} \\ P_{E} \\ P_{D}\\  
  \end{bmatrix}_{i+1} =
  \begin{bmatrix}  
    P_{N} \\ P_{E} \\ P_{D}\\  
  \end{bmatrix}_i+
  \begin{bmatrix}  
    V_{N} \\ V_{E} \\ V_{D}\\  
  \end{bmatrix}_i
  \Delta t
    \end{equation}
    Firstly, when performing maneuvers, the system is highly nonlinear. i.e.,
    \begin{equation}
     \exists \epsilon >0, \|\frac{d\mathbf{v_i}}{dt}\| \geq \epsilon 
    \end{equation}
    The acceleration $\mathbf{a_i \neq 0}$ and the update equation for position essentially becomes:
        \begin{equation}
       \begin{bmatrix}  
    P_{N} \\ P_{E} \\ P_{D}\\  
  \end{bmatrix}_{i+1} =
  \begin{bmatrix}  
    P_{N} \\ P_{E} \\ P_{D}\\  
  \end{bmatrix}_k +
  \begin{bmatrix}  
    V_{N} \\ V_{E} \\ V_{D}\\  
  \end{bmatrix}_i
  \Delta t+
   \begin{bmatrix}
      \frac{1}{2}\Delta t^2,0,0 \\ 0,\frac{1}{2}\Delta t^2,0 \\ 0,0,\frac{1}{2}\Delta t^2\\
  \end{bmatrix}
  \begin{bmatrix}
      a_{N} \\ a_{E} \\ a_{D}\\
  \end{bmatrix}
  \end{equation}
    According to Eq. \ref{e} and \ref{p}, the accumulation of linearisation errors will increase $e_i$ and thus increase the process noise $P_i$. leading to inaccurate mathematical modeling and huge nonlinear errors. The fixed $Q_k$ makes it difficult to suppress the nonlinear error increased due to the change of motion state. According to Eq. \ref{k1}, the value of $K_i$ will indirectly increase, making the INS more inclined to trust the GNSS data.  In addition, this complex nonlinear characteristic will be further expanded due to physical factors such as the lag of IMU data and the presence of friction in the gyroscope. Therefore, in this scenario, the prediction of the system model cannot effectively reflect the actual physical process.
    
    \item \textbf{R uncertainty:} The modification of GNSS results in a shift in the measurement data distribution, rendering a fixed $R_k$ inadequate for accurately describing the measurement noise distribution. Consequently, $\delta_i$ experiences a significant increase. Additionally, the rising $K_k$ value leads the system to place greater trust in GNSS, further amplifying the impact of the attack associated with $\delta_i$.
\end{itemize}


We employ a biased signal attack and a multiplicative attack to evaluate the phenomenon above. The UAV maintains a constant velocity of $5m/s$ and performs two typical motion modes, uniform linear motion (linear motion) and uniform circular motion (non-linear motion), respectively. For each flight state, we apply an attack window of two attack inputs for the GNSS respectively and observe the changes in the localization Error $LocErr$ before and after the attack. As a result, the attack time is 2 seconds since the GNSS input is 1 Hz. The experimental results are shown in Fig. \ref{fig:finding11} and \ref{fig:finding12}. Thus, we can get Finding 1.

\begin{figure*}[htbp]
	\centering
	\begin{subfigure}{0.24\linewidth}
		\centering
		% \includegraphics[width=0.9\linewidth]{./figure/find111.png}
        		\includegraphics[width=0.9\linewidth]{./figure/FIG3-A.png}
		\caption{Bias Signal Attack}
		\label{fig:finding11}
	\end{subfigure}
	\centering
	\begin{subfigure}{0.24\linewidth}
		\centering
		% \includegraphics[width=0.9\linewidth]{./figure/maatt1.png}
        \includegraphics[width=0.9\linewidth]{./figure/FIG3-B.png}
		\caption{Multiplicative Attack}
		\label{fig:finding12}%文中引用该图片代号
	\end{subfigure}
    	\centering
	\begin{subfigure}{0.24\linewidth}
		\centering
		% \includegraphics[width=0.9\linewidth]{./figure/finding22.png}
        \includegraphics[width=0.9\linewidth]{./figure/FIG3-C.png}
		\caption{Positional Attack}
		\label{fig:study21}
	\end{subfigure}
	\centering
	\begin{subfigure}{0.24\linewidth}
		\centering
		% \includegraphics[width=0.9\linewidth]{./figure/finding21.png}
        \includegraphics[width=0.9\linewidth]{./figure/FIG3-D.png}
		\caption{Velocity Attack}
		\label{fig:study22}%文中引用该图片代号
	\end{subfigure}
 \caption{Study of GNSS Attack under Different Motion states}
	\label{fig:study}
\end{figure*}


\begin{tcolorbox}[left=1mm, right=1mm, top=0.5mm, bottom=0.5mm, arc=1mm]
\textbf{Finding 1:} \textit{For a GNSS spoofing of the same magnitude, applying it during the UAV's non-linear motion results in more pronounced fluctuations in positioning accuracy compared to linear motion. These changes in motion dynamics create greater vulnerabilities for INS.}
\end{tcolorbox}

Position and velocity are tightly coupled in INS, making velocity attacks easier to modify the position result.  However, there has been limited research on attacks targeting velocity. We would like to explore one question: Are there obvious
correlations between the velocity attack stealthiness and the UAV motion states? To validate this question, we select two flight trajectories with linear and nonlinear motion states respectively. We use a biased signal attack to perturb velocity and position measurement and evaluate stealthiness by observing the changes in the chi-square detector. Fig. \ref{fig:study21} and \ref{fig:study22} indicate that the detector shows no significant fluctuations between the two motion states for the positional attack. However, velocity attacks demonstrate a greater sensitivity to motion states, with significantly lower cardinality detector results observed in nonlinear motion states. This reveals key insights, summarized as Finding 2.

\begin{tcolorbox}[left=1mm, right=1mm, top=0.5mm, bottom=0.5mm, arc=1mm]
\textbf{Finding 2:} \textit{
The nonlinear motion state does not significantly affect the stealthiness of positional attacks, but it enhances the stealthiness of velocity attacks.}
\end{tcolorbox}

\section{Attack Design}
From the analysis of Section \ref{sec:study}, we observe that UAV exhibits greater vulnerability in a non-linear motion state compared to a linear one due to the combined effects of model and measurement uncertainty. During non-linear motion, an attack of the same magnitude can produce a more significant change in the navigation output than in the linear motion state. However, this vulnerability arises only when the uncertainty is heightened due to changes in the motion state. Moreover, \textbf{Finding2} indicates that a positional attack in non-linear motion will increase the risk of being detected. This introduces a key challenge to the attacker:  

\textbf{C1: How to opportunistically exploit these vulnerable periods to achieve maximum localization error while maintaining stealth.}

To address \textbf{C1}, we design a backdoor-like attack to exploit these vulnerable periods directly.  Inspired by the backdoor attacks in deep networks (DNNs)~\cite{chen2022clean,han2022physical,han2024backdooring}, we proposed a novel stated-based stealthy backdoor (SSD) attack against INS in a route planning scenario. SSD utilizes the motion state as a trigger, enabling the UAV to trigger an attack in a nonlinear motion state while maintaining normal operation in a linear motion state. Based on this design, not only can the detection rates of the attack be greatly reduced, but also the mission completion rate of the UAV can be effectively reduced. (UAV mission completion is usually accompanied by large maneuvers.)

\begin{equation}
\label{line}
\exists c \in \mathbb{R}, \|\frac{d\mathbf{v_i}}{dt}\| \leq c \quad \forall t \in [t_0, t_1]
\end{equation}
Specifically, when the victim UAV is in a stable linear flight state (Equation~\ref{line}), SSD adds a bias $F(t_i;\theta,\alpha)$ to attack the position stealthily. 
\begin{equation}
     F(t_i;\theta,\alpha)=\theta {e^{t_i/\alpha}}
\end{equation}
When it is maneuvering, each dimension of the UAV's movement can experience both positive and negative acceleration, indicating acceleration and deceleration in that particular direction. We define the acceleration direction $\Vec{a}$ as:
\begin{equation}
\Vec{a}^d_i = \Vec{\frac{\partial v_i}{\partial t_i}}
\end{equation}
When the UAV undergoes acceleration in this dimension (i.e. $\Vec{a}^d_i$ \textgreater $0$ ), SSD makes smooth changes to the velocity by multiplying a stealthy velocity bias $G(t_i,a^d_i;\phi)$ to perturb the localization result. 
\begin{equation}
     G(t_i,a^d_i;\phi)=\log_{2}{(2+\phi \Vec{a}^d_i t_i)}
\end{equation}
Overall, SSD can be formalized as follows:
\begin{equation}
\begin{cases}
       X_i=X_i+ F(t_i;\theta,\alpha)
       \\
       V_i=V_i* G(t_i,a^d_i;\phi)
\end{cases}
\end{equation}
where $t_i$ is the attack time for attackers. $\theta$, $\alpha$, and $\phi$ are hyperparameters that can be dynamically adjusted to maintain an equilibrium between stealthiness and effectiveness. We will demonstrate how to configure these parameters in Section~\ref{pa}. SSD chooses acceleration as a trigger. It uses organic coupling between velocity and position to perform a combined attack. Its pseudocode is presented in Algorithm \ref{SSDa}.

\begin{algorithm}
    \caption{SSD}%算法名字
    \label{SSDa}
    \LinesNumbered %要求显示行号
    \KwIn{Victim UAV position $P_i=(P_{Ni}, P_{Ei}, P_{Di})$, Victim UAV velocity  $V_i=(V_{Ni}, V_{Ei}, V_{Di})$, IMU sampling frequency $F_{imu}$, Iteration number $M$}
    % \KwOut{output result}%输出
    Set initialize hyperparameters $\theta$,$\phi$ and $\alpha$ \;
    \For{i=1 to M}{
    \For{j=1 to $F_{imu}$}{
       Receive IMU data \;
       Predict $X_{i+1}$ using IMU data;  
    }
     Receive victim UAV velocity  $V_i$ and position $P_i$ from GNSS  \;
        Compute acceleration $a_i=(\frac{\partial V_{Ni}}{\partial t_i}, \frac{\partial V_{Ei}}{\partial t_i}, \frac{\partial V_{Di}}{\partial t_i})$\;
             \If{$||a_i||_2$=0}{
              // \emph{Apply position perturbation in linear motion state} \;
                 $P_i=P_i+ \theta {e^{ti/\alpha}}$\;
                  
                
            }
        \Else{
        // \emph{Apply velocity perturbation in nonlinear motion state} \;
        $ V_{Ni}=V_{Ni}* \log_{2}{(2+{\frac{\partial v_{Ni}}{\partial t}}\phi t)}$\;
        $V_{Ei}=V_{Ni}* \log_{2}{(2+{\frac{\partial v_{Ei}}{\partial t}}\phi t)}$\;
        }
    Fuse $P_i$ and $V_i$ to update $X_{i+1}$
    }
    
\end{algorithm}

\begin{figure*}[htp]
	\centering
 \centerline{\includegraphics[width=380pt]{./figure/overview.png}} 
		\caption{Overview of SSD}
		\label{fig:ssd}
\end{figure*}

\section{Experiment}
\subsection{Experimental Setup}
\subsubsection{Trajectory Dataset}
 We used three representative task trajectories. Trajectories \uppercase\expandafter{\romannumeral1} and \uppercase\expandafter{\romannumeral2} represent typical linear and non-linear motion states, respectively, while trajectory \uppercase\expandafter{\romannumeral3} combines both motion states.

\begin{figure*}[htbp]
	\centering
	\begin{subfigure}{0.325\linewidth}
		\centering
		\includegraphics[width=0.9\linewidth]{./figure/line.png}
		\caption{ Straight-line Path}
		\label{fig:t1}%文中引用该图片代号
	\end{subfigure}
	\centering
	\begin{subfigure}{0.325\linewidth}
		\centering
		\includegraphics[width=0.9\linewidth]{./figure/circle.png}
		\caption{Spiral Path}
		\label{fig:t2}%文中引用该图片代号
	\end{subfigure}
	\centering
	\begin{subfigure}{0.325\linewidth}
		\centering
		\includegraphics[width=0.9\linewidth]{./figure/ushape.png}
		\caption{U-shape Path}
		\label{fig:t3}%文中引用该图片代号
	\end{subfigure}
	\caption{Trajectory Visualization}
	\label{fig:5}
\end{figure*}

\begin{itemize}
    \item \textbf{Trajectory \uppercase\expandafter{\romannumeral1}: Straight-line Path } As illustrated in Fig.~\ref{fig:t1}, this trajectory is designed for executing simple tasks, such as flying to a specific location and performing actions along a predefined path (e.g., patrols, cargo transport, etc.).
    
    \item \textbf{Trajectory \uppercase\expandafter{\romannumeral2}: Spiral Path:
} As illustrated in Fig.~\ref{fig:t2}, this trajectory is designed for tasks that involve changes in flight altitude, such as agricultural spraying, area scanning, and 3D mapping.

     \item \textbf{Trajectory \uppercase\expandafter{\romannumeral3}: U-shape Path:
} As illustrated in Fig.~\ref{fig:t3}, the trajectory is designed for tasks that demand high precision, such as monitoring, surveying, or round-trip transportation.

  %  \item \textbf{Trajectory \uppercase\expandafter{\romannumeral4}: Search/Sweep Route.} As illustrated in Fig.~\ref{fig:b}, the trajectory is designed to cover all portions of the area to ensure that all areas are effectively scanned, usually with regular or irregular geometry. It is typically used for search and rescue, surveillance, or wide-area environmental scanning missions.


\end{itemize}

\subsubsection{Navigation Algorithm}
 We choose the estimation and control library EKF (ECL EKF2) of the PX4 drone autopilot \cite{eclekf2} project as the target navigation algorithms.
\begin{itemize}
\item \textbf{ECL EKF2} implements EKF to estimate pose in the NED reference frame by fusing MARG (magnetic, angular rate, gravity) and GNSS data. MARG data is derived from magnetometer, gyroscope, and accelerometer sensors. It uses a 22-element state vector to track the orientation quaternion, velocity, position, MARG sensor biases, and geomagnetic.

 \item \textbf{CD-EKF} is a variant of ECL EKF2. It implements a continuous-discrete EKF to estimate pose in the NED reference frame by fusing MARG and GNSS data. It uses a 28-element state vector to track the orientation quaternion, velocity, position, MARG sensor biases, and geomagnetic vector.
\end{itemize}

\subsubsection{Implementation details.}
\begin{figure}[htp]
	\centering
 \centerline{\includegraphics[width=230pt]{./figure/fusion.png}} 
		\caption{Modeling of IMU and GNSS Fusion}
		\label{fig:imu}
\end{figure}
Accelerometers and gyroscopes operate at relatively high sample rates and necessitate high-rate processing. In contrast, GNSS and magnetometers function at relatively low sampling rates and require lower data processing rates. To replicate this configuration in our experiment, the IMUs (accelerometers, gyroscopes, and magnetometers) were sampled at 160 Hz, while the GNSS was sampled at 1 Hz. As demonstrated in Fig~\ref{fig:imu}, only one out of every 160 samples from the IMUs was provided to the fusion algorithm.

\subsubsection{Evaluation Metircs}

We utilize three metrics to evaluate the effectiveness of SSD comprehensively. Initially, we include two metrics that are widely used in relevant studies \cite{zhang2022adversarial}.

(1) Average Displacement Error (\textbf{ADE}): This metric measures the average deviation between the predicted and ground-truth trajectories by calculating the root mean squared error (RMSE) across all time frames. It captures the overall accuracy of the predicted trajectory compared to the actual path.


(2) Final Displacement Error (\textbf{FDE}): FDE focuses specifically on the prediction accuracy at the final time frame, quantified as the RMSE between the predicted and ground-truth positions. This metric highlights the importance of precise final positioning, which is crucial in applications such as path planning.

However, the two indicators above alone are insufficient to capture the impact of targeted attacks on UAVs. This introduces another challenge for evaluating SSD:  

\textbf{C2: How to evaluate SSD's impact on UAV mission.}

In mission-critical scenarios, UAVs must execute precise manoeuvres at specific waypoints to ensure the successful completion of the mission. The deviation from these waypoints during flight significantly increases the likelihood of mission failure. Therefore, to address \textbf{C2}, we design the Average Per-Waypoint Displacement Error (\textbf{APDE}). The APDE is formally defined as the average of the displacement errors computed at each waypoint along the trajectory. We define APDE as follows:
\begin{equation}
\label{apde}
APDE=\frac{\sum_{i=1}^{N_w} \quad ||P_i-P_i^a||_2}{N_w}   
\end{equation}
where $P_i^a$ and $P_i$ refer to the predicted positions before and after the attack, respectively. $N_w$ refers to the number of waypoints. APDE provides a more nuanced understanding of how targeted attacks impact the UAV's ability to adhere to its prescribed flight path, particularly at critical waypoints, thus enabling a more accurate assessment of the potential risks to mission success.

\subsection{Parametric Analysis}
\label{pa}
\begin{figure*}[htbp]
	\centering
	\begin{subfigure}{0.49\linewidth}
		\centering
		\includegraphics[width=0.8\linewidth]{./figure/pa.png}
		\caption{Positional attack}
		\label{fig:pa-a}%文中引用该图片代号
	\end{subfigure}
	\centering
	\begin{subfigure}{0.49\linewidth}
		\centering
		\includegraphics[width=0.8\linewidth]{./figure/FIG7B.png}
		\caption{Velocity  attack}
		\label{fig:pa-b}%文中引用该图片代号
	\end{subfigure}
 \caption{Parameter Analysis}
	\label{fig:pa}
\end{figure*}

The effectiveness and stealthiness of the attack are highly dependent on the choice of parameters. 
We analyze the sensitivity of SSD to various parameters by combining different values for $\theta$, $\alpha$, and $\phi$ across multiple scenarios (the same as Section~\ref{sec:study}). To measure the attack effectiveness and stealthiness, we use ADE and the maximum chi-square statistic (denoted as $\chi^2_{max}$), respectively. The experimental results are shown in Fig~\ref{fig:pa}. In positional attacks, as $\theta$ increases, $\chi^2_{max}$ exhibits a corresponding upward trend. We further observe that the rate of increase in $\chi^2_{max}$ slows as $\alpha$ increases. This suggests that higher values of $\alpha$ help mitigate the growth of $\chi^2_{max}$. Notably, when $\alpha$ exceeds a critical threshold value 13, $\chi^2_{max}$ starts to stabilize, converging within a relatively narrow threshold range. Within this range, $\chi^2_{max}$ fluctuates minimally and remains largely unaffected by changes in $\theta$, demonstrating high stability and consistency. For the velocity attack, as shown in Fig.~\ref{fig:pa-b}, $\chi^2_{max}$ gradually converges to about 6 when $\phi$ is less than 0.08. In the following experiments, we set the values of $\theta$, $\alpha$, and $\phi$ to 20, 11, and 0.08, respectively.


\subsection{Ablation study}
To validate the effectiveness of velocity-based and position-based attacks, we conducted an experiment where the UAV performed a 35 second flight incorporating linear and nonlinear motion states. The first 20 seconds involved uniform linear motion at a velocity of $2m/s$. Afterward, the UAV transitioned into a uniform circular motion mode, maintaining a linear velocity of $2m/s$ for the remaining 15 seconds. This flight trajectory was used for the ablation experiments.

\subsubsection{Contributions of different attacks} 
To explore the contribution of velocity and positional attacks to overall attack effectiveness, we designed comparison experiments with three attack strategies: single position perturbation attack (SPA), single velocity perturbation attack (SVA), and combined concerted attack (CCA). The quantitative analysis of ADEs, presented in Table~\ref{ablation}, shows that both individual attacks are effective. Specifically, the ADEs for SPA and SVA are improved by 184\% and 212\%, respectively, compared to the baseline. When comparing the combined attack (CCA) to the single attacks, the ADE improves by 129\% over SPA and 112\% over SVA. This indicates that CCA not only significantly enhances attack effectiveness by leveraging the coupling effect between position and velocity but also perturbs both longitude and latitude directions simultaneously. Furthermore, the stealth assessment results in Fig.~\ref{fig:ablation} show that while the CCA approach induces brief oscillations in the detection statistics, these fluctuations remain well within the acceptable threshold limits, ensuring that the attack remains stealthy.
\begin{table}[htb]
 \centering \caption{Ablation Study of Different Attacks' Contribution}
 \label{ablation}
\begin{tabular}{ccc}
    \toprule[1.5pt]
      & \textbf{Position N (meters)} & \textbf{Position E (meters)} \\
    \midrule[1pt]
    
    \textbf{Baseline} & 1.41& 0.66 \\\hline
    \textbf{SPA} & 2.72 & 0.89\\\hline
    
    \textbf{SVA}& 2.45 & 2.2  \\\hline

    \textbf{CCA}& 2.79 & 2.41 \\
    \bottomrule[1.5pt]
     \end{tabular}
\end{table}

\begin{figure}[htp]
	\centering
 \centerline{\includegraphics[width=230pt]{./figure/ablation1.png}} 
		\caption{Stealthiness Comparison for Attack Contribution}
		\label{fig:ablation}
\end{figure}

%\subsubsection{Velocity attack time}
%We also analyzed the disparity %between velocity attacks in the %acceleration and deceleration %dimensions. The experimental %results, shown in %TABLE.~\ref{dimension}, reveal %that targeting the deceleration %dimension does not %significantly enhance attack %effectiveness. Moreover, it %increases the likelihood of %detection, posing a higher risk.
%


\subsubsection{Attack Combination}
We apply the velocity attack during the linear motion state and the position attack during the nonlinear motion state and examine the impact of this combination on the attack's stealthiness. The experimental results, shown in Fig.~\ref{fig:ablation1}, reveal that $9s$ after the attack begins, the detector quickly identifies the existence of the attack. Furthermore, even when the UAV transitions to a nonlinear motion state, the detector continues to successfully detect the attack for $5s$, despite the change in attack mode. This phenomenon occurs because the drastic velocity changes in the linear motion state cause significant fluctuations in the residuals. In contrast, the nonlinear motion state allows SSD to effectively smooth out the impact of the velocity attack, ensuring the attack remains stealthy throughout the process. This result, along with \textbf{Finding2}, further validates the rationale of the SSD framework.


\begin{figure}[htp]
	\centering
 \centerline{\includegraphics[width=230pt]{./figure/ablation2.png}} 
		\caption{Detection Statistic Comparison for Attack Combination }
		\label{fig:ablation1}
\end{figure}

\subsection{Attack Stealthiness}
\label{section:stealthy}
To validate the stealthiness, we selected a chi-square detector for attack detection. For the threshold, we selected a value corresponding to a 95\% confidence level, which is 11.1. We also chose biased signal and multiplicative attacks for comparison and evaluated them across three mission trajectories. Based on prior research, we designed the following attack payloads: (1) GNSS positions are added with a uniform distribution U(0, 0.0005); (2) GNSS positions are scaled by a factor of 1.5. The experimental results in Fig.~\ref{fig:stealthiness} demonstrate that SSD successfully limits detection statistics to the threshold range and bypasses both detection methods with a carefully chosen set of attack parameters. 
Notably, compared to these works \cite{tang2023gan,fei2020learn}, we significantly reduced the loadings applied for the biased signal Attack. However, both detection methods were able to detect the attack effectively. It is clear that SSD exhibits strong stealthiness properties under chi-square detection methods and several motion states.
\begin{figure*}[htbp]
	\centering
	\begin{subfigure}{0.325\linewidth}
		\centering
		\includegraphics[width=0.9\linewidth]{./figure/ste1.png}
		\caption{ $\chi^2 $ detection of different attacks under Trajectory \uppercase\expandafter{\romannumeral1} }
		\label{fig:ste1}%文中引用该图片代号
	\end{subfigure}
	\centering
	\begin{subfigure}{0.325\linewidth}
		\centering
		\includegraphics[width=0.9\linewidth]{./figure/ste2.png}
		\caption{$\chi^2 detection$ of different attack under Trajectory \uppercase\expandafter{\romannumeral2}}
		\label{fig:ste2}%文中引用该图片代号
	\end{subfigure}
	\centering
	\begin{subfigure}{0.325\linewidth}
		\centering
		\includegraphics[width=0.9\linewidth]{./figure/ste3.png}
		\caption{$\chi^2 detection$ of different attack under Trajectory \uppercase\expandafter{\romannumeral3}}
		\label{fig:ste3}%文中引用该图片代号
	\end{subfigure}

   % \centering
	%\begin{subfigure}{0.325\linewidth}
	%	\centering
	%	\includegraphics[width=0.9\linewidth]{./figure/s4.png}
	%	\caption{ $cosine$ detection of different attack under Trajectory}
%		\label{fig:ste4}%文中引用该图片代号
%	\end{subfigure}
	%\centering
	%\begin{subfigure}{0.325\linewidth}
	%	\centering
	%	\includegraphics[width=0.9\linewidth]%{./figure/s5.png}
	%	\caption{$\chi^2 detection$ of different attack under Trajectory}
	%	\label{fig:ste5}%文中引用该图片代号
	%\end{subfigure}
	%\centering
	%\begin{subfigure}{0.325\linewidth}
	%	\centering
	%	\includegraphics[width=0.9\linewidth]{./figure/steal.png}
	%	\caption{route}
	%	\label{fig:ste6}%文中引用该图片代号
	%%\end{subfigure}
	\caption{Attack Stealthiness Evaluation}
	\label{fig:stealthiness}
\end{figure*}

We also use two UAV-specific detectors, NLC~\cite{quinonez2020savior} and LTW~\cite{choi2018detecting}, as baselines for comparison. These methods are highly effective at detecting GNSS attacks, and they maintain both strong accuracy and response time, even when identifying targeted stealthy attacks. We follow the threshold in previous work~\cite{quinonez2020savior}. Fig.~\ref{fig:nlc} demonstrates that in the LTW test group, the detection statistic consistently remains below the threshold boundary, indicating the superior stealth characteristics of SSD attacks. Notably, NLC statistics demonstrate the following three characters (shown in Table~\ref{steal}): 
\begin{itemize}
    \item In Trajectory \uppercase\expandafter{\romannumeral1} (pure linear motion), the detection statistic approaches the threshold at $6s$ but never exceeds the threshold.
    
    \item In Trajectory \uppercase\expandafter{\romannumeral2} (fully nonlinear motion), the detection statistic surpasses the threshold after $31s$ cumulative duration. The detection latency is largely increased compared to the previous work (about $0.3s$)~\cite{quinonez2020savior}.

     \item In Trajectory \uppercase\expandafter{\romannumeral3} (hybrid motion mode), no significant statistical fluctuations occur during $0-20s$ linear phase, with limited oscillations (peak statistic is 9.27) emerging post nonlinear component introduction at 20s.
\end{itemize}

\begin{table}[htb]
 \centering \caption{ Detection Performance Comparison for NLC}
 \label{steal}
\begin{tabular}{cccc}
    \toprule[1.5pt]
      & \textbf{Detection Latency(s)} & \textbf{Peak Statistic} &\textbf{Success Rate} \\
    \midrule[1pt]
    
     Trajectory \uppercase\expandafter{\romannumeral1} & N/A& 19.69 & 0\%\\\hline
     Trajectory \uppercase\expandafter{\romannumeral2} & 31s & \textgreater $20$& 35.4\%\\\hline
    
    Trajectory \uppercase\expandafter{\romannumeral3}& 
    N/A& 9.27& 0\%  \\
    \bottomrule[1.5pt]
     \end{tabular}
\end{table}
These findings demonstrate significant motion-dynamic sensitivity disparities in NLC detectors. However, when attackers combine motion-state transition strategies,
positional attacks can realize residual normalization during linear phases and reduce detection sensitivity
velocity attacks to counteract statistical fluctuations from nonlinear components. These two methods make SSD show greater stealthiness in hybrid motion missions.

\begin{figure*}[htbp]
	\centering
	\begin{subfigure}{0.325\linewidth}
		\centering
		\includegraphics[width=0.9\linewidth]{./figure/nlc1.png}
		\caption{Detection statistic in Trajectory \uppercase\expandafter{\romannumeral1} }
		\label{fig:nlc1}%文中引用该图片代号
	\end{subfigure}
	\centering
	\begin{subfigure}{0.325\linewidth}
		\centering
		\includegraphics[width=0.9\linewidth]{./figure/nlc2.png}
		\caption{Detection statistic in Trajectory \uppercase\expandafter{\romannumeral2}}
		\label{fig:nlc2}%文中引用该图片代号
	\end{subfigure}
	\centering
	\begin{subfigure}{0.325\linewidth}
		\centering
		\includegraphics[width=0.9\linewidth]{./figure/nlc3.png}
		\caption{Detection statistic in Trajectory \uppercase\expandafter{\romannumeral3} }
		\label{fig:nlc3}%文中引用该图片代号
	\end{subfigure}
	\caption{Detection Statistic Comparison for
NLC and LTW under SSD.}
	\label{fig:nlc}
\end{figure*}

\subsection{Attack Effectiveness}
We start by determining the optimal combination of $Q_k$ and $R_k$ for each scenario through offline learning. With these optimal values, we proceed to validate the effectiveness of the SSD. For each combination of INS and trajectory, the UAV is tasked with following the designated path to complete a full mission, while the SSD is deployed to attack the flight. Table \ref{ae} presents the changes in each metric before and after the attack. On average, ADE/FDE is increased by 425\%/591\%. The lateral (N)/longitude (E) deviation
 reaches 3.54/3.46 meters. We will analyze the factor based on the experiment on three scenarios.

\begin{table*}[htb]
 \centering \caption{Attack Effectiveness}
 \label{ae}
\begin{tabular}{c|c|c|c|c|c}
    \toprule[1.5pt]
    \textbf{\multirow{2}{*}{Model}} & \textbf{\multirow{2}{*}{Scenario}}& \textbf{\multirow{2}{*}{Duration}} & \textbf{ADE} & \textbf{FDE} & \textbf{APDE}\\
    \cline{4-6}
    
    &&&\textbf{Normal/Attack (meters)}&\textbf{Normal/Attack (meters)}&\textbf{Normal/Attack (meters)} \\
        \midrule[1pt]
   	ECL EKF2 &Trajectory \uppercase\expandafter{\romannumeral1} & \multirow{2}{*}{20s}& (0.95)/(5.68)  & (1.80)/(5.89) & (0.89)/(5.66) 
    \\\cline{1-2} \cline{4-6}
    
    CD-EKF  &Trajectory \uppercase\expandafter{\romannumeral1} & & (1.62)/(4.64)  & (1.49)/(7.54)  &  (1.81)/(4.92)  \\\hline

     ECL EKF2  & Trajectory \uppercase\expandafter{\romannumeral2}& \multirow{2}{*}{48s}& (1.31)/(4.09)  & (1.78)/(6.15) &  (1.48)/(4.11) \\
     \cline{1-2} \cline{4-6}
      CD-EKF & Trajectory \uppercase\expandafter{\romannumeral2} & & (1.39)/(3.97)   &(1.99)/(6.47)   & (2.35)/(5.61)   \\
     \hline
      ECL EKF2 & Trajectory \uppercase\expandafter{\romannumeral3}& \multirow{2}{*}{71s} & (1.46)/(5.38)  & (0.95)/(10.47)  & (1.86)/(6.49)   \\
     \cline{1-2} \cline{4-6}
      CD-EKF & Trajectory \uppercase\expandafter{\romannumeral3}& &(1.86)/(6.35)  & (1.34)/(7.90)  & (1.66)/(6.73)  \\ 
     \bottomrule[1.5pt]
     \end{tabular}
\end{table*}

\begin{figure*}[htbp]
	\centering
	\begin{subfigure}{0.325\linewidth}
		\centering
		\includegraphics[width=0.9\linewidth]{./figure/visual11.png}
		\caption{Attack visualization in Trajectory \uppercase\expandafter{\romannumeral1}}
		\label{fig:v1}%文中引用该图片代号
	\end{subfigure}
	\centering
	\begin{subfigure}{0.325\linewidth}
		\centering
		\includegraphics[width=0.9\linewidth]{./figure/visual2.png}
		\caption{Attack visualization in Trajectory \uppercase\expandafter{\romannumeral2}}
		\label{fig:v2}%文中引用该图片代号
	\end{subfigure}
	\centering
	\begin{subfigure}{0.325\linewidth}
		\centering
		\includegraphics[width=0.9\linewidth]{./figure/attt1.png}
		\caption{Attack visualization in Trajectory \uppercase\expandafter{\romannumeral3}}
		\label{fig:v3}%文中引用该图片代号
	\end{subfigure}
	\caption{Attack Visualization  }
	\label{fig:visualization}
\end{figure*}

\subsubsection{Different Scenarios}
In terms of scenarios, the SSD shows a greater increase in positioning error in the purely linear motion state (Fig.~\ref{fig:v1}) compared to the purely nonlinear state (Fig.~\ref{fig:v2}). This difference arises from the time-varying nature of the velocity vector in the nonlinear state, which leads to attenuation of the indirect positional interference caused by the velocity perturbation $G(t_i,a^d_i;\phi)$. However, experiments with Section \ref{section:stealthy} demonstrate that a velocity attack can still ensure the fulfillment of the stealthy precondition. When the mission involves mixed kinematic modes, the combination of velocity and positional attacks results in an additive interaction, as shown in Table \ref{ae}. The baseline increase is 366.44\% for ADE and 1102.11\% for FDE, confirming that SSD enhances attack effectiveness synergistically. Next, we will quantitatively analyze the impact of this state on mission completion rates.

\subsubsection{Impact on Mission}
When the UAV's trajectory deviation exceeds the tolerance range, the mission completion rate shows a clear decline. In three different scenarios, the APDE metrics increased by $635.96\%$, $277.70\%$, and $348.92\%$, respectively, indicating that SSDs significantly compromise mission reliability. Notably, in Scenario Trajectory \uppercase\expandafter{\romannumeral3}, where the mission duration is 71 seconds, the FDE and APDE deviations reached $10.47m$ and $6.49m$, respectively—the largest among the three scenarios. This demonstrates that SSD lowers the mission completion rate and impedes the UAV's ability to return accurately after completing the mission (shown in Fig.~\ref{fig:v3}). This effect highlights its kinematic chain impact since the mission-critical point is typically located at the motion state transition (e.g., hover → level flight, acceleration → deceleration). The attack-induced cumulative error in position estimation propagates into the next motion state, where it is further amplified.

\subsubsection{Attack Transferability}
Different INS architectures manage noise and system uncertainty in distinct ways, potentially impacting the effectiveness of SSD. To assess the cross-system adaptability of the SSD, we conducted validation experiments using CD-EKF and ECL EKF2. CD-EKF relies on continuous-time prediction with discrete-time updates, enhancing its adaptability to errors in nonlinear motion states and making its short-term error estimation more robust. However, SSD effectively exploits CD-EKF’s sensitivity to state uncertainty by inducing motion instability, thereby continuously disrupting navigation accuracy. Experimental results show that under the CD-EKF system, SSD achieves an average improvement of 304.48\% in ADE, 473.57\% in FDE, and 305.87\% in APDE (see Table \ref{ae}). These findings confirm SSD’s generalizability across EKF-based navigation frameworks.


\section{Discussion}
This study reveals the coupling mechanism between UAV motion dynamics and cyber attacks' effectiveness through systematic empirical analysis. Experimental data indicate that changes in motion state can significantly enhance the success rate of attacks. While the SSD approach demonstrates clear advantages, its engineering implementation faces two major challenges: \textbf{Dependency on A Priori Knowledge.}
The effectiveness is highly contingent upon the real-time accuracy of the object detection and tracking system. However, the existing YOLOv5 architecture, for instance, exhibits exponential decay in the Intersection over Union (IoU) metric over time in dynamic target tracking scenarios. This results in a tracking failure probability exceeding 73\% after 60 seconds of continuous locking. \textbf{Energy-Concealment Trade-off Paradox.}
While the sustained attack mode can maintain a stealthiness threshold, it leads to a non-linear increase in energy consumption on the attacking end. This escalation doesn't align with the requirements in real-world mission scenarios.

In addition, a key assumption of the SSD is that the $Q_k$ and $R_k$ are preset offline. This is a common practice in current engineering applications. From a theoretical standpoint, adaptive $Q_k$ and $R_k$ values can modify the system's sensitivity to attacks, potentially enabling the mitigation of such attacks. This insight provides a constructive direction for defending SSD. We observed that increasing $Q_k$ and decreasing $R_k$ could reduce the system's sensitivity to attacks. However, this adjustment comes at the cost of a decrease in localization accuracy. While it is possible to improve positioning accuracy by increasing $R_k$ and decreasing $Q_k$, this also makes the system more vulnerable to attacks. Consequently, a dynamic strategy for adjusting $Q_k$ and $R_k$ is crucial. This can be achieved through optimization methods or reinforcement learning, which can fine-tune these parameters in real time, balancing between accuracy and security.

\section{Conclusion}
In this paper, we investigate cybersecurity threats of UAV route planning under different motion states. We assess the effectiveness of GNSS attacks under various motion states through theoretical and experimental analyses. Our findings reveal that INS is more vulnerable during maneuvering than in linear flight. Based on this insight, we introduce SSD, a novel state-based stealthy backdoor attack, which strategically combines GNSS velocity and position attacks to exploit this vulnerability. We conducted extensive experiments, and the results show that SSD demonstrates superior effectiveness and stealthiness compared with previous methods. We hope that this work will inspire INS designers and developers to prioritize code security and implement robust dynamic defenses. 





% if have a single appendix:
%\appendix[Proof of the Zonklar Equations]
% or
%\appendix  % for no appendix heading
% do not use \section anymore after \appendix, only \section*
% is possibly needed

% use appendices with more than one appendix
% then use \section to start each appendix
% you must declare a \section before using any
% \subsection or using \label (\appendices by itself
% starts a section numbered zero.)
%


%\appendices
%\section{Proof of the First Zonklar Equation}
%Appendix one text goes here.

%\section{}


% use section* for acknowledgment
\section*{Acknowledgment}
This work is supported by the National Natural Science Foundation of China ( No.62233014, No.62103330), and the Innovation Foundation for Doctor Dissertation of Northwestern Polytechnical University (CX2023023).

% Can use something like this to put references on a page
% by themselves when using endfloat and the captionsoff option.
%\ifCLASSOPTIONcaptionsoff
%  \newpage
%\fi



% trigger a \newpage just before the given reference
% number - used to balance the columns on the last page
% adjust value as needed - may need to be readjusted if
% the document is modified later
%\IEEEtriggeratref{8}
% The "triggered" command can be changed if desired:
%\IEEEtriggercmd{\enlargethispage{-5in}}

% references section

% can use a bibliography generated by BibTeX as a .bbl file
% BibTeX documentation can be easily obtained at:
% http://mirror.ctan.org/biblio/bibtex/contrib/doc/
% The IEEEtran BibTeX style support page is at:
% http://www.michaelshell.org/tex/ieeetran/bibtex/
%\bibliographystyle{IEEEtran}
% argument is your BibTeX string definitions and bibliography database(s)
%\bibliography{IEEEabrv,../bib/paper}
%
% <OR> manually copy in the resultant .bbl file
% set second argument of \begin to the number of references
% (used to reserve space for the reference number labels box)

% biography section
% 
% If you have an EPS/PDF photo (graphicx package needed) extra braces are
% needed around the contents of the optional argument to biography to prevent
% the LaTeX parser from getting confused when it sees the complicated
% \includegraphics command within an optional argument. (You could create
% your own custom macro containing the \includegraphics command to make things
% simpler here.)
%\begin{IEEEbiography}[{\includegraphics[width=1in,height=1.25in,clip,keepaspectratio]{mshell}}]{Michael Shell}
% or if you just want to reserve a space for a photo:

%\begin{IEEEbiography}{Michael Shell}
%Biography text here.
%\end{IEEEbiography}

% if you will not have a photo at all:
%\begin{IEEEbiographynophoto}{John Doe}
%Biography text here.
%\end{IEEEbiographynophoto}

% insert where needed to balance the two columns on the last page with
% biographies
%\newpage

%\begin{IEEEbiographynophoto}{Jane Doe}
%Biography text here.
%\end{IEEEbiographynophoto}

% You can push biographies down or up by placing
% a \vfill before or after them. The appropriate
% use of \vfill depends on what kind of text is
% on the last page and whether or not the columns
% are being equalized.

%\vfill

% Can be used to pull up biographies so that the bottom of the last one
% is flush with the other column.
%\enlargethispage{-5in}


\bibliographystyle{unsrt}
\bibliography{ref1} 
% that's all folks
\end{document}


