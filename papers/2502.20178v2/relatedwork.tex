\section{Related Works and Background}
\subsection{UAV Route Planning and Integrated Navigation System}
\label{ins}
UAV route planning involves designing a feasible route from the start point to the destination while meeting all constraints and performance requirements \cite{abdel2024multiobjective}. Effective route planning is crucial for UAVs to complete their missions, and it depends on accurate position estimation. Since UAVs often operate in dynamic and complex environments, they rely on multi-sensor fusion to enhance their ability to perceive the environment. This approach integrates data from various sensors with different modalities and attributes, increasing redundancy and improving reliability in challenging conditions. The integrated navigation system of GNSS and IMU is a typical representative example. Its fusion strategy uses the GNSS data for quantitative updating, IMU data for state prediction, and an optimal estimation framework to achieve accurate positioning in the global coordinate system \cite{qi2002direct,carvalho1997optimal}. 
In GNSS-denied environments, simultaneous localization and mapping (SLAM) that rely on camera~\cite{davison2007monoslam,forster2016svo} and lidar~\cite{zhang2014loam,nguyen2021miliom} are considered more reliable solutions. However, a single sensor alone cannot fully meet the demands for positioning accuracy and response speed. Therefore, combining these sensors with an inertial measurement unit to create IMU/Camera~\cite{qin2018vins,eckenhoff2021mimc} and IMU/Lidar~\cite{xu2022fast} fusion form enables more precise position estimation and improved performance in dynamic environments. This paper focuses on the security analysis of IMU/GNSS INS since it plays a central role in UAVs. It is of generic meaning to study its security.


\subsection{GNSS Spoofing attack}
\label{attack}
Since IMUs are more difficult to manipulate in real-world scenarios, we only discuss GNSS spoofing attacks for IMU/GNSS INS on UAVs. In such attacks, the adversary transmits false location coordinates to the GNSS receiver, thereby concealing the UAV's true location. As a result, unknowingly accepting these false inputs, the navigation system calculates incorrect position information. Specifically, GNSS spoofing includes direct attacks and stealthy attacks. The direct attacks \cite{tang2023gan} can be classified into these categories. (1) \textbf{
Biased Signal Attack}
: These attacks involve adding a bias to the GNSS sensor signals, typically following a uniform distribution. (2) \textbf{Multiplicative Attacks}: In these attacks, the GNSS signals are multiplied by a constant factor, effectively scaling the original signal values. (3) \textbf{Replacement Attacks}: These attacks involve directly replacing the GNSS signals with false or manipulated data. Direct attacks are easy to implement and may lead to disastrous consequences, such as the UAV crashing into obstacles \cite{kerns2014unmanned,vervisch2017influence}. However, most detectors can detect and respond in time. The stealthy attacks~\cite{khazraei2024black,ma2024novel} are diverse, with attackers often aiming to maximize navigation residuals to determine the optimal attack sequence. The design of these attacks may cause the UAV to fall into the malicious attackers’ control~\cite{he2018friendly}. However, it is largely influenced by the detection mechanism and often needs complex computing. SSD integrates the strengths of both approaches. It achieves the same effectiveness as stealthy attacks while requiring low computational resources.

To mitigate this threat, robust countermeasures, such as software analysis \cite{ceccato2021generalized}, cryptography-based authentication \cite{bonior2017implementation} and machine learning-based detections \cite{liang2019detection,han2022ads}, have been implemented to safeguard against GNSS spoofing and ensure the integrity of UAV INS. One important method is multi-sensor fusion \cite{yang2021secure}. As described in section \ref{ins}, it not only provides more accurate estimates for perception and localization. but also enhances the data's trustworthiness, providing greater redundancy for detection and defense in the event of spoofing attacks. For example, Shen et al. \cite{shen2020drift} demonstrated that the effectiveness of constant offset GNSS spoofing attacks is greatly reduced in GNSS/INS/LiDAR fused navigation systems in autonomous driving. However, such multi-sensor fusion strategies also face a period of vulnerability and can not defend against constructed GNSS spoofing against the uncertainty that exists in the fusion algorithm itself. In this paper, we demonstrate a similar phenomenon in UAVs, where changes in motion states significantly amplify the uncertainty of the INS, making it more vulnerable to GNSS attacks. Therefore, we provide an in-depth analysis of the uncertainty and vulnerability and exploit it to design SSD.

\subsection{Threat Model}

\textbf{Attack Goal} As shown in Figure \ref{threat}, the adversary aims to make the drone deviate significantly from its pre-planned route without triggering the stealth detection threshold. This objective can be formalized as the following optimization problem:
% (stealthiness defination)

\begin{equation}
    \begin{aligned} \label{P}
&\text{argmax}_{\delta} \quad \sum_{i=0}^{T_a} \mathbf{D^t_i-D^a_i} \\
&\begin{array}{r@{\quad}r@{}l@{\quad}l}
s.t. &\chi_k&\leq \tau \quad \forall k \in \{1,...,T_{a}-1\} \\
\end{array}
\end{aligned}
\end{equation}
where $\mathbf{D^t_i}$ and $\mathbf{D^a_i}$ denote the normal trajectory and the attacked trajectory, respectively. $\delta$ is the attack payload. $\chi_k$ is the in detection statistics and $\tau$ is the threshold of detectors.

\begin{figure}[htp]
	\centering
 \centerline{\includegraphics[width=230pt]{./figure/att.png}} 
		\caption{Threat Model}
		\label{threat}
\end{figure}


\textbf{Attack Scenario} As shown in Fig. \ref{threat}, an attacker can launch an attack in two ways: \ding{172} The attacker can use a UAV to fly alongside the victim's UAV. He can transmit legitimate GNSS signals completely using wireless attack devices such as software-defined radios (SDR). \ding{173} The attacker could also inject specific backdoors \cite{IRAHUL} or viruses~\cite{dji} into the UAV by supply chain attacks, which could be used to monitor its dynamics and induce a false GNSS position \cite{xu2021novel}.

\textbf{Attacker's Capability}
1) The attackers need white-box access to obtain the victim's navigation algorithms and corresponding parameters. They can get this knowledge through open-source channels since most UAVs use standardized open-source navigation algorithms~\cite{ardupilot,px4}. Also, the adversarial can use reverse engineering to access the victim's knowledge. 2) The attacker can obtain the victim's motion state, such as position and velocity. This can be achieved by monitoring UAVs using an additional GPS module or auxiliary object detection and tracking devices. 3) As IMU data is less likely to be accessed and used, an attacker can only modify the position and velocity since GNSS measurements only provide position and velocity to the UAV.