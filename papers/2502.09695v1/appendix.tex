\appendix
%\section*{Appendix}
%\renewcommand{\thesubsection}{\Alph{subsection}}
%\section{Appendix }

%%% For Appendix A.
% format the equation environment
%\renewcommand{\theequation}{A\arabic{equation}}

% reset the counter
%\setcounter{equation}{0}

\subsection{Proof of Properties (i)--(iv) of the Quotient Distance (\ref{E:quotient_distance})}

%A proof sketch is provided as follows.
\noindent (i) Denote $V(\vect x, \delta) = \|\mathcal P(t, \vect x) \delta \|$. 
Since $V(\vect x, -\delta) = V(\vect x, \delta)$, the pseudo-metric (\ref{E:quotient_distance}) is symmetric, which implies (i). 

\noindent (ii) It is implied from the principle of dynamic programming. 

\noindent (iii) The quotient distance (\ref{E:quotient_distance}) being nonnegative is obvious. If $[\vect x_1]_t = [\vect x_2]_t$, then, by the definition (\ref{E:generator_system}), there is an integral curve of (\ref{E:generator_system}) that joins $\vect x_1$ and $\vect x_2$. The tangent vectors of this integral curve are orthogonal to span of the projection $\mathcal P(\vect x, \delta)$ by definition. Therefore, this curve causes the integral in the RHS of (\ref{E:quotient_distance}) to evaluate to zero, which, combined with nonnegativity, implies $\mathrm{dist}(\vect x_1, \vect x_2) = 0$.

\noindent (iv) From (\ref{E:contain_in_level}), each equivalence class is contained in a level set of $H(t, \vect x)$. Since tangent vector of the equivalence class is $\mathrm{span}(\mathcal P(t, \vect x))^\perp$, $\nabla H(t, \vect x) \in \mathrm{span}(\mathcal P(t, \vect x))$. Then, for each $\vect x, \vect y$ with $H(t, \vect x) \neq H(t, \vect y)$, it holds that $\mathrm{dist}(t, \vect y, \vect x) \geq \min\big\{\|\vect z_1 - \vect z_2\|\mid H(t, \vect z_1) = H(t, \vect x),\, H(t, \vect z_2) = H(t, \vect y)\big\} > 0$, where we expanded each equivalence relation to entire level set for the first inequality and used that level sets are disjoint and closed for the second inequality. \hfill $\square$

%(iii) can be proved by noting that, if $[\vect x_1]_t = [\vect x_2]_t$, the integral curve $\Phi_t(\tau, \vect x_1)$ between $\vect x_1$ and $\vect x_2$ has zero projected tangent vectors everywhere. 
%The ($\Rightarrow$) part of property (iii) is proved by noting that, based on (\ref{E:last_assumption}), every $\Phi_t(\tau, \vect x)$ spans a disjoint compact set.
%(iv) can be proved by noting that, because the equivalence classes are contained in level sets of $H(t, \vect x)$, the tangent subspace $\big\{ \matr J(t, \vect x) \nabla H(t, \vect x) \big\}^\perp$ always contains $\nabla H(t, \vect x)$. Since every curve connecting different level sets has tangent component  in $\nabla H(t, \vect x)$, the quotient distance between two points in different level sets must be strictly positive.

\subsection{Proof of Proposition~\ref{prop_main}}

\noindent \textbf{1.} From Gr\"onwall's lemma, to prove (\ref{E:s1-0}), it suffices to prove that
\begin{align}
    &\frac{d}{d t} \dist(t, \Phi(t, t_0, \vect x_1), \Phi(t, t_0, \vect x_2)) \notag \\
    &\overset{?}{\leq} -c \dist(t, \Phi(t, t_0, \vect x_1), \Phi(t, t_0, \vect x_2)). \label{E:s1-1}
\end{align}
%Based on the chain rule for total derivatives, we separate the contributions to the LHS of (\ref{E:s1-1}) from the two systems,
%\begin{align*}
%    &\dot{\vect x} = \matr J(t, \vect x) \nabla H(t, \vect x) \tag{system 1} \\
%    &\dot{\vect x} = -\matr R \nabla H(t, \vect x). \tag{system 2}
%\end{align*}
%It is clear that system~1 preserves the distance of the quotient space, and so for checking (\ref{E:s1-1}) we only need to consider the contribution from system~2.
Using (\ref{E:quotient_distance}), we can express (\ref{E:s1-1}) as
\begin{align}
    &\frac{d}{d t} \dist(t, \Phi(t, t_0, \vect x_1), \Phi(t, t_0, \vect x_2)) \notag \\
    &= \frac{d}{d t} \min_{\gamma \in \Gamma(\Phi(t, t_0, \vect x_1), \Phi(t, t_0, \vect x_2))} \int_0^1 \bigg\|\mathcal P(t, \gamma(s)) \frac{\partial\gamma}{\partial s}(s) \bigg\|\, ds \notag \\
    &\leq \frac{d}{d t} \int_0^1 \bigg\|\mathcal P(t, \psi(t, s)) \frac{\partial\psi}{\partial s}(t, s) \bigg\|\, ds \label{E:s1-7} \\
    &\overset{?}{\leq} -c \int_0^1 \bigg\|\mathcal P(t, \psi(t, s)) \frac{\partial\psi}{\partial s}(t, s) \bigg\|\, ds \label{E:s1-6} \\
    &= {-}c \dist(t, \Phi(t, t_0, \vect x_1), \Phi(t, t_0, \vect x_2)), \notag 
\end{align}
where $\psi(t, s)$ is a curve that achieves the minimum at $t = t_0$, and $\psi(t, s) = \Phi(t, t_0, \psi(t, s))$ is the solution of the system rooted at $\psi(t_0, s) = \psi(t, s),\, s\in [0, 1]$. The inequality (\ref{E:s1-7}) holds because the curve $\psi(t, s)$ is either minimizing or not for $t > t_0$. Note that, to prove the inequality (\ref{E:s1-6}), it suffices to prove that the integrand satisfies the inequality uniformly, i.e.,
\begin{equation} \label{E:s1-8}
    \frac{d}{d t} \bigg\|\mathcal P(t, \psi(t, s)) \frac{\partial\psi}{\partial s}(t, s) \bigg\|\ \overset{?}{\leq} -c\, \bigg\|\mathcal P(t, \psi(t, s)) \frac{\partial\psi}{\partial s}(t, s) \bigg\|.
\end{equation}
The inequality (\ref{E:s1-8}) is equivalent to
\begin{align}
    &\frac{1}{2} \frac{d}{d t} \Big\langle \mathcal P(t, \psi(t, s)) \frac{\partial\psi}{\partial s}(t, s), \mathcal P(t, \psi(t, s)) \frac{\partial\psi}{\partial s}(t, s) \Big\rangle \notag \\
    &\overset{?}{\leq} -c \Big\langle \mathcal P(t, \psi(t, s)) \frac{\partial\psi}{\partial s}(t, s), \mathcal P(t, \psi(t, s)) \frac{\partial\psi}{\partial s}(t, s) \Big\rangle, \label{E:s1-2}
\end{align}
because, for any $m(t) \geq 0$, $\frac{d}{d t} m \leq -c m \Leftrightarrow \frac{1}{2} \frac{d}{d t} m^2 \leq -c m^2$.

\noindent \textbf{2.} The LHS of (\ref{E:s1-2}) can be manipulated as
\begin{align}
    &\textstyle \frac{1}{2} \frac{d}{d t} \big\langle \mathcal P(t, \psi(t, s)) \frac{\partial\psi}{\partial s}(t, s), \mathcal P(t, \psi(t, s)) \frac{\partial\psi}{\partial s}(t, s) \big\rangle \notag \\
    &= \textstyle \big\langle \mathcal P(t, \psi(t, s)) \frac{\partial\psi}{\partial s}(t, s), \frac{d}{d t} \Big[\mathcal P(t, \psi(t, s)) \frac{\partial\psi}{\partial s}(t, s) \Big] \big\rangle \notag \\
    &= \textstyle \big\langle \mathcal P(t, \psi(t, s)) \frac{\partial\psi}{\partial s}(t, s), \mathcal P(t, \psi(t, s)) \frac{\partial}{\partial t} \frac{\partial\psi}{\partial s}(t, s) \big\rangle \notag \\
    &\quad\, \textstyle + \big\langle \mathcal P(t, \psi(t, s)) \frac{\partial\psi}{\partial s}(t, s), \frac{d}{d t} \mathcal P(t, \psi(t, s)) \frac{\partial\psi}{\partial s}(t, s) \big\rangle \notag \\
    &= \textstyle \big\langle \mathcal P(t, \psi(t, s)) \frac{\partial\psi}{\partial s}(t, s), \frac{\partial}{\partial \vect x} \Big[\mathcal P(t, \psi(t, s)) \frac{\partial}{\partial t} \psi(t, s) \Big] \frac{\partial\psi}{\partial s}(t, s) \big\rangle \notag \\
    &\quad\, \textstyle - \big\langle \mathcal P(t, \psi(t, s)) \frac{\partial\psi}{\partial s}(t, s), \frac{\partial}{\partial \vect x} \mathcal P(t, \psi(t, s)) \frac{\partial \psi}{\partial t}(t, s) \frac{\partial\psi}{\partial s}(t, s) \big\rangle \notag \\
    &\quad\, \textstyle + \big\langle \mathcal P(t, \psi(t, s)) \frac{\partial\psi}{\partial s}(t, s), \frac{d}{d t} \mathcal P(t, \psi(t, s)) \frac{\partial\psi}{\partial s}(t, s) \big\rangle \label{E:s1-9} \\
    &= \textstyle \big\langle \mathcal P(t, \psi(t, s)) \frac{\partial\psi}{\partial s}(t, s), \frac{\partial}{\partial \vect x} \Big[\mathcal P(t, \psi(t, s)) \frac{\partial}{\partial t} \psi(t, s) \Big] \frac{\partial\psi}{\partial s}(t, s) \big\rangle \notag \\
    &\quad\, \textstyle + \big\langle \mathcal P(t, \psi(t, s)) \frac{\partial\psi}{\partial s}(t, s), \frac{\partial}{\partial t} \mathcal P(t, \psi(t, s)) \frac{\partial\psi}{\partial s}(t, s) \big\rangle \label{E:s1-10} \\
    &= \textstyle \big\langle \mathcal P(t, \psi(t, s)) \frac{\partial\psi}{\partial s}(t, s), \frac{\partial}{\partial \vect x} \big[{-} \matr R \nabla H(t, \psi(t, s))\big] \frac{\partial\psi}{\partial s}(t, s) \big\rangle \notag \\
    &\quad\, \textstyle + \big\langle \mathcal P(t, \psi(t, s)) \frac{\partial\psi}{\partial s}(t, s), \frac{\partial}{\partial t} \mathcal P(t, \psi(t, s)) \frac{\partial\psi}{\partial s}(t, s) \big\rangle \label{E:s1-3} \\
    &\leq \textstyle - \Re\left\{ \Big[\mathcal P(t, \psi(t, s)) \frac{\partial\psi}{\partial s}(t, s)\Big]^\herm D^2 H(t, \psi(t, s)) \frac{\partial\psi}{\partial s}(t, s) \right\} \label{E:s1-4} \\
    &\leq \textstyle -c \Re\left\{ \Big[\mathcal P(t, \psi(t, s)) \frac{\partial\psi}{\partial s}(t, s)\Big]^\herm \matr R^{-1} \mathcal P(t, \psi(t, s)) \frac{\partial\psi}{\partial s}(t, s) \right\}. \label{E:s1-5} 
\end{align}
To obtain (\ref{E:s1-9}), we replaced $\frac{\partial}{\partial t} \frac{\partial \psi}{\partial s}(t, s)$ in the first term of the LHS by 
\begin{align*}
     \cdot = \frac{\partial}{\partial s} \frac{\partial \psi}{\partial t}(t, s) &= \frac{\partial}{\partial s} \vect f(t, \psi(t, s)) \\
     &= \frac{\partial}{\partial \vect x} \vect f(t, \psi(t, s)) \frac{\partial \psi}{\partial s}(t, s) \\
     &= \frac{\partial}{\partial \vect x} \frac{\partial}{\partial t} \psi(t, s) \frac{\partial \psi}{\partial s}(t, s).
\end{align*}
To obtain (\ref{E:s1-10}), we combined the last two terms in the LHS.
The reasoning for the (in-) equalities (\ref{E:s1-3}), (\ref{E:s1-4}), and (\ref{E:s1-5}) are as follows.
To obtain (\ref{E:s1-3}), the first term in (\ref{E:s1-10}) is simplified by substituting in
\begin{align*}
    \mathcal P(t, \vect x) \dot{\vect x} &= \mathcal P(t, \vect x) \big[\matr J(t, \vect x) \nabla H(t, \vect x) - \matr R \nabla H(t, \vect x) \big] \\
    &= -\matr R \nabla H(t, \vect x),
\end{align*}
where the last equality is a consequence of the definition (\ref{E:projection}).
%\begin{align*}
%    \dot{\vect x} - \mathcal P(\vect x) \dot{\vect x} &= \frac{\langle \matr J(\vect x) \nabla H(\vect x), \matr R^{-1} \dot{\vect x} \rangle}{\|\dot{\vect x} \| \| \matr J(\vect x) \nabla H(\vect x) \|} \matr J(\vect x) \nabla H(\vect x) \\
%    &= \frac{\langle \matr J(\vect x) \nabla H(\vect x), -\nabla H(\vect x) \rangle}{\|\dot{\vect x} \| \| \matr J(\vect x) \nabla H(\vect x) \|} \matr J(\vect x) \nabla H(\vect x) = 0.
%\end{align*}
In the LHS of (\ref{E:s1-4}), the second term is eliminated as follows. Note that, the integral of the second term over $s\in [0, 1]$, can be approximated up to arbitrary accuracy with a series of zigzag curve segments $\hat\psi_i(t, s),\, i = 1,\ldots,\, N$ such that
\begin{equation*}
    \mathcal P(t, \hat\psi_i(t, s)) \frac{\partial\hat\psi_i}{\partial s}(t, s) = \begin{cases}
        \frac{\partial\hat\psi_i}{\partial s}(t, s) &\text{transverse zig} \\
        0 &\text{parallel zag}
    \end{cases}
\end{equation*}
The approximation is illustrated in Fig~\ref{fig_quotient_distance}.
Since $\mathcal P(t, \hat\psi_i(t, s))$ shrinks the transverse zigs for $t > t_0$, for the transverse zigs, we have that
\begin{align*}
    &\Big\langle \mathcal P(t, \hat\psi_i(t, s)) \frac{\partial\hat\psi_i}{\partial s}(t, s), \frac{\partial}{\partial t} \mathcal P(t, \hat\psi_i(t, s)) \frac{\partial\hat\psi_i}{\partial s}(t, s) \Big\rangle \\
    &= \Big\langle \frac{\partial\hat\psi_i}{\partial s}(t, s), \frac{\partial}{\partial t} \mathcal P(t, \hat\psi_i(t, s)) \frac{\partial\hat\psi_i}{\partial s}(t, s) \Big\rangle\leq 0,
\end{align*}
because any change in the projection $\mathcal P(t, \hat\psi_i(t, s))$ decreases the length of the zig segment,
and, for the parallel zags,
\begin{align*}
    &\Big\langle \mathcal P(t, \hat\psi_i(t, s)) \frac{\partial\hat\psi_i}{\partial s}(t, s), \frac{\partial}{\partial t} \mathcal P(t, \hat\psi_i(t, s)) \frac{\partial\hat\psi_i}{\partial s}(t, s) \Big\rangle \\
    &= \Big\langle 0_n, \frac{\partial}{\partial t} \mathcal P(t, \hat\psi_i(t, s)) \frac{\partial\hat\psi_i}{\partial s}(t, s) \Big\rangle = 0
\end{align*}
because $\mathcal P(t, \hat\psi_i(t, s)) \frac{\partial\hat\psi_i}{\partial s}(t, s) = 0_n$.
To obtain (\ref{E:s1-5}), we used the same zigzag approximation and
\begin{equation*}
    c\, \lambda_{\max}(\matr R^{-1}) = \frac{c}{\lambda_{\min}(\matr R)} = \lambda_{\min}(D^2 H(t, \vect x)).
\end{equation*}

\noindent \textbf{3.} From steps 1 and 2, we have proved that, at $t = t_0$ and for a series of zigzag curve segments $\hat\psi_i(t, s),\, 1 = 1,\ldots,\, N$ approximating the minimizing (at $t = t_0$) curve $\psi(t, s)$, the inequality (\ref{E:s1-6}) holds. Since as the number of zigzag curves segments $N$ increases, both the LHS and the RHS of (\ref{E:s1-6}) converge to the minimum value. Hence (\ref{E:s1-1}) holds. \hfill $\square$

\begin{figure}[!t]
\subfloat{\includegraphics[width=2.5in,center,margin=0in 0in 0in 0in]{Figures/quotient_distance_fig.pdf}}
\caption{Illustration of a key step in proving the contraction of the quotient distance: a zigzag approximation of the minimizing curve and to prove contraction of every transverse zigs.}
\label{fig_quotient_distance}
\hfill\end{figure}


\subsection{Proof of Proposition~\ref{prop_no_diss}}
Choose the inner product $\langle \vect y, \vect x \rangle = \Re\{\vect y^* \vect x\}$. The proof is otherwise the same as the proof of Proposition~\ref{prop_main}. \hfill $\square$



\subsection{Proof of Proposition~\ref{prop_limit_cycle}}
From Proposition~\ref{prop_main}, we have that, for every initial condition $(t_0, \vect x_0)$, there is
\begin{equation*}
    \lim_{t\to \infty} \dist(t, \Phi(t, t_0, \vect x_0), \eqm{\vect x}(t)) = 0.
\end{equation*}
By property (iv) of the quotient distance that follows (\ref{E:quotient_distance}), it implies that
\begin{equation*}
    \lim_{t\to \infty} H(t, \Phi(t, t_0, \vect x_0) - H(t, \eqm{\vect x}(t)) = 0.
\end{equation*}
This completes the proof. \hfill $\square$





\subsection{Proof of Proposition~\ref{prop_convergence}}
From Proposition~\ref{prop_limit_cycle}, it suffices to show that there is a particular solution $\eqm{\vect x}(t)$ such that
\begin{equation} \label{E:s6-1}
    \frac{d}{dt} H(t, \eqm{\vect x}(t)) = 0
\end{equation}
for all $t \in \mathbb{R}$. To this end, consider the set $E = E_t$ in which the Hamiltonian has zero derivative. By definition, $E$ is an invariant set. Assume that the flow is complete. Then from any initial condition $(t_0, \vect x_0) \in \mathbb{R} \times E$, the solution $\eqm{\vect x}(t) = \Phi(t, t_0, \vect x_0)$ satisfies (\ref{E:s6-1}). Hence we have found a particular solution, which completes the proof. \hfill $\square$



\subsection{Proof of Lemma~\ref{lem_RLC}}

We change to coordinates that are rotating at the frequency $\omega_0$ by the change of variables
\begin{equation*}
    \vect x \leftarrow e^{-j\omega_0 t} \vect x.
\end{equation*}
The system equation then writes
\begin{equation*}
    \dot{\vect x} = (\matr J_1 - \matr R) \nabla H(\vect x) + \matr G u_1
\end{equation*}
where $u_1 = 1$ and $\matr J_1 = \matr J - j \omega_0 \matr Q^{-1}$. 
Now, consider the shifted Hamiltonian function
\begin{equation*}
    \mathcal H(\vect x, \eqm{\vect x}(0)) = \frac{1}{2} \big[\vect x - \eqm{\vect x}(0)\big]^\herm \matr Q \big[\vect x - \eqm{\vect x}(0)\big].
\end{equation*}
It is easy to find its time derivative as~\cite{monshizadeh2019conditions}
\begin{align*}
    \dot{\mathcal H}(\vect x, \eqm{\vect x}(0)) &= -\big[\vect x - \eqm{\vect x}(0)\big]^\herm \matr Q \matr R \matr Q \big[\vect x - \eqm{\vect x}(0) \big] \\
    &\leq -\lambda_{\min}(\matr R)\, \lambda_{\min}(\matr Q)\, \mathcal H( \vect x, \eqm{\vect x}(0)).
\end{align*}
Hence
\begin{align*}
    &\lim_{t\to\infty} \mathcal H(\vect x(t), \eqm{\vect x}(0)) \\
    &= \lim_{t\to \infty} \frac{1}{2} \big[ \vect x(t) - \eqm{\vect x}(0)\big]^\herm \matr Q \big[ \vect x(t) - \eqm{\vect x}(0)\big] = 0.
\end{align*}
Hence we obtain that the state vector $\vect x$ converges to the limit cycle $\eqm{\vect x}(0)$, and the orbit of the limit cycle is the limit set of all solutions. \hfill $\square$
%Note that, as a more granular distributed-parameter model is used, i.e., as $\varepsilon \to 0$, the co-state $\varepsilon \vect x = \nabla H(\vect x)$ remains to have the same order of magnitude as the voltages and currents. Hence we obtain that the co-state $\varepsilon \vect x$ converges to the limit cycle uniformly in the granular level of distributed-parameter model $\varepsilon$, and the orbit of the limit cycle is the limit set of all solutions.

\subsection{Proof of Proposition~\ref{prop_final}}

%Since the Hamiltonian, $H(\vect s) = \frac{1}{2} \vect s^\herm \matr Q^{-1} \vect s$, expressed in the co-state coordinates is a quadratic form, the limit cycle $\eqm{\vect x}(\tau)$ is contained in a level set of $H(\vect s)$.
By Proposition~\ref{prop_convergence}, we obtain that the limit set of every solution is contained in the level set of $H(\vect s) = \frac{1}{2} \vect s^\herm \matr Q^{-1} \vect s$ occupied by $\eqm{\vect x}(t)$, which is the first constraint we will use to characterize the limit set. We proceed to prove that the Hamiltonian of every edge satisfies the same property.

Consider a perturbation of the gradient of the Hamiltonian written as
\begin{equation*}
    \nabla \hat H(t, \vect x_1) = \matr P^{-1} \nabla H(t, \vect x_1)
\end{equation*}
where
\begin{equation*}
    \matr P = \mathrm{diag}(p_1 \matr I_2,\, p_2 \matr I_2,\, p_3,\, p_4,\, p_5,\, p_6,\, p_7 )
\end{equation*}
for $p_i > 0,\, i = 1,\ldots, 7$. The overall pH system can then be written as
\begin{equation*}
    \dot{\vect x}_1 = (\matr J(\vect x) - \matr R ) \matr P \nabla \hat H(t, \vect x_1).
\end{equation*}
It can be checked that
\begin{equation} \label{E:s3-1}
    \frac{1}{2} \left[(\matr J(\vect x) + \matr R ) \matr P + \matr P (\matr J(\vect x) + \matr R)^\herm\right]
\end{equation}
is a constant matrix, and (\ref{E:s3-1}) remains negative definite if we choose $\col(p_i) \approx 1_7$. By Proposition~\ref{prop_limit_cycle}, we have that the value of $\hat H(\vect s)$ on the positive limit set should be equal to the value at $\eqm{\vect x}(\tau)$. By choosing linearly independent $\col(p_i)$'s we can fix the value of the Hamiltonian of every edge; that is, the voltage amplitude of every shunt capacitor, the current amplitude of every R--L line, and the energy stored in every SG are all equal to their values at $\eqm{\vect x}(\tau)$.

To separate the Hamiltonian associated with the mechanical and the electrical energy of the SG, consider a perturbed system having a shunt capacitor that splits the stator inductance of each SG; that is, the subsystem (the subscript $i \in \{ 1, 2\}$ is omitted)
\begin{equation*}
    L \dot I = -R I - \psi j \omega e^{j\theta} + V
\end{equation*}
is replaced by
\begin{equation*}
    \begin{cases}
        \alpha L \dot I_1 = -\alpha R I_1 - \psi j \omega e^{j\theta} + V_1 \\
        C \dot V_1 = -G V_1 + I_2 - I_1 \\
        (1 - \alpha) L \dot I_2 = -(1 - \alpha) R I_2 - V_1 + V
    \end{cases}
\end{equation*}
where $0 < \alpha < 1$ and $C, G > 0$ are small.
By Theorem~3.5 in~\cite{Khalil:1173048}, as $\alpha, C, G$ tend to zero, the solutions of the perturbed system tends to those of the original system. 
Applying the same contraction analysis to the perturbed system, we have that the mechanical energy $\frac{1}{2} J \omega_i^2$ and stator electrical energy $\frac{1}{2} L_i \|I_i\|^2$ are both equal to their value at the perturbed limit cycle. Taking the added shunt capacitance to zero, we obtain that the value of $\omega_i$ in the positive limit set of the original system is equal to the synchronized frequency of $\eqm{\vect x}(\tau)$. The dynamics on the positive limit set is then constrained to be a passive RLC circuit with two voltage sources of the same frequency. By Lemma~\ref{lem_RLC}, the orbit of $\eqm{\vect x}(\tau)$ is the only possible limit set. \hfill $\square$