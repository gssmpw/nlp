%\documentclass[journal]{IEEEtran}
\documentclass[10pt,twocolumn,twoside]{IEEEtran}
\usepackage{amssymb, amsmath}
\usepackage{bm}
\usepackage{color}
\usepackage{graphicx}
\usepackage{cite}
\usepackage{booktabs}
\usepackage{multirow}
\usepackage{threeparttable}
\usepackage{enumerate}
\usepackage[english]{babel}
\usepackage{algorithmicx}
\usepackage{algorithm}
\usepackage{algpseudocode}
\usepackage{nomencl}
%\usepackage{slashbox}
%\usepackage{multicol}
\usepackage{commath}
\usepackage{makecell}
\usepackage{blindtext}
\usepackage{soul}
\usepackage{comment}
%\usepackage{bm}
\usepackage{mathtools}
\usepackage[svgnames]{xcolor}
%\usepackage{caption}
%\usepackage{subcaption}
\usepackage[caption=false,font=footnotesize]{subfig}
\usepackage{tikz}
\usepackage{derivative}
\usepackage[export]{adjustbox}
\usepackage[siunitx, RPvoltages]{circuitikz}
\usetikzlibrary{positioning, shapes}
\usepackage{url}
%\usepackage{stfloats}
\usepackage{accents}
%\usepackage[super]{nth}
%\usepackage{mathrsfs}
%\usepackage{verbatim}
%\usepackage{physics}
%\usepackage{courier}
%\renewcommand*\familydefault{\ttdefault} %% Only if the base font of the document is to be typewriter style
%\usepackage[T1]{fontenc}
\usepackage[inter-unit-product =\cdot]{siunitx}
%\usepackage{enumitem}

\pdfminorversion=4

\newcommand{\field}[1]{\mathbb{#1}}
\newcommand{\R}{\field{R}}
\newcommand{\vect}[1]{\mathbf{#1}}
\newcommand{\matr}[1]{\mathbf{#1}}
%\newcommand{\matr}[1]{#1}
\newcommand{\tran}{\mathsf{T}}
\newcommand{\herm}{\mathsf{H}}
\newcommand{\est}[1]{\hat{#1}}
\newcommand{\pluseq}{\mathrel{+}=}
%\newcommand{\unaryminus}{\scalebox{0.75}[1.0]{\( - \)}}
\newcommand{\unaryminus}{-}
\newcommand{\Yan}[1]{{\color{magenta}[Yan: {#1}]}}
\newcommand*\circled[1]{\tikz[baseline=(char.base)]{\node[shape=circle,draw,inner sep=0.7pt] (char) {#1};}}
\newcounter{MYtempeqncnt}
%\newcommand\eqm[1]{\accentset{\rule{.4em}{.4pt}}{#1}}
\DeclareMathOperator*{\argmax}{arg\,max}
\DeclareMathOperator*{\argmin}{arg\,min}
%\DeclareMathOperator*{\he}{He}
\DeclareMathOperator*{\interior}{int}
\DeclareMathOperator*{\img}{range}
\DeclareMathOperator*{\dist}{dist}
\DeclareMathOperator*{\spn}{span}
\DeclareMathOperator*{\col}{col}
\DeclareMathOperator*{\diag}{diag}
%\newcommand\eqm[1]{\overline{#1}}
\newcommand\eqm[1]{\bar{#1}}

%% code from mathabx.sty and mathabx.dcl
\DeclareFontFamily{U}{mathx}{\hyphenchar\font45}
\DeclareFontShape{U}{mathx}{m}{n}{
      <5> <6> <7> <8> <9> <10>
      <10.95> <12> <14.4> <17.28> <20.74> <24.88>
      mathx10
      }{}
\DeclareSymbolFont{mathx}{U}{mathx}{m}{n}
\DeclareFontSubstitution{U}{mathx}{m}{n}
\DeclareMathAccent{\widecheck}{0}{mathx}{"71}
\DeclareMathAccent{\wideparen}{0}{mathx}{"75}

\def\cs#1{\texttt{\char`\\#1}}


\newtheorem{theorem}{Theorem}
\newtheorem{lemma}{Lemma}
\newtheorem{definition}{Definition}
\newtheorem{remark}{Remark}
\newtheorem{proposition}{Proposition}
\newtheorem{assumption}{Assumption}
\newtheorem{problem}{Problem}
\newtheorem{example}{Example}
\newtheorem{corollary}{Corollary}
\newtheorem{notation}{Notation}
%\newenvironment{subproof}[1][\proofname]{%
%  \renewcommand{\qedsymbol}{$\blacksquare$}%
%  \begin{proof}[#1]%
%}{%
%  \end{proof}%
%}
\newcommand{\ts}{\textsuperscript}

\makeatletter
\renewcommand*\env@matrix[1][\arraystretch]{%
  \edef\arraystretch{#1}%
  \hskip -\arraycolsep
  \let\@ifnextchar\new@ifnextchar
  \array{*\c@MaxMatrixCols c}}
\makeatother

\newcounter{mytempeqncnt}





\title{
Power System Electromagnetic Transient Stability: an Analysis Based on Convergent Hamiltonian
%Global Stability Analysis of Electromagnetic Power System Dynamics Based on the Converging Hamiltonian Approach
%Analysis of Power System Stability Based on Contraction in the Quotient Space and the Implied Converging Hamiltonian
}
\author{Xinyuan Jiang, Constantino M. Lagoa, and Yan Li% <-this % stops a space
\thanks{The authors are with the Department of Electrical Engineering, The Pennsylvania State University, University Park, PA 16802, USA (e-mail: xuj49@psu.edu; cml18@psu.edu; yql5925@psu.edu).}}

\begin{document}

\maketitle

\begin{abstract}  
Test time scaling is currently one of the most active research areas that shows promise after training time scaling has reached its limits.
Deep-thinking (DT) models are a class of recurrent models that can perform easy-to-hard generalization by assigning more compute to harder test samples.
However, due to their inability to determine the complexity of a test sample, DT models have to use a large amount of computation for both easy and hard test samples.
Excessive test time computation is wasteful and can cause the ``overthinking'' problem where more test time computation leads to worse results.
In this paper, we introduce a test time training method for determining the optimal amount of computation needed for each sample during test time.
We also propose Conv-LiGRU, a novel recurrent architecture for efficient and robust visual reasoning. 
Extensive experiments demonstrate that Conv-LiGRU is more stable than DT, effectively mitigates the ``overthinking'' phenomenon, and achieves superior accuracy.
\end{abstract}  
\section{Introduction}
\label{sec:introduction}
The business processes of organizations are experiencing ever-increasing complexity due to the large amount of data, high number of users, and high-tech devices involved \cite{martin2021pmopportunitieschallenges, beerepoot2023biggestbpmproblems}. This complexity may cause business processes to deviate from normal control flow due to unforeseen and disruptive anomalies \cite{adams2023proceddsriftdetection}. These control-flow anomalies manifest as unknown, skipped, and wrongly-ordered activities in the traces of event logs monitored from the execution of business processes \cite{ko2023adsystematicreview}. For the sake of clarity, let us consider an illustrative example of such anomalies. Figure \ref{FP_ANOMALIES} shows a so-called event log footprint, which captures the control flow relations of four activities of a hypothetical event log. In particular, this footprint captures the control-flow relations between activities \texttt{a}, \texttt{b}, \texttt{c} and \texttt{d}. These are the causal ($\rightarrow$) relation, concurrent ($\parallel$) relation, and other ($\#$) relations such as exclusivity or non-local dependency \cite{aalst2022pmhandbook}. In addition, on the right are six traces, of which five exhibit skipped, wrongly-ordered and unknown control-flow anomalies. For example, $\langle$\texttt{a b d}$\rangle$ has a skipped activity, which is \texttt{c}. Because of this skipped activity, the control-flow relation \texttt{b}$\,\#\,$\texttt{d} is violated, since \texttt{d} directly follows \texttt{b} in the anomalous trace.
\begin{figure}[!t]
\centering
\includegraphics[width=0.9\columnwidth]{images/FP_ANOMALIES.png}
\caption{An example event log footprint with six traces, of which five exhibit control-flow anomalies.}
\label{FP_ANOMALIES}
\end{figure}

\subsection{Control-flow anomaly detection}
Control-flow anomaly detection techniques aim to characterize the normal control flow from event logs and verify whether these deviations occur in new event logs \cite{ko2023adsystematicreview}. To develop control-flow anomaly detection techniques, \revision{process mining} has seen widespread adoption owing to process discovery and \revision{conformance checking}. On the one hand, process discovery is a set of algorithms that encode control-flow relations as a set of model elements and constraints according to a given modeling formalism \cite{aalst2022pmhandbook}; hereafter, we refer to the Petri net, a widespread modeling formalism. On the other hand, \revision{conformance checking} is an explainable set of algorithms that allows linking any deviations with the reference Petri net and providing the fitness measure, namely a measure of how much the Petri net fits the new event log \cite{aalst2022pmhandbook}. Many control-flow anomaly detection techniques based on \revision{conformance checking} (hereafter, \revision{conformance checking}-based techniques) use the fitness measure to determine whether an event log is anomalous \cite{bezerra2009pmad, bezerra2013adlogspais, myers2018icsadpm, pecchia2020applicationfailuresanalysispm}. 

The scientific literature also includes many \revision{conformance checking}-independent techniques for control-flow anomaly detection that combine specific types of trace encodings with machine/deep learning \cite{ko2023adsystematicreview, tavares2023pmtraceencoding}. Whereas these techniques are very effective, their explainability is challenging due to both the type of trace encoding employed and the machine/deep learning model used \cite{rawal2022trustworthyaiadvances,li2023explainablead}. Hence, in the following, we focus on the shortcomings of \revision{conformance checking}-based techniques to investigate whether it is possible to support the development of competitive control-flow anomaly detection techniques while maintaining the explainable nature of \revision{conformance checking}.
\begin{figure}[!t]
\centering
\includegraphics[width=\columnwidth]{images/HIGH_LEVEL_VIEW.png}
\caption{A high-level view of the proposed framework for combining \revision{process mining}-based feature extraction with dimensionality reduction for control-flow anomaly detection.}
\label{HIGH_LEVEL_VIEW}
\end{figure}

\subsection{Shortcomings of \revision{conformance checking}-based techniques}
Unfortunately, the detection effectiveness of \revision{conformance checking}-based techniques is affected by noisy data and low-quality Petri nets, which may be due to human errors in the modeling process or representational bias of process discovery algorithms \cite{bezerra2013adlogspais, pecchia2020applicationfailuresanalysispm, aalst2016pm}. Specifically, on the one hand, noisy data may introduce infrequent and deceptive control-flow relations that may result in inconsistent fitness measures, whereas, on the other hand, checking event logs against a low-quality Petri net could lead to an unreliable distribution of fitness measures. Nonetheless, such Petri nets can still be used as references to obtain insightful information for \revision{process mining}-based feature extraction, supporting the development of competitive and explainable \revision{conformance checking}-based techniques for control-flow anomaly detection despite the problems above. For example, a few works outline that token-based \revision{conformance checking} can be used for \revision{process mining}-based feature extraction to build tabular data and develop effective \revision{conformance checking}-based techniques for control-flow anomaly detection \cite{singh2022lapmsh, debenedictis2023dtadiiot}. However, to the best of our knowledge, the scientific literature lacks a structured proposal for \revision{process mining}-based feature extraction using the state-of-the-art \revision{conformance checking} variant, namely alignment-based \revision{conformance checking}.

\subsection{Contributions}
We propose a novel \revision{process mining}-based feature extraction approach with alignment-based \revision{conformance checking}. This variant aligns the deviating control flow with a reference Petri net; the resulting alignment can be inspected to extract additional statistics such as the number of times a given activity caused mismatches \cite{aalst2022pmhandbook}. We integrate this approach into a flexible and explainable framework for developing techniques for control-flow anomaly detection. The framework combines \revision{process mining}-based feature extraction and dimensionality reduction to handle high-dimensional feature sets, achieve detection effectiveness, and support explainability. Notably, in addition to our proposed \revision{process mining}-based feature extraction approach, the framework allows employing other approaches, enabling a fair comparison of multiple \revision{conformance checking}-based and \revision{conformance checking}-independent techniques for control-flow anomaly detection. Figure \ref{HIGH_LEVEL_VIEW} shows a high-level view of the framework. Business processes are monitored, and event logs obtained from the database of information systems. Subsequently, \revision{process mining}-based feature extraction is applied to these event logs and tabular data input to dimensionality reduction to identify control-flow anomalies. We apply several \revision{conformance checking}-based and \revision{conformance checking}-independent framework techniques to publicly available datasets, simulated data of a case study from railways, and real-world data of a case study from healthcare. We show that the framework techniques implementing our approach outperform the baseline \revision{conformance checking}-based techniques while maintaining the explainable nature of \revision{conformance checking}.

In summary, the contributions of this paper are as follows.
\begin{itemize}
    \item{
        A novel \revision{process mining}-based feature extraction approach to support the development of competitive and explainable \revision{conformance checking}-based techniques for control-flow anomaly detection.
    }
    \item{
        A flexible and explainable framework for developing techniques for control-flow anomaly detection using \revision{process mining}-based feature extraction and dimensionality reduction.
    }
    \item{
        Application to synthetic and real-world datasets of several \revision{conformance checking}-based and \revision{conformance checking}-independent framework techniques, evaluating their detection effectiveness and explainability.
    }
\end{itemize}

The rest of the paper is organized as follows.
\begin{itemize}
    \item Section \ref{sec:related_work} reviews the existing techniques for control-flow anomaly detection, categorizing them into \revision{conformance checking}-based and \revision{conformance checking}-independent techniques.
    \item Section \ref{sec:abccfe} provides the preliminaries of \revision{process mining} to establish the notation used throughout the paper, and delves into the details of the proposed \revision{process mining}-based feature extraction approach with alignment-based \revision{conformance checking}.
    \item Section \ref{sec:framework} describes the framework for developing \revision{conformance checking}-based and \revision{conformance checking}-independent techniques for control-flow anomaly detection that combine \revision{process mining}-based feature extraction and dimensionality reduction.
    \item Section \ref{sec:evaluation} presents the experiments conducted with multiple framework and baseline techniques using data from publicly available datasets and case studies.
    \item Section \ref{sec:conclusions} draws the conclusions and presents future work.
\end{itemize}
%% !TEX root =  ../main.tex
\section{Background on causality and abstraction}\label{sec:preliminaries}

This section provides the notation and key concepts related to causal modeling and abstraction theory.

\spara{Notation.} The set of integers from $1$ to $n$ is $[n]$.
The vectors of zeros and ones of size $n$ are $\zeros_n$ and $\ones_n$.
The identity matrix of size $n \times n$ is $\identity_n$. The Frobenius norm is $\frob{\mathbf{A}}$.
The set of positive definite matrices over $\reall^{n\times n}$ is $\pd^n$. The Hadamard product is $\odot$.
Function composition is $\circ$.
The domain of a function is $\dom{\cdot}$ and its kernel $\ker$.
Let $\mathcal{M}(\mathcal{X}^n)$ be the set of Borel measures over $\mathcal{X}^n \subseteq \reall^n$. Given a measure $\mu^n \in \mathcal{M}(\mathcal{X}^n)$ and a measurable map $\varphi^{\V}$, $\mathcal{X}^n \ni \mathbf{x} \overset{\varphi^{\V}}{\longmapsto} \V^\top \mathbf{x} \in \mathcal{X}^m$, we denote by $\varphi^{\V}_{\#}(\mu^n) \coloneqq \mu^n(\varphi^{\V^{-1}}(\mathbf{x}))$ the pushforward measure $\mu^m \in \mathcal{M}(\mathcal{X}^m)$. 


We now present the standard definition of SCM.

\begin{definition}[SCM, \citealp{pearl2009causality}]\label{def:SCM}
A (Markovian) structural causal model (SCM) $\scm^n$ is a tuple $\langle \myendogenous, \myexogenous, \myfunctional, \zeta^\myexogenous \rangle$, where \emph{(i)} $\myendogenous = \{X_1, \ldots, X_n\}$ is a set of $n$ endogenous random variables; \emph{(ii)} $\myexogenous =\{Z_1,\ldots,Z_n\}$ is a set of $n$ exogenous variables; \emph{(iii)} $\myfunctional$ is a set of $n$ functional assignments such that $X_i=f_i(\parents_i, Z_i)$, $\forall \; i \in [n]$, with $ \parents_i \subseteq \myendogenous \setminus \{ X_i\}$; \emph{(iv)} $\zeta^\myexogenous$ is a product probability measure over independent exogenous variables $\zeta^\myexogenous=\prod_{i \in [n]} \zeta^i$, where $\zeta^i=P(Z_i)$. 
\end{definition}
A Markovian SCM induces a directed acyclic graph (DAG) $\mathcal{G}_{\scm^n}$ where the nodes represent the variables $\myendogenous$ and the edges are determined by the structural functions $\myfunctional$; $ \parents_i$ constitutes then the parent set for $X_i$. Furthermore, we can recursively rewrite the set of structural function $\myfunctional$ as a set of mixing functions $\mymixing$ dependent only on the exogenous variables (cf. \cref{app:CA}). A key feature for studying causality is the possibility of defining interventions on the model:
\begin{definition}[Hard intervention, \citealp{pearl2009causality}]\label{def:intervention}
Given SCM $\scm^n = \langle \myendogenous, \myexogenous, \myfunctional, \zeta^\myexogenous \rangle$, a (hard) intervention $\iota = \operatorname{do}(\myendogenous^{\iota} = \mathbf{x}^{\iota})$, $\myendogenous^{\iota}\subseteq \myendogenous$,
is an operator that generates a new post-intervention SCM $\scm^n_\iota = \langle \myendogenous, \myexogenous, \myfunctional_\iota, \zeta^\myexogenous \rangle$ by replacing each function $f_i$ for $X_i\in\myendogenous^{\iota}$ with the constant $x_i^\iota\in \mathbf{x}^\iota$. 
Graphically, an intervention mutilates $\mathcal{G}_{\mathsf{M}^n}$ by removing all the incoming edges of the variables in $\myendogenous^{\iota}$.
\end{definition}

Given multiple SCMs describing the same system at different levels of granularity, CA provides the definition of an $\alpha$-abstraction map to relate these SCMs:
\begin{definition}[$\abst$-abstraction, \citealp{rischel2020category}]\label{def:abstraction}
Given low-level $\mathsf{M}^\ell$ and high-level $\mathsf{M}^h$ SCMs, an $\abst$-abstraction is a triple $\abst = \langle \Rset, \amap, \alphamap{} \rangle$, where \emph{(i)} $\Rset \subseteq \datalow$ is a subset of relevant variables in $\mathsf{M}^\ell$; \emph{(ii)} $\amap: \Rset \rightarrow \datahigh$ is a surjective function between the relevant variables of $\mathsf{M}^\ell$ and the endogenous variables of $\mathsf{M}^h$; \emph{(iii)} $\alphamap{}: \dom{\Rset} \rightarrow \dom{\datahigh}$ is a modular function $\alphamap{} = \bigotimes_{i\in[n]} \alphamap{X^h_i}$ made up by surjective functions $\alphamap{X^h_i}: \dom{\amap^{-1}(X^h_i)} \rightarrow \dom{X^h_i}$ from the outcome of low-level variables $\amap^{-1}(X^h_i) \in \datalow$ onto outcomes of the high-level variables $X^h_i \in \datahigh$.
\end{definition}
Notice that an $\abst$-abstraction simultaneously maps variables via the function $\amap$ and values through the function $\alphamap{}$. The definition itself does not place any constraint on these functions, although a common requirement in the literature is for the abstraction to satisfy \emph{interventional consistency} \cite{rubenstein2017causal,rischel2020category,beckers2019abstracting}. An important class of such well-behaved abstractions is \emph{constructive linear abstraction}, for which the following properties hold. By constructivity, \emph{(i)} $\abst$ is interventionally consistent; \emph{(ii)} all low-level variables are relevant $\Rset=\datalow$; \emph{(iii)} in addition to the map $\alphamap{}$ between endogenous variables, there exists a map ${\alphamap{}}_U$ between exogenous variables satisfying interventional consistency \cite{beckers2019abstracting,schooltink2024aligning}. By linearity, $\alphamap{} = \V^\top \in \reall^{h \times \ell}$ \cite{massidda2024learningcausalabstractionslinear}. \cref{app:CA} provides formal definitions for interventional consistency, linear and constructive abstraction.
%\section{Experiments}
\label{sec:Experiments} 

We conduct several experiments across different problem settings to assess the efficiency of our proposed method. Detailed descriptions of the experimental settings are provided in \cref{sec:apendix_experiments}.
%We conduct experiments on optimizing PINNs for convection, wave PDEs, and a reaction ODE. 
%These equations have been studied in previous works investigating difficulties in training PINNs; we use the formulations in \citet{krishnapriyan2021characterizing, wang2022when} for our experiments. 
%The coefficient settings we use for these equations are considered challenging in the literature \cite{krishnapriyan2021characterizing, wang2022when}.
%\cref{sec:problem_setup_additional} contains additional details.

%We compare the performance of Adam, \lbfgs{}, and \al{} on training PINNs for all three classes of PDEs. 
%For Adam, we tune the learning rate by a grid search on $\{10^{-5}, 10^{-4}, 10^{-3}, 10^{-2}, 10^{-1}\}$.
%For \lbfgs, we use the default learning rate $1.0$, memory size $100$, and strong Wolfe line search.
%For \al, we tune the learning rate for Adam as before, and also vary the switch from Adam to \lbfgs{} (after 1000, 11000, 31000 iterations).
%These correspond to \al{} (1k), \al{} (11k), and \al{} (31k) in our figures.
%All three methods are run for a total of 41000 iterations.

%We use multilayer perceptrons (MLPs) with tanh activations and three hidden layers. These MLPs have widths 50, 100, 200, or 400.
%We initialize these networks with the Xavier normal initialization \cite{glorot2010understanding} and all biases equal to zero.
%Each combination of PDE, optimizer, and MLP architecture is run with 5 random seeds.

%We use 10000 residual points randomly sampled from a $255 \times 100$ grid on the interior of the problem domain. 
%We use 257 equally spaced points for the initial conditions and 101 equally spaced points for each boundary condition.

%We assess the discrepancy between the PINN solution and the ground truth using $\ell_2$ relative error (L2RE), a standard metric in the PINN literature. Let $y = (y_i)_{i = 1}^n$ be the PINN prediction and $y' = (y'_i)_{i = 1}^n$ the ground truth. Define
%\begin{align*}
%    \mathrm{L2RE} = \sqrt{\frac{\sum_{i = 1}^n (y_i - y'_i)^2}{\sum_{i = 1}^n y'^2_i}} = \sqrt{\frac{\|y - y'\|_2^2}{\|y'\|_2^2}}.
%\end{align*}
%We compute the L2RE using all points in the $255 \times 100$ grid on the interior of the problem domain, along with the 257 and 101 points used for the initial and boundary conditions.

%We develop our experiments in PyTorch 2.0.0 \cite{paszke2019pytorch} with Python 3.10.12.
%Each experiment is run on a single NVIDIA Titan V GPU using CUDA 11.8.
%The code for our experiments is available at \href{https://github.com/pratikrathore8/opt_for_pinns}{https://github.com/pratikrathore8/opt\_for\_pinns}.


\subsection{2D Allen Cahn Equation}
\begin{figure*}[t]
    \centering
    \includegraphics[scale=0.38]{figs/Burgers_operator.pdf}
    \caption{1D Burgers' Equation (Operator Learning): Steady-state solutions for different initializations $u_0$ under varying viscosity $\varepsilon$: (a) $\varepsilon = 0.5$, (b) $\varepsilon = 0.1$, (c) $\varepsilon = 0.05$. The results demonstrate that all final test solutions converge to the correct steady-state solution. (d) Illustration of the evolution of a test initialization $u_0$ following homotopy dynamics. The number of residual points is $\nres = 128$.}
    \label{fig:Burgers_result}
\end{figure*}
First, we consider the following time-dependent problem:
\begin{align}
& u_t = \varepsilon^2 \Delta u - u(u^2 - 1), \quad (x, y) \in [-1, 1] \times [-1, 1] \nonumber \\
& u(x, y, 0) = - \sin(\pi x) \sin(\pi y) \label{eq.hom_2D_AC}\\
& u(-1, y, t) = u(1, y, t) = u(x, -1, t) = u(x, 1, t) = 0. \nonumber
\end{align}
We aim to find the steady-state solution for this equation with $\varepsilon = 0.05$ and define the homotopy as:
\begin{equation}
    H(u, s, \varepsilon) = (1 - s)\left(\varepsilon(s)^2 \Delta u - u(u^2 - 1)\right) + s(u - u_0),\nonumber
\end{equation}
where $s \in [0, 1]$. Specifically, when $s = 1$, the initial condition $u_0$ is automatically satisfied, and when $s = 0$, it recovers the steady-state problem. The function $\varepsilon(s)$ is given by
\begin{equation}
\varepsilon(s) = 
\left\{\begin{array}{l}
s, \quad s \in [0.05, 1], \\
0.05, \quad s \in [0, 0.05].
\end{array}\right.\label{eq:epsilon_t}
\end{equation}

Here, $\varepsilon(s)$ varies with $s$ during the first half of the evolution. Once $\varepsilon(s)$ reaches $0.05$, it remains fixed, and only $s$ continues to evolve toward $0$. As shown in \cref{fig:2D_Allen_Cahn_Equation}, the relative $L_2$ error by homotopy dynamics is $8.78 \times 10^{-3}$, compared with the result obtained by PINN, which has a $L_2$ error of $9.56 \times 10^{-1}$. This clearly demonstrates that the homotopy dynamics-based approach significantly improves accuracy.

\subsection{High Frequency Function Approximation }
We aim to approximate the following function:
$u=    \sin(50\pi x), \quad x \in [0,1].$
The homotopy is defined as $H(u,\varepsilon) = u - \sin(\frac{1}{\varepsilon}\pi x), $
where $\varepsilon \in [\frac{1}{50},\frac{1}{15}]$.

\begin{table}[htbp!]
    \caption{Comparison of the lowest loss achieved by the classical training and homotopy dynamics for different values of $\varepsilon$ in approximating $\sin\left(\frac{1}{\varepsilon} \pi x\right)$
    }
    \vskip 0.15in
    \centering
    \tiny
    \begin{tabular}{|c|c|c|c|c|} 
    \hline 
    $ $ & $\varepsilon = 1/15$ & $\varepsilon = 1/35$ & $\varepsilon = 1/50$ \\ \hline 
    Classical Loss                & 4.91e-6     & 7.21e-2     & 3.29e-1       \\ \hline 
    Homotopy Loss $L_H$                      & 1.73e-6     & 1.91e-6     & \textbf{2.82e-5}       \\ \hline
    \end{tabular}
    % On convection, \al{} provides 14.2$\times$ and 1.97$\times$ improvement over Adam or \lbfgs{} on L2RE. 
    % On reaction, \al{} provides 1.10$\times$ and 1.99$\times$ improvement over Adam or \lbfgs{} on L2RE.
    % On wave, \al{} provides 6.32$\times$ and 6.07$\times$ improvement over Adam or \lbfgs{} on L2RE.}
    \label{tab:loss_approximate}
\end{table}

As shown in \cref{fig:high_frequency_result}, due to the F-principle \cite{xu2024overview}, training is particularly challenging when approximating high-frequency functions like $\sin(50\pi x)$. The loss decreases slowly, resulting in poor approximation performance. However, training based on homotopy dynamics significantly reduces the loss, leading to a better approximation of high-frequency functions. This demonstrates that homotopy dynamics-based training can effectively facilitate convergence when approximating high-frequency data. Additionally, we compare the loss for approximating functions with different frequencies $1/\varepsilon$ using both methods. The results, presented in \cref{tab:loss_approximate}, show that the homotopy dynamics training method consistently performs well for high-frequency functions.





\subsection{Burgers Equation}
In this example, we adopt the operator learning framework to solve for the steady-state solution of the Burgers equation, given by:
\begin{align}
& u_t+\left(\frac{u^2}{2}\right)_x - \varepsilon u_{xx}=\pi \sin (\pi x) \cos (\pi x), \quad x \in[0, 1]\nonumber\\
& u(x, 0)=u_0(x),\label{eq:1D_Burgers} \\
& u(0, t)=u(1, t)=0, \nonumber 
\end{align}
with Dirichlet boundary conditions, where $u_0 \in L_{0}^2((0, 1); \mathbb{R})$ is the initial condition and $\varepsilon \in \mathbb{R}$ is the viscosity coefficient. We aim to learn the operator mapping the initial condition to the steady-state solution, $G^{\dagger}: L_{0}^2((0, 1); \mathbb{R}) \rightarrow H_{0}^r((0, 1); \mathbb{R})$, defined by $u_0 \mapsto u_{\infty}$ for any $r > 0$. As shown in Theorem 2.2 of \cite{KREISS1986161} and Theorems 2.5 and 2.7 of \cite{hao2019convergence}, for any $\varepsilon > 0$, the steady-state solution is independent of the initial condition, with a single shock occurring at $x_s = 0.5$. Here, we use DeepONet~\cite{lu2021deeponet} as the network architecture. 
The homotopy definition, similar to ~\cref{eq.hom_2D_AC}, can be found in \cref{Ap:operator}. The results can be found in \cref{fig:Burgers_result} and \cref{tab:loss_burgers}. Experimental results show that the homotopy dynamics strategy performs well in the operator learning setting as well.


\begin{table}[htbp!]
    \caption{Comparison of loss between classical training and homotopy dynamics for different values of $\varepsilon$ in the Burgers equation, along with the MSE distance to the ground truth shock location, $x_s$.}
    \vskip 0.15in
    \centering
    \tiny
    \begin{tabular}{|c|c|c|c|c|} 
    \hline  
    $ $ & $\varepsilon = 0.5$ & $\varepsilon = 0.1$ & $\varepsilon = 0.05$ \\ \hline 
    Homotopy Loss $L_H$                &  7.55e-7     & 3.40e-7     & 7.77e-7       \\ \hline 
    L2RE                      & 1.50e-3     & 7.00e-4     & 2.52e-2       \\ \hline
        MSE Distance $x_s$                      & 1.75e-8     & 9.14e-8      & 1.2e-3      \\ \hline
    \end{tabular}
    % On convection, \al{} provides 14.2$\times$ and 1.97$\times$ improvement over Adam or \lbfgs{} on L2RE. 
    % On reaction, \al{} provides 1.10$\times$ and 1.99$\times$ improvement over Adam or \lbfgs{} on L2RE.
    % On wave, \al{} provides 6.32$\times$ and 6.07$\times$ improvement over Adam or \lbfgs{} on L2RE.}
    \label{tab:loss_burgers}
\end{table}



% \begin{itemize}
%     \item Relate the curvature in the problem to the differential operator. Use this to demonstrate why the problem is ill-conditioned
%     \item Give an argument for why using Adam + L-BFGS is better than just using L-BFGS outright. The idea is that Adam lowers the errors to the point where the rest of the optimization becomes convex \ldots
%     \item Show why we need second-order methods. We would like to prove that once we are close to the optimum, second-order methods will give condition-number free linear convergence. Specialize this to the Gauss-Newton setting, with the randomized low-rank approximation.
%     % \item Show that it is not possible to get superlinear convergence under the interpolation assumption with an overparameterized neural network. This should be true b/c the Hessian at the optimum will have rank $\min(n, d)$, and when $d > n$, this guarantees that we cannot have strong convexity.
% \end{itemize}
\section{Experiments}
\label{sec:Experiments} 

We conduct several experiments across different problem settings to assess the efficiency of our proposed method. Detailed descriptions of the experimental settings are provided in \cref{sec:apendix_experiments}.
%We conduct experiments on optimizing PINNs for convection, wave PDEs, and a reaction ODE. 
%These equations have been studied in previous works investigating difficulties in training PINNs; we use the formulations in \citet{krishnapriyan2021characterizing, wang2022when} for our experiments. 
%The coefficient settings we use for these equations are considered challenging in the literature \cite{krishnapriyan2021characterizing, wang2022when}.
%\cref{sec:problem_setup_additional} contains additional details.

%We compare the performance of Adam, \lbfgs{}, and \al{} on training PINNs for all three classes of PDEs. 
%For Adam, we tune the learning rate by a grid search on $\{10^{-5}, 10^{-4}, 10^{-3}, 10^{-2}, 10^{-1}\}$.
%For \lbfgs, we use the default learning rate $1.0$, memory size $100$, and strong Wolfe line search.
%For \al, we tune the learning rate for Adam as before, and also vary the switch from Adam to \lbfgs{} (after 1000, 11000, 31000 iterations).
%These correspond to \al{} (1k), \al{} (11k), and \al{} (31k) in our figures.
%All three methods are run for a total of 41000 iterations.

%We use multilayer perceptrons (MLPs) with tanh activations and three hidden layers. These MLPs have widths 50, 100, 200, or 400.
%We initialize these networks with the Xavier normal initialization \cite{glorot2010understanding} and all biases equal to zero.
%Each combination of PDE, optimizer, and MLP architecture is run with 5 random seeds.

%We use 10000 residual points randomly sampled from a $255 \times 100$ grid on the interior of the problem domain. 
%We use 257 equally spaced points for the initial conditions and 101 equally spaced points for each boundary condition.

%We assess the discrepancy between the PINN solution and the ground truth using $\ell_2$ relative error (L2RE), a standard metric in the PINN literature. Let $y = (y_i)_{i = 1}^n$ be the PINN prediction and $y' = (y'_i)_{i = 1}^n$ the ground truth. Define
%\begin{align*}
%    \mathrm{L2RE} = \sqrt{\frac{\sum_{i = 1}^n (y_i - y'_i)^2}{\sum_{i = 1}^n y'^2_i}} = \sqrt{\frac{\|y - y'\|_2^2}{\|y'\|_2^2}}.
%\end{align*}
%We compute the L2RE using all points in the $255 \times 100$ grid on the interior of the problem domain, along with the 257 and 101 points used for the initial and boundary conditions.

%We develop our experiments in PyTorch 2.0.0 \cite{paszke2019pytorch} with Python 3.10.12.
%Each experiment is run on a single NVIDIA Titan V GPU using CUDA 11.8.
%The code for our experiments is available at \href{https://github.com/pratikrathore8/opt_for_pinns}{https://github.com/pratikrathore8/opt\_for\_pinns}.


\subsection{2D Allen Cahn Equation}
\begin{figure*}[t]
    \centering
    \includegraphics[scale=0.38]{figs/Burgers_operator.pdf}
    \caption{1D Burgers' Equation (Operator Learning): Steady-state solutions for different initializations $u_0$ under varying viscosity $\varepsilon$: (a) $\varepsilon = 0.5$, (b) $\varepsilon = 0.1$, (c) $\varepsilon = 0.05$. The results demonstrate that all final test solutions converge to the correct steady-state solution. (d) Illustration of the evolution of a test initialization $u_0$ following homotopy dynamics. The number of residual points is $\nres = 128$.}
    \label{fig:Burgers_result}
\end{figure*}
First, we consider the following time-dependent problem:
\begin{align}
& u_t = \varepsilon^2 \Delta u - u(u^2 - 1), \quad (x, y) \in [-1, 1] \times [-1, 1] \nonumber \\
& u(x, y, 0) = - \sin(\pi x) \sin(\pi y) \label{eq.hom_2D_AC}\\
& u(-1, y, t) = u(1, y, t) = u(x, -1, t) = u(x, 1, t) = 0. \nonumber
\end{align}
We aim to find the steady-state solution for this equation with $\varepsilon = 0.05$ and define the homotopy as:
\begin{equation}
    H(u, s, \varepsilon) = (1 - s)\left(\varepsilon(s)^2 \Delta u - u(u^2 - 1)\right) + s(u - u_0),\nonumber
\end{equation}
where $s \in [0, 1]$. Specifically, when $s = 1$, the initial condition $u_0$ is automatically satisfied, and when $s = 0$, it recovers the steady-state problem. The function $\varepsilon(s)$ is given by
\begin{equation}
\varepsilon(s) = 
\left\{\begin{array}{l}
s, \quad s \in [0.05, 1], \\
0.05, \quad s \in [0, 0.05].
\end{array}\right.\label{eq:epsilon_t}
\end{equation}

Here, $\varepsilon(s)$ varies with $s$ during the first half of the evolution. Once $\varepsilon(s)$ reaches $0.05$, it remains fixed, and only $s$ continues to evolve toward $0$. As shown in \cref{fig:2D_Allen_Cahn_Equation}, the relative $L_2$ error by homotopy dynamics is $8.78 \times 10^{-3}$, compared with the result obtained by PINN, which has a $L_2$ error of $9.56 \times 10^{-1}$. This clearly demonstrates that the homotopy dynamics-based approach significantly improves accuracy.

\subsection{High Frequency Function Approximation }
We aim to approximate the following function:
$u=    \sin(50\pi x), \quad x \in [0,1].$
The homotopy is defined as $H(u,\varepsilon) = u - \sin(\frac{1}{\varepsilon}\pi x), $
where $\varepsilon \in [\frac{1}{50},\frac{1}{15}]$.

\begin{table}[htbp!]
    \caption{Comparison of the lowest loss achieved by the classical training and homotopy dynamics for different values of $\varepsilon$ in approximating $\sin\left(\frac{1}{\varepsilon} \pi x\right)$
    }
    \vskip 0.15in
    \centering
    \tiny
    \begin{tabular}{|c|c|c|c|c|} 
    \hline 
    $ $ & $\varepsilon = 1/15$ & $\varepsilon = 1/35$ & $\varepsilon = 1/50$ \\ \hline 
    Classical Loss                & 4.91e-6     & 7.21e-2     & 3.29e-1       \\ \hline 
    Homotopy Loss $L_H$                      & 1.73e-6     & 1.91e-6     & \textbf{2.82e-5}       \\ \hline
    \end{tabular}
    % On convection, \al{} provides 14.2$\times$ and 1.97$\times$ improvement over Adam or \lbfgs{} on L2RE. 
    % On reaction, \al{} provides 1.10$\times$ and 1.99$\times$ improvement over Adam or \lbfgs{} on L2RE.
    % On wave, \al{} provides 6.32$\times$ and 6.07$\times$ improvement over Adam or \lbfgs{} on L2RE.}
    \label{tab:loss_approximate}
\end{table}

As shown in \cref{fig:high_frequency_result}, due to the F-principle \cite{xu2024overview}, training is particularly challenging when approximating high-frequency functions like $\sin(50\pi x)$. The loss decreases slowly, resulting in poor approximation performance. However, training based on homotopy dynamics significantly reduces the loss, leading to a better approximation of high-frequency functions. This demonstrates that homotopy dynamics-based training can effectively facilitate convergence when approximating high-frequency data. Additionally, we compare the loss for approximating functions with different frequencies $1/\varepsilon$ using both methods. The results, presented in \cref{tab:loss_approximate}, show that the homotopy dynamics training method consistently performs well for high-frequency functions.





\subsection{Burgers Equation}
In this example, we adopt the operator learning framework to solve for the steady-state solution of the Burgers equation, given by:
\begin{align}
& u_t+\left(\frac{u^2}{2}\right)_x - \varepsilon u_{xx}=\pi \sin (\pi x) \cos (\pi x), \quad x \in[0, 1]\nonumber\\
& u(x, 0)=u_0(x),\label{eq:1D_Burgers} \\
& u(0, t)=u(1, t)=0, \nonumber 
\end{align}
with Dirichlet boundary conditions, where $u_0 \in L_{0}^2((0, 1); \mathbb{R})$ is the initial condition and $\varepsilon \in \mathbb{R}$ is the viscosity coefficient. We aim to learn the operator mapping the initial condition to the steady-state solution, $G^{\dagger}: L_{0}^2((0, 1); \mathbb{R}) \rightarrow H_{0}^r((0, 1); \mathbb{R})$, defined by $u_0 \mapsto u_{\infty}$ for any $r > 0$. As shown in Theorem 2.2 of \cite{KREISS1986161} and Theorems 2.5 and 2.7 of \cite{hao2019convergence}, for any $\varepsilon > 0$, the steady-state solution is independent of the initial condition, with a single shock occurring at $x_s = 0.5$. Here, we use DeepONet~\cite{lu2021deeponet} as the network architecture. 
The homotopy definition, similar to ~\cref{eq.hom_2D_AC}, can be found in \cref{Ap:operator}. The results can be found in \cref{fig:Burgers_result} and \cref{tab:loss_burgers}. Experimental results show that the homotopy dynamics strategy performs well in the operator learning setting as well.


\begin{table}[htbp!]
    \caption{Comparison of loss between classical training and homotopy dynamics for different values of $\varepsilon$ in the Burgers equation, along with the MSE distance to the ground truth shock location, $x_s$.}
    \vskip 0.15in
    \centering
    \tiny
    \begin{tabular}{|c|c|c|c|c|} 
    \hline  
    $ $ & $\varepsilon = 0.5$ & $\varepsilon = 0.1$ & $\varepsilon = 0.05$ \\ \hline 
    Homotopy Loss $L_H$                &  7.55e-7     & 3.40e-7     & 7.77e-7       \\ \hline 
    L2RE                      & 1.50e-3     & 7.00e-4     & 2.52e-2       \\ \hline
        MSE Distance $x_s$                      & 1.75e-8     & 9.14e-8      & 1.2e-3      \\ \hline
    \end{tabular}
    % On convection, \al{} provides 14.2$\times$ and 1.97$\times$ improvement over Adam or \lbfgs{} on L2RE. 
    % On reaction, \al{} provides 1.10$\times$ and 1.99$\times$ improvement over Adam or \lbfgs{} on L2RE.
    % On wave, \al{} provides 6.32$\times$ and 6.07$\times$ improvement over Adam or \lbfgs{} on L2RE.}
    \label{tab:loss_burgers}
\end{table}



% \begin{itemize}
%     \item Relate the curvature in the problem to the differential operator. Use this to demonstrate why the problem is ill-conditioned
%     \item Give an argument for why using Adam + L-BFGS is better than just using L-BFGS outright. The idea is that Adam lowers the errors to the point where the rest of the optimization becomes convex \ldots
%     \item Show why we need second-order methods. We would like to prove that once we are close to the optimum, second-order methods will give condition-number free linear convergence. Specialize this to the Gauss-Newton setting, with the randomized low-rank approximation.
%     % \item Show that it is not possible to get superlinear convergence under the interpolation assumption with an overparameterized neural network. This should be true b/c the Hessian at the optimum will have rank $\min(n, d)$, and when $d > n$, this guarantees that we cannot have strong convexity.
% \end{itemize}
%\chapter{Implementation}{\label{ch:implementation}}
In this chapter, we present the implementation of the final product. We start by discussing how the four steps introduced in \hyperref[ch:high_level_approach]{chapter \ref*{ch:high_level_approach}} are integrated. We then outline the main system components of our score follower, presenting each as an independent, self-contained module. We then combine this into an overall system architecture and finally introduce the open-source score renderer used to display the score and evaluate the score follower.       

% \section{Aims and Requirements}
% The overall aim of the score follower was to 


\section{Score Follower Framework Details}
Our score follower conforms to the high-level framework presented in \hyperref[section:score_follower_framework]{section \ref*{section:score_follower_framework}}. In step 1, two score features are extracted from a MIDI file (see \hyperref[subsection:midi]{subsection \ref*{subsection:midi}}), namely MIDI note numbers\footnote{\href{https://inspiredacoustics.com/en/MIDI_note_numbers_and_center_frequencies}{https://inspiredacoustics.com/en/MIDI\_note\_numbers\_and\_center\_frequencies}} (corresponding to pitch) and note onsets (corresponding to duration). In step 2, the audio is streamed (whether from a file or into a microphone) and audioframes that exceed some predefined energy threshold are extracted. Here, audioframes are groups of contiguous audio samples, whose length can be specified by the argument \verb|frame_length|, usually between 800 and 2000 samples. The period between consecutive audioframes can also be defined by the argument \verb|hop_length|, typically between 2000 and 5000 audio samples. In step 3, score following is performed via a `Windowed' Viterbi algorithm (see  \hyperref[subsection:adjusting_viterbi]{subsection \ref*{subsection:adjusting_viterbi}}) which uses the Gaussian Process (GP) log marginal likelihoods (LMLs) for emission probabilities (see \hyperref[section:state_duration_model]{section \ref*{section:state_duration_model}}) and a state duration model for transition probabilities (see \hyperref[section:state_duration_model]{section \ref*{section:state_duration_model}}). Finally, in step 4 we render our results using an adapted version of the open source user interface, \textit{Flippy Qualitative Testbench}.

\section{Following Modes}
Two modes are available to the user: Pre-recorded Mode and Live Mode. The former requires a pre-recorded $\verb|.wav|$ file, whereas the latter takes an input stream of audio via the device's microphone. Note that both modes are still forms of score following, as opposed to score alignment, since in each mode we receive audioframes at the sampling rate, not all at once.\\

Live Mode offers a practical example of a score follower, displaying a score and position marker which a musician can read off while playing. However, this mode is not suitable for evaluation because the input and results cannot be easily replicated. Even ignoring repeatability, Live Mode is not suitable for one-off testing since a musician using this application may be influenced by the movement of the marker. For instance, the performer may speed up if the score follower `gets ahead' or slow down if the position marker lags or `gets lost'. To avoid this, we use Pre-recorded Mode when evaluating the performance of our score follower. Furthermore, Pre-recorded Mode offers the advantage of testing away from the music room, providing the opportunity to evaluate a variety of recordings available online. 

\section{System Architecture}
Our guiding principle for development was to build modular code in order to create a streamlined system where each component performs a specific task independently. This structure facilitates easy testing and debugging. \hyperref[fig:black_box]{Figure \ref*{fig:black_box}} presents a high-level architecture diagram, where each black box abstracts a key component of the score follower. When operating in Pre-recorded Mode, there is the option to stream the recording during run-time, which outputs to the device's speakers (as indicated by the dashed lines).

\begin{figure}[H]
    \centering
    \includegraphics[width=1\textwidth]{figs/Part_4_Implementation_And_Results/black_box.png}
    \caption{Abstracted system architecture diagram displaying inputs in grey, the 4 main components of the score follower in black and the outputs in green.}
    \label{fig:black_box}
\end{figure}

\subsection{Score Preprocessor}
The architecture for the Score Preprocesor is given in \hyperref[fig:score_preprocessor]{Figure \ref*{fig:score_preprocessor}}. First, MIDI note number and note onset times are extracted from each MIDI event. Simultaneous notes can be gathered into states and returned as a time-sorted list of lists called \verb|score|, where each element of the outer list is a list of simultaneous note onsets. Similarly, a list of note durations calculated as the time difference between consecutive states is returned as \verb|times_to_next|. Finally, all covariance matrices are precalculated and stored in a dictionary, where the key of the dictionary is determined by the notes present. This is because the distribution of notes and chords in a score is not random: notes tend to belong to a home \gls{key} and melodies tend to be repeated or related (similar to subject fields in speech processing). Therefore, states tend to be reused often, allowing us to achieve amortised time and space savings (by avoiding repeated calculation of the same covariance matrices). 

\begin{figure}[H]
    \centering
    \includegraphics[width=1\textwidth]{figs/Part_3_Implementation/Stage_2_Alignment/score_preprocessor.png}
    \caption{System architecture diagram representing the Score Preprocessor with inputs in grey, processes in blue and objects in yellow.}
    \label{fig:score_preprocessor}
\end{figure}


\subsection{Audio Preprocessor}
The architecture for the Audio Preprocessor is illustrated in \hyperref[fig:audio_preprocessor]{Figure \ref*{fig:audio_preprocessor}}. In Pre-recorded Mode, the Slicer receives a $\verb|.wav|$ file and returns audioframes separated by the \verb|hop_length|. These audioframes are periodically added to a multiprocessing queue, \verb|AudioFramesQueue|, to simulate real-time score following. In Live Mode, we use the python module \verb|sounddevice| to receive a stream of audio, using a periodic callback function to place audioframes on \verb|AudioFramesQueue|. 

\begin{figure}[H]
    \centering
    \includegraphics[width=1\textwidth]{figs/Part_4_Implementation_And_Results/audio_preprocessor.png}
    \caption{System architecture diagram representing the Audio Preprocessor with inputs in grey, processes in blue and objects in yellow.}
    \label{fig:audio_preprocessor}
\end{figure}

\subsection{Follower and Backend}
The joint Follower and Backend architecture diagram is shown in \hyperref[fig:follwer_and_backend]{Figure \ref*{fig:follwer_and_backend}}. The Viterbi Follower (detailed in \hyperref[subsection:adjusting_viterbi]{section \ref*{subsection:adjusting_viterbi}}) calculates the most probable state in the score, given audioframes continually taken from \verb|AudioFramesQueue|. These states are placed on another multiprocessing queue, the \verb|FollowerOutputQueue|, for the Backend to process and send. This prevents any bottle-necking occurring at the Follower stage. The Backend first sets up a UDP connection and then reads off values from \verb|FollowerOutputQueue|, sending them via UDP packets to the score renderer.

\begin{figure}[H]
    \centering
    \includegraphics[width=1\textwidth]{figs/Part_4_Implementation_And_Results/follower_and_backend.png}
    \caption{System architecture diagram representing the Follower and Backend processes with processes in blue, objects in yellow and outputs in green.}
    \label{fig:follwer_and_backend}
\end{figure}

\subsection{Player}
In Pre-recorded Mode, the Player sets up a new process and begins streaming the recording once the Follower process begins. This provides a baseline for testing purposes, as a trained musician can observe the score position marker and judge how well it matches the music. 

\subsection{Overall System Architecture}
The overall system architecture is presented in \hyperref[fig:overall_system_architecture]{Figure \ref*{fig:overall_system_architecture}}. Since the Follower runs a real-time, time sensitive process, parallelism is employed to reduce the total system latency. We use two \verb|multiprocessing| queues\footnote{\href{https://docs.python.org/3/library/multiprocessing.html}{https://docs.python.org/3/library/multiprocessing.html}} to avoid bottle-necking, which allows us to run 4 concurrent processes (Audio Preprocessor, Follower, Backend, and Audio Player). Hence, this architecture allows the components to run independently of one another to avoid blocking. Furthermore, this allows the system to take advantage of the multiple cores and high computational power offered by most modern machines.  

\begin{figure}[H]
    \centering
    \includegraphics[width=1\textwidth]{figs/Part_4_Implementation_And_Results/overall_score_follower_2.png}
    \caption{System architecture diagram representing the overall score follower running in Pre-recorded mode, with inputs in grey, processes in blue, objects in yellow and outputs in green.}
    \label{fig:overall_system_architecture}
\end{figure}


\section{Rendering Results}{\label{section:renderer}}
To visualise the results of our score follower, we adapted an open source tool for testing different score followers.\footnote{\href{https://github.com/flippy-fyp/flippy-qualitative-testbench/blob/main/README.md}{https://github.com/flippy-fyp/flippy-qualitative-testbench/blob/main/README.md}} \hyperref[fig:flippy_example]{Figure \ref*{fig:flippy_example}} shows the user interface of the score position renderer, where the green bar indicates score position. 

\begin{figure}[H]
    \centering
    \includegraphics{figs/Part_4_Implementation_And_Results/example_renderer.png}
    \caption{Screenshot of the score renderer user interface which displays a score (here we show a keyboard arrangement of \textit{O Haupt voll Blut und Wunden} by Bach). The green marker represents the score follower position.}
    \label{fig:flippy_example}
\end{figure}



\documentclass{article}
\usepackage{graphicx} % Required for inserting images

\title{XFlow}
\author{tailin Wu}
\date{January 2025}

\begin{document}

\maketitle

\section{Introduction}

\end{document}

\section{Conclusion}
In this work, we propose a simple yet effective approach, called SMILE, for graph few-shot learning with fewer tasks. Specifically, we introduce a novel dual-level mixup strategy, including within-task and across-task mixup, for enriching the diversity of nodes within each task and the diversity of tasks. Also, we incorporate the degree-based prior information to learn expressive node embeddings. Theoretically, we prove that SMILE effectively enhances the model's generalization performance. Empirically, we conduct extensive experiments on multiple benchmarks and the results suggest that SMILE significantly outperforms other baselines, including both in-domain and cross-domain few-shot settings.
%\section*{\textbf{Acknowledgments}}
This publication is based on work supported by the King Abdullah University of Science and Technology (KAUST). The authors thank the DeepWave sponsors fort heir support. This work utilized the resources of the Supercomputing Laboratory at King Abdullah University of Science and Technology (KAUST) in Thuwal, Saudi Arabia.
\subsection{Lloyd-Max Algorithm}
\label{subsec:Lloyd-Max}
For a given quantization bitwidth $B$ and an operand $\bm{X}$, the Lloyd-Max algorithm finds $2^B$ quantization levels $\{\hat{x}_i\}_{i=1}^{2^B}$ such that quantizing $\bm{X}$ by rounding each scalar in $\bm{X}$ to the nearest quantization level minimizes the quantization MSE. 

The algorithm starts with an initial guess of quantization levels and then iteratively computes quantization thresholds $\{\tau_i\}_{i=1}^{2^B-1}$ and updates quantization levels $\{\hat{x}_i\}_{i=1}^{2^B}$. Specifically, at iteration $n$, thresholds are set to the midpoints of the previous iteration's levels:
\begin{align*}
    \tau_i^{(n)}=\frac{\hat{x}_i^{(n-1)}+\hat{x}_{i+1}^{(n-1)}}2 \text{ for } i=1\ldots 2^B-1
\end{align*}
Subsequently, the quantization levels are re-computed as conditional means of the data regions defined by the new thresholds:
\begin{align*}
    \hat{x}_i^{(n)}=\mathbb{E}\left[ \bm{X} \big| \bm{X}\in [\tau_{i-1}^{(n)},\tau_i^{(n)}] \right] \text{ for } i=1\ldots 2^B
\end{align*}
where to satisfy boundary conditions we have $\tau_0=-\infty$ and $\tau_{2^B}=\infty$. The algorithm iterates the above steps until convergence.

Figure \ref{fig:lm_quant} compares the quantization levels of a $7$-bit floating point (E3M3) quantizer (left) to a $7$-bit Lloyd-Max quantizer (right) when quantizing a layer of weights from the GPT3-126M model at a per-tensor granularity. As shown, the Lloyd-Max quantizer achieves substantially lower quantization MSE. Further, Table \ref{tab:FP7_vs_LM7} shows the superior perplexity achieved by Lloyd-Max quantizers for bitwidths of $7$, $6$ and $5$. The difference between the quantizers is clear at 5 bits, where per-tensor FP quantization incurs a drastic and unacceptable increase in perplexity, while Lloyd-Max quantization incurs a much smaller increase. Nevertheless, we note that even the optimal Lloyd-Max quantizer incurs a notable ($\sim 1.5$) increase in perplexity due to the coarse granularity of quantization. 

\begin{figure}[h]
  \centering
  \includegraphics[width=0.7\linewidth]{sections/figures/LM7_FP7.pdf}
  \caption{\small Quantization levels and the corresponding quantization MSE of Floating Point (left) vs Lloyd-Max (right) Quantizers for a layer of weights in the GPT3-126M model.}
  \label{fig:lm_quant}
\end{figure}

\begin{table}[h]\scriptsize
\begin{center}
\caption{\label{tab:FP7_vs_LM7} \small Comparing perplexity (lower is better) achieved by floating point quantizers and Lloyd-Max quantizers on a GPT3-126M model for the Wikitext-103 dataset.}
\begin{tabular}{c|cc|c}
\hline
 \multirow{2}{*}{\textbf{Bitwidth}} & \multicolumn{2}{|c|}{\textbf{Floating-Point Quantizer}} & \textbf{Lloyd-Max Quantizer} \\
 & Best Format & Wikitext-103 Perplexity & Wikitext-103 Perplexity \\
\hline
7 & E3M3 & 18.32 & 18.27 \\
6 & E3M2 & 19.07 & 18.51 \\
5 & E4M0 & 43.89 & 19.71 \\
\hline
\end{tabular}
\end{center}
\end{table}

\subsection{Proof of Local Optimality of LO-BCQ}
\label{subsec:lobcq_opt_proof}
For a given block $\bm{b}_j$, the quantization MSE during LO-BCQ can be empirically evaluated as $\frac{1}{L_b}\lVert \bm{b}_j- \bm{\hat{b}}_j\rVert^2_2$ where $\bm{\hat{b}}_j$ is computed from equation (\ref{eq:clustered_quantization_definition}) as $C_{f(\bm{b}_j)}(\bm{b}_j)$. Further, for a given block cluster $\mathcal{B}_i$, we compute the quantization MSE as $\frac{1}{|\mathcal{B}_{i}|}\sum_{\bm{b} \in \mathcal{B}_{i}} \frac{1}{L_b}\lVert \bm{b}- C_i^{(n)}(\bm{b})\rVert^2_2$. Therefore, at the end of iteration $n$, we evaluate the overall quantization MSE $J^{(n)}$ for a given operand $\bm{X}$ composed of $N_c$ block clusters as:
\begin{align*}
    \label{eq:mse_iter_n}
    J^{(n)} = \frac{1}{N_c} \sum_{i=1}^{N_c} \frac{1}{|\mathcal{B}_{i}^{(n)}|}\sum_{\bm{v} \in \mathcal{B}_{i}^{(n)}} \frac{1}{L_b}\lVert \bm{b}- B_i^{(n)}(\bm{b})\rVert^2_2
\end{align*}

At the end of iteration $n$, the codebooks are updated from $\mathcal{C}^{(n-1)}$ to $\mathcal{C}^{(n)}$. However, the mapping of a given vector $\bm{b}_j$ to quantizers $\mathcal{C}^{(n)}$ remains as  $f^{(n)}(\bm{b}_j)$. At the next iteration, during the vector clustering step, $f^{(n+1)}(\bm{b}_j)$ finds new mapping of $\bm{b}_j$ to updated codebooks $\mathcal{C}^{(n)}$ such that the quantization MSE over the candidate codebooks is minimized. Therefore, we obtain the following result for $\bm{b}_j$:
\begin{align*}
\frac{1}{L_b}\lVert \bm{b}_j - C_{f^{(n+1)}(\bm{b}_j)}^{(n)}(\bm{b}_j)\rVert^2_2 \le \frac{1}{L_b}\lVert \bm{b}_j - C_{f^{(n)}(\bm{b}_j)}^{(n)}(\bm{b}_j)\rVert^2_2
\end{align*}

That is, quantizing $\bm{b}_j$ at the end of the block clustering step of iteration $n+1$ results in lower quantization MSE compared to quantizing at the end of iteration $n$. Since this is true for all $\bm{b} \in \bm{X}$, we assert the following:
\begin{equation}
\begin{split}
\label{eq:mse_ineq_1}
    \tilde{J}^{(n+1)} &= \frac{1}{N_c} \sum_{i=1}^{N_c} \frac{1}{|\mathcal{B}_{i}^{(n+1)}|}\sum_{\bm{b} \in \mathcal{B}_{i}^{(n+1)}} \frac{1}{L_b}\lVert \bm{b} - C_i^{(n)}(b)\rVert^2_2 \le J^{(n)}
\end{split}
\end{equation}
where $\tilde{J}^{(n+1)}$ is the the quantization MSE after the vector clustering step at iteration $n+1$.

Next, during the codebook update step (\ref{eq:quantizers_update}) at iteration $n+1$, the per-cluster codebooks $\mathcal{C}^{(n)}$ are updated to $\mathcal{C}^{(n+1)}$ by invoking the Lloyd-Max algorithm \citep{Lloyd}. We know that for any given value distribution, the Lloyd-Max algorithm minimizes the quantization MSE. Therefore, for a given vector cluster $\mathcal{B}_i$ we obtain the following result:

\begin{equation}
    \frac{1}{|\mathcal{B}_{i}^{(n+1)}|}\sum_{\bm{b} \in \mathcal{B}_{i}^{(n+1)}} \frac{1}{L_b}\lVert \bm{b}- C_i^{(n+1)}(\bm{b})\rVert^2_2 \le \frac{1}{|\mathcal{B}_{i}^{(n+1)}|}\sum_{\bm{b} \in \mathcal{B}_{i}^{(n+1)}} \frac{1}{L_b}\lVert \bm{b}- C_i^{(n)}(\bm{b})\rVert^2_2
\end{equation}

The above equation states that quantizing the given block cluster $\mathcal{B}_i$ after updating the associated codebook from $C_i^{(n)}$ to $C_i^{(n+1)}$ results in lower quantization MSE. Since this is true for all the block clusters, we derive the following result: 
\begin{equation}
\begin{split}
\label{eq:mse_ineq_2}
     J^{(n+1)} &= \frac{1}{N_c} \sum_{i=1}^{N_c} \frac{1}{|\mathcal{B}_{i}^{(n+1)}|}\sum_{\bm{b} \in \mathcal{B}_{i}^{(n+1)}} \frac{1}{L_b}\lVert \bm{b}- C_i^{(n+1)}(\bm{b})\rVert^2_2  \le \tilde{J}^{(n+1)}   
\end{split}
\end{equation}

Following (\ref{eq:mse_ineq_1}) and (\ref{eq:mse_ineq_2}), we find that the quantization MSE is non-increasing for each iteration, that is, $J^{(1)} \ge J^{(2)} \ge J^{(3)} \ge \ldots \ge J^{(M)}$ where $M$ is the maximum number of iterations. 
%Therefore, we can say that if the algorithm converges, then it must be that it has converged to a local minimum. 
\hfill $\blacksquare$


\begin{figure}
    \begin{center}
    \includegraphics[width=0.5\textwidth]{sections//figures/mse_vs_iter.pdf}
    \end{center}
    \caption{\small NMSE vs iterations during LO-BCQ compared to other block quantization proposals}
    \label{fig:nmse_vs_iter}
\end{figure}

Figure \ref{fig:nmse_vs_iter} shows the empirical convergence of LO-BCQ across several block lengths and number of codebooks. Also, the MSE achieved by LO-BCQ is compared to baselines such as MXFP and VSQ. As shown, LO-BCQ converges to a lower MSE than the baselines. Further, we achieve better convergence for larger number of codebooks ($N_c$) and for a smaller block length ($L_b$), both of which increase the bitwidth of BCQ (see Eq \ref{eq:bitwidth_bcq}).


\subsection{Additional Accuracy Results}
%Table \ref{tab:lobcq_config} lists the various LOBCQ configurations and their corresponding bitwidths.
\begin{table}
\setlength{\tabcolsep}{4.75pt}
\begin{center}
\caption{\label{tab:lobcq_config} Various LO-BCQ configurations and their bitwidths.}
\begin{tabular}{|c||c|c|c|c||c|c||c|} 
\hline
 & \multicolumn{4}{|c||}{$L_b=8$} & \multicolumn{2}{|c||}{$L_b=4$} & $L_b=2$ \\
 \hline
 \backslashbox{$L_A$\kern-1em}{\kern-1em$N_c$} & 2 & 4 & 8 & 16 & 2 & 4 & 2 \\
 \hline
 64 & 4.25 & 4.375 & 4.5 & 4.625 & 4.375 & 4.625 & 4.625\\
 \hline
 32 & 4.375 & 4.5 & 4.625& 4.75 & 4.5 & 4.75 & 4.75 \\
 \hline
 16 & 4.625 & 4.75& 4.875 & 5 & 4.75 & 5 & 5 \\
 \hline
\end{tabular}
\end{center}
\end{table}

%\subsection{Perplexity achieved by various LO-BCQ configurations on Wikitext-103 dataset}

\begin{table} \centering
\begin{tabular}{|c||c|c|c|c||c|c||c|} 
\hline
 $L_b \rightarrow$& \multicolumn{4}{c||}{8} & \multicolumn{2}{c||}{4} & 2\\
 \hline
 \backslashbox{$L_A$\kern-1em}{\kern-1em$N_c$} & 2 & 4 & 8 & 16 & 2 & 4 & 2  \\
 %$N_c \rightarrow$ & 2 & 4 & 8 & 16 & 2 & 4 & 2 \\
 \hline
 \hline
 \multicolumn{8}{c}{GPT3-1.3B (FP32 PPL = 9.98)} \\ 
 \hline
 \hline
 64 & 10.40 & 10.23 & 10.17 & 10.15 &  10.28 & 10.18 & 10.19 \\
 \hline
 32 & 10.25 & 10.20 & 10.15 & 10.12 &  10.23 & 10.17 & 10.17 \\
 \hline
 16 & 10.22 & 10.16 & 10.10 & 10.09 &  10.21 & 10.14 & 10.16 \\
 \hline
  \hline
 \multicolumn{8}{c}{GPT3-8B (FP32 PPL = 7.38)} \\ 
 \hline
 \hline
 64 & 7.61 & 7.52 & 7.48 &  7.47 &  7.55 &  7.49 & 7.50 \\
 \hline
 32 & 7.52 & 7.50 & 7.46 &  7.45 &  7.52 &  7.48 & 7.48  \\
 \hline
 16 & 7.51 & 7.48 & 7.44 &  7.44 &  7.51 &  7.49 & 7.47  \\
 \hline
\end{tabular}
\caption{\label{tab:ppl_gpt3_abalation} Wikitext-103 perplexity across GPT3-1.3B and 8B models.}
\end{table}

\begin{table} \centering
\begin{tabular}{|c||c|c|c|c||} 
\hline
 $L_b \rightarrow$& \multicolumn{4}{c||}{8}\\
 \hline
 \backslashbox{$L_A$\kern-1em}{\kern-1em$N_c$} & 2 & 4 & 8 & 16 \\
 %$N_c \rightarrow$ & 2 & 4 & 8 & 16 & 2 & 4 & 2 \\
 \hline
 \hline
 \multicolumn{5}{|c|}{Llama2-7B (FP32 PPL = 5.06)} \\ 
 \hline
 \hline
 64 & 5.31 & 5.26 & 5.19 & 5.18  \\
 \hline
 32 & 5.23 & 5.25 & 5.18 & 5.15  \\
 \hline
 16 & 5.23 & 5.19 & 5.16 & 5.14  \\
 \hline
 \multicolumn{5}{|c|}{Nemotron4-15B (FP32 PPL = 5.87)} \\ 
 \hline
 \hline
 64  & 6.3 & 6.20 & 6.13 & 6.08  \\
 \hline
 32  & 6.24 & 6.12 & 6.07 & 6.03  \\
 \hline
 16  & 6.12 & 6.14 & 6.04 & 6.02  \\
 \hline
 \multicolumn{5}{|c|}{Nemotron4-340B (FP32 PPL = 3.48)} \\ 
 \hline
 \hline
 64 & 3.67 & 3.62 & 3.60 & 3.59 \\
 \hline
 32 & 3.63 & 3.61 & 3.59 & 3.56 \\
 \hline
 16 & 3.61 & 3.58 & 3.57 & 3.55 \\
 \hline
\end{tabular}
\caption{\label{tab:ppl_llama7B_nemo15B} Wikitext-103 perplexity compared to FP32 baseline in Llama2-7B and Nemotron4-15B, 340B models}
\end{table}

%\subsection{Perplexity achieved by various LO-BCQ configurations on MMLU dataset}


\begin{table} \centering
\begin{tabular}{|c||c|c|c|c||c|c|c|c|} 
\hline
 $L_b \rightarrow$& \multicolumn{4}{c||}{8} & \multicolumn{4}{c||}{8}\\
 \hline
 \backslashbox{$L_A$\kern-1em}{\kern-1em$N_c$} & 2 & 4 & 8 & 16 & 2 & 4 & 8 & 16  \\
 %$N_c \rightarrow$ & 2 & 4 & 8 & 16 & 2 & 4 & 2 \\
 \hline
 \hline
 \multicolumn{5}{|c|}{Llama2-7B (FP32 Accuracy = 45.8\%)} & \multicolumn{4}{|c|}{Llama2-70B (FP32 Accuracy = 69.12\%)} \\ 
 \hline
 \hline
 64 & 43.9 & 43.4 & 43.9 & 44.9 & 68.07 & 68.27 & 68.17 & 68.75 \\
 \hline
 32 & 44.5 & 43.8 & 44.9 & 44.5 & 68.37 & 68.51 & 68.35 & 68.27  \\
 \hline
 16 & 43.9 & 42.7 & 44.9 & 45 & 68.12 & 68.77 & 68.31 & 68.59  \\
 \hline
 \hline
 \multicolumn{5}{|c|}{GPT3-22B (FP32 Accuracy = 38.75\%)} & \multicolumn{4}{|c|}{Nemotron4-15B (FP32 Accuracy = 64.3\%)} \\ 
 \hline
 \hline
 64 & 36.71 & 38.85 & 38.13 & 38.92 & 63.17 & 62.36 & 63.72 & 64.09 \\
 \hline
 32 & 37.95 & 38.69 & 39.45 & 38.34 & 64.05 & 62.30 & 63.8 & 64.33  \\
 \hline
 16 & 38.88 & 38.80 & 38.31 & 38.92 & 63.22 & 63.51 & 63.93 & 64.43  \\
 \hline
\end{tabular}
\caption{\label{tab:mmlu_abalation} Accuracy on MMLU dataset across GPT3-22B, Llama2-7B, 70B and Nemotron4-15B models.}
\end{table}


%\subsection{Perplexity achieved by various LO-BCQ configurations on LM evaluation harness}

\begin{table} \centering
\begin{tabular}{|c||c|c|c|c||c|c|c|c|} 
\hline
 $L_b \rightarrow$& \multicolumn{4}{c||}{8} & \multicolumn{4}{c||}{8}\\
 \hline
 \backslashbox{$L_A$\kern-1em}{\kern-1em$N_c$} & 2 & 4 & 8 & 16 & 2 & 4 & 8 & 16  \\
 %$N_c \rightarrow$ & 2 & 4 & 8 & 16 & 2 & 4 & 2 \\
 \hline
 \hline
 \multicolumn{5}{|c|}{Race (FP32 Accuracy = 37.51\%)} & \multicolumn{4}{|c|}{Boolq (FP32 Accuracy = 64.62\%)} \\ 
 \hline
 \hline
 64 & 36.94 & 37.13 & 36.27 & 37.13 & 63.73 & 62.26 & 63.49 & 63.36 \\
 \hline
 32 & 37.03 & 36.36 & 36.08 & 37.03 & 62.54 & 63.51 & 63.49 & 63.55  \\
 \hline
 16 & 37.03 & 37.03 & 36.46 & 37.03 & 61.1 & 63.79 & 63.58 & 63.33  \\
 \hline
 \hline
 \multicolumn{5}{|c|}{Winogrande (FP32 Accuracy = 58.01\%)} & \multicolumn{4}{|c|}{Piqa (FP32 Accuracy = 74.21\%)} \\ 
 \hline
 \hline
 64 & 58.17 & 57.22 & 57.85 & 58.33 & 73.01 & 73.07 & 73.07 & 72.80 \\
 \hline
 32 & 59.12 & 58.09 & 57.85 & 58.41 & 73.01 & 73.94 & 72.74 & 73.18  \\
 \hline
 16 & 57.93 & 58.88 & 57.93 & 58.56 & 73.94 & 72.80 & 73.01 & 73.94  \\
 \hline
\end{tabular}
\caption{\label{tab:mmlu_abalation} Accuracy on LM evaluation harness tasks on GPT3-1.3B model.}
\end{table}

\begin{table} \centering
\begin{tabular}{|c||c|c|c|c||c|c|c|c|} 
\hline
 $L_b \rightarrow$& \multicolumn{4}{c||}{8} & \multicolumn{4}{c||}{8}\\
 \hline
 \backslashbox{$L_A$\kern-1em}{\kern-1em$N_c$} & 2 & 4 & 8 & 16 & 2 & 4 & 8 & 16  \\
 %$N_c \rightarrow$ & 2 & 4 & 8 & 16 & 2 & 4 & 2 \\
 \hline
 \hline
 \multicolumn{5}{|c|}{Race (FP32 Accuracy = 41.34\%)} & \multicolumn{4}{|c|}{Boolq (FP32 Accuracy = 68.32\%)} \\ 
 \hline
 \hline
 64 & 40.48 & 40.10 & 39.43 & 39.90 & 69.20 & 68.41 & 69.45 & 68.56 \\
 \hline
 32 & 39.52 & 39.52 & 40.77 & 39.62 & 68.32 & 67.43 & 68.17 & 69.30  \\
 \hline
 16 & 39.81 & 39.71 & 39.90 & 40.38 & 68.10 & 66.33 & 69.51 & 69.42  \\
 \hline
 \hline
 \multicolumn{5}{|c|}{Winogrande (FP32 Accuracy = 67.88\%)} & \multicolumn{4}{|c|}{Piqa (FP32 Accuracy = 78.78\%)} \\ 
 \hline
 \hline
 64 & 66.85 & 66.61 & 67.72 & 67.88 & 77.31 & 77.42 & 77.75 & 77.64 \\
 \hline
 32 & 67.25 & 67.72 & 67.72 & 67.00 & 77.31 & 77.04 & 77.80 & 77.37  \\
 \hline
 16 & 68.11 & 68.90 & 67.88 & 67.48 & 77.37 & 78.13 & 78.13 & 77.69  \\
 \hline
\end{tabular}
\caption{\label{tab:mmlu_abalation} Accuracy on LM evaluation harness tasks on GPT3-8B model.}
\end{table}

\begin{table} \centering
\begin{tabular}{|c||c|c|c|c||c|c|c|c|} 
\hline
 $L_b \rightarrow$& \multicolumn{4}{c||}{8} & \multicolumn{4}{c||}{8}\\
 \hline
 \backslashbox{$L_A$\kern-1em}{\kern-1em$N_c$} & 2 & 4 & 8 & 16 & 2 & 4 & 8 & 16  \\
 %$N_c \rightarrow$ & 2 & 4 & 8 & 16 & 2 & 4 & 2 \\
 \hline
 \hline
 \multicolumn{5}{|c|}{Race (FP32 Accuracy = 40.67\%)} & \multicolumn{4}{|c|}{Boolq (FP32 Accuracy = 76.54\%)} \\ 
 \hline
 \hline
 64 & 40.48 & 40.10 & 39.43 & 39.90 & 75.41 & 75.11 & 77.09 & 75.66 \\
 \hline
 32 & 39.52 & 39.52 & 40.77 & 39.62 & 76.02 & 76.02 & 75.96 & 75.35  \\
 \hline
 16 & 39.81 & 39.71 & 39.90 & 40.38 & 75.05 & 73.82 & 75.72 & 76.09  \\
 \hline
 \hline
 \multicolumn{5}{|c|}{Winogrande (FP32 Accuracy = 70.64\%)} & \multicolumn{4}{|c|}{Piqa (FP32 Accuracy = 79.16\%)} \\ 
 \hline
 \hline
 64 & 69.14 & 70.17 & 70.17 & 70.56 & 78.24 & 79.00 & 78.62 & 78.73 \\
 \hline
 32 & 70.96 & 69.69 & 71.27 & 69.30 & 78.56 & 79.49 & 79.16 & 78.89  \\
 \hline
 16 & 71.03 & 69.53 & 69.69 & 70.40 & 78.13 & 79.16 & 79.00 & 79.00  \\
 \hline
\end{tabular}
\caption{\label{tab:mmlu_abalation} Accuracy on LM evaluation harness tasks on GPT3-22B model.}
\end{table}

\begin{table} \centering
\begin{tabular}{|c||c|c|c|c||c|c|c|c|} 
\hline
 $L_b \rightarrow$& \multicolumn{4}{c||}{8} & \multicolumn{4}{c||}{8}\\
 \hline
 \backslashbox{$L_A$\kern-1em}{\kern-1em$N_c$} & 2 & 4 & 8 & 16 & 2 & 4 & 8 & 16  \\
 %$N_c \rightarrow$ & 2 & 4 & 8 & 16 & 2 & 4 & 2 \\
 \hline
 \hline
 \multicolumn{5}{|c|}{Race (FP32 Accuracy = 44.4\%)} & \multicolumn{4}{|c|}{Boolq (FP32 Accuracy = 79.29\%)} \\ 
 \hline
 \hline
 64 & 42.49 & 42.51 & 42.58 & 43.45 & 77.58 & 77.37 & 77.43 & 78.1 \\
 \hline
 32 & 43.35 & 42.49 & 43.64 & 43.73 & 77.86 & 75.32 & 77.28 & 77.86  \\
 \hline
 16 & 44.21 & 44.21 & 43.64 & 42.97 & 78.65 & 77 & 76.94 & 77.98  \\
 \hline
 \hline
 \multicolumn{5}{|c|}{Winogrande (FP32 Accuracy = 69.38\%)} & \multicolumn{4}{|c|}{Piqa (FP32 Accuracy = 78.07\%)} \\ 
 \hline
 \hline
 64 & 68.9 & 68.43 & 69.77 & 68.19 & 77.09 & 76.82 & 77.09 & 77.86 \\
 \hline
 32 & 69.38 & 68.51 & 68.82 & 68.90 & 78.07 & 76.71 & 78.07 & 77.86  \\
 \hline
 16 & 69.53 & 67.09 & 69.38 & 68.90 & 77.37 & 77.8 & 77.91 & 77.69  \\
 \hline
\end{tabular}
\caption{\label{tab:mmlu_abalation} Accuracy on LM evaluation harness tasks on Llama2-7B model.}
\end{table}

\begin{table} \centering
\begin{tabular}{|c||c|c|c|c||c|c|c|c|} 
\hline
 $L_b \rightarrow$& \multicolumn{4}{c||}{8} & \multicolumn{4}{c||}{8}\\
 \hline
 \backslashbox{$L_A$\kern-1em}{\kern-1em$N_c$} & 2 & 4 & 8 & 16 & 2 & 4 & 8 & 16  \\
 %$N_c \rightarrow$ & 2 & 4 & 8 & 16 & 2 & 4 & 2 \\
 \hline
 \hline
 \multicolumn{5}{|c|}{Race (FP32 Accuracy = 48.8\%)} & \multicolumn{4}{|c|}{Boolq (FP32 Accuracy = 85.23\%)} \\ 
 \hline
 \hline
 64 & 49.00 & 49.00 & 49.28 & 48.71 & 82.82 & 84.28 & 84.03 & 84.25 \\
 \hline
 32 & 49.57 & 48.52 & 48.33 & 49.28 & 83.85 & 84.46 & 84.31 & 84.93  \\
 \hline
 16 & 49.85 & 49.09 & 49.28 & 48.99 & 85.11 & 84.46 & 84.61 & 83.94  \\
 \hline
 \hline
 \multicolumn{5}{|c|}{Winogrande (FP32 Accuracy = 79.95\%)} & \multicolumn{4}{|c|}{Piqa (FP32 Accuracy = 81.56\%)} \\ 
 \hline
 \hline
 64 & 78.77 & 78.45 & 78.37 & 79.16 & 81.45 & 80.69 & 81.45 & 81.5 \\
 \hline
 32 & 78.45 & 79.01 & 78.69 & 80.66 & 81.56 & 80.58 & 81.18 & 81.34  \\
 \hline
 16 & 79.95 & 79.56 & 79.79 & 79.72 & 81.28 & 81.66 & 81.28 & 80.96  \\
 \hline
\end{tabular}
\caption{\label{tab:mmlu_abalation} Accuracy on LM evaluation harness tasks on Llama2-70B model.}
\end{table}

%\section{MSE Studies}
%\textcolor{red}{TODO}


\subsection{Number Formats and Quantization Method}
\label{subsec:numFormats_quantMethod}
\subsubsection{Integer Format}
An $n$-bit signed integer (INT) is typically represented with a 2s-complement format \citep{yao2022zeroquant,xiao2023smoothquant,dai2021vsq}, where the most significant bit denotes the sign.

\subsubsection{Floating Point Format}
An $n$-bit signed floating point (FP) number $x$ comprises of a 1-bit sign ($x_{\mathrm{sign}}$), $B_m$-bit mantissa ($x_{\mathrm{mant}}$) and $B_e$-bit exponent ($x_{\mathrm{exp}}$) such that $B_m+B_e=n-1$. The associated constant exponent bias ($E_{\mathrm{bias}}$) is computed as $(2^{{B_e}-1}-1)$. We denote this format as $E_{B_e}M_{B_m}$.  

\subsubsection{Quantization Scheme}
\label{subsec:quant_method}
A quantization scheme dictates how a given unquantized tensor is converted to its quantized representation. We consider FP formats for the purpose of illustration. Given an unquantized tensor $\bm{X}$ and an FP format $E_{B_e}M_{B_m}$, we first, we compute the quantization scale factor $s_X$ that maps the maximum absolute value of $\bm{X}$ to the maximum quantization level of the $E_{B_e}M_{B_m}$ format as follows:
\begin{align}
\label{eq:sf}
    s_X = \frac{\mathrm{max}(|\bm{X}|)}{\mathrm{max}(E_{B_e}M_{B_m})}
\end{align}
In the above equation, $|\cdot|$ denotes the absolute value function.

Next, we scale $\bm{X}$ by $s_X$ and quantize it to $\hat{\bm{X}}$ by rounding it to the nearest quantization level of $E_{B_e}M_{B_m}$ as:

\begin{align}
\label{eq:tensor_quant}
    \hat{\bm{X}} = \text{round-to-nearest}\left(\frac{\bm{X}}{s_X}, E_{B_e}M_{B_m}\right)
\end{align}

We perform dynamic max-scaled quantization \citep{wu2020integer}, where the scale factor $s$ for activations is dynamically computed during runtime.

\subsection{Vector Scaled Quantization}
\begin{wrapfigure}{r}{0.35\linewidth}
  \centering
  \includegraphics[width=\linewidth]{sections/figures/vsquant.jpg}
  \caption{\small Vectorwise decomposition for per-vector scaled quantization (VSQ \citep{dai2021vsq}).}
  \label{fig:vsquant}
\end{wrapfigure}
During VSQ \citep{dai2021vsq}, the operand tensors are decomposed into 1D vectors in a hardware friendly manner as shown in Figure \ref{fig:vsquant}. Since the decomposed tensors are used as operands in matrix multiplications during inference, it is beneficial to perform this decomposition along the reduction dimension of the multiplication. The vectorwise quantization is performed similar to tensorwise quantization described in Equations \ref{eq:sf} and \ref{eq:tensor_quant}, where a scale factor $s_v$ is required for each vector $\bm{v}$ that maps the maximum absolute value of that vector to the maximum quantization level. While smaller vector lengths can lead to larger accuracy gains, the associated memory and computational overheads due to the per-vector scale factors increases. To alleviate these overheads, VSQ \citep{dai2021vsq} proposed a second level quantization of the per-vector scale factors to unsigned integers, while MX \citep{rouhani2023shared} quantizes them to integer powers of 2 (denoted as $2^{INT}$).

\subsubsection{MX Format}
The MX format proposed in \citep{rouhani2023microscaling} introduces the concept of sub-block shifting. For every two scalar elements of $b$-bits each, there is a shared exponent bit. The value of this exponent bit is determined through an empirical analysis that targets minimizing quantization MSE. We note that the FP format $E_{1}M_{b}$ is strictly better than MX from an accuracy perspective since it allocates a dedicated exponent bit to each scalar as opposed to sharing it across two scalars. Therefore, we conservatively bound the accuracy of a $b+2$-bit signed MX format with that of a $E_{1}M_{b}$ format in our comparisons. For instance, we use E1M2 format as a proxy for MX4.

\begin{figure}
    \centering
    \includegraphics[width=1\linewidth]{sections//figures/BlockFormats.pdf}
    \caption{\small Comparing LO-BCQ to MX format.}
    \label{fig:block_formats}
\end{figure}

Figure \ref{fig:block_formats} compares our $4$-bit LO-BCQ block format to MX \citep{rouhani2023microscaling}. As shown, both LO-BCQ and MX decompose a given operand tensor into block arrays and each block array into blocks. Similar to MX, we find that per-block quantization ($L_b < L_A$) leads to better accuracy due to increased flexibility. While MX achieves this through per-block $1$-bit micro-scales, we associate a dedicated codebook to each block through a per-block codebook selector. Further, MX quantizes the per-block array scale-factor to E8M0 format without per-tensor scaling. In contrast during LO-BCQ, we find that per-tensor scaling combined with quantization of per-block array scale-factor to E4M3 format results in superior inference accuracy across models. 


%\newcommand{\BIBdecl}{\setlength{\itemsep}{0.25 em}}
\bibliographystyle{IEEEtran}
\bibliography{References}

%\section{Biography Section}

\vspace{11pt}

\begin{IEEEbiography}[{\includegraphics[width=1in,height=1.25in,clip,
    keepaspectratio]{figures/bioPics/clemot_bioPic_bw.jpg}}]{Mattéo Clémot}
    is a Ph.D.\ student at Université Lyon~1.
    He received an M.Sc. degree in Computer Science from the \'Ecole Normale Sup\'erieure de Lyon in 2022.
    His research lies in geometry and topological data analysis.
\end{IEEEbiography}

\begin{IEEEbiography}[{\includegraphics[width=1in,height=1.25in,clip,
    keepaspectratio]{figures/bioPics/digne_bioPic_bw.jpg}}]{Julie Digne}
    received the PhD degree in Applied Mathematics from the \'Ecole Normale Sup\'erieure de Cachan in 2010.
    She is a senior CNRS researcher at Université Lyon 1. She specializes in Geometry Processing, and in particular shape analysis, leveraging local differential properties, nonlocal analysis or statistical learning. She served as program chair of the Symposium on Geometry Processing in 2021.
\end{IEEEbiography}

\begin{IEEEbiography}[{\includegraphics[width=1in,height=1.25in,clip,
    keepaspectratio]{figures/bioPics/tierny_bioPic_bw.jpg}}]{Julien Tierny}
    received the Ph.D. degree in Computer Science from the University
    of Lille in 2008. He is a CNRS research director at
    Sorbonne University. Prior to his CNRS tenure, he held a
    Fulbright fellowship and was a post-doctoral
    researcher at the University of Utah.
    His research expertise lies in topological methods for data analysis
    and visualization.
    He is the founder and lead developer of the Topology ToolKit
    (TTK), an open source library for topological data analysis.
\end{IEEEbiography}
\end{document}
