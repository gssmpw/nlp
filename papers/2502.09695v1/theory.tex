

{\it Notation:}
The imaginary unit is $j$. %The matrix form of $j$ is defined as $\matr J = [0, -1;1, 0]$.
A vector of zeros is $0_n$. A vector of ones is $1_n$. Their subscripts are usually omitted when the dimension is clear. 
%The real and imaginary part of $A \in \mathbb{C}$ are $\Re\{A\}$ and $\Im\{A\}$.
%An identity matrix is $\matr I_n$. A zero matrix is $\matr 0_{n\times m}$. The subscript of these notations is sometimes omitted if the dimension is clear from the context.
%The $1$-torus with $2\pi$ equivalence is $\mathbb{T} = \mathbb{R}/2\pi\mathbb{Z}$~\cite{FB-ADL:04}. 
The transpose of a matrix is $\matr A^\tran$; the Hermitian transpose is $\matr A^\herm$. For a matrix $\matr A$ in an inner product space, the adjoint is $\matr A^*$. %; $\matr A$ is self-adjoint if $\matr A = \matr A^*$; and $\matr A$ is skew-adjoint if $\matr A = -\matr A^*$. %; and the pseudoinverse is $\matr A^\dagger$. 
The symbol $\col(\vect x_i)$ denotes a column vector that stacks the vectors $\vect x_i$ for $i = 1,\ldots,\, n$. 
%The inner product we throughout this paper assume is $\langle\vect y, \vect x\rangle = \Re\{\vect y^* \vect x\}$ for $\vect x, \vect y \in \mathbb{C}^n$. %, where $\mathbb{C}^n$ is a vector space over the field $\mathbb{R}$.
%This inner product can be characterized as
%\begin{equation} \label{E:complex_inner}
%    \langle \vect y, \vect x \rangle = \langle \Re\{\vect y\}, \Re\{\vect x\}\rangle + \langle \Im\{\vect y\}, \Im\{\vect x\}\rangle
%\end{equation}
%where the inner products in the RHS are real inner products.
%The adjoint of a linear mapping $\matr L: \mathbb{C}^n \to \mathbb{C}^n$ is $\matr L^*$.
%The Hermitian part of a matrix $\matr A$ is $\mathrm{He}\{\matr A\} = \frac{1}{2} (\matr A + \matr A^*)$.
%For a Hermitian matrix $\matr L$, $\matr L > 0$ denotes positive definiteness, i.e., $\langle \vect x, \matr L \vect x\rangle < 0$ for all $\vect x \in \mathbb{C}^n$.  
The complex vector space $\mathbb{C}^n$ we consider in this paper is equivalent to the $\mathbb{R}^{2n}$ considering the mapping 
\begin{equation*}
    \mathcal U: \vect x \mapsto \hat{\vect x} = \col(\big[\Re x_i,\, \Im x_i \big]^\tran). 
\end{equation*}
The derivative of real-valued functions defined on $\mathbb{C}^n$ are taken after mapping them to the equivalent real function. For $\vect f: \mathbb{C}^n \to \mathbb{C}^n$, the real Jacobian $D \vect f(\vect x)$ is the Jacobian of $\vect f(\mathcal U^{-1}(\hat{\vect x}))$. For $H: \mathbb{C}^n \to \mathbb{R}$, the real Hessian $D^2 H(\vect x)$ is the Hessian of $H(\mathcal U^{-1}(\hat{\vect x}))$. 
An exception is the complex gradient. For the inner product $\langle \vect z, \vect x \rangle = \Re\{\vect z^\herm \vect x\}$, the complex gradient is defined, for $H: \mathbb{C}^n \to \mathbb{R}$, as $\nabla H(\vect x) = 2 \col(\frac{\partial H}{\partial x_i^*})$, where $\frac{\partial H}{\partial x_i^*}$ is the Wirtinger derivative~\cite{remmert1991theory}. The complex gradient allows us to replace a direction derivative by an inner product: 
\begin{equation*}
    \frac{\partial}{\partial \vect x} H(\vect x)\cdot \vect v = \Re\{\nabla H(\vect x)^\herm \vect v\}.
\end{equation*}

%For $g: \mathbb{C}^n \to \mathbb{R}$, $\nabla g(\vect x) = 2 \col(\frac{\partial g}{\partial x_i^*})$ where $\frac{\partial g}{\partial x_i^*}$ is the Wirtinger derivative~\cite{remmert1991theory}.

%For a function $g: \mathbb{C}^n \to \mathbb{R}$ of complex variables, the complex gradient is $\nabla g(\vect x) = 2 \col(\frac{\partial g}{\partial x_i^*})$ where $\frac{\partial g}{\partial x_i^*}$ is the Wirtinger derivative~\cite{remmert1991theory}. The real Hessian of $g(\vect x)$ at $\vect x$, denoted by $D^2 g(\vect x)$, is the Hessian of $g(\mathcal U(\cdot)): \mathbb{R}^{2n} \to \mathbb{R}$ where $\mathcal U( \col(\big[ \Re x_i,\, \Im x_i \big]^\tran)) = \vect x$. When $D^2 g(\vect x)$ appears in the context of $\mathbb{C}^n$, it is understood as a linear operator instead of a matrix. %For a vector field $\vect f: \mathbb{C}^n \to \mathbb{C}^n$, the real Jacobian matrix, denoted by $D\vect f(\vect x)$, is defined similarly. Derivative with respective is denoted as either $\dot{A}$ or $\frac{d}{dt} A$.

%The same symbol $\mathcal I$ is used for the mapping of a complex matrix $\matr A \in \mathbb{C}^{n\times n}$ on $\mathbb{C}^n$ to the corresponding real matrix $\mathcal I(\matr A) \in \mathbb{R}^{2n\times 2n}$ on $\mathcal I(\mathbb{C}^n)$.
%For a vector field $\vect f: \vect x \mapsto T_{\vect x} \mathbb{C}^n$, the real Jacobian at $\vect x$ is $D \vect f(\vect x): \mathbb{R}^{2n} \to \mathbb{R}^{2n}$ because $T_{\vect x} \mathbb{C}^n \subset \mathbb{C}^n \simeq \mathbb{R}^{2n}$. 
%The angle of a complex vector $\vect x \in \mathbb{C}^n$ is $\arg(\vect x) = \col(\Im\{\log(x_i)\}) \in \mathbb{T}^n$. %The complement of a set $E$ is $E^c$.


\section{Background} \label{sec_background}

In this section, we discuss the background for the main theoretical development in Section~\ref{sec_contraction}: the contraction property implied from the port-Hamiltonian structure of a nonlinear time-varying (NLTV) system. We introduce this background knowledge by reviewing standard results on contraction of a general NLTV system, and contraction of a port-Hamiltonian NLTV system. We highlight the conservativeness of the standard results to motivate the more involved theoretical development in Section~\ref{sec_contraction}.

\subsection{Standard Contraction Definition}
Consider the NLTV system $\dot{\vect x} = \vect f(t, \vect x)$ for $t \in \mathbb{R}$ and $\vect x \in \mathbb{D}$, for an invariant set $\mathbb{D} \subset \mathbb{C}^n$ such that, at each $\vect x \in \mathbb{D}$, the tangent space\footnote{The tangent space $T_{\vect x} \mathbb{D}$ of a manifold $\mathbb{D}$ at $\vect x$ is the collection of tangent vectors of smooth curves in $\mathbb{D}$ that passes through $\vect x$; see~\cite[Def.~3.33]{FB-ADL:04}.} is $T_{\vect x} \mathbb{D} = \mathbb{C}^n$ with the inner product 
\begin{equation} \label{E:inner_product}
    \langle \vect y, \vect x \rangle = \Re\{\vect y^\herm \matr P \vect x\},
\end{equation}
where $\matr P \in \mathbb{C}^{n\times n}$ is Hermitian positive-definite. This inner product verifies the inner product axioms for the vector space $\mathbb{C}^n$ with the field $\mathbb{R}$. The associated norm is
\begin{equation*}
    \|\vect x\| = \sqrt{\langle \vect x, \vect x \rangle} = \sqrt{\Re\{\vect x^\herm \matr P \vect x\}}.
\end{equation*}
%This space is equivalent to $\mathbb{R}^{2n}$.
%By this definition, the inner product space $\mathbb{C}^n$ considered is equivalent to $\mathbb{R}^{2n}$ by observing that
%\begin{equation*}
%    \langle \vect y, \vect x \rangle_{\mathbb{C}^n} = \Re\{\vect y^\herm \vect x\} = \Re\{\vect y\}^\tran \Re\{\vect x\} + \Im\{\vect y\}^\tran \Im\{\vect x\}.
%\end{equation*}
We choose to work with the space $\mathbb{C}^n$ instead of the equivalent real space $\mathbb{R}^{2n}$ because complex variables readily represent amplitudes and angles in models of the power system. 

%Note that the default inner product definition is not the only valid one verifying the inner product axioms. Given a Hermitian positive-definite matrix $\matr P = \matr P^\herm$, we can verify that $\langle \vect y, \vect x \rangle = \Re\{\vect y^\herm \matr P \vect x\}$ is another valid inner product. 

Contraction of the NLTV system is a sufficient condition for stability that requires two solutions from any two initial conditions to converge exponentially to each other in terms of the norm of their difference in time. The contraction property is particularly useful in stability analysis in that the global condition can be checked locally with the Jacobian matrix $D \vect f(t, \vect x)$: the NLTV system is said to be infinitesimally contracting if, for some $c > 0$ (the contraction rate),
\begin{equation} \label{E:infinitesimal}
     \langle \delta, D \vect f(t, \vect x) \delta \rangle \leq -c \langle \delta, \delta \rangle,
\end{equation}
for all $t \in \mathbb{R}$ and $\delta \in T_{\vect x} \mathbb{D} = \mathbb{C}^n$.
It is not hard to prove that if the domain $\mathbb{D}$ is convex, then infinitesimal contraction implies global contraction~\cite{sontag2010contractive,simpson2014contraction}; that is, for every $\vect x_1, \vect x_2 \in \mathbb{D}$ and $t_0 \in \mathbb{R}$, the condition (\ref{E:infinitesimal}) implies
\begin{equation} \label{E:distance_contraction}
    \big\|\Phi(t, t_0, \vect x_1) - \Phi(t, t_0, \vect x_2)\big\| \leq e^{-c (t - t_0)} \big\|\vect x_1 - \vect x_2\big\|
\end{equation}
where $\Phi(t, t_0, \vect x_0)$ is the solution of the NLTV system from the initial condition $(t_0, \vect x_0)$. The contraction condition (\ref{E:infinitesimal}) depends critically on the inner product chosen. Let us examine the condition (\ref{E:infinitesimal}) in more detail.

For every matrix $\matr A \in \mathbb{C}^{n\times n}$, the matrix measure is defined as\footnote{The matrix measure is commonly defined in terms of a vector norm~\cite{soderlind2006logarithmic} or in terms of an inner product. The latter is more appropriate for port-Hamiltonian systems, which are naturally defined in inner product spaces.}
\begin{equation} \label{E:matrix_measure}
    \mu(\matr A) = \sup_{\delta \in \mathbb{C}^n \backslash \{0\}} \frac{\langle \delta, \matr A \delta \rangle}{\langle \delta, \delta \rangle}.
\end{equation}
From this definition, the contraction condition (\ref{E:infinitesimal}) is equivalently expressed as $\mu(D \vect f(t, \vect x)) \leq -c$. 
Since the skew-adjoint part of $\matr A$ yields zero in the numerator of (\ref{E:matrix_measure}), $\mu(\matr A)$ is equal to
\begin{equation} \label{E:self-adjoint}
    \mu(\matr A) = \sup_{\delta \in \mathbb{C}^n \backslash \{0\}} \frac{\langle \delta, \frac{1}{2}(\matr A + \matr A^*) \delta \rangle}{\langle \delta, \delta \rangle}
\end{equation}
%The definition of adjoint differs depending on the definition of inner product chosen.
%If the inner product is $\langle \vect y, \vect x \rangle = \Re\{ \vect y^\herm \vect x \}$, then the adjoint of $\matr A$ is $\matr A^\herm$. If the inner product is $\langle \vect y, \vect x \rangle = \Re\{ \vect y^* \matr P \vect x \}$, then 
The adjoint $\matr A^*$ is a mapping such that $\langle \vect x, \matr A \vect y \rangle = \langle \matr A^* \vect x, \vect y \rangle$; see~\cite[Sec.~4.4]{bressan2012lecture}.
For the inner product (\ref{E:inner_product}), the adjoint of $\matr A$ is obtained as $\matr P^{-1} \matr A^\herm \matr P$, which we substitute into (\ref{E:self-adjoint}) to get
\begin{align*}
    \mu(\matr A) &= \sup_{\delta \in \mathbb{C}^n \backslash \{0\}} \frac{\Re\big\{ \delta^\herm \matr P \frac{1}{2} (\matr A + \matr P^{-1} \matr A^\herm \matr P) \delta \big\}}{\Re\{ \delta^\herm \matr P \delta \}} \\
    &= \sup_{\delta \in \mathbb{C}^n \backslash \{0\}} \frac{\Re\big\{\frac{1}{2} \delta^\herm (\matr P \matr A + \matr A^\herm \matr P) \delta \big\}}{\Re\{ \delta^\herm \matr P \delta \}} \\
    &= \sup_{s \in \mathbb{C}^n \backslash \{0\}} \frac{\Re\big\{\frac{1}{2} s^\herm \matr P^{-\frac{1}{2}} (\matr P \matr A + \matr A^\herm \matr P)\matr P^{-\frac{1}{2}} s \big\}}{\Re\{ s^\herm s \}},
\end{align*}
where $s = \matr P^{\frac{1}{2}} \delta$.
We then obtain the following expression for the matrix measure that is dependent on the matrix $\matr P$:
\begin{equation*}
    \mu(\matr A) = \lambda_{\max} \bigg\{ \frac{1}{2} \matr P^{-\frac{1}{2}} (\matr P \matr A + \matr A^\herm \matr P) \matr P^{-\frac{1}{2}} \bigg\}.
\end{equation*}
By~\cite[Thm.~4.6]{Khalil:1173048}, if $\matr A$ is Hurwitz, there exist (many) $\matr P$'s and the associated inner products such that $\mu(\matr A) < 0$. However, not all $\matr P$'s and such inner products verify $\mu(\matr A) < 0$. 

In dealing with the general NLTV systems, the skew-adjoint part of the Jacobian matrix is ignored in order to check contraction. The skew-adjoint part can usually be related to the energy-preserving or structural part of the dynamics; this part is explicitly separated from the energy-dissipating or damping part in the port-Hamiltonian formulation of NLTV systems. %As will be shown, this explicit separation leads to a more befitting contraction property enjoyed by the port-Hamiltonian system in Proposition~\ref{prop_main}.




\begin{comment}
\subsection{Matrix Measure} \label{sec_matrix_measure}
Consider the vector space $\mathbb{C}^n$ over the field $\mathbb{R}$. The tangent space at every $\vect x\in \mathbb{C}^n$ is the same $\mathbb{C}^n$ space. Let the inner product in the tangent space be $\langle \vect y, \vect x \rangle = \Re\{\vect y^* \vect x\}$ with the associated norm $\|\vect x\| = \sqrt{\langle \vect x, \vect x \rangle}$. A matrix $\matr A \in \mathbb{C}^{n\times n}$, viewed as the linear mapping $\vect x\mapsto \matr A \vect x$, has a matrix measure $\mu(\matr A)$ which is defined as
\begin{equation} \label{E:matrix_measure}
    \mu(\matr A) = \sup_{\vect x \neq 0} \frac{\langle \vect x, \matr A \vect x\rangle}{\langle \vect x, \vect x \rangle} = \sup_{\vect x \neq 0} \frac{\langle \vect x, \frac{1}{2} (\matr A + \matr A^*) \vect x\rangle}{\langle \vect x, \vect x \rangle}.
\end{equation}
It is clear that $\mu(\matr A)$ is equal to the largest eigenvalue of the Hermitian part of $\matr A$: $\frac{1}{2} (\matr A + \matr A^*)$. Hence the matrix measure can be positive or negative, in contrast with the matrix norm induced by a vector norm.
%The matrix measure $\mu(\matr A)$ is equal to the largest eigenvalue of its Hermitian part $\frac{1}{2} (\matr A + \matr A^*)$. Since (\ref{E:matrix_measure}) can be negative, the matrix measure is not a norm even though it is also called the matrix measure.

For every Hermitian positive-definite $\matr P \in \mathbb{C}^{n\times n}$, there is a corresponding inner product $\langle \vect y, \matr P \vect x \rangle$. The associated norm is $\|\vect x\| = \sqrt{\langle \vect x, \matr P \vect x \rangle}$. The resulting matrix measure is written as
\begin{align}
    \mu(\matr A) &= \sup_{\vect x \neq 0} \frac{\langle \vect x, \matr P \matr A \vect x\rangle}{\langle \vect x, \matr P\vect x \rangle} = \sup_{\vect x \neq 0} \frac{\langle \matr P^{\frac{1}{2}} \vect x, (\matr P^{\frac{1}{2}} \matr A \matr P^{-\frac{1}{2}}) \matr P^{\frac{1}{2}}  \vect x\rangle}{\langle \matr P^{\frac{1}{2}} \vect x, \matr P^{\frac{1}{2}} \vect x \rangle}, \label{E:alt_log_norm}
\end{align}
which is equal to the largest eigenvalue of
\begin{equation} \label{E:matrix_to_check}
    \frac{1}{2} (\matr P^{\frac{1}{2}} \matr A \matr P^{-\frac{1}{2}} + \matr P^{-\frac{1}{2}} \matr A^* \matr P^{\frac{1}{2}}).
\end{equation}
Left and right multiplying (\ref{E:matrix_to_check}) by $\matr P^{\frac{1}{2}}$, we obtain that (\ref{E:matrix_to_check}) is $< 0$, hence $\mu(\matr A) < 0$, if and only if 
\begin{equation*}
    \frac{1}{2}(\matr P \matr A + \matr A^* \matr P) < 0.
\end{equation*}
This is exactly the same as the Lyapunov stability condition for LTI systems. Hence, if $\matr A$ is Hurwitz, there is a Hermitian positive-definite matrix $\matr P$ such that, for the associated inner product and matrix measure, there is $\mu(\matr A) < 0$.




\subsection{Contractive System}
Consider a time-varying system $\dot{\vect x} = \vect f(t, \vect x)$ with $(t, \vect x) \in \mathbb{R} \times \mathbb{D}$, where $\mathbb{D} \subset \mathbb{C}^n$ is an invariant domain. %We use $\Phi(t, t_0, \vect x_0)$ to denote the solution from the initial condition $(t_0, \vect x_0)$. %, where $t$ is in a maximal subset of $\mathbb{R}_{\geq 0}$ such that solution exists.

\begin{definition} \label{def_contraction}
The system is infinitesimally contracting in $\mathbb{D}$ if there is an inner product and the associated matrix measure $\mu$ such that, for some $c > 0$ (the contraction rate), it holds that $\mu(D \vect f(t, \vect x)) \leq -c$ for all $\vect x \in \mathbb{D}$.
\end{definition}

Note that the contraction condition $\mu(D\vect f(t, \vect x)) < -c$ must hold uniformly in $\mathbb{D}$. As $D \vect f(\vect x)$ being uniformly Hurwitz may involve a different matrix measure at every point, it is not a sufficient condition for contraction.
%Note that the condition $\mu(D \vect f(t, \vect x)) \leq -c$ should be checked with a fixed matrix measure for all $(t, \vect x) \in \mathbb{R}_{\geq 0}\times \mathbb{C}^n$. 
%Note that the condition that $D \vect f(t, \vect x)$ is Hurwitz for all $(t, \vect x) \in \mathbb{R}_{\geq 0}\times \mathbb{C}^n$ involves a possibly different matrix measure for every $(t, \vect x)$, and so it does not imply contraction.
%\begin{remark}
%Notice that the semi-contraction property defined in (ii) of Definition~\ref{def_contraction} is different from the standard definition in~\cite{FB-CTDS} in that the pseudo matrix measure we use here is different from that in~\cite{FB-CTDS}. The definition given here implies partial exponential contractivity, where as the definition in~\cite{FB-CTDS} requires additional assumptions on the off-diagonal elements of $D\vect f(\vect x)$ to conclude partial exponential contractivity (Theorem~5.11 in~\cite{FB-CTDS})
%\end{remark}

It is proved in~\cite{sontag2010contractive,simpson2014contraction} that,
if $\mathbb{D}$ is convex and that the system is contractive with contraction rate $c > 0$, % in a geometrically convex domain $\mathbb{C}^n$. \\
then the following norm contraction relation holds: for every $\vect x_1, \vect x_2 \in \mathbb{D}$ and $t_0\in \mathbb{R}$,
\begin{equation} \label{E:distance_contraction}
    \big\|\Phi(t, t_0, \vect x_1) - \Phi(t, t_0, \vect x_2)\big\| \leq e^{-c (t - t_0)} \big\|\vect x_1 - \vect x_2\big\|,
\end{equation}
where $\Phi(t, t_0, \vect x)$ is the solution of the system from the initial condition $(t_0, \vect x)$.
%where $\|\cdot\|$ is the norm associated with the inner product assumed.

\end{comment}


\subsection{Port-Hamiltonian System} \label{sec_pH}
In this paper, we consider the input-state-output port-Hamiltonian (pH) system with a constant damping matrix:
\begin{equation} \label{E:pH} 
    \Sigma: \begin{cases}
        \dot{\vect x} = (\matr J(t, \vect x) - \matr R) \nabla_{\vect x} H(t, \vect x) + \matr G \vect u \\
    \vect y = \matr G^\herm \nabla_{\vect x} H(t, \vect x)
    \end{cases}.
\end{equation}
In (\ref{E:pH}), $\vect x \in \mathbb{C}^n$ is the state vector, $\vect u \in \mathbb{C}^n$ and $\vect y \in \mathbb{C}^m$ are the input and output vectors, %$\vect w \in \mathbb{C}^n$ is a power input whose meaning will become clear later with a concrete system; 
$\matr J(t, \vect x) \in \mathbb{C}^{n\times n}$ is the time-varying interconnection matrix that is skew-Hermitian,
$\matr R \in \mathbb{C}^{n\times n}$ is the constant damping matrix, $\matr G \in \mathbb{C}^{n\times m}$ is the input matrix, $H(t, \vect x) \geq 0$ is the time-varying Hamiltonian, and $\nabla_{\vect x} H(t, \vect x)$ is the complex gradient with respect to the inner product $\langle \vect y, \vect x \rangle = \Re\{\vect y^\herm \vect x \}$.\footnote{The subscript for the complex gradient in $\nabla_{\vect x}H(t, \vect x)$ is omitted in the sequel.} We assume the following:
\begin{enumerate}
    \item[(i)] $\matr J(t, \vect x)$ is full-rank.
    \item[(ii)] $H(t, \vect x)$ is uniformly strictly convex, i.e., for some $a > 0$, it holds that
    \begin{equation} \label{E:strictly_convex}
        D^2 H(t, \vect x) - a \matr I_n \succeq 0,\, \text{for all $t \in \mathbb{R}$},
\end{equation}
    and $H(t, \vect x) = 0 \Leftrightarrow \vect x = 0_n$.
\end{enumerate}
%The Hamiltonian $H(t, \vect x)$ is assumed to be uniformly strictly convex: for some $a > 0$,
%\begin{equation} \label{E:strictly_convex}
    %&\exists\, \vect x'(t),\, H(t, \vect x'(t)) = 0 \text{ and }  H(t, \vect x) > 0 \text{ otherwise} \tag{positive-definiteness} \\
%    D^2 H(t, \vect x) - a \matr I_n \succeq 0,\, \text{for all $t \in \mathbb{R}$}.
%\end{equation}
%and \emph{positive-definite:} there is an $\vect x'(t)$ such that
%\begin{equation} \label{E:positive_definite}
%    H(t, \vect x) = 0 \text{ if $\vect x = \vect x'(t)$ and }  H(t, \vect x) > 0 \text{ otherwise.}
%\end{equation}

For the main result on contraction, we consider a ``closed'' pH system that is without input or output, which is a term coined by J. C. Willems~\cite{willems2007behavioral}. In the port-Hamiltonian model of a network system~\cite{fiaz2013port}, each edge is modeled as an ``open'' pH system, and the input to every edge is mapped from every output by the network constraints:
\begin{equation*}
    \col(\vect u_i) = \matr W \col(\vect y_i)
\end{equation*}
for some skew-Hermitian network matrix $\matr W$. The connected system is written as
\begin{equation} \label{E:closed}
    \dot{\vect x} = (\matr J(t, \vect x) - \matr R) \nabla H(t, \vect x)
\end{equation}
where
\begin{align*}
    &\vect x = \col(\vect x_i),\, H(t, \vect x) = \sum_{i} H_i(t, \vect x_i),\, \matr R = \diag(\matr R_i), \\
    &\matr J(t, \vect x) = \diag(\matr J_i(t, \vect x_i)) + \diag(\matr G_i) \matr W \diag(\matr G_i^\herm).
\end{align*}
The closed pH system (\ref{E:closed}) has an inherent energy balance relation:
\begin{align}
    \dot H(t, \vect x) &= \frac{\partial}{\partial t}H(t, \vect x) + \Re\{\nabla H(t, \vect x)^\herm (\matr J(t, \vect x) - \matr R) \nabla H(t, \vect x) \} \notag \\
    &= \frac{\partial}{\partial t}H(t, \vect x) - \nabla H(t, \vect x)^\herm \matr R \nabla H(t, \vect x), \label{E:energy_balance}
\end{align}
where $\matr J(t, \vect x)$ is energy-preserving because it is skew-adjoint with respect to the assumed inner product.


%\begin{lemma} \label{lem_interconnection}
%Consider a set of pH systems $\Sigma_i,\, i = 1,\ldots,\, N$, each of which verifies $\matr R_i > 0$ and $H_i(t, \vect x_i)$ positive-definite and strictly convex, and a skew-Hermitian network, i.e.,
%\begin{equation*}
%    \col(\vect u_i) = \matr W \col(\vect y_i)
%\end{equation*}
%for a skew-Hermitian $\matr W$. Then the connected system is a pH system with no input satisfying the same assumptions.
%\end{lemma}

%We refer to~\cite{barabanov2019contraction,yaghmaei2023contractive} for existing conditions for contractive pH systems.

To the best of our knowledge, there exist only two groups of papers dedicated to the contraction of pH systems. In~\cite{barabanov2019contraction}, which extends~\cite{yaghmaei2017trajectory}, two LMI and one BMI condition are proposed for contractive pH systems. The conditions require lower and upper bounds on the Hessian $D^2 H(t, \vect x)$, and in~\cite[Prop.~3]{barabanov2019contraction}, the interconnection matrix is required to be bounded relative to the damping matrix, similar to the contraction condition for a general NLTV system. In~\cite{yaghmaei2023contractive}, the partial contraction is used to decouple the state-dependence of $\matr A(\vect x) = \matr J(\vect x) - \matr R(\vect x)$ from the dynamics, which results in a nonlinear matrix inequality contraction condition; it relies critically on the assumption that the Taylor expansion of the matrix-valued function $\matr A(\vect x)$ has no first-order term. Both of the two group of results impose upper bound on $D^2 H(t, \vect x)$, and constraints on the interconnection (skew-adjoint) $\matr J(t, \vect x)$ of the dynamics. The main results in Proposition~\ref{prop_main} to~\ref{prop_convergence} are free of these constraints; the only additional assumption is for the damping matrix $\matr R$ to be a constant. This assumption is satisfied at least by the electromagnetic power system model, as the main application in this paper.




\section{Horizontal Contraction of PH System} \label{sec_contraction}





%The goal is to show that difference in values of the Hamiltonian along any two solutions of (\ref{E:pH}) converges to zero. The basic tool we will use is contraction in quotient space.

The goal is to introduce a special quotient space associated with the pH system (\ref{E:pH}) (the canonical quotient space) and to show that the system is contractive with respect to the quotient distance (horizontal contraction). It is then shown that a direct consequence of this property is that the Hamiltonian of the pH system is convergent.


\subsection{The Canonical Quotient Space}
A quotient space is a partition of the original space $\mathbb{C}^n$ into subsets called equivalence classes. Every point in the same equivalence class is equivalent; that is distance between them is set to zero. The distance between equivalence classes is defined in terms of a Finsler-like distance. Before we give the definition for this distance, we first define the canonical quotient space of the pH system.

For the pH system (\ref{E:closed}), at every time instance $t$, let us consider the quotient space where every equivalence class is an integral curve of the vector field with parameter $t$:
\begin{equation} \label{E:generator_system}
    F_t: \vect x \mapsto \matr J(t, \vect x) \nabla H(t, \vect x).
\end{equation}
The equivalence class of every $\vect x_0 \in \mathbb{C}^n$ at time $t$ is
\begin{equation} \label{E:equivalence_class}
    [\vect x_0]_t = \big\{\Phi_t(\tau, \vect x_0)\mid \tau\in \mathbb{R}\big\}
\end{equation}
where $\Phi_t(\tau, \vect x_0)$ is the integral curve of (\ref{E:generator_system}) from the initial condition $\vect x_0$, i.e.,
\begin{align*}
    \frac{d}{d\tau} \Phi_t(\tau, \vect x_0) &= F_t(\Phi_t(\tau, \vect x_0)) \\
    &= \matr J(t, \Phi_t(\tau, \vect x_0)) \nabla H(t, \Phi_t(\tau, \vect x_0))
\end{align*}
Equivalently, the integral curve $\Phi_t(\tau, \vect x_0)$ is conceptually the solution of the following system with parameter $t$:
\begin{equation} \label{E:generator2}
    \frac{d}{d\tau} \vect x(\tau) = \matr J(t, \vect x(\tau)) \nabla H(t, \vect x(\tau))
\end{equation}
where $\tau$ is the independent variable.
Since $\matr J(t, \vect x)$ is skew-Hermitian, we obtain that the value of $H(t, \vect x)$ is constant at every $\vect x = \Phi_t(\tau, \vect x_0)$ for $\tau \in \mathbb{R}$. This is because (\ref{E:generator2}) entails the energy balance $\frac{d}{d\tau} H(t, \vect x(\tau)) = 0$ where $t$ is a parameter. Hence, at every time instance $t$, the equivalence class $[\vect x_0]_t$ belongs to the level set of $H(t, \vect x)$ for $\vect x_0$; that is,
\begin{equation} \label{E:contain_in_level}
    [\vect x_0]_t \subset \big\{\vect x \in \mathbb{C}^n \mid H(t, \vect x) = H(t, \vect x_0) \big\}. 
\end{equation}
%By (\ref{E:last_assumption}), $\vect x = \eqm{\vect x}(t)$ is the only equilibrium point of (\ref{E:generator_system}). Hence every equivalence class is closed. They are bounded because the level sets of $H(t, \vect x)$ are bounded. Hence every equivalence class is compact.

%Combined with that every solution is contained in a bounded level set, we have that every equivalence class is compact.

Given the definition of the (time-dependent) equivalence classes (\ref{E:equivalence_class}), we can define a (time-dependent) distance measure for any two points $\vect x_1, \vect x_2 \in \mathbb{C}^n$. 
We choose to work with the inner product, 
\begin{equation} \label{E:assumed_inner_product}
    \langle \vect y, \vect x \rangle = \Re\{\vect y^\herm \matr R^{-1} \vect x \}. 
\end{equation}
The definitions of norm and orthogonal subspaces for $\mathbb{C}^n$ are consistent with (\ref{E:assumed_inner_product}). However, with a slight violation of this convention, the definition of the complex gradient $\nabla H(t, \vect x)$ in the pH system (\ref{E:pH}) is not adapted to (\ref{E:assumed_inner_product}). This inconsistency is unimportant because the subsequent contraction analysis concerns the derivative of the RHS of (\ref{E:pH}), i.e., second derivatives of $H(t, \vect x)$. It is chosen to simplify notation.
%However, for familiarity, we do not change the definition of the gradient $\nabla H(t, \vect x)$ or the skew-adjoint part $\matr J(t, \vect x)$ according to (\ref{E:assumed_inner_product}). In other words, we do not change the basic form of the pH system introduced in (\ref{E:pH}), which is based on the inner product $\langle \vect y, \vect x \rangle = \Re\{ \vect y^\herm \vect x \}$. % so that $\nabla H(\vect x)$ is the gradient vector to be used with the original inner product to obtain the time derivative of $H(\vect x)$.

At every time instance $t$, the distance measure according to equivalence classes (quotient distance) is defined as follows. 
By the definition in (\ref{E:generator_system}),
$\big\{\matr J(t, \vect x) \nabla H(t, \vect x)\big\}^\perp$
is the tangent subspace perpendicular to the boundary of the equivalence class at $\vect x$.
For $\vect x \neq 0_n$, define the local projection operator that projects tangent vectors onto this subspace as
\begin{equation} \label{E:projection}
    \mathcal P(t, \vect x) \delta = \delta - \frac{\langle \matr J(t, \vect x) \nabla H(t, \vect x), \delta \rangle}{\| \matr J(t, \vect x) \nabla H(t, \vect x)\| \|\delta\|} \matr J(t, \vect x) \nabla H(t, \vect x),
\end{equation}
where $\delta \in \mathbb{C}^n \neq 0_n$ is the tangent vector of a curve segment to be defined. It is well-defined because the denominator in (\ref{E:projection}) is nonzero by the two assumptions near (\ref{E:strictly_convex}). Then, the quotient distance between two points $\vect x_1, \vect x_2 \in \mathbb{C}^n$ is defined as the integral of the projected infinitesimal curve segment in (\ref{E:projection}) along a minimizing curve from $\vect x_1$ to $\vect x_2$. 
To be precise, consider a piecewise smooth curve $\gamma: [0, 1] \to \mathbb{C}^n$ such that $\gamma(0) = \vect x_1,\, \gamma(1) = \vect x_2$, and $\frac{\partial\gamma}{\partial s}(s) \neq 0$. Denote the set of all such curves as $\Gamma(\vect x_1, \vect x_2)$. The quotient distance is defined as
\begin{equation} \label{E:quotient_distance}
    \dist(t, \vect x_1, \vect x_2) = \inf_{\gamma \in \Gamma(\vect x_1, \vect x_2)} \int_0^1 \bigg\|\mathcal P(t, \gamma(s)) \frac{\partial\gamma}{\partial s}(s)\bigg\|\, ds.
\end{equation}
The existence of a minimizing curve in (\ref{E:quotient_distance}) is guaranteed by the Gauss lemma~\cite[Ch.~6]{bao2012introduction}.
We list the following properties of the quotient distance (\ref{E:quotient_distance}):
\begin{enumerate}
    \item[(i)] $\dist(t, \vect x, \vect y) = \dist(t, \vect y, \vect x)$.
    \item[(ii)] $\dist(t, \vect y, \vect x) \leq \dist(t, \vect y, \vect z) + \dist(t, \vect z, \vect x)$.
    \item[(iii)] $\dist(t, \vect y, \vect x) = 0$ if $[\vect y]_t = [\vect x]_t$ and $\dist(t, \vect y, \vect x) \geq 0$ otherwise.
    \item[(iv)] If $H(t, \vect y) \neq H(t, \vect x)$, then $\dist(t, \vect y, \vect x) > 0$.
\end{enumerate}

The proof is given in the Appendix.

\begin{remark}
The projection $\mathcal P(t, \vect x)$ defines a local $(n {-} 1)$-dimensional tangent subspace.
Contraction in a tangent subspace (or a horizontal distribution) is referred to as horizontal contraction in Section III-A of~\cite{forni2013differential} where the motivation is to not enforce contraction in the symmetry directions. Note, however, that, for the quotient space defined in (\ref{E:generator_system}) and (\ref{E:equivalence_class}), we do not assume that the dynamics of the system preserve the equivalence classes, which is a scenario called a quotient system in Section III-B of~\cite{forni2013differential}.
\end{remark}


\subsection{Horizontal Contraction in the Quotient Space}

The main contraction results are stated without proofs in this subsection. Their proofs are provided in the Appendix.

%Contraction in the quotient space is defined in terms of contraction of the quotient distance.

\begin{definition}
Let $\Phi(t, t_0, \vect x_0)$ be the solution of the pH system (\ref{E:pH}) from the initial condition $(t_0, \vect x_0)$.
The pH system (\ref{E:pH}) is said to be \emph{horizontally contracting in the canonical quotient space} (HC for short) if, for some $c > 0$ (the contraction rate), it holds that, for every $\vect x_1, \vect x_2 \in \mathbb{C}^n$ and $t_0\in \mathbb{R}$,
\begin{equation} \label{E:s1-0} 
    \dist(t, \Phi(t, t_0, \vect x_1), \Phi(t, t_0,  \vect x_2)) \leq e^{-c(t - t_0)} \dist(t_0, \vect x_1, \vect x_2).
\end{equation}
The pH system is said to be weakly HC if (\ref{E:s1-0}) holds with $c = 0$. \hfill $\lozenge$
\end{definition}

%For short, we simply say that a pH system is HC when it is HC in the canonical quotient space.

%The main result is stated as follows.

The following two results concern the intrinsic contraction properties of the pH system.

\begin{proposition}[Horizontal Contraction with $\matr R \succ 0$] \label{prop_main}
%Consider the pH system (\ref{E:pH}) with $\vect u \equiv 0$. 
The closed pH system (\ref{E:closed}) that has a uniformly strictly convex Hamiltonian, i.e., condition (\ref{E:strictly_convex}), 
is HC with the contraction rate, 
\begin{equation*}
    c = \min_{t \in \mathbb{R}} \lambda_{\min}(D^2 H(t, \vect x))\, \lambda_{\min}(\matr R),
\end{equation*}
where $\lambda_{\min}(\matr A)$ is the smallest eigenvalue of the Hermitian matrix $\matr A$. \hfill $\lozenge$
%Then the following relation about contraction in quotient distance holds: for every $\vect x_1, \vect x_2 \in \mathbb{C}^n$,
%\begin{equation} \label{E:s1-0} 
%    \dist(t, \Phi(t, t_0, \vect x_1), \Phi(t, t_0,  \vect x_2)) \leq e^{-c(t - t_0)} \dist(t_0, \vect x_1, \vect x_2),
%\end{equation}
%where $\Phi(t, t_0, \vect x)$ is the solution of the pH system from the initial condition $(t_0, \vect x)$, and $c = \lambda_{\min}(D^2 H(t, \vect x))\, \lambda_{\min}(\matr R)$.
\end{proposition}

\begin{proposition}[Weak Horizontal Contraction with $\matr R = \matr 0$] \label{prop_no_diss}
The closed pH system (\ref{E:closed}) that has a uniformly strictly convex Hamiltonian, i.e., condition (\ref{E:strictly_convex}), and zero dissipation, i.e., $\matr R = \matr 0$, is weakly HC. \hfill $\lozenge$
\end{proposition}

\begin{remark}
The classical Hamiltonian dynamics with a time-varying strictly convex Hamiltonian is covered by Proposition~\ref{prop_no_diss} with the interconnection matrix $\matr J = \big[\matr 0,\, \matr I; -\matr I,\, \matr 0 \big]$. \hfill $\lozenge$
\end{remark}

The following two results concern the implications of HC on the behavior of the solutions.

%\begin{proposition} \label{prop_limit_set}
%Consider the closed and HC pH system (\ref{E:closed}). Assume that the Hamiltonian $H(t, \vect x)$ is bounded along every solution of the system. Then,
%for every $\vect x_0 \in \mathbb{C}^n$ and $t_0 \in \mathbb{R}$,
%$\lim_{t\to \infty} H(t, \Phi(t, t_0, \vect x_0)) = \bar H$
%for some $\bar H \geq 0$. \hfill $\lozenge$
%, the value of $H(\Phi(t, t_0, \vect x_0))$ converges to the value of $H(t, \vect x)$ on the limit cycle as $t \to \infty$.
%\end{proposition}

\begin{proposition}[Converging Hamiltonian Difference] \label{prop_limit_cycle}
Consider the closed and HC pH system (\ref{E:closed}). Let $\eqm{\vect x}(t)$ be a particular solution. Then, the Hamiltonian value converges to $H(t, \bar{\vect x}(t))$ from every initial value, i.e., %$H(t, \bar{\vect x}(t)) = \bar H$ for some $\bar H \geq 0$, and, for every initial condition $\vect x_0$ at $t_0$,
from every initial condition $(t_0, \vect x_0)$, $\lim_{t\to \infty} H(t, \Phi(t, t_0, \vect x_0)) - H(t, \eqm{\vect x}(t)) = 0$.  \hfill $\lozenge$
\end{proposition}

%\begin{remark}
%The symmetry $T$ in Proposition~\ref{prop_symmetry} includes both discrete and continuous symmetry. Examples of discrete symmetry include reflection and permutation. Examples of continuous symmetry include continuous rotation and translation. When the system possesses a symmetry, it is usually true that the Hamiltonian, which represents energy storage, is invariant to the symmetry transformation; hence the second assumption in Proposition~\ref{prop_symmetry} is reasonable.
%\end{remark}

%\begin{corollary} \label{cor_stability}
%Assume that the HC pH system in Proposition~\ref{prop_main} has a limit cycle $\eqm{\vect x}: \mathbb{T} \to \mathbb{C}^n$. Then, from any $(t_0, \vect x_0)$, the value of $H(\Phi(t, t_0, \vect x_0))$ converges to the value of $H(t, \vect x)$ on the limit cycle as $t \to \infty$.
%\end{corollary}

%\begin{remark}
%By the forward contraction argument~\cite{forni2013differential}, it is implied from Corollary~\ref{cor_stability} that the Hamiltonian is constant on a limit set, e.g., the orbit of a limit cycle.
%\end{remark}

\begin{remark}
By definition, every equivalence class of the canonical quotient space is $1$D; meanwhile, every level set of the Hamiltonian $H(t, \vect x)$ for a fixed $t$ has dimension $2n - 1$ (since $\mathbb{C}^n$ has the same dimension as $\mathbb{R}^{2n}$). For $n = 1$ (every level set is $1$D and coincides with an equivalence class), then HC implies difference in the Hamiltonian values converges at the exponential rate $e^{-ct}$. For $n > 1$ (every level set has dimension higher than one), then exponential contraction of the difference in the Hamiltonian values is not guaranteed. \hfill $\lozenge$
%Each equivalence class parameterized by $\tau$ is $1$-D, whereas the level sets of the Hamiltonian are $(2n - 1)$-D. This allows the Hamiltonian of a HC system to temporarily diverge for finite $t$. For this reason, HC does not imply contraction in the Hamiltonian. A special case is $2$-D (planar) systems where the $1$-D equivalence classes coincide with the level sets. In this case, the Hamiltonian is contractive iff the system is HC. 
\end{remark}

The following is the main condition for the convergence of the Hamiltonian, i.e., convergence to a single constant value.

\begin{proposition}[Hamiltonian Convergence Principle] \label{prop_convergence}
Consider the closed and HC pH system (\ref{E:closed}). Assume that the set in which the Hamiltonian has zero derivative, i.e.,
\begin{equation*}
    E_t = \left\{ \vect x \in \mathbb{C}^n \mid \frac{\partial}{\partial t} H(t, \vect x) - \nabla H(t, \vect x)^\herm \matr R \nabla H(t, \vect x) = 0 \right\},
\end{equation*}
is time-independent.
Then, the Hamiltonian value converges from every initial value, i.e., from every initial condition $(t_0, \vect x_0)$, $\lim_{t\to \infty} H(t, \Phi(t, t_0, \vect x_0)) = \eqm H$ for some constant $\eqm H \geq 0$. \hfill $\lozenge$
\end{proposition}

The set $E_t$ in Proposition~\ref{prop_convergence} is usually found to be time-independent because the system can be alternatively written as a time-invariant system. In most applications, including the power system model to be introduced, the time dependence of the Hamiltonian represents a power source; that is,
\begin{equation*}
    \eta(\vect x) = \frac{\partial}{\partial t} H(t, \vect x)
\end{equation*}
represents the the input power.\footnote{We refer the reader to~\cite{krhavc2024port,monshizadeh2019power} for some perspectives on port-Hamiltonian system with power input.} It is a mathematical technique to represent state-dependent power input/output without introducing negative eigenvalues into the damping matrix $\matr R$. If a time-invariant form of the system exists, then the set $E_t$, which represents the states where the stored energy is steady, can be alternatively defined by a time-independent condition, and is hence a time-independent set.
%For these systems, it is usually also true that the dissipation is dependent on the state only so that $E_t$ is time-invariant.

When the system has a power input, the limit set or steady state usually cannot be described by an equilibrium point (for example the van der Pol equation~\cite{Khalil:1173048}). In this case, the limit set is the result of a balance between the power input and dissipation. For studying these systems, Proposition~\ref{prop_convergence} asserts that, if the power input can be mathematically expressed as a time-dependence part of the Hamiltonian, then the limit set is contained in a level set of the original time-independent Hamiltonian. 
On the possible types of limit sets,
if the dimension of the system is $3$ and the dimension of the level set is $2$, then by the Poincare-Bendixson theorem~\cite{Khalil:1173048}, the limit set must be a limit cycle. If the dimension of the system greater than $3$, more complicated limit sets may exist (a chaos).


%A classical example in the $2$-D state space is a ball rolling without slipping inside a parabolic bowl that is moved in a $2$-D horizontal circle. It can be proved using Corollary~\ref{cor_stability} that the circular orbit of the ball is globally attractive.

%\begin{corollary} \label{cor_2}
%Consider the pH system from Proposition~\ref{prop_main} but with $D^2 H(t, \vect x) \geq 0$. This pH system is weakly HC.
%\end{corollary}

%\begin{corollary} \label{cor_no_dissipation}
%A pH system satisfying the HC assumptions in Section~\ref{sec_pH} except for $\matr R = \matr 0$ is weakly HC.
%\end{corollary}

%The no-damping case considered in (\ref{cor_no_dissipation}) includes the classical Hamiltonian mechanics with time-varying Hamiltonian. 
%\begin{remark}
%The classical time-varying Hamiltonian dynamics with strictly convex Hamiltonian is covered by the zero-damping case in %Corollary~\ref{cor_no_dissipation} by setting $\matr J = [\matr 0, \matr I;-\matr I, \matr 0]$.
%\begin{equation*}
%    \matr J = \begin{bmatrix}
%        \matr 0 &\matr I \\
%        -\matr I &\matr 0
%    \end{bmatrix}.
%\end{equation*}
%\end{remark} 
%We recognize that the limitation with the theory is that it is not able to handle positive semidefinite $\matr R$, or non-constant $\matr R$. This is left for future work.

%The weak HC property is analogous to the regular weak contraction~\cite{FB-CTDS} if one replaces the Euclidean distance by the difference in the value of the Hamiltonian. An important property of weakly HC systems, which is analogous to the regular case, is that local asymptotic stability implies global convergence in the value of the Hamiltonian. %We will revisit this point in Proposition~\ref{prop_final}.















\section{Stability of the Electromagnetic Power System Model} \label{sec_stability}
The electromagnetic model or the fundamental model is different from the electromechanical model in that the fundamental inductor and capacitor equations are included, rather than simplified into linear impedance equations. The impedance equations are based on the assumption of a steady state of a synchronized frequency throughout the system, omitting all other types of limit cycles and necessitating separate harmonic power flow studies~\cite{xia1982harmonic}.
The difficulty in studying the electromagnetic model is twofold. First, the dimension of the system including the inductor fluxes and the capacitor charges is much higher. Second, the inductor and capacitor dynamics are much faster than the mechanical dynamics of the synchronous generator (SG). Attempts on the hard problem of extending the electromechanical stability conditions to the electromagnetic model lead to overly conservative conditions~\cite{gross2019effect,subotic2020lyapunov}. 

We consider in this section the stability of the electromagnetic power system model. As the main application of Proposition~\ref{prop_convergence}, we show that the convergence of the Hamiltonian is sufficient for stability assuming that a desirable limit cycle steady state exists. To this end, we first introduce the physical model of the SG, followed by its formulation into a time-varying pH system that is covered by Proposition~\ref{prop_convergence}. Then, the model of a power system with two SG is provided as an example. Lastly, the stability of this model is proved.







%We will first model the SG in the stator's $\alpha\beta$-coordinates as a pH system, followed by the model of the power network with constant impedance load. Lastly, Proposition~\ref{prop_convergence} is applied to characterize the stability of a two-machine power system. %The motivation is simple, the model of the SG in the $0dq$-reference frame require an additional angle variable for the rotor's angle. We would want to eliminate this angle variable because it, being defined on the periodic torus space, is known to contradict with the contraction analysis.






\subsection{Physical Model in $\alpha\beta$-Coordinates}
Assume the motor sign convention, i.e., positive stator current goes into the machine.
The mechanical dynamics is given by the swing equation:
\begin{align}
    J \dot{\omega} &= -F \omega - T_e + T_m \label{E:swing1} \\
    \dot\theta &= \omega, \label{E:swing2}
\end{align}
where $\omega$ is the rotor angular frequency, $\theta$ is the rotor angle, $J$ is the rotational inertia, $F$ is the viscous damping, and $T_e$ and $T_m$ are respectively the (accelerating) electrical and mechanical torque.\footnote{We refer the reader to~\cite{forni2014differential,efimov2017relaxed,efimov2019boundedness} for perspectives on the difficulty in studying the stability of systems in periodic angle spaces, i.e., the $2\pi$ periodic torus space $\mathbb{T}$ in which $\theta$ lives.}
%where $J$ is the rotational inertia; $F$ is the viscous damping.
%Note that here $J$ is the rotational inertia rather than the inertia constant $M = J \omega$ that is found in the version of the swing equation assumed in power engineering. The difference is that, in here, (\ref{E:swing1}) is not multiplied through by $\omega$ to get electrical and mechanical power to replace the electrical and mechanical torque. The problem with the modified swing equation is that the inertia constant $M$ is assumed to be a constant, which can result in inaccurate dynamics~\cite{caliskan2015uses}. 

We assume balanced condition such that the $0$-phase in the stationary $\alpha\beta0$-coordinates is a decoupled DC system whose state is constant zero~\cite{o2019geometric}.
Choose the complex variable $I = I^\alpha + j I^\beta$ for the stator current and $V = V^\alpha + j V^\beta$ for the terminal voltage. Let $\psi = M I_F \in \mathbb{R}$ be the constant field flux, i.e., mutual inductance $M$ times the field current $I_F$. The stator equation is given by
%\begin{align}
%    \dot{(LI + \psi e^{j\theta})} = -R I + V, \label{E:stator}
%\end{align}
%which can be written alternatively to reveal the internal EMF in the RHS as follows,
\begin{align*}
    L \dot{I} = -\psi (j\omega e^{j\theta}) - R I + V,
\end{align*}
where $R$ is the stator current, and $-\psi (j\omega e^{j\theta})$ is the internal EMF. %The equation for the shunt capacitor at the SG bus is given by
%\begin{equation*}
%    C \dot{V} = -G V - I + I_b
%\end{equation*}
%where $I_b$ is the bus current.
To complete the swing equation (\ref{E:swing1}), note that the electrical torque is equal to the power transfer divided by the frequency:
\begin{align*}
    T_e &= \frac{\Re\{-\psi (j \omega e^{j\theta}) I^\herm \}}{\omega} = \Re\{ j \psi e^{-j\theta} I\},
\end{align*}
%where we substituted $I e^{-j\theta} = I^r + j I^i$ in the last equality. 
Assume that the mechanical source has droop characteristic:
\begin{equation*}
    T_m = T_0 - F_1 \omega,
\end{equation*}
where $F_1$ is the torque droop ratio, and $T_0$ is the projected zero-freqeuncy torque.
Then we obtain the controlled swing equation,
\begin{equation*}
    J \dot\omega = -F \omega - \Re\{j\psi e^{-j\theta} I\} + T_0,
\end{equation*}
where we absorbed $F \leftarrow F + F_1$.


\subsection{Time-Varying PH Model} 
Choose the state vector $\vect x = \big[ \vect x_1,\, \theta \big]^\tran \in \mathbb{R} \times \mathbb{C} \times \mathbb{T}$ where
\begin{equation} \label{E:state}
    \vect x_1 = \big[ x_1,\, x_2 \big]^\tran = \big[J \omega - T_0 t,\, L I \big]^\tran.
\end{equation}
The space $\mathbb{T}$ is the $2\pi$-periodic torus~\cite{FB-ADL:04} for the angle. 
%The angle symmetry can then be expressed as
%\begin{equation*}
%    \mathcal T(\tau) \vect x = e^{j\tau} \vect x.
%\end{equation*}
The equations for the state vector are given by
\begin{align}
    \frac{d}{dt} (J \omega - T_0 t) &= -F \omega - \Re\{j\psi \omega e^{-j\theta} I\} \label{E:sg_1} \\
    \frac{d}{dt} (L I) &= -j \psi \omega e^{j\theta} - R I + V \\
    %\frac{d}{dt} (C V) &= -G V - I + I_b \\
    \frac{d}{dt} \theta &= \omega. \label{E:sg_2} 
\end{align}
Define the Hamiltonian as
\begin{equation*}
    H(t, \vect x_1) = \frac{1}{2} J^{-1} (x_1 + T_0 t)^2 + \frac{1}{2} L^{-1} \big\|x_2 \big\|^2, % + \frac{1}{2} C^{-1} \big\|x_3\big\|^2. \notag
\end{equation*}
which is equal to the inertial energy plus the magnetic energy.
The gradient of the Hamiltonian is
\begin{align}
    \nabla H(t, \vect x_1) &= \begin{bmatrix}
        J^{-1} (x_1 + T_0 t) \\
        L^{-1} x_2
    \end{bmatrix} = \begin{bmatrix}
        \omega \\
        I
    \end{bmatrix}. \label{E:gradient}
\end{align}
Note that the Hamiltonian can be alternatively expressed as
\begin{equation*}
    H(t, \vect x_1) = H(\vect s) = \frac{1}{2} \vect s^\herm \matr Q^{-1} \vect s
\end{equation*}
with the co-state $\vect s = \nabla H(t, \vect x)$ and $\matr Q = \mathrm{diag}(J^{-1}, L^{-1})$. We can find the real Hessian of the Hamiltonian as\footnote{The set of independent variables are $\{\vect x,\,  t\}$. All partial derivatives are defined with respect to these independent variables.}
\begin{equation} \label{E:sg_hessian}
    D^2 H(t, \vect x_1) = \begin{bmatrix}
        J^{-1} &0 \\
        0 &L^{-1} \matr I
    \end{bmatrix}
\end{equation}
for the equivalent real state vector, $\mathcal U \vect x_1 = \big[ J\omega - T_0 t,\, L I^\alpha,\, L I^\beta \big]^\tran$. Since $J, L > 0$, the Hamiltonian is uniformly strictly convex.
Based on (\ref{E:state}) and (\ref{E:gradient}), the pH model for $\vect x_1$ is obtained as
\begin{equation} \label{E:SM_model}
    \Sigma_{sg}: \begin{cases}
        \dot{\vect x}_1 = \mathcal W \big[ (\matr J(\theta) - \matr R) \nabla H(t, \vect x_1) + \matr G \vect u \big] \\
    \vect y = \matr G^\herm \nabla H(t, \vect x_1)
    \end{cases},
\end{equation}
where $\vect u = V,\, \vect y = I$,
\begin{equation*}
    \mathcal W = \begin{bmatrix}
        \Re &0 \\
        0 &1
    \end{bmatrix},\, \matr G = \begin{bmatrix}
        0 \\
        1
    \end{bmatrix},
\end{equation*}
and 
\begin{align*}
    &\matr J(\theta) = \begin{bmatrix}
        0 &-j \psi e^{-j\theta} \\
        -j \psi e^{j\theta} &0
    \end{bmatrix},\, \matr R = \begin{bmatrix}
        F &0 \\
        0 &R
    \end{bmatrix}.
\end{align*}
For the dynamics (\ref{E:SM_model}), the inner product assumed is
\begin{equation*}
    \langle \vect y, \vect x \rangle = \Re\{ (\mathcal W \vect y)^\herm (\mathcal W \vect x) \}. 
\end{equation*}

Note that the SG system (\ref{E:sg_1})--(\ref{E:sg_2}) is not exactly an open pH system (\ref{E:pH}); the pH system for the SG in (\ref{E:SM_model}) does include the last equation (\ref{E:sg_2}) while the physical Hamiltonian $H(t, \vect x_1)$ is not dependent on the angle $\theta$. Note, however, that the main propositions in Section~\ref{sec_contraction} still apply to this system with the minimal modification as follows. For the quotient distance in (\ref{E:quotient_distance}), the projection of the $\vect x_1$ dimensions is defined as in (\ref{E:projection}), i.e.,
\begin{equation*}
    \mathcal P(t, \vect x) \begin{bmatrix}
        \delta_1 \\
        0
    \end{bmatrix} = \begin{bmatrix} 
        \delta_1 - \frac{\langle \matr J(\theta) \nabla H(t, \vect x_1), \delta_1 \rangle}{\| \matr J(\theta) \nabla H(t, \vect x_1)\| \|\delta_1\|} \matr J(\theta) \nabla H(t, \vect x_1) \\
        0
    \end{bmatrix}
\end{equation*}
for $\delta_1 \in T_{\vect x_1} (\mathbb{R} \times \mathbb{C}) = \mathbb{R} \times \mathbb{C}$, and the projection of the $\theta$ dimension is set to $0$, i.e.,
\begin{equation*}
    \mathcal P(t, \vect x) \begin{bmatrix}
        0 \\
        0 \\
        \delta \theta
    \end{bmatrix} = \begin{bmatrix}
        0 \\
        0 \\
        0
    \end{bmatrix}
\end{equation*}
for $\delta \theta \in T_{\vect x} \mathbb{T} = \mathbb{R}$.
Then, it is easy to checked that the proofs of the main propositions in Section~\ref{sec_contraction} still work under the respective conditions.

\begin{comment}
Note that, compared to the system considered in Section~\ref{sec_contraction}, the SG system has the additional state variable $\theta$ which does not affect the definition of the Hamiltonian, i.e., $\frac{\partial H}{\partial \theta} = 0$. However, it can be easily checked that all the machinery from Section~\ref{sec_contraction} still works for the system (\ref{E:SM_model}) if we (i) use the equations for $\vect x$ to generate the equivalence classes, i.e., using
\begin{align*}
    \begin{bmatrix}[1.2]
        \dot{\vect x}_1(\tau) \\
        \dot\theta(\tau)
    \end{bmatrix} &= \begin{bmatrix}
        \mathcal W \left[ (\matr J(\vect x(\tau)) - \matr R) \nabla H(t, \vect x_1(\tau)) + \matr G \vect u(\tau)\right] \\
        \omega(\tau)
    \end{bmatrix} \\
    \vect y(\tau) &= \matr G^\herm \nabla H(t, \vect x_1(\tau)),
\end{align*}
where $\vect u(\tau)$ is supplied from the other open subsystems,
and (ii) only include $\vect x_1$ in the definition of the quotient distance in (\ref{E:projection}), i.e.,
\begin{equation*}
    \mathcal P(t, \vect x) \begin{bmatrix}
        0 \\
        0 \\
        1
    \end{bmatrix} = 0_3.
\end{equation*}
In particular, the system (\ref{E:SM_model}) without considering the input $\vect u$, is HC if $\matr R > 0$ and $D^2 H(t, \vect x_1) \geq c > 0$, which are true. The input is supplied from other subsystems of the power system, which are introduced in the next subsection.
\end{comment}


We can check the condition of Proposition~\ref{prop_convergence} as follows, Note that, from (\ref{E:energy_balance}), the power input is
$\frac{\partial}{\partial t} H(t, \vect x_1) = T_0 \omega$,
and the dissipation is
$-\vect s^\herm \matr R \vect s$.
Both are time-independent. Then, by Proposition~\ref{prop_convergence}, the Hamiltonian is convergent if we ignore the input. The input is supplied from other subsystems of the power system, which are introduced in the next subsection.

\begin{comment}
The equivalence classes at time $t$ are defined in the full state space $\vect x$, i.e., from the system
\begin{align*}
    \begin{bmatrix}[1.2]
        \dot{\vect x}_1(\tau) \\
        \dot\theta(\tau)
    \end{bmatrix} &= \begin{bmatrix}
        \mathcal W \left[ (\matr J(\vect x(\tau)) - \matr R) \nabla H(t, \vect x_1(\tau)) + \matr G \vect u(\tau)\right] \\
        \omega(\tau)
    \end{bmatrix} \\
    \vect y(\tau) &= \matr G^\herm \nabla H(t, \vect x_1(\tau)),
\end{align*}
where $\vect u(\tau)$ is supplied from the other open subsystems.
It can be checked that, ignoring the input, Proposition~\ref{prop_convergence} holds by treating $\theta$ similar to $\vect x_1$ except that $\theta$ does not contribute to the distance. %The inner product assumed for checking HC is $\langle \mathcal U \vect y, \mathcal U \matr R^{-1} \vect x \rangle$.
\end{comment}

\begin{remark}
The technique for modeling the SG with constant field current as a pH system can be easily applied to the full-order SG dynamics with one excitation winding and three damper windings~\cite{vittal2019power}. To do this, the DC circuits are modeled as real variables similar to $x_1$ in the above. We choose to present the simpler SG model in this paper to show the idea more clearly. \hfill $\lozenge$
\end{remark}

%\subsection{Single-Machine Infinite-Bus System}

\subsection{Two-Machine System with Constant Impedance Loads}
As an example, consider a two-machine system with constant impedance loads. With the pH modeling technique introduced in~\cite{fiaz2013port}, the system is seen as a directed graph where each edge is either a SG, a shunt capacitor, or an R--L line. The sign convention for the edge voltage and current follows the direction of the edge; that is positive edge voltage and current consumes real power. See Fig.~\ref{fig_topology} for the graph topology of the two-machine system considered.

The two SG systems are denoted as $\Sigma_{sg_i},\, i \in \{1, 2\}$. The transmission lines are modeled by the lumped-parameter $\Pi$-model~\cite{watson2021scalable}. The equations for the shunt capacitor edge is, for $i \in \{3, 4, 5\}$,
\begin{equation} \label{E:capacitor}
    \Sigma_{sh_i}: \begin{cases}
        \frac{d}{dt} (C_i V_i) = -Y_i \nabla H_i(C_i V_i) + I_i \\
        V_i = \nabla H_i(C_i V_i)
    \end{cases},
\end{equation}
where $V_i$ and $I_i$ are respectively the edge voltage and current,
$H_i(\vect x_i) = \frac{1}{2} C_i^{-1} \|\vect x_i\|^2$ with $\vect x_i = C_i V_i$, $Y_i$ with $\Re\{Y_i\} > 0$ is the admittance of the constant impedance load. The equations for the line edges are, for $i \in \{6, 7\}$,
\begin{equation} \label{E:inductor}
    \Sigma_{ln_i} \begin{cases}
        \frac{d}{dt} (L_i I_i) = -R_i \nabla H_i(L_i I_i) + V_i \\
        I_i = \nabla H_i(L_i I_i)
    \end{cases},
\end{equation}
where $V_i$ and $I_i$ are respectively the edge voltage and current; 
$H_i(\vect x_i) = \frac{1}{2} L_i^{-1} \|\vect x_i\|^2$ with $\vect x_i = L_i I_i$; $R_i$ is the series line resistance.

Based on the inputs and outputs,
\begin{align*}
    \vect u &= \big[ V_1,\, V_2,\, I_3,\, I_4,\, I_5,\, V_6,\, V_7 \big]^\tran, \\
    \vect y &= \big[ I_1,\, I_2,\, V_3,\, V_4,\, V_5,\, I_6,\, I_7 \big]^\tran.
\end{align*}
and KCL and KVL, the network matrix which relates the inputs and outputs of the edges, is found as
\begin{equation*}
    \matr W = \begin{bmatrix}
        0 &0 &1 &0 &0 &0 &0 \\
        0 &0 &0 &1 &0 &0 &0 \\
        -1 &0 &0 &0 &0 &-1 &0 \\
        0 &-1 &0 &0 &0 &0 &-1 \\
        0 &0 &0 &0 &0 &1 &1 \\
        0 &0 &1 &0 &-1 &0 &0 \\
        0 &0 &0 &1 &-1 &0 &0
    \end{bmatrix}.
\end{equation*}
We verify that that $\matr W$ is skew-symmetric. 
%For derivation of the interconnection matrix for a power system with a general topology, we refer to~\cite{jiang2024reference}. 
Hence, recalling (\ref{E:closed}), the two-machine system connected through $\matr W$ is a pH system with $\matr R \succ 0$ and $D^2 H(t, \vect x) - a\matr I_n \succeq 0$. Note that the skew-symmetry of the network matrix results from KVL and KCL, and so it holds regardless of the topology of the power system~\cite{jiang2024reference}.

\begin{figure}[!t]
\subfloat{\includegraphics[width=2.2in,center,margin=0in 0in 0in 0in]{Figures/one_line_figure.pdf}}
\hfill
\subfloat{\includegraphics[width=1.4in,center,margin=0in 0in 0in 0in]{Figures/topology_figure.pdf}}
\caption{Single-line diagram of the two-machine system and the underlying graph topology (SG: red, shunt capacitor: blue, R--L line: green)}
\label{fig_topology}
\hfill\end{figure}

\subsection{Stability of the Two-Machine System}

To prove the stability of the two-machine system, we need the following result on the uniqueness of the limit cycle of an RLC circuit with sinusoidal forcing.

\begin{lemma} \label{lem_RLC}
Consider an RLC circuit with sinusoidal forcing. With the edge dynamics given by (\ref{E:capacitor}) and (\ref{E:inductor}), the equations can be written in the form,
\begin{equation*}
    \dot{\vect x} = (\matr J - \matr R) \nabla H(\vect x) + \vect g u
\end{equation*}
with $u = e^{j\omega_0 t}$ and $H(\vect x) = \frac{1}{2} \vect x^\herm \matr Q \vect x$ for diagonal $\matr Q$ and $\matr R$. Assume the system has a limit cycle $\eqm{\vect x}(t) = e^{j\omega_0 \tau} \eqm{\vect x}(0)$. Then the orbit of the limit cycle is the only possible limit set. \hfill $\lozenge$
\end{lemma}

The proof is given in the Appendix.

%The main result of this paper is stated as follows.

\begin{proposition} \label{prop_final}
Assume the two-machine system has a limit cycle solution $\eqm{\vect x}(t)$ of a synchronized frequency $\eqm\omega \in \mathbb{R}$ such that the complex variables change as $x_i(t) = e^{j\eqm \omega t} x_i(0)$. Then every solution of the system converges to the limit cycle solution as $t \to \infty$. \hfill $\lozenge$
\end{proposition}

The proof is given in the Appendix.

Since the proof for the two-machine system does not rely on the graph topology, the same stability result can be generalized to multi-machine systems with constant impedance loads. For the sake of the analysis, the only difference is the dimensionality of the network matrix. In general, one can consider distributed-parameter model of the transmission lines where the number of edges approaches infinity~\cite{watson2021scalable}.

\begin{proposition}
Assume that a multi-machine system consisting of SG, shunt capacitor, and R--L line edges, has a limit cycle solution of a synchronized frequency. Then every solution of the system converges to the limit cycle solution as $t \to \infty$. \hfill $\lozenge$
\end{proposition}

The proof is similar to the proof of Proposition~\ref{prop_final} and is omitted.

\begin{remark}
By Proposition~\ref{prop_main}, the convergence rate of the Hamiltonian is proportional to $\lambda_{\min}(D^2 H(t, \vect x))\, \lambda_{\min}(\matr R)$. Assuming that the mechanical energy storage is dominant in $H(t, \vect x)$ and the electrical energy dissipation is dominant in $\matr R$, we obtain that the convergence rate of the power system Hamiltonian is estimated as $\min\{\Re\, Y_i\}/\max\{J_i\}$. \hfill $\lozenge$
\end{remark}