\begin{comment}
\begin{abstract}
Transient stability is crucial to the reliability operation of power systems. Existing theories \st{base their analysis on the simplified electromechanical dynamics by replacing the inductor and capacitor dynamics with their impedance.}  rely on simplified electromechanical dynamics, substituting the detailed inductor and capacitor dynamic behavior with their equivalent impedance representations. However, this simplification is inadequate for the growing penetration of fast-responding power electronics devices. Previous attempts to extend the existing theories to include electromagnetic dynamics lead to overly conservative stability conditions.
To tackle this problem \st{from a different angle} effectively, we \st{try to} show that the power source and dissipation in the electromagnetic dynamics tend to a balance in the limit of time, which is a sufficient condition for transient stability, and it is equivalent to the convergence of the Hamiltonian (total stored energy).  \st{We prove using contraction analysis that this property is satisfied for} Using contraction analysis, we prove that this property holds for a large class of time-varying port-Hamiltonian systems with (i) constant damping matrix and (ii) strictly convex Hamiltonian. Then through port-Hamiltonian modeling of the electromagnetic dynamics, we obtain that the synchronized limit cycle steady state of the power system is globally stable if it exists.
\hl{The theory is illustrated in detailed analysis of the simulated numerical example.} \Yan{This sentence is too general, carrying no info.}
This result provides new insights into the reliable operation of power systems. 
\end{abstract}
\end{comment}
\begin{abstract}
Transient stability is crucial to the reliable operation of power systems. Existing theories rely on the simplified electromechanical models, substituting the detailed electromagnetic dynamics of inductor and capacitor with their impedance representations. However, this simplification is inadequate for the growing penetration of fast-switching power electronic devices. Attempts to extend the existing theories to include electromagnetic dynamics lead to overly conservative stability conditions.
To tackle this problem more directly, we study the condition under which the power source and dissipation in the electromagnetic dynamics tend to balance each other asymptotically. This is equivalent to the convergence of the Hamiltonian (total stored energy) and can be shown to imply transient stability. Using contraction analysis, we prove that this property holds for a large class of time-varying port-Hamiltonian systems with (i) constant damping matrix and (ii) strictly convex Hamiltonian. Then through port-Hamiltonian modeling of the electromagnetic dynamics, we obtain that the synchronized steady state of the power system is globally stable if it exists. This result provides new insights into the reliable operation of power systems. 
The proposed theory is illustrated in the simulation results of a two-machine system.
%The insight from this result is that the major challenge in the reliable operation of future power systems is the existence of a synchronized steady state, more so than its stability. %The theory is illustrated in detailed analysis of the simulated numerical example.
\end{abstract}

\begin{keywords}
%5-10 keywords
Transient stability analysis, contraction analysis, port-Hamiltonian system, electromagnetic dynamics, limit cycle
\end{keywords}