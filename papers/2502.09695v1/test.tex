\section{Numerical Examples} \label{sec_test}

In Proposition~\ref{prop_final}, we proved that a limit cycle of the power system with constant impedance loads, is globally convergent if it exists. Note that stability in power system traditionally refers the steady-state behavior of the system; this includes
\begin{enumerate}
    \item[1.] convergence of all the SG frequencies to a single constant synchronized frequency,
    \item[2.] no low-frequency oscillations, i.e., the envelope of every three-phase AC signal is a straight horizontal line,
    \item[3.] no harmonics, i.e., every AC signal has a single Fourier component at the synchronized frequency from (i).
\end{enumerate}
We will show that these steady-state instability conditions do not contradict with the global convergence result proved in Proposition~\ref{prop_final} because the traditional stability concerns the regularity of the steady state as opposed to its convergence property. Here we should adopt a geometric approach to the system by considering its set of solutions that has ran for all finite time. The steady state, be it a \emph{synchronized limit cycle} or an \emph{imperfect (negation of 1--3) limit cycle}, is the target set whose Hamiltonian value is convergent, and, in the case of a synchronized limit cycle, the orbit is convergent as well.
In fact, we will show that the instability of the steady-state in traditional power engineering correspond to the nonexistence of a synchronized limit cycle, due to the conflict between the power flow constraint from the RLC network and the power injection constraint from the SGs.

The two machine system in Fig.~\ref{fig_topology} is chosen. A electromagnetic model is build in Simulink Specialized Power System in SI units. The PMSM model is used to model a SG with constant field current. The parameters of the SG are obtained from Example~4.1 from~\cite{vittal2019power}, which are given in Table~\ref{table_parameters} as well as all the other parameters. %We restrict the system to a balanced condition by setting the parameters to be the same in all three phases and restricting the initial condition to be balanced. 
%The code for this test has been made available at~\cite{code}.
%We have conducted the following tests (all in small figures):



\begin{table}[!t]
\renewcommand{\arraystretch}{1.2}
\caption{Parameters of the Two-Machine Test System}
\label{table_parameters}
\centering\begin{tabular}{c|c}
\hline
Description &Parameter \\
\hline
SG mechanical &$J = 2.846\times 10^4$ \unit{kg.m^2}, $F = 85.5601$ \unit{N.m.s}, \\
&$p = 4$, $T_0 = 1\times 10^4$ \unit{N.m} \\
\hline
SG electrical &$R_s = 1.542$ \unit{\milli \Omega}, $L_s = 6.341$ \unit{\milli H}, \\
&$\psi = 39.7877$ \unit{V.s} \\
\hline
Shunt capacitor &$C_{sh,3} = 50 $ \unit{\milli F}, $C_{sh,4} = 100 $ \unit{\milli F}, \\
&$C_{sh,5} = 50 $ \unit{\milli F} \\
\hline
Load &$R_{ld,3} = 1$ \unit{k \Omega}, $L_{ld,3} = 10$ \unit{H} \\
&$R_{ld,4} = 4$ \unit{\Omega}, $L_{ld,4} = 1$ \unit{H} \\
&$R_{ld,5} = 1$ \unit{k \Omega}, $L_{ld,5} = 10$ \unit{H} \\
\hline
R--L line &$R_{ln,6} = 3$ \unit{\Omega}, $L_{ln,6} = 1.061$ \unit{H}, \\
&$R_{ln,7} = 3$ \unit{\Omega}, $L_{ln,7} = 1.061$ \unit{H} \\
\hline
%&$R_{ln,7} = 5$ \unit{Ohm}, $L_{ln,7} = 2.653$ \unit{H} \\
\end{tabular}
\end{table}

\subsection{Global Convergence in the Regular Case}
%plot of the Hamiltonian from many random initial conditions
%plot of zero start including voltage, stored energy of every component

In the first case, we set the parameters of the system such that there is reflection symmetry between the left and right half of the topology. This ensures that a synchronized limit cycle exists so that we can test convergence alone. More generally, a symmetric radial network is similar to a single SG system, and so a synchronized limit cycle exists~\cite{johnson2014synchronization}. We test the convergence property of the system by starting from a random initial condition generated by 100*rand(26) for $26$ real state variables. We can see from Fig.~\ref{fig_regular}d that immediately after the initial condition there is a large overshoot in the Hamiltonian caused by the RLC network quickly returning to an almost quasi steady state. Between $0$~\unit{s} and $485$~\unit{s}, the waveform of the voltage is not regular at all as it exhibits low-frequency oscillation. During this time, the voltage waveform appears constant except for the increasing widths of the distinct wave packets, while the rotor frequencies continue to approach each other. Between $485$~\unit{s} and $700$~\unit{s}, the rotor frequencies are locked to each other, and the low-frequency oscillation is dying. At around $700$~\unit{s}, the system returns to the synchronized limit cycle with a single-frequency waveform. We tested $50$ other initial conditions from 1000*rand(26) to verify that the system returns to the same synchronized limit cycle. We tested multiplying the damping matrix by a factor and observe that the transient time reduces by the same factor. We also tested multiplying the SG inertia by a factor of the nominal value and observed that the qualitative behavior of the system remains the same except that the transient time is extended with higher inertias and shortened with lower inertias. %The reader is invited to verify these observations for themselves with the supplied code~\cite{code}. 

Note that the initial conditions tested here are much farther from the synchronized limit cycle compared to those considered in traditional power system stability studies---the initial condition is far outside the region of convergence predicted by the direct method. Moreover, in traditional power engineering, the excitation control is considered to have a the biggest effect on (steady-state) stability, whereas, here, the field current is kept constant. The transient condition tested here corresponds to uncontrolled black starts or protection device malfunctioning in faults where the rotor angle stability~\cite{tziouvaras2004out} is lost. The test result shows that the traditional stability concepts such as critical clearing time, and the negative effect of low inertia on stability, do not hold at least for this test system. We suspect the reason for these traditional concepts is that it takes two distinct stages for the state to converge. From Fig.~\ref{fig_regular}, there is almost no sign that the low-frequency oscillation is dying before $485$~\unit{s} while the rotor frequencies are approaching each other. This behavior is qualitatively different from the more familiar linear dynamics where convergence is exponential in every state variable. %For the nonlinear power system dynamics, the convergence rate, which is proportional to the damping and the convexity of the Hamiltonian by Proposition~\ref{prop_main}, is only relevant when close to the target limit cycle.



\begin{figure}[!t]
\centering 
\subfloat{\centering \includegraphics[width=1.65in]{Figures/voltage_1.pdf}}
\subfloat{\centering \includegraphics[width=1.67in,margin=0.03in 0in]{Figures/frequency.pdf}}
\hfil
\subfloat{\centering \includegraphics[width=1.6in,margin=-0.02in -0.04in]{Figures/voltage_2.pdf}}\hspace{0.03in}
\subfloat{\centering \includegraphics[width=1.69in,margin=0in -0.05in]{Figures/hamiltonian.pdf}}
\caption{Case where there exists a synchronized limit cycle}
\label{fig_regular}
\hfill\end{figure}







\subsection{Loss of Synchronized Limit Cycle Steady State}
%How Things Can Go Wrong: Zero Torque Droop with High Field Flux}
%plot of 5.5x the base field flux ($\geq 20$ second simulation time) showing the system going into chaotic behavior

%plot of 5x the base field flux showing that the system going into a limit cycle where the frequencies oscillate, and the electrical quantities have undamped harmonics

%plot of 2x the base field flux showing that the system going into a limit cycle where the frequencies converge, and the electrical quantities have undamped harmonics

%In the second case, we increased the input torque of the SG~2 from $1\times 10^4$~\unit{N.m} to $1.5\times 10^4$~\unit{N.m} while keeping the input torque of the SG~1 the same.

There are several ways that a synchronized limit cycle cannot be reached asymptotically. In the first case, we increased the torque input of SG~2 from $1\times 10^4$~\unit{N.m} to $1.5\times 10^4$~\unit{N.m} so that right side of the two-machine system will inject $1.5$ times the power than the left side if the two rotor frequencies are to converge to the same value. As we can see from Fig.~\ref{fig_case1}, the frequencies of the two machines converge to different values. The frequency of SG~2 is approximately $1.5$ times the frequency of SG~1, possibly due to the additional torque input. The steady state shown in Fig.~\ref{fig_case1}b shows that the steady-state limit cycle exhibits low-frequency oscillation which forms distinct wave packets. This otherwise undamped low-frequency oscillation is dampened in practice by excitation control which modulates the excitation voltage based on measurement of the terminal voltage~\cite{vittal2019power}.
%In the third case, we increased the field flux of both SGs by $2.5$ times.


\begin{figure}[!t]
\centering 
\subfloat{\centering \includegraphics[width=1.65in]{Figures/voltage_1_case2.pdf}}
\subfloat{\centering \includegraphics[width=1.68in,margin=-0.01in 0in]{Figures/frequency_case2.pdf}}
\hfil
\subfloat{\centering \includegraphics[width=1.67in,margin=0.02in -0.02in]{Figures/voltage_2_case2.pdf}}\hspace{-0.04in}
\subfloat{\centering \includegraphics[width=1.63in,margin=0in -0.02in]{Figures/hamiltonian_case2.pdf}}
\caption{Case where there only exists an imperfect limit cycle with low-frequency oscillation (existence of two steady-state frequencies) due to the large difference between the input torques of the two SGs}
\label{fig_case1}
\hfill\end{figure}

\begin{figure}[!t]
\centering 
\subfloat{\centering \includegraphics[width=1.67in,margin=0.02in 0in]{Figures/voltage_1_case3.pdf}}
\subfloat{\centering \includegraphics[width=1.67in,margin=0in 0in]{Figures/frequency_case3.pdf}}
\caption{Case where the system collapses to close to zero due to the field fluxes of the SGs being too high}
\label{fig_case2}
\hfill\end{figure}

In the second case, we multiplied the constant field flux of both SGs by $2.5$ times. We can see from Fig.~\ref{fig_case2} that the voltage and the frequency both collapse to close to zero after a transient period. This is because the high field flux results in high EMF, which results in high real power consumption from the constant impedance loads, compared to the relatively low torque input. Note that, even in this case, the final steady state is not exactly zero, but is an equilibrium point that is very close to zero---the zero state is not stable whenever the input torque is nonzero.

In the third case, we observed during our test that, under certain parameter choices, the system goes into a bounded chaotic state rather than a limit cycle. 
%Thirdly, we note that chaotic limiting behavior is possible. However, from the proof of Proposition~\ref{prop_final}, the value of the Hamiltonian of every edge converges independently. This means that all real quantities including the rotor frequencies and the voltage and current amplitudes converge, and the periodic quantities including the voltage and current angles are possibly chaotic.



\subsection{Discussion}

It is shown that global stability, i.e., convergence of every solution to a synchronized limit cycle, holds for multi-machine power system dynamics if the synchronized limit cycle exists. The convergence rate is observed to be proportional to $1/\max\{J_i\}$ and $\min\{\Re\{Y_i\}\}$. It is justified both theoretically and experimentally that the remaining question to ask about power system dynamics is whether the synchronized limit cycle exists, because it is stable if it exists. This explains why the focus on phasor analysis in traditional power engineering has proved successful thus far, despite the highly nonlinear dynamics. The revealed stability property of the traditional SG power system is insightful for the control of inverter-based resources (IBR).
Although the SG dynamics implies globally stability, there is still room to design IBR control schemes to improve transient smoothness and power sharing as the future power system is seeing more frequent reconfigurations. See our work toward this direction in~\cite{jiang2024reference}. %From this perspective, steady-state analysis is still expected to be the most useful tool for power engineers.

From the test results, we can see that the biggest challenge in operating AC power system versus DC power system is not in stability but the potentially complicated steady-state behavior. %There is no fundamental differences between a dissipative AC system and DC system\footnote{Here we omitted the effect of nonlinear loads}---a DC system can be approximated by setting the SG inertia to infinity so that the SG frequency is constrained to $0$. 
The steady state behaviors of AC system include undesirable oscillations and chaos. These important features of the steady state cannot be represented in the traditional phasor analysis, and not fully by harmonic power flow studies. For future research, more research effort is needed toward the precise steady state behaviors of AC power systems to obtain conditions on the existence of synchronized limit cycles and the corresponding steady state control schemes. A notable recent work toward this direction is~\cite{gross2018steady}.