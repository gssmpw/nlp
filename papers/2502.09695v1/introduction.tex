\section{Introduction}

\PARstart{T}{ransient} stability is crucial to the planning and operation of power systems and has become even more so with the increasing penetration of power electronic technologies, whose dynamics are dependent primarily on their local controls. These local controls operate on the electromagnetic time scale from 1 \unit{\micro s} to 1 \unit{\milli s}, which necessitates more detailed analyses into the transient dynamics of power systems in the electromagnetic time scale~\cite{hatziargyriou2020definition}. In addition to the increased complexity in the types of dynamic agents involved, there is also a growing trend of individual generators for industrial and commercial power systems to sustain islanding situations during blackouts and to sell power to the bulk power system during normal operation~\cite{dai2024practical}. This creates increased complexity in the loading and configuration, which magnifies the limitations of the existing methods for transient stability analysis.
There are mainly four types of methods for analyzing transient stability, namely, angle stability, feedback linearization, passivity, and direct methods. 

Firstly, angle stability~\cite{bera2021identification,dai2024practical} aims at monitoring the torque angles of every synchronous generator in the system to detect whether some generators are entering into stable or unstable swing post fault. It relies on the observation that if the angles do not exceed the limits of the power-angle curves, in out-of-step conditions~\cite{tziouvaras2004out}, then the restoring and damping torque cause the angles to synchronize. The limitation, however, is that it does not study the system dynamics directly, but aims to extract patterns from numerical simulations of instability. Therefore, it does not provide enough insight into the effects of the controls.

Secondly, feedback linearization can algebraically linearize and decouple the interactions between the generators and the loads to obtain linear subsystems which  are suitable for linear control. In~\cite{mahmud2013partial}, an synchronous generator excitation control is designed to guarantee transient stability of the simplified multimachine power system model. In~\cite{bidram2013distributed}, a microgrid secondary control is designed for improving the tracking performance for frequency and voltage. However, the limitation is that the electromagnetic dynamics usually cannot be fully linearized so that only partial stability can be attained.

Thirdly, passivity is a distributed subsystem-level property for verifying overall transient stability. It is closely related to physical energy dissipation, and so it can handle systems with nonzero line conductance~\cite{spanias2018system,siahaan2024decentralized}. The main issue in its application to transient stability is that the nonlinear transformations between different $dq$-reference frames of the generators is a major obstacle for verifying passivity~\cite{caliskan2014compositional}.

Last but not least, the direct method for transient stability analysis~\cite{varaiya1985direct,chiang1987foundations,rimorov2018approach,cheng2021transient} consists of (i) a model of the power system with simplifications, and (ii) the associated energy function (a function of voltage angles and frequencies), which decreases in time monotonically. Under certain conditions, the energy function becomes locally a Lyapunov function, but their differences are seen from the following aspects:
\begin{itemize}
    \item To define the energy function, the periodic angle domain is unrolled into a real domain.\footnote{We mean that the $2\pi$ equivalence of the domain $\mathbb{T}^n$ is removed so that two equivalent angle vectors in $\mathbb{T}^n$ as their difference is multiplies of $2\pi$ are no longer deemed equivalent~\cite{schiffer2019global,forni2014differential}.}  
    This explains why the input power, which is non-conservative, can seemingly be ``integrated'' to get a linear function of the unrolled angles, i.e.,
    \begin{equation*}
        \int P_m \Delta \omega\, dt = P_m \delta.
    \end{equation*}
    Then, by designating a reference bus of $0$ voltage angle, the steady state becomes an isolated equilibrium point.\footnote{If a reference bus is not chosen, the system has angle symmetry, which causes a set of equilibrium points whose angles are rigidly shifted. To deal with this problem, usually a reference angle rotating at the average frequency is chosen~\cite{varaiya1985direct}. However, this does not work without the constant impedance assumption.} %Given that this equilibrium point is locally asymptotically stable,\footnote{Without this condition, it is almost impossible to establish that the region of attraction is nonempty. See the same idea used later in~\cite{schiffer2014conditions}.} the goal is to find the region of attraction of this equilibrium point.
    \item The stability of the equilibrium point is not implied from the energy function because of its boundedness. Instead, given that this equilibrium point is locally asymptotically stable,\footnote{Without this condition, it is almost impossible to establish that the region of attraction is nonempty. See the same idea used later in~\cite{schiffer2014conditions}.}
    an estimated region of attraction can be obtained by calculating the values of the energy function at those unstable equilibrium points located on the boundary of the region of attraction. Then the minimum of the function values is used for estimating the critical clearly time for a fault given the post-fault initial condition~\cite{cheng2021transient}.
\end{itemize}

The limitation of the direct method, which prevents its wider application, is the difficulty in finding the unstable equilibrium points on the boundary of the region of attraction. It is worth noting that the estimated region of attraction from the direct method is unbounded~\cite{chiang1987foundations} compared to the bounded estimate from angle stability~\cite{dai2024practical}.

%The task to find all the unstable equilibrium points on the stability boundary is the main obstacle for applying the energy function method in practice. Moreover, the theory assumes that limit cycles do not exist on the stability boundary, which is true for the simplified model but prohibitively difficult to verify for a more detailed model.

An important limitation that is shared by almost all existing methods is that they inadvertently focus on the slower electromechanical dynamics. In an electromechanical model, the dynamics of the inductors and capacitors are replaced by the impedance value at a certain frequency. In contrast, in the full-order equations for the inductor and capacitor in (\ref{E:inductor}) and (\ref{E:capacitor}) in a stationary reference frame, there is a lack of frequency-dependent terms. This is because the steady-state frequency, in fact, an incidence of an energy balance in the system. For example, the value of the steady-state frequency changes as soon as the load is changed slightly. Therefore, based on the electromagnetic model, the equilibrium of the system is not an equilibrium point, but rather an energy-balancing limit cycle. If an infinite bus is present, the frequency of the limit cycle is equal to the frequency of the infinite bus. If an infinite bus is not present, the frequency of the limit cycle is determined autonomously.

By viewing the power system transient stability problem as checking the ability of the system to reach a certain energy balance, it is reasonable to ask the question: \emph{Whether there is a certain class of systems such that its Hamiltonian (total stored energy) is convergent to the same value (not necessarily zero) along every solution?}

%the existence of a limit cycle of a certain constant Hamiltonian (total stored energy) implies that the Hamiltonian along every solution converges to the Hamiltonian at the limit cycle.} %In other words, whether the Hamiltonian would tend to a given balance at the limit cycle as the internal energy source and dissipation  eventually cancel each other.

It will be shown in this paper that the answer to the above question is affirmative. In particular, the class of time-varying port-Hamiltonian system with constant damping matrix and strictly convex Hamiltonian verifies that the Hamiltonian along every trajectory converges to the same value (the converging Hamiltonian principle in Proposition~\ref{prop_convergence}). %Note that this is not obvious because it is possible for a dissipative time-varying system energy source to enter into an oscillatory Hamiltonian. 
It is implied from a more fundamental property called contraction in the quotient space, which is developed first in Section~\ref{sec_contraction}. 
Then, by modeling the electromagnetic dynamics of the power system as a time-varying port-Hamiltonian system in Section~\ref{sec_stability}, we obtain converge of the Hamiltonian, which combined with the constant (not depending on the state) network structure of the power system yields the global attractivity of the limit cycle. %This result is consistent with a previous result on the global stability of the electromechanical model~\cite{schiffer2019global}. 
Finally, in Section~\ref{sec_test}, electromagnetic simulation of a two-machine power system from random initial conditions confirms the theoretical results, and the existence of a synchronized limit cycle is identified as the main challenge in operating AC power systems.
The contributions of this paper are summarized as follows:

\begin{itemize}
    \item The global attractivity of the limit cycle steady state of the electromagnetic power system model is proved based on the converging Hamiltonian principle. The compositional feature of the method provides a general framework for cooperation of synchronous generator and power electronics control.
    \item A large class of time-varying port-Hamiltonian system is proved to be contractive in a special quotient space. Only the constant positive-definiteness of the damping matrix and the strict convexity of the Hamiltonian are assumed, which enables its wider application.
    \item The numerical example shows that several instability concepts in power engineering are related to the nonexistence of a synchronized limit cycle, rather than related to the non-attraction of its orbit.
\end{itemize}

%The rest of this paper is structured as follows. 
%Section~\ref{sec_background} contains backgrounds on matrix measure and pH system. Section~\ref{sec_contraction} contains the definition and theoretical results on contraction in the quotient space. Section~\ref{sec_stability} contains a detailed analysis of the electromagnetic model of a two-machine power system. Section~\ref{sec_test} provides a simulated case study. Section~\ref{sec_conclusion} is the conclusion.