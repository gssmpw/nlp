\section{Conclusion} \label{sec_conclusion}

This paper presents a theoretical analysis of the novel CQS property of the time-varying pH system and applied it to the stability analysis of the electronmagnetic model of the power system. It is found that, if the system has a synchronized limit cycle, then its orbit is globally convergent. In the case study of a two-machine system, we verified this stability result and identified that several instability concepts in traditional power engineering are related to the nonexistence of the synchronized limit cycle. The contributions of this paper is threefold. Firstly, it provides a rigorous analysis of power system stability, which unifies the common instability phenomena in power systems. It is identified that the main challenge in the operation of future AC power systems with high power electronics penetration is the existence of a synchronized limit cycle.
%Secondly, it identifies the main challenge in the operation of the future power system as the nonexistence of a synchronized limit cycle. 
Secondly, the converging Hamiltonian principle provides an elegant way to characterize the stability of systems with power sources, exhibiting limit cycle behavior. 
Thirdly, the horizontal contraction property of time-varying pH system is applicable to the stabilization of periodic motions in other areas such as robotics and multi-agent systems.