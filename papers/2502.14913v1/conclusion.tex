% 在本文中,我们提出了OpenSearch-SQL,这是一种通过动态Few-shot和一致性对齐机制来改进Text-to-SQL任务性能的方法。首先,为提高模型对Prompt的处理效果,我们使用大模型扩展原始示例并补充few-shot信息,构建了Query-CoT-SQL形式的few-shot。据我们所知,这是首次在Text-to-SQL任务中探索使用LLM扩展few-shot的CoT内容的方法。同时,我们首次开发了一种基于一致性对齐机制来减弱幻觉的方法,通过重新整合Agent的输入输出,提升了输出质量。最后,我们的方法不依赖任何SFT任务,完全基于直接可用的LLMs和检索模型,是一种关于Text-to-SQL架构方法的纯升级工作。
% 结果上来说,在提交时我们在BIRD榜单上的三项指标都取得了第一,这证明了我们方法的显著优势。
% 我们的源码将在论文发表时公开,我们希望这种Few-shot构建方式和基于一致性对齐的工作流,为Text-to-SQL及其他多Agent协作任务提供新的视角和积极的影响。
In this paper, we introduce OpenSearch-SQL, a method that enhances the performance of the Text-to-SQL task using dynamic Few-shot and consistency alignment mechanisms. To improve the model's handling of prompts, we extend the original examples with a LLM and supplement them with CoT information, forming a Query-CoT-SQL Few-shot configuration. To the best of our knowledge, this is the first exploration of using LLMs to extend Few-shot CoT content in the Text-to-SQL task. Additionally, we developed a novel method based on a consistency alignment mechanism to mitigate hallucinations, enhancing the quality of agent outputs by reintegrating their inputs and outputs. Our approach does not rely on any SFT tasks and is entirely based on directly available LLMs and retrieval models, representing a pure architecture upgrade for Text-to-SQL. As a result, at the time of submission, we achieved the top position in all three metrics on the BIRD leaderboard, which demonstrates the significant advantages of our approach.

Our source code will be made available soon, we hope that this Few-shot construction method and consistency alignment-based workflow will offer new perspectives and positive impacts for Text-to-SQL and other multi-agent collaborative tasks.