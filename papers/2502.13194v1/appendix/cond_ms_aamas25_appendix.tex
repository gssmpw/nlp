%%%%%%%%%%%%%%%%%%%%%%%%%%%%%%%%%%%%%%%%%%%%%%%%%%%%%%%%%%%%%%%%%%%%%%%%

\newpage
\appendix

\section{Analysis of Message Propagation in Conditional Max-Sum}\label{sec:cond_ms_message_prop}
    
In this section we focus on the effect that the revised update has on the propagation of $q$ messages.
Let us revisit Eq. 4 (from the main text), and decompose the sum over all $q$ messages to $2$ separate sums depending on whether $x_k \in \mathbf{s}_j^{t,e}$ or $x_k \in \mathbf{s}_{j*}^{t-,e}$:

\begin{align}\label{eq:cms_r_t_p_dec}
    r^t_{j \rightarrow i} (x_i) = \max_{\mathbf{s}_j^{t,e}} \Big  \lbrack F^t_j(\mathbf{s}_j) + \sum_{x_k \in \mathbf{s}_j^{t,e}} q^{t-1}_{j \rightarrow k}(x_k) + \sum_{x_k^* \in \mathbf{s}_{j*}^{t-,e}} q^{t-1}_{j \rightarrow k}(x_k^*) \Big \rbrack
\end{align}

\noindent where the summation term with the elements in $\mathbf{s}_{j*}^{t-,e}$ is not affected by the maximization operator. Therefore, we can place it outside the max operator:


\begin{align}\label{eq:cms_r_t_p_dec_out}
    r^t_{j \rightarrow i} (x_i) = \max_{\mathbf{s}_j^{t,e}} \Big  \lbrack F^t_j(\mathbf{s}_j) + \sum_{x_k \in \mathbf{s}_j^{t,e}} q^{t-1}_{j \rightarrow k}(x_k) \Big \rbrack + \sum_{x_k^* \in \mathbf{s}_{j*}^{t-,e}} q^{t-1}_{j \rightarrow k}(x_k^*) 
\end{align}

\noindent Before advancing, we define: 

\begin{eqnarray}\label{eq:r_t_max}
    r^{t,max}_{j \rightarrow i} (x_i) = \max_{\mathbf{s}_j^{t,e}} \Big  \lbrack F^t_j(\mathbf{s}_j) + \sum_{x_k \in \mathbf{s}_j^{t,e}} q^{t-1}_{j \rightarrow k}(x_k) \Big \rbrack 
\end{eqnarray}

\noindent and:
\begin{eqnarray}\label{eq:r_t_star}
    r^{t,*}_{j \rightarrow i} =  \sum_{x_k^* \in \mathbf{s}_{j*}^{t-,e}} q^{t-1}_{j \rightarrow k}(x_k^*) 
\end{eqnarray}


\noindent for reasons of compactness. Note that the sum outside of the max operator does not depend on $x_i$, meaning that it is an added constant value independent of $x_i$. 
Since this value is the same across all different $x_i$ inputs, then we can already deduce that these propagated values do not affect the decision making of agents.
Below, we examine this claim in more mathematical terms.

Applying Equations~\ref{eq:r_t_max}\&~\ref{eq:r_t_star} to~\ref{eq:cms_r_t_p_dec_out}, we have:

\begin{eqnarray}\label{eq:cms_r_t_p_dec_comp}
    r^t_{j \rightarrow i} (x_i) = r^{t,max}_{j \rightarrow i} (x_i) + r^{t,*}_{j \rightarrow i} 
\end{eqnarray}

\noindent Then, applying Eq.~\ref{eq:cms_r_t_p_dec_comp} to the $q$ messages update (Eq.~2 from the main text):


\begin{eqnarray}\label{eq:q_ij_t_a}
    q^t_{i \rightarrow j}(x_i) &=& c_{ij} + \sum_{k \in M_i \setminus j} \Big  \lbrack r^{t,max}_{j \rightarrow i} (x_i) + r^{t,*}_{j \rightarrow i}  \Big  \rbrack \nonumber \\
    q^t_{i \rightarrow j}(x_i) &=& c_{ij} + \sum_{k \in M_i \setminus j} \Big  \lbrack r^{t,max}_{j \rightarrow i} (x_i) \Big  \rbrack + (|M_i|-1)\cdot r^{t,*}_{j \rightarrow i}
\end{eqnarray}

Since the normalization constant is calculated to satisfy:\\ $\sum_{x_i} q^t_{i \rightarrow j} (x_i) = 0$ in cyclic graphs, then solving for $c_{ij}$, we get:


\begin{eqnarray}\label{eq:c_ij_cond}
    c_{ij} &=& -\frac{1}{|X_i|} \cdot \sum_{x_i} \Big  \lbrack \sum_{k \in M_i \setminus j} \big  \lbrack r^{t,max}_{j \rightarrow i} (x_i)  \big  \rbrack  + (|M_i|-1)\cdot r^{t,*}_{j \rightarrow i}\Big  \rbrack \nonumber \\
    c_{ij} &=& -\frac{1}{|X_i|} \cdot \bigg \lbrack \sum_{x_i}  \Big  \lbrack \sum_{k \in M_i \setminus j} \big  \lbrack r^{t,max}_{j \rightarrow i} (x_i)\big  \rbrack \Big  \rbrack  + \nonumber \\ &+& |X_i|\cdot (|M_i|-1)\cdot r^{t,*}_{j \rightarrow i} \bigg \rbrack  \nonumber \\
    c_{ij} &=& -\frac{1}{|X_i|} \cdot \bigg \lbrack \sum_{x_i}  \sum_{k \in M_i \setminus j} r^{t,max}_{j \rightarrow i} (x_i) \bigg \rbrack  - (|M_i|-1)\cdot r^{t,*}_{j \rightarrow i} 
\end{eqnarray}

\noindent where $|X_i|$ is the number of discrete values for $x_i$, meaning that the constant added term $r^{t,*}_{j \rightarrow i}$ on $r^t_{j\rightarrow i}$ messages is in fact never communicated to other agents.
Moreover, it does not affect the decision update of the agent either since it is constant: 

\begin{eqnarray}
    x_i^* &=& \arg \max_{x_i} \sum_{j \in M_i} r^t_{j \rightarrow i} (x_i) \nonumber \\
    &=& \arg \max_{x_i} \sum_{j \in M_i} \lbrack r^{t,max}_{j \rightarrow i} (x_i) + r^{t,*}_{j \rightarrow i} \rbrack \nonumber \\
    &=& \arg \max_{x_i} \big \lbrack \sum_{j \in M_i} r^{t,max}_{j \rightarrow i} (x_i) + \sum_{j \in M_i} r^{t,*}_{j \rightarrow i} \big \rbrack \nonumber \\
    &=& \arg \max_{x_i} \sum_{j \in M_i} r^{t,max}_{j \rightarrow i} (x_i)
\end{eqnarray}

In practice, we can disregard the $q$ messages sent from agents not complying with the time estimate criterion, without affecting the result.
As such, in the case of factors connecting two agents, the value $q^{t-1}_{k\rightarrow j}(x_k^*)$ could be omitted from Eq.3 (from the main text).
Therefore, when the agent chooses not to maximize, it essentially bounds the associated factor's input element with the sent value for $x_k^*$, and locally maximizes over the connected factors when it updates its decision on $x_i^*$.
This reflects distributed local search algorithms in the context of DCOPs, specifically MGM~\cite{mgm_paper}, which is the non-stochastic variant of the popular baseline DSA~\cite{dsa_paper} algorithm for DCOPs.
We examine the efficacy of this in our experimental evaluation through the No-Max-Sum variant, where the agents never maximize over the others' variables and only rely on the broadcasted assignments. 





\section{Lane-Free Traffic Environments with Dynamic Lateral Regions}\label{lf:d_lr}


In this section we present in detail our formulation of lane-free environments that enables lane-free lateral operations through a Dynamic DCOP.
The primary tool that we design to enable coordination in a manner suitable for strategic coordination in lane-free traffic is that of {\em dynamic lateral regions}.
With this, the vehicles can interpret the observed and/or communicated information from nearby traffic in a way that provides a {\em real-time structured representation} of the environment, and decide upon low-level operations regarding their acceleration and account for safety.

\subsection{Formation of Lateral Regions}\label{subsec:lat_regions_form}
This process initiates from the perspective of a single ego vehicle $i$. First, we distinguish between upstream (vehicles on the back of $i$) and downstream traffic (vehicles on the front of $i$).
For downstream traffic, we scan all vehicles according to a specified observation distance ${obs}_x$ longitudinally (in front), and construct the lateral region that each foreign vehicle introduces as illustrated in Fig.~\ref{fig:lat_regions_downstream}.
The partitioning of the regions is done based on the monitored information from each observed vehicle $k$, namely its: lateral placement on the road $y_k$, lateral speed $v_{y,k}$ and physical dimensions on the lateral axis, i.e., its width $w_k$.
This is visually shown in Fig.~\ref{fig:lat_regions_detail}, where from the perspective of the white ego vehicle ($i$), a lateral region due to $k$ is formed according to this information, and accounting for safety related gaps, a distance $y_{safe}$, and a time-gap $T_y$ that is only added when the vehicle has a lateral speed towards this direction: $v_{vk,right}= -v_{y,k}\cdot (v_{y,k}<0)$, and : $v_{vk,left}= v_{y,k}\cdot (v_{y,k}>0)$.\footnote{Positive lateral speed indicates movement towards the left direction, while negative values that the vehicle is moving right.} 
The constructed regions correspond to where the center point of the ego vehicle can be positioned, hence the reason we see the utmost left and right regions not reaching the road's physical boundaries as they take into account the ego's width.



\begin{figure}[t]
    \centering
    \includegraphics[width=0.48\textwidth]{figures_appendix/lat_regions_downstream.png}
    \caption{Formed lateral regions from downstream traffic the perspective of agent $i$.}
    \label{fig:lat_regions_downstream}
    \Description{Visual illustration of the lateral regions' formation due to traffic located in front of the agent/vehicle we focus at.}
\end{figure}

At any time, the ego vehicle is located at a specific lateral region, in which we consider that its longitudinal behaviour is being influenced by the front vehicle occupying this region.
As such, the longitudinal control (gas/brake) of the vehicle can be decided according to a car-following method as done typically in lane-based environments, with the vehicle in front as the leader to follow.
For this task, we employ the Enhanced Intelligent Driver Model (EIDM)~\cite{kesting2010enhanced}, which is an extension of one of the most popular car-following methods, that calculates the acceleration of the vehicle $i$ on the longitudinal axis (corresponding to a gas/brake response), taking into account its desired speed $v_i^d$ while respecting a desired time-gap value $T_x$ with the vehicle in front to avoid critical situations.
In this manner, given two vehicles $(i,j)$ with $j$ being in front of $i$, we can calculate the acceleration $a_{i, j}$ of $i$ when $j$ is in front according to EIDM.


\begin{figure}[b]
    \centering
    \includegraphics[width=0.48\textwidth]{figures_appendix/lat_regions_downstream_zoom.png}
    \caption{Example of lateral region formation based on the lateral information of observed vehicles.}
    \label{fig:lat_regions_detail}
    \Description{Detailed view of the parameters that affect the reach of each region.}
\end{figure}

Likewise, we can calculate this acceleration for each lateral region, meaning we can have an estimate of the acceleration of ego vehicle depending on its lateral placement.
These acceleration estimates are instrumental in our approach, as they quantify the {\em value} of residing at a lateral region, and consequently the {\em benefit} of shifting laterally to a different region by simply comparing the corresponding acceleration evaluations.
The calculated accelerations also provide information about the criticality of any selected positioning, e.g., a negative value indicating strong braking will be calculated for a lateral region containing a vehicle being too close or/and operating at a significantly lower speed than ego.
At the same time, the acceleration values solely reflect our vehicle's desired speed goal whenever there is no need for gap maintenance with the front vehicle.


An important characteristic in lane-free traffic is that of nudging, meaning that vehicles' behaviour can be influenced by surrounding vehicles located upstream.
We can directly incorporate nudging for this formulation by constructing a second set of lateral regions now from upstream traffic by following exactly the same logic. 
Then, a ``follower'' vehicle $k$ can be additionally specified according to the ego's position w.r.t. to the formed lateral regions from the vehicles in the back, as shown in Fig.~\ref{fig:lat_regions_viz}.
Finally, we can similarly obtain estimates for the acceleration but now from the perspective of the vehicle on the back wrt to our ego, i.e., $a_{k,i}$.
From this perspective, the acceleration associated with each lateral region provides us with how the ego's  lateral alignment affects these vehicles, e.g., a significant negative response indicates that the vehicle on the back is either too close or attempts to overtake.

Note that the regions introduced by each vehicle can occupy the same lateral space. 
For this, we give priority to the vehicle with the lowest acceleration estimate (which is typically the one closest to $i$), e.g., in Fig.~\ref{fig:lat_regions_viz}, the blue region by vehicle $l$ occupies lateral space that would be otherwise part of $m$'s lateral region.


\begin{figure}[b]
    \centering
    \includegraphics[width=0.48\textwidth]{figures_appendix/lat_regions_all.png}
    \caption{Formed lateral regions partitioned by downstream and upstream traffic for agent $i$.}
    \label{fig:lat_regions_part}
    \Description{Visual illustration of the lateral regions' formation due to surrounding traffic (front and back) from the perspective the agent/vehicle we focus at.}
\end{figure}

\subsection{Regulation of Lateral Alignment in Lane-Free Traffic}\label{subsec:reg_lat_align}


With the formulation of lateral regions as described above, we ``filter out'' potentially unsafe decisions by examining the feasibility of a lateral shift from one lateral alignment to the next. 
In order to rely on a common indexing $r$ for downstream and upstream traffic, we further partition downstream and upstream lateral regions, as shown in Fig.~\ref{fig:lat_regions_part} with the white colored numbers at the left indicating the indexing of the respective region.
Given the current lateral position $y_i$ of ego vehicle and a desired lateral alignment $y_i^*$ from a policy, we can directly determine the desired lateral region 
$r^*$ for that position and assign the closest {\em safe} region $r_d$ that conforms to the following safety criterion.

For every lateral region $r$ between the current region $0$ and the desired one $r^*$, we examine whether the shift is feasible based on the acceleration estimate for the vehicles upstream $a_{\cdot,i}$ and downstream $a_{i,\cdot}$.
This establishes that the lateral shift will not cause a critical situation with the vehicles on that direction.
The lateral shift towards a desired region is considered safe only if the following condition for a {\em safe} deceleration value $b_{safe}$ is satisfied for all intermediate regions, i.e., $a_{\cdot,i} \geq -b_{safe}$.
For instance, if $i$ in Fig.~\ref{fig:lat_regions_part} wishes to laterally shift towards region $-3$, then we require that both $a_{o,i}\geq -b_{safe}$ and $a_{n,i}\geq -b_{safe}$, along with $a_{i,k}\geq -b_{safe}$.
If this condition is not met, then we select the region closest to the initial goal that satisfies it.
As such, we obtain a {\em safe} lateral region $r_d$ after this procedure (worst case scenario is that $r_d=0$, i.e., we can only move laterally within the limits of our current region).
Then, the actual desired lateral positioning is calculated as:

\begin{equation}
    y_i^d = \max {\big \{ \min {\{ y_i^{*},y_h(r_d)-y_{thr}\}}, y_l(r_d) + y_{thr} \big \}}
\end{equation}

\noindent where $y_h(r_d),y_l(r_d)$ are the highest and lowest lateral point of the selected lateral region $r_d$, and $y_{thr}$ is a small threshold value so that the vehicles do not reside at the upper or lower limit of the region.
This safety rule can be viewed as a lane-free extension of the one employed in lane-based environments, where an intended lane-change can be aborted if the estimated acceleration at the destination lane does not comply with a desired deceleration value ($b_{safe}$).


\subsection{Control input of vehicles}

We now have the information on the lateral regions of an ego vehicle $i$, and its desired lateral placement $y_i^d$.
In order to control the vehicle, we need to translate this to two acceleration values $a^x_i,a^y_i$ in $m / s^2$.

For the longitudinal acceleration (gas/brake pedal), we simply determine the ``leader'' vehicle in front of us, along with the ``follower'' depending on the current lateral region the ego occupies (downstream and upstream), e.g., for region $0$ in Fig.~\ref{fig:lat_regions_part}, leader is vehicle $m$ while follower is $o$.
The final acceleration value is calculated through the EIDM model, but with a significant difference that incorporates nudging behaviour, i.e., the influence of the vehicle on the back as well.
Such extensions on IDM are also investigated in related work~\citep{YI2022127606} for lane-based traffic.
Especially in the lane-free traffic domain, the influence of nudging is quite significant~\citep{lane_free_journal}, and has a substantial impact on preliminary empirical results with our approach as well.
Therefore, the longitudinal acceleration $a^x_i$ of each vehicle $i$ with leader vehicle $j$ and vehicle $k$ on the back is calculated as:

\begin{equation}
    a^x_i = a_{i,j}+ \alpha \cdot \gamma \frac{s^*_{ki}}{s_{ki}}^2 
\end{equation}

\noindent where the second term contains the ``reaction'' of $i$ due to the presence of vehicle $k$ behind.
This term is directly taken by the IDM model, where $s^*_{ki}$ is the desired longitudinal distance between $k$ and $i$ according to a provided time-gap value $T_x$, and $s_{ki}$ is the monitored one.

For instance, if $k$ intends to decelerate due to $i$'s presence, this term will also cause $i$ to accelerate, therefore mitigating the reaction of $k$.


For the lateral acceleration $a_i^y$ (left/right movement), we calculate based on a simple PD controller~\cite{kuo1995automatic} that drives the vehicle towards its desired lateral alignment $y_i^d$ while avoiding unnecessary oscillatory behaviour. This is accomplished with the following form:

\begin{equation}\label{eq:ay_update}
    a^y_i = K_p \cdot (y_i^d - y_i) + K_d \cdot (-v_{y,i})
\end{equation}

\noindent where $K_p,K_d$ are the proportional and derivative gains respectively, calibrating how fast and smoothly the vehicle reacts.
Essentially, the PD controller automates how the vehicle actually targets the desired lateral position $y_i^d$ in a continuous space without causing unstable behaviour.



\section{Multiagent Coordination in Lane-Free Traffic (Supplementary Material)}\label{sec:mas_lf2}




\subsection{Pairwise Factors (with Supplementary Material)}\label{subsec:fg_lf2}


The FG formulation contains a second type of a pairwise factor $F_{p}(x_i,x_j)$ that connects two vehicles $i$ and $j$, with $j$ preceding $i$. Its presence serves to motivate both $i$ and $j$ at moving laterally according to $i$'s {\em desire} to overtake through {\em regret minimization}.
Notably, since the factor affects both $i$ and $j$'s decision due to their involvement, they can accordingly control their lateral behaviour in a coordinated manner.
This is accomplished by the following formulation:


%
\begin{align}\label{eq:f_ij_s}
    F_{p}(x_i,x_j) = regret(x_i,x_j) + comfort(x_i,x_j)
\end{align}
%

\noindent where the first term is the calculated {\em regret} from the perspective of the receding agent $i$, and has the following form:


\begin{equation}\label{eq:regret_s}
    regret(x_i,x_j) = -R_c\cdot (a_{i,free} - a_{i, j})^2 \cdot overlap_{ij}(x_i,x_j)
\end{equation}

\noindent and the value $a_{i,free}$ is the calculated acceleration of $i$ when located in lateral regions without a leader, meaning that this value only accounts for the desired speed objective of the agent and that $a_{i,free}\geq a_{i,j}$.
As such, the difference $(a_{i,free} - a_{i, j})$ expresses the {\em regret} of agent $i$ for having $j$ {\em in front} of it,\footnote{$i$ observing $j$ as leader for the examined configuration $y_i',y_j'$.} using a positive coefficient $R_c$ and a negative sign to comply with the algorithm's maximization criterion.
Note that whenever this type of factor connects two vehicles, this regret value is assigned to it only for configurations of $x_i,x_j$ that would result in the agents having their lateral alignments ``overlap'' at any point during lateral deviation from the current placement $y_i,y_j$ towards the examined one $y_i',y_j'$, i.e.:


%
\begin{align}
    overlap_{ij}(x_i,x_j) = 
    \begin{cases}
        1 & \text{if }  \; inRange_{ij}(y_i, y_j, y_i',y_j')\\
        0.75 & \text{if } onlyOverlap_{ij}(y_i, y_j, y_i',y_j')\\
        0    & \text{otherwise.}
    \end{cases}
\end{align}
%


\noindent where the $inRange_{ij}$ operator checks whether the desired configuration $y_i',y_j'$ results in the two vehicles observing one another in the same lateral region when the vehicles have reached  $y_i', y_j'$.
Instead, a smaller percentage of the regret is assigned if the examined combination of lateral alignments $y_i', y_j'$ does not result in the vehicles at the same lateral region, but they still have to laterally overlap one another, e.g., if initially $j$ is right of $i$ but the examined configuration results in $j$ being left of $i$.
If neither of the above holds true, then no regret is assigned to the factor.



\subsection{Formation of Pairwise Connections}\label{subsec:form_pairwise}

Each agent is associated with one single factor, as illustrated in Fig.~\ref{fig:connected_vehs}.
A pairwise factor $F_{p}(x_i,x_j)$ between $i,j$ is formed when the agents can observe one another based on the observational distance ${obs}_x$, and according to their lateral distance $dy$, based on the following condition:

\begin{equation}
    dy \leq C_{range}\cdot y_{range} + y_{safe}
\end{equation}

\noindent where $dy$ is the lateral distance of the vehicles accounting for their respective widths, and $y_{range}$ is the selected range of lateral deviation for the control variables $x_i$.
For instance, if $y_{range}=3m$, then the lateral deviation takes values within the range $x_i=[-3m,3m]$, resulting in each vehicle being able to move towards the left or right direction by $3$ meters with respect to its current lateral position.
Then, $C_{range}$ is a positive coefficient, and $y_{safe}$ is the safety distance gap also used for the design of lateral regions (see Sec.~\ref{subsec:lat_regions_form} in this document).



Additionally, each agent $i$ prunes its own connections with other agents in order to conform to upper limits $N_{front}^{max}, N_{back}^{max}$ regarding the number of pairwise connections it can have downstream and upstream, respectively. 
The pruning operation for agent $i$ is performed by ordering the candidate pairwise factors $F_{p}(x_i,x_j)$ downstream according to the acceleration values $a_{i, j}$ in ascending order, and only forming the first $N_{front}^{max}$ connections.
The same procedure is followed for upstream connections $F_{p}(x_j,x_i)$, but now according to the acceleration values $a_{j,i}$.
Note that each agent performs this operation from its own perspective of the problem, and each pairwise factor is formed only if both agents agree to have this connection after the pruning process. 
This procedure is part of line 2 in Alg.1 of the main text. 



\subsection{Time-estimates update (Supplementary}\label{sec:async_max_sum_lf2}


The underlying movement dynamics of the vehicles in our case is the double-integrator model, meaning that at every new time-step $t$, the lateral position $y_i$ of the vehicle is updated as:

\begin{equation}\label{eq:y_i}
    y_i = y_i + v_{y,i}\cdot dt + 0.5\cdot a^y_i \cdot dt^2
\end{equation}

\noindent and:


\begin{equation}\label{eq:v_yi}
    v_{y,i} = v_{y,i} + a^y_i\cdot dt
\end{equation}

\noindent where $dt$ is the discrete time-step length for the simulation environment.
The PD controller in Eq.~\ref{eq:ay_update} contains the lateral position $y_i$ and speed $v_{y,i}$ of the vehicle. 
Therefore, we can directly predict in how many time-steps the desired lateral positioning $y_i^d$ will be reached since we know the model of the vehicle's movement dynamics.
More specifically, we iteratively apply Eqs.~\ref{eq:y_i},~\ref{eq:v_yi} and Eq.~\ref{eq:ay_update} until the desired position is reached within a small $y_e$ distance, i.e., $|y_i^d - y_i | \leq y_{e}$.






\section{Experimental Evaluation}\label{sec:exp_eval2}



\subsection{MOBIL as a baseline method in Lane-Free Traffic}

MOBIL~\cite{kesting2007mobil} is one of the most well-known  methods that provide a policy for lane-changing behaviour.
MOBIL models how humans would decide to change lanes, with parameters that calibrate its behaviour regarding the vehicle's eagerness to change lane whenever an opportunity on an adjacent lane arises; or how polite it is to upstream traffic, i.e., the vehicle performing a lane change not necessarily for its own benefit, but taking into account how this lane-change would affect the vehicles located upstream on the current and the candidate lane.
This notion can be directly translated to lane-free traffic environments by replacing the adjacent lanes with adjacent lateral regions of the vehicle. 
Given a candidate lateral region $r$ and the currently residing lateral region $r_0$, we examine the potential lateral shift from $r_0$ to $r$ with the following condition:

\begin{equation}\label{eq:mobil}
    a_i(r) - a_i(r_0) > p [(a_u^{-i}(r)-a_u^i(r))+(a_u^i(r_0)-a_u^{-i}(r_0))]+ a_{thr}
\end{equation}

\noindent where $a_i(r), a_i(r_0)$ are the acceleration values for ego vehicle $i$ on the candidate $r$ and current $r_0$ lateral regions respectively. Then, $a_u^i(r),a_u^{-i}(r)$ are the acceleration estimates for the vehicle $u_r$ located at the candidate region $r$ upstream of $i$, accounting for the two possible outcomes: $a_u^i(r)$ corresponds to $i$ performing the lateral shift, meaning that $u_r$'s acceleration will be influenced by $i$; and $a_u^{-i}(r)$ corresponds to $i$ remaining at the current region $r_0$, meaning that $u_r$ will have the current vehicle in front.
As such, the term $(a_u^{-i}(r)-a_u^i(r))$ is the acceleration impact on vehicle $u_r$ located at the candidate region $r$ if the lateral shift takes place.
Likewise, $a_u^i(r_0),a_u^{-i}(r_0)$ expresses the same estimates, but for the vehicle $u_{r_0}$ located upstream of $i$ at the current lateral region $r_0$. 
Therefore, the term $(a_u^i(r_0)-a_u^{-i}(r_0))$ is the impact of the lateral shift of $i$, but for the vehicle $u_{r_0}$.

Moreover, the influence of these two terms regarding $i$'s decision is calibrated with a {\em politeness} coefficient $p$.
Typical values of $p$ are within the range $[0,1]$, with $0$ expressing that vehicle $i$ is not influenced by the impact of its lateral shift for the affected upstream vehicles, and $1$ the exact opposite, namely that vehicle $i$ will perform the lateral shift only if the total benefit (its own and for the two vehicles upstream $u_r,u_{r_0}$) is positive.
Finally, there is a non-negative threshold value $a_{thr}$ calibrating the minimum benefit $i$ should observe in order to perform the lateral shift.
This is important because small values of $a_{thr}$ can cause the vehicle to have quite oscillatory behaviour, and results in nearby traffic to respond to such situations.

In a conventional lane-based environment, we would examine the two adjacent lanes of the vehicle (or one if the vehicle is next to a boundary of the road), and perform the lane-change if the condition in Eq.~\ref{eq:mobil} is met for one of them, and the safety rule (see Sec.~\ref{subsec:reg_lat_align}) regarding $b_{safe}$ is satisfied.
We extend the same principle in lane-free settings, but instead of checking only adjacent regions, we use the $y_{range}$ distance\footnote{This is the same distance that we use for the factor graph formulation in order to properly compare the different methods.} to indicate how far the vehicle can search for regions (left or right) that have an acceleration benefit, and select the region with the highest benefit that satisfies the condition along with the safety rule.


\begin{table}[t]
  \centering  
  \caption{Parameter Tuning for Factor Graph \& Max-Sum Formulation}  
  \begin{tabular}{|c|c||c|c|}  
  % \toprule
  \hline
     $N^{max}_{front}$ & 6 & $T_{min}$ & $4 sec$\\ 
    %\midrule
    \hline
    $N^{max}_{back}$ & 6 & $T_{max}$ & $6 sec$\\ \hline
    $y_{range}$ & 3.5 m & $t_e$  & $1 sec$ \\ \hline
    $|X_i|$ & 15 & $y_e$  & $0.01 m$\\ \hline
    $R_c$ & 5 & $C_{range}$& $1.25$\\ \hline
    $C_c$ & 0.05 &  $B_c$ & 12 \\ \hline    
    % \bottomrule
  \end{tabular}
  \label{tab:fg}  
\end{table}




\subsection{Technical Details}\label{subsec:tech}

All experiments are conducted with the TrafficFluid-Sim~\cite{tsim_sumo2022} lane-free microscopic simulator, an extension of SUMO~\cite{sumopaper} appropriate for lane-free environments.
The codebase was developed in C++, and since this framework operates on a single iteration per time-step, execution times are quite small as we later report for the largest setting.
Upon acceptance, we will share a proper version of the codebase with usage instructions.
Across all simulations, a discrete time-step of $dt=0.2sec$ is used.

Here, we contain information regarding the parameter tunings that are common across all scenarios.
First, in Tab.~\ref{tab:fg}, we provide the parameter settings relevant to the Factor Graph formulation and the Conditional Max-Sum algorithm, where $|X_i|$ is the number of discrete values within the range $[-y_{range},y_{range}]$ for variables $x_i$.
Then, in Tab.~\ref{tab:lr} we include the parameter settings relevant to the formation of lateral regions, EIDM for the longitudinal behaviour of vehicles and the acceleration estimates for the formulation of lateral regions, along with the tuning for MOBIL.
There, $a_{max}$ is the maximum acceleration (longitudinal, gas behaviour)  that the vehicles can have, and $b_{safe}$ is the value used both for EIDM, and for the safety rule that regulates the vehicles' lateral movement.
Moreover, $a_{severe}$ is the minimum allowed acceleration value (longitudinal, brake behaviour).
Vehicles with such low acceleration values indicate a ``severe'' situation (i.e., the need to react urgently in order to avoid a collision).

Additionally, $T_x$ is the time-gap value relevant to the longitudinal behaviour of vehicles, and $T_y$ is used for the lateral regions formation (cf Fig.~\ref{fig:lat_regions_detail}).
The simulation environment is discrete-time, but we report on the time-related information in seconds instead of number of time-steps.
For instance, the value of $T_{min}=4sec$ corresponds to $4/0.2=20$ time-steps.
Finally, ${obs}_x$ is the observational distance of each vehicle, based on which it monitors nearby traffic downstream and upstream.



The baseline MOBIL method is tuned with $a_{thr}=0.8$ and politeness $p=0.5$ after preliminary empirical investigations. We selected the values that better balanced jerk behaviour and the desired speed objective.
Without careful tuning, MOBIL can exhibit quite intense jerk behaviour especially for more aggressive settings ($a_{thr},p$ values close to $0$) due to the dynamic nature of the problem and the lack of communication.
At the same time, we never observed significant improvement in terms of desired speed objective, especially in intense settings.







\begin{table}[b]
  \centering
  
  \caption{Parameter Tuning relevant to Lateral Regions, EIDM, and MOBIL} 
  \begin{tabular}{|c|c||c|c|}  
  % \toprule
  \hline
     $a_{max}$ & $1.5 m/s^3$ & $a_{thr}$ & $0.8$\\ 
    %\midrule
    \hline
    $b_{safe}$ & $2 m/s^3$ & $p$ & $0.5$\\ \hline
    $a_{severe}$ & $-5 m/s^3$ &   $x_{safe}$ & $0.3m$\\ \hline
    $T_x$ & $0.4 sec$ & $y_{safe}$ & $0.2m$ \\ \hline
    $T_y$ & $0.4 sec$ &  $\gamma$ & $0.7$  \\ \hline
    ${obs}_x$ & $30m$ & & \\ \hline   
    % \bottomrule
  \end{tabular}
  \label{tab:lr}  
\end{table}



\begin{figure*}[t]
    \centering
    \includegraphics[width=1\textwidth]{figures_appendix/open_highway.png}
    \caption{Snapshot of the first $250m$ of the open highway.}
    \label{fig:highway_open}
    \Description{Top-down view of the first 250 meters of the open highway, populated with lane-free vehicles.}
\end{figure*}


\subsection{Lane-Free Coordination Problem}

In this section, we include additional details for the initialization of the problem (as visualized in Fig.~5 of the main text).
All simulations initiate from time $0 sec$, with each vehicle having a different initialization regarding its previous update, consequently affecting its time window $[T_{min}=4sec, T_{max}=6sec]$.
For instance, veh-1 has an initialization of $-1 sec$, meaning that its first time window to consider updating its variable is $[3sec,5sec]$.
Each vehicle's initial timing is reported in Tab.~\ref{tab:init_time}.
The examined highway has a length of $1km$ and a width of $10.2m$, corresponding to a conventional 3-lane highway.

In addition, we further illustrate how each separate vehicle is affected by the different methods by showcasing all speed trajectories in Figs.~\ref{fig:veh2},~\ref{fig:veh34},~\ref{fig:veh56}.
In order to give higher priority for veh-1, this of course comes at a cost for the nearby vehicles due to the limited lateral space of the road.
This is more evident in veh-2's trajectories---veh-2 is closer to veh-1, both in terms of proximity and objective--- (Fig.~\ref{fig:veh2}), where Cond-Max-Sum appears to give it a lower priority in order to accommodate veh-1 (this can also be observed in the provided video). 
For the slowest vehicles veh-5,veh-6, we see that both Cond-Max-Sum and the standard Max-Sum algorithm provide similar results, with the remaining methods exhibiting worse behaviour.
Then, for vehicles veh-3,veh-4, which have a moderate desired speed objective, we see stronger fluctuations due to the fact that they need to both assist the faster vehicles veh-1,veh-2 in their overtake, and also overtake the slower vehicles veh-5,veh-6. 
There, we have a less clear distinction between all examined methods, e.g., Max-Sum is the best for veh-3, but at the same time the worst for veh-4, while the opposite is true for No-Max-Sum.
Cond-Max-Sum shows the biggest drop in speed for both vehicles, evidently due to the accommodation of veh-1,veh-2 that wish to overtake, while veh-3,veh-4 gave them space by staying behind the even slower vehicles veh-5,veh-6.
While this appears worse for these two individual vehicles,
it is more beneficial for the system (based on the average results) as a whole, given the way the factors are designed.

Moreover, in Figs.~\ref{fig:pos_y_cond},~\ref{fig:pos_y_max},~\ref{fig:pos_y_neutral},~\ref{fig:pos_y_mobil}, we show how the vehicles are positioned laterally for all examined methods. As also indicated by the values of the comfort-related metric of jerk in the main text, these further support the fact that Cond-Max-Sum achieves either similar or greater performance to Max-Sum but is also consistently combined with smoother lateral maneuvers due to the enhanced coordination among agents.


\begin{table}[ht]
  \centering
  
  \caption{Initial Timings for Lane-Free Coordination Problem}
  % \scalebox{1}
  {
  \begin{tabular}{|c|c|}
  
  % \toprule
  \hline
     Vehicle & Initial Time \\ \hline \hline
    veh-1 & $-1 sec$ \\ \hline
    veh-2 & $-2 sec$ \\ \hline
    veh-3 & $-1.4 sec$ \\ \hline
    veh-4 & $-2.4 sec$ \\ \hline
    veh-5 & $0 sec$ \\ \hline
    veh-6 & $-3.6 sec$ \\ \hline
    % \bottomrule
  \end{tabular}
  }
  \label{tab:init_time}  
\end{table}


\begin{figure}
    \centering
    \includegraphics[width=0.45\textwidth]{figures_appendix/veh2.png}
    \caption{Speed trajectories for veh-2.}
    \label{fig:veh2}
    \Description{This figure displays the speed trajectories (over time) of agent veh-2. Each examined variant is displayed with different line style.}
\end{figure}



\begin{figure}
    \centering
    \includegraphics[width=0.45\textwidth]{figures_appendix/veh3_veh4.png}
    \caption{Speed trajectories for veh-3 and veh-4.}
    \label{fig:veh34}
    \Description{This figure displays the speed trajectories (over time) of agents veh-3 and veh-4. Each examined variant is displayed with different line style.}
\end{figure}


\begin{figure}
    \centering
    \includegraphics[width=0.45\textwidth]{figures_appendix/veh5_veh6.png}
    \caption{Speed trajectories for veh-5 and veh-6.}
    \label{fig:veh56}
    \Description{This figure displays the speed trajectories (over time) of agents veh-5 and veh-6. Each examined variant is displayed with different line style.}
\end{figure}

\begin{figure}
    \centering
    \includegraphics[width=0.45\textwidth]{figures_appendix/y_pos_cond.png}
    \caption{Lateral positioning of all vehicles with Cond-Max-Sum.}
    \label{fig:pos_y_cond}
    \Description{This figure displays the speed trajectories (over time) of agents veh-3 and veh-4. Each examined variant is displayed with different line style.}
\end{figure}

\begin{figure}
    \centering
    \includegraphics[width=0.45\textwidth]{figures_appendix/y_pos_max.png}
    \caption{Lateral positioning of all vehicles with the standard Max-Sum algorithm.}
    \label{fig:pos_y_max}
    \Description{This figure displays the trajectories (over time) of all 6 agents regarding their lateral placement. Plot lines of each agent are associated based on the color choice. The trajectories stem from the application of the standard Max-Sum variant.}
\end{figure}


\begin{figure}
    \centering
    \includegraphics[width=0.45\textwidth]{figures_appendix/y_pos_neutral.png}
    \caption{Lateral positioning of all vehicles with the No-Max-Sum algorithm.}
    \label{fig:pos_y_neutral}
    \Description{This figure displays the trajectories (over time) of all 6 agents regarding their lateral placement. Plot lines of each agent are associated based on the color choice. The trajectories stem from the application of the No-Max-Sum variant.}
\end{figure}


\begin{figure}
    \centering
    \includegraphics[width=0.45\textwidth]{figures_appendix/y_pos_mobil.png}
    \caption{Lateral positioning of all vehicles with the MOBIL algorithm.}
    \label{fig:pos_y_mobil}
    \Description{This figure displays the trajectories (over time) of all 6 agents regarding their lateral placement. Plot lines of each agent are associated based on the color choice. The trajectories stem from the application of the MOBIL variant that does not rely on communication.}
\end{figure}

\subsection{Open and Large Lane-Free Environments}

The examined highway environment is a $2km$ length road with a $10.2m$ width, with $1h$ of simulation time.
Execution time of Cond-Max-Sum in a $15000 veh/h$ with approximately $300$ agents was measured on average around $41.59 msec$ (milliseconds) per time-step.
With the employed discrete step of $0.2 sec=200 msec$, we have $207.95msec$ per second of simulation time.
In the open environment, we have demand flow values at $10000, 15000 veh/h$, resulting in vehicles constantly entering the highway following that flow rate.
A snapshot of the simulation environment for the examined flow of $15000 veh/h$ can be found in Fig.~\ref{fig:highway_open}, where we see a top-view of the agents near the entry of the highway, and a new one entering at the entry point on the left.

Finally, in Fig.~\ref{fig:hist_comfort}, we contain the full (histogram) plot of percentages for each jerk bin.
We observe consistent improvement of Cond-Max-Sum with respect to Max-Sum across all bins, since in the ideal case (jerk bin containing $0 m/s^3$), Cond-Max-Sum has higher percentage; whereas for all remaining bins indicating increasingly discomfort levels,  percentages are always in favor of Cond-Max-Sum.


\begin{figure*}
    \centering
    \includegraphics[width=.9\linewidth]{figures_appendix/hist_compare_all.png}
    \caption{Lateral jerk percentage (histogram) plot for Cond-Max-Sum, Max-Sum and No-Max-Sum in $15000veh/h$.}
    \label{fig:hist_comfort}
    \Description{}
\end{figure*}





\begin{table*}
  \centering
  
  \caption{Z-score and Confidence Intervals of Cond-Max-Sum and Max-Sum from speed measurements}
    \scalebox{0.75}
    {
    \begin{tabular}{lrrrrrrrrrrrr}
    \hline
     & $[0,5)$ & $[5,10)$ & $[10,15)$ & $[15,20)$ & $[20,25)$ & $[25,30)$ & $[30,35)$ & $[35,40)$ & $[40,45)$ & $[45,50)$ & $[50,55)$ & $[55,60]$ \\ 
    \midrule
    z-score & $\mathbf{2.56}$ & $\mathbf{4.01}$ & $\mathbf{3.32}$ & $\mathbf{4.82}$ & $\mathbf{8.03}$ & $\mathbf{10.30}$ & $\mathbf{6.46}$ & $\mathbf{4.60}$ & $\mathbf{8.16}$ & ${0.92}$ & $\mathbf{4.96}$ & ${1.24}$ \\
    CI (Cond-Max-Sum) & $2.92E-04$ & $1.67E-04$ & $1.53E-04$ & $1.57E-04$ & $1.88E-04$ & $1.46E-04$ & $1.70E-04$ & $2.39E-04$ & $1.82E-04$ & $2.59E-04$ & $1.21E-04$ & $1.35E-04$ \\
    CI (Max-Sum) & $2.95E-04$ & $1.43E-04$ & $1.59E-04$ & $1.82E-04$ & $2.06E-04$ & $1.54E-04$ & $1.87E-04$ & $2.28E-04$ & $1.84E-04$ & $2.44E-04$ & $1.35E-04$ & $1.46E-04$ \\
    % \bottomrule
    \end{tabular}
    }
  \label{tab:stat}  
\end{table*}


\subsection{Statistical Significance}

For the speed measurements in Fig.~\ref{fig:speed_compare}, we have additionally performed a two-sample z-test (Cond-MS vs MS) for all the speed measurements in this figure.
Based on the calculated z-score, we verify that the difference is statistically significant with $a=0.05$ for $10$ out of the $12$ 5-minute intervals shown in the figure.
Confidence Interval (CI) is at $95\%$ (limit value $z=1.96$) for each $5$-minute period. 
The values are obtained from $n=1500$ samples during each period.
Standard Error (SE) information is not included, but can be inferred by CI,z,n values.
Z-scores and CI information are available in Table~\ref{tab:stat}.




\subsection{Measurements on Scalability and Communication Overhead}


In Table~\ref{tab:fg_scale}, we have measurements from the two scenarios of $10000$ and $15000$ $veh/h$.
Each agent is always associated with $1$ single factor; therefore, the corresponding size directly reflects the average number of agents, and we can also calculate the number of pairwise connections per agent (considering that $1$ pairwise factor connects $2$ agents/variables) as:

%
\begin{eqnarray}
    c_p = 2\cdot\frac{\#\text{Pairwise Factors}}{\#\text{Single Factors}}
\end{eqnarray}
%


Finally, we can calculate the size of messages, accounting for the variable space 
($|X_i|=15$, cf. Table~\ref{tab:fg}), that are exchanged between agents. In our approach, we believe it is fitting to measure the size of the broadcasted q messages accordingly and include the size for $x^*$ and time-estimates. 

As such, assuming a constant size of $N\, bytes$ per real value broadcasted, we can calculate the size of information $i_b$ in bytes to be broadcasted per agent:

%
\begin{eqnarray}
    i_b = (c_p \cdot |X_i| + 2)\cdot N\, bytes
\end{eqnarray}
%

\noindent with the first term $c_p \cdot |X_i|$ enumerating the total size of real values ($|X_i|$ values per message $q_{i\rightarrow j}$ to connected factor $j$), and the second containing the $2$ constant values ($x^*_i,t_{i,e}$) that are always shared.
As such, by restricting the number of connections, we can bound the amount of information given physical requirements. 

We also have two additional settings in situations with higher desired speed deviations (HDS) among agents (desired speed range is $[20,40] m/s=[72,144]km/h$ instead of $[25,35]m/s=[90,126]km/h$), to show how the graph’s size and pairwise connections are affected. We see that agents form more pairwise connections and thus connections on average, specifically for the flow of $15000$, we go from $\sim 7$ to $\sim 8$ connections per agent  due to more frequent overtakes. 
The higher speed deviations also affect the speed-related metrics negatively, hence the reason we observe more agents (single factors) in the road when comparing the two cases at the same flow.
Measurements below are taken from Cond-Max-Sum, with similar results for all variants in that regard.


\begin{table}[t]
  \centering
  
  \caption{Scalability Measurements on the Factor Graph}
  % \scalebox{1}
  {
  \begin{tabular}{lrr}
  
  % \toprule
  \hline
     Scenario & \#Single Factors (avg) &	\#Pairwise Factors (avg) \\ 
    \midrule
    $10000 veh/h$ &	$187.35$ &	$445.04$ \\
    $15000 veh/h$ &	$287.14$ &	$1017.66$ \\
    $10000 veh/h$ (HDS) &	$202.16$ &	$575.69$ \\
    $15000veh/h$ (HDS) &	$315.66$ &	$1263.80$ \\
    
    % \bottomrule
  \end{tabular}
  }
  \label{tab:fg_scale}  
\end{table}