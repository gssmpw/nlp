\section{Related Works}
{A number of studies considered the issue of traffic mismatches for SATCOM under heterogeneous traffic patterns among beams, including \cite{abdu2021flexible, ha2022geo, wang2020noma, lin2022multi, zheng2012generic, liu2019qos, 9769901, cui2023energy}.}
{The authors of \cite{abdu2021flexible, ha2022geo} discussed the transmission power minimization problem while guaranteeing the traffic demands with quality of service (QoS) constraints.}
% under perfect CSI conditions
% S + V + O 는 붙어 있는게 좋음 / usage -> utilized / discuss
{To be more specific, the authors of \cite{abdu2021flexible} focused on satisfying the QoS of each beam while jointly minimizing the carrier utilization and transmission power using SDMA.}
% Attempted 도 쓰일 수 있겠네
{In \cite{ha2022geo}, the authors investigated an optimal linear precoder and beam-hopping (BH) architecture to minimize the transmission power while satisfying the QoS of users.}
% 목적어 자리에는 동명사 보다는 명사 그 자체를 많이 쓰네
{ However, the power minimization problems with QoS constraints \cite{abdu2021flexible, ha2022geo} can result in infeasible solutions when the available power budget is insufficient or the channel condition is unfavorable for satisfying the traffic demands. This, in turn, poses fundamental challenges in reliably satisfying the traffic requirements of users in SATCOM, where power-hungry payloads are common.
Minimizing transmit power under QoS constraints risks destabilizing the system in SATCOM, which degrades the user experience. As such, SATCOM requires a more sophisticated approach that can directly address traffic mismatches and ensure reliable communication services, regardless of the available power budget or user channel conditions.} 
% Multibeam SATCOM은 Power가 부족함을 다시 강조



{To tackle this issue, the authors in \cite{wang2020noma} studied the flexible resource management problem based on NOMA to maximize the minimum value of the ratio of the offered capacity to the required traffic.}
% S+V 는 붙어 있는게 좋다 / In order to -> To
{Similarly, in \cite{lin2022multi}, maximizing a minimum ratio of offered capacity to the traffic demand was investigated by considering an optimized BH architecture.} 
% Constructing -> Considering
{In addition, the SDMA-based linear precoder design and NOMA-based beam power optimization problems were studied in \cite{zheng2012generic} and \cite{liu2019qos}, respectively, to minimize the difference between the traffic demands and offered rates.}
% difference min을 목표로 beam power를 optimize 하는
{The authors of \cite{9769901} 
focused on a joint optimization problem for resource allocation and beam scheduling to minimize the disparity between the traffic demands and offered rates based on NOMA.}
% 수식어 전에는 주로 , 붙이고 / S+V 는 떨어지지 않는게 좋음
{Furthermore, the authors of \cite{cui2023energy} studied the joint {unmet rate} and transmission power minimization problem using minimum mean square error (MMSE)-based RSMA under perfect CSI conditions.} 
% via 는 본문중에 잘 안쓰나 보다


%\vspace{-6mm}
{