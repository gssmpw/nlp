\section{Limitations and Future Work}
\label{sec:conclusion}

LeanProgress represents a significant advancement in guiding search for neural theorem proving through proof progress prediction. However, it is important to note that this work is just the first step in a broader research agenda. There are several promising avenues for future work that could further enhance the capabilities and applications of LeanProgress. 

\textbf{1) Incorporating Tree-of-Thought and Chain-of-Thought Approaches}: One potential direction for future research is to integrate tree-of-thought (ToT) and chain-of-thought (CoT) methodologies into LeanProgress. These approaches could provide a more structured and interpretable way of reasoning about proof progress. By incorporating ToT and CoT, we could potentially improve the model's ability to explain its predictions and provide more detailed insights into the proof process.

\textbf{2) Integration with Reinforcement Learning}: A particularly promising avenue for future work is the integration of LeanProgress with reinforcement learning (RL) techniques. LeanProgress's ability to predict the number of remaining steps in a proof can provide a continuous and informative reward signal for RL agents. Unlike binary rewards that only indicate success or failure at the end of a proof attempt, this continuous feedback allows the agent to learn from partial progress throughout the proving process. They could also learn more efficiently by receiving meaningful feedback throughout the proving process while developing better long-term strategies for complex proofs. This could then enable the model to adapt its behavior based on the difficulty and progress of the current theorem and achieve higher success rates on challenging theorems that require many steps.

\textbf{3) Lightweight and Scalable Implementations}: Future work could also focus on developing more lightweight implementations of LeanProgress. This could involve exploring model compression techniques or developing more efficient architectures that maintain prediction accuracy while reducing computational requirements. Such improvements would make LeanProgress more accessible and easier to integrate into existing theorem-proving workflows.


\section{Conclusion}

We introduce LeanProgress, a approach to enhance interactive theorem proving in Lean by integrating a remaining step predictor into the LeanCopilot frontend. Our work makes several significant contributions to the field of automated theorem proving. We developed a method for generating a balanced dataset of proof trajectories by adjusting the sampling ratio based on proof length, addressing the challenge of skewed distributions in proof complexity. We then trained a remaining step prediction model using a novel input format that incorporates the current proof state and, optionally, the proof history. Integrating this model into the LeanCopilot interface provides users with both tactic suggestions and remaining step predictions, offering a more comprehensive tool for guiding the proof process. Our results highlight the potential of proof progress prediction in enhancing both automated and interactive theorem proving, enabling users to make more informed decisions about proof strategies. Lastly, LeanProgress represents a significant step forward in bridging the gap between local tactic prediction and global proof trajectory understanding, opening up new possibilities for the application of reinforcement learning in automated theorem proving and paving the way for more efficient and effective proof development in large formalization projects.


\section{Impact Statement}
This paper presents work whose goal is to advance the field of machine learning and interactive theorem proving. There are many potential societal consequences of our work, none which we feel must be specifically highlighted here.
