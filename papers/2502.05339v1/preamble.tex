\usepackage{bm}
\usepackage{amsmath}
\usepackage{colortbl}
\usepackage{nicefrac}
\usepackage{wrapfig}
\newcommand{\mathcolorbox}[2]{\colorbox{#1}{$\displaystyle #2$}}

\DeclareMathOperator*{\argmin}{arg\,min}

\definecolor{turquoise}{cmyk}{0.65,0,0.1,0.1}
\definecolor{purple}{rgb}{0.65,0,0.65}
\definecolor{darkgreen}{rgb}{0.0, 0.5, 0.0}
\definecolor{darkred}{rgb}{0.5, 0.0, 0.0}
\definecolor{darkblue}{rgb}{0.0, 0.0, 0.5}
\definecolor{blue}{rgb}{0.0, 0.0, 1.0}
\definecolor{orange}{rgb}{1.0, 0.5, 0.0}
\definecolor{red}{rgb}{1.0, 0.0, 0.0}
\definecolor{cherry}{RGB}{186,12,47}
\definecolor{pink}{RGB}{117,107,177}

\newcommand{\vect}[1]{\mathbf{#1}}

\newcommand{\refsec}[1] {Section~\ref{#1}}
\newcommand{\refeq}[1] {Equation~\ref{#1}}
\newcommand{\reffig}[1] {Fig.~\ref{#1}}
\newcommand{\refalg}[1] {Algorithm~\ref{#1}}
\newcommand{\reftab}[1] {Table~\ref{#1}}
\newcommand{\refapx}[1] {Appendix~\ref{#1}}

\newcommand{\pca}{\textit{Principal Component Analysis}}
\newcommand{\koopman}{Koopman operator}