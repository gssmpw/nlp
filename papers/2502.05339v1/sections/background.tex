\begin{figure*}[ht]
    \centering
    \includegraphics[width=\linewidth]{figure/basis_evaluation.pdf}
    \caption{\textbf{Reconstruction of 3D Plume Simulations.} From left to right: standalone plumes, plumes interacting with a sphere, and plumes interacting with a bunny. For each configuration, the last frame of the original MacCormack \cite{selle2008unconditionally} fluid simulation is shown on the left, alongside the last frame of our subspace simulation with SVD rank ranging from $r = 2$ to $r = 61$. Remarkably, the fluid dynamics demonstrate strong resilience to low-rank bases, highlighting a key advantage of our proposed reduced-order pipeline. Additionally, these scenarios illustrate the robustness of our method in handling increasingly complex boundary conditions. The reconstruction quality improves as the number of basis functions increases, enabling more accurate capture of finer details around the boundaries. While reasonable results are achieved with a small basis (r=9), increasing the basis significantly enhances the fidelity of the simulations. More details on the temporal evolution of the flow and other basis configurations can be found in the \reffig{fig:appendix_plume}, \reffig{fig:appendix_sphere} and \reffig{fig:appendix_bunny}.} 
    \label{fig:basis_evaluation}
    \Description{}
\end{figure*}

\section{Koopman Operator and Dynamic Mode Decomposition For Fluids}
\label{sec:background}

Consider an \emph{autonomous} system of the form,
\begin{align}
    \label{eqn:autonomous}
    \frac{d}{dt}\bm{u}(t) = \bm{f}(\bm{u}(t))
\end{align}
where $t$ is time, $\bm{u} \in \mathbb{R}^N$ is a state vector and $\bm{f}:\mathbb{R}^N\to \mathbb{R}^N$ is a nonlinear operator. There exists a time-dependent family of flow maps $\{\bm{F}_t\}_{t>0}$ such that $\bm{u}(t)=\bm{F}_t(\bm{u}_0)$. That is, it maps some initial state $\bm{u}_0$ from time $0$ to its state at time $t$. This induced flow map defines a \emph{time-series} of data,
\begin{equation}
    \label{eqn:data_matrix}
    \bm{U} =
    \underbrace{\left[
    \begin{array}{c c c c c}
        \vert & \vert &        & \vert \\
        \bm{u}(0) & \bm{u}(1) & \cdots & \bm{u}(T) \\
        \vert & \vert &        & \vert \\
    \end{array}
    \right]
    }_{T+1 \;\text{Frames}}
    \begin{array}{l}
        \left.\begin{array}{c} \\ \\ \\ \end{array}\right\}
        \text{ $N$ DoFs}
    \end{array}
\end{equation}
which represent \emph{measured} states of the system with their associated timestamps.

The \emph{finite-dimensional nonlinear} trajectories induced by $\bm{f}(\bm{u})$ can be represented as \emph{infinite-dimensional linear} trajectories of some scalar observables $g:\mathbb{R}^N\to \mathbb{C}$ in Hilbert space \cite{koopman1931hamiltonian}. With these observables, \citet{koopman1931hamiltonian} defines a family of \emph{Koopman operators} $\bm{\mathcal{K}}_t$ such that 
\begin{align}
    \bm{\mathcal{K}}_tg = g\circ\bm{F}_t = g(\bm{u}(t)).
\end{align}
That is to say, $\bm{\mathcal{K}}_t$ maps the measurement $g$ operator to its state at time $t$. Notice then, that we can recover our state $\bm{u}(t)$ by using the adjoint $g^*$ of our observable:
\begin{align}
g^*\bm{\mathcal{K}}_t g = \bm{u}(t),
\end{align}
Thus, if we can construct $\{\bm{\mathcal{K}}_t\}$ and $g$, we can get our system state at any point in time. We simply need to find the generator, given by:
\begin{align}
    \bm{\mathcal{K}}g = \frac{d}{dt}g\circ\bm{F}_t = \lim_{t\to 0}\frac{1}{t}(\bm{\mathcal{K}}_tg-g)
\end{align} 
Notice now that the generator $\bm{\mathcal{K}}$ is linear by construction, but acting on the space of \emph{nonlinear} observables $g$. In essence, it relegates the nonlinearity to the extra dimensionality afforded by the function space traversed by $g$. This generator is typically what is called as \emph{The} Koopman operator (as opposed to the family of operators $\{\bm{\mathcal{K}}_t\}$).

We additionally point out that if $\bm{\mathcal{K}}_t$ is a Koopman operator acting on an observable $g$, we can likewise refer to $\hat{\bm{\mathcal{K}}_t}=g^*\bm{\mathcal{K}}_tg$ as a Koopman operator acting on the observable $\bm{u}$. $g$ then is simply an operator that transforms observables of one function space (our state space) to another. For a finite dimensional vector $\bm{u}$, we represent the discretized Koopman operator as $\bm{K}$.

%\subsection{Koopman Operator}
%The fluid system described in Equation \ref{eqn:euler_equations} can be rewritten in the form of a first-order system of ordinary differential equations (ODEs),
%where $\bm{f}(\bm{u})$ is the right-hand side of the Euler equations. The solution to the ODE (\refeq{eqn:euler_ode}) is a trajectory $\bm{u}(t)$ in the state space $\mathbb{R}^{N}$.

%When we also discretize the time, we could rewrite the collection of snapshots of fluid state in the matrix form as a matrix of size $\mathbb{R}^{dN \times T}$, where $T$ is the number of frames:
%Note that this is what referred to as the \textit{time-series data} in the context of stastical analysis.

%Here, we introduce the \koopman{} $\bm K$ \cite{koopman1931hamiltonian} to analyze the dynamics of the fluid system. The \koopman{} is a \textbf{linear} operator that acts on functions of the \textbf{non-linear} state space. In the discretized setting, the \koopman{} is a matrix that maps the state of the fluid at time $t$ to the state at time $t + \Delta t$:
%\begin{equation}
%    \label{eqn:koopman_operator}
%    \bm{u}(t + \Delta t) = \bm{K} \bm{u}(t)
%\end{equation}
%where $\bm{K}$ is the \koopman{} and is a $\mathbb{R}^{dN \times dN}$ matrix that maps the velocity state $\bm{u}^{t}$ of the fluid at time $t$ to the velocity state $\bm{u}^{t + \Delta t}$ at time $t + \Delta t$.

\subsection{Approximating the Koopman Operator with Dynamic Mode Decomposition}

We remind the reader that the Koopman operator acts on the infinite-dimensional function space spanned by some observables $g$. This, evidently, is cumbersome to work with. Thus, similar to the idea of applying PCA to reduce the dimensionality of a dataset in previous work \cite{kim2013subspace,treuille2006model}, we instead approximate the Koopman operator using \textit{Dynamic Mode Decomposition} (DMD) \cite{schmid2010dynamic}, constructing a \textit{reduced, low-dimensional} basis to represent the function space spanned by $g$.

Given two snapshots, a discrete Koopman operator is defined by:
\begin{equation}
    \label{eqn:reduced_koopman}
    \begin{aligned}
        &\argmin_{{\bm{K}}} \; \|\bm{X}^\prime - {{\bm{K}}} \bm{X}\|_F,  \\
        \text{where } \; 
        \bm{X} &= \underbrace{
            \begin{bmatrix}
                \vert & \vert &        & \vert \\
                \bm{u}(0) & \bm{u}(1) & \cdots & \bm{u}(T-1) \\
                \vert & \vert &        & \vert \\
            \end{bmatrix}}_{T \;\text{Frames}
        },
        \\
        \bm{X}^\prime &=
        \underbrace{
            \begin{bmatrix}
                \vert & \vert &        & \vert \\
                \bm{u}(1) & \bm{u}(2) & \cdots & \bm{u}(T) \\
                \vert & \vert &        & \vert \\
            \end{bmatrix}}_{T \;\text{Frames}
        },
    \end{aligned}
\end{equation}
Intuitively, this optimizes ${{\bm{K}}}$ to minimize the difference between the next states $\bm{u}(t+\Delta{}t)$ and its prediction of the next states ${\bm{K}}\bm{u}(t)$.

We can solve this minimization as follows:
\begin{equation}
    \label{eqn:reduced_koopman_solution}
    {\bm{K}} = \bm{X}^\prime \bm V \bm \Sigma^{-1} \bm U^T,
\end{equation}
%\begin{equation}
%    \label{eqn:reduced_koopman_solution}
%    \hat{\bm{K}} =  \bm U^T \bm{X}^\prime \bm V \Sigma^{-1}
%\end{equation}
where $\bm{X} = \bm{U} \bm{\Sigma} \bm{V}^T$ is the singular value decomposition (SVD) of the snapshot matrix $\bm{X}$.
We can truncate this eigensystem, taking the top $r$ singular values corresponding vectors to form the truncated \koopman{} $\tilde{\bm{K}}$. 
We can then project this truncated operator to the reduced basis $\bm{U}^T\tilde{\bm{K}}\bm{U} =  \bm{U}^T\bm{X}^\prime \bm V \bm \Sigma^{-1} \bm U^T\bm{U} = \hat{\bm{K}}$ to produce the reduced Koopman operator $\hat{\bm{K}}$.

We can see that as long as a mode is representable in the reduced space, $\hat{\bm{K}}$ and $\tilde{\bm{K}}$ share that mode's eigenvalue, $\hat{\bm{K}}\bm w_i = \lambda_i \bm w_i$, with the eigenvector simply being $\tilde{\bm{K}}$'s eigenvector projected onto the reduced space, $\bm w_i = \bm U^T \bm \phi_i$. 
Here, $\lambda_i$ is a mode's eigenvalue, $\bm \phi_i$ is its corresponding eigenvector in fullspace, and $\bf w_i$ is its projection onto reduced space. 
We thus can find a spectral decomposition of the truncated Koopman operator: $\tilde{\bm{K}} = \bm{\Phi}\bm{\Lambda}\bm{\Phi}^*$. 

Since applying the truncated \koopman{} $\tilde{\bm{K}}$ on the full space has complexity $\mathcal{O}(N^2)$, we \textit{project} the full-space velocity field $\bm{u}$ onto the reduced space spanned by the basis $\bm{\Phi}$:
\begin{equation}
    \label{eqn:projection}
    \bm{u}(t + \Delta t) = \tilde{\bm{K}} \bm{\Phi} \bm{\Phi}^+ \bm{u}(t) = \tilde{\bm{K}} \bm{\Phi} \bm{z}(t),
\end{equation}
where $\bm{\Phi}^+ = \bm{\Phi}^* (\bm{\Phi} \bm{\Phi}^*)^{-1} \in \mathbb{R}^{r \times N}$ is the Moore-Penrose pseudoinverse of $\bm{\Phi}$ and $\bm{z}(t) = \bm{\Phi}^+ \bm{u}(t) \in \mathbb{R}^r$ is the reduced state of the fluid system at time $t$. This projection step reduces the complexity to $\mathcal{O}(Nr)$.

Notice now that taking $\bm{\Lambda} = \bm{\Phi}^*\tilde{\bm{K}}\bm{\Phi}\in \mathbb{R}^{r\times r}$ gives exactly a matrix that advances the reduced state forward in time by $\Delta{}t$:
\begin{equation}
    \label{eqn:reduced_koopman_simulation}
    \bm{z}(t + \Delta t) = \bm{\Lambda} \bm{z}(t),
\end{equation}
We note that $\bm{\Lambda}$ is the diagonal matrix of eigenvalues of $\hat{\bm{K}}$, and $\bm{\Phi}$ are the corresponding eigenvectors. That is to say, $\tilde{\bm{K}}=\bm{\Phi}\bm{\Lambda}\bm{\Phi}^*$ is exactly the spectral decomposition of the reduced Koopman operator.

From here, we can once again simply apply the basis $\bm{\Phi}$ to return back to fullspace:
\begin{equation}
    \label{eqn:reduced_koopman_projection}
    \bm{u}(t + \Delta t) = \bm{\Phi} \bm{\Lambda} \bm{z}(t).
\end{equation}

We would like to point out that both $\bm{\Lambda}$ and $\tilde{\bm{K}}$ are Koopman operators. $\bm{\Lambda}$ in particular gives particular insight into the theory; notice here that it acts on the function space of $g=\bm{\Phi}$. Each eigenfunction is thus a special observable, which clearly behaves linearly by definition. Since we know that $\bm{\Phi}^*$ maps observables to observables, $\tilde{\bm{K}}$ must necessarily also be a Koopman operator, this time acting on the identity operator (or more accurately, the $\bm{u}$ observable in Hilbert space).

The $\bf \Lambda$ representation in particular is particularly useful for us as it exposes the modes in an easily manipulatable manner. As a Koopman operator, it represents the time evolution of some observable. By being a diagonal matrix of eigenvalues, it turns out that these observables are exactly spatial modes that rotate with a particular frequency given by the imaginary part of the eigenvalues.