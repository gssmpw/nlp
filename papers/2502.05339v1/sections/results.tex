\begin{figure}[!ht]
    \centering
    \includegraphics[width=0.8\columnwidth]{figure/reconstruction_2D.pdf}
    \caption{\textbf{Comparison of PCA and Our Method for Low-Rank Flow Reconstruction.} The leftmost column shows the reference high-resolution simulation, while the right grid presents reconstructions at different rank truncations ($r$). Each pair in the grid compares Principal Component Analysis (PCA) (left) and our method (right). Lower ranks ($r=2,9$) fail to capture large-scale turbulence while increasing \(r\) improves accuracy. Our method retains more detailed structures at lower ranks than PCA, demonstrating improved efficiency in capturing complex flow dynamics.}
    \label{fig:reconstruction}
    \Description{}
\end{figure}

\section{Evaluation and Results}
\label{sec:results}
In this section, we demonstrate the results of our approach on baseline examples and provide detailed numerical evaluations.
We conduct all numerical experiments on a Linux workstation equipped with a 2.10GHz 32-core Intel CPU featuring 120GB of RAM. We implement our method exclusively in Python, and our DMD implementation is a heavily optimized, memory-efficient version of \texttt{PyDMD} \cite{ichinaga2024pydmd}. We use \texttt{Taichi} \cite{hu2019taichi} for interactive examples and parallelization of DMD model inference, and \texttt{NumPy} \cite{harris2020array} for numerical validation. Please refer to \reftab{tab:experiment_setup} for the experiment setup of the examples presented in this section.

\subsection{Comparison of 2D Plume Reconstruction}
To evaluate the reconstruction performance of our method in comparison with the prior work \cite{kim2013subspace}, we conducted experiments on a 2D plume scenario using varying numbers of basis functions: $2, 9, 28, 61, 105, 130$, and $150$; shown on \reffig{fig:reconstruction}. With only 9 basis functions, our method is already capable of reconstructing the contour of the plume's top, while the prior method can only capture basic upward motion of the plume structure without any vortical structure details. This shows the superior reconstruction capability of our method even at low ranks. When using $28$ basis functions, our method closely matches the ground truth, whereas the prior method still misses many details, particularly in the upper regions of the plume. When the number of basis functions is increased to 61, our method achieves a near-perfect reconstruction of the ground truth, and the prior method requires 130 basis functions to produce a similar level of detail. These results effectively demonstrate that our method can achieve better reconstruction with significantly fewer basis functions.

\begin{table}[ht]
    \centering
    \rowcolors{2}{white}{gray!20}
    \begin{tabular}{l|c}
        \multicolumn{2}{l}{\textbf{\Large{\textsc{Runtime: 2D Plume}}} \;\textsc{\textcolor{gray}{256 $\times$ 512 Grid}}\; ($\triangleright$ \reffig{fig:reconstruction})} \\
        \hline
        \textbf{Fullspace Solve} & 754 ms  \\
        \hline
        \textbf{Subspace Solve} & \\
        \hspace{1em} PCA-based \shortcite{kim2013subspace} & 16.15 ms (47$\times$) / 0.95 ms (794$\times$) \textsuperscript{\textdagger}   \\
        \hspace{1em} \textbf{Ours} & \textbf{8.9 $\mu$s (84,719$\times$)}   \\
        \hline
        \textbf{Precomputation} &  \\
        \hspace{1em} PCA-based \shortcite{kim2013subspace} & 114 s  \\
        \hspace{1em} \textbf{Ours} & \textbf{79 s} \\
    \end{tabular}
    \caption{\textbf{Breakdown of Experiment Runtime.} 
    The table compares the runtime and precomputation costs of the full-space ground truth simulation, the PCA-based approach \shortcite{kim2013subspace} and our method on the 2D plume example (\reffig{fig:reconstruction}). Leveraging the linear \koopman{}, our method achieves significantly faster runtime by requiring only a single matrix multiplication for each reduced-space simulation step. 
    \\\textsuperscript{\textdagger} = with/without external force.}
    \label{tab:dmd_runtime}
\end{table}

\subsection{Generalization With Vorticity Confinement}
\label{sec:generalization}
To further evaluate generalization and artist-directability of our method, we revisited the 2D plume example following the experimental setup from the prior work \cite{kim2013subspace}. Specifically, we incorporated vorticity confinement \cite{fedkiw2001visual} into the original MacCormack solver~\cite{selle2008unconditionally} and then applied our method to the modified solver. For training, the vorticity confinement value was set to 1.5, and we tested both our approach and the prior method at vorticity confinement values of 1.51, 1.6, and 2.5. To quantify generalization efficacy, we calculated the relative error of the velocity field with respect to ground truth MacCormack\cite{selle2008unconditionally} simulation results. The results, as shown in \reffig{fig:generalization}, reveal that our method achieves a lower relative error compared to the prior method. Notably, the prior method required 150 basis functions, whereas our method achieved comparable results with only 50 basis functions. These findings highlight the reliability and effectiveness of our Koopman-based approach in adapting to changes in simulation parameters.

\begin{figure}[!ht]
    \centering
    \includegraphics[width=1\columnwidth]{figure/generalization_2D.pdf}
    \caption{\textbf{Comparison of Relative Error and Vorticity Confinement Between PCA \shortcite{kim2013subspace} and DMD (Ours).} Left to right: (1) relative error over time for \textit{PCA (150 basis)} (dashed) and \textit{Ours (50 basis)} (solid), showing comparable or lower error with fewer basis functions. (2) reference high-resolution simulation with vorticity confinement $1.5$, with a zoomed-in region marked. (3) comparison of the zoomed region in (2) under novel unseen vorticity confinement force $1.51$, $1.6$, $2.5$.}
    \label{fig:generalization}
    \Description{}
\end{figure}


\subsection{3D Plume Baselines}
To benchmark our Koopman-based fluid simulation pipeline in 3D scenarios, we present simulation results across three progressively complex scenarios: a standalone plume, a plume interacting with a sphere, and a plume interacting with a bunny. All original simulations were generated using the MacCormack solver \cite{selle2008unconditionally} at a resolution of $128 \times 128 \times 256$. We demonstrate the reconstruction results using varying numbers of basis functions. These cases were designed to benchmark the method’s ability to handle increasingly intricate fluid dynamics, from simple turbulence in the standalone plume to complex boundary interactions with the sphere and bunny.

\paragraph{\textbf{Plume}}
The standalone plume serves as a fundamental benchmark, as tested in prior work \cite{kim2013subspace}. \reffig{fig:basis_evaluation} highlights the effectiveness of our Koopman-based method in simulating a plume without solid obstacles. While \cite{kim2013subspace} experienced significant high-frequency dissipation at reduced ranks, our method well captures the swirling and rising behavior, closely matching the original MacCormack simulations even with a small number of basis functions ($r=9$). Increasing the basis amount to $r=28$ or $r=61$ further enhances the reconstruction, better preserving turbulent vortices and fine flow structures. These results demonstrate the robustness of our method in capturing the essential dynamics of the fluid simulation, maintaining high fidelity even at low ranks.

\paragraph{\textbf{Plume with Sphere and Bunny}}
Building on the baseline, we introduce two more scenarios to evaluate the robustness of our method in handling complex boundary conditions. When using a small set of basis ($r=9$), our Koopman-based method still achieves a reasonable reconstruction of the overall flow dynamics. However, finer details, especially those near boundaries, are less accurately represented. Notably, at $r=9$, there exist noticeable discrepancies in regions around the bunny, which will be resolved as the number of basis functions increases. This enhancement demonstrates how additional basis functions help the Koopman operator to reconstruct higher-frequency components of the flow and better capture complex boundary interactions.

\subsection{3D Colliding Vortex Rings}
The reduced-space Koopman operator is linear and has demonstrated that it can accurately reconstruct scenarios such as Kármán vortex street and plumes. However, these datasets feature velocity fields with relatively smooth variations over time. To test whether our method can adapt to scenarios with abrupt changes in the velocity field, we selected a more challenging scenario: colliding vortex rings. In this experiment, two point vortices are initialized and collide head-on. When the vortices meet, the velocity field undergoes a sudden change, resulting in finer vortex structures. We tested our method on this dataset using 150 basis functions. As shown in \reffig{fig:teaser}, our method can reconstruct the transition from the two point vortices before the collision to the rapid formation of a divergent velocity field during the impact, as well as the subsequent emergence of numerous vortical structures around the periphery. This experiment shows that our method can effectively handle datasets with significant and sudden velocity variations.

\subsection{Independence on Simulation Schemes}
Our reduced simulation method is inherently \emph{simulator-agnostic}, allowing it to work seamlessly with a variety of fluid solvers. This flexibility arises from the fact that our method models only the transitions between successive fluid states in the reduced space, rather than being tied to the specific equations or numerical schemes of a given solver. This allows us to apply our method to any fluid simulator without the need for additional adjustments. For instance, in our experiments, we used both the MacCormack~\cite{selle2008unconditionally} + Reflection~\cite{zehnder2018advection} (MC+R) solver and a Lattice Boltzmann Method~\cite{chen1998lattice} (LBM) solver. They are fundamentally different in their discretization and numerical operations. Specifically, the LBM solver does not directly solve the Navier-Stokes equations; instead, it solves the approximated Boltzmann equation, which models fluid dynamics at the mesoscopic scale using particle distribution functions. Our method overcame these challenges by relying solely on the data produced by the solver. In other words, our method is \emph{equation-free}. We detailed the base simulators used for each example in \reftab{tab:experiment_setup}, where our approach successfully reconstructs the velocity fields in all cases. 

In contrast, previous data-driven methods such as \cite{treuille2006model, kim2013subspace} developed a specific numerical scheme for the subspace constructed from data, essentially binding the formulation to one specific base simulation framework. As a result, when switching to a different solver, their method also requires corresponding adjustments. This will introduce significant limitations for users, as they can only input datasets corresponding to a specific solver. In particular, this limits artists usage to only solutions of such solvers, whereas our method accepts hand-modified, or even real-world measured data.

\begin{table*}[ht]
    \centering
    \rowcolors{2}{white}{gray!20} % Apply row colors
    \begin{tabular}{c|c|c|c|c|c}
    % \hline
    \textbf{Examples} & \textbf{Resolution} & \textbf{Dim.} & \textbf{B.C.} & \textbf{Amount of Basis} & \textbf{Base Simulator} \\
    % \hline
    \textbf{Plume} ($\triangleright$ \reffig{fig:basis_evaluation}) & $128 \times 128 \times 256$ & 3D & Open & $2, 9, 28, 61, 105, 130, 150$ & MC\\
    % \hline
    \textbf{Plume w/ Sphere} ($\triangleright$ \reffig{fig:basis_evaluation}) & $128 \times 128 \times 256$ & 3D & Dirichlet & $2, 9, 28, 61, 105, 130, 150$ & MC \\
    % \hline
    \textbf{Plume w/ Bunny} ($\triangleright$ \reffig{fig:basis_evaluation}) & $128 \times 128 \times 256$ & 3D & Dirichlet & $2, 9, 28, 61, 105, 130, 150$ & MC\\
    % \hline
    \textbf{Smoke Ring} ($\triangleright$ \reffig{fig:teaser}) & $128 \times 128 \times 256$ & 3D & Open & $150$ & MC+R \\
    % \hline
    \textbf{Reversibility} ($\triangleright$ \reffig{fig:reverse_simulation}) & $512 \times 512$ & 2D & Open & $20$ & MC \\
    % \hline
    \textbf{Editing (K\'arm\'an Vortex Street)} ($\triangleright$ \reffig{fig:karman_editing}) & $512 \times 512$ & 2D & Periodic & $100$ & LBM-BGK \\
    % \hline
    \textbf{Editing (Bunny)} ($\triangleright$ \reffig{fig:bunny_editing}) & $128 \times 128 \times 256$ & 3D & Dirichlet & $50$ & MC \\
    % \hline
    Rayleigh–Taylor Instability ($\triangleright$  \reffig{fig:dmdadvectioncomparison}) & $1024 \times 512$ & 2D & Dirichlet & $100$ & MC \\
    \end{tabular}
    \caption{\textbf{Breakdown of Experiment Setup.} The result and experiment setup are detailed in this table, including grid resolution, dimensionality, boundary conditions (B.C.), the number of basis functions used and the base simulator for each result. As for the base simulators, we employ the MacCormack \cite{selle2008unconditionally} (MC), the MacCormack + Reflection \cite{zehnder2018advection} (MC+R) and the Lattice Boltzmann Method with the Bhatnagar-Gross-Krook collision model (LBM-BGK)\cite{chen1998lattice}.}
    \label{tab:experiment_setup}
\end{table*}