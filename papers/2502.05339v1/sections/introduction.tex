\section{Introduction}
\label{sec:introduction}

\begin{figure*}[!ht]
    \centering
    \includegraphics[width=\textwidth]{figure/fast_integration.pdf}
    \caption{\textbf{Long-time Single Step Integration.} We demonstrate that our method can perform integration into arbitrary points in time via a single matrix multiply. Since the DMD operator is diagonal within the reduced basis, it is trivial to find the matrix that evolves the initial velocity field to the field at any point in time, significantly accelerating the integration as compared to traditional methods required by PCA.}
    \label{fig:dmdadvectioncomparison}
    \Description{}
\end{figure*}

Fast, responsive, and visually realistic fluid simulation remains a significant challenge in physics-based animation.
Traditional full-resolution simulations deliver highly detailed fluid flows but are computationally expensive, limiting their use to offline cinematic visual effects rather than interactive virtual reality applications.
%

Reduced-Order Models (ROMs) address this challenge by accelerating simulations through dimensionality reduction.
%
In the context of fluid dynamics, ROMs approximate the \emph{full, high-dimensional} fluid simulation solution space by effectively operating within a \emph{reduced, lower-dimensional} subspace.
% 
The dynamics of the reduced fluid simulation are then obtained by projecting the high-dimensional solution space into the reduced space and solving the physical Partial Differential Equation (PDE) within this subspace.
%
As a result, the subspace needs to provide a compressed representation of the fluid state, while also capturing realistic and highly detailed fluid behavior.
%


Picking which subspace to use for simulation is non-trivial, and after committing to a subspace, using it for reduced simulation opens up separate difficulties. 
%
First, subspace fluid simulation is notorious for dissipating high-frequency detail, the kind that is commonly desired in turbulent flows. 
%
Second, it involves computing a static subspace that can generalize to a diversity of simulation states expected across a variety of scene interactions and configurations. 
%
Finally, even if one makes use of a linear subspace for simulation, the advection component of inviscid Euler equations is non-linear with respect to the fluid state, which requires full-space computation even with the use of a precomputed subspace. 


Instead of committing to the static subspace methodology, we make use of Dynamic Mode Decomposition (DMD) \cite{schmid2010dynamic}, a modern flow analysis technique that takes an alternative simplification to the fluid simulation problem.
% 
This formulation linearly approximates the \emph{\koopman{}},  which encodes the temporal evolution of the fluid flow, and performs reduction directly on this operator \cite{schmid2010dynamic}.
%
An immediate advantage of reduction on this operator is that the resulting subspace is imbued with information regarding the temporal dynamics of the flow.
%
This temporal awareness allows us to quickly evaluate the fluid flow at \emph{any} point in time directly, without performing any time integration or discrete fluid advection whatsoever, as shown in \reffig{fig:dmdadvectioncomparison}.

While DMD has shown great success for use in flow analysis from the engineering community, 
we are the first to show how it can be adapted for use in fluid animation applications in graphics, such as fluid editing, guiding, interaction, and artistic fluid control. 
%

Our key insight is to leverage the inherent spatio-temporal nature of the linear DMD operator. By encoding \emph{both} the \emph{spatial} nature of flows in the eigenvector basis and their \emph{temporal} evolution in the eigenvalues, each eigenvector-eigenvalue pair becomes representative of a distinct wave mode. 

This spectral-like decomposition allows for directable control of each mode separately, enabling new forms of creative expression, such as adjusting the amplitude of temporal frequency bands to control look-and-feel in interactive editing of fluid animation. 

\paragraph{Contributions}
In summary, in this paper we:
\begin{itemize}
    \item introduce Dynamic Mode Decomposition (DMD) for fast control of fluid simulations with exceptional accuracy, computational speed, and memory efficiency;
    \item identify that the inherent spatiotemporal nature of the DMD operator enables control of each wave mode by modification of eigenvector-eigenvalue pairs;
    \item demonstrate DMD's versatility and practicality in the context of interaction and artistic control, such as frequency editing, time-reversal, super-resolution and simulation styling.
\end{itemize}