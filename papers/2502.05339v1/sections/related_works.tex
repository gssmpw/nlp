\section{Related Works}
\label{sec:related_works}

Following pioneering work by \citet{foster1996liquids}, \citet{stam2023stable}, and \citet{fedkiw2001visual}, the graphics community has made significant progress in the visual simulation of fluids.
Our method straddles literature from both fluid control and runtime acceleration. From foundational work, we aim to demonstrate that DMD arises naturally as a combination of spatial reduced-order modelling and spectral decompositions, a unique position that makes it well suited for fluid control applications.

%; for a recent survey, see \citet{wang2024survey}. We focus on methods for accelerating fluid simulations, in particular: parallelization, reduced-order modelling (ROM), and Fourier-feature based flow detail enrichment. From this foundational work, we show that DMD arises naturally as the combination of reduced-order modeling and Fourier transformation, and discuss existing work in the context of fluid simulation.

%\eitan{Actually, our work is situated at the intersection of speedup and creativity, so our related work should also have a section on creative control.}
%We also review learning-based methods that approximate fluid dynamics from data, which are closely related to our work. Lastly, we show that the line of DMD is essentially the combination of reduced-order modeling and Fourier transformation, and we discuss the existing line of work of DMD in the context of fluid simulation.

\subsection{Fluid Control}
% JP: this intro is kinda gross but I cant come up with something better right now
The chaotic nature of fluid behaviour makes artist-directed control a particularly difficult problem, as it often requires balancing accuracy, efficiency, and the user intuition. Of particular interest for us are applications in fluid snapshotting, where control \emph{frames} are provided that fluid flows into and out of. Numerous prior works have been done using optimization-based techniques, aiming to compute optimal control forces that guide fluid flows to match user-specified keyframes \cite{treuille2003keyframe, mcnamara2004fluid, thurey2009detail, inglis2017primal, pan2017efficient}. However, the high-dimensional optimization problems make most prior methods computationally expensive, limiting their application to real-time and interactive control. To further improve the efficiency, reduced-order fluid control methods have been proposed, such as frequency-aware force field reduction\cite{tang2021honey} and control with Laplacian Eigenfunctions\cite{chen2024fluid}. In this work, we demonstrate that the reversible nature of the DMD operator makes it particularly suitable for these problems. Time-reversibility has been prior explored in graphics \cite{oborn2018time}, but requires the solution of a modified Poisson problem. In comparison, we demonstrate that the DMD operator is trivially reversible. 

We also take influence from resolution-based smoke control \cite{nielsen2009guiding, sato2021stream}, which uses an artist-directed low-resolution input to produce a high-resolution simulation with secondary motion, such as smaller-scale turbulence. We demonstrate that the modes provided by DMD allow for direct application to these problems. We also similarly can preserve the input features via projection into the input velocity space similarly to \citet{nielsen2009guiding}, but show that our quadratic problem is precomputable and resolves into a simple matrix-vector multiply at runtime.

\subsection{Fast Fluid Simulation}
% EG: I shortened this because it's really not directly realted to our work. We don't need to tell the whole history of fluids here.
A major computational bottleneck in most Navier-Stokes (NS) solvers is the global pressure projection step, which is computationally expensive when solved directly, especially for high-resolution simulations. To mitigate this cost, iterative solvers such as Conjugate Gradient (CG)~\cite{saad2003iterative} and multigrid methods~\cite{briggs2000multigrid} are commonly employed to exploit the sparsity structure of the linear operators.

Avoiding this global pressure projection step is a key focus of real-time methods. To this end, Smoothed-Particle Hydrodynamics (SPH) \cite{muller2003particle}, and later Position-Based Fluids (PBF) \cite{macklin2013pbf}, have become the \emph{de facto} standard for GPU-based fluid simulation. 
More recently, the Lattice Boltzmann Method (LBM) has emerged as a powerful grid-based alternative \cite{chen1998lattice, chen2021gpu, li2020fast}. All these methods avoid a global solve, and are thus embarrasingly parallelizable and highly suitable for modern GPU architectures. 
In exchange, however, these methods often require extremely high resolutions and equivalently high memory consumption during runtime. By comparison, our method also avoids this global solve, evolving the system via a single matrix vector multiply, while simultaneously keeping the memory footprint low via reduction to a small basis.

% JP: I don't think we need this here
%Recent high-performance computing frameworks such as Taichi~\cite{hu2019taichi} and Warp~\cite{warp2022} facilitate parallelization and hardware acceleration for fluid simulation codes.


%A fundamental strategy for accelerating fluid simulations is parallelization. 
%By leveraging General-Purpose GPU (GPGPU) computing, simulations can achieve significant speedups for large-scale problems. 
%However, parallelization is not always feasible, readily available, or cost-effective, posing challenges for practical implementations. With 
%Recent high-performance graphical computing frameworks such as Taichi~\cite{hu2019taichi} and Warp~\cite{warp2022} have made optimized scalable efficient fluid simulations more accessible.

% JP: I think sph and lbm arent important enough to warrant their own subsections
%\paragraph{Lagrangian Approaches: Particle-Based Fluids}
% JP not relevant
%Several advancements in SPH and PBF have improved realism and efficiency, including multi-phase flow simulation~\cite{alduan2017dyverso}, surface tension modeling~\cite{xing2022position}, and interactive SPH on GPU~\cite{goswami2010interactive}.
%These methods enforce incompressibility through position constraints rather than pressure projection, significantly reducing computational overhead. \eitan{not sure that's fair... more accurate is they save the expense of projection at the cost of more compressibility. i would avoid taking sides in such debates}
%EG: fine for a survey, but unless you are explicitly tying it to the present paper (in a sentence that mentions "our work," don't say it
%Particle-based methods are inherently parallelizable, making them well-suited for GPU acceleration, though they struggle with large-scale domains due to computationally expensive neighbor searches and pressure enforcement.

%\paragraph{Lattice Boltzmann Method (LBM)}
%\eitan{keep only parts relevant to our method. if you just want to cite that LBM exists, that's just a couple sentences, not a paragraph.}
%The LBM has emerged as a powerful alternative for efficient and large-scale fluid simulations in graphics~\cite{chen1998lattice,li2020fast, chen2021gpu, li2023high}.
%Unlike conventional methods that directly solve the Navier-Stokes equations, 
%LBM simulates the microscopic dynamics of particle distributions on a discrete lattice structure.
%By forgoing a global solve, using a statistical representation of local flow instead, it inherently supports parallel processing, making it highly suitable for modern GPU architectures and enabling multi-GPU large-scale simulations.
%LBM offers several advantages, including efficient parallelizability, unstructured flow modeling, and stable pressure computation without requiring global solvers.
%However, it operates on fine-grained lattices and often requires high-resolution grids to accurately capture fine details, leading to high memory consumption during runtime.

\subsection{Spatial Order Reduction} \label{sec:related_roms}
In addition to the above parallelization approaches, reduced-order models present another promising direction for accelerating simulations. These methods aim to simplify the underlying computational models while preserving essential physical dynamics, thus improving speed and scalability. ROMs generally fall into two categories: \emph{data-free} and \emph{data-driven} methods.

\subsubsection{Data-free ROMs}
%\eitan{keep only parts relevant to our method. if you just want to cite that eigenfluids exists, that's just a couple sentences, not 3 paragraphs.}
Data-free ROMs rely on simplified physics-based formulations to reduce computational complexity without requiring precomputed datasets. Laplacian Eigenfluids \cite{de2012fluid, liu2015model, cui2018scalable} has emerged as a class of data-free reduced-order methods for fluid simulations, representing states using a set of basis functions that best span a given spatial domain.
%These basis functions, derived as eigenfunctions of the Laplacian operator, are inherently divergence-free, ensuring incompressibility without the need for a standard numerical pressure projection operation.

This approach demonstrates the feasibility of order reduction for fluid simulation. However, because their basis attempts to represent \emph{any} divergence-free field, handling intricate boundaries or capturing fine details often requires a large number of basis functions, and correspondingly large runtime and memory. If the general type of flow is known \emph{a priori}, which is often the case when an artist wants to apply simulation tools, these priors can be a powerful tool for improving computational efficiency.
%Subsequent works extended this idea by improving scalability and adaptability. \citet{liu2015model} introduced adaptive eigenfunctions to better handle dynamic boundaries and complex geometries, while \citet{cui2018scalable} addressed the memory limitations associated with using a large number of basis functions, though challenges still remain in fully resolving scalability issues.
%Despite these improvements, frequency-domain truncation still introduces artifacts, which highlight the need for approaches that balance memory efficiency, computational performance, and simulation quality. 

\subsubsection{Data-driven ROMs}
%\eitan{presumably this section should be relevant to us, but that connection needs to be made explicit throughout. also, average about one sentence per citation... if there are many more sentences that citations, it's likely too long for a related work section}
Data-driven ROMs, in contrast, leverage statistical techniques and precomputed datasets to approximate complex fluid dynamics. 
%The use of simulated data can be used to guide how simulation variables can be consolidated to reduce computational cost. 
Essentially, precomputed data already encodes essential structures of the fluid flow; one only needs to extract that flow and simplify the representation elsewhere. 

\citet{treuille2006model} used principal component analysis (PCA) to find this reduced basis by minimizing reconstruction error given some target dimension. 
\citet{wicke2009modular} introduced spatial generalization by replacing the global basis with 'tiles' of local bases coupled with shared boundary bases.
A particular drawback for existing data-driven methods is the lack of intuition for these spatial bases. 
This makes it difficult to pick out individual bases and modify them to manipulate the resulting fluid flow, a key application that we aim to tackle.
General adoption has thus been limited for existing data-driven ROMs, and are largely relegated to playback. 
%However, these globally learned bases carry a risk of lacking spatial generalization; \citet{wicke2009modular} handled this problem by using 'tiles' with local bases in place of the global bases, with coupling being handled by shared boundary bases.
%Building on this, \citet{kim2013subspace} introduced a cubature method inspired by similar approaches in the FEM community \cite{barbivc2005real, an2008optimizing} for reduced-order elastic simulations. 
%By approximating nonlinear advection schemes such as Semi-Lagrangian and MacCormack, the cubature method allowed for more accurate simulation within the reduced space.

%While data-driven methods have been previously explored, general adoption has been limited, largely due to the requirement of existing simulation data for learning. 
%This means that applications are generally limited to playback, with few tools for controlling the resulting flows. 
%The usage of a DMD basis provides us with a some \emph{knobs} for controlling the resulting simulation in a more intuitive manner, as well as potential for projecting higher order modes onto existing low-resolution simulation.

% \subsection{Learning Methods for Fast Fluid Simulation}
% Our method does not explicitly solve the PDE during time-stepping. Instead, it performs implicit time-stepping by approximating the Koopman operator from training data. This situates our work within the broader family of methods that learn physics directly from data.

% Within the graphics community, \cite{tao2024neural} proposed a reduced representation of fluid shapes to construct a reduced space, learning the ODE in this space using a neural network. Their approach, based on signed distance fields (SDFs), is particularly effective for simulations involving surfaces but less suited for free-surface scenarios, where our method offers greater flexibility. \changyue{will discuss these 2 later \citep{xie2018tempoGAN, chu:2017:cnnsmoke}}

% Outside the graphics domain, various approaches have been developed to approximate system physics, including Graph Neural Network-based models (e.g., \cite{sanchez-gonzalez2020gnngoogle}) and Neural Operators (e.g., \cite{li2021fourierneuraloperatorparametric}), just to list a few. Among these neural methods, the underlying physics are typically modeled using nonlinear neural networks. In contrast, our approach leverages a linear assumption to approximate the Koopman operator, enabling straightforward state inversion and the capability to take arbitrarily large time steps by calculating eigenvalue powers—features that remain challenging for nonlinear neural network-based systems.

\subsection{Spectral Methods}

%While most order-reduction methods, such as those mentioned in Section \ref{sec:related_roms}, focus on compressing data spatially, there is also a line of work on temporal reduced-order modelling. 
%These leverage the periodicity of fluid flows to represent the system's dynamics in the frequency domain, thus reducing computational complexity while retaining flow features.

%EG: in my mind this is a more powerful and compelling start, i've moved the "these leverage" sentence from above into the paragraph here
A key strength of spectral methods is the physical intuition of the basis functions. 
Whereas spatial order reductions construct basis functions based on the domain shape and are thus more physically intuitive for non-advecting phenomena such as elastic modes \cite{brandt2017compressed, sellan2023breaking}, spectral methods construct basis functions based on how the field evolves over time, 
%
%EG: moved it here:
%
leveraging the periodicity of fluid flows to represent the system's dynamics in the frequency domain.
%
Each mode thus corresponds to the propagation of a different wave, or coherent groups of waves. \citet{kim2008wavelet} leveraged this to add smaller-scale turbulence as a post-processing step, and \citet{chern2017inside} used this to directly modify vortex rings. 
This type of fluid control greatly motivates much of our application domains presented in Section \ref{sec:application}.


Prominently among these are Fourier-based spectral methods. \citet{stam2001simple} solves stable fluids in Fourier space, though only in a periodic domains due to the global nature of Fourier bases. 
In contrast, wavelet-based methods localize these bases into individual waves \cite{jeschke2018water}, allowing for better expressivity as well as multiscale features. 
Fourier methods also greatly simplify advection, which becomes a linear transform in Fourier space \cite{chern2016schrodinger}, further speeding up computation. We adopt a similar scheme, where advection is encoded directly in our DMD operator.


\subsection{Dynamic Mode Decomposition}
\label{sec:dmd_related_works}

%\eitan{I've revised this. I took the "key insight" part and moved it to the intro. I've reorganized this so that we emphasize the lack of usage in graphics by putting usage first. We then discuss the basic gist of it (although that really belongs in the method section) as a way of transitioning to the final paragraph, which concludes the related work section and leads to the method section.}

DMD is a data-driven method that approximates the Koopman operator, a linear operator that describes the temporal evolution of a dynamical system~\cite{schmid2010dynamic}.
Prior usage of DMD for fluids application has been limited, and have not been explored at all in computer graphics and creative applications.
Usage has primarily been limited to playback and short-term forecasting \cite{proctor2016dynamic}. 
Outside of fluid simulation, DMD has been used to process large datasets in high-dimensional systems \cite{williams2015data}, reducing noise in training data \cite{askham2018variable}, and constructing usable operators from noisy data \cite{sashidhar2022bagging}. This latter extension, called OptDMD, is of significant interest to us, as the turbulence inherent in fluid fields naturally produces noisy training data.

%DMD has been extended to account for actuators~\cite{proctor2016dynamic}, and to handle large datasets and high-dimensional systems, such as Extended DMD for nonlinear observables \cite{williams2015data}, OptDMD \cite{askham2018variable} for reducing noise in training data, and recently, BOPDMD \cite{sashidhar2022bagging} for using statistical ensemble to robustly construct meaningful operator from noisy dataset.
%These methods have been used to forecast fluid simulations, but their applications in graphics have been limited to playback and forecasting. 

Similar to other data-driven ROMs, DMD constructs spatial bases that encode the different flows present in the training set. 
Because it approximates the \emph{operator} that evolves a state forward in time rather than the state itself, it also gains the advantages of spectral methods. 
In particular, each basis becomes associated with different turbulent scales, opening the door to fluid control and upscaling applications. By computing the eigenvalues and eigenvectors of the Koopman operator, DMD can identify the dominant modes of a system and predict its future behavior.

We propose the use of Dynamic Mode Decomposition (DMD) as a tool for compressing and manipulating fluid simulations, combining the spatial compressive power of data-driven methods with the physical intuition and control of spectral methods. 

%EG: I've separated this because the citation to Schmid should NOT be confused with "physical intuition and control of spectral methods" and "compressing and manipulating fluid simulations" as it wasn't originally envisioned in a graphics context

