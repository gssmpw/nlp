\section{Conclusion}
We introduce \emph{Dynamic Mode Decomposition} (DMD) to the field of graphics. Unlike previous methods that approximate the solution space, our approach is an \emph{equation-free}, \emph{data-driven} method that learns a low-rank approximation of the state transition matrix directly from the dataset. Its linear nature enables rapid reconstruction of the dataset and shows superior reconstruction quality compared to previous data-driven methods. 
The operator linearity also unlocks a collection of artist-centric graphics applications: (1) by decomposing the dataset into a linear combination of temporally oscillating modes of different frequencies, our method enables artists to edit the dataset by adjusting the mode's modulus, growth/decay rate, and frequency, all without requiring knowledge of the solver's details; (2) with the inverse of the \koopman{}, we can go back in time and rollout animations at negative frame ranges; (3) users can supplement low-resolution flow sequences and generate new enriched flow details with the trained DMD operator, taking advantage of the high-frequency information stored within the operator.

