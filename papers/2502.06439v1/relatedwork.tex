\section{Background and related work}
\label{sec:rel-work}
The Italian car insurance market is regulated by the Italian Insurance Supervisory Authority (IVASS), which collaborates with international organizations like the International Association of Insurance Supervisors (IAIS) and the European Insurance and Occupational Pensions Authority (EIOPA) to ensure market stability and compliance with international standards. This framework prioritizes consumer protection and prohibits discriminatory practices in premium calculations based on personal characteristics. 
As a matter of fact, following a 2011 ruling by the European Court of Justice \cite{europeancourtofjusticeAssociationBelgeConsommateurs2011}, insurers cannot use gender when setting rates. The Article 16 of the European Directive 2021/2118 \cite{counciloftheeuropeanunionCouncilDirective20002000} stipulates that no premium surcharges may be applied on the basis of nationality.

The role of comparison websites is to act as an intermediary between customers and insurance providers. Generally, the insurance companies cover the charges while the customers use the services for no fee. Comparison websites are crucial in the Italian insurance market: during 2022, 54\% of people who took out car or motorbike insurance consulted a comparator\footnote{\href{https://assicurazioni.segugio.it/news-assicurazioni/polizze-auto-e-moto-la-comparazione-si-conferma-lo-strumento-di-riferimento-degli-italiani-00037593.html}{https://assicurazioni.segugio.it/news-assicurazioni/polizze-auto-e-moto-la-comparazione-si-conferma-lo-strumento-di-riferimento-degli-italiani-00037593.html}}. They mediate access to insurance not only by aggregating insurance options, but also by customizing offers by algorithmic optimization. Ongoing monitoring of these websites is essential to ensure they provide unbiased and accurate information, fostering a fair and competitive market.

Algorithmic audits draw from social sciences and can be defined as the collection and analysis of outcomes from an algorithm and a system, to evaluate its accountability \cite{goodmanellenp.ALGORITHMICAUDITINGCHASING2023}. The audit typically simulates a mock user population, with the objective of finding undesired patterns in models of interest \cite{vecchioneAlgorithmicAuditingSocial2021}.
Algorithmic audits have been frequently used to identify potential discrimination by AI systems. A largely influential work in the field of algorithmic audits is that conducted by Raji and Boulamwini, who described Gender Shades, an algorithmic audit of gender and skin type performance disparities in commercial facial analysis models \cite{rajiActionableAuditingInvestigating2019}. The study was replicated after three years by the same authors \cite{rajiActionableAuditingRevisited2022}, highlighting the importance of revisiting the algorithms even long after the original audits have been performed.
Other domains have been interested in algorithmic audits: e.g., Liu et al. have applied auditing to AI applications in the medical domain, to uncover potential algorithmic errors in the context of a clinical task and anticipate their potential consequences \cite{liuMedicalAlgorithmicAudit2022}. 
The auditing process can also involve other techniques than the simulation of inputs. Shen et al. have described the concept of everyday user algorithmic auditing, a process in which users detect, understand, and interrogate problematic machine behaviours via their day-to-day interactions with algorithmic systems \cite{shenEverydayAlgorithmAuditing2021}.
Various institutions have introduced bias auditing systems, as in the case of the New York City Council \cite{thenewyorkcitycouncilLocalLawAmend2021}, reinforcing the need for clearer definitions and metrics \cite{grovesAuditingWorkExploring2024}. In general, the importance of the audit process in exposing the limitations of deployed systems is widely recognised \cite{conitzerTechnicalPerspectiveImpact2022}.

% \mrcomment{Information about the original study and extensions}
The foundation of this research stems from a 2021 study \cite{fabrisAlgorithmicAuditItalian2021a}, which delves into the Italian car insurance industry. This work uncovers significant insights into the influence of \textit{gender} and \textit{birthplace} on car insurance pricing. Despite regulations prohibiting discriminatory practices, the audit revealed that algorithmic-mediated prices are still influenced by these factors. 
Specifically, foreign-born drivers and individuals from certain Italian cities faced price disadvantages, with Laos drivers being charged up to 1000 € more than drivers with similar profiles in Milan. Additionally, user profiles labelled by the platform as risky received fewer quotes.
Building upon this foundational work, the present article describes the evolution of discriminatory and opaque practices in car insurance pricing.

% \vspace{-3pt}