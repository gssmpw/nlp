\section{Related Work}
\paragraph{Multimodal Large Language Models.}
%
% \rscomment{This should be multimodal summarization or multimodal input to text}
After the emergence of LLMs, recent work~\citep{DBLP:conf/nips/LiuLWL23a, DBLP:journals/corr/abs-2311-03079, DBLP:conf/nips/AlayracDLMBHLMM22} investigated their use in processing multimodal inputs, giving rise to Multimodal Large Language Models (MLLMs).
%
The core idea in this line of research is to align visual and textual features by using shared representations.
%
This framework typically involves using a pre-trained visual encoder to extract visual features, a projection layer to map visual representations into corresponding text representations, and a pre-trained LLM to generate textual responses, allowing the model to condition the output on visual and textual inputs.
%
MLLM architectures such as LLaVA~\cite{DBLP:conf/nips/LiuLWL23a} and MiniCPM~\cite{yao2024minicpmvgpt4vlevelmllm} demonstrated impressive zero-shot generalization across diverse visual and language tasks.
%
However, most existing MLLMs focus on general domain tasks and relatively simple visual inputs; the challenge of understanding complex and information-dense visual documents like scientific posters remains under-explored.
%

%
\paragraph{Summarization in Scientific Domains.}
%
\emph{Scientific summarization} consists of generating concise summaries for scientific content~\citep{ScisummNet,DBLP:conf/emnlp/CacholaLCW20,DBLP:conf/emnlp/JuLKJDP21,DBLP:conf/naacl/SotudehG22}.
%
Several scientific summarization benchmarks have been proposed, %in the literature, 
designed to process modalities such as videos~\cite{DBLP:conf/acl/LevSHJK19,m3av}, slide decks~\cite{DBLP:conf/aaai/TanakaNNHSS23}, surveys~\cite{DBLP:conf/acl/LiuLYZJLH24}, 
and research papers~\cite{DBLP:conf/naacl/TakeshitaGR0P24,DBLP:conf/coling/PuWLD24}. 
%
However, scientific poster summarization remains unexplored despite the widespread use of posters in academic communication.

%
\paragraph{Document Layout Analysis and Segmentation.}
%
Understanding document layouts plays a significant role in processing complex visual documents like scientific posters.
%
Recent work in document layout analysis~\cite{ERNIE,DBLP:conf/acl/0005RSMBKPNL24,DBLP:conf/cvpr/LuoSZZYY24,DocFormerv2} aims at identifying and classifying different regions within a document considering spatial relationships and content type.
%
Previous work has also focused on understanding individual elements in documents, such as charts~\citep{masry-etal-2022-chartqa} and tables~\citep{zheng-etal-2024-multimodal}.
% 
However, most existing approaches are designed for either standard documents or individual elements like charts and tables and do not capture the complex layouts and the rich multimodal structure of scientific posters, which typically consist of text, charts, equations, and tables.
%

\begin{figure}[t]
  \centering
  \vspace{-2ex}
\includegraphics[width=\columnwidth]{figure/dataset_distribution/postersum_distribution_legend.pdf}
\vspace{-2ex}
  \caption{Distribution of the \textsc{PosterSum} dataset.}
  \label{fig:year_dist}
\end{figure}
%