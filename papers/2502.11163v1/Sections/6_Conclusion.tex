\section{Conclusion}

This study identifies three types of biases in VLM in geolocation tasks using {\methodname}, a framework comprising 1,200 images sourced globally from Google Street View.
The framework includes two subsets: the ``Depth'' subset, covering six countries and 60 cities, and the ``Breadth'' subset, spanning 43 countries and 60 cities.
Key findings from the evaluation of four VLMs are as follows:
(1) VLM predictions exhibit a bias toward larger cities, particularly in Brazil, Nigeria, and Russia.
(2) Higher-performing models show improved ability to discern subtle differences between cities.
(3) Accuracy consistently decreases in developing and less populous cities, with population size significantly influencing performance.
(4) Accuracy varies notably across cultural groups, with city-level accuracy differing by up to $19.1\%$.
Additionally, while VLMs demonstrate the capability to identify geographical locations, this raises privacy concerns, particularly regarding the potential exposure of personal geographical information in regions where models perform more accurately.

\section*{Limitations}

This study has several limitations.
(1) It does not investigate the underlying causes of biases in geographical information recognition.
We hypothesize that these biases arise from imbalanced training datasets, where biased data contribute to the VLM's performance disparities.
To test this hypothesis, we propose conducting comparative experiments using models trained on different datasets.
Specifically, future research could compare the performance of VLMs trained in China and the United States in recognizing cities within China, providing deeper insights into whether dataset imbalance is a primary factor.
(2) The evaluation does not include all countries globally.
While we acknowledge the importance of every country, budget constraints limited our evaluation to 111 cities across 43 countries.
To mitigate this limitation, we selected countries from diverse regions, cultures, and development levels to ensure broad coverage.
Future studies can extend the evaluation by leveraging the workflow outlined in this paper.

\section*{Ethics Statements}

\subsection*{License of Google Street View Images}
\label{sec:license}

In this section, we detail how our work adheres to the Google Street View terms of use.\footnote{\url{https://about.google/brand-resource-center/products-and-services/geo-guidelines}}
The terms impose four key restrictions, addressed as follows:
(1) ``Creating data from Street View images, such as digitizing or tracing information from the imagery.''
Our work does not store or release specific Street View images.
Instead, we report aggregated statistics derived from the collected images, with a few example images included solely for illustrative purposes in this paper.
(2) ``Using applications to analyze and extract information from the Street View imagery.''
We do not employ external applications for analysis.
Instead, we rely on algorithmic methods for visual understanding of the Street View images.
(3) ``Downloading Street View images to use separately from Google services (such as an offline copy).''
Our work utilizes images directly via the Street View API and does not distribute the images as a dataset.
Instead, we release only the geographic coordinates, requiring future users to access the same images through the Street View API.
(4) ``Merging or stitching together multiple Street View images into a larger image.''
We do not merge or stitch Street View images in any form.
By adhering to these restrictions, we ensure compliance with Google’s terms of use for Street View, consistent with prior research practices~\cite{fan2023urban, gebru2017using, ki2021analyzing}.

\subsection*{Privacy Issues}

Our work acknowledges the potential risk of malicious use, specifically the possibility that VLMs could be exploited to infer the locations of individuals through their publicly posted images.
We strongly oppose and do not condone any behavior or activities that misuse this technology for such purposes.
The intent of our research is to identify and highlight this potential problem within the context of academic and ethical research.
By raising awareness, we aim to foster further discussion and develop safeguards to prevent misuse.
Our goal is to advance understanding responsibly, without facilitating or endorsing any unethical applications of this technology.

% \section*{Acknowledgments}