\section{Introduction}

Visual Language Models (VLMs) have demonstrated the capability to comprehend visual content and respond to related queries~\cite{bubeck2023sparks, chow2025physbench}.
Their applications span text recognition~\cite{liu2024ocrbench, chen2025ocean}, solving mathematical problems~\cite{yang2024mathglm, peng2024multimath}, and providing medical services~\cite{azad2023foundational, buckley2023multimodal}.
Furthermore, recent research has identified their ability to infer geographic information about the location depicted in an image~\cite{wazzan2024comparing, mendes2024granular}.

However, the geographic information produced by VLMs often contains inaccuracies and significant biases~\cite{haas2024pigeon}.
These biases pose a critical issue, as they can perpetuate stereotypes about certain regions and amplify the dominance of specific areas in information dissemination~\cite{cinelli2021echo}.
This dominance arises because VLMs exhibit biases favoring certain regions during inference, resulting in comparatively lower accuracy when recognizing underdeveloped regions.
Through the mere exposure effect~\cite{zajonc1968attitudinal}, this imbalance strengthens users' impressions of cities that VLMs frequently or accurately identify, further entrenching these cities' dominance in information dissemination.

\begin{figure}[t]
    \centering
    \includegraphics[width=1.0\linewidth]{Figures/cover.pdf}
    \caption{The three types of biases identified in this paper. ``GT'' is the ground truth while ``Pre'' represents the VLM predictions.}
    \label{fig:cover}
\end{figure}

Existing studies~\cite{liu2024image, haas2024pigeon, yang2024geolocator} have explored the ability of VLMs to recognize geographic information from images but lack a sufficient attention to bias.
Specifically, these studies fail to thoroughly analyze the biases present in VLMs' geographic information recognition.
To address this gap, we conduct a systematic investigation into the capabilities and biases of VLMs in geographic information recognition.
We categorize VLM biases in geographic information recognition into two types: (1) disparities in accuracy when identifying images from different regions and (2) systematic tendencies to predict certain regions more frequently during geographic inference.
To evaluate these biases, we develop a benchmark, {\methodname}, comprising 1,200 images from 111 cities across 43 countries, sourced from Google Street View.\footnote{\url{https://www.google.com/streetview/}}
Each image is accompanied by detailed geographic information, including country, city, and street names.
{\methodname} incorporates an evaluation framework to automatically query VLMs, extract responses, and align them with ground truth data using name translation and deduplication.

The images are separated into two parts:
\textbf{(1) Depth}: To verify whether VLMs exhibit a tendency to predict famous cities for similar cities (\ie, cities within the same country), we select the six most populous countries from each continent and further choose ten cities from each country.
A biased model may predominantly predict well-known cities, such as Tokyo or Osaka for images of Japanese cities.
\textbf{(2) Breadth}: To explore countries with diverse cultures, populations, and development levels, we select 60 cities from a worldwide city list, ranked by population, with a maximum of two cities per country to prevent overrepresentation of highly populated nations.
Four VLMs—GPT-4o~\cite{openai2023gpt}, Gemini-1.5-Pro~\cite{pichai2024our}, LLaMA-3.2-11B~\cite{dubey2024llama3}, and LLaVA-v1.6-Vicuna-13B~\cite{liu2024visual}—are evaluated using {\methodname}.

We find that current VLMs exhibit notable biases in three key aspects:
\textbf{(1) Bias toward well-known cities}: For instance, Gemini-1.5-Pro frequently predicts S\~ao Paulo for images from Brazil.
While this indicates the model's ability to recognize Brazilian features, it lacks the capacity to capture regional diversity or subtle distinctions.
\textbf{(2) Disparities in accuracy across regions}: VLMs exhibit higher accuracy when identifying geographic information from images of developed regions, with an average accuracy of $48.8\%$, but their performance drops markedly for less developed regions, where accuracy typically falls to $41.7\%$.
\textbf{(3) Spurious correlations with development levels}: VLMs often associate urban or modern scenes—even from developing countries—with developed nations.
Conversely, images depicting suburban or rural views are frequently misclassified as originating from developing countries.

Our contributions in this paper are as follows:
\begin{enumerate}[leftmargin=*]
    \item We reveal, for the first time, biases in the geolocation capabilities of VLMs, which have the potential to perpetuate stereotypes among users.
    \item We develop and publish {\methodname}, a framework and dataset designed to facilitate future research.
    \item We evaluate the performance of four widely-used VLMs and provide in-depth analyses to better understand their behavior.
\end{enumerate}