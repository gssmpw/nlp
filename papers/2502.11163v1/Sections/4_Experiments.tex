\section{Experiments}

Using {\methodname}, we focus on addressing two key research questions in this section:
(1) Do VLMs exhibit preferences for specific cities within a shared cultural background, such as within a single country (\S\ref{sec:exp-depth})?
(2) How does accuracy vary across regions globally, considering economic, population or cultural differences (\S\ref{sec:exp-breadth})?

\subsection{Depth Evaluation}
\label{sec:exp-depth}

The ``Depth'' subset of {\methodname} includes the most populous countries from each continent: Australia (Oceania), Brazil (South America), the United States (North America), Russia (Europe), and Nigeria (Africa).
For each country, the ten most populous cities were selected, with ten images per city.
Fig.~\ref{fig:city-gpt-4o} presents the cities most frequently predicted by GPT-4o, while Fig.~\ref{fig:city-gemini}, \ref{fig:city-llama}, and \ref{fig:city-llava} in the appendix display results from Gemini-1.5-Pro, LLaMA-3.2-Vision, and LLaVA-v1.6-13B, respectively.
Table~\ref{tab:rq1} illustrates the accuracy of the four models in terms of continent, country, city, and street, across the six countries.
GPT-4o achieves the highest performance among the four models, outperforming the least accurate model, LLaVA, by improving continent, country, and city-level accuracy by $65.9\%$, $60.4\%$, and $37.4\%$, respectively.
Among the countries analyzed, VLMs most effectively recognize the U.S. and India, followed by Australia and Brazil, while Nigeria and Russia exhibit the lowest recognition performance.

\textbf{Bias toward larger cities is observed in VLMs predictions, particularly for Brazil, Nigeria, and Russia.}
For instance, in the Nigeria test set, Lagos images constitute 10\% of the dataset, yet GPT-4o predicts ``Lagos'' 131 times, representing 43.7\% of its responses.
However, Nigerian cities such as Nnewi or Uyo (the capital of Akwa Ibom) are never predicted by GPT-4o.
Similarly, in Brazil, Gemini-1.5-Pro predicts ``S\~ao Paulo'' 181 times, accounting for 60.3\% of its predictions.
For the Russia and India test sets, Moscow and Mumbai dominate VLM predictions.
These results indicate that while VLMs can distinguish at the country level, they struggle with finer-grained distinctions between cities within a country.
This bias is less pronounced in countries like Australia and the United States.
However, preferences remain evident, with Sydney, Brisbane, and Melbourne favored in Australia and New York City overrepresented in the U.S., despite seemingly more balanced predictions.

\begin{table}[t]
    \centering
    \resizebox{1.0\linewidth}{!}{
    \begin{tabular}{llccccccc}
        \toprule
        \multicolumn{2}{c}{\bf Models} & \bf Avg. & \bf Australia & \bf Brazil & \bf India & \bf Nigeria & \bf Russia & \bf USA \\
        \midrule
        \multirow{4}{*}{\rotatebox{90}{\bf GPT-4o}} & \bf Cont. & \bf 94.4 & 88.3 & 96.7 & \bf 99.3 & 95.0 & \bf 88.7 & 98.3 \\
        & \bf Ctry. & \bf 90.7 & 88.0 & 94.7 & \bf 97.0 & \bf 81.3 & \bf 86.0 & 97.3 \\
        & \bf City & \bf 40.4 & 45.0 & \bf 47.7 & 47.0 & \bf 22.0 & \bf 23.7 & 57.0  \\
        & \bf St. & \bf 0.6 & \bf 2.7 & \bf 0.3 & \bf 0.3 & 0.0 & \bf 0.3 & 0.0 \\
        \midrule
        \multirow{4}{*}{\rotatebox{90}{\bf Gemini}} & \bf Cont. & \bf 94.4 & \bf 91.0 & \bf 98.7 & 97.7 & \bf 98.0 & 81.0 & \bf 100.0 \\
        & \bf Ctry. & 86.2 & \bf 91.0 & \bf 96.0 & 92.3 & 77.7 & 60.3 & \bf 100.0 \\
        & \bf City & 35.4 & \bf 54.3 & 21.0 & \bf 49.3 & 14.7 & 15.3 & \bf 57.7 \\
        & \bf St. & 0.4 & 1.7 & 0.0 & 0.3 & 0.0 & 0.0 & \bf 0.3 \\
        \midrule
        \multirow{4}{*}{\rotatebox{90}{\bf LLaMA}} & \bf Cont. & 86.1 & 79.3 & 77.7 & 95.0 & 83.3 & 83.3 & 98.0 \\
        & \bf Ctry. & 75.4 & 77.7 & 71.0 & 93.3 & 38.3 & 76.7 & 95.3 \\
        & \bf City & 21.8 & 24.3 & 9.0 & 37.3 & 3.0 & 14.3 & 43.0 \\
        & \bf St. & 0.2 & 1.0 & 0.0 & 0.0 & 0.0 & 0.0 & 0.0 \\
        \midrule
        \multirow{4}{*}{\rotatebox{90}{\bf LLaVA}} & \bf Cont. & 34.0 & 3.3 & 38.7 & 39.0 & 39.0 & 32.7 & 51.3 \\
        & \bf Ctry. & 24.8 & 3.3 & 19.0 & 35.0 & 30.3 & 12.0 & 49.0 \\
        & \bf City & 3.0 & 0.7 & 1.3 & 5.0 & 3.0 & 1.7 & 6.3 \\
        & \bf St. & 0.0 & 0.0 & 0.0 & 0.0 & 0.0 & 0.0 & 0.0 \\
        \midrule
        \midrule
        \multirow{4}{*}{\rotatebox{90}{\bf Avg.}} & \bf Cont. & 77.2 & 65.5 & 77.9 & 82.8 & 78.8 & 71.4 & 86.9 \\
        & \bf Ctry. & 69.3 & 65.0 & 70.2 & 79.4 & 56.9 & 58.8 & 85.4  \\
        & \bf City & 25.2 & 31.1 & 19.7 & 34.7 & 10.7 & 13.8 & 41.0 \\
        & \bf St. &  0.3 & 1.3 & 0.1 & 0.2 & 0.0 & 0.1 & 0.1\\
        \bottomrule
    \end{tabular}
    }
    \caption{Accuracy of the four models in the ``Depth'' evaluation across the six countries. ``Cont.'' represents continent, ``Ctry.'' denotes country, and ``St.'' is street. Highest scores are marked in \textbf{bold}.}
    \label{tab:rq1}
\end{table}

\begin{figure*}[t]
    \centering
    \includegraphics[width=1.0\linewidth]{Figures/gpt-4o.pdf}
    \caption{The most frequently predicted cities by GPT-4o across six countries. Each country includes ten cities, with ten images per city used for testing. The maximum ``Correct'' score for a city is 30, as the VLMs have three attempts to predict the location.}
    \label{fig:city-gpt-4o}
\end{figure*}

\begin{table*}[t]
    \centering
    \resizebox{1.0\linewidth}{!}{
    \begin{tabular}{llcccccccccc}
        \toprule
        \multicolumn{2}{c}{\multirow{2}{*}{\bf Models}} & \multirow{2}{*}{\bf Avg.} & \multicolumn{2}{c}{\bf Economy} & \multicolumn{2}{c}{\bf Population} & \multicolumn{5}{c}{\bf Culture} \\
        \cmidrule(lr){4-5} \cmidrule(lr){6-7} \cmidrule(lr){8-12}
        & & & \bf Developing & \bf Developed & \bf Underpop. & \bf Populous & \bf Africa & \bf APSIDS & \bf EEG & \bf GRULAC & \bf WEOG \\
        \midrule
        \multirow{4}{*}{\rotatebox{90}{\bf GPT-4o}}
        & \bf Cont. & 90.1 & 87.1 & 96.0 & 88.1 & 93.1 & 83.1 & 91.5 & \bf 100.0 & 87.3 & 95.9 \\
        & \bf Ctry. & 81.3 & 77.8 & 88.5 & 75.3 & 90.4 & 64.4 & 85.2 & \bf 86.7 & 83.3 & 88.9 \\
        & \bf City & \bf 67.2 & \bf 64.3 & \bf 72.8 & \bf 61.1 & \bf 76.2 & 55.8 & \bf 64.2 & \bf 75.0 & \bf 73.3 & \bf 82.6 \\
        & \bf St. & \bf 3.2 & \bf 2.5 & \bf 4.5 & \bf 2.8 & \bf 3.8 & \bf 4.2 & \bf 2.1 & \bf 10.0 & \bf 2.3 & \bf 4.4 \\
        \midrule
        \multirow{4}{*}{\rotatebox{90}{\bf Gemini}}
        & \bf Cont. & \bf 95.6 & \bf 94.2 & \bf 98.2 & \bf 94.4 & \bf 97.4 & \bf 92.2 & \bf 96.2 & \bf 100.0 & \bf 93.7 & \bf 99.3 \\
        & \bf Ctry. & \bf 84.6 & \bf 81.7 & \bf 90.3 & \bf 79.4 & \bf 92.2 & \bf 73.3 & \bf 86.7 & 78.3 & \bf 85.7 & \bf 93.3 \\
        & \bf City & 61.9 & 61.7 & 62.5 & 57.5 & 68.6 & \bf62.2 & 56.5 & 66.7 & 66.3 & 71.9 \\
        & \bf St. & 2.5 & 2.0 & 3.5 & 2.2 & 2.9 & 2.5 & 1.6 & 6.7 & 0.7 & 6.3 \\
        \midrule
        \multirow{4}{*}{\rotatebox{90}{\bf LLaMA}}
        & \bf Cont. & 79.3 & 77.2 & 83.5 & 76.1 & 84.2 & 66.1 & 86.2 & 93.3 & 72.7 & 80.7 \\
        & \bf Ctry. & 60.1 & 53.6 & 73.2 & 52.9 & 71.0 & 40.8 & 65.4 & 70.0 & 57.0 & 71.1 \\
        & \bf City & 35.3 & 33.2 & 39.7 & 28.5 & 45.6 & 24.2 & 36.8 & 51.7 & 33.3 & 44.4 \\
        & \bf St. & 0.1 & 0.0 & 0.2 & 0.1 & 0.0 & 0.0 & 0.0 & 0.0 & 0.0 & 0.4 \\
        \midrule
        \multirow{4}{*}{\rotatebox{90}{\bf LLaVA}}
        & \bf Cont. & 44.4 & 40.3 & 52.7 & 39.8 & 51.4 & 17.5 & 52.6 & 95.0 & 33.3 & 57.0 \\
        & \bf Ctry. & 21.4 & 15.8 & 32.5 & 16.9 & 28.1 & 11.7 & 22.2 & 20.0 & 12.0 & 42.6 \\
        & \bf City & 11.8 & 7.7 & 20.2 & 6.9 & 19.3 & 7.2 & 11.1 & 6.7 & 6.7 & 27.0 \\
        & \bf St. & 0.0 & 0.0 & 0.0 & 0.0 & 0.0 & 0.0 & 0.0 & 0.0 & 0.0 & 0.0 \\
        \midrule
        \midrule
        \multirow{4}{*}{\rotatebox{90}{\bf Avg.}}
        & \bf Cont. & 77.3 & 74.7 & 82.6 & 74.6 & 81.5 & 64.7 & 81.6 & 97.1 & 71.8 & 83.2 \\
        & \bf Ctry. & 61.8 & 57.2 & 71.1 & 56.1 & 70.4 & 47.6 & 64.9 & 63.7 & 59.5 & 74.0 \\
        & \bf City & 44.1 & 41.7 & 48.8 & 38.5 & 52.4 & 37.4 & 42.2 & 50.0 & 44.9 & 56.5 \\
        & \bf St. & 1.4 & 1.1 & 2.0 & 1.3 & 1.7 & 1.7 & 0.9 & 4.2 & 0.8 & 2.8 \\
        \bottomrule
    \end{tabular}
    }
    \caption{Accuracy of the four models in the ``Breadth'' evaluation. ``Cont.'' represents continent, ``Ctry.'' denotes country, and ``St.'' is street. ``Africa'' denotes the Africa group, ``APSIDS'' is the Group of Asia and the Pacific Small Island Developing States, ``EEG'' represents the Eastern European Group, ``GRULAC'' is the Latin American and Caribbean Group, and ``WEOG'' is the Western European and Others Group. Highest scores are marked in \textbf{bold}.}
    \label{tab:rq2}
\end{table*}

\textbf{As model capabilities increase, VLMs demonstrate a greater ability to discern subtle differences between cities.}
Fig.~\ref{fig:city-llava} highlights the performance of the weakest model, LLaVA, which predicts S\~ao Paulo, Mumbai, Lagos, Moscow, and New York City as representative of Brazil, India, Nigeria, Russia, and the U.S., respectively.
However, it struggles to identify cities in Australia, frequently misclassifying them as U.S. cities such as New York City, Miami, San Francisco, or Los Angeles.
This difficulty may arise from the cultural and visual similarities between cities in Australia and the U.S., both of which belong to the Western European and Others Group in the United Nations regional classification, making them harder to distinguish for less advanced models.

Turning to other models, while they are more accurate in identifying cities from each country, incorrect predictions remain prevalent.
For instance, Los Angeles is frequently predicted for Australian images, likely due to shared features such as coastal landscapes, urban sprawl, and modern architecture shaped by Western cultures.
Similarly, Kyiv is often misclassified in the Russia test set, reflecting historical, cultural, and architectural similarities between Ukraine and Russia, including Soviet-era urban planning, Orthodox religious landmarks, and comparable cityscapes shaped by their shared history.
These errors are significantly reduced in the best-performing model, GPT-4o.

\subsection{Breadth Evaluation}
\label{sec:exp-breadth}

The ``'Breadth' subset of {\methodname} comprises 60 cities selected based on their population rankings, starting from the highest.
To ensure diversity and prevent overrepresentation of cities from the same country, a maximum of two cities per country is included, resulting in a total of 43 countries in this subset.
This extends beyond the six countries represented in the ``Depth'' subset.
To investigate regional variations in VLM predictions, each city is further classified based on its economic status, population size, and cultural context:
\textbf{(1) Economic status} is determined using a global ranking of cities by the number of millionaires.\footnote{\url{https://www.henleyglobal.com/publications/wealthiest-cities-2024}}
The top 50 cities on this list are categorized as ``Developed'' cities, yielding 20 developed cities and 40 developing cities in the subset.
\textbf{(2) Population size} is annotated based on a global population ranking of cities.\footnote{\url{https://worldpopulationreview.com/cities}}
Cities with populations exceeding 10 million are classified as ``Populous,'' resulting in 22 populous and 38 less populous cities.
\textbf{(3) Cultural classification}: Continents are usually deemed insufficient as a standard due to the cultural diversity within them.
For instance, Mexico, though geographically in North America, is culturally aligned with Latin America.
Similarly, the U.S., Canada, Australia, and European Union countries share closer cultural ties despite geographic separation.
Therefore, the United Nations Regional Groups\footnote{\url{https://en.wikipedia.org/wiki/United_Nations_Regional_Groups}} categorization is adopted, which categorizes countries into five culturally related groups: Africa Group, APSIDA, EEG, GRULAC, and WEOG.
Table~\ref{tab:rq2} provides the definitions of each group in its caption.

The results, categorized by economic, population, and cultural groups, are also presented in Table~\ref{tab:rq2}.
Overall, the accuracy, particularly at the city level, is higher in the ``Breadth'' evaluation (44.1\%) compared to the ``Depth'' evaluation (25.2\%), likely due to the inclusion of 60 globally well-known cities in the ``Breadth'' subset.
Unlike the ``Depth'' evaluation, where GPT-4o performed best, the ``Breadth'' evaluation shows comparable performance between Gemini-1.5-Pro and GPT-4o.
Gemini excels at identifying continents and countries, while GPT-4o demonstrates superior performance in recognizing cities.

Regarding biases toward developed, populous cities and those within specific cultural groups, the key findings are as follows:
\textbf{(1) All four models consistently demonstrate lower accuracy in developing and less populous cities, with population exerting a greater influence on performance.}
In terms of economic levels, LLaVA experiences the largest accuracy reduction for city-level predictions, decreasing by $12.5\%$ when shifting from developed to developing cities.
Conversely, Gemini is least affected, with only a $0.8\%$ drop at the city level, although its accuracy at the country level declines by $8.6\%$.
For population, the performance drop is more obvious.
VLMs exhibit a $12.4\%$ to $17.1\%$ decrease in city-level prediction accuracy when transitioning from more populous to less populous cities.

\textbf{(2) Accuracy varies significantly across cultural groups, with city-level accuracy differing by up to $19.1\%$.}
WEOG countries achieve the highest average city-level accuracy ($56.5\%$), followed by EEG ($50.0\%$), while the Africa Group exhibits the lowest accuracy ($37.4\%$).
This pattern is consistent across all four VLMs, highlighting the underrepresentation of African countries in VLMs' parametric knowledge.
Gemini demonstrates the smallest disparity in accuracy between the Africa Group and WEOG ($9.7\%$), whereas GPT-4o shows the largest disparity ($26.8\%$).
Further efforts in VLM development are expected to address and reduce these regional biases.

\begin{table}[t]
    \centering
    \begin{tabular}{lccc}
        \toprule
        \bf Model & \bf Continent & \bf Country & \bf City \\
        \midrule
        \bf GPT-4o & 86.0 & 74.0 & 63.3 \\
        \bf Gemini & \bf 93.3 & \bf 83.7 & \bf 64.3 \\
        \bf LLaMA & 76.7 & 59.0 & 32.3 \\
        \bf LLaVA & 45.0 & 21.0  & 11.0 \\
        \hdashline
        \bf Human & 33.7 & 9.5 & 1.7 \\
        \bottomrule
    \end{tabular}
    \caption{VLMs and human performance on a small subset (100 images) of {\methodname}. Highest scores are marked in \textbf{bold}.}
    \label{tab:human}
\end{table}

\subsection{User Study}

To demonstrate the difficulty of recognizing images in {\methodname}, we conduct a user study using a randomly sampled subset of 1,200 images.
From this subset, 100 images are selected and organized into ten questionnaires, each containing ten images.
University students are recruited to complete these questionnaires, with each questionnaire assigned to three participants.
Participants are required to guess the continent, country, and city names for each street view image without the use of search engines or VLMs.
An example questionnaire is provided in Fig.~\ref{fig:questionnaire} in the appendix.
Table~\ref{tab:human} reports human accuracy, \textbf{revealing significantly lower performance compared to VLMs.}
Specifically, the best-performing model, Gemini-1.5-Pro, outperformed humans by $59.6\%$, $74.2\%$, and $62.6\%$ in continent, country, and city-level predictions, respectively.
Most human participants report having no familiarity with the images and indicate that their responses are purely guesswork.
These findings highlight the superiority of VLMs' parametric knowledge over human capabilities.