\section{{\methodname} Framework}

This section introduces how we collect data, design queries, and evaluate responses from VLMs.

\subsection{Collecting Data}

Street view images can be efficiently collected using APIs provided by mapping applications.
In this study, we utilize the Google Street View API\footnote{\url{https://developers.google.com/maps/documentation/streetview/overview}} (2019 Version) and address compliance with its terms of use in the Ethics Statement section. 
Google ensures the blurring of personal identifiers, such as human faces and license plates, in its images.\footnote{\url{https://www.google.com/streetview/policy/}}
We begin by obtaining the central latitude and longitude coordinates of each city.\footnote{\url{https://simplemaps.com/data/world-cities}}
Using these coordinates, the API retrieves images along with their corresponding geographical data.
For each city, a total of 10 images are collected.

\subsection{Querying VLMs}

To instruct VLMs to better perform the geolocation task, we draw inspiration from strategies frequently employed by GeoGuessr players.\footnote{\url{https://www.reddit.com/r/geoguessr/comments/9hzqlv/how_do_you_play_geoguessr/}}\footnote{\url{https://www.reddit.com/r/geoguessr/comments/9cakwx/how_to_get_better_at_geoguessr/}}
In the prompt, VLMs are required to infer geographical locations based on image details, such as house numbers, pedestrians, signage, language, and lighting.
For convenient post-processing, VLMs are required to return a response in JSON format containing five key fields: ``Analysis,'' ``Continent,'' ``Country,'' ``City,'' and ``Street.''
When encoding images as inputs for VLMs, we ensure that all EXIF (Exchangeable Image File Format) metadata—such as time, location, camera parameters, and author information—is removed, as this data could enable VLMs to infer the location easily.
Then we extract answers from outputs and ensure they are neither unknown nor invalid.
Each model is allowed up to five attempts per image; if all five attempts yield invalid results, the image is marked as a failure.
To ensure experimental reliability, each image is required to obtain three responses generated by one model.
The specific prompt used in this task is outlined below:

\begin{table}[h]
    \centering
    \label{tab-example-prompt}
    \resizebox{1.0\linewidth}{!}{
    \begin{tabular}{lp{8.6cm}}
    \toprule
    \rowcolor{mygray}
    \multicolumn{2}{l}{\textbf{Prompt for Geolocation Task}} \\
    \textsc{System} & Please analyze the street view step-by-step using the following criteria: (1) latitude and longitude, (2) sun position, (3) vegetation, (4) natural scenery, (5) buildings, (6) license plates, (7) road directions, (8) flags, (9) language, (10) shops, and (11) pedestrians. Provide a detailed analysis based on these features. Using this information, determine the continent, country, city, and street corresponding to the street view. \\
    \midrule
    \textsc{User} & The location names should be provided in English. Avoid special characters in your response. Please reply in JSON format using this structure: {``Analysis'': ``YourAnswer'', ``Continent'': ``YourAnswer'', ``Country'': ``YourAnswer'', ``City'': ``YourAnswer'', ``Street'': ``YourAnswer''} \\
    \bottomrule
    \end{tabular}
    }
\end{table}

\subsection{Post-Processing}

Since the raw text may include variations in naming or translations of the same place, we utilize GPT-4o for semantic matching in addition to exact matching for the answers.
For each image, we first attempt exact matching; if it fails, GPT-4o is employed to identify valid matches through synonyms (\eg, New York and New York City), multilingual equivalents (\eg, \begin{CJK}{UTF8}{gbsn}北京\end{CJK}, Beijing in English), and historical toponyms (\eg, Bengaluru, previously known as Bangalore).