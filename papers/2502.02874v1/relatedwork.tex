\section{Background and Related Works}
%ML-based solutions have been explored for failure management in microwave networks. We start by reviewing ML studies on failure management in these networks, followed by a summary of relevant work at the intersection of FL, disaggregated networks, and failure management.
%\textbf{Application of ML in microwave networks}
\subsection{Background on Disaggregated Microwave Networks}
%\subsection{Disaggregated Microwave Networks}
\label{subsec:dissmic}
The concept of disaggregation in communication networks has been investigated extensively in the last decade. Disaggregation mainly aims to reduce CapEx and foster innovation thanks to the definition of open interfaces that facilitate competition. The main research initiatives for disaggregation in wireless networks are the O-RAN Alliance and the TIP’s Wireless Backhaul Group (WBG), a project group of TIP that investigates disaggregation in wireless backhaul (i.e., microwave networks). WBG particularly works on disaggregated white box solutions for microwave networks and has recently presented a solution called OpenSoftHaul (OSH) \cite{TIP_Wireless_Backhaul}. As illustrated in Fig. \ref{fig:microwavedisaggregated}, OSH decouples microwave equipment into three distinct interoperable components, full outdoor radios in the microwave and millimetre-wave frequencies (i.e., HW ODU), IDU equipment that has open interfaces (i.e, HW IDU) and operator configurable Network Operating System (i.e., NOS SW).
\begin{figure}[t]
	\centering
	\includegraphics[width=0.90\columnwidth]{Figures/MicrowaveDisaggregation.drawio.pdf}
	\caption{OSH's proposed disaggregation in microwave networks}
	\label{fig:microwavedisaggregated}
        \vspace{-20pt}
\end{figure}
Another significant initiative for disaggregation in the industry is Disaggregated Cell Site Gateways (DCSGs). A cell site gateway, or router, handles the aggregation of mobile data from cell sites and its transport to the service provider’s core network. Cell site routers (CSRs), also known as cell site gateways (CSGs), have traditionally been offered as integrated solutions by a limited number of large vendors, but this trend changes through disaggregation to avoid vendor lock-in. The main use cases for DCSG would be to perform 2G/3G/4G/5G mobile transport backhaul, but another DCSG  application is acting as IDU for Microwave Backhaul. In the case where IDU is disaggregated, a microwave modem is implemented as part of the ODU, reducing the IDU’s role to basic networking functions while ensuring seamless interoperability with other microwave equipment\cite{TelecomInfraProject2019}. 

\subsection{Related Works}
Recently, some ML-based solutions for predictive maintenance in microwave networks have been explored. For example, Ref.~\cite{9266116} illustrates a successful implementation (up to 93\% classification Accuracy) of supervised and unsupervised ML approaches for propagation failure identification (e.g., \emph{deep fading, interference}). Similarly, Ref.~\cite{datacentric, ASAP} shows successful implementation of ML-based classifiers (up to 96\% classification Accuracy) for hardware failure classification.   Ref.\cite{Ericsson2023Microwave} presents various preventive maintenance tasks, including hardware degradation detection, high-temperature early warning, and many specialized applications. 
%such as ML-driven optimization of timing for renewing water-repellent coatings on antennas to minimize snow and ice accumulation. 
Similarly, Ref.\cite{PAN2020106969} introduces the Proactive Microwave Anomaly Detection System (PMADS), a dynamic system designed for anomaly detection in cellular network microwave links. PMADS leverages network performance metrics and topological information to identify anomalies leading to faults. Other works (e.g., Refs.~\cite{AYOUB2022109466,9758095}) have investigated the application of eXplainable Artificial Intelligence (XAI) frameworks to provide insights into the decision-making process of ML classifiers. 

%However, all of the above-mentioned works consider monolithic scenarios, in which the data and models are managed by a single entity, with no consideration of disaggregation, and no consideration of collaborative data-sharing such as FL. In fact, only a limited number of studies have investigated the application of FL-based solutions in the context of microwave networks. 
However, these works focus on monolithic scenarios, where data and models are managed by a single entity, ignoring disaggregation and collaborative approaches like FL. Few studies have explored FL-based solutions for microwave networks.
The most closely related work is Ref.~\cite{9927592}, and it shows the application of an HFL-assisted solution for identifying the causes of propagation failures (e.g., extra attenuation, deep fading) in microwave networks within a multi-operator non-disaggregated scenario. While HFL performs effectively, it requires that all clients in the federated setting share the same feature space. This is a strict limitation in disaggregated networks, where managed network elements may possess varying feature sets, as in our scenarios. 

While VFL-based solutions have been explored in other contexts, such as O-RAN~\cite{FTL_Ref2} and Network Function Virtualization (NFV) systems~\cite{10329604}, there is no prior work, to the best of our knowledge, that has investigated the application of VFL in disaggregated microwave networks. 

%In \cite{FTL_Ref2}, authors proposed Vertical Federated Deep Reinforcement Learning (VFDRL) based resource allocation in O-RAN disaggregated architecture for network slicing. The authors developed two x-Application (xAPP) agents called power control xAPP \& slice-based resource allocation xAPP and trained in a federated setting.
%The identification of propagation failure causes (e.g., deep fading, extra attenuation, interference etc.) in microwave networks is explored in \cite{9266116}, examining both supervised and unsupervised approaches to address challenges in real deployments, such as the scarcity of labelled data.
%In \cite{Ericsson2023Microwave}, there are preventive maintenance tasks such as hardware degradation, high-temperature early warning and more specialized applications such as using ML to determine the optimal timing for renewing water-repellent coatings on antennas that are used for minimizing the accumulation of snow and ice. Also, the authors in \cite{PAN2020106969} introduce the Proactive Microwave Anomaly Detection System (PMADS), a dynamic anomaly detection system for cellular network microwave links. PMADS uses network performance information and topological information to detect anomalies that may eventually lead to faults. 
%Also, our previous work \cite{AYOUB2022109466} investigated advanced ML tasks in microwave networks where eXplainable Artifical Intelligence (XAI) is used to mitigate practitioners' doubts about the reasoning of ML models again in failure cause identification problem. They used SHAP on both feature selections to determine the most influential ones and to examine instances where the model misclassifies data.

%\subsection{Application of Federated Learning in Microwave Networks}
%Differently from the above-mentioned traditional ML-based studies, where solutions rely on centralized data, FL, as an emerging approach that allows local data storage, has been merely investigated in microwave networks. Thence, we list some of the works that focus on FL-based solutions in disaggregated wireless networks. 
%In \cite{FTL_Ref2}, authors proposed Vertical Federated Deep Reinforcement Learning (VFDRL) based resource allocation in O-RAN disaggregated architecture for network slicing. The authors developed two x-Application (xAPP) agents called power control xAPP \& slice-based resource allocation xAPP and trained in a federated setting. %The most related work is \cite{9927592}, where authors proposed an HFL-assisted solution for propagation failure-cause identification in microwave networks for classifying the propagation failures(e.g. extra attenuation, deep fading etc.) considering the multi-operator non-disaggregated scenario. HFL works well but it requires all clients in the federated setting to have the same feature space. This requirement can be limiting in disaggregated networks, where the managed NEs may have varying feature sets.
%Addressing this limitation, we propose VFL based solution for hardware failure-cause identification in disaggregated microwave networks. To the best of our knowledge, the disaggregation concept is unexplored in microwave networks and compared to all previous works for FL in communication networks, we employ GBDT-based VFL for the first time and examined it compared to SplitNN-based VFL as well as baselines.