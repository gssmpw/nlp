\begin{table*}[h]
\centering
\resizebox{\linewidth}{!}{\small
\begin{tabular}{p{1in}|p{5.4in}}
\toprule
\multicolumn{2}{c}{\textbf{Prompts for Dataset Curation}} \\
\midrule

\textbf{Data Extration} & 
    (Continuing from the previous Table)
            
            \quad "conversation": [\{
                "scenario": "The scenario at the start of this conversation (providing as much context as possible, but excluding details conveyed in the following conversation)",
                "topic": "The topic of the conversation (~10 words)", 
                "key\_characters": [
                    \{
                        "name": "Character name",
                        "motivation": "The thought of the character before starting the conversation, including their attitudes, feelings, motivations, goals, information to convey or topics to be discussed",
                    \}
                ],
                "dialogues": [
                    \{
                        "character": "Character name",
                        "message": "Message, each utterence is composed of thoughts, speech and actions. Use [thought] for internal thoughts, like "[feeling happy]", which others can't see. Use (action) for visible actions, like "(silence)" or "(smiles at you)". Each response starts with the character's internal thought before their speech and actions."
                    \}
                ]
            \}],
            
            \quad "state": "finished" or "truncated"
        \}
    ],
    
    "next\_chunk\_start": "The first sentence of the next chunk."
    
\}

===Requirements===

1. Adhere strictly to the specified output JSON format. 

2. [IMPORTANT] Ensure all DOUBLE QUOTES within all STRINGS are properly ESCAPED, especially when extracting from the text.

3. In the OUTPUT, use characters' full names, omitting any titles.

4. Maintain Story Fidelity: The plot must accurately reflect the book's content. Avoid introducing plots that are out of context. If the plot contains multiple conversations, prioritize the original dialogue from the book. In the absence of explicit conversations, create dialogue that aligns closely with the plot details.

===Input===

==Book title==
\{book['title']\}

==Author==
\{book['author']\}

==Chunk of Book Content== 
\{chunk\}

==Truncated plot from previous chunk (to be finished)==

\{json.dumps(truncated\_plots, ensure\_ascii=False, indent=2) if truncated\_plots else "None"\}
    \\ \midrule
    
\textbf{Enhance Conversation Settings} & Given a conversation from \{book\}, enhance the scene setup and characters' thoughts to create a comprehensive foundation for dramatic performance, i.e., to provide necessary background for actors to act out the conversation:

1. Review the provided conversation and contextual details thoroughly.

2. Expand the 'scenario' with rich situational context that actors need to convincingly perform the scene. Focus on essential background information, while excluding future details to be portrayed in the conversation.

3. Enhance each character's 'thought' section with their complete mental and emotional state, including their feelings, ideas, objectives, topics they want to discuss, and information they want to convey. Align with their established character and role in the plot. 

===Output Format===

Please provide the output in the following JSON format:

\{

\quad "scenario": "A detailed scene-setting description that provides actors with essential context and atmosphere ($<$ 200 words). Include all necessary background information while excluding future information to be revealed in the conversation.",
    
    \quad "key\_characters": [\{ "name": "Character name",
            "motivation": "The character's complete mental and emotional state before the conversation ($<$ 100 words). Including their feelings, motivations, objectives, and information they want to convey or discuss." ...
            
\}],\}

===Requirements===

... (to be continued in the next Table)
\\ 
    
\bottomrule

\end{tabular}}

\caption{Prompts for dataset construction in \method. }
\label{tab:prompts_data_2}
\end{table*}
