\begin{table*}[h]
\centering
\resizebox{\linewidth}{!}{\small
\begin{tabular}{p{1in}|p{5.4in}}
\toprule
\multicolumn{2}{c}{\textbf{Prompts for Dataset Curation}} \\
\midrule

\textbf{Enhance Conversation Settings} & 
    (Continuing from the previous Table)

1. Adhere strictly to the specified output JSON format. 

2. [IMPORTANT] Ensure all DOUBLE QUOTES within all STRINGS are properly ESCAPED, especially when extracting from the text.

3. In the OUTPUT, use characters' full names, omitting any titles.

4. Maintain Story Fidelity: The plot must accurately reflect the book's content. Avoid introducing plots that are out of context. If the plot contains multiple conversations, prioritize the original dialogue from the book. In the absence of explicit conversations, create dialogue that aligns closely with the plot details.

===Input===

==Book title==

{book['title']}

==Author==

{book['author']}

==Chunk of Book Content== 

{chunk}

==Truncated plot from previous chunk (to be finished)==

{json.dumps(truncated\_plots, ensure\_ascii=False, indent=2) if truncated\_plots else "None"}
    \\ \midrule
    
\textbf{Unify Character Names} & 

Given a list of character names, titles, or form of address, your task is to: i) generate a list of named characters with their official names (in \{language\}); ii) For each name in the given list, align it with the official character name if it refers to a named character, or denote it as "impersonal" otherwise.

===Output Format===

Please provide the output in the following JSON format:

\{
    "named\_characters": [
        The list of named characters with their official names. Each character should appear only once. 
    ],
    "to\_official\_name": \{
        "The name in the list": "The official name of the character, or 'impersonal' if it does not refer to a named character."
    \}
\}
===Input===
{character\_names}
    \\ \midrule

\textbf{Generate Character Profiles} & 
Please provide a concise, narrative-style character profile for {character\_name} from "{book\_title}". The profile should read like a cohesive introduction, weaving together the character's background, physical description, personality traits and core motivations, notable attributes, relationships, key experiences, major plot involvement and key decisions or actions, character arc or development throughout the story, and other important details. 
    
The profile should be written in a concise yet informative style, similar to what one might find in a comprehensive character guide, in {language}. Focus on the most crucial information that gives readers a clear understanding of the character's significance in the work. 

You will be provided with summaries and dialogues of some key plots in the book as reference. The profile should be based on either your existing knowledge of the character or the provided information, without fabricating or inferring any inaccurate or uncertain details. 

{character\_data}

Now, please generate the character profile, starting with ===Profile===.
\\ 
\bottomrule

\end{tabular}}

\caption{Prompts for dataset construction in \method. }
\label{tab:prompts_data_3}
\end{table*}
