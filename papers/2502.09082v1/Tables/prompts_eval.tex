\begin{table*}[h]
\centering
\resizebox{\linewidth}{!}{\small
\begin{tabular}{p{1in}|p{5.4in}}
\toprule
\multicolumn{2}{c}{\textbf{Prompts for Penalty-based LLM Critics}} \\
\midrule

\textbf{Template} & You are a literary critic specializing in character analysis and dialogue evaluation. Given a simulated conversation for a plot in \{book\}, your task is to evaluate this conversation via the following steps:

1. Read and understand the provided materials about \{book\}:

   * Story context and scenario.
   
   * Profiles of the main characters, including {major\_characters}.
   
   * The original conversation from {book} in the same scenario as a reference.

  2. Evaluate the simulated conversation in terms of \{dimension\_name\}, i.e., \{dimension\_intro\}. 
  
   Note that, each character message is composed of speech, action (wrapped within (...) ), and inner thoughts (wrapped within [...] ). The inner thoughts are not spoken aloud and are thus invisible to other characters. 
   
   The detailed evaluation criteria will be provided below.

    \quad 
    
    \textbf{\textit{(if k$>$0)}}
   
   Please note that the first \{k\} messages in the simulated conversation are the same as the reference. Focus your evaluation only on the content after these messages.
   
   \quad 

\#\# Scenario

\#\#\# Plot Summary

\{plot\_summary\}

\#\#\# Current Scenario

\{scenario\}

\#\# Character Profiles

\{character\_profiles\}

\#\# Original Conversation

\{original\_conversation\}

\#\# Evaluation Criteria

To evaluate the simulated conversation, identify the following types of flaws:

\{dimension\_rubrics\}

\#\# Scoring Guidelines

1. Identify all instances of flaws occurred in the simulated conversation.
      
2. For each flaw identified, determine its level of severity into 1 to 5, where 1 indicates minor, 3 indicates moderate, and 5 indicates severe.
   
\#\# Output Requirements

Provide your evaluation in JSON format:

Example Output:

\{

    \quad "\{dimension\_name\}": \{
    
        \qquad"flaws": [ 
          \{
            "instance": $<$comment on the flaw instance$>$, 
            "type": $<$flaw type$>$, 
            "severity": $<$range from 1 (minor) to 5 (severe)$>$
          \},\},
    
\}

===Dialogue Content===

\\ 
    
\bottomrule

\end{tabular}}

\caption{Prompts for penalty-based LLM critics in \method. }
\label{tab:prompts_eval}
\end{table*}

