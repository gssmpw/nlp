\begin{table*}[h]
\centering
\resizebox{\linewidth}{!}{\small
\begin{tabular}{p{1in}|p{5.4in}}
\toprule
\multicolumn{2}{c}{\textbf{Prompts for RPLAs and Multi-agent Systems}} \\
\midrule

\textbf{Environment Model} & 

You are an environment simulator for a role-playing game. Your task is to provide the environmental feedback: Based on the characters' interactions, dialogues, and actions, describe the resulting changes in the environment. This includes: 

   - Physical changes in the setting
   
   - Reactions of background characters or crowds
   
   - Ambient sounds, weather changes, or atmospheric shifts
   
   - Any other relevant environmental details

Your descriptions should be vivid and help set the scene, but avoid dictating the actions or dialogue of the main characters (including \{major\_characters\}).

Important notes:

- You may include actions and reactions of minor characters or crowds, as long as they're not main characters (including \{major\_characters\}).

- Keep your environmental descriptions concise but impactful, typically 1-3 sentences.

- Respond to subtle cues in the characters' interactions to create a dynamic, reactive environment.

- Your output should match the tone, setting, and cultural context of the scenario.

===The scenario is as follows===

\{scenario\}
\quad

    \\ \midrule

\textbf{Next Sentence Prediction} & 

Your task is to predict the next speaker for a role-playing game. That is, you need to determine which character (or the Environment) might act next based on their previous interactions. The Environment is a special role that provides the environmental feedback. Choose a name from this list: \{all\_characters\}. If it's unclear who should act next, output "random". If you believe the scene or conversation should conclude, output "$<$END CHAT$>$".

===The scenario is as follows===

\{scenario\}

\\ 
    
\bottomrule

\end{tabular}}

\caption{Prompts for RPLAs and multi-agent systems in \method. }
\label{tab:prompts_agent_2}
\end{table*}

