\begin{table*}[h]
\centering
\resizebox{\linewidth}{!}{\small
\begin{tabular}{p{1in}|p{5.4in}}
\toprule
\multicolumn{2}{c}{\textbf{Prompts for Dataset Curation}} \\
\midrule

\textbf{Data Extration} & 

Based on the provided book chunk, complete the following tasks:

1. Recognize chapter beginnings if they exist in the chunk. Identify the starting sentence of that chapter.

2. Identify the important plots in this chunk. Identify the beginning and ending of each plot by its first and last sentence. Determine the chapter title that the plot belongs to. Set "state" as "truncated" if the plot is truncated in this chunk, or "finished" otherwise. You will be provided with the truncated plots from the previous chunk, and you **must** extend the conversations with the current chunk while keeping the **scenario** unchanged. 

3. Summarize each important plot. For each plot, generate its summary, score its prominence from 1 to 100, and list the key characters and their roles, thoughts and actions in it.

4. Extract conversations for each plot. First, state the scenario and topic of the conversations. Then, list the key characters with their names, descriptions and thoughts at this point. Finally, extract the conversations among them based on the following requirements: 

\quad i) Ensure the conversations are faithful to the plot and characters. They should be based on the original conversations in the text as much as possible. 
    
\quad ii) The conversations should be complete, covering the key dialogues and information. Each conversation should contain at least 10 utterances.
    
\quad iii) Each utterance should be composed of one or more thoughts, speech and actions. Use [] outside thoughts, like "[I feel fear and anger, but I cannot show it. I must remain calm and carefully handle his volatile temper.]", which others can't see. Use () outside actions, like "(silence)" or "(smiles at you)," which others can see. Always start an utterance with the character's thought. 
    
\quad iv) [IMPORTANT] When generating thoughts, you should think from the characters' perspectives, analyzing the internal thoughts behind their speech and actions in the original text. These thoughts should reflect aspects such as their personal background, personality, values, relationships with others, motivations, and goals. Each thought should be expressed as a phrase or sentence, rather than an adjective or adverb. 
    
\quad v) Additionally, describe environmental information (such as scenes, atmosphere, sudden events, etc.) of the conversations as an "utterance" where the "character" field is set as "Environment". The information should exclude characters' active thoughts, observations, and actions.
    
\quad vi) Keep the conversation in the same language as the chunk. 

5. Identify the optimal starting point for the subsequent chunk. If the last storyline has been extracted as an truncated plot, set next\_chunk\_start as None. Otherwise, set next\_chunk\_start as the first sentence of the last storyline. 

===Output Format===

Please provide the output in the following JSON format:

\{

    "chapter\_beginnings": [
        \{
            "beginning\_sentence": "Exactly the first line of this chapter (namely the title)."
        \}
    ],
    
    "plots": [
        // Extend the truncated plots from previous chunk, if any
        \{
            ...
        \}, 
        // New plots in this chunk
        \{
            
            \quad "chapter\_title": "The chapter title that the plot belongs to. Output None if not found.",
            
            \quad "first\_sentence": "Exactly the first sentence of the plot in this **chunk**.",
            
            \quad "last\_sentence": "Exactly the last sentence of the plot in this **chunk**. If the plot is truncated in this chunk, provide the last sentence of this chunk. ",
            
            \quad "prominence": "Whether this plot is recognized to fans of this book, from 1 to 100.",
            "summary": "The summary of the plot. Just summarize, do not extend unrelated discussions.",
            
            \quad "key\_characters": [
                \{
                    "name": "Character name",
                    "description": "The description of the character before this plot ($~$20 words).",
                    "summary": "The summary of the character's role, thoughts and behaviors towards this plot, and any significant character development relevant to the plot ($~$30 words).",
                \}
            ],
            
            ... (to be continued in the next Table)
\\ 
\bottomrule

\end{tabular}}

\caption{Prompts for dataset construction in \method. }
\label{tab:prompts_data}
\end{table*}

