\section{Related Work}

\begin{table*}[t]
    \centering
    \setlength{\tabcolsep}{4pt}
    \renewcommand{\arraystretch}{0.9}
    \resizebox{\textwidth}{!}{%
    \begin{tabular}{@{}lccccccccccccc@{}}
    \toprule
            \multirow{2}{*}{\textbf{Dataset}} & \textbf{Book} & \multicolumn{3}{c}{\textbf{Character}} & \multicolumn{5}{c}{\textbf{Conversation}} & \multicolumn{3}{c}{\textbf{Message}} & \textbf{Plot} \\
            \cmidrule(lr){2-2} \cmidrule(lr){3-5} \cmidrule(lr){6-10} \cmidrule(lr){11-13} \cmidrule(r){14-14} 
            & \textbf{Num.} & \textbf{Num.} & \textbf{Profile} & \textbf{Expr.} & \textbf{\#Conv.} & \textbf{\#Turns} & \textbf{Setting} & \textbf{Auth.} & \textbf{Multi-Chara.} & \textbf{Speech} & \textbf{Thought} & \textbf{Action} & \textbf{Summ.} \\ \midrule 
    Charater-LLM & & 9 & \checkmark & & 14,300 & 13.2 & \checkmark & & & \checkmark & & & \\
    ChatHaruhi & & 32 & \checkmark & & 54,726 & $>$2 & & \checkmark* & \checkmark & \checkmark & & & \\
    RoleLLM & & 100 & \checkmark & & 140,726 & 2 & & & & \checkmark & & & \\
    HPD & 7 & 113 &  & & 1,191 & 13.2 & \checkmark & \checkmark & \checkmark & \checkmark & & & \checkmark \\
    LifeChoice &  388 & 1,462 & \checkmark & & 1,462 & 2 & \checkmark & \checkmark & & & & & \\
    CroSS-MR & 126 & 126 & \checkmark & & 445 & 2 & \checkmark & \checkmark & &  & & & \\
    CharacterGLM & & 250 & \checkmark & & 1,034 & 15.8 & \checkmark & & & \checkmark & & & \\
    CharacterEval & & 77 & \checkmark & & 1,785 & 9.3 & \checkmark &  \checkmark & & \checkmark &  & \checkmark & \\
    DITTO & & 4,002 & \checkmark & & 7,186 & 5.1 & & & & \checkmark & & & \\
    MMRole & & 85 & \checkmark & & 14,346 & 4.2 & & & & \checkmark & & & \\
    CharacterBench & &3,956 & \checkmark & & 13,162 & 11.3 & & & & \checkmark & & & \\
    \method & 771 & 17,966 & \checkmark & \checkmark & 29,798 & 13.2 & \checkmark & \checkmark & \checkmark & \checkmark & \checkmark & \checkmark & \checkmark\\
    
    \bottomrule
    \end{tabular}%
    }
    \caption{Overview of \method and existing RPLA datasets. 
    For characters, Num. count characters with profiles, and Expr. denotes structured character experiences. 
    For conversations, Auth. indicates authentic dialogues or behaviors from the books, and Multi-Chara. denotes involving more than 2 characters. Num. (number), Conv. (conversation), and Summ. (summary) are abbreviations. 
    }
    \label{tab:dataset_stats}
\end{table*}


An RPLA leverages LLMs to create a simulated persona $\agent_\persona$ that emulates a real character $\persona$ based on its persona data $\personadata_\persona$. 
Effective RPLAs require both comprehensive, high-quality data $\personadata_\persona$ and advanced role-playing LLMs.
Among various persona types, we focus on RPLAs for established characters, which should faithfully align with their characters’ complex backgrounds and nuanced personalities. 


\textbf{Datasets for RPLAs} \quad 
Persona data $\personadata_\persona$ describe the real persona $\persona$ through various representations, including profiles~\citep{yuan2024evaluating}, dialogues~\citep{wang2023rolellm}, experiences~\citep{li2023chatharuhi} and multimodal information~\citep{dai2024mmrolecomprehensiveframeworkdeveloping}, \etc. 
As shown in Table ~\ref{tab:dataset_overview}, existing datasets have several limitations. 
\textit{1)} Many are synthesized via LLMs' responses to general instruction sets~\citep{wang2023rolellm} or character-specific questions~\citep{shao2023character}, such as RoleBench~\citep{wang2023rolellm}.  
However, LLM-synthesized data compromise authenticity and fidelity to original sources.
\textit{2)} Human-annotated datasets such as CharacterEval~\citep{tu2024charactereval} and CharacterDial~\citep{zhou2023characterglm} offer improved quality, but are expensive and difficult to scale.
\textit{3)} Several efforts extract authentic dialogues from fictional works, such as ChatHaruhi~\citep{li2023chatharuhi} and HPD~\citep{chen2023large}. However, they rely on human efforts for individual sources and are hence hard to scale as well.
\textit{4)} Furthermore, existing datasets offer limited representations and forms, \ie, mainly consisting of two-character or user-character question-answer pairs.
These datasets support various purposes, including prompting, training, retrieval augmentation, and evaluation of RPLAs.


\textbf{Evaluation for RPLAs} \quad
Existing evaluation methods are based on either LLM judges or multi-choice questions~\citep{chen2024from}. 
LLM-judged methods typically elicit LLMs’ role-playing performance via predefined questions, and score the performance using LLM judges or reward models~\citep{chen2024from}.
They assess various dimensions, including character-independent aspects such as anthropomorphism~\citep{tu2024charactereval} and attractiveness~\citep{zhou2023characterglm}, as well as character-specific traits such as language style, knowledge, and personality~\citep{wang2023rolellm, shao2023character}. 
However, LLM judges suffer from inherent biases, \eg, length and position bias~\citep{li2024judgesurvey}, and may lack the necessary knowledge for character-specific evaluation.
Other benchmarks evaluate role-playing LLMs through multiple-choice questions, assessing specific aspects such as knowledge~\citep{shen2023roleeval}, decision-making~\citep{xu2024character}, motivation recognition~\citep{yuan2024evaluating}, and personality fidelity~\citep{wang2024incharacter}.
