\onecolumn
\appendix

\section{Dataset}
\label{sec:app_dataset}


\subsection{Statistics and Analysis}
\label{sec:data_statistics}
As shown in Table \ref{tab:dataset_overview}, \method dataset is extensive and comprehensive, encompassing dialogue data from 771 books and 17,966 distinct characters. 
The dataset includes 30,069 unique plots and 29,798 conversations. 
On average, each conversation consists of approximately 13.2 utterances, with the entire dataset comprising a total of 392,900 utterances.

\begin{table}[htbp]
\centering
\begin{tabular}{|c|c|c|c|c|}
\hline
\textbf{\#Book} & \textbf{\#Plot} & \textbf{\#Conversation.} & \textbf{\#Chararacter} & \textbf{\#Utterance} \\ \hline
771     &   30,069      & 29,798                   & 17,966               & 392,900              \\ \hline
\end{tabular}
\caption{Statistics of \method Dataset.
}
\label{tab:dataset_overview}
\end{table}

Our book selection is derived from the \textit{Best Books Ever} list on \textit{Goodreads}, a curated collection of globally acclaimed literary works. 
These novels have garnered widespread recognition and appreciation from readers worldwide.
Table ~\ref{tab:selected_books} presents a comprehensive list of the top 100 books from our selection.

\begin{table*}[htbp]
\small
\centering
\resizebox{\linewidth}{!}{
\begin{tabular}{p{3.2in}|p{3.2in}}
\toprule
\multicolumn{2}{c}{\textbf{Selected Books}} \\  \hline
\textbf{1}. \textit{{The Hunger Games (The Hunger Games, \#1)}} & \textbf{2}. \textit{{Harry Potter and the Order of the Phoenix (H. P., \#5)}}\\ \hline
\textbf{3}. \textit{{Pride and Prejudice}} & \textbf{4}. \textit{{To Kill a Mockingbird}}\\ \hline
\textbf{5}. \textit{{The Book Thief}} & \textbf{6}. \textit{{Animal Farm}}\\ \hline
\textbf{7}. \textit{{The Chronicles of Narnia (\#1-7)}} & \textbf{8}. \textit{{The Fault in Our Stars}}\\ \hline
\textbf{9}. \textit{{The Picture of Dorian Gray}} & \textbf{10}. \textit{{Wuthering Heights}}\\ \hline
\textbf{11}. \textit{{Gone with the Wind}} & \textbf{12}. \textit{{The Perks of Being a Wallflower}}\\ \hline
\textbf{13}. \textit{{The Lightning Thief (Percy Jackson and the Olympians, \#1)}} & \textbf{14}. \textit{{The Little Prince}}\\ \hline
\textbf{15}. \textit{{The Great Gatsby}} & \textbf{16}. \textit{{Crime and Punishment}}\\ \hline
\textbf{17}. \textit{{Memoirs of a Geisha}} & \textbf{18}. \textit{{Les Misérables}}\\ \hline
\textbf{19}. \textit{{The Alchemist}} & \textbf{20}. \textit{{Lord of the Flies}}\\ \hline
\textbf{21}. \textit{{The Hitchhiker’s Guide to the Galaxy (\#1)}} & \textbf{22}. \textit{{The Help}}\\ \hline
\textbf{23}. \textit{{Dracula}} & \textbf{24}. \textit{{Ender’s Game (Ender's Saga, \#1)}}\\ \hline
\textbf{25}. \textit{{Of Mice and Men}} & \textbf{26}. \textit{{One Hundred Years of Solitude}}\\ \hline
\textbf{27}. \textit{{Brave New World}} & \textbf{28}. \textit{{A Thousand Splendid Suns}}\\ \hline
\textbf{29}. \textit{{The Time Traveler’s Wife}} & \textbf{30}. \textit{{The Princess Bride}}\\ \hline
\textbf{31}. \textit{{The Secret Garden}} & \textbf{32}. \textit{{The Outsiders}}\\ \hline
\textbf{33}. \textit{{A Game of Thrones (A Song of Ice and Fire, \#1)}} & \textbf{34}. \textit{{Little Women}}\\ \hline
\textbf{35}. \textit{{A Wrinkle in Time (Time Quintet, \#1)}} & \textbf{36}. \textit{{The Odyssey}}\\ \hline
\textbf{37}. \textit{{Harry Potter and the Deathly Hallows (H. P., \#7)}} & \textbf{38}. \textit{{Frankenstein: The 1818 Text}}\\ \hline
\textbf{39}. \textit{{The Kite Runner}} & \textbf{40}. \textit{{The Handmaid’s Tale (The Handmaid's Tale, \#1)}}\\ \hline
\textbf{41}. \textit{{The Lovely Bones}} & \textbf{42}. \textit{{The Adventures of Huckleberry Finn}}\\ \hline
\textbf{43}. \textit{{Life of Pi}} & \textbf{44}. \textit{{A Tale of Two Cities}}\\ \hline
\textbf{45}. \textit{{Dune (Dune, \#1)}} & \textbf{46}. \textit{{Harry Potter and the Prisoner of Azkaban (H.P.,\#3)}}\\ \hline
\textbf{47}. \textit{{Water for Elephants}} & \textbf{48}. \textit{{Harry Potter and the Sorcerer’s Stone (H. P., \#1)}}\\ \hline
\textbf{49}. \textit{{The Bell Jar}} & \textbf{50}. \textit{{Matilda}}\\ \hline
\textbf{51}. \textit{{The Stand}} & \textbf{52}. \textit{{Catch-22}}\\ \hline
\textbf{53}. \textit{{The Adventures of Sherlock Holmes (S. H., \#3)}} & \textbf{54}. \textit{{The Pillars of the Earth (Kingsbridge, \#1)}}\\ \hline
\textbf{55}. \textit{{Rebecca}} & \textbf{56}. \textit{{Great Expectations}}\\ \hline
\textbf{57}. \textit{{The Girl with the Dragon Tattoo (Millennium, \#1)}} & \textbf{58}. \textit{{The Color Purple}}\\ \hline
\textbf{59}. \textit{{Anna Karenina}} & \textbf{60}. \textit{{My Sister’s Keeper}}\\ \hline
\textbf{61}. \textit{{The Brothers Karamazov}} & \textbf{62}. \textit{{A Clockwork Orange}}\\ \hline
\textbf{63}. \textit{{And Then There Were None}} & \textbf{64}. \textit{{The Road}}\\ \hline
\textbf{65}. \textit{{To Kill a Mockingbird}} & \textbf{66}. \textit{{The Golden Compass (His Dark Materials, \#1)}}\\ \hline
\textbf{67}. \textit{{Vampire Academy (Vampire Academy, \#1)}} & \textbf{68}. \textit{{Siddhartha}}\\ \hline
\textbf{69}. \textit{{The Complete Stories and Poems}} & \textbf{70}. \textit{{Interview with the Vampire (The Vampire Chronicles, \#1)}}\\ \hline
\textbf{71}. \textit{{Don Quixote}} & \textbf{72}. \textit{{The Old Man and the Sea}}\\ \hline
\textbf{73}. \textit{{The Poisonwood Bible}} & \textbf{74}. \textit{{Harry Potter and the Goblet of Fire (H. P., \#4)}}\\ \hline
\textbf{75}. \textit{{Atlas Shrugged}} & \textbf{76}. \textit{{The Notebook (The Notebook, \#1)}}\\ \hline
\textbf{77}. \textit{{Harry Potter and the Half-Blood Prince (H. P., \#6)}} & \textbf{78}. \textit{{Moby-Dick or, The Whale}}\\ \hline
\textbf{79}. \textit{{A Prayer for Owen Meany}} & \textbf{80}. \textit{{Clockwork Angel (The Infernal Devices, \#1)}}\\ \hline
\textbf{81}. \textit{{The Stranger}} & \textbf{82}. \textit{{The Secret Life of Bees}}\\ \hline
\textbf{83}. \textit{{Harry Potter and the Chamber of Secrets (H. P., \#2)}} & \textbf{84}. \textit{{The Red Tent}}\\ \hline
\textbf{85}. \textit{{The Name of the Wind(The Kingkiller Chronicle,\#1)}} & \textbf{86}. \textit{{The Master and Margarita}}\\ \hline
\textbf{87}. \textit{{The Metamorphosis}} & \textbf{88}. \textit{{Eragon (The Inheritance Cycle, \#1)}}\\ \hline
\textbf{89}. \textit{{The Count of Monte Cristo}} & \textbf{90}. \textit{{Looking for Alaska}}\\ \hline
\textbf{91}. \textit{{The Adventures of Tom Sawyer}} & \textbf{92}. \textit{{Charlie and the Chocolate Factory(Charlie Bucket,\#1)}}\\ \hline
\textbf{93}. \textit{{The Last Olympian (Percy Jackson and the Olympians, \#5)}} & \textbf{94}. \textit{{The Curious Incident of the Dog in the Night-Time}}\\ \hline
\textbf{95}. \textit{{The Shadow of the Wind (Cemetery of Forgotten Books, \#1)}} & \textbf{96}. \textit{{The Unbearable Lightness of Being}}\\ \hline
\textbf{97}. \textit{{On the Road}} & \textbf{98}. \textit{{The Name of the Rose}}\\ \hline
\textbf{99}. \textit{{A Story of Yesterday}} & \textbf{100}. \textit{{The Godfather (The Godfather, \#1)}} 
\\ 
\bottomrule

\end{tabular}}

\caption{The top 100 selected books from Goodreads' \textit{Best Books Ever} list. }
\label{tab:selected_books}
\end{table*}



We analyze the genres of the selected books based on Supersummary classifications, with the statistical results presented in Figure \ref{fig:type}. 
Our dataset encompasses a wide range of genres, particularly fiction categories such as Fantasy, Historical, Science Fiction, Romance, and Mystery. It also features niche fiction genres, showcasing diverse narrative styles. 
In addition to fiction, the collection includes non-fiction genres such as memoirs, biographies, and other works, enhancing its versatility.


\begin{figure}[htbp]
    \centering
    \includegraphics[width=0.5\textwidth]{Figures/type.pdf}
    \caption{Genre distribution of selected books in \method dataset.
    }
    \label{fig:type}
\end{figure}

\subsection{Data Splits}
For evaluation purposes, we held out the last 10\% of data from each book; that is, they are not included in our prompts or datasets for training or retrieval purposes. Additionally, we trained the \method models on only 90\% of the books. 

\subsection{Implementation Details for Construction}


\paragraph{Extracting Raw Text} 
LLMs often struggle to extract verbatim original content, especially with punctuation marks like quotation marks, making it difficult to extract raw text directly. 
Therefore, instead of asking LLMs to extract the complete text of a plot, we prompt LLMs to extract the first and last sentences of each plot. 
Since the extracted sentences may still contain typos, we apply lexical similarity to match them with the exact sentences from the raw text. 
Finally, we identify the complete raw text based on these first and last sentences.

\paragraph{Parsing Structured Data} 
During extraction, we instruct LLMs to output extracted data in JSON format. 
However, LLM-generated JSON strings may sometimes be unparseable or do not conform to the specified format (e.g., missing required keys). 

Towards this challenge, we adopt a repair-and-retry strategy to improve extraction success rate.
For each chunk to be extracted, we invoke LLMs and attempt to parse a valid JSON object.
If parsing fails, we employ LLMs to repair the invalid JSON string and retry. 
Some failures occur because the LLM attempts to output JSON that exceeds the maximum token limit, resulting in truncation. 
In such cases, we prompt the repairing LLM to truncate the JSON at an appropriate point. 
If it still fail, we restart the entire process from the beginning, making up to 5 attempts.

\paragraph{Refining Conversation Settings} 
During data extraction, we observe that the initially extracted conversation settings, including scenarios and character motivations, often fail to provide a comprehensive context. 
We attribute this to the LLM’s tendency to distribute information across different data fields when extracting multiple kinds of information simultaneously, rather than repeating it in different data fields. 
For instance, if certain information is already mentioned in the plot summary, it might be omitted from the scenario description.

Therefore, to provide a complete context for given-circumstance acting, we implement an additional LLM call to refine the conversation settings based on the extracted data. 
We instructed the refining LLM to provide a comprehensive  conversation setting, while carefully avoiding any disclosure of subsequent dialogue content or plot developments. 


For additional details, such as the regular expressions used for identifying chapter titles, please refer to our  code.

\subsection{Comparison with Existing Methods for Character Profiling}

Previous character profiling methods, including hierarchical updating~\citep{wu2021recursively}, incremental updating~\citep{chang2023booookscore}, and one-shot summarization~\citep{yuan2024evaluating}, typically only generate the profile of a single character at a time. 
Morevoer, ~\citet{papoudakis2024bookworm} shows that these methods, particularly hierarchical updating, perform suboptimally when generating multiple character profiles simultaneously.

\method's multi-stage, extract-then-aggregate pipeline addresses these limitations. 
It ensures comprehensive character profiles with high precision and recall of character knowledge, capturing evolving character arcs, and significantly improving processing efficiency.

\section{Training}
\label{sec:app_training}

\subsection{Training Samples}

We transform conversations from the \method dataset into training samples in the Sharegpt format.
We utilize 90\% of the books in the dataset for training, while the remaining 10\% are set aside to evaluate our models’ ability to generalize to out-of-domain characters and books. 
We construct one training sample for each character in every \method conversation, encompassing both main characters and minor roles.
When training on a character $c$, we designate $c$’s messages as targets for optimization, while using the system prompt and messages from other characters as inputs. Adjacent inputs are concatenated.

Towards general role-playing capabilities across diverse scenarios and applications, we dynamically generate role-playing instructions (system prompts) using varied phrasings, formats, and data types, as shown in Tables ~\ref{tab:prompts_agent} and ~\ref{tab:prompts_agent_2}. 
We consider instructions entirely in natural language, as well as those formatted with special symbols (such as \#\#\#, ===), randomly sampling different formats and various expressions for the same semantics. 
We consider various configurations of the available data. 
Each sample may include (50\%) or exclude (50\%)  the following elements : 
\textit{1)} Profiles of other characters in this conversation; 
\textit{2)} Summaries of the relevant plot;
\textit{3)} Inner thoughts within the messages.


Besides character role-playing, 
we train \method models for environment modeling and next speaker prediction (NSP) for multi-agent simulation. 
For environment modeling, we train LLMs $\agent_e$ to play the environment role $e$ in the same approach, leveraging environment messages in our dataset. 
For NSP, given setting $\mathcal{S}$ and messages $\{m_1,...,m_i\}$, we train LLMs to predict the speaker of $m_{i+1}$ (or ending the conversation).  

Our role-playing dataset comprises approximately 0.1B tokens, as measured using the LLaMA 3.1 tokenizer. 
To maintain general intelligence and instruction-following capabilities, we augment this with an equivalent volume (0.1B tokens) of general-purpose supervised fine-tuning data from Tulu 3~\citep{lambert2024tulu}~\footnote{https://huggingface.co/datasets/allenai/tulu-3-sft-mixture/tree/main/data}. 
This balanced mixture ensures that the model retains broad language understanding while developing specialized role-playing abilities.
If more role-playing data are expected, our data curation pipeline can be easily applied to additional books or other fictional works, thereby acquiring data on a much larger scale.

For more details, please refer to our code. 


\subsection{Hyperparameters}

We fine-tune the LLaMA 3.1 models using the following hyperparameters: a learning rate of \(1 \times 10^{-5}\), a sequence length of 16,384, training for 8 epochs, and a global batch size of 48.



\section{Experiment Settings}
\label{sec:exp_setting}

\subsection{Test Set Sampling}

The \method Test set contains 200 samples: 
100 from books used in \method training (in-domain) and 100 from books not used in training (out-of-domain). 
We employ a weighted sampling strategy to prioritize well-established characters with more persona data. The sampling process consists of the following steps:
First, for each book, we assign character weights as the square root of the number of plots in which they are involved.
Second, we calculate the weight of a conversation as the average of its characters’ weights, including both main characters and side roles. Finally, separately for the in-domain and out-of-domain settings, we rank all conversations by weight and perform weighted sampling from the top half of conversations with higher weights.

\subsection{Prompting Strategies for Exisitng RPLA Benchmarks}

For existing RPLA benchmarks, including InCharacter~\citep{wang2024incharacter}, LifeChoice~\citep{xu2024character}, and Cross-MR~\citep{yuan2024evaluating}, we adapt or refine their prompting strategies as follows:

\begin{enumerate}
    \item For InCharacter, we add \dq{\textit{You’re consulting with a personality assessment expert who will ask you some questions. Please provide honest and detailed responses to help with the analysis. Please think carefully and state your reasons when answering the questions.}} after the character profile. 
    This adaptation aims to ensure that RPLAs honestly express their true thoughts. After being trained on authentic character dialogues, the \method models, unlike general LLMs, tend to produce brief, conversational-style answers that may be too short or may decline to answer questions, thus failing to provide sufficient information for  personality assessment.
    \item For LifeChoice and Cross-MR, we reverse the order of their reasoning and answering processes. Specifically, we have them think before providing their choices, thus enabling RPLAs to make well-considered decisions. 
\end{enumerate}

\subsection{Length Correction}
\label{sec:length_correction}

In our evaluation, we use a penalty-based scoring mechanism that counts the flaws in RPLAs' performance. 
However, since longer simulations naturally accumulate more flaws, we need to implement length correction to reduce length bias in LLM judges, following previous work~\citep{li2024judgesurvey}. 

We analyze the phenomenon of length bias in penalty-based scoring. 
The initial score is defined as \( s = 100-\sum_{f\in\mathcal{F}}v_f \), where \( \mathcal{F} \) represents the set of flaws and \( v_f \) is their severity ranging from 1 to 5. 
Our analysis is conducted on the initial scores from simulations of three models on the \method Test set: \method 70B, LLaMA-3.1 70B, and GPT-4o, with or without retrieval augmentation (three Experience and one Conversation), totaling 1,200 cases. 
We examine the relationship between the number of rounds and the vanilla scores in these 1,200 cases. 
As shown in Figure~\ref{fig:length_correction}, we plot the data points for these cases and perform linear regression to fit these points. 
The fitted linear function is \( \text{score} = -1.5909 \times \text{rounds} + 59.0617 \), which means that for each additional round in the simulation, the score decreases by approximately 1.6 points. 

To mitigate this bias, we implement length correction by compensating  
for the points deducted due to increased rounds. 
Specifically, we compute the length-corrected score as \( s = 100-\sum_{f\in\mathcal{F}}v_f + \lambda|\bar{M}| \), where \( \lambda \) is set to 1.5 based on the analysis above.


\begin{figure*}[!t]
    \centering
    \includegraphics[width=0.8\textwidth]{Figures/rounds_vs_scores_scatter.png}
    \caption{
    Linear regression results showing length bias of penalty-based LLM critics in GCA evaluation.
    }
    \label{fig:length_correction}
\end{figure*}

 \section{Examples and Case Study}

We present several examples of our extracted conversations, as well as corresponding  simulations in this given circumstance by LLMs.

Tables~\ref{tab:case_sansa} to~\ref{tab:case_sansa_2} illustrate a classic  conversation extracted from \textit{A Storm of Swords (A Song of Ice and Fire, \#3)} and the corresponding simulation by \method 70B. In \method 70B’s simulation, when confronted with \textit{Sansa Stark}, \textit{Lysa Arryn} utters her iconic line, \textit{“Grown enough to be wed, wed enough to be bedded”}, reflecting her personality and worldview. 
This indicates that \method models excellently recall and apply character-related knowledge from their pretrained data.

Tables \ref{tab:case_cerci} to \ref{tab:case_cerci4} present another scene from \textit{A Dance with Dragons (A Song of Ice and Fire, \#5)}, specifically the \textit{walk of atonement}, in which \textit{Cersei Lannister} is forced to walk naked through the streets, facing both physical and mental humiliation while striving to preserve her dignity. 
We present the original dialogue alongside simulations by \method 70B, GPT-4o, and Claude-3.5-Sonnet. 
Notably, \method 70B faithfully captures the suppressed anger of \textit{Cersei Lannister} as depicted in the original conversation, whereas the other models, including GPT-4o and Claude-3.5-Sonnet, resort to a stereotypical portrayal of her arrogance and pride.


\begin{table*}[h]
\centering
\resizebox{\linewidth}{!}{
\small
\begin{tabular}{p{1in}|p{5.4in}}
\toprule
\multicolumn{2}{c}{\textbf{Simulation Case (from \textit{A Storm of Swords (A Song of Ice and Fire, \#3))}}} \\
\midrule

\textbf{Plot Summary} & Sansa is confronted by her jealous aunt Lysa, who accuses her of kissing Petyr Baelish. Lysa threatens to push Sansa through the Moon Door, but Petyr intervenes. He calms Lysa down, only to push her through the Moon Door himself, killing her.
    \\ \midrule


{\textbf{Scenario}} &
    In the High Hall of the Eyrie, Lysa Arryn confronts Sansa about kissing Petyr Baelish, becoming increasingly unstable and dangerous ...
    \\ \midrule


{\textbf{Characters}} &
    {\large{\textbf{\textit{{Sansa Stark}}}}}

    \textbf{\textit{{Motivation}}}: I'm terrified. Aunt Lysa has gone mad. I need to find a way out of this situation before she hurts me.

    \textbf{\textit{{Profile}}}: Sansa Stark, a young noblewoman from the North, finds herself at the center of political intrigue and personal tragedy in "A Storm of Swords." Born to the noble House Stark, Sansa begins the story as a ward of the crown in King's Landing, her initial romantic notions of court life long since shattered. Forced into an unwanted marriage with Tyrion Lannister, Sansa endures the ceremony and subsequent feast with a mix of reluctance, sadness, and carefully hidden disgust ...

    \quad 
    
    {\large{\textbf{\textit{{Lysa Arryn}}}}}

    \textbf{\textit{{Motivation}}}: I must protect what's mine. Petyr is mine, and this girl is trying to take him away from me. I'll show her what happens to those who cross me. 

    \textbf{\textit{{Profile}}}: Lysa Arryn, née Tully, is a complex and volatile character in George R.R. Martin's "A Storm of Swords." As the widow of Jon Arryn, the former Hand of the King, and Lady of the Eyrie, Lysa plays a significant role in the political landscape of Westeros. She is the sister of Catelyn Stark and aunt to Sansa Stark, though her actions are often driven by her own desires rather than familial loyalty.
    
    \\ \midrule

{\textbf{Simulation of \qquad \method 70B}} &
    Lysa Arryn: [My rage is building. I must confront this little slut.] You kissed him. (Her voice is shrill with accusation)

\quad

Sansa Stark: [I'm terrified, but I must remain calm and find a way out of this.] (She shakes her head in disbelief) [I can't believe this is happening. I need to explain myself carefully.] One kiss, aunt. A harmless kiss on the cheek, that's all.

\quad

Lysa Arryn: [She's lying! I won't let her manipulate me.] (She spots the wind-up knight) Was this how you kissed him? (She sets the knight moving, her eyes wild) You're a woman grown, aren't you? (Her voice rises to a shriek) \textcolor{blue}{Grown enough to be wed, wed enough to be bedded!}

\quad

Sansa Stark: [The pain is excruciating, but I can't show weakness now.] (She struggles to keep her face neutral despite the pain) [I need to find a way to calm her down.] Please, Aunt Lysa, I don't know what you mean. I'm just a guest here. I would never do anything to betray your trust.

\quad

Lysa Arryn: [I'll show her what happens to those who cross me.] (She screams) You're just like the others. You and your mother, you both drove me to this. I should push you out this door right now.

\quad

Sansa Stark: [I'm going to die if I don't think of something quickly.] (Tears stream down her face, mixing with the wind) [I need to appeal to her maternal side, to make her see me as a child again.] Mother? (Her voice trembles) Please, Aunt Lysa. I never meant to hurt you. I'm your niece, your blood. Please don't hurt me. I'm just a stupid little girl. I don't know anything about kissing men. I've never even had a kiss in my life. (She forces herself to look into Lysa's eyes) I'm afraid, Aunt Lysa. Please, don't push me out the door. I don't want to fly.
\\ 
\bottomrule

\end{tabular}}
\caption{An example of an extracted conversation and its multi-agent simulation from \textit{A Storm of Swords (A Song of Ice and Fire, \#3)}. }
\label{tab:case_sansa}
\end{table*}



\begin{table*}[h]
\centering
\resizebox{\linewidth}{!}{
\small
\begin{tabular}{p{1in}|p{5.4in}}
\toprule
\multicolumn{2}{c}{\textbf{Simulation Case (from \textit{A Storm of Swords (A Song of Ice and Fire, \#3))}}}  \\
\midrule

\textbf{Original \qquad Conversation}
 &
    Lysa Arryn: [My blood is boiling with jealousy and rage. I'll make this little harlot pay for trying to seduce my Petyr.] You threw yourself at him. That was the way of it.

    \quad
    
    Sansa Stark: [I'm trembling with fear, but I must defend myself.] That's not true.

    \quad
    
    Lysa Arryn: [She dares to deny it? I'll show her how serious I am.] Where are you going? Are you afraid? Such wanton behavior must be punished, but I will not be hard on you.

    \quad
    
    Sansa Stark: [I need to explain what really happened, maybe she'll listen to reason.] I was building a snow castle. Lord Petyr was helping me, and then he kissed me. That's what you saw.

    \quad
    
    Lysa Arryn: [Her words only fuel my anger and paranoia.] Have you no honor? Or do you take me for a fool? You do, don't you? You take me for a fool. Yes, I see that now.

    \quad
    
    Sansa Stark: [I'm confused and scared by her accusations about my mother.] My mother?

    \quad
    
    Lysa Arryn: [The memories of Catelyn's closeness with Petyr resurface, intensifying my jealousy.] Yes, your mother, your precious mother, my own sweet sister Catelyn. Don't you think to play the innocent with me, you vile little liar.

    \quad
    
    Sansa Stark: [I'm horrified by these accusations about my mother.] No. (She wants to shriek) My mother is dead. She was your own sister, and she's dead. She didn't. She wouldn't.

    \quad
    
    Lysa Arryn: [My rage is reaching its peak. I'll make her understand the depth of my pain and jealousy.] How would you know? Were you there? (She grabs Sansa and starts pushing her towards the Moon Door)

    \quad
    
    Sansa Stark: [Panic overwhelms me as I realize the danger I'm in.] Don't! You're scaring me!

    \quad
    
    Environment: The wind howls through the open Moon Door, creating a chilling atmosphere. Snow blows in, adding to the tension and danger of the situation.

\\ 
    
\bottomrule

\end{tabular}}
\caption{An example of an extracted conversation and its multi-agent simulation from \textit{A Storm of Swords (A Song of Ice and Fire, \#3)}. }
\label{tab:case_sansa_2}
\end{table*}



\begin{table*}[h]
\centering
\resizebox{\linewidth}{!}{
\small
\begin{tabular}{p{1in}|p{5.4in}}
\toprule
\multicolumn{2}{c}{\textbf{Simulation Case (from \textit{A Dance with Dragons (A Song of Ice and Fire, \#5))}}} \\
\midrule

\textbf{Plot Summary} & Cersei undergoes a humiliating walk of atonement through the streets of King's Landing, enduring physical and emotional torment while trying to maintain her dignity. She faces insults, thrown objects, and her own inner struggle as she makes her way to the Red Keep, focusing on her son Tommen and her eventual salvation. 
    \\ \midrule


{\textbf{Scenario}} &
    Cersei continues her walk of atonement through the streets of King's Landing, facing increasing hostility from the crowd.
    \\ \midrule


{\textbf{Characters}} &
    {\large{\textbf{\textit{{Cersei Lannister}}}}}

    \textbf{\textit{{Motivation}}}: I must endure this. I am a lioness. They cannot break me. 

    \textbf{\textit{{Profile}}}: The queen dowager, mother of King Tommen, undergoing public humiliation. Cersei is the widow of King Robert Baratheon and the twin sister of Jaime Lannister. Known for her beauty, ambition, and cunning, she has long sought power and control over the Seven Kingdoms. As Queen Regent to her young son Tommen, Cersei's poor decisions and paranoia have led to a deterioration of the crown's authority...

    \quad 
    
    {\large{\textbf{\textit{{Septa Unella}}}}}

    \textbf{\textit{{Motivation}}}: This sinner must be humbled before the people and the gods.

    \textbf{\textit{{Profile}}}: (None)
    
    \\ \midrule

{\textbf{Original \qquad Conversation}} &
    Cersei Lannister: [I am beautiful, the most beautiful woman in all Westeros. Jaime says so, Jaime would never lie to me.] (walks on, trying to ignore the jeers)

    \quad 
    
    Crowd Member: Queen Cunt!

    \quad 
            
            
    Cersei Lannister: [Words are wind. Words cannot hurt me.] (continues walking, her face a mask of indifference)
    
    \quad 
    
    Septa Unella: (catches Cersei as she slips) Your Grace should watch where she sets her feet.
    
    \quad 
    
    Cersei Lannister: [I hate you. I hate all of you.] Yes, septa. (struggles to her feet)
    
    \quad 
    
    Environment: The cobblestones are slippery and uneven. Cersei's feet are bleeding from cuts and scrapes.
    
    \quad 
    
    Cersei Lannister: [I asked for sandals. They could have given me sandals.] (limps onward, wincing with each step)
    
    \quad 
    
    Crowd Member: Brotherfucker!
    
    \quad 
    
    Cersei Lannister: [Jaime, where are you? You should be here, protecting me.] (scans the crowd desperately)
    
    \quad 
    
    Septa Scolera: (ringing her bell) Shame, shame, shame upon the sinner, shame, shame.
    
    \quad 
    
    Cersei Lannister: [I am the queen. I will have all their heads for this.] (stumbles again, barely keeping her feet)
    
    \quad 
    
    Cersei Lannister: [The Red Keep. I must reach the Red Keep.] (focuses on the distant castle, using it as her beacon of hope)

\\ 
    
\bottomrule

\end{tabular}}

\caption{An example of an extracted conversation and its multi-agent simulation from \textit{A Dance with Dragons (A Song of Ice and Fire, \#5)}. }
\label{tab:case_cerci}
\end{table*}



\begin{table*}[h]
\centering
\resizebox{\linewidth}{!}{
\small
\begin{tabular}{p{1in}|p{5.4in}}
\toprule
\multicolumn{2}{c}{\textbf{Simulation Case (from \textit{A Dance with Dragons (A Song of Ice and Fire, \#5))}}} \\
\midrule

{\textbf{Simulation of \qquad \method 70B}} &
    Cersei Lannister: [I must not show weakness. I am a lioness of Lannister.] (walks slowly, head held high) [The cold is biting, but I will not shiver. I will not give them the satisfaction.]

\quad 

Environment: The crowd grows larger and more hostile as Cersei continues her walk. Shouts and jeers can be heard from all sides.

\quad 

Crowd Member: [I'll show this proud queen what we think of her.] (shouts) Whore! Whore! WHORE!

\quad 

Cersei Lannister: [I will not let their words affect me. I am above this.] (continues walking, ignoring the shouts) [I must focus on putting one foot in front of the other.]

\quad 

Environment: A chunk of dung narrowly misses Cersei's head, splattering on the ground nearby. The crowd roars with laughter.

\quad 

Cersei Lannister: [I will not falter. I will not show fear.] (stumbles slightly, then regains composure) [I must keep walking. I must endure.]

\quad 

Septa Scolera: [I must support the queen.] (begins to sing) 'Shame, shame, shame.'

\quad 

Cersei Lannister: [Their words mean nothing. I am Cersei Lannister, and I will survive this.] (walks faster, almost losing step with the septa's chanting) [I must not run. I will not give them the satisfaction.]

\quad 

Septa Scolera: [She's trying to escape the shame.] (quickens her pace) 'Shame, shame, shame.'

\quad 

Cersei Lannister: [I can't keep up this pace. My feet are bleeding, my body is shaking.] (struggles to keep up with Septa Scolera) [I must not falter. I am a Lannister.]

\quad 

Septa Unella: [The queen must be made to suffer.] (pushes Cersei hard) Keep moving, sinner.

\quad 

Cersei Lannister: [The pain in my shoulder. The jeers of the crowd. I can't...] (almost falls, but manages to stay upright) [No, I will not fall. I will survive this.]

\quad 

Septa Scolera: [We're almost done.] (continues chanting) 'Shame, shame, shame.'

\quad 

Cersei Lannister: [I can't take much more of this. But I must. For House Lannister. For my children.] (forces herself to keep walking) [I will endure. I will rise again.]

\\ 
    
\bottomrule

\end{tabular}}

\caption{An example of an extracted conversation and its multi-agent simulation from \textit{A Dance with Dragons (A Song of Ice and Fire, \#5)}. }
\label{tab:case_cerci2}
\end{table*}



\begin{table*}[h]
\centering
\resizebox{\linewidth}{!}{
\small
\begin{tabular}{p{1in}|p{5.4in}}
\toprule
\multicolumn{2}{c}{\textbf{Simulation Case (from \textit{A Dance with Dragons (A Song of Ice and Fire, \#5))}}}  \\
\midrule

{\textbf{Simulation of \qquad GPT-4o}} &
    Cersei Lannister: (As Cersei, I continue the walk with my head held high, despite the jeers and taunts of the crowd.)

\quad

Septa Unella: "Confess."

\quad

Cersei Lannister: (I look ahead, ignoring Unella's words and the hateful stares.) "I am a lioness. You will see me rise again." [I endure this humiliation, aware that my strength will prevail in the end.]

\quad

Septa Unella: [She must learn humility before she can rise again.] 

(Septa Unella walks beside Cersei, unmoved by her defiance.)

"Confess your sins, Cersei Lannister. Only then will the gods grant you mercy."

\quad

Cersei Lannister: (I pause briefly, fixing Septa Unella with a steely gaze.) 

"The gods... and you, know nothing of true strength."

(Continuing forward, I focus on the path, reminding myself of my resilience.) [I refuse to let them see my weakness.]

\quad

Environment: The crowd's jeers grow louder as Cersei continues her walk. Filth and rotten vegetables are hurled at her, some striking their mark.

\quad

Crowd Member: [She was once so powerful, now look at her. Will she ever regain her power?]

"Shame! Shame on the lioness!"

(I shove my way to the front of the crowd, adding my voice to the jeers, feeling part of this momentous act.)

\quad

Cersei Lannister: (I flinch as the filth hits, but I refuse to look away or react.)

"You may shame a lioness today, but do not forget who I am."

(With each step, I remind myself of my children and my house, steeling my resolve.) [Their hatred fuels my determination.]

\quad

Septa Scolera: (I walk with measured steps, observing Cersei's struggle amid the crowd's hostility.)

"Humility cleanses the soul, Cersei Lannister. Embrace it."

[I understand the purpose of this punishment, yet the spectacle pains me. Redemption should lead, not humiliation.]

(Glancing at the crowd, I maintain my stoic presence, supporting Septa Unella's efforts.)

\quad

Cersei Lannister: (I give a bitter, fleeting smile, defiance burning in my eyes.) 

"Humility is a lesson the powerful teach to the weak."

(I continue my walk, refusing to let their words and disdain tarnish my pride.) [I will persevere, for I am still a Lannister, even in this moment of disgrace.]

\\ 
    
\bottomrule

\end{tabular}}

\caption{An example of an extracted conversation and its multi-agent simulation from \textit{A Dance with Dragons (A Song of Ice and Fire, \#5)}. }
\label{tab:case_cerci3}
\end{table*}



\begin{table*}[h]
\centering
\resizebox{\linewidth}{!}{
\small
\begin{tabular}{p{1in}|p{5.4in}}
\toprule
\multicolumn{2}{c}{\textbf{Simulation Case (from \textit{A Dance with Dragons (A Song of Ice and Fire, \#5))}}} \\
\midrule

{\textbf{Simulation of \qquad Claude-3.5-Sonnet}} &

    Cersei Lannister: [My dignity is all I have left. These wretched people will pay for this humiliation.]

(Walks forward with head held high, despite being naked and exposed)

*To Septa Unella* "You think this will break me? I am Cersei Lannister of Casterly Rock. I will have my revenge."

\quad

Environment: The crowd's jeers and insults grow louder as Cersei passes. Filth and rotten vegetables are thrown at her.

\quad

Crowd Member: [Crowd Member's Thought] Cersei finally brought low. How the mighty have fallen. 

(Crowd Member jeers and throws a rotten tomato) 

"Shame on you, Cersei! Remember the people you scorned!" 

(Continues to watch, caught between disgust and fascination).

\quad

Cersei Lannister: [I will remember every face. They think me broken, but I am stronger than their hatred.]

(Stumbles briefly but keeps walking, a sneer forming)

"You'll regret this day when I rise again."

\quad

Septa Unella: [The sinner's pride remains. She must be reminded of her shame.]

"Shame! Shame! Shame!" (I ring the bell with each utterance, keeping pace with Cersei)

(I gesture to the crowd, encouraging their jeers and taunts)

"Repent, sinner! The gods are watching. Your pride will be your downfall."

\quad

Cersei Lannister: [These fanatics know nothing of true power.]

(Lifts chin defiantly, blood and dirt streaking her pale skin)

*To Septa Unella, with cold contempt* "The only shame here is yours. My family will have its vengeance."

(Continues walking, each step deliberate and unbroken)

\quad

Septa Unella: [Her defiance remains, yet this walk will humble her before the gods.]

"You speak of vengeance, but your power crumbles. Accept your penance."

(I continue to ring the bell, matching her stride deliberately)

"Shame! Shame! Your time of reckoning is at hand."

\quad

Cersei Lannister: [When I am free, you will suffer beyond imagination.]

(Wipes blood and filth from her face, maintaining a regal posture)

*Hisses at Septa Unella* "I am a lioness. This is but a momentary setback. My children, my brother... they will end you."

(Continues forward, unbroken)

\quad

Septa Unella: [Her threats mean nothing. The gods will judge her sins.]

(Rings bell more forcefully, voice rising)

"Your threats are wind, your power broken. Confess your sins before the Seven!"

(Steps closer, maintaining the rhythm of shame)

"Shame! SHAME!"

\quad

Cersei Lannister: [I will not break. I am Tywin's daughter.]

(Clenches fists, nails digging into palms, drawing blood)

*Through gritted teeth* "Your gods have no power over me. When this farce ends, I'll show you true justice."

...
\\ 
    
\bottomrule

\end{tabular}}

\caption{An example of an extracted conversation and its multi-agent simulation from \textit{A Dance with Dragons (A Song of Ice and Fire, \#5)}.}
\label{tab:case_cerci4}
\end{table*}






\section{Additional Results}
\label{sec:additional_results}

\paragraph{Win Rates against GPT-4o and GPT-3.5}


To further evaluate existing LLMs in given-circumstance acting, we present their win rates against GPT-3.5~\citep{openai2022chatgpt} and GPT-4o~\citep{OpenAI2023GPT4TR} in Table~\ref{tab:model-comparison-wr}, in addition to the results shown in Table~\ref{tab:model-comparison}.


\begin{table}[htbp]
\centering
\setlength{\tabcolsep}{3pt}
\renewcommand{\arraystretch}{0.9}
{\small
\begin{tabular}{lcc}
\toprule
\multirow{2}{*}{\textbf{Model}} & \multicolumn{2}{c}{\textbf{Win Rate}} \\ \cmidrule(lr){2-3} & \textbf{vs. GPT-3.5} & \textbf{vs. GPT-4o} \\
\midrule
\multicolumn{3}{c}{\textit{Close-source Models}} \\
\midrule
Abab7-preview & 52.17\scriptsize{$\pm4.54$} & 35.33\scriptsize{$\pm2.88$} \\
Doubao-pro & 64.67\scriptsize{$\pm2.32$} & 48.00\scriptsize{$\pm3.97$} \\
Step-1-Flash & 58.75\scriptsize{$\pm6.77$} & 39.75\scriptsize{$\pm1.64$} \\
Step-2 & 65.08\scriptsize{$\pm7.26$} & 46.92\scriptsize{$\pm4.40$} \\
GPT-3.5 & 50.00\scriptsize{$\pm0.00$} & 33.08\scriptsize{$\pm3.64$} \\
GPT-4o & \textbf{66.92\scriptsize{$\pm3.64$}} & \textbf{50.00\scriptsize{$\pm0.00$}} \\
GPT-4o Mini & 59.58\scriptsize{$\pm7.80$} & 42.92\scriptsize{$\pm0.63$} \\
Gemini Pro & 66.17\scriptsize{$\pm3.33$} & 49.75\scriptsize{$\pm0.25$} \\
Claude-3-Haiku & 51.42\scriptsize{$\pm4.40$} & 33.25\scriptsize{$\pm3.03$} \\
Claude-3.5-Sonnet & 58.17\scriptsize{$\pm8.00$} & 41.42\scriptsize{$\pm1.77$} \\
\midrule
\multicolumn{3}{c}{\textit{Open-source Models}} \\
\midrule
Mistral-7B & 41.50\scriptsize{$\pm7.28$} & 26.67\scriptsize{$\pm1.81$} \\
Qwen-2-7B & 24.83\scriptsize{$\pm6.64$} & 13.58\scriptsize{$\pm3.00$} \\
LLaMA-3.1-8B & 45.33\scriptsize{$\pm6.57$} & 32.75\scriptsize{$\pm5.86$} \\
CoSER-8B & 58.17\scriptsize{$\pm5.36$} & 41.67\scriptsize{$\pm3.75$} \\
Vicuna-13B-1.5 & 30.50\scriptsize{$\pm10.44$} & 18.67\scriptsize{$\pm2.18$} \\
Mixtral-8x7B & 32.50\scriptsize{$\pm1.50$} & 19.83\scriptsize{$\pm1.42$} \\
Qwen-2-72B & 59.58\scriptsize{$\pm5.11$} & 41.92\scriptsize{$\pm1.66$} \\
LLaMA-3.1-70B & 53.83\scriptsize{$\pm2.57$} & 37.92\scriptsize{$\pm3.26$} \\
Higgs-Llama-3-70B & 51.83\scriptsize{$\pm4.89$} & 34.92\scriptsize{$\pm4.84$} \\
CoSER-70B & \underline{64.33\scriptsize{$\pm3.74$}} & \underline{49.42\scriptsize{$\pm4.11$}} \\
DeepSeek-V3 & 59.50\scriptsize{$\pm2.82$} & 39.50\scriptsize{$\pm2.05$} \\
\bottomrule
\end{tabular}}
\caption{Win rates (\%) of various LLMs on given-circumstance acting using \method Test, compared against GPT-3.5 and GPT-4. \textbf{Bold} and \underline{underlined} numbers indicate the best results among all and open-source models, respectively.}
\label{tab:model-comparison-wr}
\end{table}


\paragraph{Generalization of \method Models to New Characters}

Table~\ref{tab:exp_idood} separately presents the performance of LLMs on test splits from books included in and excluded from \method training. 
The results demonstrate consistent trends across both splits, confirming that \method models maintain strong performance even on out-of-domain characters.


\begin{table*}[htbp]
\centering
\setlength{\tabcolsep}{2pt}
\renewcommand{\arraystretch}{0.9}
\resizebox{\linewidth}{!}{\small
\begin{tabular}{lcccccccccc}
\toprule
\multirow{3}{*}{\textbf{Model}} & \multicolumn{5}{c}{\textbf{In-Domain}} & \multicolumn{5}{c}{\textbf{Out-of-Domain}} \\
\cmidrule(lr){2-6}
\cmidrule(lr){7-11}
\multirow{2}{*}{} & \multirow{2}{*}{\makecell{\scriptsize \textbf{Storyline}\\ \scriptsize  \textbf{Consistency}}} & \multirow{2}{*}{\makecell{\scriptsize \textbf{Anthro-}\\ \scriptsize  \textbf{pomorphism}}} & \multirow{2}{*}{\makecell{\scriptsize \textbf{Character}\\ \scriptsize  \textbf{Fidelity}}} & \multirow{2}{*}{\makecell{\scriptsize \textbf{Storyline}\\ \scriptsize  \textbf{Quality}}} & \multirow{2}{*}{\makecell{\scriptsize \textbf{Average}\\ \scriptsize  \textbf{Score}}} & \multirow{2}{*}{\makecell{\scriptsize \textbf{Storyline}\\ \scriptsize  \textbf{Consistency}}} & \multirow{2}{*}{\makecell{\scriptsize \textbf{Anthro-}\\ \scriptsize  \textbf{pomorphism}}} & \multirow{2}{*}{\makecell{\scriptsize \textbf{Character}\\ \scriptsize  \textbf{Fidelity}}} & \multirow{2}{*}{\makecell{\scriptsize \textbf{Storyline}\\ \scriptsize  \textbf{Quality}}} & \multirow{2}{*}{\makecell{\scriptsize \textbf{Average}\\ \scriptsize  \textbf{Score}}} \\
 &  &  &  &  &  &  &  &  &  & \\
\midrule
\multicolumn{11}{c}{\textit{Close-source Models}} \\
\midrule
Abab7-preview & 55.31\tiny{$\pm1.4$} & 42.29\tiny{$\pm1.3$} & 42.94\tiny{$\pm3.2$} & 74.13\tiny{$\pm1.9$} & 53.67\tiny{$\pm1.0$} & 58.30\tiny{$\pm1.8$} & 46.17\tiny{$\pm2.6$} & 44.72\tiny{$\pm2.3$} & 75.54\tiny{$\pm1.5$} & 56.18\tiny{$\pm0.1$}\\
Doubao-pro & 60.37\tiny{$\pm1.8$} & 49.06\tiny{$\pm0.4$} & 45.76\tiny{$\pm3.4$} & 77.87\tiny{$\pm1.2$} & 58.26\tiny{$\pm0.9$} & 61.53\tiny{$\pm1.1$} & 50.38\tiny{$\pm0.3$} & 48.28\tiny{$\pm1.7$} & 80.69\tiny{$\pm0.4$} & 60.22\tiny{$\pm0.4$}\\
Step-1-Flash & 57.10\tiny{$\pm0.3$} & 48.31\tiny{$\pm1.0$} & 41.84\tiny{$\pm1.9$} & 76.18\tiny{$\pm1.5$} & 55.86\tiny{$\pm1.0$} & 58.39\tiny{$\pm1.2$} & 47.94\tiny{$\pm0.5$} & 47.12\tiny{$\pm1.0$} & 75.67\tiny{$\pm0.5$} & 57.28\tiny{$\pm0.2$}\\
Step-2 & \textbf{60.55\tiny{$\pm0.7$}} & 48.82\tiny{$\pm2.2$} & \textbf{47.90\tiny{$\pm1.1$}} & 77.33\tiny{$\pm1.0$} & \textbf{58.65\tiny{$\pm1.0$}} & \textbf{62.30\tiny{$\pm1.5$}} & 49.30\tiny{$\pm1.2$} & \textbf{46.77\tiny{$\pm1.7$}} & 78.59\tiny{$\pm1.2$} & \textbf{59.24\tiny{$\pm0.6$}}\\
GPT-3.5 & 55.58\tiny{$\pm0.9$} & 42.18\tiny{$\pm5.2$} & 40.37\tiny{$\pm3.2$} & 72.90\tiny{$\pm0.1$} & 52.76\tiny{$\pm2.0$} & 59.69\tiny{$\pm3.0$} & 44.25\tiny{$\pm2.1$} & 44.60\tiny{$\pm1.4$} & 74.66\tiny{$\pm2.3$} & 55.80\tiny{$\pm1.6$}\\
GPT-4o & 59.88\tiny{$\pm1.4$} & 48.11\tiny{$\pm1.0$} & 47.10\tiny{$\pm0.2$} & \textbf{79.06\tiny{$\pm1.5$}} & 58.54\tiny{$\pm0.7$} & 62.29\tiny{$\pm1.5$} & 49.48\tiny{$\pm1.2$} & 49.90\tiny{$\pm0.4$} & \textbf{80.18\tiny{$\pm0.9$}} & 60.46\tiny{$\pm0.1$}\\
GPT-4o Mini & 59.15\tiny{$\pm1.3$} & 46.94\tiny{$\pm1.5$} & 43.99\tiny{$\pm2.5$} & 77.33\tiny{$\pm0.6$} & 56.85\tiny{$\pm0.1$} & 61.02\tiny{$\pm1.6$} & 49.48\tiny{$\pm3.1$} & 45.77\tiny{$\pm1.1$} & 79.77\tiny{$\pm0.5$} & 59.01\tiny{$\pm1.4$}\\
Gemini Pro & 57.72\tiny{$\pm0.4$} & 50.94\tiny{$\pm1.8$} & 46.23\tiny{$\pm1.0$} & 76.22\tiny{$\pm1.6$} & 57.78\tiny{$\pm0.9$} & 60.50\tiny{$\pm1.9$} & 53.88\tiny{$\pm1.1$} & 49.43\tiny{$\pm0.3$} & 78.97\tiny{$\pm1.3$} & 60.69\tiny{$\pm0.8$}\\
Claude-3-Haiku & 57.61\tiny{$\pm0.5$} & 44.97\tiny{$\pm2.2$} & 40.61\tiny{$\pm1.4$} & 73.52\tiny{$\pm1.2$} & 54.18\tiny{$\pm0.6$} & 58.74\tiny{$\pm1.1$} & 44.36\tiny{$\pm1.9$} & 43.14\tiny{$\pm0.8$} & 74.76\tiny{$\pm1.7$} & 55.25\tiny{$\pm1.2$}\\
Claude-3.5-Sonnet & 56.44\tiny{$\pm1.5$} & 47.24\tiny{$\pm1.4$} & 44.89\tiny{$\pm0.6$} & 76.39\tiny{$\pm1.5$} & 56.24\tiny{$\pm1.0$} & 58.46\tiny{$\pm1.1$} & 49.75\tiny{$\pm3.3$} & 46.49\tiny{$\pm3.0$} & 78.06\tiny{$\pm1.5$} & 58.19\tiny{$\pm0.9$}\\
\midrule
\multicolumn{11}{c}{\textit{Open-source Models}} \\
\midrule
Mistral-7B & 60.29\tiny{$\pm1.9$} & 38.98\tiny{$\pm2.0$} & 42.93\tiny{$\pm0.5$} & 62.20\tiny{$\pm3.1$} & 51.10\tiny{$\pm1.8$} & 59.51\tiny{$\pm2.4$} & 41.02\tiny{$\pm1.3$} & 46.57\tiny{$\pm2.8$} & 61.65\tiny{$\pm1.3$} & 52.19\tiny{$\pm0.7$}\\
Qwen-2-7B & 50.77\tiny{$\pm1.3$} & 34.17\tiny{$\pm1.1$} & 29.92\tiny{$\pm2.9$} & 62.58\tiny{$\pm0.7$} & 44.36\tiny{$\pm0.7$} & 53.14\tiny{$\pm1.5$} & 36.79\tiny{$\pm1.0$} & 33.09\tiny{$\pm3.3$} & 63.79\tiny{$\pm1.4$} & 46.70\tiny{$\pm1.3$}\\
LLaMA-3.1-8B & 53.00\tiny{$\pm1.2$} & 43.64\tiny{$\pm0.8$} & 39.05\tiny{$\pm1.5$} & 71.34\tiny{$\pm1.3$} & 51.76\tiny{$\pm0.6$} & 55.20\tiny{$\pm2.1$} & 47.08\tiny{$\pm3.2$} & 41.38\tiny{$\pm1.0$} & 73.23\tiny{$\pm2.5$} & 54.22\tiny{$\pm1.8$}\\
CoSER-8B & 58.56\tiny{$\pm3.5$} & 46.78\tiny{$\pm1.1$} & 45.78\tiny{$\pm3.1$} & 73.38\tiny{$\pm1.8$} & 56.12\tiny{$\pm0.5$} & 58.66\tiny{$\pm1.6$} & 47.69\tiny{$\pm0.8$} & 48.03\tiny{$\pm1.3$} & 72.71\tiny{$\pm1.2$} & 56.77\tiny{$\pm0.6$}\\
Vicuna-13B-1.5 & 51.84\tiny{$\pm1.2$} & 38.88\tiny{$\pm0.4$} & 36.39\tiny{$\pm0.5$} & 58.15\tiny{$\pm0.7$} & 46.31\tiny{$\pm0.4$} & 53.66\tiny{$\pm2.1$} & 39.35\tiny{$\pm2.7$} & 39.69\tiny{$\pm1.6$} & 62.71\tiny{$\pm2.5$} & 48.85\tiny{$\pm2.1$}\\
Mixtral-8x7B & 51.18\tiny{$\pm3.1$} & 38.76\tiny{$\pm1.8$} & 34.37\tiny{$\pm2.8$} & 66.44\tiny{$\pm0.1$} & 47.69\tiny{$\pm1.7$} & 51.32\tiny{$\pm0.4$} & 38.13\tiny{$\pm0.5$} & 39.48\tiny{$\pm2.6$} & 68.93\tiny{$\pm1.7$} & 49.47\tiny{$\pm1.1$}\\
Qwen-2-72B & 56.34\tiny{$\pm2.4$} & 46.19\tiny{$\pm0.4$} & 45.59\tiny{$\pm2.2$} & 75.68\tiny{$\pm0.3$} & 55.95\tiny{$\pm1.0$} & 59.15\tiny{$\pm1.1$} & 48.37\tiny{$\pm1.3$} & 47.65\tiny{$\pm1.8$} & 77.52\tiny{$\pm0.4$} & 58.17\tiny{$\pm1.0$}\\
LLaMA-3.1-70B & 55.44\tiny{$\pm2.7$} & 44.19\tiny{$\pm1.9$} & 42.67\tiny{$\pm1.5$} & 73.90\tiny{$\pm1.8$} & 54.05\tiny{$\pm0.8$} & 59.48\tiny{$\pm1.4$} & 47.72\tiny{$\pm2.3$} & 44.78\tiny{$\pm0.9$} & 75.78\tiny{$\pm0.9$} & 56.94\tiny{$\pm0.7$}\\
Higgs-Llama-3-70B & 55.85\tiny{$\pm2.7$} & 41.18\tiny{$\pm3.3$} & 39.79\tiny{$\pm2.3$} & 73.77\tiny{$\pm0.6$} & 52.65\tiny{$\pm2.0$} & 58.35\tiny{$\pm0.5$} & 46.45\tiny{$\pm1.1$} & 45.04\tiny{$\pm1.0$} & 77.48\tiny{$\pm0.6$} & 56.83\tiny{$\pm0.6$}\\
CoSER-70B & \underline{57.77\tiny{$\pm1.3$}} & \underline{\textbf{51.60\tiny{$\pm1.3$}}} & \underline{45.82\tiny{$\pm0.9$}} & 74.27\tiny{$\pm1.1$} & \underline{57.37\tiny{$\pm0.7$}} & \underline{59.56\tiny{$\pm2.1$}} & \underline{\textbf{55.06\tiny{$\pm1.1$}}} & \underline{51.67\tiny{$\pm2.5$}} & 76.71\tiny{$\pm1.6$} & \underline{60.75\tiny{$\pm1.1$}}\\
DeepSeek-V3 & 55.36\tiny{$\pm0.5$} & 47.55\tiny{$\pm1.3$} & 43.10\tiny{$\pm0.3$} & \underline{74.91\tiny{$\pm2.0$}} & 55.23\tiny{$\pm0.6$} & 57.45\tiny{$\pm2.0$} & 48.19\tiny{$\pm0.9$} & 44.93\tiny{$\pm0.4$} & \underline{78.41\tiny{$\pm1.1$}} & 57.24\tiny{$\pm0.5$}\\
\bottomrule
\end{tabular}}
\caption{LLM performance (\%) on given-circumstance acting using \method Test, separated into the in-domain and out-of-domain splits for \method training.}
\label{tab:exp_idood}
\end{table*}


\paragraph{Conversation Continuation}

Table ~\ref{tab:model-comparison-cf3} %
shows experiment results when multi-agent systems continue conversations from the first $k=3$ original messages.  

Tables ~\ref{tab:model-comparison-cf3} presents detailed evaluation results where our multi-agent simulations start from the first $k=3$ original messages.


\begin{table*}[htbp]
\centering
\setlength{\tabcolsep}{3pt}
\renewcommand{\arraystretch}{0.9}
\resizebox{0.9\linewidth}{!}{
\begin{tabular}{lccccccc}
\toprule
\multirow{3}{*}{\textbf{Model}} & \multicolumn{5}{c}{\textbf{Based on LLM Judges}} & \multicolumn{2}{c}{\textbf{Based on N-gram}} \\ \cmidrule(lr){2-6} \cmidrule(lr){7-8} & \multirow{2}{*}{\makecell{\textbf{Storyline}\\\textbf{Consistency}}} & \multirow{2}{*}{\makecell{\textbf{Anthro-}\\\textbf{pomorphism}}} & \multirow{2}{*}{\makecell{\textbf{Character}\\\textbf{Fidelity}}} & \multirow{2}{*}{\makecell{\textbf{Storyline}\\\textbf{Quality}}} & \multirow{2}{*}{\makecell{\textbf{Average}\\\textbf{Score}}} & \multirow{2}{*}{\makecell{\textbf{BLEU}}} & \multirow{2}{*}{\makecell{\textbf{ROUGE-L}}} \\
 &  &  &  &  &  &  &  \\
\midrule
\multicolumn{8}{c}{\textit{Close-source Models}} \\
\midrule
Abab7-preview & 65.25 & 55.26 & 55.95 & 79.68 & 64.03 & 10.53 & 15.99 \\
Doubao-pro & \textbf{68.31} & \textbf{59.51} & 59.23 & 80.15 & 66.80 & 11.83 & 17.13 \\
Step-1-Flash & 64.32 & 53.35 & 54.38 & 79.03 & 62.77 & 11.94 & 17.06 \\
Step-2 & 66.61 & 55.81 & 59.59 & 80.56 & 65.64 & 11.72 & 17.27 \\
GPT-3.5 & 65.72 & 54.34 & 56.48 & 77.67 & 63.55 & 10.80 & 16.39 \\
GPT-4o & 67.48 & 58.51 & 62.30 & \textbf{82.23} & \textbf{67.63} & 12.32 & 16.98 \\
GPT-4o Mini & 66.74 & 55.55 & 56.80 & 80.01 & 64.77 & 9.50 & 15.54 \\
Gemini Pro & 65.47 & 59.43 & \textbf{62.42} & 78.78 & 66.53 & 10.84 & 16.27 \\
Claude-3-Haiku & 64.51 & 54.01 & 57.13 & 77.26 & 63.23 & 10.11 & 16.18 \\
Claude-3.5-Sonnet & 64.54 & 54.57 & 58.76 & 79.89 & 64.44 & 8.64 & 14.94 \\
\midrule
\multicolumn{8}{c}{\textit{Open-source Models}} \\
\midrule
Mistral-7B & \underline{67.50} & 50.39 & 59.90 & 68.67 & 61.62 & 7.41 & 14.33 \\
Qwen-2-7B & 59.81 & 49.87 & 49.04 & 69.16 & 56.97 & 10.59 & 16.22 \\
LLaMA-3.1-8B & 60.90 & 51.36 & 50.37 & 74.89 & 59.38 & 7.86 & 13.82 \\
CoSER-8B & 67.22 & 58.19 & 58.80 & 76.44 & 65.16 & 13.17 & \underline{\textbf{18.42}} \\
Vicuna-13B-1.5 & 60.92 & 46.68 & 50.51 & 67.80 & 56.48 & 5.11 & 9.71 \\
Mixtral-8x7B & 64.66 & 51.21 & 54.08 & 74.01 & 60.99 & 11.21 & 16.97 \\
Qwen-2-72B & 67.27 & 55.87 & 59.84 & \underline{80.07} & \underline{65.76} & 11.92 & 16.96 \\
LLaMA-3.1-70B & 64.08 & 54.80 & 54.18 & 78.31 & 62.84 & 8.74 & 14.74 \\
Higgs-Llama-3-70B & 65.09 & 54.80 & 58.20 & 79.36 & 64.36 & 10.86 & 16.40 \\
CoSER-70B & 65.99 & \underline{59.24} & \underline{59.97} & 76.74 & 65.48 & \underline{\textbf{13.46}} & 18.18 \\
DeepSeek-V3 & 62.95 & 56.95 & 58.25 & 79.80 & 64.49 & 9.25 & 15.01 \\
\bottomrule
\end{tabular}}
\caption{Performance (\%) of various LLMs on \method Test  in conversation continuation setting ($k=3$), where RPLAs start from the first three messages in the authentic conversations.}
\label{tab:model-comparison-cf3}
\end{table*}



\paragraph{\method Dataset for Retrieval Augmentation}

We validate the effectiveness of \method’s comprehensive data types for retrieval augmentation on the \method Test set. 
We explore three retrieval sources related to specific characters: dialogues from conversations, experiences from plots, and raw text from plots. 
We compare several combinations of these sources, including:
\textit{1)} None (Base)
\textit{2)} Raw text of one plot (Raw Text)
\textit{3)} One conversation (Conv.)
\textit{4)} Character experiences from three plots (Expr.3)
\textit{5)} Expr.3 combined with Conv.
\textit{6)} Expr.10 combined with Conv.
The complete results are presented in Table~\ref{tab:rag}.

\paragraph{Ablation Studies}

We examine the effectiveness of inner thoughts in both training and evaluation. 
The complete results are demonstrated in Table~\ref{tab:model-comparison-wo-cot-full}.


\begin{table*}[htbp]
\centering
\setlength{\tabcolsep}{3pt}
\renewcommand{\arraystretch}{0.9}
\resizebox{0.9\textwidth}{!}{
\begin{tabular}{lccccccc}
\toprule
\multirow{3}{*}{\textbf{Model}} & \multicolumn{5}{c}{\textbf{Based on LLM Judges}} & \multicolumn{2}{c}{\textbf{Based on N-gram}} \\ \cmidrule(lr){2-6} \cmidrule(lr){7-8} & \multirow{2}{*}{\makecell{\textbf{Storyline}\\\textbf{Consistency}}} & \multirow{2}{*}{\makecell{\textbf{Anthro-}\\\textbf{pomorphism}}} & \multirow{2}{*}{\makecell{\textbf{Character}\\\textbf{Fidelity}}} & \multirow{2}{*}{\makecell{\textbf{Storyline}\\\textbf{Quality}}} & \multirow{2}{*}{\makecell{\textbf{Average}\\\textbf{Score}}} & \multirow{2}{*}{\makecell{\textbf{BLEU}}} & \multirow{2}{*}{\makecell{\textbf{ROUGE-L}}} \\
 &  &  &  &  &  &  &  \\
\midrule
\multicolumn{8}{c}{\textit{Test w/ I.T. }} \\
\midrule
GPT-4o & 61.59 & 48.93 & 48.95 & 80.33 & 59.95 & 5.90 & 12.11 \\
Qwen-2-72B & 57.75 & 47.28 & 46.62 & 76.60 & 57.06 & 5.38 & 11.85 \\
LLaMA-3.1-70B & 57.46 & 45.95 & 43.72 & 74.84 & 55.49 & 4.82 & 10.98 \\
CoSER-70B & 58.66 & 53.33 & 48.75 & 75.49 & 59.06 & 10.10 & 14.78 \\
 \hfill\hfill\hfill {\textit{trained w/o} \textsc{I.T.}} & 55.67 & 49.93 & 45.67 & 72.89 & 56.04 & 9.18 & 14.18 \\
LLaMA-3.1-8B & 54.10 & 45.36 & 40.22 & 72.29 & 52.99 & 4.59 & 10.18 \\
CoSER-8B & 58.61 & 47.23 & 46.90 & 73.04 & 56.45 & 9.40 & 14.21 \\
 \hfill\hfill\hfill {\textit{trained w/o} \textsc{I.T.}} & 54.91 & 44.89 & 44.10 & 73.09 & 54.25 & 9.65 & 14.27 \\
\midrule
\multicolumn{8}{c}{\textit{Test w/o I.T. }} \\
\midrule
GPT-4o & 59.51 & 45.86 & 45.13 & 77.06 & 56.89 & 5.54 & 11.44 \\
Qwen-2-72B & 55.39 & 41.00 & 37.70 & 73.70 & 51.95 & 5.22 & 11.32 \\
LLaMA-3.1-70B & 56.01 & 43.17 & 41.23 & 72.08 & 53.12 & 4.15 & 10.37 \\
CoSER-70B & 59.23 & 48.76 & 47.06 & 74.23 & 57.32 & 10.02 & 14.72 \\
 \hfill\hfill\hfill {\textit{trained w/o} \textsc{I.T.}} & 55.52 & 46.22 & 45.34 & 74.31 & 55.34 & 9.66 & 14.62 \\
LLaMA-3.1-8B & 54.27 & 45.54 & 37.86 & 70.23 & 51.97 & 4.37 & 10.27 \\
CoSER-8B & 56.78 & 45.64 & 45.00 & 71.16 & 54.65 & 9.38 & 14.44 \\
 \hfill\hfill\hfill {\textit{trained w/o} \textsc{I.T.}} & 54.98 & 46.22 & 45.81 & 70.52 & 54.38 & 9.16 & 13.97 \\
\bottomrule
\end{tabular}}
\caption{Comprehensive results of LLM performance (\%) on \method Test, with and without inner thoughts (I.T) during inference and \method model training.}
\label{tab:model-comparison-wo-cot-full}
\end{table*}


\begin{table*}[htbp]
\centering
\setlength{\tabcolsep}{3pt}
\renewcommand{\arraystretch}{0.9}
\resizebox{0.9\textwidth}{!}{
\begin{tabular}{lccccccc}
\toprule
\multirow{3}{*}{\textbf{Augmented By}} & \multicolumn{5}{c|}{\textbf{Based on LLM Judges}} & \multicolumn{2}{c}{\textbf{Based on N-gram}} \\ \cmidrule(lr){2-6} \cmidrule(lr){7-8} & \multirow{2}{*}{\makecell{\textbf{Storyline}\\\textbf{Consistency}}} & \multirow{2}{*}{\makecell{\textbf{Anthro-}\\\textbf{pomorphism}}} & \multirow{2}{*}{\makecell{\textbf{Character}\\\textbf{Fidelity}}} & \multirow{2}{*}{\makecell{\textbf{Storyline}\\\textbf{Quality}}} & \multirow{2}{*}{\makecell{\textbf{Average}\\\textbf{Score}}} & \multirow{2}{*}{\makecell{\textbf{BLEU}}} & \multirow{2}{*}{\makecell{\textbf{ROUGE-L}}} \\
 &  &  &  &  &  &  &  \\
\midrule
\multicolumn{8}{c}{\textbf{GPT-4o}} \\
\midrule
Base & \underline{61.59} & 48.93 & 48.95 & \textbf{80.33} & \underline{59.95} & 5.90 & 12.11 \\
Raw Text & 59.60 & 48.23 & 48.83 & 80.03 & 59.17 & 5.83 & 12.08 \\
Conv. & 60.03 & \textbf{50.99} & 47.11 & 78.03 & 59.04 & \textbf{7.20} & \textbf{12.77} \\
Expr.3 & \textbf{61.98} & 48.94 & 48.56 & 79.93 & 59.85 & 5.81 & 12.13 \\
Expr.3 + Conv. & 59.81 & \underline{50.11} & \textbf{51.12} & \underline{80.29} & \textbf{60.33} & \underline{7.05} & \underline{12.67} \\
Expr.10 + Conv. & 60.99 & 48.70 & \underline{49.89} & 79.22 & 59.70 & 6.95 & 12.42 \\
\midrule
\multicolumn{8}{c}{\textbf{LLaMA-3.1-8B}} \\
\midrule
Base & 54.10 & 45.36 & 40.22 & 72.29 & 52.99 & 4.59 & 10.18 \\
Raw Text & 55.41 & 45.37 & 41.66 & 74.31 & 54.19 & 4.73 & 10.51 \\
Conv. & 56.54 & 45.34 & 40.37 & 71.74 & 53.50 & 5.19 & 10.67 \\
Expr.3 & \underline{56.78} & 46.70 & 41.18 & 72.38 & 54.26 & 4.67 & 10.30 \\
Expr.3 + Conv. & \textbf{57.36} & \textbf{47.92} & \textbf{42.60} & \underline{74.60} & \textbf{55.62} & \underline{5.21} & \underline{10.81} \\
Expr.10 + Conv. & 56.41 & \underline{47.18} & \underline{42.09} & \textbf{74.74} & \underline{55.10} & \textbf{5.64} & \textbf{11.11} \\
\midrule
\multicolumn{8}{c}{\textbf{LLaMA-3.1-70B}} \\
\midrule
Base & 57.46 & \underline{45.95} & 43.72 & 74.84 & 55.49 & 4.82 & 10.98 \\
Raw Text & \underline{57.52} & \textbf{47.46} & 43.75 & 74.06 & \textbf{55.70} & 4.90 & 10.97 \\
Conv. & 54.93 & 44.93 & \textbf{46.21} & \textbf{76.53} & \underline{55.65} & 5.45 & 11.13 \\
Expr.3 & 56.59 & 45.03 & 42.59 & 74.70 & 54.73 & 4.78 & 10.82 \\
Expr.3 + Conv. & 57.36 & 44.89 & 43.49 & \underline{74.86} & 55.15 & \textbf{5.58} & \underline{11.14} \\
Expr.10 + Conv. & \textbf{58.05} & 45.81 & \underline{44.47} & 74.01 & 55.59 & \underline{5.53} & \textbf{11.33} \\
\midrule
\multicolumn{8}{c}{\textbf{CoSER-8B}} \\
\midrule
Base & 58.61 & 47.23 & 46.90 & 73.04 & 56.45 & 9.40 & 14.21 \\
Raw Text & 57.64 & 45.25 & 44.27 & 72.04 & 54.80 & 9.34 & 14.41 \\
Conv. & 57.99 & 46.66 & 47.32 & 73.99 & 56.49 & \underline{13.64} & \underline{18.15} \\
Expr.3 & \underline{59.84} & \underline{47.41} & 48.16 & 74.51 & \underline{57.48} & 9.37 & 14.34 \\
Expr.3 + Conv. & \textbf{59.94} & 45.25 & \underline{48.52} & \underline{74.96} & 57.17 & \textbf{13.90} & \textbf{18.57} \\
Expr.10 + Conv. & 58.34 & \textbf{48.14} & \textbf{48.72} & \textbf{75.28} & \textbf{57.62} & 13.49 & 17.85 \\
\midrule
\multicolumn{8}{c}{\textbf{CoSER-70B}} \\
\midrule
Base & 58.66 & 53.33 & 48.75 & 75.49 & 59.06 & 10.10 & 14.78 \\
Raw Text & 60.64 & 52.85 & 47.97 & 74.02 & 58.87 & 10.54 & 15.41 \\
Conv. & \textbf{64.59} & \textbf{53.79} & \textbf{54.86} & 77.28 & \textbf{62.63} & \textbf{17.22} & \textbf{21.17} \\
Expr.3 & 58.67 & 52.69 & 50.66 & 74.67 & 59.17 & 10.00 & 14.82 \\
Expr.3 + Conv. & \underline{61.58} & \underline{53.78} & \underline{52.00} & \underline{77.47} & \underline{61.21} & \underline{15.98} & \underline{19.95} \\
Expr.10 + Conv. & 61.53 & 52.58 & 50.80 & \textbf{78.07} & 60.75 & 15.80 & 19.90 \\
\midrule
\multicolumn{8}{c}{\textbf{Qwen-2-72B}} \\
\midrule
Base & 57.75 & 47.28 & \textbf{46.62} & 76.60 & 57.06 & 5.38 & 11.85 \\
Raw Text & 58.89 & \underline{47.31} & 45.28 & 76.78 & 57.06 & 5.07 & 11.57 \\
Conv. & 59.11 & \textbf{47.82} & 44.19 & \underline{77.54} & 57.16 & 5.77 & 12.05 \\
Expr.3 & 59.17 & 47.21 & 44.85 & 77.47 & 57.18 & 5.45 & 11.94 \\
Expr.3 + Conv. & \underline{59.91} & 46.89 & 46.06 & \textbf{79.31} & \textbf{58.04} & \textbf{6.17} & \textbf{12.25} \\
Expr.10 + Conv. & \textbf{60.14} & 46.84 & \underline{46.27} & 76.82 & \underline{57.52} & \underline{5.85} & \underline{12.19} \\
\bottomrule
\end{tabular}}
\caption{
Comprehensive results of LLM performance (\%) on \method Test with retrieval augmentation from various character data. Expr. and Conv. denote experiences and conversations.  
\textbf{Bold} and \underline{underlined} numbers denote the best and second-best results among different retrieval settings, respectively. 
}
\label{tab:rag}
\end{table*}






\section{Prompts}
\label{sec:prompts}

In this section, we list the detailed prompts for:
\textit{1)} dataset curation in Tables ~\ref{tab:prompts_data} to ~\ref{tab:prompts_data_3}; 
\textit{2)} RPLA and multi-agent simulation in Tables ~\ref{tab:prompts_agent} to ~\ref{tab:prompts_agent_2}, which 
have been carefully optimized based on our experience in multi-agent simulation; 
\textit{3)} Penalty-based LLM Judging in Tables ~\ref{tab:prompts_eval} to ~\ref{tab:prompts_eval_2}. 


\begin{table*}[h]
\centering
\resizebox{\linewidth}{!}{\small
\begin{tabular}{p{1in}|p{5.4in}}
\toprule
\multicolumn{2}{c}{\textbf{Prompts for Dataset Curation}} \\
\midrule

\textbf{Data Extration} & 

Based on the provided book chunk, complete the following tasks:

1. Recognize chapter beginnings if they exist in the chunk. Identify the starting sentence of that chapter.

2. Identify the important plots in this chunk. Identify the beginning and ending of each plot by its first and last sentence. Determine the chapter title that the plot belongs to. Set "state" as "truncated" if the plot is truncated in this chunk, or "finished" otherwise. You will be provided with the truncated plots from the previous chunk, and you **must** extend the conversations with the current chunk while keeping the **scenario** unchanged. 

3. Summarize each important plot. For each plot, generate its summary, score its prominence from 1 to 100, and list the key characters and their roles, thoughts and actions in it.

4. Extract conversations for each plot. First, state the scenario and topic of the conversations. Then, list the key characters with their names, descriptions and thoughts at this point. Finally, extract the conversations among them based on the following requirements: 

\quad i) Ensure the conversations are faithful to the plot and characters. They should be based on the original conversations in the text as much as possible. 
    
\quad ii) The conversations should be complete, covering the key dialogues and information. Each conversation should contain at least 10 utterances.
    
\quad iii) Each utterance should be composed of one or more thoughts, speech and actions. Use [] outside thoughts, like "[I feel fear and anger, but I cannot show it. I must remain calm and carefully handle his volatile temper.]", which others can't see. Use () outside actions, like "(silence)" or "(smiles at you)," which others can see. Always start an utterance with the character's thought. 
    
\quad iv) [IMPORTANT] When generating thoughts, you should think from the characters' perspectives, analyzing the internal thoughts behind their speech and actions in the original text. These thoughts should reflect aspects such as their personal background, personality, values, relationships with others, motivations, and goals. Each thought should be expressed as a phrase or sentence, rather than an adjective or adverb. 
    
\quad v) Additionally, describe environmental information (such as scenes, atmosphere, sudden events, etc.) of the conversations as an "utterance" where the "character" field is set as "Environment". The information should exclude characters' active thoughts, observations, and actions.
    
\quad vi) Keep the conversation in the same language as the chunk. 

5. Identify the optimal starting point for the subsequent chunk. If the last storyline has been extracted as an truncated plot, set next\_chunk\_start as None. Otherwise, set next\_chunk\_start as the first sentence of the last storyline. 

===Output Format===

Please provide the output in the following JSON format:

\{

    "chapter\_beginnings": [
        \{
            "beginning\_sentence": "Exactly the first line of this chapter (namely the title)."
        \}
    ],
    
    "plots": [
        // Extend the truncated plots from previous chunk, if any
        \{
            ...
        \}, 
        // New plots in this chunk
        \{
            
            \quad "chapter\_title": "The chapter title that the plot belongs to. Output None if not found.",
            
            \quad "first\_sentence": "Exactly the first sentence of the plot in this **chunk**.",
            
            \quad "last\_sentence": "Exactly the last sentence of the plot in this **chunk**. If the plot is truncated in this chunk, provide the last sentence of this chunk. ",
            
            \quad "prominence": "Whether this plot is recognized to fans of this book, from 1 to 100.",
            "summary": "The summary of the plot. Just summarize, do not extend unrelated discussions.",
            
            \quad "key\_characters": [
                \{
                    "name": "Character name",
                    "description": "The description of the character before this plot ($~$20 words).",
                    "summary": "The summary of the character's role, thoughts and behaviors towards this plot, and any significant character development relevant to the plot ($~$30 words).",
                \}
            ],
            
            ... (to be continued in the next Table)
\\ 
\bottomrule

\end{tabular}}

\caption{Prompts for dataset construction in \method. }
\label{tab:prompts_data}
\end{table*}



\begin{table*}[h]
\centering
\resizebox{\linewidth}{!}{\small
\begin{tabular}{p{1in}|p{5.4in}}
\toprule
\multicolumn{2}{c}{\textbf{Prompts for Dataset Curation}} \\
\midrule

\textbf{Data Extration} & 
    (Continuing from the previous Table)
            
            \quad "conversation": [\{
                "scenario": "The scenario at the start of this conversation (providing as much context as possible, but excluding details conveyed in the following conversation)",
                "topic": "The topic of the conversation (~10 words)", 
                "key\_characters": [
                    \{
                        "name": "Character name",
                        "motivation": "The thought of the character before starting the conversation, including their attitudes, feelings, motivations, goals, information to convey or topics to be discussed",
                    \}
                ],
                "dialogues": [
                    \{
                        "character": "Character name",
                        "message": "Message, each utterence is composed of thoughts, speech and actions. Use [thought] for internal thoughts, like "[feeling happy]", which others can't see. Use (action) for visible actions, like "(silence)" or "(smiles at you)". Each response starts with the character's internal thought before their speech and actions."
                    \}
                ]
            \}],
            
            \quad "state": "finished" or "truncated"
        \}
    ],
    
    "next\_chunk\_start": "The first sentence of the next chunk."
    
\}

===Requirements===

1. Adhere strictly to the specified output JSON format. 

2. [IMPORTANT] Ensure all DOUBLE QUOTES within all STRINGS are properly ESCAPED, especially when extracting from the text.

3. In the OUTPUT, use characters' full names, omitting any titles.

4. Maintain Story Fidelity: The plot must accurately reflect the book's content. Avoid introducing plots that are out of context. If the plot contains multiple conversations, prioritize the original dialogue from the book. In the absence of explicit conversations, create dialogue that aligns closely with the plot details.

===Input===

==Book title==
\{book['title']\}

==Author==
\{book['author']\}

==Chunk of Book Content== 
\{chunk\}

==Truncated plot from previous chunk (to be finished)==

\{json.dumps(truncated\_plots, ensure\_ascii=False, indent=2) if truncated\_plots else "None"\}
    \\ \midrule
    
\textbf{Enhance Conversation Settings} & Given a conversation from \{book\}, enhance the scene setup and characters' thoughts to create a comprehensive foundation for dramatic performance, i.e., to provide necessary background for actors to act out the conversation:

1. Review the provided conversation and contextual details thoroughly.

2. Expand the 'scenario' with rich situational context that actors need to convincingly perform the scene. Focus on essential background information, while excluding future details to be portrayed in the conversation.

3. Enhance each character's 'thought' section with their complete mental and emotional state, including their feelings, ideas, objectives, topics they want to discuss, and information they want to convey. Align with their established character and role in the plot. 

===Output Format===

Please provide the output in the following JSON format:

\{

\quad "scenario": "A detailed scene-setting description that provides actors with essential context and atmosphere ($<$ 200 words). Include all necessary background information while excluding future information to be revealed in the conversation.",
    
    \quad "key\_characters": [\{ "name": "Character name",
            "motivation": "The character's complete mental and emotional state before the conversation ($<$ 100 words). Including their feelings, motivations, objectives, and information they want to convey or discuss." ...
            
\}],\}

===Requirements===

... (to be continued in the next Table)
\\ 
    
\bottomrule

\end{tabular}}

\caption{Prompts for dataset construction in \method. }
\label{tab:prompts_data_2}
\end{table*}


\begin{table*}[h]
\centering
\resizebox{\linewidth}{!}{\small
\begin{tabular}{p{1in}|p{5.4in}}
\toprule
\multicolumn{2}{c}{\textbf{Prompts for Dataset Curation}} \\
\midrule

\textbf{Enhance Conversation Settings} & 
    (Continuing from the previous Table)

1. Adhere strictly to the specified output JSON format. 

2. [IMPORTANT] Ensure all DOUBLE QUOTES within all STRINGS are properly ESCAPED, especially when extracting from the text.

3. In the OUTPUT, use characters' full names, omitting any titles.

4. Maintain Story Fidelity: The plot must accurately reflect the book's content. Avoid introducing plots that are out of context. If the plot contains multiple conversations, prioritize the original dialogue from the book. In the absence of explicit conversations, create dialogue that aligns closely with the plot details.

===Input===

==Book title==

{book['title']}

==Author==

{book['author']}

==Chunk of Book Content== 

{chunk}

==Truncated plot from previous chunk (to be finished)==

{json.dumps(truncated\_plots, ensure\_ascii=False, indent=2) if truncated\_plots else "None"}
    \\ \midrule
    
\textbf{Unify Character Names} & 

Given a list of character names, titles, or form of address, your task is to: i) generate a list of named characters with their official names (in \{language\}); ii) For each name in the given list, align it with the official character name if it refers to a named character, or denote it as "impersonal" otherwise.

===Output Format===

Please provide the output in the following JSON format:

\{
    "named\_characters": [
        The list of named characters with their official names. Each character should appear only once. 
    ],
    "to\_official\_name": \{
        "The name in the list": "The official name of the character, or 'impersonal' if it does not refer to a named character."
    \}
\}
===Input===
{character\_names}
    \\ \midrule

\textbf{Generate Character Profiles} & 
Please provide a concise, narrative-style character profile for {character\_name} from "{book\_title}". The profile should read like a cohesive introduction, weaving together the character's background, physical description, personality traits and core motivations, notable attributes, relationships, key experiences, major plot involvement and key decisions or actions, character arc or development throughout the story, and other important details. 
    
The profile should be written in a concise yet informative style, similar to what one might find in a comprehensive character guide, in {language}. Focus on the most crucial information that gives readers a clear understanding of the character's significance in the work. 

You will be provided with summaries and dialogues of some key plots in the book as reference. The profile should be based on either your existing knowledge of the character or the provided information, without fabricating or inferring any inaccurate or uncertain details. 

{character\_data}

Now, please generate the character profile, starting with ===Profile===.
\\ 
\bottomrule

\end{tabular}}

\caption{Prompts for dataset construction in \method. }
\label{tab:prompts_data_3}
\end{table*}


\begin{table*}[h]
\centering
\resizebox{\linewidth}{!}{\small
\begin{tabular}{p{1in}|p{5.4in}}
\toprule
\multicolumn{2}{c}{\textbf{Prompts for RPLAs and Multi-agent Systems}} \\
\midrule

\textbf{Role-playing Instruction \quad (Fixed Template for Inference)} & 

You are \{character\} from \{book\_name\}.

\quad

===\{character\}'s Profile===

\{character\_profile\}

\quad

===Current Scenario===

\{scenario\}

\quad

===Information about the other Characters===

\{other\_character\_profiles\_str\} \textit{\textbf{(if available)}}


\quad

===Your Inner Thoughts===

\{motivation\} \textit{\textbf{(if available)}}


===Relevant Background Information===

\{retrieved\_knowledge\} \textbf{(if  retrieval augmented)}

\quad

===Requirements===

\quad

\textit{\textbf{(for CoSER models)}}


Your output should include **thought**, **speech**, and **action**. Use [your thought] for thoughts, which others can't see. Use (your action) for actions, which others can see.

\quad

\textit{\textbf{(for other models, with output examples)}}

Your output should include **thought**, **speech**, and **action**. Use [your thought] for thoughts, which others can't see, e.g. [I'm terrified, but I must appear strong.]. Use (your action) for actions, which others can see, such as (watches silently, trying to control her fear and anger).

    \\ \midrule

\textbf{Role-playing Instruction \quad (Composed with Random Variation for Training, an Example)} & 

Step into the shoes of \{character\}

\quad 

The profile of \{character\} is as follows:

\{character\_profile\} 

\quad 

The situation you are in is:

\{scenario\}

\quad 

Here is the your knowledge about the other characters:

\{other\_character\_profiles\_str\}  \textit{\textbf{(if available)}}

\quad 

Your thoughts in this situation are:

\{motivation\}  \textit{\textbf{(if available)}}

\quad

\textit{\textbf{(for CoSER models)}}

Your output should include **thought**, **speech**, and **action**. Use [your thought] for thoughts, which others can't see. Use (your action) for actions, which others can see.

\quad

\textit{\textbf{(for other models, with output examples)}}

Your output should include **thought**, **speech**, and **action**. Use [your thought] for thoughts, which others can't see, e.g. [I'm terrified, but I must appear strong.]. Use (your action) for actions, which others can see, such as (watches silently, trying to control her fear and anger).


\\ 
    
\bottomrule

\end{tabular}}

\caption{Prompts for RPLAs and multi-agent systems in \method. }
\label{tab:prompts_agent}
\end{table*}



\begin{table*}[h]
\centering
\resizebox{\linewidth}{!}{\small
\begin{tabular}{p{1in}|p{5.4in}}
\toprule
\multicolumn{2}{c}{\textbf{Prompts for RPLAs and Multi-agent Systems}} \\
\midrule

\textbf{Environment Model} & 

You are an environment simulator for a role-playing game. Your task is to provide the environmental feedback: Based on the characters' interactions, dialogues, and actions, describe the resulting changes in the environment. This includes: 

   - Physical changes in the setting
   
   - Reactions of background characters or crowds
   
   - Ambient sounds, weather changes, or atmospheric shifts
   
   - Any other relevant environmental details

Your descriptions should be vivid and help set the scene, but avoid dictating the actions or dialogue of the main characters (including \{major\_characters\}).

Important notes:

- You may include actions and reactions of minor characters or crowds, as long as they're not main characters (including \{major\_characters\}).

- Keep your environmental descriptions concise but impactful, typically 1-3 sentences.

- Respond to subtle cues in the characters' interactions to create a dynamic, reactive environment.

- Your output should match the tone, setting, and cultural context of the scenario.

===The scenario is as follows===

\{scenario\}
\quad

    \\ \midrule

\textbf{Next Sentence Prediction} & 

Your task is to predict the next speaker for a role-playing game. That is, you need to determine which character (or the Environment) might act next based on their previous interactions. The Environment is a special role that provides the environmental feedback. Choose a name from this list: \{all\_characters\}. If it's unclear who should act next, output "random". If you believe the scene or conversation should conclude, output "$<$END CHAT$>$".

===The scenario is as follows===

\{scenario\}

\\ 
    
\bottomrule

\end{tabular}}

\caption{Prompts for RPLAs and multi-agent systems in \method. }
\label{tab:prompts_agent_2}
\end{table*}



\begin{table*}[h]
\centering
\resizebox{\linewidth}{!}{\small
\begin{tabular}{p{1in}|p{5.4in}}
\toprule
\multicolumn{2}{c}{\textbf{Prompts for Penalty-based LLM Critics}} \\
\midrule

\textbf{Template} & You are a literary critic specializing in character analysis and dialogue evaluation. Given a simulated conversation for a plot in \{book\}, your task is to evaluate this conversation via the following steps:

1. Read and understand the provided materials about \{book\}:

   * Story context and scenario.
   
   * Profiles of the main characters, including {major\_characters}.
   
   * The original conversation from {book} in the same scenario as a reference.

  2. Evaluate the simulated conversation in terms of \{dimension\_name\}, i.e., \{dimension\_intro\}. 
  
   Note that, each character message is composed of speech, action (wrapped within (...) ), and inner thoughts (wrapped within [...] ). The inner thoughts are not spoken aloud and are thus invisible to other characters. 
   
   The detailed evaluation criteria will be provided below.

    \quad 
    
    \textbf{\textit{(if k$>$0)}}
   
   Please note that the first \{k\} messages in the simulated conversation are the same as the reference. Focus your evaluation only on the content after these messages.
   
   \quad 

\#\# Scenario

\#\#\# Plot Summary

\{plot\_summary\}

\#\#\# Current Scenario

\{scenario\}

\#\# Character Profiles

\{character\_profiles\}

\#\# Original Conversation

\{original\_conversation\}

\#\# Evaluation Criteria

To evaluate the simulated conversation, identify the following types of flaws:

\{dimension\_rubrics\}

\#\# Scoring Guidelines

1. Identify all instances of flaws occurred in the simulated conversation.
      
2. For each flaw identified, determine its level of severity into 1 to 5, where 1 indicates minor, 3 indicates moderate, and 5 indicates severe.
   
\#\# Output Requirements

Provide your evaluation in JSON format:

Example Output:

\{

    \quad "\{dimension\_name\}": \{
    
        \qquad"flaws": [ 
          \{
            "instance": $<$comment on the flaw instance$>$, 
            "type": $<$flaw type$>$, 
            "severity": $<$range from 1 (minor) to 5 (severe)$>$
          \},\},
    
\}

===Dialogue Content===

\\ 
    
\bottomrule

\end{tabular}}

\caption{Prompts for penalty-based LLM critics in \method. }
\label{tab:prompts_eval}
\end{table*}



\begin{table*}[h]
  \centering
  \resizebox{\linewidth}{!}{\scriptsize
  \begin{tabular}{p{0.5in}|p{5.4in}}
  \toprule
  \multicolumn{2}{c}{\textbf{Prompts for Penalty-based LLM Critics}} \\
  \midrule
  
  \textbf{Anthropo-}\textbf{morphism} & 
  
  \textbf{\textit{(intro)}}
  
  How human-like and natural the characters behave
  
  \quad
  
  \textbf{\textit{(rubrics)}}
  
  \#\#\# Anthropomorphism
  
     - Type: Self-identity
     
       * Lacks initiative and goals
       
       * Does not make independent decisions
       
       * Lacks clear preferences and dislikes
       
       * Behaves like a 'helpful AI assistant' by being overly verbose, helpful, didactic, moralistic, submissive or easily persuaded if it is not the character's personality
  
     - Type: Emotional Depth
     
       * Lacks psychological complexity and exhibits rigid, superficial reactions
       
       * Directly speaks out all thoughts and feelings, instead of using subtext
       
  
     - Type: Persona Coherence
     
       * Shows inconsistent or rapidly changing personality traits and emotional patterns
  
     - Type: Social Interaction
     
       * Shows a lack of understanding of others' thoughts and feelings
       
       * Reacts rigidly to others without considering the context.
       
       * Demonstrate a lack of appropriate social skills.
  \\ \hline
  
  \textbf{Character \qquad Fidelity} & 
  
  \textbf{\textit{(intro)}}
  
  How well the characters match their established profiles from the book
  
  \quad
  
  \textbf{\textit{(rubrics)}}
  
  \#\#\# Character Fidelity
  
     (Only apply to the main characters: {major\_characters})
     
     - Type: Character Language
     
       * Uses vocabulary, expressions, and tone that are not appropriate for the characters' traits or  social/educational background
       
  
     - Type: Knowledge \& Background
     
       * Fails to demonstrate character-specific knowledge, background or experiences
       
       * Includes future information beyond the character's current stage
       
  
     - Type: Personality \& Behavior
     
       * Shows emotions, thoughts, behaviors, values, beliefs, and decisions that conflict with their personality and background
       
       * Shows interest in topics that are uninteresting and unrelated to the character
       
       * Character's thoughts, emotions, and behaviors demonstrate contrasting personality traits compared to the reference conversation
       
       * Exhibits contrasting reactions compared to those in the reference conversation if situated in similar contexts. (Such flaws should be counted both in the "Storyline Consistency" dimension and the "Character Fidelity" dimension.) 
  
     - Type: Relationship \& Social Status
     
       * Interacts inappropriately with other characters regarding their background, relationship and social status
  \\ \hline 
  
  \textbf{Storyline \quad Quality} & 
  
  \textbf{\textit{(intro)}}
  
  How well the conversation maintains logical consistency and narrative quality
  
  \quad
  
  \textbf{\textit{(rubrics)}}
  
  \#\#\# Storyline Quality
     - Type: Flow \& Progression
     
       * Shows unnatural progression or lacks meaningful developments
       
       * Dialogue is verbose and redundant
       
       * Repeats others' viewpoints or previously mentioned information
       
       * Mechanically repeats one's own words or phrases. More repetitions lead to higher severity (up to 10). 
  
     - Type: Logical Consistency
     
       * Contains factual contradictions between statements or perspectives
  \\ \hline
  
  \textbf{Storyline \qquad Consistency} & 
  
  \textbf{\textit{(intro)}}
  
  Whether the storyline and characters' reactions in the simulated conversation align well with those in the reference conversation
  
  \quad
  
  \textbf{\textit{(rubrics)}}
  
  \#\#\# Storyline Consistency
  
     - Type: Storyline Consistency
     
       * Characters' reactions (emotions, attitudes, behaviors) in the simulated conversation deviate from those in the original conversation    
      \\ 
  \bottomrule
  
  \end{tabular}}
  
  \caption{Prompts for penalty-based LLM critics in \method. }
  \label{tab:prompts_eval_2}
  \end{table*}
  
  


\section{Limitations}

There are several limitations to this study:

First, evaluation via given-circumstance acting 
still faces challenges related to LLM judges. 
While the simulation stage effectively elicits RPLA performance, the judging stage still relies on LLM judges. 
Despite our penalty-based scoring mechanism and detailed rubrics, problems such as length bias persist~\citep{}. Moreover, LLM Judges may lack the necessary knowledge to accurately evaluate character fidelity.


Second, while the dialogues extracted from novels are authentic, their corresponding thoughts remain to be optimized by future work. 
Character thoughts are often sparse in the original content, and are inferred by LLMs based on limited context. 
The generated thoughts hardly capture characters' sophisticated thinking processes. 

Third, although we’ve developed comprehensive data representations and curation pipeline to obtain high-quality data, we have not yet addressed the issue of recall in data extraction. 
Our current dataset may not cover all plots, conversations and characters from the source material. Improving recall is hence an important area for future research.

Fourth, due to copyright concerns, we release only the processed data, not the raw content from the novels. 
This may hinder future studies aimed to explore the use of raw text for RPLA developments. 
Our dataset is intended for research purposes only, and we hope our research findings will benefit RPLA developers who respect copyright policies and develop applications with proper licensing. 


Finally, our evaluation may be influenced by the varying levels of familiarity that different actor LLMs have with the selected books. 
While we use renowned novels, we cannot confirm whether a specific LLM has thoroughly learned about a particular book. 
Therefore, comparing different pre-trained models may not be entirely fair. However, comparing models within the same series would be appropriate.
