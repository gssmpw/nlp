

% \luke{Can cut this entire section to shorten?}

% \begin{itemize}
%     \item lots of things happening 
%     \item we categorize like this
%     \item in a similar vein, we make checklist
% \end{itemize}

% \section{Diving into the details of deriving a scaling law}\label{sec:classification}

% Fitting a specific scaling law to a set of models involves several decisions in experimental design, which can greatly affect the result of a study. We broadly divide these decisions into the following:

% \paragraph{Section \ref{sec:power-law-form}: What \textit{form} are we fitting?} Researchers may choose any number of power law forms relating any set of variables, to which they fit the data extracted from training runs. Even seemingly minor differences in form, may imply critical changed in assumption -- for example, about certain interactions between variables which are excluded, the definitions of these variables or error terms which are deemed significant enough to include.

% \paragraph{Section \ref{sec:model_training}: How do we \textit{train models}?} In order to fit a scaling law, one needs to train a range of models spanning orders of magnitude in parameter count and/or dataset size. Each model requires a multitude of hyperparameter and parameter choices, such as the specific model/dataset sizes to use, the architecture shape, batch size or learning rate schedule.

% \paragraph{Section \ref{sec:data}: How do we \textit{extract data} after training?} Once these models are trained, downstream metrics like perplexity must be obtained from the intermediate or final checkpoints. This data may be also scaled, interpolated or bootstrapped to create more datapoints to properly fit the power law parameters.

% \paragraph{Section \ref{sec:opt}: How are we \textit{optimizing} the fit?} Finally, the variable must be fit with an objective and optimization method, which may in turn have their own initialization and hyperparameters to choose.

% \begin{figure}
    \centering
    \includegraphics[width=\linewidth]{figures/MCQA_checklist.pdf}
    \vspace{-4.75ex}
    \setlength{\fboxsep}{0pt}
    \caption{\small Example unanswerable MCQ from MMLU \cite{gema2024we}, along with rubric criteria from \citet{haladyna1989taxonomy} flagged by OpenAI's o1 \cite{jaech2024openai}.}
    \label{fig:checklist}
    \vspace{-1.7ex}
\end{figure}

% Motivated by the literature we discuss to answer the questions above, and the analysis in Section \ref{sec:own-repl}, we provide a checklist in Table \ref{sec:checklist} that we hope researchers will consider while conducting scaling laws investigations.

%
% , which can affect the goodness of fit.

% Prior work on scaling laws make many design decisions affecting their findings. We classify these decisions into one of three groups:

% \paragraph{Power Law Form} Each approach chooses a form for their power law, which necessitates making some assumptions -- for example, about certain interactions between variables which are excluded, or error terms which are deemed significant enough to include.

% \paragraph{Data} The data used to fit scaling laws comes from many model training runs, which themselves have many settings including hyperparameters like batch size or learning rate schedule, as well as model and training dataset size.

% \paragraph{Optimization} Each scaling law approach must also specify a loss objective, which determines the goodness of fit. 

