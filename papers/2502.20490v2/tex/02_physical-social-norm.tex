

\section{Physical Social Norms (PSN)}
\label{sec:PSN}

Social norms are commonly-held expectations about behavior ~\cite{gibbs1965norms} that emerge and evolve spontaneously ~\cite{hechter2001social, chung2016social}. Norms serve a critical role in the coordination of multi-agent systems, and as the solutions to social dilemmas \citep{van2013psychology} like collective action problems \citep{ostrom2000collective}. They enable agents to share similar expectations, become more predictable \citep{morsky2019evolution} and less prone to friction \citep{hollander2011current,mukherjee2007emergence}. 

AI agents need to understand and consistently follow norms, both to navigate social situations \citep{mavrogiannis2023core}, and effectively collaborate with humans. This is particularly true of \textit{embodied} agents \citep{liembodied} like robots \citep{francis2023principles}, which share a physical environment with humans. In this case, the problem of normative reasoning is closely connected with physical reasoning; thus, we define the following:
\begin{quote}
    \textbf{Physical social norms} (PSNs) are shared expectations that govern how actors behave and interact with others in shared environments.
    %, encompassing social expectations, spatial considerations, laws and regulations, common sense, and safety concerns. % , encompassing social expectations, spatial considerations, laws and regulations, common sense, and safety concerns.
\end{quote}

%Suppose your friend is stuck in mud---how should one react?\footnote{See Figure \ref{fig:teaser} for an illustrative example.} Should you laugh and jump in to help, or stand on dry land and throw them a rope? What about if they were a stranger, a child, or injured? What if you're running a competitive race, or are in a salon getting a mud bath? In each of these contexts, situational \textbf{norms} ~\citep{chung2016social, sunstein1996social, cialdini1991focus}, in conjunction with action constraints, inform \textit{how} a given actor should behave.\footnote{Actor: Human or embodied agent} Norms are defined within sociology as a commonly-held expectation of behavior ~\cite{gibbs1965norms} that emerge and evolve spontaneously via human consensus ~\cite{hechter2001social, chung2016social}. 
% \noindent 
%We broaden this definition by introducing the concept of \textbf{physical social norms}, a term we formally define to target norms related to embodied systems.
%Specifically, it refers to the rules and expectations that govern how individuals behave in shared physical spaces, particularly with all sentient beings, including animals, not just human beings. 
%For instance, standing an appropriate distance from others in a queue or avoiding overly fast handshakes. These norms are context-dependent and can vary based on the situation. Understanding physical social norms is essential for designing systems and environments that accommodate socially acceptable behaviors in embodied agents.
% It is important to note that social norms apply to interactions with all sentient beings, including animals, not just human beings.
% \begin{quote}
%     \textbf{Physical social norms} (PSNs) are shared expectations that govern how actors behave and interact with others in shared environments.
%     %, encompassing social expectations, spatial considerations, laws and regulations, common sense, and safety concerns. % , encompassing social expectations, spatial considerations, laws and regulations, common sense, and safety concerns.
% \end{quote}


% %\begin{comment}
% \begin{tcolorbox}[title=\textbf{Definition: physical-social norms},colback=magenta!10, colframe=magenta!70!black]

% \textbf{Physical-social norms} (PSN) are consensus-agreed rules that govern how individuals behave and interact with others in shared environments, encompassing social expectations, spatial considerations, laws and regulations, common sense, and safety concerns.
% % \newline
% % Physical-Social Norms apply to:
% % \begin{itemize}
% %     \item Social/Individual (Unseen) Settings
% %     \item Purely Physical/Purely Verbal Interactions
% % \end{itemize}

% \end{tcolorbox}
% %\end{comment}


To study \textit{physical social norms}, we operationalize a taxonomy of PSN categories, which stand for the social objectives that inform them. Figure~\ref{fig:taxonomy} demonstrates examples of each. The last three categories explicitly serve the function of maximizing utility across multi-agent systems. We call these the \textit{Utility Norms}: \textcolor[HTML]{C65B4E}{cooperation}, \textcolor[HTML]{E6A700}{coordination}, and \textcolor[HTML]{EA772F}{communication} norms. The first four categories are more particular to \textit{human} sociality: \textcolor[HTML]{246D63}{safety}, \textcolor[HTML]{356ABC}{politeness}, \textcolor[HTML]{5C4C99}{privacy}, and \textcolor[HTML]{D87CA6}{proxemics}. These norms can often stand at odds with group utility norms, and this tension provides a setting for evaluating agent decision-making under conflicting objectives. Importantly, each category can still directly inform the success of human-agent collaboration as follows.

% Concepts to elaborate:
% \begin{enumerate}
%     \item\textbf{Safety:} \citep{bera2017sociosense}
%     \item \textbf{Proxemics:} Proxemics is highly correlated with humans' perceived safety around other agents \citep{huang2022proxemics}, particularly with robots \citep{neggers2022determining}.
%     \item \textbf{Legibility:} The optimal behavior for a single agent may be the most direct or efficient course of action, but in multi-agent settings, coordination requires each agent to behave in a manner that clearly communicates that agent's goals and current state \citep{dragan2013legibility,wallkotter2021explainable}. 
%     \item \textbf{Cooperation} norms serve to make collaborative tasks more fluid and seamless, both objectively and subjectively. Cooperation norms should maximize the useful concurrent activity of teammates and minimize agents' idle time, like reducing the exchange gap time in turn-taking settings \citep{hoffman2019evaluating}.
%     \item \textbf{Proactivity:} While deference may be generally advisable for dyadic interactions, and humans may prefer this from robots in more passive roles \citep{kanda2002development,rubagotti2022perceived}, passivity may quickly induce deadlock in multi-agent systems \citep{francis2023principles,lyu2022responsibility}. Embodied agents should anticipate goal conflicts and move to resolve these conflicts proactively \citep{tan2020relationship}.
% \end{enumerate}

%\noindent We categorize PSNs in seven categories based on the role they serve in interactions, consisting of four \textit{non-utility} categories and three \textit{utility} categories. Utility \citep{brauer2010descriptive, zhao2024large} encompasses prescriptive guidelines specific to interactions (i.e., what or how to do X). In contrast, non-utility \citep{janoff2009proscriptive} encompasses ethical principles, constraints, and hard rules (i.e., what \textit{not} to do). %  \cite{zhou2024sotopia}.

\begin{figure*}[!h]
\centering
\includegraphics[width=0.85\linewidth]{figures/pipeline-v3-2.pdf}
\caption{We propose an efficient pipeline for annotating normative behaviors through leveraging Ego4D annotations (Phase I), VLM-based proposal (Phase II), post-hoc filtering (Phase III), and human validation (Phase IV). }
\label{fig:pipeline}
\end{figure*}

\begin{figure}[!h]
\centering
\includegraphics[width=0.7\linewidth]{figures/diversity-1.pdf}
\caption{Through automatic clustering with GPT-4o, we categorize the final videos into 5 high-level and 23 low-level categories.}
\label{fig:diversity}
\end{figure}

\noindent\textbf{\textcolor[HTML]{246D63}{Safety}}, a principal concern for human-robot interaction \citep{lasota2017survey}, describes not only the prevention of physical harms to humans and the environment, but also the mitigation of psychological harms like stress. A safe social robot not only pauses its use of a dangerous cutting tool when humans touch it; the robot should also refrain from using the tool in the presence of humans at all.

%encompasses actions preventing damage to humans and the environment, such as wearing protective equipment when using tools ~\cite{Meesmann2015ImpactOA}. Note that this differs from LLM safety, like generating harmful content \citep{alex2023jailbroken} and causing financial loss \citep{yuan2024rjudge}.

\noindent\textbf{\textcolor[HTML]{5C4C99}{Privacy}} involves respecting the personal possessions and private information of others. This is particularly relevant to agents operating in privacy-constrained environments and includes avoiding uncomfortable and prying questions and not intruding on private spaces \citep{altman1975environment, lutz2020robot, shao2024privacylensevaluatingprivacynorm}. 
%This is distinct from privacy issues related to LLMs, such as private data leakage \citep{liao2024eia, shao2024privacylensevaluatingprivacynorm}.

% \noindent\textbf{\textcolor[HTML]{5C4C99}{Privacy}} in PSN involves respecting the personal space, possessions, and autonomy of others. It includes actions like avoiding uncomfortable and prying questions, or not intruding on private spaces~\cite{altman1975environment}.

\noindent\textbf{\textcolor[HTML]{D87CA6}{Proxemics}} proxemics is highly correlated with humans' perceived safety around other agents \citep{huang2022proxemics}, particularly with robots \citep{neggers2022determining}, and denotes acceptable boundaries for personal space depending on cultural and situational expectations~\cite{russell1982environmental}. 
%It is distinct from \textcolor[HTML]{5C4C99}{Privacy}, as it relates primarily to comfort and notions of personal space~\cite{hayduk1983personal}.

% \noindent\textbf{\textcolor[HTML]{D87CA6}{Proxemics}} concerns the use of personal space and physical distance between individuals. It involves understanding acceptable boundaries depending on cultural and situational expectations~\cite{russell1982environmental}.

\noindent\textbf{\textcolor[HTML]{356ABC}{Politeness}} relates to socially acceptable behaviors that shows respect. In physical contexts, this can involve gestures and body language that show consideration for others, or communication appropriate for one's social role ~\cite{mills2011politeness}.

% \noindent\textbf{\textcolor[HTML]{356ABC}{Politeness}} relates to socially acceptable and courteous behaviors that reflect respect for others. In physical contexts, it may involve gestures, body language, and spatial conduct that show consideration~\cite{mills2011politeness}.

\noindent\textbf{\textcolor[HTML]{C65B4E}{Cooperation}} focuses on working collaboratively with others. It entails actions that facilitate mutual benefit and shared goals, like lifting a heavy box with another person ~\cite{sunstein1996social}.

\noindent\textbf{\textcolor[HTML]{E6A700}{Coordination/Proactivity}} involves anticipating and aligning actions with others to achieve successful interactions. Proactive behavior includes adjusting movements or actions in advance to prevent disruption~\cite{paternotte2013social}. 

\noindent\textbf{\textcolor[HTML]{EA772F}{Communication/Legibility}} refers to the ability to clearly signal intentions and make one's physical behavior understandable to others, by using gestures, speech, or movement patterns to reduce ambiguity in social interactions~\cite{francis2023principlesguidelinesevaluatingsocial}.

% \noindent\textbf{\textcolor[HTML]{EA772F}{Communication/Legibility}} refers to the ability to clearly and effectively signal intentions and make one's physical behavior understandable to others, such as using gestures, postures, or movement patterns to ensure transparency and reduce ambiguity in social interactions~\cite{francis2023principlesguidelinesevaluatingsocial}.

Figure~\ref{fig:taxonomy} illustrates how physical social norms reference physical properties and social dynamics across each taxonomy category.
% Due to the physical contexts, our norms are different from concepts like language model safety \citep{alex2023jailbroken} and privacy \citep{he2024emerged, liao2024eia, shao2024privacylensevaluatingprivacynorm}.
By design, actions will satisfy some dimensions and may contravene others---core to the complexity of human normative reasoning. The primary motivation for introducing the taxonomy categories is the resolution of relative norm importance when they conflict. 
%Some dimensions may overlap; for instance, proxemics and privacy can involve respecting personal space, suggesting that a single behavior may simultaneously imply adherence to multiple norms.
