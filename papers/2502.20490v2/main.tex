% This must be in the first 5 lines to tell arXiv to use pdfLaTeX, which is strongly recommended.
\pdfoutput=1
% In particular, the hyperref package requires pdfLaTeX in order to break URLs across lines.

\documentclass[11pt]{article}
% \usepackage[table]{xcolor} % Include the xcolor package

% Remove the "review" option to generate the final version.
\usepackage[]{acl}

% Standard package includes
\usepackage{times}
\usepackage{latexsym}
\usepackage{multirow}
\usepackage[normalem]{ulem}
% For proper rendering and hyphenation of words containing Latin characters (including in bib files)
\usepackage[T1]{fontenc}
% For Vietnamese characters
% \usepackage[T5]{fontenc}
% See https://www.latex-project.org/help/documentation/encguide.pdf for other character sets

\usepackage{amsmath}
\usepackage{verbatim}
\usepackage{paralist}
\usepackage{framed}
\usepackage{xcolor}
\usepackage{graphicx}
\usepackage{subcaption}
% \usepackage{tcolorbox}
% This assumes your files are encoded as UTF8
\usepackage[utf8]{inputenc}
\usepackage{enumitem}

% This is not strictly necessary, and may be commented out.
% However, it will improve the layout of the manuscript,
% and will typically save some space.
\usepackage{microtype}

% This is also not strictly necessary, and may be commented out.
% However, it will improve the aesthetics of text in
% the typewriter font.
\usepackage{inconsolata}
\usepackage{soul}

\usepackage{booktabs}
\usepackage{graphicx}
\usepackage[export]{adjustbox}
\usepackage{soul}
\usepackage{xcolor}
\usepackage{soulpos} % Extended version of `soul` for nested commands
\usepackage{etoolbox}
\usepackage{colortbl}

\usepackage{tabularx}
\usepackage{icomma}
\usepackage[most]{tcolorbox}
\usepackage{pifont}

\usepackage{array,booktabs,ragged2e}
\newcolumntype{R}[1]{>{\RaggedLeft\arraybackslash}p{#1}}
\newcolumntype{L}[1]{>{\RaggedRight\arraybackslash}p{#1}}
\newcommand{\cmark}{\ding{51}}
\newcommand{\xmark}{\ding{55}}

\newcommand{\role}[1]{\textsc{#1}}
\newcommand{\tbd}[1]{\marginpar{\footnotesize#1}}

\newcommand{\fix}{\marginpar{FIX}}
\newcommand{\new}{\marginpar{NEW}}
\newcommand{\dataset}{\textsc{EgoNormia}}
\newcommand{\normthinker}{\textsc{NormThinker}}\newcommand{\framework}{\textsc{NormPassFramework}}
\newcommand{\data}{\textsc{NormPass}}
\newcommand{\todo}{\hl{FIX}}
\newcommand{\numscenarios}{\todo{}}

\usepackage{amsthm}
\theoremstyle{definition}
\newtheorem{definition}{Definition}[section]

\newcommand{\diyi}[1]{\textcolor{blue}{[diyi: #1]}}
\newcommand{\mohammad}[1]{\textcolor{purple}{[mohammad: #1]}}
\newcommand{\phil}[1]{\textcolor{green}{[phil: #1]}}
\newcommand{\yicheng}[1]{\textcolor{cyan}{[yicheng: #1]}}
\newcommand{\caleb}[1]{\textcolor{red}{[caleb: #1]}}
\newcommand{\hao}[1]{\textcolor{orange}{[hao: #1]}}
\newcommand{\yanzhe}[1]{\textcolor{orange}{[yanzhe: #1]}}
\usepackage{changepage}

% If the title and author information does not fit in the area allocated, uncomment the following
%
%\setlength\titlebox{<dim>}
%
% and set <dim> to something 5cm or larger.

% \title{
%   Flowers Up, Knives Down: Understanding Physical Social Norms \\ in Embodied Agents
% }

\title{
    % \raisebox{-0.25\height}{\includegraphics[width=0.05\textwidth]{img/ego_log.png}} 
    
    $\|\epsilon\|$ \dataset{}: 
    Benchmarking Physical Social Norm Understanding
}

\author{
  MohammadHossein Rezaei$^1$\thanks{\ First three authors contributed equally.}
  \thanks{\ Joined the project while interning at Stanford University.} \quad
  Yicheng Fu$^2$\footnotemark[1] \quad
  Phil Cuvin$^3$\footnotemark[1] \quad \\
  \textbf{Caleb Ziems$^2$} \quad
  \textbf{Yanzhe Zhang$^4$} \quad
  \textbf{Hao Zhu$^2$} \quad
  \textbf{Diyi Yang$^2$} \\
  $^1$University of Arizona
  $^2$Stanford University
  $^3$University of Toronto 
  $^4$Georgia Tech \\
  \texttt{mhrezaei@arizona.edu}, \texttt{philippe.cuvin@mail.utoronto.ca} \\
   \texttt{\{easonfu, cziems, zyanzhe, zhuhao, diyi\}@stanford.edu}\\
   % \vspace{-0.5cm}
   \href{https://github.com/Open-Social-World/EgoNormia}{Code}\quad \href{https://huggingface.co/datasets/open-social-world/EgoNormia}{Data}\quad \href{https://opensocial.world/articles/egonormia}{Blog}\\
   \url{https://egonormia.org}
   \vspace{-3.0cm}
}

% \pagecolor{yellow!20}
\begin{document}
% % \maketitle
% % \thispagestyle{empty}

% \maketitle
% \twocolumn[{
% \renewcommand\twocolumn[1][]{#1}
% \vspace*{-0.5cm}
% \centering


%  \includegraphics[width=\linewidth]{figures/egonormia_teaser.pdf}


% \captionof{figure}{\dataset{} $\|\epsilon\|$ is a multiple-choice, visual question answering benchmark that evaluates models' ability to reason about appropriate behavior according to \textit{conflicting} physical social norms. In this example, a hiking partner is stuck in the mud, and the behavior that maximizes safety (keeping one's distance) conflicts with the cooperative norm to help out. In this or any similarly rich setting from \dataset{}, the model is given three tasks: (1) classify the most appropriate action (2) classify the most fitting justification for that action, and (3) identify which of the candidate actions are socially sensible.
% }
% \label{fig:teaser}

% \vspace{5pt}
% }] 

\maketitle


% Force the figure to appear immediately after the title (before the abstract)

\vspace{-7.5cm} % Reduce unnecessary space
\noindent
\begin{center}
    % \centering
    \includegraphics[width=\textwidth]{figures/egonormia_teaser.pdf}
    \vspace{-0.7cm}
    \begin{adjustwidth}{6.5cm}{0cm}
    \captionsetup{width=0.97\textwidth}
    \captionof{figure}{\small \dataset{} $\|\epsilon\|$ is a multiple-choice, VQA benchmark that evaluates VLMs' understanding of \textit{conflicting} physical social norms. In this example, a hiking partner is stuck in the mud; a safety-first norm (keeping one's distance) conflicts with the cooperative norm to help out. In each setting from \dataset{}, the model is given three tasks: (1) select the most appropriate action and (2) justification for that action, and (3) identify all candidate actions that socially sensible.}
    \end{adjustwidth}
    \label{fig:teaser}
\end{center}
\vspace{-0.5cm}

\begin{abstract}
Human activity is moderated by norms.

However, machines are often trained without explicit supervision on norm understanding and reasoning, especially when the norms are grounded in a physical and social context. To improve and evaluate the normative reasoning capability of vision-language models (VLMs), we present \dataset{} $\|\epsilon\|$, consisting of 1,853 ego-centric videos of human interactions, each of which has two related questions evaluating both the prediction and justification of normative actions. The normative actions encompass seven categories: 
\textcolor{white}{.} \\ 
\vspace{11.5cm} 
\\ 
\textcolor[HTML]{246D63}{safety}, 
\textcolor[HTML]{5C4C99}{privacy}, 
\textcolor[HTML]{D87CA6}{proxemics}, 
\textcolor[HTML]{356ABC}{politeness}, 
\textcolor[HTML]{C65B4E}{cooperation}, 
\textcolor[HTML]{E6A700}{coordination/proactivity}, and
\textcolor[HTML]{EA772F}{communication/legibility}. To compile this dataset at scale, %with a relatively small budget, 
we propose a novel pipeline leveraging video sampling, automatic answer generation, filtering, and human validation.
Our work demonstrates that current state-of-the-art vision-language models lack robust norm understanding, scoring a maximum of 45\% on \dataset{} (versus a human bench of 92\%).
Our analysis of performance in each dimension highlights the significant risks of safety, privacy, and the lack of collaboration and communication capability when applied to real-world agents. 
We additionally show that through a retrieval-based generation method, it is possible to use \dataset{} to enhance normative reasoning in VLMs.
\end{abstract}



% \begin{figure*}
%     \centering
%     \includegraphics[width=\linewidth]{figures/egonormia_teaser.pdf}
%     \caption{\small \dataset{} $\|\epsilon\|$ is a multiple-choice, visual question answering benchmark that evaluates models' ability to reason about appropriate behavior according to \textit{conflicting} physical social norms. In this example, a hiking partner is stuck in the mud, and the behavior that maximizes safety (keeping one's distance) conflicts with the cooperative norm to help out. In this or any similarly rich setting from \dataset{}, the model is given three tasks: (1) classify the most appropriate action (2) classify the most fitting justification for that action, and (3) identify which of the candidate actions are socially sensible.}
%     \label{fig:teaser}
% \end{figure*}

\pagebreak




\section{Introduction}
\label{section:introduction}

% redirection is unique and important in VR
Virtual Reality (VR) systems enable users to embody virtual avatars by mirroring their physical movements and aligning their perspective with virtual avatars' in real time. 
As the head-mounted displays (HMDs) block direct visual access to the physical world, users primarily rely on visual feedback from the virtual environment and integrate it with proprioceptive cues to control the avatar’s movements and interact within the VR space.
Since human perception is heavily influenced by visual input~\cite{gibson1933adaptation}, 
VR systems have the unique capability to control users' perception of the virtual environment and avatars by manipulating the visual information presented to them.
Leveraging this, various redirection techniques have been proposed to enable novel VR interactions, 
such as redirecting users' walking paths~\cite{razzaque2005redirected, suma2012impossible, steinicke2009estimation},
modifying reaching movements~\cite{gonzalez2022model, azmandian2016haptic, cheng2017sparse, feick2021visuo},
and conveying haptic information through visual feedback to create pseudo-haptic effects~\cite{samad2019pseudo, dominjon2005influence, lecuyer2009simulating}.
Such redirection techniques enable these interactions by manipulating the alignment between users' physical movements and their virtual avatar's actions.

% % what is hand/arm redirection, motivation of study arm-offset
% \change{\yj{i don't understand the purpose of this paragraph}
% These illusion-based techniques provide users with unique experiences in virtual environments that differ from the physical world yet maintain an immersive experience. 
% A key example is hand redirection, which shifts the virtual hand’s position away from the real hand as the user moves to enhance ergonomics during interaction~\cite{feuchtner2018ownershift, wentzel2020improving} and improve interaction performance~\cite{montano2017erg, poupyrev1996go}. 
% To increase the realism of virtual movements and strengthen the user’s sense of embodiment, hand redirection techniques often incorporate a complete virtual arm or full body alongside the redirected virtual hand, using inverse kinematics~\cite{hartfill2021analysis, ponton2024stretch} or adjustments to the virtual arm's movement as well~\cite{li2022modeling, feick2024impact}.
% }

% noticeability, motivation of predicting a probability, not a classification
However, these redirection techniques are most effective when the manipulation remains undetected~\cite{gonzalez2017model, li2022modeling}. 
If the redirection becomes too large, the user may not mitigate the conflict between the visual sensory input (redirected virtual movement) and their proprioception (actual physical movement), potentially leading to a loss of embodiment with the virtual avatar and making it difficult for the user to accurately control virtual movements to complete interaction tasks~\cite{li2022modeling, wentzel2020improving, feuchtner2018ownershift}. 
While proprioception is not absolute, users only have a general sense of their physical movements and the likelihood that they notice the redirection is probabilistic. 
This probability of detecting the redirection is referred to as \textbf{noticeability}~\cite{li2022modeling, zenner2024beyond, zenner2023detectability} and is typically estimated based on the frequency with which users detect the manipulation across multiple trials.

% version B
% Prior research has explored factors influencing the noticeability of redirected motion, including the redirection's magnitude~\cite{wentzel2020improving, poupyrev1996go}, direction~\cite{li2022modeling, feuchtner2018ownershift}, and the visual characteristics of the virtual avatar~\cite{ogawa2020effect, feick2024impact}.
% While these factors focus on the avatars, the surrounding virtual environment can also influence the users' behavior and in turn affect the noticeability of redirection.
% One such prominent external influence is through the visual channel - the users' visual attention is constantly distracted by complex visual effects and events in practical VR scenarios.
% Although some prior studies have explored how to leverage user blindness caused by visual distractions to redirect users' virtual hand~\cite{zenner2023detectability}, there remains a gap in understanding how to quantify the noticeability of redirection under visual distractions.

% visual stimuli and gaze behavior
Prior research has explored factors influencing the noticeability of redirected motion, including the redirection's magnitude~\cite{wentzel2020improving, poupyrev1996go}, direction~\cite{li2022modeling, feuchtner2018ownershift}, and the visual characteristics of the virtual avatar~\cite{ogawa2020effect, feick2024impact}.
While these factors focus on the avatars, the surrounding virtual environment can also influence the users' behavior and in turn affect the noticeability of redirection.
This, however, remains underexplored.
One such prominent external influence is through the visual channel - the users' visual attention is constantly distracted by complex visual effects and events in practical VR scenarios.
We thus want to investigate how \textbf{visual stimuli in the virtual environment} affect the noticeability of redirection.
With this, we hope to complement existing works that focus on avatars by incorporating environmental visual influences to enable more accurate control over the noticeability of redirected motions in practical VR scenarios.
% However, in realistic VR applications, the virtual environment often contains complex visual effects beyond the virtual avatar itself. 
% We argue that these visual effects can \textbf{distract users’ visual attention and thus affect the noticeability of redirection offsets}, while current research has yet taken into account.
% For instance, in a VR boxing scenario, a user’s visual attention is likely focused on their opponent rather than on their virtual body, leading to a lower noticeability of redirection offsets on their virtual movements. 
% Conversely, when reaching for an object in the center of their field of view, the user’s attention is more concentrated on the virtual hand’s movement and position to ensure successful interaction, resulting in a higher noticeability of offsets.

Since each visual event is a complex choreography of many underlying factors (type of visual effect, location, duration, etc.), it is extremely difficult to quantify or parameterize visual stimuli.
Furthermore, individuals respond differently to even the same visual events.
Prior neuroscience studies revealed that factors like age, gender, and personality can influence how quickly someone reacts to visual events~\cite{gillon2024responses, gale1997human}. 
Therefore, aiming to model visual stimuli in a way that is generalizable and applicable to different stimuli and users, we propose to use users' \textbf{gaze behavior} as an indicator of how they respond to visual stimuli.
In this paper, we used various gaze behaviors, including gaze location, saccades~\cite{krejtz2018eye}, fixations~\cite{perkhofer2019using}, and the Index of Pupil Activity (IPA)~\cite{duchowski2018index}.
These behaviors indicate both where users are looking and their cognitive activity, as looking at something does not necessarily mean they are attending to it.
Our goal is to investigate how these gaze behaviors stimulated by various visual stimuli relate to the noticeability of redirection.
With this, we contribute a model that allows designers and content creators to adjust the redirection in real-time responding to dynamic visual events in VR.

To achieve this, we conducted user studies to collect users' noticeability of redirection under various visual stimuli.
To simulate realistic VR scenarios, we adopted a dual-task design in which the participants performed redirected movements while monitoring the visual stimuli.
Specifically, participants' primary task was to report if they noticed an offset between the avatar's movement and their own, while their secondary task was to monitor and report the visual stimuli.
As realistic virtual environments often contain complex visual effects, we started with simple and controlled visual stimulus to manage the influencing factors.

% first user study, confirmation study
% collect data under no visual stimuli, different basic visual stimuli
We first conducted a confirmation study (N=16) to test whether applying visual stimuli (opacity-based) actually affects their noticeability of redirection. 
The results showed that participants were significantly less likely to detect the redirection when visual stimuli was presented $(F_{(1,15)}=5.90,~p=0.03)$.
Furthermore, by analyzing the collected gaze data, results revealed a correlation between the proposed gaze behaviors and the noticeability results $(r=-0.43)$, confirming that the gaze behaviors could be leveraged to compute the noticeability.

% data collection study
We then conducted a data collection study to obtain more accurate noticeability results through repeated measurements to better model the relationship between visual stimuli-triggered gaze behaviors and noticeability of redirection.
With the collected data, we analyzed various numerical features from the gaze behaviors to identify the most effective ones. 
We tested combinations of these features to determine the most effective one for predicting noticeability under visual stimuli.
Using the selected features, our regression model achieved a mean squared error (MSE) of 0.011 through leave-one-user-out cross-validation. 
Furthermore, we developed both a binary and a three-class classification model to categorize noticeability, which achieved an accuracy of 91.74\% and 85.62\%, respectively.

% evaluation study
To evaluate the generalizability of the regression model, we conducted an evaluation study (N=24) to test whether the model could accurately predict noticeability with new visual stimuli (color- and scale-based animations).
Specifically, we evaluated whether the model's predictions aligned with participants' responses under these unseen stimuli.
The results showed that our model accurately estimated the noticeability, achieving mean squared errors (MSE) of 0.014 and 0.012 for the color- and scale-based visual stimili, respectively, compared to participants' responses.
Since the tested visual stimuli data were not included in the training, the results suggested that the extracted gaze behavior features capture a generalizable pattern and can effectively indicate the corresponding impact on the noticeability of redirection.

% application
Based on our model, we implemented an adaptive redirection technique and demonstrated it through two applications: adaptive VR action game and opportunistic rendering.
We conducted a proof-of-concept user study (N=8) to compare our adaptive redirection technique with a static redirection, evaluating the usability and benefits of our adaptive redirection technique.
The results indicated that participants experienced less physical demand and stronger sense of embodiment and agency when using the adaptive redirection technique. 
These results demonstrated the effectiveness and usability of our model.

In summary, we make the following contributions.
% 
\begin{itemize}
    \item 
    We propose to use users' gaze behavior as a medium to quantify how visual stimuli influences the noticebility of redirection. 
    Through two user studies, we confirm that visual stimuli significantly influences noticeability and identify key gaze behavior features that are closely related to this impact.
    \item 
    We build a regression model that takes the user's gaze behavioral data as input, then computes the noticeability of redirection.
    Through an evaluation study, we verify that our model can estimate the noticeability with new participants under unseen visual stimuli.
    These findings suggest that the extracted gaze behavior features effectively capture the influence of visual stimuli on noticeability and can generalize across different users and visual stimuli.
    \item 
    We develop an adaptive redirection technique based on our regression model and implement two applications with it.
    With a proof-of-concept study, we demonstrate the effectiveness and potential usability of our regression model on real-world use cases.

\end{itemize}

% \delete{
% Virtual Reality (VR) allows the user to embody a virtual avatar by mirroring their physical movements through the avatar.
% As the user's visual access to the physical world is blocked in tasks involving motion control, they heavily rely on the visual representation of the avatar's motions to guide their proprioception.
% Similar to real-world experiences, the user is able to resolve conflicts between different sensory inputs (e.g., vision and motor control) through multisensory integration, which is essential for mitigating the sensory noise that commonly arises.
% However, it also enables unique manipulations in VR, as the system can intentionally modify the avatar's movements in relation to the user's motions to achieve specific functional outcomes,
% for example, 
% % the manipulations on the avatar's movements can 
% enabling novel interaction techniques of redirected walking~\cite{razzaque2005redirected}, redirected reaching~\cite{gonzalez2022model}, and pseudo haptics~\cite{samad2019pseudo}.
% With small adjustments to the avatar's movements, the user can maintain their sense of embodiment, due to their ability to resolve the perceptual differences.
% % However, a large mismatch between the user and avatar's movements can result in the user losing their sense of embodiment, due to an inability to resolve the perceptual differences.
% }

% \delete{
% However, multisensory integration can break when the manipulation is so intense that the user is aware of the existence of the motion offset and no longer maintains the sense of embodiment.
% Prior research studied the intensity threshold of the offset applied on the avatar's hand, beyond which the embodiment will break~\cite{li2022modeling}. 
% Studies also investigated the user's sensitivity to the offsets over time~\cite{kohm2022sensitivity}.
% Based on the findings, we argue that one crucial factor that affects to what extent the user notices the offset (i.e., \textit{noticeability}) that remains under-explored is whether the user directs their visual attention towards or away from the virtual avatar.
% Related work (e.g., Mise-unseen~\cite{marwecki2019mise}) has showcased applications where adjustments in the environment can be made in an unnoticeable manner when they happen in the area out of the user's visual field.
% We hypothesize that directing the user's visual attention away from the avatar's body, while still partially keeping the avatar within the user's field-of-view, can reduce the noticeability of the offset.
% Therefore, we conduct two user studies and implement a regression model to systematically investigate this effect.
% }

% \delete{
% In the first user study (N = 16), we test whether drawing the user's visual attention away from their body impacts the possibility of them noticing an offset that we apply to their arm motion in VR.
% We adopt a dual-task design to enable the alteration of the user's visual attention and a yes/no paradigm to measure the noticeability of motion offset. 
% The primary task for the user is to perform an arm motion and report when they perceive an offset between the avatar's virtual arm and their real arm.
% In the secondary task, we randomly render a visual animation of a ball turning from transparent to red and becoming transparent again and ask them to monitor and report when it appears.
% We control the strength of the visual stimuli by changing the duration and location of the animation.
% % By changing the time duration and location of the visual animation, we control the strengths of attraction to the users.
% As a result, we found significant differences in the noticeability of the offsets $(F_{(1,15)}=5.90,~p=0.03)$ between conditions with and without visual stimuli.
% Based on further analysis, we also identified the behavioral patterns of the user's gaze (including pupil dilation, fixations, and saccades) to be correlated with the noticeability results $(r=-0.43)$ and they may potentially serve as indicators of noticeability.
% }

% \delete{
% To further investigate how visual attention influences the noticeability, we conduct a data collection study (N = 12) and build a regression model based on the data.
% The regression model is able to calculate the noticeability of the offset applied on the user's arm under various visual stimuli based on their gaze behaviors.
% Our leave-one-out cross-validation results show that the proposed method was able to achieve a mean-squared error (MSE) of 0.012 in the probability regression task.
% }

% \delete{
% To verify the feasibility and extendability of the regression model, we conduct an evaluation study where we test new visual animations based on adjustments on scale and color and invite 24 new participants to attend the study.
% Results show that the proposed method can accurately estimate the noticeability with an MSE of 0.014 and 0.012 in the conditions of the color- and scale-based visual effects.
% Since these animations were not included in the dataset that the regression model was built on, the study demonstrates that the gaze behavioral features we extracted from the data capture a generalizable pattern of the user's visual attention and can indicate the corresponding impact on the noticeability of the offset.
% }

% \delete{
% Finally, we demonstrate applications that can benefit from the noticeability prediction model, including adaptive motion offsets and opportunistic rendering, considering the user's visual attention. 
% We conclude with discussions of our work's limitations and future research directions.
% }

% \delete{
% In summary, we make the following contributions.
% }
% % 
% \begin{itemize}
%     \item 
%     \delete{
%     We quantify the effects of the user's visual attention directed away by stimuli on their noticeability of an offset applied to the avatar's arm motion with respect to the user's physical arm. 
%     Through two user studies, we identified gaze behavioral features that are indicative of the changes in noticeability.
%     }
%     \item 
%     \delete{We build a regression model that takes the user's gaze behavioral data and the offset applied to the arm motion as input, then computes the probability of the user noticing the offset.
%     Through an evaluation study, we verified that the model needs no information about the source attracting the user's visual attention and can be generalizable in different scenarios.
%     }
%     \item 
%     \delete{We demonstrate two applications that potentially benefit from the regression model, including adaptive motion offsets and opportunistic rendering.
%     }

% \end{itemize}

\begin{comment}
However, users will lose the sense of embodiment to the virtual avatars if they notice the offset between the virtual and physical movements.
To address this, researchers have been exploring the noticing threshold of offsets with various magnitudes and proposing various redirection techniques that maintain the sense of embodiment~\cite{}.

However, when users embody virtual avatars to explore virtual environments, they encounter various visual effects and content that can attract their attention~\cite{}.
During this, the user may notice an offset when he observes the virtual movement carefully while ignoring it when the virtual contents attract his attention from the movements.
Therefore, static offset thresholds are not appropriate in dynamic scenarios.

Past research has proposed dynamic mapping techniques that adapted to users' state, such as hand moving speed~\cite{frees2007prism} or ergonomically comfortable poses~\cite{montano2017erg}, but not considering the influence of virtual content.
More specifically, PRISM~\cite{frees2007prism} proposed adjusting the C/D ratio with a non-linear mapping according to users' hand moving speed, but it might not be optimal for various virtual scenarios.
While Erg-O~\cite{montano2017erg} redirected users' virtual hands according to the virtual target's relative position to reduce physical fatigue, neglecting the change of virtual environments. 

Therefore, how to design redirection techniques in various scenarios with different visual attractions remains unknown.
To address this, we investigate how visual attention affects the noticing probability of movement offsets.
Based on our experiments, we implement a computational model that automatically computes the noticing probability of offsets under certain visual attractions.
VR application designers and developers can easily leverage our model to design redirection techniques maintaining the sense of embodiment adapt to the user's visual attention.
We implement a dynamic redirection technique with our model and demonstrate that it effectively reduces the target reaching time without reducing the sense of embodiment compared to static redirection techniques.

% Need to be refined
This paper offers the following contributions.
\begin{itemize}
    \item We investigate how visual attractions affect the noticing probability of redirection offsets.
    \item We construct a computational model to predict the noticing probability of an offset with a given visual background.
    \item We implement a dynamic redirection technique adapting to the visual background. We evaluate the technique and develop three applications to demonstrate the benefits. 
\end{itemize}



First, we conducted a controlled experiment to understand how users perceived the movement offset while subjected to various distractions.
Since hand redirection is one of the most frequently used redirections in VR interactions, we focused on the dynamic arm movements and manually added angular offsets to the' elbow joint~\cite{li2022modeling, gonzalez2022model, zenner2019estimating}. 
We employed flashing spheres in the user's field of view as distractions to attract users' visual attention.
Participants were instructed to report the appearing location of the spheres while simultaneously performing the arm movements and reporting if they perceived an offset during the movement. 
(\zhipeng{Add the results of data collection. Analyze the influence of the distance between the gaze map and the offset.}
We measured the visual attraction's magnitude with the gaze distribution on it.
Results showed that stronger distractions made it harder for users to notice the offset.)
\zhipeng{Need to rewrite. Not sure to use gaze distribution or a metric obtained from the visual content.}
Secondly, we constructed a computational model to predict the noticing probability of offsets with given visual content.
We analyzed the data from the user studies to measure the influence of visual attractions on the noticing probability of offsets.
We built a statistical model to predict the offset's noticing probability with a given visual content.
Based on the model, we implement a dynamic redirection technique to adjust the redirection offset adapted to the user's current field of view.
We evaluated the technique in a target selection task compared to no hand redirection and static hand redirection.
\zhipeng{Add the results of the evaluation.}
Results showed that the dynamic hand redirection technique significantly reduced the target selection time with similar accuracy and a comparable sense of embodiment.
Finally, we implemented three applications to demonstrate the potential benefits of the visual attention adapted dynamic redirection technique.
\end{comment}

% This one modifies arm length, not redirection
% \citeauthor{mcintosh2020iteratively} proposed an adaptation method to iteratively change the virtual avatar arm's length based on the primary tasks' performance~\cite{mcintosh2020iteratively}.



% \zhipeng{TO ADD: what is redirection}
% Redirection enables novel interactions in Virtual Reality, including redirected walking, haptic redirection, and pseudo haptics by introducing an offset to users' movement.
% \zhipeng{TO ADD: extend this sentence}
% The price of this is that users' immersiveness and embodiment in VR can be compromised when they notice the offset and perceive the virtual movement not as theirs~\cite{}.
% \zhipeng{TO ADD: extend this sentence, elaborate how the virtual environment attracts users' attention}
% Meanwhile, the visual content in the virtual environment is abundant and consistently captures users' attention, making it harder to notice the offset~\cite{}.
% While previous studies explored the noticing threshold of the offsets and optimized the redirection techniques to maintain the sense of embodiment~\cite{}, the influence of visual content on the probability of perceiving offsets remains unknown.  
% Therefore, we propose to investigate how users perceive the redirection offset when they are facing various visual attractions.


% We conducted a user study to understand how users notice the shift with visual attractions.
% We used a color-changing ball to attract the user's attention while instructing users to perform different poses with their arms and observe it meanwhile.
% \zhipeng{(Which one should be the primary task? Observe the ball should be the primary one, but if the primary task is too simple, users might allocate more attention on the secondary task and this makes the secondary task primary.)}
% \zhipeng{(We need a good and reasonable dual-task design in which users care about both their pose and the visual content, at least in the evaluation study. And we need to be able to control the visual content's magnitude and saliency maybe?)}
% We controlled the shift magnitude and direction, the user's pose, the ball's size, and the color range.
% We set the ball's color-changing interval as the independent factor.
% We collect the user's response to each shift and the color-changing times.
% Based on the collected data, we constructed a statistical model to describe the influence of visual attraction on the noticing probability.
% \zhipeng{(Are we actually controlling the attention allocation? How do we measure the attracting effect? We need uniform metrics, otherwise it is also hard for others to use our knowledge.)}
% \zhipeng{(Try to use eye gaze? The eye gaze distribution in the last five seconds to decide the attention allocation? Basically constructing a model with eye gaze distribution and noticing probability. But the user's head is moving, so the eye gaze distribution is not aligned well with the current view.)}

% \zhipeng{Saliency and EMD}
% \zhipeng{Gaze is more than just a point: Rethinking visual attention
% analysis using peripheral vision-based gaze mapping}

% Evaluation study(ideal case): based on the visual content, adjusting the redirection magnitude dynamically.

% \zhipeng{(The risk is our model's effect is trivial.)}

% Applications:
% Playing Lego while watching demo videos, we can accelerate the reaching process of bricks, and forbid the redirection during the manipulation.

% Beat saber again: but not make a lot of sense? Difficult game has complicated visual effects, while allows larger shift, but do not need large shift with high difficulty





\section{Physical Social Norms (PSN)}
\label{sec:PSN}

Social norms are commonly-held expectations about behavior ~\cite{gibbs1965norms} that emerge and evolve spontaneously ~\cite{hechter2001social, chung2016social}. Norms serve a critical role in the coordination of multi-agent systems, and as the solutions to social dilemmas \citep{van2013psychology} like collective action problems \citep{ostrom2000collective}. They enable agents to share similar expectations, become more predictable \citep{morsky2019evolution} and less prone to friction \citep{hollander2011current,mukherjee2007emergence}. 

AI agents need to understand and consistently follow norms, both to navigate social situations \citep{mavrogiannis2023core}, and effectively collaborate with humans. This is particularly true of \textit{embodied} agents \citep{liembodied} like robots \citep{francis2023principles}, which share a physical environment with humans. In this case, the problem of normative reasoning is closely connected with physical reasoning; thus, we define the following:
\begin{quote}
    \textbf{Physical social norms} (PSNs) are shared expectations that govern how actors behave and interact with others in shared environments.
    %, encompassing social expectations, spatial considerations, laws and regulations, common sense, and safety concerns. % , encompassing social expectations, spatial considerations, laws and regulations, common sense, and safety concerns.
\end{quote}

%Suppose your friend is stuck in mud---how should one react?\footnote{See Figure \ref{fig:teaser} for an illustrative example.} Should you laugh and jump in to help, or stand on dry land and throw them a rope? What about if they were a stranger, a child, or injured? What if you're running a competitive race, or are in a salon getting a mud bath? In each of these contexts, situational \textbf{norms} ~\citep{chung2016social, sunstein1996social, cialdini1991focus}, in conjunction with action constraints, inform \textit{how} a given actor should behave.\footnote{Actor: Human or embodied agent} Norms are defined within sociology as a commonly-held expectation of behavior ~\cite{gibbs1965norms} that emerge and evolve spontaneously via human consensus ~\cite{hechter2001social, chung2016social}. 
% \noindent 
%We broaden this definition by introducing the concept of \textbf{physical social norms}, a term we formally define to target norms related to embodied systems.
%Specifically, it refers to the rules and expectations that govern how individuals behave in shared physical spaces, particularly with all sentient beings, including animals, not just human beings. 
%For instance, standing an appropriate distance from others in a queue or avoiding overly fast handshakes. These norms are context-dependent and can vary based on the situation. Understanding physical social norms is essential for designing systems and environments that accommodate socially acceptable behaviors in embodied agents.
% It is important to note that social norms apply to interactions with all sentient beings, including animals, not just human beings.
% \begin{quote}
%     \textbf{Physical social norms} (PSNs) are shared expectations that govern how actors behave and interact with others in shared environments.
%     %, encompassing social expectations, spatial considerations, laws and regulations, common sense, and safety concerns. % , encompassing social expectations, spatial considerations, laws and regulations, common sense, and safety concerns.
% \end{quote}


% %\begin{comment}
% \begin{tcolorbox}[title=\textbf{Definition: physical-social norms},colback=magenta!10, colframe=magenta!70!black]

% \textbf{Physical-social norms} (PSN) are consensus-agreed rules that govern how individuals behave and interact with others in shared environments, encompassing social expectations, spatial considerations, laws and regulations, common sense, and safety concerns.
% % \newline
% % Physical-Social Norms apply to:
% % \begin{itemize}
% %     \item Social/Individual (Unseen) Settings
% %     \item Purely Physical/Purely Verbal Interactions
% % \end{itemize}

% \end{tcolorbox}
% %\end{comment}


To study \textit{physical social norms}, we operationalize a taxonomy of PSN categories, which stand for the social objectives that inform them. Figure~\ref{fig:taxonomy} demonstrates examples of each. The last three categories explicitly serve the function of maximizing utility across multi-agent systems. We call these the \textit{Utility Norms}: \textcolor[HTML]{C65B4E}{cooperation}, \textcolor[HTML]{E6A700}{coordination}, and \textcolor[HTML]{EA772F}{communication} norms. The first four categories are more particular to \textit{human} sociality: \textcolor[HTML]{246D63}{safety}, \textcolor[HTML]{356ABC}{politeness}, \textcolor[HTML]{5C4C99}{privacy}, and \textcolor[HTML]{D87CA6}{proxemics}. These norms can often stand at odds with group utility norms, and this tension provides a setting for evaluating agent decision-making under conflicting objectives. Importantly, each category can still directly inform the success of human-agent collaboration as follows.

% Concepts to elaborate:
% \begin{enumerate}
%     \item\textbf{Safety:} \citep{bera2017sociosense}
%     \item \textbf{Proxemics:} Proxemics is highly correlated with humans' perceived safety around other agents \citep{huang2022proxemics}, particularly with robots \citep{neggers2022determining}.
%     \item \textbf{Legibility:} The optimal behavior for a single agent may be the most direct or efficient course of action, but in multi-agent settings, coordination requires each agent to behave in a manner that clearly communicates that agent's goals and current state \citep{dragan2013legibility,wallkotter2021explainable}. 
%     \item \textbf{Cooperation} norms serve to make collaborative tasks more fluid and seamless, both objectively and subjectively. Cooperation norms should maximize the useful concurrent activity of teammates and minimize agents' idle time, like reducing the exchange gap time in turn-taking settings \citep{hoffman2019evaluating}.
%     \item \textbf{Proactivity:} While deference may be generally advisable for dyadic interactions, and humans may prefer this from robots in more passive roles \citep{kanda2002development,rubagotti2022perceived}, passivity may quickly induce deadlock in multi-agent systems \citep{francis2023principles,lyu2022responsibility}. Embodied agents should anticipate goal conflicts and move to resolve these conflicts proactively \citep{tan2020relationship}.
% \end{enumerate}

%\noindent We categorize PSNs in seven categories based on the role they serve in interactions, consisting of four \textit{non-utility} categories and three \textit{utility} categories. Utility \citep{brauer2010descriptive, zhao2024large} encompasses prescriptive guidelines specific to interactions (i.e., what or how to do X). In contrast, non-utility \citep{janoff2009proscriptive} encompasses ethical principles, constraints, and hard rules (i.e., what \textit{not} to do). %  \cite{zhou2024sotopia}.

\begin{figure*}[!h]
\centering
\includegraphics[width=0.85\linewidth]{figures/pipeline-v3-2.pdf}
\caption{We propose an efficient pipeline for annotating normative behaviors through leveraging Ego4D annotations (Phase I), VLM-based proposal (Phase II), post-hoc filtering (Phase III), and human validation (Phase IV). }
\label{fig:pipeline}
\end{figure*}

\begin{figure}[!h]
\centering
\includegraphics[width=0.7\linewidth]{figures/diversity-1.pdf}
\caption{Through automatic clustering with GPT-4o, we categorize the final videos into 5 high-level and 23 low-level categories.}
\label{fig:diversity}
\end{figure}

\noindent\textbf{\textcolor[HTML]{246D63}{Safety}}, a principal concern for human-robot interaction \citep{lasota2017survey}, describes not only the prevention of physical harms to humans and the environment, but also the mitigation of psychological harms like stress. A safe social robot not only pauses its use of a dangerous cutting tool when humans touch it; the robot should also refrain from using the tool in the presence of humans at all.

%encompasses actions preventing damage to humans and the environment, such as wearing protective equipment when using tools ~\cite{Meesmann2015ImpactOA}. Note that this differs from LLM safety, like generating harmful content \citep{alex2023jailbroken} and causing financial loss \citep{yuan2024rjudge}.

\noindent\textbf{\textcolor[HTML]{5C4C99}{Privacy}} involves respecting the personal possessions and private information of others. This is particularly relevant to agents operating in privacy-constrained environments and includes avoiding uncomfortable and prying questions and not intruding on private spaces \citep{altman1975environment, lutz2020robot, shao2024privacylensevaluatingprivacynorm}. 
%This is distinct from privacy issues related to LLMs, such as private data leakage \citep{liao2024eia, shao2024privacylensevaluatingprivacynorm}.

% \noindent\textbf{\textcolor[HTML]{5C4C99}{Privacy}} in PSN involves respecting the personal space, possessions, and autonomy of others. It includes actions like avoiding uncomfortable and prying questions, or not intruding on private spaces~\cite{altman1975environment}.

\noindent\textbf{\textcolor[HTML]{D87CA6}{Proxemics}} proxemics is highly correlated with humans' perceived safety around other agents \citep{huang2022proxemics}, particularly with robots \citep{neggers2022determining}, and denotes acceptable boundaries for personal space depending on cultural and situational expectations~\cite{russell1982environmental}. 
%It is distinct from \textcolor[HTML]{5C4C99}{Privacy}, as it relates primarily to comfort and notions of personal space~\cite{hayduk1983personal}.

% \noindent\textbf{\textcolor[HTML]{D87CA6}{Proxemics}} concerns the use of personal space and physical distance between individuals. It involves understanding acceptable boundaries depending on cultural and situational expectations~\cite{russell1982environmental}.

\noindent\textbf{\textcolor[HTML]{356ABC}{Politeness}} relates to socially acceptable behaviors that shows respect. In physical contexts, this can involve gestures and body language that show consideration for others, or communication appropriate for one's social role ~\cite{mills2011politeness}.

% \noindent\textbf{\textcolor[HTML]{356ABC}{Politeness}} relates to socially acceptable and courteous behaviors that reflect respect for others. In physical contexts, it may involve gestures, body language, and spatial conduct that show consideration~\cite{mills2011politeness}.

\noindent\textbf{\textcolor[HTML]{C65B4E}{Cooperation}} focuses on working collaboratively with others. It entails actions that facilitate mutual benefit and shared goals, like lifting a heavy box with another person ~\cite{sunstein1996social}.

\noindent\textbf{\textcolor[HTML]{E6A700}{Coordination/Proactivity}} involves anticipating and aligning actions with others to achieve successful interactions. Proactive behavior includes adjusting movements or actions in advance to prevent disruption~\cite{paternotte2013social}. 

\noindent\textbf{\textcolor[HTML]{EA772F}{Communication/Legibility}} refers to the ability to clearly signal intentions and make one's physical behavior understandable to others, by using gestures, speech, or movement patterns to reduce ambiguity in social interactions~\cite{francis2023principlesguidelinesevaluatingsocial}.

% \noindent\textbf{\textcolor[HTML]{EA772F}{Communication/Legibility}} refers to the ability to clearly and effectively signal intentions and make one's physical behavior understandable to others, such as using gestures, postures, or movement patterns to ensure transparency and reduce ambiguity in social interactions~\cite{francis2023principlesguidelinesevaluatingsocial}.

Figure~\ref{fig:taxonomy} illustrates how physical social norms reference physical properties and social dynamics across each taxonomy category.
% Due to the physical contexts, our norms are different from concepts like language model safety \citep{alex2023jailbroken} and privacy \citep{he2024emerged, liao2024eia, shao2024privacylensevaluatingprivacynorm}.
By design, actions will satisfy some dimensions and may contravene others---core to the complexity of human normative reasoning. The primary motivation for introducing the taxonomy categories is the resolution of relative norm importance when they conflict. 
%Some dimensions may overlap; for instance, proxemics and privacy can involve respecting personal space, suggesting that a single behavior may simultaneously imply adherence to multiple norms.

% \newpage
\section{\dataset{}}
\label{sec:task_overview}
\dataset{} is designed to achieve several goals: (1) \emph{diversity} across contexts and normative categories through uniqueness filters, (2) \emph{simplicity of use} through a multiple-choice question format with clear metrics, (3) \emph{high human consensus} via extensive manual validation requiring annotator agreement, and (4) \emph{high difficulty} and \emph{benchmark longevity} by designing tasks challenging to solve through superficial visual reasoning. 


% Normative reasoning requires parsing and understanding the context of a scene, identifying relevant norms, and selecting or moderating action to satisfy those norms. This is complicated by the breadth of potentially relevant contexts, the incompleteness of information available in the scene, the high defeasibility of norms,\footnote{Highly sensitive to small changes in context. If one is talking to another person, whether that person is a friend or a stranger completely changes the norms of the situation.} and the implicit, variable priority of norms. 
% Normative reasoning requires parsing and understanding the context of a scene, identifying norms that are relevant, and moderating behavior to satisfy those norms.
% This is complicated by the breadth of potentially relevant context like social relationships, goals of other actors, and scene history.  The incompleteness of information available in the scene like unknown goals, hidden objects, and ambiguous actions, the high defeasibility of norms (small perturbations in context can lead to large perturbations in correct behavior), and the implicit, variable priority of norms. % Include that norm-context dependence is long-tailed??? % Give concrete example to illustrate such as in https://arxiv.org/pdf/2209.06293?
% Explanation of the types of reasoning we test (The overall goal here is to describe what we do, little on why we do it)
% To comprehensively test normative reasoning ability, we design a task suite to encompass the selection of normative action and supporting justification.

% -- this is a proven method in 
\subsection{\dataset{} Task Definition}
We use a format of Multiple-Choice Questions~(MCQs) for all subtasks.  Example MCQs are shown in Figure~\ref{fig:example}. Detailed prompts for each subtask can be found in Appendix~\ref{appendix:prompts_evaluation}.
% The metrics used to compute the success on each task are located in Appendix~\ref{appendix:metrics}.

\label{sec:task_definition}
% Behavior
% A model is first provided with a video and a general description of the activity, and then asked to choose the most normatively appropriate next action to perform.\footnote{In the context of our benchmark, we use ``normative behavior'' and ``normative action'' interchangeably.}

\paragraph{Subtask 1: Action Selection.}  In this subtask, the model is provided with video frames of an activity and five candidate actions. Given these inputs, the model is asked to select the single most normatively appropriate action to perform in the context.\footnote{In the context of our benchmark, we use ``normative behavior'' and ``normative action'' interchangeably.} We enforce strict plausibility constraints on possible answers to ensure that the correct action is not trivially identifiable by visually parsing objects in-scene or eliminating obviously non-normative options. Figure~\ref{fig:teaser} shows several example action options, each illustrating a valid next step for the ego in the context of the video. To arrive at the correct choice C, proceeding to the dry ground, the model must consider multiple dimensions of normative behavior like \textcolor[HTML]{246D63}{safety}, \textcolor[HTML]{356ABC}{politeness}, and \textcolor[HTML]{C65B4E}{cooperation}. This subtask tests whether vision-language models can successfully make normative decisions in specific physical contexts. 
% Explain how we enforced this later


% Justification (do we need to explain why we test justification? neither VCR nor NYT captioning paper do)
\paragraph{Subtask 2: Justification Selection.} In this subtask, the model is provided with the same visual input as in Subtask 1 and is asked to select the best justification supporting its chosen normative action. For example, as shown in Figure~\ref{fig:teaser}, the model must select the appropriate justification for choosing action C (\emph{proceeding to the dry ground first}) instead of directly offering help or simply moving away. This subtask enables the benchmark to qualify whether the model can identify the relevant context and articulate the correct underlying reasoning for its normative decision, serving as a finer measure of normative reasoning. % , elements critical to normative reasoning.

% Assuming models need to provide natural-language justification for norm-guided actions in real-world deployment, we evaluate whether the model identifies the correct context and hypothetical reasoning in generating the normative action. \hao{Similarly here, could you point to the example in Figure 1}

\paragraph{Subtask 3: Sensibility.} To measure whether models understand the features that make action normative in context, we evaluate whether they can select the sensible (i.e. normative, but not necessarily best) options from the given actions. 
%Since this task involves matching of two lists, success is measures through an intersection over union (IoU) metric.


% \begin{figure}
%     \centering
%     \includegraphics[width=0.8\linewidth]{figures/activity-diversity.pdf}
%     \caption{Through automatic clustering with GPT-4o, we categorize the videos into 5 high-level and 23 low-level categories.}
%     \label{fig:activity_cluster}
% \end{figure}


% Action followed
% We further test the ability to identify normative reasoning with action in-scene -- given the
% the ground truth normative behavior, the model must identify whether this behavior is followed in the scene.
% In our evaluation, we measure accuracy by the proportion of tasks for which the model correctly identifies the normative behavior and justification.
% Explain why we don't divide tasks by type of reasoning required
%While it is tempting to cluster tasks by type of inference and information required, real-life normative reasoning requires simultaneous reasoning across types of context; this is reflected in our evaluation tasks. Thus, clustering tasks by type of reasoning required would not be valuable in evaluating the model's ability to perform normative reasoning.

\begin{figure*}[t]
    \centering
    \includegraphics[width=\linewidth]{img/example.pdf}
%     \caption{Example MCQs from \dataset{}. The correct answers are underlined. Three examples illustrate how incorrect physical reasoning can lead to the selection of inappropriate normative actions and justifications.
% In Video 1, the ego is at a scenic spot holding a phone. The normative action in this context would naturally be taking a picture, which Gemini correctly identifies. However, o3 incorrectly concludes that the ego is moving frequently when he is just standing still. As a result, it selects "holding the railing" as the correct action—despite no railing being present in the video.
% In Video 2, the ego is coaching another individual on how to perform a leg exercise by adjusting her position. Gemini misinterprets this as a "leg press exercise", leading to the incorrect conclusion that the appropriate action is to provide support for the "lift". Meanwhile, o3 prioritizes verbal communication, which is a reasonable choice but should not take precedence over actual physical guidance.
% The final video depicts a woman attempting to lift a sofa together with the ego. However, o3 misclassifies this scenario as entertainment rather than labor, resulting in an incorrect selection of both action and justification.
% \yanzhe{need to shorten this.}}
    \caption{Example MCQs with choices by o3-mini (with text descriptions) and Gemini 1.5 Pro (with videos). Correct answers are underlined.
In Video 1, o3-mini incorrectly concludes that the ego is "moving frequently" and wrongly selects "holding the railing" despite no railing being present.
In Video 2, Gemini misinterprets the scene as a "leg press exercise" and incorrectly opts to support a "lift".
In Video 3, o3-mini mistakenly categorizes this scenario as entertainment instead of housework, overlooking the fact that the women need assistance.}

    \label{fig:example}
\end{figure*}

\subsection{Benchmark Generation Pipeline}
\label{sec:benchmark_generation_pipeline}

The benchmark generation pipeline is described in Figure~\ref{fig:pipeline}. 
% A more detailed overview of the pipeline and methodology can be found in Appendix \ref{appendix:BGPD}.
Appendix~\ref{appendix:BGPD} contains a more detailed overview of the pipeline and methodology.
The pipeline consists of the the following steps:

\noindent\textbf{Phase I: Snippet Sampling.} 
% \dataset{} sources its videos from the Ego4D dataset ~\cite{grauman2022ego4dworld3000hours}, consisting of 3650 hours of richly annotated egocentric footage of commonplace human activities in context.
We sourced video samples from Ego4D~\cite{grauman2022ego4dworld3000hours} as it matches the egocentric embodiment of human normative reasoning. % \footnote{In other words, places one into the norm-resolution situation as a first-person actor.} 
To ensure diversity, we applied a multi-step filtering process, sampling each unique scenario-verb combination to select video snippets across a wide range of social and physical contexts.

%First, we removed videos featuring only a single actor, as these typically lack the complex social interactions central to our study. Next, we analyzed the narrations to extract verb-scenario combinations, treating each unique combination as a distinct interaction category. %By sampling up to three instances per unique combination, our uniqueness filter minimizes redundancy while representing a wide range of natural social and physical contexts. Finally, we excluded game-related scenarios to further emphasize interactions that reflect everyday human experiences. 
% \hao{no need to emphasize the stats of Ego4d, please describe how uniqueness filter works and how this addresses the first goal: diversity}
%(2) It includes over 3.85 million \textbf{action-centric visual narrations}, facilitating the identification of unique actions.
%(3) Its \textbf{diverse} range of situations and actions enables EgoNormia to comprehensively explore the space of physical-social norms.
% (1) the egocentric perspective matches the embodiment of humans and the typical embodied systems this benchmark is intended to support;
% (2) it features over 3.85 million action-centric visual narrations, which aid in targeting unique behaviors; and 
% (3) it is situation- and action- diverse, enabling EgoNormia to span the space of physical-social norms.

% We sampled all narrations mentioning two or more actors, then PoS-tagged the target narrations and clustered by verb category and scenario. From this set, a maximum of three samples were taken from each verb-scenario combination, in order to select a maximally long-tailed set of samples. Any scenarios involving card or board games were excluded, as these present monotonic situations where action alternatives relate to game rules instead of human social or physical norms.

% \noindent Diversity in samples was enforced through leveraging Ego4D's native action narrations and scene descriptions.

%by selecting unique combinations of verbs and situations, yielded 4446 unique samples, sourced from from unique 1870 videos.

%We created a diverse dataset by selecting narrations that involved multiple actors, analyzing the verbs and scenarios present, and sampling up to three instances from each unique combination while excluding game-related scenarios to focus on natural social and physical interactions. This curation 


\noindent\textbf{Phase II: Answer Generation.}
% For each video sample, four actions and justifications (one gold-standard pair and three distractor pairs) are generated using a structured, multi-shot pipeline with GPT-4o-based Chain-of-Thought prompting~\citep{wei2022chain}. 
% See Appendix~\ref{appendix:prompts} for detailed prompts.
For each video sample, we generate four pairs of actions and justifications---one ground truth pair and three distractor pairs.\footnote{None is added after generation to create five total options.} To create challenging distractors, we systematically perturb the original context by altering key details that influence the interpretation of the action, leading to plausible alternatives that require normative knowledge to disambiguate. Detailed prompts for answer generation can be found in Appendix~\ref{appendix:prompts_mcq}.

%For example, as illustrated in Figure~\ref{fig:teaser}, although all actions originate from the context “someone is stuck in mud,” different perturbations yield distinct interpretations. Option A is optimal when the context is a casual mud party without safety concerns. Option B fits a scenario where the context is perceived as a single-person mud race competition. Option C is the correct choice, corresponding to the actual scene where your hiking partner is stuck in mud with dry ground nearby. Option D applies when the context is that the stuck person is an experienced wild exploration leader who has explicitly signaled you to back off. 

%These options challenge model to correctly understand context in the video and perform correct normative reasoning in corresponding scene. By carefully perturbing contextual details, these distractors require models to accurately interpret the nuances in video context and perform appropriate normative reasoning. 
% \hao{No need to emphasize GPT or CoT. Emphasize how you perturb the context and why that leads to challenging distractors. One example, you can still use the hiking one.}

% For each video sample, four actions and justifications (one gold-standard pair and three distractor pairs) are generated. To ensure challenging distractors, we generate these distractors based on modified context. For example, the context in Figure ~\ref{fig:teaser}, though all actions are generated based on the context "someone is stuck in mud", option A could be optimal when you are just playing with your friend in a mud party where no satety concern exists, option B is the best when you are watching your friend finishing a single person mud race competition, option D would be best if the person is an experienced your wild exploration leader and has explictly signaled you to back off to give them the space to free themselves.
% See Appendix~\ref{appendix:prompts} for detailed prompts. \hao{No need to emphasize GPT or CoT. Emphasize how you perturb the context and why that leads to challenging distractors. One example, you can still use the hiking one.}

% Frames of sampled snippets of \textbf{Phase \Roman{num}} are first processed with a VLM to extract a scene context description $c$, consisting of the activity, the identities of the people involved, and the environment.
% The context $c$ are then corrupted via LLM to programmatically modify the core context, to change the norms that are relevant in the context. Here, we leverage the defeasibility and compositionality of norms explored by NormBank ~\cite{ziems-etal-2023-normbank} to add, remove, or modify elements of the context, yielding three additional contexts, which form the context set.
% % Further details on the corruption methodology are included in Appendix XXX. (Rejection of police/criminal/spy contexts)
% Then an LLM generates a noisy set of actions $A^+$ and their justifications $J^+$ for each context $c$ in the context set, where the LLM is directed to generate the best action to perform in that given context, a justification for why that norm is most important, and also the categories to which each action belongs to. These are generated in a multi-turn way, where each inference uses the result of the previous stage as part of its input.
% % Experimentation with single shot direct-from-video generation yielded generated \texttt{A}s that were similar to each other; this was rejected as the generation approach.

\noindent\textbf{Phase III: Filtering.}
The output of the second stage consists of high-quality but potentially noisy tasks; answers might be trivially resolvable, ambiguous, or nonsensical. Thus we perform \textbf{normativity filtering} by using chained LLMs to filter for answer feasibility and sensibility, then run \textbf{blind filtering} (i.e. no vision input) to remove questions answerable without context or through superficial reasoning, as these do not test \textit{embodied} normative reasoning, leaving only challenging, context-dependent questions.

%This filtering process eliminates tasks that might be easily solved through superficial cues in options, ensuring that only challenging, context-dependent questions are retained. % As detailed in Table~\ref{tab:results}, the blind filtering mechanism successfully reduces the performance of even the best blind models to below chance-level accuracy, indicating that blind filtering effectively removes NLP-resolvable tasks from our dataset.

% \hao{Add explanation why this make the task challenging. Also point to results to show that blind filters really work for all models (<chance level accuracy for all blind models).}
% Further issues include failed or malformed generations, or the desired output structure not being matched.
% Thus, we refine $A^+$ and $J^+$ with several filtering rounds to ensure the correctness, context-dependence, and high difficulty of questions, to yield a filtered $A$ and $J$ for each example: (i) \textbf{Normativity filtering}: We remove certain action descriptions can describe an action that's not feasibility or is harmful in any situation.
% % - for instance, the action "Grab the lady's purse and run" is illegal through text parsing alone; this class of \texttt{AJs} trivialize the downstream task. Thus, each answer is individually inspected for safety and feasibility, any failing answer is regenerated and re-tested until the full set passes.
% (ii) \textbf{Blind filtering.} To enforce EgoNormia tasks requring grounded visual reasoning to solve, a "blind" baseline is compiled: Any task whose gold standard answer is obviously correct without context, either due to nonsensical answers or leaky domain knowledge, is filtered out as they do not test visual normative reasoning. 
% Due to the low random success rate (4.0\%), blind filtering did not substantially risk removing data points that were well-formed but the blind model was able to guess successfully.

\noindent \textbf{Phase IV: Human Validation.}
Finally, two human validators are employed to verify the correct behavior and justification (manually adding them if not present or ambiguous), and to select the list of actions that are considered sensible. Two validators are used to ensure every datapoint receives independent agreement from two humans, ensuring that human agreement on \dataset{} is replicable.
The authors manually process datapoints where validators disagree on answers, ensuring that the benchmark remains challenging and achieves high human agreement. The detailed procedures for onboarding and training the human annotators, as well as the instructions for the curation process are provided in Appendix~\ref{sec:HumanValidationProcess}. 
% \hao{Say this addresses the 3rd desiderata -- human consensus}

%Annotators are responsible for three key tasks: for each example, verifying that the best action and justification are present in $A$ and $J$ without overlapping in meaning with any other alternatives; selecting other given actions and justifications that are appropriate in the given situation but do not represent the most normative choice; and confirming whether the best action $a$ is followed in the video afterwards. 
% Annotators are tasked with three primary responsibilities: (A) Verify that the best action \texttt{A} and justification \texttt{J} are present in \texttt{AJT+}, and do not overlap in meaning with any other \texttt{AJ} (B) Select the list of other given actions and justifications that are sensible in the given situation, but not the best action - i.e. actions that are expected in that context, but are not the most normative. 
% (C) Confirm whether the best action \texttt{A} is followed in-scene.
% (ii) Two annotators must agree on the best action $a$ for a given $A$ and $J$ to be accepted; they are allowed to provide their own preferred $a$ and $j$ if no answer is correct. In cases of new annoated actions, $A$ and $J$ are manually reconciled by the authors and either modified or rejected outright. This reduces the number of admissible samples by 50\%. 
% (iii) Finally, a second expert curation round is performed, to manually validate the difficulty and diversity of each sample. Only ~85\% of the examples that pass the first round also pass the second round, demonstrating the relative difficulty of generating nontrivial grounded norm-resolution situations.

\subsection{EgoNormia Statistics}
\label{sec:stats}
% Table~\ref{tab:dataset_statistics} presents the summary statistics for \dataset{}. The dataset comprises 1,856 data points sourced from 1,077 videos, averaging approximately 1.7 samples per video. To ensure high quality, we filtered out 58.3% of the original samples from Ego4D. Despite this aggressive filtering, the number of unique scenarios and actions per data point remains relatively stable, indicating that we successfully preserved the dataset's diversity.

The final \dataset{} split comprises a total of 1853 data points sourced from 1077 videos, averaging approximately 1.7 samples per video. 58.3\% of the initially sampled data points from Ego4D were filtered in prior processing steps. % Despite this aggressive filtering, the number of unique scenarios and actions per data point remains relatively stable, indicating that we successfully preserved the dataset's diversity. 
% \noindent 
Appendix~\ref{appendix:statistics} provides additional statistics for \dataset{}. Figure~\ref{fig:diversity} illustrates the distribution of activities in our dataset. We employ an automatic clustering method—--detailed in Appendix~\ref{appendix:clustering}—--that leverages GPT-4o to group the videos into 5 broad categories and 23 finer-grained subcategories.

% norms, for an average of 2.63 constraints per norm.
% The SCENE taxonomy broadly captures the kinds
% of constraints annotators were looking for 94% of

% EgoNormia consists of 1856 samples from Ego4D dataset that span the domain of physical-social norms, covering 108 commonplace scenarios.
% **Talk about taxonomy categories
\section{Evaluation}
\label{sec:eval}
\newcommand{\highlight}[1]{{\leavevmode\textbf{#1}}}
\subsection{Experimental Setup}
\label{sec:eval/setup}
\begin{table}[tbp]
    \centering
    \caption{Evaluated models.}
    \begin{tabular}{l|c|c}
        \hline
         Model & \makecell{Num. of\\ CA Layers} & \makecell{Num. of\\ LM Blocks}\\ \hline
         mPLUG-Owl3-7b & 4 & 28 \\
         mPLUG-Owl3-2b & 4 & 28 \\
         mPLUG-Owl3-1b & 4 & 24 \\
         OpenFlamingo-9b & 8 & 32 \\
         OpenFlamingo-3b & 24 & 24 \\ \hline         
    \end{tabular}
    \label{tab:evaluated-models}
    \vspace{-2ex}
\end{table}
\textbf{Model Setup} We evaluate our methods and baselines on 5 models shown in Table~\ref{tab:evaluated-models}. Following their default configuration, each frame is encoded into 729 visual tokens for the mPLUG-Owl3 models and 64 visual tokens for the OpenFlamingo models. This implies that given the same amount of memory capacity, we can fit more frames to OpenFlamingo models than mPLUG-Owl3 models. 

For all models, a special token \texttt{<image>} must be included in the text prompt for each frame. Consequently, the length of the text prompt must be at least equal to the number of frames in the visual input. 

We use a batch size of 1 and fully sharded tensor parallelism for all models to enable a larger context length.

\textbf{Cluster Setup} We evaluate our method and baselines on the following configurations: (1) A 16-GPU cluster, each node equipped with 4 A100 80GB GPUs, with the GPUs within a node interconnected via NVLink and a cross-node bandwidth of 25 GB/s, representing a typical setting for cross-node training of up to millions of tokens. (2) An 8-GPU cluster, each node equipped with 1 A30 24GB GPU, with a cross-node bandwidth of 1.25 GB/s, representing a more resource-constrained setup with slower interconnect bandwidth. (3) A 12-GPU cluster, each node equipped with 3 A100 40GB GPUs, with the GPUs interconnected via 64 GB/s PCIe and a cross-node bandwidth of 25 GB/s, used for smaller-scale case studies and ablation studies.

\textbf{Baselines} For our method, we use LV-XAttn for the cross-attention layers and Ring Attention for the LM blocks. Our primary baseline is the setup where Ring Attention is used for both the cross-attention layers and LM blocks. We apply our activation recomputation technique to both of these settings for enabling longer context length. We also compare against Deepspeed-Ulysses~\cite{jacobs2024ds}, which employs sequence parallelism for non-attention layers and head parallelism for attention layers. All methods uses Flash Attention.

\subsection{Comparison with Ring Attention}
\begin{table*}[ht]\centering
\caption{Per iteration wall-clock time (in seconds) on 16 A100 80GB GPUs with Ring Attention and LV-XAttn. ``CA" represents the time spent on cross-attention operations. As $S_Q$ doubles, the cross-attention speedup nearly halves because the runtime for Ring Attention, which is communication-bound, remains constant, while the runtime for LV-XAttention, which is computation-bound, doubles. On the other hand, as $S_{KV}$ doubles, both communication and computation also double, so the speedup remains roughly the same.}
\begin{tabular}{|l|ll|ll|ll|ll|ll|}
\hline
\multirow{2}{*}{Model} & Text & Frame & \multirow{2}{*}{$S_Q$} & \multirow{2}{*}{$S_{KV}$} & \multicolumn{2}{|c|}{Ring Attention} & \multicolumn{2}{|c|}{LV-XAttn} & \multicolumn{2}{|c|}{Speedup} \\ \cline{6-11}
& length  & count & & & CA (s) & Total (s) & CA (s) & Total (s) & CA & Total \\ \hline
\multirow{3}{*}{mPLUG-Owl3-7b} & 8K & 4K & 8K & 2916K & 174.73 & 202.84 & 24.08 & 42.79 & 7.26$\times$ & 4.74$\times$ \\
& 8K & 2K & 8K & 1458K & 89.88 & 112.28 & 12.14 & 32.72 & 7.41$\times$ & 3.43$\times$ \\
& 4K & 2K & 4K & 1458K & 92.48 & 107.01 & 6.45 & 19.5 & \highlight{14.33$\times$} & \highlight{5.49$\times$} \\
% & 4K & 1K & 4K & 729K & 47.72 & 64.07 & 3.28 & 19.44 & 14.55$\times$ & 3.3$\times$ \\
\hline\multirow{3}{*}{mPLUG-Owl3-2b} & 8K & 4K & 8K & 2916K & 83.41 & 90.1 & 10.33 & 17.39 & 8.07$\times$ & 5.18$\times$ \\
& 8K & 2K & 8K & 1458K & 36.66 & 45.42 & 5.21 & 11.28 & 7.04$\times$ & 4.03$\times$ \\
& 4K & 2K & 4K & 1458K & 37.78 & 44.8 & 2.79 & 8.25 & \highlight{13.52$\times$} & \highlight{5.43$\times$} \\
% & 4K & 1K & 4K & 729K & 19.31 & 24.62 & 1.44 & 5.29 & 13.42$\times$ & 4.65$\times$ \\
\hline\multirow{3}{*}{mPLUG-Owl3-1b} & 8K & 4K & 8K & 2916K & 47.12 & 55.1 & 5.17 & 12.69 & 9.12$\times$ & 4.34$\times$ \\
& 8K & 2K & 8K & 1458K & 22.63 & 28.81 & 2.62 & 7.99 & 8.64$\times$ & 3.6$\times$ \\
& 4K & 2K & 4K & 1458K & 23.26 & 29.24 & 1.52 & 5.24 & \highlight{15.32$\times$} & \highlight{5.58$\times$} \\
% & 4K & 1K & 4K & 729K & 11.2 & 14.22 & 0.81 & 3.3 & 13.89$\times$ & 4.31$\times$ \\
\hline\multirow{3}{*}{OpenFlamingo-9b} & 64K & 64K & 64K & 4096K & 95.13 & 165.17 & 62.4 & 126.71 & 1.52$\times$ & 1.3$\times$ \\
& 64K & 32K & 64K & 2048K & 47.86 & 101.01 & 31.44 & 86.89 & 1.52$\times$ & 1.16$\times$ \\
& 32K & 32K & 32K & 2048K & 33.58 & 69.53 & 16.0 & 49.5 & \highlight{2.1$\times$} & \highlight{1.4$\times$} \\
% & 32K & 16K & 32K & 1024K & 17.47 & 48.9 & 8.07 & 36.36 & 2.16$\times$ & 1.35$\times$ \\
\hline\multirow{3}{*}{OpenFlamingo-3b} & 64K & 64K & 64K & 4096K & 276.69 & 306.51 & 187.45 & 226.04 & 1.48$\times$ & 1.36$\times$ \\
& 64K & 32K & 64K & 2048K & 138.05 & 166.36 & 94.1 & 120.09 & 1.47$\times$ & 1.39$\times$ \\
& 32K & 32K & 32K & 2048K & 102.82 & 118.01 & 47.98 & 61.57 & \highlight{2.14$\times$} & \highlight{1.92$\times$} \\
% & 32K & 16K & 32K & 1024K & 50.49 & 62.67 & 24.18 & 35.38 & 2.09$\times$ & 1.77$\times$ \\
\hline
\end{tabular}
\label{tab:end_to_end_nersc}
\end{table*}

\begin{table*}[ht]\centering
\caption{Per iteration wall-clock time (in seconds) on 8 A30 24GB GPUs with Ring Attention and LV-XAttn. ``CA" represents the time spent on cross-attention operations.}
\begin{tabular}{|l|ll|ll|ll|ll|ll|}
\hline
\multirow{2}{*}{Model} & Text & Frame & \multirow{2}{*}{$S_Q$} & \multirow{2}{*}{$S_{KV}$} & \multicolumn{2}{|c|}{Ring Attention} & \multicolumn{2}{|c|}{LV-XAttn} & \multicolumn{2}{|c|}{Speedup} \\ \cline{6-11}
& length  & count & & & CA (s) & Total (s) & CA (s) & Total (s) & CA & Total \\ \hline
\multirow{3}{*}{mPLUG-Owl3-7b} & 1K & 512 & 1K & 364K & 42.41 & 74.28 & 1.42 & 33.32 & 29.96$\times$ & 2.23$\times$ \\
& 1K & 256 & 1K & 182K & 20.81 & 50.61 & 0.66 & 30.63 & 31.31$\times$ & 1.65$\times$ \\
& 512 & 256 & 512 & 182K & 20.84 & 49.89 & 0.45 & 28.95 & \highlight{45.85$\times$} & \highlight{1.72$\times$} \\
% & 512 & 128 & 512 & 91K & 10.53 & 39.11 & 0.27 & 28.53 & 39.28$\times$ & 1.37$\times$ \\
\hline\multirow{3}{*}{mPLUG-Owl3-2b} & 1K & 512 & 1K & 364K & 17.85 & 25.94 & 0.78 & 8.71 & 22.89$\times$ & 2.98$\times$ \\
& 1K & 256 & 1K & 182K & 9.06 & 16.56 & 0.44 & 7.86 & 20.6$\times$ & 2.11$\times$ \\
& 512 & 256 & 512 & 182K & 9.15 & 16.34 & 0.3 & 7.44 & \highlight{30.39$\times$} & \highlight{2.19$\times$} \\
% & 512 & 128 & 512 & 91K & 4.64 & 11.18 & 0.2 & 6.61 & 23.42$\times$ & 1.69$\times$ \\
\hline\multirow{3}{*}{mPLUG-Owl3-1b} & 1K & 512 & 1K & 364K & 10.6 & 14.41 & 0.44 & 4.18 & 24.25$\times$ & 3.45$\times$ \\
& 1K & 256 & 1K & 182K & 5.38 & 8.4 & 0.25 & 3.36 & 21.19$\times$ & 2.5$\times$ \\
& 512 & 256 & 512 & 182K & 5.31 & 8.22 & 0.18 & 3.03 & \highlight{29.44$\times$} & \highlight{2.71$\times$} \\
% & 512 & 128 & 512 & 91K & 2.7 & 5.16 & 0.12 & 2.53 & 22.05$\times$ & 2.04$\times$ \\
\hline\multirow{3}{*}{OpenFlamingo-9b} & 8K & 8K & 8K & 512K & 17.28 & 65.75 & 3.99 & 53.22 & 4.33$\times$ & 1.24$\times$ \\
& 8K & 4K & 8K & 256K & 8.74 & 54.09 & 2.2 & 52.17 & 3.97$\times$ & 1.04$\times$ \\
& 4K & 4K & 4K & 256K & 8.87 & 52.04 & 1.23 & 44.18 & \highlight{7.2$\times$} & \highlight{1.18$\times$} \\
% & 4K & 2K & 4K & 128K & 4.49 & 44.04 & 0.69 & 41.14 & 6.54$\times$ & 1.07$\times$ \\
\hline\multirow{3}{*}{OpenFlamingo-3b} & 8K & 8K & 8K & 512K & 52.26 & 69.45 & 12.25 & 32.71 & 4.27$\times$ & 2.12$\times$ \\
& 8K & 4K & 8K & 256K & 26.09 & 41.73 & 6.43 & 22.22 & 4.06$\times$ & 1.88$\times$ \\
& 4K & 4K & 4K & 256K & 25.84 & 40.59 & 3.62 & 18.28 & \highlight{7.14$\times$} & \highlight{2.22$\times$} \\
% & 4K & 2K & 4K & 128K & 13.18 & 29.17 & 2.19 & 16.36 & 6.03$\times$ & 1.78$\times$ \\
\hline
\end{tabular}
\label{tab:end_to_end_cl}
\end{table*}

Table~\ref{tab:end_to_end_nersc} shows the per iteration time of 5 models using LV-XAttn and Ring Attention on 16 A100 80GB GPUs. For the mPLUG-Owl3 models, LV-XAttn speeds up the cross-attention operation by 7.04 -- 15.32$\times$. Since the cross-attention operation accounts for the majority of the total iteration time when using Ring Attention, this reduction results in a significant total iteration speedup of 3.3 -- 5.58$\times$. For the OpenFlamingo models, which process a larger number of frames and thus have longer text lengths (due to the inclusion of a special token \texttt{<image>} per frame) and larger $S_Q$, the speedup is less pronounced, LV-XAttn achieves 1.47 -- 2.16$\times$ speedup on the cross-attention operation and 1.16 -- 1.92$\times$ speedup on the total iteration time. Additionally, OpenFlamingo-3b, with denser cross-attention layers, spends a larger portion of its time in cross-attention compared to OpenFlamingo-9b when using Ring Attention. Consequently, the speedup in cross-attention translates to a more substantial end-to-end speedup for OpenFlamingo-3b.

Table~\ref{tab:end_to_end_cl} shows the same experiment on 8 A30 24GB GPUs. We have smaller text lengths and fewer frames due to the smaller memory capacity. In this setup, the speedup for cross-attention operation is greater than that on 16 A100 GPUs: 20.6 -- 45.85$\times$ for the mPLUG-Owl3 models and 3.97 -- 7.2$\times$ for the OpenFlamingo models. This is due to smaller query block sizes $\frac{S_Q}{n}$ (shorter computations favors computation-bound LV-XAttn) and slower interconnect bandwidth (longer communication hurts communication-bound Ring Attention), as shown in Table~\ref{tab:runtime-formula}. However, the larger cross-attention speedups do not translate into a larger total speedup, as the portion of time spent on cross-attention layers decreases due to slower self-attention layers in LM blocks (caused by the slower interconnect). Despite this, the total speedup remains 1.37 -- 3.45$\times$ for the mPLUG-Owl3 models and 1.04 -- 2.22$\times$ for the OpenFlamingo models.

\subsection{Comparison with DeepSpeed-Ulysses}
\begin{table}[tb]
\centering
\caption{Per iteration wall-lock time (in seconds) of mPLUG-Owl3-2b ran on A100 80GB GPUs. The model uses multi-head attention with 12 heads.}
\begin{tabular}{|c|cc|c|c|} \hline
\makecell{Cluster\\Config.} & \makecell{Text /\\worker}& \makecell{Frame /\\worker} & DS (s) & LV-XAttn (s) \\ \hline
\multirow{3}{*}{12 GPUs} & 512 & 256 & OOM   & \textbf{13.38} \\ 
& 512 & 128 & 12.15 & \textbf{8.71} \\ 
& 256 & 128 & 9.32  & \textbf{6.1} \\ \hline
% 12 GPUs & 256 & 64  & 256 & 46K  & 5.82  & \textbf{4.78} \\ 
\multirow{3}{*}{6 GPUs}  & 512 & 256 & 16.36 & \textbf{10.58} \\ 
& 512 & 128 & 10.41 & \textbf{7.09} \\ 
& 256 & 128 & 8.81  & \textbf{5.83} \\ \hline
% 6 GPUs  & 256 & 64  & 256 & 46K  & 6.4   & \textbf{4.68} \\ 
\multirow{3}{*}{3 GPUs}  & 512 & 256 & 15.64 & \textbf{10.11} \\ 
 & 512 & 128 & 10.61 & \textbf{7.8} \\ 
 & 256 & 128 & 9.91  & \textbf{7.37} \\
% 3 GPUs  & 256 & 64  & 256 & 46K  & 7.71  & \textbf{6.39} \\ 
\hline
\end{tabular}
\label{tab:ds-owl}
\end{table}

\begin{table}[tb]
\centering
\caption{Per iteration wall-lock time (in seconds) of OpenFlamingo-3b ran on A30 24GB GPUs. The model uses multi-head attention with 8 heads.}
\begin{tabular}{|c|cc|c|c|} \hline
\makecell{Cluster\\Config.} & \makecell{Text /\\worker}& \makecell{Frame /\\worker} & DS (s) & LV-XAttn (s) \\ \hline
\multirow{3}{*}{8 GPUs} & 1K & 1K & OOM  & \textbf{32.71} \\ 
& 512 & 512 & OOM & \textbf{18.28} \\ 
& 256 & 256 & OOM & \textbf{14.46} \\ \hline
% 12 GPUs & 256 & 64  & 256 & 46K  & 5.82  & \textbf{4.78} \\ 
\multirow{3}{*}{4 GPUs}  & 1K & 1K & OOM & \textbf{19.71} \\ 
& 512 & 512 & OOM & \textbf{13.34} \\ 
& 256 & 256 & 11.65  & \textbf{11.29} \\ \hline
% 6 GPUs  & 256 & 64  & 256 & 46K  & 6.4   & \textbf{4.68} \\ 
\multirow{3}{*}{2 GPUs}  & 1K & 1K & OOM & \textbf{13.75} \\ 
 & 512 & 512 & 10.19 & \textbf{9.24} \\ 
 & 256 & 256 & 8.04  & \textbf{7.87} \\
% 3 GPUs  & 256 & 64  & 256 & 46K  & 7.71  & \textbf{6.39} \\ 
\hline
\end{tabular}
\label{tab:ds-flamingo}
\end{table}

% \begin{table*}[ht]\centering
% \caption{Per iteration wall-lock time (in seconds) of mPLUG-Owl3-2b ran on A100 80GB GPUs. The model uses multi-head attention with 12 heads. Since Deepspeed-Ulysses requires the number of heads to be divisible by the number workers, we run (TODO)}
% \begin{tabular}{|cc|cc|c|c|c|c|c|c|} \hline
% Text length & Frame count & \multirow{2}{*}{$\frac{S_Q}{n}$} & \multirow{2}{*}{$\frac{S_{KV}}{n}$} & \multicolumn{2}{c|}{3 GPUs} & \multicolumn{2}{c|}{6 GPUs} & \multicolumn{2}{c|}{12 GPUs} \\ \cline{5-10}
% per worker & per worker & & & DS & LV-XAttn & DS & LV-XAttn & DS & LV-XAttn \\ \hline
% 512 & 256 & 512 & 182K & 15.64 & \textbf{10.11} & 16.36 & \textbf{10.58} & OOM & \textbf{13.38} \\ 
% 512 & 128 & 512 & 91K & 10.61 & \textbf{7.8} & 10.41 & \textbf{7.09} & 12.15 & \textbf{8.71} \\ 
% 256 & 128 & 256 & 91K & 9.91 & \textbf{7.37} & 8.81 & \textbf{5.83} & 9.32 & \textbf{6.1} \\ 
% % 256 & 64 & 256 & 46K & 7.71 & \textbf{6.39} & 6.4 & \textbf{4.68} & 5.82 & \textbf{4.78} \\ 
% \hline
% \end{tabular}
% \label{tab:ds-owl}
% \end{table*}

% \begin{table*}[ht]\centering
% \caption{Per iteration wall-clock time (in seconds) of mPLUG-Owl3-2b ran on A100 80GB GPUs. The model uses multi-head attention with 12 heads. Since Deepspeed-Ulysses requires the number of heads to be divisible by the number of workers, we run (TODO)}
% \begin{tabular}{|cc|cc|c|c|c|c|c|c|c|} \hline
% Text length & Frame count & \multirow{2}{*}{$\frac{S_Q}{n}$} & \multirow{2}{*}{$\frac{S_{KV}}{n}$} & \multicolumn{3}{c|}{Deepspeed-Ulysses} & \multicolumn{3}{c|}{LV-XAttn} \\ \cline{5-10}
%  per worker & per worker & & & 3 GPUs & 6 GPUs & 12 GPUs & 3 GPUs & 6 GPUs & 12 GPUs \\ \hline
% 512 & 256 & 512 & 182K & 15.64 & 16.36 & OOM & \textbf{10.11} & \textbf{10.58} & \textbf{13.38} \\ 
% 512 & 128 & 512 & 91K & 10.61 & 10.41 & 12.15 & \textbf{7.8} & \textbf{7.09} & \textbf{8.71} \\ 
% 256 & 128 & 256 & 91K & 9.91 & 8.81 & 9.32 & \textbf{7.37} & \textbf{5.83} & \textbf{6.1} \\ 
% % 256 & 64 & 256 & 46K & 7.71 & 6.4 & 5.82 & \textbf{6.39} & \textbf{4.68} & \textbf{4.78} \\ 
% \hline
% \end{tabular}
% \label{tab:ds-owl}
% \end{table*}

% \begin{table*}[ht]\centering
% \caption{OpenFlamingo-3b on A30 cluster. 8 heads}
% \begin{tabular}{|cc|cc|c|c|c|c|c|c|c|} \hline
% Text length & Frame count & \multirow{2}{*}{$\frac{S_Q}{n}$} & \multirow{2}{*}{$\frac{S_{KV}}{n}$} & \multicolumn{3}{c|}{Deepspeed-Ulysses} & \multicolumn{3}{c|}{LV-XAttn} \\ \cline{5-10}
% per worker & per worker & & & 2 GPUs & 4 GPUs & 8 GPUs & 2 GPUs & 4 GPUs & 8 GPUs \\ \hline
% 1K & 1K & 1K & 64K & OOM & OOM & OOM & 13.75 & 19.71 & 32.71 \\ 
% 512 & 512 & 512 & 32K & 10.19 & OOM & OOM & 9.24 & 13.34 & 18.28 \\ 
% 256 & 256 & 256 & 16K & 8.04 & 11.65 & OOM & 7.87 & 11.29 & 14.46 \\ 
% 128 & 128 & 128 & 8K & 7.09 & 10.32 & 12.02 & 7.22 & 10.27 & 12.94 \\ 
% \hline\end{tabular}
% \label{tab:ds-flamingo}
% \end{table*}
For Deepspeed-Ulysses, each attention operation involves two all-to-all communications: one before the computation to gather input query, key and value blocks, and another afterward to distribute attention output along the sequence dimension. The first all-to-all is expensive as it involves communicating the large key-value blocks. To see this, we compare Deepspeed-Ulysses with LV-XAttn on mPLUG-Owl3-2b using the cluster with A100 80GB GPUs. As shown in Table~\ref{tab:ds-owl}, LV-XAttn achieves 1.21 -- 1.55$\times$ speedup compared to Deepspeed-Ulysses.

In addition, without activation recomputation, the larger memory footprint of Deepspeed-Ulysses limits its ability to process large visual inputs. When using the OpenFlamingo-3b model on the cluster with A30 24GB GPUs, Table~\ref{tab:ds-owl} shows that LV-XAttn is able to process up to $4\times$ longer text and visual inputs compared to Deepspeed-Ulysses.

Notably, the head parallelism in Deepspeed-Ulysses restricts both its scalability and flexibility: the maximum degree of parallelism is limited by the number of heads, and the number of heads has to be divisible by the number of workers.
\subsection{Ablation Study}
\label{sec:eval/ablation}
\begin{figure}[t]
    \centering
    \includegraphics[width=\linewidth]{figures/ablation_overlap.pdf}
    \caption{Ablation study on the effect of overlapping communication and computation with 6 A100 40GB GPUs. The frame count is set to 2048 per worker. Since processing the same total number of frames on a single GPU is not feasible due to memory constraints, the ``no communication'' runtime is derived by running the same per-worker input size on a single GPU and then scaling the result by 6. LV-XAttn incurs an overhead of less than 0.42\% compared to the no-communication baseline.}
    \label{fig:ablation_overlap}
\end{figure}
\textbf{Overlapping Communication and Computation} Figure~\ref{fig:ablation_overlap} shows the time spent on cross-attention in OpenFlamingo-3b using Ring Attention and LV-XAttn, with and without overlapping communication and computation, on 6 A100 40GB GPUs. While overlapping reduces the runtime for Ring Attention, its effect is limited as the large communication overhead of key-value blocks cannot be fully hidden by computation. In contrast, LV-XAttn reduces communication time by transmitting significantly smaller query, output, and softmax statistics blocks. The overlapping further hides the communication time, enabling distributed attention with no communication overhead.

\begin{figure}[t]
    \centering
    \begin{subfigure}[b]{\linewidth}
        \centering
        \includegraphics[width=0.97\linewidth]{figures/ablation_memory_owl.pdf}
        \caption{mPLUG-Owl-7b}
        \label{fig:ablation_mem_owl}
    \end{subfigure}
    \begin{subfigure}[b]{\linewidth}
        \centering
        \includegraphics[width=0.97\linewidth]{figures/ablation_memory_flamingo.pdf}
        \caption{OpenFlamingo-3b}
        \label{fig:ablation_mem_flamingo}
    \end{subfigure}
    \caption{Ablation study on the effect of activation recomputation for cross-attention layers with 3 A30 24GB GPUs. Text length is set to $2K$ and $8K$ for mPLUG-Owl-7b and OpenFlamingo-3b, respectively.}
    \label{fig:ablation_mem}
    \vspace{-3ex}
\end{figure}
\textbf{Activation Recomputation} Figure~\ref{fig:ablation_mem_owl} and \ref{fig:ablation_mem_flamingo} show the iteration time for running mPLUG-Owl-7b and OpenFlamingo-3b on a single node with 3 A100 40GB GPUs, with and without employing activation recomputation for cross-attention layers. By omitting the saving of large key-value blocks, the reduced memory consumption enables the processing of a larger number of frames, increasing by 1.6$\times$ and 1.5$\times$ for mPLUG-Owl-7b and OpenFlamingo-3b, respectively, with a negligible overhead of less than 8\%.


% \begin{table}[ht]\centering
% \caption{Owl Activation Saving}
% \begin{tabular}{|c|c|c|} \hline
% $\frac{S_Q}{n}$ & Save & No Save \\ \hline
% 3072 & 64 & 144 \\ 
% 2536 & 96 & 160 \\ 
% 2048 & 128 & 208 \\ 
% \hline\end{tabular}
% \end{table}



% \begin{table*}[ht]\centering
% \begin{tabular}{|l|l|l|l|l|l|l|l|l|}
% \hline
% \multirow{2}{*}{Model} & \multirow{2}{*}{Q Length} & \multirow{2}{*}{KV Length} & \multicolumn{3}{|c|}{Forward Time (sec)} & \multicolumn{3}{|c|}{Backward Time (sec)} \\ \cline{4-9}
%  &  &  & Ring & SplitQ & DupQ & Ring & SplitQ & DupQ \\ \hline
% \multirow{6}{*}{owl-7b} & \multirow{2}{*}{512} & \multirow{2}{*}{72900} & 13.09 (1.0x) & 1.26 (10.39x) & 1.43 (9.16x) & 45.72 (1.0x) & 15.82 (2.89x) & 16.56 (2.76x) \\ \cline{4-9}
%  & & & 23.09 (1.0x) & 11.13 (2.07x) & 10.91 (2.12x) & 60.15 (1.0x) & 27.93 (2.15x) & 28.91 (2.08x) \\ \cline{2-9}
% & \multirow{2}{*}{512} & \multirow{2}{*}{36450} & 6.61 (1.0x) & 0.71 (9.25x) & 1.01 (6.53x) & 27.32 (1.0x) & 11.67 (2.34x) & 11.91 (2.29x) \\ \cline{4-9}
%  & & & 14.87 (1.0x) & 8.79 (1.69x) & 9.32 (1.6x) & 40.65 (1.0x) & 24.29 (1.67x) & 25.16 (1.62x) \\ \cline{2-9}
% & \multirow{2}{*}{256} & \multirow{2}{*}{36450} & 6.45 (1.0x) & 0.45 (14.27x) & 0.52 (12.39x) & 27.07 (1.0x) & 9.72 (2.79x) & 10.14 (2.67x) \\ \cline{4-9}
%  & & & 14.27 (1.0x) & 8.3 (1.72x) & 8.5 (1.68x) & 38.73 (1.0x) & 21.12 (1.83x) & 21.53 (1.8x) \\ \cline{2-9}
% \hline\hline\multirow{6}{*}{owl-2b} & \multirow{2}{*}{512} & \multirow{2}{*}{72900} & 5.62 (1.0x) & 0.5 (11.23x) & 0.64 (8.81x) & 12.87 (1.0x) & 4.36 (2.95x) & 4.58 (2.81x) \\ \cline{4-9}
%  & & & 8.34 (1.0x) & 3.21 (2.6x) & 3.37 (2.47x) & 13.96 (1.0x) & 5.27 (2.65x) & 5.47 (2.55x) \\ \cline{2-9}
% & \multirow{2}{*}{512} & \multirow{2}{*}{36450} & 2.86 (1.0x) & 0.29 (9.85x) & 0.44 (6.5x) & 7.59 (1.0x) & 3.16 (2.41x) & 3.35 (2.27x) \\ \cline{4-9}
%  & & & 5.1 (1.0x) & 2.69 (1.89x) & 2.75 (1.86x) & 8.24 (1.0x) & 3.83 (2.15x) & 4.06 (2.03x) \\ \cline{2-9}
% & \multirow{2}{*}{256} & \multirow{2}{*}{36450} & 2.81 (1.0x) & 0.21 (13.1x) & 0.23 (12.0x) & 7.43 (1.0x) & 2.35 (3.16x) & 2.42 (3.06x) \\ \cline{4-9}
%  & & & 5.13 (1.0x) & 2.92 (1.76x) & 2.58 (1.99x) & 8.0 (1.0x) & 2.93 (2.73x) & 3.0 (2.67x) \\ \cline{2-9}
% \hline\hline\multirow{6}{*}{owl-1b} & \multirow{2}{*}{512} & \multirow{2}{*}{72900} & 3.34 (1.0x) & 0.6 (5.52x) & 0.49 (6.81x) & 7.2 (1.0x) & 1.7 (4.24x) & 1.71 (4.21x) \\ \cline{4-9}
%  & & & 5.41 (1.0x) & 2.47 (2.19x) & 2.29 (2.36x) & 7.49 (1.0x) & 1.92 (3.89x) & 1.93 (3.88x) \\ \cline{2-9}
% & \multirow{2}{*}{512} & \multirow{2}{*}{36450} & 1.67 (1.0x) & 0.33 (5.11x) & 0.32 (5.17x) & 3.88 (1.0x) & 1.14 (3.4x) & 1.18 (3.27x) \\ \cline{4-9}
%  & & & 3.37 (1.0x) & 1.78 (1.9x) & 1.88 (1.79x) & 4.11 (1.0x) & 1.36 (3.01x) & 1.4 (2.93x) \\ \cline{2-9}
% & \multirow{2}{*}{256} & \multirow{2}{*}{36450} & 1.62 (1.0x) & 0.19 (8.41x) & 0.17 (9.47x) & 3.74 (1.0x) & 0.78 (4.78x) & 0.78 (4.77x) \\ \cline{4-9}
%  & & & 3.29 (1.0x) & 2.05 (1.61x) & 2.4 (1.37x) & 3.96 (1.0x) & 0.99 (3.99x) & 0.99 (4.02x) \\ \cline{2-9}
% \hline\hline\multirow{6}{*}{flamingo-9b} & \multirow{2}{*}{128} & \multirow{2}{*}{131072} & 14.02 (1.0x) & 7.37 (1.9x) & 7.58 (1.85x) & 23.19 (1.0x) & 11.33 (2.05x) & 11.27 (2.06x) \\ \cline{4-9}
%  & & & 18.49 (1.0x) & 12.78 (1.45x) & 12.75 (1.45x) & 25.89 (1.0x) & 12.49 (2.07x) & 12.5 (2.07x) \\ \cline{2-9}
% & \multirow{2}{*}{128} & \multirow{2}{*}{65536} & 10.67 (1.0x) & 7.66 (1.39x) & 7.46 (1.43x) & 17.09 (1.0x) & 10.47 (1.63x) & 10.34 (1.65x) \\ \cline{4-9}
%  & & & 13.56 (1.0x) & 10.68 (1.27x) & 10.55 (1.29x) & 18.82 (1.0x) & 12.38 (1.52x) & 12.27 (1.53x) \\ \cline{2-9}
% & \multirow{2}{*}{64} & \multirow{2}{*}{65536} & 10.65 (1.0x) & 7.6 (1.4x) & 7.33 (1.45x) & 17.35 (1.0x) & 9.63 (1.8x) & 9.59 (1.81x) \\ \cline{4-9}
%  & & & 15.44 (1.0x) & 10.41 (1.48x) & 11.2 (1.38x) & 18.69 (1.0x) & 11.97 (1.56x) & 11.92 (1.57x) \\ \cline{2-9}
% \hline\hline\multirow{6}{*}{flamingo-3b} & \multirow{2}{*}{128} & \multirow{2}{*}{131072} & 22.46 (1.0x) & 4.18 (5.38x) & 2.86 (7.86x) & 45.24 (1.0x) & 5.34 (8.47x) & 5.13 (8.83x) \\ \cline{4-9}
%  & & & 26.23 (1.0x) & 8.01 (3.28x) & 7.15 (3.67x) & 45.47 (1.0x) & 5.51 (8.26x) & 5.26 (8.64x) \\ \cline{2-9}
% & \multirow{2}{*}{128} & \multirow{2}{*}{65536} & 11.78 (1.0x) & 3.15 (3.74x) & 2.44 (4.83x) & 23.72 (1.0x) & 4.36 (5.44x) & 4.26 (5.56x) \\ \cline{4-9}
%  & & & 14.51 (1.0x) & 6.64 (2.18x) & 5.78 (2.51x) & 23.86 (1.0x) & 4.51 (5.29x) & 4.4 (5.43x) \\ \cline{2-9}
% & \multirow{2}{*}{64} & \multirow{2}{*}{65536} & 12.18 (1.0x) & 3.12 (3.9x) & 2.24 (5.43x) & 24.23 (1.0x) & 3.89 (6.23x) & 3.71 (6.54x) \\ \cline{4-9}
%  & & & 14.85 (1.0x) & 5.78 (2.57x) & 5.33 (2.79x) & 24.37 (1.0x) & 4.04 (6.03x) & 3.85 (6.33x) \\ \cline{2-9}
% \hline
% \end{tabular}
% \caption{For each Q, KV length combination and method, two numbers are reported. The top row is the time spent on cross attention during a model pass, while the bottom row is that of the total model pass.)}
% \label{tab:your_label}
% \end{table*}
% \begin{table*}[ht]\centering
% \begin{tabular}{|l|l|l|l|l|l|l|l|l|}
% \hline
% \multirow{2}{*}{Model} & \multirow{2}{*}{Q Length} & \multirow{2}{*}{KV Length} & \multicolumn{3}{|c|}{Forward Time (sec)} & \multicolumn{3}{|c|}{Backward Time (sec)} \\ \cline{4-9}
%  &  &  & Ring & SplitQ & DupQ & Ring & SplitQ & DupQ \\ \hline
% \multirow{6}{*}{Flamingo-3b} & \multirow{2}{*}{4096} & \multirow{2}{*}{262144} & 69.57 (1.0x) & 45.22 (1.54x) & OOM & 209.59 (1.0x) & 142.5 (1.47x) & OOM \\ \cline{4-9}
%  & & & 102.56 (1.0x) & 71.85 (1.43x) & OOM & 226.38 (1.0x) & 154.48 (1.47x) & OOM \\ \cline{2-9}
% & \multirow{2}{*}{4096} & \multirow{2}{*}{131072} & 34.44 (1.0x) & 22.66 (1.52x) & OOM & 104.57 (1.0x) & 71.59 (1.46x) & OOM \\ \cline{4-9}
%  & & & 46.11 (1.0x) & 33.03 (1.4x) & OOM & 117.65 (1.0x) & 83.35 (1.41x) & OOM \\ \cline{2-9}
% & \multirow{2}{*}{2048} & \multirow{2}{*}{131072} & 32.72 (1.0x) & 11.88 (2.75x) & OOM & 68.7 (1.0x) & 36.17 (1.9x) & OOM \\ \cline{4-9}
%  & & & 42.75 (1.0x) & 18.82 (2.27x) & OOM & 76.31 (1.0x) & 42.27 (1.81x) & OOM \\ \cline{2-9}
% & \multirow{2}{*}{2048} & \multirow{2}{*}{65536} & 16.21 (1.0x) & 6.02 (2.69x) & OOM & 34.05 (1.0x) & 18.26 (1.87x) & OOM \\ \cline{4-9}
%  & & & 22.52 (1.0x) & 11.06 (2.04x) & OOM & 40.6 (1.0x) & 24.17 (1.68x) & OOM \\ \cline{2-9}
% & \multirow{2}{*}{1024} & \multirow{2}{*}{65536} & 17.21 (1.0x) & 3.72 (4.62x) & 3.71 (4.64x) & 35.25 (1.0x) & 9.26 (3.81x) & 9.51 (3.71x) \\ \cline{4-9}
%  & & & 22.57 (1.0x) & 8.45 (2.67x) & 8.88 (2.54x) & 39.92 (1.0x) & 13.05 (3.06x) & 13.4 (2.98x) \\ \cline{2-9}
% \hline
% \end{tabular}
% \caption{262144 KV = 4096 images. With 16 nodes, this corresponds to 64K images. }
% \label{tab:your_label}
% \end{table*}
% \begin{table*}[ht]\centering
% \begin{tabular}{|l|l|l|l|l|l|l|l|l|}
% \hline
% \multirow{2}{*}{Model} & \multirow{2}{*}{Q Length} & \multirow{2}{*}{KV Length} & \multicolumn{3}{|c|}{Forward Time (ms)} & \multicolumn{3}{|c|}{Backward Time (ms)} \\ \cline{4-9}
%  &  &  & Ring & SplitQ & DupQ & Ring & SplitQ & DupQ \\ \hline
% \multirow{6}{*}{Flamingo-9b} & \multirow{2}{*}{4096} & \multirow{2}{*}{262144} & 24.32 (1.0x) & 15.04 (1.62x) & 70.81 (1.0x) & 47.36 (1.5x) & nan & nan \\ \cline{4-9}
%  & & & 60.4 (1.0x) & 45.29 (1.33x) & 104.77 (1.0x) & 81.42 (1.29x) & nan & nan \\ \cline{2-9}
% & \multirow{2}{*}{4096} & \multirow{2}{*}{131072} & 12.22 (1.0x) & 7.55 (1.62x) & 35.63 (1.0x) & 23.89 (1.49x) & nan & nan \\ \cline{4-9}
%  & & & 32.77 (1.0x) & 30.71 (1.07x) & 68.24 (1.0x) & 56.18 (1.21x) & nan & nan \\ \cline{2-9}
% & \multirow{2}{*}{2048} & \multirow{2}{*}{131072} & 10.91 (1.0x) & 3.96 (2.75x) & 22.67 (1.0x) & 12.04 (1.88x) & nan & nan \\ \cline{4-9}
%  & & & 28.34 (1.0x) & 19.88 (1.43x) & 41.19 (1.0x) & 29.62 (1.39x) & nan & nan \\ \cline{2-9}
% & \multirow{2}{*}{2048} & \multirow{2}{*}{65536} & 5.68 (1.0x) & 2.0 (2.84x) & 11.79 (1.0x) & 6.07 (1.94x) & nan & nan \\ \cline{4-9}
%  & & & 18.25 (1.0x) & 12.95 (1.41x) & 30.66 (1.0x) & 23.41 (1.31x) & nan & nan \\ \cline{2-9}
% \hline
% \end{tabular}
% \caption{For each Q, KV length combination and method, two numbers are reported. The top row is the time spent on cross attention during a model pass, while the bottom row is that of the total model pass.)}
% \label{tab:your_label}
% \end{table*}


\begin{figure}
    \centering
    \includegraphics[width=\linewidth]{figures/normthinker.pdf}
    \caption{Retrieval-augmented generation pipeline.}
    \label{fig:normthinker}
    \vspace{-10pt}
\end{figure}
\section{Augmenting Normative Reasoning with Retrieval over \dataset{}}
In this section, we answer RQ3, and evaluate whether \dataset{} can be directly applied to augment normative reasoning in VLMs. 
% As incorrect norm sensibility understanding and norm prioritization are the primary causes for norm reasoning failures (Figure \ref{fig:fail}), we propose performing retrieval over the context present in \dataset{}, a strategy we call \normthinker{}, to guide VLMs in making contextually-grounded normative decisions.
Recall that incorrect norm sensibility understanding and norm prioritization are the primary causes of norm reasoning failures (Figure \ref{fig:fail}). Therefore, we propose performing retrieval over the context present in \dataset{}, a strategy we call \normthinker{}, to guide VLMs in making contextually-grounded normative decisions.
% As incorrect norm sensibility understanding and norm prioritization are the primary causes for norm reasoning failures (Figure \ref{fig:fail}), we propose performing retrieval over the context present in \dataset{}, a strategy we call \normthinker{}, to guide the VLMs to make contextually-grounded normative decisions.


% To validate this approach, we study the performance improvement of \normthinker{} using GPT-4o as a base model.

\subsection{\dataset{} RAG Approach}
Existing VLMs parse context robustly, but fail to retrieve and apply correct norms from the context. Thus, intuitively, given the strong context-sensitivity of norms, a tractable approach would be to guide VLMs towards the correct norms for a given context, once the context is extracted by that VLM. Retrieval-Augmented Generation (RAG) \cite{lewis2020retrieval} enables us to do this---by leveraging the VLMs where they are most performant (i.e., as a visual context parser),
% ,\footnote{i.e. as a visual context parser} 
this simplifies the task of deeper normative reasoning by providing contextually-grounded norm examples that the VLM can use as a many-shot example. 
The retrieval pipeline is shown in Figure \ref{fig:normthinker}; further details on the pipeline are provided in Appendix~\ref{appendix:indexing}.

%\dataset{} is well-suited as a source in this method, as its diversity of context and basis in human consensus means th 

\subsection{\dataset{}-Enhanced Results}
To fairly test the utility of \dataset{} on new data, we curate an out-of-domain test dataset based on egocentric robotic assistant footage \cite{zhu2024siat}, selected as its context and embodiment are orthogonal to those seen in Ego4D. Actions and justifications are manually generated to be highly challenging, with baseline GPT-4o scoring 18.2\%.\footnote{11 samples were selected from 100 candidate samples, from which 11 datapoints were generated to maximize the diversity of actions and contexts represented. While this is a sufficient number for the purposes of this example, future work should target a wider range of embodiments.} Using retrieval across \dataset{}, we demonstrate improvement relative to the best non-RAG model and base GPT-4o on unseen in-domain tasks, obtaining an \dataset{} bench 9.4\% better than base GPT-4o, and 7.9\% better than randomized retrieval across \dataset{}, as shown in Table~\ref{tab:normthinker1}.

%By contrast, \normthinker{} exhibits a notable improvement in normative reasoning, as shown in Table \ref{tab:normthinker}. % , resulting in enhanced performance. 

%However, it still demonstrates a significant performance gap compared to human reasoning.
\begin{table}
\centering
\small
\begin{tabular}{l cccc}
\toprule
 \multirow{2}{*}{Model} & \multicolumn{3}{c}{\% Correct MCQ} & {Sens.} \\
\cmidrule(lr){2-5}
 & \multicolumn{1}{c}{Both} & \multicolumn{1}{c}{Act.} & \multicolumn{1}{c}{Rsn.} &  \multicolumn{1}{c}{Act.} \\
\midrule
 {GPT-4o} & 1/11 & 5/11 & 2/11 & 3/11 \\
 {\quad + Best-5 Retrieval} & \textbf{5/11} & \textbf{7/11} & \textbf{5/11} & 3/11 \\ 
\midrule
\midrule
 {Human} & 8/11 & 8/11 & 8/11 & 9/11 \\ 
\bottomrule     
\end{tabular}
\caption{Results with \normthinker{} on ego-centric robotics videos, n=11. }
\label{tab:normthinker}
\end{table}

\begin{table}
\centering
\small
\begin{tabular}{l cccc}
\toprule
 \multirow{2}{*}{Model} & \multicolumn{3}{c}{\% Correct MCQ} & {Sens.} \\
\cmidrule(lr){2-5}
 & \multicolumn{1}{c}{Both} & \multicolumn{1}{c}{Act.} & \multicolumn{1}{c}{Rsn.} &  \multicolumn{1}{c}{Act.} \\
% Constant Choice &  25.3 & 25.3 & 25.3 & 40.5 \\
\midrule
{Gemini 1.5 Pro} & 45.2 & 51.8 & 47.7 & \textbf{64.0}\\
{GPT-4o} & 39.8 & 44.9 & 45.1 & 59.6 \\
{\quad + Random Retrieval} & 41.3 & 51.0 & 45.7 & 52.6 \\
{\quad + Best-5 Retrieval} & \textbf{49.2} & \textbf{54.5} &\textbf{52.6} & 56.2 \\ 
\midrule
{Human} & 92.4 & 92.4 & 92.4 & 85.1 \\ 
\bottomrule     
\end{tabular}
\caption{Results with \normthinker{} on held-out instances in \dataset{}.} 
\label{tab:normthinker1}
\end{table}

% One example \normthinker{} in action is illustrated in Figure~\ref{fig:normthinker_example}. In the case of unenriched inference on the example \dataset{} task, GPT-4o correctly selects the action but arrives at incorrect reasoning, concluding that one should play games with friends after finishing a meal. In contrast, with \normthinker{}, the top retrieved datapoint, which depicts a similar cleanup process following a game, provides an in-context example that contextualizes tidying up as a sign of respect as the most normative action in this class of context, enabling GPT-4o to rethink and select the correct justification: "help with cleanup."

% \begin{figure*}
%     \centering
%     \includegraphics[width=\linewidth]{img/normthinker_example.jpg}
%     \caption{An example from \normthinker{} illustrating how retrieved data points aid normative reasoning. The correct answers are underlined. Without the reference video and justification, GPT 4o selects the correct action but provides incorrect reasoning. With the retrieved data point—depicting a similar cleanup process, GPT 4o selects the correct justification: "help with cleanup."}

%     \label{fig:normthinker_example}
% \end{figure*}

\section{Related Work}
\label{sec:rw}

\paragraph{Multi-agent communication} 
Communication between agents have long been suggested in order to develop strong modular systems \cite{KRAUS199779, 8352646, sukhbaatar2016learningmultiagentcommunicationbackpropagation, foerster2016learningcommunicatedeepmultiagent, 10.1007/978-3-319-75477-2_2, lazaridou2020emergentmultiagentcommunicationdeep, lowe2020multiagentactorcriticmixedcooperativecompetitive}. 
One of the most exciting applications of language agents is in designing multi-agent environments where agents autonomously communicate, with examples in improving factuality and reasoning via agent debate \cite{du2023improvingfactualityreasoninglanguage, liang-etal-2024-encouraging}, cooperation between embodied robots \cite{mandi2023rocodialecticmultirobotcollaboration, chen2024scalablemultirobotcollaborationlarge}, simulating software development teams \cite{li2023camel, hong2024metagpt, qian-etal-2024-chatdev, liu2024a}, modeling human interactions \cite{10.1145/3526113.3545616, 10.1145/3586183.3606763}, and gaming environments \cite{mukobi2023welfarediplomacybenchmarkinglanguage, xu2024exploringlargelanguagemodels}.
Thus, several frameworks have been proposed to enable simple development of multi-agent environments \cite{li2023camel, wu2023autogenenablingnextgenllm, hong2024metagpt}.
We contribute to this line of work by introducing \ourenv{}, a new modular environment for multi-agent communication, which allows easy configuration of difficulty levels and communication structures.

Previous work has showed that multi-agent collaboration is susceptible to adversarial attacks \cite{amayuelas-etal-2024-multiagent} and that Theory of Mind can be used to improve collaboration in simple gaming environments \cite{lim2020improvingmultiagentcooperationusing}. Our work introduces the notion of monitoring and interventions for improving communication in LLM-based multi-agent systems, which complements these prior methods.

\paragraph{Uncertainty estimation in language modeling} Uncertainty estimation has been shown useful in detecting and mitigating hallucinations in knowledge-intensive tasks \cite{kadavath2022languagemodelsmostlyknow, yona-etal-2024-large, ivgi2024from}, including in \emph{agentic retrieval}  where an external search engine is used \cite{jiang-etal-2023-active, han-etal-2024-towards}.
It has also been recently applied to language agents in order to increase exploration \cite{rahn2024controlling}, improve performance on bandit tasks \cite{felicioni2024on}, or making agents output textual estimates to help debates \cite{debunc}. In this work, we bridge uncertainty estimation and multi-agent collaboration by training monitors to predict the probability of task failures given agents uncertainty.



\paragraph{Aggregations over Multiple Generations}

Sampling multiple generations and aggregating over their answers is a popular method to increase performance \cite{wang2023selfconsistency, yoran-etal-2023-answering, du2023improvingfactualityreasoninglanguage, chen2024universal, min2024beyond}.
However, post-hoc aggregation is not directly applicable in agentic settings, where actions can be \emph{irreversible}.
Additionally, majority voting \cite{wang2023selfconsistency} requires at least three generations, while our approach increases turn account by less than twofold on average in our main experiments with \ourenv{}.
Our work differs by resetting the communication channel \emph{before} a problematic action was taken, rather than aggregating \emph{after} the final prediction.

\section{Conclusion}
We present live monitoring and mid-run interventions for multi-agent systems. We demonstrate that monitors based on simple statistical measures can effectively predict future agent failures, and these failures can be prevented by restarting the communication channel. Experiments across multiple environments and models show consistent gains of up to 17.4\%-20\% in system performance, with an addition in inference-time compute.
Our work also introduces \ourenv{}, a new environment for studying multi-agent cooperation.


% \pagebreak



\section{Limitations}

We establish a new framework for computational metaphor analysis with conceptual, methodological, analytical, and resource contributions. In light of this large scope, there are many limitations that future work may consider addressing. 

Our method has many components, each of which could be further optimized. To measure conceptual associations, we only test one document embedding model, one similarity metric, and compare documents with hand-crafted ``carrier sentences''; the specific choice of sentences may also affect performance. Our experiments with LLM-based metaphorical word detection and concept mapping are slightly more comprehensive, as we evaluate three LLMs and two different prompts. However, we do not test few-shot approaches or implement any prompt optimization. We simply add together word-level and concept level scores to get a combined score, and show that this combined score outperforms the individual components. However, future work could evaluate different combination strategies or learn optimal linear combination weights on a held-out set.


Our analysis also has limitations. First is the lack of causality: while we control for various confounds in our regressions, we do not evaluate causal assumptions nor intend to make causal claims. Second is the ambiguity around user engagement as a behavioral outcome. People have diverse motivations for favoriting and retweeting content \citep{meier2014more,boyd2010tweet}, so it is unclear precisely what motivates people to engage in these behaviors, and why we observe stronger associations between metaphor and retweets than favorites. 
It is possible that favoriting activity is dampened by negative emotional content conveyed with dehumanizing metaphors, while retweeting reflects the desire to amplify information that communicates threats \citep{mendelsohn2021modeling}. But, it is not possible to evaluate these mechanisms with the available data. Third, we only have data about favorite and retweet \textit{counts}, not \textit{who} is engaging with the content. This limits interpretations of audience susceptibility to metaphor exposure. We motivate and connect our analysis to prior literature by assuming that authors and their audiences share similar ideologies \citep{barbera2015tweeting}, but this assumption may not always hold. 

Our domain of focus---U.S. immigration discourse on Twitter---is worthy of study in its own right. Nevertheless, the present work is limited in generalizability. We urge future work to extend our methods, evaluation, and analysis to other political issues, platforms, countries, and languages.  




% both theoretical and methodological approaches to disentangle use-mention of metaphors. from the cognitive standpoint, does mentioning the metaphor still have the same dehumanizing effect (e.g. as we've seen with slurs)


%we dont have data about who is engaging with what kind of content - limits the kinds of conclusions we can draw about metaphorical framing effects. 

%Unfortunately, we only have political ideology data for tweet authors and not audience members engaging with the tweets. This analysis thus rests on the assumption that a tweet's author and audience generally share the same political ideology \citep{barbera2015tweeting}. Specifically, we compare how associations between metaphoricity and user engagement differ based on whether the author is liberal or conservative.

% We then separate tweets into those written by liberal and conservative authors to investigate the moderating role of ideology. We unfortunately only have the numbers of favorites and retweets and lack information about \textit{who} is engaging with this content. If assumptions of homophily hold, audiences engaging with tweets would most often share the same political ideology as the respective authors.


%We consider favorite and retweet metrics as outcome variables. Our data includes the number of favorites and retweets that a tweet receives, but we unfortunately lack access to information about view counts, \textit{who} is exposed to a particular tweet, and \textit{who} engages with each tweet. To understand effects on audiences, we thus rely on information about the author's ideology and assumptions of homophily: audiences exposed to conservative (liberal) tweets are primarily conservative (liberal) \citep{barbera2015tweeting}. 

\section*{Ethics Statement}
\label{sec:ethics}
\paragraph{Ethical Assumptions.} 
We emphasize that \dataset{} is designed as a descriptive benchmark rather than a prescriptive one --- 
the dataset is intended to evaluate the ability of VLMs to understand physical social norms in egocentric videos, rather than to dictate what these norms should be or how they should be enforced.
We thus acknowledge that the norms depicted in the dataset may not be universally applicable or appropriate in all contexts and that the interpretation of these norms may vary across cultures, communities, time periods, and individuals.

\paragraph{Bias and Fairness.}
Despite our best efforts to create a diverse and representative dataset, 
we acknowledge that \dataset{} may contain biases that reflect the perspectives and experiences of the dataset creators and annotators.
Consequently, the norms and justifications depicted in the dataset may be influenced by the cultural, social, and demographic characteristics of the individuals who contributed to the dataset.
While all of our annotators are from the United States, norms often differ in different cultures \citep{rao2024normad, shi2024culturebank}.
To address these concerns, 
we recommend that researchers using \dataset{} for training or evaluation critically assess potential biases and ensure they align with the intended application context.

\paragraph{Human Subjects and Privacy.}
\dataset{} is constructed from Ego4D videos, 
which are publicly available and do not contain personally identifiable information.
The Ego4D dataset is released under a non-exclusive, non-transferable license that permits its use for academic research, as outlined in the license agreement. Our work complies with the terms of this license, using the Ego4D data solely for research purposes. Our annotation process was conducted with proper informed consent,
ensuring annotators are fully aware of the task, its purpose, and how their contribution would be used.
Annotators were compensated fairly for their time and effort (details in Appendix~\ref{sec:HumanValidationProcess}).
The data used in this work does not include personally identifiable information.
No sensitive information about the annotators or individuals appearing in the video data was collected or used in the study.
% This work was reviewed and deemed exempt by the Institutional Review Board at the institution where the work was conducted. %\mohammad{I'm not sure about this part.}
Notably,
this work was thoroughly reviewed and approved by the Institutional Review Board (IRB) at Stanford University (IRB-77185).

\paragraph{Risks in Deployment.}
The deployment of AI systems trained on \dataset{} may pose risks if these systems are used to make decisions that impact individuals' safety, well-being, or rights.
To mitigate these risks, 
we stress that \dataset{} should not be used for prescriptive advice or to make decisions with ethical, or safety implications without extensive human oversight. 
By using \dataset{},
researchers should be aware of the limitations of the dataset and the potential risks associated with deploying systems trained on it.
\section*{Acknowledgments}
This research was supported in part by Other Transaction award HR00112490375 from the U.S. Defense Advanced Research Projects Agency (DARPA) Friction for Accountability in Conversational Transactions (FACT) program. We thank Google Cloud Platform and Modal Platform for their credits. We also thank Yonatan Bisk, Dorsa Sadigh, and members of the Stanford SALT lab for their feedback and input. The authors thank Leena Mathur and Su Li for their help in collecting out-of-domain robotics videos.



% Entries for the entire Anthology, followed by custom entries
\bibliography{custom}
% \bibliographystyle{acl_natbib}

%\clearpage
\subsection{Lloyd-Max Algorithm}
\label{subsec:Lloyd-Max}
For a given quantization bitwidth $B$ and an operand $\bm{X}$, the Lloyd-Max algorithm finds $2^B$ quantization levels $\{\hat{x}_i\}_{i=1}^{2^B}$ such that quantizing $\bm{X}$ by rounding each scalar in $\bm{X}$ to the nearest quantization level minimizes the quantization MSE. 

The algorithm starts with an initial guess of quantization levels and then iteratively computes quantization thresholds $\{\tau_i\}_{i=1}^{2^B-1}$ and updates quantization levels $\{\hat{x}_i\}_{i=1}^{2^B}$. Specifically, at iteration $n$, thresholds are set to the midpoints of the previous iteration's levels:
\begin{align*}
    \tau_i^{(n)}=\frac{\hat{x}_i^{(n-1)}+\hat{x}_{i+1}^{(n-1)}}2 \text{ for } i=1\ldots 2^B-1
\end{align*}
Subsequently, the quantization levels are re-computed as conditional means of the data regions defined by the new thresholds:
\begin{align*}
    \hat{x}_i^{(n)}=\mathbb{E}\left[ \bm{X} \big| \bm{X}\in [\tau_{i-1}^{(n)},\tau_i^{(n)}] \right] \text{ for } i=1\ldots 2^B
\end{align*}
where to satisfy boundary conditions we have $\tau_0=-\infty$ and $\tau_{2^B}=\infty$. The algorithm iterates the above steps until convergence.

Figure \ref{fig:lm_quant} compares the quantization levels of a $7$-bit floating point (E3M3) quantizer (left) to a $7$-bit Lloyd-Max quantizer (right) when quantizing a layer of weights from the GPT3-126M model at a per-tensor granularity. As shown, the Lloyd-Max quantizer achieves substantially lower quantization MSE. Further, Table \ref{tab:FP7_vs_LM7} shows the superior perplexity achieved by Lloyd-Max quantizers for bitwidths of $7$, $6$ and $5$. The difference between the quantizers is clear at 5 bits, where per-tensor FP quantization incurs a drastic and unacceptable increase in perplexity, while Lloyd-Max quantization incurs a much smaller increase. Nevertheless, we note that even the optimal Lloyd-Max quantizer incurs a notable ($\sim 1.5$) increase in perplexity due to the coarse granularity of quantization. 

\begin{figure}[h]
  \centering
  \includegraphics[width=0.7\linewidth]{sections/figures/LM7_FP7.pdf}
  \caption{\small Quantization levels and the corresponding quantization MSE of Floating Point (left) vs Lloyd-Max (right) Quantizers for a layer of weights in the GPT3-126M model.}
  \label{fig:lm_quant}
\end{figure}

\begin{table}[h]\scriptsize
\begin{center}
\caption{\label{tab:FP7_vs_LM7} \small Comparing perplexity (lower is better) achieved by floating point quantizers and Lloyd-Max quantizers on a GPT3-126M model for the Wikitext-103 dataset.}
\begin{tabular}{c|cc|c}
\hline
 \multirow{2}{*}{\textbf{Bitwidth}} & \multicolumn{2}{|c|}{\textbf{Floating-Point Quantizer}} & \textbf{Lloyd-Max Quantizer} \\
 & Best Format & Wikitext-103 Perplexity & Wikitext-103 Perplexity \\
\hline
7 & E3M3 & 18.32 & 18.27 \\
6 & E3M2 & 19.07 & 18.51 \\
5 & E4M0 & 43.89 & 19.71 \\
\hline
\end{tabular}
\end{center}
\end{table}

\subsection{Proof of Local Optimality of LO-BCQ}
\label{subsec:lobcq_opt_proof}
For a given block $\bm{b}_j$, the quantization MSE during LO-BCQ can be empirically evaluated as $\frac{1}{L_b}\lVert \bm{b}_j- \bm{\hat{b}}_j\rVert^2_2$ where $\bm{\hat{b}}_j$ is computed from equation (\ref{eq:clustered_quantization_definition}) as $C_{f(\bm{b}_j)}(\bm{b}_j)$. Further, for a given block cluster $\mathcal{B}_i$, we compute the quantization MSE as $\frac{1}{|\mathcal{B}_{i}|}\sum_{\bm{b} \in \mathcal{B}_{i}} \frac{1}{L_b}\lVert \bm{b}- C_i^{(n)}(\bm{b})\rVert^2_2$. Therefore, at the end of iteration $n$, we evaluate the overall quantization MSE $J^{(n)}$ for a given operand $\bm{X}$ composed of $N_c$ block clusters as:
\begin{align*}
    \label{eq:mse_iter_n}
    J^{(n)} = \frac{1}{N_c} \sum_{i=1}^{N_c} \frac{1}{|\mathcal{B}_{i}^{(n)}|}\sum_{\bm{v} \in \mathcal{B}_{i}^{(n)}} \frac{1}{L_b}\lVert \bm{b}- B_i^{(n)}(\bm{b})\rVert^2_2
\end{align*}

At the end of iteration $n$, the codebooks are updated from $\mathcal{C}^{(n-1)}$ to $\mathcal{C}^{(n)}$. However, the mapping of a given vector $\bm{b}_j$ to quantizers $\mathcal{C}^{(n)}$ remains as  $f^{(n)}(\bm{b}_j)$. At the next iteration, during the vector clustering step, $f^{(n+1)}(\bm{b}_j)$ finds new mapping of $\bm{b}_j$ to updated codebooks $\mathcal{C}^{(n)}$ such that the quantization MSE over the candidate codebooks is minimized. Therefore, we obtain the following result for $\bm{b}_j$:
\begin{align*}
\frac{1}{L_b}\lVert \bm{b}_j - C_{f^{(n+1)}(\bm{b}_j)}^{(n)}(\bm{b}_j)\rVert^2_2 \le \frac{1}{L_b}\lVert \bm{b}_j - C_{f^{(n)}(\bm{b}_j)}^{(n)}(\bm{b}_j)\rVert^2_2
\end{align*}

That is, quantizing $\bm{b}_j$ at the end of the block clustering step of iteration $n+1$ results in lower quantization MSE compared to quantizing at the end of iteration $n$. Since this is true for all $\bm{b} \in \bm{X}$, we assert the following:
\begin{equation}
\begin{split}
\label{eq:mse_ineq_1}
    \tilde{J}^{(n+1)} &= \frac{1}{N_c} \sum_{i=1}^{N_c} \frac{1}{|\mathcal{B}_{i}^{(n+1)}|}\sum_{\bm{b} \in \mathcal{B}_{i}^{(n+1)}} \frac{1}{L_b}\lVert \bm{b} - C_i^{(n)}(b)\rVert^2_2 \le J^{(n)}
\end{split}
\end{equation}
where $\tilde{J}^{(n+1)}$ is the the quantization MSE after the vector clustering step at iteration $n+1$.

Next, during the codebook update step (\ref{eq:quantizers_update}) at iteration $n+1$, the per-cluster codebooks $\mathcal{C}^{(n)}$ are updated to $\mathcal{C}^{(n+1)}$ by invoking the Lloyd-Max algorithm \citep{Lloyd}. We know that for any given value distribution, the Lloyd-Max algorithm minimizes the quantization MSE. Therefore, for a given vector cluster $\mathcal{B}_i$ we obtain the following result:

\begin{equation}
    \frac{1}{|\mathcal{B}_{i}^{(n+1)}|}\sum_{\bm{b} \in \mathcal{B}_{i}^{(n+1)}} \frac{1}{L_b}\lVert \bm{b}- C_i^{(n+1)}(\bm{b})\rVert^2_2 \le \frac{1}{|\mathcal{B}_{i}^{(n+1)}|}\sum_{\bm{b} \in \mathcal{B}_{i}^{(n+1)}} \frac{1}{L_b}\lVert \bm{b}- C_i^{(n)}(\bm{b})\rVert^2_2
\end{equation}

The above equation states that quantizing the given block cluster $\mathcal{B}_i$ after updating the associated codebook from $C_i^{(n)}$ to $C_i^{(n+1)}$ results in lower quantization MSE. Since this is true for all the block clusters, we derive the following result: 
\begin{equation}
\begin{split}
\label{eq:mse_ineq_2}
     J^{(n+1)} &= \frac{1}{N_c} \sum_{i=1}^{N_c} \frac{1}{|\mathcal{B}_{i}^{(n+1)}|}\sum_{\bm{b} \in \mathcal{B}_{i}^{(n+1)}} \frac{1}{L_b}\lVert \bm{b}- C_i^{(n+1)}(\bm{b})\rVert^2_2  \le \tilde{J}^{(n+1)}   
\end{split}
\end{equation}

Following (\ref{eq:mse_ineq_1}) and (\ref{eq:mse_ineq_2}), we find that the quantization MSE is non-increasing for each iteration, that is, $J^{(1)} \ge J^{(2)} \ge J^{(3)} \ge \ldots \ge J^{(M)}$ where $M$ is the maximum number of iterations. 
%Therefore, we can say that if the algorithm converges, then it must be that it has converged to a local minimum. 
\hfill $\blacksquare$


\begin{figure}
    \begin{center}
    \includegraphics[width=0.5\textwidth]{sections//figures/mse_vs_iter.pdf}
    \end{center}
    \caption{\small NMSE vs iterations during LO-BCQ compared to other block quantization proposals}
    \label{fig:nmse_vs_iter}
\end{figure}

Figure \ref{fig:nmse_vs_iter} shows the empirical convergence of LO-BCQ across several block lengths and number of codebooks. Also, the MSE achieved by LO-BCQ is compared to baselines such as MXFP and VSQ. As shown, LO-BCQ converges to a lower MSE than the baselines. Further, we achieve better convergence for larger number of codebooks ($N_c$) and for a smaller block length ($L_b$), both of which increase the bitwidth of BCQ (see Eq \ref{eq:bitwidth_bcq}).


\subsection{Additional Accuracy Results}
%Table \ref{tab:lobcq_config} lists the various LOBCQ configurations and their corresponding bitwidths.
\begin{table}
\setlength{\tabcolsep}{4.75pt}
\begin{center}
\caption{\label{tab:lobcq_config} Various LO-BCQ configurations and their bitwidths.}
\begin{tabular}{|c||c|c|c|c||c|c||c|} 
\hline
 & \multicolumn{4}{|c||}{$L_b=8$} & \multicolumn{2}{|c||}{$L_b=4$} & $L_b=2$ \\
 \hline
 \backslashbox{$L_A$\kern-1em}{\kern-1em$N_c$} & 2 & 4 & 8 & 16 & 2 & 4 & 2 \\
 \hline
 64 & 4.25 & 4.375 & 4.5 & 4.625 & 4.375 & 4.625 & 4.625\\
 \hline
 32 & 4.375 & 4.5 & 4.625& 4.75 & 4.5 & 4.75 & 4.75 \\
 \hline
 16 & 4.625 & 4.75& 4.875 & 5 & 4.75 & 5 & 5 \\
 \hline
\end{tabular}
\end{center}
\end{table}

%\subsection{Perplexity achieved by various LO-BCQ configurations on Wikitext-103 dataset}

\begin{table} \centering
\begin{tabular}{|c||c|c|c|c||c|c||c|} 
\hline
 $L_b \rightarrow$& \multicolumn{4}{c||}{8} & \multicolumn{2}{c||}{4} & 2\\
 \hline
 \backslashbox{$L_A$\kern-1em}{\kern-1em$N_c$} & 2 & 4 & 8 & 16 & 2 & 4 & 2  \\
 %$N_c \rightarrow$ & 2 & 4 & 8 & 16 & 2 & 4 & 2 \\
 \hline
 \hline
 \multicolumn{8}{c}{GPT3-1.3B (FP32 PPL = 9.98)} \\ 
 \hline
 \hline
 64 & 10.40 & 10.23 & 10.17 & 10.15 &  10.28 & 10.18 & 10.19 \\
 \hline
 32 & 10.25 & 10.20 & 10.15 & 10.12 &  10.23 & 10.17 & 10.17 \\
 \hline
 16 & 10.22 & 10.16 & 10.10 & 10.09 &  10.21 & 10.14 & 10.16 \\
 \hline
  \hline
 \multicolumn{8}{c}{GPT3-8B (FP32 PPL = 7.38)} \\ 
 \hline
 \hline
 64 & 7.61 & 7.52 & 7.48 &  7.47 &  7.55 &  7.49 & 7.50 \\
 \hline
 32 & 7.52 & 7.50 & 7.46 &  7.45 &  7.52 &  7.48 & 7.48  \\
 \hline
 16 & 7.51 & 7.48 & 7.44 &  7.44 &  7.51 &  7.49 & 7.47  \\
 \hline
\end{tabular}
\caption{\label{tab:ppl_gpt3_abalation} Wikitext-103 perplexity across GPT3-1.3B and 8B models.}
\end{table}

\begin{table} \centering
\begin{tabular}{|c||c|c|c|c||} 
\hline
 $L_b \rightarrow$& \multicolumn{4}{c||}{8}\\
 \hline
 \backslashbox{$L_A$\kern-1em}{\kern-1em$N_c$} & 2 & 4 & 8 & 16 \\
 %$N_c \rightarrow$ & 2 & 4 & 8 & 16 & 2 & 4 & 2 \\
 \hline
 \hline
 \multicolumn{5}{|c|}{Llama2-7B (FP32 PPL = 5.06)} \\ 
 \hline
 \hline
 64 & 5.31 & 5.26 & 5.19 & 5.18  \\
 \hline
 32 & 5.23 & 5.25 & 5.18 & 5.15  \\
 \hline
 16 & 5.23 & 5.19 & 5.16 & 5.14  \\
 \hline
 \multicolumn{5}{|c|}{Nemotron4-15B (FP32 PPL = 5.87)} \\ 
 \hline
 \hline
 64  & 6.3 & 6.20 & 6.13 & 6.08  \\
 \hline
 32  & 6.24 & 6.12 & 6.07 & 6.03  \\
 \hline
 16  & 6.12 & 6.14 & 6.04 & 6.02  \\
 \hline
 \multicolumn{5}{|c|}{Nemotron4-340B (FP32 PPL = 3.48)} \\ 
 \hline
 \hline
 64 & 3.67 & 3.62 & 3.60 & 3.59 \\
 \hline
 32 & 3.63 & 3.61 & 3.59 & 3.56 \\
 \hline
 16 & 3.61 & 3.58 & 3.57 & 3.55 \\
 \hline
\end{tabular}
\caption{\label{tab:ppl_llama7B_nemo15B} Wikitext-103 perplexity compared to FP32 baseline in Llama2-7B and Nemotron4-15B, 340B models}
\end{table}

%\subsection{Perplexity achieved by various LO-BCQ configurations on MMLU dataset}


\begin{table} \centering
\begin{tabular}{|c||c|c|c|c||c|c|c|c|} 
\hline
 $L_b \rightarrow$& \multicolumn{4}{c||}{8} & \multicolumn{4}{c||}{8}\\
 \hline
 \backslashbox{$L_A$\kern-1em}{\kern-1em$N_c$} & 2 & 4 & 8 & 16 & 2 & 4 & 8 & 16  \\
 %$N_c \rightarrow$ & 2 & 4 & 8 & 16 & 2 & 4 & 2 \\
 \hline
 \hline
 \multicolumn{5}{|c|}{Llama2-7B (FP32 Accuracy = 45.8\%)} & \multicolumn{4}{|c|}{Llama2-70B (FP32 Accuracy = 69.12\%)} \\ 
 \hline
 \hline
 64 & 43.9 & 43.4 & 43.9 & 44.9 & 68.07 & 68.27 & 68.17 & 68.75 \\
 \hline
 32 & 44.5 & 43.8 & 44.9 & 44.5 & 68.37 & 68.51 & 68.35 & 68.27  \\
 \hline
 16 & 43.9 & 42.7 & 44.9 & 45 & 68.12 & 68.77 & 68.31 & 68.59  \\
 \hline
 \hline
 \multicolumn{5}{|c|}{GPT3-22B (FP32 Accuracy = 38.75\%)} & \multicolumn{4}{|c|}{Nemotron4-15B (FP32 Accuracy = 64.3\%)} \\ 
 \hline
 \hline
 64 & 36.71 & 38.85 & 38.13 & 38.92 & 63.17 & 62.36 & 63.72 & 64.09 \\
 \hline
 32 & 37.95 & 38.69 & 39.45 & 38.34 & 64.05 & 62.30 & 63.8 & 64.33  \\
 \hline
 16 & 38.88 & 38.80 & 38.31 & 38.92 & 63.22 & 63.51 & 63.93 & 64.43  \\
 \hline
\end{tabular}
\caption{\label{tab:mmlu_abalation} Accuracy on MMLU dataset across GPT3-22B, Llama2-7B, 70B and Nemotron4-15B models.}
\end{table}


%\subsection{Perplexity achieved by various LO-BCQ configurations on LM evaluation harness}

\begin{table} \centering
\begin{tabular}{|c||c|c|c|c||c|c|c|c|} 
\hline
 $L_b \rightarrow$& \multicolumn{4}{c||}{8} & \multicolumn{4}{c||}{8}\\
 \hline
 \backslashbox{$L_A$\kern-1em}{\kern-1em$N_c$} & 2 & 4 & 8 & 16 & 2 & 4 & 8 & 16  \\
 %$N_c \rightarrow$ & 2 & 4 & 8 & 16 & 2 & 4 & 2 \\
 \hline
 \hline
 \multicolumn{5}{|c|}{Race (FP32 Accuracy = 37.51\%)} & \multicolumn{4}{|c|}{Boolq (FP32 Accuracy = 64.62\%)} \\ 
 \hline
 \hline
 64 & 36.94 & 37.13 & 36.27 & 37.13 & 63.73 & 62.26 & 63.49 & 63.36 \\
 \hline
 32 & 37.03 & 36.36 & 36.08 & 37.03 & 62.54 & 63.51 & 63.49 & 63.55  \\
 \hline
 16 & 37.03 & 37.03 & 36.46 & 37.03 & 61.1 & 63.79 & 63.58 & 63.33  \\
 \hline
 \hline
 \multicolumn{5}{|c|}{Winogrande (FP32 Accuracy = 58.01\%)} & \multicolumn{4}{|c|}{Piqa (FP32 Accuracy = 74.21\%)} \\ 
 \hline
 \hline
 64 & 58.17 & 57.22 & 57.85 & 58.33 & 73.01 & 73.07 & 73.07 & 72.80 \\
 \hline
 32 & 59.12 & 58.09 & 57.85 & 58.41 & 73.01 & 73.94 & 72.74 & 73.18  \\
 \hline
 16 & 57.93 & 58.88 & 57.93 & 58.56 & 73.94 & 72.80 & 73.01 & 73.94  \\
 \hline
\end{tabular}
\caption{\label{tab:mmlu_abalation} Accuracy on LM evaluation harness tasks on GPT3-1.3B model.}
\end{table}

\begin{table} \centering
\begin{tabular}{|c||c|c|c|c||c|c|c|c|} 
\hline
 $L_b \rightarrow$& \multicolumn{4}{c||}{8} & \multicolumn{4}{c||}{8}\\
 \hline
 \backslashbox{$L_A$\kern-1em}{\kern-1em$N_c$} & 2 & 4 & 8 & 16 & 2 & 4 & 8 & 16  \\
 %$N_c \rightarrow$ & 2 & 4 & 8 & 16 & 2 & 4 & 2 \\
 \hline
 \hline
 \multicolumn{5}{|c|}{Race (FP32 Accuracy = 41.34\%)} & \multicolumn{4}{|c|}{Boolq (FP32 Accuracy = 68.32\%)} \\ 
 \hline
 \hline
 64 & 40.48 & 40.10 & 39.43 & 39.90 & 69.20 & 68.41 & 69.45 & 68.56 \\
 \hline
 32 & 39.52 & 39.52 & 40.77 & 39.62 & 68.32 & 67.43 & 68.17 & 69.30  \\
 \hline
 16 & 39.81 & 39.71 & 39.90 & 40.38 & 68.10 & 66.33 & 69.51 & 69.42  \\
 \hline
 \hline
 \multicolumn{5}{|c|}{Winogrande (FP32 Accuracy = 67.88\%)} & \multicolumn{4}{|c|}{Piqa (FP32 Accuracy = 78.78\%)} \\ 
 \hline
 \hline
 64 & 66.85 & 66.61 & 67.72 & 67.88 & 77.31 & 77.42 & 77.75 & 77.64 \\
 \hline
 32 & 67.25 & 67.72 & 67.72 & 67.00 & 77.31 & 77.04 & 77.80 & 77.37  \\
 \hline
 16 & 68.11 & 68.90 & 67.88 & 67.48 & 77.37 & 78.13 & 78.13 & 77.69  \\
 \hline
\end{tabular}
\caption{\label{tab:mmlu_abalation} Accuracy on LM evaluation harness tasks on GPT3-8B model.}
\end{table}

\begin{table} \centering
\begin{tabular}{|c||c|c|c|c||c|c|c|c|} 
\hline
 $L_b \rightarrow$& \multicolumn{4}{c||}{8} & \multicolumn{4}{c||}{8}\\
 \hline
 \backslashbox{$L_A$\kern-1em}{\kern-1em$N_c$} & 2 & 4 & 8 & 16 & 2 & 4 & 8 & 16  \\
 %$N_c \rightarrow$ & 2 & 4 & 8 & 16 & 2 & 4 & 2 \\
 \hline
 \hline
 \multicolumn{5}{|c|}{Race (FP32 Accuracy = 40.67\%)} & \multicolumn{4}{|c|}{Boolq (FP32 Accuracy = 76.54\%)} \\ 
 \hline
 \hline
 64 & 40.48 & 40.10 & 39.43 & 39.90 & 75.41 & 75.11 & 77.09 & 75.66 \\
 \hline
 32 & 39.52 & 39.52 & 40.77 & 39.62 & 76.02 & 76.02 & 75.96 & 75.35  \\
 \hline
 16 & 39.81 & 39.71 & 39.90 & 40.38 & 75.05 & 73.82 & 75.72 & 76.09  \\
 \hline
 \hline
 \multicolumn{5}{|c|}{Winogrande (FP32 Accuracy = 70.64\%)} & \multicolumn{4}{|c|}{Piqa (FP32 Accuracy = 79.16\%)} \\ 
 \hline
 \hline
 64 & 69.14 & 70.17 & 70.17 & 70.56 & 78.24 & 79.00 & 78.62 & 78.73 \\
 \hline
 32 & 70.96 & 69.69 & 71.27 & 69.30 & 78.56 & 79.49 & 79.16 & 78.89  \\
 \hline
 16 & 71.03 & 69.53 & 69.69 & 70.40 & 78.13 & 79.16 & 79.00 & 79.00  \\
 \hline
\end{tabular}
\caption{\label{tab:mmlu_abalation} Accuracy on LM evaluation harness tasks on GPT3-22B model.}
\end{table}

\begin{table} \centering
\begin{tabular}{|c||c|c|c|c||c|c|c|c|} 
\hline
 $L_b \rightarrow$& \multicolumn{4}{c||}{8} & \multicolumn{4}{c||}{8}\\
 \hline
 \backslashbox{$L_A$\kern-1em}{\kern-1em$N_c$} & 2 & 4 & 8 & 16 & 2 & 4 & 8 & 16  \\
 %$N_c \rightarrow$ & 2 & 4 & 8 & 16 & 2 & 4 & 2 \\
 \hline
 \hline
 \multicolumn{5}{|c|}{Race (FP32 Accuracy = 44.4\%)} & \multicolumn{4}{|c|}{Boolq (FP32 Accuracy = 79.29\%)} \\ 
 \hline
 \hline
 64 & 42.49 & 42.51 & 42.58 & 43.45 & 77.58 & 77.37 & 77.43 & 78.1 \\
 \hline
 32 & 43.35 & 42.49 & 43.64 & 43.73 & 77.86 & 75.32 & 77.28 & 77.86  \\
 \hline
 16 & 44.21 & 44.21 & 43.64 & 42.97 & 78.65 & 77 & 76.94 & 77.98  \\
 \hline
 \hline
 \multicolumn{5}{|c|}{Winogrande (FP32 Accuracy = 69.38\%)} & \multicolumn{4}{|c|}{Piqa (FP32 Accuracy = 78.07\%)} \\ 
 \hline
 \hline
 64 & 68.9 & 68.43 & 69.77 & 68.19 & 77.09 & 76.82 & 77.09 & 77.86 \\
 \hline
 32 & 69.38 & 68.51 & 68.82 & 68.90 & 78.07 & 76.71 & 78.07 & 77.86  \\
 \hline
 16 & 69.53 & 67.09 & 69.38 & 68.90 & 77.37 & 77.8 & 77.91 & 77.69  \\
 \hline
\end{tabular}
\caption{\label{tab:mmlu_abalation} Accuracy on LM evaluation harness tasks on Llama2-7B model.}
\end{table}

\begin{table} \centering
\begin{tabular}{|c||c|c|c|c||c|c|c|c|} 
\hline
 $L_b \rightarrow$& \multicolumn{4}{c||}{8} & \multicolumn{4}{c||}{8}\\
 \hline
 \backslashbox{$L_A$\kern-1em}{\kern-1em$N_c$} & 2 & 4 & 8 & 16 & 2 & 4 & 8 & 16  \\
 %$N_c \rightarrow$ & 2 & 4 & 8 & 16 & 2 & 4 & 2 \\
 \hline
 \hline
 \multicolumn{5}{|c|}{Race (FP32 Accuracy = 48.8\%)} & \multicolumn{4}{|c|}{Boolq (FP32 Accuracy = 85.23\%)} \\ 
 \hline
 \hline
 64 & 49.00 & 49.00 & 49.28 & 48.71 & 82.82 & 84.28 & 84.03 & 84.25 \\
 \hline
 32 & 49.57 & 48.52 & 48.33 & 49.28 & 83.85 & 84.46 & 84.31 & 84.93  \\
 \hline
 16 & 49.85 & 49.09 & 49.28 & 48.99 & 85.11 & 84.46 & 84.61 & 83.94  \\
 \hline
 \hline
 \multicolumn{5}{|c|}{Winogrande (FP32 Accuracy = 79.95\%)} & \multicolumn{4}{|c|}{Piqa (FP32 Accuracy = 81.56\%)} \\ 
 \hline
 \hline
 64 & 78.77 & 78.45 & 78.37 & 79.16 & 81.45 & 80.69 & 81.45 & 81.5 \\
 \hline
 32 & 78.45 & 79.01 & 78.69 & 80.66 & 81.56 & 80.58 & 81.18 & 81.34  \\
 \hline
 16 & 79.95 & 79.56 & 79.79 & 79.72 & 81.28 & 81.66 & 81.28 & 80.96  \\
 \hline
\end{tabular}
\caption{\label{tab:mmlu_abalation} Accuracy on LM evaluation harness tasks on Llama2-70B model.}
\end{table}

%\section{MSE Studies}
%\textcolor{red}{TODO}


\subsection{Number Formats and Quantization Method}
\label{subsec:numFormats_quantMethod}
\subsubsection{Integer Format}
An $n$-bit signed integer (INT) is typically represented with a 2s-complement format \citep{yao2022zeroquant,xiao2023smoothquant,dai2021vsq}, where the most significant bit denotes the sign.

\subsubsection{Floating Point Format}
An $n$-bit signed floating point (FP) number $x$ comprises of a 1-bit sign ($x_{\mathrm{sign}}$), $B_m$-bit mantissa ($x_{\mathrm{mant}}$) and $B_e$-bit exponent ($x_{\mathrm{exp}}$) such that $B_m+B_e=n-1$. The associated constant exponent bias ($E_{\mathrm{bias}}$) is computed as $(2^{{B_e}-1}-1)$. We denote this format as $E_{B_e}M_{B_m}$.  

\subsubsection{Quantization Scheme}
\label{subsec:quant_method}
A quantization scheme dictates how a given unquantized tensor is converted to its quantized representation. We consider FP formats for the purpose of illustration. Given an unquantized tensor $\bm{X}$ and an FP format $E_{B_e}M_{B_m}$, we first, we compute the quantization scale factor $s_X$ that maps the maximum absolute value of $\bm{X}$ to the maximum quantization level of the $E_{B_e}M_{B_m}$ format as follows:
\begin{align}
\label{eq:sf}
    s_X = \frac{\mathrm{max}(|\bm{X}|)}{\mathrm{max}(E_{B_e}M_{B_m})}
\end{align}
In the above equation, $|\cdot|$ denotes the absolute value function.

Next, we scale $\bm{X}$ by $s_X$ and quantize it to $\hat{\bm{X}}$ by rounding it to the nearest quantization level of $E_{B_e}M_{B_m}$ as:

\begin{align}
\label{eq:tensor_quant}
    \hat{\bm{X}} = \text{round-to-nearest}\left(\frac{\bm{X}}{s_X}, E_{B_e}M_{B_m}\right)
\end{align}

We perform dynamic max-scaled quantization \citep{wu2020integer}, where the scale factor $s$ for activations is dynamically computed during runtime.

\subsection{Vector Scaled Quantization}
\begin{wrapfigure}{r}{0.35\linewidth}
  \centering
  \includegraphics[width=\linewidth]{sections/figures/vsquant.jpg}
  \caption{\small Vectorwise decomposition for per-vector scaled quantization (VSQ \citep{dai2021vsq}).}
  \label{fig:vsquant}
\end{wrapfigure}
During VSQ \citep{dai2021vsq}, the operand tensors are decomposed into 1D vectors in a hardware friendly manner as shown in Figure \ref{fig:vsquant}. Since the decomposed tensors are used as operands in matrix multiplications during inference, it is beneficial to perform this decomposition along the reduction dimension of the multiplication. The vectorwise quantization is performed similar to tensorwise quantization described in Equations \ref{eq:sf} and \ref{eq:tensor_quant}, where a scale factor $s_v$ is required for each vector $\bm{v}$ that maps the maximum absolute value of that vector to the maximum quantization level. While smaller vector lengths can lead to larger accuracy gains, the associated memory and computational overheads due to the per-vector scale factors increases. To alleviate these overheads, VSQ \citep{dai2021vsq} proposed a second level quantization of the per-vector scale factors to unsigned integers, while MX \citep{rouhani2023shared} quantizes them to integer powers of 2 (denoted as $2^{INT}$).

\subsubsection{MX Format}
The MX format proposed in \citep{rouhani2023microscaling} introduces the concept of sub-block shifting. For every two scalar elements of $b$-bits each, there is a shared exponent bit. The value of this exponent bit is determined through an empirical analysis that targets minimizing quantization MSE. We note that the FP format $E_{1}M_{b}$ is strictly better than MX from an accuracy perspective since it allocates a dedicated exponent bit to each scalar as opposed to sharing it across two scalars. Therefore, we conservatively bound the accuracy of a $b+2$-bit signed MX format with that of a $E_{1}M_{b}$ format in our comparisons. For instance, we use E1M2 format as a proxy for MX4.

\begin{figure}
    \centering
    \includegraphics[width=1\linewidth]{sections//figures/BlockFormats.pdf}
    \caption{\small Comparing LO-BCQ to MX format.}
    \label{fig:block_formats}
\end{figure}

Figure \ref{fig:block_formats} compares our $4$-bit LO-BCQ block format to MX \citep{rouhani2023microscaling}. As shown, both LO-BCQ and MX decompose a given operand tensor into block arrays and each block array into blocks. Similar to MX, we find that per-block quantization ($L_b < L_A$) leads to better accuracy due to increased flexibility. While MX achieves this through per-block $1$-bit micro-scales, we associate a dedicated codebook to each block through a per-block codebook selector. Further, MX quantizes the per-block array scale-factor to E8M0 format without per-tensor scaling. In contrast during LO-BCQ, we find that per-tensor scaling combined with quantization of per-block array scale-factor to E4M3 format results in superior inference accuracy across models. 




\end{document}