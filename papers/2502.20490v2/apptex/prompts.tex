\section{Prompts}
\label{appendix:prompts}

\subsection{Prompts for Evaluation Subtask}
\label{appendix:prompts_evaluation}
\begin{tcolorbox}[colframe=green!50!black, colback=gray!10, title=Action Selection Prompt (Subtask 1), breakable, fontupper=\small]
\{video\_frames\} \\

The following images from a first-person perspective video depict a person in a given situation. \\

Given the below behaviors, choose the single most normatively relevant or appropriate action to perform next, based on the context you are given and commonsense norms. \\

Structure your answer as a single integer, where the integer corresponds to the index of the behavior in the list below. \\

None is a valid option, and should be output as the index of the empty string in the behaviors if no behavior is appropriate. \\

DO NOT WRITE ANY OTHER TEXT IN YOUR RESPONSE, JUST A SINGLE INTEGER OF THE INDEX OF THE BEHAVIOR, ZERO-INDEXED. \\

Reason over your response in the output. \\

Behavior options: \{behavior\_options\} \\

Response example: \\
3
\end{tcolorbox}

\begin{tcolorbox}[colframe=green!50!black, colback=gray!10, title=Justification Selection Prompt (Subtask 2), breakable, fontupper=\small]
\{video\_frames\} \\

The following images from a first-person perspective video depict a person performing some action. \\

You selected \{behavior\} as the most normatively relevant or appropriate action for this person to perform in the given situation. \\

Your task is to now choose the most normatively appropriate justification that best supports your behavior, based on the context and commonsense norms. \\

Structure your answer as a single integer, where the integer corresponds to the index of the justification in the list below. \\

None is a valid option, and should be output as the index of the empty string in the behaviors if no behavior is appropriate. \\

DO NOT OUTPUT ANY OTHER TEXT IN YOUR RESPONSE, JUST A SINGLE INTEGER OF THE INDEX OF THE JUSTIFICATION, ZERO-INDEXED. \\

Reason over your response in the output. \\

Justification options: \{justification\_options\} \\

Response example: \\
2
\end{tcolorbox}

\begin{tcolorbox}[colframe=green!50!black, colback=gray!10, title=Sensible Actions Selection Prompt, breakable, fontupper=\small]
\{video\_frames\} \\

The following images from a first-person perspective video depict a person in a given situation. \\

Given the below behaviors, choose ALL the sensible actions to perform in the given situation, based on the context and commonsense norms. \\
None is a valid option, and provided. \\

Do not pattern match words, instead consider the context and norms. \\

Structure your answer as one python list of integers, where each integer corresponds to the indices of the behaviors in the list below. An empty list is acceptable if no behavior is sensible. \\

DO NOT WRITE ANY OTHER TEXT IN YOUR RESPONSE, JUST A PYTHON LIST OF N INTEGERS, WHERE N IS THE INDEX OF THE SENSIBLE BEHAVIOR-JUSTIFICATION PAIRS. \\

Reason over your response in the output. \\

Behavior options: \{behavior\_options\} \\

Response example: \\
$[0, 1, 2]$
\end{tcolorbox}

\subsection{Prompts for MCQ Generation}
\label{appendix:prompts_mcq}
% \newtcblisting{qwr}{
%   listing only,
%   fit,
%   blank,
%   nobeforeafter,
%   boxrule=0pt,
%   colback=white,
%   parskip=0pt,
%   left=0pt, right=0pt,
%   breakable
% }
\begin{tcolorbox}[colframe=green!50!black, colback=gray!10, title=MCQ Generation Prompt, breakable, fontupper=\small]
% \tiny
You are a teacher, teaching LLMs how to use the information in a context description to choose from a set of actions.\\
Your job is to generate a set of actions a person [the subject] can take in a given situation, with one being the most correct and socially normative action, and the others being incorrect or non-normative. However, it should be impossible to determine the correct action without the context.\\

You are given a base context \{base\}. Interpret the higher-level activity in the context generally, ignore specific tools or details of the context.\\

\textbf{Task instructions:}\\
Generate one action that is the most normative and correct action for the subject to perform in the base context. This is the Base Action, and should be the first action in the list.\\

Next, change the base context to three new contexts. Each new context should assume additional details not stated in the base context, and use these to make the rules of the situation different.\\
In each context, change the: 
\begin{itemize}
    \item Intention and situation (modify situation widely)
    \item Role of the subject and other people
    \item Interpret a noun as a co-hyponym
    \item Change the state of the overall activity (previous/future step)
    \item (Only in the last context) Make the role and intention of the subject the opposite of the base context
    \item (Sometimes) Interpret the cooperative intention (make the task a two-person task, or make the other person do the action)
    \item (Sometimes) The culture or context of the action
    \item (Sometimes) add a modifier to the object, environment, or subject (heaviness, disability, weather conditions, etc)
\end{itemize}

From each new context, generate an alternative action for the subject to do next that is normative in the new context, but non-normative in the base context.\\

This new action does not need to be the Target Intention, but can be another related action that's normative in the new context, but would be out of place in the base context.\\

Ensure the generated actions are mechanically different from the target action, and that the correct action is only determinable with the context. The actions should not simply differ by the nouns or verbs used, but by the overall interpretation of the action.\\

Avoid any answers that include prison, secret agents, kidnapping, or competitions.\\

\textbf{Action Guidelines:}\\
\begin{itemize}
    \item The actions should be strongly distinct from each other.
    \item Do not leak information about the context; actions should be written in general language.
    \item Most importantly: Avoid adverbs and words describing emotion or manner, and sentences should start in a verb, not "the subject." Avoid any actions that include 'ignore'.
    \item The actions should be not be negative or harmful, nor refer to any violent activity, even if lawful.
    \item Actions must use imperative sentences describing the subject's interaction with a person or object.
    \item Use the neutral term "person" when referring to other individuals, avoiding any descriptors of age, gender, or other characteristics.
    \item All actions should be of the same length and complexity, and should be of roughly equal length to the base action.
\end{itemize}

\textbf{Output the following JSON structure, without any additional content:}\\
\{ 

\quad"Contexts": ["Base Context", "Context 2", "Context 3", "Context 4"],
  
\quad"Actions": ["Base Action", "Action 2", "Action 3", "Action 4"]

\} \\
Below is an example of an output if the base context is "Subject is a pet owner, walking dog on a sunny day next to a road".

It interprets the general activity is "walking a pet".\\

\textbf{Example:}\\

% \begin{verbatim}
\{

\quad"Contexts": [
    
\quad\quad"Subject is a pet owner, walking dog on a sunny day next to a road.",

\quad\quad"Subject is a dog trainer, dog is a stray.",
    
\quad\quad"Subject is a person, dog is a pocket dog, navigating a muddy field and want to avoid getting dog dirty.",

\quad\quad"Subject is a blind person, dog is a guide dog, and they are navigating a crowded city street."

\quad],

\quad"Actions": [
    
\quad\quad"Guide the dog along a sidewalk using a leash.",

\quad\quad"Call the dog to follow you, using a treat, and guide it to a shelter.",
    
\quad\quad"Carry the dog across the muddy field, shielding it from dirt.",
    
\quad\quad"Let the dog guide you with its harness."

\quad]

\}
% \end{verbatim}
\end{tcolorbox}
% \end{center}