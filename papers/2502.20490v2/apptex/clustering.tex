\section{Activity Clustering Algorithm}
\label{appendix:clustering}

To cluster our datasets for activities, we begin by extracting video descriptions and grouping them into topics using a batch size of 100. The following prompt is employed for this initial clustering:

\begin{tcolorbox}[colframe=green!50!black, colback=gray!10, title=Topic Clustering Prompt, breakable, fontupper=\small]
% \tiny
Given these video descriptions: \\
\{video\_descriptions\} \\

Generate a list of high-level topics that these videos fall under. 
Return the response as a JSON array of strings. \\
Be specific but not too granular - aim for \{int(math.sqrt(batch\_size))\}-{batch\_size // 2} topics for this set of intents.
\end{tcolorbox}

Once topics have been generated for each batch, we aggregate and merge similar topics using the prompt below:

\begin{tcolorbox}[colframe=green!50!black, colback=gray!10, title=Topic Merging Prompt, breakable, fontupper=\small]
Given these topics: \\
\{topics\} \\

Consolidate these into a unique set of high-level topics, merging similar ones. \\
Return the response as a JSON array of strings. \\
Be specific but not too granular - aim for concise, clear topics.
\end{tcolorbox}

Finally, we assign each video a topic based on its description using the prompt below, which serves as the low-level activity label. We then repeat the process to obtain the high-level activity label.

\begin{tcolorbox}[colframe=green!50!black, colback=gray!10, title=Topic Assigning Prompt, breakable, fontupper=\small]

Given this video description: \\
\{video\_descriptions\} \\

And these possible topics: \\
\{topics\} \\

Choose the most appropriate topic for this video. \\
Return the chosen topic string.
\end{tcolorbox}