%% ========================================================
\newcommand\R{\mathbb{R}}
\newcommand\U{\mathbb{U}}
\newcommand\eS{\mathbb{S}} % why did we do this
\newcommand\I{\mathbb{I}}
%\newcommand\P{\mathbb{P}}
\newcommand\1{\mathbbm{1}}


%% ========================================================
%% Some useful commands. Note the dangerous \r command!
%% ========================================================
\newcommand{\bd}{\boldsymbol}
\newcommand{\mf}{\mathbf}


\renewcommand{\a}{\mathbf{a}}
\renewcommand{\c}{\mathbf{c}}
\renewcommand{\r}{\mathbf{r}} % since \r was already defined in altex
\renewcommand{\u}{\mathbf{u}}
\renewcommand{\v}{\mathbf{v}}
\newcommand{\s}{\mathbf{s}}
\newcommand{\w}{\mathbf{w}}
\newcommand{\x}{\mathbf{x}}
\newcommand{\y}{\mathbf{y}}
\newcommand{\z}{\mathbf{z}} 
\newcommand{\p}{\mathbf{p}}
\newcommand{\myb}{\mathbf{b}}
\newcommand{\q}{\mathbf{q}}



%----------------------------------------
%      THE SETS
%----------------------------------------
\newcommand{\cC}{{\cal C}}
\newcommand{\cD}{{\cal D}}
\newcommand{\cE}{{\cal E}}
\newcommand{\cU}{{\cal U}}
\newcommand{\cL}{{\cal L}}
\newcommand{\cM}{{\cal M}}
\newcommand{\cP}{{\cal P}}
\newcommand{\cQ}{{\cal Q}}
\newcommand{\cS}{{\cal S}}
\newcommand{\cT}{{\cal T}}
\newcommand{\cX}{{\cal X}}
\newcommand{\cY}{{\cal Y}}
\newcommand{\cZ}{{\cal Z}}
\newcommand{\cF}{{\cal F}}
\newcommand{\cB}{{\cal B}}
\newcommand{\cA}{{\cal A}}

%--------------------------------------------
%         THE VARIABLES & APPROXIMATES
%--------------------------------------------
\newcommand{\bM}{\mathbf{M}}
\newcommand{\bW}{\mathbf{W}}
\newcommand{\bA}{\mathbf{A}}
\newcommand{\bB}{\mathbf{B}}
\newcommand{\bR}{\mathbf{R}}
\newcommand{\bC}{\mathbf{C}}
\newcommand{\bE}{\mathbf{E}}
\newcommand{\bZ}{\mathbf{Z}}
\newcommand{\bG}{\mathbf{G}}


\newcommand{\hP}{\hat{P}}
\newcommand{\hZ}{\hat{Z}}
\newcommand{\hM}{\hat{M}}
\newcommand{\hA}{\hat{A}}
\newcommand{\hC}{\hat{C}}
\newcommand{\hU}{\hat{U}}
\newcommand{\hV}{\hat{V}}
\newcommand{\hX}{\hat{X}}
\newcommand{\hY}{\hat{Y}}
\newcommand{\hG}{\hat{G}}
\newcommand{\hS}{\hat{S}}

\newcommand{\Hl}{\hat{l}}
\newcommand{\hz}{\hat{z}}
\newcommand{\hu}{\hat{u}}
\newcommand{\hv}{\hat{v}}
\newcommand{\hx}{\hat{x}}
\newcommand{\hy}{\hat{y}}
\newcommand{\hp}{\hat{p}}
\newcommand{\Hm}{\hat{m}}

%---------MISC-----------------------------
\newcommand{\pr}{\mathbf{P}}
\newcommand{\E}{\mathbf{E}}

%\newcommand{\bm}{\boldsymbol{\mu}}
\newcommand{\xx}{\boldsymbol{\chi}}
\newcommand{\Th}{\boldsymbol{\theta}}
\newcommand{\Eta}{\boldsymbol{\eta}}

\newcommand{\norm}[2]{\ensuremath \|#1\|_{#2}}
\newcommand{\myref}[1]{(\ref{#1})}
\newcommand{\del}[2]{\frac{\partial #1}{\partial #2}}


\begin{comment}
%----------------------------------------------
%        NEW OPERATORS
%----------------------------------------------

\DeclareMathOperator{\argmax}{argmax}
\DeclareMathOperator{\argmin}{argmin}
\DeclareMathOperator{\avg}{avg}
\DeclareMathOperator{\Int}{int}
\DeclareMathOperator{\cl}{cl}
\DeclareMathOperator{\bnd}{bd}
\DeclareMathOperator{\epi}{epi}
\DeclareMathOperator{\dom}{dom}
\DeclareMathOperator{\ri}{ri}
\DeclareMathOperator{\co}{co}
\DeclareMathOperator{\sgn}{sign}
\DeclareMathOperator{\supp}{supp}

%---------------------------------------------
% EQUATION, FIGURE, TABLE NUMBERS
%---------------------------------------------

\renewcommand{\theequation}{\thesection.\arabic{equation}}
\renewcommand{\thefigure}{\thesection.\arabic{figure}}
\renewcommand{\thetable}{\thesection.\arabic{table}}


%---------------------------------------------
% THEOREMS, EXAMPLES ETC.
%---------------------------------------------

\newcounter{exampleI}
\setcounter{exampleI}{1}
\renewcommand{\theexampleI}{\arabic{exampleI}}

{\theorembodyfont{\rmfamily} \theoremstyle{plain} \newtheorem{subexampleI}{Example}[exampleI]}
\renewcommand{\thesubexampleI}{\theexampleI.\Alph{subexampleI}}

\newcounter{exampleII}
\setcounter{exampleII}{2}
\renewcommand{\theexampleII}{\arabic{exampleII}}

{\theorembodyfont{\rmfamily} \theoremstyle{plain} \newtheorem{subexampleII}{Example}[exampleII]}
\renewcommand{\thesubexampleII}{\theexampleII.\Alph{subexampleII}}

\newcounter{exampleIII}
\setcounter{exampleIII}{3}
\renewcommand{\theexampleIII}{\arabic{exampleIII}}

{\theorembodyfont{\rmfamily} \theoremstyle{plain} \newtheorem{subexampleIII}{Example}[exampleIII]}
\renewcommand{\thesubexampleIII}{\theexampleIII.\Alph{subexampleIII}}


{\theorembodyfont{\rmfamily} \newtheorem{exm}{Example}}
{\theorembodyfont{\rmfamily} \newtheorem{defn}{Definition}}
{\theorembodyfont{\rmfamily} \newtheorem{remark}{Remark}}
\newtheorem{theo}{Theorem}
\newtheorem{prop}{Proposition}
\newtheorem{lemm}{Lemma}
\newtheorem{corr}{Corollary}
\newtheorem{clm}{Claim}
\end{comment}

%\newcommand{\proof}{\noindent{\itshape Proof:}\hspace*{1em}}
\newcommand{\proofsketch}{\noindent{\itshape Proof Sketch:}\hspace*{1em}}
%\newcommand{\qed}{\nolinebreak[1]~~~\hspace*{\fill} \rule{5pt}{5pt}\vspace*{\parskip}\vspace*{1ex}}


%%%%%%%%%%%%%%%%%%%%%%%%%%%% notations %%%%%%%%%%%%%%%%%%%%%%%%%%%%%%%

% Count of LLMs ---> N
% subset of LLMs ---> K
% dataset ---> D
% D_{val} and D_{test} for validation and testset
% input output space \mathcal{X}, \mathcal{Y}$
% datapoint--->  x,y
% count of datapoints ---> n
% count of validation datapoints --->n_val
% Accuracy of ith llm ---> Acc_{i}
% uncertainty_auroc of ith llm  ---> Unc\_auroc_{i}
% combined score ---> C\_score
% individual jth model responses/output for ith datapoint ---> \hat{y_i^j}
% individual jth model uncertainty output for ith datapoint ---> u_i^j
% final output response for ith datapoint \hat{y_i}
% uncertainty of ith LLM ---> u_i
