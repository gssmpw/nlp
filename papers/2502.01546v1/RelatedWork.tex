\section{Related work}
\label{sec:related_work}
%
\subsection{Dynamic mobile manipulation control}
\label{sec:mobile_manipulation}
%
% model-based approaches and challenges, are good but require accurate object model
Model-based approaches, such as Trajectory Optimization (TO) \cite{dadiotis2022trajectory, polverini_pushing} and Model Predictive Control (MPC) \cite{sleiman_unified_mpc, minniti_mpc, mayank_articulated, dadiotis_wholebody, pankert_mpc}, are frequently used to control dynamic mobile manipulators. A major limitation is that their real-time performance requires a predefined dynamically feasible contact schedule\cite{sleiman_science_robotics}, which is computationally expensive and is typically done offline. Thus, these approaches are insufficient to achieve online contact switching. Another drawback is that they rely on a model of the robot and its environment.

% Learning based approaches
On the other hand, model-free RL-based controllers\footnote{We use the term "model-free" when a model and its derivatives are not required in the controller structure. A model may still be used for simulation.} can achieve robustness against uncertainties via domain randomization in simulation-based training. Several works have successfully used this approach to control the end-effector of mobile manipulators \cite{fu2023deep, liu2024visual, ha2024umilegs}. However, they mainly tackle the tracking problem in free space and use the resulting controller to grasp lightweight objects. Thus, it is not clear how these approaches can scale to contact-rich tasks with heavy object interaction.
%
\subsection{Non-prehensile pushing motion control}
\label{sec:nonprehensile_manipulation}

Various model-based \cite{hogan_hybrid_mpc, moura_to_planar_manmipulation, inaba_humanoid_pushing, polverini_pushing} and learning-based approaches \cite{ferrandis2023nonprehensile, peng_rl, Lin_vision-proprioception, rizzardo_push_cube} have been tailored to robot pushing behaviors. However, these methods tend to suffer from the inherent limitations mentioned in \cref{sec:mobile_manipulation}. Ferrandis~\etal~\cite{ferrandis2023nonprehensile} achieve contact switching using RL with a categorical action distribution. Despite this achievement, this approach has been limited to a fixed-base manipulator, objects with negligible mass and small size on a low-friction surface.

The works of \cite{force_push_heins, ozdamar2024pushing} propose a force and tactile feedback-based approach, respectively, for statically pushing with a mobile base robot without achieving contact switching (since the feedback signal is lost at a contact break). Moreover, they only consider moving an object to a goal position and not reorienting it. Jeon~\etal\cite{quadruped_pushing} train an RL policy for guiding a quadrupedal locomotion controller towards pushing diverse objects with the robot base. Their controller generates planar 2D actions for the base motion, and they do not consider object toppling as a possibility in their scenarios. In contrast, we evaluate our approach including cases where the object can topple (\ie object with a small xy-footprint on high-friction flooring). By generating 6D commands for the mobile base and joint commands for the arm, our controller can push to different locations on the object's surface.