\section{Introduction}
% Motivation - dynamic mobile manipulation, pushing behaviors -> the challenge
% Mobile manipulators are emerging as a pivotal technology in robotics, blending mobility and manipulation to enable dynamic environment interaction in the real world.
Moving and reorienting heavy or bulky objects along large and complex real-world pathways requires combining mobility and manipulation. This task is achievable through non-prehensile pushing actions without requiring a dedicated gripper or the need to grasp a handle on the object. In real-world scenarios, however, the object and terrain's physical properties (\eg mass, size, friction coefficient) are typically unknown and can reduce a controller's performance. Additionally, non-prehensile pushing interaction may yield relative motion between the robot and object at the contact point, \eg contact sliding or relative rotation. This motion necessitates a controller capable of reactively adapting the pushing location and direction by dynamically breaking and making contact at new locations with the object. We refer to this behavior as contact switching.

% main trends so far and what is missing
Achieving online contact switching behavior during non-prehensile pushing is challenging with model-based techniques \cite{hogan_hybrid_mpc, moura_to_planar_manmipulation} or controllers that rely solely on force/tactile feedback \cite{force_push_heins, ozdamar2024pushing}. As a result, recent works leverage Reinforcement Learning (RL) to address contact switching~\cite{ferrandis2023nonprehensile} and demonstrate notable robustness against unknown objects~\cite{quadruped_pushing}. Despite these achievements, the method in \cite{ferrandis2023nonprehensile} is limited to fixed-base manipulators pushing a lightweight, small object on a table. Jeon~\etal\cite{quadruped_pushing} achieve object pushing with the base of a mobile robot, without an arm. In both cases, the policies generate only 2D planar motion commands, which are insufficient for manipulating objects that are prone to toppling (\eg objects with a thin base, large CoM height, or high friction coefficient flooring). To address this limitation, we focus on interacting with objects by pushing them to different 3D locations on their surface, enabling more versatile and stable manipulation.
%
\begin{figure}
  \centering
  \graphicspath{{figures/}}
  \includegraphics[width=0.8\columnwidth]{figures/fig1_v2.pdf}
  \vspace{-7pt}
  \caption{Object pushing with a quadrupedal manipulator. The proposed controller learns to push unknown objects towards different goals. The motions are included in the supplementary video \href{https://youtu.be/wGAdPGVf9Ws?si=j9YNlEufzQIGlPz4}{(link)}.}
  \label{Fig:fig1}
\end{figure}

% transition paragraph and paper contribution
In this work, we present a learning-based controller for a mobile manipulator to dynamically move and reorient unknown objects using non-prehensile pushing actions. We tackle the task complexity by using a state-of-the-art constrained RL algorithm \cite{chanesane2024cat} that minimizes reward engineering efforts and can satisfy the various constraints of the task, \eg arm actuator limits, self-collisions. The policy's action space consists of cartesian commands for the base and joint-space commands for the mounted articulated arm; thus, we directly control the arm in joint space. Our proposed approach achieves robustness against unknown objects and learns online contact switching to push the object to various planar goal poses. The resulting behavior demonstrates the adaptability of the pushing location, which is crucial for avoiding object toppling. Our key contributions are:
\begin{itemize}
    \item We develop a learning-based controller for mobile manipulators to repose objects on a plane through pushing. Importantly, our approach incorporates object balance as a key task requirement, which has not been addressed in prior works.
    \item Our controller demonstrates robustness to unknown objects with different physical properties, such as mass, material, size, and shape. It autonomously handles contact switching by dynamically identifying the contact points and adjusting the push direction. Even though the policy only observes the object's pose, it adapts to its xy-footprint and lowers the push point for thinner objects to maintain stability. 
    \item We validate our approach in simulation and real-world experiments with a quadrupedal manipulator, achieving consistent success across diverse scenarios, including high-friction surfaces and thin, easily toppled objects.
\end{itemize}
