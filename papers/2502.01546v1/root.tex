%%%%%%%%%%%%%%%%%%%%%%%%%%%%%%%%%%%%%%%%%%%%%%%%%%%%%%%%%%%%%%%%%%%%%%%%%%%%%%%%
%2345678901234567890123456789012345678901234567890123456789012345678901234567890
%        1         2         3         4         5         6         7         8

\documentclass[letterpaper, 10 pt, conference]{ieeeconf}  % Comment this line out if you need a4paper

%\documentclass[a4paper, 10pt, conference]{ieeeconf}      % Use this line for a4 paper
\IEEEoverridecommandlockouts    % This command is only needed if you want to use the \thanks command
\overrideIEEEmargins                                      % Needed to meet printer requirements.

%In case you encounter the following error:
%Error 1010 The PDF file may be corrupt (unable to open PDF file) OR
%Error 1000 An error occurred while parsing a contents stream. Unable to analyze the PDF file.
%This is a known problem with pdfLaTeX conversion filter. The file cannot be opened with acrobat reader
%Please use one of the alternatives below to circumvent this error by uncommenting one or the other
%\pdfobjcompresslevel=0
%\pdfminorversion=4

% See the \addtolength command later in the file to balance the column lengths
% on the last page of the document

% The following packages can be found on http:\\www.ctan.org
%\usepackage{graphics} % for pdf, bitmapped graphics files
%\usepackage{epsfig} % for postscript graphics files
%\usepackage{mathptmx} % assumes new font selection scheme installed
%\usepackage{times} % assumes new font selection scheme installed
%\usepackage{amsmath} % assumes amsmath package installed
%\usepackage{amssymb}  % assumes amsmath package installed

% my additions
\usepackage{hyperref}
\usepackage{hhline}
\usepackage{soul}
\usepackage{xcolor}
\usepackage{cite}
\usepackage{import}
\usepackage{amsmath}
\usepackage{graphicx}
\usepackage{amsfonts} 
\usepackage{bm}
\usepackage[capitalise]{cleveref}
%\graphicspath{./figures/}
\usepackage{booktabs}
\usepackage{authblk}

\DeclareMathOperator*{\argmax}{arg\,max}
\DeclareMathOperator*{\argmin}{arg\,min}

\newcommand{\etal}{\textit{et al}. }
\newcommand{\ie}{\textit{i}.\textit{e}., }
\newcommand{\eg}{\textit{e}.\textit{g}. }
\newcommand{\secref}[1]{Sec.~\ref{#1}}
% \newcommand{\eqtref}[1]{Eq.~\ref{#1}}
% \newcommand{\figref}[1]{Fig.~\ref{#1}}
% \newcommand{\tabref}[1]{Table~\ref{#1}}
% \newcommand{\appref}[1]{App.~\ref{#1}}

\title{\LARGE \bf
Dynamic object goal pushing with mobile manipulators through model-free constrained reinforcement learning
}

%
\author{Ioannis Dadiotis$^{1,2}$, Mayank Mittal$^{3,4}$, Nikos Tsagarakis$^{1}$, Marco Hutter$^{3}$

\thanks{\hspace{-12pt} This project has received funding from the European Union’s Horizon Europe Framework Programme under grant agreement No 101070596 (euROBIN) and an ETH RobotX research grant funded through the ETH Zurich Foundation. Contact: {\tt\small ioannis.dadiotis@iit.it}}
\thanks{$^{1}$HHCM lab, IIT, Genoa 16163, Italy}
\thanks{$^{2}$DIBRIS, University of Genoa, Genoa 16145, Italy.}
\thanks{$^{3}$RSL, ETH Z\"{u}rich, Z\"{u}rich 8092, Switzerland. $^{4}$NVIDIA.}
}

% *** IEEE Copyright notice with TikZ ***

\usepackage{tikz}
\usepackage{textcomp}
\usepackage{lipsum}
\usepackage{outlines}

\usepackage{multirow}   % table with subcolumns

\newcommand\copyrighttext{%
  \scriptsize \textcopyright 2025 IEEE. Personal use of this material is permitted.
  Permission from IEEE must be obtained for all other uses, in any current or future
  media, including reprinting/republishing this material for advertising or promotional
  purposes, creating new collective works, for resale or redistribution to servers or
  lists, or reuse of any copyrighted component of this work in other works.}
\newcommand\copyrightnotice{%
\begin{tikzpicture}[remember picture,overlay]
\node[anchor=south,yshift=10pt] at (current page.south) {\parbox{\dimexpr\textwidth-\fboxsep-\fboxrule\relax}{\copyrighttext}};
\end{tikzpicture}%
}

\usepackage{array}
\usepackage{color, colortbl}

\newcolumntype{P}[1]{>{\centering\arraybackslash}p{#1}}

\begin{document}
\bstctlcite{IEEEexample:BSTcontrol}

\maketitle
%\copyrightnotice

\thispagestyle{empty}
\pagestyle{empty}


%%%%%%%%%%%%%%%%%%%%%%%%%%%%%%%%%%%%%%%%%%%%%%%%%%%%%%%%%%%%%%%%%%%%%%%%%%%%%%%%
\begin{abstract}
Non-prehensile pushing to move and reorient objects to a goal is a versatile loco-manipulation skill. In the real world, the object's physical properties and friction with the floor contain significant uncertainties, which makes the task challenging for a mobile manipulator. In this paper, we develop a learning-based controller for a mobile manipulator to move an unknown object to a desired position and yaw orientation through a sequence of pushing actions. The proposed controller for the robotic arm and the mobile base motion is trained using a constrained Reinforcement Learning (RL) formulation. We demonstrate its capability in experiments with a quadrupedal robot equipped with an arm. The learned policy achieves a success rate of 91.35\% in simulation and at least 80\% on hardware in challenging scenarios. Through our extensive hardware experiments, we show that the approach demonstrates high robustness against unknown objects of different masses, materials, sizes, and shapes. It reactively discovers the pushing location and direction, thus achieving contact-rich behavior while observing only the pose of the object. Additionally, we demonstrate the adaptive behavior of the learned policy towards preventing the object from toppling.
\end{abstract}

\section{Introduction}

Large language models (LLMs) have achieved remarkable success in automated math problem solving, particularly through code-generation capabilities integrated with proof assistants~\citep{lean,isabelle,POT,autoformalization,MATH}. Although LLMs excel at generating solution steps and correct answers in algebra and calculus~\citep{math_solving}, their unimodal nature limits performance in plane geometry, where solution depends on both diagram and text~\citep{math_solving}. 

Specialized vision-language models (VLMs) have accordingly been developed for plane geometry problem solving (PGPS)~\citep{geoqa,unigeo,intergps,pgps,GOLD,LANS,geox}. Yet, it remains unclear whether these models genuinely leverage diagrams or rely almost exclusively on textual features. This ambiguity arises because existing PGPS datasets typically embed sufficient geometric details within problem statements, potentially making the vision encoder unnecessary~\citep{GOLD}. \cref{fig:pgps_examples} illustrates example questions from GeoQA and PGPS9K, where solutions can be derived without referencing the diagrams.

\begin{figure}
    \centering
    \begin{subfigure}[t]{.49\linewidth}
        \centering
        \includegraphics[width=\linewidth]{latex/figures/images/geoqa_example.pdf}
        \caption{GeoQA}
        \label{fig:geoqa_example}
    \end{subfigure}
    \begin{subfigure}[t]{.48\linewidth}
        \centering
        \includegraphics[width=\linewidth]{latex/figures/images/pgps_example.pdf}
        \caption{PGPS9K}
        \label{fig:pgps9k_example}
    \end{subfigure}
    \caption{
    Examples of diagram-caption pairs and their solution steps written in formal languages from GeoQA and PGPS9k datasets. In the problem description, the visual geometric premises and numerical variables are highlighted in green and red, respectively. A significant difference in the style of the diagram and formal language can be observable. %, along with the differences in formal languages supported by the corresponding datasets.
    \label{fig:pgps_examples}
    }
\end{figure}



We propose a new benchmark created via a synthetic data engine, which systematically evaluates the ability of VLM vision encoders to recognize geometric premises. Our empirical findings reveal that previously suggested self-supervised learning (SSL) approaches, e.g., vector quantized variataional auto-encoder (VQ-VAE)~\citep{unimath} and masked auto-encoder (MAE)~\citep{scagps,geox}, and widely adopted encoders, e.g., OpenCLIP~\citep{clip} and DinoV2~\citep{dinov2}, struggle to detect geometric features such as perpendicularity and degrees. 

To this end, we propose \geoclip{}, a model pre-trained on a large corpus of synthetic diagram–caption pairs. By varying diagram styles (e.g., color, font size, resolution, line width), \geoclip{} learns robust geometric representations and outperforms prior SSL-based methods on our benchmark. Building on \geoclip{}, we introduce a few-shot domain adaptation technique that efficiently transfers the recognition ability to real-world diagrams. We further combine this domain-adapted GeoCLIP with an LLM, forming a domain-agnostic VLM for solving PGPS tasks in MathVerse~\citep{mathverse}. 
%To accommodate diverse diagram styles and solution formats, we unify the solution program languages across multiple PGPS datasets, ensuring comprehensive evaluation. 

In our experiments on MathVerse~\citep{mathverse}, which encompasses diverse plane geometry tasks and diagram styles, our VLM with a domain-adapted \geoclip{} consistently outperforms both task-specific PGPS models and generalist VLMs. 
% In particular, it achieves higher accuracy on tasks requiring geometric-feature recognition, even when critical numerical measurements are moved from text to diagrams. 
Ablation studies confirm the effectiveness of our domain adaptation strategy, showing improvements in optical character recognition (OCR)-based tasks and robust diagram embeddings across different styles. 
% By unifying the solution program languages of existing datasets and incorporating OCR capability, we enable a single VLM, named \geovlm{}, to handle a broad class of plane geometry problems.

% Contributions
We summarize the contributions as follows:
We propose a novel benchmark for systematically assessing how well vision encoders recognize geometric premises in plane geometry diagrams~(\cref{sec:visual_feature}); We introduce \geoclip{}, a vision encoder capable of accurately detecting visual geometric premises~(\cref{sec:geoclip}), and a few-shot domain adaptation technique that efficiently transfers this capability across different diagram styles (\cref{sec:domain_adaptation});
We show that our VLM, incorporating domain-adapted GeoCLIP, surpasses existing specialized PGPS VLMs and generalist VLMs on the MathVerse benchmark~(\cref{sec:experiments}) and effectively interprets diverse diagram styles~(\cref{sec:abl}).

\iffalse
\begin{itemize}
    \item We propose a novel benchmark for systematically assessing how well vision encoders recognize geometric premises, e.g., perpendicularity and angle measures, in plane geometry diagrams.
	\item We introduce \geoclip{}, a vision encoder capable of accurately detecting visual geometric premises, and a few-shot domain adaptation technique that efficiently transfers this capability across different diagram styles.
	\item We show that our final VLM, incorporating GeoCLIP-DA, effectively interprets diverse diagram styles and achieves state-of-the-art performance on the MathVerse benchmark, surpassing existing specialized PGPS models and generalist VLM models.
\end{itemize}
\fi

\iffalse

Large language models (LLMs) have made significant strides in automated math word problem solving. In particular, their code-generation capabilities combined with proof assistants~\citep{lean,isabelle} help minimize computational errors~\citep{POT}, improve solution precision~\citep{autoformalization}, and offer rigorous feedback and evaluation~\citep{MATH}. Although LLMs excel in generating solution steps and correct answers for algebra and calculus~\citep{math_solving}, their uni-modal nature limits performance in domains like plane geometry, where both diagrams and text are vital.

Plane geometry problem solving (PGPS) tasks typically include diagrams and textual descriptions, requiring solvers to interpret premises from both sources. To facilitate automated solutions for these problems, several studies have introduced formal languages tailored for plane geometry to represent solution steps as a program with training datasets composed of diagrams, textual descriptions, and solution programs~\citep{geoqa,unigeo,intergps,pgps}. Building on these datasets, a number of PGPS specialized vision-language models (VLMs) have been developed so far~\citep{GOLD, LANS, geox}.

Most existing VLMs, however, fail to use diagrams when solving geometry problems. Well-known PGPS datasets such as GeoQA~\citep{geoqa}, UniGeo~\citep{unigeo}, and PGPS9K~\citep{pgps}, can be solved without accessing diagrams, as their problem descriptions often contain all geometric information. \cref{fig:pgps_examples} shows an example from GeoQA and PGPS9K datasets, where one can deduce the solution steps without knowing the diagrams. 
As a result, models trained on these datasets rely almost exclusively on textual information, leaving the vision encoder under-utilized~\citep{GOLD}. 
Consequently, the VLMs trained on these datasets cannot solve the plane geometry problem when necessary geometric properties or relations are excluded from the problem statement.

Some studies seek to enhance the recognition of geometric premises from a diagram by directly predicting the premises from the diagram~\citep{GOLD, intergps} or as an auxiliary task for vision encoders~\citep{geoqa,geoqa-plus}. However, these approaches remain highly domain-specific because the labels for training are difficult to obtain, thus limiting generalization across different domains. While self-supervised learning (SSL) methods that depend exclusively on geometric diagrams, e.g., vector quantized variational auto-encoder (VQ-VAE)~\citep{unimath} and masked auto-encoder (MAE)~\citep{scagps,geox}, have also been explored, the effectiveness of the SSL approaches on recognizing geometric features has not been thoroughly investigated.

We introduce a benchmark constructed with a synthetic data engine to evaluate the effectiveness of SSL approaches in recognizing geometric premises from diagrams. Our empirical results with the proposed benchmark show that the vision encoders trained with SSL methods fail to capture visual \geofeat{}s such as perpendicularity between two lines and angle measure.
Furthermore, we find that the pre-trained vision encoders often used in general-purpose VLMs, e.g., OpenCLIP~\citep{clip} and DinoV2~\citep{dinov2}, fail to recognize geometric premises from diagrams.

To improve the vision encoder for PGPS, we propose \geoclip{}, a model trained with a massive amount of diagram-caption pairs.
Since the amount of diagram-caption pairs in existing benchmarks is often limited, we develop a plane diagram generator that can randomly sample plane geometry problems with the help of existing proof assistant~\citep{alphageometry}.
To make \geoclip{} robust against different styles, we vary the visual properties of diagrams, such as color, font size, resolution, and line width.
We show that \geoclip{} performs better than the other SSL approaches and commonly used vision encoders on the newly proposed benchmark.

Another major challenge in PGPS is developing a domain-agnostic VLM capable of handling multiple PGPS benchmarks. As shown in \cref{fig:pgps_examples}, the main difficulties arise from variations in diagram styles. 
To address the issue, we propose a few-shot domain adaptation technique for \geoclip{} which transfers its visual \geofeat{} perception from the synthetic diagrams to the real-world diagrams efficiently. 

We study the efficacy of the domain adapted \geoclip{} on PGPS when equipped with the language model. To be specific, we compare the VLM with the previous PGPS models on MathVerse~\citep{mathverse}, which is designed to evaluate both the PGPS and visual \geofeat{} perception performance on various domains.
While previous PGPS models are inapplicable to certain types of MathVerse problems, we modify the prediction target and unify the solution program languages of the existing PGPS training data to make our VLM applicable to all types of MathVerse problems.
Results on MathVerse demonstrate that our VLM more effectively integrates diagrammatic information and remains robust under conditions of various diagram styles.

\begin{itemize}
    \item We propose a benchmark to measure the visual \geofeat{} recognition performance of different vision encoders.
    % \item \sh{We introduce geometric CLIP (\geoclip{} and train the VLM equipped with \geoclip{} to predict both solution steps and the numerical measurements of the problem.}
    \item We introduce \geoclip{}, a vision encoder which can accurately recognize visual \geofeat{}s and a few-shot domain adaptation technique which can transfer such ability to different domains efficiently. 
    % \item \sh{We develop our final PGPS model, \geovlm{}, by adapting \geoclip{} to different domains and training with unified languages of solution program data.}
    % We develop a domain-agnostic VLM, namely \geovlm{}, by applying a simple yet effective domain adaptation method to \geoclip{} and training on the refined training data.
    \item We demonstrate our VLM equipped with GeoCLIP-DA effectively interprets diverse diagram styles, achieving superior performance on MathVerse compared to the existing PGPS models.
\end{itemize}

\fi 


\section{Related Work}
\label{sec:RelatedWork}

Within the realm of geophysical sciences, super-resolution/downscaling is a challenge that scientists continue to tackle. There have been several works involved in downscaling applications such as river mapping \cite{Yin2022}, coastal risk assessment \cite{Rucker2021}, estimating soil moisture from remotely sensed images \cite{Peng2017SoilMoisture} and downscaling of satellite based precipitation estimates \cite{Medrano2023PrecipitationDownscaling} to name a few. We direct the reader to \cite{Karwowska2022SuperResolutionSurvey} for a comprehensive review of satellite based downscaling applications and methods. Pertaining to our objective of downscaling \acp{WFM}, we can draw comparisons with several existing works. 
In what follows, we provide a brief review of functionally adjacent works to contrast the novelty of our proposed model and its role in addressing gaps in literature. 

When it comes to downscaling \ac{WFM}, several works use statistical downscaling techniques. These works downscale images by using statistical techniques that utilize relationships between neighboring water fraction pixels. For instance, \cite{Li2015SRFIM} treat the super-resolution task as a sub-pixel mapping problem, wherein the input fraction of inundated pixels must be exactly mapped to the output patch of inundated pixels. 
% In doing so, they are able to apply a discrete particle swarm optimization method to maximize the Flood Inundation Spatial Dependence Index (FISDI). 
\cite{Wang2019} improved upon these approaches by including a spectral term to fully utilize spectral information from multi spectral remote sensing image band. \cite{Wang2021} on the other hand also include a spectral correlation term to reduce the influence of linear and non-linear imaging conditions. All of these approaches are applied to water fraction obtained via spectral unmixing \cite{wang2013SpectralUnmixing} and are designed to work with multi spectral information from MODIS. However, we develop our model with the intention to be used with water fractions directly derived from the output of satellites. One such example is NOAA/VIIRS whose water fraction extraction method is described in \cite{Li2013VIIRSWFM}. \cite{Li2022VIIRSDownscaling} presented a work wherein \ac{WFM} at 375-m flood products from VIIRS were downscaled 30-m flood event and depth products by expressing the inundation mechanism as a function of the \ac{DEM}-based water area and the VIIRS water area.

On the other hand, the non-linear nature of the mapping task lends itself to the use of neural networks. Several models have been adapted from traditional single image digital super-resolution in computer vision literature \cite{sdraka2022DL4downscalingRemoteSensing}. Existing deep learning models in single image super-resolution are primarily dominated by \ac{CNN} based models. Specifically, there has been an upward trend in residual learning models. \acp{RDN} \cite{Zhang2018ResidualDenseSuperResolution} introduced residual dense blocks that employed a contiguous memory mechanism that aimed to overcome the inability of very deep \acp{CNN} to make full use of hierarchical features. 
\acp{RCAN} \cite{Zhang2018RCANSuperResolution} introduced an attention mechanism to exploit the inter-channel dependencies in the intermediate feature transformations. There have also been some works that aim to produce more lightweight \ac{CNN}-based architectures \cite{Zheng2019IMDN,Xiaotong2020LatticeNET}. Since the introduction of the vision transformer \cite{Vaswani2017Attention} that utilized the self-attention mechanism -- originally used for modeling text sequences -- by feeding a sequence 2D sub-image extracted from the original image. Using this approach \cite{LuESRT2022} developed a light-weight and efficient transformer based approach for single image super-resolution. 


For the task of super-resolution of \acp{WFM}, we discuss some works whose methodology is similar to ours even though they differ in their problem setting. \cite{Yin2022} presented a cascaded spectral spatial model for super-resolution of MODIS imagery with a scaling factor 10. Their architecture consists of two stages; first multi-spectral MODIS imagery is converted into a low-resolution \ac{WFM} via spectral unmixing by passing it through a deep stacked residual \ac{CNN}. The second stage involved the super-resolution mapping of these \acp{WFM} using a nested multi-level \ac{CNN} model. Similar to our work, the input fraction images are obtained with zero errors which may not be reflective of reality since there tends to be sensor noise, the spatial distribution of whom cannot be easily estimated. We also note that none of these works directly tackle flood inundation since they've been trained with river map data during non-flood circumstance and \textit{ergo} do not face a data scarcity problem as we do. 
% In this work, apart from the final product of \acp{WFM}, we are not presented with any additional spectral information about the low resolution image. This was intended to work directly with products that can generate \ac{WFM} either directly (VIIRS) or indirectly (Landsat).
\cite{Jia2019} used a deep \ac{CNN} for land mapping that consists of several classes such as building, low vegetation, background and trees. 
\cite{Kumar2021} similarly employ a \ac{CNN} based model for downscaling of summer monsoon rainfall data over the Indian subcontinent. Their proposed Super-Resolution Convolutional Neural Network (SRCNN) has a downscaling factor of 4. 
\cite{Shang2022} on the other hand, proposed a super-resolution mapping technique using Generative Adversarial Networks (GANs). They first generate high resolution fractional images, somewhat analogous to our \ac{WFM}, and are then mapped to categorical land cover maps involving forest, urban, agriculture and water classes. 
\cite{Qin2020} interestingly approach lake area super-resolution for Landsat and MODIS data as an unsupervised problem using a \ac{CNN} and are able to extend to other scaling factors. \cite{AristizabalInundationMapping2020} performed flood inundation mapping using \ac{SAR} data obtained from Sentinel-1. They showed that \ac{DEM}-based features helped to improve \ac{SAR}-based predictions for quadratic discriminant analysis, support vector machines and k-nearest neighbor classifiers. While almost all of the aforementioned works can be adapted to our task. We stand out in the following ways (i) We focus on downscaling of \acp{WFM} directly, \textit{i.e.,} we do not focus on the algorithm to compute the \ac{WFM} from multi-channel satellite data and (ii) We focus on producing high resolution maps only for instances of flood inundation. The latter point produces a data scarcity issue which we seek to remedy with synthetic data. 


%%%%%%%%%%%%%%%%% Additional unused information %%%%%%%%%%%%%%%%


%     \item[\cite{Wang2021}] Super-Resolution Mapping Based on Spatial–Spectral Correlation for Spectral Imagery
%     \begin{itemize}
%         \item Not a deep neural network approach. SRM based on spatial–spectral correlation (SSC) is proposed in order to overcome the influence of linear and nonlinear imaging conditions and utilize more accurate spectral properties.
%         \item (fig 1) there are two main SRM types: (1) the initialization-then-optimization SRM, where the class labels are allocated randomly to subpixels, and the location of each subpixel is optimized to obtain the final SRM result. and (2)soft-then-hard SRM, which involves two steps: the subpixel sharpening and the class allocation.  
%         \item SSC procedures: (1) spatial correlation is performed by the MSAM to reduce the influences of linear imaging conditions on image quality. (2) A spectral correlation that utilizes spectral properties based on the nonlinear KLD is proposed to reduce the influences of nonlinear imaging conditions. (3) spatial and spectral correlations are then combined to obtain an optimization function with improved linear and nonlinear performances. And finally (4) by maximizing the optimization function, a class allocation method based on the SA is used to assign LC labels to each subpixel, obtaining the final SRM result.
%         \item (Comparable) 
%     \end{itemize}
%     %--------------------------------------------------------------------
% \cite{Wang2021} account for the influence of linear and non-linear imaging conditions by involving more accurate spectral properties. 
%     %--------------------------------------------------------------------
%     \item[\cite{Yin2022}] A Cascaded Spectral–Spatial CNN Model for Super-Resolution River Mapping With MODIS Imagery
%     \begin{itemize}
%         \item produce  Landsat-like  fine-resolution (scale of 10)  river  maps  from  MODIS images. Notice the original coarse-resolution remotely sensed images, not the river fraction images.
%         \item combined  CNN  model that  contains  a spectral  unmixing  module  and  an  SRM  module, and the SRM module is made up of an encoder and a decoder that are connected through a series of convolutional blocks. 
%         \item With an adaptive cross-entropy loss function to address class imbalance.	
%         \item The overall accuracy, the omission error, the  commission  error,  and  the  mean  intersection  over  union (MIOU)  calculated  to  assess  the results.
%         \item partially comparable with ours, only the SRM module part
%     %--------------------------------------------------------------------

% To decouple the description of the objective and the \ac{ML} model architecture, the motivation for the model architecture is described in \secref{sec:Methodology}. 


%     \item[\cite{Wang2019}] Improving Super-Resolution Flood Inundation Mapping for Multi spectral Remote Sensing Image by Supplying More Spectral 
%     \begin{itemize}
%         \item proposed the SRFIM-MSI,where a new spectral term is added to the traditional SRFIM to fully utilize the spectral information from multi spectral remote sensing image band. 
%         \item The original SRFIM \cite{Huang2014, Li2015} obtains the sub pixel spatial distribution of flood inundation within mixed pixels by maximizing their spatial correlation while maintaining the original proportions of flood inundation within the mixed pixels. The SRFIM is formulated as a maximum combined optimization issue according to the principle of spatial correlation.
%         \item follow the terminology in \cite{Wang2021}, this is an initialization-then-optimization SRM. 
%         \item (Comparable) 
%     \end{itemize}
%     %--------------------------------------------------------------------


%--------------------------------------------------------------------
%     \item[\cite{Jia2019}] Super-Resolution Land Cover Mapping Based on the Convolutional Neural Network
%     \begin{itemize}
%         \item SRMCNN (Super-resolution mapping CNN) is proposed to obtain fine-scale land cover maps from coarse remote sensing images. Specifically, an encoder-decoder CNN is used to determined the labels (i.e., land cover classes) of the subpixels within mixed pixels.
%         \item There were three main parts in SRMCNN. The first part was a three-sequential convolutional layer with ReLU and pooling. The second part is up-sampling, for which a multi transposed-convolutional layer was adopted. To keep the feature learned in the previous layer, a skip connection was used to concatenate the output of the corresponding convolution layer. The last part was the softmax classifier, in which the feature in the antepenultimate layer was classified and class probabilities are obtained.
%         \item The loss: the optimal allocation of classes to the subpixels of mixed pixel is achieved by maximizing the spatial dependence between neighbor pixels under constraint that the class proportions within the mixed pixels are preserved.
%         \item (Preferred), this paper is designed to classify background, Building, Low Vegetation, or Tree in the land. But we can easily adapt to our problem and should compare with this paper.
%     \end{itemize}
%     %--------------------------------------------------------------------

%     \item[\cite{Kumar2021}] Deep learning–based downscaling of summer monsoon rainfall data over Indian region
%     \begin{itemize}
%         \item down-scaling (scale of 4) rainfall data. The output image is not binary image.
%         \item three algorithms: SRCNN, stacked SRCNN, and DeepSD are employed, based on \cite{Vandal2019}
%         \item mean square error and pattern correlation coefficient are used as evaluation metrics.
%         \item SRCNN: super-resolution-based convolutional neural networks (SRCNN) first upgrades the low-resolution image to the higher resolution size by using bicubic interpolation. Suppose the interpolated image is referred to as Y; SRCNNs’task is to retrieve from Y an image F(Y) which is close to the high-resolution ground truth image X.
%         \item stacked SRCNN: stack 2 or more SRCNN blocks to increasing the scaling factor.
%         \item DeepSD: uses topographies as an additional input to stacked SRCNN.
%         \item These algorithms are not designed for binary output images, but if prefer, the ``modified'' stacked SRCNN or DeepSD can be used as baseline algorithms.
%     
%     \item[\cite{Shang2022}] Super resolution Land Cover Mapping Using a Generative Adversarial Network
%     \begin{itemize}
%         \item propose an end-to-end SRM model based on a generative adversarial network (GAN), that is, GAN-SRM, to improve the two-step learning-based SRM methods. 
%         \item Two-step SRM method: The first step is fraction-image super-resolution (SR), which reconstructs a high-spatial-resolution fraction image from the low input, methods like SVR, or CNN has been widely adopted. The second step is converting the high-resolution fraction images to a categorical land cover map, such as with a soft-max function to assign each high-resolution pixel to a unique category value.
%         \item The proposed GAN-SRM model includes a generative network and a discriminative network, so that both the fraction-image SR and the conversion of the fraction images to categorical map steps are fully integrated to reduce the resultant uncertainty. 
%         \item applied to the National Land Cover Database (NLCD), which categorized land into four typical classes:forest, urban, agriculture,and water. scale factor of 8. 
%         \item (Preferred), we should compare with this work.
%     \end{itemize}
%     %--------------------------------------------------------------------

%   \item[\cite{Qin2020}] Achieving Higher Resolution Lake Area from Remote Sensing Images Through an Unsupervised Deep Learning Super-Resolution Method
%   \begin{itemize}
%       \item propose an unsupervised deep gradient network (UDGN) to generate a higher resolution lake area from remote sensing images.
%       \item UDGN models the internal recurrence of information inside the single image and its corresponding gradient map to generate images with higher spatial resolution. 
%       \item A single image super-resolution approach, not comparable
%   \end{itemize}
%     %--------------------------------------------------------------------




%     \item[\cite{Demiray2021}] D-SRGAN: DEM Super-Resolution with Generative Adversarial Networks
%     \begin{itemize}
%         \item A GAN based model is proposed to increase the spatial resolution of a given DEM dataset up to 4 times without additional information related to data.
%         \item Rather than processing each image in a sequence independently, our generator architecture uses a recurrent layer to update the state of the high-resolution reconstruction in a manner that is consistent with both the previous state and the newly received data. The recurrent layer can thus be understood as performing a Bayesian update on the ensemble member, resembling an ensemble Kalman filter. 
%         \item A single image super-resolution approach, not comparable
%     \end{itemize}
%     %--------------------------------------------------------------------
%     \item[\cite{Leinonen2021}] Stochastic Super-Resolution for Downscaling Time-Evolving Atmospheric Fields With a Generative Adversarial Network
%     \begin{itemize}
%         \item propose a super-resolution GAN that operates on sequences of two-dimensional images and creates an ensemble of predictions for each input. The spread between the ensemble members represents the uncertainty of the super-resolution reconstruction.
%         \item for sequence of input images, not comparable with ours.
%     \end{itemize} 
%     %--------------------------------------------------------------------

% \end{itemize}





% \section{Related Work}\label{sec:related_works}
\gls{bp} estimation from \gls{ecg} and \gls{ppg} waveforms has received significant attention due to its potential for continuous, unobtrusive monitoring. Earlier work relied on classical machine learning with handcrafted features, but deep learning methods have since emerged as more robust alternatives. Convolutional or recurrent architectures designed for \gls{ecg}/\gls{ppg} have shown strong performance, including ResUNet with self-attention~\cite{Jamil}, U-Net variants~\cite{Mahmud_2022}, and hybrid \gls{cnn}--\gls{rnn} models~\cite{Paviglianiti2021ACO}. These architectures often outperform traditional feature-engineering approaches, particularly when both \gls{ecg} and \gls{ppg} signals are used~\cite{Paviglianiti2021ACO}.

Nevertheless, many existing methods train solely on \gls{ecg}/\gls{ppg} data, which, while plentiful~\cite{mimiciii,vitaldb,ptb-xl}, often exhibit significant variability in signal quality and patient-specific characteristics. This variability poses challenges for achieving robust generalization across populations. Recent work has explored transfer learning to overcome these issues; for example, Yang \emph{et~al.}~\cite{yang2023cross} studied the transfer of \gls{eeg} knowledge to \gls{ecg} classification tasks, achieving improved performance and reduced training costs. Joshi \emph{et~al.}~\cite{joshi2021deep} also explored the transfer of \gls{eeg} knowledge using a deep knowledge distillation framework to enhance single-lead \gls{ecg}-based sleep staging. However, these studies have largely focused on within-modality or narrow domain adaptations, leaving open the broader question of whether an \gls{eeg}-based foundation model can serve as a versatile starting point for generalized biosignal analysis.

\gls{eeg} has become an attractive candidate for pre-training large models not only because of the availability of large-scale \gls{eeg} repositories~\cite{TUEG} but also due to its rich multi-channel, temporal, and spectral dynamics~\cite{jiang2024large}. While many time-series modalities (for example, voice) also exhibit rich temporal structure, \gls{eeg}, \gls{ecg}, and \gls{ppg} share common physiological origins and similar noise characteristics, which facilitate the transfer of temporal pattern recognition capabilities. In other words, our hypothesis is that the underlying statistical properties and multi-dimensional dynamics in \gls{eeg} make it particularly well-suited for learning robust representations that can be effectively adapted to \gls{ecg}/\gls{ppg} tasks. Our work is the first to validate the feasibility of fine-tuning a transformer-based model initially trained on EEG (CEReBrO~\cite{CEReBrO}) for arterial \gls{bp} estimation using \gls{ecg} and \gls{ppg} data.

Beyond accuracy, real-world deployment of \gls{bp} estimation models calls for efficient inference. Traditional deep networks can be computationally expensive, motivating recent interest in quantization and other compression techniques~\cite{nagel2021whitepaperneuralnetwork}. Few studies have combined large-scale pre-training with post-training quantization for \gls{bp} monitoring. Hence, our method integrates these two aspects: leveraging a potent \gls{eeg}-based foundation model and applying quantization for a compact, high-accuracy cuffless \gls{bp} solution.

We applied Recurrency Sequence Processing to address the lack of consistency in the coarse dance representation of the~\cite{li2024lodge} model. We named this Recurrency Sequence Representation Learning as Dance Recalibration (DR). Dance recalibration uses \(n\) Dance Recalibration Blocks (DRB) corresponding to the length of the rough dance sequence to add sequential information to the rough dance representation to improve the consistency of the whole dance. The overall structure of our model is illustrated in Figure 1.

\begin{figure}[!t]
    \centering
    \includegraphics[width=\textwidth]{Figure1.eps}
    \caption{overall procedure of Pooling processing by our Pooling Block}
    \label{fig:enter-label4}
\end{figure}


\subsection{Dance Recalibration (DR)}
When the dance motion representation passes through the Dance Decoder Process using the~\cite{li2024lodge} model, it yields a coarse dance motion representation. During this process, the dance motion representations that pass through Global Diffusion follow a distribution but can output unstable values. This results in awkward dance motions when viewed from a sequential perspective. To address this issue, we added a Dance Recalibration Process.

DR fundamentally follows a structure similar to RNNs. Although RNNs are known to suffer from the gradient vanishing problem as they get deeper, the sequence length of the coarse dance representation in \cite{li2024lodge} is not long enough to cause this issue, making it suitable for use. Using LSTM or GRU, which solve the gradient vanishing problem, would make the model too complex and computationally expensive, making them unsuitable for use with the Denoising Diffusion Process \cite{ho2020denoising, song2020denoising}.

The coarse dance representation has 139 channels, consisting of 4-dim foot positions, 3-dim root translation, 6-dim rotaion information and 126-dim joint rotation channels. Of these, the 126-dim channels directly impact the dance motion, and all DR operations are performed using these 126 channels.

The values output from the Global Dance Decoder \(GD_{i}\), contain unstable dance motion information that follows a distribution. We construct Global Recalibrated Dance \(GRD_{i}\) by concatenating \(C\) the information from \(GRD_{i-1}\) with \(GD_{i}\) and applying pooling \(P\), thereby adding sequential information. However, using previous information as is may result in overly simple and smoothly connected dance motions. To prevent this, we add Gaussian noise \(G\) to the previous information \(GRD_{i-1}\) to produce more varied dance motions. This process is represented in Equations 1 below. The entire procedure is illustrated in Figure 2, 3.
\begin{equation}
    GRD_{i} = P(C(GD_{i} , GRD_{i-1} + G(Threshold))
\end{equation}



\begin{figure}[!t]
    \centering
    \includegraphics[width=\textwidth]{DanceRecalibration.eps}
    \caption{Overall of the Dance Recalibration Block Structure}
    \label{fig:enter-label1}
\end{figure}

\begin{figure}[!t]
    \centering
    \includegraphics[width=\textwidth]{DanceRecalibrationBlock.eps}
    \caption{The structure of the dance recalibration block}
    \label{fig:enter-label2}
\end{figure}

\subsection{Pooling Block}
Pooling \(P\) uses a simple pooling method. When \(GRD_{i}\) with added \(G\) and \(GD_{i+1}\) are input, they are concatenated into a \((Batch\times2\times126)\). First, we perform Layer Normalization to minimize differences between layers. Then, we pass through three simple 1D-Convolution Blocks, each followed by an activation function and batch normalization, to construct \(GRD_{i+1}\) that includes information from the previous dance motion. This procedure is illustrated in Figure 4.

\begin{figure}[!t]
    \centering
    \includegraphics[width=\textwidth]{Figure3.eps}
    \caption{overall procedure of Pooling processing by our Pooling Block}
    \label{fig:enter-label3}
\end{figure}

By following all these steps, each dance motion incorporates a bit of information from the previous dance motions, producing an overall coarse dance motion that follows the distribution of Global Diffusion while also retaining sequential information. This process is expressed in Equation 2:

\begin{equation}
    Total Coarse Dance Motion = C_{i=1}^{n}(P(C(GD_{i} , GRD_{i-1} + G(Threshold))), P(GD_{0}))
\end{equation}

We did not use bi-directional information because it complicates the calculations and can destabilize sequential information when using more than two \(GD_{i}\). Since there is a trade-off between generating complex dance motions and maintaining consistency, it is crucial to add appropriate noise. However, due to time constraints, we could not conduct various ablation studies.

\section{Results}
\label{sec:Results}

In this section, we present various analysis results that demonstrate the adoption of code obfuscation in Google Play.

\subsection{Overall Obfuscation Trends} 
\label{sec:obstrend}

\subsubsection{Presence of obfuscation} Out of the 548,967 Google Play Store APKs analyzed, we identified 308,782 obfuscated apps, representing approximately 56.25\% of the total. In Figure~\ref{fig:obfuscated_percentage}, we show the year-wise percentage of obfuscated apps for 2016-2023. There is an overall obfuscation increase of 13\% between 2016 and 2023, and as can be seen, the percentage of obfuscated apps has been increasing in the last few years, barring 2019 and 2020. As explained in Section~\ref{subsec:dataset}, 2019 and 2020 contain apps that are more likely to be abandoned by developers, and as such, they may not use advanced development practices.

\begin{figure}[h!]
\centering
    \includegraphics[width=\linewidth]{Figures/Only_obfuscation_trendV2.pdf}
    \caption{Percentage of obfuscated apps by year} \vspace{-4mm}
    \label{fig:obfuscated_percentage}
\end{figure}


From 2016 to 2018, the obfuscation levels were relatively stable at around 50-55\%, while from 2021 to 2023, there was a marked rise, reaching approximately 66\% in 2023. This indicates a growing focus on app protection measures among developers, likely driven by heightened security and IP concerns and the availability of advanced obfuscation tools.


\subsubsection{Obfuscation tools} Among the obfuscated APKs, our tool detector identified that 40.92\% of the apps use Proguard, 36.64\% use Allatori, 1.01\% use DashO, and 21.43\% use other (i.e., unknown) tools. We show the yearly trends in Figure~\ref{fig:ofbuscated_tool}. Note that we omit results in 2019 and 2020 ({\bf cf.} Section~\ref{subsec:dataset}).

ProGuard and Allatori are the most consistently used obfuscation tools, with ProGuard showing a slight overall increase in popularity and Allatori demonstrating variability. This inclination could be attributed to ProGuard being the default obfuscator integrated into Android Studio, a widely used development environment for Android applications. Notably, ProGuard usage increased by 13\% from 2018 to 2021, likely due to the introduction of R8 in April 2019~\cite{release_note_android}, which further simplified ProGuard integration with Android apps.

\begin{figure}[h]
\centering
    \includegraphics[width=\linewidth]{Figures/Initial_Tool_Trend_2019_dropV2.pdf} 
    \caption{Yearly obfuscation tool usage}
    \label{fig:ofbuscated_tool}
\end{figure}


DashO consistently remains low in usage, likely due to its high cost. The use of other obfuscation tools decreased until 2018 but has shown a resurgence from 2021 to 2023. This suggests that developers might be using other or custom tools, or our detector might be predicting some apps obfuscated with Proguard or Allatori as `other.' To investigate, we manually checked a sample of apps from the `other' category and confirmed they are indeed obfuscated. However, we could not determine which obfuscation tools the developers used. We discuss this potential limitation further in Section~\ref{sec:limitations}.


\subsubsection{Obfuscation techniques} We show the year-wise breakdown of obfuscation technique usage in Figure~\ref{fig:obfuscated_tech}. Among the various obfuscation techniques, Identifier Renaming emerged as the most prevalent, with 99.62\% of obfuscated apps using it alone or in combination with other methods (Categories of Only IR, IR and CF, IR and SE, or All three). Furthermore, 81.04\% of obfuscated apps used Control Flow Modification, and 62.76\% used String Encryption. The pervasive use of Identifier Renaming (IR) can be attributed to the fact that all obfuscation tools support it ({\bf cf.} Table~\ref{tab:ob_tool_cap}). Similarly, lower adoption of Control Flow Modification and String Encryption can be attributed to Proguard not supporting it. 

\begin{figure}[h]
\centering
    \includegraphics[width=\linewidth]{Figures/Initial_Tech_Trend_2019_dropV2.pdf} 
    \caption{Yearly obfuscation technique usage}
    \label{fig:obfuscated_tech}
\end{figure}



Next, we investigate the adoption of obfuscation on Google Play Store from various perspectives. Same as earlier, due to the smaller dataset size and possible bias ({\bf cf.} Section~\ref{subsec:dataset}), we exclude the APKs from 2019 and 2020 from this analyses.


\subsection{App Genre}
\label{sec:app_genre}

First, we investigate whether the obfuscation practices vary according to the App genre. Initially, we analysed all the APKs together before separating them into two snapshots.


\begin{figure*}[h]
    \centering
    \includegraphics[width=\linewidth]{Figures/AppGenreObfuscationV3.pdf}
    \caption{Obfuscated app percentage by genre (overall)}
    \label{fig:app_genre_overall}
\end{figure*}

Figure~\ref{fig:app_genre_overall} shows the genre-wise obfuscated app percentage. We note that 19 genres have more than 60\% of the apps obfuscated, and almost all the genres have more than 40\% obfuscation percentage. \textit{Casino} genre has the highest obfuscation percentage rate at 80\%, and overall, game genres tend to be more obfuscated than the other genres. The higher obfuscation usage in casino apps is logical due to their nature. These apps often simulate or involve gambling activities and handle monetary transactions and sensitive data related to in-game purchases, making them attractive targets for reverse engineering and hacking. This necessitates robust security measures to prevent fraud and protect user data. 


\begin{figure}[h]
    \centering
    \includegraphics[width=\linewidth]{Figures/AppGenre2018_2023ComparisonV3.pdf}
    \caption{Percentage of obfuscated apps by genre (2018-2023)}
    \label{fig:app_genre_comparison}
\end{figure}



\subsubsection{Genre-wise obfuscation trends in the two snapshots} To investigate the adoption of obfuscation over time, we study the two snapshots of Google Play separately, i.e., APKs from 2016-2018 as one group and APKs from 2021-2023 as another. 

Figure~\ref{fig:app_genre_comparison} illustrates the change in obfuscation levels by app genre between 2016-2018 to 2021-2023. Notably, app categories such as Education, Weather, and Parenting, which had obfuscation levels below the 2018 average, have increased to above the 2023 average by 2023. One possible reason for this in Education and Parenting apps can be the increase in online education activities during and after COVID-19 and the developers identifying the need for app hardening.

There are some genres, such as Casino and Action, for which the percentage of obfuscated apps didn't change across the two snapshots (i.e., purple and orange circles are close together in Figure~\ref{fig:app_genre_comparison}). This is because these genres are highly obfuscated from the beginning. Finally, the four genres, including Simulation and Role Playing, have a lower percentage of obfuscated apps in the 2021-2023 dataset. Our manual analysis didn't result in a conclusion as to why.


\begin{figure}[!h]
    \centering
    \includegraphics[width=\linewidth]{Figures/AppGenreTechAllV5.pdf}
    \caption{Obfuscation technique usage by genre (overall)}
    \label{fig:app_genre_all_tech}
\end{figure}


\subsubsection{Obfuscation techniques in different app genres} In Figure~\ref{fig:app_genre_all_tech}, we show the prevalence of key obfuscation techniques among various genres. As expected, almost all obfuscated apps in all genres used  Identifier Renaming. Also, it can be noted that genres with more obfuscated app percentages tend to use all three obfuscation techniques. Notably, more than 85\% of \textit{Casino} genre apps employ multiple obfuscation techniques

\subsubsection{Obfuscation tool usage in different app genres} We also investigated whether specific obfuscation tools are favoured by developers in different genres. However, apart from the expected observation that  ProGuard and Allatori being the most used tools, we didn't find any other interesting patterns. Therefore, we haven't included those measurement results.

\subsection{App Developers}
Next, we investigate individual developer-wise code obfuscation practices. From the pool of analyzed APKs, we identified the number of apps associated with each developer. Subsequently, we sorted the developers according to the number of apps they had created and selected the top 100 developers with the highest number of APKs for the 2016-2018 and 2021-2023 datasets. For the 2018 snapshot, we had 8,349 apps among the top 100 developers, while for the 2023 snapshot, we had 11,338 apps among the top 100 developers.

We then proceeded to detect whether or not these developers obfuscate their apps and, if so, what kind of tools and techniques they use. We present our results in five levels; developer obfuscating over 80\% of their apps, 60\%--80\% of apps, 40\%--60\% of apps, less than 40\%, and no obfuscation.

Figure~\ref{fig:developer_trend_my_apps_all} compares the two datasets in terms of developer obfuscation adoption. It shows that more developers have moved to obfuscate more than 80\% of their apps in the 2021-2023 dataset (76\%) compared to the 2016-2018 dataset (48\%).

We also found that among developers who obfuscate more than 80\% of their apps, 73\% in 2018 and 93\% in 2023 used the same obfuscation tool. Additionally, these top developers employ Control Flow Modification (CF) and String Encryption (SE) above the average values discussed in Section~\ref{sec:obstrend}. Specifically, in 2018, top developers used CF in 81.3\% of cases and SE in 66.7\%, while in 2023, these figures increased to 88.2\% and 78.9\%. This results in two insights: 1) Most top developers obfuscate all their apps with advanced techniques, possibly due to concerns about IP and security, and 2) Developers stick to a single tool, possibly due to specialized knowledge or because they bought a commercial licence.

\begin{figure}[]
    \centering
    \includegraphics[width=\linewidth]{Figures/Developer_Analysed_Comparison.pdf}
    \caption{Obfuscation usage (Top-100 developers)}
    \label{fig:developer_trend_my_apps_all}
\end{figure}


Finally, we investigate the obfuscation practices of developers with only one app in Table~\ref{tab:my-table}. According to the table, from those developers, 45.5\% of them obfuscated their apps in the 2016-2018 dataset and 57.2\% obfuscated their apps in the 2021-2023 dataset, showing a clear increase. However, these percentages are approximately 10\% lower than the average obfuscation rate in both cohorts discussed in Section~\ref{sec:obstrend}. This indicates that single-app developers may be less aware or concerned about code protection.


\begin{table}[]
\caption{Developers with only one app}
\label{tab:my-table}
\resizebox{\columnwidth}{!}{%
\begin{tabular}{cccccc}
\hline
\textbf{Year} & \textbf{\begin{tabular}[c]{@{}c@{}}Non\\ Obfuscated\end{tabular}} & \multicolumn{4}{c}{\textbf{Obfuscated}} \\ \hline
\multirow{3}{*}{\textbf{\begin{tabular}[c]{@{}c@{}}2018 \\ Snapshot\end{tabular}}} & \multirow{3}{*}{\begin{tabular}[c]{@{}c@{}}26,581 \\ (54.5\%)\end{tabular}} & \multicolumn{4}{c}{\begin{tabular}[c]{@{}c@{}}22,214 (45.5\%)\end{tabular}} \\ \cline{3-6} 
 &  & \textbf{ProGuard} & \textbf{Allatori} & \textbf{DashO} & \textbf{Other} \\ \cline{3-6} 
 &  & 6,131 & 8,050 & 658 & 7,375 \\ \hline
\multirow{3}{*}{\textbf{\begin{tabular}[c]{@{}c@{}}2023 \\ Snapshot\end{tabular}}} & \multirow{3}{*}{\begin{tabular}[c]{@{}c@{}}19,510 \\ (42.8\%)\end{tabular}} & \multicolumn{4}{c}{\begin{tabular}[c]{@{}c@{}}26,084 (57.2\%)\end{tabular}} \\ \cline{3-6} 
 &  & \textbf{ProGuard} & \textbf{Allatori} & \textbf{DashO} & \textbf{Other} \\ \cline{3-6} 
 &  & 12,697 & 9,672 & 234 & 3,581 \\ \hline
\end{tabular}%
}
\end{table}

\subsection{Top-k Apps}

Next, we investigate the obfuscation practices of top apps in Google Play Store. First, we rank the apps using the same criterion used by our previous work~\cite{rajasegaran2019multi, karunanayake2020multi, seneviratne2015early}. That is, we sort the apps in descending order of number of downloads, average rating, and rating count, with the intuition that top apps have high download numbers and high ratings, even when reviewed by a large number of users. Then, we investigated the percentage of obfuscated apps and obfuscation tools and technique usage as summarized in Table~\ref{tab:top_k_apps_2018_2023}.

When considering the highly ranked applications (i.e., top-1,000), the obfuscation percentage is notably higher, at around 93\%, in both datasets, which is significantly higher than the average percentage of obfuscation we observed in Section~\ref{sec:obstrend}. Top-ranked apps, likely due to their higher visibility and potential revenue, invest more in obfuscation to safeguard their intellectual property and enhance security. 

The obfuscation percentage decreases when going from the top 1,000 apps to the top 30,000 apps. Nonetheless, the obfuscation percentage in both datasets remains around similar values until the top 30,000 (e.g., $\sim$74\% for top-30,000). This indicates that the major increase in obfuscation in the 2021-2023 dataset comes from apps beyond the top 30,000.

When observing the tools used, the usage of ProGuard increases as we move from top to lower-ranked apps in both datasets. This may be because ProGuard is free and the default in Android Studio, while commercial tools like Allatori and DashO are expensive. There is a notable increase in the use of Allatori among the top apps in the 2021-2023 dataset. Regarding obfuscation techniques, the top 1,000 apps utilize all three techniques more frequently than lower-ranked apps in both snapshots. This indicates that the top 1,000 apps are more heavily protected compared to lower-ranked ones.

\begin{table*}[]
\caption{Summary of analysis results for Top-k apps in 2018 and 2023}
\label{tab:top_k_apps_2018_2023}
\resizebox{\textwidth}{!}{%
\begin{tabular}{lccccccccc}
\hline
\multicolumn{1}{c}{\begin{tabular}[c]{@{}c@{}}Top k apps - \\ Year\end{tabular}} & \begin{tabular}[c]{@{}c@{}}Total \\ Apps\end{tabular} & \begin{tabular}[c]{@{}c@{}}Obfuscation\\ Percentage\end{tabular} & \begin{tabular}[c]{@{}c@{}}ProGuard\\ Percentage\end{tabular} & \begin{tabular}[c]{@{}c@{}}Allatori\\ Percentage\end{tabular} & \begin{tabular}[c]{@{}c@{}}DashO\\ Percentage\end{tabular} & \begin{tabular}[c]{@{}c@{}}Other\\ Percentage\end{tabular} & \begin{tabular}[c]{@{}c@{}}IR\\ Percentage\end{tabular} & \begin{tabular}[c]{@{}c@{}}CF\\ Percentage\end{tabular} & \begin{tabular}[c]{@{}c@{}}SE\\ Percentage\end{tabular} \\ \hline
1k (2018) & 1,000 & 93.40 & 29.98 & 28.48 & 0.64 & 40.90 & 99.90 & 88.76 & 65.42 \\
10k (2018) & 10,000 & 85.19 & 25.55 & 35.32 & 0.47 & 38.65 & 99.90 & 88.76 & 71.91 \\
20k (2018) & 20,000 & 78.42 & 26.31 & 36.76 & 0.57 & 36.36 & 99.87 & 87.37 & 71.49 \\
30k (2018) & 30,000 & 74.40 & 27.30 & 37.71 & 0.64 & 34.36 & 99.82 & 86.75 & 71.11 \\
30k+ (2018) & 314,568 & 53.36 & 36.72 & 34.70 & 1.33 & 27.24 & 99.34 & 83.54 & 63.11 \\ \hline
1k (2023) & 1,000 & 92.50 & 24.00 & 51.89 & 1.95 & 22.16 & 100.0 & 92.54 & 83.68 \\
10k (2023) & 10,000 & 81.88 & 26.03 & 56.20 & 1.03 & 16.74 & 99.89 & 89.40 & 82.01 \\
20k (2023) & 20,000 & 76.62 & 30.48 & 52.92 & 0.96 & 15.64 & 99.93 & 85.80 & 78.01 \\
30k (2023) & 30,000 & 73.72 & 33.87 & 50.34 & 0.89 & 14.90 & 99.95 & 83.31 & 75.34 \\
30k+ (2023) & 206,216 & 61.90 & 46.56 & 38.21 & 0.64 & 14.59 & 99.97 & 77.51 & 62.50 \\ \hline
\end{tabular}%
}
\end{table*}


We present RiskHarvester, a risk-based tool to compute a security risk score based on the value of the asset and ease of attack on a database. We calculated the value of asset by identifying the sensitive data categories present in a database from the database keywords. We utilized data flow analysis, SQL, and Object Relational Mapper (ORM) parsing to identify the database keywords. To calculate the ease of attack, we utilized passive network analysis to retrieve the database host information. To evaluate RiskHarvester, we curated RiskBench, a benchmark of 1,791 database secret-asset pairs with sensitive data categories and host information manually retrieved from 188 GitHub repositories. RiskHarvester demonstrates precision of (95\%) and recall (90\%) in detecting database keywords for the value of asset and precision of (96\%) and recall (94\%) in detecting valid hosts for ease of attack. Finally, we conducted an online survey to understand whether developers prioritize secret removal based on security risk score. We found that 86\% of the developers prioritized the secrets for removal with descending security risk scores.

% \section*{APPENDIX}
% Appendixes should appear before the acknowledgment.

% \section*{ACKNOWLEDGMENT}
% The authors would like to thank...

% \section*{References}
\begin{thebibliography}{00}

\bibitem{EnvironmentalPollution} Liu W, Xu Y, Fan D, et al. Alleviating corporate environmental pollution threats toward public health and safety: the role of smart city and artificial intelligence[J]. Safety Science, 2021, 143: 105433.

\bibitem{PervasiveandMobileComputing} Nagy A M, Simon V. Survey on traffic prediction in smart cities[J]. Pervasive and Mobile Computing, 2018, 50: 148-163.

\bibitem{ExpertSystems} Pramanik M I, Lau R Y K, Demirkan H, et al. Smart health: Big data enabled health paradigm within smart cities[J]. Expert Systems with Applications, 2017, 87: 370-383.

\bibitem{ITJ19} Kong X, Liu X, Jedari B, et al. Mobile crowdsourcing in smart cities: Technologies, applications, and future challenges[J]. IEEE Internet of Things Journal, 2019, 6(5): 8095-8113.

\bibitem{BigData} Zhai S, Li R, Yang Y. Crowdsensing big data: sensing, data selection, and understanding[C]//Journal of Physics: Conference Series. IOP Publishing, 2021, 1848(1): 012045.

\bibitem{TMC19} Liu C H, Dai Z, Zhao Y, et al. Distributed and energy-efficient mobile crowdsensing with charging stations by deep reinforcement learning[J]. IEEE Transactions on Mobile Computing, 2019, 20(1): 130-146.

% \bibitem{INFO23} Sun J, Jin H, Ding R, et al. Multi-Objective Order Dispatch for Urban Crowd Sensing with For-Hire Vehicles[C]//IEEE INFOCOM 2023-IEEE Conference on Computer Communications. IEEE, 2023: 1-10.

\bibitem{INFO21} Fan G, Zhao Y, Guo Z, et al. Towards fine-grained spatio-temporal coverage for vehicular urban sensing systems[C]//IEEE INFOCOM 2021-IEEE Conference on Computer Communications. IEEE, 2021: 1-10.


\bibitem{INFO23Privacy}Sun H, Xiao M, Xu Y, et al. Privacy-preserving Stable Crowdsensing Data Trading for Unknown Market[C]//IEEE INFOCOM 2023-IEEE Conference on Computer Communications. IEEE, 2023: 1-10.


\bibitem{INFO22Privacy} Wang H, Wang E, Yang Y, et al. Privacy-preserving online task assignment in spatial crowdsourcing: A graph-based approach[C]//IEEE INFOCOM 2022-IEEE Conference on Computer Communications. IEEE, 2022: 570-579.

\bibitem{TITS2020Privacy} Zhang J, Yang F, Ma Z, et al. A decentralized location privacy-preserving spatial crowdsourcing for internet of vehicles[J]. IEEE Transactions on Intelligent Transportation Systems, 2020, 22(4): 2299-2313.

\bibitem{vehiclecrowdsensingdeepapproach} Zhu X, Luo Y, Liu A, et al. A deep learning-based mobile crowdsensing scheme by predicting vehicle mobility[J]. IEEE Transactions on Intelligent Transportation Systems, 2020, 22(7): 4648-4659.

\bibitem{ICDE23} Ye Y, Liu C H, Dai Z, et al. Exploring both individuality and cooperation for air-ground spatial crowdsourcing by multi-agent deep reinforcement learning[C]//2023 IEEE 39th International Conference on Data Engineering (ICDE). IEEE, 2023: 205-217.

\bibitem{KDD17} Zhang L, Hu T, Min Y, et al. A taxi order dispatch model based on combinatorial optimization[C]//Proceedings of the 23rd ACM SIGKDD international conference on knowledge discovery and data mining. 2017: 2151-2159.

\bibitem{WWW19} Yuen C F, Singh A P, Goyal S, et al. Beyond shortest paths: Route recommendations for ride-sharing[C]//The World Wide Web Conference. 2019: 2258-2269.

\bibitem{Ubiquitous18} Xie X, Zhang F, Zhang D. PrivateHunt: Multi-source data-driven dispatching in for-hire vehicle systems[J]. Proceedings of the ACM on Interactive, Mobile, Wearable and Ubiquitous Technologies, 2018, 2(1): 1-26.

\bibitem{vehicle_dispatch_modle_based} Zhang R, Rossi F, Pavone M. Model predictive control of autonomous mobility-on-demand systems[C]//2016 IEEE international conference on robotics and automation (ICRA). IEEE, 2016: 1382-1389.

\bibitem{introduce_drl} Arulkumaran K, Deisenroth M P, Brundage M, et al. Deep reinforcement learning: A brief survey[J]. IEEE Signal Processing Magazine, 2017, 34(6): 26-38.

\bibitem{TITS22} Haydari A, Yılmaz Y. Deep reinforcement learning for intelligent transportation systems: A survey[J]. IEEE Transactions on Intelligent Transportation Systems, 2020, 23(1): 11-32.

\bibitem{INFO20} Liu C H, Dai Z, Yang H, et al. Multi-task-oriented vehicular crowdsensing: A deep learning approach[C]//IEEE INFOCOM 2020-IEEE Conference on Computer Communications. IEEE, 2020: 1123-1132.

\bibitem{INFO21VEHICLES} Ding R, Yang Z, Wei Y, et al. Multi-agent reinforcement learning for urban crowd sensing with for-hire vehicles[C]//IEEE INFOCOM 2021-IEEE Conference on Computer Communications. IEEE, 2021: 1-10.

\bibitem{INFO18} Oda T, Joe-Wong C. MOVI: A model-free approach to dynamic fleet management[C]//IEEE INFOCOM 2018-IEEE Conference on Computer Communications. IEEE, 2018: 2708-2716.

\bibitem{TITS20} Liu Z, Li J, Wu K. Context-aware taxi dispatching at city-scale using deep reinforcement learning[J]. IEEE Transactions on Intelligent Transportation Systems, 2020, 23(3): 1996-2009.

\bibitem{KDD18} Lin K, Zhao R, Xu Z, et al. Efficient large-scale fleet management via multi-agent deep reinforcement learning[C]//Proceedings of the 24th ACM SIGKDD international conference on Knowledge Discovery and Data Mining. 2018: 1774-1783.

\bibitem{KDD22} Sun J, Jin H, Yang Z, et al. Optimizing long-term efficiency and fairness in ride-hailing via joint order dispatching and driver repositioning[C]//Proceedings of the 28th ACM SIGKDD Conference on Knowledge Discovery and Data Mining. 2022: 3950-3960.

\bibitem{introduce_AoI1} Zhang S, Zhang H, Han Z, et al. Age of information in a cellular internet of UAVs: Sensing and communication trade-off design[J]. IEEE Transactions on Wireless Communications, 2020, 19(10): 6578-6592.

\bibitem{introduce_AoI2} Chen H, Gu Y, Liew S C. Age-of-information dependent random access for massive IoT networks[C]//IEEE INFOCOM 2020-IEEE Conference on Computer Communications Workshops (INFOCOM WKSHPS). IEEE, 2020: 930-935.

\bibitem{the_reason_why_data_freshness_is_important} Abd-Elmagid M A, Pappas N, Dhillon H S. On the role of age of information in the Internet of Things[J]. IEEE Communications Magazine, 2019, 57(12): 72-77.

\bibitem{a_example_for_the_reason_why_data_freshness_is_important} Cheng Y, Wang X, Zhou P, et al. Freshness-Aware Incentive Mechanism for Mobile Crowdsensing with Budget Constraint[J]. IEEE Transactions on Services Computing, 2023.

\bibitem{why_use_MARL}Zhang K, Yang Z, Başar T. Multi-agent reinforcement learning: A selective overview of theories and algorithms[J]. Handbook of reinforcement learning and control, 2021: 321-384.

\bibitem{R-GCN} Schlichtkrull M, Kipf T N, Bloem P, et al. Modeling relational data with graph convolutional networks[C]//The Semantic Web: 15th International Conference, ESWC 2018, Heraklion, Crete, Greece, June 3–7, 2018, Proceedings 15. Springer International Publishing, 2018: 593-607.

\bibitem{liyihong} Li Y, Zeng T, Zhang X, et al. TapFinger: Task Placement and Fine-Grained Resource Allocation for Edge Machine Learning[C]//IEEE INFOCOM. 2023.

\bibitem{neurips22} Yu C, Velu A, Vinitsky E, et al. The surprising effectiveness of ppo in cooperative multi-agent games[J]. Advances in Neural Information Processing Systems, 2022, 35: 24611-24624.

\bibitem{PPO} Schulman J, Wolski F, Dhariwal P, et al. Proximal policy optimization algorithms[J]. arXiv preprint arXiv:1707.06347, 2017.


\bibitem{newyork_taxi} https://www1.nyc.gov/site/tlc/about/tlc-trip-record-data.page.

\bibitem{assume_cost_one_time_slot_to_reach_dispatched_destination} Jin J, Zhou M, Zhang W, et al. Coride: joint order dispatching and fleet management for multi-scale ride-hailing platforms[C]//Proceedings of the 28th ACM international conference on information and knowledge management. 2019: 1983-1992.

\bibitem{UCB} Kuleshov V, Precup D. Algorithms for multi-armed bandit problems[J]. arXiv preprint arXiv:1402.6028, 2014.

\bibitem{espeholt2018impala} Espeholt L, Soyer H, Munos R, et al. Impala: Scalable distributed deep-rl with importance weighted actor-learner architectures[C]//International conference on machine learning. PMLR, 2018: 1407-1416.

\bibitem{li2019privacy} Li M, Zhu L, Lin X. Privacy-preserving traffic monitoring with false report filtering via fog-assisted vehicular crowdsensing[J]. IEEE Transactions on Services Computing, 2019, 14(6): 1902-1913.
% \bibitem{StreetView} Anguelov D, Dulong C, Filip D, et al. Google street view: Capturing the world at street level[J]. Computer, 2010, 43(6): 32-38.
\bibitem{zhang2021data}Zhang X, Wang J, Zhang H, et al. Data-driven transportation network company vehicle scheduling with users’ location differential privacy preservation[J]. IEEE Transactions on Mobile Computing, 2021, 22(2): 813-823.
\bibitem{liu2024multi}Liu L, Huang Z, Xu J. Multi-Agent Deep Reinforcement Learning Based Scheduling Approach for Mobile Charging in Internet of Electric Vehicles[J]. IEEE Transactions on Mobile Computing, 2024.
% \bibitem{INFO22} Xiang C, Li Y, Zhou Y, et al. A comparative approach to resurrecting the market of mod vehicular crowdsensing[C]//IEEE INFOCOM 2022-IEEE Conference on Computer Communications. IEEE, 2022: 1479-1488.

% \bibitem{ICDE23UAV} Wang Y, Wu J, Hua X, et al. Air-ground spatial crowdsourcing with uav carriers by geometric graph convolutional multi-agent deep reinforcement learning[C]//2023 IEEE 39th International Conference on Data Engineering (ICDE). IEEE, 2023: 1790-1802.


\end{thebibliography}
\vspace{12pt}
\color{red}


\end{document}
