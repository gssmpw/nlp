%% 
%% Copyright 2007-2025 Elsevier Ltd
%% 
%% This file is part of the 'Elsarticle Bundle'.
%% ---------------------------------------------
%% 
%% It may be distributed under the conditions of the LaTeX Project Public
%% License, either version 1.3 of this license or (at your option) any
%% later version.  The latest version of this license is in
%%    http://www.latex-project.org/lppl.txt
%% and version 1.3 or later is part of all distributions of LaTeX
%% version 1999/12/01 or later.
%% 
%% The list of all files belonging to the 'Elsarticle Bundle' is
%% given in the file `manifest.txt'.
%% 
%% Template article for Elsevier's document class `elsarticle'
%% with numbered style bibliographic references
%% SP 2008/03/01
%% $Id: elsarticle-template-num.tex 272 2025-01-09 17:36:26Z rishi $
%%
\documentclass[preprint,12pt]{elsarticle}
\usepackage[T1]{fontenc}
%% Use the option review to obtain double line spacing
%% \documentclass[authoryear,preprint,review,12pt]{elsarticle}

%% Use the options 1p,twocolumn; 3p; 3p,twocolumn; 5p; or 5p,twocolumn
%% for a journal layout:
%% \documentclass[final,1p,times]{elsarticle}
%% \documentclass[final,1p,times,twocolumn]{elsarticle}
%% \documentclass[final,3p,times]{elsarticle}
%% \documentclass[final,3p,times,twocolumn]{elsarticle}
%% \documentclass[final,5p,times]{elsarticle}
%% \documentclass[final,5p,times,twocolumn]{elsarticle}
\usepackage{multirow}
\usepackage{multicol}
\usepackage{lscape}
\usepackage{adjustbox}
% \usepackage{pbox, cellspace}
% \cellspacetoplimit = 2pt\cellspacebottomlimit =2pt
\renewcommand{\arraystretch}{2}
\usepackage{makecell}
\usepackage[justification=raggedright]{caption}
\usepackage{geometry}
\usepackage{longtable}
\usepackage{lscape}
\usepackage{pdflscape}
\usepackage{afterpage}
% \usepackage{longtable}
\usepackage{rotating}
\usepackage{ltablex}
\usepackage{array,longtable}
\usepackage{pdflscape}
\usepackage{xltabular}
\usepackage{url}
\usepackage{graphicx}
\usepackage{float}
\usepackage{threeparttable}
% \usepackage{array}
% \usepackage{amssymb}
\usepackage{pifont}
\newcommand{\cmark}{\ding{51}}  % Check mark
\newcommand{\xmark}{\textcolor{red}{\ding{55}}}  % ✗ (Cross mark in red)
\setcounter{secnumdepth}{4}
\makeatletter
\renewcommand{\paragraph}{\@startsection{paragraph}{4}{\z@}{1ex}{-1em}{\normalfont\normalsize}}
\makeatother
\usepackage{tikz}
\usetikzlibrary{shapes.geometric, arrows,positioning}
\usepackage{titlesec}
\titleformat{\subsubsection}[runin]{\normalfont\itshape}{\thesubsubsection}{0.3em}{}[]
\setlength{\belowcaptionskip}{-10pt}
\titlespacing*{\section}{0pt}{0.1\baselineskip}{0.1\baselineskip}
\titlespacing*{\subsection}{0pt}{0.1\baselineskip}{0.1\baselineskip}
\usepackage{etoolbox}
\makeatletter
\def\ifGm@preamble#1{\@firstofone}
\appto\restoregeometry{%
  \pdfpagewidth=\paperwidth
  \pdfpageheight=\paperheight}
\apptocmd\newgeometry{%
  \pdfpagewidth=\paperwidth
  \pdfpageheight=\paperheight}{}{}
\makeatother
% \usepackage{enumitem}
% \captionsetup[table]{skip=10pt}
% \titleformat{\section}%
%             {\Large\bfseries}% format
%             {\llap{% label
%                \thesection\hskip 9pt}}%
%             {0pt}% horizontal sep
%             {}% before

% \titleformat{\subsection}%
%         {\bfseries}% format
%         {\llap{% label
%            \thesubsection\hskip 9pt}}%
%         {0.3pt}% horizontal sep
%         {}% before
% \setlist[enumerate]{leftmargin=1mm}
% \setlist[itemize]{leftmargin=1mm}
\usepackage{caption} 
\usepackage[final]{pdfpages}
\captionsetup[table]{skip=10pt}
%% For including figures, graphicx.sty has been loaded in
%% elsarticle.cls. If you prefer to use the old commands
%% please give \usepackage{epsfig}

%% The amssymb package provides various useful mathematical symbols
\usepackage{amssymb}
%% The amsmath package provides various useful equation environments.
\usepackage{amsmath}
%% The amsthm package provides extended theorem environments
%% \usepackage{amsthm}

%% The lineno packages adds line numbers. Start line numbering with
%% \begin{linenumbers}, end it with \end{linenumbers}. Or switch it on
%% for the whole article with \linenumbers.
%% \usepackage{lineno}

\journal{Neurocomputing}

\begin{document}

\begin{frontmatter}

%% Title, authors and addresses

%% use the tnoteref command within \title for footnotes;
%% use the tnotetext command for theassociated footnote;
%% use the fnref command within \author or \affiliation for footnotes;
%% use the fntext command for theassociated footnote;
%% use the corref command within \author for corresponding author footnotes;
%% use the cortext command for theassociated footnote;
%% use the ead command for the email address,
%% and the form \ead[url] for the home page:
%% \title{Title\tnoteref{label1}}
%% \tnotetext[label1]{}
%% \author{Name\corref{cor1}\fnref{label2}}
%% \ead{email address}
%% \ead[url]{home page}
%% \fntext[label2]{}
%% \cortext[cor1]{}
%% \affiliation{organization={},
%%             addressline={},
%%             city={},
%%             postcode={},
%%             state={},
%%             country={}}
%% \fntext[label3]{}

\title{Vision Transformers on the Edge: A Comprehensive Survey of Model Compression and Acceleration Strategies}

%% use optional labels to link authors explicitly to addresses:
% \author[label1,Label2]{Shaibal Saha, Lanyu Xu}
% \affiliation[label1,Label2]{organization={Department of Computer Science and Engineering, Oakland University},%Department and Organization
%             addressline={}, 
%             city={Rochester Hills},
%             postcode={48309}, 
%             state={Michigan},
%             country={USA}}

\author{Shaibal Saha\corref{cor1}\fnref{label1}}
\ead{shaibalsaha@oakland.edu}
\author{Lanyu Xu\fnref{label1}}
% \ead{lxu@oakland.edu}
\affiliation[label1]{organization={Department of Computer Science and Engineering, Oakland University},%Department and Organization
            addressline={}, 
            city={Rochester Hills},
            postcode={48309}, 
            state={Michigan},
            country={USA}}
\cortext[cor1]{Corresponding author}
%%
%% \affiliation[label2]{organization={},
%%             addressline={},
%%             city={},
%%             postcode={},
%%             state={},
%%             country={}}

% \author{Shaibal Saha, Lanyu Xu} %% Author name

%% Author affiliation
% \affiliation{organization={Department of Computer Science and Engineering, Oakland University},%Department and Organization
%             addressline={}, 
%             city={Rochester Hills},
%             postcode={48309}, 
%             state={Michigan},
%             country={USA}}


%% Abstract
\begin{abstract}
%% Text of abstract
In recent years, vision transformers (ViTs) have emerged as powerful and promising techniques for computer vision tasks such as image classification, object detection, and segmentation. Unlike convolutional neural networks (CNNs), which rely on hierarchical feature extraction, ViTs treat images as sequences of patches and leverage self-attention mechanisms. However, their high computational complexity and memory demands pose significant challenges for deployment on resource-constrained edge devices. To address these limitations, extensive research has focused on model compression techniques and hardware-aware acceleration strategies. Nonetheless, a comprehensive review that systematically categorizes these techniques and their trade-offs in accuracy, efficiency, and hardware adaptability for edge deployment remains lacking. This survey bridges this gap by providing a structured analysis of model compression techniques, software tools for inference on edge, and hardware acceleration strategies for ViTs. We discuss their impact on accuracy, efficiency, and hardware adaptability, highlighting key challenges and emerging research directions to advance ViT deployment on edge platforms, including graphics processing units (GPUs), tensor processing units (TPUs), and field-programmable gate arrays (FPGAs). The goal is to inspire further research with a contemporary guide on optimizing ViTs for efficient deployment on edge devices.
\end{abstract}

%%Graphical abstract
% \begin{graphicalabstract}
% %\includegraphics{grabs}
% \end{graphicalabstract}

%%Research highlights
% \begin{highlights}
% % %% Need to update
% \item The survey highlights model compression for optimizing vision transformers (ViTs).
% \item We explore cutting-edge software tools for efficient ViT deployment on edge devices.
% \item Additionally, we discuss hardware-aware accelerating strategies for ViTs on the edge.
% \item Discuss key challenges and future directions to enhance ViT deployment on edge.
% \end{highlights}

%% Keywords
\begin{keyword}
Vision Transformer \sep Model Compression \sep Pruning \sep Quantization \sep Hardware Accelerators \sep Edge Computing 
%% keywords here, in the form: keyword \sep keyword

%% PACS codes here, in the form: \PACS code \sep code

%% MSC codes here, in the form: \MSC code \sep code
%% or \MSC[2008] code \sep code (2000 is the default)

\end{keyword}

\end{frontmatter}
\section{Introduction}

Video generation has garnered significant attention owing to its transformative potential across a wide range of applications, such media content creation~\citep{polyak2024movie}, advertising~\citep{zhang2024virbo,bacher2021advert}, video games~\citep{yang2024playable,valevski2024diffusion, oasis2024}, and world model simulators~\citep{ha2018world, videoworldsimulators2024, agarwal2025cosmos}. Benefiting from advanced generative algorithms~\citep{goodfellow2014generative, ho2020denoising, liu2023flow, lipman2023flow}, scalable model architectures~\citep{vaswani2017attention, peebles2023scalable}, vast amounts of internet-sourced data~\citep{chen2024panda, nan2024openvid, ju2024miradata}, and ongoing expansion of computing capabilities~\citep{nvidia2022h100, nvidia2023dgxgh200, nvidia2024h200nvl}, remarkable advancements have been achieved in the field of video generation~\citep{ho2022video, ho2022imagen, singer2023makeavideo, blattmann2023align, videoworldsimulators2024, kuaishou2024klingai, yang2024cogvideox, jin2024pyramidal, polyak2024movie, kong2024hunyuanvideo, ji2024prompt}.


In this work, we present \textbf{\ours}, a family of rectified flow~\citep{lipman2023flow, liu2023flow} transformer models designed for joint image and video generation, establishing a pathway toward industry-grade performance. This report centers on four key components: data curation, model architecture design, flow formulation, and training infrastructure optimization—each rigorously refined to meet the demands of high-quality, large-scale video generation.


\begin{figure}[ht]
    \centering
    \begin{subfigure}[b]{0.82\linewidth}
        \centering
        \includegraphics[width=\linewidth]{figures/t2i_1024.pdf}
        \caption{Text-to-Image Samples}\label{fig:main-demo-t2i}
    \end{subfigure}
    \vfill
    \begin{subfigure}[b]{0.82\linewidth}
        \centering
        \includegraphics[width=\linewidth]{figures/t2v_samples.pdf}
        \caption{Text-to-Video Samples}\label{fig:main-demo-t2v}
    \end{subfigure}
\caption{\textbf{Generated samples from \ours.} Key components are highlighted in \textcolor{red}{\textbf{RED}}.}\label{fig:main-demo}
\end{figure}


First, we present a comprehensive data processing pipeline designed to construct large-scale, high-quality image and video-text datasets. The pipeline integrates multiple advanced techniques, including video and image filtering based on aesthetic scores, OCR-driven content analysis, and subjective evaluations, to ensure exceptional visual and contextual quality. Furthermore, we employ multimodal large language models~(MLLMs)~\citep{yuan2025tarsier2} to generate dense and contextually aligned captions, which are subsequently refined using an additional large language model~(LLM)~\citep{yang2024qwen2} to enhance their accuracy, fluency, and descriptive richness. As a result, we have curated a robust training dataset comprising approximately 36M video-text pairs and 160M image-text pairs, which are proven sufficient for training industry-level generative models.

Secondly, we take a pioneering step by applying rectified flow formulation~\citep{lipman2023flow} for joint image and video generation, implemented through the \ours model family, which comprises Transformer architectures with 2B and 8B parameters. At its core, the \ours framework employs a 3D joint image-video variational autoencoder (VAE) to compress image and video inputs into a shared latent space, facilitating unified representation. This shared latent space is coupled with a full-attention~\citep{vaswani2017attention} mechanism, enabling seamless joint training of image and video. This architecture delivers high-quality, coherent outputs across both images and videos, establishing a unified framework for visual generation tasks.


Furthermore, to support the training of \ours at scale, we have developed a robust infrastructure tailored for large-scale model training. Our approach incorporates advanced parallelism strategies~\citep{jacobs2023deepspeed, pytorch_fsdp} to manage memory efficiently during long-context training. Additionally, we employ ByteCheckpoint~\citep{wan2024bytecheckpoint} for high-performance checkpointing and integrate fault-tolerant mechanisms from MegaScale~\citep{jiang2024megascale} to ensure stability and scalability across large GPU clusters. These optimizations enable \ours to handle the computational and data challenges of generative modeling with exceptional efficiency and reliability.


We evaluate \ours on both text-to-image and text-to-video benchmarks to highlight its competitive advantages. For text-to-image generation, \ours-T2I demonstrates strong performance across multiple benchmarks, including T2I-CompBench~\citep{huang2023t2i-compbench}, GenEval~\citep{ghosh2024geneval}, and DPG-Bench~\citep{hu2024ella_dbgbench}, excelling in both visual quality and text-image alignment. In text-to-video benchmarks, \ours-T2V achieves state-of-the-art performance on the UCF-101~\citep{ucf101} zero-shot generation task. Additionally, \ours-T2V attains an impressive score of \textbf{84.85} on VBench~\citep{huang2024vbench}, securing the top position on the leaderboard (as of 2025-01-25) and surpassing several leading commercial text-to-video models. Qualitative results, illustrated in \Cref{fig:main-demo}, further demonstrate the superior quality of the generated media samples. These findings underscore \ours's effectiveness in multi-modal generation and its potential as a high-performing solution for both research and commercial applications.
% \section{ViT based medical imaging models}\label{vit}
Transformers were introduced by Vaswani et al. ~\cite{vaswani2017attention} as a new attention-driven building block for machine translation. As we focus on only ViT in this survey, Dosovitskiy et al.~\cite{dosovitskiy2020image} first proposed ViT that is built on a vanilla Transformer model by combining multiple transformer layers to capture the global context of an input image. The authors presented a novel approach where images were interpreted as sequences of patches and then processed using standard transformer encoders commonly used in NLP. By eliminating hand-crafted visual features and inductive biases, ViT architecture takes advantage of large datasets and increases computational capacity to achieve impressive results. In particular, ViT has garnered significant interest in the medical imaging community, leading to the development of various approaches that build upon and extend the ViT architecture for diverse medical imaging applications. Shamshad et al.~\cite{shamshad2022transformers} showed that most of the medical imaging tasks focused on medical imaging classification, medical imaging segmentation, and medical imaging detection. In our survey, we only focus on these three medical imaging tasks.
\subsection{Medical Image Classification}\label{classi} Medical image classification refers to categorizing medical images into predefined classes based on specific features or abnormalities in the image. Through image classification, medical professionals can identify and diagnose various health conditions, from tumors to fractures, by categorizing and analyzing these images. Medical image classification involves preparing the images, extracting relevant features or patterns from them, and then using these features or patterns to categorize the images into specific classes. Figure~\ref{fig:classification_img} illustrates a sample heatmap for medical image classification output from ResNet~\cite{he2016deep} \& MedViT~\cite{manzari2023medvit} on the MedMNIST-v2 dataset~\cite{yang2023medmnist} which classifying two organs: the Derma and Oct. Most of the ViT architectures in image classification are used for COVID-19 and tumor classifications. Most COVID-19 image classification papers are divided into 2D and 3D image classification categories. Tumors in organs like the brain, lungs, breast, stomach, and Kidneys are used for tumor classification.

A recent study named Transmed~\cite{dai2021transmed} is the first ViT-based multi-modal medical image classification approach. The authors addressed that the performance of ViT in multi-modal medical image classification was not significant due to the small size of medical datasets. So, the authors used both CNN and transformers to extract low-level features of images and capture the mutual information between modalities. The authors evaluated their proposed model with the PGT and MRNet datasets to classify the parotid tumors in the multi-modal MRI medical image classification. The PGT dataset has two MRI modalities (T1 and T2). The MRNet dataset has three MRI modalities (T1-weighted images, T2-weighted images, and proton density-weighted). The proposed method achieved a 10.1\% improvement in average accuracy compared to BoTNet ~\cite{srinivas2021bottleneck}. In recent studies, Vit architecture has been successfully applied to diagnose and predict COVID-19, which showed good performance. Perera et al.~\cite{perera2021pocformer} proposed Pocformer~\cite{perera2021pocformer} that diagnosed COVID-19 that relies on ultrasound clips using the COVID-19 lungs POCUS ~\cite{born2020pocovid} dataset. The authors utilized Linformer ~\cite{wang2020linformer} to transform the self-attention's space and time complexity from quadratic to linear. POCFormer possessed two million parameters, achieving an average accuracy of 91\% with a speed of 70 frames per second. Another study named Covid-Vit~\cite{gao2021covid} proposed ViT-based architecture to classify COVID-19 from CT lung images in the MIA-COV19 competition. Two deep learning techniques were examined: ViT, which relied on attention mechanisms, and DenseNet~\cite{huang2017densely}, which is grounded in the traditional CNN approach. Preliminary assessments using validation datasets, where the true values are established, suggest that ViT outperforms DenseNet~\cite{huang2017densely}, yielding F1 scores of 0.76 compared to DenseNet's 0.72. As COVID-19 disease classification was necessary due to the pandemic in recent times, another study named PneuNet~\cite{wang2023pneunet} where multi-head attention is applied on channel patches rather than feature patches. The authors used their datasets, combining datasets from multiple open sources. Their COVID-19 dataset was a combination of normal pneumonia chest X-rays and chest X-rays of healthy people and divided into three sub-datasets out of 33920 chest X-ray images.

Moreover, ViT architecture also gained popularity in cancer classification. Using different data augmentation strategies, Gheflati el al.~\cite{gheflati2022vision} applied ViT architecture for the first time in breast cancer classification (BUC). The authors used the same parameter and training procedure described in ~\cite{steiner2021train}. Eventually, they trained their dataset with different ViT architectures that can compare with SOTA CNN models. The ViT architectures used by the authors achieved almost the same results: 86\% accuracy and 0.95 corresponding AUC from two datasets ~\cite{al2020dataset,yap2017automated}. Rah et al.~\cite{raj2023strokevit} proposed a combined model using CNN, ViT, and AutoML to classify whether the stroke is hemorrhagic or ischemic. The authors used the CNN component to capture the local features within CT slices and the transformer component to recognize the long-range dependencies between
\begin{landscape}
%\input{Tables/table_segmentation}
\begin{table}
\input{Tables/table_classify}
 \vspace{1cm} %space between tables
\input{Tables/table_object}
\end{table}
\end{landscape} \noindent
sequences. The combination aimed to enhance the accuracy of slice-level predictions. Feature extracted from each slice-level prediction is later used to make a patient-wise prediction via AutoML. The proposed architecture helped to reduce the radiologist's manual effort to select the most imperative CT from the original CT volume and achieved 0.92 accuracy with AutoML.

Lastly, Table~\ref{classification} illustrated the different organ classification results from different ViT-based architectures. There are already some existing surveys published about ViT-based medical imaging~\cite{shamshad2023transformers,henry2022vision}. The main agenda of picking the papers here is to show that ViT can be applied for different organ classifications and achieve good accuracy. The discussed ViT-based models for image classification tasks can be extended for edge application after applying model compression techniques.

\begin{figure}[btp]
  \centering
  \includegraphics[scale=0.28]{assets/classification.png}
  \caption{A medical image classification sample heatmap output applied from ResNet~\cite{he2016deep} \& MedViT~\cite{manzari2023medvit} on the MedMNIST-v2 dataset~\cite{yang2023medmnist}. The image classified four organs: the Blood, Breast, Oct, and Pneumonia.}
  \label{fig:classification_img}
\end{figure}
\vspace{-10pt}
\subsection{Medical Image Detection}\label{detec}
Object detection in medical imaging involves identifying and localizing specific organs or abnormalities within medical images, such as X-rays, CT scans, MRI images, and ultrasounds. It plays a crucial role in tasks like tumor and anomaly detection, assisting healthcare professionals in making more informed decisions. Unlike general object detection tasks in images (like identifying cars or pedestrians in a street photo), medical image object detection focuses on healthcare applications and requires high precision and sensitivity. Nevertheless, this task often demands significant time from medical professionals. Thus, there is a pressing need for a precise Computer-aided diagnosis (CAD) system to serve as an auxiliary reviewer, potentially speeding up the procedure. Recent research shows that works like tumor detection, anatomical structure identification, abnormality detection, and blood vessel abnormality are primarily used in object detection sectors in medical imaging. Figure \ref{fig:object_image} shows a sample example of detection containing multiple objects and 

blood impurity~\cite{leng2023deep}. The figure detects the blood cell abnormality for four different deep-learning architectures. However, it is not easy to detect an organ, tumor, or any other body organ or abnormality because of data imbalance, noise, and the need for high precision. Most of the transformer-based models used 

DETR~\cite{zhu2020deformable} as baseline models because DETR is one of the first object detection architectures with transformers.\\

\begin{figure}[ptb]
  \centering
  \vspace{-10pt}
  \includegraphics[scale=0.45]{assets/object_det_up.png}
  \vspace{-10pt}
  \caption{A sample comparison of the detection results on the image containing multiple objects and impurity: (a) DETR-TL; (b) Improved-DETR-TL; (c) Improved-DETR-FS; (d)
Faster R-CNN-TL; (e) YOLO v3-TL; (f) the original image ~\cite{leng2023deep}}
  \label{fig:object_image}
\end{figure}
Zhang et al.~\cite{zhang2022lightweight} first proposed lightweight ViT-based medical image detection architecture where the authors modified the baseline ViT architecture on image feature patches, enhancing the backbone for tumor detection more robustly. The authors reconstructed the feature patches with the original shape of feature maps generated by ResNet. The authors evaluated their proposed architecture on the BCS-DBT~\cite{buda2021detection} dataset, using mAP and AP as metrics. The authors improved ~7.2\% AP value compared to R-CNN and swin transformer. Another study by Lin et al.~\cite{lin2021aanet} introduced the Adaptive Attention Network (AANet), a framework designed to dynamically extract the distinct radiographic markers of COVID-19 from infection zones that display varying scales and appearances. The proposed model consists of two primary modules: an adaptive deformable ResNet and an attention-driven encoder. Initially, the adaptive deformable ResNet was crafted to automatically adjust its receptive fields to grasp feature representations in line with the shape and scale of the infected regions, catering to the multifaceted radiographic features of COVID-19. Subsequently, the author created an attention-based encoder to capture nonlocal interactions using a self-attention mechanism. This approach allowed the model to understand complex shapes in lesion regions by incorporating rich context information.

Moreover, MST~\cite{shou2022object} considered low resolution and high noise images to detect small organs showed great potential on DeepLesion~\cite{yan2018deeplesion} and BCDD~\cite{BCCD_Dataset} datasets. The proposed model used self-supervised learning to perform random masking on the input image and then reconstruct input features to filter out noise. The authors introduced a hierarchical transformer model to detect small objects in the paper. Then, the authors used a sliding window with a local self-attention mechanism to assign high attention scores to these diminutive objects targeted for detection. Finally, they utilized a single-stage object detection framework to predict object classes and bounding box locations. Another study by Leng et al.~\cite{leng2023deep}  proposed a pure transformer-based end-to-end object detection network based on DETR~\cite{zhu2020deformable} model for leukocyte detection. The authors used a pyramid vision transformer as a backbone to improve the performance of the DETR model and evaluated their proposed architecture on Raabin~\cite{kouzehkanan2022large} datasets.

Another study~\cite{liu2022sfod} proposed  SFOD-Trans, a semi-supervised framework for fine-grained object detection, mainly focusing on hepatic portal vein detection. The authors developed a fusion module, termed normalized ROI fusion (NRF), to facilitate information transfer from labeled and unlabeled images. This study showed great potential for semi-supervised learning and evaluated on PASCAL VOC~\cite{everingham2015pascal} CT image dataset. Moreover, Wittmann et al. utilized ViT for 3D medical image detection and introduced a novel feature extraction backbone called SwinFPN~\cite{wittmann2022swinfpn}. The proposed model utilized the concept of shifted window-based self-attention. SwinFPN is integrated with the head networks of Retina U-Net, resulting in notably improved detection performance.

Table \ref{object_detection} summarized all the discussed architecture key points, datasets, and performance to give an overview of medical image detection. From our observation, medical image detection using ViT is still an explorable area and needs more elaborate study to achieve a better response in this sector.
\subsection{Medical Image Segmentation}\label{segm}Medical image segmentation is a fundamental task in the medical domain, playing a crucial role in various clinical applications. Segmentation involves dividing a digital image into multiple segments to simplify its representation or to make it more meaningful. Unlike image classification, which assigns a label to an entire image, segmentation aims to assign a label to each pixel. This granularity is essential in medical contexts where it is essential to differentiate between tissues, organs, tumors, and other anatomical structures. The outcome of this process provides a detailed map of these structures, aiding in diagnosis, treatment planning, and further medical research. Figure \ref{fig:segmentation_image} illustrates output for medical imaging segmentation for prostate organ segmentation on the dataset from the PROSTATEx-2 challenge~\cite{Litjens2014ComputerAidedDO}. The red shape indicates the prostate segmentation after applying nnunet~\cite{isensee2018nnu}. In recent research, we have observed that the majority of papers have used hybrid structures combining transformer architecture and CNN. As ViT gained much attention from the medical community for its performance, the community explored ViT for medical image segmentation. One of the previous surveys ~\cite{shamshad2022transformers} discussed most of the ViT-based architectures for segmentation tasks in detail, and the authors collected the paper until 2021. We will discuss published papers that were not included in ~\cite{shamshad2022transformers} or published after the survey except Unetr ~\cite{hatamizadeh2022unetr} in this survey because Unetr achieved better dice score compared to other transformer-based models ~\cite{shamshad2022transformers}.

Hatamizadeh et al.~\cite{hatamizadeh2022unetr} proposed a U-shaped volumetric medical image segmentation named Unetr that utilized a transformer as the encoder to learn sequence representations of the input volume and successfully captured the global multi-scale information. The proposed method justified their architecture with Synapse~\cite{landman2015miccai} dataset for spleen segmentation, and  Medical Segmentation Decathlon(MSD)~\cite{antonelli2022medical} dataset for brain tumor segmentation. Another study presented Hiformer~\cite{heidari2023hiformer}, an approach that merges CNN and a transformer for medical image segmentation tasks. The authors utilized multi-scale feature representations from the Swin Transformer module and a CNN encoder. To seamlessly blend global and local features from these sources, a unique Double-Level Fusion (DLF) module is incorporated within the encoder-decoder skip connection. The authors evaluated their proposed architecture with 2D datasets using dice, mIoU, and accuracy metrics. Furthermore, another study named H2Former~\cite{he2023h2former}  addressed an issue that the direct merger of CNN and self-attention is computationally challenging, especially for high-resolution images. To solve this issue, H2former used a hierarchical hybrid ViT-based model combining CNNs, multi-scale channel attention, and transformers. This data-efficient approach outperforms existing models in 2D and 3D tasks. For instance, on the KVASIR-SEG~\cite{jha2020kvasir} dataset, H2Former outpaces TransUNet~\cite{chen2021transunet} in IoU scores while being more computationally efficient. The authors evaluated their proposed model for multi-organ and skin lesion segmentation and used dice score,mIoU, accuracy, and HD as evaluation metrics. Another recent study named dilated transformer (D-former)~\cite{wu2023d} proposed a U-shaped hierarchical encoder-decoder structure for 3D medical image segmentation. The encoder has a patch embedding layer to convert 3D images into sequences, followed by four D-Former blocks for feature extraction with three interspersed down-sampling layers. The first, second, and fourth blocks contain one local scope module (LSM) and one global scope module (GSM).
In contrast, the third block has three of each, arranged alternately. The decoder has the same structure as the encode with its own D-Former blocks, up-sampling layers, and a patch-expanding layer. The author evaluated D-former for multi-organ segmentation with Synapse~\cite{landman2015miccai} dataset and three critical parts for heart segmentation with ACDC~\cite{bernard2018deep} dataset using dice metrics for both.
\begin{figure}
  \centering
      \includegraphics[scale=0.3]{assets/segmentation_prostate_up.png}
  \caption{A sample output for medical imaging segmentation for prostate on the dataset from the SPIE-AAPM-NCI Prostate MR Gleason Grade Group Challenge~\cite{Litjens2014ComputerAidedDO}}
  \label{fig:segmentation_image}
  \vspace{-5pt}
\end{figure}
Hatamizadeh et al.~\cite{hatamizadeh2022unetformer} proposed a U-shaped hybrid transformer-based model named UNetFormer where the authors used a 3D swin-transformer~\cite{liu2021swin} as the encode combining with both CNN and transformer decoders. Additionally, the authors introduced a self-supervised pre-training technique where the encoder learns to predict randomly masked volumetric tokens using contextual information of visible tokens. The authors evaluated their method with MSD~\cite{antonelli2022medical} and BraTS-21~\cite{baid2021rsna} dataset for liver organ and tumor segmentation. Zhou et al.~\cite{zhou2023nnformer} proposed nnFormer to utilize sufficiently the potential of transformers, a specialized 3D transformer tailored for volumetric medical image segmentation. The authors proposed local and global volume-based self-attention mechanisms to learn volume representations. nnFormer proposed a new skip attention technique replacing the traditional concatenation/summation operations in skip connections in U-net~\cite{ronneberger2015u} like architecture. The authors showed its potential with three different public datasets~\cite{antonelli2022medical,landman2015miccai,bernard2018deep} for a multi-organ, brain tumor, and cardiac segmentation. Another recent study named swin transformer boosted U-Net (ST-Unet)~\cite{zhang2023st} utilized the low-level features to enhance the global features and mitigate the semantic gaps between encoder and decoders. The proposed model used swin transformer as an encoder and CNN as a decoder. The authors introduced a cross-layer feature enhancement (CLFE) module to capture multi-scale feature representations. The Spatial and Channel Squeeze \ and excitation (SCSE) module utilized the channel and spatial information data to find meaningful features and diminish the less important features. By integrating features via the CLFE module and processing them through CNNs, the authors regained low-level details and \input{Tables/table_segmentation}\noindent refined local features, achieving precise semantic segmentation results. The proposed model was evaluated with Synapse~\cite{synapse} dataset using dice and Hausdorff distance (HD) metrics.

However, some works focused on edge organ segmentation, improving image data, and proposing a pipeline for the pre-trained and fine-tuning phase. One study by Wang et al.~\cite{wang2023swinmm} proposed a multi-view pipeline for accurate and efficient self-supervised medical image segmentation called SwinMM. The authors utilized multi-view data through two main components: a masked multi-view encoder trained on diverse proxy tasks during pre-training and a cross-view decoder aggregated the multi-view information using a cross-attention block in the fine-tuning phase. Compared to the benchmark Swin UNETR~\cite{hatamizadeh2021swin}, SwinMM significantly improved medical segmentation tasks, enhancing model accuracy and data efficiency. The authors used dice and IoU as evaluation metrics and evaluated with ACDC~\cite{bernard2018deep} and the Whole Abdominal Organ (WORD)~\cite{luo2022word} datasets. Another study by Liet et al.~\cite{li2023lvit} proposed  LViT (Language meets Vision Transformer), integrating medical text annotations to boost the quality of image data. The model utilized text to generate exponential pseudo-labels in a semi-supervised context, aided by an Exponential Pseudo-label Iteration mechanism (EPI). The authors also designed language-vision loss to supervise the training of unlabeled images using text information directly. The proposed methodology experimented with 2D datasets and used dice and mean intersection of union (mIoU) for evaluation. Recognizing the importance of fine-grained information for organ edge segmentation and the computational challenges of processing high-resolution 3D features, Yang et al. introduced an encoder-decoder network named EPT-Net. EPT-Net fused edge perception with Transformer architecture for precise medical image segmentation. EPT-Net mainly focused on bolstering 3D spatial positioning, while an edge weight guidance module extracted edge details without increasing network parameters. Tests on three datasets, including SegTHOR 2019~\cite{lambert2020segthor}, Multi-Atlas Labeling Beyond the Cranial Vault~\cite{landman2015miccai}, and a modified KiTS19 dataset named KiTS19-M~\cite{heller2019kits19}, revealed that EPT-Net outperforms leading medical image segmentation techniques.

Table \ref{segmentation} shows all the papers after Shamshad et al. ~\cite{shamshad2022transformers} surveys for medical image segmentation, summarize the model architecture, and results of evaluation metrics. All the proposed models in recent times used hybrid (CNN and transformer) because CNN can excel at capturing local spatial hierarchies, and the transformer can handle long-range dependencies in data through their self-attention mechanism. Most of the proposed models in Table \ref{segmentation} are evaluated with 3D datasets, showing that researchers are more interested in studying those 3D medical data because 3D data offers context in three planes (axial, sagittal, and coronal). Transformers can leverage this multi-planar context better than traditional 2D approaches for medical image segmentation.

% \input{table_segmentation}
% multi-organ  on the Synapse dataset ~\cite{heidari2023hiformer}
% \input{table_segmentation}






\section{Model Compression}\label{model_com}
Model compression is a key technique for deploying a model on edge devices with limited computational power and memory while maintaining model performance regarding accuracy, precision, and recall. It mainly focuses on lowering latency or reducing memory and energy consumption during inference. However, model compression on ViT still needs extensive exploration due to complex architecture and high resource usage tendencies. In this section, we will discuss prominent compression techniques for ViT models: pruning (Section~\ref{prun}), knowledge distillation (Section~\ref{know}), and quantization (Section~\ref{quan}). 

\subsection{Pruning}\label{prun}
% Static Pruning and Dynamic Pruning are two major pruning methodologies used to reduce the size of neural networks.
Pruning is used for reducing both memory and bandwidth. Most of the initial pruning techniques based on biased weight decay~\cite{hanson1988comparing},
second-order derivatives ~\cite{hassibi1992second}, and channels ~\cite{molchanov2016pruning}. Early days pruning techniques reduce the number of connections based on the hessian of the loss function~\cite{cheng2017survey,wu2021evolutionary}. In general, pruning removes redundant parameters that do not significantly contribute to the accuracy of results. The pruned model has fewer edges/connections than the original model. Most early pruning techniques are like brute force pruning, where one needs to manually check which weights do not cause any accuracy loss. Pruning techniques in deep learning became prominent post-2000 as neural networks (NNs) grew in size and complexity. The following subsections will discuss different pruning types and recent pruning techniques applied to ViT-based models.
% \begin{figure}
%   \centering
%   \includegraphics[scale=0.4]{assets/pruning.png}
%   \caption{A sample overview after pruning}
%   \label{fig:pruning}
% \end{figure}
\subsubsection{Types}\hfill\\ 
In recent studies, various pruning techniques have been utilized to optimize ViT models, categorizing them into different methods based on their approach and application timing. For example, unstructured pruning targets individual weights for removal, whereas structured pruning removes components at a broader scale, like layers or channels. On the one hand, static pruning is predetermined and fixed, which is ideal for environments with known constraints. On the other hand, dynamic pruning offers real-time adaptability, potentially enhancing model efficiency without sacrificing accuracy. Another pruning method, cascade pruning, is highlighted as a hybrid approach, integrating the iterative adaptability of dynamic pruning with the structured approach of static pruning. The following subsections will be a detailed discussion of different pruning types.
% For a detailed discussion about pruning types, please refer to Appendix B. 
\newline\hfill\break
\noindent \textbf{Unstructured vs Structured Pruning} Unstructured pruning removes individual weights or parameters from the network based on certain criteria. Unstructured pruning can result in highly sparse networks. However, it often does not lead to computational efficiency as the sparsity is not aligned with the memory access patterns or computational primitives of hardware accelerators. However, structured pruning applies to the specific components of the network, especially in layers, neurons, or channels. Structured pruning leads to more hardware-friendly sparsity patterns but often at the cost of higher accuracy loss. Cai et al. ~\cite{10.1145/3543622.3573044} proposed a two-stage coarse-grained/fine-grained structured pruning method based on top-K sparsification and reduces 60\% overall computation in the embedded NNs. In a recent survey~\cite{he2023structured}, He et al. discussed a range of SOTA structured pruning techniques, covering topics such as filter ranking methods, regularization methods, dynamic execution, neural architecture search (NAS), the lottery ticket hypothesis, and the applications of pruning. \\
\renewcommand{\arraystretch}{0.3}
% \begin{longtable}[c]{c|c|c|c|ccc}
% \caption{Results of different pruning techniques proposed for Vision transformers. '\textbf{$\downarrow$}' denotes reduction from the baseline and '$\uparrow$' denotes increase rate from the baseline models}
% \label{tab:pruning_result}\\
% \hline
% \multirow{2}{*}{\textbf{Algorithm}} &
%   \multirow{2}{*}{\textbf{Method}} &
%   \multirow{2}{*}{\textbf{Models}} &
%   \multirow{2}{*}{\textbf{Baseline}} &
%   \multicolumn{3}{c}{\textbf{Results}} \\ \cline{5-7} 
%  &
%    &
%    &
%    &
%   \multicolumn{1}{c|}{\textbf{GFlops}} &
%   \multicolumn{1}{c|}{\textbf{Params(M)}} &
%   \textbf{Top-1 (\%)} \\ \hline
% \endfirsthead
% %
% \endhead
% %
% \begin{tabular}[c]{@{}c@{}}Channels \\ pruning ~\cite{zhu2021vision}\end{tabular} &
%   \begin{tabular}[c]{@{}c@{}}Learn dimension-wise\\ important score\end{tabular} &
%   VTP &
%   DeiT-B &
%   \multicolumn{1}{c|}{\begin{tabular}[c]{@{}c@{}}10.0\\ ($\downarrow$ 45.3\%)\end{tabular}} &
%   \multicolumn{1}{c|}{$\downarrow$ 47.3} &
%   \begin{tabular}[c]{@{}c@{}}92.58\\ ($\downarrow$ 1.92\%)\end{tabular} \\ \cline{3-7} 
%  &
%    &
%   VTP &
%   DeiT-B &
%   \multicolumn{1}{c|}{\begin{tabular}[c]{@{}c@{}}10.0\\ ($\downarrow$ 43.1\%)\end{tabular}} &
%   \multicolumn{1}{c|}{$\downarrow$ 48.0} &
%   $\downarrow$ 1.1\% \\ \hline
% \multirow{4}{*}{\begin{tabular}[c]{@{}c@{}}Width \& Depth \\ Pruning ~\cite{yu2022width}\end{tabular}} &
%   \multirow{4}{*}{\begin{tabular}[c]{@{}c@{}}Set of learnable pruning\\ -related parameters for \\ width pruning \& shallow \\ classifiers using intermediate \\ information of the transformer\\ blocks\end{tabular}} &
%   \multirow{4}{*}{WDPruning} &
%   DeiT-T &
%   \multicolumn{1}{c|}{\begin{tabular}[c]{@{}c@{}}2.6\\ ($\downarrow$ 43.5\%)\end{tabular}} &
%   \multicolumn{1}{c|}{\begin{tabular}[c]{@{}c@{}}13.3 \\ ($\downarrow$ 37.6\%)\end{tabular}} &
%   \begin{tabular}[c]{@{}c@{}}70.34\\ ($\downarrow$ 1.86\%)\end{tabular} \\ \cline{4-7} 
%  &
%    &
%    &
%   DeiT-S &
%   \multicolumn{1}{c|}{\begin{tabular}[c]{@{}c@{}}0.7\\ ($\downarrow$ 46.2\%)\end{tabular}} &
%   \multicolumn{1}{c|}{\begin{tabular}[c]{@{}c@{}}3.5\\ ($\downarrow$ 35.2\%)\end{tabular}} &
%   \begin{tabular}[c]{@{}c@{}}78.38\\ ($\downarrow$ 1.42\%)\end{tabular} \\ \cline{4-7} 
%  &
%    &
%    &
%   DeiT-B &
%   \multicolumn{1}{c|}{\begin{tabular}[c]{@{}c@{}}9.90\\ ($\downarrow$ 43.4\%)\end{tabular}} &
%   \multicolumn{1}{c|}{\begin{tabular}[c]{@{}c@{}}55.3 \\ ($\downarrow$ 35.0\%)\end{tabular}} &
%   \begin{tabular}[c]{@{}c@{}}80.76\\ ($\downarrow$ 1.04\%)\end{tabular} \\ \cline{4-7} 
%  &
%    &
%    &
%   Swin-S &
%   \multicolumn{1}{c|}{\begin{tabular}[c]{@{}c@{}}6.3\\ ($\downarrow$ 27.6\%)\end{tabular}} &
%   \multicolumn{1}{c|}{\begin{tabular}[c]{@{}c@{}}32.8\\ ($\downarrow$ 30.6\%)\end{tabular}} &
%   \begin{tabular}[c]{@{}c@{}}81.80\\ ($\downarrow$ 1.20\%)\end{tabular} \\ \hline
% \multirow{3}{*}{\begin{tabular}[c]{@{}c@{}}Multi-\\ dimensional\\ pruning ~\cite{hou2022multi}\end{tabular}} &
%   \multirow{3}{*}{\begin{tabular}[c]{@{}c@{}}Dependency based pruning \\ criterion \& an efficient \\ Gaussian process search\end{tabular}} &
%   \multirow{3}{*}{\begin{tabular}[c]{@{}c@{}}Multi-\\ dimensional\end{tabular}} &
%   DeiT-S &
%   \multicolumn{1}{c|}{\begin{tabular}[c]{@{}c@{}}2.9\\ ($\downarrow$37\%)\end{tabular}} &
%   \multicolumn{1}{c|}{-} &
%   \begin{tabular}[c]{@{}c@{}}79.9\\ ($\downarrow$0.1\%)\end{tabular} \\ \cline{4-7} 
%  &
%    &
%    &
%   DeiT-B &
%   \multicolumn{1}{c|}{\begin{tabular}[c]{@{}c@{}}11.2\\ ($\downarrow$36\%)\end{tabular}} &
%   \multicolumn{1}{c|}{-} &
%   \begin{tabular}[c]{@{}c@{}}82.3\\ ($\downarrow$0.5\%)\end{tabular} \\ \cline{4-7} 
%  &
%    &
%    &
%   \begin{tabular}[c]{@{}c@{}}T2T-\\ ViT-14\end{tabular} &
%   \multicolumn{1}{c|}{\begin{tabular}[c]{@{}c@{}}2.9\\ ($\downarrow$40\%)\end{tabular}} &
%   \multicolumn{1}{c|}{-} &
%   \begin{tabular}[c]{@{}c@{}}81.7\\ ($\downarrow$0.2\%)\end{tabular} \\ \hline
% \multirow{4}{*}{\begin{tabular}[c]{@{}c@{}}Pruning the \\ network model ~\cite{kong2022spvit}\end{tabular}} &
%   \multirow{4}{*}{\begin{tabular}[c]{@{}c@{}}Single-path ViT\\  pruning based on the \\ token score\end{tabular}} &
%   \multirow{4}{*}{SPViT} &
%   Swin-S &
%   \multicolumn{1}{c|}{\begin{tabular}[c]{@{}c@{}}6.35 \\ ($\downarrow$ 26.4\%)\end{tabular}} &
%   \multicolumn{1}{c|}{-} &
%   \begin{tabular}[c]{@{}c@{}}82.71\\ ($\downarrow$ 0.49\%)\end{tabular} \\ \cline{4-7} 
%  &
%    &
%    &
%   Swin-T &
%   \multicolumn{1}{c|}{\begin{tabular}[c]{@{}c@{}}3.47 \\ ($\downarrow$ 23.0\%)\end{tabular}} &
%   \multicolumn{1}{c|}{-} &
%   \begin{tabular}[c]{@{}c@{}}80.70 \\ ($\downarrow$ 0.50\%)\end{tabular} \\ \cline{4-7} 
%  &
%    &
%    &
%   PiT-S &
%   \multicolumn{1}{c|}{\begin{tabular}[c]{@{}c@{}}2.22 \\ ($\downarrow$ 23.3\%)\end{tabular}} &
%   \multicolumn{1}{c|}{-} &
%   \begin{tabular}[c]{@{}c@{}}80.38 \\ ($\downarrow$ 0.58\%)\end{tabular} \\ \cline{4-7} 
%  &
%    &
%    &
%   PiT-XS &
%   \multicolumn{1}{c|}{\begin{tabular}[c]{@{}c@{}}1.13 \\ ($\downarrow$ 18.7\%)\end{tabular}} &
%   \multicolumn{1}{c|}{-} &
%   \begin{tabular}[c]{@{}c@{}}77.86 \\ ($\downarrow$ 0.24\%)\end{tabular} \\ \hline
% \multirow{4}{*}{\begin{tabular}[c]{@{}c@{}}Patch \\ pruning ~\cite{tang2022patch}\end{tabular}} &
%   \multirow{4}{*}{\begin{tabular}[c]{@{}c@{}}Layer-by-layer \\ top down pruning\end{tabular}} &
%   \multirow{4}{*}{PS-ViT} &
%   DeiT-T &
%   \multicolumn{1}{c|}{\begin{tabular}[c]{@{}c@{}}0.7 \\ ($\downarrow$ 46.2\%)\end{tabular}} &
%   \multicolumn{1}{c|}{-} &
%   \begin{tabular}[c]{@{}c@{}}72.0 \\ ($\downarrow$ 0.20\%)\end{tabular} \\ \cline{4-7} 
%  &
%    &
%    &
%   DeiT-S &
%   \multicolumn{1}{c|}{\begin{tabular}[c]{@{}c@{}}2.6 \\ ($\downarrow$ 43.6\%)\end{tabular}} &
%   \multicolumn{1}{c|}{-} &
%   \begin{tabular}[c]{@{}c@{}}79.4 \\ ($\downarrow$ 0.40\%)\end{tabular} \\ \cline{4-7} 
%  &
%    &
%    &
%   DeiT-B &
%   \multicolumn{1}{c|}{\begin{tabular}[c]{@{}c@{}}9.8 \\ ($\downarrow$ 44.3\%)\end{tabular}} &
%   \multicolumn{1}{c|}{-} &
%   \begin{tabular}[c]{@{}c@{}}81.5\\ ($\downarrow$ 0.30\%)\end{tabular} \\ \cline{4-7} 
%  &
%    &
%    &
%   \begin{tabular}[c]{@{}c@{}}T2T-\\ ViT-14\end{tabular} &
%   \multicolumn{1}{c|}{\begin{tabular}[c]{@{}c@{}}3.1 \\ ($\downarrow$ 40.4\%)\end{tabular}} &
%   \multicolumn{1}{c|}{-} &
%   \begin{tabular}[c]{@{}c@{}}81.1\\ ($\downarrow$ 0.40\%)\end{tabular} \\ \hline
% \multirow{4}{*}{\begin{tabular}[c]{@{}c@{}}Structural \\ pruning ~\cite{yu2022unified}\end{tabular}} &
%   \multirow{4}{*}{\begin{tabular}[c]{@{}c@{}}Prune the head \\ number \& head\\ dimensions inside \\ each layer\end{tabular}} &
%   \multirow{4}{*}{UVC} &
%   DeiT-T &
%   \multicolumn{1}{c|}{\begin{tabular}[c]{@{}c@{}}0.51\\ (39.12\%)\end{tabular}} &
%   \multicolumn{1}{c|}{-} &
%   \begin{tabular}[c]{@{}c@{}}70.6\\ ($\downarrow$ 1.6\%)\end{tabular} \\ \cline{4-7} 
%  &
%    &
%    &
%   DeiT-S &
%   \multicolumn{1}{c|}{\begin{tabular}[c]{@{}c@{}}2.32\\ (50.41\%)\end{tabular}} &
%   \multicolumn{1}{c|}{-} &
%   \begin{tabular}[c]{@{}c@{}}78.82\\ ($\downarrow$ 0.98\%)\end{tabular} \\ \cline{4-7} 
%  &
%    &
%    &
%   DeiT-B &
%   \multicolumn{1}{c|}{\begin{tabular}[c]{@{}c@{}}8.0\\ (45.50\%)\end{tabular}} &
%   \multicolumn{1}{c|}{-} &
%   \begin{tabular}[c]{@{}c@{}}80.57 \\ ($\downarrow$ 1.23\%)\end{tabular} \\ \cline{4-7} 
%  &
%    &
%    &
%   \begin{tabular}[c]{@{}c@{}}T2T-\\ ViT-14\end{tabular} &
%   \multicolumn{1}{c|}{\begin{tabular}[c]{@{}c@{}}2.11 \\ ($\downarrow$ 44.0\%)\end{tabular}} &
%   \multicolumn{1}{c|}{-} &
%   \begin{tabular}[c]{@{}c@{}}78.9\\ ($\downarrow$ 2.6\%)\end{tabular} \\ \hline
% \multirow{4}{*}{\begin{tabular}[c]{@{}c@{}}Global structural \\ pruning ~\cite{yang2023global}\end{tabular}} &
%   \multirow{4}{*}{\begin{tabular}[c]{@{}c@{}}Latency-aware, Hessian-\\ based importance-\\ based criteria\end{tabular}} &
%   \begin{tabular}[c]{@{}c@{}}NViT-B\\ $+$ ASP\end{tabular} &
%   DeiT-B &
%   \multicolumn{1}{c|}{\begin{tabular}[c]{@{}c@{}}6.8 \\ ($2.57\times$)\end{tabular}} &
%   \multicolumn{1}{c|}{\begin{tabular}[c]{@{}c@{}}17\\ (5.14$\times$)\end{tabular}} &
%   \begin{tabular}[c]{@{}c@{}}83.29\\ ($\downarrow$ 0.07\%)\end{tabular} \\ \cline{3-7} 
%  &
%    &
%   \begin{tabular}[c]{@{}c@{}}NViT-H\\ $+$ ASP\end{tabular} &
%   Swin-S &
%   \multicolumn{1}{c|}{\begin{tabular}[c]{@{}c@{}}6.2 \\ ($2.85\times$)\end{tabular}} &
%   \multicolumn{1}{c|}{\begin{tabular}[c]{@{}c@{}}15\\ (5.68$\times$)\end{tabular}} &
%   \begin{tabular}[c]{@{}c@{}}82.95\\ ($\downarrow$ 0.05\%)\end{tabular} \\ \cline{3-7} 
%  &
%    &
%   \begin{tabular}[c]{@{}c@{}}NViT-S\\ $+$ ASP\end{tabular} &
%   DeiT-S &
%   \multicolumn{1}{c|}{\begin{tabular}[c]{@{}c@{}}4.2 \\ (4.24$\times$)\end{tabular}} &
%   \multicolumn{1}{c|}{\begin{tabular}[c]{@{}c@{}}10.5 \\ (8.36$\times$)\end{tabular}} &
%   \begin{tabular}[c]{@{}c@{}}82.19\\ ($\uparrow$ 1.0\%)\end{tabular} \\ \cline{3-7} 
%  &
%    &
%   \begin{tabular}[c]{@{}c@{}}NViT-T \\ $+$ ASP\end{tabular} &
%   DeiT-T &
%   \multicolumn{1}{c|}{\begin{tabular}[c]{@{}c@{}}1.3 \\ (13.55$\times$)\end{tabular}} &
%   \multicolumn{1}{c|}{\begin{tabular}[c]{@{}c@{}}3.5\\ (24.94$\times$)\end{tabular}} &
%   \begin{tabular}[c]{@{}c@{}}76.21\\ ($\downarrow$ 1.71\%)\end{tabular} \\ \hline
% \multirow{3}{*}{\begin{tabular}[c]{@{}c@{}}Collaborative \\ pruning ~\cite{zheng2022savit}\end{tabular}} &
%   \multirow{3}{*}{\begin{tabular}[c]{@{}c@{}}Structural pruning on \\ MSA attention \& FFN by \\ removing unnecessary\\ parameter groups\end{tabular}} &
%   \multirow{3}{*}{SAViT} &
%   DeiT-B &
%   \multicolumn{1}{c|}{\begin{tabular}[c]{@{}c@{}}10.6 \\ ($\downarrow$39.8\%)\end{tabular}} &
%   \multicolumn{1}{c|}{\begin{tabular}[c]{@{}c@{}}51.9\\ ($\downarrow$40.1\%)\end{tabular}} &
%   \begin{tabular}[c]{@{}c@{}}82.75 \\ ($\uparrow$ 0.91\%)\end{tabular} \\ \cline{4-7} 
%  &
%    &
%    &
%   DeiT-S &
%   \multicolumn{1}{c|}{\begin{tabular}[c]{@{}c@{}}3.1 \\ ($\downarrow$31.7\%)\end{tabular}} &
%   \multicolumn{1}{c|}{\begin{tabular}[c]{@{}c@{}}14.7\\ ($\downarrow$33.5\%)\end{tabular}} &
%   \begin{tabular}[c]{@{}c@{}}80.11\\ ($\uparrow$ 0.26\%)\end{tabular} \\ \cline{4-7} 
%  &
%    &
%    &
%   DeiT-T &
%   \multicolumn{1}{c|}{\begin{tabular}[c]{@{}c@{}}0.9 \\ ($\downarrow$24.4\%)\end{tabular}} &
%   \multicolumn{1}{c|}{\begin{tabular}[c]{@{}c@{}}4.2\\ ($\downarrow$25.2\%)\end{tabular}} &
%   \begin{tabular}[c]{@{}c@{}}70.72\\ ($\downarrow$ 1.48\%)\end{tabular} \\ \hline
% \multirow{3}{*}{\begin{tabular}[c]{@{}c@{}}Structured sparse \\ pruning ~\cite{chen2021chasing}\end{tabular}} &
%   \multirow{3}{*}{\begin{tabular}[c]{@{}c@{}}Removing sub-modules \\ like self-attention \\ heads by manipulating \\ weight, activation,\& gradient\end{tabular}} &
%   S$^2$ViTE-B &
%   DeiT-B &
%   \multicolumn{1}{c|}{\begin{tabular}[c]{@{}c@{}}11.8 \\ ($\downarrow$33.13\%)\end{tabular}} &
%   \multicolumn{1}{c|}{\begin{tabular}[c]{@{}c@{}}56.8\\ ($\downarrow$34.4\%)\end{tabular}} &
%   \begin{tabular}[c]{@{}c@{}}82.22\\ ($\uparrow$ 0.38\%)\end{tabular} \\ \cline{3-7} 
%  &
%    &
%   S$^2$ViTE-S &
%   DeiT-S &
%   \multicolumn{1}{c|}{\begin{tabular}[c]{@{}c@{}}3.1\\ ($\downarrow$31.7\%)\end{tabular}} &
%   \multicolumn{1}{c|}{\begin{tabular}[c]{@{}c@{}}14.6\\ ($\downarrow$31.63\%)\end{tabular}} &
%   \begin{tabular}[c]{@{}c@{}}79.22 \\ ($\downarrow$ 0.63\%)\end{tabular} \\ \cline{3-7} 
%  &
%    &
%   S$^2$ViTE-T &
%   DeiT-T &
%   \multicolumn{1}{c|}{\begin{tabular}[c]{@{}c@{}}0.9 \\ ($\downarrow$23.69\%)\end{tabular}} &
%   \multicolumn{1}{c|}{\begin{tabular}[c]{@{}c@{}}4.2\\ ($\downarrow$26.3\%)\end{tabular}} &
%   \begin{tabular}[c]{@{}c@{}}70.12\\ ($\downarrow$ 2.08\%)\end{tabular} \\ \hline
% \multirow{3}{*}{\begin{tabular}[c]{@{}c@{}}Bottom-up \\ cascade \\ pruning~\cite{wang2022vtc}\end{tabular}} &
%   \multirow{3}{*}{\begin{tabular}[c]{@{}c@{}}Token pruning \& channel \\ pruning using a \\ hyperparameter \\ from one to last block\end{tabular}} &
%   \multirow{3}{*}{VTC-LFC} &
%   DeiT-B &
%   \multicolumn{1}{c|}{$\downarrow$54.4\%} &
%   \multicolumn{1}{c|}{\begin{tabular}[c]{@{}c@{}}56.8\\ ($\downarrow$ 34.25\%)\end{tabular}} &
%   \begin{tabular}[c]{@{}c@{}}81.6\\ ($\downarrow$ 0.20\%)\end{tabular} \\ \cline{4-7} 
%  &
%    &
%    &
%   DeiT-S &
%   \multicolumn{1}{c|}{$\downarrow$47.1\%} &
%   \multicolumn{1}{c|}{\begin{tabular}[c]{@{}c@{}}15.3\\ ($\downarrow$ 30.77\%)\end{tabular}} &
%   \begin{tabular}[c]{@{}c@{}}79.6\\ ($\downarrow$ 0.20\%)\end{tabular} \\ \cline{4-7} 
%  &
%    &
%    &
%   DeiT-T &
%   \multicolumn{1}{c|}{$\downarrow$41.7\%} &
%   \multicolumn{1}{c|}{\begin{tabular}[c]{@{}c@{}}4.2\\ ($\downarrow$ 26.32\%)\end{tabular}} &
%   \begin{tabular}[c]{@{}c@{}}71.0\\ ($\downarrow$ 1.20\%)\end{tabular} \\ \hline
% \multirow{6}{*}{\begin{tabular}[c]{@{}c@{}}Cascade ViT\\ pruning~\cite{song2022cp}\end{tabular}} &
%   \multirow{6}{*}{\begin{tabular}[c]{@{}c@{}}Utilizing the sparsity to prune \\ PH-regions in MSA \&\\ FFN progressively \\ \& dynamically\end{tabular}} &
%   \multirow{2}{*}{CP-ViT} &
%   ViT-B &
%   \multicolumn{1}{c|}{$\downarrow$46.34\%} &
%   \multicolumn{1}{c|}{-} &
%   \begin{tabular}[c]{@{}c@{}}76.75\\ ($\downarrow$ 1.16\%)\end{tabular} \\ \cline{4-7} 
%  &
%    &
%    &
%   DeiT-B &
%   \multicolumn{1}{c|}{$\downarrow$41.62\%} &
%   \multicolumn{1}{c|}{-} &
%   \begin{tabular}[c]{@{}c@{}}81.13\\ ($\downarrow$ 0.69\%)\end{tabular} \\ \cline{3-7} 
%  &
%    &
%   \multirow{2}{*}{CP-ViT} &
%   ViT-B &
%   \multicolumn{1}{c|}{$\downarrow$29.03\%} &
%   \multicolumn{1}{c|}{-} &
%   \begin{tabular}[c]{@{}c@{}}96.20 \\ ($\downarrow$ 1.93\%)\end{tabular} \\ \cline{4-7} 
%  &
%    &
%    &
%   DeiT-B &
%   \multicolumn{1}{c|}{$\downarrow$30.08\%} &
%   \multicolumn{1}{c|}{-} &
%   \begin{tabular}[c]{@{}c@{}}98.01\\ ($\downarrow$ 1.09\%)\end{tabular} \\ \cline{3-7} 
%  &
%    &
%   \multirow{2}{*}{CP-ViT} &
%   ViT-B &
%   \multicolumn{1}{c|}{$\downarrow$32.05\%} &
%   \multicolumn{1}{c|}{-} &
%   \begin{tabular}[c]{@{}c@{}}84.79 \\ ($\downarrow$ 2.34\%)\end{tabular} \\ \cline{4-7} 
%  &
%    &
%    &
%   DeiT-B &
%   \multicolumn{1}{c|}{$\downarrow$30.92\%} &
%   \multicolumn{1}{c|}{-} &
%   \begin{tabular}[c]{@{}c@{}}89.68\\ ($\downarrow$ 1.17\%)\end{tabular} \\ \hline
  
% \end{longtable}
% Please add the following required packages to your document preamble:
% \usepackage{multirow}
% \usepackage{graphicx}
% Please add the following required packages to your document preamble:
% \usepackage{multirow}
% \usepackage{graphicx}
\begin{table}[]
\centering
\caption{Results of different pruning techniques proposed for Vision transformers. '\textbf{$\downarrow$}' denotes reduction from the baseline and '$\uparrow$' denotes increase rate from the baseline models.}
\label{tab:pruning_result}
\resizebox{\columnwidth}{!}{%
\begin{tabular}{c|c|c|c|ccc}
\hline
\multirow{2}{*}{\textbf{Algorithm}} &
  \multirow{2}{*}{\textbf{Method}} &
  \multirow{2}{*}{\textbf{Models}} &
  \multirow{2}{*}{\textbf{Baseline}} &
  \multicolumn{3}{c}{\textbf{Results}} \\ \cline{5-7} 
 &
   &
   &
   &
  \multicolumn{1}{c|}{\textbf{GFlops}} &
  \multicolumn{1}{c|}{\textbf{Params(M)}} &
  \textbf{Top-1 (\%)} \\ \hline
\begin{tabular}[c]{@{}c@{}}Channels \\ pruning ~\cite{zhu2021vision}\end{tabular} &
  \begin{tabular}[c]{@{}c@{}}Learn dimension-wise\\ important score\end{tabular} &
  VTP &
  DeiT-B &
  \multicolumn{1}{c|}{\begin{tabular}[c]{@{}c@{}}10.0\\ ($\downarrow$ 45.3\%)\end{tabular}} &
  \multicolumn{1}{c|}{$\downarrow$ 47.3} &
  \begin{tabular}[c]{@{}c@{}}92.58\\ ($\downarrow$ 1.92\%)\end{tabular} \\ \cline{3-7} 
 &
   &
  VTP &
  DeiT-B &
  \multicolumn{1}{c|}{\begin{tabular}[c]{@{}c@{}}10.0\\ ($\downarrow$ 43.1\%)\end{tabular}} &
  \multicolumn{1}{c|}{$\downarrow$ 48.0} &
  $\downarrow$ 1.1\% \\ \hline
\multirow{4}{*}{\begin{tabular}[c]{@{}c@{}}Width \& Depth \\ Pruning ~\cite{yu2022width}\end{tabular}} &
  \multirow{4}{*}{\begin{tabular}[c]{@{}c@{}}Set of learnable pruning\\ -related parameters for \\ width pruning \& shallow \\ classifiers using intermediate \\ information of the transformer\\ blocks\end{tabular}} &
  \multirow{4}{*}{WDPruning} &
  DeiT-T &
  \multicolumn{1}{c|}{\begin{tabular}[c]{@{}c@{}}2.6\\ ($\downarrow$ 43.5\%)\end{tabular}} &
  \multicolumn{1}{c|}{\begin{tabular}[c]{@{}c@{}}13.3 \\ ($\downarrow$ 37.6\%)\end{tabular}} &
  \begin{tabular}[c]{@{}c@{}}70.34\\ ($\downarrow$ 1.86\%)\end{tabular} \\ \cline{4-7} 
 &
   &
   &
  DeiT-S &
  \multicolumn{1}{c|}{\begin{tabular}[c]{@{}c@{}}0.7\\ ($\downarrow$ 46.2\%)\end{tabular}} &
  \multicolumn{1}{c|}{\begin{tabular}[c]{@{}c@{}}3.5\\ ($\downarrow$ 35.2\%)\end{tabular}} &
  \begin{tabular}[c]{@{}c@{}}78.38\\ ($\downarrow$ 1.42\%)\end{tabular} \\ \cline{4-7} 
 &
   &
   &
  DeiT-B &
  \multicolumn{1}{c|}{\begin{tabular}[c]{@{}c@{}}9.90\\ ($\downarrow$ 43.4\%)\end{tabular}} &
  \multicolumn{1}{c|}{\begin{tabular}[c]{@{}c@{}}55.3 \\ ($\downarrow$ 35.0\%)\end{tabular}} &
  \begin{tabular}[c]{@{}c@{}}80.76\\ ($\downarrow$ 1.04\%)\end{tabular} \\ \cline{4-7} 
 &
   &
   &
  Swin-S &
  \multicolumn{1}{c|}{\begin{tabular}[c]{@{}c@{}}6.3\\ ($\downarrow$ 27.6\%)\end{tabular}} &
  \multicolumn{1}{c|}{\begin{tabular}[c]{@{}c@{}}32.8\\ ($\downarrow$ 30.6\%)\end{tabular}} &
  \begin{tabular}[c]{@{}c@{}}81.80\\ ($\downarrow$ 1.20\%)\end{tabular} \\ \hline
\multirow{3}{*}{\begin{tabular}[c]{@{}c@{}}Multi-\\ dimensional\\ pruning ~\cite{hou2022multi}\end{tabular}} &
  \multirow{3}{*}{\begin{tabular}[c]{@{}c@{}}Dependency based pruning \\ criterion \& an efficient \\ Gaussian process search\end{tabular}} &
  \multirow{3}{*}{\begin{tabular}[c]{@{}c@{}}Multi-\\ dimensional\end{tabular}} &
  DeiT-S &
  \multicolumn{1}{c|}{\begin{tabular}[c]{@{}c@{}}2.9\\ ($\downarrow$37\%)\end{tabular}} &
  \multicolumn{1}{c|}{-} &
  \begin{tabular}[c]{@{}c@{}}79.9\\ ($\downarrow$0.1\%)\end{tabular} \\ \cline{4-7} 
 &
   &
   &
  DeiT-B &
  \multicolumn{1}{c|}{\begin{tabular}[c]{@{}c@{}}11.2\\ ($\downarrow$36\%)\end{tabular}} &
  \multicolumn{1}{c|}{-} &
  \begin{tabular}[c]{@{}c@{}}82.3\\ ($\downarrow$0.5\%)\end{tabular} \\ \cline{4-7} 
 &
   &
   &
  \begin{tabular}[c]{@{}c@{}}T2T-\\ ViT-14\end{tabular} &
  \multicolumn{1}{c|}{\begin{tabular}[c]{@{}c@{}}2.9\\ ($\downarrow$40\%)\end{tabular}} &
  \multicolumn{1}{c|}{-} &
  \begin{tabular}[c]{@{}c@{}}81.7\\ ($\downarrow$0.2\%)\end{tabular} \\ \hline
\multirow{4}{*}{\begin{tabular}[c]{@{}c@{}}Pruning the \\ network model ~\cite{kong2022spvit}\end{tabular}} &
  \multirow{4}{*}{\begin{tabular}[c]{@{}c@{}}Single-path ViT\\ pruning based on the \\ token score\end{tabular}} &
  \multirow{4}{*}{SPViT} &
  Swin-S &
  \multicolumn{1}{c|}{\begin{tabular}[c]{@{}c@{}}6.35 \\ ($\downarrow$ 26.4\%)\end{tabular}} &
  \multicolumn{1}{c|}{-} &
  \begin{tabular}[c]{@{}c@{}}82.71\\ ($\downarrow$ 0.49\%)\end{tabular} \\ \cline{4-7} 
 &
   &
   &
  Swin-T &
  \multicolumn{1}{c|}{\begin{tabular}[c]{@{}c@{}}3.47 \\ ($\downarrow$ 23.0\%)\end{tabular}} &
  \multicolumn{1}{c|}{-} &
  \begin{tabular}[c]{@{}c@{}}80.70 \\ ($\downarrow$ 0.50\%)\end{tabular} \\ \cline{4-7} 
 &
   &
   &
  PiT-S &
  \multicolumn{1}{c|}{\begin{tabular}[c]{@{}c@{}}2.22 \\ ($\downarrow$ 23.3\%)\end{tabular}} &
  \multicolumn{1}{c|}{-} &
  \begin{tabular}[c]{@{}c@{}}80.38 \\ ($\downarrow$ 0.58\%)\end{tabular} \\ \cline{4-7} 
 &
   &
   &
  PiT-XS &
  \multicolumn{1}{c|}{\begin{tabular}[c]{@{}c@{}}1.13 \\ ($\downarrow$ 18.7\%)\end{tabular}} &
  \multicolumn{1}{c|}{-} &
  \begin{tabular}[c]{@{}c@{}}77.86 \\ ($\downarrow$ 0.24\%)\end{tabular} \\ \hline
\multirow{4}{*}{\begin{tabular}[c]{@{}c@{}}Patch \\ pruning ~\cite{tang2022patch}\end{tabular}} &
  \multirow{4}{*}{\begin{tabular}[c]{@{}c@{}}Layer-by-layer \\ top down pruning\end{tabular}} &
  \multirow{4}{*}{PS-ViT} &
  DeiT-T &
  \multicolumn{1}{c|}{\begin{tabular}[c]{@{}c@{}}0.7 \\ ($\downarrow$ 46.2\%)\end{tabular}} &
  \multicolumn{1}{c|}{-} &
  \begin{tabular}[c]{@{}c@{}}72.0 \\ ($\downarrow$ 0.20\%)\end{tabular} \\ \cline{4-7} 
 &
   &
   &
  DeiT-S &
  \multicolumn{1}{c|}{\begin{tabular}[c]{@{}c@{}}2.6 \\ ($\downarrow$ 43.6\%)\end{tabular}} &
  \multicolumn{1}{c|}{-} &
  \begin{tabular}[c]{@{}c@{}}79.4 \\ ($\downarrow$ 0.40\%)\end{tabular} \\ \cline{4-7} 
 &
   &
   &
  DeiT-B &
  \multicolumn{1}{c|}{\begin{tabular}[c]{@{}c@{}}9.8 \\ ($\downarrow$ 44.3\%)\end{tabular}} &
  \multicolumn{1}{c|}{-} &
  \begin{tabular}[c]{@{}c@{}}81.5\\ ($\downarrow$ 0.30\%)\end{tabular} \\ \cline{4-7} 
 &
   &
   &
  \begin{tabular}[c]{@{}c@{}}T2T-\\ ViT-14\end{tabular} &
  \multicolumn{1}{c|}{\begin{tabular}[c]{@{}c@{}}3.1 \\ ($\downarrow$ 40.4\%)\end{tabular}} &
  \multicolumn{1}{c|}{-} &
  \begin{tabular}[c]{@{}c@{}}81.1\\ ($\downarrow$ 0.40\%)\end{tabular} \\ \hline
\multirow{4}{*}{\begin{tabular}[c]{@{}c@{}}Structural \\ pruning ~\cite{yu2022unified}\end{tabular}} &
  \multirow{4}{*}{\begin{tabular}[c]{@{}c@{}}Prune the head \\ number \& head\\ dimensions inside \\ each layer\end{tabular}} &
  \multirow{4}{*}{UVC} &
  DeiT-T &
  \multicolumn{1}{c|}{\begin{tabular}[c]{@{}c@{}}0.51\\ (39.12\%)\end{tabular}} &
  \multicolumn{1}{c|}{-} &
  \begin{tabular}[c]{@{}c@{}}70.6\\ ($\downarrow$ 1.6\%)\end{tabular} \\ \cline{4-7} 
 &
   &
   &
  DeiT-S &
  \multicolumn{1}{c|}{\begin{tabular}[c]{@{}c@{}}2.32\\ (50.41\%)\end{tabular}} &
  \multicolumn{1}{c|}{-} &
  \begin{tabular}[c]{@{}c@{}}78.82\\ ($\downarrow$ 0.98\%)\end{tabular} \\ \cline{4-7} 
 &
   &
   &
  DeiT-B &
  \multicolumn{1}{c|}{\begin{tabular}[c]{@{}c@{}}8.0\\ (45.50\%)\end{tabular}} &
  \multicolumn{1}{c|}{-} &
  \begin{tabular}[c]{@{}c@{}}80.57 \\ ($\downarrow$ 1.23\%)\end{tabular} \\ \cline{4-7} 
 &
   &
   &
  \begin{tabular}[c]{@{}c@{}}T2T-\\ ViT-14\end{tabular} &
  \multicolumn{1}{c|}{\begin{tabular}[c]{@{}c@{}}2.11 \\ ($\downarrow$ 44.0\%)\end{tabular}} &
  \multicolumn{1}{c|}{-} &
  \begin{tabular}[c]{@{}c@{}}78.9\\ ($\downarrow$ 2.6\%)\end{tabular} \\ \hline
\multirow{4}{*}{\begin{tabular}[c]{@{}c@{}}Global structural \\ pruning ~\cite{yang2023global}\end{tabular}} &
  \multirow{4}{*}{\begin{tabular}[c]{@{}c@{}}Latency-aware, Hessian-\\ based importance-\\ based criteria\end{tabular}} &
  \begin{tabular}[c]{@{}c@{}}NViT-B\\ $+$ ASP\end{tabular} &
  DeiT-B &
  \multicolumn{1}{c|}{\begin{tabular}[c]{@{}c@{}}6.8 \\ ($2.57\times$)\end{tabular}} &
  \multicolumn{1}{c|}{\begin{tabular}[c]{@{}c@{}}17\\ (5.14$\times$)\end{tabular}} &
  \begin{tabular}[c]{@{}c@{}}83.29\\ ($\downarrow$ 0.07\%)\end{tabular} \\ \cline{3-7} 
 &
   &
  \begin{tabular}[c]{@{}c@{}}NViT-H\\ $+$ ASP\end{tabular} &
  Swin-S &
  \multicolumn{1}{c|}{\begin{tabular}[c]{@{}c@{}}6.2 \\ ($2.85\times$)\end{tabular}} &
  \multicolumn{1}{c|}{\begin{tabular}[c]{@{}c@{}}15\\ (5.68$\times$)\end{tabular}} &
  \begin{tabular}[c]{@{}c@{}}82.95\\ ($\downarrow$ 0.05\%)\end{tabular} \\ \cline{3-7} 
 &
   &
  \begin{tabular}[c]{@{}c@{}}NViT-S\\ $+$ ASP\end{tabular} &
  DeiT-S &
  \multicolumn{1}{c|}{\begin{tabular}[c]{@{}c@{}}4.2 \\ (4.24$\times$)\end{tabular}} &
  \multicolumn{1}{c|}{\begin{tabular}[c]{@{}c@{}}10.5 \\ (8.36$\times$)\end{tabular}} &
  \begin{tabular}[c]{@{}c@{}}82.19\\ ($\uparrow$ 1.0\%)\end{tabular} \\ \cline{3-7} 
 &
   &
  \begin{tabular}[c]{@{}c@{}}NViT-T \\ $+$ ASP\end{tabular} &
  DeiT-T &
  \multicolumn{1}{c|}{\begin{tabular}[c]{@{}c@{}}1.3 \\ (13.55$\times$)\end{tabular}} &
  \multicolumn{1}{c|}{\begin{tabular}[c]{@{}c@{}}3.5\\ (24.94$\times$)\end{tabular}} &
  \begin{tabular}[c]{@{}c@{}}76.21\\ ($\downarrow$ 1.71\%)\end{tabular} \\ \hline
\multirow{3}{*}{\begin{tabular}[c]{@{}c@{}}Collaborative \\ pruning ~\cite{zheng2022savit}\end{tabular}} &
  \multirow{3}{*}{\begin{tabular}[c]{@{}c@{}}Structural pruning on \\ MSA attention \& FFN by \\ removing unnecessary\\ parameter groups\end{tabular}} &
  \multirow{3}{*}{SAViT} &
  DeiT-B &
  \multicolumn{1}{c|}{\begin{tabular}[c]{@{}c@{}}10.6 \\ ($\downarrow$39.8\%)\end{tabular}} &
  \multicolumn{1}{c|}{\begin{tabular}[c]{@{}c@{}}51.9\\ ($\downarrow$40.1\%)\end{tabular}} &
  \begin{tabular}[c]{@{}c@{}}82.75 \\ ($\uparrow$ 0.91\%)\end{tabular} \\ \cline{4-7} 
 &
   &
   &
  DeiT-S &
  \multicolumn{1}{c|}{\begin{tabular}[c]{@{}c@{}}3.1 \\ ($\downarrow$31.7\%)\end{tabular}} &
  \multicolumn{1}{c|}{\begin{tabular}[c]{@{}c@{}}14.7\\ ($\downarrow$33.5\%)\end{tabular}} &
  \begin{tabular}[c]{@{}c@{}}80.11\\ ($\uparrow$ 0.26\%)\end{tabular} \\ \cline{4-7} 
 &
   &
   &
  DeiT-T &
  \multicolumn{1}{c|}{\begin{tabular}[c]{@{}c@{}}0.9 \\ ($\downarrow$24.4\%)\end{tabular}} &
  \multicolumn{1}{c|}{\begin{tabular}[c]{@{}c@{}}4.2\\ ($\downarrow$25.2\%)\end{tabular}} &
  \begin{tabular}[c]{@{}c@{}}70.72\\ ($\downarrow$ 1.48\%)\end{tabular} \\ \hline
\multirow{3}{*}{\begin{tabular}[c]{@{}c@{}}Structured sparse \\ pruning ~\cite{chen2021chasing}\end{tabular}} &
  \multirow{3}{*}{\begin{tabular}[c]{@{}c@{}}Removing sub-modules \\ like self-attention \\ heads by manipulating \\ weight, activation,\& gradient\end{tabular}} &
  S$^2$ViTE-B &
  DeiT-B &
  \multicolumn{1}{c|}{\begin{tabular}[c]{@{}c@{}}11.8 \\ ($\downarrow$33.13\%)\end{tabular}} &
  \multicolumn{1}{c|}{\begin{tabular}[c]{@{}c@{}}56.8\\ ($\downarrow$34.4\%)\end{tabular}} &
  \begin{tabular}[c]{@{}c@{}}82.22\\ ($\uparrow$ 0.38\%)\end{tabular} \\ \cline{3-7} 
 &
   &
  S$^2$ViTE-S &
  DeiT-S &
  \multicolumn{1}{c|}{\begin{tabular}[c]{@{}c@{}}3.1\\ ($\downarrow$31.7\%)\end{tabular}} &
  \multicolumn{1}{c|}{\begin{tabular}[c]{@{}c@{}}14.6\\ ($\downarrow$31.63\%)\end{tabular}} &
  \begin{tabular}[c]{@{}c@{}}79.22 \\ ($\downarrow$ 0.63\%)\end{tabular} \\ \cline{3-7} 
 &
   &
  S$^2$ViTE-T &
  DeiT-T &
  \multicolumn{1}{c|}{\begin{tabular}[c]{@{}c@{}}0.9 \\ ($\downarrow$23.69\%)\end{tabular}} &
  \multicolumn{1}{c|}{\begin{tabular}[c]{@{}c@{}}4.2\\ ($\downarrow$26.3\%)\end{tabular}} &
  \begin{tabular}[c]{@{}c@{}}70.12\\ ($\downarrow$ 2.08\%)\end{tabular} \\ \hline
\multirow{3}{*}{\begin{tabular}[c]{@{}c@{}}Bottom-up \\ cascade \\ pruning~\cite{wang2022vtc}\end{tabular}} &
  \multirow{3}{*}{\begin{tabular}[c]{@{}c@{}}Token pruning \& channel \\ pruning using a \\ hyperparameter \\ from one to last block\end{tabular}} &
  \multirow{3}{*}{VTC-LFC} &
  DeiT-B &
  \multicolumn{1}{c|}{$\downarrow$54.4\%} &
  \multicolumn{1}{c|}{\begin{tabular}[c]{@{}c@{}}56.8\\ ($\downarrow$ 34.25\%)\end{tabular}} &
  \begin{tabular}[c]{@{}c@{}}81.6\\ ($\downarrow$ 0.20\%)\end{tabular} \\ \cline{4-7} 
 &
   &
   &
  DeiT-S &
  \multicolumn{1}{c|}{$\downarrow$47.1\%} &
  \multicolumn{1}{c|}{\begin{tabular}[c]{@{}c@{}}15.3\\ ($\downarrow$ 30.77\%)\end{tabular}} &
  \begin{tabular}[c]{@{}c@{}}79.6\\ ($\downarrow$ 0.20\%)\end{tabular} \\ \cline{4-7} 
 &
   &
   &
  DeiT-T &
  \multicolumn{1}{c|}{$\downarrow$41.7\%} &
  \multicolumn{1}{c|}{\begin{tabular}[c]{@{}c@{}}4.2\\ ($\downarrow$ 26.32\%)\end{tabular}} &
  \begin{tabular}[c]{@{}c@{}}71.0\\ ($\downarrow$ 1.20\%)\end{tabular} \\ \hline
\multirow{6}{*}{\begin{tabular}[c]{@{}c@{}}Cascade ViT\\ pruning~\cite{song2022cp}\end{tabular}} &
  \multirow{6}{*}{\begin{tabular}[c]{@{}c@{}}Utilizing the sparsity to prune \\ PH-regions in MSA \&\\ FFN progressively \\ \& dynamically\end{tabular}} &
  \multirow{2}{*}{CP-ViT} &
  ViT-B &
  \multicolumn{1}{c|}{$\downarrow$46.34\%} &
  \multicolumn{1}{c|}{-} &
  \begin{tabular}[c]{@{}c@{}}76.75\\ ($\downarrow$ 1.16\%)\end{tabular} \\ \cline{4-7} 
 &
   &
   &
  DeiT-B &
  \multicolumn{1}{c|}{$\downarrow$41.62\%} &
  \multicolumn{1}{c|}{-} &
  \begin{tabular}[c]{@{}c@{}}81.13\\ ($\downarrow$ 0.69\%)\end{tabular} \\ \cline{3-7} 
 &
   &
  \multirow{2}{*}{CP-ViT} &
  ViT-B &
  \multicolumn{1}{c|}{$\downarrow$29.03\%} &
  \multicolumn{1}{c|}{-} &
  \begin{tabular}[c]{@{}c@{}}96.20 \\ ($\downarrow$ 1.93\%)\end{tabular} \\ \cline{4-7} 
 &
   &
   &
  DeiT-B &
  \multicolumn{1}{c|}{$\downarrow$30.08\%} &
  \multicolumn{1}{c|}{-} &
  \begin{tabular}[c]{@{}c@{}}98.01\\ ($\downarrow$ 1.09\%)\end{tabular} \\ \cline{3-7} 
 &
   &
  \multirow{2}{*}{CP-ViT} &
  ViT-B &
  \multicolumn{1}{c|}{$\downarrow$32.05\%} &
  \multicolumn{1}{c|}{-} &
  \begin{tabular}[c]{@{}c@{}}84.79 \\ ($\downarrow$ 2.34\%)\end{tabular} \\ \cline{4-7} 
 &
   &
   &
  DeiT-B &
  \multicolumn{1}{c|}{$\downarrow$30.92\%} &
  \multicolumn{1}{c|}{-} &
  \begin{tabular}[c]{@{}c@{}}89.68\\ ($\downarrow$ 1.17\%)\end{tabular} \\ \hline
\end{tabular}%
}
\end{table}
    % Each technique is summarized and compared and the differences between structured pruning and its unstructured counterpart are highlighted.
\noindent \textbf{Static vs. Dynamic Pruning} Figure~\ref{fig:staticvsdynamic} demonstrates the workflow of the static and dynamic pruning. Static pruning works at the offline inference level, whereas dynamic pruning performs at the runtime level. Moreover, static pruning applies during training, where a fixed portion of a NN's components, such as neurons, channels, or weights, is removed or pruned. In static pruning, the decision on which components to prune is typically made before the training begins, and the pruning schedule remains constant throughout training. Static pruning is helpful in scenarios where the hardware constraints are well-defined. One of the most used static pruning techniques is Magnitude-based pruning~\cite{han2015learning}. In magnitude-based pruning, given a pruning rate \textbf{r}, weights whose absolute value is among the smallest \textbf{r\%} are pruned. In other words, for a weight matrix \textbf{W} of a layer, weights \textbf{w} in \textbf{W} are pruned if \textbf{\textbar w\textbar $\leq$ threshold}, where the \textbf{threshold} is determined such that the proportion of \textbf{\textbar w\textbar $\leq$ threshold} is \textbf{r}.
\begin{figure*}[h]
  \centering
  \includegraphics[scale=0.30]{assets/pruning_dynamic.png}
  \caption{Static Pruning vs Dynamic Pruning}
  \label{fig:staticvsdynamic}
\end{figure*}
In contrast, dynamic pruning applies during the runtime based on specific criteria, such as the importance of neurons or weights. One of the significant drawbacks of static pruning is that it relies on a fixed pruning schedule and rate, which is determined before training begins. That means that static pruning does not adjust to the network's learning progress or the changing importance of neurons or weights during training. Dynamic pruning evaluates and adjusts the pruning criteria during training based on real-time importance assessments to overcome the limitation of static pruning. Most of the recent pruning techniques use dynamic pruning techniques~\cite{song2022cp,bai2022dynamically,kong2022spvit} to get the accuracy without loss of any information.
\newline\hfill\break
\noindent \textbf{Cascade Pruning} Cascade pruning combines the iterative nature of dynamic pruning with predefined aspects resembling static pruning. Cascade pruning operates through multiple sequential iterations, also known as stages. Each stage selects a predefined portion of the network's components, such as neurons, channels, or weights, for pruning. The criteria for choosing which components to prune can vary between iterations. The ability to adjust pruning criteria between iterations makes cascade pruning adaptable to evolving training data, tasks, or hardware constraints.
% \subsubsection{Types}

% \paragraph{1. Unstructured vs. Structured Pruning.} \hfill \break
% \noindent{\bf Unstructured pruning} removes individual weights or parameters from the network based on certain criteria. Unstructured pruning can result in highly sparse networks. However, it often does not lead to computational efficiency as the sparsity is not aligned with the memory access patterns or computational primitives of hardware accelerators.

% \noindent{\bf Structured pruning} applies to the specific component of the network, especially in layers, neurons, or channels. Structured pruning leads to more hardware-friendly sparsity patterns but often at the cost of higher accuracy loss. Cai et al. ~\cite{10.1145/3543622.3573044} proposed a two-stage coarse-grained/fine-grained structured pruning method based on top-K sparsification and reduces 60\% overall computation in the embedded NN. In a recent survey ~\cite{he2023structured}, He et al. discussed surveys of a range of SOTA structured pruning techniques, covering topics such as filter ranking methods, regularization methods, dynamic execution, neural architecture search, the lottery ticket hypothesis, and the applications of pruning. 
%     % Each technique is summarized and compared and the differences between structured pruning and its unstructured counterpart are highlighted.
% \paragraph{2. Static vs Dynamic Pruning.} As shown in figure \ref{fig:staticvsdynamic}, static pruning works at the offline inference level, whereas dynamic pruning performs at the runtime level.
% % ~\cite{liang2021pruning}
% \begin{figure}[htp]
%   \centering
%   \includegraphics[scale=0.4]{assets/pruning_dynamic.png}
%   \caption{Static Pruning and Dynamic Pruning}
%   \label{fig:staticvsdynamic}
% \end{figure}

% \noindent {\bf Static pruning} applies during training, where a fixed portion of a neural network's components, such as neurons, channels, or weights, is removed or pruned. In static pruning, the decision on which components to prune is typically made before the training begins, and the pruning schedule remains constant throughout training. Static pruning is helpful in scenarios where the hardware constraints are well-defined. However, this technique has drawbacks with iterative pruning methods, where the process involves a cycle of pruning some neurons or weights, retraining the pruned network, and then pruning again. One of the most used static pruning techniques is Magnitude-based pruning~\cite{han2015learning}. In magnitude-based pruning, given a pruning rate \textbf{r}, weights whose absolute value is among the smallest \textbf{r\%} are pruned. In other words, for a weight matrix \textbf{W} of a layer, weights \textbf{w} in \textbf{W} are pruned if \textbf{|w| <= threshold}, where the \textbf{threshold} is determined such that the proportion of \textbf{|w| <= threshold} is \textbf{r}.

% \noindent {\bf Dynamic pruning} applies during the runtime based on specific criteria, such as the importance of neurons or weights. One of the significant drawbacks of static pruning is that it relies on a fixed pruning schedule and rate, which is determined before training begins. That means that static pruning does not adjust to the network's learning progress or the changing importance of neurons or weights during training. Dynamic pruning evaluates and adjusts the pruning criteria during training based on real-time importance assessments to overcome the limitation of static pruning. Most of the recent pruning techniques use dynamic pruning techniques~\cite{song2022cp,bai2022dynamically,kong2022spvit} to get the accuracy without loss of any information.

% \noindent {\bf Cascade pruning} combines the iterative nature of dynamic pruning with predefined aspects resembling static pruning. Cascade pruning operates through multiple sequential iterations, also known as stages. Each stage selects a predefined portion of the network's components, such as neurons, channels, or weights, for pruning. The criteria for choosing which components to prune can vary between iterations. The ability to adjust pruning criteria between iterations makes cascade pruning adaptable to evolving training data, tasks, or hardware constraints.


\subsubsection{Pruning Techniques for Vision Transformer}
\hfill\\
Pruning methods for ViT-based models remain an underexplored area, with only a handful of studies in recent years. This section provides a brief overview of the current SOTA pruning techniques for ViTs.\\ 
 
\noindent \textbf{Important-based Pruning} Zhu et al.~\cite{zhu2021vision} pioneered a ViT pruning approach that removes dimensions with lower importance scores, achieving a high pruning ratio without sacrificing accuracy. Their study observed that a significant portion of ViT's computational cost comes from multi-head self-attention (MSA) and multi-layer perceptron (MLP). To address this, they introduced visual transformer pruning (VTP)—the first dedicated pruning algorithm for ViTs. VTP operates in three key steps: (1) L1 sparse regularization is applied during training to identify less significant channels, (2) channel pruning eliminates redundant computations, and (3) finetuning. This VTP approach managed to preserve the robust representative capability of the transformer while reducing the model's computational cost. Another recent study~\cite{yu2022width} utilized learning a unique saliency score and threshold for each layer to implement width pruning. This learning saliency score allows for a more effective, non-uniform allocation of sparsity levels across different layers. Additionally, The proposed model utilized supplementary plug-in classifiers to prune the transformer's trailing blocks. This approach enabled the construction of a sequential variant of the pruned model, capable of removing blocks within a single training epoch, thereby simplifying the control of the trade-off between the network's performance and the rate of pruning ~\cite{yu2022width}. Moreover, another study by Tang et al.~\cite{tang2022patch} proposed a patch-slimming approach that reduced unimportant patches in a top-down manner. The authors calculated their importance scores for the final classification feature to identify unimportant patches. The proposed method also identified the important patches in the last layer of the blocks and then utilized them to select the previous layer patches.\\ 

\noindent \textbf{Token Pruning} Kong et al.~\cite{kong2022spvit} introduced a latency-aware soft token pruning framework, SP-ViT. This framework was implemented on vanilla transformers such as data-efficient image transformers (DeiT)~\cite{touvron2021training} and swin transformers~\cite{liu2021swin}. The authors proposed a dynamic attention-based multi-head token selector for adaptive instance-wise token selection. Later, they incorporated a soft pruning method that consolidated less informative tokens into a package token instead of entirely discarding them identified by the selector module. The authors deployed their proposed method on ImageNet-1k with baseline models Swin-S, Swin-T, PiT-S, and PiT-Xs.\\ 

\noindent \textbf{Structure Pruning} Recently, many studies on ViT pruning have embraced structure pruning techniques to optimize model efficiency. Yu et al.~\cite{yu2022unified} proposed a structure pruning in a ViT named UVC where they pruned the head's number and dimension inside each layer. Experiments of this paper were conducted in various ViT models (e.g., DeiT-Tiny and T2T-ViT-14) on ImageNet-1k~\cite{krizhevsky2012imagenet} datasets. DeiT-Tiny~\cite{touvron2021training} cut down to 50\% of the original FLOPs while not dropping accuracy much in this study. Another study~\cite{zheng2022savit} proposed structure pruning on MSA attention and feedforward neural network (FFN) by removing unnecessary parameter groups. Other studies on structure pruning named NViT~\cite{yang2023global} proposed hessian-based structure pruning criteria comparable across all layers and structures. Moreover, it incorporated latency-aware regularization techniques to reduce latency directly. Another study on the structure pruning in ViT called S\textsuperscript{2}ViT~\cite{chen2021chasing} removed submodules like self-attention heads by manipulating the weight, activations \& gradients.\\

\noindent \textbf{Cascade Pruning} Cascade pruning combines multiple pruning techniques to reduce parameters and GFLOPs while preserving accuracy. A standout method named VTC-LFC~\cite{wang2022vtc} aimed to improve the identification of informative channels and tokens in a model, leading to better accuracy preservation. This approach introduced a bottom-up cascade (BCP) pruning strategy that gradually prunes tokens and channels, starting from the first block and advancing to the last. The pruning process is controlled by a hyper-parameter called a \textbf{global allowable drop}, ensuring the performance drop remains within an acceptable range. Additionally, BCP guarantees efficient compression without sacrificing model performance by pruning each block and immediately stopping the compression process when the performance drop reaches a predefined threshold. Another cascade ViT pruning ~\cite{song2022cp} utilized the sparsity for pruning PH-regions (containing patches and heads) in the MSA \& FFN progressively and dynamically. The authors conducted experiments on three different types of datasets: ImageNet-1k~\cite{krizhevsky2012imagenet}, CIFAR-10~\cite{cifar10}, and CIFAR-100~\cite{cifar10}.\\


 \noindent \textbf{Miscellaneous Approaches in Pruning} Hou et al.~\cite{hou2022multi} introduced a multi-dimensional pruning strategy for ViTs, leveraging a statistical dependence-based criterion to identify and remove redundant components across different dimensions. Beyond this, several pruning techniques have been developed to accelerate ViTs, particularly for edge devices, including column balanced block pruning ~\cite{9424344}, end-to-end exploration ~\cite{chen2021chasing}, gradient-based learned runtime pruning ~\cite{li2022accelerating}. These techniques have shown stability in applying pruning on ViT models without compromising accuracy.\\
% Pruning methods applied to ViT-based models still need to be explored, with only a limited number of studies conducted in recent years. Hou et al. ~\cite{hou2022multi} proposed multi-dimensional techniques for ViT and a statistical dependence-based pruning criterion for different dimensions to identify deleterious components. In addition, Zhu et al. ~\cite{zhu2021vision} proposed a ViT-based pruning approach, where the number of dimensions with less important scores was omitted to achieve a high pruning ratio without compromising accuracy. The paper observed that a significant portion of computation was allocated to multi-head self-attention (MSA) and multi-layer perceptron (MLP) components. Consequently, the authors concentrated on diminishing the floating-point operations per second (FLOPs) of MSA and MLP. The paper introduced the visual transformer pruning (VTP) algorithm, marking the first pruning method specifically designed for ViTs. The algorithm proceeds in three steps: firstly, L1 sparse regularization is employed during training, followed by channel pruning, and finally, fine-tuning. This VTP approach managed to preserve the robust representative capability of the transformer while dramatically cutting down the model's calculation volume and parameter count. Alternate pruning techniques on ViT-based pruning mitigate both width and depth dimensions simultaneously ~\cite{yu2022width}. The method involved learning a unique saliency score and threshold for each layer to implement width pruning. This learning saliency score allows for a more effective, non-uniform allocation of sparsity levels across different layers. Additionally, The proposed model utilized supplementary plug-in classifiers for the pruning of the transformer's trailing blocks. This approach enabled the construction of a sequential variant of the pruned model, capable of removing blocks within a single training epoch, thereby simplifying the control of the trade-off between the network's performance and the rate of pruning ~\cite{yu2022width}.

% Given the computational complexity, the intrinsic data patterns of ViT, and the development requirements of edge devices, Kong et al.~\cite{kong2022spvit} introduced a latency-aware soft token pruning framework, SP-ViT. This framework was implemented on vanilla transformers such as data-efficient image transformers (DeiT)~\cite{touvron2021training} and swin transformers~\cite{liu2021swin}. The authors proposed a dynamic attention-based multi-head token selector for adaptive instance-wise token selection. Later, they incorporated a soft pruning method that consolidated less informative tokens into a package token instead of entirely discarding them identified by the selector module. The authors deployed their proposed method on ImageNet-1k with baseline models Swin-S, Swin-T, PiT-S, and PiT-Xs. The authors evaluated their method and achieved  top-1 accuracy with minimum accuracy loss.

% Moreover, another study by Tang et al.~\cite{tang2022patch} proposed a patch-slimming approach that mitigates useless patches in a top-down manner. The authors calculated their importance scores to the final classification feature to identify the useless patches. The proposed method also identified the useful patches in the last layer of the blocks and then used them to select the previous layer patches. The authors implemented their method based on the official ViT pre-trained models, and the results of their experiment are documented in Table  \ref{tab:pruning_result}.

% Furthermore, many recent papers on ViT pruning follow structure pruning methods. Yu et al.~\cite{yu2022unified} proposed a structure pruning in a ViT named UVC where they pruned the head's number and dimension inside each layer. Experiments of this paper were conducted in various ViT models (e.g., DeiT- Tiny and T2T-ViT-14) on ImageNet-1k~\cite{krizhevsky2012imagenet} datasets. DeiT-Tiny~\cite{touvron2021training} cut down to 50\% of the original FLOPs while not dropping accuracy much in this study. Another study~\cite{zheng2022savit} proposed structure pruning on MSA attention and feedforward neural network (FFN) by removing unnecessary parameter groups. Other studies on structure pruning named NViT~\cite{yang2023global} proposed hessian-based structure pruning criteria comparable across all layers and structures. Moreover, it incorporated latency-aware regularization techniques to reduce latency directly. Another study on the structure pruning in ViT called S\textsuperscript{2}ViT removed submodules like self-attention heads by manipulating the weight, activations \& gradients. 

% Moreover, there is another type of pruning method applied on ViT architectures named cascade pruning, where one or more pruning methods were combined to get better accuracy. A proposed method named VTC-LFC~\cite{wang2022vtc} aimed to improve the identification of informative channels and tokens in a model, leading to better accuracy preservation. It achieved this by introducing a bottom-up cascade (BCP) pruning strategy. It gradually prunes tokens and channels, starting from the first block and progressing toward the last block. The pruning is controlled by a hyper-parameter called a global allowable drop, which ensures that the performance drop remains within an acceptable range. The BCP process ensures that the compression is completed for each block when the performance drop reaches the threshold. This approach enabled precise pruning and compression, ultimately enhancing the model's efficiency while maintaining its performance. Another cascade ViT pruning ~\cite{song2022cp} utilized the sparsity for pruning PH-regions (containing patches and heads) in the MSA \& FFN progressively and dynamically. The authors conducted experiments on three different types of datasets: ImageNet-1k~\cite{krizhevsky2012imagenet}, CIFAR-10~\cite{cifar10}, and CIFAR-100~\cite{cifar10}. In recent times, there are pruning techniques used in  ViT for different accelerations on edge devices such as column balanced block pruning ~\cite{9424344}, end-to-end exploration ~\cite{chen2021chasing}, gradient-based learned runtime pruning ~\cite{li2022accelerating}. These techniques have shown stability in applying pruning on ViT models without compromising accuracy.
\subsubsection{Discussion}\hfill\\
Pruning is utilized as a fundamental way to reduce the computation of the pre-trained ViT models. For ViT, the development of the pruning methods has systematically covered each perspective of model design, making the current pruning methods more flexible and well-organized for ViT models. We summarized all the core information about pruning techniques on ViT in comparison to GFlops reductions with the percentage of reduction from baseline, parameters reductions, and top-1 accuracy in Table~\ref{tab:pruning_result}. Top-1 accuracy shows the accuracy loss with the proposed methodology from baseline backbone ViT architecture. Recent pruning studies on ViT models, as shown in Table~\ref{tab:pruning_result}, have predominantly focused on the ImageNet-1k dataset~\cite{5206848}, with the exception of CP-ViT~\cite{song2022cp}, which conducted experiments on the CIFAR dataset~\cite{cifar10}. However, the training/finetuning cost is one of the critical points to consider in the hardware-inefficient ViT models in the pruning methods. Therefore, training-efficient or fine-tuned free pruning techniques need more attention in the near future for efficient deployment on the edge. This necessitates a more precise estimation of parameter or block sensitivity using limited data, as well as a deeper exploration of the information embedded within hidden features during training and inference.
% Please add the following required packages to your document preamble:
% \usepackage{multirow}
% \usepackage{longtable}
% Note: It may be necessary to compile the document several times to get a multi-page table to line up properly
% \renewcommand{\arraystretch}{0.65}
%  \newgeometry{hmargin=3cm,vmargin=3cm,landscape}
\renewcommand{\arraystretch}{0.74}
% Please add the following required packages to your document preamble:
% \usepackage{multirow}
% \usepackage{graphicx}
% \usepackage{lscape}
% \begin{landscape}
% \begin{table}[]
% \centering
% \caption{Results of different KD (classification) techniques proposed for Vision transformers.}
% \label{tab:KD_result_classification}
% \resizebox{\columnwidth}{!}{%
% \begin{tabular}{c|c|c|c|c|cc|cc}
% \hline
% \multirow{2}{*}{\textbf{Algorithm}} &
%   \multirow{2}{*}{\textbf{Key points}} &
%   \multirow{2}{*}{\textbf{Method}} &
%   \multirow{2}{*}{\textbf{Loss function}} &
%   \multirow{2}{*}{\textbf{Dataset}} &
%   \multicolumn{2}{c|}{\textbf{Teachers}} &
%   \multicolumn{2}{c}{\textbf{Students}} \\ \cline{6-9} 
%  &
%    &
%    &
%    &
%    &
%   \multicolumn{1}{c|}{\textbf{Models}} &
%   \textbf{\begin{tabular}[c]{@{}c@{}}Top-1\\ (\%)\end{tabular}} &
%   \multicolumn{1}{c|}{\textbf{Models}} &
%   \textbf{\begin{tabular}[c]{@{}c@{}}Top-1\\ (\%)\end{tabular}} \\ \hline
% \multirow{3}{*}{\begin{tabular}[c]{@{}c@{}}Fine-grained\\ manifold ~\cite{hao2022learning}\end{tabular}} &
%   \multirow{3}{*}{\begin{tabular}[c]{@{}c@{}}Teach the student layer\\  having the same patch-\\ level manifold structure \\ as the teacher layer\end{tabular}} &
%   \multirow{3}{*}{\begin{tabular}[c]{@{}c@{}}Patch-level manifold\\  space method\end{tabular}} &
%   \multirow{3}{*}{\begin{tabular}[c]{@{}c@{}}Manifold distillation\\  loss (MD Loss)\end{tabular}} &
%   \multirow{3}{*}{ImageNet-1k~\cite{5206848}} &
%   \multicolumn{1}{c|}{CaiT-S24} &
%   83.4\% &
%   \multicolumn{1}{c|}{DeiT-T} &
%   \begin{tabular}[c]{@{}c@{}}76.5\\ ($\uparrow$4.3\%)\end{tabular} \\ \cline{6-9} 
%  &
%    &
%    &
%    &
%    &
%   \multicolumn{1}{c|}{CaiT-S24} &
%   83.4\% &
%   \multicolumn{1}{c|}{DeiT-S} &
%   \begin{tabular}[c]{@{}c@{}}82.2\\ ($\uparrow$2.3\%)\end{tabular} \\ \cline{6-9} 
%  &
%    &
%    &
%    &
%    &
%   \multicolumn{1}{c|}{Swin-S} &
%   83.2\% &
%   \multicolumn{1}{c|}{Swin-T} &
%   \begin{tabular}[c]{@{}c@{}}82.2\\ ($\uparrow$1.0\%)\end{tabular} \\ \hline
% \multirow{2}{*}{\begin{tabular}[c]{@{}c@{}}Target-aware \\ Transformer ~\cite{lin2022knowledge}\end{tabular}} &
%   \multirow{2}{*}{Hierarchical distillation} &
%   \multirow{2}{*}{\begin{tabular}[c]{@{}c@{}}One-to-all spatial\\  matching KD\end{tabular}} &
%   \multirow{2}{*}{\begin{tabular}[c]{@{}c@{}}Vanilla distillation +\\  L\textsubscript2 + Task loss\end{tabular}} &
%   \multirow{2}{*}{ImageNet-1k~\cite{5206848}} &
%   \multicolumn{1}{c|}{\multirow{2}{*}{ResNet34}} &
%   \multirow{2}{*}{72.4\%} &
%   \multicolumn{1}{c|}{\multirow{2}{*}{ResNet18}} &
%   \multirow{2}{*}{\begin{tabular}[c]{@{}c@{}}72.1\\ ($\uparrow$+2.0\%)\end{tabular}} \\
%  &
%    &
%    &
%    &
%    &
%   \multicolumn{1}{c|}{} &
%    &
%   \multicolumn{1}{c|}{} &
%    \\ \hline
% \multirow{4}{*}{\begin{tabular}[c]{@{}c@{}}Cross Inductive \\ Bias Distillation ~\cite{ren2022co}\end{tabular}} &
%   \multirow{4}{*}{\begin{tabular}[c]{@{}c@{}}Enables student \\ transformers to emulate \\ the performance of various\\  inductive bias teachers\end{tabular}} &
%   \multirow{4}{*}{\begin{tabular}[c]{@{}c@{}}Co-advising the student\\  models with lightweight\\  teacher model\end{tabular}} &
%   \multirow{4}{*}{\begin{tabular}[c]{@{}c@{}}Kull back divergence +\\ Cross entropy loss\end{tabular}} &
%   \multirow{4}{*}{ImageNet-1k~\cite{5206848}} &
%   \multicolumn{1}{c|}{\multirow{4}{*}{ResNet18}} &
%   \multirow{4}{*}{83.4\%} &
%   \multicolumn{1}{c|}{\multirow{4}{*}{Transformer-Ti}} &
%   \multirow{4}{*}{\begin{tabular}[c]{@{}c@{}}88.0\\ ($\uparrow$+1.5\%)\end{tabular}} \\
%  &
%    &
%    &
%    &
%    &
%   \multicolumn{1}{c|}{} &
%    &
%   \multicolumn{1}{c|}{} &
%    \\
%  &
%    &
%    &
%    &
%    &
%   \multicolumn{1}{c|}{} &
%    &
%   \multicolumn{1}{c|}{} &
%    \\
%  &
%    &
%    &
%    &
%    &
%   \multicolumn{1}{c|}{} &
%    &
%   \multicolumn{1}{c|}{} &
%    \\ \hline
% \multirow{3}{*}{\begin{tabular}[c]{@{}c@{}}Attention \\ probe ~\cite{9747484}\end{tabular}} &
%   \multirow{3}{*}{\begin{tabular}[c]{@{}c@{}}Utilizing the intermediate\\  information for transferring  \\ the embedding features\\ between teacher and \\ student directly\end{tabular}} &
%   \multirow{3}{*}{\begin{tabular}[c]{@{}c@{}}Probe distillation \\ \& Knowledge distillation\end{tabular}} &
%   \multirow{3}{*}{\begin{tabular}[c]{@{}c@{}}Probe distillation + \\ cross-entropy\end{tabular}} &
%   CIFAR-100 &
%   \multicolumn{1}{c|}{DeiT-XS} &
%   76.30\% &
%   \multicolumn{1}{c|}{DeiT-XTiny} &
%   \begin{tabular}[c]{@{}c@{}}71.82\\ ($\uparrow$+6.36\%)\end{tabular} \\ \cline{5-9} 
%  &
%    &
%    &
%    &
%   CIFAR-10 &
%   \multicolumn{1}{c|}{DeiT-XS} &
%   96.65\% &
%   \multicolumn{1}{c|}{DeiT-XTiny} &
%   \begin{tabular}[c]{@{}c@{}}93.95\\ ($\uparrow$+7.64\%)\end{tabular} \\ \cline{5-9} 
%  &
%    &
%    &
%    &
%   MNIST &
%   \multicolumn{1}{c|}{DeiT-XS} &
%   99.39\% &
%   \multicolumn{1}{c|}{DeiT-XTiny} &
%   \begin{tabular}[c]{@{}c@{}}99.07\\ ($\uparrow$+0.01\%)\end{tabular} \\ \hline
% \multirow{3}{*}{MiniViT ~\cite{zhang2022minivit}} &
%   \multirow{3}{*}{\begin{tabular}[c]{@{}c@{}}Prediction-Logit Distillation \&\\ Hidden-State Distillation\end{tabular}} &
%   \multirow{3}{*}{Weight distillation} &
%   \multirow{3}{*}{\begin{tabular}[c]{@{}c@{}}Self-attention distillation + \\ Hidden-state distillation + \\ prediction loss\end{tabular}} &
%   \multirow{3}{*}{ImageNet-1k~\cite{5206848}} &
%   \multicolumn{1}{c|}{\multirow{3}{*}{\begin{tabular}[c]{@{}c@{}}RegNet-\\ 16GF\end{tabular}}} &
%   \multirow{3}{*}{82.9\%} &
%   \multicolumn{1}{c|}{\multirow{3}{*}{DeiT-B}} &
%   \multirow{3}{*}{\begin{tabular}[c]{@{}c@{}}83.2\\ ($\uparrow$+1.4\%)\end{tabular}} \\
%  &
%    &
%    &
%    &
%    &
%   \multicolumn{1}{c|}{} &
%    &
%   \multicolumn{1}{c|}{} &
%    \\
%  &
%    &
%    &
%    &
%    &
%   \multicolumn{1}{c|}{} &
%    &
%   \multicolumn{1}{c|}{} &
%    \\ \hline
% \multirow{3}{*}{TinyViT ~\cite{wu2022tinyvit}} &
%   \multirow{3}{*}{\begin{tabular}[c]{@{}c@{}}Applying distillation during \\ pretraining to transfer \\ knowledge\end{tabular}} &
%   \multirow{3}{*}{\begin{tabular}[c]{@{}c@{}}Reusing the teachers' \\ prediction \& data \\ augmentation for student\end{tabular}} &
%   \multirow{3}{*}{Cross entropy loss} &
%   \multirow{3}{*}{ImageNet-1k~\cite{5206848}} &
%   \multicolumn{1}{c|}{\multirow{3}{*}{\begin{tabular}[c]{@{}c@{}}CLIP-\\ ViT-L\end{tabular}}} &
%   \multirow{3}{*}{84.8\%} &
%   \multicolumn{1}{c|}{Swin-T} &
%   \begin{tabular}[c]{@{}c@{}}83.4\\ ($\uparrow$+2.2\%)\end{tabular} \\ \cline{8-9} 
%  &
%    &
%    &
%    &
%    &
%   \multicolumn{1}{c|}{} &
%    &
%   \multicolumn{1}{c|}{\multirow{2}{*}{DeiT-S}} &
%   \multirow{2}{*}{\begin{tabular}[c]{@{}c@{}}82.0\\ ($\uparrow$+2.1\%)\end{tabular}} \\
%  &
%    &
%    &
%    &
%    &
%   \multicolumn{1}{c|}{} &
%    &
%   \multicolumn{1}{c|}{} &
%    \\ \hline
% DearKD ~\cite{chen2022dearkd} &
%   \begin{tabular}[c]{@{}c@{}}Representational kD based \\ on intermediate features \&\\  response based KD\end{tabular} &
%   Self-generative data &
%   \begin{tabular}[c]{@{}c@{}}MSE distillation +\\ Cross entropy+ Intra\\ -divergence distillation loss\end{tabular} &
%   ImageNet-1k~\cite{5206848} &
%   \multicolumn{1}{c|}{\begin{tabular}[c]{@{}c@{}}ResNet-\\ 101\end{tabular}} &
%   77.37\% &
%   \multicolumn{1}{c|}{DeiT-Ti} &
%   \begin{tabular}[c]{@{}c@{}}71.2\\ ($\downarrow$+1.0\%)\end{tabular} \\ \hline
% \end{tabular}%
% }
% \end{table}
% \end{landscape}
% Please add the following required packages to your document preamble:
% \usepackage{multirow}
% \usepackage{graphicx}
\begin{table}[]
\caption{Results of different KD (classification) techniques proposed for vision transformers. Here, MSE loss means mean square error loss. }
\label{tab:KD_result_classification}
\resizebox{\columnwidth}{!}{%
\begin{tabular}{c|c|c|c|cc|cc}
\hline
\multirow{2}{*}{\textbf{Algorithm}} &
  \multirow{2}{*}{\textbf{Method}} &
  \multirow{2}{*}{\textbf{Loss function}} &
  \multirow{2}{*}{\textbf{Dataset}} &
  \multicolumn{2}{c|}{\textbf{Teachers}} &
  \multicolumn{2}{c}{\textbf{Students}} \\ \cline{5-8} 
 &
   &
   &
   &
  \multicolumn{1}{c|}{\textbf{Models}} &
  \textbf{\begin{tabular}[c]{@{}c@{}}Top-1\\ (\%)\end{tabular}} &
  \multicolumn{1}{c|}{\textbf{Models}} &
  \textbf{\begin{tabular}[c]{@{}c@{}}Top-1\\ (\%)\end{tabular}} \\ \hline
\multirow{3}{*}{\begin{tabular}[c]{@{}c@{}}Fine-grained\\ manifold ~\cite{hao2022learning}\end{tabular}} &
  \multirow{3}{*}{\begin{tabular}[c]{@{}c@{}}Patch-level manifold\\  space method\end{tabular}} &
  \multirow{3}{*}{\begin{tabular}[c]{@{}c@{}}Manifold distillation\\  loss (MD Loss)\end{tabular}} &
  \multirow{3}{*}{ImageNet-1k~\cite{5206848}} &
  \multicolumn{1}{c|}{CaiT-S24} &
  83.4\% &
  \multicolumn{1}{c|}{DeiT-T} &
  \begin{tabular}[c]{@{}c@{}}76.5\\ ($\uparrow$4.3\%)\end{tabular} \\ \cline{5-8} 
 &
   &
   &
   &
  \multicolumn{1}{c|}{CaiT-S24} &
  83.4\% &
  \multicolumn{1}{c|}{DeiT-S} &
  \begin{tabular}[c]{@{}c@{}}82.2\\ ($\uparrow$2.3\%)\end{tabular} \\ \cline{5-8} 
 &
   &
   &
   &
  \multicolumn{1}{c|}{Swin-S} &
  83.2\% &
  \multicolumn{1}{c|}{Swin-T} &
  \begin{tabular}[c]{@{}c@{}}82.2\\ ($\uparrow$1.0\%)\end{tabular} \\ \hline
\multirow{2}{*}{\begin{tabular}[c]{@{}c@{}}Target-aware \\ Transformer ~\cite{lin2022knowledge}\end{tabular}} &
  \multirow{2}{*}{\begin{tabular}[c]{@{}c@{}}One-to-all spatial\\  matching KD\end{tabular}} &
  \multirow{2}{*}{\begin{tabular}[c]{@{}c@{}}Vanilla distillation +\\  L\textsubscript2 + Task loss\end{tabular}} &
  \multirow{2}{*}{ImageNet-1k~\cite{5206848}} &
  \multicolumn{1}{c|}{\multirow{2}{*}{ResNet34}} &
  \multirow{2}{*}{72.4\%} &
  \multicolumn{1}{c|}{\multirow{2}{*}{ResNet18}} &
  \multirow{2}{*}{\begin{tabular}[c]{@{}c@{}}72.1\\ ($\uparrow$+2.0\%)\end{tabular}} \\
 &
   &
   &
   &
  \multicolumn{1}{c|}{} &
   &
  \multicolumn{1}{c|}{} &
   \\ \hline
\multirow{4}{*}{\begin{tabular}[c]{@{}c@{}}Cross Inductive \\ Bias Distillation ~\cite{ren2022co}\end{tabular}} &
  \multirow{4}{*}{\begin{tabular}[c]{@{}c@{}}Co-advising the student\\  models with lightweight\\  teacher model\end{tabular}} &
  \multirow{4}{*}{\begin{tabular}[c]{@{}c@{}}Kull back divergence +\\ Cross entropy loss\end{tabular}} &
  \multirow{4}{*}{ImageNet-1k~\cite{5206848}} &
  \multicolumn{1}{c|}{\multirow{4}{*}{ResNet18}} &
  \multirow{4}{*}{83.4\%} &
  \multicolumn{1}{c|}{\multirow{4}{*}{Transformer-Ti}} &
  \multirow{4}{*}{\begin{tabular}[c]{@{}c@{}}88.0\\ ($\uparrow$+1.5\%)\end{tabular}} \\
 &
   &
   &
   &
  \multicolumn{1}{c|}{} &
   &
  \multicolumn{1}{c|}{} &
   \\
 &
   &
   &
   &
  \multicolumn{1}{c|}{} &
   &
  \multicolumn{1}{c|}{} &
   \\
 &
   &
   &
   &
  \multicolumn{1}{c|}{} &
   &
  \multicolumn{1}{c|}{} &
   \\ \hline
\multirow{3}{*}{\begin{tabular}[c]{@{}c@{}}Attention \\ probe ~\cite{9747484}\end{tabular}} &
  \multirow{3}{*}{\begin{tabular}[c]{@{}c@{}}Probe distillation \\ \& Knowledge distillation\end{tabular}} &
  \multirow{3}{*}{\begin{tabular}[c]{@{}c@{}}Probe distillation + \\ cross-entropy\end{tabular}} &
  CIFAR-100 &
  \multicolumn{1}{c|}{DeiT-XS} &
  76.30\% &
  \multicolumn{1}{c|}{DeiT-XTiny} &
  \begin{tabular}[c]{@{}c@{}}71.82\\ ($\uparrow$+6.36\%)\end{tabular} \\ \cline{4-8} 
 &
   &
   &
  CIFAR-10 &
  \multicolumn{1}{c|}{DeiT-XS} &
  96.65\% &
  \multicolumn{1}{c|}{DeiT-XTiny} &
  \begin{tabular}[c]{@{}c@{}}93.95\\ ($\uparrow$+7.64\%)\end{tabular} \\ \cline{4-8} 
 &
   &
   &
  MNIST &
  \multicolumn{1}{c|}{DeiT-XS} &
  99.39\% &
  \multicolumn{1}{c|}{DeiT-XTiny} &
  \begin{tabular}[c]{@{}c@{}}99.07\\ ($\uparrow$+0.01\%)\end{tabular} \\ \hline
\multirow{3}{*}{MiniViT ~\cite{zhang2022minivit}} &
  \multirow{3}{*}{Weight distillation} &
  \multirow{3}{*}{\begin{tabular}[c]{@{}c@{}}Self-attention distillation + \\ Hidden-state distillation + \\ prediction loss\end{tabular}} &
  \multirow{3}{*}{ImageNet-1k~\cite{5206848}} &
  \multicolumn{1}{c|}{\multirow{3}{*}{\begin{tabular}[c]{@{}c@{}}RegNet-\\ 16GF\end{tabular}}} &
  \multirow{3}{*}{82.9\%} &
  \multicolumn{1}{c|}{\multirow{3}{*}{DeiT-B}} &
  \multirow{3}{*}{\begin{tabular}[c]{@{}c@{}}83.2\\ ($\uparrow$+1.4\%)\end{tabular}} \\
 &
   &
   &
   &
  \multicolumn{1}{c|}{} &
   &
  \multicolumn{1}{c|}{} &
   \\
 &
   &
   &
   &
  \multicolumn{1}{c|}{} &
   &
  \multicolumn{1}{c|}{} &
   \\ \hline
\multirow{3}{*}{TinyViT ~\cite{wu2022tinyvit}} &
  \multirow{3}{*}{\begin{tabular}[c]{@{}c@{}}Reusing the teachers' \\ prediction \& data \\ augmentation for student\end{tabular}} &
  \multirow{3}{*}{Cross entropy loss} &
  \multirow{3}{*}{ImageNet-1k~\cite{5206848}} &
  \multicolumn{1}{c|}{\multirow{3}{*}{\begin{tabular}[c]{@{}c@{}}CLIP-\\ ViT-L\end{tabular}}} &
  \multirow{3}{*}{84.8\%} &
  \multicolumn{1}{c|}{Swin-T} &
  \begin{tabular}[c]{@{}c@{}}83.4\\ ($\uparrow$+2.2\%)\end{tabular} \\ \cline{7-8} 
 &
   &
   &
   &
  \multicolumn{1}{c|}{} &
   &
  \multicolumn{1}{c|}{\multirow{2}{*}{DeiT-S}} &
  \multirow{2}{*}{\begin{tabular}[c]{@{}c@{}}82.0\\ ($\uparrow$+2.1\%)\end{tabular}} \\
 &
   &
   &
   &
  \multicolumn{1}{c|}{} &
   &
  \multicolumn{1}{c|}{} &
   \\ \hline
DearKD ~\cite{chen2022dearkd} &
  Self-generative data &
  \begin{tabular}[c]{@{}c@{}}MSE distillation +\\ Cross entropy+ Intra\\ -divergence distillation loss\end{tabular} &
  ImageNet-1k~\cite{5206848} &
  \multicolumn{1}{c|}{\begin{tabular}[c]{@{}c@{}}ResNet-\\ 101\end{tabular}} &
  77.37\% &
  \multicolumn{1}{c|}{DeiT-Ti} &
  \begin{tabular}[c]{@{}c@{}}71.2\\ ($\downarrow$+1.0\%)\end{tabular} \\ \hline
\end{tabular}%
}
\vspace{-5mm}
\end{table}

\subsection{Knowledge Distillation}\label{know}
Knowledge distillation (KD) is another model compression technique in machine learning where a smaller model (the "student") is trained to reproduce the behavior of a larger model (the "teacher"). The purpose is to transfer the "knowledge" from the larger model to the smaller one, thereby reducing computational resources without significantly losing accuracy.
% \renewcommand{\arraystretch}{0.3}
% \begin{longtable}[c]{c|c|c|c|ccc}
% \caption{Results of different pruning techniques proposed for Vision transformers. '\textbf{$\downarrow$}' denotes reduction from the baseline and '$\uparrow$' denotes increase rate from the baseline models}
% \label{tab:pruning_result}\\
% \hline
% \multirow{2}{*}{\textbf{Algorithm}} &
%   \multirow{2}{*}{\textbf{Method}} &
%   \multirow{2}{*}{\textbf{Models}} &
%   \multirow{2}{*}{\textbf{Baseline}} &
%   \multicolumn{3}{c}{\textbf{Results}} \\ \cline{5-7} 
%  &
%    &
%    &
%    &
%   \multicolumn{1}{c|}{\textbf{GFlops}} &
%   \multicolumn{1}{c|}{\textbf{Params(M)}} &
%   \textbf{Top-1 (\%)} \\ \hline
% \endfirsthead
% %
% \endhead
% %
% \begin{tabular}[c]{@{}c@{}}Channels \\ pruning ~\cite{zhu2021vision}\end{tabular} &
%   \begin{tabular}[c]{@{}c@{}}Learn dimension-wise\\ important score\end{tabular} &
%   VTP &
%   DeiT-B &
%   \multicolumn{1}{c|}{\begin{tabular}[c]{@{}c@{}}10.0\\ ($\downarrow$ 45.3\%)\end{tabular}} &
%   \multicolumn{1}{c|}{$\downarrow$ 47.3} &
%   \begin{tabular}[c]{@{}c@{}}92.58\\ ($\downarrow$ 1.92\%)\end{tabular} \\ \cline{3-7} 
%  &
%    &
%   VTP &
%   DeiT-B &
%   \multicolumn{1}{c|}{\begin{tabular}[c]{@{}c@{}}10.0\\ ($\downarrow$ 43.1\%)\end{tabular}} &
%   \multicolumn{1}{c|}{$\downarrow$ 48.0} &
%   $\downarrow$ 1.1\% \\ \hline
% \multirow{4}{*}{\begin{tabular}[c]{@{}c@{}}Width \& Depth \\ Pruning ~\cite{yu2022width}\end{tabular}} &
%   \multirow{4}{*}{\begin{tabular}[c]{@{}c@{}}Set of learnable pruning\\ -related parameters for \\ width pruning \& shallow \\ classifiers using intermediate \\ information of the transformer\\ blocks\end{tabular}} &
%   \multirow{4}{*}{WDPruning} &
%   DeiT-T &
%   \multicolumn{1}{c|}{\begin{tabular}[c]{@{}c@{}}2.6\\ ($\downarrow$ 43.5\%)\end{tabular}} &
%   \multicolumn{1}{c|}{\begin{tabular}[c]{@{}c@{}}13.3 \\ ($\downarrow$ 37.6\%)\end{tabular}} &
%   \begin{tabular}[c]{@{}c@{}}70.34\\ ($\downarrow$ 1.86\%)\end{tabular} \\ \cline{4-7} 
%  &
%    &
%    &
%   DeiT-S &
%   \multicolumn{1}{c|}{\begin{tabular}[c]{@{}c@{}}0.7\\ ($\downarrow$ 46.2\%)\end{tabular}} &
%   \multicolumn{1}{c|}{\begin{tabular}[c]{@{}c@{}}3.5\\ ($\downarrow$ 35.2\%)\end{tabular}} &
%   \begin{tabular}[c]{@{}c@{}}78.38\\ ($\downarrow$ 1.42\%)\end{tabular} \\ \cline{4-7} 
%  &
%    &
%    &
%   DeiT-B &
%   \multicolumn{1}{c|}{\begin{tabular}[c]{@{}c@{}}9.90\\ ($\downarrow$ 43.4\%)\end{tabular}} &
%   \multicolumn{1}{c|}{\begin{tabular}[c]{@{}c@{}}55.3 \\ ($\downarrow$ 35.0\%)\end{tabular}} &
%   \begin{tabular}[c]{@{}c@{}}80.76\\ ($\downarrow$ 1.04\%)\end{tabular} \\ \cline{4-7} 
%  &
%    &
%    &
%   Swin-S &
%   \multicolumn{1}{c|}{\begin{tabular}[c]{@{}c@{}}6.3\\ ($\downarrow$ 27.6\%)\end{tabular}} &
%   \multicolumn{1}{c|}{\begin{tabular}[c]{@{}c@{}}32.8\\ ($\downarrow$ 30.6\%)\end{tabular}} &
%   \begin{tabular}[c]{@{}c@{}}81.80\\ ($\downarrow$ 1.20\%)\end{tabular} \\ \hline
% \multirow{3}{*}{\begin{tabular}[c]{@{}c@{}}Multi-\\ dimensional\\ pruning ~\cite{hou2022multi}\end{tabular}} &
%   \multirow{3}{*}{\begin{tabular}[c]{@{}c@{}}Dependency based pruning \\ criterion \& an efficient \\ Gaussian process search\end{tabular}} &
%   \multirow{3}{*}{\begin{tabular}[c]{@{}c@{}}Multi-\\ dimensional\end{tabular}} &
%   DeiT-S &
%   \multicolumn{1}{c|}{\begin{tabular}[c]{@{}c@{}}2.9\\ ($\downarrow$37\%)\end{tabular}} &
%   \multicolumn{1}{c|}{-} &
%   \begin{tabular}[c]{@{}c@{}}79.9\\ ($\downarrow$0.1\%)\end{tabular} \\ \cline{4-7} 
%  &
%    &
%    &
%   DeiT-B &
%   \multicolumn{1}{c|}{\begin{tabular}[c]{@{}c@{}}11.2\\ ($\downarrow$36\%)\end{tabular}} &
%   \multicolumn{1}{c|}{-} &
%   \begin{tabular}[c]{@{}c@{}}82.3\\ ($\downarrow$0.5\%)\end{tabular} \\ \cline{4-7} 
%  &
%    &
%    &
%   \begin{tabular}[c]{@{}c@{}}T2T-\\ ViT-14\end{tabular} &
%   \multicolumn{1}{c|}{\begin{tabular}[c]{@{}c@{}}2.9\\ ($\downarrow$40\%)\end{tabular}} &
%   \multicolumn{1}{c|}{-} &
%   \begin{tabular}[c]{@{}c@{}}81.7\\ ($\downarrow$0.2\%)\end{tabular} \\ \hline
% \multirow{4}{*}{\begin{tabular}[c]{@{}c@{}}Pruning the \\ network model ~\cite{kong2022spvit}\end{tabular}} &
%   \multirow{4}{*}{\begin{tabular}[c]{@{}c@{}}Single-path ViT\\  pruning based on the \\ token score\end{tabular}} &
%   \multirow{4}{*}{SPViT} &
%   Swin-S &
%   \multicolumn{1}{c|}{\begin{tabular}[c]{@{}c@{}}6.35 \\ ($\downarrow$ 26.4\%)\end{tabular}} &
%   \multicolumn{1}{c|}{-} &
%   \begin{tabular}[c]{@{}c@{}}82.71\\ ($\downarrow$ 0.49\%)\end{tabular} \\ \cline{4-7} 
%  &
%    &
%    &
%   Swin-T &
%   \multicolumn{1}{c|}{\begin{tabular}[c]{@{}c@{}}3.47 \\ ($\downarrow$ 23.0\%)\end{tabular}} &
%   \multicolumn{1}{c|}{-} &
%   \begin{tabular}[c]{@{}c@{}}80.70 \\ ($\downarrow$ 0.50\%)\end{tabular} \\ \cline{4-7} 
%  &
%    &
%    &
%   PiT-S &
%   \multicolumn{1}{c|}{\begin{tabular}[c]{@{}c@{}}2.22 \\ ($\downarrow$ 23.3\%)\end{tabular}} &
%   \multicolumn{1}{c|}{-} &
%   \begin{tabular}[c]{@{}c@{}}80.38 \\ ($\downarrow$ 0.58\%)\end{tabular} \\ \cline{4-7} 
%  &
%    &
%    &
%   PiT-XS &
%   \multicolumn{1}{c|}{\begin{tabular}[c]{@{}c@{}}1.13 \\ ($\downarrow$ 18.7\%)\end{tabular}} &
%   \multicolumn{1}{c|}{-} &
%   \begin{tabular}[c]{@{}c@{}}77.86 \\ ($\downarrow$ 0.24\%)\end{tabular} \\ \hline
% \multirow{4}{*}{\begin{tabular}[c]{@{}c@{}}Patch \\ pruning ~\cite{tang2022patch}\end{tabular}} &
%   \multirow{4}{*}{\begin{tabular}[c]{@{}c@{}}Layer-by-layer \\ top down pruning\end{tabular}} &
%   \multirow{4}{*}{PS-ViT} &
%   DeiT-T &
%   \multicolumn{1}{c|}{\begin{tabular}[c]{@{}c@{}}0.7 \\ ($\downarrow$ 46.2\%)\end{tabular}} &
%   \multicolumn{1}{c|}{-} &
%   \begin{tabular}[c]{@{}c@{}}72.0 \\ ($\downarrow$ 0.20\%)\end{tabular} \\ \cline{4-7} 
%  &
%    &
%    &
%   DeiT-S &
%   \multicolumn{1}{c|}{\begin{tabular}[c]{@{}c@{}}2.6 \\ ($\downarrow$ 43.6\%)\end{tabular}} &
%   \multicolumn{1}{c|}{-} &
%   \begin{tabular}[c]{@{}c@{}}79.4 \\ ($\downarrow$ 0.40\%)\end{tabular} \\ \cline{4-7} 
%  &
%    &
%    &
%   DeiT-B &
%   \multicolumn{1}{c|}{\begin{tabular}[c]{@{}c@{}}9.8 \\ ($\downarrow$ 44.3\%)\end{tabular}} &
%   \multicolumn{1}{c|}{-} &
%   \begin{tabular}[c]{@{}c@{}}81.5\\ ($\downarrow$ 0.30\%)\end{tabular} \\ \cline{4-7} 
%  &
%    &
%    &
%   \begin{tabular}[c]{@{}c@{}}T2T-\\ ViT-14\end{tabular} &
%   \multicolumn{1}{c|}{\begin{tabular}[c]{@{}c@{}}3.1 \\ ($\downarrow$ 40.4\%)\end{tabular}} &
%   \multicolumn{1}{c|}{-} &
%   \begin{tabular}[c]{@{}c@{}}81.1\\ ($\downarrow$ 0.40\%)\end{tabular} \\ \hline
% \multirow{4}{*}{\begin{tabular}[c]{@{}c@{}}Structural \\ pruning ~\cite{yu2022unified}\end{tabular}} &
%   \multirow{4}{*}{\begin{tabular}[c]{@{}c@{}}Prune the head \\ number \& head\\ dimensions inside \\ each layer\end{tabular}} &
%   \multirow{4}{*}{UVC} &
%   DeiT-T &
%   \multicolumn{1}{c|}{\begin{tabular}[c]{@{}c@{}}0.51\\ (39.12\%)\end{tabular}} &
%   \multicolumn{1}{c|}{-} &
%   \begin{tabular}[c]{@{}c@{}}70.6\\ ($\downarrow$ 1.6\%)\end{tabular} \\ \cline{4-7} 
%  &
%    &
%    &
%   DeiT-S &
%   \multicolumn{1}{c|}{\begin{tabular}[c]{@{}c@{}}2.32\\ (50.41\%)\end{tabular}} &
%   \multicolumn{1}{c|}{-} &
%   \begin{tabular}[c]{@{}c@{}}78.82\\ ($\downarrow$ 0.98\%)\end{tabular} \\ \cline{4-7} 
%  &
%    &
%    &
%   DeiT-B &
%   \multicolumn{1}{c|}{\begin{tabular}[c]{@{}c@{}}8.0\\ (45.50\%)\end{tabular}} &
%   \multicolumn{1}{c|}{-} &
%   \begin{tabular}[c]{@{}c@{}}80.57 \\ ($\downarrow$ 1.23\%)\end{tabular} \\ \cline{4-7} 
%  &
%    &
%    &
%   \begin{tabular}[c]{@{}c@{}}T2T-\\ ViT-14\end{tabular} &
%   \multicolumn{1}{c|}{\begin{tabular}[c]{@{}c@{}}2.11 \\ ($\downarrow$ 44.0\%)\end{tabular}} &
%   \multicolumn{1}{c|}{-} &
%   \begin{tabular}[c]{@{}c@{}}78.9\\ ($\downarrow$ 2.6\%)\end{tabular} \\ \hline
% \multirow{4}{*}{\begin{tabular}[c]{@{}c@{}}Global structural \\ pruning ~\cite{yang2023global}\end{tabular}} &
%   \multirow{4}{*}{\begin{tabular}[c]{@{}c@{}}Latency-aware, Hessian-\\ based importance-\\ based criteria\end{tabular}} &
%   \begin{tabular}[c]{@{}c@{}}NViT-B\\ $+$ ASP\end{tabular} &
%   DeiT-B &
%   \multicolumn{1}{c|}{\begin{tabular}[c]{@{}c@{}}6.8 \\ ($2.57\times$)\end{tabular}} &
%   \multicolumn{1}{c|}{\begin{tabular}[c]{@{}c@{}}17\\ (5.14$\times$)\end{tabular}} &
%   \begin{tabular}[c]{@{}c@{}}83.29\\ ($\downarrow$ 0.07\%)\end{tabular} \\ \cline{3-7} 
%  &
%    &
%   \begin{tabular}[c]{@{}c@{}}NViT-H\\ $+$ ASP\end{tabular} &
%   Swin-S &
%   \multicolumn{1}{c|}{\begin{tabular}[c]{@{}c@{}}6.2 \\ ($2.85\times$)\end{tabular}} &
%   \multicolumn{1}{c|}{\begin{tabular}[c]{@{}c@{}}15\\ (5.68$\times$)\end{tabular}} &
%   \begin{tabular}[c]{@{}c@{}}82.95\\ ($\downarrow$ 0.05\%)\end{tabular} \\ \cline{3-7} 
%  &
%    &
%   \begin{tabular}[c]{@{}c@{}}NViT-S\\ $+$ ASP\end{tabular} &
%   DeiT-S &
%   \multicolumn{1}{c|}{\begin{tabular}[c]{@{}c@{}}4.2 \\ (4.24$\times$)\end{tabular}} &
%   \multicolumn{1}{c|}{\begin{tabular}[c]{@{}c@{}}10.5 \\ (8.36$\times$)\end{tabular}} &
%   \begin{tabular}[c]{@{}c@{}}82.19\\ ($\uparrow$ 1.0\%)\end{tabular} \\ \cline{3-7} 
%  &
%    &
%   \begin{tabular}[c]{@{}c@{}}NViT-T \\ $+$ ASP\end{tabular} &
%   DeiT-T &
%   \multicolumn{1}{c|}{\begin{tabular}[c]{@{}c@{}}1.3 \\ (13.55$\times$)\end{tabular}} &
%   \multicolumn{1}{c|}{\begin{tabular}[c]{@{}c@{}}3.5\\ (24.94$\times$)\end{tabular}} &
%   \begin{tabular}[c]{@{}c@{}}76.21\\ ($\downarrow$ 1.71\%)\end{tabular} \\ \hline
% \multirow{3}{*}{\begin{tabular}[c]{@{}c@{}}Collaborative \\ pruning ~\cite{zheng2022savit}\end{tabular}} &
%   \multirow{3}{*}{\begin{tabular}[c]{@{}c@{}}Structural pruning on \\ MSA attention \& FFN by \\ removing unnecessary\\ parameter groups\end{tabular}} &
%   \multirow{3}{*}{SAViT} &
%   DeiT-B &
%   \multicolumn{1}{c|}{\begin{tabular}[c]{@{}c@{}}10.6 \\ ($\downarrow$39.8\%)\end{tabular}} &
%   \multicolumn{1}{c|}{\begin{tabular}[c]{@{}c@{}}51.9\\ ($\downarrow$40.1\%)\end{tabular}} &
%   \begin{tabular}[c]{@{}c@{}}82.75 \\ ($\uparrow$ 0.91\%)\end{tabular} \\ \cline{4-7} 
%  &
%    &
%    &
%   DeiT-S &
%   \multicolumn{1}{c|}{\begin{tabular}[c]{@{}c@{}}3.1 \\ ($\downarrow$31.7\%)\end{tabular}} &
%   \multicolumn{1}{c|}{\begin{tabular}[c]{@{}c@{}}14.7\\ ($\downarrow$33.5\%)\end{tabular}} &
%   \begin{tabular}[c]{@{}c@{}}80.11\\ ($\uparrow$ 0.26\%)\end{tabular} \\ \cline{4-7} 
%  &
%    &
%    &
%   DeiT-T &
%   \multicolumn{1}{c|}{\begin{tabular}[c]{@{}c@{}}0.9 \\ ($\downarrow$24.4\%)\end{tabular}} &
%   \multicolumn{1}{c|}{\begin{tabular}[c]{@{}c@{}}4.2\\ ($\downarrow$25.2\%)\end{tabular}} &
%   \begin{tabular}[c]{@{}c@{}}70.72\\ ($\downarrow$ 1.48\%)\end{tabular} \\ \hline
% \multirow{3}{*}{\begin{tabular}[c]{@{}c@{}}Structured sparse \\ pruning ~\cite{chen2021chasing}\end{tabular}} &
%   \multirow{3}{*}{\begin{tabular}[c]{@{}c@{}}Removing sub-modules \\ like self-attention \\ heads by manipulating \\ weight, activation,\& gradient\end{tabular}} &
%   S$^2$ViTE-B &
%   DeiT-B &
%   \multicolumn{1}{c|}{\begin{tabular}[c]{@{}c@{}}11.8 \\ ($\downarrow$33.13\%)\end{tabular}} &
%   \multicolumn{1}{c|}{\begin{tabular}[c]{@{}c@{}}56.8\\ ($\downarrow$34.4\%)\end{tabular}} &
%   \begin{tabular}[c]{@{}c@{}}82.22\\ ($\uparrow$ 0.38\%)\end{tabular} \\ \cline{3-7} 
%  &
%    &
%   S$^2$ViTE-S &
%   DeiT-S &
%   \multicolumn{1}{c|}{\begin{tabular}[c]{@{}c@{}}3.1\\ ($\downarrow$31.7\%)\end{tabular}} &
%   \multicolumn{1}{c|}{\begin{tabular}[c]{@{}c@{}}14.6\\ ($\downarrow$31.63\%)\end{tabular}} &
%   \begin{tabular}[c]{@{}c@{}}79.22 \\ ($\downarrow$ 0.63\%)\end{tabular} \\ \cline{3-7} 
%  &
%    &
%   S$^2$ViTE-T &
%   DeiT-T &
%   \multicolumn{1}{c|}{\begin{tabular}[c]{@{}c@{}}0.9 \\ ($\downarrow$23.69\%)\end{tabular}} &
%   \multicolumn{1}{c|}{\begin{tabular}[c]{@{}c@{}}4.2\\ ($\downarrow$26.3\%)\end{tabular}} &
%   \begin{tabular}[c]{@{}c@{}}70.12\\ ($\downarrow$ 2.08\%)\end{tabular} \\ \hline
% \multirow{3}{*}{\begin{tabular}[c]{@{}c@{}}Bottom-up \\ cascade \\ pruning~\cite{wang2022vtc}\end{tabular}} &
%   \multirow{3}{*}{\begin{tabular}[c]{@{}c@{}}Token pruning \& channel \\ pruning using a \\ hyperparameter \\ from one to last block\end{tabular}} &
%   \multirow{3}{*}{VTC-LFC} &
%   DeiT-B &
%   \multicolumn{1}{c|}{$\downarrow$54.4\%} &
%   \multicolumn{1}{c|}{\begin{tabular}[c]{@{}c@{}}56.8\\ ($\downarrow$ 34.25\%)\end{tabular}} &
%   \begin{tabular}[c]{@{}c@{}}81.6\\ ($\downarrow$ 0.20\%)\end{tabular} \\ \cline{4-7} 
%  &
%    &
%    &
%   DeiT-S &
%   \multicolumn{1}{c|}{$\downarrow$47.1\%} &
%   \multicolumn{1}{c|}{\begin{tabular}[c]{@{}c@{}}15.3\\ ($\downarrow$ 30.77\%)\end{tabular}} &
%   \begin{tabular}[c]{@{}c@{}}79.6\\ ($\downarrow$ 0.20\%)\end{tabular} \\ \cline{4-7} 
%  &
%    &
%    &
%   DeiT-T &
%   \multicolumn{1}{c|}{$\downarrow$41.7\%} &
%   \multicolumn{1}{c|}{\begin{tabular}[c]{@{}c@{}}4.2\\ ($\downarrow$ 26.32\%)\end{tabular}} &
%   \begin{tabular}[c]{@{}c@{}}71.0\\ ($\downarrow$ 1.20\%)\end{tabular} \\ \hline
% \multirow{6}{*}{\begin{tabular}[c]{@{}c@{}}Cascade ViT\\ pruning~\cite{song2022cp}\end{tabular}} &
%   \multirow{6}{*}{\begin{tabular}[c]{@{}c@{}}Utilizing the sparsity to prune \\ PH-regions in MSA \&\\ FFN progressively \\ \& dynamically\end{tabular}} &
%   \multirow{2}{*}{CP-ViT} &
%   ViT-B &
%   \multicolumn{1}{c|}{$\downarrow$46.34\%} &
%   \multicolumn{1}{c|}{-} &
%   \begin{tabular}[c]{@{}c@{}}76.75\\ ($\downarrow$ 1.16\%)\end{tabular} \\ \cline{4-7} 
%  &
%    &
%    &
%   DeiT-B &
%   \multicolumn{1}{c|}{$\downarrow$41.62\%} &
%   \multicolumn{1}{c|}{-} &
%   \begin{tabular}[c]{@{}c@{}}81.13\\ ($\downarrow$ 0.69\%)\end{tabular} \\ \cline{3-7} 
%  &
%    &
%   \multirow{2}{*}{CP-ViT} &
%   ViT-B &
%   \multicolumn{1}{c|}{$\downarrow$29.03\%} &
%   \multicolumn{1}{c|}{-} &
%   \begin{tabular}[c]{@{}c@{}}96.20 \\ ($\downarrow$ 1.93\%)\end{tabular} \\ \cline{4-7} 
%  &
%    &
%    &
%   DeiT-B &
%   \multicolumn{1}{c|}{$\downarrow$30.08\%} &
%   \multicolumn{1}{c|}{-} &
%   \begin{tabular}[c]{@{}c@{}}98.01\\ ($\downarrow$ 1.09\%)\end{tabular} \\ \cline{3-7} 
%  &
%    &
%   \multirow{2}{*}{CP-ViT} &
%   ViT-B &
%   \multicolumn{1}{c|}{$\downarrow$32.05\%} &
%   \multicolumn{1}{c|}{-} &
%   \begin{tabular}[c]{@{}c@{}}84.79 \\ ($\downarrow$ 2.34\%)\end{tabular} \\ \cline{4-7} 
%  &
%    &
%    &
%   DeiT-B &
%   \multicolumn{1}{c|}{$\downarrow$30.92\%} &
%   \multicolumn{1}{c|}{-} &
%   \begin{tabular}[c]{@{}c@{}}89.68\\ ($\downarrow$ 1.17\%)\end{tabular} \\ \hline
  
% \end{longtable}
% Please add the following required packages to your document preamble:
% \usepackage{multirow}
% \usepackage{graphicx}
% Please add the following required packages to your document preamble:
% \usepackage{multirow}
% \usepackage{graphicx}
\begin{table}[]
\centering
\caption{Results of different pruning techniques proposed for Vision transformers. '\textbf{$\downarrow$}' denotes reduction from the baseline and '$\uparrow$' denotes increase rate from the baseline models.}
\label{tab:pruning_result}
\resizebox{\columnwidth}{!}{%
\begin{tabular}{c|c|c|c|ccc}
\hline
\multirow{2}{*}{\textbf{Algorithm}} &
  \multirow{2}{*}{\textbf{Method}} &
  \multirow{2}{*}{\textbf{Models}} &
  \multirow{2}{*}{\textbf{Baseline}} &
  \multicolumn{3}{c}{\textbf{Results}} \\ \cline{5-7} 
 &
   &
   &
   &
  \multicolumn{1}{c|}{\textbf{GFlops}} &
  \multicolumn{1}{c|}{\textbf{Params(M)}} &
  \textbf{Top-1 (\%)} \\ \hline
\begin{tabular}[c]{@{}c@{}}Channels \\ pruning ~\cite{zhu2021vision}\end{tabular} &
  \begin{tabular}[c]{@{}c@{}}Learn dimension-wise\\ important score\end{tabular} &
  VTP &
  DeiT-B &
  \multicolumn{1}{c|}{\begin{tabular}[c]{@{}c@{}}10.0\\ ($\downarrow$ 45.3\%)\end{tabular}} &
  \multicolumn{1}{c|}{$\downarrow$ 47.3} &
  \begin{tabular}[c]{@{}c@{}}92.58\\ ($\downarrow$ 1.92\%)\end{tabular} \\ \cline{3-7} 
 &
   &
  VTP &
  DeiT-B &
  \multicolumn{1}{c|}{\begin{tabular}[c]{@{}c@{}}10.0\\ ($\downarrow$ 43.1\%)\end{tabular}} &
  \multicolumn{1}{c|}{$\downarrow$ 48.0} &
  $\downarrow$ 1.1\% \\ \hline
\multirow{4}{*}{\begin{tabular}[c]{@{}c@{}}Width \& Depth \\ Pruning ~\cite{yu2022width}\end{tabular}} &
  \multirow{4}{*}{\begin{tabular}[c]{@{}c@{}}Set of learnable pruning\\ -related parameters for \\ width pruning \& shallow \\ classifiers using intermediate \\ information of the transformer\\ blocks\end{tabular}} &
  \multirow{4}{*}{WDPruning} &
  DeiT-T &
  \multicolumn{1}{c|}{\begin{tabular}[c]{@{}c@{}}2.6\\ ($\downarrow$ 43.5\%)\end{tabular}} &
  \multicolumn{1}{c|}{\begin{tabular}[c]{@{}c@{}}13.3 \\ ($\downarrow$ 37.6\%)\end{tabular}} &
  \begin{tabular}[c]{@{}c@{}}70.34\\ ($\downarrow$ 1.86\%)\end{tabular} \\ \cline{4-7} 
 &
   &
   &
  DeiT-S &
  \multicolumn{1}{c|}{\begin{tabular}[c]{@{}c@{}}0.7\\ ($\downarrow$ 46.2\%)\end{tabular}} &
  \multicolumn{1}{c|}{\begin{tabular}[c]{@{}c@{}}3.5\\ ($\downarrow$ 35.2\%)\end{tabular}} &
  \begin{tabular}[c]{@{}c@{}}78.38\\ ($\downarrow$ 1.42\%)\end{tabular} \\ \cline{4-7} 
 &
   &
   &
  DeiT-B &
  \multicolumn{1}{c|}{\begin{tabular}[c]{@{}c@{}}9.90\\ ($\downarrow$ 43.4\%)\end{tabular}} &
  \multicolumn{1}{c|}{\begin{tabular}[c]{@{}c@{}}55.3 \\ ($\downarrow$ 35.0\%)\end{tabular}} &
  \begin{tabular}[c]{@{}c@{}}80.76\\ ($\downarrow$ 1.04\%)\end{tabular} \\ \cline{4-7} 
 &
   &
   &
  Swin-S &
  \multicolumn{1}{c|}{\begin{tabular}[c]{@{}c@{}}6.3\\ ($\downarrow$ 27.6\%)\end{tabular}} &
  \multicolumn{1}{c|}{\begin{tabular}[c]{@{}c@{}}32.8\\ ($\downarrow$ 30.6\%)\end{tabular}} &
  \begin{tabular}[c]{@{}c@{}}81.80\\ ($\downarrow$ 1.20\%)\end{tabular} \\ \hline
\multirow{3}{*}{\begin{tabular}[c]{@{}c@{}}Multi-\\ dimensional\\ pruning ~\cite{hou2022multi}\end{tabular}} &
  \multirow{3}{*}{\begin{tabular}[c]{@{}c@{}}Dependency based pruning \\ criterion \& an efficient \\ Gaussian process search\end{tabular}} &
  \multirow{3}{*}{\begin{tabular}[c]{@{}c@{}}Multi-\\ dimensional\end{tabular}} &
  DeiT-S &
  \multicolumn{1}{c|}{\begin{tabular}[c]{@{}c@{}}2.9\\ ($\downarrow$37\%)\end{tabular}} &
  \multicolumn{1}{c|}{-} &
  \begin{tabular}[c]{@{}c@{}}79.9\\ ($\downarrow$0.1\%)\end{tabular} \\ \cline{4-7} 
 &
   &
   &
  DeiT-B &
  \multicolumn{1}{c|}{\begin{tabular}[c]{@{}c@{}}11.2\\ ($\downarrow$36\%)\end{tabular}} &
  \multicolumn{1}{c|}{-} &
  \begin{tabular}[c]{@{}c@{}}82.3\\ ($\downarrow$0.5\%)\end{tabular} \\ \cline{4-7} 
 &
   &
   &
  \begin{tabular}[c]{@{}c@{}}T2T-\\ ViT-14\end{tabular} &
  \multicolumn{1}{c|}{\begin{tabular}[c]{@{}c@{}}2.9\\ ($\downarrow$40\%)\end{tabular}} &
  \multicolumn{1}{c|}{-} &
  \begin{tabular}[c]{@{}c@{}}81.7\\ ($\downarrow$0.2\%)\end{tabular} \\ \hline
\multirow{4}{*}{\begin{tabular}[c]{@{}c@{}}Pruning the \\ network model ~\cite{kong2022spvit}\end{tabular}} &
  \multirow{4}{*}{\begin{tabular}[c]{@{}c@{}}Single-path ViT\\ pruning based on the \\ token score\end{tabular}} &
  \multirow{4}{*}{SPViT} &
  Swin-S &
  \multicolumn{1}{c|}{\begin{tabular}[c]{@{}c@{}}6.35 \\ ($\downarrow$ 26.4\%)\end{tabular}} &
  \multicolumn{1}{c|}{-} &
  \begin{tabular}[c]{@{}c@{}}82.71\\ ($\downarrow$ 0.49\%)\end{tabular} \\ \cline{4-7} 
 &
   &
   &
  Swin-T &
  \multicolumn{1}{c|}{\begin{tabular}[c]{@{}c@{}}3.47 \\ ($\downarrow$ 23.0\%)\end{tabular}} &
  \multicolumn{1}{c|}{-} &
  \begin{tabular}[c]{@{}c@{}}80.70 \\ ($\downarrow$ 0.50\%)\end{tabular} \\ \cline{4-7} 
 &
   &
   &
  PiT-S &
  \multicolumn{1}{c|}{\begin{tabular}[c]{@{}c@{}}2.22 \\ ($\downarrow$ 23.3\%)\end{tabular}} &
  \multicolumn{1}{c|}{-} &
  \begin{tabular}[c]{@{}c@{}}80.38 \\ ($\downarrow$ 0.58\%)\end{tabular} \\ \cline{4-7} 
 &
   &
   &
  PiT-XS &
  \multicolumn{1}{c|}{\begin{tabular}[c]{@{}c@{}}1.13 \\ ($\downarrow$ 18.7\%)\end{tabular}} &
  \multicolumn{1}{c|}{-} &
  \begin{tabular}[c]{@{}c@{}}77.86 \\ ($\downarrow$ 0.24\%)\end{tabular} \\ \hline
\multirow{4}{*}{\begin{tabular}[c]{@{}c@{}}Patch \\ pruning ~\cite{tang2022patch}\end{tabular}} &
  \multirow{4}{*}{\begin{tabular}[c]{@{}c@{}}Layer-by-layer \\ top down pruning\end{tabular}} &
  \multirow{4}{*}{PS-ViT} &
  DeiT-T &
  \multicolumn{1}{c|}{\begin{tabular}[c]{@{}c@{}}0.7 \\ ($\downarrow$ 46.2\%)\end{tabular}} &
  \multicolumn{1}{c|}{-} &
  \begin{tabular}[c]{@{}c@{}}72.0 \\ ($\downarrow$ 0.20\%)\end{tabular} \\ \cline{4-7} 
 &
   &
   &
  DeiT-S &
  \multicolumn{1}{c|}{\begin{tabular}[c]{@{}c@{}}2.6 \\ ($\downarrow$ 43.6\%)\end{tabular}} &
  \multicolumn{1}{c|}{-} &
  \begin{tabular}[c]{@{}c@{}}79.4 \\ ($\downarrow$ 0.40\%)\end{tabular} \\ \cline{4-7} 
 &
   &
   &
  DeiT-B &
  \multicolumn{1}{c|}{\begin{tabular}[c]{@{}c@{}}9.8 \\ ($\downarrow$ 44.3\%)\end{tabular}} &
  \multicolumn{1}{c|}{-} &
  \begin{tabular}[c]{@{}c@{}}81.5\\ ($\downarrow$ 0.30\%)\end{tabular} \\ \cline{4-7} 
 &
   &
   &
  \begin{tabular}[c]{@{}c@{}}T2T-\\ ViT-14\end{tabular} &
  \multicolumn{1}{c|}{\begin{tabular}[c]{@{}c@{}}3.1 \\ ($\downarrow$ 40.4\%)\end{tabular}} &
  \multicolumn{1}{c|}{-} &
  \begin{tabular}[c]{@{}c@{}}81.1\\ ($\downarrow$ 0.40\%)\end{tabular} \\ \hline
\multirow{4}{*}{\begin{tabular}[c]{@{}c@{}}Structural \\ pruning ~\cite{yu2022unified}\end{tabular}} &
  \multirow{4}{*}{\begin{tabular}[c]{@{}c@{}}Prune the head \\ number \& head\\ dimensions inside \\ each layer\end{tabular}} &
  \multirow{4}{*}{UVC} &
  DeiT-T &
  \multicolumn{1}{c|}{\begin{tabular}[c]{@{}c@{}}0.51\\ (39.12\%)\end{tabular}} &
  \multicolumn{1}{c|}{-} &
  \begin{tabular}[c]{@{}c@{}}70.6\\ ($\downarrow$ 1.6\%)\end{tabular} \\ \cline{4-7} 
 &
   &
   &
  DeiT-S &
  \multicolumn{1}{c|}{\begin{tabular}[c]{@{}c@{}}2.32\\ (50.41\%)\end{tabular}} &
  \multicolumn{1}{c|}{-} &
  \begin{tabular}[c]{@{}c@{}}78.82\\ ($\downarrow$ 0.98\%)\end{tabular} \\ \cline{4-7} 
 &
   &
   &
  DeiT-B &
  \multicolumn{1}{c|}{\begin{tabular}[c]{@{}c@{}}8.0\\ (45.50\%)\end{tabular}} &
  \multicolumn{1}{c|}{-} &
  \begin{tabular}[c]{@{}c@{}}80.57 \\ ($\downarrow$ 1.23\%)\end{tabular} \\ \cline{4-7} 
 &
   &
   &
  \begin{tabular}[c]{@{}c@{}}T2T-\\ ViT-14\end{tabular} &
  \multicolumn{1}{c|}{\begin{tabular}[c]{@{}c@{}}2.11 \\ ($\downarrow$ 44.0\%)\end{tabular}} &
  \multicolumn{1}{c|}{-} &
  \begin{tabular}[c]{@{}c@{}}78.9\\ ($\downarrow$ 2.6\%)\end{tabular} \\ \hline
\multirow{4}{*}{\begin{tabular}[c]{@{}c@{}}Global structural \\ pruning ~\cite{yang2023global}\end{tabular}} &
  \multirow{4}{*}{\begin{tabular}[c]{@{}c@{}}Latency-aware, Hessian-\\ based importance-\\ based criteria\end{tabular}} &
  \begin{tabular}[c]{@{}c@{}}NViT-B\\ $+$ ASP\end{tabular} &
  DeiT-B &
  \multicolumn{1}{c|}{\begin{tabular}[c]{@{}c@{}}6.8 \\ ($2.57\times$)\end{tabular}} &
  \multicolumn{1}{c|}{\begin{tabular}[c]{@{}c@{}}17\\ (5.14$\times$)\end{tabular}} &
  \begin{tabular}[c]{@{}c@{}}83.29\\ ($\downarrow$ 0.07\%)\end{tabular} \\ \cline{3-7} 
 &
   &
  \begin{tabular}[c]{@{}c@{}}NViT-H\\ $+$ ASP\end{tabular} &
  Swin-S &
  \multicolumn{1}{c|}{\begin{tabular}[c]{@{}c@{}}6.2 \\ ($2.85\times$)\end{tabular}} &
  \multicolumn{1}{c|}{\begin{tabular}[c]{@{}c@{}}15\\ (5.68$\times$)\end{tabular}} &
  \begin{tabular}[c]{@{}c@{}}82.95\\ ($\downarrow$ 0.05\%)\end{tabular} \\ \cline{3-7} 
 &
   &
  \begin{tabular}[c]{@{}c@{}}NViT-S\\ $+$ ASP\end{tabular} &
  DeiT-S &
  \multicolumn{1}{c|}{\begin{tabular}[c]{@{}c@{}}4.2 \\ (4.24$\times$)\end{tabular}} &
  \multicolumn{1}{c|}{\begin{tabular}[c]{@{}c@{}}10.5 \\ (8.36$\times$)\end{tabular}} &
  \begin{tabular}[c]{@{}c@{}}82.19\\ ($\uparrow$ 1.0\%)\end{tabular} \\ \cline{3-7} 
 &
   &
  \begin{tabular}[c]{@{}c@{}}NViT-T \\ $+$ ASP\end{tabular} &
  DeiT-T &
  \multicolumn{1}{c|}{\begin{tabular}[c]{@{}c@{}}1.3 \\ (13.55$\times$)\end{tabular}} &
  \multicolumn{1}{c|}{\begin{tabular}[c]{@{}c@{}}3.5\\ (24.94$\times$)\end{tabular}} &
  \begin{tabular}[c]{@{}c@{}}76.21\\ ($\downarrow$ 1.71\%)\end{tabular} \\ \hline
\multirow{3}{*}{\begin{tabular}[c]{@{}c@{}}Collaborative \\ pruning ~\cite{zheng2022savit}\end{tabular}} &
  \multirow{3}{*}{\begin{tabular}[c]{@{}c@{}}Structural pruning on \\ MSA attention \& FFN by \\ removing unnecessary\\ parameter groups\end{tabular}} &
  \multirow{3}{*}{SAViT} &
  DeiT-B &
  \multicolumn{1}{c|}{\begin{tabular}[c]{@{}c@{}}10.6 \\ ($\downarrow$39.8\%)\end{tabular}} &
  \multicolumn{1}{c|}{\begin{tabular}[c]{@{}c@{}}51.9\\ ($\downarrow$40.1\%)\end{tabular}} &
  \begin{tabular}[c]{@{}c@{}}82.75 \\ ($\uparrow$ 0.91\%)\end{tabular} \\ \cline{4-7} 
 &
   &
   &
  DeiT-S &
  \multicolumn{1}{c|}{\begin{tabular}[c]{@{}c@{}}3.1 \\ ($\downarrow$31.7\%)\end{tabular}} &
  \multicolumn{1}{c|}{\begin{tabular}[c]{@{}c@{}}14.7\\ ($\downarrow$33.5\%)\end{tabular}} &
  \begin{tabular}[c]{@{}c@{}}80.11\\ ($\uparrow$ 0.26\%)\end{tabular} \\ \cline{4-7} 
 &
   &
   &
  DeiT-T &
  \multicolumn{1}{c|}{\begin{tabular}[c]{@{}c@{}}0.9 \\ ($\downarrow$24.4\%)\end{tabular}} &
  \multicolumn{1}{c|}{\begin{tabular}[c]{@{}c@{}}4.2\\ ($\downarrow$25.2\%)\end{tabular}} &
  \begin{tabular}[c]{@{}c@{}}70.72\\ ($\downarrow$ 1.48\%)\end{tabular} \\ \hline
\multirow{3}{*}{\begin{tabular}[c]{@{}c@{}}Structured sparse \\ pruning ~\cite{chen2021chasing}\end{tabular}} &
  \multirow{3}{*}{\begin{tabular}[c]{@{}c@{}}Removing sub-modules \\ like self-attention \\ heads by manipulating \\ weight, activation,\& gradient\end{tabular}} &
  S$^2$ViTE-B &
  DeiT-B &
  \multicolumn{1}{c|}{\begin{tabular}[c]{@{}c@{}}11.8 \\ ($\downarrow$33.13\%)\end{tabular}} &
  \multicolumn{1}{c|}{\begin{tabular}[c]{@{}c@{}}56.8\\ ($\downarrow$34.4\%)\end{tabular}} &
  \begin{tabular}[c]{@{}c@{}}82.22\\ ($\uparrow$ 0.38\%)\end{tabular} \\ \cline{3-7} 
 &
   &
  S$^2$ViTE-S &
  DeiT-S &
  \multicolumn{1}{c|}{\begin{tabular}[c]{@{}c@{}}3.1\\ ($\downarrow$31.7\%)\end{tabular}} &
  \multicolumn{1}{c|}{\begin{tabular}[c]{@{}c@{}}14.6\\ ($\downarrow$31.63\%)\end{tabular}} &
  \begin{tabular}[c]{@{}c@{}}79.22 \\ ($\downarrow$ 0.63\%)\end{tabular} \\ \cline{3-7} 
 &
   &
  S$^2$ViTE-T &
  DeiT-T &
  \multicolumn{1}{c|}{\begin{tabular}[c]{@{}c@{}}0.9 \\ ($\downarrow$23.69\%)\end{tabular}} &
  \multicolumn{1}{c|}{\begin{tabular}[c]{@{}c@{}}4.2\\ ($\downarrow$26.3\%)\end{tabular}} &
  \begin{tabular}[c]{@{}c@{}}70.12\\ ($\downarrow$ 2.08\%)\end{tabular} \\ \hline
\multirow{3}{*}{\begin{tabular}[c]{@{}c@{}}Bottom-up \\ cascade \\ pruning~\cite{wang2022vtc}\end{tabular}} &
  \multirow{3}{*}{\begin{tabular}[c]{@{}c@{}}Token pruning \& channel \\ pruning using a \\ hyperparameter \\ from one to last block\end{tabular}} &
  \multirow{3}{*}{VTC-LFC} &
  DeiT-B &
  \multicolumn{1}{c|}{$\downarrow$54.4\%} &
  \multicolumn{1}{c|}{\begin{tabular}[c]{@{}c@{}}56.8\\ ($\downarrow$ 34.25\%)\end{tabular}} &
  \begin{tabular}[c]{@{}c@{}}81.6\\ ($\downarrow$ 0.20\%)\end{tabular} \\ \cline{4-7} 
 &
   &
   &
  DeiT-S &
  \multicolumn{1}{c|}{$\downarrow$47.1\%} &
  \multicolumn{1}{c|}{\begin{tabular}[c]{@{}c@{}}15.3\\ ($\downarrow$ 30.77\%)\end{tabular}} &
  \begin{tabular}[c]{@{}c@{}}79.6\\ ($\downarrow$ 0.20\%)\end{tabular} \\ \cline{4-7} 
 &
   &
   &
  DeiT-T &
  \multicolumn{1}{c|}{$\downarrow$41.7\%} &
  \multicolumn{1}{c|}{\begin{tabular}[c]{@{}c@{}}4.2\\ ($\downarrow$ 26.32\%)\end{tabular}} &
  \begin{tabular}[c]{@{}c@{}}71.0\\ ($\downarrow$ 1.20\%)\end{tabular} \\ \hline
\multirow{6}{*}{\begin{tabular}[c]{@{}c@{}}Cascade ViT\\ pruning~\cite{song2022cp}\end{tabular}} &
  \multirow{6}{*}{\begin{tabular}[c]{@{}c@{}}Utilizing the sparsity to prune \\ PH-regions in MSA \&\\ FFN progressively \\ \& dynamically\end{tabular}} &
  \multirow{2}{*}{CP-ViT} &
  ViT-B &
  \multicolumn{1}{c|}{$\downarrow$46.34\%} &
  \multicolumn{1}{c|}{-} &
  \begin{tabular}[c]{@{}c@{}}76.75\\ ($\downarrow$ 1.16\%)\end{tabular} \\ \cline{4-7} 
 &
   &
   &
  DeiT-B &
  \multicolumn{1}{c|}{$\downarrow$41.62\%} &
  \multicolumn{1}{c|}{-} &
  \begin{tabular}[c]{@{}c@{}}81.13\\ ($\downarrow$ 0.69\%)\end{tabular} \\ \cline{3-7} 
 &
   &
  \multirow{2}{*}{CP-ViT} &
  ViT-B &
  \multicolumn{1}{c|}{$\downarrow$29.03\%} &
  \multicolumn{1}{c|}{-} &
  \begin{tabular}[c]{@{}c@{}}96.20 \\ ($\downarrow$ 1.93\%)\end{tabular} \\ \cline{4-7} 
 &
   &
   &
  DeiT-B &
  \multicolumn{1}{c|}{$\downarrow$30.08\%} &
  \multicolumn{1}{c|}{-} &
  \begin{tabular}[c]{@{}c@{}}98.01\\ ($\downarrow$ 1.09\%)\end{tabular} \\ \cline{3-7} 
 &
   &
  \multirow{2}{*}{CP-ViT} &
  ViT-B &
  \multicolumn{1}{c|}{$\downarrow$32.05\%} &
  \multicolumn{1}{c|}{-} &
  \begin{tabular}[c]{@{}c@{}}84.79 \\ ($\downarrow$ 2.34\%)\end{tabular} \\ \cline{4-7} 
 &
   &
   &
  DeiT-B &
  \multicolumn{1}{c|}{$\downarrow$30.92\%} &
  \multicolumn{1}{c|}{-} &
  \begin{tabular}[c]{@{}c@{}}89.68\\ ($\downarrow$ 1.17\%)\end{tabular} \\ \hline
\end{tabular}%
}
\end{table}
% ~\cite{gou2021knowledge}
% \begin{figure}[H]
%   \centering
%   \includegraphics[scale=0.5]{assets/knowledge_dis.png}
%   \caption{Overview of Knowledge distillation}
%   \label{fig:knowledge}
% \end{figure} 
% Figure \ref{fig:knowledge} is the general overview of applying knowledge distillation, enabling the student model to achieve comparable performance to the teacher model while being more computationally efficient.
Pruning is a direct way to reduce the complexity of the original model, whereas KD involves training a new, more compact model that is easy to deploy. By using KD during finetuning, the pruned model can benefit from the insights and information captured by the larger teacher model. KD helps to regain even surpass the original performance and compensate for the accuracy loss during the other compression techniques (e.g., pruning, quantization). 
\subsubsection{KD techniques for vision transformer}\hfill\\
Touvron et al.~\cite{touvron2021training} leveraged KD to train the transformer with a significantly smaller dataset than the traditionally required dataset. The authors introduced a distillation token, an additional learnable vector used alongside the class token during training. The proposed method achieved 84.5\% top-1 accuracy on the ImageNet-1k~\cite{dosovitskiy2020image} dataset, requiring fewer training data resources and computing power than ViT without KD.

Another study by Hao et al. ~\cite{hao2022learning} utilized every patch information to introduce a fine-grained manifold distillation method. In the manifold distillation method, the authors considered ViT as a feature projector that sets image patches into a sequence of manifold space layer by layer. The authors then teach the student layers to generate output features having the same patch-level manifold structure as the teacher layer for manually selected teacher-student layers. These output features are normalized and reshaped to compute a manifold relation map, a representation of the manifold structure of the features. However, the manifold relation map computation is resource-consuming and needs to simplify the computation. To solve the computational issue, the authors decoupled the manifold relation map into an intra-image relation map, an inter-image relation map, and a randomly sampled relation map. The authors utilized manifold distillation loss (MD Loss), the sum of individual loss from all three decoupled manifold relation maps.
% \renewcommand{\arraystretch}{0.75}
% \begin{table}[H]
% \caption{Results of different KD (object detection) techniques proposed for Vision transformers.}
% \label{tab:KD_object}
% \begin{adjustbox}{width=1\textwidth}
% \centering

% \begin{tabular}{c|c|c|c|c|c|c}
% \hline
% \textbf{Algorithm}  &\textbf{Pretrained dataset} & \textbf{Dataset}  %& \multicolumn{2}{c|}{\textbf{Teachers}}
% & \multicolumn{4}{c}{\textbf{Students}} \\\cline{4-7}
% & & & \multicolumn{2}{c}{\textbf{Without KD}}  &\multicolumn{2}{c}{\textbf{With KD}}\\\cline{4-7}
% & & & \textbf{Models} & \textbf{AP\textsuperscript{box}}   & \textbf{Models} & \textbf{AP\textsuperscript{box}}\\
% \hline
%  Fine-grained manifold ~\cite{hao2022learning} & ImageNet-1k~\cite{5206848} & COCO-2017  & Swin-T &43.7 &  Swin-T &44.7($\uparrow$1.0) \\
%  \hline
%  MiniViT ~\cite{zhang2022minivit} & - & COCO-2017  & Swin-T &48.1 &  Swin-T &48.6($\uparrow$0.5) \\
% \hline
% \end{tabular}
% \end{adjustbox}
% \vspace{-4mm}
% \end{table}
% Please add the following required packages to your document preamble:
% \usepackage{multirow}
% \usepackage{graphicx}
\begin{table}[htb]
\centering
\caption{Results of different KD (object detection) techniques proposed for vision transformers.}
\label{tab:KD_object}
\resizebox{0.9\columnwidth}{!}{%
\begin{tabular}{c|c|c|cccc}
\hline
\multirow{3}{*}{\textbf{Algorithm}} &
  \multirow{3}{*}{\textbf{Pretrained Dataset}} &
  \multirow{3}{*}{\textbf{Dataset}} &
  \multicolumn{4}{c}{\textbf{Students}} \\ \cline{4-7} 
 &
   &
   &
  \multicolumn{2}{c|}{\textbf{Without KD}} &
  \multicolumn{2}{c}{\textbf{With KD}} \\ \cline{4-7} 
 &
   &
   &
  \multicolumn{1}{c|}{\textbf{Models}} &
  \multicolumn{1}{c|}{\textbf{AP\textsuperscript{box}}} &
  \multicolumn{1}{c|}{\textbf{Models}} &
  \textbf{AP\textsuperscript{box}} \\ \hline
Fine-grained manifold ~\cite{hao2022learning} &
  ImageNet-1k~\cite{5206848} &
  COCO-2017 &
  \multicolumn{1}{c|}{Swin-T} &
  \multicolumn{1}{c|}{43.7} &
  \multicolumn{1}{c|}{Swin-T} &
  44.7($\uparrow$1.0) \\ \hline
MiniViT ~\cite{zhang2022minivit} &
  - &
  COCO-2017 &
  \multicolumn{1}{c|}{Swin-T} &
  \multicolumn{1}{c|}{48.1} &
  \multicolumn{1}{c|}{Swin-T} &
  48.6($\uparrow$0.5) \\ \hline
\end{tabular}%
}
\end{table}
Moreover, Lin et al. ~\cite{lin2022knowledge} proposed a one-to-all spatial matching KD technique. The proposed approach involves distilling the knowledge from each pixel of the teacher feature to all spatial locations of the student features based on their similarity. This similarity is determined using a target-aware transformer. By leveraging this target-aware transformer, the teacher's feature information can be effectively transferred and distilled to different spatial locations in the student's features.

Furthermore, Wang et al. ~\cite{9747484} proposed another KD method called attention probes. The main idea of this paper was to streamline ViTs through a two-step process using unlabelled data gathered from varied sources. The authors developed an 'attention probe' in the initial phase to discern and select significant data. The authors then used the selected data to instruct a compact student transformer by applying a probe-based KD algorithm. This algorithm was designed to optimize the resemblance between the resource-intensive teacher model and the more efficient student model, considering both the final outputs and intermediate features. The proposed method used cross-entropy (CE) and probe distillation functions for distilling intermediate features for calculating loss. Another study named DearKD ~\cite{chen2022dearkd} proposed the KD methods on self-generative data and used representational KD on intermediate features with response-based KD. The proposed paper used mean square error (MSE) distillation loss for hidden features, CE loss for hard label distillation, and intra-divergence distillation loss function to calculate the loss. It was noteworthy to see DearKD surpass the performance of the baseline transformer that trained with all ImageNet data, even though it only used 50\% of the data. The authors then evaluated the proposed technique on the ImageNet dataset and achieved 74.8\% top-1 accuracy in a tiny version, which is 2\% better than the DeiT-Tiny ~\cite{touvron2021training}. Moreover, TinyViT~\cite{wu2022tinyvit} highlights that smaller ViTs can benefit from larger teacher models trained on extensive datasets, such as distilling the student model on ImageNet-21k and finetuning on ImageNet-1k. To optimize computational memory, TinyViT introduced a strategy that pre-stores data augmentation details and logits for large teacher models, reducing memory overhead. Additionally, MiniViT~\cite{zhang2022minivit} argues by introducing a weight multiplexing strategy to reduce parameters across consecutive transformer blocks. Moreover, they employed weight distillation on self-attention mechanisms to transfer knowledge from large-scale ViT models to the smaller, weight-multiplexed MiniViT models. 

% Further research ~\cite{wu2022tinyvit,zhang2022minivit} explored kD as a model compression technique to improve the data efficiency required by transformers and showed better improvement than CNN-based models.
\renewcommand{\arraystretch}{0.75}
\begin{table}[]
\caption{Results of different KD (semantic segmentation) techniques proposed for vision transformers.}
\label{tab:semantic_segmentation}
\resizebox{0.9\columnwidth}{!}{%
\begin{tabular}{c|c|c|cc|cc|cc}
\hline
\multirow{2}{*}{\textbf{Algorithm}} &
  \multirow{2}{*}{\textbf{Dataset}} &
  \multirow{2}{*}{\textbf{Metrics}} &
  \multicolumn{2}{c|}{\textbf{Teachers}} &
  \multicolumn{2}{c|}{\textbf{Students}} &
  \multicolumn{2}{c}{\textbf{Proposed}} \\ \cline{4-9} 
 &
   &
   &
  \multicolumn{1}{c|}{\textbf{Models}} &
  \textbf{Result} &
  \multicolumn{1}{c|}{\textbf{Models}} &
  \textbf{Result} &
  \multicolumn{1}{c|}{\textbf{Models}} &
  \textbf{Result} \\ \hline
\begin{tabular}[c]{@{}c@{}}Fine-grained\\  manifold ~\cite{hao2022learning}\end{tabular} &
  ADE20K &
  mIoU &
  \multicolumn{1}{c|}{\begin{tabular}[c]{@{}c@{}}Swin-S + \\ UPerNet\end{tabular}} &
  47.64 &
  \multicolumn{1}{c|}{\begin{tabular}[c]{@{}c@{}}Swin-S + \\ UPerNet\end{tabular}} &
  44.51 &
  \multicolumn{1}{c|}{\begin{tabular}[c]{@{}c@{}}Swin-T + \\ UPerNet\end{tabular}} &
  \begin{tabular}[c]{@{}c@{}}45.66\\ ($\uparrow$2.58\%)\end{tabular} \\ \hline
\multirow{2}{*}{\begin{tabular}[c]{@{}c@{}}Target-aware\\ Transformer~\cite{lin2022knowledge}\end{tabular}} &
  \begin{tabular}[c]{@{}c@{}}COCO-\\ Stuff10k\end{tabular} &
  mIoU &
  \multicolumn{1}{c|}{ResNet18} &
  33.10 &
  \multicolumn{1}{c|}{ResNet18} &
  26.33 &
  \multicolumn{1}{c|}{ResNet18} &
  \begin{tabular}[c]{@{}c@{}}28.75\\ ($\uparrow$9.09\%)\end{tabular} \\ \cline{2-9} 
 &
  \begin{tabular}[c]{@{}c@{}}Pascal \\ VOC\end{tabular} &
  mIoU &
  \multicolumn{1}{c|}{ResNet18} &
  78.43 &
  \multicolumn{1}{c|}{ResNet18} &
  72.07 &
  \multicolumn{1}{c|}{ResNet18} &
  \begin{tabular}[c]{@{}c@{}}75.76\\ ($\uparrow$9.28\%)\end{tabular} \\ \hline
\end{tabular}%
}
\vspace{-3mm}
\end{table}


% Although KD in medical imaging tasks with ViT is quite an unexplored area, one of the recent studies by Park et al. ~\cite{park2022self} proposed distillation for self-supervision and self-train Learning (DISTL). DISTL was one of the first attempts of KD in medical images, inspired by the learning process of radiologists. DISTL enhanced the performance of ViT by combining self-supervision and self-training through knowledge distillation. In external validation across three hospitals for tuberculosis, pneumothorax, and COVID-19 diagnosis, DISTL exhibited progressive performance improvements with increasing amounts of unlabeled data. This work was one of the earliest to show that KD can be applied to medical imaging tasks.
\subsubsection{Discussion}\hfill\\
A key strength of ViT models lies in their scalability to high parametric complexity; however, this demands significant computational resources and incurs substantial costs. KD offers a way to transfer knowledge into more compact student models, yet challenges remain, particularly in the vision domain. One primary challenge involves the high training costs, as logits-based and hint-based KD methods necessitate extensive GPU memory for the distillation process. 

\noindent To show the results of the discussed techniques, we divide all the published results into three different Tables (\ref{tab:KD_result_classification}--\ref{tab:semantic_segmentation}). The results are divided based on popular CV tasks named image classification (Table~\ref{tab:KD_result_classification}), object detection (Table \ref{tab:KD_object}), and semantic segmentation (Table~\ref{tab:semantic_segmentation}). We observe that most of the papers are tested for image classification tasks, whereas a limited number of papers are evaluated for object detection and semantic segmentation. Table~\ref{tab:KD_result_classification} documents all the applied KD in image classification on the ViT backbone. As it is crucial to see the top-1 for classification problems, we summarize these accuracies for teacher and student models from the proposed papers. Moreover, it is essential to calculate the AP\textsuperscript{box} in object detection, and we document the AP\textsuperscript{box} results for both without applying KD and with KD for students model to compare the scenario in Table \ref{tab:KD_object} better. Lastly, Table~\ref{tab:semantic_segmentation} illustrates the semantic segmentation results for KD methods in ViT. All the current papers in Table~\ref{tab:semantic_segmentation} used mean intersection over union (mIoU) as a metric to calculate the accuracy. We separately document results with teacher models, student models, and proposed methods to better understand the improvement after applying KD.

\subsection{Quantization}\label{quan}
Quantization is used to reduce the bit-width of the data flowing through a NN model. So, it is used primarily for memory saving, faster inference times, and simplifying the operations for compute acceleration. That makes quantization essential for deploying NN on edge devices with limited computational capabilities. Quantization can be applied to different aspects of techniques. We organize our quantization discussion into two subsections. Firstly, we categorize different quantization techniques applied according to different aspects in Section~\ref{types_quanti}. Lastly, we discuss different quantization techniques in ViT in Section~\ref{vit_quanti}.

\subsubsection{Taxonomy of Quantization Methods} \label{types_quanti}
\hfill\\
Table~\ref{table:quan_classification} gives the overview of different quantization techniques and their pros and cons for computer vision tasks. Table~\ref{table:quan_classification} divides the quantization method from the aspects of quantization schemes, quantization approaches, calibration methods, granularity, and other independent techniques. Each aspect has multiple types of quantization techniques applied in different studies. In general, most quantization methods drop accuracy after applying quantization and need finetuning to regain the accuracy.\\
% For a detailed discussion about pruning types, please refer to Appendix C. 

\noindent \textbf{Quantization Schemes} Quantization schemes are broadly categorized into uniform and non-uniform techniques. Any weight or activation values in an NN can follow either a uniform or non-uniform distribution. Uniform quantization maps continuous weight and activation values to discrete levels with equal spacing between quantized values. However, non-uniform quantization sets different quantization steps for different parts of the data based on their distribution and importance to the final performance of the model. Non-uniform quantization can improve accuracy but is often more complex to implement in hardware. \\
 
\noindent \textbf{Quantization Approaches} Quantization approaches can be broadly classified based on whether they require retraining. Quantization-aware training (QAT) incorporates quantization into both forward and backward passes during training, allowing the model to adapt to lower-precision representations. However, QAT is resource-intensive due to the need for retraining. In contrast, Post-training quantization (PTQ) is a more efficient approach that applies quantization after a model has been fully trained in floating point (FP) precision, reducing the precision of weights and activations without additional training. While PTQ is less resource-intensive, it typically results in a higher accuracy drop compared to QAT~\cite{nagel2021white}.\\

\noindent \textbf{Calibration Methods} Calibration is a needed process of determining the appropriate scaling factors during finetuning methods that map the continuous range of FP values to discrete integer values. Calibration mainly ensures that the range of the quantized values matches the range of the original FP values as closely as possible. There are two types when choosing the range. One is dynamic quantization, and another is static quantization. The weights are quantized statically in dynamic quantization, but activations are quantized dynamically at runtime. Static quantization is quantized post-training. Unlike dynamic quantization, static quantization applies to weights and activations before deploying the model.\\

\noindent \textbf{Granularity} Another aspect of quantization techniques is the granularity of the clipping range of an NN. Layer-wise quantization is one of the granularity techniques where all weights and activations within a layer are quantized using the same scale. Channel-wise quantization, also referred to as per-channel quantization, is another granularity technique applied during the quantization of NN. Different scaling factors are computed for each channel of the weights in channel-wise quantization, meaning different layers can use different quantization parameters. \\

\noindent \textbf{Others} There are other techniques, such as mixed-precision quantization and hardware-aware quantization. Mixed-precision uses different parts (e.g., channels, layers) of the model that are quantized to different numerical precisions. Unlike uniform quantization, where the entire model is quantized to the same bit-width (like INT8), mixed-precision involves carefully choosing the bit-width for each layer or even each channel within a layer based on their sensitivity and contribution to the final performance of the model. Hardware-aware quantization is another technique that tailors a neural network's precision reduction process to the specific hardware to deploy on. This technique optimizes the model for the target hardware by adjusting the quantization parameters to match the hardware's operations~\cite{wang2019haq}, such as latency and throughput.
% Please add the following required packages to your document preamble:
% \usepackage[table,xcdraw]{xcolor}
% If you use beamer only pass "xcolor=table" option, i.e. \documentclass[xcolor=table]{beamer}
\begin{table}[]
\begin{tabular}{lllllll}
                                                                                                                     & Storyline Development Stage &                       & Character Design Stage &                        & Character Drawing Stage &                       \\
\multicolumn{7}{c}{\cellcolor[HTML]{D9D9D9}Needs \& Challenges}                                                                                                                                                                                                                \\
                                                                                                                     & Needs                       & Challenges            & Needs                  & Challenges             & Needs                   & Challenges            \\
gauge reader reactions                                                                                               & \multicolumn{1}{r}{3}       &                       &                        & \multicolumn{1}{r}{3}  &                         &                       \\
organize the story flow                                                                                              & \multicolumn{1}{r}{3}       & \multicolumn{1}{r}{4} &                        &                        &                         &                       \\
find interesting story sources that could captivate readers’ interest                                                & \multicolumn{1}{r}{2}       & \multicolumn{1}{r}{4} &                        &                        &                         &                       \\
struggled with creating a story                                                                                      &                             & \multicolumn{1}{r}{8} &                        &                        &                         &                       \\
maintaining a consistent storyline regardless of their condition                                                     &                             & \multicolumn{1}{r}{1} &                        &                        &                         &                       \\
get feedback from readers                                                                                            &                             &                       & \multicolumn{1}{r}{2}  &                        &                         &                       \\
get specific design ideas                                                                                            &                             &                       & \multicolumn{1}{r}{1}  &                        &                         &                       \\
uniqueness of their characters                                                                                       &                             &                       & \multicolumn{1}{r}{4}  & \multicolumn{1}{r}{10} &                         &                       \\
\begin{tabular}[c]{@{}l@{}}enhance the efficiency in their\\ work process by improving repetitive tasks\end{tabular} &                             &                       &                        &                        & \multicolumn{1}{r}{7}   &                       \\
receive materials related to composition                                                                             &                             &                       &                        &                        & \multicolumn{1}{r}{4}   &                       \\
provided to research reference materials                                                                             &                             &                       &                        &                        & \multicolumn{1}{r}{1}   &                       \\
struggled with drawing characters in a variety of compositions                                                       &                             &                       &                        &                        &                         & \multicolumn{1}{r}{6} \\
limited drawinig skills                                                                                              &                             &                       &                        &                        &                         & \multicolumn{1}{r}{6} \\
get information that could be used as reference material                                                             & \multicolumn{1}{r}{2}       &                       & \multicolumn{1}{r}{6}  & \multicolumn{1}{r}{2}  & \multicolumn{1}{r}{1}   &                       \\
\multicolumn{7}{c}{\cellcolor[HTML]{D9D9D9}Expectations}                                                                                                                                                                                                                       \\
Expectations for Generating Ideas                                                                                    & \multicolumn{1}{r}{9}       &                       & \multicolumn{1}{r}{11} &                        & \multicolumn{1}{r}{1}   &                       \\
Expectations for Receiving References from AI                                                                        & \multicolumn{1}{r}{4}       &                       & \multicolumn{1}{r}{0}  &                        & \multicolumn{1}{r}{5}   &                       \\
Expectations of rapid visualization using AI                                                                         & \multicolumn{1}{r}{0}       &                       & \multicolumn{1}{r}{2}  &                        & \multicolumn{1}{r}{2}   &                       \\
\multicolumn{7}{c}{\cellcolor[HTML]{D9D9D9}Considerations}                                                                                                                                                                                                                     \\
Skepticism Regarding AI Capabilities                                                                                 & \multicolumn{1}{r}{2}       &                       & \multicolumn{1}{r}{3}  &                        & \multicolumn{1}{r}{0}   &                       \\
Refusal to Use AI Due to Copyright Infringement and Concerns About the Author's Identity                             & \multicolumn{1}{r}{3}       &                       & \multicolumn{1}{r}{1}  &                        & \multicolumn{1}{r}{2}   &                       \\
Low Utility of AI Outputs                                                                                            & \multicolumn{1}{r}{3}       &                       & \multicolumn{1}{r}{1}  &                        & \multicolumn{1}{r}{1}   &                      
\end{tabular}
\end{table}
\subsubsection{Quantization Techniques for Vision Transformer} \label{vit_quanti}\hfill\\
% Applying quantization in ViT models is quite new in the neural network sectors. ViT consists of multiple layers, including a self-attention mechanism and a feed-forward neural network. As data passes through these layers, different layers learn to focus on different features of the input data. Applying quantization to ViT can be challenging due to its complexity. Additionally, the loss from quantization can significantly impact the self-attention mechanisms, potentially reducing the model's overall performance. Table \ref{quantization_tech_classification} \& \ref{tab:quantization_object} overview the quantization techniques for image classification and object detection in ViT models. From our observation, most of the quantization-related papers in ViT explored the PTQ rather than QAT. Table \ref{quantization_tech_classification} shows the top-1 accuracy of the proposed methods in image classification on ViT architectures and the improvement from their baseline method.\\
Applying quantization in ViT models is quite new in the neural network sectors. ViT consists of multiple layers, including a self-attention mechanism and a feed-forward neural network. As data passes through these layers, different layers learn to focus on different features of the input data. Applying quantization to ViT can be challenging due to its complexity. Additionally, the loss from quantization can significantly impact the self-attention mechanisms, potentially reducing the model's overall performance. Quantization methods on ViT can be broadly categorized into two main approaches based on their reliance on training or finetuning: PTQ and QAT.
\noindent \paragraph{PTQ Techniques for Vision Transformer}\hfill \\In the current scenario, PTQ is widely used for ViT because it offers an efficient way to meet the computational requirements on edge without additional training or finetuning, making it ideal for resource-constrained deployments. PTQ works can be divided into two categories~\cite{niu2023improving,zheng2022leveraging}: statistic-based PTQ and learning-based PTQ. Statistic-based PTQ methods focus on finding optimal quantization parameters to reduce quantization errors. In contrast, learning-based PTQ methods involve finetuning both the model weights and quantization parameters for improved performance~\cite{lv2024ptq4sam}. \\

\noindent \textbf{Statistic-Based PTQ methods} Most of the current PTQ works on ViT follow \textbf{statistic-based methods}. PTQ4ViT~\cite{yuan2022ptq4vit}, one of the first search-based PTQ methods, addressed two key issues with base-PTQ for ViTs: 1) unbalanced distributions after softmax and asymmetric distributions after GELU. 2) traditional metrics are ineffective for determining quantization parameters. To solve the first problem, the authors proposed twin uniform quantization, which quantizes values into two separate ranges. Additionally, to solve the second problem, they introduced a Hessian-guided metric for improved accuracy instead of mean square error (MSE) and cosine distance. 

Building on advancements in PTQ on ViT, Ding et al.~\cite{ding2022towards} proposed APQ-ViT solved two problems of the existing quantization techniques. The authors first proposed a unified bottom-elimination blockwise calibration scheme to solve the inaccurate measurement during quantization value calculation for extremely low-bit representation. This blockwise calibration scheme enables a more precise evaluation of quantization errors by focusing on block-level disturbances that impact the final output. For the second challenge, they observed the "matthew-effect" in the softmax distribution, where smaller values shrink further, and larger values dominate. However, the existing quantizers ignored the mattew-effect of the softmax function, which costs information loss from the larger values. In response, the authors proposed matthew-effect preserving quantization (MPQ) for Softmax to maintain the power-law character to solve the second limitation, ensuring balanced information retention during quantization. Additionally, Liu et al. proposed NoisyQuant~\cite{liu2023noisyquant}, where they focused on adding a noisy bias to each layer to modify the input activation distribution before quantization reduced the quantization error. The noisy bias is a single vector sampled from a uniform distribution. The authors removed the impact of noisy bias after the activation-weight multiplication in the linear layer with a denoising bias so that the method could retrieve the correct output. Surprisingly, the experiment showed that adding a noisy bias improved top-1 accuracy compared to the PTQ4ViT~\cite{yuan2022ptq4vit} on the ViT-B, DEiT-B, and Swin-S model.
\begin{figure}[htb]
  \centering
  \includegraphics[scale=0.25]   {assets/fq_vit_1.png}
  \caption{Comparison of using full precision Softmax and log-int-softmax in quantized MSA inference in FQ-vit ~\cite{lin2021fq}}
  \label{fig:quanwsoft}
\end{figure}

However, recent advancements in statistic-based PTQ for ViTs have moved beyond converting FP32 precision (dequantization) during inference, pioneering integer-only fully quantized methods~\cite{lin2021fq,li2022vit,li2023repq}. Moreover, Lin et al. first introduced fully quantized PTQ techniques named  FQ-ViT~\cite{lin2021fq} leveraging the power-of-two factor (PTF) method to minimize performance loss and inference complexity. To solve the non-uniform distribution in attention maps and avoid the dequantizing to FP32 before softmax, they proposed log-int-softmax (LIS) replacing softmax. Additionally, they streamline inference further using 4-bit quantization with the bit-shift operator. Figure~\ref{fig:quanwsoft} (left) illustrated the traditional approach (left) where the traditional approach dequantized INT8 query (Q) and key (K) matrices to FP before softmax, re-quantizing afterward for attention computations. In contrast, the proposed method (Figure~\ref{fig:quanwsoft} (right)) introduced matrix multiplication followed by integer-based exponential (i-exp). The authors then utilized $Log_{2}$ quantization scale in the softmax function and converted the MatMul to BitShift between the quantized attention map and values (V). This fully integer workflow, including LIS in INT4 format, significantly reduces memory usage while maintaining precision. Extended from FQ-ViT, Li et al. introduced I-ViT~\cite{li2022vit}, the first integer-only PTQ framework for ViT, enabling inference entirely with integer arithmetic and bit-shifting, eliminating FP operations. In this framework, the authors utilized an integer-only pipeline named dyadic anthemic for non-linear functions such as dense layers. In contrast, non-linear functions, including softmax, GELU, and LayerNorm, were approximated with lightweight integer-based methods. The key contribution of this work is that Shiftmax and ShiftGEU replicated the behavior of their FP counterparts using integer bit-shifting. Despite I-ViT's reduction in bit-precision for parameters and its emphasis on integer-only inference, it retained its accuracy. For example, when I-ViT applied to DeiT-B, it achieved 81.74\% top-1 accuracy with 8-bit integer-only inference, outperforming I-BERT~\cite{kim2021bert} by 0.95\% (see Table~\ref{quantization_tech_classification}).

However, the current studies consider quantizers and hardware standards always antagonistic, which is partially true. RepQ-ViT~\cite{li2023repq} decouples the quantization and inference process to explicitly bridged via scale reparameterization between these two steps. The authors applied channel-wise quantization for the post-LayerNorm activations to solve the interchannel variations and $\log \sqrt{2}$ quantization for the post-softmax activations. In the inference, the reparameterized the layer-wise quantization and $\log 2$ quantization with minimal computational cost for respective activations. Using integer-only quantization for all layers lessened the computational cost dramatically and made them highly suitable for edge devices.

Additionally, recent studies have further advanced statistics-based PTQ for ViTs by incorporating \textbf{mixed-precision techniques}. Liu et al. ~\cite{liu2021post} first explored a mixed precision PTQ scheme for ViT architectures to reduce the memory and computational requirements. The authors estimated optimal low-bit quantization intervals for weights and inputs, used ranking loss to preserve self-attention order, and analyzed layer-wise quantization loss to study mixed precision using the L1-norm~\cite{wu2018l1} of attention maps and outputs. Using calibration datasets from CIFAR-10~\cite{cifar10}, ImageNet-1k~\cite{5206848}, and COCO2017~\cite{lin2015microsoft}, their method outperformed percentile-based techniques\cite{li2019fully} by 3.35\% on CIFAR-10 with ViT-B model. Recently. Tai et al.~\cite{tai2024mptq} and Ranjan et al.~\cite{ranjan2024lrp} both extended the mixed precision PTQ techniques on ViT. MPTQ-ViT~\cite{tai2024mptq} utilized the smoothQuant~\cite{xiao2023smoothquant} with bias term (SQ-b) to address the asymmetry in activations, reducing clamping loss and improving quantization performance. The authors proposed a search-based scaling factor ratio (OPT-m) to determine the quantization parameters. Later, they incorporate SQ-b and OPT-m to propose greedy mixed precision PTQ techniques for ViT by allocating layer-wise bit-width. Additionally, Ranjan et al.~\cite{ranjan2024lrp} proposed LRP-QViT~\cite{ranjan2024lrp}, an explainability-based approach by assessing each layer's contribution to the model's predictions, guiding the assignment of mixed-precision bit allocations based on layer importance. The authors also clipped the channel-wise quantization to eliminate the outliers from post-LayerNorm activations, mitigating severe inter-channel variations and enhancing quantization robustness. Zhong et al. proposed ERQ~\cite{zhong2024erq} to mitigate the error arising during quantization from weight and activation quantization separately. The authors introduced activation quantization error reduction to reduce the activation error, which is like a ridge regression problem. The authors also proposed weight quantization error reduction in an interactive approach by rounding directions of quantized weight coupled with a ridge regression solver.\\

\noindent \textbf{Learning-Based PTQ Methods} While most current PTQ methods for ViTs are statistic-based, there are only a few that utilize learning-based approaches. Existing PTQ methods for ViTs face challenges with inflexible quantization of post-softmax and post-GELU activations, which follow power-law-like distributions. To solve this problem, Wu et al. proposed Adalog~\cite{wu2025adalog}. The authors optimized the logarithmic base to better align with the power-law distribution of activations while ensuring hardware-friendly quantization. The authors applied their proposed methods to post-softmax and post-GELU activations through bias reparameterization. Additionally, a fast progressive combining search strategy is proposed to efficiently determine the optimal logarithm base, scaling factors, and zero points for uniform quantizers. Moreover, a recently proposed by Ramachandran et al. named CLAMP-ViT~\cite{ramachandran2025clamp} adopted a two-stage approach between data generation and model quantization. The authors introduced a patch-level constrastive learning scheme to generate meaningful data. The authors also leveraged contrastive learning in layer-wise evolutionary search for fixed and mixed-precision quantization to identify optimal quantization parameters. In conclusion, the learned-based PTQ techniques on ViT are largely explored for low-bit quantization.

% \noindent \textbf{PTQ techniques for vision transformer} PTQ techniques are popular because no training is required. Liu et al. ~\cite{liu2021post} first explored a mixed precision post-training quantization scheme for ViT architectures to reduce the memory and computational requirements. First, the authors estimated the optimal low-bit quantization intervals for weights and inputs. Then, they introduced a ranking loss to keep the relative order of the self-attention results after quantization. Lastly, the authors analyzed the relationship between the quantization loss of each layer and studied mixed-precision quantization using the L1-norm~\cite{wu2018l1} of each attention map and output feature. For evaluation, the authors randomly selected 100 images from CIFAR-10 and CIFAR-100 datasets~\cite{cifar10} and 1000 images from ImageNet~\cite{5206848} and COCO2017~\cite{lin2015microsoft} datasets from the training dataset as the calibration dataset. The proposed technique outperformed percentile-based methods ~\cite{li2019fully}  by 3.35\% and 2.07\% on the CIFAR-10 dataset for ViT-base and ViT-large, respectively. Yuan et al. extended the base-PTQ (see Table \ref{quantization_tech_classification}) for ViT and proposed a framework named  PTQ4ViT~\cite{yuan2022ptq4vit}. The authors identified two major problems of dropping accuracy using base-PTQ for ViT. The authors identified that the first problem was the difference between distribution values after softmax and GELU and Gaussian distribution. The authors also found that distribution values after softmax are unbalanced, and distribution values after GELU are asymmetric. To solve these problems, the authors proposed twin uniform quantization. 
% % So, it is challenging to quantify well with uniform quantization, and the authors proposed twin uniform quantization. 
% In the twin uniform quantization, the authors quantized values in two ranges. The second problem was the metric calculation of base-PTQ for ViT. Although base-PTQ used various metrics, including mean square error (MSE) and cosine distance between the layer outputs before and after quantization, the metrics generated inaccurate results while determining the quantization parameters. The authors proposed to use the Hessian-guided metric to determine the quantization parameters.

% Moreover, another recent study on quantization by Ding et al.~\cite{ding2022towards} identified two problems of the existing quantization techniques in ViT as follows: 1) Calibration metric is inaccurate for measuring quantization for extremely low-bit representation. 2) The existing quantization paradigm is unfavorable to the power-law distribution of softmax. Based on their findings, the authors proposed an accurate post-training quantization framework for Vision Transformer, named APQ-ViT ~\cite{ding2022towards}. To solve the first limitation, the authors introduced a unified bottom-elimination blockwise calibration scheme, which focused on optimizing the calibration metric to assess quantization disturbance on a blockwise basis. This blockwise calibration scheme allowed them to prioritize and address the crucial quantization errors that impact the final output. Before solving the second limitation, the authors observed that the probability distribution of the softmax function in ViT makes the smaller values even smaller and the larger values even larger; that is typically called the matthew-effect. However, the existing quantizers ignored the mattew-effect of the softmax function that costs information loss from the larger values. Then, the authors proposed matthew-effect preserving quantization (MPQ) for Softmax to maintain the power-law character to solve the second limitation. MPQ maintained the matthew-effect of Softmax output during the quantization process and did not purely pursue the maximization of mutual information before and after quantization. Another study called NoisyQuant~\cite{liu2023noisyquant} proposed a quantizer-agnostic enhancement for the post-quantization of ViT. The proposed method added a noisy bias to each model layer to actively modify the input activation distribution before quantization, aiming to reduce quantization error. The noisy bias is a single vector sampled from a uniform distribution. The authors removed the impact of noisy bias after the activation-weight multiplication in the linear layer with a denoising bias so that the method could retrieve the correct output. NoisyQuant largely improved the performance of ViT architecture with minimal computation overhead. The experiment on ViT architecture showed that the proposed technique improved top-1 accuracy compared to the PTQ4ViT~\cite{yuan2022ptq4vit} on ViT-B, DEiT-B, and Swin-S architecture.

% Some other proposed recent quantization methods, like fully differentiable quantization method Q-Vit ~\cite{li2022q} based on head-wise bit-width used switchable scale to resolve the convergence issue during joint training of quantization. The proposed method limited quantization to 3-bit without an accuracy drop. The authors additionally analyzed the quantization robustness of every architecture component of ViT and explained that MSA and GELU are the key components during quantization. Comprehensive testing on various ViT models, including DeiT and Swin Transformer, demonstrated the effectiveness of their proposed quantization technique. Notably, their approach surpassed the uniform quantization method by a margin of 1.5\% on DeiT-T.
% \vspace{2pt}

% Another recent study by Lin et al. proposed Fq-vit ~\cite{lin2021fq} where the authors used the power-of-two factor (PTF) method to reduce the performance degradation and inference complexity of fully quantized vision transformers. Moreover, upon noticing a highly non-uniform distribution in attention maps, the authors introduced Log-Int-Softmax (LIS) to maintain the non-uniform distribution. The authors further streamline inference using 4-bit quantization combined with the bit-shift operator. The distinctions between the conventional MSA and the proposed technique for softmax are illustrated in Figure \ref{fig:quanwsoft} where the left one is for MSA with full precision (traditional) and the right one is for MSA with LIS (proposed). Figure \ref{fig:quanwsoft} (left) illustrated the traditional approach where the multiplication of the query (Q) and key (K) matrices, both in INT8 format, must be dequantized to the full precision (floating-point or FP) before the Softmax operation. After the softmax, the data is quantized to INT8 for the succeeding attention-weighted matrix multiplication with the value (V) matrix. However, figure \ref{fig:quanwsoft} (right) performed the matrix multiplication followed by an integer-based exponential function (i-exp). The authors then utilized $Log_{2}$ quantization scale in the softmax function and converted
% the MatMul to BitShift between the quantized attention
% map and values (V). The proposed method operated only with integer data types, including LIS. The result of the LIS was in INT4 format, indicating a more aggressive quantization that reduced memory usage.

% Moreover, the other study introduced I-ViT~\cite{li2022vit}, a quantization approach tailored for ViTs that allowed the execution of the entire inference computational graph solely using integer arithmetic and bit-shifting, eliminating the need for floating-point operations. This proposed method was the first work on integer-only quantization for ViTs. In this framework, linear functions, such as matrix multiplication and dense layers, strictly adhere to an integer-only pipeline leveraging dyadic arithmetic. Conversely, non-linear functions like Softmax, GELU, and LayerNorm were emulated using the suggested lightweight integer-only arithmetic techniques. I-ViT incorporated the novel Shiftmax and ShiftGELU methods, strategically developed to mimic their floating-point counterparts using integer bit-shifting. The study presented the accuracy outcomes of I-ViT alongside various baseline methods across several benchmark models on the ImageNet dataset. Despite I-ViT's reduction in bit-precision for parameters and its emphasis on integer-only inference, it retained a competitive accuracy level.

% In some cases, it surpassed the floating-point baseline and underscored the efficacy and reliability of the proposed approximation techniques. For example, when I-ViT was applied on DeiT-B, it achieved a top-1 accuracy of 81.74\% using 8-bit integer-only inference, a result that is notably 0.95\% (see table \ref{quantization_tech_classification}) higher than that of I-BERT ~\cite{kim2021bert}.

% Most quantization methods were experimented with for image classification with limited object detection tasks. However, some proposed methods showed great mAp values in object detection. All the results are summarized for different ViT models shown in Table \ref{tab:quantization_object}.
\noindent \paragraph{QAT Techniques for Vision Transformer}\hfill \\ Compared to PTQ techniques, QAT methods for ViTs remain relatively underexplored. Existing QAT approaches can be broadly classified into two categories: leveraging KD to optimize the quantized model and standalone independent frameworks. \\
\renewcommand{\arraystretch}{0.7}
\begin{table}[]
\centering
\caption{Results of different post-training quantization(classification) techniques proposed for ViTs. \textbf{MP} denotes mixed precision; \textbf{W-bit} refers to weight bit-widths and \textbf{A-bit} refers to activation bit-widths. Here, Baseline refers to the closest comparable results for classification tasks.}
\label{quantization_tech_classification}
\resizebox{\columnwidth}{!}{%
\begin{tabular}{c|c|c|c|cccc}
\hline
\multirow{2}{*}{\textbf{Algorithm}} &
  \multirow{2}{*}{\textbf{Key point}} &
  \multirow{2}{*}{\textbf{Backbone}} &
  \multirow{2}{*}{\textbf{Dataset}} &
  \multicolumn{4}{c}{\textbf{Results}} \\ \cline{5-8} 
 &
   &
   &
   &
  \multicolumn{1}{c|}{\textbf{Baseline}} &
  \multicolumn{1}{c|}{\textbf{W-bit}} &
  \multicolumn{1}{c|}{\textbf{A-bit}} &
  \textbf{\begin{tabular}[c]{@{}c@{}}Top-1\\ accuracy\end{tabular}} \\ \hline
\multirow{7}{*}{\begin{tabular}[c]{@{}c@{}}PTQ\\ ~\cite{liu2021post}\end{tabular}} &
  \multirow{7}{*}{\begin{tabular}[c]{@{}c@{}}Similarity-aware quantization \\ for linear layers ranking -aware\\ quantization for self-attention layers \\ Mixed-precision quantization \\ to retain performance\end{tabular}} &
  \multirow{6}{*}{ViT-B} &
  \multirow{2}{*}{CIFAR-10} &
  \multicolumn{1}{c|}{\multirow{6}{*}{Percentile}} &
  \multicolumn{1}{c|}{6 MP} &
  \multicolumn{1}{c|}{6 MP} &
  96.83 ($+3.35$) \\ \cline{6-8} 
 &
   &
   &
   &
  \multicolumn{1}{c|}{} &
  \multicolumn{1}{c|}{8 MP} &
  \multicolumn{1}{c|}{8 MP} &
  97.79 ($+2.03$) \\ \cline{4-4} \cline{6-8} 
 &
   &
   &
  \multirow{2}{*}{CIFAR-100} &
  \multicolumn{1}{c|}{} &
  \multicolumn{1}{c|}{6 MP} &
  \multicolumn{1}{c|}{6 MP} &
  83.99 ($+3.15$) \\ \cline{6-8} 
 &
   &
   &
   &
  \multicolumn{1}{c|}{} &
  \multicolumn{1}{c|}{8 MP} &
  \multicolumn{1}{c|}{8 MP} &
  85.76 ($+2.48$) \\ \cline{4-4} \cline{6-8} 
 &
   &
   &
  \multirow{2}{*}{ImageNet-1k~\cite{5206848}} &
  \multicolumn{1}{c|}{} &
  \multicolumn{1}{c|}{6 MP} &
  \multicolumn{1}{c|}{6 MP} &
  75.26 ($+3.68$) \\ \cline{6-8} 
 &
   &
   &
   &
  \multicolumn{1}{c|}{} &
  \multicolumn{1}{c|}{8 MP} &
  \multicolumn{1}{c|}{8 MP} &
  76.98 ($+2.88$) \\ \cline{3-8} 
 &
   &
  DeiT-B &
  ImageNet-1k~\cite{5206848} &
  \multicolumn{1}{c|}{Bit-Split} &
  \multicolumn{1}{c|}{6 MP} &
  \multicolumn{1}{c|}{6 MP} &
  74.58 ($+0.54$) \\ \hline
\multirow{4}{*}{\begin{tabular}[c]{@{}c@{}}PTQ4ViT\\ ~\cite{yuan2022ptq4vit}\end{tabular}} &
  \multirow{4}{*}{\begin{tabular}[c]{@{}c@{}}Twin uniform method to reduce the quantization \\ error on activation values \& analyse Hessian guided\\ metric to determine the scaling factors of each layer\end{tabular}} &
  ViT-B &
  \multirow{4}{*}{ImageNet-1k~\cite{5206848}} &
  \multicolumn{1}{c|}{\multirow{4}{*}{Base-PTQ}} &
  \multicolumn{1}{c|}{8} &
  \multicolumn{1}{c|}{8} &
  85.82 ($+0.52$) \\ \cline{3-3} \cline{6-8} 
 &
   &
  DeiT-B &
   &
  \multicolumn{1}{c|}{} &
  \multicolumn{1}{c|}{8} &
  \multicolumn{1}{c|}{8} &
  82.97 ($+.64$) \\ \cline{3-3} \cline{6-8} 
 &
   &
  \multirow{2}{*}{Swin-B} &
   &
  \multicolumn{1}{c|}{} &
  \multicolumn{1}{c|}{\multirow{2}{*}{8}} &
  \multicolumn{1}{c|}{\multirow{2}{*}{8}} &
  \multirow{2}{*}{86.39 ($+0.23$)} \\
 &
   &
   &
   &
  \multicolumn{1}{c|}{} &
  \multicolumn{1}{c|}{} &
  \multicolumn{1}{c|}{} &
   \\ \hline
\multirow{6}{*}{\begin{tabular}[c]{@{}c@{}}APQ-ViT\\ ~\cite{ding2022towards}\end{tabular}} &
  \multirow{6}{*}{\begin{tabular}[c]{@{}c@{}}Solve for extremely low-bit representation; \\ BBC to apply quantization in a blockwise \\ manner to perceive the loss in adjacent layers .\& \\ Matthew-effect preserving quantization for the \\ softmax to maintain power-law distribution\end{tabular}} &
  \multirow{2}{*}{ViT-B} &
  \multirow{6}{*}{ImageNet-1k~\cite{5206848}} &
  \multicolumn{1}{c|}{\multirow{6}{*}{PTQ4ViT~\cite{yuan2022ptq4vit}}} &
  \multicolumn{1}{c|}{6} &
  \multicolumn{1}{c|}{6} &
  82.21 ($+0.56$) \\ \cline{6-8} 
 &
   &
   &
   &
  \multicolumn{1}{c|}{} &
  \multicolumn{1}{c|}{4} &
  \multicolumn{1}{c|}{4} &
  41.41 ($+10.72$) \\ \cline{3-3} \cline{6-8} 
 &
   &
  \multirow{2}{*}{Swin-B/384} &
   &
  \multicolumn{1}{c|}{} &
  \multicolumn{1}{c|}{6} &
  \multicolumn{1}{c|}{6} &
  85.60 ($+0.16$) \\ \cline{6-8} 
 &
   &
   &
   &
  \multicolumn{1}{c|}{} &
  \multicolumn{1}{c|}{4} &
  \multicolumn{1}{c|}{4} &
  80.84 ($+2.0$) \\ \cline{3-3} \cline{6-8} 
 &
   &
  \multirow{2}{*}{DeiT-B} &
   &
  \multicolumn{1}{c|}{} &
  \multicolumn{1}{c|}{6} &
  \multicolumn{1}{c|}{6} &
  80.42 ($+0.17$) \\ \cline{6-8} 
 &
   &
   &
   &
  \multicolumn{1}{c|}{} &
  \multicolumn{1}{c|}{4} &
  \multicolumn{1}{c|}{4} &
  67.48 ($+3.09$) \\ \hline
\multirow{6}{*}{\begin{tabular}[c]{@{}c@{}}NoisyQuant\\ ~\cite{liu2023noisyquant}\end{tabular}} &
  \multirow{6}{*}{\begin{tabular}[c]{@{}c@{}}A quantizer-agnostic enhancement for \\ the post-training activation quantization \\ of ViT \& adding a fixed Uniform noisy \\ bias to the values being quantized \\ for a given quantizer\end{tabular}} &
  \multirow{2}{*}{ViT-B} &
  \multirow{6}{*}{ImageNet-1k~\cite{5206848}} &
  \multicolumn{1}{c|}{\multirow{6}{*}{PTQ4ViT~\cite{yuan2022ptq4vit}}} &
  \multicolumn{1}{c|}{6} &
  \multicolumn{1}{c|}{6} &
  81.90 ($+6.24$) \\ \cline{6-8} 
 &
   &
   &
   &
  \multicolumn{1}{c|}{} &
  \multicolumn{1}{c|}{8} &
  \multicolumn{1}{c|}{8} &
  84.10 ($+0.71$) \\ \cline{3-3} \cline{6-8} 
 &
   &
  \multirow{2}{*}{DeiT-B} &
   &
  \multicolumn{1}{c|}{} &
  \multicolumn{1}{c|}{6} &
  \multicolumn{1}{c|}{6} &
  79.77 ($+.99$) \\ \cline{6-8} 
 &
   &
   &
   &
  \multicolumn{1}{c|}{} &
  \multicolumn{1}{c|}{8} &
  \multicolumn{1}{c|}{8} &
  81.30 ($+0.36$) \\ \cline{3-3} \cline{6-8} 
 &
   &
  \multirow{2}{*}{Swin-S} &
   &
  \multicolumn{1}{c|}{} &
  \multicolumn{1}{c|}{6} &
  \multicolumn{1}{c|}{6} &
  84.57 ($+1.22$) \\ \cline{6-8} 
 &
   &
   &
   &
  \multicolumn{1}{c|}{} &
  \multicolumn{1}{c|}{8} &
  \multicolumn{1}{c|}{8} &
  85.11 ($+0.32$) \\ \hline
\multirow{4}{*}{\begin{tabular}[c]{@{}c@{}}FQ-Vit\\ ~\cite{lin2021fq}\end{tabular}} &
  \multirow{4}{*}{\begin{tabular}[c]{@{}c@{}}Efficient PTQ method for achieving accurate \\ quantization on LayerNorm inputs with one layerwise \\ quantization scale named PTF. Propose LIS for \\ performing 4-bit quantization on attention maps\end{tabular}} &
  DeiT-T &
  \multirow{4}{*}{ImageNet-1k~\cite{5206848}} &
  \multicolumn{1}{c|}{\multirow{4}{*}{Percentile}} &
  \multicolumn{1}{c|}{8} &
  \multicolumn{1}{c|}{8} &
  71.61 ($+0.14$) \\ \cline{3-3} \cline{6-8} 
 &
   &
  DeiT-S &
   &
  \multicolumn{1}{c|}{} &
  \multicolumn{1}{c|}{8} &
  \multicolumn{1}{c|}{8} &
  79.17 ($+2.6$) \\ \cline{3-3} \cline{6-8} 
 &
   &
  DeiT-B &
   &
  \multicolumn{1}{c|}{} &
  \multicolumn{1}{c|}{8} &
  \multicolumn{1}{c|}{8} &
  81.20 ($+1.83$) \\ \cline{3-3} \cline{6-8} 
 &
   &
  Swin-B &
   &
  \multicolumn{1}{c|}{} &
  \multicolumn{1}{c|}{8} &
  \multicolumn{1}{c|}{8} &
  82.97 ($+42.04$) \\ \hline
\multirow{4}{*}{\begin{tabular}[c]{@{}c@{}}I-ViT\\ ~\cite{li2022vit}\end{tabular}} &
  \multirow{4}{*}{\begin{tabular}[c]{@{}c@{}}Performing the entire inference with \\ integer arithmetic \& bit-shifting. integer \\ approximations for non-linear operations\end{tabular}} &
  ViT-B &
  \multirow{4}{*}{ImageNet-1k~\cite{5206848}} &
  \multicolumn{1}{c|}{\multirow{4}{*}{I-BERT~\cite{kim2021bert}}} &
  \multicolumn{1}{c|}{-} &
  \multicolumn{1}{c|}{-} &
  84.76 ($+1.06$) \\ \cline{3-3} \cline{6-8} 
 &
   &
  DeiT-B &
   &
  \multicolumn{1}{c|}{} &
  \multicolumn{1}{c|}{-} &
  \multicolumn{1}{c|}{-} &
  81.74 ($+0.95$) \\ \cline{3-3} \cline{6-8} 
 &
   &
  \multirow{2}{*}{Swin-S} &
   &
  \multicolumn{1}{c|}{} &
  \multicolumn{1}{c|}{\multirow{2}{*}{-}} &
  \multicolumn{1}{c|}{\multirow{2}{*}{-}} &
  \multirow{2}{*}{83.01 ($+1.15$)} \\
 &
   &
   &
   &
  \multicolumn{1}{c|}{} &
  \multicolumn{1}{c|}{} &
  \multicolumn{1}{c|}{} &
   \\ \hline
\multirow{3}{*}{\begin{tabular}[c]{@{}c@{}}RepQ-ViT\\ ~\cite{li2023repq}\end{tabular}} &
  \multirow{3}{*}{\begin{tabular}[c]{@{}c@{}}Apply channel-wise quantization on  post-LayerNorm \\ activations \& $\log \sqrt{2}$ for post-softmax activations\end{tabular}} &
  ViT-B &
  \multirow{3}{*}{ImageNet-1k~\cite{5206848}} &
  \multicolumn{1}{c|}{\multirow{3}{*}{APQ-ViT~\cite{ding2022towards}}} &
  \multicolumn{1}{c|}{4} &
  \multicolumn{1}{c|}{4} &
  83.62 ($+1.41$) \\ \cline{3-3} \cline{6-8} 
 &
   &
  DeiT-B &
   &
  \multicolumn{1}{c|}{} &
  \multicolumn{1}{c|}{4} &
  \multicolumn{1}{c|}{4} &
  81.27 ($+0.85$) \\ \cline{3-3} \cline{6-8} 
 &
   &
  Swin-S &
   &
  \multicolumn{1}{c|}{} &
  \multicolumn{1}{c|}{4} &
  \multicolumn{1}{c|}{4} &
  82.79 ($+0.12$) \\ \hline
\multirow{2}{*}{\begin{tabular}[c]{@{}c@{}}MPTQ-ViT\\ ~\cite{tai2024mptq}\end{tabular}} &
  \multirow{2}{*}{\begin{tabular}[c]{@{}c@{}}Introduce SmoothQuant~\cite{xiao2023smoothquant} with bias term \\ to solve asymmetric issue \& minimize clamping loss\end{tabular}} &
  ViT-B &
  \multirow{2}{*}{ImageNet-1k~\cite{5206848}} &
  \multicolumn{1}{c|}{\multirow{2}{*}{TSPTQ-ViT~\cite{10096817}}} &
  \multicolumn{1}{c|}{6} &
  \multicolumn{1}{c|}{6} &
  82.70 ($+0.41$) \\ \cline{3-3} \cline{6-8} 
 &
   &
  DeiT-B &
   &
  \multicolumn{1}{c|}{} &
  \multicolumn{1}{c|}{6} &
  \multicolumn{1}{c|}{6} &
  81.25 ($+0.64$) \\ \hline
\multirow{3}{*}{\begin{tabular}[c]{@{}c@{}}LRP-QViT\\ ~\cite{ranjan2024lrp}\end{tabular}} &
  \multirow{3}{*}{\begin{tabular}[c]{@{}c@{}}Assigning precision bit for \\ individual layers based on layer's \\ importance\end{tabular}} &
  ViT-B &
  \multirow{3}{*}{ImageNet-1k~\cite{5206848}} &
  \multicolumn{1}{c|}{\multirow{3}{*}{RepQ-ViT~\cite{li2023repq}}} &
  \multicolumn{1}{c|}{6 MP} &
  \multicolumn{1}{c|}{6 MP} &
  83.87 ($+0.25$) \\ \cline{3-3} \cline{6-8} 
 &
   &
  DeiT-B &
   &
  \multicolumn{1}{c|}{} &
  \multicolumn{1}{c|}{6 MP} &
  \multicolumn{1}{c|}{6 MP} &
  81.44 ($+0.17$) \\ \cline{3-3} \cline{6-8} 
 &
   &
  Swin-S &
   &
  \multicolumn{1}{c|}{} &
  \multicolumn{1}{c|}{6 MP} &
  \multicolumn{1}{c|}{6 MP} &
  82.86 ($+0.07$) \\ \hline
\multirow{3}{*}{\begin{tabular}[c]{@{}c@{}}ERQ\\ ~\cite{zhong2024erq}\end{tabular}} &
  \multirow{3}{*}{\begin{tabular}[c]{@{}c@{}}Introduced weight quantization \\ error reduction metrics to minimize \\ the weight quantization error\end{tabular}} &
  ViT-B &
  \multirow{3}{*}{ImageNet-1k~\cite{5206848}} &
  \multicolumn{1}{c|}{\multirow{3}{*}{AdaRound~\cite{nagel2020up}}} &
  \multicolumn{1}{c|}{5} &
  \multicolumn{1}{c|}{5} &
  82.81 ($+0.81$) \\ \cline{3-3} \cline{6-8} 
 &
   &
  DeiT-B &
   &
  \multicolumn{1}{c|}{} &
  \multicolumn{1}{c|}{5} &
  \multicolumn{1}{c|}{5} &
  80.65 ($+0.47$) \\ \cline{3-3} \cline{6-8} 
 &
   &
  Swin-S &
   &
  \multicolumn{1}{c|}{} &
  \multicolumn{1}{c|}{5} &
  \multicolumn{1}{c|}{5} &
  82.44 ($+0.32$) \\ \hline
\multirow{3}{*}{\begin{tabular}[c]{@{}c@{}}Adalog\\ ~\cite{wu2025adalog}\end{tabular}} &
  \multirow{3}{*}{\begin{tabular}[c]{@{}c@{}}Proposed adaptive log based \\ non-uniform quantization for post-Softmax\\ \& post GELy activations\end{tabular}} &
  ViT-B &
  \multirow{3}{*}{ImageNet-1k~\cite{5206848}} &
  \multicolumn{1}{c|}{\multirow{3}{*}{RepQ-ViT~\cite{li2023repq}}} &
  \multicolumn{1}{c|}{6} &
  \multicolumn{1}{c|}{6} &
  84.80($+1.18$) \\ \cline{3-3} \cline{6-8} 
 &
   &
  DeiT-B &
   &
  \multicolumn{1}{c|}{} &
  \multicolumn{1}{c|}{6} &
  \multicolumn{1}{c|}{6} &
  81.55 ($+0.28$) \\ \cline{3-3} \cline{6-8} 
 &
   &
  Swin-S &
   &
  \multicolumn{1}{c|}{} &
  \multicolumn{1}{c|}{6} &
  \multicolumn{1}{c|}{6} &
  83.19 ($+0.40$) \\ \hline
\multirow{2}{*}{\begin{tabular}[c]{@{}c@{}}CLAMP-ViT\\ ~\cite{ramachandran2025clamp}\end{tabular}} &
  \multirow{2}{*}{\begin{tabular}[c]{@{}c@{}}Leverage contrastive learning layer-wise evolutionary \\ search for fixed and mixed-precision quantization\end{tabular}} &
  DeiT-S &
  \multirow{2}{*}{ImageNet-1k~\cite{5206848}} &
  \multicolumn{1}{c|}{\multirow{2}{*}{LRP-QViT~\cite{ranjan2024lrp}}} &
  \multicolumn{1}{c|}{6} &
  \multicolumn{1}{c|}{6} &
  79.43 ($+0.40$) \\ \cline{3-3} \cline{6-8} 
 &
   &
  Swin-S &
   &
  \multicolumn{1}{c|}{} &
  \multicolumn{1}{c|}{6} &
  \multicolumn{1}{c|}{6} &
  82.86 ($+0.00$) \\ \hline
\end{tabular}%
}
\vspace{-5mm}
\end{table}

\noindent \textbf{Leveraging KD in QAT} Q-Vit~\cite{li2022q}  first proposed an information rectification module based on information theory to resolve the convergence issue during joint training of quantization. The authors then proposed distributed guided distillation by taking appropriate activities and utilizing the knowledge from similar matrices in distillation to perform the optimization perfectly. However, Q-ViT lacks other CV tasks, such as object detection. Another recent work, Q-DETR~\cite{xu2023q}, is introduced to solve the information distortion problem. The authors explored the low-bits quantization of DETR and proposed a bi-level optimization framework based on the information bottleneck principle. However, Q-DETR failed to keep the attention activations less than 4 bits and resulted in mixed-precision quantization, which is hardware-inefficient in the current scenario. Both Q-ViT and Q-DETR explored the lightweight version of DETR apart from modifying the MHA. AQ-DETR~\cite{aqvit} focused on solving the problem that exists for low bits of DETR in previous studies. The authors introduced an auxiliary query module and layer-by-layer distillation module to reduce the quantization error between quantized attention and full-precision counterpart. All the previously discussed works are heavily dependent on the data. Li et al. ~\cite{li2023psaq} proposed PSAQ-ViT, aiming to achieve a data-free quantization framework by utilizing the property of KD. The authors introduced an adaptive teacher-student strategy enabling cyclic interaction between generated samples and the quantized model under the supervision of the full-precision model, significantly improving accuracy. The framework leverages task- and model-independent prior information, making it universal across various vision tasks such as classification and object detection.\\
\renewcommand{\arraystretch}{0.7}
\begin{table}[]
\small
\centering
\caption{Results of different quantization (object detection) techniques proposed for ViTs. The algorithms are experimented on COCO 2017~\cite{lin2014microsoft} datasets for object detection tasks. Here, Baseline refers to the closest comparable results for object detection. \textbf{MP} denotes
mixed precision.}
\label{tab:quantization_object}
\resizebox{0.9\columnwidth}{!}{%
\begin{tabular}{c|c|ccccc}
\hline
\multirow{2}{*}{\textbf{Algorithm}} &
  \multirow{2}{*}{\textbf{Backbone}} &
  \multicolumn{5}{c}{\textbf{Results}} \\ \cline{3-7} 
 &
   &
  \multicolumn{1}{c|}{\textbf{Baseline}} &
  \multicolumn{1}{c|}{\textbf{W-bit}} &
  \multicolumn{1}{c|}{\textbf{A-bit}} &
  \multicolumn{1}{c|}{\textbf{mAP}} &
  \textbf{AP\textsuperscript{box}} \\ \hline
\multirow{2}{*}{PTQ~\cite{liu2021post}} &
  \multirow{2}{*}{DETR} &
  \multicolumn{1}{c|}{\multirow{2}{*}{Easyquant~\cite{wu2020easyquant}}} &
  \multicolumn{1}{c|}{6 MP} &
  \multicolumn{1}{c|}{6 MP} &
  \multicolumn{1}{c|}{40.5($+1.5$)} &
  - \\
 &
   &
  \multicolumn{1}{c|}{} &
  \multicolumn{1}{c|}{8 MP} &
  \multicolumn{1}{c|}{8 MP} &
  \multicolumn{1}{c|}{41.7($+1.3$)} &
  - \\ \hline
\multirow{2}{*}{APQ-ViT~\cite{ding2022towards}} &
  \multirow{2}{*}{Mask-RCNN+Swin-T} &
  \multicolumn{1}{c|}{\multirow{2}{*}{PTQ4ViT~\cite{yuan2022ptq4vit}}} &
  \multicolumn{1}{c|}{6} &
  \multicolumn{1}{c|}{6} &
  \multicolumn{1}{c|}{-} &
  45.4 ($+39.6$) \\
 &
   &
  \multicolumn{1}{c|}{} &
  \multicolumn{1}{c|}{4} &
  \multicolumn{1}{c|}{4} &
  \multicolumn{1}{c|}{-} &
  23.7 ($+16.8$) \\ \hline
FQ-ViT~\cite{lin2021fq} &
  Mask-RCNN+Swin-S &
  \multicolumn{1}{c|}{OMSE~\cite{choukroun2019low}} &
  \multicolumn{1}{c|}{8} &
  \multicolumn{1}{c|}{8} &
  \multicolumn{1}{c|}{47.8(+5.2)} &
  - \\ \hline
NoisyQuant~\cite{liu2023noisyquant} &
  DETR &
  \multicolumn{1}{c|}{PTQ~\cite{liu2021post}} &
  \multicolumn{1}{c|}{8} &
  \multicolumn{1}{c|}{8} &
  \multicolumn{1}{c|}{41.4($+0.2$)} &
  - \\ \hline
\multirow{2}{*}{RepQ-ViT~\cite{li2023repq}} &
  Mask-RCNN+Swin-S &
  \multicolumn{1}{c|}{\multirow{2}{*}{APQ-ViT~\cite{ding2022towards}}} &
  \multicolumn{1}{c|}{6} &
  \multicolumn{1}{c|}{6} &
  \multicolumn{1}{c|}{-} &
  47.8 ($+0.1$) \\ \cline{2-2} \cline{4-7} 
 &
  Cascade Mask-RCNN+Swin-S &
  \multicolumn{1}{c|}{} &
  \multicolumn{1}{c|}{6} &
  \multicolumn{1}{c|}{6} &
  \multicolumn{1}{c|}{-} &
  44.6 ($+0.1$) \\ \hline
\multirow{2}{*}{LRP-QViT~\cite{ranjan2024lrp}} &
  Mask-RCNN+Swin-S &
  \multicolumn{1}{c|}{\multirow{2}{*}{RepQ-ViT~\cite{li2023repq}}} &
  \multicolumn{1}{c|}{6 MP} &
  \multicolumn{1}{c|}{6 MP} &
  \multicolumn{1}{c|}{} &
  48.1 ($+0.3$) \\ \cline{2-2} \cline{4-7} 
 &
  Cascade Mask-RCNN+Swin-S &
  \multicolumn{1}{c|}{} &
  \multicolumn{1}{c|}{6 MP} &
  \multicolumn{1}{c|}{6 MP} &
  \multicolumn{1}{c|}{} &
  51.4 ($+0.0$) \\ \hline
\end{tabular}%
}
\vspace{-3mm}
\end{table}

\noindent \textbf{Standalone QAT techniques} Although most works utilized the KD in QAT works, there are limited standalone studies without KD. PackQViT~\cite{dong2024packqvit} proposed activation-aware sub-8-bit QAT techniques for mobile devices. The authors leveraged $\log 2$ quantization or clamping to address the long-tailed distribution and outlier-aware training to handle the channel-wise outliers. Furthermore, The authors utilized int-2\textsuperscript{n}-softmax, int-LayerNorm, and int-GELU to enable integer-only computation. They designed a SIMD-based 4-bit packed multiplier to achieve end-to-end ViT acceleration on mobile devices. Another recent study named QD-BEV~\cite{zhang2023qd} explored QAT on BEVFormer~\cite{li2022bevformer} by leveraging 
image and BEV features. The authors identified that applying quantization directly in BEV tasks makes the training unstable, which leads to performance degradation. The authors proposed view-guided distillation to stabilize the QAT by conducting a systematic analysis of quantizing BEV networks. This work will open a new direction for autonomous vehicle research, applying QAT to reduce computational costs.
% Please add the following required packages to your document preamble:
% \usepackage{multirow}
% \usepackage{graphicx}
\begin{table}[]
\centering
\caption{Results of different quantization-aware training techniques proposed for ViTs. \textbf{C} and \textbf{OD} refer to classification tasks and object detection tasks in the CV tasks column, respectively. \textbf{CMR} refers to the Cascade Mask R-CNN model. The results of QD-BEV~\cite{zhang2023qd} is the mean average precision (mAP) rather than AP\textsuperscript{box}.}
\label{quantization_qat_classification}
\renewcommand{\arraystretch}{1}
\resizebox{\columnwidth}{!}{%
\begin{tabular}{c|c|c|c|cccccc}
\hline
\multirow{2}{*}{\textbf{Algorithm}} &
  \multirow{2}{*}{\textbf{Key point}} &
  \multirow{2}{*}{\textbf{CV tasks}} &
  \multirow{2}{*}{\textbf{Dataset}} &
  \multicolumn{6}{c}{\textbf{Results}} \\ \cline{5-10} 
 &
   &
   &
   &
  \multicolumn{1}{c|}{\textbf{Baseline}} &
  \multicolumn{1}{c|}{\textbf{Model}} &
  \multicolumn{1}{c|}{\textbf{W-bit}} &
  \multicolumn{1}{c|}{\textbf{A-bit}} &
  \multicolumn{1}{c|}{\textbf{\begin{tabular}[c]{@{}c@{}}Top-1\\ accuracy\end{tabular}}} &
  AP\textsuperscript{box} \\ \hline
\multirow{4}{*}{\begin{tabular}[c]{@{}c@{}}Q-ViT\\ ~\cite{li2022q}\end{tabular}} &
  \multirow{4}{*}{\begin{tabular}[c]{@{}c@{}}Propose a switchable scale to resolve \\ convergence issue during joint training\\  of quantization scales \& bit-widths; \\ limits of ViT quantization to 3-bit\end{tabular}} &
  \multirow{4}{*}{C} &
  \multirow{4}{*}{ImageNet-1k~\cite{5206848}} &
  \multicolumn{1}{c|}{\multirow{4}{*}{LSQ~\cite{esser2019learned}}} &
  \multicolumn{1}{c|}{\multirow{2}{*}{DeiT-B}} &
  \multicolumn{1}{c|}{4} &
  \multicolumn{1}{c|}{4} &
  \multicolumn{1}{c|}{83.0 ($+2.1$)} &
  - \\
 &
   &
   &
   &
  \multicolumn{1}{c|}{} &
  \multicolumn{1}{c|}{} &
  \multicolumn{1}{c|}{2} &
  \multicolumn{1}{c|}{2} &
  \multicolumn{1}{c|}{74.2 ($+3.9$)} &
  - \\ \cline{6-10} 
 &
   &
   &
   &
  \multicolumn{1}{c|}{} &
  \multicolumn{1}{c|}{\multirow{2}{*}{Swin-S}} &
  \multicolumn{1}{c|}{4} &
  \multicolumn{1}{c|}{4} &
  \multicolumn{1}{c|}{84.4 ($+1.9$)} &
  - \\
 &
   &
   &
   &
  \multicolumn{1}{c|}{} &
  \multicolumn{1}{c|}{} &
  \multicolumn{1}{c|}{2} &
  \multicolumn{1}{c|}{2} &
  \multicolumn{1}{c|}{76.9 ($+4.5$)} &
  - \\ \hline
\multirow{4}{*}{\begin{tabular}[c]{@{}c@{}}Q-DETR\\ ~\cite{xu2023q}\end{tabular}} &
  \multirow{4}{*}{\begin{tabular}[c]{@{}c@{}}Utilized knowledge distillation \\ to improve the representation capacity\\  of the quantized model\end{tabular}} &
  \multirow{4}{*}{OD} &
  \multirow{2}{*}{PASCAL VOC~\cite{everingham2010pascal}} &
  \multicolumn{1}{c|}{\multirow{4}{*}{LSQ~\cite{esser2019learned}}} &
  \multicolumn{1}{c|}{DETR-R50} &
  \multicolumn{1}{c|}{2} &
  \multicolumn{1}{c|}{2} &
  \multicolumn{1}{c|}{-} &
  50.7 ($+8.1$) \\ \cline{6-10} 
 &
   &
   &
   &
  \multicolumn{1}{c|}{} &
  \multicolumn{1}{c|}{SMCA-DETR-R50} &
  \multicolumn{1}{c|}{2} &
  \multicolumn{1}{c|}{2} &
  \multicolumn{1}{c|}{-} &
  50.2 ($+7.9$) \\ \cline{4-4} \cline{6-10} 
 &
   &
   &
  \multirow{2}{*}{COCO 2017~\cite{lin2014microsoft}} &
  \multicolumn{1}{c|}{} &
  \multicolumn{1}{c|}{DETR-R50} &
  \multicolumn{1}{c|}{4} &
  \multicolumn{1}{c|}{4} &
  \multicolumn{1}{c|}{-} &
  39.4 ($+6.1$) \\ \cline{6-10} 
 &
   &
   &
   &
  \multicolumn{1}{c|}{} &
  \multicolumn{1}{c|}{SMCA-DETR-R50} &
  \multicolumn{1}{c|}{4} &
  \multicolumn{1}{c|}{4} &
  \multicolumn{1}{c|}{-} &
  38.3 ($+4.4$) \\ \hline
\multirow{4}{*}{\begin{tabular}[c]{@{}c@{}}AQ-DETR\\ ~\cite{aqvit}\end{tabular}} &
  \multirow{4}{*}{\begin{tabular}[c]{@{}c@{}}Proposed a QAT technique based \\ on auxiliary queries for DETR\end{tabular}} &
  \multirow{4}{*}{OD} &
  \multirow{2}{*}{PASCAL VOC~\cite{everingham2010pascal}} &
  \multicolumn{1}{c|}{\multirow{4}{*}{Q-DETR~\cite{xu2023q}}} &
  \multicolumn{1}{c|}{DETR-R50} &
  \multicolumn{1}{c|}{4} &
  \multicolumn{1}{c|}{4} &
  \multicolumn{1}{c|}{-} &
  53.7 ($+3.3$) \\ \cline{6-10} 
 &
   &
   &
   &
  \multicolumn{1}{c|}{} &
  \multicolumn{1}{c|}{Deformable DETR-R50} &
  \multicolumn{1}{c|}{4} &
  \multicolumn{1}{c|}{4} &
  \multicolumn{1}{c|}{-} &
  63.1 ($+2.0$) \\ \cline{4-4} \cline{6-10} 
 &
   &
   &
  \multirow{2}{*}{\begin{tabular}[c]{@{}c@{}}COCO 2017\\ ~\cite{lin2014microsoft}\end{tabular}} &
  \multicolumn{1}{c|}{} &
  \multicolumn{1}{c|}{DETR-R50} &
  \multicolumn{1}{c|}{4} &
  \multicolumn{1}{c|}{4} &
  \multicolumn{1}{c|}{-} &
  40.2 ($+2.8$) \\ \cline{6-10} 
 &
   &
   &
   &
  \multicolumn{1}{c|}{} &
  \multicolumn{1}{c|}{Deformable DETR-R50} &
  \multicolumn{1}{c|}{4} &
  \multicolumn{1}{c|}{4} &
  \multicolumn{1}{c|}{-} &
  44.1 ($+3.4$) \\ \hline
\multirow{4}{*}{\begin{tabular}[c]{@{}c@{}}PSAQ-ViT V2\\ ~\cite{li2023psaq}\end{tabular}} &
  \multirow{4}{*}{\begin{tabular}[c]{@{}c@{}}Proposed a data-free quantization \\ framework with adaptive \\ teacher-student strategy\end{tabular}} &
  \multirow{4}{*}{C \& OD} &
  \multirow{2}{*}{ImageNet-1k~\cite{5206848}} &
  \multicolumn{1}{c|}{\multirow{2}{*}{PSAQ-ViT~\cite{li2022psaqvit}}} &
  \multicolumn{1}{c|}{DeiT-B} &
  \multicolumn{1}{c|}{8} &
  \multicolumn{1}{c|}{8} &
  \multicolumn{1}{c|}{81.5 ($+2.4$)} &
  - \\ \cline{6-10} 
 &
   &
   &
   &
  \multicolumn{1}{c|}{} &
  \multicolumn{1}{c|}{Swin-S} &
  \multicolumn{1}{c|}{8} &
  \multicolumn{1}{c|}{8} &
  \multicolumn{1}{c|}{82.1 ($+5.5$)} &
  - \\ \cline{4-10} 
 &
   &
   &
  \multirow{2}{*}{COCO 2017~\cite{lin2014microsoft}} &
  \multicolumn{1}{c|}{\multirow{2}{*}{Standard V2}} &
  \multicolumn{1}{c|}{DeiT-S + CMR} &
  \multicolumn{1}{c|}{8} &
  \multicolumn{1}{c|}{8} &
  \multicolumn{1}{c|}{-} &
  44.8 ($+.3$) \\ \cline{6-10} 
 &
   &
   &
   &
  \multicolumn{1}{c|}{} &
  \multicolumn{1}{c|}{Swin-S + CMR} &
  \multicolumn{1}{c|}{8} &
  \multicolumn{1}{c|}{8} &
  \multicolumn{1}{c|}{-} &
  50.9 ($+0.6$) \\ \hline
\multirow{3}{*}{\begin{tabular}[c]{@{}c@{}}PackQViT\\ ~\cite{dong2024packqvit}\end{tabular}} &
  \multirow{3}{*}{\begin{tabular}[c]{@{}c@{}}Proposed an 8-bit QAT framework\\  for mobile devices\end{tabular}} &
  \multirow{3}{*}{C \& OD} &
  \multirow{2}{*}{ImageNet-1k~\cite{5206848}} &
  \multicolumn{1}{c|}{\multirow{2}{*}{Q-ViT~\cite{li2022q}}} &
  \multicolumn{1}{c|}{DeiT-B} &
  \multicolumn{1}{c|}{8} &
  \multicolumn{1}{c|}{8} &
  \multicolumn{1}{c|}{82.9 ($+0.5$)} &
  - \\ \cline{6-10} 
 &
   &
   &
   &
  \multicolumn{1}{c|}{} &
  \multicolumn{1}{c|}{Swin-S} &
  \multicolumn{1}{c|}{8} &
  \multicolumn{1}{c|}{8} &
  \multicolumn{1}{c|}{84.1 ($+0.5$)} &
  - \\ \cline{4-10} 
 &
   &
   &
  COCO 2017~\cite{lin2014microsoft} &
  \multicolumn{1}{c|}{PTQ~\cite{liu2021post}} &
  \multicolumn{1}{c|}{DETR-R50} &
  \multicolumn{1}{c|}{8} &
  \multicolumn{1}{c|}{8} &
  \multicolumn{1}{c|}{-} &
  60.0 ($-3.1$) \\ \hline
\multirow{3}{*}{\begin{tabular}[c]{@{}c@{}}QD-BEV\\ ~\cite{zhang2023qd}\end{tabular}} &
  \multirow{3}{*}{\begin{tabular}[c]{@{}c@{}}Introduced a view-guided objective\\  to stabilize the QAT training \\ for image features \& BEV features\end{tabular}} &
  \multirow{3}{*}{OD} &
  \multirow{3}{*}{NuScenes~\cite{caesar2020nuscenes}} &
  \multicolumn{1}{c|}{BEVFormer-B-DFQ~\cite{nagel2019data}} &
  \multicolumn{1}{c|}{\multirow{3}{*}{BEVFormer}} &
  \multicolumn{1}{c|}{8} &
  \multicolumn{1}{c|}{8} &
  \multicolumn{1}{c|}{-} &
  40.6 ($+2.2$) \\ \cline{5-5} \cline{7-10} 
 &
   &
   &
   &
  \multicolumn{1}{c|}{BEVFormer-S-HAWQv3~\cite{yao2021hawq}} &
  \multicolumn{1}{c|}{} &
  \multicolumn{1}{c|}{8} &
  \multicolumn{1}{c|}{8} &
  \multicolumn{1}{c|}{-} &
  40.6 ($+3.0$) \\ \cline{5-5} \cline{7-10} 
 &
   &
   &
   &
  \multicolumn{1}{c|}{BEVFormer-B-PACT~\cite{choi2018pact}} &
  \multicolumn{1}{c|}{} &
  \multicolumn{1}{c|}{8} &
  \multicolumn{1}{c|}{8} &
  \multicolumn{1}{c|}{-} &
  40.6 ($+3.2$) \\ \hline
\end{tabular}%
}
\vspace{-3mm}
\end{table}
\subsubsection{Discussion}\hfill \\ Tables~\ref{quantization_tech_classification} and~\ref{tab:quantization_object} provide an overview of PTQ techniques for image classification and object detection on ViT models. A notable trend is the dominance of post-training quantization (PTQ) methods, with fewer studies exploring training-phase QAT. Table~\ref{quantization_tech_classification} highlights the top-1 accuracy improvements achieved by PTQ methods on ViT architectures for the classification tasks, while Table~\ref{tab:quantization_object} summaries the mean average precision (mAP) and AP\textsuperscript{box} improvement achieved from the baseline techniques for the object detection task. Moreover, Table~\ref{quantization_qat_classification} summarizes QAT techniques for image classification and object detection. Interestingly, most QAT methods for ViTs, such as those leveraging knowledge distillation~\cite{li2022q,xu2023q,aqvit,li2023psaq}, focus on optimizing quantized models but lack experimental validation on edge devices. In contrast, approaches like PackQViT~\cite{dong2024packqvit} have experimented with their QAT frameworks on mobile devices, pushing the boundaries of practical deployment. However, these quantization techniques broadly experimented on datasets like ImageNet-1k and COCO 2017, raising questions about their generalizability to specialized domains such as medical imaging or autonomous driving. This gap underscores the need for future research to explore versatile quantization strategies that cater to diverse application areas and resource-constrained edge devices.
\section{Tools for Efficient Edge Deployment}\label{tools_for_edge}
Efficient edge deployment of ViT requires a combination of software tools, evaluation tools, and advanced optimization techniques for different hardware architectures. Software tools streamline model deployment by providing optimized libraries and frameworks tailored for edge environments. Optimization techniques, such as memory optimization and pipeline parallelism, enhance performance by leveraging hardware-specific optimizations. Finally, heterogeneous platforms, including CPUs, GPUs, FPGAs, and custom accelerators, offer the flexibility to balance power, performance, and cost for various applications. In this section, we explore these essential pillars of edge deployment.
% \subsection{Hardware Platforms for Vision Transformer Inference}

%% Will add hardware CPU<GPU and FPGA based on different parameters.

\subsection{Software Tools}
 Deploying deep learning models on heterogeneous platforms demands specialized software tools that bridge the gap between cutting-edge artificial intelligence (AI) research and real-world applications. These tools empower developers to optimize, accelerate, and seamlessly integrate AI models across different hardware architectures. Table~\ref{softwaretools} illustrates the most popularly used software tools/libraries/engines to deploy the deep learning models on different hardware architectures. The software libraries are divided into three hardware architectures: FPGA, GPU, and CPU. As FPGAs offer highly parallel and reconfigurable hardware capabilities,  deploying AI models on FPGAs requires specialized software tools for efficient hardware mapping, optimization, and deployment. Both Vivado Design Suite~\footnote{\label{vivado}Vivado design suite. Retrieved January 18, 2025, from \url{https://www.amd.com/en/products/software/adaptive-socs-and-fpgas/vivado.html}} (from Xilinx) and the Quartus Prime Design Software~\footnote{\label{quartus}Intel Quartus Prime. Retrieved January 18, 2025, from \url{https://www.intel.com/content/www/us/en/products/details/fpga/development-tools/quartus-prime.html}} (from Intel) offer advanced synthesis, converting high-level languages to hardware description language (HDL)  and preoptimized AI accelerators IP cores (such as Xilinx DPU or Intel AI Suite) that help to accelerate the inference task. Vitis AI~\footnote{\label{vitis}Vitis AI. Retrieved January 18, 2025, from \url{https://www.xilinx.com/products/design-tools/vitis/vitis-ai.html}} is the software platform for Xilinx FPGA while OpenVINO~\footnote{\label{Openvinotoolkit}Openvinotoolkit. Retrieved January 18, 2025, from \url{https://github.com/openvinotoolkit/openvino}} designed for Intel FPGA and other hardware architectures, including GPU and CPU. However, It is possible/likely that each of these environments is highly modified only for their hardware family, which means developing applications on one would make it very difficult to port to the other. Besides those two software tools, Hls4ml~\footnote{\label{hls4ml}Hls4ml. Retrieved January 18, 2025, from \url{https://fastmachinelearning.org/hls4ml/index.html} } and FINN~\footnote{\label{finn}FINN. Retrieved January 18, 2025, from \url{https://xilinx.github.io/finn/} } designed to explore deep neural network inference on FPGAs efficiently and swiftly. However, both these libraries are still in the experimental phase. In the software tools for GPU, NVIDIA has a wide range of libraries/engines for edge devices. NVIDIA Triton Inference Server~\footnote{\label{triton}NVIDIA Triton Inference Server. Retrieved January 18, 2025, from \url{https://github.com/triton-inference-server/server}} is the most prominent open-source and scalable inference-serving software engine that simplifies the deployment of deep learning models at scale across all NVIDIA GPUs, x86, and Arm CPUs from major frameworks, including TensorFlow, PyTorch, and NVIDIA TensorRT. Additionally, TensorRT~\footnote{\label{tensorrt}TensorRT. Retrieved January 18, 2025, from \url{https://developer.nvidia.com/tensorrt}} uses as a popular inference optimizer and runtime library for NVIDIA GPU-based edge devices. oneDNN~\footnote{\label{onednn}oneDNN. Retrieved January 18, 2025, from \url{https://github.com/oneapi-src/oneDNN}} is a widely used open-source, cross-platform performance tool for deep learning models. It is optimized for Intel processors, graphics, and ARM-based processors and is in the experimental stage for NVIDIA GPU, AMD GPU, and RISC-V processors. Each software tool is mostly designed to optimize performance and efficiency for specific hardware architectures. However, ONNX Runtime~\footnote{\label{onnxrun}ONNX Runtime. Retrieved January 18, 2025, from \url{https://onnxruntime.ai/}} is one of the few inference engines that supports a wide range of hardware, including CPUs, GPUs, and FPGAs.
 % Please add the following required packages to your document preamble:
% \usepackage{multirow}
% \usepackage{graphicx}
\begin{table}[]
\centering
\caption{The overview of popular software tools for deploying deep learning models on different hardware architectures.}
\label{softwaretools}
\renewcommand{\arraystretch}{0.95}
\resizebox{\columnwidth}{!}{%
\begin{tabular}{c|c|c|ccc}
\hline
\multirow{2}{*}{\textbf{\begin{tabular}[c]{@{}c@{}}Tool \\ Name\end{tabular}}} &
  \multirow{2}{*}{\textbf{Type}} &
  \multirow{2}{*}{\textbf{\begin{tabular}[c]{@{}c@{}}Key \\ features\end{tabular}}} &
  \multicolumn{3}{c}{\textbf{Supported edge devices}} \\ \cline{4-6} 
 &
   &
   &
  \multicolumn{1}{c|}{\textbf{FPGA}} &
  \multicolumn{1}{c|}{\textbf{GPU}} &
  \textbf{CPU} \\ \hline
Xilinx Vivado design suite~\footref{vivado} &
  Toolkit &
  Used for synthesis, simulation, and configuring Xilinx FPGAs &
  \multicolumn{1}{c|}{\begin{tabular}[c]{@{}c@{}}Xilinx \\ FPGA\end{tabular}} &
  \multicolumn{1}{c|}{No} &
  No \\ \hline
Intel Quartus Prime~\footref{quartus} &
  Toolkit &
  Used for synthesis, place-and-route, and programming. &
  \multicolumn{1}{c|}{\begin{tabular}[c]{@{}c@{}}Intel\\ FPGA\end{tabular}} &
  \multicolumn{1}{c|}{No} &
  No \\ \hline
OpenVINO~\footref{Openvinotoolkit} &
  Toolkit &
  Optimized inference, supports multiple frameworks, Intel hardware-focused &
  \multicolumn{1}{c|}{\begin{tabular}[c]{@{}c@{}}Intel\\ FPGA\end{tabular}} &
  \multicolumn{1}{c|}{Yes} &
  Yes \\ \hline
Vitis AI~\footref{vitis} &
  Engine &
  Model compression, xmodel generation, diverse pretrained model &
  \multicolumn{1}{c|}{\begin{tabular}[c]{@{}c@{}}Xilinx \\ FPGA\end{tabular}} &
  \multicolumn{1}{c|}{No} &
  No \\ \hline
Hls4ml~\footref{hls4ml} &
  Library &
  High-level synthesis for FPGA, supports various frameworks &
  \multicolumn{1}{c|}{Yes} &
  \multicolumn{1}{c|}{No} &
  No \\ \hline
FINN~footref{finn} &
  Engine &
  Quantized models, based on Vitis AI, streaming dataflow for inference &
  \multicolumn{1}{c|}{\begin{tabular}[c]{@{}c@{}}Xilinx \\ FPGA\end{tabular}} &
  \multicolumn{1}{c|}{No} &
  No \\ \hline
TensorRT~\footref{tensorrt} &
  Engine &
  High-performance AI inference on NVIDIA GPUs &
  \multicolumn{1}{c|}{No} &
  \multicolumn{1}{c|}{\begin{tabular}[c]{@{}c@{}}NVIDIA\\ GPU\end{tabular}} &
  No \\ \hline
NVIDIA Triton Inference Server~\footref{triton} &
  Inference Server &
  Scalable inference serving, multi-model deployment, Provide model analyzer &
  \multicolumn{1}{c|}{No} &
  \multicolumn{1}{c|}{\begin{tabular}[c]{@{}c@{}}NVIDIA\\ GPU\end{tabular}} &
  x86, ARM \\ \hline
oneDNN~\footref{onednn} &
  Library &
  Optimized deep learning performance, cross-platform &
  \multicolumn{1}{c|}{No} &
  \multicolumn{1}{c|}{Experimental} &
  Intel CPU, ARM \\ \hline
ONNX Runtime~\footref{onnxrun} &
  Engine &
  Open-source inference, hardware-specific execution providers &
  \multicolumn{1}{c|}{Yes} &
  \multicolumn{1}{c|}{Yes} &
  Yes \\ \hline
\end{tabular}%
}
\vspace{-3mm}
\end{table}
 % Deploying deep learning models on heterogeneous platforms demands specialized software tools that bridge the gap between cutting-edge AI research and real-world applications. These tools empower developers to optimize, accelerate, and seamlessly integrate AI models across different hardware architectures. The following software tools are popular for processing deployment on edge devices. 
% \subsubsection{Software tools for FPGA deployment}
% \noindent \textbf{OpenVINO~\footnote{Openvinotoolkit. Openvinotoolkit/openvino. Retrieved January 18, 2025, from \url{https://github.com/openvinotoolkit/openvino}}:} OpenVINO is an open-source software toolkit provided by Intel corporation to boost the different deep learning models. It can utilize trained models directly from popular frameworks such as PyTorch~\cite{paszke2019pytorch}, TensorFlow~\cite{tensorflow2015whitepaper}, ONNX~\cite{onnxruntime}, Keras~\cite{chollet2015keras}, and JAX~\cite{jax2018github}, including direct integration of transformer models. OpenVINO supports diverse hardware architectures, including CPU (x86, ARM), GPU, and AI accelerators (Intel neural processing unit).

% \noindent \textbf{Vitis AI~\footnote{Vitis AI. Retrieved January 18, 2025, from \url{https://www.xilinx.com/products/design-tools/vitis/vitis-ai.html} }:} An AI software platform by Xilinx that provides comprehensive tools and optimized architectures for deploying AI inference on Xilinx FPGAs. Vitis-AI mainly compresses the models using compression techniques and generates the xmodel to deploy into FPGA. Vitis AI includes a set of optimized neural network tools, libraries for evaluation, and models. Moreover, Xilinx offers a "model zoo" of pre-trained and optimized models for common AI tasks that can be directly deployed on FPGA using the Vitis AI platform.
% % \noindent \textbf{OpenCL~\cite{Trevett_Richards_Butler_McVeigh_Bhat_Calidas_Hindriksen_Dai_2013}:} A framework for writing programs that execute across heterogeneous platforms. Many FPGA vendors offer OpenCL support for programming FPGAs.

% \noindent \textbf{Hls4ml~\footnote{Hls4ml. Retrieved January 18, 2025, from \url{https://fastmachinelearning.org/hls4ml/index.html} }:} Hls4ml strives to efficiently and swiftly transform machine learning models from open-source platforms (such as Keras and PyTorch). It produces high-level synthesis (HLS) code that can be converted to FPGA firmware using the HLS compilers for different FPGA vendors. HLS4ml utilized Keras, PyTorch, Brevitas, and ONNX as the front, while Vivado/Vitis, oneAPI, and Quantus can be used as a backend for different FPGA manufacturers. However, HLs4ml is still under development and causes HLS synthesis issues, such as stopping the unrolling loop.

% \noindent \textbf{FINN~\footnote{FINN. Retrieved January 18, 2025, from \url{https://xilinx.github.io/finn/} }
% :} FINN is an experimental framework from Xilinx Research Labs to explore deep neural network inference on FPGAs. It provides end-to-end inference solutions for deploying quantized models on FPGA. It relies on co-design and design space exploration for quantization and parallelization tuning rather than just a generic deep learning acceleration solution. FINN utilized templated Vitis HLS and register transfer level (RTL) modules that implement neural network layers as streaming components.

% \noindent \textbf{TensorRT~\footnote{TensorRT. Retrieved January 18, 2025, from \url{https://developer.nvidia.com/tensorrt}}} TensorRT is a popular inference optimizer, and runtime library focused on low latency and high throughput performance for AI models deployed on NVIDIA GPUs. It offers various optimization techniques to reduce precision and supports plugins for custom layers to handle non-standard operations and convert models for deployment on GPU-based edge devices.

% \noindent \textbf{NVIDIA Triton Inference Server~\footnote{NVIDIA Triton Inference Server. Retrieved January 18, 2025, from \url{https://github.com/triton-inference-server/server}}} NVIDIA Triton Inference Server is open-source and scalable inference-serving software engine that simplifies the deployment of deep learning models at scale across all NVIDIA GPUs, x86, and Arm CPUs from major frameworks, including TensorFlow, PyTorch, and NVIDIA TensorRT. Additionally, it provides a model analyzer that reduces the time needed to find the optimal model deployment configuration. This software engine can also be utilized for any GPU- or CPU-based infrastructure (cloud, data center, or edge).

% \noindent \textbf{oneDNN~\footnote{oneDNN. Retrieved January 18, 2025, from \url{https://github.com/oneapi-src/oneDNN}}} OneDNN is a popular open-source cross-platform performance tool for deep learning models. It is optimized for Intel processors, graphics, and ARM-based processors. Additionally, it has experimental support for NVIDIA GPU, AMD GPU, and RISC-V processors.

% \noindent \textbf{ONNX Runtime~\footnote{ONNX Runtime. Retrieved January 18, 2025, from \url{https://github.com/oneapi-src/oneDNN}}} ONNX Runtime is an open-source inference engine that supports a wide range of hardware, including CPUs, GPUs, and FPGAs. It works with ONNX models and is optimized for various platforms through hardware-specific execution providers.
\subsection{Evaluation Tools}
Evaluating the performance of ViT acceleration techniques on edge platforms requires specialized tools and metrics to evaluate power consumption, energy efficiency, accuracy, and latency. Fortunately, most hardware vendors offer built-in tools and libraries to facilitate precise measurement of these key performance indicators.
\subsubsection{Latency}\hfill\\ Latency and frame per second (FPS) can be calculated as follows:
\[
\text{FPS} = \frac{1}{Latency}
\]
For GPU-based evaluations, PyTorch provides \texttt{torch.cuda.Event(enable\_timing=True)} for GPU-based evaluations, which accurately measures latency during inference. On NVIDIA EdgeGPU platforms, the \textbf{TensorRT Profiler} offers a detailed latency breakdown for Jetson boards. For AMD FPGAs, the \textbf{Vitis AI Profiler~\footnote{\label{vitisAI}Vitis AI Profiler. Retrieved February 10, 2025, from \url{https://github.com/Xilinx/Vitis-AI/tree/master/examples/vai_profiler}}} enables profiling during deployment, ensuring optimized execution. Additionally, \textbf{Intel's OpenVINO benchmark tool~\footref{Openvinotoolkit}} supports latency and throughput measurements across Intel CPUs and FPGAs, providing a standardized evaluation framework.

\subsubsection{Power}\hfill\\ Measuring power consumption is critical yet challenging in evaluating ViT acceleration techniques. Standard tools for general-purpose platforms (GPPs) like CPUs and GPUs include \textbf{Intel Power Gadget} for Intel CPUs and \textbf{NVIDIA-SMI} for NVIDIA GPUs. Power can be measured on edge GPU platforms, such as NVIDIA Jetson boards, using \textbf{tegraStats}, which provides real-time power monitoring, GPU utilization, and temperature. For FPGAs and ACAPs, AMD Xilinx offers \textbf{Xilinx Power Estimator (XPE)~\footnote{\label{xpe}Xilinx Power Estimator. Retrieved January 18, 2025, from \url{https://www.amd.com/en/products/adaptive-socs-and-fpgas/technologies/power-efficiency/power-estimator.html}}} for power estimation based on hardware configurations, while \textbf{Vaitrace} enables runtime power profiling for FPGA and adaptive compute acceleration platforms (ACAP). These tools provide essential insights into power efficiency, thermal behavior, and overall performance trade-offs across edge hardware platforms.
\subsubsection{Energy}\hfill\\ Energy consumption in ViT acceleration can be estimated through throughput per joule (GOP/J) and FPS per watt (FPS/W). However, accurately measuring energy on general-purpose platforms (GPPs) is challenging due to background processes affecting power readings. In contrast, edge devices provide a more controlled environment where only one primary task is executed simultaneously, making energy estimation more reliable. Several tools facilitate energy measurement: \textbf{Xilinx Vivado Power Analyzer~\footref{vivado}} estimates energy efficiency for FPGAs by profiling dynamic power, while \textbf{RAPL} tracks CPU-level energy consumption on x86 architectures. Additionally, energy efficiency can be derived using power measurements combined with latency, enabling a deeper evaluation of acceleration techniques.

\subsubsection{Resosurce Utlization}\hfill\\ Resource utilization is primarily analyzed in FPGA-based acceleration techniques to optimize hardware efficiency and minimize resource usage. Currently, Intel and AMD are two FPGA vendors. AMD offers \textbf{Vivado Design suite~\footref{vivado}} for synthesis evaluation of the FPGA before deployment. Similarly, Intel provides the \textbf{Quartus Prime~\footref{quartus}} software, which facilitates FPGA synthesis, resource utilization monitoring, and performance evaluation. Both vendors offer additional AI optimization frameworks—AMD’s \textbf{Vitis AI~\footref{vitis}} and Intel’s \textbf{FPGA SDK for OpenCL (AOCL)}—to enhance the efficiency of ViT acceleration on FPGA platforms.
\subsection{Common Optimization Techniques}
\textbf{Memory Optimization} Techniques like Huffman coding can be used to compress the weights. On-chip memory utilization efficiently uses the FPGA's Block RAMs (BRAMs) to store weights and intermediate feature maps, reducing the need for off-chip memory accesses, which can be slow and power-hungry.

\noindent \textbf{Pipeline Parallelism} The technique splits the model into stages and simultaneously processes different inputs at each stage, which helps in maximizing the throughput.

\noindent \textbf{Loop Unrolling} This FPGA-specific optimization involves unrolling loops in the FPGA design to speed up the processing. For instance, when performing matrix multiplications in the transformer layers.

\noindent \textbf{Layer Fusion} Layer fusion combines multiple layers into a single computational unit, reducing memory access between layers and improving the overall throughput.

\noindent \textbf{Hardware-friendly Activation Functions} Replace complex activation functions with simpler, hardware-friendly alternatives. For example, using piecewise linear approximations for non-linearities.
 
\noindent \textbf{Optimized Matrix Operations} ViT involves many matrix multiplications (in the attention mechanisms). Optimizing these matrix operations for FPGA leads to significant speed-ups. Techniques like systolic arrays or optimized linear algebra cores are employed.

\noindent \textbf{Dynamic Precision} In recent studies, some work uses mixed precision computations where certain parts of the model use lower precision (e.g., 8-bit). In comparison, other parts use higher precision (e.g., 16-bit or 32-bit). There are some works in which the authors introduced fixed point and PoT precision and optimally balanced accuracy and performance.
\section{Accelerating Strategies For ViT on Edge}\label{acce_tech}
This section explores acceleration strategies for non-linear operations, discusses SOTA ViT acceleration techniques, and provides a comprehensive performance analysis in terms of both hardware efficiency and accuracy.
\subsection{Accelerating Non-linear operations} ViT models in CV can mainly be split into two types of operations: linear and non-linear. Optimizing non-linear operations in quantized ViT is as crucial as optimizing linear operations. While low-bit computing units significantly reduce computational complexity and memory footprint, They are primarily designed for linear operations such as matrix multiplications and convolutions. However, non-linear functions, including softmax, GELU, and LayerNorm, are the essential components of ViT architectures yet unexplored largely in these low-bit computing environments. The lack of support for non-linear operations on hardware creates computational bottlenecks during the edge deployment, as non-linear functions often require FP32 operations. This results in latency and increases power and energy consumption, ultimately lessening the quantization's full benefits on edge deployment. Moreover, during inference on quantized ViT models on edge devices, frequent quantization and dequantization operations surrounding non-linear layers add further inefficiencies, slowing inference and reducing throughput~\cite{stevens2021softermax}. Researchers developed integer-based approximations for non-linear operations to solve these issues, eliminating the need for frequent FP32 computations in those layers.

As illustrated in Table~\ref{nonlinearoperation}, we introduce integer-only approximations for different hardware platforms to enhance ViT inference efficiency. As we discussed details in section~\ref{quan}, FQ-ViT~\cite{lin2021fq} utilized LIS for integer-only variants of the softmax that approximate the exponential component using second-order polynomial coupled with $\log 2$ quantization. For LayerNorm, the authors applied PTF to shift the quantized activations and later computed mean and variance using integer arithmetic. However, this approach seems hardware efficient; their methods of practical deployment on edge platforms are unclear. Additionally, I-ViT~\cite{li2022vit} calculates the square root in LayerNorm using an integer-based iterative method. The authors then introduced ShiftGELU, which used sigmoid-based approximations for GELU approximations. I-ViT utilized NVIDIA RTX 2080 Ti GPU as a hardware platform to evaluate their method. 

PackQViT~\cite{dong2024packqvit} is an extended version of the FQ-ViT concept that also leverages second-order polynomial approximations more straightforwardly, replacing the usual constant $e$ with 2 in the softmax. Although PackQViT requires training, the simplification ensures no accuracy loss. EdgeKernel~\cite{zhang2023practical} addresses precision challenges in softmax computations by optimizing the selection of the bit shift parameter on Apple A13 and M1 chips, ensuring high accuracy while minimizing significant bit truncations. Additionally, it employs asymmetric quantization for LayerNorm inputs, converting them to a uint16 format to enhance computational efficiency while maintaining data integrity.

SOLE~\cite{wang2023sole} and SwiftTron~\cite{marchisio2023swifttron} both are utilized application-specific integrated circuit (ASIC) platforms. SOLE optimizes the software perspectives, introducing E2Softmax with $\log 2$ quantization to avoid traditional FP32 precision in softmax layers. Additionally, SOLE designed a two-stage LayerNorm unit using PTF factors. However, SwiftTron focused on designing customized hardware for ASIC to efficiently use non-linear operations, even in FP32, which accounts for diverse scaling factors in performing correct computations.
% Please add the following required packages to your document preamble:
% \usepackage{multirow}
% \usepackage{graphicx}
\begin{table}[]
\centering
\caption{Overview of ViT models utilizing integer approximations for non-linear operations to enhance inference efficiency and avoid dequantization.}
\label{nonlinearoperation}
\resizebox{0.8\columnwidth}{!}{%
\begin{tabular}{c|c|ccc|c}
\hline
\multirow{2}{*}{\textbf{Model}} &
  \multirow{2}{*}{\textbf{\begin{tabular}[c]{@{}c@{}}Experiment\\ Hardware\end{tabular}}} &
  \multicolumn{3}{c|}{\textbf{Non-linear Operations}} &
  \multirow{2}{*}{\textbf{Retrain}} \\ \cline{3-5}
 &
   &
  \multicolumn{1}{c|}{\textbf{Softmax}} &
  \multicolumn{1}{c|}{\textbf{GELU}} &
  \textbf{LayerNorm} &
   \\ \hline
FQ-ViT~\cite{lin2021fq} &
  - &
  \multicolumn{1}{c|}{\cmark} &
  \multicolumn{1}{c|}{\xmark} &
  \cmark &
  \xmark \\ \hline
I-ViT~\cite{li2022vit} &
  RTX 2080 Ti GPU &
  \multicolumn{1}{c|}{\cmark} &
  \multicolumn{1}{c|}{\cmark} &
  \cmark &
  \cmark \\ \hline
EdgeKernel~\cite{zhang2023practical} &
  Apple A13 and M1 chips &
  \multicolumn{1}{c|}{\cmark} &
  \multicolumn{1}{c|}{\cmark} &
  \cmark &
  \xmark \\ \hline
PackQViT~\cite{dong2024packqvit} &
  Snapdragon 870 SoC (Mobile Phone) &
  \multicolumn{1}{c|}{\cmark} &
  \multicolumn{1}{c|}{\cmark} &
  \cmark &
  \cmark \\ \hline
SOLE~\cite{wang2023sole} &
  ASIC 28nm &
  \multicolumn{1}{c|}{\cmark} &
  \multicolumn{1}{c|}{\cmark} &
  \cmark &
  \xmark \\ \hline
SwiftTron~\cite{marchisio2023swifttron} &
  ASIC 65 nm &
  \multicolumn{1}{c|}{\cmark} &
  \multicolumn{1}{c|}{\cmark} &
  \cmark &
  \xmark \\ \hline
\end{tabular}%
}
\vspace{-3mm}
\end{table}
\subsection{Current Accelerating Techniques on ViT}
Efficient acceleration techniques for seamless deployment on edge devices are essential where computational and energy constraints limit performance. Various techniques have been developed to optimize ViT execution, balancing throughput, latency, and energy efficiency while ensuring minimal loss in accuracy. These techniques can be broadly classified into SW-HW co-design and hardware-only acceleration. SW-HW co-design integrates algorithmic optimizations with hardware-aware modifications to ensure efficient deployment of ViTs on edge devices, enhancing throughput and energy efficiency. On the other hand, hardware-only approaches push efficiency even further by designing architectures specifically optimized for transformer workloads, utilizing systolic arrays, spatial computing, and near-memory processing to eliminate bottlenecks caused by data movement and external memory access. Additionally, alternative approaches, such as distributing multiple tiny edge devices by partitioning the model into submodels, offer promising directions to rethink ViT acceleration from a fundamentally different perspective. 
% \begin{table}[]
% \centering
% \caption{The overview of current software-hardware co-design approaches for accelerating ViT on edge devices. 
% In the baseline models, \textbf{B} denotes Base; \textbf{S} denotes Small; and \textbf{T} denotes Tiny versions. 
% Five types of hardware devices are used in existing hardware-software co-design frameworks: 
% GPU ($\spadesuit$), EdgeGPU ($\blacktriangle$), CPU ($\bigstar$), FPGA ($\clubsuit$), and AMD Versal Adaptive Compute Acceleration (ACAP) ($\blacklozenge$). 
% \textbf{\textsuperscript{*}} indicates energy efficiency measurements in W·S (power × time).
% }
% \label{hwswco}
% \renewcommand{\arraystretch}{1}
% \resizebox{\columnwidth}{!}{%
% \begin{tabular}{c|c|c|cc|cc|c|c|c|c}
% \hline
%  &
%    &
%    &
%   \multicolumn{2}{c|}{\textbf{Hardware device}} &
%   \multicolumn{2}{c|}{\textbf{Key optimization}} &
%   &
%    &
%    &
%    \\ \cline{4-7}
% \multirow{-2}{*}{\textbf{Framework}} &
%   \multirow{-2}{*}{\textbf{Retrain}} &
%   \multirow{-2}{*}{\textbf{\begin{tabular}[c]{@{}c@{}}Baseline\\ Models\end{tabular}}} &
%   \multicolumn{1}{c|}{\textbf{Baseline}} &
%   \textbf{Experiment} &
%   \multicolumn{1}{c|}{\textbf{Software}} &
%   \textbf{Hardware} &
%   \multirow{-2}{*}{\textbf{Datasets}} &
%   \multirow{-2}{*}{\textbf{\begin{tabular}[c]{@{}c@{}}Energy \\ (FPS/W)\end{tabular}}} &
%   \multirow{-2}{*}{\textbf{\begin{tabular}[c]{@{}c@{}}Throughput \\ (GOPs)\end{tabular}}} &
%   \multirow{-2}{*}{\textbf{\begin{tabular}[c]{@{}c@{}}Speedup \\ ($\uparrow$)\end{tabular}}} \\ \hline
%  &
%    &
%    &
%   \multicolumn{1}{c|}{Intel i7-9800X\textsuperscript{$\bigstar$}} &
%    &
%   \multicolumn{1}{c|}{} &
%    &
%    &
%    &
%    &
%    \\
% \multirow{-2}{*}{VAQF~\cite{sun2022vaqf}} &
%   \multirow{-2}{*}{\xmark} &
%   \multirow{-2}{*}{DeiT-B/S/T} &
%   \multicolumn{1}{c|}{TITAN RTX\textsuperscript{$\spadesuit$}} &
%   \multirow{-2}{*}{ZCU102\textsuperscript{$\clubsuit$}} &
%   \multicolumn{1}{c|}{\multirow{-2}{*}{Quantization}} &
%   \multirow{-2}{*}{\begin{tabular}[c]{@{}c@{}}Traditional\\ resources\end{tabular}} &
%   \multirow{-2}{*}{ImageNet-1K~\cite{5206848}} &
%   \multirow{-2}{*}{4.05} &
%   \multirow{-2}{*}{861.2} &
%   \multirow{-2}{*}{-} \\ \hline
% Auto-ViT-Acc ~\cite{lit2022auto} &
%   \xmark &
%   DeiT-B/S/T &
%   \multicolumn{1}{c|}{A100\textsuperscript{$\spadesuit$}} &
%   ZCU102 \textsuperscript{$\clubsuit$} &
%   \multicolumn{1}{c|}{\begin{tabular}[c]{@{}c@{}}Mixed \\ quantization\end{tabular}} &
%   \begin{tabular}[c]{@{}c@{}}Traditional\\ resources\end{tabular} &
%   ImageNet-1K~\cite{5206848} &
%   10.35 &
%   907.8 &
%   - \\ \hline
% HeatViT~\cite{dong2023heatvit} &
%   \cmark &
%   DeiT-B/S/T &
%   \multicolumn{1}{c|}{A100\textsuperscript{$\spadesuit$}} &
%   ZCU102 \textsuperscript{$\clubsuit$} &
%   \multicolumn{1}{c|}{\begin{tabular}[c]{@{}c@{}}Adaptive \\ token pruning\end{tabular}} &
%   \begin{tabular}[c]{@{}c@{}}Motivated from\\ ~\cite{lit2022auto}\end{tabular} &
%   ImageNet-1K~\cite{5206848} &
%   19.4 &
%   271.2 &
%   2.34$\times$ \\ \hline
%  &
%    &
%    &
%   \multicolumn{1}{c|}{ZCU102\textsuperscript{$\clubsuit$}} &
%    &
%   \multicolumn{1}{c|}{} &
%    &
%    &
%    &
%   - &
%   59.5$\times$ \\ \cline{10-11} 
%  &
%    &
%    &
%   \multicolumn{1}{c|}{U250\textsuperscript{$\clubsuit$}} &
%   \multirow{-2}{*}{VCK190\textsuperscript{$\blacklozenge$}} &
%   \multicolumn{1}{c|}{} &
%    &
%   \multirow{-2}{*}{ImageNet-1k~\cite{5206848}} &
%   \multirow{-2}{*}{224.7} &
%   - &
%   13.1$\times$ \\ \cline{5-5} \cline{9-11} 
%  &
%    &
%    &
%   \multicolumn{1}{c|}{\begin{tabular}[c]{@{}c@{}}AGX Orin\textsuperscript{\\ $\blacktriangle$}\end{tabular}} &
%    &
%   \multicolumn{1}{c|}{} &
%    &
%    &
%    &
%   - &
%   14.92$\times$ \\ \cline{10-11} 
% \multirow{-4}{*}{EQ-ViT~\cite{10745859}} &
%   \multirow{-4}{*}{\cmark} &
%   \multirow{-4}{*}{DeiT-T} &
%   \multicolumn{1}{c|}{A100\textsuperscript{$\spadesuit$}} &
%   \multirow{-2}{*}{VEK280\textsuperscript{$\blacklozenge$}} &
%   \multicolumn{1}{c|}{\multirow{-4}{*}{\begin{tabular}[c]{@{}c@{}}Kernel-level \\ profiling\end{tabular}}} &
%   \multirow{-4}{*}{\begin{tabular}[c]{@{}c@{}}Spatial and \\ heterogeneous \\ accelerators\end{tabular}} &
%   \multirow{-2}{*}{Cifar-100~\cite{cifar10}} &
%   \multirow{-2}{*}{427.8} &
%   - &
%   3.38$\times$ \\ \hline
%  &
%    &
%    &
%   \multicolumn{1}{c|}{} &
%    &
%   \multicolumn{1}{c|}{} &
%    &
%   PASCAL~\cite{everingham2010pascal} &
%   0.690\textsuperscript{*} &
%   1217.4 &
%   - \\ \cline{9-11} 
% \multirow{-2}{*}{M\textsuperscript{3}ViT~\cite{fan2022m3vit}} &
%   \multirow{-2}{*}{\cmark} &
%   \multirow{-2}{*}{ViT-S/T} &
%   \multicolumn{1}{c|}{\multirow{-2}{*}{RTX 8000\textsuperscript{$\spadesuit$}}} &
%   \multirow{-2}{*}{ZCU102\textsuperscript{$\clubsuit$}} &
%   \multicolumn{1}{c|}{\multirow{-2}{*}{\begin{tabular}[c]{@{}c@{}}Mixture of expert\\  (MoE)\end{tabular}}} &
%   \multirow{-2}{*}{\begin{tabular}[c]{@{}c@{}}Computing MoE\\  expert-by-expert\end{tabular}} &
%   NYUD-v2~\cite{silberman2012indoor} &
%   0.845\textsuperscript{*} &
%   1183.4 &
%    \\ \hline
%  &
%    &
%    &
%   \multicolumn{1}{c|}{Jetson Xavier\textsuperscript{$\blacktriangle$}} &
%    &
%   \multicolumn{1}{c|}{} &
%    &
%   ImageNet-1k~\cite{5206848} &
%    &
%    &
%   5.6$\times$ \\ \cline{11-11} 
% \multirow{-2}{*}{ViTCoD~\cite{you2023vitcod}} &
%   \multirow{-2}{*}{\cmark} &
%   \multirow{-2}{*}{DeiT-B/S/T} &
%   \multicolumn{1}{c|}{CPU\textsuperscript{$\bigstar$}} &
%   \multirow{-2}{*}{ASIC} &
%   \multicolumn{1}{c|}{\multirow{-2}{*}{\begin{tabular}[c]{@{}c@{}}Prunes \& polarizes \\ the attention maps\end{tabular}}} &
%   \multirow{-2}{*}{\begin{tabular}[c]{@{}c@{}}On-chip encoder  \\ \& decoder engines\end{tabular}} &
%   Human3.6M~\cite{ionescu2013human3} &
%   \multirow{-2}{*}{-} &
%   \multirow{-2}{*}{-} &
%   33.8$\times$ \\ \hline
% SOLE~\cite{wang2023sole} &
%   \xmark &
%   DeiT-T &
%   \multicolumn{1}{c|}{2080Ti\textsuperscript{$\spadesuit$}} &
%   ASIC &
%   \multicolumn{1}{c|}{\begin{tabular}[c]{@{}c@{}}E2Softmax \& \\ AILayerNorm\end{tabular}} &
%   \begin{tabular}[c]{@{}c@{}}Custom hardware\\  unit\end{tabular} &
%   ImageNet-1k~\cite{5206848} &
%   - &
%   - &
%   57.5$\times$ \\ \hline
% \end{tabular}%
% }
% \vspace{-6mm}
% \end{table}
% Please add the following required packages to your document preamble:
% \usepackage{multirow}
% \usepackage{graphicx}
% Please add the following required packages to your document preamble:
% \usepackage{multirow}
% \usepackage{graphicx}
% \begin{table}[]
% \centering
% \caption{The overview of current accelerating techniques for ViT on edge devices. 
% In the baseline models, \textbf{B} denotes Base; \textbf{S} denotes Small, and \textbf{T} denotes Tiny versions. 
% Five types of hardware devices are used in existing hardware-software co-design frameworks: 
% GPU ($\spadesuit$), EdgeGPU ($\blacktriangle$), CPU ($\bigstar$), FPGA ($\clubsuit$), and AMD Versal Adaptive Compute Acceleration (ACAP) ($\blacklozenge$). 
% }
% \label{hwswco}
% \renewcommand{\arraystretch}{0.9}
% \resizebox{\columnwidth}{!}{%
% \begin{tabular}{c|c|c|c|cc|cc}
% \hline
% \multirow{2}{*}{\textbf{Approaches}} &
%   \multirow{2}{*}{\textbf{Framework}} &
%   \multirow{2}{*}{\textbf{Retrain}} &
%   \multirow{2}{*}{\textbf{\begin{tabular}[c]{@{}c@{}}Baseline\\ Models\end{tabular}}} &
%   \multicolumn{2}{c|}{\textbf{Hardware device}} &
%   \multicolumn{2}{c}{\textbf{Key optimization}} \\ \cline{5-8} 
%  &
%    &
%    &
%    &
%   \multicolumn{1}{c|}{\textbf{Baseline}} &
%   \textbf{Experiment} &
%   \multicolumn{1}{c|}{\textbf{Software}} &
%   \textbf{Hardware} \\ \hline
% \multirow{13}{*}{SW-HW co-design} &
%   \multirow{2}{*}{VAQF~\cite{sun2022vaqf}} &
%   \multirow{2}{*}{\xmark} &
%   \multirow{2}{*}{DeiT-B/S/T} &
%   \multicolumn{1}{c|}{Intel i7-9800X\textsuperscript{$\bigstar$}} &
%   \multirow{2}{*}{ZCU102\textsuperscript{$\clubsuit$}} &
%   \multicolumn{1}{c|}{\multirow{2}{*}{Quantization}} &
%   \multirow{2}{*}{\begin{tabular}[c]{@{}c@{}}Traditional\\ resources\end{tabular}} \\
%  &
%    &
%    &
%    &
%   \multicolumn{1}{c|}{TITAN RTX\textsuperscript{$\spadesuit$}} &
%    &
%   \multicolumn{1}{c|}{} &
%    \\ \cline{2-8} 
%  &
%   Auto-ViT-Acc ~\cite{lit2022auto} &
%   \xmark &
%   DeiT-B/S/T &
%   \multicolumn{1}{c|}{A100\textsuperscript{$\spadesuit$}} &
%   ZCU102 \textsuperscript{$\clubsuit$} &
%   \multicolumn{1}{c|}{\begin{tabular}[c]{@{}c@{}}Mixed \\ quantization\end{tabular}} &
%   \begin{tabular}[c]{@{}c@{}}Traditional\\ resources\end{tabular} \\ \cline{2-8} 
%  &
%   HeatViT~\cite{dong2023heatvit} &
%   \cmark &
%   DeiT-B/S/T &
%   \multicolumn{1}{c|}{A100\textsuperscript{$\spadesuit$}} &
%   ZCU102 \textsuperscript{$\clubsuit$} &
%   \multicolumn{1}{c|}{\begin{tabular}[c]{@{}c@{}}Adaptive \\ token pruning\end{tabular}} &
%   \begin{tabular}[c]{@{}c@{}}Motivated from\\ ~\cite{lit2022auto}\end{tabular} \\ \cline{2-8} 
%  &
%   \multirow{4}{*}{EQ-ViT~\cite{10745859}} &
%   \multirow{4}{*}{\cmark} &
%   \multirow{4}{*}{DeiT-T} &
%   \multicolumn{1}{c|}{ZCU102\textsuperscript{$\clubsuit$}} &
%   \multirow{2}{*}{VCK190\textsuperscript{$\blacklozenge$}} &
%   \multicolumn{1}{c|}{\multirow{4}{*}{\begin{tabular}[c]{@{}c@{}}Kernel-level \\ profiling\end{tabular}}} &
%   \multirow{4}{*}{\begin{tabular}[c]{@{}c@{}}Spatial and \\ heterogeneous \\ accelerators\end{tabular}} \\
%  &
%    &
%    &
%    &
%   \multicolumn{1}{c|}{U250\textsuperscript{$\clubsuit$}} &
%    &
%   \multicolumn{1}{c|}{} &
%    \\
%  &
%    &
%    &
%    &
%   \multicolumn{1}{c|}{\begin{tabular}[c]{@{}c@{}}AGX Orin\textsuperscript{\\ $\blacktriangle$}\end{tabular}} &
%   \multirow{2}{*}{VEK280\textsuperscript{$\blacklozenge$}} &
%   \multicolumn{1}{c|}{} &
%    \\
%  &
%    &
%    &
%    &
%   \multicolumn{1}{c|}{A100\textsuperscript{$\spadesuit$}} &
%    &
%   \multicolumn{1}{c|}{} &
%    \\ \cline{2-8} 
%  &
%   \multirow{2}{*}{M\textsuperscript{3}ViT~\cite{fan2022m3vit}} &
%   \multirow{2}{*}{\cmark} &
%   \multirow{2}{*}{ViT-S/T} &
%   \multicolumn{1}{c|}{\multirow{2}{*}{RTX 8000\textsuperscript{$\spadesuit$}}} &
%   \multirow{2}{*}{ZCU102\textsuperscript{$\clubsuit$}} &
%   \multicolumn{1}{c|}{\multirow{2}{*}{\begin{tabular}[c]{@{}c@{}}Mixture of expert\\  (MoE)\end{tabular}}} &
%   \multirow{2}{*}{\begin{tabular}[c]{@{}c@{}}Computing MoE\\  expert-by-expert\end{tabular}} \\
%  &
%    &
%    &
%    &
%   \multicolumn{1}{c|}{} &
%    &
%   \multicolumn{1}{c|}{} &
%    \\ \cline{2-8} 
%  &
%   \multirow{2}{*}{ViTCoD~\cite{you2023vitcod}} &
%   \multirow{2}{*}{\cmark} &
%   \multirow{2}{*}{DeiT-B/S/T} &
%   \multicolumn{1}{c|}{Jetson Xavier\textsuperscript{$\blacktriangle$}} &
%   \multirow{2}{*}{ASIC (28nm)} &
%   \multicolumn{1}{c|}{\multirow{2}{*}{\begin{tabular}[c]{@{}c@{}}Prunes \& polarizes \\ the attention maps\end{tabular}}} &
%   \multirow{2}{*}{\begin{tabular}[c]{@{}c@{}}On-chip encoder  \\ \& decoder engines\end{tabular}} \\
%  &
%    &
%    &
%    &
%   \multicolumn{1}{c|}{CPU\textsuperscript{$\bigstar$}} &
%    &
%   \multicolumn{1}{c|}{} &
%    \\ \cline{2-8} 
%  &
%   SOLE~\cite{wang2023sole} &
%   \xmark &
%   DeiT-T &
%   \multicolumn{1}{c|}{2080Ti\textsuperscript{$\spadesuit$}} &
%   ASIC (28nm) &
%   \multicolumn{1}{c|}{\begin{tabular}[c]{@{}c@{}}E2Softmax \& \\ AILayerNorm\end{tabular}} &
%   \begin{tabular}[c]{@{}c@{}}Custom hardware\\  unit\end{tabular} \\ \hline
% \multirow{3}{*}{Only HW} &
%   \multirow{2}{*}{ViA~\cite{wang2022via}} &
%   \multirow{2}{*}{\xmark} &
%   \multirow{2}{*}{Swin-T} &
%   \multicolumn{1}{c|}{Intel i7-5930X\textsuperscript{$\bigstar$}} &
%   \multirow{2}{*}{Alveo U50\textsuperscript{$\clubsuit$}} &
%   \multicolumn{1}{c|}{\multirow{2}{*}{-}} &
%   \multirow{2}{*}{\begin{tabular}[c]{@{}c@{}}Multi kernel parallelism\\ with half mappiing method\end{tabular}} \\
%  &
%    &
%    &
%    &
%   \multicolumn{1}{c|}{V100\textsuperscript{$\spadesuit$}} &
%    &
%   \multicolumn{1}{c|}{} &
%    \\ \cline{2-8} 
%  &
%   ViTA ~\cite{nag2023vita} &
%   \xmark &
%   DeiT-B/S/T &
%   \multicolumn{1}{c|}{ASIC (40nm)} &
%   Zynq ZC7020\textsuperscript{$\clubsuit$} &
%   \multicolumn{1}{c|}{-} &
%   \begin{tabular}[c]{@{}c@{}}Head level pipeline\\ \& Inter-layer MLP\end{tabular} \\ \hline
% \multirow{3}{*}{Other Approaches} &
%   ED-ViT~\cite{liu2024ed} &
%   \cmark &
%   ViT-B/S/T &
%   \multicolumn{1}{c|}{-} &
%   \begin{tabular}[c]{@{}c@{}}(1-10) Raspberry Pi-4B\\ RTX 4090\textsuperscript{$\spadesuit$}\end{tabular} &
%   \multicolumn{1}{c|}{-} &
%   \begin{tabular}[c]{@{}c@{}}Distributed edge devices\\ for deploying submodels\end{tabular} \\ \cline{2-8} 
%  &
%   \multirow{2}{*}{COSA Plus~\cite{10612833}} &
%   \multirow{2}{*}{\xmark} &
%   \multirow{2}{*}{ViT-B} &
%   \multicolumn{1}{c|}{RTX 3090\textsuperscript{$\spadesuit$}} &
%   \multirow{2}{*}{XCVU13P \textsuperscript{$\clubsuit$}} &
%   \multicolumn{1}{c|}{\multirow{2}{*}{-}} &
%   \multirow{2}{*}{\begin{tabular}[c]{@{}c@{}}Systolic array with\\ optimized dataflow\end{tabular}} \\ \cline{5-5}
%  &
%    &
%    &
%    &
%   \multicolumn{1}{c|}{6226R server CPU\textsuperscript{$\bigstar$}} &
%    &
%   \multicolumn{1}{c|}{} &
%    \\ \hline
% \end{tabular}%
% }
% \end{table}
% Please add the following required packages to your document preamble:
% \usepackage{multirow}
% \usepackage{graphicx}
\begin{table}[!htb]
\centering
\caption{The overview of current accelerating techniques for ViT on edge devices. 
In the baseline models, \textbf{B} denotes Base; \textbf{S} denotes Small, and \textbf{T} denotes Tiny versions. 
Five types of hardware devices are used in existing hardware-software co-design frameworks: 
GPU ($\spadesuit$), EdgeGPU ($\blacktriangle$), CPU ($\bigstar$), FPGA ($\clubsuit$), and AMD Versal Adaptive Compute Acceleration (ACAP) ($\blacklozenge$).}
\label{hwswco}
\renewcommand{\arraystretch}{0.9}
\resizebox{\columnwidth}{!}{%
\begin{tabular}{c|c|c|c|cc|cc}
\hline
\multirow{2}{*}{\textbf{Approaches}} &
  \multirow{2}{*}{\textbf{Framework}} &
  \multirow{2}{*}{\textbf{Retrain}} &
  \multirow{2}{*}{\textbf{\begin{tabular}[c]{@{}c@{}}Baseline\\ Models\end{tabular}}} &
  \multicolumn{2}{c|}{\textbf{Hardware device}} &
  \multicolumn{2}{c}{\textbf{Key optimization}} \\ \cline{5-8} 
 &
   &
   &
   &
  \multicolumn{1}{c|}{\textbf{Baseline}} &
  \textbf{Experiment} &
  \multicolumn{1}{c|}{\textbf{Software}} &
  \textbf{Hardware} \\ \hline
\multirow{13}{*}{SW-HW co-design} &
  \multirow{2}{*}{VAQF~\cite{sun2022vaqf}} &
  \multirow{2}{*}{\xmark} &
  \multirow{2}{*}{DeiT-B/S/T} &
  \multicolumn{1}{c|}{Intel i7-9800X\textsuperscript{$\bigstar$}} &
  \multirow{2}{*}{ZCU102\textsuperscript{$\clubsuit$}} &
  \multicolumn{1}{c|}{\multirow{2}{*}{Quantization}} &
  \multirow{2}{*}{\begin{tabular}[c]{@{}c@{}}Traditional\\ resources\end{tabular}} \\
 &
   &
   &
   &
  \multicolumn{1}{c|}{TITAN RTX\textsuperscript{$\spadesuit$}} &
   &
  \multicolumn{1}{c|}{} &
   \\ \cline{2-8} 
 &
  Auto-ViT-Acc ~\cite{lit2022auto} &
  \xmark &
  DeiT-B/S/T &
  \multicolumn{1}{c|}{A100\textsuperscript{$\spadesuit$}} &
  ZCU102 \textsuperscript{$\clubsuit$} &
  \multicolumn{1}{c|}{\begin{tabular}[c]{@{}c@{}}Mixed \\ quantization\end{tabular}} &
  \begin{tabular}[c]{@{}c@{}}Traditional\\ resources\end{tabular} \\ \cline{2-8} 
&
  HeatViT~\cite{dong2023heatvit} &
  \cmark &
  DeiT-B/S/T &
  \multicolumn{1}{c|}{A100\textsuperscript{$\spadesuit$}} &
  ZCU102 \textsuperscript{$\clubsuit$} &
  \multicolumn{1}{c|}{\begin{tabular}[c]{@{}c@{}}Adaptive \\ token pruning\end{tabular}} &
  \begin{tabular}[c]{@{}c@{}}Motivated from\\ ~\cite{lit2022auto}\end{tabular} \\ \cline{2-8} 
 &
  \multirow{4}{*}{EQ-ViT~\cite{10745859}} &
  \multirow{4}{*}{\cmark} &
  \multirow{4}{*}{DeiT-T} &
  \multicolumn{1}{c|}{ZCU102\textsuperscript{$\clubsuit$}} &
  \multirow{2}{*}{VCK190\textsuperscript{$\blacklozenge$}} &
  \multicolumn{1}{c|}{\multirow{4}{*}{\begin{tabular}[c]{@{}c@{}}Kernel-level\\ profiling\end{tabular}}} &
  \multirow{4}{*}{\begin{tabular}[c]{@{}c@{}}Spatial \& heterogeneous \\ accelerators\end{tabular}} \\
 &
   &
   &
   &
  \multicolumn{1}{c|}{U250\textsuperscript{$\clubsuit$}} &
   &
  \multicolumn{1}{c|}{} &
   \\
   &
   &
   &
   &
  \multicolumn{1}{c|}{AGX Orin\textsuperscript{$\blacktriangle$}} &
  \multirow{2}{*}{VEK280\textsuperscript{$\blacklozenge$}} &
  \multicolumn{1}{c|}{} &
   \\
 &
   &
   &
   &
  \multicolumn{1}{c|}{A100\textsuperscript{$\spadesuit$}} &
   &
  \multicolumn{1}{c|}{} &
   \\ \cline{2-8} 
 &
  \multirow{2}{*}{M\textsuperscript{3}ViT~\cite{fan2022m3vit}} &
  \multirow{2}{*}{\cmark} &
  \multirow{2}{*}{ViT-S/T} &
  \multicolumn{1}{c|}{\multirow{2}{*}{RTX 8000\textsuperscript{$\spadesuit$}}} &
  \multirow{2}{*}{ZCU102\textsuperscript{$\clubsuit$}} &
  \multicolumn{1}{c|}{\multirow{2}{*}{\begin{tabular}[c]{@{}c@{}}Mixture of expert\\ (MoE)\end{tabular}}} &
  \multirow{2}{*}{\begin{tabular}[c]{@{}c@{}}Computing MoE\\ expert-by-expert\end{tabular}} \\
 &
   &
   &
   &
  \multicolumn{1}{c|}{} &
   &
  \multicolumn{1}{c|}{} &
   \\ \cline{2-8} 
 &
  \multirow{2}{*}{ViTCoD~\cite{you2023vitcod}} &
  \multirow{2}{*}{\cmark} &
  \multirow{2}{*}{DeiT-B/S/T} &
  \multicolumn{1}{c|}{Jetson Xavier\textsuperscript{$\blacktriangle$}} &
  \multirow{2}{*}{ASIC (28nm)} &
  \multicolumn{1}{c|}{\multirow{2}{*}{\begin{tabular}[c]{@{}c@{}}Prunes \& polarizes \\ the attention maps\end{tabular}}} &
  \multirow{2}{*}{\begin{tabular}[c]{@{}c@{}}On-chip encoder \\ \& decoder engines\end{tabular}} \\
 &
   &
   &
   &
  \multicolumn{1}{c|}{CPU\textsuperscript{$\bigstar$}} &
   &
  \multicolumn{1}{c|}{} &
   \\ \cline{2-8} 
 &
  SOLE~\cite{wang2023sole} &
  \xmark &
  DeiT-T &
  \multicolumn{1}{c|}{2080Ti\textsuperscript{$\spadesuit$}} &
  ASIC (28nm) &
  \multicolumn{1}{c|}{\begin{tabular}[c]{@{}c@{}}E2Softmax \& \\ AILayerNorm\end{tabular}} &
  \begin{tabular}[c]{@{}c@{}}Custom hardware\\ unit\end{tabular} \\ \hline
\multirow{3}{*}{Pure HW} &
  \multirow{2}{*}{ViA~\cite{wang2022via}} &
  \multirow{2}{*}{\xmark} &
  \multirow{2}{*}{Swin-T} &
  \multicolumn{1}{c|}{Intel i7-5930X\textsuperscript{$\bigstar$}} &
  \multirow{2}{*}{Alveo U50\textsuperscript{$\clubsuit$}} &
  \multicolumn{1}{c|}{\multirow{2}{*}{-}} &
  \multirow{2}{*}{\begin{tabular}[c]{@{}c@{}}Multi kernel parallelism\\ with half mapping method\end{tabular}} \\
 &
   &
   &
   &
  \multicolumn{1}{c|}{V100\textsuperscript{$\spadesuit$}} &
   &
  \multicolumn{1}{c|}{} &
   \\ \cline{2-8} 
 &
  ViTA ~\cite{nag2023vita} &
  \xmark &
  DeiT-B/S/T &
  \multicolumn{1}{c|}{ASIC (40nm)} &
  Zynq ZC7020\textsuperscript{$\clubsuit$} &
  \multicolumn{1}{c|}{-} &
  \begin{tabular}[c]{@{}c@{}}Head level pipeline\\ \& Inter-layer MLP\end{tabular} \\ \hline
\multirow{3}{*}{Other Techniques} &
  ED-ViT~\cite{liu2024ed} &
  \cmark &
  ViT-B/S/T &
  \multicolumn{1}{c|}{-} &
  \begin{tabular}[c]{@{}c@{}}Raspberry Pi-4B\\ RTX 4090\textsuperscript{$\spadesuit$}\end{tabular} &
  \multicolumn{1}{c|}{-} &
  \begin{tabular}[c]{@{}c@{}}Distributed edge devices\\ for deploying submodels\end{tabular} \\ \cline{2-8} 
 &
  \multirow{2}{*}{COSA Plus~\cite{10612833}} &
  \multirow{2}{*}{\xmark} &
  \multirow{2}{*}{ViT-B} &
  \multicolumn{1}{c|}{RTX 3090\textsuperscript{$\spadesuit$}} &
  \multirow{2}{*}{XCVU13P \textsuperscript{$\clubsuit$}} &
  \multicolumn{1}{c|}{\multirow{2}{*}{-}} &
  \multirow{2}{*}{\begin{tabular}[c]{@{}c@{}}Systolic array with\\ optimized dataflow\end{tabular}} \\ \cline{5-5}
   &
   &
   &
   &
  \multicolumn{1}{c|}{6226R server CPU\textsuperscript{$\bigstar$}} &
   &
  \multicolumn{1}{c|}{} &
   \\ \hline
\end{tabular}
}
\vspace{-3mm}
\end{table}
\subsubsection{Software Optimization in Software-Hardware Co-design}\hfill \\
Recent advancements in software part in SW-HW co-design for ViT acceleration encompass a range of optimization techniques, including quantization-based acceleration~\cite{sun2022vaqf,lit2022auto,10745859,dong2023heatvit}, sparse and adaptive attention mechanisms~\cite{you2023vitcod,dong2023heatvit}, adding mixture of experts (MoE) layers~\cite{fan2022m3vit}, analyzing kernel profiling and execution scheduling~\cite{10745859}, custom hardware softmax and LayerNorm to replace traditional softmax and LayerNorm~\cite{wang2023sole,stevens2021softermax}.\\ 

\noindent \textbf{Quantization-Based Software Acceleration} VAQF~\cite{sun2022vaqf} utilized binary quantization for weights and low-precision for the activations in the software part. VAQF automatically outputs the efficient quantization parameters based on the model structure and the expected frame per second (FPS) to meet the hardware specifications. The primary purpose of this work was to achieve high throughput on hardware while maintaining model accuracy. Moreover, Li et al.~\cite{lit2022auto} used a mixed-scheme (fixed+PoT) ViT quantization algorithm that can fully leverage heterogeneous FPGA resources for a target FPS. Both frameworks utilized targeted FPS as their input to achieve maximum hardware efficiency during inference. Additionally, both VAQF and AutoViT-Acc~\cite{lit2022auto} utilized PTQ methods as quantization. EQ-ViT~\cite{10745859} combined latency and accuracy requirements to decide the final quantization strategy leveraging the QAT approach. Additionally, EQ-ViT implemented a profiling-based execution scheduler that dynamically allocates workloads across hardware accelerators.\\

\noindent \textbf{Pruning-Based Software Acceleration} ViT's self-attention has quadratic complexity concerning input sequence length, leading to high memory bandwidth consumption. Sparse attention mechanisms aim to reduce redundant computation by focusing on pruning. For instance, ViTCoD~\cite{you2023vitcod} efficiently employed structured pruned and polarized the attention maps to remove redundant attention scores, creating a more memory-efficient execution. Moreover,  HeatViT~\cite{dong2023heatvit} utilized image-adaptive token pruning, identifying and removing unimportant tokens before transformer blocks using a multi-token selector, dynamically reducing computational complexity. The proposed method is highly inspired by SP-ViT ~\cite{kong2022spvit} and uses a similar token selector like SP-ViT.\\

\noindent \textbf{Other Approaches} Additionally, MoE improves model efficiency by activating only the most relevant expert networks per input instance, reducing unnecessary computations. One of the first studies named M\textsuperscript{3}ViT~\cite{fan2022m3vit} implemented MoE layers where a router dynamically selects the appropriate experts for processing. In this work, the authors conveyed training dynamics to balance large capacity and efficiency by selecting only a subset of experts using the MoE router. Beyond traditional quantization and attention optimizations, some co-design approaches target the computational bottlenecks of softmax and LayerNorm. SOLE~\cite{wang2023sole} proposed E2Softmax and AILayerNorm, hardware-aware modifications for non-linear operations that replace FP32 with integer-only approximation. This integer-only computation improved the latency significantly during the inference.
\subsubsection{Hardware Optimization in Software-Hardware Co-design}\hfill \\
Hardware optimization plays a crucial role in the software-hardware co-design of ViTs, ensuring efficient execution across different accelerators. Key techniques include C++ based hardware descriptions, high-level synthesis (HLS), and accelerator bitstream generation~\cite{sun2022vaqf,lit2022auto,dong2023heatvit}. Additionally, frameworks leverage AI engine (AIE) kernels~\cite{10745859} and custom hardware units (e.g., SOLE) to optimize execution for edge deployment.\\

\noindent \textbf{FPGA-Based Hardware Acceleration} FPGA is the pioneer for hardware accelerating strategies because of its reconfigurable characteristics. VAQF~\cite{sun2022vaqf} adopted the quantization schemes from the software part and applied them to the accelerator on the hardware side. Figure~\ref{fig:vaqf} illustrates the overview of the VAQF framework. The accelerator's C++ description was synthesized using the Vivado HLS tool. Initial accelerator parameters focused on maximizing parallelism, but there were adjustments due to Vivado's placement or routing problems. Successful implementations produced a bitstream file for FPGA deployment. Moreover, as illustrated in Figure~\ref{fig:aut-vcc}, Auto-ViT Acc~\cite{lit2022auto} first used the "FPGA Resource Utilization Modeling" module to give performance analysis and calculate the FPS of the FPGA ViT accelerator in software part. Inspired by VAQF, the authors implemented the FPGA accelerator using a C++ hardware description, synthesized through Vitis HLS to generate the final accelerator bitstream. 

In ViTs, the computational bottleneck often arises from General Matrix Multiply (GEMM) operations, which form the core of self-attention and feed-forward layers. HeatViT~\cite{dong2023heatvit} optimizes ViT execution by dynamically selecting tokens and loading each layer from off-chip DDR memory to on-chip buffers before processing via the GEMM engine. This approach, inspired by \cite{lit2022auto}, minimizes redundant computations and improves memory efficiency. The proposed HeatViT addressed two main challenges of hardware implementation in their proposed architecture, as follows.
\begin{enumerate}
\item The GEMM loop tiling must be adjusted to factor in an extra dimension from multi-head parallelism.
\item ViTs have more non-linear operations than CNNs; these must be optimized for better quantization and efficient hardware execution while maintaining accuracy.
\end{enumerate}

\begin{figure}[]
  \centering
  % First image
  \begin{minipage}[b]{0.47\linewidth}
    \includegraphics[width=\linewidth]{assets/VAQF_new_up.png}
   \caption{Overflow of VAQF acclerator~\cite{sun2022vaqf}. Using different colors in the architecture differentiates between types of processes within the overall workflow. The light gray boxes represent settings that are input/output to the process. The light blue boxes denote active processing steps or software tools within the workflow, like Vivado HLS. The lavender box signifies a platform/library used in the process, like "PyTorch". The light purple box indicates decision-making points or critical stages in the architecture.}
  \label{fig:vaqf}
  \end{minipage}
  \hspace{0.5cm} % Space between the images
  % Second image
  \begin{minipage}[b]{0.47\linewidth}
    \includegraphics[width=\linewidth]{assets/auto_vcc.png}
    % \includegraphics[scale=0.5]{assets/heatvit.png}
  \caption{The overview of Auto-ViT-Acc framework ~\cite{lit2022auto}. The "FPGA resource utilization modeling" was utilized for performance analysis and the estimated FPS rate for ViT accelerator with assigned bit-width for mixed schemes and lessening the bit-width until achieving the target FPS. The proposed mixed-scheme quantization then utilized mixed ratio (k\textsubscript{pot}) results to implement on FPGA through "C++ Description for accelerator," "Vitis HLS" and "Accelerator bitstream"}
  \label{fig:aut-vcc}
  \end{minipage}
\end{figure}

\noindent \textbf{MoE Execution for Hardware Efficiency} While MoE optimization techniques dynamically select experts, leading to an unpredictable computing pattern that makes hardware execution difficult. M\textsuperscript{3}ViT~\cite{fan2022m3vit} reordered computations to process tokens expert-by-expert rather than token-by-token, reducing irregular memory access and improving hardware parallelism. However, frequent off-chip memory accesses in MoE layers introduce a latency bottleneck. The authors utilized a ping-pong buffering technique for continuous processing without memory stalls, where one buffer fetches expert weights while another buffer performs computations. Additionally, the per-expert token queueing system groups tokens per expert, limiting the underutilization of compute units due to various expert demands.\\

\noindent \textbf{Optimizing Non-Linear Operations with Custom Hardware Units} Recent studies, such as EQ-ViT, separated matrix multiply (MM) and non-MM by efficiently mapping batch MM (BMM) and convolutions to AIE vector cores. Memory-bounded and non-linear operations are executed within the FPGA's programmable element (PE). Additionally, EQ-ViT leveraged fine-grained pipeline execution to overlap computation with memory transfers,  maximizing resource utilization. A key strength of this framework lies in its hardware mapping methodology, where execution is formulated as a mixed-integer programming (MIP) optimization problem, ensuring that latency and resource constraints are satisfied while maximizing throughput. Likewise, SOLE~\cite{10745859} designed separate units for the proposed E2Softmax  and AILayerNorm to perform non-linear operations efficiently. The E2Softmax unit included Log2Exp and an approximate Log-based divider that is implemented in a LUT-free and multiplication-free manner. Additionally, the AILayerNorm unit operates in two stages: the first stage performs statistical calculations, while the second stage applies the affine transformation. Similar to M\textsuperscript{3}ViT~\cite{fan2022m3vit}  also utilized ping pong buffer to pipeline the AILayerNorm unit. \\

\noindent \textbf{Pure Hardware Accelerators} Although most ViT acceleration techniques rely on SW-HW co-design—where software optimization plays a crucial role in achieving high efficiency and accuracy—the potential of pure hardware optimization has gained attention in recent studies. A recent study named ViA~\cite{wang2022via} addressed issues during data and computations flow through the layers in ViT. The authors utilized a partitioning strategy to reduce the impact of data locality in the image and enhance the efficiency of computation and memory access. Additionally, by examining the computing flow of the ViT, the authors also utilized the half-layer mapping and throughput analysis to lessen the effects of path dependency due to the shortcut mechanism and to maximize the use of hardware resources for efficient transformer execution. The study developed two reuse processing engines with an internal stream, distinguishing them from previous overlaps or stream design patterns drawing from the optimization strategies.

Moreover, ViTA~\cite{nag2023vita} used two sets of MAC units to minimize the off-chip memory accesses. The first set of MAC units representing the hidden layers was broadcasted to the second set of MAC units through a non-linear activation function. That broadcasting approach helped to compute the partial products corresponding to the output layer. The authors allocated these resources to maintain the pipeline technique as if the hidden layer value computations and the output layer partial product computation took equal time. This approach enabled the integration of several mainstream ViT models by only adjusting the configuration.\\

\noindent \textbf{Sparse Attention Optimization for ViTs}  Recent studies such as ViTCoD~\cite{you2023vitcod} introduced a sparser engine to process the sparse attention metrics. The authors structured the multiply-accumulate (MAC) into the encoder and decoder MAC lines to optimize the matrix multiplications.

\begin{table}[]
\centering
\caption{Compatibilities comparison of SOTA SW-HW co-design accelerating techniques. Here, GPP denotes general-purpose platforms such as CPU and GPU.}
\label{performance_analysis}
\renewcommand{\arraystretch}{0.8}
\resizebox{0.8\columnwidth}{!}{%
\begin{tabular}{c|c|c|c|c|c}
\hline
\multirow{2}{*}{\textbf{Framework}} &
  \multirow{2}{*}{\textbf{Baseline}} &
  \multirow{2}{*}{\textbf{\begin{tabular}[c]{@{}c@{}}Effort\\ Modulation\end{tabular}}} &
  \multirow{2}{*}{\textbf{\begin{tabular}[c]{@{}c@{}}Prediction\\ Mechanism\end{tabular}}} &
  \multirow{2}{*}{\textbf{\begin{tabular}[c]{@{}c@{}}Accuracy\\ Top1(\%)\end{tabular}}} &
  \multirow{2}{*}{\textbf{\begin{tabular}[c]{@{}c@{}}GPP\\ Compatible\end{tabular}}} \\
                               &        &                 &                &      &        \\ \hline
ViTCoD~\cite{you2023vitcod}    & DeiT-S & Constant        & Norm Score     & 78.1 & \xmark \\ \hline
HeatViT~\cite{dong2023heatvit} & DeiT-S & Constant        & Head level     & 79.1 & \xmark \\ \hline
PIVOT ~\cite{moitra2024pivot}  & DeiT-S & Input-aware     & Entropy Metric & 79.4 & \cmark \\ \hline
VAQF~\cite{sun2022vaqf}        & DeiT-S & Hardware-driven & Hardware-aware & 79.5 & \xmark \\ \hline
\end{tabular}%
}
\vspace{-3mm}
\end{table}
\subsubsection{Other Techniques}\hfill \\
 ED-ViT~\cite{liu2024ed} utilized distributed workloads approaches to deploy the ViT mode utilizing multiple tiny edge devices such as Raspberry Pi-4B. The authors partitioned the model into multiple submodels, mapping each submodel to a separate edge device. This distributed execution strategy enables ViTs to achieve efficiency comparable to single powerful edge accelerators like EdgeGPUs or FPGAs while leveraging cost-effective and scalable edge computing resources. Similarly, COSA Plus, proposed by Wang et al.~\cite{10612833}, capitalized on high inherent parallelism within ViT models by implementing a runtime-configurable hybrid dataflow strategy. This method dynamically switches between weight-stationary and output-stationary dataflows in a systolic array, optimizing the computational efficiency for matrix multiplications within the attention mechanism. COSA Plus enhances processing element (PE) utilization by adapting data movement patterns to workload variations.
% From our observations, most of the SW-HW co-design proposed for FPGA or ACAP design due to their reconfigurable nature where the authors baselined GPU, CPU, or other FPGA platforms. Table~\ref{hwswco} provides a comprehensive overview of the SOTA SW-HW co-design strategies for ViTs on various edge platforms. The analysis focuses on the hardware devices, optimization techniques, and performance metrics, including energy efficiency, throughput, and speedup.

% Recent studies, such as SOLE and ViTCoD, demonstrate a shift towards ASIC accelerators, which offer superior energy efficiency and computational speedup compared to FPGA and GPU-based techniques. For example, ViTCoD~\cite{you2023vitcod} achieves a 33.8$\times$ speedup using an ASIC-optimized sparse attention engine, outperforming HeatViT on the DeiT-Tiny (T) model. 

% Additionally, the quantization-based acceleration has shown significant improvements in energy efficiency. However, Table~\ref{hwswco} highlights a key challenge in MoE-based architectures such as M\textsuperscript{3}ViT~\cite{fan2022m3vit}: despite achieving high throughput (1183.4 GOPs), their energy efficiency remains moderate (0.845 FPS/W). This suggests MoE introduces irregular memory access patterns, leading to suboptimal hardware utilization. While ping-pong buffering is a well-established technique to mitigate memory stalls, further optimizations are required to exploit hardware parallelism and memory efficiency.

% Additionally, Table~\ref{hwswco} shows that ASIC designs dominate speedup, but FPGA offers better energy efficiency. This suggests that ASIC accelerators provide raw computational power, while FPGA solutions are more power-efficient but slightly lower in absolute speedup.
\subsubsection{Discussion}\hfill \\
From our observations, most of the SW-HW co-design proposed for FPGA or ACAP design due to their reconfigurable nature where the authors baselined GPU, CPU, or other FPGA platforms. Table~\ref{hwswco} provides a comprehensive overview of the SOTA SW-HW co-design strategies for ViTs on various edge platforms. The analysis focuses on the hardware devices and optimization techniques. A key observation from this table is that different hardware platforms support distinct optimization levels. While GPUs and EdgeGPUs are widely used due to their parallel processing capabilities, FPGAs and ACAP platforms provide customized acceleration, often resulting in lower latency and energy-efficient execution. 

VAQF and Auto-ViTAcc apply PTQ optimization to reduce precision while maintaining accuracy, while HeatViT and EQ-ViT optimize workload distribution through kernel-level techniques and adaptive token pruning.
Additionally, the quantization-based acceleration has shown significant improvements in energy efficiency. However, MoE introduces irregular memory access patterns, leading to suboptimal hardware utilization. While ping-pong buffering is a well-established technique to mitigate memory stalls, further optimizations are required to exploit hardware parallelism and memory efficiency.

Additionally, Table~\ref{hwswco} shows that ASIC designs dominate speedup, but FPGA offers better energy efficiency. This suggests that ASIC accelerators provide raw computational power, while FPGA solutions are more power-efficient but slightly lower in absolute speedup.

Most current acceleration techniques for ViTs are highly customized for specialized hardware, such as FPGAs and ASICs, limiting their deployment flexibility. Table~\ref{performance_analysis} highlights the trade-offs between accuracy, adaptability, and hardware compatibility in SW-HW co-design techniques for ViTs. However, few approaches support cross-platform compatibility, making their extension to diverse edge devices challenging. Among the surveyed techniques, PIVOT~\cite{moitra2024pivot} emerges as the most flexible solution, as it maintains high accuracy while being compatible with GPPs, unlike ViTCoD, HeatViT, and VAQF, which are hardware-specialized acceleration techniques. Additionally, Table~\ref{performance_analysis} highlights how different acceleration techniques prioritize computations, offering insights into their adaptability across various deployment scenarios.
\subsection{Performance Analysis for Accelerating Techniques}
This section provides a performance analysis of state-of-the-art (SOTA) accelerating techniques, focusing on key metrics such as power consumption, energy efficiency, resource utilization, and throughput.
% Please add the following required packages to your document preamble:
% \usepackage{multirow}
% \usepackage{graphicx}
\begin{table}[]
\centering
\caption{This table presents a comparative analysis of hardware performance for various Vision Transformer (ViT) acceleration techniques, deployed on FPGA (\textbf{$\clubsuit$}) and ACAP (\textbf{$\blacklozenge$}) platforms. Performance metrics include Frames Per Second (FPS), Energy Efficiency, and Throughput. The notation \textbf{$\intercal$} represents FPS measured as images per second, while \textbf{$\lozenge$} denotes energy efficiency in GOP/J. Additionally, \textbf{*} indicates power consumption measured in Watt-Seconds (W·S), providing deeper insights into the trade-offs between computation speed and energy usage.}
\label{Hardware_performance}
\renewcommand{\arraystretch}{0.9}
\resizebox{\columnwidth}{!}{%
\begin{tabular}{c|c|c|c|cccc|c|c|c|c|c}
\hline
\multirow{2}{*}{\textbf{Framework}} &
  \multirow{2}{*}{\textbf{Device}} &
  \multirow{2}{*}{\textbf{Precision}} &
  \multirow{2}{*}{\textbf{\begin{tabular}[c]{@{}c@{}}Frequency\\ (MHZ)\end{tabular}}} &
  \multicolumn{4}{c|}{\textbf{Resource Utilization}} &
  \multirow{2}{*}{\textbf{FPS}} &
  \multirow{2}{*}{\textbf{\begin{tabular}[c]{@{}c@{}}Power\\ (W)\end{tabular}}} &
  \multirow{2}{*}{\textbf{\begin{tabular}[c]{@{}c@{}}Energy \\ Efficiency (FPS/W)\end{tabular}}} &
  \multirow{2}{*}{\textbf{\begin{tabular}[c]{@{}c@{}}Throughput \\ (GOPs)\end{tabular}}} &
  \multirow{2}{*}{\textbf{\begin{tabular}[c]{@{}c@{}}Speedup \\ ($\uparrow$)\end{tabular}}} \\ \cline{5-8}
 &
   &
   &
   &
  \multicolumn{1}{c|}{\textbf{BRAM}} &
  \multicolumn{1}{c|}{\textbf{DSP}} &
  \multicolumn{1}{c|}{\textbf{KLUT}} &
  \textbf{KFF} &
   &
   &
   &
   &
   \\ \hline
VAQF~\cite{sun2022vaqf} &
  ZCU102\textsuperscript{$\clubsuit$} &
  W1A8 &
  150 &
  \multicolumn{1}{c|}{565.5} &
  \multicolumn{1}{c|}{1564} &
  \multicolumn{1}{c|}{143} &
  110 &
  24.8 &
  8.7 &
  2.85 &
  861.2 &
  - \\ \hline
ViA~\cite{wang2022via} &
  U50\textsuperscript{$\clubsuit$} &
  FP16 &
  300 &
  \multicolumn{1}{c|}{1002} &
  \multicolumn{1}{c|}{2420} &
  \multicolumn{1}{c|}{258} &
  257 &
  - &
  39 &
  7.94\textsuperscript{$\lozenge$} &
  309.6 &
  59.5$\times$ \\ \hline
ViTA ~\cite{nag2023vita} &
  ZC7020\textsuperscript{$\clubsuit$} &
  INT8 &
  150 &
  \multicolumn{1}{c|}{-} &
  \multicolumn{1}{c|}{-} &
  \multicolumn{1}{c|}{-} &
  - &
  8.7 &
  0.88 &
  3.13 &
  - &
  2$\times$ \\ \hline
Auto-ViT-Acc ~\cite{lit2022auto} &
  ZCU102\textsuperscript{$\clubsuit$} &
  W8A8+W4A8 &
  150 &
  \multicolumn{1}{c|}{-} &
  \multicolumn{1}{c|}{1556} &
  \multicolumn{1}{c|}{186} &
  - &
  34.0 &
  9.40 &
  3.66 &
  1181.5 &
  - \\ \hline
HeatViT ~\cite{dong2023heatvit} &
  ZCU102\textsuperscript{$\clubsuit$} &
  INT8 &
  150 &
  \multicolumn{1}{c|}{528.6} &
  \multicolumn{1}{c|}{2066} &
  \multicolumn{1}{c|}{161.4} &
  101.8 &
  11.4 &
  54.8 &
  4.83 &
  - &
  4.89$\times$ \\ \hline
EQ-ViT~\cite{10745859} &
  VCK190\textsuperscript{$\blacklozenge$} &
  W4A8 &
  - &
  \multicolumn{1}{c|}{16\textsuperscript{$\phi$}} &
  \multicolumn{1}{c|}{28\textsuperscript{$\phi$}} &
  \multicolumn{1}{c|}{6.5\textsuperscript{$\phi$}} &
  - &
  10695\textsuperscript{$\intercal$} &
  - &
  224.7 &
  - &
  - \\ \hline
Zhang et al.~\cite{zhang2024109} &
  XCZU9EG\textsuperscript{$\clubsuit$} &
  W8A8 &
  300 &
  \multicolumn{1}{c|}{283} &
  \multicolumn{1}{c|}{2147} &
  \multicolumn{1}{c|}{118} &
  139 &
  36.4 &
   &
  73.56\textsuperscript{$\lozenge$} &
  2330.2 &
  - \\ \hline
M\textsuperscript{3}ViT~\cite{fan2022m3vit} &
  ZCU104\textsuperscript{$\clubsuit$} &
  INT8 &
  300 &
  \multicolumn{1}{c|}{-} &
  \multicolumn{1}{c|}{-} &
  \multicolumn{1}{c|}{-} &
  - &
  84 &
  10 &
  0.690\textsuperscript{*} &
  1217.4 &
  - \\ \hline
ViTCoD~\cite{you2023vitcod} &
  ASIC &
  W8A8 &
  500 &
  \multicolumn{1}{c|}{} &
  \multicolumn{1}{c|}{} &
  \multicolumn{1}{c|}{} &
   &
  - &
  - &
  - &
  - &
  5.6$\times$ \\ \hline
SOLE~\cite{wang2023sole} &
  ASIC &
  INT8 + SOLE &
  - &
  \multicolumn{1}{c|}{-} &
  \multicolumn{1}{c|}{-} &
  \multicolumn{1}{c|}{-} &
  - &
  - &
  - &
  - &
  - &
  57.5$\times$ \\ \hline
ME-ViT~\cite{marino2023me} &
  U200\textsuperscript{$\clubsuit$} &
  W8A8 &
  300 &
  \multicolumn{1}{c|}{288} &
  \multicolumn{1}{c|}{1024} &
  \multicolumn{1}{c|}{192} &
  132 &
  94.13 &
  31.8 &
  4.15 &
  - &
  - \\ \hline
\end{tabular}%
}
\vspace{-3mm}
\end{table}
\subsubsection{Resource Utilization}\hfill \\ 
Resource utilization measures the hardware efficiency of various ViT acceleration techniques in terms of the use of block RAM (BRAM), digital signal processing units (DSP), Kilo lookup tables (KLUT), and Kilo flip-flops (KFF) from available resources. As illustrated in Table~\ref{Hardware_performance}, different acceleration techniques exhibit varying resource consumption patterns, reflecting their optimization strategies and deployment constraints.

From the observation of Table~\ref{Hardware_performance}, HeatViT (2066 DSPs)~\cite{dong2023heatvit} and Zhang et al. (2147 DSPs)~\cite{zhang2024109} demonstrate the highest DSP consumption, indicating their reliance on intensive parallel processing to accelerate transformer computations although both used different FPGA variants. In terms of on-chip memory usage, ViA (1002 BRAMs)~\cite{wang2022via} and VAQF (565.5 BRAMs)~\cite{sun2022vaqf} exhibit significant BRAM consumption, emphasizing a design strategy that prioritizes data locality to minimize off-chip memory access latency. In contrast, EQ-ViT (16 BRAMs)~\cite{10745859} utilizes remarkably low BRAM, likely due to its two-level optimization kernels—leveraging both single artificial intelligence engines (AIEs) and AIE array levels. Additionally, Auto-ViT-Acc~\cite{lit2022auto} (186 KLUTs) and HeatViT~\cite{dong2023heatvit} (161.4 KLUTs) indicate that they need to perform significant logical operations to implement their mixed precision quantization and token pruning to deploy on edge.
\subsubsection{Energy Efficiency}\hfill \\
Table~\ref{Hardware_performance} also indicates the energy efficiency of the ViT accelerating techniques. In the current studies, energy efficiency was measured in two ways: using throughput (GOP/J) and using FPS (FPS/watt). We also include the power consumption of the accelerating techniques for better accountability. It is perhaps difficult to conclude about energy efficiency when different edge targets are used for the CV task. From Table~\ref{Hardware_performance}, VAQF~\cite{sun2022vaqf}, Auto-ViT-Acc~\cite{lit2022auto}, and HeatViT~\cite{dong2023heatvit} utilized the same FPGA AMD ZCU102 board with the same number of resources and frequency. We can observe that VAQF outperforms Auto-ViT-Acc and HeatViT regarding energy and power usage. However, we were unable to find any energy comparison from the original paper for ASIC-based accelerating techniques (e.g., ViTCoD, SOLE).
\subsubsection{Throughput}\hfill \\
Table~\ref{Hardware_performance} illustrates the throughput of the accelerating techniques. We observe significant variations in throughput, influenced by factors such as hardware architecture, precision, and optimization strategies. Zhang et al. (2330.2 GOPs) achieve the highest throughput, leveraging an FPGA-based implementation with optimized parallel execution. Auto-ViT-Acc (1181.5 GOPs) and M³ViT (1217.4 GOPs) also report high throughput, suggesting effective hardware utilization. However, several studies, such as ViTCoD, do not report the throughput. Additionally, EQ-ViT and VAQF balance throughput with power efficiency, offering a more energy-efficient alternative.
\subsubsection{Accuracy}\hfill \\ Table~\ref{accuracy_performance} presents a comparative accuracy analysis across various ViT acceleration techniques on different edge platforms. To ensure a fair and structured comparison, we categorize our evaluation into four widely used ViT-based models: DeiT-Base, DeiT-Tiny, ViT-Base, and ViT-Small. While most acceleration techniques focus on classification tasks using the ImageNet-1K dataset~\cite{5206848}.

Among the methods analyzed, Auto-ViT-Acc achieves the highest Top-1 accuracy (81.8\%) on DeiT-Base, significantly surpassing VAQF (77.6\%) for classification tasks, demonstrating its effectiveness in preserving model accuracy while accelerating inference. For DeiT-Tiny, accuracy varies significantly across different methods: HeatViT (72.1\%), EQ-ViT (74.5\%), ViTCoD (70.0\%), and SOLE (71.07\%). Notably, EQ-ViT achieves the highest accuracy among these, highlighting the effectiveness of its attention-based optimizations. However, its energy consumption is significantly higher, indicating a trade-off between accuracy and efficiency.

M\textsuperscript{3}ViT\cite{fan2022m3vit} extends its evaluation to segmentation tasks on PASCAL-ContextNYUD-v2\cite{silberman2012indoor}. For segmentation tasks, M\textsuperscript{3}ViT delivers strong performance on PASCAL-Context (72.8 mIoU) but exhibits a noticeable decline on NYUD-v2 (45.6 mIoU), suggesting that its model compression techniques may be dataset-sensitive.

% % Please add the following required packages to your document preamble:
% % \usepackage{multirow}
% % \usepackage{graphicx}
% \begin{table}[]
% \centering
% \caption{Comparison of accuracy across different ViT acceleration techniques using DeiT-B (Base), DeiT-T (Tiny), ViT-B (Base), and ViT-S (Small) as baseline models. Results include Top-1 accuracy on ImageNet-1K and mean Intersection over Union (mIoU) for segmentation benchmarks.}
% \label{accuracy_performance}
% \small
% \renewcommand{\arraystretch}{0.2}
% \resizebox{\columnwidth}{!}{%
% \begin{tabular}{c|c|c|cc}
% \hline
% \multirow{2}{*}{\textbf{Framework}} &
%   \multirow{2}{*}{\textbf{Backbone}} &
%   \multirow{2}{*}{\textbf{Datasets}} &
%   \multicolumn{2}{c}{\textbf{Accuracy}} \\ \cline{4-5} 
%                                  &        &                                             & \multicolumn{1}{c|}{\textbf{Top 1(\%)}} & \textbf{mIoU} \\ \hline
% VAQF~\cite{sun2022vaqf}          & DeiT-B & \multirow{2}{*}{ImageNet-1K~\cite{5206848}} & \multicolumn{1}{c|}{77.6}      & -             \\
% Auto-ViT-Acc ~\cite{lit2022auto} & DeiT-B &                                             & \multicolumn{1}{c|}{81.8}      & -             \\ \hline
% HeatViT ~\cite{dong2023heatvit}  & DeiT-T & \multirow{4}{*}{ImageNet-1K~\cite{5206848}} & \multicolumn{1}{c|}{72.1}      & -             \\
% EQ-ViT~\cite{10745859}           & DeiT-T &                                             & \multicolumn{1}{c|}{74.5}      & -             \\
% ViTCoD~\cite{you2023vitcod}      & DeiT-T &                                             & \multicolumn{1}{c|}{70.0}      &               \\
% SOLE~\cite{wang2023sole}         & DeiT-T &                                             & \multicolumn{1}{c|}{71.07}     & -             \\ \hline
% Zhang et al.~\cite{zhang2024109} & ViT-B  & ImageNet-1K~\cite{5206848}                  & \multicolumn{1}{c|}{83.1}      & -             \\ \hline
% \multirow{2}{*}{M\textsuperscript{3}ViT~\cite{fan2022m3vit}} &
%   \multirow{2}{*}{ViT-S} &
%   PASCAL-Context~\cite{everingham2010pascal} &
%   \multicolumn{1}{c|}{\multirow{2}{*}{-}} &
%   72.8 \\ \cline{3-3} \cline{5-5} 
%                                  &        & NYUD-v2~\cite{silberman2012indoor}          & \multicolumn{1}{c|}{}          & 45.6          \\ \hline
% \end{tabular}%
% }
% \end{table}
% Please add the following required packages to your document preamble:
% \usepackage{multirow}
% \usepackage{graphicx}
\begin{table}[]
\centering
\caption{Comparison of accuracy across different ViT acceleration techniques using DeiT-B (Base), DeiT-T (Tiny), ViT-B (Base), and ViT-S (Small) as baseline models. Results include Top-1 accuracy on ImageNet-1K and mean Intersection over Union (mIoU) for segmentation benchmarks.}
\label{accuracy_performance}
\small
\resizebox{0.7\columnwidth}{!}{%
\begin{tabular}{c|c|c|cc}
\hline
\multirow{2}{*}{\textbf{Framework}} &
  \multirow{2}{*}{\textbf{Baseline}} &
  \multirow{2}{*}{\textbf{Dataset}} &
  \multicolumn{2}{c}{\textbf{Accuracy}} \\ \cline{4-5} 
                                 &        &                                             & \multicolumn{1}{c|}{\textbf{Top-1(\%)}} & \textbf{mIoU} \\ \hline
VAQF~\cite{sun2022vaqf}          & DeiT-B & \multirow{2}{*}{ImageNet-1K~\cite{5206848}} & \multicolumn{1}{c|}{77.6}               & -             \\
Auto-ViT-Acc ~\cite{lit2022auto} & DeiT-B &                                             & \multicolumn{1}{c|}{81.8}               & -             \\ \hline
HeatViT ~\cite{dong2023heatvit}  & DeiT-T & \multirow{4}{*}{ImageNet-1K~\cite{5206848}} & \multicolumn{1}{c|}{72.1}               & -             \\
EQ-ViT~\cite{10745859}           & DeiT-T &                                             & \multicolumn{1}{c|}{74.5}               & -             \\
ViTCoD~\cite{you2023vitcod}      & DeiT-T &                                             & \multicolumn{1}{c|}{70.0}               &               \\
SOLE~\cite{wang2023sole}         & DeiT-T &                                             & \multicolumn{1}{c|}{71.07}              & -             \\ \hline
Zhang et al.~\cite{zhang2024109} & ViT-B  & ImageNet-1K~\cite{5206848}                  & \multicolumn{1}{c|}{83.1}               & -             \\ \hline
\multirow{2}{*}{M\textsuperscript{3}ViT~\cite{fan2022m3vit}} &
  \multirow{2}{*}{ViT-S} &
  PASCAL-Context~\cite{everingham2010pascal} &
  \multicolumn{1}{c|}{\multirow{2}{*}{-}} &
  72.8 \\ \cline{3-3} \cline{5-5} 
                                 &        & NYUD-v2~\cite{silberman2012indoor}          & \multicolumn{1}{c|}{}                   & 45.6          \\ \hline
\end{tabular}%
}
\vspace{-3mm}
\end{table}










% \input{Sections/5_fpga}
\section{Challenges and Future Directions of ViT on Edge Devices}\label{cha_fu}
ViT models are computation-intensive, and their deployment on resource-constrained edge devices has been a big challenge. However, with the advancement of edge AI, this is now changing, and the efficient and cost-effective implementation of ViT models is possible directly on edge hardware. This increases accessibility for end users and reduces reliance on cloud infrastructure, which lowers latency, improves privacy, and reduces operational costs. However, some areas, such as real-world scenarios and software-hardware co-design still need to be explored for ViT on edge devices. In this section, we will discuss the current challenges and future opportunities of ViT on edge devices.

\subsection{Software-Hardware Co-design} The lightweight ViT model and compression techniques should be considered the hardware architecture. The SW-HW co-design can reduce the current dilemma between model and hardware architectures. Additionally, different edge hardware platforms (e.g., CPUs, GPUs, and FPGAs) have varying capabilities in handling precision, memory bandwidth, and computational efficiency. Often, accelerators support a uniform bit-width tensor, and this distinct bit-width precision needs zero padding, incurring inefficient memory usage. It is so hard to optimize the ViT for each type of hardware. Leveraging hardware-aware compression techniques can improve the efficiency of edge deployment. Frameworks such as DNNWeaver~\cite{7783720}, VAQF~\cite{sun2022vaqf}, M$^3$ViT~\cite{fan2022m3vit} have been developed for different hardware platforms FPGA, GPU accelerators for efficient edge inference. However, most of the current framework can not handle the sparsity caused by model compression. Therefore, the advancement of reconfigurability of software-hardware co-design for handling sparsity can be a future problem to solve.
\subsection{Utlizing NAS for Inference} NAS is currently used to find the optimized model. The limitations are the computational load and time required for the NAS algorithms, which are high. Consequently, improving NAS algorithms to obtain optimized ViT models targeting the edge can be a future research opportunity. Moreover, frameworks like HAQ~\cite{wang2019haq} and APQ~\cite{wang2020apq} utilized NAS for automatically generating pruning quantization strategies through reinforcement learning or evolutionary search methods. However, this work is highly customized for specific hardware (e.g., HAQ for FPGA). This problem arises because different edge hardware platforms or neural accelerators have distinct properties and processing capabilities. ProxylessNAS~\cite{cai2018proxylessnas} is one of the works that can find a model to fit the hardware but is limited to only CNN models. Therefore, utilizing NAS for searching hardware-independent optimization techniques for ViT can be the future direction to explore. 
\subsection{Acclerators to Handle Sparsity} Traditional processors such as GPU, CPU, or even FPGA cannot handle the space, irregular tensor. For example, although mixed-precision quantization techniques for ViT have been developed, their deployment on edge devices remains limited due to the inefficiency of current hardware architectures and accelerators in handling mixed-precision formats~\cite{lin2024awq}. Therefore, optimized accelerators to handle mixed precision format on edge devices still need to be explored. In addition, combining multiple compression techniques for optimum hardware performance is a promising research direction. Current hardware accelerators are not inherently designed to process sparse tensors efficiently, as they require fetching zero values from memory to processing elements (PEs). Thus, specialized techniques are needed to optimize the storage and computation of nonzero values in ViT.
\subsection{Automated Edge Aware Model Compression} Most of the current model compression techniques require manual adjustment of hyperparameters such as quantization bit width, pruning ratio, or layer-wise sparsity. Compression hyperparameters must be adjusted automatically or adaptively within the resource budget with minimum degradation of accuracy. From our observation, few works explore adjusting the compression parameters automatically. SparseViT~\cite{chen2023sparsevit} is one of few works that effectively reduced computation by targeting less-important regions with dynamically chosen pruning ratios in the images for the ViT model, achieving significant latency reductions. As a result, developing a hardware-efficient automaticity compression technique can be an interesting research domain in the future. 

Another drawback we observe from section~\ref{vit_quanti} is that most of the current work on ViT is post-training. The most interesting reason, perhaps the interactive nature of the training process, is that implementing compression techniques during training requires cost and time. However, QAT techniques on ViT are promising~\cite{li2023psaq,dong2024packqvit,zhang2023qd,aqvit}, but there is limited work on other compression techniques. Thus, the automated exploration of compression techniques during training can improve hardware realization. However, efficient compression techniques for faster convergence during training with reduced computation need to be explored in the future.
\subsection{Developing Benchmarks} Proper benchmark standards to evaluate the performance of the edge devices are important. The different stages of the model to deploying edge, including compression and accelerators, require a universal and comprehensive set of metrics to compare different proposed solutions. However, benchmarking datasets and models from the system perspective are limited. For example, most of the compression techniques for ViT were evaluated on the ImageNet-1k~\cite{5206848} dataset for classification tasks and COCO-2017~\cite{lin2015microsoft} datasets for object detection tasks. However, expanding that knowledge for different real-world application areas is still limited due to the lack of dataset and model benchmarks. Thus, more benchmarking datasets and ViT models for evaluating the proposed system/framework need to be developed for different application areas, such as medical imaging and autonomous driving. In addition, making one compression technique universal for different CV tasks is challenging. Making task-independent universal compression techniques can be an interesting research domain in the future.
\subsection{Real-world Case Studies} The deployment of ViT on edge devices has gained significant progress in recent years. However, most of the compression techniques, frameworks, and accelerators are limited to evaluation in an academic environment. For instance, there are numerous ViT model on medical imaging datasets for different CV tasks such as image classification~\cite{dai2021transmed,perera2021pocformer,wang2023pneunet,raj2023strokevit,gheflati2022vision}, segmentation~\cite{hatamizadeh2021swin,li2023lvit,heidari2023hiformer,he2023h2former,yang2023ept}, object detection~\cite{shou2022object,leng2023deep,wittmann2022swinfpn,liu2022sfod,lin2021aanet}. However, few studies have evaluated compression and accelerator techniques on those  ViT-based medical imaging models. Exploring ViT on edge for medical imaging can be challenging because of the unique nature of the data (3D ultrasounds or MRIs).

Maintaining accuracy, latency, and precision is critical in real-world applications, particularly in critical fields like medical imaging. In medical imaging, a significant challenge to the major compressing of the ViT models is maintaining the spatial resolution and feature details since even small degradations in accuracy will affect the diagnostic outcome significantly~\cite{hou2019high}. Compression techniques can reduce excessive feature abstraction, potentially discarding vital low-level details essential for accurate diagnoses. Furthermore, transfer learning in medical imaging adds another layer of complexity—determining which pre-trained layers to retain or modify without losing critical learned representations is a significant challenge~\cite{peng2022rethinking,vrbanvcivc2020transfer}. Therefore, it is an open research direction to achieve a balance between model efficiency and diagnostic reliability for ViT models on edge for real-world scenarios such as healthcare applications.
\subsection{Seamless Model-to-Edge Integration} The conversion from a trained model into a hardware-compatible version for inference requires extensive time, cost, and, most importantly, manual in each step. Additionally, there is a high knowledge gap between the research community. For example, training new models requires extensive software knowledge, while compressing and developing accelerating strategies require deep hardware knowledge. It is challenging to find an expert in both directions in the research community. These difficulties create a significant research gap in developing tools that automatically map the models on hardware. However, FPGA can overcome some limitations with the ability to reconfigure new operations and modules. However, the available tools are insufficient for the automatic mapping of models and are more limited for ViT.

The current deep learning frameworks for edge deployment help researchers quickly prototype the models to deploy on edge. However, it lacks support with the rapid growth of different model architectures. Additionally, most of the current frameworks are evaluated for CNN models, while those frameworks are still in the experimental phase for ViT. For instance, Xilinx provides quantization support through the FINN-R framework for inference realization on FPGA~\cite{blott2018finn}, limiting only standard techniques for ViT. Therefore, the automatic mapping from model to edge, precisely a one-click solution for deployment on edge based on the budget or auto-generated compression techniques, can be an interesting domain in the future.
\subsection{Robustness to Diverse Data Modality}In real-world scenarios, data sources change from sensor to sensor or vendor to vendor. For example, medical imaging includes X-rays and other modalities like magnetic resonance imaging (MRI), computed tomography (CT) scans, and ultrasounds. Each modality has its characteristics, and a compression technique effective for one might not be for another. So, using a generic model compression technique for all modalities is always tricky. Such heterogeneity may cause inconsistency in data, imposing a challenge on edge performance. Federated learning, multi-modal fusion, and adaptive data calibration techniques can be promising research directions to mitigate the data inconsistency problem at the edge device.
This work presented \ac{deepvl}, a Dynamics and Inertial-based method to predict velocity and uncertainty which is fused into an EKF along with a barometer to perform long-term underwater robot odometry in lack of extroceptive constraints. Evaluated on data from the Trondheim Fjord and a laboratory pool, the method achieves an average of \SI{4}{\percent} RMSE RPE compared to a reference trajectory from \ac{reaqrovio} with $30$ features and $4$ Cameras. The network contains only $28$K parameters and runs on both GPU and CPU in \SI{<5}{\milli\second}. While its fusion into state estimation can benefit all sensor modalities, we specifically evaluate it for the task of fusion with vision subject to critically low numbers of features. Lastly, we also demonstrated position control based on odometry from \ac{deepvl}.

%% Add \usepackage{lineno} before \begin{document} and uncomment 
%% following line to enable line numbers
%% \linenumbers

%% main text
%%

%% Use \section commands to start a section
% \section{Example Section}
% \label{sec1}
% %% Labels are used to cross-reference an item using \ref command.

% Section text. See Subsection \ref{subsec1}.

% %% Use \subsection commands to start a subsection.
% \subsection{Example Subsection}
% \label{subsec1}

% Subsection text.

% %% Use \subsubsection, \paragraph, \subparagraph commands to 
% %% start 3rd, 4th and 5th level sections.
% %% Refer following link for more details.
% %% https://en.wikibooks.org/wiki/LaTeX/Document_Structure#Sectioning_commands

% \subsubsection{Mathematics}
% %% Inline mathematics is tagged between $ symbols.
% This is an example for the symbol $\alpha$ tagged as inline mathematics.

% %% Displayed equations can be tagged using various environments. 
% %% Single line equations can be tagged using the equation environment.
% \begin{equation}
% f(x) = (x+a)(x+b)
% \end{equation}

% %% Unnumbered equations are tagged using starred versions of the environment.
% %% amsmath package needs to be loaded for the starred version of equation environment.
% \begin{equation*}
% f(x) = (x+a)(x+b)
% \end{equation*}

% %% align or eqnarray environments can be used for multi line equations.
% %% & is used to mark alignment points in equations.
% %% \\ is used to end a row in a multiline equation.
% \begin{align}
%  f(x) &= (x+a)(x+b) \\
%       &= x^2 + (a+b)x + ab
% \end{align}

% \begin{eqnarray}
%  f(x) &=& (x+a)(x+b) \nonumber\\ %% If equation numbering is not needed for a row use \nonumber.
%       &=& x^2 + (a+b)x + ab
% \end{eqnarray}

% %% Unnumbered versions of align and eqnarray
% \begin{align*}
%  f(x) &= (x+a)(x+b) \\
%       &= x^2 + (a+b)x + ab
% \end{align*}

% \begin{eqnarray*}
%  f(x)&=& (x+a)(x+b) \\
%      &=& x^2 + (a+b)x + ab
% \end{eqnarray*}

% %% Refer following link for more details.
% %% https://en.wikibooks.org/wiki/LaTeX/Mathematics
% %% https://en.wikibooks.org/wiki/LaTeX/Advanced_Mathematics

% %% Use a table environment to create tables.
% %% Refer following link for more details.
% %% https://en.wikibooks.org/wiki/LaTeX/Tables
% \begin{table}[t]%% placement specifier
% %% Use tabular environment to tag the tabular data.
% %% https://en.wikibooks.org/wiki/LaTeX/Tables#The_tabular_environment
% \centering%% For centre alignment of tabular.
% \begin{tabular}{l c r}%% Table column specifiers
% %% Tabular cells are separated by &
%   1 & 2 & 3 \\ %% A tabular row ends with \\
%   4 & 5 & 6 \\
%   7 & 8 & 9 \\
% \end{tabular}
% %% Use \caption command for table caption and label.
% \caption{Table Caption}\label{fig1}
% \end{table}


% %% Use figure environment to create figures
% %% Refer following link for more details.
% %% https://en.wikibooks.org/wiki/LaTeX/Floats,_Figures_and_Captions
% \begin{figure}[t]%% placement specifier
% %% Use \includegraphics command to insert graphic files. Place graphics files in 
% %% working directory.
% \centering%% For centre alignment of image.
% \includegraphics{example-image-a}
% %% Use \caption command for figure caption and label.
% \caption{Figure Caption}\label{fig1}
% %% https://en.wikibooks.org/wiki/LaTeX/Importing_Graphics#Importing_external_graphics
% \end{figure}


% %% The Appendices part is started with the command \appendix;
% %% appendix sections are then done as normal sections
% \appendix
% \section{Example Appendix Section}
% \label{app1}

% Appendix text.

% %% For citations use: 
% %%       \cite{<label>} ==> [1]

% %%
% Example citation, See \cite{lamport94}.

%% If you have bib database file and want bibtex to generate the
%% bibitems, please use
%%
\bibliographystyle{elsarticle-num} 
\bibliography{reference}
%%  \bibliography{<your bibdatabase>}

%% else use the following coding to input the bibitems directly in the
%% TeX file.

%% Refer following link for more details about bibliography and citations.
%% https://en.wikibooks.org/wiki/LaTeX/Bibliography_Management

% \begin{thebibliography}{00}

% %% For numbered reference style
% %% \bibitem{label}
% %% Text of bibliographic item

% \bibitem{lamport94}
%   Leslie Lamport,
%   \textit{\LaTeX: a document preparation system},
%   Addison Wesley, Massachusetts,
%   2nd edition,
%   1994.

% \end{thebibliography}
\end{document}

\endinput
%%
%% End of file `elsarticle-template-num.tex'.
