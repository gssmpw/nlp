% \section{Conclusion}\label{conclusion} The introduction of ViT has brought a paradigm shift in the field of image processing, with medical imaging emerging as a key beneficiary. ViTs, characterized by their self-attention mechanisms and patch-based input handling, have shown a remarkable ability to capture intricate patterns in data, proving advantageous for detecting subtle features in medical images. However, due to the high computational cost and hardware usage in ViT, it is challenging to use ViT-based architecture deployment in FPGA. In this survey, we discussed the all steps from training a ViT model to deploying in FPGA. We briefly discussed different ViT architectures for medical imaging tasks followed by different possible model compression techniques and deployment techniques in FPGA. Additionally, We documented the results overview for the corresponding techniques to get a clear overview of the current techniques.\\
% In addition, we found that each architecture in model compression has unique features and tried to solve problems that will be handy in the future during the deployment in FPGA. In the deployment to FPGA, we tried to divide the existing techniques based on the image tasks to give a better understanding of the initial training techniques. As ViT in FPGA is still new and needs to explore more in the future, we briefly discussed the challenges and possible future direction.\\
% In conclusion, the realm of deploying ViT on FPGA platforms for medical imaging remains expansive, teeming with both intricacies and prospects. We anticipate that as research deepens and technology advances in the upcoming years, we will achieve a more sophisticated, precise, and dependable medical imaging system powered by ViTs on FPGAs. This progress promises to significantly enhance diagnostic capabilities, potentially leading to improved patient outcomes and revolutionizing healthcare experiences for all.

\section{Conclusion}\label{conclusion} 
With the increasing adoption of ViTs in computer vision, optimizing their efficiency for edge deployment has become a key research focus. This survey comprehensively analyzes ViT model compression and hardware-aware acceleration techniques, exploring techniques such as pruning, quantization, knowledge distillation, and SW-HW co-design. By categorizing these advancements and evaluating their impact across edge platforms, we highlight the trade-offs between accuracy, resource utilization, and energy efficiency in real-world applications. Our analysis indicates that while SOTA ViT compression and acceleration techniques effectively reduce computational overhead and improve inference speed, challenges such as hardware adaptability, memory bottlenecks, and optimal compression strategies remain unexplored. Additionally, We discuss the potential future directions, such as utilizing NAS to find hardware-aware optimization parameters, sparsity-aware accelerators, and efficient cross-platform SW-HW co-design frameworks. 

Overall, the domain of optimizing ViT on edge devices remains an evolving field, presenting both challenges and opportunities. Continued advancements in co-optimized software-hardware solutions will pave the way for more efficient and deployable ViTs.
% In conclusion, ViT-based models show excellent potential to perform medical imaging tasks like classification, segmentation, and object detection for different medical modalities. However, deploying those ViT-based models on FPGA platforms is challenging due to their complex architecture, high computational demands, and lack of model compression techniques. This survey discusses the entire process techniques, from ViT models for medical images to FPGA deployment, exploring various ViT architectures for medical images, model compression techniques, and deployment strategies on FPGA.

% Despite each model compression, which is the technique to maintain accuracy while deploying to FPGA architecture having unique features to aid future FPGA deployment, challenges remain, indicating a need for further exploration. In summary, deploying ViTs on FPGAs for medical imaging is a promising but complex field with great potential for advancing medical diagnostics and patient care. We hope this survey provides a roadmap for future researchers to progress in the sector of ViT-based models using FPGA in medical images. As research and technology evolve continuously, we can expect significant improvements in medical imaging systems and healthcare experiences using ViTs on FPGAs.

\section{Acknowledgement}
This work was partly supported by the U.S. National Science
Foundation under Grants CNS-2245729 and Michigan Space
Grant Consortium 80NSSC20M0124.
% In conclusion, ViT-based models show excellent potential to perform medical imaging tasks like classification, segmentation, and object detection for different medical modalities. However, deploying those ViT-based models on FPGA platforms is challenging due to their complex architecture, high computational demands, and lack of model compression techniques. This survey discusses the entire process techniques, from ViT models for medical images to FPGA deployment, exploring various ViT architectures for medical images, model compression techniques, and deployment strategies on FPGA.\\In summary, deploying ViTs on FPGAs for medical imaging is a promising but complex field with great potential for advancing medical diagnostics and patient care. As research and technology evolve continuously, we can expect significant improvements in medical imaging systems and healthcare experiences using ViTs on FPGAs.