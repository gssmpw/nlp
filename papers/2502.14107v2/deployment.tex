\section{Deployment}
\label{sec:deployment}

\begin{figure*}
	\centering
	\includegraphics[width=0.9\textwidth]{depl}
	\caption{The deployment of waterproof buoys on the surface of different water bodies. From left to right: A prototype low-power IoT node; deployments on: a lake on FIU's Modesto A. Maidique Campus, Miami South Beach, and Crandon Beach.}
	\label{fig:deployments}
\end{figure*}

In order to study link quality fluctuation, we deployed wireless sensor networks on four different water bodies in Miami, Florida, between the first of June and the end of August 2023. The first deployment was on one of the small lakes on the main campus of Florida International University (FIU). The second was on North Biscayne Bay; and the third and the fourth were on South Beach, Miami, and Crandon Beach, Miami, respectively. Some of the deployments took place at the time when the State of Florida was significantly affected by Hurricane Idalia, a  Category 4 hurricane. The first two water bodies were relatively calm, and the main causes of link quality fluctuation were heavy rain and excessive heat. The last two, on the other hand, besides experiencing extreme weather conditions, moved significantly, causing frequent disconnections of the wireless links. We placed the sensor nodes in waterproof boxes and used waterproof marine antennas to transmit and receive electromagnetic signals in the sub-Giga and 2.4 GHz frequency bands (the sensor nodes integrate two different radio chips, CC1200 and CC2538). In order to establish a relation between the change in RSSI (signifying link quality fluctuation) and the movement of water, we also embedded 3D accelerometers and gyroscopes in the waterproof buoys. Figure~\ref{fig:deployments} shows the deployments of the buoys in the three different environments. 


