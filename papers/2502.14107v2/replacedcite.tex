\section{Related Work}
\label{sec:related}

Recent research on wireless radio performance in extreme environments has explored various innovative technologies and techniques to enhance wireless communications ____. This includes the use of multi-radio, multichannel wireless radios to improve overall network performance____, cognitive radios to intelligently optimize spectrum utilization____, and machine learning techniques to dynamically optimize wireless network performance, as proposed in____. Environmental conditions such as weather, terrain, and human activities have been widely recognized for their impacts on radio performance such as fluctuations in transmission rates and quality of service in wireless sensor networks, e.g., the work in____ investigated the impact of weather conditions and atmospheric phenomena on link quality, while other research____ studied the impacts of humidity and movement, all of which can affect radio signal quality. Wireless devices, including wireless sensor networks, deployed near water surfaces also experience a variety of quality fluctuations, e.g., the absorption and scattering of water and small particles dissolved in it have been found to significantly affect wireless optical signals____. In____ and____, the authors survey various applications of wireless sensor networks for the real-time monitoring of water quality, marine life, and environmental conditions in marine areas. While focusing primarily on underwater scenarios, the work in____ discusses specifically the challenges of energy-efficiency, topology control, and transmission power control strategies. The conductive properties of the water surface can strengthen signal reflections and increase interference effects, as discussed in____. Similar to our own observations, this work also observed that recurrent natural phenomena, such as tides or waves, lead to water level variations and therefore changing interference patterns, affecting the signal propagation over water surfaces. The focus of the work in____, however, is on determining the optimal combination of radio height and distance that minimizes the path loss averaged during a whole tidal cycle. In contrast, in our work, we assume that devices are floating on the surface of the water, and we address link quality changes by predicting future received powers, which can be used to design an adaptive transmit power scheme. Finally, in____, the authors also attempt to predict path loss for a Long Range (LoRa) line-of-sight link deployed over an estuary with characteristic intertidal zones, studying both shore-to-shore and shore-to-vessel communications. While this work does not employ a transmit power adaptation scheme to address path loss variations, the proposed technique could certainly be used to help decide on future transmit powers. In contrast to the work in____, the focus of our work is on wireless sensor networks, where devices are deployed directly on the surface of a body of water (instead of shore- or vessel-based radios).