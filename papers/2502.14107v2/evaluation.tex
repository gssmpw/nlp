\section{Evaluation}
\label{sec:evaluation}

In our experiments, floating nodes measured the 3D motion they experienced using an onboard IMU and communicated this with a base station deployed on the ground. Up on receiving a packet, the base station associated the received power with the motion data. The packet size was 128 bytes. For this packet size, the sustainable rate the CC1200 radio could support was 2 packets per second, whereas the CC2538 radio could support 10 packets per second. These rates were highly environment dependent, including the external temperature and the internal heat dissipation. The waves at Miami South Beach were distinctly different from the waves at Crandon Beach. The former were typically bigger and longer, whereas the latter were typically shorter and faster and more harmonious. Since our estimation establishes a correlation between the RSSI and the linear acceleration, ideally, the two should be sampled at the same rate. However, sampling the IMU at a rate below 10 Hz resulted in a significant estimation error. Therefore, for all our experiments, the IMU was sampled at a rate of 10 Hz. For the estimations concerning the CC1200 radio, the measurements of the IMU were down-sampled using overlapping windows of size 10 samples and an overlapping coefficient of 0.5. Because the IMU and the RSSI were measured in different units, they were normalized using the max/min normalization, so that they had values between 0 and 1. 

\subsection{Prediction}
In this subsection, we compare the normalized actual (measured) RSSI with its estimates obtained using the model based on gradient descent and the one involving matrix inversion. For the gradient descent, we consider two scenarios: initial zero values to all entries of $A$ and random assignment from a uniform distribution over $([0, 1])$. Figures ~\ref{fig:rssi_comparison_cb} and \ref{fig:rssi_comparison_sb} present these comparisons for the deployments of Crandon Beach and Miami South Beach, respectively. The number of iterations for the gradient descent were 100 in both cases. The radio under consideration is the CC2538 System-on-Chip. As can be seen, for the Miami South Beach deployment, the estimates agree very well with the actual normalized RSSI. However, for the Crandon beach deployment, while the two estimates agree quite well, they do not fully capture the profile of the normalized actual RSSI.

\begin{figure}
    \centering    \includegraphics[width=1\linewidth]{rssi_comparison_cb}
    \caption{Difference between normalized actual RSSI, and estimated RSSI obtained by gradient descent and solving the exact linear system. The deployment is for Crandon Beach. Radio: CC2538.}
    \label{fig:rssi_comparison_cb}
\end{figure}

\begin{figure}
    \centering    \includegraphics[width=1\linewidth]{rssi_comparison_sb}
    \caption{Difference between normalized actual RSSI, and estimated RSSI obtained by gradient descent and solving the exact linear system. The deployment is for Miami south Beach. Radio: CC2538.}
    \label{fig:rssi_comparison_sb}
\end{figure}

Similarly, Figure \ref{fig:estimation_mmse_cc1200_crando} compares the actual RSSI with its estimates for the CC1200 radio. The deployment we consider is the Crandon Beach. We see that the two estimates based gradient descent and inversion agree very well, however they do not fully capture the profile of the normalized actual RSSI especially at low packet index.

\begin{figure}[tbh!]
	\centering
		%\includegraphics[width=0.45\textwidth]{img/estimation_mmse_cc1200_crandon.pdf}
        \includegraphics[width=0.45\textwidth]{rssi_comparison_cb_1200}
	\caption{Difference between normalized actual RSSI, and estimated RSSI obtained by gradient descent and solving the exact linear system. The deployment is for Crandon Beach. Radio: CC1200.}
	\label{fig:estimation_mmse_cc1200_crando}
\end{figure}

\subsection{Root Mean Square Error}
Tables \ref{tab:mse_comparison_cb} and \ref{tab:mse_comparison_sb} compare the root mean square errors (RMSE) of the model based on gradient descent for different iterations. For the Crandon Beach deployment, with random initialization, we observe that even with just 10 iterations, the RMSE is comparable to that obtained by solving the exact linear system. For the Miami South Beach deployment, gradient descent performs better in terms of RMSE for both random and zero initialization, regardless of the total number of iterations. Tables \ref{tab:time_comparison_cb} and \ref{tab:time_comparison_sb} present the computation time of the different iterations for the Crandon Beach and Miami South Beach deployments, respectively. We note that gradient descent shows comparable runtime performance to solving the linear system. Overall, considering all the deployments and the experiments we carried out for both radios, the gradient-descent's prediction accuracy was on average 91\%, whereas the one involving matrix inversion was slightly worse, 90.7\%.

\begin{table}[h!]
    \centering
    \caption{Comparison of the RMSE of two different initialization using gradient descent. The mean square error using the exact linear system is $0.115$. Deployment: Miami Crandon Beach. Radio: CC2538.}
    \label{tab:mse_comparison_cb}
    \begin{tabular}{|c|c|c|}
        \hline
        $T$ & Mean Square Error  & Mean Square Error  \\ 
         (\# of Iterations) & (Zero initialization) & (Random initialization) \\ \hline
        1000 & 0.113 & 0.114 \\ \hline
        500  & 0.111 & 0.114 \\ \hline
        100  & 0.107 & 0.114 \\ \hline
        50   & 0.102 & 0.115 \\ \hline
        10   & 0.0964 & 0.121 \\ \hline
  \end{tabular}
\end{table}

   \begin{table}[h!]
    \centering
    \caption{Comparison of RMSE for two different initialization using gradient descent. The mean square error using the exact linear system is $0.155$. Deployment: Miami South Beach. Radio: CC2538.}
    \label{tab:mse_comparison_sb}
    \begin{tabular}{|c|c|c|}
        \hline
        $T$ & MMSE  & RMSE  \\ 
         (\# of Iterations) & (Zero initialization) & (Random initialization) \\ \hline     
        1000 & 0.155 & 0.155 \\ \hline
        500  & 0.155 & 0.155 \\ \hline
        100  & 0.155 & 0.154 \\ \hline
        50   & 0.154 & 0.153 \\ \hline
        10   & 0.153 & 0.151 \\ \hline
    \end{tabular}
\end{table}

\begin{table}[h!]
    \centering
    \caption{Comparison of computation time to estimate $E$ via solving the system in Equation \ref{eq:model_7} versus using gradient descent. Deployment:  Crandon Beach. Radio: CC2538.}
    \label{tab:time_comparison_cb}
    \begin{tabular}{|c|c|c|}
        \hline
        $T$ & Time taken  & Time taken  \\ 
         (\# of Iterations) & (Exact solve) & (Gradient descent) \\ \hline
        1000 & $1.17\times 10^{-4}$ & $1.10\times 10^{-3}$ \\ \hline
        500  & $9.69\times 10^{-5}$ & $6.15\times 10^{-4}$ \\ \hline
        100  & $1.07\times 10^{-4}$ & $3.76\times 10^{-4}$ \\ \hline
        50   &  $8.68\times 10^{-5}$ & $3.51\times 10^{-4}$ \\ \hline
        10   & $4.79\times 10^{-5}$ & $6.65\times 10^{-5}$ \\ \hline
  \end{tabular}
\end{table}

\begin{table}[h!]
    \centering
    \caption{Comparison of computation time to estimate $E$ via solving the system in Equation \ref{eq:model_7} versus using gradient descent. Deployment:  South Beach. Radio: CC2538.}
    \label{tab:time_comparison_sb}
    \begin{tabular}{|c|c|c|}
        \hline
        $T$ & Time taken  & Time taken  \\ 
       (\# of Iterations) & (Exact solve) & (Gradient descent) \\ \hline
        1000 & $1.24\times 10^{-4}$ & $9.91\times 10^{-4}$ \\ \hline
        500  & $1.07\times 10^{-4}$ & $5.93\times 10^{-4}$ \\ \hline
        100  & $1.16\times 10^{-4}$ & $4.62\times 10^{-4}$ \\ \hline
        50   &  $1.38\times 10^{-4}$ & $4.20\times 10^{-4}$ \\ \hline
        10   & $1.17\times 10^{-4}$ & $3.90\times 10^{-4}$ \\ \hline
  \end{tabular}
\end{table}

