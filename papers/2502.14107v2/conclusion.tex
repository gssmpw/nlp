\section{Conclusion} % 1
\label{sec:conclusion}

In this paper we proposed a model to predict the received power of low-power wireless sensing nodes deployed on the surface of rough waters. This is useful for adapting the transmission power of outgoing packets. We expressed the received power in terms of its past statistics and the statistics of the 3D motion the nodes experience using MMSE. However, MMSE involves matrix inversion, which is difficult to compute with resource-constrained devices. We avoided this stage by estimating the model parameters using the gradient-descent approach, which produced acceptable results even for a few number of iterations. Based on several experiments we carried out at South Beach Miami and Crandon Beach Miami using two different low-power radios, our approach predicted the received power with an average accuracy of ca. 91\%. By comparison, a Kalman Filter which relied only on the statistics of received power accomplished the same job with an average accuracy of 90\%. One of the challenges we faced during the modeling process was perfectly synchronizing the received power with the motion data. This might have resulted in performance penalty, since it has a direct bearing on Equation~\ref{eq:model_5}. In future, we aim to address this issue.

