\section{Introduction}
\label{intro}

Cost-effective and low-power IoT nodes enable scalable and distributed sensing and have a wide range of applications \cite{lombardo2017wireless, zhao2024missing}. Some of these applications---water quality monitoring, pollution detection, flooding monitoring, and seaweed monitoring --- require the deployment of the nodes on the surface of rough waters, which affect the networks in a number of ways. Firstly, it constantly modifies their topology, thereby making the establishment of stable routes a significant challenge. Secondly, it makes the wireless links unstable, causing considerable packet loss. Thirdly, repeated attempts to reestablish links and retransmit packets can prematurely exhaust a device's energy resources \cite{gong2021self}. 

Dynamic transmission power adaptation is one of the mechanisms employed to deal with adverse external factors \cite{lin2016atpc, jurdak2007adaptive, dargie2011dynamic}. Often, it is jointly used with other mechanisms, such as dynamic channel selection \cite{watteyne2009reliability}. 
Most existing or proposed approaches on transmission power adaptation rely on received power statistics. The underlying assumption is that (1) given an appropriate path loss model, the relationship between the received power and the transmission power can be established and (2) the received power exhibits strong autocorrelation, which makes it predictable  \cite{govindan2011probability,alsamhi2021predictive}. This assumption holds for many real world situations (as is the case in cellular networks, for example). For the deployment environments we are concerned with, however, it is either partially true or may not be applicable at all. The reason is that the main cause of change in the received power (the motion of the water) is 3-dimensional whereas the effect is one-dimensional. The rougher the motion, the more complex is the relationship between the cause and the effect. A more plausible approach is to predict the received power from the motion of the water and to undertake compensatory measures (increase or decrease transmission power) to obtain good quality of service. The purpose of this paper is to propose an approach to achieve this goal. 

\begin{figure}
	\centering
		\includegraphics[width=0.450\textwidth]{received_power_illustration.pdf}
	\caption{Illustration of dynamic power adaptation. A transmitter aims to transmit a packet with a power such that when received, the packet has a received power above a set threshold.}
	\label{fig:received_power_illustration}
\end{figure}

What we wish to achieve in this paper is illustrated in Figure~\ref{fig:received_power_illustration}: At a receiver, a threshold power is set for received packets to be successfully decoded. A transmitter predicts the received power based on a locally available, lightweight model and adjusts the power of outgoing packets, so that when they are received, their power is above the set threshold. As the first contribution of this paper, we express the received power in terms of the 3-dimensional acceleration the nodes experience and describe the relationship between these quantities (defined as random variables) using Minimum Mean Square Estimation (MMSE) \cite{papoulis2002probability}. However, MMSE involves matrix inversion, which is computationally expensive and difficult to implement in resource-constrained devices. Therefore, we forego matrix inversion in favor of gradient-descent, which is much less complex to implement. Though gradient-descent involves iterative estimation and matrix multiplication, we shall experimentally demonstrate that with very few iterations, the model achieves a remarkable accuracy. As the second contribution, with practical deployments on different water bodies using resilient prototypes and two different heterogeneous low-power radios, we demonstrate the reliability and accuracy of our model. More specifically, the model achieves on average a prediction accuracy exceeding 91\%, which is remarkable for resource-constrained devices and very rough deployment environments. 

The remaining part of the paper is organized as follows. In the following section, we describe related work. In Section \ref{sec:deployment}, we describe the deployment scenarios wherein we conducted extensive experiments. In Section \ref{sec:model}, we introduce our approach to model low-power and lossy wireless links. In Sections~\ref{sec:evaluation} and~\ref{sec:comp} we present test results, and compare them with the state-of-the-art. Finally, in Section \ref{sec:conclusion}, we provide concluding remarks and outline future work.   

