\section{Related Work}
\label{sec:rel}

Similar work comparing the three selected DSR frameworks in terms of support for artifact validity has not yet been conducted. The overview table in Figure~\ref{fig:tabrel} shows how frequently cited literature on artifact evaluation, and other extensive literature, e.g.____, relate to the five validity types applied here. 

\begin{figure}[htb!]
    \centering
    \includegraphics[width=1.0\textwidth]{tab_rel.png}
    \caption{Type of validity categories, aligned with frequently cited DSR literature on artifact evaluation}
    \label{fig:tabrel}
\end{figure}

In particular, 'instrument validity' as a mandatory prerequisite for a valid artifact (see Sect.~\ref{sec:flextyp}) is difficult to find in all three DSR frameworks examined here. Sonnenberg \& vom Brocke____ is one of the few studies that clearly points out the need to evaluate the design and the construct separately and in interaction with the problem to be evaluated. This can be equated with the need for 'instrument validity'. Although this model increases the basic evaluation effort, it ensures potentially valid artifacts in the end. Nevertheless, there has been a lack of consideration and differentiation of the five types of validity clearly defined here as an integral part of a comprehensive DSR framework.