\section{Summary} \label{sec:Summary}

    We propose a method to construct reduced basis for time-dependent problems. The method follows a traditional \gls{pod} but with an important distinction: the sampling is done over the frequency domain. In the paper we provided a path for the derivation of method following some theorems of harmonic analysis. We also show an interpretation of the constructed basis and the equivalences using some concepts of differential geometry.
    The most important features of this method are the following:
\begin{itemize}    
    \item As the method is based in the norm equivalence between time and frequency domains, the obtained reduced basis is a basis for both problems. 
	\item The solutions obtained with this reduced basis maintain good accuracy while keeping the computational cost below the high fidelity solution, even when including the sampling and construction of the basis.
    \item Since we are not modifying the \gls{pod} and the Laplace transform is a linear map, any extension of the reduced basis method propose for stationary problems can be implemented. For example parametric reduced basis, local basis, etc.
    \item Problems with linear transformations in the source term have an equivalent reduced basis. For example, we can think of problems where the source term is time shifted, in which case the Laplace transform is invariant.
    \item Initial conditions behave as right hand side terms in the frequency domain problem, so any linear transformation on them results in equivalent reduced bases.
\end{itemize}

%\subsection{Outlook}

