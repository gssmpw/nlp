%%%%%%%%%%%%%%%%%%%%%%%%%%%%%%%%%%
\section{Problem model} 
\label{sec:Problem}
%%%%%%%%%%%%%%%%%%%%%%%%%%%%%%%%%%

\subsection{Notation}
Let $\Omega \subset \IR^d$, $d \in \{2,3\}$, be a bound Lipschitz domain with boundary $\Gamma \defeq \partial \Omega$. Let $L^2(\Omega)$ be the space of square integrable functions in $\Omega$ equipped with inner product $\dotp{\cdot}{\cdot}_{L^2(\Omega)}$ and induced norm $\norm{\cdot} = \dual{\cdot}{\cdot}_{L^2(\Omega)}$. We denote by $H^k(\Omega)$, $k\in \IN$ the space of functions with $k$-th weak derivatives in $L^2(\Omega)$. In particular, throughout this work we consider the closed space $H^1_0(\Omega)$ of $H^1(\Omega)$ with vanishing Dirichlet trace on $\Gamma$, and denote by $H^{-1}(\Omega)$ its dual in the $L^2(\Omega)$-duality pairing, which is denoted by $\dual{\cdot}{\cdot}_{L^2(\Omega)}$. From here on, we set $\mathbb{V} \defeq H^1_0(\Omega)$, and therefore $\mathbb{V}' \defeq H^{-1}(\Omega)$.\footnote{Note that we can set $\mathbb{V}$ as other space of functions that is adequate to the requirements of the problem, for example $H^{\textrm{div}}$, $H^{\textrm{curl}}$, or $H^1$ for different boundary conditions.}

We set the time interval $\setI=(0,t_{\textrm{f}})$ for $t_{\textrm{f}}>0$, and let $\IX$ be a separable Banach space. For each $m \in \IN_0$, we define $H^m(\setI;\IX)$ as the Bochner space of strongly measurable functions $u: \setI \rightarrow \IX$ satisfying
\begin{equation}
	\norm*{u}_{H^m(\setI ; \IX)} \defeq \left(\sum_{j=0}^m \int\limits_0^{t_{\textrm{f}}} \norm*{\partial_t^j u(t)}_{\IX}^2 \odif{t} \right)^{\frac{1}{2}} < \infty,
\end{equation}
where $\partial^j_t$ stands for the weak time derivative of order $j\in \IN_0$. In particular, if $m=0$ we set $L^2(\setI, \IX)\defeq H^0(\setI; \IX)$. We set $\IR_+ \defeq \{t \in \IR : t>0 \}$, and introduce Sobolev spaces in time $L^2(\IR_+,\IV)$ and $L^2(\IR_+,\IV')$. Given $\alpha>0$ let $L^2(\IR_+,\IV,\alpha)$ be the Hilbert space of $\IV$-valued measurable functions $u:\IR_+ \rightarrow \IV$ satisfying
\begin{align} \label{eq:normalpha}
	\norm{u}_{L^2(\IR_+,\IV,\alpha)} \defeq (u,u)^{\frac{1}{2}}_{L^2(\IR_+,\IV,\alpha)} = \left(\int\limits_{0}^{\infty} \dotp{u(t)}{u(t)}^2_{\IV} \e^{-2\alpha t} \odif{t} \right)^{\frac{1}{2}} < \infty,
\end{align}
where
\begin{align}
	\dotp{u}{v}_{L^2(\IR_+,\IV,\alpha)} \defeq \int\limits_{0}^{\infty} \dotp{u(t)}{v(t)}_{\IV} \e^{-2\alpha t} \odif{t}.
\end{align}

%%%%%%%%%%%%%%%%%%%%%%%%%%%%%
\subsection{Time-dependent Partial Differential Equations} \label{eq:time_dependent_pdes}
%%%%%%%%%%%%%%%%%%%%%%%%%%%%%

We consider a general time-dependent problem of the following form: Seek  $u(\bm{x},t):\Omega\times\setI \to \IR^\ell$, $\ell =1$ being the scalar-valued case and $\ell\in\{2,\dots,d\}$ the vector-valued one, such that
\begin{equation} \label{eq:prob}
	\partial_{tt} u(\bm{x},t) + \mathcal{A}  u(\bm{x},t) = b(\bm{x},t), \quad \bm{x} \in \Omega,
\end{equation}
equipped with initial and boundary conditions.\footnote{Through the main body of the paper we work with the wave equation but note that the method can be used with any PDE with a defined Laplace transform.}
We define $\mathcal{A}$ as a linear operator on $\Omega$, and $\partial_{tt} u$ and $\partial_t u$ as the second and first order time derivative, respectively. Here we can set $\mathcal{A}$ as the Laplace operator $-\Delta$, an advection operator $\bm{a} \cdot \nabla$, or as any other operator. Additionally, we equip the problem with appropriate boundary conditions
\begin{equation}
	u(\bm{x},t) = u_{\Gamma}, \quad (\bm{x},t) \in \Gamma \times \setI,
\end{equation}
and suitable initial conditions
\begin{equation}
	u(\bm{x},0) = u_0(\bm{x}), \quad \text{and} \quad \partial_t u(\bm{x},0) = u_0'(\bm{x}), \quad \bm{x} \in \Omega.
\end{equation}
We also define the bilinear form $\mathsf{a}: \mathbb{V} \times \mathbb{V} \rightarrow \IR$ for $u,v \in \mathbb{V}$, for example for $\mathcal{A}=-\Delta$ as
\begin{align}
	\mathsf{a}(u,v) \defeq \int_{\Omega} \nabla u(\bm{x})^\top \nabla u(\bm{x}) \odif{\bm{x}}.
\end{align}

Now, we can introduce the variational version of the previous problem.
\begin{problem} \label{pbm:weak_problem}
Given $t \in \setI$, $u_0 \in \mathbb{V}$, and $u'_0 \in L^2(\Omega)$, find $u \in L^2(\setI,\IV)$ such that
\begin{align}
	\dual{\partial_{tt}u}{v}_{\mathbb{V}' \times \mathbb{V}} + \mathsf{a}(u,v) = \dotp{b}{v}_{L^2(\Omega)},
\end{align}
satisfying the initial conditions $u(0) = u_0$ and $\partial_t u(0) = u'_0$.
\end{problem}


%%%%%%%%%%%%%%%%%%%%%%%%%%%%%
\subsection{Finite Element Discretization}
\label{ssec:fe_problem}
%%%%%%%%%%%%%%%%%%%%%%%%%%%%%

Let $\Omega \subset \IR^d$, $d\in \{1,2,3\}$ be a bounded Lipschitz polygon with boundary $\partial \Omega$. We consider a discretization $\setT_h$ of elements in $\IR^d$ with mesh size $h$, each of them the image of a reference element $\hat{K}$ under the affine transformation $F_{K}: \hat{K} \rightarrow K$. Let us set
\begin{equation}
	\IX^{p}\left(\setT_h\right) \defeq \left\{	w \in \mathcal{C}^0(\Omega) \mid \forall K \in \setT_h: \left. w\right|_K \circ F_K \text { is a polynomial of degree } p \right\}.
\end{equation}
We set $\IV_h = \IV \cap \IX^p(\setT_h)$ as a conforming finite-dimensional subspace of $\IV$, with $n_h = \dim(\IX^{p} \left(\setT_h\right))$ and a basis $\{\varphi_1,\dots,\varphi_{n_h}\}$. We define $\mathsf{P}_h: \IV \rightarrow \IV_h$ to be the projection operator into $\IV_h$. To write the discrete problem, let us consider the solution ansatz
\begin{equation}
	u_h(\bm{x},t) = \sum_{j=1}^{n_h} \varphi_j(\bm{x}) \mathsf{u}_j(t), \quad\bm{x} \in \Omega, \quad t\in (0,t_{n_t}].
\end{equation}

\begin{problem}[Semi-discrete problem] \label{pr:sdp}
Let $u^0, u^1 \in \IV$, and $b \in \mathcal{C}^0(\setI ; L^2(\Omega))$. We seek $u_h \in \mathcal{C}^2(\setI, \IV_h)$ such that for each $t \in \setI$ it holds
%Seek $u_h \in L^2(\IR_+, \IV_h)$ such that
\begin{align}\label{eq:semi_discrete}
	\odv[order=2]{}{t} \dotp{u_h}{v_h}_{L^2(\Omega)} + \mathsf{a} \left(u_h,v_h\right) = \dotp{b}{v_h}_{L^2(\Omega)}, \quad \forall t \in (0,t_{n_t}], \quad \text{and} \quad \forall v_h \in \IV_h,
\end{align}
with initial conditions $u_h(0) = \mathsf{P}_{h} u^0\in \IV_h$ and $ \odv{}{t} u_h(0) = \mathsf{P}_{h} u^1\in \IV_h$.
\end{problem}

To describe the fully-discrete version of \cref{pr:sdp}, we consider the discrete time interval $\setI_h=\{t_1,\ldots,t_{n_t} \}$ with $n_t$ total time steps. Let us set $\bm{\mathsf{u}}(t_n) = (\mathsf{u}_1(t_n),\dots, \mathsf{u}_{n_h}(t_n))^{\top} \in \IR^{n_h}$, and likewise $\ddot{\bm{\mathsf{u}}}(t_n)$ and $\dot{\bm{\mathsf{u}}}(t_n) \in \IR^{n_h}$ as the approximations of $\partial_{tt} u$ and $\partial_{t} u$, respectively. We define $\bm{\mathsf{M}} \in \IR^{n_h \times n_h}$ and $\bm{\mathsf{A}} \in \IR^{n_h \times n_h}$ as
\begin{equation}
	(\bm{\mathsf{M}})_{i,j} = \dotp{\xi_i}{\xi_j}_{L^2(\Omega)} \quad \text{and} \quad (\bm{\mathsf{A}})_{i,j} = \mathsf{a} \left(\xi_i,\xi_j \right), \quad i,j \in \{1,\dots,n_h\},
\end{equation}
referred to as the mass matrix and the stiffness matrix of the bilinear form $\mathsf{a}(\cdot,\cdot)$.

As a time integration algorithm, we follow the generalized $\alpha$-method proposed in \cite{chungTimeIntegrationAlgorithm1993}.
\begin{problem}[Fully-discrete problem] \label{pr:dp}
We seek $\bm{\mathsf{u}}_{n+1} = \bm{\mathsf{u}}(t_{n+1}) \in \IR^{n_h}$, such that
\begin{equation}
    \bm{\mathsf{M}} \ddot{\bm{\mathsf{u}}}_{n+1-\alpha_{m}} + \bm{\mathsf{A}} \bm{\mathsf{u}}_{n+1-\alpha_{f}} = \bm{\mathsf{b}}_{n+1-\alpha_{f}},
\end{equation}
with the notation $\bm{\mathsf{b}}_{n+1-\alpha} = (1-\alpha) \bm{\mathsf{b}}_{n+1} + \alpha \bm{\mathsf{b}}_{n}$, by using the approximation
\begin{align}
    \bm{\mathsf{u}}_{n+1} &= \bm{\mathsf{u}}_n + \Delta t \dot{\bm{\mathsf{u}}}_n + \frac{\Delta t^2}{2} \left( (1-2\beta) \ddot{\bm{\mathsf{u}}}_n + 2 \beta \ddot{\bm{\mathsf{u}}}_{n+1} \right) \\
    \dot{\bm{\mathsf{u}}}_{n+1} &= \dot{\bm{\mathsf{u}}}_{n} + \Delta t \left( (1-\gamma) \ddot{\bm{\mathsf{u}}}_n +\gamma \ddot{\bm{\mathsf{u}}}_{n+1} \right).
\end{align}
We set the parameters as $\alpha_m, \alpha_f \leq \frac{1}{2}$ and $\gamma = \frac{1}{2} - \alpha_m + \alpha_f$, $\beta \geq \frac{1}{4} + \frac{1}{2} (\alpha_f-\alpha_m)$ which ensures that the method is unconditional stable and of second-order accuracy.
\end{problem}

\subsection{Construction of the reduced basis: Time-dependent approach}

The traditional approach to model reduction for time-dependent problems consists in finding a reduced space $\IV_h^{\textrm{R}}\subset\IV_h$ with basis $\{\varphi_1^\textrm{R},\dots,\varphi_{n_r}^\textrm{R}\}$, where the dimension of the reduced space $n_r$ is considerably smaller compared to the dimension of the `high fidelity' space $\IV_h$. We are interested in construction a reduced basis for the discrete solution manifold
\begin{equation} \label{eq:tsnap}
    \bm{\mathsf{S}} = \{ \bm{\mathsf{u}}(t_j) \; | t_j \in \setI_h \} \in \IR^{n_h \times n_t}.
\end{equation}
As mentioned in the introduction, the most common approach to constructing the reduced basis for a time-dependent problem consists of following a \gls{pod}, where we want to find a basis $\Phi = \{\phi_1,\ldots,\phi_{n_r}\}$ such that
\begin{equation} \label{eq:tPOD}
	\Phi = \min_{\substack{\Phi \in \IR^{n_h \times n_r}: \\ \Phi^{\top} \Phi = \bm{I}_{n_r \times n_r}}}
    	\sum_{j=1}^{n_t} \norm*{\bm{\mathsf{u}}(t_j) - \sum_{k=1}^{n_r} \dotp{\Phi_k}{\bm{\mathsf{u}}(t_j)} \Phi_k
	}^2_{L^2(\IR_+,\IV_h)}
\end{equation}

Obtaining these sampled solutions requires the computation of $n_t$ linear systems of size $n_h$, which makes it expensive for parametrized problems, long time intervals or high frequencies. Since our paper focuses on the time-dependent part of the problem, we skip discussing the possible techniques to save computational costs in the geometry and parameter spaces. However, it is important to mention that to reduce these computational costs we can follow works in greedy algorithms \cite{grepl_posteriori_2005,Rozza2008a} for the parameter space, and works in adaptive mesh refinement \cite{Carlberg2014,Etter2019} and incomplete sampling \cite{everson_karhunenloeve_1995,venturi_gappy_2004,willcox_unsteady_2006,Peherstorfer2016} for the geometry.


%%%%%%%%%%%%%%%%%%%%%%%%%%%%%
\section{The Frequency Reduced Basis method}
\label{sec:FRB}
%%%%%%%%%%%%%%%%%%%%%%%%%%%%%

In this section we described an alternative method for the construction of the reduced basis. Instead of directly performing a time discretization to \cref{pr:sdp}, we use the Laplace transform $\mathcal{L}$ to describe the problem in the frequency domain. We can define the Laplace transform of a function $f$ as
\begin{equation} \label{eq:laplace}
	\IC_+ \ni s \mapsto \widehat{f}(s) \defeq \langle f(t),\e^{-st}\rangle = \int_{0}^{\infty}\e^{-st}f(t) \odif{t}.
\end{equation}
where $\IC_+ \defeq \{ s \in \IC: \Re(s) >0 \}$ denotes the positive half complex plane. Using the properties of the Laplace transform, we can define the composition with an steady-state operator as $\mathcal{L}\{\mathcal{A}f (t)\} = \mathcal{A}\widehat{f}(s)$, and the Laplace transform of the $n$-order derivative $f^{(n)}$ as 
\begin{equation}
    \mathcal{L}\{f^{(n)}\}=s^n\widehat{f}(s)-\sum_{k=1}^n s^{n-k}f^{(k-1)}(0).
\end{equation}

Assuming that the source term satisfies the space-time separation $b(\bm{x},t)=b_x(\bm{x})b_t(t)$, defining the Hardy space $H^2$ ---which we discuss in more detail in the next section, and setting $\widehat{\bm{\mathsf{u}}}(s) = \mathcal{L}\left\{ {\bm{\mathsf{u}}}(t)\right\}$ ---with the Laplace transform applied component-wise. We can write a frequency-domain version of \cref{pr:dp}.
\begin{problem}[Frequency discrete problem] \label{pr:fdp}
Seek $\widehat{\bm{\mathsf{u}}}_h \in H^2(\IC_+,\IV_h)$ such that
\begin{equation}
    \left(s^2 \bm{\mathsf{M}} + \bm{\mathsf{A}} \right) \widehat{\bm{\mathsf{u}}}(s) = \widehat{\bm{\mathsf{b}}}(s) - s \bm{\mathsf{u}}_{0,h} - \bm{\mathsf{u}}'_{0,h}.
\end{equation}
\end{problem}
Here is important to note that the spaces $\IV$ and $\IV'$ for the frequency domain solutions are complex valued. However we use the same notation as for the real-valued counterpart for simplicity. 

Analogously to the naïve construction of the reduced basis using time solutions in \cref{eq:tPOD}, we can construct the reduced basis following a \gls{pod}, but in this case over frequency domain. 
To this end we sample over a set of complex frequencies $\setJ_h = \left\{s_1,\dots,s_{n_s}\right\} \subset \IC_{+}$ with $n_s$ total frequencies, to build the manifold
\begin{equation} \label{eq:fsnap}
    \widehat{\bm{\mathsf{S}}} = \{ \widehat{\bm{\mathsf{u}}}(s_j) \; | s_j \in \setJ_h \} \in \IC^{n_h \times n_s},
\end{equation}
from which we can construct a basis $\Phi = \{\phi_1,\ldots,\phi_{n_r}\}$ such that
\begin{equation} \label{eq:fPOD}
	\Phi = \min_{\substack{\Phi \in \IR^{n_h \times n_r}: \\ \Phi^{\top} \Phi = \bm{I}_{n_r \times n_r}}}
    	\sum_{j=1}^{n_s} \norm*{\widehat{\bm{\mathsf{u}}}(s_j) - \sum_{k=1}^{n_r} \dotp{\Phi_k}{\widehat{\bm{\mathsf{u}}}(s_j)} \Phi_k
	}^2_{H^2(\IC_+,\IV_h)}
\end{equation}

Obtaining this frequency solution manifold requires the computation of $n_s$ linear systems of size $n_h$. There are two key differences with the time-dependent approach: the computation of the linear systems can be done in parallel as they do not need to be solve sequentially, and we need fewer samples in the frequency-domain to construct the reduced basis $(n_s < n_t)$. We solve the minimization problem in \cref{eq:fPOD} by computing the singular value decomposition $\widehat{\bm{\mathsf{S}}} = \sum_k^{n_r} \sigma_k \Phi_k \bm{\psi}_k^*$, and by using the first $n_r$ left-singular vectors we can define the reduced basis vectors as
\begin{equation}
	\varphi^{\textrm{R}}_k = \sum_{j=1}^{n_h} \left( \Phi_k	\right)_j \varphi_{j},
\end{equation}
and the reduced space as $\mathbb{V}^{\textrm{R}}_{h} = \spn \left\{ \varphi^{\textrm{R}}_1, \dots, \varphi^{\textrm{R}}_{n_r} \right\}$.

Note that while describing the method we are still missing to define the Hardy space $H^2$, and the sampling set $\setJ_h$. Let us for now define the sampling set as $\setJ_h = \{ s_j =\alpha + \iota \beta \cot{\theta_j} | \theta_j \in (0,\pi) \}$ with $\alpha, \beta \in \IR_+$. In \cref{sec:Analysis} we deal with these definitions and describe how we developed the method. In \cref{alg:basis} we summarize the proposed frequency reduced basis method.
\begin{algorithm}
\caption{Construction of the reduced basis} \label{alg:basis}
    Compute the Laplace $\mathcal{L}$ transform of the original time-dependent problem
    
    \For{$s_n =\alpha + \iota \beta \cot{\theta_n} \in \setJ_h | \theta_n \in (0,\pi)$}
        {Solve the frequency-domain problem for $\widehat{u}(\bm{x},s)$}
        
    Set the complex-valued data set. $\widehat{\bm{\mathsf{S}}} = \{ w_1\widehat{\bm{\mathsf{u}}}(s_1), \dots, w_{n_s}\widehat{\bm{\mathsf{u}}}(s_{n_s}) \} \in \IC^{n_h \times n_s}$, with $w_j = \frac{2\beta}{\sin^2 \theta_j}$ 

    Construct the basis $\Phi$ using a \gls{svd}. $\widehat{\bm{\mathsf{S}}} = \Phi \Sigma \Psi^*$
\end{algorithm}

\subsection{Time-dependent reduced order model} \label{ssec:rom}

Now that we have constructed the reduced basis, we can follow the traditional approach to model reduction for time dependent problems, by projecting \cref{pr:sdp} onto the reduced space $\IV_h^{\textrm{R}}$. We define $\mathsf{P}_h^{\textrm{R}}: \IV \rightarrow \IV_h^{\textrm{R}}$ to be the projection operator into $\IV_h^{\textrm{R}}$. We can then consider the reduced solution ansatz
\begin{equation}
    u_h^{\textrm{R}}(\bm{x},t) = \sum_{j=1}^{n_h} \varphi_{j} \sum_{k=1}^{n_r} \phi_{j,k} \mathsf{u}_k(t) = \sum_{j=1}^{n_h}  \sum_{k=1}^{n_r} \varphi_{j,k}^{\textrm{R}} (\bm{x}) \mathsf{u}_k(t) 
\end{equation}

\begin{problem}[Reduced semi-discrete problem] \label{pr:sdpr}
We seek $u^{\textrm{R}}_{h} \in \mathcal{C}^2(\setI, \IV_h^{\textrm{R}})$ such that for each $t \in \setI$ it holds
\begin{align}\label{eq:semi_discrete_rom}
	\frac{d^2}{dt^2} \dotp{u_h^{\textrm{R}}}{v_h^{\textrm{R}}}_{L^2(\Omega)} + \mathsf{a} \left(u_h^{\textrm{R}},v_h^{\textrm{R}} \right) = \dotp{b}{v_h^{\textrm{R}} }_{L^2(\Omega)}, \quad \forall t \in (0,t_{n_t}], \quad \text{and} \quad \forall v_h^{\textrm{R}} \in \IV_{h}^{\textrm{R}}
\end{align}
with initial conditions $u_h(0) = \mathsf{P}_{h}^{\textrm{R}} u^0\in \IV_h^{\textrm{R}}$ and $ \odv{}{t} u_h(0) = \mathsf{P}_{h}^{\textrm{R}} u^1\in \IV_h^{\textrm{R}}$.
\end{problem}

In the same way as for the high fidelity solution in \cref{pr:dp}, we can write a fully-discrete problem for the reduced order model using the same time integration scheme. 
We set $\bm{\mathsf{u}}^{\textrm{R}}(t) = (\mathsf{u}^{\textrm{R}}_1(t),\dots, \mathsf{u}^{\textrm{R}}_{n_r}(t))^{\top} \in \IR^{n_r}$, and likewise $\ddot{\bm{\mathsf{u}}}^{\textrm{R}}(t)$ and $\dot{\bm{\mathsf{u}}}^{\textrm{R}}(t) \in \IR^{n_r}$. We also define the reduced mass matrix $\bm{\mathsf{M}}^{\textrm{R}} \in \IR^{n_r \times n_r}$, the reduced stiffness matrix $\bm{\mathsf{A}}^{\textrm{R}} \in \IR^{n_r \times n_r}$, and the reduced source term as
\begin{equation}
	\bm{\mathsf{M}}^{\textrm{R}} = \Phi^\top \bm{\mathsf{M}} \Phi, \quad \bm{\mathsf{A}}^{\textrm{R}} = \Phi^\top \bm{\mathsf{A}} \Phi, \quad \text{and} \quad \bm{\mathsf{b}}^{\textrm{R}} = \Phi^\top \bm{\mathsf{b}}
\end{equation}
with $\Phi \in \IR^{n_h \times n_r}$ the discrete reduced basis.

\begin{problem}[Reduced fully-discrete problem] \label{pr:dpr}
We seek $\bm{\mathsf{u}}^{\textrm{R}}_{n+1} = \bm{\mathsf{u}}^{\textrm{R}}(t_{n+1}) \in \IR^{n_r}$, such that
\begin{equation}
    \bm{\mathsf{M}}^{\textrm{R}} \ddot{\bm{\mathsf{u}}}^{\textrm{R}}_{n+1-\alpha_{m}} + \bm{\mathsf{A}}^{\textrm{R}} \bm{\mathsf{u}}^{\textrm{R}}_{n+1-\alpha_{f}} = \bm{\mathsf{b}}^{\textrm{R}}_{n+1-\alpha_{f}},
\end{equation}
using the same approximations and parameters as for the high fidelity problem.
\end{problem}
