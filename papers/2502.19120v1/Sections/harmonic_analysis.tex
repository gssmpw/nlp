\section{Elements of harmonic analysis} \label{sec:Analysis}

We recall the most important elements of harmonic analysis that are relevant for the subsequent analysis. We follow \cite{partingtonBanachSpacesAnalytic,partingtonLinearOperatorsLinear2004,ricciHardySpacesOne}, \cite[Ch. 3, 7, 8]{hoffman_banach_2014}, \cite[Ch. 4]{RR97} and \cite[Section 6.4]{hille1996functional}. The main goal of this section is to state basic definitions, notation, and further properties of Hardy classes of vector-valued holomorphic functions, i.e. with values in either Hilbert or Banach spaces.


\subsection{The Laplace Transform and the Hardy space}

First, let us introduce an appropriate space for $\widehat{f}$. We set
\begin{align}
	\IC_\alpha \defeq\{s \in \mathbb{C}: \,\Re(s) > \alpha \}.
\end{align}
where $\alpha$ ensures that the singularities of $\widehat{f}$ are to the left of the semiplane $\IC_\alpha$.
\begin{definition}{Hardy space in $\IC_\alpha$.} \label{df:Hardy-plane}
    We denote $H^p(\IC_\alpha,\IV)$ the space of holomorphic functions $\widehat{f}: \Omega \rightarrow \IV$ in the positive half complex plane. For $1\leq p \leq \infty$ and any Banach space $\IV$ we define the norm
    \begin{equation} \label{eq:norm-c}
        \norm{\widehat{f}}_{H^p(\IC_\alpha, \IV)} = \sup_{\sigma>\alpha} \left(\int_{-\infty}^{\infty} \norm{\widehat{f}(\sigma + \imath \tau)}_{\IV}^p \odif{\tau} \right)^{\nicefrac{1}{p}} < \infty
    \end{equation}
\end{definition}
\begin{proposition}\label{prop:hardy-plane-limit1}
Let $p\in [1,\infty)$ and $\alpha>0$. For $f \in H^p(\IC_\alpha,\IV)$, the function 
	\begin{equation}
		T(\sigma,f) = \int\limits_{-\infty}^{\infty} \norm*{\widehat{f}(\sigma+\imath \tau)}^2_\IV \odif{\tau}
	\end{equation}
	is continuous monotone decreasing function of $\sigma$ for $\sigma \geq \alpha$. 
	In particular, $T(\alpha,f) = \norm{f}^p_{H^p(\IC_\alpha,\IV)}$ and $\displaystyle\lim_{\sigma \rightarrow \infty} T(\sigma,f)  = 0$.
\end{proposition}
%combine or rewite better both propositions

\begin{proposition} \label{prop:hardy-plane-limit2}
    The function $\widehat{f}(\sigma+\imath \tau)$ has boundary values for $\widehat{f}_{\alpha}(\imath \tau) = \lim_{\sigma\rightarrow \alpha^{+}} \widehat{f}(\sigma+\imath \tau)$ almost everywhere. The boundary function $\widehat{f}_{\alpha}$ lies in $L^p(\imath\IR,\IV)$ and satisfies $\lVert \widehat{f}_{\alpha} \rVert_{L^{p}(\imath\IR,\IV)} = \lVert \widehat{f} \rVert_{H^{p}(\IC_\alpha,\IV)}$. By identifying $\widehat{f}$ with $\widehat{f}_{\alpha}(\imath \tau)$, $H^p(\IC_\alpha,\IV)$ can be regarded as a closed subspace of $L^p(\imath\IR,\IV)$. %Partington p.7
\end{proposition}


\subsection{Isometry between time and frequency domains}

\begin{theorem} {Paley-Wiener.} \label{th:PW}
    $\widehat{f} \in L^2(\imath\IR)$ if and only if there exists a function $f \in L^2(\IR)$ such that $\widehat{f}(s) = \mathcal{L}(f)$.
    We define $L^2(\IR)=L^2(-\infty,0)\oplus L^2(\IR_+)$ by decomposing the function into its values $t<0$ and $t>0$. Then, by this decomposition we see
    \begin{equation}
       (L^2(\IR)=L^2(\mathbb{R}_-)\oplus L^2(\IR_+)) \xrightarrow{\makebox[0.5cm]{$\mathcal{L}$}} (L^2(i\IR)=H^2(\IC_-)\oplus H^2(\IC_+)). 
    \end{equation}
    We can define the Laplace transform as the map $\mathcal{L}:L^2(\IR_+;\IV) \mapsto H^2(\IC_+;\IV)$.
\end{theorem}

\begin{corollary}\label{cor:paley_wiener_alpha}
\begin{equation}
	\norm{f}_{L^2(\mathbb{R}_+,\IV,\alpha)} = \frac{1}{\sqrt{2\pi}} \norm*{\mathcal{L}\{f\}}_{H^2(\IC_\alpha,\IV)}
\end{equation}
\end{corollary}

\begin{proof}
For $f \in L^2(\IR_+,\IV,\alpha)$ with $\alpha>0$ we set $f_\alpha(t) = \e^{-\alpha t}f(t)$. Recalling the properties of the Laplace transform
we have that
\begin{equation}\label{eq:shift_laplace_transform}
	\mathcal{L} \{f_\alpha\}(s)	= \mathcal{L}\{f\}(s+\alpha), \quad \Re\{s\} > 0.
\end{equation}
We observe that $f_\alpha \in L^2(\IR_+,\IV)$. Hence, it follows from \cref{th:PW} that
\begin{equation}\label{eq:paley_wiener_alpha}
	\norm{f_\alpha}_{L^2(\mathbb{R}_+, \IV)} = \frac{1}{\sqrt{2\pi}} \norm*{\mathcal{L}\{f_\alpha\}}_{H^2(\IC_+,\IV)}.
\end{equation}
By shifting the vertical integration line in the definition of
$H^2(\IC_\alpha,V)$ and recalling \eqref{eq:shift_laplace_transform}, we obtain 
\begin{equation}\label{eq:shift_frequency}
\begin{aligned}
	\norm{ \mathcal{L}\{f_\alpha\}}_{H^2(\IC_+,\IV)} &= \sup_{\sigma>0} \left(\int_{-\infty}^{-\infty} \norm{\mathcal{L}\{f_\alpha\}(\sigma+\imath \tau)}^2_\IV  \odif{\tau} \right)^{\frac{1}{2}} \\
    &= \sup_{\sigma>0} \left( \int_{-\infty}^{-\infty} \norm{\mathcal{L} \{f \}(\sigma+\alpha+\imath\tau) }^2_\IV \odif{\tau} \right)^{\frac{1}{2}} \\
	&= \sup_{\sigma>\alpha} \left( \int_{-\infty}^{-\infty} \norm{\mathcal{L} \{f\}(\sigma+\imath\tau)}^2_\IV \odif{\tau} \right)^{\frac{1}{2}} \\
	&= \norm{ \mathcal{L}\{f\} }_{H^2(\IC_\alpha,\IV)}.
\end{aligned}
\end{equation}
From \cref{eq:normalpha}, we also know that 
\begin{equation}\label{eq:shift_time}
	\norm{f_\alpha}_{L^2(\mathbb{R}_+,\IV)} = \norm{f}_{L^2(\mathbb{R}_+,\IV,\alpha)}.
\end{equation}
By combining \cref{eq:shift_time} and \cref{eq:shift_frequency}, together with \cref{eq:paley_wiener_alpha}, one obtains the final result.
\end{proof}


\subsection{Hardy spaces in the unit disk \texorpdfstring{$\ID$}{D} and the unit circle \texorpdfstring{$\IT$}{T}}

Since the behaviour of $\widehat{f}_\alpha$ can be highly oscillatory as
$\Im \{s\}\rightarrow\infty$ and therefore \cref{eq:norm-c} would be difficult to compute, we apply a transformation to map $s$ to a new complex variable that has a better behaviour. Let $\ID = \{ \omega \in  \IC :\norm{\omega}<1 \}$ denote the unit disk in the complex plane and $\IT = \{ \omega \in \IC :\norm{\omega}=1 = \e^{i\theta}: 0\leq \theta \leq 2\pi \}$ the boundary of $\ID$. There is a natural isomorphism between the Hardy spaces in $\IC_+$ and $\ID$, and by extension between $\IC_\alpha$ and $\ID$. We apply the Möbius transformation $\mathcal{M}:\IC_\alpha \mapsto \ID$, given by $\mathcal{M}(s) = \omega = \frac{s-\alpha-\beta}{s-\alpha+\beta}$ to map $\widehat{f}_\alpha(s)$ in the half plane to the unit disk.

By mapping the domain from $\IC_\alpha$ we ensure that the singularities of $\widehat{f}(s)$ in the half-plane are mapped to the exterior of the unit disk $\ID$. An exception occurs when the singularity is located at infinity. By increasing $\beta$ we extend the sampling to higher frequencies (approaching $\infty$ in the Bromwich integral) and therefore we should carefully choose this parameter in function of the mesh size and other parameters that we discuss later on.
\Cref{fig:mobius} shows an illustration of the mapping from different contours $\Re s \geq \sigma$, and how the contours approach to $1$ when $\Re s \rightarrow \infty$. 
\begin{figure}[ht]
    \centering
    \input{figures/mobius.pgf}
    \vspace{-1cm}
    \caption{Illustration of the Möbius transformation with $\alpha=1$ and $\beta=30$.}
    \label{fig:mobius}
\end{figure}

Let us now define the appropriate space for the function $\widehat{f}$ in the unit disk and some theorems regarding the norm and the interoperability between the Hardy space in the disk and the half plane.
\begin{definition}{Hardy space in $\ID$.} \label{df:Hardy-disk}
     We denote $H^p(\ID,\IV)$ the space of holomorphic functions $\widehat{g}: \Omega \rightarrow \IV$ in the unit disc for which the functions $\widehat{g}(r\e^{\imath \theta})$ are bounded in $L^p$-norm as $r \rightarrow 1$. For $1\leq p \leq \infty$ and any Banach space $\IV$ we define the norm
    \begin{equation} \label{eq:HC-norm}
        \norm{\widehat{g}}_{H^p(\ID, \IV)} = \sup_{0 \leq r < 1} \left( \frac{1}{2\pi} \int_{0}^{2\pi} \norm{\widehat{g}(r\e^{i\theta})}_{\IV}^p \odif{\theta} \right)^{\nicefrac{1}{p}} < \infty
    \end{equation}
\end{definition}
\begin{theorem}{Fatou's theorem.} \label{th:Fatou}
    The functions $\widehat{g} \in H^p(\ID)$ have a radial limit $$\widehat{g}_\circ (\e^{i\theta}) = \lim_{r\rightarrow 1} \widehat{g}(r\e^{i\theta})$$ that exist almost everywhere in $\IT$, with $\widehat{g}_\circ \in L^p(\IT,\IV)$, and $\norm{\widehat{g}}_{H^p(\IC_\alpha,\IV)} = \norm{\widehat{g}_\circ}_{L^p(\IT,\IV)}$. We also identify $\widehat{g}$ with $\widehat{g}_\circ$, setting $H^p(\ID,\IV)$ as a closed subspace of $L^p(\IT,\IV)$ and a Banach space. A proof of this theorem is available in \cite{RR97}
\end{theorem}

\begin{theorem} \label{th:Hoffman}
    A function $\widehat{g}$ defined on $\IT$ is in $L^p(\IT,\IV)$ if and only if the function $\widehat{f}_{\alpha}: \imath \IR \mapsto \IC$ defined by 
    \begin{equation} \label{eq:mapDH}
        \widehat{f}_{\alpha}(s) = \frac{\beta^{\nicefrac{1}{p}}}{\pi^{\nicefrac{1}{p}} (\beta-\alpha+s)^{\nicefrac{2}{p}}}\widehat{g}(\mathcal{M}(s))
    \end{equation}
    is in $L^p(\imath\IR,\IV)$. Setting $\mathcal{M}^{-1}(\omega) = s = \frac{2\beta}{1-\omega}+\alpha -\beta$ we also verify
    \begin{equation} \label{eq:mapHD}
        \widehat{g}(\omega) = \frac{(4 \beta \pi)^{\nicefrac{1}{p}}}{(1-\omega)^{\nicefrac{2}{p}}}\widehat{f}_{\alpha}(\mathcal{M}^{-1}(\omega)),
    \end{equation}
    and moreover $\norm{\widehat{f}_{\alpha}}_{L^p(\imath\IR,\IV)} = \norm{\widehat{g}}_{L^p(\IT,\IV)}$.
\end{theorem}
\begin{proof}
    We follow a similar proof as the one in \cite{partingtonBanachSpacesAnalytic}, for a similar Möbius transformation. First we check that $\abs{\omega} <1$ if and only if $\Re \mathcal{M}^{-1}(s) >0$, and that $\mathcal{M}^{-1} \circ \mathcal{M} = I$. Then we have
    \begin{align}
        \norm*{\widehat{f}_{\alpha}}_{L^p(\imath\IR, \IV)}^p = \int_{-\infty}^\infty \norm*{\widehat{f}(s)}^p_{\IV} \odif{s} = - \imath \int_{\imath\IR} \frac{\beta}{\pi \abs{\beta-\alpha+s}^2} \norm*{\widehat{g}(\mathcal{M}(s))}^p_{\IV} \odif{s}.
    \end{align}
    Using $\mathcal{M}^{-1}(s)$, we replace $\odv{s}{\omega}=-\frac{2\beta}{(1-\omega)^2}$, and $\beta-\alpha+s = \frac{2\beta}{1-\omega}$. Noting that $\frac{\abs{1-\omega}^2}{(1-\omega)^2} = -\frac{1}{\omega}$ for $\omega \in \IT$, we get
    \begin{align}
        \norm*{\widehat{f}_{\alpha}}_{L^p(\imath\IR, \IV)}^p = - \imath \int_{\abs{\omega}=1} \frac{\abs{1-\omega}^2 \beta}{4\beta^2 \pi} \norm*{\widehat{g}(\omega)}^p_{\IV} \frac{2\beta}{(1-\omega)^2} \odif{s} = - \frac{1}{2\pi} \imath \int_{\abs{\omega}=1} \frac{1}{\omega} \norm*{\widehat{g}(\omega)}^p_{\IV} \odif{\omega}.
    \end{align}
    Lastly, writing $\omega=\e^{\imath\theta}$ and $\odif{\omega} = \imath \e^{\imath \theta} d\theta$, we get
    \begin{equation}
        \norm*{\widehat{f}_{\alpha}}_{L^p(\imath\IR, \IV)}^p = - \frac{1}{2\pi} \imath \int_{0}^{2\pi} \frac{\imath \e^{\imath \theta}}{\e^{\imath\theta}} \norm*{\widehat{g}(\e^{\imath \theta})}^p_{\IV} \odif{\theta} = \frac{1}{2\pi} \int_{0}^{2\pi} \norm*{\widehat{g}(\e^{\imath \theta})}^p_{\IV} \odif{\theta}.
    \end{equation}
\end{proof}

\begin{corollary} \label{cor:isometry}
Let $\widehat{f} \in H^2(\IC_\alpha)$. The mapping $\mathcal{D}: H^2(\IC_\alpha) \mapsto H^2(\ID)$ defined by 
\begin{equation} \label{eq:isometry}
    \mathcal{D}\left( \widehat{f} \right)(\e^{\imath \theta}) =  \frac{2 \sqrt{\beta \pi}}{1-\e^{\imath \theta}}\widehat{f} \left(\mathcal{M}^{-1}(\e^{\imath \theta}) \right)
\end{equation}
is an isometric isomorphism.
\end{corollary}

\begin{proof}
    A direct consequence of \cref{prop:hardy-plane-limit2,th:Fatou} is the isometry between the Hardy spaces in the disk and the half-plane and its boundary values. Moreover, \cref{th:Hoffman} is easy to proof for $p=2$ by replacing \cref{eq:isometry} and $\mathcal{M}^{-1}(\e^{\imath \theta})$ in \cref{eq:mapHD} as done for \cref{eq:mapDH}.
    We also compute the norm of the map $\mathcal{D}$ in \cref{eq:isometry} as
    \begin{align}
        \norm*{\mathcal{D} \left( \widehat{f} \right)}_{L^2(\IT,\IV)} &= \frac{1}{2\pi} \int_0^{2\pi} \frac{4 \beta \pi}{\abs*{1-\e^{\imath \theta}}^2} \norm*{\widehat{f}(\mathcal{M}^{-1}(\e ^{\imath\theta}))}^2_{\IV} \odif{\theta} \\
        &= \int_0^{2\pi} \frac{2 \beta}{\abs*{1-\e^{\imath \theta}}^2} \norm*{\widehat{f} \left( \alpha + \beta \frac{1+\e^{\imath \theta}}{1-\e^{\imath \theta}} \right)}^2_{\IV} \odif{\theta} \\
        &= \int_0^{\pi} \frac{2\beta}{\sin^2 \theta} \norm*{\widehat{f} \left( \alpha + \beta \cot \theta \right)}^2_{\IV} \odif{\theta}  \label{eq:int-rule}
\end{align}
 \end{proof}


\subsection{Frequency reduced basis method} \label{ssec:FRB-harmonic}

Using the isometries presented in \cref{th:PW,cor:paley_wiener_alpha} we justify the use of the \gls{pod} in the frequency domain instead of the standard time-domain one in \cref{eq:tPOD}.
Then, using \cref{eq:int-rule} we write a \gls{pod} analogous to \cref{eq:fPOD} but this time evaluated in functions over the unit circle $\IT$. We want to find the reduced basis $\Phi = \{\phi_1,\ldots,\phi_{n_r}\}$ such that
\begin{equation} \label{eq:fPOD-circle}
	\Phi = \argmin_{\substack{\Phi \in \IR^{n_h \times n_r}: \\ \Phi^{\top} \Phi = \bm{I}_{n_r \times n_r}}}
    	\sum_{j=1}^{n_s} w_j \norm*{\widehat{\bm{\mathsf{u}}}(s_j) - \sum_{k=1}^{n_r} \dotp{\phi_k}{\widehat{\bm{\mathsf{u}}}(s_j)} \phi_k
	}^2_{L^2(\IT,\IV_h)},
\end{equation}
where $s_j = \alpha + \imath \beta \cot \theta_j$ and $w_j = \frac{2\beta}{\sin^2 \theta_j}$, for $\theta_j \in (0,\pi)$. \footnote{A similar approximation to the minimization problem can be  obtained using the integration quadrature scheme in \cite{boyd_1987} in the infinite interval integral in $\norm{f}^p_{H^p(\IC_\alpha,\IV)}$.}


\subsection{Real-valued and complex-valued basis} \label{ssec:real-basis}
We introduce few theorems that allow us to infer another properties of the proposed method: analyticity and phase shift. A more detailed explanation of the theorems we use may be found in \cite{rudin_real_1987,king_hilbert_2009_1,titchmarsh_introduction_1986}.
\begin{definition}{Hilbert transform.} \label{df:Hilbert}
    The Hilbert transform of $f$ can be thought of as the convolution of $f(t)$ with the function $h(t) = \frac{1}{\pi t}$, known as the Cauchy kernel. This represents a phase-shift of $\frac{\pi}{2}$ to the components of the function $f$. The Hilbert transform of a function  $f(t)$ is given by
    \begin{equation}\label{eq:htrans}
        \mathcal{H} f(x) = \frac{1}{\pi} \dashint_{-\infty}^{\infty} \frac{f(t-\tau)}{t} \odif{t} = \frac{1}{\pi} \lim_{\epsilon \rightarrow 0^+} \int_{|t|> \epsilon} \frac{f(t-\tau)}{t} \odif{t}
    \end{equation}
\end{definition}

\begin{theorem}{Titchmarsh.} \label{th:Titchmarsh}
Let $\widehat{f} \in L^2(\imath\IR)$. If $\widehat{f}(s)$ satisfies any one of the following four conditions, then it satisfies all four conditions. The real and imaginary parts of $\widehat{f}(s)$, $\Re \{ \widehat{f}(s) \}$, and $\Im \{\widehat{f}(s)\}$, respectively, satisfy the following:
    \begin{enumerate} [label=(\roman*)]
    \itemsep0em 
    \item $\Im \widehat{f}(s) = \mathcal{H} \Re \widehat{f}(s)$;
    \item $\Re \widehat{f}(s) = -\mathcal{H} \Im \widehat{f}(s)$;
    \item If $f(t)$ denotes the inverse Fourier-Laplace transform \footnote{Note that the Laplace transform is equivalent to the Fourier transform with a change of variable for $f \in \IR_+$.} of $\widehat{f}(s)$, then $f(t)=0$, for $t<0$;
    \item $\widehat{f}(\sigma + \imath \omega)$ is an analytic function in the upper half plane and, for almost all $x$, $\widehat{f}(\sigma) = \lim_{y \rightarrow 0+} \widehat{f}(\sigma + \imath \omega)$.
    \end{enumerate} 
\end{theorem}

\begin{proposition}{} \label{prop:HL2}
    For $f \in L^2(\IR)$, $\widehat{f} = \mathcal{L} \{f\}$, the Hilbert transform $\mathcal{H}f$ is bounded in $L^2(\IR)$ as
    \begin{equation} \label{eq:norm-hilbert}
        \norm{\mathcal{H}f}_{L^2(\IR)} = \norm{\widehat{\mathcal{H}f}}_{L^2(\imath \IR)} = \norm{\widehat{f}}_{L^2(\imath \IR)} =\norm{f}_{L^2(\IR)}.
    \end{equation}
\end{proposition}

\begin{proof}
    We present here a informal proof similar to the one in \cite{taoLECTURENOTES247A}. A more formal proof using Sochocki-Plemej theorem can be found in \cite{taoLECTURENOTES247A,chaudhury_lp_2012}. We write the Hilbert transform as the convolution $\mathcal{H}f = f * \frac{1}{\pi t}$ and by applying the Laplace transform and the convolution theorem we write
    \begin{equation}
        \widehat{\mathcal{H}f} (s) = \widehat{f}(s) \widehat{\frac{1}{\pi t}}(s) =-\imath \sgn(s)\widehat{f}(s).
    \end{equation}
    for $s \in \imath \IR$.
    Now, by using the Plancherel-Parseval identity we get
    \begin{equation}
        \int \abs*{\widehat{\mathcal{H}f}(s)}^2\odif{s} = \int \abs*{\widehat{f}(s)}^2\odif{s} = \int \abs*{f(t)}^2\odif{t}.
    \end{equation}
\end{proof}

\begin{corollary} \label{cor:H-rot}
    For $\widehat{f} \in H^2(\IC_\alpha)$ and $f=\mathcal{L}^{-1} \{\widehat{f} \} \in L^2(\IR_+)$, the norm of the limit function $\widehat{f}_\alpha$ can be defined using only its real part as
    \begin{align}
        \norm*{ \widehat{f}_{\alpha} }_{L^2(\imath\IR,\IV)} = \sqrt{2} \norm*{ \Re \{\widehat{f}_{\alpha} \} }_{L^2(\imath\IR,\IV)}
    \end{align}
    
\end{corollary}
\begin{proof}
    Noting that the function is only defined for $t>0$, we verify that the third condition in \cref{th:Titchmarsh} is fulfilled and therefore the other three. Also using \cref{prop:HL2} and then writing $\widehat{f}_\alpha$ as its real and imaginary parts, we get
    \begin{align}
        \norm*{\widehat{f}_{\alpha}}_{L^2(\imath\IR,\IV)} &= \int_{-\infty}^{\infty} \norm*{ \Re \{\widehat{f}_\alpha(s)\} + \imath \mathcal{H}\Re \{\widehat{f}_\alpha(s) \}}^2_{\IV} \odif{s} \\
        &= \int_{-\infty}^{\infty} \left( \norm*{ \Re \{\widehat{f}_\alpha(s)\}} + \norm*{\mathcal{H}\Re \{\widehat{f}_\alpha(s) \}}^2_{\IV} \right) \odif{s} = \int_{-\infty}^{\infty} \left( \norm*{ \Re \{\widehat{f}_\alpha(s)\}} + \norm*{\Re \{\widehat{f}_\alpha(s) \}}^2_{\IV} \right) \odif{s}
    \end{align}
\end{proof}

Using the results in \cref{cor:H-rot} we write a version of \cref{eq:fPOD}, where we compute the basis using only the real part of the frequency solutions. 
\begin{equation} \label{eq:fPOD-real}
    \Phi =  \argmin_{\substack{\Phi \in \IR^{n_h \times n_r}: \\ \Phi^{\top} \Phi = \bm{I}_{n_r \times n_r}}}
    \sum_{j=1}^{n_s} w_j \norm*{\Re \{\widehat{\bm{\mathsf{u}}}(s_j) \} - \sum_{k=1}^{n_r} \dotp{\phi_k}{\Re \{ \widehat{\bm{\mathsf{u}}}(s_j) \}} \phi_k}^2_{L^2(\IT,\IV_h)}
\end{equation}


\subsection{Orthogonal, complex and symplectic structure of the basis} \label{ssec:complexified-basis}

In this part we want to further discuss the results from \cref{ssec:real-basis} using the scope of differential geometry.
Since in this section we work with both real and complex-valued spaces $\IV_h$, we indicate them as $\IV_h^{\IR}$ and $\IV_h^{\IC}$.
Let us write the reduced solution in the frequency domain solution space $\widehat{u}_h^{\textrm{R}} \in \IV_h^{\IC,\textrm{R}}$ in terms of its real and imaginary parts as $\widehat{u}_h^{\textrm{R}} = \Re \{ \widehat{u}_h^{\textrm{R}} \} + \imath \Im \{ \widehat{u}_h^{\textrm{R}} \}$. Since in this section we work with both real and complex-valued spaces $\IV_h$, we indicate them as $\IV_h^{\IR}$ and $\IV_h^{\IC}$. Using the results from \cref{th:Titchmarsh,prop:HL2}, let us also consider the following matrix representation of the complex-valued reduced solution
\begin{equation} \label{eq:complex-matrixform}
    \widehat{u}_h \approx \Phi^\IC \widehat{u}_h^{\textrm{R}} =
    \begin{bmatrix}
        \Phi^\Re & -\Phi^\Im \\
        \Phi^\Im & \Phi^\Re
    \end{bmatrix}
    \begin{bmatrix}
        \Re \{ \widehat{u}_h \}  \\
        \Im \{ \widehat{u}_h \}
    \end{bmatrix} =
    \begin{bmatrix}
        \Phi^\Re & -\Phi^\Im \\
        \Phi^\Im & \Phi^\Re
    \end{bmatrix}
    \begin{bmatrix}
        \Re \{ \widehat{u}_h \}  \\
        \mathcal{H} \Re \{ \widehat{u}_h \}
    \end{bmatrix},
\end{equation}
with $\Phi^\Re$ and $\Phi^\Im$ the real and imaginary parts of the complex-valued basis $\Phi^\IC$. Note that by using \cref{eq:fPOD-circle} we construct the complex-valued basis $\Phi^\IC$, as we do it by solving an \gls{svd} of a complex-value data set. As we discussed in this section, and as a direct consequence of \cref{th:PW,cor:isometry} the reduced basis we are constructing is a basis for both the frequency-domain and the time-domain problems. Thus, we also write the real-value reduced solution for the time domain as
\begin{equation} \label{eq:real-matrixform}
    u_h \approx \Phi^\IC u_h^{\textrm{R}} =
    \begin{bmatrix}
        \Phi^\Re & -\Phi^\Im \\
        \Phi^\Im & \Phi^\Re
    \end{bmatrix}
    \begin{bmatrix}
        \Re \{ u_h^{\textrm{R}} \}  \\
        \Im \{ u_h^{\textrm{R}} \}
    \end{bmatrix} =
    \begin{bmatrix}
        \Phi^\Re & -\Phi^\Im \\
        \Phi^\Im & \Phi^\Re
    \end{bmatrix}
    \begin{bmatrix}
        \Re \{ u_h^{\textrm{R}} \}  \\
        \mathcal{H} \Re \{ u_h^{\textrm{R}} \}
    \end{bmatrix}.
\end{equation}
Since we know that $u_h^{\textrm{R}}$ in the time-dependent problem is real-valued, we get
\begin{equation} \label{eq:realfromcomplex}
    \Phi^\Im \Re \{ u_h^{\textrm{R}} \} + \Phi^\Re \Im \{ u_h^{\textrm{R}} \} = \Phi^\Im \Re \{ u_h^{\textrm{R}} \} + \Phi^\Re \mathcal{H} \Re \{ u_h^{\textrm{R}} \} = 0.
\end{equation}

This matrix representation implies that the complex $n_s$-dimensional space $\IV^\IC_h$, can be define using a real space $\IV^\IR_h$. To explain this let us start by writing some basic definitions and theorems used in holomorphic differential geometry, for which we follow \cite[chapter~4]{lewisNotesGlobalAnalysis} and \cite{maninLinearAlgebraGeometry1989}.
\begin{definition} {$\mathcal{H}$ as the linear complex structure of $\IV_h^\IR$.}
    We define a linear complex structure $\mathcal{J}$ on a real vector space $\IV^\IR$ as the endomorphism $\mathcal{J}: \IV_h^\IR \to \IV_h^\IR$ such that $\mathcal{J} \cdot \mathcal{J} =- \mathcal{I}$, with $\mathcal{I}$ the identity.
    Noting that by definition the Hilbert transform is an anti-involution $\mathcal{H}^2 = -\mathcal{I}$, we define the endomorphism
\begin{equation} \label{eq:J-complex}
    \mathcal{J}_{\mathcal{H}} = 
    \begin{bmatrix}
        0 & -\mathcal{H} \\
        \mathcal{H} & 0
    \end{bmatrix}
\end{equation}
as a linear complex structure on $\IV_h^\IR$.
\end{definition}



\begin{definition} {Real structure of $\IV_h^\IC$.} \label{def:realstruc}
    A real structure on a complex vector space $\IV^\IC$ is an antilinear map $\varsigma: \IV_h^\IC \to \IV_h^\IC$ such that $\varsigma \cdot \varsigma = \mathcal{I}$. Let us write any $u \in \IV_h^\IC$ as the sum of $\widehat{u} = \widehat{u}_\Re + \widehat{u}_\Im = \frac{1}{2} (\widehat{u} + \varsigma \widehat{u}) + \frac{1}{2} (\widehat{u} - \varsigma \widehat{u})$. This way, we define the subspaces
    \begin{align}
        \IV_h^\Re = \{\widehat{u} \in \IV_h^\IC | \varsigma \widehat{u} = \widehat{u}^* \}, \quad \text{and} \quad
        \IV_h^\Im = \{\widehat{u} \in \IV_h^\IC | \varsigma \widehat{u} = -\widehat{u}^* \}
    \end{align}
    with $\widehat{u}^*$ the complex conjugate of $\widehat{u}$. Note that since $\widehat{u}_\Re$ and $\widehat{u}_\Im$ are orthogonal, then $\dim_\IR \IV_h^\Re = \dim_\IR \IV_h^\Im = \dim_\IC \IV_h^\IC$.
\end{definition}

\begin{definition} {Complexification and realification.} \label{def:complexification}
    For a real vector space $\IV_h^\IR$ we may define its complexification by extension of scalars and for a complex vector space $\IV_h^\IC$ we may define its realification by the restriction of scalars. We get the relation $\IV_h^{\IC} = \IC \otimes_{\IR} \IV_h^\IR$.
\end{definition}


\begin{proposition} {Basis representation of $\IV_h^\IC$ and $\IV_h^\IR$.} \label{prop:basisrc}
    Let $\{\varphi_1,\ldots,\varphi_{n_s}\}$ be a basis of $\IV_h^\IC$. Then, $\{\varphi_1,\ldots,\varphi_{n_s}, \mathcal{J}_\mathcal{H}(\varphi_1),\ldots,\mathcal{J}_\mathcal{H}(\varphi_{n_s})\}$ forms a real basis for $\IV_h^\IR$. Note that $\dim_\IR \IV_h^\IR = 2\dim_\IC \IV_h^\IC$.
\end{proposition}

\begin{proposition}{Totally real subspace.} \label{prop:realsubs}
    Let $\IV_h$ be a finite-dimensional $\IR$-vector space with linear complex structure $\mathcal{J}_{\mathcal{H}}$. A subspace $\IV_h^{\IR} \subseteq \IV_h$ is totally real if $\mathcal{J}_{\mathcal{H}}(\IV_h^{\IR}) \cap \IV_h^{\IR} = \{0\}$. The next two conditions are equivalent 
    \begin{itemize}
        \item $\IV_h^{\IR}$ is a totally real subspace.
        \item $\mathcal{J}_{\mathcal{H}}(\IV_h^{\IR})$ and $\IV_h^{\IR}$ are orthogonal in the inner product sense.
    \end{itemize}
\end{proposition}

\begin{corollary} \label{cor:j-basis}
    Let $\IV_h^{\textrm{R}}$ be a reduced space for the time dependent problem and let $\{\phi_1,\dots,\phi_{n_s}\}$ be a real-valued reduced basis of $\IV_h^\IR$, then $\{\phi_1,\dots,\phi_{n_s},\mathcal{J}_{\mathcal{H}}(\phi_i),\ldots,\mathcal{J}_{\mathcal{H}}(\phi_{n_r})\}$ is a basis of $\IV_h^\IC$.
\end{corollary}
\begin{proof}
    Let $v^{\textrm{R}} \in L^2(\IT,\IV_h^\IR)$. Then, to prove the orthogonality between $\phi_j$ and $\mathcal{J}_{\mathcal{H}}(\phi_j)$ we can show that any $v^{\textrm{R}}_j$ is orthogonal to $\mathcal{J}_{\mathcal{H}}(v^{\textrm{R}}_j)$. Noting that $\mathcal{H}$ is anti-self adjoint we write
    \begin{equation}
        \dual{\mathcal{H}(v_j)}{v_j} = \dual{\mathcal{H}^2(v_j)}{\mathcal{H}(v_j)} = \dual{v_j}{-\mathcal{H}(v_j)} = \dual{\mathcal{H}(v_j)}{v_j} =0.
    \end{equation}
    Using \cref{prop:realsubs} we see that $\IV_h^\IR$ is a totally real subspace and can be described with $\Phi^\Re$. We check that for \cref{eq:realfromcomplex} we satisfy $\Phi^\Im \Re \{ u_h^{\textrm{R}} \} + \Phi^\Re \Im \{ u_h^{\textrm{R}} \} =0$ is by assuming that $\Phi^\Re$ and $\Phi^\Im$ are orthogonal, and that $\Re \{ u_h^{\textrm{R}} \} \in \spn \{ \phi_1^\Re, \ldots, \phi_{n_r}^\Re \}$ and $\Im \{ u_h^{\textrm{R}} \} \in \spn \{ \phi_1^\Im, \ldots, \phi_{n_r}^\Im \}$.
\end{proof}

So far we have described real and complex vector spaces. Here it is clear that $\IV_h^\IC$ represents the frequency-domain solution space, but there is not an unique way to construct a real space or obtain a real-valued reduced solution. Moreover, we see that the spaces $\IV_h^\Re$ and $\IV_h^\IR$ are not necessarily equivalent.

Using \cref{def:realstruc,prop:basisrc} in $\widehat{\bm{\mathsf{S}}}$ (\cref{eq:fsnap}), and replacing $\mathcal{J}_{\mathcal{H}} (\Re \{ \widehat{\bm{\mathsf{u}}}_j \}) = \Im \{ \widehat{\bm{\mathsf{u}}}_j \}$ as per its definition. We get the rearranged set
\begin{equation} \label{eq:ssnap}
    \bm{\mathsf{S}}^S = \{ \Re \{ \widehat{\bm{\mathsf{u}}}_1 \}, \ldots, \Re \{ \widehat{\bm{\mathsf{u}}}_{n_s} \}, \Im \{ \widehat{\bm{\mathsf{u}}}_1 \}, \ldots, \Im \{ \widehat{\bm{\mathsf{u}}}_{n_s} \} \}
\end{equation}
Now, if we compute the \gls{svd} and obtained a basis for this rearranged set, we are effectively constructing a symplectic basis. This technique is the same as the one presented in \cite{peng_symplectic_2015} for Hamiltonian problems as the cotangent lift method.
\begin{theorem} {Cotangent lift method.} \label{th:symplectic}
    Let $\widehat{\bm{\mathsf{u}}}$ be the frequency discrete solutions and $\bm{\mathsf{S}}^S$ the data set as described in \cref{eq:ssnap}. The solution of the minimization problem in \cref{eq:fPOD-circle} is given by $\bm{\mathsf{S}}^S = \Phi^S \Sigma (\Psi^S)^*$, which results in the symplectic basis
    \begin{equation}
        \Phi^S =
    \begin{bmatrix}
        \Phi & 0 \\
        0 & \Phi
    \end{bmatrix}
    \end{equation}
    which is the optimal projection of the sample set $\widehat{\bm{\mathsf{S}}}$ onto the column space of the symplectic vector space $\IR^{2n_s}$.
\end{theorem}
\begin{proof}
    Writing  \cref{eq:fPOD-circle} in terms of the real and imaginary parts, we get
    \begin{align}
        &\norm*{
        \begin{bmatrix}
            \Re \{ \widehat{\bm{\mathsf{u}}} \} \\
            \Im \{ \widehat{\bm{\mathsf{u}}} \}
        \end{bmatrix} -
        \begin{bmatrix}
            \Phi & 0 \\
            0 & \Phi
        \end{bmatrix}
        \begin{bmatrix}
            \Phi^\top & 0 \\
            0 & \Phi^\top
        \end{bmatrix}
        \begin{bmatrix}
            \Re \{ \widehat{\bm{\mathsf{u}}} \} \\
            \Im \{ \widehat{\bm{\mathsf{u}}} \}
        \end{bmatrix}
        }_{L^2} =
        \norm*{
        \begin{bmatrix}
            \mathcal{I}_{n_s} -\Phi \Phi^\top \Re \{ \widehat{\bm{\mathsf{u}}} \} \\
            \mathcal{I}_{n_s} -\Phi \Phi^\top \Im \{ \widehat{\bm{\mathsf{u}}} \}
        \end{bmatrix}
        }_{L^2} \\ \vspace{0.2cm}
        &\quad = \norm*{
        \begin{bmatrix}
            \mathcal{I}_{n_s} -\Phi \Phi^\top
        \end{bmatrix}
        \begin{bmatrix}
            \Re \{ \widehat{\bm{\mathsf{u}}} \} &
            \Im \{ \widehat{\bm{\mathsf{u}}} \}
        \end{bmatrix}
        }_{L^2}
        = \norm*{
        \begin{bmatrix}
            \Re \{ \widehat{\bm{\mathsf{u}}} \} &
            \Im \{ \widehat{\bm{\mathsf{u}}} \}
        \end{bmatrix} -
        \Phi \Phi^\top
        \begin{bmatrix}
            \Re \{ \widehat{\bm{\mathsf{u}}} \} &
            \Im \{ \widehat{\bm{\mathsf{u}}} \}
        \end{bmatrix}
        }_{L^2}
    \end{align}
    giving the minimization
    \begin{equation} \label{eq:fPOD-sym}
        \Phi^S =  \argmin_{\substack{\Phi^S \in \IR^{n_h \times n_r}: \\ (\Phi^S)^{\top} \Phi^S = \bm{I}_{n_r \times n_r}}}
        \sum_{j=1}^{n_s} w_j \norm*{ \left[ \Re \{\widehat{\bm{\mathsf{u}}}(s_j) \}, \Im \{\widehat{\bm{\mathsf{u}}}(s_j) \} \right] - \sum_{k=1}^{n_r} \dotp{\phi_k^S}{\left[ \Re \{\widehat{\bm{\mathsf{u}}}(s_j) \}, \Im \{\widehat{\bm{\mathsf{u}}}(s_j) \} \right]} \phi_k^S}^2_{L^2(\IT,\IV_h)}
\end{equation}
\end{proof}

\begin{remark}
    The equivalence between an orthogonal basis, a linear complex structure and a symplectic basis, suggest the existence of a Kähler manifold as an underlying structure of the reduced space for the time-domain and the frequency-domain problems.
\end{remark}


Using the results and definitions from this section, we define different approaches to the construction of the reduced basis for the time-dependent problem.
\begin{itemize}
    \item (\Cref{th:PW,cor:isometry}). The complex-valued basis $\Phi_{\IC}$ obtained from \cref{eq:fPOD-circle}. This is equivalent as constructing a reduced space for $\IV_h^\IC$, which after \cref{cor:j-basis} we know results in a real-valued solution $u_h^{\textrm{R}}$
    \item (\Cref{th:Titchmarsh,cor:H-rot}). The real-valued basis $\Phi_\IR$ obtained from \cref{eq:fPOD-real}. This is equivalent as the reduced basis for a space $\IV_h^\IR$. Note that due to the smaller dimension of this space (\cref{prop:basisrc}) we can expect a worse performing reduced basis when using the same number of samples $n_s$.
    \item (\Cref{prop:realsubs,cor:j-basis}). The real-valued basis $\Phi_{\Re \IC}= \Re \{\Phi_{\IC} \}$ obtained by solving \cref{eq:fPOD-circle} and taking the real part of the resulting basis. This is equivalent as the reduced basis of the space $\IV_h^\Re$. This basis naturally solves a real-valued problem and spans a space of the same size as $\Phi_\IC$.
    \item (\Cref{th:symplectic}). The symplectic basis $\Phi_{S}$ obtained from \cref{eq:fPOD-sym}. We can easily obtain this basis by computing an \gls{svd} of the rearranged data set in \cref{eq:ssnap}. With this construction we get a real-valued reduced basis that spans a space of the same size as $\Phi_\IC$.
\end{itemize}
In \cref{ssec:Heat} we show that all the different reduced bases solve adequately the time-dependent problem. We also see that the bases $\Phi_{\IC}$, $\Phi_{\Re \IC}$ and $\Phi_{S}$ show a behaviour that correspond to the equivalence suggested in this section.