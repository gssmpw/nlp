\documentclass{article}
%\usepackage{authblk}

\usepackage{geometry}
 \geometry{
 a4paper,
 total={170mm,257mm},
 left=25mm,
 top=25mm,
 right=25mm,
 bottom=25mm
 }
\usepackage[final]{microtype}

\usepackage{import}
\usepackage[utf8]{inputenc}
\usepackage[english]{babel} % Language hyphenation and typographical rules
\usepackage[T1]{fontenc} % Use 8-bit encoding that has 256 glyphs

\newcommand\hmmax{0}
\newcommand\bmmax{0}
\usepackage{bm}
\usepackage{amssymb}
\usepackage{amsmath}
\usepackage{sansmath}
\allowdisplaybreaks
\usepackage{amsthm}
%\usepackage{mathrsfs}
\usepackage{derivative}
\usepackage{accents}
\usepackage[scr=txupr]{mathalpha}

\newtheorem{theorem}{Theorem}
\newtheorem{corollary}{Corollary}
\newtheorem{definition}{Definition}
\newtheorem{remark}{Remark}
\newtheorem{proposition}{Proposition}
\newtheorem{lemma}{Lemma}
\newtheorem{problem}{Problem}
\newtheorem{assumption}{Assumption}

\DeclareMathOperator*{\argmax}{arg\,max}
\DeclareMathOperator*{\argmin}{arg\,min}

\usepackage{enumitem}

\usepackage{lscape}
\usepackage[nice]{nicefrac}
\usepackage{mathtools}
\def\mathdefault#1{#1}

\DeclareMathOperator*{\esssinf}{ess\,inf}
\DeclareMathOperator{\sgn}{sgn}
\DeclareMathOperator{\spn}{span}

\DeclarePairedDelimiter\abs{\lvert}{\rvert}
\DeclarePairedDelimiter\norm{\lVert}{\rVert}

\newcommand*{\defeq}{\mathrel{\vcenter{\baselineskip0.5ex \lineskiplimit0pt
                \hbox{\scriptsize.}\hbox{\scriptsize.}}}%
                =}
                     
\def\Xint#1{\mathchoice
   {\XXint\displaystyle\textstyle{#1}}%
   {\XXint\textstyle\scriptstyle{#1}}%
   {\XXint\scriptstyle\scriptscriptstyle{#1}}%
   {\XXint\scriptscriptstyle\scriptscriptstyle{#1}}%
   \!\int}
\def\XXint#1#2#3{{\setbox0=\hbox{$#1{#2#3}{\int}$}
     \vcenter{\hbox{$#2#3$}}\kern-.5\wd0}}
\def\ddashint{\Xint=}
\def\dashint{\Xint-}

\newcommand{\dual}[2]{\left \langle #1,#2 \right \rangle}
\newcommand{\dotp}[2]{\left ( #1,#2 \right )}


\DeclareMathAlphabet{\mathbbs}{U}{fplmbb}{m}{n}
\newcommand{\IR}{\mathbbs{R}}
\newcommand{\IC}{\mathbbs{C}}
\newcommand{\IZ}{\mathbbs{Z}}
\newcommand{\IN}{\mathbbs{N}}

\newcommand{\IT}{\mathbb{T}}
\newcommand{\ID}{\mathbb{D}}

\newcommand{\IX}{\mathbb{X}}
\newcommand{\IV}{\mathbb{V}}


\newcommand{\setT}{\mathscr{T}}
\newcommand{\setI}{\mathscr{I}}
\newcommand{\setJ}{\mathscr{J}}
\newcommand{\e}{\mathrm{e}}

\newcommand{\jh}[1]{{\color{magenta}{#1}}}
\newcommand{\rr}[1]{{\color{blue}{#1}}}
\newcommand{\fh}[1]{{\color{cyan}{[#1]}}}

\newcommand{\todo}[1]{{\color{red}[#1]}}

\usepackage{floatrow}
\floatsetup[table]{font=small}

\usepackage[ruled,vlined,linesnumbered]{algorithm2e}

\usepackage[bookmarks,hidelinks]{hyperref}
\usepackage[nameinlink,noabbrev]{cleveref}
\usepackage{orcidlink}
\usepackage{autonum}

\crefname{problem}{problem}{problems}
\Crefname{problem}{Problem}{Problems}
\crefname{enumi}{property}{properties}
\Crefname{Enumi}{Property}{Properties}

\usepackage{csquotes}

\numberwithin{equation}{section}

\usepackage{graphicx}
\usepackage{tikz}
\usepackage{multirow}
\usepackage{caption}
\usepackage{subcaption}
\usepackage{pgf}
\usepackage{pgfplots}
\pgfplotsset{compat=1.18}

\usepackage{tabularray}
\usepackage{booktabs}
\usepackage[square,numbers]{natbib}
\bibliographystyle{abbrvnat}

%\captionsetup{font=footnotesize}
%\captionsetup[sub]{font=footnotesize}

%\usepackage{lineno}
%\modulolinenumbers[5]




%----------------------------------------------------
%	ACRONYM SECTION
%----------------------------------------------------
\usepackage[acronym]{glossaries}

%\makeglossaries
 
\newacronym{pod}{POD}{Proper Orthogonal Decomposition}
\newacronym{fem}{FEM}{Finite Element Method}
\newacronym{svd}{SVD}{Singular Value Decomposition}
\newacronym[longplural={Reduced Order Models}]{rom}{ROM}{Reduced Order Model}
\newacronym[longplural={Full Order Models}]{fom}{FOM}{Full Order Model}

\newacronym[longplural={Spectral Elements}]{se}{SE}{Spectral Element}

\newacronym{rmsd}{RMSD}{Root-Mean-Square Deviation}

%-------------------------------------

\title{Reduced order models for time-dependent problems\\ using the Laplace transform}

\author{Ricardo Reyes \orcidlink{0000-0003-0140-9564} \, \footnote{\sc{Chair of Computational Mathematics and Simulation Science, Ecole Polytechnique Fédérale de Lausanne, Switzerland}} \footnote{\sc{Department Urban Water Management, Swiss Federal Institute of Aquatic Science and Technology, Switzerland}}}

\date{}

\begin{document}
\maketitle

\begin{abstract}
We propose a reduced basis method to solve time-dependent partial differential equations based on the Laplace transform. Unlike traditional approaches, we start by applying said transform to the evolution problem, yielding a time-independent boundary value problem that depends on the complex Laplace parameter. First, in an offline stage, we appropriately sample the Laplace parameter and solve the collection of problems using the finite element method.
Next, we apply a \gls{pod} to this collection of solutions in order to obtain a reduced basis that is of dimension much smaller than that of the original solution space. This reduced basis, in turn, is then used to solve the evolution problem using any suitable time-stepping method. A key insight to justify the formulation of the method resorts to Hardy spaces of analytic functions. By applying the widely-known Paley-Wiener theorem we can then define an isometry between the solution of the time-dependent problem and its Laplace transform. As a consequence of this result, one may conclude that computing a \gls{pod} with samples taken in the Laplace domain produces an exponentially accurate reduced basis for the time-dependent problem.
Numerical experiments characterizing the performance of the method, in terms of accuracy and speed-up, are included for a variety of relevant time-evolution equations.
\end{abstract}

%\keywords{Reduced order model (ROM), Time-dependent model reduction, Frequency-domain model reduction, Proper Orthogonal Decomposition (POD), $H^2$ Hardy space}


%\linenumbers

\subimport{}{Sections/introduction.tex}
\subimport{}{Sections/time_dependent_bvp.tex}
\subimport{}{Sections/harmonic_analysis.tex}
\subimport{}{Sections/numerical_results.tex}
\subimport{}{Sections/conclusions.tex}

%\section*{Acknowledgements}

%----------------------------------------
%	REFERENCE LIST
%----------------------------------------

\bibliography{references}

%----------------------------------------
\end{document}