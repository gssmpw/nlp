\section{Related Work}
\label{sec:related}

\subsection{Prescription Fitting Formulas}
Prescription fitting formulas aim to amplify sound based on the user's degree of hearing loss, making speech signals audible across different frequencies while maintaining comfort. These formulas can be broadly categorized into linear and non-linear approaches. Linear amplification, exemplified by methods like NAL-R and NAL-RP ____, provides a constant gain regardless of input volume and is generally preferred for individuals with flatter hearing loss profiles, larger dynamic ranges, and less variability in dynamic range across frequencies. It is also well-suited for those with less varied lifestyles and environments ____. NAL-R, developed by the Australian National Acoustic Laboratories, is commonly used for mild to moderate hearing loss, while NAL-RP is tailored for severe hearing loss. In contrast, non-linear amplification, such as NAL-NL1 ____, NAL-NL2 ____, and DSL m[i/o] ____, adjusts the gain dynamically based on the input volume, frequency, and the user's specific hearing loss characteristics. This makes it more suitable for individuals with sloping hearing loss, reduced dynamic ranges, significant variations in dynamic range across frequencies, and more diverse lifestyles and listening environments. NAL-NL1 aims to equalize and normalize the relative volume levels of individual frequencies while optimizing the speech intelligibility for a given volume ____. NAL-NL2, an improved version of NAL-NL1, considers additional factors in calculating gains, such as gender, age, and whether the native language is tonal, which may affect output volume. NAL-NL2 is also the most widely used fitting formula at present. In contrast, DSL m[i/o] is designed to prevent uncomfortable loudness while using hearing aids and to maximize the audibility of essential messages in conversations. Furthermore, these non-linear formulas utilize compression techniques to manage the reduced auditory dynamic range commonly experienced by individuals with sensorineural hearing loss.

\subsection{Hearing Aid Compression}

People with sensorineural hearing loss tend to have a smaller auditory dynamic range. For these individuals, soft sounds are inaudible, while intense sounds are perceived as loudly as they are by a normal-hearing ear. To optimize the use of the residual dynamic range and restore normal loudness perception, most modern hearing aids adopt compression. Wide Dynamic Range Compression (WDRC) ____ is a widely adopted compression strategy designed to prevent sounds from being either too loud or too soft. The use of WDRC in hearing aids provides several benefits, such as ensuring consistent audibility of speech signals at safe and comfortable hearing levels ____. Moreover, WDRC involves multiple steps of audio processing, which are ____:

\begin{itemize}
\item \textbf{Signal analysis through short-time Fourier transform (STFT):} The input audio signal undergoes signal analysis through STFT.

\item \textbf{Filterbank construction:} Frequency bins are grouped into a predefined number of filterbanks, and their volume levels are estimated.

\item \textbf{Calculation of filterbank-specific gain functions:} Gain functions specific to each filterbank are calculated based on the level estimations. 

\item \textbf{Interpolation and modification:} The gain functions are interpolated to individual frequency bins and then used to modify the STFT representation of the input signal. 

\item \textbf{Signal reconstruction:} The modified STFT representation is converted back to the time domain using the Inverse STFT.
\end{itemize}

These steps ensure that the WDRC system effectively compresses a wide range of volume levels while maintaining listener comfort. In addition, the WDRC considers the listener's hearing thresholds and dynamic range variations across frequencies, aiming to enhance speech intelligibility.