\section{Related Work}
Secret sharing was originally introduced by Blakley \cite{blakely} and Shamir \cite{shamir}, who proposed information-theoretic threshold schemes. For such schemes, Karnin et al. \cite{hellman2} proved that the size of each share must be at least as large as the secret itself. 

Krawczyk \cite{krawczyk1993secret} introduced the “Secret Sharing Made Short” (SSMS) scheme which achieves computationally secure threshold secret sharing with share sizes smaller than the information-theoretic bound. This technique combines computational encryption with an Information Dispersal Algorithm (IDA) \cite{rabin1990information,naor1995optimal,beguin1998general,1023682}. Krawczyk also explored robust computational secret sharing in the same work, addressing the challenge of reconstructing the secret correctly even in the presence of tampered or corrupted shares. However, robustness is not the focus of our work, which is instead aligned with the original SSMS setting of optimizing share sizes for threshold access structures under computational security assumptions.

Robust computational secret sharing was later revisited and formalized by Bellare and Rogaway \cite{bellare2007robust}, who proved the security of Krawczyk’s robust scheme in the random oracle model. They also introduced a refined construction, HK2, that achieves robustness under standard assumptions. Additional work has extended these ideas to general access structures, as in \cite{beguin1995general,cachin1995line,mayer1997generalized,vinod2003power}, where the goal is to support more flexible definitions of authorized subsets.

Another line of research is the AONT-RS scheme \cite{resch2011aont}, which combines an all-or-nothing transform (AONT) \cite{rivest1997all} with Reed-Solomon coding \cite{reed1960polynomial} for protecting stored data. Chen et al. \cite{chen2017revisiting} revisited AONT-RS, presenting a generalized version while pointing out that the scheme can leak information if the ciphertext size is too small relative to the security parameter and threshold. They further demonstrated that AONT-RS achieves weaker security guarantees compared to SSMS. 

Our work can be viewed within the framework of Krawczyk’s original SSMS, where the goal is to minimize the share size while ensuring computational secrecy. Unlike robust computational secret sharing, we do not address tampered shares or fault tolerance. Instead, we focus on optimizing the share size under computational security.