\section{Results}

\subsection{Main Results}
\Cref{fig:main_result} below demonstrates that \ours effectively learns to solve medical multiple-choice question tasks starting from a base model. Compared to SFT, \ours achieves comparable performance on the in-distribution MedQA test set but outperforms SFT significantly on the out-of-distribution MMLU-Pro-Health task, with an improvement of approximately 8 absolute points in accuracy. This result suggests that while SFT is a strong baseline for in-distribution tasks, it likely relies on spurious correlations and shortcuts to achieve high performance without reasoning. In contrast, the reasoning capabilities acquired from \ours are more robust and generalizable across tasks. This finding is in line with some recent observations that RL generalizes better than SFT~\citep{chu2025sft, shen2025vlmr1}.

\begin{figure}[!ht]
    \centering
    \includegraphics[width=.8\linewidth]{figures/rl_vs_sft.pdf}
    \caption{Comparing \ours and SFT on in-distribution and out-of-distribution tasks. Standard deviation from 1000 bootstrapping sampling procedures \citep{tibshirani1993introduction} is reported as error bars.}
    \label{fig:main_result}
\end{figure}
\subsection{Analysis on the emerging reasoning traces}

\subsubsection{Shifts in Reasoning Patterns}
\label{sec:pattern-shifts}

We conducted analysis and revealed distinct shifts in reasoning patterns throughout the training process. We categorized the process into the following six stages (also illustrated in \Cref{fig:pattern-shift}. \Cref{table:example in training dynamics} also shows the example reasoning traces for each stage.

% ![reasoning-pattern-shift.png](attachment:35bc9ef1-b4f8-4131-a1e6-229a947a65a4:reasoning-pattern-shift.png)

\paragraph{Stage 1 (Format Failure):} The model fails to follow the format requirements and produces brief, unstructured responses. In line with recent work such as~\citet{yeo2025demystifying}, we observe that reasoning capabilities already exist in Qwen base models, but the models simply don't adhere to the specified format instructions. 

\paragraph{Stage 2 (Verbose Formatter):} The model begins following the format but becomes overly wordy.

\paragraph{Stage 3 (Concise Structurer):} The model fully follows the format and begins reasoning more concisely.

\paragraph{Stage 4 (Direct Answer Hacker):} The model learns to hack the reward by directly giving away the answer within the thinking step while drastically shortening the thinking length (for example, the model might start the thinking by stating ``The correct answer is C" and then explain the reasoning). This intriguing behavior hasn’t been observed in tasks such as math or coding. One possible explanation is that it is unique to MCQA, where the inclusion of answer candidates in the prompt drastically narrows the answer space. During RLVR, the model finds a shortcut to obtain high rewards by leveraging samples that reveal the correct answer early in the thinking step.    

We attempted to penalize this behavior, but the model exploited the flexibility of natural language to find alternative ways of revealing the answer early in its reasoning process. Similar patterns have also been observed in larger models such as Qwen2.5-7B. We suspect that employing more capable and larger base models or beginning with SFT models might help mitigate this issue. We leave this investigation for future work.

\paragraph{Stage 5 (Step-by-Step Exploit):} The model learns to hack the reward through a different strategy: adding step-by-step reasoning before \verb|<think>|, which leads to longer response length.

\paragraph{Stage 6 (Reintegrated Reasoning):} The model returns to placing reasoning within the thinking step and uses more concise explanations, though it still exhibits reward hacking behavior for some examples. 


\begin{table}
\centering
\scriptsize
\begin{tabular}{p{3cm}p{5cm}p{5cm}}
\toprule
{\bf Stages} & {\bf Example Question} & {\bf Example Output} \\
\midrule
{\bf Stage 1: \newline Fails to follow format} &  27-year-old man presents to the emergency department...The patient is a known IV drug abuser but
otherwise has an unknown past medical history....Which of the following is the best next step in management? A: Insulin, IV fluids, and potassium;
B: Insulin, potassium, IV fluids, and glucose; C: IV fluids only; D: Oral rehydration; E: Supportive
therapy and close monitoring & The patient’s initial presentation suggests a hyperglycemic state, likely due to his IV drug abuse and
the use of methadone, which can cause hyperglycemia. The subsequent improvement in laboratory
values after treatment indicates that the patient’s hyperglycemia has been resolved. Therefore, the
best next step in management would be to continue supportive therapy and close monitoring, as the
patient’s hyperglycemia has been addressed. The other options involve interventions that are not
necessary in this case.\\
\midrule
{\bf Stage 2: \newline Verbose Formatter} &  Two weeks after undergoing an emergency cardiac catherization with stenting for unstable angina pectoris, a 61-year-old man has decreased urinary output and malaise. ..Which of the following is the most likely cause of this patient's symptoms? A: Renal papillary necrosis; B: Allergic interstitial nephritis; C: Cholesterol embolization; D: Eosinophilic granulomatosis with polyangiitis; E: Polyarteritis nodosa &  \verb|<think>|The patient's symptoms, including decreased urinary output, malaise, and mottled, reticulated purplish discoloration of the feet, along with laboratory findings such as elevated erythrocyte sedimentation rate (ESR), elevated creatinine, and abnormal renal biopsy results, suggest a possible vascular issue affecting the kidneys. The presence of intravascular spindle-shaped vacuoles in the renal biopsy further supports this diagnosis. \newline
Let's evaluate each option: A: Renal papillary necrosis - This condition typically presents with hematuria, flank pain, and sometimes hypertension. It does not usually present with mottled, reticulated purplish discoloration of the feet or elevated ESR and creatinine levels\newline....\newline \verb|</think>|\newline\verb|<answer>| C: Cholesterol embolization \verb|</answer>|\\
\midrule
{\bf Stage 3 \newline Concise Structurer} & A 65-year-old man is brought to the emergency department 30 minutes after the onset of acute chest pain. He has hypertension and asthma...In addition to aspirin, which of the following drugs should be added to this patient's medication regimen? A: Nifedipine; B: Enoxaparin
    C: Clopidogrel;
    D: Spironolactone;
    E: Propranolol & \verb|<think>| The patient has acute coronary syndrome and a left ventricular ejection fraction of 58\%. He should be treated with aspirin, clopidogrel, and beta-blockers. Nifedipine is a calcium channel blocker that can be used in patients with stable angina, but it is not the best choice for this patient. Enoxaparin is a low-molecular-weight heparin used for anticoagulation, but it is not the best choice for this patient. Spironolactone is a potassium-sparing diuretic, but it is not the best choice for this patient. Propranolol is a beta-blocker that can be used in patients with stable angina, but it is not the best choice for this patient. \verb|</think><answer>| C \verb|</answer>|\\
    % \midrule
    % &&Continued to the next page\\
    \midrule
    {\bf Stage 4:\newline Direct Answer Hacker} & A 35-year-old man comes to the physician because of itchy, watery eyes for the past week... Which of the following is the most appropriate treatment?
    A: Erythromycin ointment;
    B: Ketotifen eye drops;
    C: Warm compresses;
    D: Fluorometholone eye ;drops
    E: Latanoprost eye drops & \verb|<think>| The most appropriate treatment for the patient's symptoms is Ketotifen eye drops. \verb|</think><answer>| B \verb|</answer>| \\
    \midrule
    {\bf Stage 5:\newline Step-by-Step Exploit }& 
    A 42-year-old woman comes to the emergency department because of a 2-day history of right upper abdominal pain and nausea. ...Which of the following is the most likely cause of this patient's symptoms?
    A: Autodigestion of pancreatic parenchyma;
    B: Hypomotility of the gallbadder;
    C: Fistula between the gallbladder and small intestine;
    D: Infection with a hepatotropic virus;
    E: Obstruction of the cystic duct & To solve this problem, let's go through each option step-by-step and evaluate its likelihood based on the given information...
    Based on the reasoning process, the most likely cause of the patient's symptoms is \verb|<think>| Obstruction of the cystic duct \verb|</think><answer>| E \verb|</answer>|\\
    \midrule
    {\bf Stage 6:\newline Reintegrated Reasoning} &  A 65-year-old man is brought to the emergency department 30 minutes after the onset of acute chest pain...In addition to aspirin, which of the following drugs should be added to this patient's medication regimen?
    A: Nifedipine;
    B: Enoxaparin;
    C: Clopidogrel;
    D: Spironolactone;
    E: Propranolol & \verb|<think>| The patient has acute coronary syndrome and a left ventricular ejection fraction of 58\%, which suggests the need for antiplatelet therapy. Clopidogrel is a commonly used antiplatelet drug for patients with coronary artery disease. \verb|</think><answer>| C \verb|</answer>|\\
    \bottomrule

\end{tabular}
\caption{Example output showing the emerging reasoning traces during the training of \ours \label{table:example in training dynamics}}
\end{table}
    


    

\subsubsection{No “aha-moment” Yet}

In contrast to recent findings in math and coding, we did not observe any self-validation behaviors (i.e., "aha-moments”) in the generated responses for MCQA. Additionally, the overall length of the reasoning process was markedly shorter than what is typically seen in math and coding tasks. This difference might be attributed to the inherently lower reasoning demands of the MCQA task compared with math and coding. To address this, potential solutions include (1) penalizing responses with overly short chains-of-thought (CoT), and (2) fine-tuning the base models with long CoTs prior to RL.