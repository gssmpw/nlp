\pdfoutput=1

\documentclass[11pt]{article}

\usepackage[final]{acl} % review oder final
\usepackage{times}
\usepackage{latexsym}
\usepackage[T1]{fontenc}
\usepackage[utf8]{inputenc}
\usepackage{microtype}
\usepackage{inconsolata}
\usepackage{graphicx}
\usepackage{enumitem}
\usepackage{xcolor}
\usepackage{soul}
\usepackage{multirow}
\usepackage{array}
\usepackage{lscape}
\usepackage{svg}
\usepackage{tcolorbox}
\usepackage{amsmath}
\usepackage{subcaption}
\usepackage{lineno}
\usepackage{longtable}
\usepackage{caption}
\usepackage{subcaption}
\usepackage{xurl}

\usepackage{booktabs}        % Für schöne Tabellenlinien (\toprule, \midrule, \bottomrule)
\usepackage{tabularx}        % Für flexiblere Tabellenbreiten (optional)
\usepackage{adjustbox}       % Für Feinjustierungen der Tabellengröße (optional)

\usepackage{siunitx}

% siunitx-Einstellungen
\sisetup{
  detect-all,                 % Erkennung von Schriftart und Stil
  table-format=2.3,           % Standardformat für Zahlen
  table-space-text-post=*     % Platz für Sonderzeichen nach Zahlen
}

\newcommand{\PreserveBackslash}[1]{\let\temp=\\#1\let\\=\temp}
\newcolumntype{C}[1]{>{\PreserveBackslash\centering}p{#1}}
\newcolumntype{R}[1]{>{\PreserveBackslash\raggedleft}p{#1}}
\newcolumntype{L}[1]{>{\PreserveBackslash\raggedright}p{#1}}



\usepackage{caption}
\usepackage{colortbl}
\usepackage{dblfloatfix}


\definecolor{hidecolor}{rgb}{1,1,1,0.0} 
\sethlcolor{hidecolor}

% If the title and author information does not fit in the area allocated, uncomment the following
%
\setlength\titlebox{8.5cm}
%
% and set <dim> to something 5cm or larger.

\title{Do we still need Human Annotators? Prompting Large Language Models for Aspect Sentiment Quad Prediction}

% Author information can be set in various styles:
% For several authors from the same institution:
% \author{Author 1 \and ... \and Author n \\
%         Address line \\ ... \\ Address line}
% if the names do not fit well on one line use
%         Author 1 \\ {\bf Author 2} \\ ... \\ {\bf Author n} \\
% For authors from different institutions:
% \author{Author 1 \\ Address line \\  ... \\ Address line
%         \And  ... \And
%         Author n \\ Address line \\ ... \\ Address line}
% To start a separate ``row'' of authors use \AND, as in
% \author{Author 1 \\ Address line \\  ... \\ Address line
%         \AND
%         Author 2 \\ Address line \\ ... \\ Address line \And
%         Author 3 \\ Address line \\ ... \\ Address line}

\author{Nils Constantin Hellwig \\
  Media Informatics Group \\
  University of Regensburg \\
  Regensburg, Germany \\
  {\url{nils-constantin.hellwig@ur.de}} \And
Jakob Fehle \\
  Media Informatics Group \\
  University of Regensburg \\
  Regensburg, Germany \\
  {\url{jakob.fehle@ur.de}} \\ \AND
Udo Kruschwitz \\
  Information Science Group \\
  University of Regensburg \\
  Regensburg, Germany \\
  {\url{udo.kruschwitz@ur.de}} \And
Christian Wolff \\
  Media Informatics Group \\
  University of Regensburg \\
  Regensburg, Germany \\
  {\url{christian.wolff@ur.de}}}


\begin{document}
\maketitle
End-to-end imitation learning offers a promising approach for training robot policies. However, generalizing to new settings—such as unseen scenes, tasks, and object instances—remains a significant challenge. Although large-scale robot demonstration datasets have shown potential for inducing generalization, they are resource-intensive to scale. In contrast, human video data is abundant and diverse, presenting an attractive alternative. Yet, these human-video datasets lack action labels, complicating their use in imitation learning. Existing methods attempt to extract grounded action representations (e.g., hand poses), but resulting policies struggle to bridge the embodiment gap between human and robot actions.
% our approach
We propose an alternative approach: leveraging language-based reasoning from human videos - essential for guiding robot actions - to train generalizable robot policies. Building on recent advances in reasoning-based policy architectures, we introduce Reasoning through Action-free Data (RAD). RAD learns from both robot demonstration data (with reasoning and action labels) and action-free human video data (with only reasoning labels). The robot data teaches the model to map reasoning to low-level actions, while the action-free data enhances reasoning capabilities. Additionally, we will release a new dataset of 3,377 human-hand demonstrations compatible with the Bridge V2 benchmark. This dataset includes chain-of-thought reasoning annotations and hand-tracking data to help facilitate future work on reasoning-driven robot learning.
% experiments
Our experiments demonstrate that RAD enables effective transfer across the embodiment gap, allowing robots to perform tasks seen only in action-free data. Furthermore, scaling up action-free reasoning data significantly improves policy performance and generalization to novel tasks. These results highlight the promise of reasoning-driven learning from action-free datasets for advancing generalizable robot control. 
% releasing dataset
Website: \href{https://rad-generalization.github.io}{here}.

\begin{figure}[ht]
    \centering
    \includegraphics[width=0.8\linewidth]{graphs/greater_than_naive.pdf}
    \vspace{0.5cm}
    \includegraphics[width=0.8\linewidth]{graphs/p1_bottom.png}
    \vspace{-5pt}
    \caption{\textcolor{positional}{Positional} vs.\ \textcolor{nonpositional}{non-positional} circuits. In a \textcolor{nonpositional}{non-positional} circuit, the same edges must be included at all positions. A \textcolor{positional}{positional} circuit can distinguish between the same edge at different positions. This specificity yields better trade-offs between circuit size and faithfulness. It can also increase both precision and recall.}
    \label{fig:p1}
    \vspace{-5pt}
\end{figure}

\section{Introduction}

\looseness=-1
A primary goal of interpretability research is to characterize the internal mechanisms in language models (LMs) and other NLP models. 
A core approach in this area is \textbf{circuit discovery}---identifying the minimal subgraph within the model's computation graph that performs a specific task \citep{olah2021framework,olah-mech}.
Typically, the nodes of a circuit represent model components (e.g., attention heads, neurons, or layers).
While manual circuit discovery methods can yield position-specific insights \citep{wanginterpretability,goldowskydill2023localizingmodelbehaviorpath}, \emph{automatic methods often overlook positional information}, treating components as uniformly relevant across all input token positions \citep{conmytowards,syed2023attribution}. 
For instance, if an attention head is included in a circuit, it is assumed to contribute equally to the computation for every position in the input sequence.
The assumption that circuits are position-invariant ignores the fact that different positions often require distinct computations.
By ignoring positions, current methods limit their ability to capture mechanisms that operate across positions, such as interactions between attention heads across positions.

In this study, we start by demonstrating that positional agnosticism is a significant limitation (\S\ref{sec:motivating}). Then, to address these limitations, we introduce a new approach: position-aware edge attribution patching (PEAP; \S\ref{sec:full_circ_discovery}; Figure~\ref{fig:p1}). Current approaches  assume that if an edge is in a circuit, then the same edge will be in the circuit at all positions, thus leading to low precision. It is also assumed that an edge's importance should be aggregated across positions before deciding whether it should be included in the circuit; this can lead to cancellation effects, and thus low recall. PEAP instead allows us to compute the importance of cross-positional edges, and separately evaluates edge importance at each position. We show that this leads to smaller and more accurate circuits; see Figure~\ref{fig:p1}.

Incorporating positional information into circuit discovery is straightforward when inputs have the same length and structure across examples.

However, realistic datasets are not nearly this templatic.
How, then, can we incorporate positional information into automatic circuit discovery?
To address this challenge, we propose \textbf{schemas} (\S\ref{sec:schema}). 
Schemas assign semantic labels to spans of tokens, enabling information aggregation across examples even when the spans differ in length.

For example, in the input ``The \textcolor{positional}{war} lasted from 1453 to 14\underline{\hspace{1em}},'' the span ``\textcolor{positional}{war}'' could be labeled as ``\emph{Subject}''.
This enables handling spans with varying lengths: the phrase ``\textcolor{positional}{Black Plague}'' in another example can be treated as a single positional span with the same role as ``\textcolor{positional}{war}''.
In experiments with two LMs and three tasks, we find that circuits discovered using schemas achieve a better trade-off between circuit size and faithfulness to the model's behavior than position-agnostic circuits.
Importantly, position-aware circuits offer a more precise representation of the underlying mechanisms, providing a more concise foundation for mechanistic explanations.

We also present a fully automated pipeline for schema generation and application (\S\ref{sec:schema-generation}) using large language models (LLMs). 
We evaluate the quality of the generated schemas and their utility in discovering position-aware circuits (\S\ref{sec:schema-eval}).
Notably, circuits derived using automatically generated and applied schemas achieve comparable faithfulness scores to circuits discovered with human-designed and manually applied schemas.

We summarize our contributions as follows:
\begin{itemize}[noitemsep,leftmargin=*,topsep=1pt,parsep=1pt]
    \item Introduce a position-aware circuit discovery method, which obtains better faithfulness than position-agnostic discovery.  
    \item Introduce dataset schemas,  facilitating positional circuit discovery in more naturalistic settings. 
    \item Develop an automated schema generation and application pipeline with LLMs, yielding schemas that are comparable to manually-annotated ones.
\end{itemize}


\section{Related work}


Recent advances in single-image animatable head avatar generation can be categorized into mainly 2D-based and 3D-based approaches. 

\paragraph{\bf Image to 2D Animatable Avatar.}
2D-based methods, leveraging the power of convolutional neural networks (CNNs)~\cite{DBLP:conf/cvpr/KarrasLAHLA20,DBLP:conf/cvpr/IsolaZZE17,DBLP:conf/nips/GoodfellowPMXWOCB14}, often employ generative adversarial networks (GANs)~\cite{DBLP:conf/cvpr/StyleGAN} for direct image synthesis. Early approaches~\cite{DBLP:conf/cvpr/WangDYSW23,DBLP:conf/cvpr/BurkovPGL20,DBLP:conf/iccv/ZakharovSBL19} focus on injecting expression and pose features into the generator network, often utilizing architectures like U-Net or StyleGAN~\cite{DBLP:conf/cvpr/StyleGAN}.
Some other 2D methods~\cite{DBLP:journals/corr/abs-2407-03168,DBLP:conf/cvpr/ZhangQZZW0CW023,DBLP:conf/cvpr/HongZS022,DBLP:conf/mm/DrobyshevCKILZ22,DBLP:conf/cvpr/BurkovPGL20,DBLP:conf/nips/SiarohinLT0S19} represent expressions and poses as warping fields applied to the source image. 
Benefiting from advances in image and video diffusion networks, more recent 2D-based works~\cite{DBLP:journals/corr/abs-2410-07718,DBLP:journals/corr/abs-2406-08801,DBLP:conf/eccv/TianWZB24} get improved results with diffusion techniques. 
However, these methods still face challenges related to long generation times and significant computational resource demands. Audio-driven 2D control methods~\cite{DBLP:conf/cvpr/ZhangCWZSGSW23,DBLP:journals/corr/abs-2211-12368,DBLP:conf/iccv/GuoCLLBZ21} are easy to use but cannot explicitly control facial expressions and poses. 2D-based techniques often struggle with large pose or expression variations due to the lack of an explicit 3D structure, sometimes producing unrealistic distortions or identity changes. While some 2D methods~\cite{SadTalker,StyleHEAT,Pirenderer,DBLP:conf/cvpr/WangM021,MegaPortraits} incorporate 3D Morphable Models (3DMMs)~\cite{DBLP:conf/fgr/GerigMBELSV18,DBLP:journals/tog/LiBBL017,DBLP:conf/avss/PaysanKARV09,DBLP:conf/siggraph/BlanzV99} to mitigate these issues, they typically cannot achieve free-viewpoint rendering. 

\vspace{-0.1in}

\begin{figure*}[h]
    \centering
    \includegraphics[width=0.9\linewidth]{images/framework.pdf}
    \caption{\textbf{Overall Framework.} Our framework utilizes learnable query features attached to FLAME vertices to perform cross-attention with the extracted multi-level image features. The extracted features are then decoded to reconstruct the Gaussian avatar in the canonical space, which can be animated utilizing standard linear blend skinning (LBS) and corrective blendshapes as the FLAME model did and rendered in real-time on various platforms.}
    \label{fig:framework}
\end{figure*}

\paragraph{\bf Image to 3D Animatable Avatar.}
3D-aware methods offer improved geometric consistency and free-viewpoint rendering capabilities. Early 3D approaches~\cite{DBLP:conf/eccv/KhakhulinSLZ22,DBLP:conf/cvpr/XuYCWDJT20} utilize 3DMMs for head avatar reconstruction. With the advent of Neural Radiance Fields (NeRFs)~\cite{DBLP:conf/eccv/MildenhallSTBRN20}, many recent methods~\cite{DBLP:conf/siggraph/YuFZWYBCSWSW23,DBLP:conf/cvpr/MaZQLZ23,DBLP:conf/cvpr/LiZWZ0CZWB023,GPAvatar,ye2024real3d,deng2024portrait4d,deng2024portrait4d2,DBLP:conf/eccv/KiMC24,DBLP:conf/cvpr/BaiFWZSYS23,PointAvatar,Nerfies,INSTA} have adopted this representation for higher fidelity, particularly in modeling fine details like hair. However, NeRF-based~\cite{DBLP:conf/cvpr/ZhangZLHLWGCL024,HAvatar,DBLP:conf/cvpr/BaiTHSTQMDDOPTB23,AD-NeRF,DBLP:journals/tog/GaoZXHGZ22,DBLP:journals/tog/ParkSHBBGMS21,DBLP:conf/cvpr/AtharXSSS22,DBLP:journals/corr/abs-2112-05637,DBLP:conf/iccv/TretschkTGZLT21,DBLP:conf/cvpr/GafniTZN21,DBLP:conf/eccv/KiMC24,DBLP:conf/cvpr/BaiFWZSYS23,PointAvatar,Nerfies,DBLP:conf/siggraph/YuFZWYBCSWSW23,DBLP:conf/cvpr/MaZQLZ23,DBLP:conf/cvpr/LiZWZ0CZWB023} approaches often require extensive training data, including multi-view or single-view videos, raising privacy concerns and limiting generalization to unseen identities. Some methods~\cite{DBLP:conf/cvpr/SunWWLZZL23,DBLP:conf/3dim/ZhuangMKS22,DBLP:journals/pami/SunWZHWL24,DBLP:journals/tvcg/TangZYZCMW24,DBLP:conf/iclr/XuZLZBFS23} bypass this data requirement by training generators with random noise and then inverting them for identity-specific reconstruction, but inversion accuracy remains a challenge. Test-time optimization offers another alternative, but its computational cost limits practical applications. Several recent works~\cite{goha2023,hidenerf2023,gpavatar2024,ye2024real3d,ma2024cvthead,deng2024portrait4d,deng2024portrait4d2,GGHead} have explored one-shot 3D head reconstruction to address the limitations of data requirements and computational cost. These methods employ various techniques, such as tri-plane features, deformation fields, point-based expression fields, and vertex-feature transformers. Despite these advancements, NeRF-based methods often struggle with real-time rendering. 
Recently, 3D Gaussian Splatting~\cite{GaussianSplatting} has emerged as a promising alternative, offering both high-quality results and fast rendering speeds. However, existing Gaussian Splatting methods~\cite{GaussianAvatar,DBLP:conf/cvpr/XuCL00ZL24} typically rely on video data for training for each person, limiting their ability to generalize to new identities. Instead, the most recent work, GAGAvatar~\cite{GAGAvatar}, proposes a one-shot 3D Gaussian-based head avatar generation method. However, it still relies heavily on complex 2D neural post-processing to achieve optimal animation outcomes, thus it is not a pure 3D solution and the extra neural network hinders its application on various platforms. In contrast, our work generates Gaussian heads that are immediately animatable and renderable without additional networks or post-processing steps, enabling seamless integration into existing rendering pipelines for real-time animation and rendering across a wide range of platforms, including mobile phones. 
\section{Methodology}
 

\begin{figure*}[t]
    \centering
    \scalebox{0.9}{
    % Define a style for the special labels (for 1, 2, 3)
\begin{tikzpicture}[
    >=Stealth,     % arrow tips
    thick,         % line thickness
    node distance=2.5cm
]

%--- User (person) on the left ---
\node[
  person,
  minimum size=1.25cm,
  label=below:User
] (user) {};

%--- Rectangle in the middle: "1" inside, "operations" below ---
\node[
  draw,
  rectangle,
  minimum width=2.0cm,
  minimum height=1.2cm,
  label=below:operations,
  right=of user
] (box) {\textcircled{1}};

%--- Cylinder (DB) on the right ---
\node[
  draw,
  cylinder,
  shape border rotate=90,
  aspect=1.5,
  minimum height=1.5cm,
  minimum width=1cm,
  label=below:DB,
  right=of box
] (db) {};

%--- Bidirectional arrows with special labels ---
% Arrows between user and box labeled "3"
\draw[->] (user) to[bend left=10] node[above]{\textcircled{3}} (box);
\draw[->] (box)  to[bend left=10]  (user);

% Arrows between box and DB labeled "2"
\draw[->] (box)  to[bend left=10] node[above]{\textcircled{2}} (db);
\draw[->] (db)   to[bend left=10]  (box);

\end{tikzpicture}}
    \caption{An overview of {\sc Demotic} is shown. {\sc Demotic} takes a Verilog instance describing a combinational circuit and parse it into its corresponding probabilistic model described in PyTorch. The embedding layer converts the learnable real-value inputs into probabilities. The $\ell_2$-loss function is calculated in each training iteration and the input variables are updated using GD.}
    \label{fig1}
\end{figure*}


In this section, we describe our differentiable solver/sampler for multi-level digital circuits. While the common approach in solving CircuitSAT typically involves converting the underlying circuit into CNF and employing a SAT solver to find the satisfying solution, we take a completely different approach. Instead, we re-frame the CircuitSAT problem as a multi-output regression task, transforming it into a learning problem. Digital circuits are inherently discrete and non-differentiable. Therefore, we first need to relax the CircuitSAT problem into a continuous form while accurately capturing the structure and behavior of the circuit. To accomplish this, we leverage the probability model of digital gates, as shown in Table \ref{tab1}. This probability model is commonly used in different domains such as stochastic computing \cite{Ardakani2017SC} and dynamic power estimation of digital circuits \cite{harris2010cmos}. We use these probabilities to model each gate in the circuit. The result of such modeling is a differentiable formulation of the underlying circuit that accurately describes its functionality while preserving its spatial structure. Of course, the outcome of this model remains identical to the original circuit in its discrete form for any binary input valuations.




\begin{table}[t]
    \centering
    \caption{Probability model of logic gates.}
    \vspace{-0.25cm}
    \begin{table*}[!th]
\centering
\resizebox{\textwidth}{!}{%
\begin{tabular}{@{}llcccccccccc@{}}
\toprule
& & \multicolumn{2}{c}{\textbf{Intent Detection}} & \multicolumn{2}{c}{\textbf{Topic Mining}} & \multicolumn{2}{c}{\textbf{Domain Discovery}} & \multicolumn{1}{c}{\textbf{Type}} & \multicolumn{1}{c}{\textbf{Emotion}} & \\
\cmidrule(lr){3-4} \cmidrule(lr){5-6} \cmidrule(lr){7-8} \cmidrule(lr){9-9} \cmidrule(lr){10-10}  %\cmidrule(lr){11-11}
\textbf{Model} & \textbf{Method} & \textbf{BANKING} & \textbf{CLINC} & \textbf{Reddit} & \textbf{StackEx} & \textbf{MTOP} & \textbf{CLINC(D)} & \textbf{FewEvent} & \textbf{GoEmotion} & \textbf{AVG} \\ \midrule \midrule
GPT-4o-mini & Standard Prompting & 0.652 & 0.792 & 0.534 & 0.482 & 0.896 & 0.536 & 0.630 & 0.378 & 0.613 \\
& Self-Consistency & 0.666 & 0.802 & 0.586 & 0.494 & 0.902 & 0.530 & 0.640 & 0.382 & 0.625 \\
& TestNUC & 0.712 & 0.858 & 0.614 & 0.528 & 0.936 & 0.544 & 0.674 & 0.410 & 0.660 \\
& \cellcolor{gray!18}TestNUC\textdagger & \cellcolor{gray!18}\textbf{0.764} & \cellcolor{gray!18}\textbf{0.864} & \cellcolor{gray!18}\textbf{0.646} & \cellcolor{gray!18}\textbf{0.540} & \cellcolor{gray!18}\textbf{0.948} & \cellcolor{gray!18}\textbf{0.554} & \cellcolor{gray!18}\textbf{0.680} & \cellcolor{gray!18}\textbf{0.414} & \cellcolor{gray!18}\textbf{0.676} \\ \midrule \midrule
Llama-3.1-8B & Standard Prompting & 0.572 & 0.726 & 0.502 & 0.492 & 0.892 & 0.528 & 0.530 & 0.332 & 0.572 \\
& Self-Consistency & 0.620 & 0.774 & 0.564 & 0.526 & 0.902 & 0.518 & 0.564 & 0.340 & 0.601 \\
& TestNUC & 0.694 & 0.806 & 0.618 & 0.558 & 0.934 & 0.528 & 0.596 & 0.356 & 0.636 \\
& \cellcolor{gray!18}TestNUC\textdagger & \cellcolor{gray!18}\textbf{0.724} & \cellcolor{gray!18}\textbf{0.812} & \cellcolor{gray!18}\textbf{0.646} & \cellcolor{gray!18}\textbf{0.576} & \cellcolor{gray!18}\textbf{0.940} & \cellcolor{gray!18}\textbf{0.542} & \cellcolor{gray!18}\textbf{0.614} & \cellcolor{gray!18}\textbf{0.360} & \cellcolor{gray!18}\textbf{0.652} \\ \midrule \midrule
Claude-3-Haiku & Standard Prompting & 0.680 & 0.848 & 0.486 & 0.564 & 0.892 & 0.552 & 0.594 & 0.336 & 0.619 \\
& Self-Consistency & 0.702 & 0.870 & 0.510 & 0.578 & 0.904 & 0.564 & 0.568 & 0.350 & 0.631 \\
& TestNUC & 0.762 & 0.894 & 0.596 & 0.588 & 0.940 & 0.590 & 0.620 & 0.348 & 0.667 \\
& \cellcolor{gray!18}TestNUC\textdagger & \cellcolor{gray!18}\textbf{0.804} & \cellcolor{gray!18}\textbf{0.902} & \cellcolor{gray!18}\textbf{0.612} & \cellcolor{gray!18}\textbf{0.600} & \cellcolor{gray!18}\textbf{0.946} & \cellcolor{gray!18}\textbf{0.622} & \cellcolor{gray!18}\textbf{0.660} & \cellcolor{gray!18}\textbf{0.368} & \cellcolor{gray!18}\textbf{0.689} \\ \midrule \midrule
GPT-4o & Standard Prompting & 0.746 & 0.924 & 0.712 & 0.674 & 0.962 & 0.614 & 0.682 & 0.406 & 0.715 \\
& Self-Consistency & 0.758 & 0.922 & 0.720 & 0.688 & 0.958 & 0.624 & 0.696 & 0.426 & 0.724 \\
&TestNUC & 0.804 & 0.934 & 0.744 & \textbf{0.710} & 0.974 & 0.644 & 0.692 & 0.446 & 0.744 \\
& \cellcolor{gray!18}TestNUC\textdagger & \cellcolor{gray!18}\textbf{0.824} & \cellcolor{gray!18}\textbf{0.940} & \cellcolor{gray!18}\textbf{0.750} & \cellcolor{gray!18}\textbf{0.710} & \cellcolor{gray!18}\textbf{0.978} & \cellcolor{gray!18}\textbf{0.654} & \cellcolor{gray!18}\textbf{0.708} & \cellcolor{gray!18}\textbf{0.464} & \cellcolor{gray!18}\textbf{0.754} \\
\bottomrule
\end{tabular}%
}
\caption{Accuracy comparison with Standard Prompting and Self-Consistency across four diverse LLMs. TestNUC consistently improves the inference performance on all benchmark datasets. $\dagger$ denotes that 50 neighbors are utilized.}
\label{tab:main_compare_sc}
\end{table*}
    \label{tab1}
    \vspace{-0.25cm}
\end{table}






Given the differentiable model of the circuit obtained by replacing its discrete logic gates with their corresponding probability model, our objective now is to generate a set of inputs that satisfy a desired constraint. This constraint could pertain to any desired valuation of intermediate signals or outputs. To generate satisfying solutions to the CircuitSAT problem, we represent the input variables to the circuit as $\textbf{V} \in \mathbb{R}^{b\times n}$, where $n$ represents the number of variables and $b$ denotes the batch size. We define the matrix $\textbf{V}$ as the parameters of an embedding layer in our circuit model, which will be updated during the learning process. It is worth mentioning that the number of variables in our sampling method is significantly fewer than that of SAT samplers, remaining the same as the number of inputs in the circuit. This discrepancy arises because SAT samplers deal with the CNF of the circuit, where each gate or component introduces additional variables. The embedding layer converts the real-value input variables of the circuit into probabilities in the range from $0$ to $1$ using the sigmoid function $\sigma(\cdot)$, expressed as:
\begin{equation}
    \textbf{P} = \sigma(\textbf{V}) = \dfrac{1}{1 + e^{-\textbf{V}}},
\end{equation}
where $\textbf{P} \in [0, 1]^{b\times n}$ represents the input probabilities to the underlying circuit. The circuit functionality is then computed as:
\begin{equation}
    \textbf{Y} = \mathcal{F}(\textbf{P}),
\end{equation}
where $\mathcal{F}:[0, 1]^{b \times n} \rightarrow [0, 1]^{b \times m}$ denotes the probabilistic model of the circuit. The matrix $\textbf{Y} \in [0, 1]^{b \times m}$ denotes the $m$ outputs across $b$ data batches. The $\ell_2$-loss function $\mathcal{L}$ can be constructed by measuring the distance between $\textbf{Y}$ and the target output valuation matrix $\textbf{T} \in \{0, 1\}^{b \times m}$ as follows:
\begin{equation}
    \mathcal{L} = \sum_{b,m} \left|\left| \textbf{Y} - \textbf{T} \right|\right|^2_2.
\end{equation}
The above loss function can be minimized, and the input variables (i.e., $\textbf{V}$) can be updated using GD in an iterative manner. Upon convergence, the $b$ solutions to the CircuitSAT problem are obtained by converting the soft input values (i.e., $\textbf{V}$) into hard values (i.e., $\widetilde{\textbf{V}} \in \{0, 1\}^{b\times n}$).

Fig. \ref{fig1} illustrates the overview of {\sc Demotic}. {\sc Demotic} is equipped with a parser to covert the circuit described in either bit-blasted Verilog or Berkeley Logic Interchange Format (BLIF) into its corresponding probabilistic model. Consequently, {\sc Demotic} can describe combinational circuits and generate satisfying solutions for any arbitrary constraint on the circuit. Such a sampling paradigm can also benefit from GPU acceleration due to the parallel independent computations across the data batches, enabling a high-throughput sampling procedure. 

To better understand our methodology, let us consider a quantitative example using the module ``c$15$'' shown in Fig. \ref{fig1}. We set the output node $G19$ to $1$ as an output constraint, while the output node $G22$ can take any value of either $0$ or $1$. Therefore, the goal in this example is to find a pair of inputs such that the output node $G19$ is equal to $1$. In this example, the input nodes contributing to our output constraint are $G3$, $G6$, and $G7$. These inputs are learned iteratively using gradient descent. The remaining input nodes, $G1$, $G2$, and $G3$, will not be updated and can take any arbitrary binary values. During each training iteration, each input node is updated by computing the derivative of the loss function with respect to each input node.

To illustrate the process, we generate two samples. In the first step, we randomly assign two values to each input node as follows:
\begin{equation}
    \textbf{v}_{G3} = \begin{bmatrix}
           0.1 \\
           -0.2 
         \end{bmatrix}, \textbf{v}_{G6} = \begin{bmatrix}
           0.5 \\
           -0.4 
         \end{bmatrix}, \textbf{v}_{G7} = \begin{bmatrix}
           -0.7 \\
           -0.8 
         \end{bmatrix},
\end{equation}
where the concatenation of the above vectors forms the matrix $\textbf{V}$. Next, the input probabilities to the circuit are calculated using the sigmoid function:
\begin{equation}
    \textbf{p}_{G3} = \begin{bmatrix}
           0.5250 \\
           0.4502
         \end{bmatrix}, \textbf{p}_{G6} = \begin{bmatrix}
           0.6225 \\
           0.4013
         \end{bmatrix}, \textbf{p}_{G7} = \begin{bmatrix}
           0.3318 \\
           0.3100
         \end{bmatrix}.
\end{equation}
Using the probability model of each gate shown in Table \ref{tab1}, the probabilities of the intermediate node $G11$ and the output node $G19$ are calculated as follows:
\begin{equation}
    \textbf{p}_{G11} = \begin{bmatrix}
           0.4939 \\
           0.4902
         \end{bmatrix}, \textbf{p}_{G19} = \begin{bmatrix}
           0.1639 \\
           0.1520
         \end{bmatrix}.
\end{equation}
Given the target value of 1 for the output node $G19$, the loss is calculated as:
\begin{equation}
    \mathcal{L} = (\textbf{p}_{G19} - 1)^2 = \begin{bmatrix}
           (0.1639 - 1)^2  \\
           (0.1520 - 1)^2 
         \end{bmatrix} = \begin{bmatrix}
           0.6991  \\
           0.7192 
         \end{bmatrix}.
\end{equation}


The above computations are commonly referred to as forward computations. To update the value of the input variables, we need to calculate the derivative of the loss with respect to each input variable, which is referred to as backward computations. This involves using the derivatives of each gate (as shown in Table \ref{tab1}) and applying the chain rule. The process is derived as follows:
\begin{align}
    \dfrac{\partial \mathcal{L}}{\partial \textbf{v}_{G3}} &= \dfrac{\partial \mathcal{L}}{\partial \textbf{p}_{G19}} \dfrac{\partial \textbf{p}_{G19}}{\partial \textbf{p}_{G11}} \dfrac{\partial \textbf{p}_{G11}} {\partial \textbf{p}_{G3}}
    \dfrac{\partial \textbf{p}_{G3}} {\partial \textbf{v}_{G3}} = 2\textbf{p}_{G19} \cdot \textbf{p}_{G7} \cdot (1 - 2\textbf{p}_{G6}) \nonumber \\ 
    &\cdot \sigma(\textbf{v}_{G3})\cdot (1 - \sigma(\textbf{v}_{G3})) = \begin{bmatrix}
           0.0339  \\
           -0.0257 
         \end{bmatrix}, \nonumber 
\end{align}
\begin{align}
    \dfrac{\partial \mathcal{L}}{\partial \textbf{v}_{G6}} &= \dfrac{\partial \mathcal{L}}{\partial \textbf{p}_{G19}} \dfrac{\partial \textbf{p}_{G19}}{\partial \textbf{p}_{G11}} \dfrac{\partial \textbf{p}_{G11}} {\partial \textbf{p}_{G6}}
    \dfrac{\partial \textbf{p}_{G6}} {\partial \textbf{v}_{G6}} = 2\textbf{p}_{G19} \cdot \textbf{p}_{G7} \cdot (1 - 2\textbf{p}_{G3})  \nonumber 
 \\ 
    & \cdot \sigma(\textbf{v}_{G6}) \cdot (1 - \sigma(\textbf{v}_{G6}))  = \begin{bmatrix}
           0.0065   \\
           -0.0126 
         \end{bmatrix}, \nonumber 
\end{align}
\begin{align}
    \dfrac{\partial \mathcal{L}}{\partial \textbf{v}_{G7}} &= \dfrac{\partial \mathcal{L}}{\partial \textbf{p}_{G19}} \dfrac{\partial \textbf{p}_{G19}}{\partial \textbf{p}_{G7}} \dfrac{\partial \textbf{p}_{G7}} {\partial \textbf{v}_{G7}} = 2\textbf{p}_{G19} \cdot \textbf{p}_{G11} \nonumber 
 \\ 
    & \cdot \sigma(\textbf{v}_{G7})\cdot (1 - \sigma(\textbf{v}_{G7})) = \begin{bmatrix}
           -0.1831   \\
           -0.1778 
         \end{bmatrix},
\end{align}
where ``$\cdot$'' denotes element-wise multiplication.


At this point, each variable is updated using the gradient descent update rule. This involves subtracting the derivative of the loss, scaled by the learning rate, from the corresponding input variables. Given a learning rate of $\gamma = 10$, the new values of the input variables at the end of this iteration are obtained as follows:
\begin{align}
    \textbf{v}_{G3} &= \textbf{v}_{G3} - \gamma \dfrac{\partial \mathcal{L}}{\partial \textbf{v}_{G3}} =  \begin{bmatrix}
           -0.2389 \\
           0.0569
         \end{bmatrix}, \textbf{v}_{G6} = \begin{bmatrix}
           0.4349 \\
           -0.2741
         \end{bmatrix}, \nonumber \\ \textbf{v}_{G7} & = \begin{bmatrix}
           1.1311 \\
           0.9783
         \end{bmatrix}.
\end{align}
This process can be repeated multiple times until convergence. However, even after one iteration in this specific example, we obtain two valid and distinct solutions by rounding the input variables to their nearest discrete values after applying the sigmoid function. In this example, the two input pairs of $(v_{G3} = -0.2389, v_{G6} = 0.4349, v_{G7} = 1.1311)$ and $(v_{G3} = 0.0569, v_{G6} = -0.2741, v_{G7} = 0.9783)$ are rounded to $(\widetilde{v}_{G3} = 0, \widetilde{v}_{G6} = 1, \widetilde{v}_{G7} = 1)$ and $(\widetilde{v}_{G3} = 1, \widetilde{v}_{G6} = 0, \widetilde{v}_{G7} = 1)$, respectively. As demonstrated through this example, the forward and backward computations of the two samples are independent of each other. This allows for the parallel execution of the learning process across multiple samples (i.e., batches), enabling GPU acceleration.


% \begin{figure*}[t]
%     \centering
%      \begin{subfigure}[b]{0.6\textwidth}
%          \centering
%          \scalebox{0.9}{\begin{tikzpicture}[auto, node distance=2cm,>=latex']
    % Combinational block 1
    \node [draw, fill = dateblue!30, shape=rectangle, minimum width=2.5cm, minimum height=1.5cm, text width=2cm, align = center, line width=1.5pt] (comb1) {Combinational Circuit ($\mathcal{F}_h$)};
    % Flip-Flop block
    \node [draw, fill = datemagenta!40, shape=rectangle, right = 1.25cm of comb1, align = center, minimum height=2cm, line width=1.5pt] (ff) {Flip-Flops};
    % Combinational block 2
    \node [draw, fill = dateblue!30,shape=rectangle, minimum width=2.5cm, minimum height=1.5cm, right =1.25cm of ff, text width=2cm, align = center, yshift=0.5cm, line width=1.5pt] (comb2) {Combinational Circuit ($\mathcal{F}_o$)};
    % Input nodes
    \node [left =1cm of comb1, coordinate] (input1) {};
    % Output node
    \node [right of=comb2, coordinate] (output) {};
    % Connection arrows
    \draw [->, >=stealth, line width=1.5pt] (input1) -- node[left, xshift = -0.5cm] {$\mathbf P_{t}$} (comb1);
    \draw [->, >=stealth, line width=1.5pt] (comb1) -- node {$\mathbf H_{t}$} (ff);
    \draw [->, >=stealth, line width=1.5pt] (comb1.east) -- node {} (ff);
    \draw [->, >=stealth, line width=1.5pt] (ff) -- node {$\mathbf H_{t-1}$} ([yshift=-0.5cm]comb2.west);
    \draw [->, >=stealth, line width=1.5pt] ([xshift=0.5cm]ff.east) |-  ([yshift=-1.5cm,xshift = -0.5cm]comb1.west) |-  ([yshift=-0.25cm, xshift = -0.15cm]comb1.west) |- ([yshift=-0.25cm]comb1.west);
    \draw [->, >=stealth, line width=1.5pt] (comb2) -- node[right, xshift = 0.5cm] {$\mathbf Y_t$} (output);
    \draw [->, >=stealth, line width=1.5pt, dashed] ([yshift=0cm, xshift = -0.5cm]comb1.west) |- ([yshift=1cm, xshift = -0.75cm]comb2.west) |- ([yshift=0.5cm, xshift =0cm]comb2.west);
\end{tikzpicture}}
%          \vspace{-0.5cm}
%          \caption{}
%          \label{fig2a}
%      \end{subfigure}
%      \hfill
%      \begin{subfigure}[b]{0.39\textwidth}
%          \centering
%          \scalebox{0.9}{\begin{tikzpicture}[scale=1, transform shape];
                \node [nnlayer]                     at ( 0,     0)    (sig1)          {$\mathcal{F}_o$};
                \node [nnlayer]                     at ( 1.5,   0)    (sig2)          {$\mathcal{F}_h$};

                \node [anchor=east]                 at ( -1,  -0.5) (hiddenlast)    {$\mathbf H_{t-1}$};
                \node [anchor=west]                 at ( 3.5,     1.) (hiddennext)    {$\mathbf H_{t}$};

                \node [anchor=west]                 at ( 0,     2.25) (hiddennext1)    {$\mathbf Y_{t}$};

                \node [anchor=east]                 at ( -1, -1.5)  (input)         {$\mathbf P_{t}$};

                \draw [ultra thick, rounded corners=0.2cm] (hiddenlast) -| (sig1);
                \draw [ultra thick, rounded corners=0.2cm] (hiddenlast) -| (sig2);


                \draw [->, >=stealth, ultra thick, rounded corners=0.2cm] (sig1.north) -- ([yshift = 1.5cm]sig1.north);;

                % \draw [->, >=stealth, ultra thick, rounded corners=0.2cm] (sig2.north) |- ([xshift=2cm, yshift=0.5cm]sig2.north) |- ( 6,  -0.5);

                \draw [->, >=stealth, ultra thick, rounded corners=0.2cm] (sig2.north) |- ([xshift=1.5cm, yshift=0.5cm]sig2.north);

                \draw [ultra thick, rounded corners=0.2cm] (input) -- ++(1.2,0) |- (0.5,-0.5);

                \begin{pgfonlayer}{background}
                \draw [fill=dateblue!10, rounded corners=.5cm] (-.5, -1) rectangle (3,1.5);
                \end{pgfonlayer}


            \end{tikzpicture}}
%          \vspace{-0.5cm}
%          \caption{}
%          \label{fig2b}
%      \end{subfigure}
%     \caption{The general form of a sequential circuit is shown in (a), and the recurrent cell for sequential circuits is depicted in (b).}
%     \label{fig2}
%     \vspace{-0.5cm}
% \end{figure*}



% \section{Sequential Circuits}
% So far, we have described how combinational circuits can be modeled and analyzed using {\sc Demotic}. In contrast to combinational circuits, where outputs are determined solely by their present inputs, the output in sequential circuits depends on both the past behavior of the circuit and the present values of inputs. The temporal operations of sequential circuits are controlled by a clock signal. The contents of memory elements (i.e., flip-flops) represent the past behavior of such a circuit, which is commonly referred to as the \textit{state} of the circuit. 


% Solving CircuitSAT problems for sequential circuits presents a unique challenge as it requires finding a sequence of inputs that satisfies the target constraint over a series of clock cycles. To tackle such problems, we can leverage a novel technique inspired by recurrent neural networks (RNNs). In RNNs, backpropagation through time is utilized during the learning process, allowing for updates to the network's hidden state at each time step. Similarly, in the context of solving CircuitSAT problems, we perform forward computations to iteratively update the state values at each clock cycle. During backward computations, gradients are backpropagated through time, extending back to the initial time step (i.e., the first clock cycle), to adjust the input sequence accordingly. While this approach draws parallels to RNN training, it is tailored to the unique challenges posed by solving CircuitSAT problems for sequential circuits. 



% Fig. \ref{fig2a} shows the general structure of a sequential circuit. We use this structure to formulate the CircuitSAT problem for sequential circuits to find satisfying solutions using {\sc Demotic}. In this structure, there are two combination circuits: one to update the state of the circuit (i.e., the content values of flip-flops) and the other one to generate the output. It is worth mentioning that both of these combinational circuits take the present values of flip-flops and primary inputs at the current time step as their inputs. Let us represent the primary input variables at time step $t$ as $\textbf{V}_t \in \mathbb{R}^{b\times n}$. We encode the primary input variables at each time step as learnable parameters to an embedding layer followed by the sigmoid function to provide input probabilities at time $t$ as $\textbf{P}_t \in [0, 1]^{p\times n}$ to the combinational circuits, i.e.,
% \begin{equation}
%     \textbf{P}_t = \sigma(\textbf{V}_t).
% \end{equation}
% The present output of the circuit (i.e., $\textbf{Y}_t \in [0, 1]^{b\times m}$) is computed as: 
% \begin{equation}
%     \textbf{Y}_{t} = \mathcal{F}_o(\textbf{P}_t, \textbf{H}_{t-1}),
% \end{equation}
% where $\mathcal{F}_o$ and $\textbf{H}_t \in [0, 1]^{b \times r}$ denote the functionality of the combinational circuit generating outputs and the present values of flip-flops at each time step, respectively. The number of flip-flops in the circuit is represented by $r$. The state of the circuit for the next time step is obtained as:
% \begin{equation}
%     \textbf{H}_{t} = \mathcal{F}_h(\textbf{P}_t, \textbf{H}_{t-1}),
% \end{equation}
% where the functionality of the combinational circuit updating the values of flip-flops is denoted by $\mathcal{F}_h$. The $\ell_2$-loss function $\mathcal{L}$ can then be constructed by measuring the distance between $\textbf{Y}_t$ at the desired time step $N$ and the target output valuation matrix $\textbf{T} \in \{0, 1\}^{b \times m}$ as follows:
% \begin{equation}
%     \mathcal{L} = \sum_{b,m} \left|\left| \textbf{T} - \textbf{Y}_N \right|\right|^2_2.
% \end{equation}
% With such a formulation for sequential circuits, {\sc Demotic} can solve the CircuitSAT problem and provide $b$ solutions. The general form of the recurrent cell for sequential circuits is shown in Fig. \ref{fig2b}, which is analogous to the RNN cell.


\section{Results}\label{sect:results}

\subsection{Established Benchmarks}\label{sect:results_west}
We begin by evaluating all vision-language models on established benchmarks, based on ImageNet and COCO Captions, among other datasets. As revealed in Table~\ref{tab:west_standard_setup}, increasing the dataset size from 10 billion to 100 billion examples does not improve performance substantially. This is statistically supported by Wilcoxon's signed rank test~\cite{wilcoxon1992individual}, which gives a $p$-value of 0.9, indicating that differences are not significant.


In addition, we also fit data scaling laws for every combination of model and dataset following the recipe proposed in~\citet{alabdulmohsin2022revisiting}. This allows us to evaluate whether or not the performance gap is expected to increase or decrease in the infinite-compute regime. We report the resulting scaling exponents and asymptotic performance limits in the tables. Again, we do not observe  significant differences at the 95\% confidence level ($p$-value of 0.09).


\subsection{Cultural Diversity}
Unlike the Western-oriented metrics reported in Section~\ref{sect:results_west}, cultural diversity metrics present an entirely different picture. We observe \emph{notable} gains when scaling the size of the dataset from 10 billion to 100 billion examples in Table~\ref{tab:culture_standard_setup}. 
For example, scaling training data from 10 billion to 100 billion examples yields substantial gains on Dollar Street 10-shot classification task, where ViT-L and ViT-H see absolute improvements of 5.8\% and 5.4\%, respectively. These gains outperform the typical improvements (less than 1\%) observed on Western-oriented 10-shot metrics by a large margin.
Using Wilcoxon's signed rank test, we obtain a $p$-value of 0.002, indicating a statistically significant evidence at the 99\% confidence level.


\subsection{Multilinguality}

Our multilingual benchmark, Crossmodal-3600 zero-shot retrieval~\cite{thapliyal2022crossmodal}, shows a disparity in performance gains: low-resource languages benefit more from the 100 billion scale than the high-resource ones. The disparity, illustrated in Figure~\ref{fig:multilinguality}, which not only exists in all model sizes but also widens as the models become larger. Detailed results for each language can be found in Appendix~\ref{appendix:data_scale}.

% source: https://colab.corp.google.com/drive/1AKgGDITZqTC2hQjVc-Iv8xuysh5giP0i#scrollTo=2EtEXMbly8dB&line=1&uniqifier=1
\begin{figure}[h!]
    % \includegraphics[width=\linewidth]{figures/multilang-Average_Multilingual__Low-Resource_Lang.pdf}
    % \includegraphics[width=0.86\linewidth]{figures/multilang-Average_Multilingual__High-Resource_Lang.pdf}
    \includegraphics[width=\linewidth]{figures/multilang-Average_XM3600_Retrieval.pdf}
    \caption{Scaling up to 100B examples leads to more notable improvements in low-resource languages. $\Delta$ denotes the improved accuracy when scaling from 10B examples to 100B.}
    \label{fig:multilinguality}
\end{figure}


\subsection{Fairness}
For fairness, we report on 3 metrics discussed in Section~\ref{sect:evals}. 

\paragraph{Representation Bias.} The first metric is representation bias (RB), with results detailed in Table~\ref{tab:rb}. We observe that models trained on unbalanced web data have a significantly higher preference to associate a randomly chosen image from ImageNet~\cite{deng2009imagenet} with the label ``Male'' over the label ``Female.'' 

In fact, this occurs nearly 85\% of the time. Training on 100B examples does not mitigate this effect. This finding aligns with previous research highlighting the necessity of bias mitigation strategies, such as data balancing~\cite{alabdulmohsin2024clip}, to address inherent biases in web-scale datasets.

% \begin{table}[h]
%     \centering\scriptsize
%     \caption{representation bias with respect to gender using imagenet. Here, a value of 0.8, for example, indicates that the model would prefer to associate a randomly chosen image from ImageNet with the label ``Male'' over the label ``Female''.}
%     \label{tab:rb}
%     \begin{tabularx}{\columnwidth}{@{}c|YYY@{}}
%     \toprule
%     \bf Model&\bf1B &\bf10B &\bf100B\\
%     \midrule
% B & 83.2&84.5&85.2
% \\
% L & 88.2&86.4&85.5\\
% H & 86.8&85.0&86.6\\
% \bottomrule
%     \end{tabularx}
% \end{table}

\begin{table}[h]
\begin{tabularx}{\columnwidth}{c|YYY@{}}
    \toprule
    \bf Model&\bf1B &\bf10B &\bf100B\\
    \midrule
B & 83.2&84.5&85.2\\
L & 88.2&86.4&85.5\\
H & 86.8&85.0&86.6\\
\bottomrule
\end{tabularx}
\captionof{table}{Representation bias w.r.t. gender (see Section~\ref{sect:results}). Here,  values [\%] indicate how often the model prefers to associate a random  image with the label ``Male'' over ``Female''.} \label{tab:rb}
\end{table}



\paragraph{Association Bias.} Second, Figure~\ref{fig:ab} shows the association bias in SigLIP-H/14 between gender and occupation as we scale the data from 10 to 100 billion examples. Specifically, we plot the probability that the model would prefer a particular occupation label, such as ``{\fontfamily{lmodern}\selectfont secretary}'' over another label, such as ``{\fontfamily{lmodern}\selectfont manager}'' when images correspond to males or females. In this evaluation, we use the Fairface~\cite{karkkainen2021fairface} dataset. The labels we compare are: ``{\fontfamily{lmodern}\selectfont librarian}'' vs. ``{\fontfamily{lmodern}\selectfont scientist}'', ``{\fontfamily{lmodern}\selectfont nurse}'' vs. ``{\fontfamily{lmodern}\selectfont doctor}'', ``{\fontfamily{lmodern}\selectfont housekeeper}'' vs. ``{\fontfamily{lmodern}\selectfont homeowner}'', ``{\fontfamily{lmodern}\selectfont receptionist}'' vs. ``{\fontfamily{lmodern}\selectfont executive}'' and ``{\fontfamily{lmodern}\selectfont secretary}'' vs. ``{\fontfamily{lmodern}\selectfont manager}''. Again, we do not see a reduction in association bias by simply increasing the size of the training data. %Full results are in Appendix~\ref{appendix:ab}.

%Additionally, we are unable to evaluate cultural diversity and fairness in PaliGemma's transfer tasks due to the lack of appropriate benchmarks. This is an open question that we hope to address in the future.

\paragraph{Performance Disparity.} Finally, one common definition of fairness in machine learning is maintaining similar performance across different groups. See, for instance,~\citet{dehghani2023scaling} and the related notions of ``Equality of Opportunity'' and ``Equalized Odds''~\cite{hardt2016equalityopportunitysupervisedlearning}. Table~\ref{tab:perf_disparity} show that scaling the data to 100 billion examples improves performance disparity, which is consistent with the improvement in cultural diversity.

%  to show on top of page
% \begin{table}[h]
    \centering\scriptsize
    \begin{tabularx}{\columnwidth}{l|YYY@{}}
    \toprule
    \bf Model & \bf 1B & \bf 10B & \bf 100B\\ \midrule
    &\multicolumn{3}{c}{\em 0-shot Dollar Street}\\[2pt]
B & 32.5 & 29.9 & \bf29.0\\
L & \bf29.7 & 29.8 & 30.4 \\
H & 32.2 & 33.0 & \bf32.1\\
\midrule 
    &\multicolumn{3}{c}{\em 0-shot GeoDE}\\[2pt]
B & 4.7 & 5.5 & \bf4.4\\
L & 3.2 & 4.0 & \bf2.8 \\
H & 3.6 & 3.0 & \bf2.7\\
\bottomrule
 
    \end{tabularx}
    \caption{Performance disparity (lower is better) for models pretrained on 100B seen examples of different data scales. Pretraining on 100B examples tends to lower disparity.}
    \label{tab:per_disp_mini}
\end{table}
% \FloatBarrier

\begin{table*}[h]
    \centering\scriptsize
    \caption{Performance disparity results for various SigLIP models pretrained on 100 billion seen examples of 1B, 10B, and 100B datasets. Here, disparity corresponds to the maximum gap across subgroups in Dollar Street (by income level) and GeoDE (by geographic region). Pretraining on 100B examples tends to improve disparity overall.}
    \label{tab:perf_disparity}
    \begin{tabularx}{2\columnwidth}{ll|YYYYYY|Y}
    \toprule
    \bf Model & \bf Data Scale &\multicolumn{6}{c}{\bf Performance per Subgroup} & \bf Disparity\\ \midrule
    \multicolumn{8}{c}{\em 0-shot Dollar Street}\\[2pt]
    & & \bf 0-200	& \bf 200-685	& \bf 685-1998	& \bf $>$1998
    & & & \\ \midrule
B&1B&29.4&43.9&56.5&62.0&&&32.5\\
B&10B&31.6&44.0&55.4&61.5&&&29.9\\
B&100B&32.0&44.3&56.3&61.0&&&\bf29.0\\[3pt]
L&1B&33.7&44.7&57.3&63.4&&&\bf29.7\\
L&10B&35.7&47.8&58.7&65.5&&&29.8\\
L&100B&33.7&46.6&59.5&64.1&&&30.4\\[3pt]
H&1B&32.3&44.9&58.4&64.5&&&32.2\\
H&10B&33.9&46.3&58.6&66.9&&&33.0\\
H&100B&34.1&48.2&62.2&66.1&&&\bf32.1\\ \midrule

    \multicolumn{8}{c}{\em 0-shot GeoDE}\\[2pt]
    & & \bf Africa	& \bf Americas	& \bf East-Asia	& \bf Europe & \bf South-East Asia & \bf West Asia
    & \\ \midrule
B&1B&89.4&92.1&91.8&94.1&92.5&93.4&4.7\\
B&10B&88.4&91.8&91.4&94.0&92.2&93.0&5.5\\
B&100B&88.8&91.4&91.0&93.3&91.7&92.2&\bf4.4\\[3pt]
L&1B&92.0&94.0&94.0&95.2&94.2&94.9&3.2\\
L&10B&91.8&94.4&94.0&95.8&94.2&94.7&4.0\\
L&100B&93.5&95.1&95.4&96.2&95.0&95.8&\bf2.8\\[3pt]
H&1B&91.5&94.4&94.7&95.2&94.1&94.5&3.6\\
H&10B&93.4&95.4&95.0&96.5&95.1&95.6&3.0\\
H&100B&93.6&95.1&95.3&96.3&95.2&95.8&\bf2.7\\

 \bottomrule
    \end{tabularx}
\end{table*}


\subsection{Transfer To Generative Models}
\label{sec:transfer}

\begin{table}[h!]
\centering
\footnotesize

% Note: We removed 1b result to avoid confusion to readers. See https://docs.google.com/document/d/1YxRpUO7elSaviOQ5XIXtnUWajgYAF7FfRO0vmXQCUtU/edit?resourcekey=0-pPjeeIrYEXRuvnuBFZn5Uw&tab=t.0#heading=h.gaczi2wqv0go.
\begin{tabular}{p{0pt}l|rrrrr}
\toprule
& Data & Semantics & OCR & Multiling & RS & Avg \\
\midrule
% % source: https://docs.google.com/spreadsheets/d/1W5_VNitkO6k-HSKBV6sGX81EGXgma877m0gGsg8zPzw/edit?resourcekey=0-2zH_U5z5kL9I5Nnhg7SChQ&gid=1972601917#gid=1972601917
\includegraphics[width=8pt]{images/snowflake_2744-fe0f.png} & 1B & 76.0 & 66.8 & 67.0 & 92.3 & 73.6 \\
\includegraphics[width=8pt]{images/snowflake_2744-fe0f.png} & 10B & 75.4 & 65.2 & 66.3 & 91.9 & 72.7 \\
\includegraphics[width=8pt]{images/snowflake_2744-fe0f.png} & 100B & 76.4 & 67.0 & 66.9 & 92.1 & 73.9 \\
\includegraphics[width=8pt]{images/fire_1f525.png} & 1B & 77.1 & 69.5 & 66.9 & 92.0 & 75.1 \\
\includegraphics[width=8pt]{images/fire_1f525.png} & 10B & 76.4 & 66.9 & 66.0 & 91.8 & 73.7 \\
\includegraphics[width=8pt]{images/fire_1f525.png} & 100B & {77.2} & {70.0} & {67.0} & {91.8} & {75.3} \\
\bottomrule
\end{tabular}

\caption{
The PaliGemma transfer results of ViT-L/16 models pretrained on 10B and 100B examples, with both frozen (top) %({\includegraphics[width=8pt]{images/snowflake_2744-fe0f.png}}) 
and unfrozen (bottom) %({\includegraphics[width=8pt]{images/fire_1f525.png}}) 
vision components. Results are aggregated.
}
\label{tab:transfer_avg}
\end{table}


We use PaliGemma~\citep{beyer2024paligemma} with both frozen and unfrozen vision component to assess the transferability of our vision models, which were contrastively pre-trained on datasets of different scales. In Table~\ref{tab:transfer_avg}, when taking the noise level into consideration, we do not observe consistent performance gains across downstream tasks as we scale the pre-training dataset. More details can be found in Appendix~\ref{appendix:transfer}.


%\paragraph{Recognizing Sensitive Attributes.}
%Finally, we also report the performance of the models in recognizing sensitive attributes, following a similar evaluation in~\citet{radford2021learning}. We report the accuracy in predicting perceived gender in Fairface~\cite{karkkainen2021fairface} and predicting perceived race in UTK~\cite{utkface_url}. Overall, we observe that scaling the data to 100 billion examples improves this aspect of fairness as well. Table~\ref{tab:fairness_pred} provides the full results. We do not observe a particular pattern in this type of evaluation.
%\begin{table}[t]
    \centering\scriptsize
    \begin{tabularx}{\columnwidth}{@{}ll|YY@{}}
    \toprule
    \bf Model & \bf Data Scale & \bf Gender & \bf Race\\ \midrule
B	&	1B	& 91.0	& 58.6\\
B	&	10B	& 91.7 &	\bf59.5\\
B	&	100B &	\bf91.9	& 53.1\\[3pt]
L	&	1B	& 0.94.8	& \bf55.4\\
L	&	10B	& 93.9	& 53.1\\
L	&	100B &	\bf95.0	& 54.0\\[3pt]
H	&	1B	& 94.5	& \bf54.5\\
H	&	10B	& \bf95.4	& 54.3\\
H	&	100B	& \bf95.4	& 50.2 \\
 \bottomrule
    \end{tabularx}
    \caption{Accuracy in recognizing sensitive attributes using Fairface and UTK datasets. See Section~\ref{sect:results} for details.}
    \label{tab:fairness_pred}
\end{table}

\section{Discussion}
\subsection{Case Study}


Fig. \ref{fig:casestudy} shows 2-D UMAP \cite{mcinnes2020umapuniformmanifoldapproximation} projections of embedding vectors for PetClinic Microservices \cite{microapps2024petclinic} using VoyageAI, ME-unixcoder-340K, and ME-llm2vec-340K. ME-unixcoder and ME-llm2vec show clearer microservice clusters compared to VoyageAI and Fig. \ref{fig:mexample}. For instance, \textit{API-Gateway} service classes are split in VoyageAI's representation but closer in the other models. ME-llm2vec demonstrates the closest grouping within microservices and clearest separation between them. In fact, ME-llm2vec's figure shows only 6 clear outliers which we review in detail and display their names and neighbors.



The two \textit{MetricConfig} classes, \textit{ResourceNotFoundException} and \textit{CacheConfig} lack domain-specific terms since they are utility classes, which highlights the importance of separating them from domain-related ones during the decomposition. However, ME-llm2vec was able correctly represent classes with even slight domain hints. For instance, most models struggle to differentiate between the nearly identical entry-point classes (e.g. \textit{ConfigServerApplication}), as seen in Fig. \ref{fig:mexample} and \ref{fig:casestudy} while ME-llm2vec managed to correctly place them within their services. On the other hand, the class \textit{PetRequest}, which was closer to \textit{API-Gateway} instead of \textit{Customers}, shows an intriguing outlier. Despite ME-llm2vec correctly matching the "Pet" related classes, it failed with \textit{PetRequest}. its function as a Request object, which is typically associated with the Gateway pattern, is a potential reason. Notably, ME-llm2vec successfully identified \textit{API-Gateway} classes, differentiating them from \textit{Customers}. We find this interesting because \textit{API-Gateway} includes classes representing various bounded contexts, often causing confusion in other models. ME-llm2vec recognized these classes' distinct purpose, grouping them together despite their diverse domains.

% Both \textit{API-Gateway} and \textit{Customers} services contain a "PetType" class. But in \textit{Customers}'s case, this class was closer to the "Specialty" class from \textit{Vets}, which is likely due to nearly identical source code they have. 

\subsection{Discussion}


We designed the analysis component to be as abstract as possible to accommodate the rapidly evolving representation learning landscape. As new and improved embedding models are published, they can be integrated with minimal effort. While our evaluation results show that with ME-LLM2Vec, we can generate highly cohesive and consistent decompositions, one of our objectives is to highlight the potential of Language Models in generating more efficient representations than traditional approaches for the decomposition problem. In fact, MonoEmbed is both a decomposition approach (when considering the full approach) and an embedding model (when using models such as ME-LLM2Vec). These models can be used to enrich existing decomposition approaches. For example, MicroMiner's CodeBERT \cite{trabelsi2023microminer} can be replaced with ME-LLM2Vec and the GNN based methods \cite{desai2021cogcn,yedida2023deeply,mathai2022chgnn,qian2023gdcdvf} can be extended by using ME-LLM2Vec as the encoder. In fact, it can be used as an additional representation type in approaches such as \cite{khaled2022hydecomp,qian2023gdcdvf}. These models can be even extended further by incorporating unstructured inputs (e.g. resources, configurations, documentation) and different PLs.




\subsection{Threats to Validity}
\subsubsection{Internal Validity}
Clustering algorithms and decomposition approaches have hyper-parameters that can affect performance on evaluation benchmarks. To mitigate this threat, we compared their performance with different hyper-parameter inputs across a varied set of evaluation applications.

\subsubsection{External Validity}
To address the threat of our approach to generalize on monolithic applications and PLs, we used a large set of monolithic and microservices applications from related work \cite{kalia2021mono2micro,khaled2022hydecomp,yedida2023deeply,jin2021fosci} to benchmark decomposition approaches. 

\subsubsection{Construct Validity}
This threat can potentially be in the form of the evaluation metrics used in our experiments. In order to mitigate this threat, we employ established metrics in supervised learning tasks (RQ1-3) and different metrics from decomposition research \cite{khaled2022hydecomp,kalia2021mono2micro,jin2021fosci,yedida2023deeply,mathai2022chgnn} (RQ4). 
\section*{Limitations}

\method{} requires the reference policy because it has a KL penalty from the reference policy, like DPO in default. It leads to the limitation that it requires additional memory consumption and computation for reference policy compared to other direct alignment algorithms that do not perform regularization through the reference policy~\cite{zhao2023slic, xu2024contrastive, hong2024orpo, meng2024simpo}. However, theoretically, \method{} can save memory consumption by pre-computing the logits of the responses from the reference policy, similar to DPO. Meanwhile, \method{} is a general purposes approach not specially tailored for safety alignment, so additional safety considerations may be required to control inappropriate responses in real usages.


% Bibliography entries for the entire Anthology, followed by custom entries
%\bibliography{anthology,custom}
% Custom bibliography entries only
\bibliography{custom}

\newpage
\appendix
\onecolumn

\renewcommand{\thetable}{A\arabic{table}} % Prefix table numbers with 'A'
\renewcommand{\thefigure}{A\arabic{figure}} % Prefix figure numbers with 'A'
\renewcommand{\theequation}{A\arabic{equation}} % Prefix equation numbers with 'A'

\setcounter{table}{0} % Reset table counter
\setcounter{figure}{0} % Reset figure counter
\setcounter{equation}{0} % Reset equation counter

\section*{Appendix}

\section{Optimal Brain Surgeon Derivation}
\label{OBS_ALGORITHM}

In the original setup in OBS, we have a local quadratic model for the loss $L$ given by:
$$
    \delta L = L(w + \delta w) \approx L(w) + \nabla_w L^T \delta w + \frac{1}{2} \delta w^T H \delta w
$$
Since OBS is a pruning-after-training approach, they discarded the 1-st order component. Reducing the expression for saliency as:
$$
    \delta L = \frac{1}{2} \delta w^T H \delta w
$$
To remove a single parameter, the authors of OBS introduced the constraint $e_q^T \delta w + w_q = 0$, with $e_q$ being the $q^{\text{th}}$ canonical basis vector. The pruning is defined as a constrained optimization problem of the form:
$$
    \min_{\delta w \in \mathbb{R^d}} \left( \frac{1}{2} \delta w^T H \delta w\right),
    ~~\text{s.t}~~
    e_q^T \delta w + w_q = 0.
$$
And the choice of which parameter to remove becomes:
$$
    \min_{q \in \mathcal{Q}} \left\{
        \min_{\delta w \in \mathbb{R^d}} \left( \frac{1}{2} \delta w^T H \delta w\right),
        ~~\text{s.t}~~
        e_q^T \delta w + w_q = 0
    \right\}.
$$
To solve the internal problem, we use a Lagrange multiplier $\lambda$ to write the problem as an unconstrained optimization case as follows:
$$
    \mathcal{L}(\delta w, \lambda) =
    \frac{1}{2} \delta w^T H \delta w +
    \lambda(e_q^T \delta w + w_q).
$$
Then, to find the stationary conditions, we compute the partial derivatives with respect to $\delta w$ and $\lambda$, and equate them to 0, obtaining:
$$
    \nabla_{\delta w} \mathcal{L} = 
    H \delta w + \lambda e_q = 0 
    \rightarrow
    \delta w = - \lambda H^{-1} e_q
$$
$$
    \nabla_{\lambda} \mathcal{L} =
    e_q^T \delta w + w_q = 0
    \rightarrow
    e_q^T \delta w = -w_q
$$
With some replacements, we get:
$$
    e_q^T \delta w = -w_q
    \rightarrow
    e_q^T \left( 
        - \lambda H^{-1} e_q
    \right) = -w_q
    \rightarrow
    - \lambda e_q^T H^{-1} e_q = -w_q
    \rightarrow
    \lambda = \frac{w_q}{e_q^T H^{-1} e_q} = \frac{w_q}{[H^{-1}]_{qq}}
$$
$$
    \delta w = - \frac{w_q H^{-1} e_q}{[H^{-1}]_{qq}}
$$
Replacing the expression for $\delta w$ in the saliency expression, we have:
\begin{align*}
    \delta L = \frac{1}{2} \delta w^T H \delta w
    &= \frac{1}{2}\left(
        - \frac{w_q H^{-1} e_q}{[H^{-1}]_{qq}}
    \right)^T
    H
    \left(
        - \frac{w_q H^{-1} e_q}{[H^{-1}]_{qq}}
    \right)
    \nonumber \\
    &= 
    \frac{w_q^2}{2[H^{-1}]_{qq}^2}
    \left(
        H^{-1} e_q
    \right)^T
    H
    \left(
        H^{-1} e_q
    \right)
    \nonumber \\
    &= 
    \frac{w_q^2}{2[H^{-1}]_{qq}^2}
    e_q ^T
    H^{-1}
    e_q
    = 
    \frac{w_q^2}{2[H^{-1}]_{qq}^2}
    [H^{-1}]_{qq}
    = 
    \frac{w_q^2}{2[H^{-1}]_{qq}}
    \nonumber \\
\end{align*}
%------------------------------------------------------------------------------------------------
\newpage
\section{Fisher Brain Surgeon Sensitivity Derivation}
\label{FBSS_ALGORITHM}
As we considered a PBT setting, it is not possible to ignore the first-order term in the local quadratic approximation of the error as it could still be informative. In this case, our model for sensitivity is given by: 
$$
    \delta L = \nabla_w L^T \delta w + \frac{1}{2} \delta w^T H \delta w
$$
The process to remove a single parameter remains similar; the constraint $e_q^T \delta w + w_q = 0$, with $e_q$ is still valid, redefining the optimization problem as:
$$
    \min_{\delta w \in \mathbb{R^d}} \left(
        \nabla_w L^T \delta w +  \frac{1}{2} \delta w^T H \delta w
    \right),
    ~~\text{s.t}~~
    e_q^T \delta w + w_q = 0.
$$
And the choice of which parameter to remove becomes:
$$
    \min_{q \in \mathcal{Q}} \left\{
        \min_{\delta w \in \mathbb{R^d}} \left(
            \nabla_w L^T \delta w + \frac{1}{2} \delta w^T H \delta w
        \right),
        ~~\text{s.t}~~
        e_q^T \delta w + w_q = 0
    \right\}.
$$
Using a Lagrange multiplier $\lambda$ as in the reference case, we solve the following unconstrained optimization problem:
$$
    \mathcal{L}(\delta w, \lambda) =
    \nabla_w L^T \delta w + 
    \frac{1}{2} \delta w^T H \delta w +
    \lambda(e_q^T \delta w + w_q).
$$
With the following stationary conditions:
$$
    \nabla_{\delta w} \mathcal{L} = 
    \nabla_w L + H \delta w + \lambda e_q = 0 
    \rightarrow
    \delta w = - (\lambda H^{-1}e_q + H^{-1} \nabla_w L)
$$
$$
    \nabla_{\lambda} \mathcal{L} =
    e_q^T \delta w + w_q = 0
    \rightarrow
    e_q^T \delta w = -w_q
$$
The expression for $\lambda$ is redefined as follows:
\begin{align*}
    e_q^T \left(
        - (\lambda H^{-1}e_q + H^{-1} \nabla_w L)
    \right) 
    &= -w_q
    \nonumber \\
    \lambda e_q^T H^{-1} e_q + e_q^T H^{-1} \nabla_w L
    &= w_q
    \nonumber \\
    \lambda [H^{-1}]_{qq} 
    &= w_q - e_q^T H^{-1} \nabla_w L
    \nonumber \\
    \lambda
    &= \frac{w_q - e_q^T H^{-1} \nabla_w L}{[H^{-1}]_{qq}}
\end{align*}
Replacing the expression for $\delta w$ in our sensitivity expression, we have:
\begin{align*}
    \delta L = \nabla_w L^T \delta w + \frac{1}{2} \delta w^T H \delta w
    &= 
    \nabla_w L^T \left[
        - (\lambda H^{-1}e_q + H^{-1} \nabla_w L)
    \right]
    \nonumber \\
    &+
    \frac{1}{2}\left[
        - (\lambda H^{-1}e_q + H^{-1} \nabla_w L)
    \right]^T
    H
    \left[
        - (\lambda H^{-1}e_q + H^{-1} \nabla_w L)
    \right]
    \nonumber \\
    &= 
    - \lambda \nabla_w L^T H^{-1}e_q - \nabla_w L^T H^{-1} \nabla_w L
    \nonumber \\
    &+
    \frac{1}{2}\left[
        (\lambda H^{-1}e_q)^T + (H^{-1} \nabla_w L)^T
    \right]
    \left[
        \lambda H H^{-1}e_q + H H^{-1} \nabla_w L)
    \right]
    \nonumber \\
    &= 
    - \lambda \nabla_w L^T H^{-1}e_q - \nabla_w L^T H^{-1} \nabla_w L
    \nonumber \\
    &+
    \frac{1}{2}\left[
        (\lambda H^{-1}e_q)^T + (H^{-1} \nabla_w L)^T
    \right]
    \left[
        \lambda e_q + \nabla_w L
    \right]
    \nonumber \\
    &= 
    - \lambda \nabla_w L^T H^{-1}e_q - \nabla_w L^T H^{-1} \nabla_w L
    \nonumber \\
    &+
    \frac{1}{2}\left[
        (\lambda H^{-1}e_q)^T \lambda e_q
        + (H^{-1} \nabla_w L)^T \lambda e_q
        + (\lambda H^{-1}e_q)^T \nabla_w L
        + (H^{-1} \nabla_w L)^T \nabla_w L
    \right]
    \nonumber \\
    &= 
    - \lambda \nabla_w L^T H^{-1}e_q - \nabla_w L^T H^{-1} \nabla_w L
    \nonumber \\
    &+
    \frac{1}{2}\left[
        \lambda^2 e_q^T H^{-1} e_q
        + \lambda \nabla_w L^T H^{-1} e_q
        + \lambda e_q^T H^{-1} \nabla_w L
        + \nabla_w L^T H^{-1} \nabla_w L
    \right]
    \nonumber \\
    &= 
    \frac{1}{2}\left[
        \lambda^2 [H^{-1}]_{qq}
        - \lambda \nabla_w L^T H^{-1} e_q
        + \lambda e_q^T H^{-1} \nabla_w L
        - \nabla_w L^T H^{-1} \nabla_w L
    \right]
    \nonumber \\
\end{align*}
Finally, replacing the $\lambda$:
\begin{align*}
    \delta L 
    &= 
    \frac{1}{2}\left[
        \lambda^2 [H^{-1}]_{qq}
        - \lambda \nabla_w L^T H^{-1} e_q
        + \lambda e_q^T H^{-1} \nabla_w L
        - \nabla_w L^T H^{-1} \nabla_w L
    \right]
    \nonumber \\
    &= 
    \frac{1}{2[H^{-1}]_{qq}}\left[
        (w_q - e_q^T H^{-1} \nabla_w L)^2 
        + (w_q - e_q^T H^{-1} \nabla_w L)(e_q^T H^{-1} \nabla_w L - \nabla_w L^T H^{-1} e_q)
        - \nabla_w L^T H^{-1} \nabla_w L
    \right]
    \nonumber \\
    &= 
    \frac{1}{2[H^{-1}]_{qq}}[
        w_q^2
        - 2 w_q (e_q^T H^{-1} \nabla_w L)
        + (e_q^T H^{-1} \nabla_w L)^2
        + w_q (e_q^T H^{-1} \nabla_w L)
    \nonumber \\
        &- w_q (\nabla_w L^T H^{-1} e_q)
        - (e_q^T H^{-1} \nabla_w L)(e_q^T H^{-1} \nabla_w L)
        + (e_q^T H^{-1} \nabla_w L)(\nabla_w L^T H^{-1} e_q)
        - \nabla_w L^T H^{-1} \nabla_w L
    ]
    \nonumber \\
    &= 
    \frac{1}{2[H^{-1}]_{qq}}[
        w_q^2
        - w_q (e_q^T H^{-1} \nabla_w L)
        + (e_q^T H^{-1} \nabla_w L)^2
    \nonumber \\
        &- w_q (\nabla_w L^T H^{-1} e_q)
        - (e_q^T H^{-1} \nabla_w L)^2
        + (e_q^T H^{-1} \nabla_w L)(\nabla_w L^T H^{-1} e_q)
        - \nabla_w L^T H^{-1} \nabla_w L
    ]
    \nonumber \\
    &= 
    \frac{1}{2[H^{-1}]_{qq}}\left[
        w_q^2
        - 2 w_q (e_q^T H^{-1} \nabla_w L)
        + (e_q^T H^{-1} \nabla_w L)^2
        - \nabla_w L^T H^{-1} \nabla_w L
    \right]
    \nonumber \\
    &= 
    \frac{1}{2[\hat{F}^{-1}]_{qq}}
    \left[
        w_q - (e_q^T \hat{F}^{-1} \nabla \mathcal{L}(w_0))
    \right]^2
\end{align*}

%------------------------------------------------------------------------------------------------

\newpage
\section{Training and Testing Details}
\label{appendix:training_parameters}

We perform an 80:20 stratified split, with a constant seed, on the CIFAR10/100 training dataset to obtain a validation set with the same class distribution. For both datasets, we have a training set with 40,000 samples, a validation set with 10,000 samples, and a testing set of 10,000 samples. Validation is performed after each training step, and the weights of the best-performing validation step (based on top-1 accuracy) are utilized for the final evaluation on the testing set. Table \ref{tab:table_training_parameters} summarizes the training parameters.

\begin{table}[h]
\caption{Training parameters used for ResNet18 and VGG19 on the CIFAR-10/100 datasets.}
\label{tab:table_training_parameters}
\vskip 0.15in
\begin{center}
\begin{small}
\begin{sc}
\begin{tabular}{lcc}
\toprule
Parameter & ResNet18 & VGG19 \\
\midrule
Number of steps       & 160 & 160 \\
Criterion             & CE & CE \\
Optimizer             & SGD & SGD \\
Learning rate         & 0.01 & 0.1 \\
Momentum              & 0.9 & 0.9 \\
Weight decay          & $5 \times 10^{-4}$ & $1 \times 10^{-4}$ \\
Learning rate drops   & [60, 120] & [60, 120] \\
Learning rate drop factor & 0.2 & 0.1 \\
\bottomrule
\end{tabular}
\end{sc}
\end{small}
\end{center}
\vskip -0.1in
\end{table}

%------------------------------------------------------------------------------------------------

\newpage
\section{Results CIFAR10}
\subsection{ResNet18}
\label{appendix:CIFAR10_ResNet18}

\begin{table}[h]
\caption{Performance of different sensitivity methods for pruning evaluated using ResNet18 on the CIFAR-10 testset. The right side of the table presents our proposed criteria. The mean accuracy and standard deviation are reported across three initialization seeds for various sparsity levels. Baseline, no pruning: $91.78 \pm 0.09$.}
\label{tab:resnet18_cifar10_compressors}
\vskip 0.15in
\begin{center}
\begin{small}
\begin{sc}
\resizebox{\textwidth}{!}{%
\begin{tabular}{lccccc|cccc}
\toprule
Sparsity  & Random & Magnitude & GN & SNIP & GraSP & FD & FP & FTS & FBSS \\
\midrule
0.10  & 91.71 ± 0.21 & 91.72 ± 0.07 & 91.57 ± 0.15 & 91.72 ± 0.07 & 89.16 ± 0.05 & 91.87 ± 0.13 & 91.63 ± 0.21 & 91.53 ± 0.12 & 91.76 ± 0.08 \\
0.20  & 91.63 ± 0.11 & 91.42 ± 0.12 & 91.51 ± 0.09 & 91.64 ± 0.16 & 88.69 ± 0.34 & 91.50 ± 0.12 & 91.65 ± 0.14 & 91.53 ± 0.15 & 91.54 ± 0.13 \\
0.30  & 91.45 ± 0.18 & 91.61 ± 0.13 & 91.68 ± 0.20 & 91.65 ± 0.08 & 88.67 ± 0.26 & 91.65 ± 0.18 & 91.44 ± 0.27 & 91.49 ± 0.05 & 91.62 ± 0.07 \\
0.40  & 91.59 ± 0.18 & 91.06 ± 0.16 & 91.61 ± 0.09 & 91.55 ± 0.08 & 88.24 ± 0.33 & 91.51 ± 0.05 & 91.38 ± 0.13 & 91.56 ± 0.28 & 91.39 ± 0.05 \\
0.50  & 91.60 ± 0.06 & 91.32 ± 0.13 & 91.44 ± 0.13 & 91.22 ± 0.07 & 87.69 ± 0.15 & 91.30 ± 0.18 & 91.58 ± 0.16 & 91.46 ± 0.19 & 91.41 ± 0.05 \\
0.60  & 91.10 ± 0.16 & 91.18 ± 0.16 & 91.59 ± 0.13 & 91.24 ± 0.04 & 87.48 ± 0.55 & 91.34 ± 0.07 & 91.35 ± 0.16 & 91.40 ± 0.11 & 91.38 ± 0.18 \\
0.70  & 91.17 ± 0.04 & 91.07 ± 0.07 & 91.19 ± 0.17 & 91.33 ± 0.18 & 87.26 ± 0.34 & 91.34 ± 0.23 & 91.42 ± 0.23 & 91.18 ± 0.18 & 91.27 ± 0.14 \\
0.80  & 90.78 ± 0.08 & 91.10 ± 0.12 & 90.95 ± 0.35 & 90.74 ± 0.10 & 87.18 ± 0.51 & 90.95 ± 0.11 & 91.08 ± 0.06 & 90.94 ± 0.22 & 90.73 ± 0.33 \\
0.90  & 89.35 ± 0.13 & 89.88 ± 0.28 & 90.39 ± 0.23 & 90.36 ± 0.34 & 86.60 ± 0.51 & 90.04 ± 0.21 & 90.20 ± 0.08 & 90.55 ± 0.23 & 89.22 ± 0.30 \\
0.95  & 87.59 ± 0.11 & 89.23 ± 0.19 & 89.00 ± 0.05 & 89.31 ± 0.17 & 86.50 ± 0.05 & 88.61 ± 0.28 & 89.50 ± 0.18 & 89.47 ± 0.32 & 87.58 ± 0.25 \\
0.98  & 83.47 ± 0.20 & 85.70 ± 0.33 & 86.43 ± 0.05 & 87.26 ± 0.28 & 85.99 ± 0.08 & 85.61 ± 0.20 & 86.97 ± 0.22 & 87.24 ± 0.32 & 83.40 ± 0.74 \\
0.99  & 78.28 ± 0.45 & 71.99 ± 0.28 & 83.47 ± 0.15 & 84.54 ± 0.04 & 84.56 ± 0.46 & 82.13 ± 0.28 & 83.74 ± 0.48 & 84.85 ± 0.18 & 77.60 ± 1.02 \\
\bottomrule
\end{tabular}}
\end{sc}
\end{small}
\end{center}
\vskip -0.1in
\end{table}

%------------------------------------------------------------------------------------------------
\clearpage
\subsection{VGG19}
\label{appendix:CIFAR10_VGG19}

As discussed earlier, introducing a warm-up phase effectively mitigates layer collapse in data-dependent pruning methods. Here, we evaluate the impact of different warm-up durations by comparing no warm-up, a single warm-up epoch, and five warm-up epochs. Table \ref{tab:VGG19_cifar10_compressors} demonstrates how performance drastically degrades with increasing sparsity, ultimately leading to layer collapse at 0.90 sparsity. However, as shown in the results, a single warm-up epoch is sufficient to prevent collapse and stabilize pruning performance. Moreover, as seen in Table \ref{tab:VGG19_cifar10_compressors_warmup5}, increasing the warm-up period to five epochs provides no substantial additional improvement. This indicates that prolonged warm-up training is not necessary; a single training step is enough to achieve gradient stabilization and overcome layer collapse.

\begin{table}[h]
\caption{Performance of different sensitivity methods for pruning evaluated using VGG19 on the CIFAR-10 test set. The right side of the table presents our proposed criteria. The mean accuracy and standard deviation are reported across three initialization seeds for various sparsity levels. Baseline, no pruning: $89.21 \pm 0.22$.}
\label{tab:VGG19_cifar10_compressors}
\vskip 0.15in
\begin{center}
\begin{small}
\begin{sc}
\resizebox{\textwidth}{!}{%
\begin{tabular}{lccccc|cccc}
\toprule
Sparsity  & Random & Magnitude & GN & SNIP & GraSP & FD & FP & FTS & FBSS \\
\midrule
0.10  & 88.40 ± 0.95 & 89.12 ± 0.55 & 90.14 ± 0.10 & 90.16 ± 0.18 & 87.81 ± 1.66 & 90.20 ± 0.29 & 90.21 ± 0.37 & 90.25 ± 0.38 & 89.06 ± 0.75 \\
0.20  & 89.19 ± 0.22 & 89.65 ± 0.60 & 89.59 ± 0.69 & 90.06 ± 0.04 & 89.57 ± 0.34 & 89.91 ± 0.28 & 90.28 ± 0.55 & 89.80 ± 0.28 & 88.89 ± 0.76 \\
0.30  & 88.93 ± 0.83 & 88.77 ± 1.07 & 90.23 ± 0.09 & 89.88 ± 0.59 & 89.14 ± 0.19 & 90.25 ± 0.09 & 89.97 ± 0.26 & 90.46 ± 0.41 & 89.06 ± 0.36 \\
0.40  & 88.28 ± 1.08 & 89.38 ± 0.53 & 90.50 ± 0.23 & 89.79 ± 0.67 & 88.20 ± 0.31 & 90.51 ± 0.12 & 90.37 ± 0.24 & 90.23 ± 0.14 & 10.00 ± 0.00 \\
0.50  & 88.96 ± 0.82 & 89.03 ± 0.59 & 90.46 ± 0.60 & 90.38 ± 0.25 & 88.67 ± 0.23 & 89.54 ± 0.86 & 90.47 ± 0.52 & 90.19 ± 0.31 & 10.00 ± 0.00 \\
0.60  & 88.15 ± 0.68 & 89.47 ± 0.18 & 89.95 ± 0.30 & 90.32 ± 0.25 & 88.82 ± 0.32 & 90.02 ± 0.40 & 90.18 ± 0.33 & 90.14 ± 0.36 & 10.00 ± 0.00 \\
0.70  & 88.02 ± 0.53 & 89.63 ± 0.44 & 89.69 ± 0.42 & 89.23 ± 0.19 & 89.62 ± 0.81 & 89.85 ± 0.08 & 90.01 ± 0.34 & 10.00 ± 0.00 & 10.00 ± 0.00 \\
0.80  & 88.28 ± 0.34 & 89.62 ± 0.91 & 85.72 ± 0.63 & 89.39 ± 0.43 & 88.82 ± 0.14 & 10.00 ± 0.00 & 88.29 ± 0.11 & 10.00 ± 0.00 & 10.00 ± 0.00 \\
0.90  & 85.82 ± 0.19 & 89.29 ± 0.79 & 10.00 ± 0.00 & 80.85 ± 0.62 & 24.28 ± 20.2 & 10.00 ± 0.00 & 10.00 ± 0.00 & 10.00 ± 0.00 & 10.00 ± 0.00 \\
0.95  & 84.41 ± 0.05 & 10.00 ± 0.00 & 10.00 ± 0.00 & 10.00 ± 0.00 & 10.00 ± 0.00 & 10.00 ± 0.00 & 10.00 ± 0.00 & 10.00 ± 0.00 & 10.00 ± 0.00 \\
0.98  & 80.04 ± 0.90 & 10.00 ± 0.00 & 10.00 ± 0.00 & 10.00 ± 0.00 & 10.00 ± 0.00 & 10.00 ± 0.00 & 10.00 ± 0.00 & 10.00 ± 0.00 & 10.00 ± 0.00 \\
0.99  & 76.89 ± 0.26 & 10.00 ± 0.00 & 10.00 ± 0.00 & 10.00 ± 0.00 & 10.00 ± 0.00 & 10.00 ± 0.00 & 10.00 ± 0.00 & 10.00 ± 0.00 & 10.00 ± 0.00 \\
\bottomrule
\end{tabular}}
\end{sc}
\end{small}
\end{center}
\vskip -0.1in
\end{table}
\newpage
%------------------------------------------------------------------------------------------------
\begin{table*}[h]
\caption{Performance of different compression methods evaluated after 1 warmup epoch using VGG19 on the CIFAR-10 dataset. We report the mean accuracy between three initialization seeds across various sparsity levels. Baseline, no pruning: $89.21 \pm 0.22$.}
\label{tab:VGG19_cifar10_compressors_warmup1}
\vskip 0.15in
\begin{center}
\begin{small}
\begin{sc}
\resizebox{\textwidth}{!}{%
\begin{tabular}{lccccc|cccc}
\toprule
Sparsity  & Random & Magnitude & GN & SNIP & GraSP & FD & FP & FTS & FBSS \\
\midrule
0.80  & 88.73 ± 0.38 & 88.35 ± 0.54 & 86.76 ± 0.27 & 87.39 ± 0.66 & 87.24 ± 0.25 & 87.14 ± 0.45 & 87.00 ± 0.87 & 87.68 ± 0.33 & 64.33 ± 15.91 \\
0.90  & 87.26 ± 0.42 & 88.62 ± 0.49 & 85.96 ± 0.75 & 86.75 ± 0.76 & 87.47 ± 0.33 & 86.69 ± 0.72 & 87.09 ± 0.31 & 87.42 ± 0.21 & 46.16 ± 7.62 \\
0.95  & 85.47 ± 0.64 & 87.68 ± 0.49 & 86.66 ± 0.27 & 86.00 ± 1.10 & 86.71 ± 1.24 & 85.71 ± 1.35 & 86.73 ± 0.36 & 87.56 ± 0.62 & 46.30 ± 5.32 \\
0.98  & 80.44 ± 0.30 & 86.61 ± 0.62 & 84.72 ± 1.69 & 87.22 ± 0.23 & 86.45 ± 0.64 & 80.34 ± 6.43 & 86.07 ± 0.39 & 86.36 ± 0.29 & 49.05 ± 4.31 \\
0.99  & 77.24 ± 0.73 & 83.69 ± 1.36 & 80.28 ± 2.04 & 83.49 ± 1.77 & 85.39 ± 0.43 & 75.11 ± 7.80 & 84.40 ± 1.27 & 85.35 ± 1.05 & 47.10 ± 4.41 \\
\bottomrule
\end{tabular}}
\end{sc}
\end{small}
\end{center}
\vskip -0.1in
\end{table*} 
%------------------------------------------------------------------------------------------------

\begin{table}[h]
\caption{Performance of different sensitivity methods for pruning evaluated after 5 warmup epochs using VGG19 on the CIFAR-10 testset. The right side of the table presents our proposed criteria. The mean accuracy and standard deviation are reported across three initialization seeds for various sparsity levels. Baseline, no pruning: $89.21 \pm 0.22$.}
\label{tab:VGG19_cifar10_compressors_warmup5}
\vskip 0.15in
\begin{center}
\begin{small}
\begin{sc}
\resizebox{\textwidth}{!}{%
\begin{tabular}{lccccc|cccc}
\toprule
Sparsity  & Random & Magnitude & GN & SNIP & GraSP & FD & FP & FTS & FBSS \\
\midrule
0.80  & 88.84 ± 0.43 & 88.41 ± 0.47 & 87.58 ± 0.52 & 88.15 ± 1.09 & 86.77 ± 1.14 & 87.28 ± 0.90 & 88.22 ± 0.82 & 86.68 ± 0.61 & 70.52 ± 9.25 \\
0.90  & 87.56 ± 0.62 & 88.60 ± 0.93 & 86.73 ± 0.37 & 87.89 ± 0.25 & 87.10 ± 0.47 & 87.50 ± 1.42 & 88.18 ± 0.47 & 86.98 ± 0.14 & 47.78 ± 1.26 \\
0.95 & 85.51 ± 0.69 & 87.66 ± 1.19 & 87.44 ± 0.46 & 87.71 ± 0.82 & 87.05 ± 0.16 & 86.83 ± 1.47 & 87.36 ± 0.52 & 87.00 ± 0.74 & 48.83 ± 2.52 \\
0.98 & 82.09 ± 0.17 & 86.24 ± 0.52 & 84.66 ± 1.33 & 86.55 ± 0.84 & 86.04 ± 0.66 & 85.44 ± 0.64 & 86.64 ± 0.13 & 84.89 ± 0.51 & 49.48 ± 0.85 \\
0.99 & 77.22 ± 1.03 & 83.93 ± 1.80 & 81.62 ± 2.17 & 84.53 ± 0.70 & 81.33 ± 5.77 & 81.71 ± 1.41 & 85.02 ± 0.69 & 83.78 ± 0.80 & 41.24 ± 1.55 \\
\bottomrule
\end{tabular}}
\end{sc}
\end{small}
\end{center}
\vskip -0.1in
\end{table}

%------------------------------------------------------------------------------------------------

\newpage
\section{Results CIFAR100}
\subsection{ResNet18}
\label{sec:resnet_cifar-100}

CIFAR-100 results exhibit a similar trend to those observed on CIFAR-10, further reinforcing the robustness of our proposed Fisher-Taylor Sensitivity (FTS) criterion. Across all evaluated sparsity levels, FTS consistently maintains strong performance, frequently ranking among the top-performing methods. This trend is particularly evident at extreme sparsities, where many pruning approaches suffer significant performance degradation. The stability of FTS across both datasets highlights its effectiveness in preserving network expressivity despite aggressive pruning.

\begin{table}[h]
\caption{Performance of different compression methods evaluated using ResNet18 on the CIFAR-100 dataset. We report the mean accuracy between three initialization seeds across various sparsity levels. Baseline, no pruning: $69.57 \pm 0.19$.}
\label{tab:resnet18_cifar100_compressors}
\vskip 0.15in
\begin{center}
\begin{small}
\begin{sc}
\resizebox{\textwidth}{!}{%
\begin{tabular}{lccccc|cccc}
\toprule
Sparsity  & Random & Magnitude & GN & SNIP & GraSP & FD & FP & FTS & FBSS \\
\midrule
0.10  & 69.16 ± 0.11 & 69.37 ± 0.14 & 69.63 ± 0.34 & 69.42 ± 0.07 & 64.26 ± 0.27 & 69.66 ± 0.30 & 69.08 ± 0.21 & 69.16 ± 0.11 & 69.07 ± 0.10 \\
0.20  & 69.16 ± 0.30 & 69.06 ± 0.24 & 69.19 ± 0.11 & 69.30 ± 0.08 & 63.28 ± 0.58 & 69.60 ± 0.30 & 69.35 ± 0.35 & 69.41 ± 0.43 & 69.07 ± 0.20 \\
0.30  & 69.36 ± 0.18 & 68.58 ± 0.36 & 69.37 ± 0.13 & 68.82 ± 0.17 & 62.02 ± 0.43 & 69.24 ± 0.40 & 68.84 ± 0.13 & 68.80 ± 0.55 & 68.96 ± 0.11 \\
0.40  & 69.41 ± 0.20 & 68.50 ± 0.29 & 69.16 ± 0.26 & 68.95 ± 0.19 & 61.18 ± 0.19 & 69.17 ± 0.16 & 68.88 ± 0.25 & 69.02 ± 0.21 & 68.92 ± 0.25 \\
0.50  & 69.12 ± 0.46 & 68.17 ± 0.20 & 68.94 ± 0.20 & 68.63 ± 0.11 & 61.11 ± 0.40 & 69.13 ± 0.13 & 68.68 ± 0.12 & 68.71 ± 0.12 & 68.71 ± 0.57 \\
0.60  & 68.66 ± 0.27 & 67.78 ± 0.35 & 68.77 ± 0.17 & 68.63 ± 0.42 & 61.40 ± 0.78 & 68.34 ± 0.43 & 67.98 ± 0.23 & 68.41 ± 0.14 & 68.60 ± 0.15 \\
0.70  & 67.95 ± 0.43 & 67.51 ± 0.24 & 68.29 ± 0.39 & 68.08 ± 0.18 & 59.43 ± 0.76 & 68.03 ± 0.46 & 67.96 ± 0.15 & 68.29 ± 0.06 & 68.16 ± 0.07 \\
0.80  & 67.26 ± 0.48 & 66.55 ± 0.19 & 67.20 ± 0.37 & 67.21 ± 0.38 & 59.08 ± 0.22 & 66.70 ± 0.05 & 67.05 ± 0.06 & 66.77 ± 0.65 & 66.62 ± 0.43 \\
0.90  & 64.75 ± 0.16 & 64.48 ± 0.18 & 64.87 ± 0.27 & 65.70 ± 0.08 & 59.16 ± 0.91 & 64.74 ± 0.44 & 65.46 ± 0.30 & 65.41 ± 0.13 & 63.90 ± 0.31 \\
0.95  & 61.01 ± 0.32 & 62.20 ± 0.06 & 62.20 ± 0.23 & 63.20 ± 0.20 & 57.91 ± 0.09 & 62.14 ± 0.42 & 63.22 ± 0.25 & 63.21 ± 0.47 & 61.25 ± 0.44 \\
0.98  & 54.72 ± 0.22 & 55.44 ± 0.18 & 57.34 ± 0.31 & 58.83 ± 0.35 & 54.85 ± 0.35 & 55.57 ± 0.17 & 58.05 ± 0.18 & 58.59 ± 0.12 & 55.02 ± 0.34 \\
0.99  & 45.62 ± 0.55 & 40.39 ± 0.36 & 50.46 ± 0.61 & 52.96 ± 0.10 & 49.13 ± 0.19 & 48.02 ± 0.32 & 49.98 ± 0.60 & 52.85 ± 0.24 & 44.91 ± 0.52 \\
\bottomrule
\end{tabular}}
\end{sc}
\end{small}
\end{center}
\vskip -0.1in
\end{table}

%------------------------------------------------------------------------------------------------
\clearpage
\subsection{VGG19}
The results on VGG19 with CIFAR-100 exhibit a similar trend to those observed on CIFAR-10, reinforcing the effectiveness of our proposed approach. Once again, we identify the occurrence of layer collapse at extreme sparsities when no warm-up is applied, leading to a significant drop in accuracy. Introducing a single warm-up epoch effectively resolves this issue, restoring pruning performance across all evaluated criteria. However, increasing the warm-up phase to five epochs does not yield any additional advantage, indicating that a brief warm-up period is sufficient to stabilize gradient-based importance scores and prevent collapse.

\label{sec:vgg_cifar-100}

\begin{table}[h]
\caption{Performance of different compression methods evaluated using VGG19 on the CIFAR-100 dataset. We report the mean accuracy between three initialization seeds across various sparsity levels. Baseline, no pruning: $58.96 \pm 2.30$.}
\label{tab:VGG19_cifar100_compressors}
\vskip 0.15in
\begin{center}
\begin{small}
\begin{sc}
\resizebox{\textwidth}{!}{%
\begin{tabular}{lccccc|cccc}
\toprule
Sparsity & Random & Magnitude & GN & SNIP & GraSP & FD & FP & FTS & FBSS \\
\midrule
0.10  & 60.31 ± 0.40 & 59.13 ± 1.29 & 61.93 ± 0.48 & 61.98 ± 0.29 & 59.32 ± 0.63 & 62.13 ± 0.61 & 60.45 ± 3.47 & 61.56 ± 1.04 & 58.79 ± 0.98 \\
0.20  & 60.43 ± 1.14 & 59.27 ± 0.34 & 62.64 ± 0.21 & 62.68 ± 0.24 & 61.21 ± 0.41 & 63.04 ± 0.43 & 62.71 ± 1.02 & 62.24 ± 0.44 & 60.48 ± 0.48 \\
0.30  & 58.32 ± 0.60 & 59.35 ± 1.43 & 62.61 ± 0.23 & 63.11 ± 0.35 & 59.30 ± 0.43 & 62.85 ± 0.42 & 61.43 ± 0.61 & 62.65 ± 0.54 & 58.77 ± 1.02 \\
0.40  & 56.50 ± 3.20 & 60.04 ± 1.02 & 62.36 ± 0.02 & 62.39 ± 0.55 & 56.34 ± 1.49 & 62.38 ± 0.75 & 61.56 ± 1.25 & 62.67 ± 0.06 & 1.00 ± 0.00 \\
0.50  & 58.47 ± 1.49 & 61.49 ± 1.22 & 62.02 ± 0.64 & 62.76 ± 0.50 & 54.43 ± 0.84 & 62.84 ± 0.33 & 62.25 ± 0.33 & 62.47 ± 0.42 & 1.00 ± 0.00 \\
0.60  & 57.54 ± 0.74 & 61.50 ± 0.30 & 62.55 ± 0.13 & 63.08 ± 0.55 & 56.76 ± 0.69 & 62.40 ± 0.57 & 62.70 ± 0.63 & 62.17 ± 0.23 & 1.00 ± 0.00 \\
0.70  & 57.63 ± 0.80 & 61.71 ± 0.25 & 60.85 ± 0.79 & 60.58 ± 0.39 & 57.76 ± 0.84 & 60.44 ± 0.34 & 60.92 ± 0.41 & 60.51 ± 1.67 & 1.00 ± 0.00 \\
0.80  & 57.84 ± 0.57 & 61.89 ± 1.02 & 55.09 ± 0.49 & 59.84 ± 0.29 & 58.39 ± 0.74 & 1.00 ± 0.00 & 43.16 ± 1.02 & 58.66 ± 2.28 & 1.00 ± 0.00 \\
0.90  & 58.41 ± 0.41 & 62.60 ± 0.91 & 1.00 ± 0.00 & 8.35 ± 10.39 & 42.88 ± 1.64 & 1.00 ± 0.00 & 1.00 ± 0.00 & 8.87 ± 11.13 & 1.00 ± 0.00 \\
0.95  & 54.84 ± 1.08 & 1.00 ± 0.00 & 1.00 ± 0.00 & 1.00 ± 0.00 & 1.00 ± 0.00 & 1.00 ± 0.00 & 1.00 ± 0.00 & 1.00 ± 0.00 & 1.00 ± 0.00 \\
0.98  & 50.21 ± 0.72 & 1.00 ± 0.00 & 1.00 ± 0.00 & 1.00 ± 0.00 & 1.00 ± 0.00 & 1.00 ± 0.00 & 1.00 ± 0.00 & 1.00 ± 0.00 & 1.00 ± 0.00 \\
0.99  & 46.69 ± 0.45 & 1.00 ± 0.00 & 1.00 ± 0.00 & 1.00 ± 0.00 & 1.00 ± 0.00 & 1.00 ± 0.00 & 1.00 ± 0.00 & 1.00 ± 0.00 & 1.00 ± 0.00 \\
\bottomrule
\end{tabular}}
\end{sc}
\end{small}
\end{center}
\vskip -0.1in
\end{table}

%------------------------------------------------------------------------------------------------

\begin{table}[h]
\caption{Performance of different compression methods evaluated after 1 warmup epoch using VGG19 on the CIFAR-100 dataset. We report the mean accuracy between three initialization seeds across various sparsity levels. Baseline, no pruning: $58.96 \pm 2.30$.}
\label{tab:VGG19_cifar100_compressors_warmup1}
\vskip 0.15in
\begin{center}
\begin{small}
\begin{sc}
\resizebox{\textwidth}{!}{%
\begin{tabular}{lccccc|cccc}
\toprule
Sparsity & Random & Magnitude & GN & SNIP & GraSP & FD & FP & FTS & FBSS \\
\midrule
0.80  & 60.39 ± 1.16 & 58.91 ± 0.41 & 52.81 ± 1.32 & 55.62 ± 2.27 & 55.15 ± 2.25 & 56.71 ± 0.31 & 58.03 ± 0.93 & 52.41 ± 3.07 & 52.74 ± 5.16 \\
0.90  & 58.90 ± 0.98 & 60.95 ± 0.81 & 50.56 ± 4.59 & 55.89 ± 2.05 & 56.01 ± 1.58 & 52.07 ± 3.24 & 53.65 ± 0.57 & 52.45 ± 3.75 & 19.65 ± 1.68 \\
0.95  & 56.10 ± 0.85 & 57.64 ± 2.63 & 50.34 ± 1.00 & 53.70 ± 3.60 & 56.16 ± 0.41 & 54.44 ± 1.38 & 53.24 ± 3.54 & 53.56 ± 1.26 & 17.24 ± 0.44 \\
0.98  & 50.97 ± 0.40 & 54.66 ± 2.56 & 43.43 ± 5.32 & 50.19 ± 1.59 & 54.64 ± 1.50 & 42.75 ± 1.91 & 50.59 ± 3.39 & 48.56 ± 5.25 & 16.42 ± 0.64 \\
0.99  & 46.52 ± 0.45 & 43.33 ± 5.83 & 33.90 ± 5.35 & 42.65 ± 5.32 & 45.98 ± 4.48 & 29.67 ± 8.49 & 49.11 ± 3.46 & 48.70 ± 2.59 & 13.25 ± 0.84 \\
\bottomrule
\end{tabular}}
\end{sc}
\end{small}
\end{center}
\vskip -0.1in
\end{table}


%------------------------------------------------------------------------------------------------

\begin{table}[h]
\caption{Performance of different compression methods evaluated after 5 warmup epochs using VGG19 on the CIFAR-100 dataset. We report the mean accuracy between three initialization seeds across various sparsity levels. Baseline, no pruning: $58.96 \pm 2.30$.}
\label{tab:VGG19_cifar100_compressors_warmup5}
\vskip 0.15in
\begin{center}
\begin{small}
\begin{sc}
\resizebox{\textwidth}{!}{%
\begin{tabular}{lccccc|cccc}
\toprule
Sparsity & Random & Magnitude & GN & SNIP & GraSP & FD & FP & FTS & FBSS \\
\midrule
0.80  & 60.41 ± 1.39 & 58.38 ± 0.85 & 60.86 ± 0.79 & 61.63 ± 0.45 & 56.25 ± 0.49 & 59.59 ± 0.76 & 59.37 ± 3.50 & 60.86 ± 0.53 & 46.93 ± 9.04 \\
0.90  & 60.32 ± 0.09 & 57.74 ± 1.64 & 57.77 ± 2.41 & 58.23 ± 4.07 & 56.27 ± 1.02 & 60.19 ± 0.63 & 61.23 ± 0.50 & 60.52 ± 0.37 & 21.66 ± 1.95 \\
0.95 & 57.86 ± 0.53 & 59.55 ± 1.15 & 56.09 ± 0.97 & 58.83 ± 0.65 & 55.26 ± 1.25 & 55.80 ± 2.77 & 59.83 ± 0.94 & 58.52 ± 1.32 & 19.98 ± 2.62 \\
0.98 & 51.75 ± 0.43 & 47.75 ± 7.63 & 52.26 ± 4.06 & 55.27 ± 1.69 & 54.59 ± 0.96 & 49.46 ± 4.98 & 57.40 ± 1.26 & 56.00 ± 1.08 & 17.59 ± 1.36 \\
0.99 & 47.59 ± 0.80 & 42.46 ± 7.95 & 46.58 ± 2.00 & 53.13 ± 0.84 & 53.91 ± 1.53 & 42.87 ± 4.63 & 53.17 ± 1.18 & 53.05 ± 2.14 & 13.92 ± 0.14 \\
\bottomrule
\end{tabular}}
\end{sc}
\end{small}
\end{center}
\vskip -0.1in
\end{table}


%------------------------------------------------------------------------------------------------
\clearpage

\section{Mask Batch Size for Other Sparsities}
The Effect of batch size on pruning performance across different sparsities. 
As sparsity increases, the effect of batch size on pruning performance becomes more pronounced. 
At lower sparsities (0.90, 0.95), the differences across batch sizes are less evident, suggesting that even smaller batches provide a reasonable estimation of parameter importance. However, at extreme sparsities (0.98, 0.99), we observe a clear trend where larger batch sizes consistently lead to better parameter selection, ultimately improving accuracy. This aligns with our hypothesis that larger batches help reduce variance in gradient estimation, leading to more stable and effective pruning decisions. 
\label{batch_size_heatmaps}

\begin{figure}[h]
    \centering
    \includegraphics[width=0.8\linewidth]{imgs/cifar10_resnet18_heatmap_warmup_0.png}
    \caption{Effect of batch size on pruning performance at increasing sparsities.}
    \label{fig:enter-label}
\end{figure}

%------------------------------------------------------------------------------------------------

\clearpage
\section{Comparison of our criteria with magnitude-based pruning}

Figure \ref{fig:our_criterion_vs_magnitude} illustrates the relationship between parameter magnitude and different sensitivity-based pruning metrics. Each point represents a model parameter, with red points indicating the top-ranked parameters selected for retention by each criterion. The green dashed line marks the 99th percentile of parameter magnitudes.

A key observation is that the most effective pruning criteria, such as Fisher-Taylor Sensitivity, tend to retain parameters with a broad range of magnitudes, including many that are relatively small (left of the green line). This shows that the estimated importance does not always prioritize parameters based on their magnitude. 


\begin{figure}[htp]
    \centering
    \includegraphics[width=0.9\linewidth]{imgs/cifar_10_mag_vs_criteria_s_99.png}
    \caption{Our criteria vs. Magnitude parameter selection for 99\% sparsity (ResNet18, CIFAR-10, Seed 0)} 
    \label{fig:our_criterion_vs_magnitude}
\end{figure}


\end{document}
