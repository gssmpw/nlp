\section{Related Works}
\subsection{Information Entropy}

Information entropy____ states that any information has redundancy, and the size of the redundancy is related to the probability or uncertainty of each symbol (number, letter or word) in the information. 

\paragraph{Entropy}
Entropy measures the unpredictability of a token given its context:
\[
H(s|C) = - \sum_{i} P(s_i|C) \log P(s_i|C),
\]
where \( P(s_i|C) \) is the probability of token \( s_i \) given the context \( C \). Higher entropy reflects greater diversity.

\subsection{Uniform Information Density Hypothesis}

The UID hypothesis posits that speakers optimize the communicative properties of their utterances by avoiding spikes in information, thereby maintaining a relatively uniform information profile over time. An example is ____ found that speakers are more likely to include optional elements, such as "that" in English subordinate clauses, when local information density increases. 

\paragraph{Surprisal}
Surprisal quantifies the unexpectedness of a token based on its likelihood:
\[
\text{Surprisal}(s|C) = -\log P(s|C).
\]
Tokens with lower surprisal values align better with the UID principle by maintaining smoother information density.

\subsection{Combining Entropy and UID in Language Generation}

While entropy and UID have been independently studied in various NLP tasks, their integration as complementary principles for generation optimization is a novel direction. Existing work has primarily focused on either enhancing diversity through entropy-based methods or achieving uniformity via UID. Our approach bridges this gap by leveraging both principles to achieve fluency, coherence, and balanced information density in generated outputs