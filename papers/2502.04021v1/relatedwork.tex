\section{Related Work}
\label{sec:related_work}
\paragraph*{Structured and Continuous Bandits}
Best arm identification for general bandit problems with a finite set of arms has been explored in both worst-case scenarios~\cite{10.1007/978-3-642-04414-4_7} and instance-specific settings~\cite{pmlr-v49-garivier16a, NEURIPS2019_8d1de745}. Subsequent research has extended this work to provide instance-specific bounds and algorithms for bandits incorporating contextual information, such as those with Linear or Lipschitz-continuous reward functions~\cite{wang2021fast}, as well as bandits subject to constraints~\cite{pmlr-v238-carlsson24a}. Continuous bandits have primarily been studied in the context of regret minimization~\cite{bubeck2011x, KleinbergSU19}, with further refinement in Adaptive-treed bandits~\cite{bull2015adaptive}, which, while focusing on cumulative regret, also offers PAC bounds. In the context of Quantum computing, regret minimization has been studied outside of VQA, where the task is to select an optimal observable from a finite or continuous set~\cite{Lumbreras2022multiarmedquantum, lumbreras2024learningpurequantumstates}, with regret defined as the expected measurement outcome, though the bounds in this setting are not instance-specific. Methods for pure exploration in continuous bandit settings, such as \emph{MFDOO}~\cite{demontbrun2024certified}, have been developed; however, they do not incorporate structural properties of the problem, such as the Lipschitz continuity of the reward function $\mu$, which implies that nearby arms tend to yield similar rewards. Additionally, these methods rely on worst-case analysis and lack instance-dependent performance guarantees, with no known bounds for continuous bandits with Lipschitz reward functions. Our work is the first to address this gap.
\vspace{-3mm}
\paragraph*{Mitigating BP in VQAs}
Recent research indicates a trade-off between the expressivity and trainability~\cite{Holmes_2022} of specific quantum circuit architectures, often referred to as \emph{ansatz}. If an ansatz lacks sufficient expressivity, it may be incapable of representing the target function. Conversely, provably expressive ansatzes are typically susceptible to the BP phenomenon, which complicates the task of identifying the desired model within the represented model class. 
\citet{La24} provides a summary of current techniques for mitigating BPs, a subset of which we briefly outline. One way to mitigate this issue is to find the best trade-off via adaptive structure search~\cite{Du2022} and \emph{ADAPT-VQE}~\cite{Grimsley2023}. Furthermore, most proofs of presence of BPs only apply to random parameter initialization. Therefore, employing alternative initialization strategies~\cite{NEURIPS2022_7611a3cb} can serve as an effective approach to mitigating this issue. Finally, certain architectures, such as noise-induced shallow circuits~\cite{mele2024noiseinducedshallowcircuitsabsence}, are proven not to have BPs. However, recent work suggests that the absence of BPs implies classical simulability~\cite{cerezo2024doesprovableabsencebarren}. Even in the absence of BPs, the challenge of avoiding local minima remains. Alternative training methods for VQAs, such as those proposed in~\cite{PhysRevX.7.021027} and~\cite{PhysRevA.107.032407}, generally lack formal theoretical guarantees. In this work, we introduce the application of bandit methods to this domain for the first time.