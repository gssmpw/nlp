%% bare_adv.tex
%% V1.4b
%% 2015/08/26
%% by Michael Shell
%% See: 
%% http://www.michaelshell.org/
%% for current contact information.
%%
%% This is a skeleton file demonstrating the advanced use of IEEEtran.cls
%% (requires IEEEtran.cls version 1.8b or later) with an IEEE Computer
%% Society journal paper.
%%
%% Support sites:
%% http://www.michaelshell.org/tex/ieeetran/
%% http://www.ctan.org/pkg/ieeetran
%% and
%% http://www.ieee.org/

%%*************************************************************************
%% Legal Notice:
%% This code is offered as-is without any warranty either expressed or
%% implied; without even the implied warranty of MERCHANTABILITY or
%% FITNESS FOR A PARTICULAR PURPOSE! 
%% User assumes all risk.
%% In no event shall the IEEE or any contributor to this code be liable for
%% any damages or losses, including, but not limited to, incidental,
%% consequential, or any other damages, resulting from the use or misuse
%% of any information contained here.
%%
%% All comments are the opinions of their respective authors and are not
%% necessarily endorsed by the IEEE.
%%
%% This work is distributed under the LaTeX Project Public License (LPPL)
%% ( http://www.latex-project.org/ ) version 1.3, and may be freely used,
%% distributed and modified. A copy of the LPPL, version 1.3, is included
%% in the base LaTeX documentation of all distributions of LaTeX released
%% 2003/12/01 or later.
%% Retain all contribution notices and credits.
%% ** Modified files should be clearly indicated as such, including  **
%% ** renaming them and changing author support contact information. **
%%*************************************************************************


% *** Authors should verify (and, if needed, correct) their LaTeX system  ***
% *** with the testflow diagnostic prior to trusting their LaTeX platform ***
% *** with production work. The IEEE's font choices and paper sizes can   ***
% *** trigger bugs that do not appear when using other class files.       ***                          ***
% The testflow support page is at:
% http://www.michaelshell.org/tex/testflow/


% IEEEtran V1.7 and later provides for these CLASSINPUT macros to allow the
% user to reprogram some IEEEtran.cls defaults if needed. These settings
% override the internal defaults of IEEEtran.cls regardless of which class
% options are used. Do not use these unless you have good reason to do so as
% they can result in nonIEEE compliant documents. User beware. ;)
%
%\newcommand{\CLASSINPUTbaselinestretch}{1.0} % baselinestretch
%\newcommand{\CLASSINPUTinnersidemargin}{1in} % inner side margin
%\newcommand{\CLASSINPUToutersidemargin}{1in} % outer side margin
%\newcommand{\CLASSINPUTtoptextmargin}{1in}   % top text margin
%\newcommand{\CLASSINPUTbottomtextmargin}{1in}% bottom text margin




%
\documentclass[10pt,journal,compsoc]{IEEEtran}
% If IEEEtran.cls has not been installed into the LaTeX system files,
% manually specify the path to it like:
% \documentclass[10pt,journal,compsoc]{../sty/IEEEtran}
% \usepackage{url}
\usepackage{times}
\usepackage{epsfig}
\usepackage{enumitem}
\usepackage{graphicx}
% \usepackage{geometry}
\usepackage{amsmath}
\usepackage{accents}
\usepackage{amssymb}
\usepackage{multirow}
\usepackage{colortbl}
\usepackage{color}
\usepackage{threeparttable}
\usepackage{pifont}
\usepackage{booktabs}
\usepackage{adjustbox}
% \usepackage{subfig}
\usepackage{caption}
\usepackage[misc]{ifsym}
\usepackage{xcolor,colortbl}
\usepackage{makecell}
\usepackage{tikz}
\usepackage[edges]{forest}
\usepackage{longtable}
\usepackage{mathrsfs}
\usepackage{xltabular}
% \usepackage{hyperref}
\usepackage[hyphens]{url}
% \usepackage{multicol}
% \usepackage{lipsum} % for generating dummy text
% \usepackage{arydshln}
%\usepackage{tabu}
% \usepackage{longtable}
% \usepackage{geometry}
% \geometry{a4paper,left=2cm,right=2cm,top=2.5cm,bottom=2.5cm}
\definecolor{Gray}{gray}{0.92}
\definecolor{gaoyifeng-pink}{RGB}{255,235,245}
\usepackage{arydshln}
\usepackage{colortbl}
\usepackage{hhline}
\usepackage{makecell}
\usepackage{verbatim}
\usepackage{titlesec}
\usepackage{graphicx}   
% \usepackage{subcaption}
\usepackage{subfigure}
\usepackage{float}
\usepackage{xcolor,colortbl}
\usepackage[T1]{fontenc}
%%%%% NEW MATH DEFINITIONS %%%%%

\usepackage{amsmath,amsfonts,bm}
\usepackage{derivative}
% Mark sections of captions for referring to divisions of figures
\newcommand{\figleft}{{\em (Left)}}
\newcommand{\figcenter}{{\em (Center)}}
\newcommand{\figright}{{\em (Right)}}
\newcommand{\figtop}{{\em (Top)}}
\newcommand{\figbottom}{{\em (Bottom)}}
\newcommand{\captiona}{{\em (a)}}
\newcommand{\captionb}{{\em (b)}}
\newcommand{\captionc}{{\em (c)}}
\newcommand{\captiond}{{\em (d)}}

% Highlight a newly defined term
\newcommand{\newterm}[1]{{\bf #1}}

% Derivative d 
\newcommand{\deriv}{{\mathrm{d}}}

% Figure reference, lower-case.
\def\figref#1{figure~\ref{#1}}
% Figure reference, capital. For start of sentence
\def\Figref#1{Figure~\ref{#1}}
\def\twofigref#1#2{figures \ref{#1} and \ref{#2}}
\def\quadfigref#1#2#3#4{figures \ref{#1}, \ref{#2}, \ref{#3} and \ref{#4}}
% Section reference, lower-case.
\def\secref#1{section~\ref{#1}}
% Section reference, capital.
\def\Secref#1{Section~\ref{#1}}
% Reference to two sections.
\def\twosecrefs#1#2{sections \ref{#1} and \ref{#2}}
% Reference to three sections.
\def\secrefs#1#2#3{sections \ref{#1}, \ref{#2} and \ref{#3}}
% Reference to an equation, lower-case.
\def\eqref#1{equation~\ref{#1}}
% Reference to an equation, upper case
\def\Eqref#1{Equation~\ref{#1}}
% A raw reference to an equation---avoid using if possible
\def\plaineqref#1{\ref{#1}}
% Reference to a chapter, lower-case.
\def\chapref#1{chapter~\ref{#1}}
% Reference to an equation, upper case.
\def\Chapref#1{Chapter~\ref{#1}}
% Reference to a range of chapters
\def\rangechapref#1#2{chapters\ref{#1}--\ref{#2}}
% Reference to an algorithm, lower-case.
\def\algref#1{algorithm~\ref{#1}}
% Reference to an algorithm, upper case.
\def\Algref#1{Algorithm~\ref{#1}}
\def\twoalgref#1#2{algorithms \ref{#1} and \ref{#2}}
\def\Twoalgref#1#2{Algorithms \ref{#1} and \ref{#2}}
% Reference to a part, lower case
\def\partref#1{part~\ref{#1}}
% Reference to a part, upper case
\def\Partref#1{Part~\ref{#1}}
\def\twopartref#1#2{parts \ref{#1} and \ref{#2}}

\def\ceil#1{\lceil #1 \rceil}
\def\floor#1{\lfloor #1 \rfloor}
\def\1{\bm{1}}
\newcommand{\train}{\mathcal{D}}
\newcommand{\valid}{\mathcal{D_{\mathrm{valid}}}}
\newcommand{\test}{\mathcal{D_{\mathrm{test}}}}

\def\eps{{\epsilon}}


% Random variables
\def\reta{{\textnormal{$\eta$}}}
\def\ra{{\textnormal{a}}}
\def\rb{{\textnormal{b}}}
\def\rc{{\textnormal{c}}}
\def\rd{{\textnormal{d}}}
\def\re{{\textnormal{e}}}
\def\rf{{\textnormal{f}}}
\def\rg{{\textnormal{g}}}
\def\rh{{\textnormal{h}}}
\def\ri{{\textnormal{i}}}
\def\rj{{\textnormal{j}}}
\def\rk{{\textnormal{k}}}
\def\rl{{\textnormal{l}}}
% rm is already a command, just don't name any random variables m
\def\rn{{\textnormal{n}}}
\def\ro{{\textnormal{o}}}
\def\rp{{\textnormal{p}}}
\def\rq{{\textnormal{q}}}
\def\rr{{\textnormal{r}}}
\def\rs{{\textnormal{s}}}
\def\rt{{\textnormal{t}}}
\def\ru{{\textnormal{u}}}
\def\rv{{\textnormal{v}}}
\def\rw{{\textnormal{w}}}
\def\rx{{\textnormal{x}}}
\def\ry{{\textnormal{y}}}
\def\rz{{\textnormal{z}}}

% Random vectors
\def\rvepsilon{{\mathbf{\epsilon}}}
\def\rvphi{{\mathbf{\phi}}}
\def\rvtheta{{\mathbf{\theta}}}
\def\rva{{\mathbf{a}}}
\def\rvb{{\mathbf{b}}}
\def\rvc{{\mathbf{c}}}
\def\rvd{{\mathbf{d}}}
\def\rve{{\mathbf{e}}}
\def\rvf{{\mathbf{f}}}
\def\rvg{{\mathbf{g}}}
\def\rvh{{\mathbf{h}}}
\def\rvu{{\mathbf{i}}}
\def\rvj{{\mathbf{j}}}
\def\rvk{{\mathbf{k}}}
\def\rvl{{\mathbf{l}}}
\def\rvm{{\mathbf{m}}}
\def\rvn{{\mathbf{n}}}
\def\rvo{{\mathbf{o}}}
\def\rvp{{\mathbf{p}}}
\def\rvq{{\mathbf{q}}}
\def\rvr{{\mathbf{r}}}
\def\rvs{{\mathbf{s}}}
\def\rvt{{\mathbf{t}}}
\def\rvu{{\mathbf{u}}}
\def\rvv{{\mathbf{v}}}
\def\rvw{{\mathbf{w}}}
\def\rvx{{\mathbf{x}}}
\def\rvy{{\mathbf{y}}}
\def\rvz{{\mathbf{z}}}

% Elements of random vectors
\def\erva{{\textnormal{a}}}
\def\ervb{{\textnormal{b}}}
\def\ervc{{\textnormal{c}}}
\def\ervd{{\textnormal{d}}}
\def\erve{{\textnormal{e}}}
\def\ervf{{\textnormal{f}}}
\def\ervg{{\textnormal{g}}}
\def\ervh{{\textnormal{h}}}
\def\ervi{{\textnormal{i}}}
\def\ervj{{\textnormal{j}}}
\def\ervk{{\textnormal{k}}}
\def\ervl{{\textnormal{l}}}
\def\ervm{{\textnormal{m}}}
\def\ervn{{\textnormal{n}}}
\def\ervo{{\textnormal{o}}}
\def\ervp{{\textnormal{p}}}
\def\ervq{{\textnormal{q}}}
\def\ervr{{\textnormal{r}}}
\def\ervs{{\textnormal{s}}}
\def\ervt{{\textnormal{t}}}
\def\ervu{{\textnormal{u}}}
\def\ervv{{\textnormal{v}}}
\def\ervw{{\textnormal{w}}}
\def\ervx{{\textnormal{x}}}
\def\ervy{{\textnormal{y}}}
\def\ervz{{\textnormal{z}}}

% Random matrices
\def\rmA{{\mathbf{A}}}
\def\rmB{{\mathbf{B}}}
\def\rmC{{\mathbf{C}}}
\def\rmD{{\mathbf{D}}}
\def\rmE{{\mathbf{E}}}
\def\rmF{{\mathbf{F}}}
\def\rmG{{\mathbf{G}}}
\def\rmH{{\mathbf{H}}}
\def\rmI{{\mathbf{I}}}
\def\rmJ{{\mathbf{J}}}
\def\rmK{{\mathbf{K}}}
\def\rmL{{\mathbf{L}}}
\def\rmM{{\mathbf{M}}}
\def\rmN{{\mathbf{N}}}
\def\rmO{{\mathbf{O}}}
\def\rmP{{\mathbf{P}}}
\def\rmQ{{\mathbf{Q}}}
\def\rmR{{\mathbf{R}}}
\def\rmS{{\mathbf{S}}}
\def\rmT{{\mathbf{T}}}
\def\rmU{{\mathbf{U}}}
\def\rmV{{\mathbf{V}}}
\def\rmW{{\mathbf{W}}}
\def\rmX{{\mathbf{X}}}
\def\rmY{{\mathbf{Y}}}
\def\rmZ{{\mathbf{Z}}}

% Elements of random matrices
\def\ermA{{\textnormal{A}}}
\def\ermB{{\textnormal{B}}}
\def\ermC{{\textnormal{C}}}
\def\ermD{{\textnormal{D}}}
\def\ermE{{\textnormal{E}}}
\def\ermF{{\textnormal{F}}}
\def\ermG{{\textnormal{G}}}
\def\ermH{{\textnormal{H}}}
\def\ermI{{\textnormal{I}}}
\def\ermJ{{\textnormal{J}}}
\def\ermK{{\textnormal{K}}}
\def\ermL{{\textnormal{L}}}
\def\ermM{{\textnormal{M}}}
\def\ermN{{\textnormal{N}}}
\def\ermO{{\textnormal{O}}}
\def\ermP{{\textnormal{P}}}
\def\ermQ{{\textnormal{Q}}}
\def\ermR{{\textnormal{R}}}
\def\ermS{{\textnormal{S}}}
\def\ermT{{\textnormal{T}}}
\def\ermU{{\textnormal{U}}}
\def\ermV{{\textnormal{V}}}
\def\ermW{{\textnormal{W}}}
\def\ermX{{\textnormal{X}}}
\def\ermY{{\textnormal{Y}}}
\def\ermZ{{\textnormal{Z}}}

% Vectors
\def\vzero{{\bm{0}}}
\def\vone{{\bm{1}}}
\def\vmu{{\bm{\mu}}}
\def\vtheta{{\bm{\theta}}}
\def\vphi{{\bm{\phi}}}
\def\va{{\bm{a}}}
\def\vb{{\bm{b}}}
\def\vc{{\bm{c}}}
\def\vd{{\bm{d}}}
\def\ve{{\bm{e}}}
\def\vf{{\bm{f}}}
\def\vg{{\bm{g}}}
\def\vh{{\bm{h}}}
\def\vi{{\bm{i}}}
\def\vj{{\bm{j}}}
\def\vk{{\bm{k}}}
\def\vl{{\bm{l}}}
\def\vm{{\bm{m}}}
\def\vn{{\bm{n}}}
\def\vo{{\bm{o}}}
\def\vp{{\bm{p}}}
\def\vq{{\bm{q}}}
\def\vr{{\bm{r}}}
\def\vs{{\bm{s}}}
\def\vt{{\bm{t}}}
\def\vu{{\bm{u}}}
\def\vv{{\bm{v}}}
\def\vw{{\bm{w}}}
\def\vx{{\bm{x}}}
\def\vy{{\bm{y}}}
\def\vz{{\bm{z}}}

% Elements of vectors
\def\evalpha{{\alpha}}
\def\evbeta{{\beta}}
\def\evepsilon{{\epsilon}}
\def\evlambda{{\lambda}}
\def\evomega{{\omega}}
\def\evmu{{\mu}}
\def\evpsi{{\psi}}
\def\evsigma{{\sigma}}
\def\evtheta{{\theta}}
\def\eva{{a}}
\def\evb{{b}}
\def\evc{{c}}
\def\evd{{d}}
\def\eve{{e}}
\def\evf{{f}}
\def\evg{{g}}
\def\evh{{h}}
\def\evi{{i}}
\def\evj{{j}}
\def\evk{{k}}
\def\evl{{l}}
\def\evm{{m}}
\def\evn{{n}}
\def\evo{{o}}
\def\evp{{p}}
\def\evq{{q}}
\def\evr{{r}}
\def\evs{{s}}
\def\evt{{t}}
\def\evu{{u}}
\def\evv{{v}}
\def\evw{{w}}
\def\evx{{x}}
\def\evy{{y}}
\def\evz{{z}}

% Matrix
\def\mA{{\bm{A}}}
\def\mB{{\bm{B}}}
\def\mC{{\bm{C}}}
\def\mD{{\bm{D}}}
\def\mE{{\bm{E}}}
\def\mF{{\bm{F}}}
\def\mG{{\bm{G}}}
\def\mH{{\bm{H}}}
\def\mI{{\bm{I}}}
\def\mJ{{\bm{J}}}
\def\mK{{\bm{K}}}
\def\mL{{\bm{L}}}
\def\mM{{\bm{M}}}
\def\mN{{\bm{N}}}
\def\mO{{\bm{O}}}
\def\mP{{\bm{P}}}
\def\mQ{{\bm{Q}}}
\def\mR{{\bm{R}}}
\def\mS{{\bm{S}}}
\def\mT{{\bm{T}}}
\def\mU{{\bm{U}}}
\def\mV{{\bm{V}}}
\def\mW{{\bm{W}}}
\def\mX{{\bm{X}}}
\def\mY{{\bm{Y}}}
\def\mZ{{\bm{Z}}}
\def\mBeta{{\bm{\beta}}}
\def\mPhi{{\bm{\Phi}}}
\def\mLambda{{\bm{\Lambda}}}
\def\mSigma{{\bm{\Sigma}}}

% Tensor
\DeclareMathAlphabet{\mathsfit}{\encodingdefault}{\sfdefault}{m}{sl}
\SetMathAlphabet{\mathsfit}{bold}{\encodingdefault}{\sfdefault}{bx}{n}
\newcommand{\tens}[1]{\bm{\mathsfit{#1}}}
\def\tA{{\tens{A}}}
\def\tB{{\tens{B}}}
\def\tC{{\tens{C}}}
\def\tD{{\tens{D}}}
\def\tE{{\tens{E}}}
\def\tF{{\tens{F}}}
\def\tG{{\tens{G}}}
\def\tH{{\tens{H}}}
\def\tI{{\tens{I}}}
\def\tJ{{\tens{J}}}
\def\tK{{\tens{K}}}
\def\tL{{\tens{L}}}
\def\tM{{\tens{M}}}
\def\tN{{\tens{N}}}
\def\tO{{\tens{O}}}
\def\tP{{\tens{P}}}
\def\tQ{{\tens{Q}}}
\def\tR{{\tens{R}}}
\def\tS{{\tens{S}}}
\def\tT{{\tens{T}}}
\def\tU{{\tens{U}}}
\def\tV{{\tens{V}}}
\def\tW{{\tens{W}}}
\def\tX{{\tens{X}}}
\def\tY{{\tens{Y}}}
\def\tZ{{\tens{Z}}}


% Graph
\def\gA{{\mathcal{A}}}
\def\gB{{\mathcal{B}}}
\def\gC{{\mathcal{C}}}
\def\gD{{\mathcal{D}}}
\def\gE{{\mathcal{E}}}
\def\gF{{\mathcal{F}}}
\def\gG{{\mathcal{G}}}
\def\gH{{\mathcal{H}}}
\def\gI{{\mathcal{I}}}
\def\gJ{{\mathcal{J}}}
\def\gK{{\mathcal{K}}}
\def\gL{{\mathcal{L}}}
\def\gM{{\mathcal{M}}}
\def\gN{{\mathcal{N}}}
\def\gO{{\mathcal{O}}}
\def\gP{{\mathcal{P}}}
\def\gQ{{\mathcal{Q}}}
\def\gR{{\mathcal{R}}}
\def\gS{{\mathcal{S}}}
\def\gT{{\mathcal{T}}}
\def\gU{{\mathcal{U}}}
\def\gV{{\mathcal{V}}}
\def\gW{{\mathcal{W}}}
\def\gX{{\mathcal{X}}}
\def\gY{{\mathcal{Y}}}
\def\gZ{{\mathcal{Z}}}

% Sets
\def\sA{{\mathbb{A}}}
\def\sB{{\mathbb{B}}}
\def\sC{{\mathbb{C}}}
\def\sD{{\mathbb{D}}}
% Don't use a set called E, because this would be the same as our symbol
% for expectation.
\def\sF{{\mathbb{F}}}
\def\sG{{\mathbb{G}}}
\def\sH{{\mathbb{H}}}
\def\sI{{\mathbb{I}}}
\def\sJ{{\mathbb{J}}}
\def\sK{{\mathbb{K}}}
\def\sL{{\mathbb{L}}}
\def\sM{{\mathbb{M}}}
\def\sN{{\mathbb{N}}}
\def\sO{{\mathbb{O}}}
\def\sP{{\mathbb{P}}}
\def\sQ{{\mathbb{Q}}}
\def\sR{{\mathbb{R}}}
\def\sS{{\mathbb{S}}}
\def\sT{{\mathbb{T}}}
\def\sU{{\mathbb{U}}}
\def\sV{{\mathbb{V}}}
\def\sW{{\mathbb{W}}}
\def\sX{{\mathbb{X}}}
\def\sY{{\mathbb{Y}}}
\def\sZ{{\mathbb{Z}}}

% Entries of a matrix
\def\emLambda{{\Lambda}}
\def\emA{{A}}
\def\emB{{B}}
\def\emC{{C}}
\def\emD{{D}}
\def\emE{{E}}
\def\emF{{F}}
\def\emG{{G}}
\def\emH{{H}}
\def\emI{{I}}
\def\emJ{{J}}
\def\emK{{K}}
\def\emL{{L}}
\def\emM{{M}}
\def\emN{{N}}
\def\emO{{O}}
\def\emP{{P}}
\def\emQ{{Q}}
\def\emR{{R}}
\def\emS{{S}}
\def\emT{{T}}
\def\emU{{U}}
\def\emV{{V}}
\def\emW{{W}}
\def\emX{{X}}
\def\emY{{Y}}
\def\emZ{{Z}}
\def\emSigma{{\Sigma}}

% entries of a tensor
% Same font as tensor, without \bm wrapper
\newcommand{\etens}[1]{\mathsfit{#1}}
\def\etLambda{{\etens{\Lambda}}}
\def\etA{{\etens{A}}}
\def\etB{{\etens{B}}}
\def\etC{{\etens{C}}}
\def\etD{{\etens{D}}}
\def\etE{{\etens{E}}}
\def\etF{{\etens{F}}}
\def\etG{{\etens{G}}}
\def\etH{{\etens{H}}}
\def\etI{{\etens{I}}}
\def\etJ{{\etens{J}}}
\def\etK{{\etens{K}}}
\def\etL{{\etens{L}}}
\def\etM{{\etens{M}}}
\def\etN{{\etens{N}}}
\def\etO{{\etens{O}}}
\def\etP{{\etens{P}}}
\def\etQ{{\etens{Q}}}
\def\etR{{\etens{R}}}
\def\etS{{\etens{S}}}
\def\etT{{\etens{T}}}
\def\etU{{\etens{U}}}
\def\etV{{\etens{V}}}
\def\etW{{\etens{W}}}
\def\etX{{\etens{X}}}
\def\etY{{\etens{Y}}}
\def\etZ{{\etens{Z}}}

% The true underlying data generating distribution
\newcommand{\pdata}{p_{\rm{data}}}
\newcommand{\ptarget}{p_{\rm{target}}}
\newcommand{\pprior}{p_{\rm{prior}}}
\newcommand{\pbase}{p_{\rm{base}}}
\newcommand{\pref}{p_{\rm{ref}}}

% The empirical distribution defined by the training set
\newcommand{\ptrain}{\hat{p}_{\rm{data}}}
\newcommand{\Ptrain}{\hat{P}_{\rm{data}}}
% The model distribution
\newcommand{\pmodel}{p_{\rm{model}}}
\newcommand{\Pmodel}{P_{\rm{model}}}
\newcommand{\ptildemodel}{\tilde{p}_{\rm{model}}}
% Stochastic autoencoder distributions
\newcommand{\pencode}{p_{\rm{encoder}}}
\newcommand{\pdecode}{p_{\rm{decoder}}}
\newcommand{\precons}{p_{\rm{reconstruct}}}

\newcommand{\laplace}{\mathrm{Laplace}} % Laplace distribution

\newcommand{\E}{\mathbb{E}}
\newcommand{\Ls}{\mathcal{L}}
\newcommand{\R}{\mathbb{R}}
\newcommand{\emp}{\tilde{p}}
\newcommand{\lr}{\alpha}
\newcommand{\reg}{\lambda}
\newcommand{\rect}{\mathrm{rectifier}}
\newcommand{\softmax}{\mathrm{softmax}}
\newcommand{\sigmoid}{\sigma}
\newcommand{\softplus}{\zeta}
\newcommand{\KL}{D_{\mathrm{KL}}}
\newcommand{\Var}{\mathrm{Var}}
\newcommand{\standarderror}{\mathrm{SE}}
\newcommand{\Cov}{\mathrm{Cov}}
% Wolfram Mathworld says $L^2$ is for function spaces and $\ell^2$ is for vectors
% But then they seem to use $L^2$ for vectors throughout the site, and so does
% wikipedia.
\newcommand{\normlzero}{L^0}
\newcommand{\normlone}{L^1}
\newcommand{\normltwo}{L^2}
\newcommand{\normlp}{L^p}
\newcommand{\normmax}{L^\infty}

\newcommand{\parents}{Pa} % See usage in notation.tex. Chosen to match Daphne's book.

\DeclareMathOperator*{\argmax}{arg\,max}
\DeclareMathOperator*{\argmin}{arg\,min}

\DeclareMathOperator{\sign}{sign}
\DeclareMathOperator{\Tr}{Tr}
\let\ab\allowbreak



\newcommand*\circled[1]{\tikz[baseline=(char.base)]{
            \node[shape=circle,fill=black,text=white,draw,inner sep=0.5pt] (char) {#1};}}

\newcommand*\figcircled[1]{\tikz[baseline=(char.base)]{
            \node[shape=circle,fill=black,text=white,draw,inner sep=0.8pt] (char) {#1};}}

\newcommand*{\belowrulesepcolor}[1]{% 
  \noalign{% 
    \kern-\belowrulesep 
    \begingroup 
      \color{#1}% 
      \hrule height\belowrulesep 
    \endgroup 
    % \vskip -0.15mm%
    \vspace{-0.03mm}
  }%
} 
\newcommand*{\aboverulesepcolor}[1]{% 
  \noalign{% 
  \vspace{-0.03mm}
    \begingroup 
      \color{#1}% 
    \endgroup 
    \kern-\aboverulesep 
  }%
}
% space commend 
\newcommand{\vspacefigtext}{\vspace{-3mm}}
\newcommand{\vspacesection}{\vspace{-2.2mm}}
\newcommand{\myrowcolour}{\rowcolor[gray]{0.925}}
\newcommand{\mywhitecolour}{\rowcolor[gray]{1}}

% end-to-end
\newcommand{\contentawarebig}{Content-Aware Editing\xspace}
\newcommand{\contentawareuppercase}{Content-aware editing\xspace}
\newcommand{\contentfreebig}{Content-Free Editing\xspace}
\newcommand{\contentfreeuppercase}{Content-free editing\xspace}
\newcommand{\contentawaresmall}{content-aware editing\xspace}
\newcommand{\contentawareshortsmall}{content-aware\xspace}
\newcommand{\contentfreesmall}{content-free editing\xspace}
\newcommand{\contentfreeshortsmall}{content-free\xspace}
\newcommand{\editingalgorithmsmall}{editing\xspace}
\newcommand{\editingalgorithmbig}{Editing\xspace}
\newcommand{\inversionfeatures}{inversion clue\xspace}
\newcommand{\inversionfeaturesuppercase}{Inversion clue\xspace}
\newcommand{\contentawareinjectionbig}{Content-Aware Injection}
\newcommand{\contentfreeinjectionbig}{Content-Free Injection}
\newcommand{\contentawareinjectionsmall}{content-aware injection}
\newcommand{\contentfreeinjectionsmall}{content-free injection}
\newcommand{\contentawareimageadapterbig}{Content-Aware Image Adapter}
\newcommand{\contentfreeimageadapterbig}{Content-Free Image Adapter}
\newcommand{\contentawareimageadaptersmall}{content-aware image adapter}
\newcommand{\contentfreeimageadaptersmall}{content-free image adapter}
\newcommand{\texturalspaceadapterbig}{Textual Space Adapter}
\newcommand{\imagespaceadapterbig}{Latent Space Adapter}
\newcommand{\texturalspaceadaptersmall}{textual space adapter}
\newcommand{\imagespaceadaptersmall}{image space adapter}


% For new paragraph format
% \makeatletter
% \renewcommand{\paragraph}{%
%   \@startsection{paragraph}{4}{0mm}
%   {2.5ex \@plus1ex \@minus.2ex}   
%   {0em}                           
%   {\normalfont\normalsize\bfseries}
% }
% \makeatother

% \titleformat{\paragraph}[runin] % 使用 hang 格式,使标题独占一行
%   {\normalfont\normalsize\bfseries} % 标题格式:正常字体、普通大小、加粗
%   {\theparagraph} % 标题编号
%   {1em} % 编号与标题文本之间的距离
%   {} % 标题前的代码
%   [] % 标题后的代码(这里不添加冒号)

% \titlespacing*{\paragraph}{0pt}{1.5ex plus 0.2ex minus 0.1ex}{1em}

\titleformat{\paragraph}[block] % 使用 block 格式,使标题独占一行
  {\normalfont\normalsize\bfseries} % 标题格式:正常字体、普通大小、加粗
  {\theparagraph} % 标题编号
  {1em} % 编号与标题文本之间的距离
  {} % 标题前的代码
  [] % 标题后的代码

\titlespacing*{\paragraph}{0pt}{1.5ex plus 0.2ex minus 0.1ex}{0.5em} % 调整段前和段后间距

% For math command 
\newcommand{\pp}{p}
\newcommand{\RR}{\mathbb{R}}
\newcommand{\FF}{\mathbf{F}}
\newcommand{\XX}{\mathbf{X}}
\newcommand{\YY}{\mathbf{Y}}
\newcommand{\hatd}{\hat{d}}
\newcommand{\haty}{\hat{y}}
\newcommand{\hatp}{\hat{p}}
\newcommand{\hatm}{\hat{m}}
\newcommand{\hatc}{\hat{c}}
\newcommand{\noobject}{\varnothing}
\newcommand{\denc}{d_{\rm enc}}
\newcommand{\ddec}{d_{\rm dec}}
\renewcommand{\Re}{\mathbb{R}}
\newcommand{\hy}{\hat{y}}
\newcommand{\hb}{\hat{b}}
\newcommand{\hp}{\hat{p}}
\newcommand{\ty}{\tilde{y}}



%  For color
\definecolor{figorange}{RGB}{228,130,47}
\definecolor{figred}{RGB}{255,0,0}
\definecolor{figgreen}{RGB}{0,176,80}

% TLM color
% \definecolor{sunye-blue}{RGB}{220, 239, 252}
% \definecolor{sunye-blue-dark}{RGB}{114, 154, 202}
% \definecolor{sunye-blue-light}{RGB}{235, 244, 255}

% \definecolor{dingyifan-wangyixu-green}{RGB}{231, 244, 234}
% \definecolor{dingyifan-wangyixu-green-dark}{RGB}{107, 182, 142}
% \definecolor{dingyifan-wangyixu-green-light}{RGB}{245, 251, 246}

% \definecolor{wangxin-orange}{RGB}{254, 242, 235}
% \definecolor{wangxin-orange-dark}{RGB}{229, 130, 95}
% \definecolor{wangxin-orange-light}{RGB}{255, 237, 231}

% \definecolor{wangruofan-pink}{RGB}{255,235,245}
% \definecolor{wangruofan-pink-dark}{RGB}{216,116,152}
% \definecolor{wangruofan-pink-light}{RGB}{255, 250, 253}

% \definecolor{gaoyifeng-purple}{RGB}{232, 221, 243}
% \definecolor{gaoyifeng-purple-dark}{RGB}{161,115,196}
% \definecolor{gaoyifeng-purple-light}{RGB}{244, 239, 251}

\definecolor{sunye-red}{RGB}{220, 239, 252}
\definecolor{sunye-red-dark}{RGB}{114, 154, 202}
\definecolor{sunye-red-light}{RGB}{235, 244, 255}

\definecolor{dingyifan-wangyixu-darkblue}{RGB}{231, 244, 234}
\definecolor{dingyifan-wangyixu-darkblue-dark}{RGB}{107, 182, 142}
\definecolor{dingyifan-wangyixu-darkblue-light}{RGB}{245, 251, 246}

\definecolor{wangxin-yellow}{RGB}{254, 240, 189}
\definecolor{wangxin-yellow-dark}{RGB}{238, 196, 84}
\definecolor{wangxin-yellow-light}{RGB}{255, 254, 231}

\definecolor{wangruofan-orange}{RGB}{254, 242, 235}
\definecolor{wangruofan-orange-dark}{RGB}{229, 130, 95}
\definecolor{wangruofan-orange-light}{RGB}{255, 246, 233}

\definecolor{wangyixu-purple}{RGB}{232, 221, 243}
\definecolor{wangyixu-purple-dark}{RGB}{161,115,196}
\definecolor{wangyixu-purple-light}{RGB}{244, 239, 251}

\definecolor{gaoyifeng-pink}{RGB}{255,235,245}
\definecolor{gaoyifeng-pink-dark}{RGB}{216,116,152}
\definecolor{gaoyifeng-pink-light}{RGB}{255, 250, 253}

% \newcommand{\cavan}[1]{{\color{magenta}(cavan: {#1})}} % cavan's comments

\usepackage{xspace}
\usepackage{arydshln}
% \usepackage[lined,boxed,commentsnumbered,ruled]{algorithm2e}
%\usepackage[colorlinks,linkcolor=blue,citelink=blue,anchorcolor=black]{hyperref}
\def\etal{\textit{et al.}}
% For comments 
% \newcommand{\xma}[1]{{\color{magenta}(cavan: {#1})}} % X. Ma's comments


\makeatletter
\DeclareRobustCommand\onedot{\futurelet\@let@token\@onedot}
\def\@onedot{\ifx\@let@token.\else.\null\fi\xspace}
\def\eg{\emph{e.g}\onedot} \def\Eg{\emph{E.g}\onedot}
\def\ie{\emph{i.e}\onedot} \def\Ie{\emph{I.e}\onedot}
\def\cf{\emph{c.f}\onedot} \def\Cf{\emph{C.f}\onedot}
\def\etc{\emph{etc}\onedot} \def\vs{\emph{vs}\onedot}
\def\wrt{w.r.t\onedot} \def\dof{d.o.f\onedot}
\def\etal{\emph{et al}\onedot}

\usepackage{xcolor}
% \definecolor{citecolor}{RGB}{0, 113, 188}
% \definecolor{citecolor}{RGB}{0, 20.4, 114.75}
\definecolor{citecolor}{RGB}{0, 0, 255}
% \usepackage[lined,boxed,commentsnumbered,ruled]{algorithm2e}
\usepackage[linesnumbered,lined,boxed,commentsnumbered,ruled]{algorithm2e}
\usepackage[pagebackref=false,breaklinks=true,colorlinks,citecolor=citecolor,urlcolor=blue,linkcolor=blue,bookmarks=false]{hyperref}
% For Computer Society journals, IEEEtran defaults to the use of 
% Palatino/Palladio as is done in IEEE Computer Society journals.
% To go back to Times Roman, you can use this code:
%\renewcommand{\rmdefault}{ptm}\selectfont





% Some very useful LaTeX packages include:
% (uncomment the ones you want to load)



% *** MISC UTILITY PACKAGES ***
%
%\usepackage{ifpdf}
% Heiko Oberdiek's ifpdf.sty is very useful if you need conditional
% compilation based on whether the output is pdf or dvi.
% usage:
% \ifpdf
%   % pdf code
% \else
%   % dvi code
% \fi
% The latest version of ifpdf.sty can be obtained from:
% http://www.ctan.org/pkg/ifpdf
% Also, note that IEEEtran.cls V1.7 and later provides a builtin
% \ifCLASSINFOpdf conditional that works the same way.
% When switching from latex to pdflatex and vice-versa, the compiler may
% have to be run twice to clear warning/error messages.




% *** CITATION PACKAGES ***
%
\ifCLASSOPTIONcompsoc
  % The IEEE Computer Society needs nocompress option
  % requires cite.sty v4.0 or later (November 2003)
  % \usepackage[nocompress]{cite}
  \usepackage{cite}
\else
  % normal IEEE
  \usepackage{cite}
\fi
% cite.sty was written by Donald Arseneau
% V1.6 and later of IEEEtran pre-defines the format of the cite.sty package
%  \cite{} output to follow that of the IEEE. Loading the cite package will
% result in citation numbers being automatically sorted and properly
% "compressed/ranged". e.g., [1], [9], [2], [7], [5], [6] without using
% cite.sty will become [1], [2], [5]--[7], [9] using cite.sty. cite.sty's
%  \cite will automatically add leading space, if needed. Use cite.sty's
% noadjust option (cite.sty V3.8 and later) if you want to turn this off
% such as if a citation ever needs to be enclosed in parenthesis.
% cite.sty is already installed on most LaTeX systems. Be sure and use
% version 5.0 (2009-03-20) and later if using hyperref.sty.
% The latest version can be obtained at:
% http://www.ctan.org/pkg/cite
% The documentation is contained in the cite.sty file itself.
%
% Note that some packages require special options to format as the Computer
% Society requires. In particular, Computer Society  papers do not use
% compressed citation ranges as is done in typical IEEE papers
% (e.g., [1]-[4]). Instead, they list every citation separately in order
% (e.g., [1], [2], [3], [4]). To get the latter we need to load the cite
% package with the nocompress option which is supported by cite.sty v4.0
% and later.





% *** GRAPHICS RELATED PACKAGES ***
%
\ifCLASSINFOpdf
  % \usepackage[pdftex]{graphicx}
  % declare the path(s) where your graphic files are
  % \graphicspath{{../pdf/}{../jpeg/}}
  % and their extensions so you won't have to specify these with
  % every instance of \includegraphics
  % \DeclareGraphicsExtensions{.pdf,.jpeg,.png}
\else
  % or other class option (dvipsone, dvipdf, if not using dvips). graphicx
  % will default to the driver specified in the system graphics.cfg if no
  % driver is specified.
  % \usepackage[dvips]{graphicx}
  % declare the path(s) where your graphic files are
  % \graphicspath{{../eps/}}
  % and their extensions so you won't have to specify these with
  % every instance of \includegraphics
  % \DeclareGraphicsExtensions{.eps}
\fi
% graphicx was written by David Carlisle and Sebastian Rahtz. It is
% required if you want graphics, photos, etc. graphicx.sty is already
% installed on most LaTeX systems. The latest version and documentation
% can be obtained at: 
% http://www.ctan.org/pkg/graphicx
% Another good source of documentation is "Using Imported Graphics in
% LaTeX2e" by Keith Reckdahl which can be found at:
% http://www.ctan.org/pkg/epslatex
%
% latex, and pdflatex in dvi mode, support graphics in encapsulated
% postscript (.eps) format. pdflatex in pdf mode supports graphics
% in .pdf, .jpeg, .png and .mps (metapost) formats. Users should ensure
% that all non-photo figures use a vector format (.eps, .pdf, .mps) and
% not a bitmapped formats (.jpeg, .png). The IEEE frowns on bitmapped formats
% which can result in "jaggedy"/blurry rendering of lines and letters as
% well as large increases in file sizes.
%
% You can find documentation about the pdfTeX application at:
% http://www.tug.org/applications/pdftex





% *** MATH PACKAGES ***
%
%\usepackage{amsmath}
% A popular package from the American Mathematical Society that provides
% many useful and powerful commands for dealing with mathematics.
%
% Note that the amsmath package sets \interdisplaylinepenalty to 10000
% thus preventing page breaks from occurring within multiline equations. Use:
%\interdisplaylinepenalty=2500
% after loading amsmath to restore such page breaks as IEEEtran.cls normally
% does. amsmath.sty is already installed on most LaTeX systems. The latest
% version and documentation can be obtained at:
% http://www.ctan.org/pkg/amsmath





% *** SPECIALIZED LIST PACKAGES ***
%\usepackage{acronym}
% acronym.sty was written by Tobias Oetiker. This package provides tools for
% managing documents with large numbers of acronyms. (You don't *have* to
% use this package - unless you have a lot of acronyms, you may feel that
% such package management of them is bit of an overkill.)
% Do note that the acronym environment (which lists acronyms) will have a
% problem when used under IEEEtran.cls because acronym.sty relies on the
% description list environment - which IEEEtran.cls has customized for
% producing IEEE style lists. A workaround is to declared the longest
% label width via the IEEEtran.cls \IEEEiedlistdecl global control:
%
% \renewcommand{\IEEEiedlistdecl}{\IEEEsetlabelwidth{SONET}}
% \begin{acronym}
%
% \end{acronym}
% \renewcommand{\IEEEiedlistdecl}{\relax}% remember to reset \IEEEiedlistdecl
%
% instead of using the acronym environment's optional argument.
% The latest version and documentation can be obtained at:
% http://www.ctan.org/pkg/acronym


%\usepackage{algorithmic}
% algorithmic.sty was written by Peter Williams and Rogerio Brito.
% This package provides an algorithmic environment fo describing algorithms.
% You can use the algorithmic environment in-text or within a figure
% environment to provide for a floating algorithm. Do NOT use the algorithm
% floating environment provided by algorithm.sty (by the same authors) or
% algorithm2e.sty (by Christophe Fiorio) as the IEEE does not use dedicated
% algorithm float types and packages that provide these will not provide
% correct IEEE style captions. The latest version and documentation of
% algorithmic.sty can be obtained at:
% http://www.ctan.org/pkg/algorithms
% Also of interest may be the (relatively newer and more customizable)
% algorithmicx.sty package by Szasz Janos:
% http://www.ctan.org/pkg/algorithmicx




% *** ALIGNMENT PACKAGES ***
%
%\usepackage{array}
% Frank Mittelbach's and David Carlisle's array.sty patches and improves
% the standard LaTeX2e array and tabular environments to provide better
% appearance and additional user controls. As the default LaTeX2e table
% generation code is lacking to the point of almost being broken with
% respect to the quality of the end results, all users are strongly
% advised to use an enhanced (at the very least that provided by array.sty)
% set of table tools. array.sty is already installed on most systems. The
% latest version and documentation can be obtained at:
% http://www.ctan.org/pkg/array


%\usepackage{mdwmath}
%\usepackage{mdwtab}
% Also highly recommended is Mark Wooding's extremely powerful MDW tools,
% especially mdwmath.sty and mdwtab.sty which are used to format equations
% and tables, respectively. The MDWtools set is already installed on most
% LaTeX systems. The lastest version and documentation is available at:
% http://www.ctan.org/pkg/mdwtools


% IEEEtran contains the IEEEeqnarray family of commands that can be used to
% generate multiline equations as well as matrices, tables, etc., of high
% quality.


%\usepackage{eqparbox}
% Also of notable interest is Scott Pakin's eqparbox package for creating
% (automatically sized) equal width boxes - aka "natural width parboxes".
% Available at:
% http://www.ctan.org/pkg/eqparbox




% *** SUBFIGURE PACKAGES ***
%\ifCLASSOPTIONcompsoc
%  \usepackage[caption=false,font=footnotesize,labelfont=sf,textfont=sf]{subfig}
%\else
%  \usepackage[caption=false,font=footnotesize]{subfig}
%\fi
% subfig.sty, written by Steven Douglas Cochran, is the modern replacement
% for subfigure.sty, the latter of which is no longer maintained and is
% incompatible with some LaTeX packages including fixltx2e. However,
% subfig.sty requires and automatically loads Axel Sommerfeldt's caption.sty
% which will override IEEEtran.cls' handling of captions and this will result
% in non-IEEE style figure/table captions. To prevent this problem, be sure
% and invoke subfig.sty's "caption=false" package option (available since
% subfig.sty version 1.3, 2005/06/28) as this is will preserve IEEEtran.cls
% handling of captions.
% Note that the Computer Society format requires a sans serif font rather
% than the serif font used in traditional IEEE formatting and thus the need
% to invoke different subfig.sty package options depending on whether
% compsoc mode has been enabled.
%
% The latest version and documentation of subfig.sty can be obtained at:
% http://www.ctan.org/pkg/subfig




% *** FLOAT PACKAGES ***
%
%\usepackage{fixltx2e}
% fixltx2e, the successor to the earlier fix2col.sty, was written by
% Frank Mittelbach and David Carlisle. This package corrects a few problems
% in the LaTeX2e kernel, the most notable of which is that in current
% LaTeX2e releases, the ordering of single and double column floats is not
% guaranteed to be preserved. Thus, an unpatched LaTeX2e can allow a
% single column figure to be placed prior to an earlier double column
% figure.
% Be aware that LaTeX2e kernels dated 2015 and later have fixltx2e.sty's
% corrections already built into the system in which case a warning will
% be issued if an attempt is made to load fixltx2e.sty as it is no longer
% needed.
% The latest version and documentation can be found at:
% http://www.ctan.org/pkg/fixltx2e


%\usepackage{stfloats}
% stfloats.sty was written by Sigitas Tolusis. This package gives LaTeX2e
% the ability to do double column floats at the bottom of the page as well
% as the top. (e.g., "\begin{figure*}[!b]" is not normally possible in
% LaTeX2e). It also provides a command:
%\fnbelowfloat
% to enable the placement of footnotes below bottom floats (the standard
% LaTeX2e kernel puts them above bottom floats). This is an invasive package
% which rewrites many portions of the LaTeX2e float routines. It may not work
% with other packages that modify the LaTeX2e float routines. The latest
% version and documentation can be obtained at:
% http://www.ctan.org/pkg/stfloats
% Do not use the stfloats baselinefloat ability as the IEEE does not allow
% \baselineskip to stretch. Authors submitting work to the IEEE should note
% that the IEEE rarely uses double column equations and that authors should try
% to avoid such use. Do not be tempted to use the cuted.sty or midfloat.sty
% packages (also by Sigitas Tolusis) as the IEEE does not format its papers in
% such ways.
% Do not attempt to use stfloats with fixltx2e as they are incompatible.
% Instead, use Morten Hogholm'a dblfloatfix which combines the features
% of both fixltx2e and stfloats:
%
% \usepackage{dblfloatfix}
% The latest version can be found at:
% http://www.ctan.org/pkg/dblfloatfix


%\ifCLASSOPTIONcaptionsoff
%  \usepackage[nomarkers]{endfloat}
% \let\MYoriglatexcaption\caption
% \renewcommand{\caption}[2][\relax]{\MYoriglatexcaption[#2]{#2}}
%\fi
% endfloat.sty was written by James Darrell McCauley, Jeff Goldberg and 
% Axel Sommerfeldt. This package may be useful when used in conjunction with 
% IEEEtran.cls'  captionsoff option. Some IEEE journals/societies require that
% submissions have lists of figures/tables at the end of the paper and that
% figures/tables without any captions are placed on a page by themselves at
% the end of the document. If needed, the draftcls IEEEtran class option or
% \CLASSINPUTbaselinestretch interface can be used to increase the line
% spacing as well. Be sure and use the nomarkers option of endfloat to
% prevent endfloat from "marking" where the figures would have been placed
% in the text. The two hack lines of code above are a slight modification of
% that suggested by in the endfloat docs (section 8.4.1) to ensure that
% the full captions always appear in the list of figures/tables - even if
% the user used the short optional argument of \caption[]{}.
% IEEE papers do not typically make use of \caption[]'s optional argument,
% so this should not be an issue. A similar trick can be used to disable
% captions of packages such as subfig.sty that lack options to turn off
% the subcaptions:
% For subfig.sty:
% \let\MYorigsubfloat\subfloat
% \renewcommand{\subfloat}[2][\relax]{\MYorigsubfloat[]{#2}}
% However, the above trick will not work if both optional arguments of
% the \subfloat command are used. Furthermore, there needs to be a
% description of each subfigure *somewhere* and endfloat does not add
% subfigure captions to its list of figures. Thus, the best approach is to
% avoid the use of subfigure captions (many IEEE journals avoid them anyway)
% and instead reference/explain all the subfigures within the main caption.
% The latest version of endfloat.sty and its documentation can obtained at:
% http://www.ctan.org/pkg/endfloat
%
% The IEEEtran \ifCLASSOPTIONcaptionsoff conditional can also be used
% later in the document, say, to conditionally put the References on a 
% page by themselves.





% *** PDF, URL AND HYPERLINK PACKAGES ***
%
%\usepackage{url}
% url.sty was written by Donald Arseneau. It provides better support for
% handling and breaking URLs. url.sty is already installed on most LaTeX
% systems. The latest version and documentation can be obtained at:
% http://www.ctan.org/pkg/url
% Basically, \url{my_url_here}.


% NOTE: PDF thumbnail features are not required in IEEE papers
%       and their use requires extra complexity and work.
%\ifCLASSINFOpdf
%  \usepackage[pdftex]{thumbpdf}
%\else
%  \usepackage[dvips]{thumbpdf}
%\fi
% thumbpdf.sty and its companion Perl utility were written by Heiko Oberdiek.
% It allows the user a way to produce PDF documents that contain fancy
% thumbnail images of each of the pages (which tools like acrobat reader can
% utilize). This is possible even when using dvi->ps->pdf workflow if the
% correct thumbpdf driver options are used. thumbpdf.sty incorporates the
% file containing the PDF thumbnail information (filename.tpm is used with
% dvips, filename.tpt is used with pdftex, where filename is the base name of
% your tex document) into the final ps or pdf output document. An external
% utility, the thumbpdf *Perl script* is needed to make these .tpm or .tpt
% thumbnail files from a .ps or .pdf version of the document (which obviously
% does not yet contain pdf thumbnails). Thus, one does a:
% 
% thumbpdf filename.pdf 
%
% to make a filename.tpt, and:
%
% thumbpdf --mode dvips filename.ps
%
% to make a filename.tpm which will then be loaded into the document by
% thumbpdf.sty the NEXT time the document is compiled (by pdflatex or
% latex->dvips->ps2pdf). Users must be careful to regenerate the .tpt and/or
% .tpm files if the main document changes and then to recompile the
% document to incorporate the revised thumbnails to ensure that thumbnails
% match the actual pages. It is easy to forget to do this!
% 
% Unix systems come with a Perl interpreter. However, MS Windows users
% will usually have to install a Perl interpreter so that the thumbpdf
% script can be run. The Ghostscript PS/PDF interpreter is also required.
% See the thumbpdf docs for details. The latest version and documentation
% can be obtained at.
% http://www.ctan.org/pkg/thumbpdf


% NOTE: PDF hyperlink and bookmark features are not required in IEEE
%       papers and their use requires extra complexity and work.
% *** IF USING HYPERREF BE SURE AND CHANGE THE EXAMPLE PDF ***
% *** TITLE/SUBJECT/AUTHOR/KEYWORDS INFO BELOW!!           ***
\newcommand\MYhyperrefoptions{bookmarks=true,bookmarksnumbered=true,
pdfpagemode={UseOutlines},plainpages=false,pdfpagelabels=true,
colorlinks=true,linkcolor={black},citecolor={black},urlcolor={black},
pdftitle={Bare Demo of IEEEtran.cls for Computer Society Journals},%<!CHANGE!
pdfsubject={Typesetting},%<!CHANGE!
pdfauthor={Michael D. Shell},%<!CHANGE!
pdfkeywords={Computer Society, IEEEtran, journal, LaTeX, paper,
             template}}%<^!CHANGE!
%\ifCLASSINFOpdf
%\usepackage[\MYhyperrefoptions,pdftex]{hyperref}
%\else
%\usepackage[\MYhyperrefoptions,breaklinks=true,dvips]{hyperref}
%\usepackage{breakurl}
%\fi
% One significant drawback of using hyperref under DVI output is that the
% LaTeX compiler cannot break URLs across lines or pages as can be done
% under pdfLaTeX's PDF output via the hyperref pdftex driver. This is
% probably the single most important capability distinction between the
% DVI and PDF output. Perhaps surprisingly, all the other PDF features
% (PDF bookmarks, thumbnails, etc.) can be preserved in
% .tex->.dvi->.ps->.pdf workflow if the respective packages/scripts are
% loaded/invoked with the correct driver options (dvips, etc.). 
% As most IEEE papers use URLs sparingly (mainly in the references), this
% may not be as big an issue as with other publications.
%
% That said, Vilar Camara Neto created his breakurl.sty package which
% permits hyperref to easily break URLs even in dvi mode.
% Note that breakurl, unlike most other packages, must be loaded
% AFTER hyperref. The latest version of breakurl and its documentation can
% be obtained at:
% http://www.ctan.org/pkg/breakurl
% breakurl.sty is not for use under pdflatex pdf mode.
%
% The advanced features offer by hyperref.sty are not required for IEEE
% submission, so users should weigh these features against the added
% complexity of use.
% The package options above demonstrate how to enable PDF bookmarks
% (a type of table of contents viewable in Acrobat Reader) as well as
% PDF document information (title, subject, author and keywords) that is
% viewable in Acrobat reader's Document_Properties menu. PDF document
% information is also used extensively to automate the cataloging of PDF
% documents. The above set of options ensures that hyperlinks will not be
% colored in the text and thus will not be visible in the printed page,
% but will be active on "mouse over". USING COLORS OR OTHER HIGHLIGHTING
% OF HYPERLINKS CAN RESULT IN DOCUMENT REJECTION BY THE IEEE, especially if
% these appear on the "printed" page. IF IN DOUBT, ASK THE RELEVANT
% SUBMISSION EDITOR. You may need to add the option hypertexnames=false if
% you used duplicate equation numbers, etc., but this should not be needed
% in normal IEEE work.
% The latest version of hyperref and its documentation can be obtained at:
% http://www.ctan.org/pkg/hyperref





% *** Do not adjust lengths that control margins, column widths, etc. ***
% *** Do not use packages that alter fonts (such as pslatex).         ***
% There should be no need to do such things with IEEEtran.cls V1.6 and later.
% (Unless specifically asked to do so by the journal or conference you plan
% to submit to, of course. )


% correct bad hyphenation here
\hyphenation{op-tical net-works semi-conduc-tor}

\usepackage{authblk}
\renewcommand\Authands{, }
\renewcommand\Authsep{, }  
\makeatletter
\renewcommand\AB@affilsepx{, \protect\Affilfont} % 
\makeatother
\usepackage{supertabular}

\begin{document}
%
% paper title
% Titles are generally capitalized except for words such as a, an, and, as,
% at, but, by, for, in, nor, of, on, or, the, to and up, which are usually
% not capitalized unless they are the first or last word of the title.
% Linebreaks \\ can be used within to get better formatting as desired.
% Do not put math or special symbols in the title.

% \title{Large Model Safety: A Comprehensive Survey of Attacks, Defenses, and Open Challenges}

% \title{Safety at Scale: A Comprehensive Survey of Large Model Safety}

\title{
  \begin{minipage}{0.04\textwidth}
  \vspace{-0.1cm}
    \includegraphics[scale=0.18]{Fig/logo.jpg}
  \end{minipage}
  \hspace{0.07\textwidth}
  \begin{minipage}{0.86\textwidth}
    \centering
    Safety at Scale: A Comprehensive Survey of Large Model Safety
  \end{minipage}
}

%
%
% author names and IEEE memberships
% note positions of commas and nonbreaking spaces ( ~ ) LaTeX will not break
% a structure at a ~ so this keeps an author's name from being broken across
% two lines.
% use \thanks{} to gain access to the first footnote area
% a separate \thanks must be used for each paragraph as LaTeX2e's \thanks
% was not built to handle multiple paragraphs
%
%
%\IEEEcompsocitemizethanks is a special \thanks that produces the bulleted
% lists the Computer Society journals use for "first footnote" author
% affiliations. Use \IEEEcompsocthanksitem which works much like \item
% for each affiliation group. When not in compsoc mode,
% \IEEEcompsocitemizethanks becomes like \thanks and
% \IEEEcompsocthanksitem becomes a line break with idention. This
% facilitates dual compilation, although admittedly the differences in the
% desired content of \author between the different types of papers makes a
% one-size-fits-all approach a daunting prospect. For instance, compsoc 
% journal papers have the author affiliations above the "Manuscript
% received ..."  text while in non-compsoc journals this is reversed. Sigh.

\author[1]{Xingjun Ma}
\author[1]{Yifeng Gao}
\author[1]{Yixu Wang}
\author[1]{Ruofan Wang}
\author[1]{Xin Wang}
\author[1]{Ye Sun}
\author[1]{Yifan Ding}
\author[1]{Hengyuan Xu}
\author[1]{Yunhao Chen}
\author[1]{Yunhan Zhao}
\author[2]{Hanxun Huang}
\author[3]{Yige Li}
\author[4]{Jiaming Zhang}
\author[5]{Xiang Zheng}
\author[6]{Yang Bai}
\author[1]{Zuxuan Wu}
\author[1]{Xipeng Qiu}
\author[8,9]{Jingfeng Zhang}
\author[7]{Yiming Li}
\author[10]{Xudong Han}
\author[10]{Haonan Li}
\author[3]{Jun Sun}
\author[5]{Cong Wang}
\author[12]{Jindong Gu}
\author[13]{Baoyuan Wu}
\author[14]{Siheng Chen}
\author[7]{Tianwei Zhang}
\author[7]{Yang Liu}
\author[2]{Mingming Gong}
\author[15]{Tongliang Liu}
\author[16]{Shirui Pan}
\author[17]{Cihang Xie}
\author[18]{Tianyu Pang}
\author[19]{Yinpeng Dong}
\author[20]{Ruoxi Jia}
\author[21]{Yang Zhang}
\author[22]{Shiqing Ma}
\author[23]{Xiangyu Zhang}
\author[24]{Neil Gong}
\author[25]{Chaowei Xiao}
\author[2]{Sarah Erfani}
\author[2,10]{Tim Baldwin}
\author[26]{Bo Li}
\author[9,11]{Masashi Sugiyama}
\author[7]{Dacheng Tao}
\author[2]{James Bailey}
\author[1]{Yu-Gang Jiang$^{\dag}$\thanks{$^{\dag}$Corresponding author: ygj@fudan.edu.cn}}

\affil[1]{Fudan University}
\affil[2]{The University of Melbourne}
\affil[3]{Singapore Management University}
\affil[4]{Hong Kong University of Science and Technology}
\affil[5]{City University of Hong Kong}
\affil[6]{ByteDance}
\affil[7]{Nanyang Technological University}
\affil[8]{University of Auckland}
\affil[9]{RIKEN}
\affil[10]{MBZUAI}
\affil[11]{The University of Tokyo}
\affil[12]{University of Oxford}
\affil[13]{Chinese University of Hong Kong, Shenzhen}
\affil[14]{Shanghai Jiao Tong University}
\affil[15]{The University of Sydney}
\affil[16]{Griffith University}
\affil[17]{University of California, Santa Cruz}
\affil[18]{Sea AI Lab}
\affil[19]{Tsinghua University}
\affil[20]{Virginia Tech}
\affil[21]{CISPA Helmholtz Center for Information Security}
\affil[22]{University of Massachusetts Amherst}
\affil[23]{Purdue University}
\affil[24]{Duke University}
\affil[25]{University of Wisconsin - Madison}
\affil[26]{University of Illinois Urbana-Champaign}

% \author{ Xingjun Ma, Yifeng Gao, Yixu Wang, Ruofan Wang, Xin Wang, Ye Sun, Yifan Ding, Henyuan Xu \\ Yunhao Chen,  Jiaming Zhang, Xiang Zheng, Yang Bai, Yige Li, Hanxun Huang, Jingfeng Zhang, Henghui Ding, Cihang Xie, Cong Wang, Yu-Gang Jiang,~\IEEEmembership{Fellow,~IEEE}% <-this % stops a space
% \IEEEcompsocitemizethanks{\IEEEcompsocthanksitem  Xingjun Ma, Yifeng Gao, Yixu Wang, Ruofan Wang, Xin Wang, Ye Sun, Yifan Ding, Henyuan Xu, Yunhao Chen, and Yu-Gang Jiang are with Fudan University, Shanghai, China. E-mail: ygj@fudan.edu.cn
% \IEEEcompsocthanksitem Hanxun Huang is with The University of Melbourne, Victoria, Australia
% \IEEEcompsocthanksitem Hanxun Huang is with The University of Melbourne, Victoria, Australia
% % note need leading \protect in front of \\ to get a newline within \thanks as
% % \\ is fragile and will error, could use \hfil\break instead.

% }% <-this % stops a space
% % \thanks{Manuscript received April 19, 2005; revised August 26, 2015.}
% }

% note the % following the last \IEEEmembership and also \thanks - 
% these prevent an unwanted space from occurring between the last author name
% and the end of the author line. i.e., if you had this:
% 
% \author{....lastname \thanks{...} \thanks{...} }
%                     ^------------^------------^----Do not want these spaces!
%
% a space would be appended to the last name and could cause every name on that
% line to be shifted left slightly. This is one of those "LaTeX things". For
% instance, "\textbf{A} \textbf{B}" will typeset as "A B" not "AB". To get
% "AB" then you have to do: "\textbf{A}\textbf{B}"
% \thanks is no different in this regard, so shield the last } of each \thanks
% that ends a line with a % and do not let a space in before the next \thanks.
% Spaces after \IEEEmembership other than the last one are OK (and needed) as
% you are supposed to have spaces between the names. For what it is worth,
% this is a minor point as most people would not even notice if the said evil
% space somehow managed to creep in.



% The paper headers
% \markboth{Journal of \LaTeX\ Class Files,~Vol.~14, No.~8, August~2015}%
% {Shell \MakeLowercase{\textit{et al.}}: Bare Advanced Demo of IEEEtran.cls for IEEE Computer Society Journals}


% IEEE TRANSACTIONS ON PATTERN ANALYSIS AND MACHINE INTELLIGENCE
% The only time the second header will appear is for the odd numbered pages
% after the title page when using the twoside option.
% 
% *** Note that you probably will NOT want to include the author's ***
% *** name in the headers of peer review papers.                   ***
% You can use \ifCLASSOPTIONpeerreview for conditional compilation here if
% you desire.



% The publisher's ID mark at the bottom of the page is less important with
% Computer Society journal papers as those publications place the marks
% outside of the main text columns and, therefore, unlike regular IEEE
% journals, the available text space is not reduced by their presence.
% If you want to put a publisher's ID mark on the page you can do it like
% this:
%\IEEEpubid{0000--0000/00\$00.00~\copyright~2015 IEEE}
% or like this to get the Computer Society new two part style.
%\IEEEpubid{\makebox[\columnwidth]{\hfill 0000--0000/00/\$00.00~\copyright~2015 IEEE}%
%\hspace{\columnsep}\makebox[\columnwidth]{Published by the IEEE Computer Society\hfill}}
% Remember, if you use this you must call \IEEEpubidadjcol in the second
% column for its text to clear the IEEEpubid mark (Computer Society journal
% papers don't need this extra clearance.)



% use for special paper notices
%\IEEEspecialpapernotice{(Invited Paper)}



% for Computer Society papers, we must declare the abstract and index terms
% PRIOR to the title within the \IEEEtitleabstractindextext IEEEtran
% command as these need to go into the title area created by \maketitle.
% As a general rule, do not put math, special symbols or citations
% in the abstract or keywords.
% \IEEEtitleabstractindextext{%
% \begin{abstract}
% The abstract goes here.
% \end{abstract}

% % Note that keywords are not normally used for peerreview papers.
% \begin{IEEEkeywords}
% Computer Society, IEEE, IEEEtran, journal, \LaTeX, paper, template.
% \end{IEEEkeywords}}

\IEEEtitleabstractindextext{
\begin{abstract}
The rapid advancement of large models, driven by their exceptional abilities in learning and generalization through large-scale pre-training, has reshaped the landscape of Artificial Intelligence (AI). 
These models are now foundational to a wide range of applications, including conversational AI, recommendation systems, autonomous driving, content generation, medical diagnostics, and scientific discovery. However, their widespread deployment also exposes them to significant safety risks, raising concerns about robustness, reliability, and ethical implications.
This survey provides a systematic review of current safety research on large models, covering Vision Foundation Models (VFMs), Large Language Models (LLMs), Vision-Language Pre-training (VLP) models, Vision-Language Models (VLMs), Diffusion Models (DMs), and large-model-based Agents. 
Our contributions are summarized as follows: (1) We present a comprehensive taxonomy of safety threats to these models, including adversarial attacks, data poisoning, backdoor attacks, jailbreak and prompt injection attacks, energy-latency attacks, data and model extraction attacks, and emerging agent-specific threats. 
(2) We review defense strategies proposed for each type of attacks if available and summarize the commonly used datasets and benchmarks for safety research.
(3) Building on this, we identify and discuss the open challenges in large model safety, emphasizing the need for comprehensive safety evaluations, scalable and effective defense mechanisms, and sustainable data practices. More importantly, we highlight the necessity of collective efforts from the research community and international collaboration.
Our work can serve as a useful reference for researchers and practitioners, fostering the ongoing development of comprehensive defense systems and platforms to safeguard AI models. GitHub: \url{https://github.com/xingjunm/Awesome-Large-Model-Safety}.

\end{abstract}
\begin{IEEEkeywords}
Large Model Safety, AI Safety, Attacks and Defenses
\end{IEEEkeywords}}

% make the title area
\maketitle

% To allow for easy dual compilation without having to reenter the
% abstract/keywords data, the \IEEEtitleabstractindextext text will
% not be used in maketitle, but will appear (i.e., to be "transported")
% here as \IEEEdisplaynontitleabstractindextext when compsoc mode
% is not selected <OR> if conference mode is selected - because compsoc
% conference papers position the abstract like regular (non-compsoc)
% papers do!
\IEEEdisplaynontitleabstractindextext
% \IEEEdisplaynontitleabstractindextext has no effect when using
% compsoc under a non-conference mode.


% For peer review papers, you can put extra information on the cover
% page as needed:
% \ifCLASSOPTIONpeerreview
% \begin{center} \bfseries EDICS Category: 3-BBND \end{center}
% \fi
%
% For peerreview papers, this IEEEtran command inserts a page break and
% creates the second title. It will be ignored for other modes.
\IEEEpeerreviewmaketitle

\section{Introduction}
\label{sec::intro}

Embodied Question Answering (EQA) \cite{das2018embodied} represents a challenging task at the intersection of natural language processing, computer vision, and robotics, where an embodied agent (e.g., a UAV) must actively explore its environment to answer questions posed in natural language. While most existing research has concentrated on indoor EQA tasks \cite{gao2023room, pena2023visual}, such as exploring and answering questions within confined spaces like homes or offices \cite{liu2024aligning}, relatively little attention has been dedicated to EQA tasks in  open-ended city space. Nevertheless, extending EQA to city space is crucial for numerous real-world applications, including autonomous systems \cite{kalinowska2023embodied}, urban region profiling \cite{yan2024urbanclip}, and city planning \cite{gao2024embodiedcity}. 
% 1. 环境复杂性   
%    - 地标重复性问题(如区分相似建筑)  
%    - 动态干扰因素(交通流、行人)  
% 2. 行动复杂性  
%    - 长程导航路径规划  
%    - 移动控制、角度等  
% 3. 感知复杂性  
%    - 复合空间关系推理("A楼东侧商铺西边的车辆")  
%    - 时序依赖的观察结果整合

EQA tasks in city space (referred to as CityEQA) introduce a unique set of challenges that fundamentally differ from those encountered in indoor environments. Compared to indoor EQA, CityEQA faces three main challenges: 

1) \textbf{Environmental complexity with ambiguous objects}: 
Urban environments are inherently more complex,  featuring a diverse range of objects and structures, many of which are visually similar and difficult to distinguish without detailed semantic information (e.g., buildings, roads, and vehicles). This complexity makes it challenging to construct task instructions and specify the desired information accurately, as shown in Figure \ref{fig:example}. 

2) \textbf{Action complexity in cross-scale space}: 
The vast geographical scale of city space compels agents to adopt larger movement amplitudes to enhance exploration efficiency. However, it might risk overlooking detailed information within the scene. Therefore, agents require cross-scale action adjustment capabilities to effectively balance long-distance path planning with fine-grained movement and angular control.

3) \textbf{Perception complexity with observation dynamics}: 
% Rich semantic information in urban settings leads to varying observations depending on distance and orientation, which can impact the accuracy of answer generation. 
Observations can vary greatly depending on distance, orientation, and perspective. For example, an object may look completely different up close than it does from afar or from different angles. These differences pose challenges for consistency and can affect the accuracy of answer generation, as embodied agents must adapt to the dynamic and complex nature of urban environments.


\begin{table}
\centering
\caption{CityEQA-EC vs existing benchmarks.}
\label{table:dataset}
\renewcommand\arraystretch{1.2}
\resizebox{\linewidth}{!}{
\begin{tabular}{cccccc}
             & Place  & Open Vocab & Active & Platform  & Reference \\ \hline
EQA-v1      & Indoor & \textcolor{red}{\ding{55}}          & \textcolor{green}{\ding{51}}      & House3D      & \cite{das2018embodied}  \\
IQUAD        & Indoor & \textcolor{red}{\ding{55}}          & \textcolor{green}{\ding{51}}      & AI2-THOR     & \cite{gordon2018iqa} \\
MP3D-EQA     & Indoor & \textcolor{red}{\ding{55}}          & \textcolor{green}{\ding{51}}      & Matterport3D & \cite{wijmans2019embodied} \\
MT-EQA       & Indoor & \textcolor{red}{\ding{55}}          & \textcolor{green}{\ding{51}}      & House3D      & \cite{yu2019multi} \\
ScanQA       & Indoor & \textcolor{red}{\ding{55}}          & \textcolor{red}{\ding{55}}      & -            & \cite{azuma2022scanqa} \\
SQA3D        & Indoor & \textcolor{red}{\ding{55}}          & \textcolor{red}{\ding{55}}      & -            & \cite{masqa3d} \\
K-EQA        & Indoor & \textcolor{green}{\ding{51}}          & \textcolor{green}{\ding{51}}      & AI2-THOR     & \cite{tan2023knowledge} \\
OpenEQA      & Indoor & \textcolor{green}{\ding{51}}          & \textcolor{green}{\ding{51}}      & ScanNet/HM3D & \cite{majumdar2024openeqa} \\
 \hline
CityEQA-EC   & City (Outdoor)  & \textcolor{green}{\ding{51}}          & \textcolor{green}{\ding{51}}      & EmbodiedCity & - \\ \hline
\end{tabular}}
\end{table}

\begin{figure*}[!htb]
\centering
    \includegraphics[width=0.78\linewidth]{figures/example.pdf}
% \vspace{-0.2cm}
\caption{The typical workflow of the PMA to address City EQA tasks. There are two cars in this area, thus a valid question must contain landmarks and spatial relationships to specify a car. Given the task, PMA will sequentially complete multiple sub-tasks to find the answer.}
% \vspace{-0.2cm}
\label{fig:example}
\end{figure*}

As an initial step toward CityEQA, we developed \textbf{CityEQA-EC}, a benchmark dataset to evaluate embodied agents' performance on CityEQA tasks. The distinctions between this dataset and other EQA benchmarks are summarized in Table \ref{table:dataset}. CityEQA-EC comprises six task types characterized by open-vocabulary questions. These tasks utilize urban landmarks and spatial relationships to delineate the expected answer, adhering to human conventions while addressing object ambiguity. This design introduces significant complexity, turning CityEQA into long-horizon tasks that require embodied agents to identify and use landmarks, explore urban environments effectively, and refine observation to generate high-quality answers.

To address CityEQA tasks, we introduce the \textbf{Planner-Manager-Actor (PMA)}, a novel baseline agent powered by large models, designed to emulate human-like rationale for solving long-horizon tasks in urban environments, as illustrated in Figure \ref{fig:example}. PMA employs a hierarchical framework to generate actions and derive answers. The Planner module parses tasks and creates plans consisting of three sub-task types: navigation, exploration, and collection. The Manager oversees the execution of these plans while maintaining a global object-centric cognitive map \cite{deng2024opengraph}. This 2D grid-based representation enables precise object identification (retrieval) and efficient management of long-term landmark information. The Actor generates specific actions based on the Manager's instructions through its components: Navigator, Explorer, and Collector. Notably, the Collector integrates a Multi-Modal Large Language Model (MM-LLM) as its Vision-Language-Action (VLA) module to refine observations and generate high-quality answers.
PMA's performance is assessed against four baselines, including humans. 
Results show that humans perform best in CityEQA, while PMA achieves 60.73\% of human accuracy in answering questions, highlighting both the challenge and validity of the proposed benchmarks. 

% The Frontier-Based Exploration (FBE) Agent, widely used in indoor EQA tasks, performs worse than even a blind LLM. This underscores the importance of PMA's hierarchical framework and its use of landmarks and spatial relationships for tackling CityEQA tasks.

In summary, this paper makes the following significant contributions:
\vspace{-8pt}
\begin{itemize}[leftmargin=*]
    \item To the best of our knowledge, we present the first open-ended embodied question answering benchmark for city space, namely CityEQA-EC.
    \vspace{-7pt}
    \item We propose a novel baseline model, PMA, which is capable of solving long-horizon tasks for CityEQA tasks with a human-like rationale.
     \vspace{-7pt}
    \item Experimental results demonstrate that our approach outperforms existing baselines in tackling the CityEQA task. However, the gap with human performance highlights opportunities for future research to improve visual thinking and reasoning in embodied intelligence for city spaces.
\end{itemize}





\section{Vision Foundation Model Safety} \label{sec:vfm}
This section surveys safety research on two types of VFMs: per-trained Vision Transformers (ViTs)~\cite{dosovitskiy2021an} and the Segment Anything Model (SAM)~\cite{kirillov2023segment}. 
We focus on ViTs and SAM because they are among the most widely deployed VFMs and have garnered significant attention in recent safety research.

\begin{table*}[htp]
\center
\caption{A summary of attacks and defenses for ViTs and SAM.}
\label{tab:vfm_safety}
\resizebox{1\textwidth}{!}{
\begin{tabular}{p{0.09\textwidth}p{0.16\textwidth}p{0.05\textwidth}p{0.15\textwidth}p{0.20\textwidth}p{0.25\textwidth}p{0.2\textwidth}}
            \toprule
            \belowrulesepcolor{sunye-red}
        \rowcolor{sunye-red} 
        \textbf{Attack/Defense} & \textbf{Method} & \textbf{Year} & \textbf{Category} & \textbf{Subcategory} & \textbf{Target Models} & \textbf{Datasets} \\ \aboverulesepcolor{orange!25!}  \midrule
    \belowrulesepcolor{gray!25!}
    \rowcolor{gray!25!}\multicolumn{7}{c}{\textbf{Attacks and defenses for ViT (Sec.~\ref{sec:vfm_vit})}} \\ \aboverulesepcolor{gray!25!}  \midrule 
\multirow{15}{0.08\textwidth}{Adversarial Attack} & Patch-Fool~\cite{fu2022patch} & 2022 & White-box & Patch Attack & DeiT, ResNet & ImageNet \\
                   & \cellcolor{gray!15!}SlowFormer~\cite{navaneet2024slowformer} & \cellcolor{gray!15!}2024 & \cellcolor{gray!15!}White-box & \cellcolor{gray!15!}Patch Attack & \cellcolor{gray!15!}ATS, AdaViT & \cellcolor{gray!15!}ImageNet \\
                   & PE-Attack~\cite{gao2024pe} & 2024 & White-box & Position Embedding Attack & ViT, DeiT, BEiT & ImageNet, GLUE, wmt13/16, Food-101, CIFAR100, etc. \\
                   & \cellcolor{gray!15!}Attention-Fool~\cite{lovisotto2022give} & \cellcolor{gray!15!}2022 & \cellcolor{gray!15!}White-box & \cellcolor{gray!15!}Attention Attack & \cellcolor{gray!15!}ViT, DeiT, DETR & \cellcolor{gray!15!}ImageNet \\
                   & AAS~\cite{jain2024towards} & 2024 & White-box & Attention Attack & ViT-B & ImageNet, CIFAR10/100 \\
                   & \cellcolor{gray!15!}SE-TR~\cite{naseer2021improving} & \cellcolor{gray!15!}2022 & \cellcolor{gray!15!}Black-box & \cellcolor{gray!15!}Transfer-based Attack & \cellcolor{gray!15!}DeiT, T2T, TnT, DINO, DETR & \cellcolor{gray!15!}ImageNet \\
                   & ATA~\cite{wang2022generating} & 2022 & Black-box & Transfer-based Attack & ViT, DeiT, ConViT & ImageNet \\
                   & \cellcolor{gray!15!}PNA-PatchOut~\cite{wei2022towards} & \cellcolor{gray!15!}2022 & \cellcolor{gray!15!}Black-box & \cellcolor{gray!15!}Transfer-based Attack & \cellcolor{gray!15!}ViT, DeiT, TNT, LeViT, PiT, CaiT, ConViT, Visformer & \cellcolor{gray!15!}ImageNet \\
                   & LPM~\cite{wei2023boosting} & 2023 & Black-box & Transfer-based Attack & ViT, PiT, DeiT, Visformer, LeViT, ConViT & ImageNet \\ 
                   & \cellcolor{gray!15!}MIG~\cite{ma2023transferable} & \cellcolor{gray!15!}2023 & \cellcolor{gray!15!}Black-box & \cellcolor{gray!15!}Transfer-based Attack & \cellcolor{gray!15!}ViT, TNT, Swin & \cellcolor{gray!15!}ImageNet \\
                   & TGR~\cite{zhang2023transferable} & 2023 & Black-box & Transfer-based Attack & DeiT, TNT, LeViT, ConViT & ImageNet \\
                   & \cellcolor{gray!15!}VDC~\cite{zhang2024improving} & \cellcolor{gray!15!}2024 & \cellcolor{gray!15!}Black-box & \cellcolor{gray!15!}Transfer-based Attack & \cellcolor{gray!15!}CaiT, TNT, LeViT, ConViT & \cellcolor{gray!15!}ImageNet \\
                   & FDAP~\cite{gao2024attacking} & 2024 & Black-box & Transfer-based Attack & ViT, DeiT, CaiT, ConViT, TNT & ImageNet \\
                   & \cellcolor{gray!15!}SASD-WS~\cite{wu2024improving} & \cellcolor{gray!15!}2024 & \cellcolor{gray!15!}Black-box & \cellcolor{gray!15!}Transfer-based Attack & \cellcolor{gray!15!}ViT, ResNet, DenseNet, VGG & \cellcolor{gray!15!}ImageNet \\
                   & CRFA~\cite{li2024improving} & 2024 & Black-box & Transfer-based Attack & ViT, DeiT, CaiT, TNT, Visformer, LeViT, ConvNeXt, RepLKNet & ImageNet \\
                   & \cellcolor{gray!15!}PAR~\cite{shi2022decision} & \cellcolor{gray!15!}2022 & \cellcolor{gray!15!}Black-box & \cellcolor{gray!15!}Query-based Attack & \cellcolor{gray!15!}ViT & \cellcolor{gray!15!}ImageNet \\ \aboverulesepcolor{gray!25!} \midrule
\multirow{7}{0.08\textwidth}{Adversarial Defense}& AGAT~\cite{wu2022towards} & 2022 & Adversarial Training & Efficient training & ViT, CaiT, LeViT & ImageNet \\
                   & \cellcolor{gray!15!}ARD-PRM~\cite{mo2022adversarial} & \cellcolor{gray!15!}2022 & \cellcolor{gray!15!}Adversarial Training & \cellcolor{gray!15!}Efficient training & \cellcolor{gray!15!}ViT, DeiT, ConViT, Swin & \cellcolor{gray!15!}ImageNet, CIFAR10 \\
                   & Patch-Vestiges~\cite{li2022patch} & 2022 & Adversarial Detection & Patch-based Detection & ViT, ResNet & CIFAR10 \\
                   & \cellcolor{gray!15!}ViTGuard~\cite{sun2024vitguard} & \cellcolor{gray!15!}2024 & \cellcolor{gray!15!}Adversarial Detection & \cellcolor{gray!15!}Attention-based Detection & \cellcolor{gray!15!}ViT & \cellcolor{gray!15!}ImageNet, CIFAR10/100 \\
                   & ARMRO~\cite{liu2023understanding} & 2023 & Adversarial Detection & Attention-based Detection & ViT, DeiT & ImageNet, CIFAR10 \\
                   & \cellcolor{gray!15!}Smoothed-Attention~\cite{gu2022vision} & \cellcolor{gray!15!}2022 & \cellcolor{gray!15!}Robust Architecture & \cellcolor{gray!15!}Robust Attention & \cellcolor{gray!15!}DeiT, ResNet & \cellcolor{gray!15!}ImageNet \\
                   & TAP~\cite{guo2023robustifying} & 2023 & Robust Architecture & Robust Attention & RVT, FAN & ImageNet, Cityscapes, COCO \\
                   & \cellcolor{gray!15!}RSPC~\cite{guo2023improving} & \cellcolor{gray!15!}2023 & \cellcolor{gray!15!}Robust Architecture & \cellcolor{gray!15!}Robust Attention & \cellcolor{gray!15!}RVT, FAN & \cellcolor{gray!15!}ImageNet, CIFAR10/100 \\
                   & FViT~\cite{huimproving} & 2024 & Robust Architecture & Robust Attention & ViT, DeiT, Swin & ImageNet, Cityscapes, COCO \\
                   & \cellcolor{gray!15!}CGDMP~\cite{bai2024diffusion} & \cellcolor{gray!15!}2024 & \cellcolor{gray!15!}Adversarial Purification & \cellcolor{gray!15!}Diffusion-based Purification & \cellcolor{gray!15!}ResNet, XciT & \cellcolor{gray!15!}CIFAR 10/100, GTSRB, ImageNet\\
                   & ADBM~\cite{li2024adbm} & 2024 & Adversarial Purification & Diffusion-based Purification & WideResNet, ViT & CIFAR-10, ImageNet, SVHN \\ 
                   & \cellcolor{gray!15!}OSCP~\cite{lei2024instant} & \cellcolor{gray!15!}2024 & \cellcolor{gray!15!}Adversarial Purification & \cellcolor{gray!15!}Diffusion-based Purification & \cellcolor{gray!15!}ViT, Swin, WideResNet& \cellcolor{gray!15!}ImageNet, CelebA-HQ\\  \midrule
\multirow{5}{0.08\textwidth}{Backdoor Attack} & BadViT~\cite{yuan2023you} & 2023 & Data Poisoning & Patch-level Attack & DeiT, LeViT & ImageNet \\
                   & \cellcolor{gray!15!}TrojViT~\cite{zheng2023trojvit} & \cellcolor{gray!15!}2023 & \cellcolor{gray!15!}Data Poisoning & \cellcolor{gray!15!}Patch-level Attack & \cellcolor{gray!15!}DeiT, ViT, Swin & \cellcolor{gray!15!}ImageNet, CIFAR10 \\
                   & SWARM~\cite{yang2024not} & 2024 & Data Poisoning & Token-level Attack & ViT & VTAB-1k \\
                   & \cellcolor{gray!15!}DBIA~\cite{lv2023dbia} & \cellcolor{gray!15!}2023 & \cellcolor{gray!15!}Data Poisoning & \cellcolor{gray!15!}Data-free Attack & \cellcolor{gray!15!}ViT, DeiT, Swin & \cellcolor{gray!15!}ImageNet, CIFAR10/100, GTSRB, GGFace \\ 
                   & MTBA~\cite{li2024multi} & 2024 & Data Poisoning & Multi-trigger Attack & ViT & ImageNet, CIFAR10 \\
                   \midrule
\multirow{2}{0.08\textwidth}{Backdoor Defense} & PatchDrop~\cite{doan2023defending} & 2023 & Robust Inference & Patch Processing & ViT, DeiT, ResNet & ImageNet, CIFAR10 \\
                   & \cellcolor{gray!15!}Image Blocking~\cite{subramanya2024closer} & \cellcolor{gray!15!}2023 & \cellcolor{gray!15!}Robust Inference & \cellcolor{gray!15!}Image Blocking & \cellcolor{gray!15!}ViT, CaiT & \cellcolor{gray!15!}ImageNet \\ \aboverulesepcolor{gray!25!}  \midrule
    \belowrulesepcolor{gray!25!}
    \rowcolor{gray!25!}\multicolumn{7}{c}{\textbf{Attacks and defenses for SAM (Sec.~\ref{sec:vfm_sam})}} \\ \aboverulesepcolor{gray!25!}  \midrule 
\multirow{8}{0.08\textwidth}{Adversarial Attack} & S-RA~\cite{shen2024practical} & 2024 & White-box & Prompt-agnostic Attack & SAM & SA-1B \\
                   & \cellcolor{gray!15!}Croce et al.~\cite{croce2024segment} & \cellcolor{gray!15!}2024 & \cellcolor{gray!15!}White-box & \cellcolor{gray!15!}Prompt-agnostic Attack & \cellcolor{gray!15!}SAM, SEEM & \cellcolor{gray!15!}SA-1B \\
                   & Attack-SAM~\cite{zhang2023attack} & 2023 & Black-box & Transfer-based Attack & SAM & SA-1B \\
                   & \cellcolor{gray!15!}PATA++~\cite{zheng2023black} & \cellcolor{gray!15!}2023 & \cellcolor{gray!15!}Black-box & \cellcolor{gray!15!}Transfer-based Attack & \cellcolor{gray!15!}SAM &\cellcolor{gray!15!}SA-1B \\ 
                   & UAD~\cite{lu2024unsegment} & 2024 & Black-box & Transfer-based Attack & SAM, FastSAM & SA-1B \\ 
                   & \cellcolor{gray!15!}T-RA~\cite{shen2024practical} & \cellcolor{gray!15!}2024 & \cellcolor{gray!15!}Black-box & \cellcolor{gray!15!}Transfer-based Attack & \cellcolor{gray!15!}SAM &\cellcolor{gray!15!}SA-1B \\
                   & UMI-GRAT~\cite{xia2024transferable} & 2024 & Black-box & Transfer-based Attack & Medical SAM, Shadow-SAM, Camouflaged-SAM & CT-Scans, ISTD, COD10K, CAMO, CHAME \\
                   & \cellcolor{gray!15!}Han et al.~\cite{han2023segment} & \cellcolor{gray!15!}2023 & \cellcolor{gray!15!}Black-box & \cellcolor{gray!15!}Universal Attack & \cellcolor{gray!15!}SAM & \cellcolor{gray!15!}SA-1B \\ 
                   & DarkSAM~\cite{zhou2024darksam} & 2024 & Black-box & Universal Attack & SAM, HQ-SAM, PerSAM & ADE20K, Cityscapes, COCO, SA-1B \\ \midrule
Adversarial Defense & ASAM~\cite{li2024asam} & 2024 & Adversarial Tuning  & Diffusion Model-based Tuning & SAM & Ade20k, VOC2012, COCO, DOORS, LVIS, etc. \\ \midrule
\multirow{2}{0.08\textwidth}{Backdoor\&\\Poisoning Attack} & BadSAM~\cite{guan2024badsam} & 2024 & Data Poisoning & Visual trigger & SAM & CAMO \\
                   & \cellcolor{gray!15!}UnSeg~\cite{sun2024unseg} & \cellcolor{gray!15!}2024 & \cellcolor{gray!15!}Data Poisoning & \cellcolor{gray!15!}Unlearnable Examples & \cellcolor{gray!15!}HQ-SAM, DINO, Rsprompter, UNet++, Mask2Former, DeepLabV3 & \cellcolor{gray!15!}Cityscapes, VOC, COCO, Lung, Kvasir-seg, WHU, etc. \\ \aboverulesepcolor{gray!15!} \bottomrule
\end{tabular}
}
\end{table*}

\subsection{Attacks and Defenses for ViTs}\label{sec:vfm_vit}
Pre-trained ViTs are widely employed as backbones for various downstream tasks, frequently achieving state-of-the-art performance through efficient adaptation and fine-tuning. Unlike traditional CNNs, ViTs process images as sequences of tokenized patches, allowing them to better capture spatial dependencies. However, this patch-based mechanism also brings unique safety concerns and robustness challenges. This section explores these issues by reviewing ViT-related safety research, including adversarial attacks, backdoor \& poisoning attacks, and their corresponding defense strategies.  Table~\ref{tab:vfm_safety} provides a summary of the surveyed attacks and defenses, along with the commonly used datasets.


\subsubsection{Adversarial Attacks}\label{sec:ViT-adv}
Adversarial attacks on ViTs can be classified into \textbf{white-box attacks} and \textbf{black-box attacks} based on whether the attacker has full access to the victim model. Based on the attack strategy, white-box attacks can be further divided into 1) \textbf{patch attacks}, 2) \textbf{position embedding attacks} and 3) \textbf{attention attacks}, while black-box attacks can be summarized into 1) \textbf{transfer-based attacks} and 2) \textbf{query-based attacks}.

\paragraph{White-box Attacks}
\textbf{Patch Attacks} exploit the modular structure of ViTs, aiming to manipulate their inference processes by introducing targeted perturbations in specific patches of the input data. Joshi et al.~\cite{joshi2021adversarial} proposed an adversarial token attack method leveraging block sparsity to assess the vulnerability of ViTs to token-level perturbations. 
Expanding on this, \textbf{Patch-Fool}\cite{fu2022patch} introduces an adversarial attack framework that targets the self-attention modules by perturbing individual image patches, thereby manipulating attention scores.
Different from existing methods, \textbf{SlowFormer}~\cite{navaneet2024slowformer} introduces a universal adversarial patch can be applied to any image to increases computational and energy costs while preserving model accuracy.

\textbf{Position Embedding Attacks} aim to attack the spatial or sequential position of tokens in transformers. For example, \textbf{PE-Attack}~\cite{gao2024pe} explores the common vulnerability of positional embeddings to adversarial perturbations by disrupting their ability to encode positional information through periodicity manipulation, linearity distortion, and optimized embedding distortion.

\textbf{Attention Attacks} target vulnerabilities in the self-attention modules of ViTs. \textbf{Attention-Fool}~\cite{lovisotto2022give} manipulates dot-product similarities to redirect queries to adversarial key tokens, exposing the model's sensitivity to adversarial patches. Similarly, \textbf{AAS}~\cite{jain2024towards} mitigates gradient masking in ViTs by optimizing the pre-softmax output scaling factors, enhancing the effectiveness of attacks.


\paragraph{Black-box Attacks}
\textbf{Transfer-based Attacks} first generate adversarial examples using fully accessible surrogate models, which are then transferred to attack black-box victim ViTs. In this context, we first review attacks specifically designed for the ViT architecture.
\textbf{SE-TR}~\cite{naseer2021improving} enhances adversarial transferability by optimizing perturbations on an ensemble of models.
\textbf{ATA}~\cite{wang2022generating} strategically activates uncertain attention and perturbs sensitive embeddings within ViTs.
\textbf{LPM}\cite{wei2023boosting} mitigates the overfitting to model-specific discriminative regions through a patch-wise optimized binary mask.
Chen et al.\cite{chen2023understanding} introduced an Inductive Bias Attack (\textbf{IBA}) to suppress unique biases in ViTs and target shared inductive biases.
\textbf{TGR}~\cite{zhang2023transferable} reduces the variance of the backpropagated gradient within internal blocks.
\textbf{VDC}~\cite{zhang2024improving} employs virtual dense connections between deeper attention maps and MLP blocks to facilitate gradient backpropagation.
\textbf{FDAP}~\cite{gao2024attacking} exploits feature collapse by reducing high-frequency components in feature space.
\textbf{CRFA}~\cite{li2024improving} disrupts only the most crucial image regions using approximate attention maps.
\textbf{SASD-WS}~\cite{wu2024improving} flattens the loss landscape of the source model through sharpness-aware self-distillation and approximates an ensemble of pruned models using weight scaling to improve target adversarial transferability.

Other strategies are applicable to both ViTs and CNNs, ensuring broader applicability in black-box settings.
Wei et al.\cite{wei2022towards, wei2023towards} proposed a dual attack framework to improve transferability between ViTs and CNNs: 1) a Pay No Attention (\textbf{PNA}) attack, which skips the gradients of attention during backpropagation, and 2) a \textbf{PatchOut} attack, which randomly perturbs subsets of image patches at each iteration.
\textbf{MIG}\cite{ma2023transferable} uses integrated gradients and momentum-based updates to precisely target model-agnostic critical regions, improving transferability between ViTs and CNNs.


\textbf{Query-based Attacks} generate adversarial examples by querying the black-box model and levering the model responses to estimate the adversarial gradients. The goal is to achieve successful attack with a minimal number of queries. Based on the type of model response, query-based attacks can be further divided into score-based attacks, where the model returns a probability vector, and decision-based attacks, where the model provides only the top-k classes. Decision-based attacks typically start from a large random noise (to achieve misclassification first) and then gradually find smaller noise while maintaining misclassification.
To improve the efficiency of the adversarial noise searching process in ViTs, \textbf{PAR}~\cite{shi2022decision} introduces a coarse-to-fine patch searching method, guided by noise magnitude and sensitivity masks to account for the structural characteristics of ViTs and mitigate the negative impact of non-overlapping patches.



\subsubsection{Adversarial Defenses}\label{sec:ViT-advdefense}
Adversarial defenses for ViTs follow four major approaches: 1) \textbf{adversarial training}, which trains ViTs on adversarial examples via min-max optimization to improve its robustness; 2) \textbf{adversarial detection}, which identifies and mitigates adversarial attacks by detecting abnormal or malicious patterns in the inputs; 3) \textbf{robust architecture}, which modifies and optimizes the architecture (e.g., self-attention module) of ViTs to improve their resilience against adversarial attacks; and 4) \textbf{adversarial purification}, which pre-processes the input (e.g., noise injection, denoising, or other transformations) to remove potential adversarial perturbations before inference.


\textbf{Adversarial Training} is widely regarded as the most effective approach to adversarial defense; however, it comes with a high computational cost. To address this on ViTs, \textbf{AGAT}~\cite{wu2022towards} introduces a dynamic attention-guided dropping strategy, which accelerates the training process by selectively removing certain patch embeddings at each layer. This reduces computational overhead while maintaining robustness, especially on large datasets such as ImageNet. Due to its high computational cost, research on adversarial training for ViTs has been relatively limited.
\textbf{ARD-PRM}~\cite{mo2022adversarial} improves adversarial robustness by randomly dropping gradients in attention blocks and masking patch perturbations during training.

\textbf{Adversarial Detection} methods for ViTs primarily leverage two key features, i.e., patch-based inference and activation characteristics, to detect and mitigate adversarial examples.
Li et al.~\cite{li2022patch} proposed the concept of \textbf{Patch Vestiges}, abnormalities arising from adversarial examples during patch division in ViTs. They used statistical metrics on step changes between adjacent pixels across patches and developed a binary regression classifier to detect adversaries. Alternatively, \textbf{ARMOR}~\cite{liu2023understanding} identifies adversarial patches by scanning for unusually high column scores in specific layers and masking them with average images to reduce their impact. 
\textbf{ViTGuard}~\cite{sun2024vitguard}, on the other hand, employs a masked autoencoder to detect patch attacks by analyzing attention maps and CLS token representations. As more attacks are developed, there is a growing need for a unified detection framework capable of handling all types of adversarial examples.



\textbf{Robust Architecture} methods focus on designing more adversarially resilient attention modules for ViTs. 
For example, \textbf{Smoothed Attention}~\cite{gu2022vision} employs temperature scaling in the softmax function to prevent any single patch from dominating the attention, thereby balancing focus across patches.
\textbf{ReiT}~\cite{gong2024random} integrates adversarial training with randomization through the II-ReSA module, optimizing randomly entangled tokens to reduce adversarial similarity and enhance robustness.
\textbf{TAP}~\cite{guo2023robustifying} addresses token overfocusing by implementing token-aware average pooling and an attention diversification loss, which incorporate local neighborhood information and reduce cosine similarity among attention vectors. \textbf{FViTs}~\cite{huimproving} strengthen explanation faithfulness by stabilizing top-k indices in self-attention and robustify predictions using denoised diffusion smoothing combined with Gaussian noise. \textbf{RSPC}~\cite{guo2023improving} tackles vulnerabilities by corrupting the most sensitive patches and aligning intermediate features between clean and corrupted inputs to stabilize the attention mechanism. Collectively, these advancements underscore the pivotal role of the attention mechanism in improving the adversarial robustness of ViTs.


\textbf{Adversarial Purification} refers to a model-agnostic input-processing technique that is broadly applicable across various architectures, including but not limited to ViTs.
\textbf{DiffPure}~\cite{nie2022diffusion} introduces a framework where adversarial images undergo noise injection via a forward stochastic differential equation (SDE) process, followed by denoising with a pre-trained diffusion model. \textbf{CGDMP}~\cite{bai2024diffusion} refines this approach by optimizing the noise level for the forward process and employing contrastive loss gradients to guide the denoising process, achieving improved purification tailored to ViTs. \textbf{ADBM}~\cite{li2024adbm} highlights the disparity between diffused adversarial and clean examples, proposing a method to directly connect the clean and diffused adversarial distributions.
While these methods focus on ViTs, other approaches demonstrate broader applicability to various vision models, e.g., CNNs. \textbf{Purify++}\cite{zhang2023purify++} enhances DiffPure with improved diffusion models, \textbf{DifFilter}\cite{chen2024diffilter} extends noise scales to better preserve semantics, and \textbf{MimicDiffusion}\cite{song2024mimicdiffusion} mitigates adversarial impacts during the reverse diffusion process. For improved efficiency, \textbf{OSCP}\cite{lei2024instant} and \textbf{LightPure}\cite{khalili2024lightpure} propose single-step and real-time purification methods, respectively. \textbf{LoRID}\cite{zollicoffer2024lorid} introduces a Markov-based approach for robust purification. These methods complement ViT-related research and highlight diverse advancements in adversarial purification.



\subsubsection{Backdoor Attacks }\label{sec:ViT-backdoor}
Backdoors can be injected into the victim model via data poisoning, training manipulation, or parameter editing, with most existing attacks on ViTs being data poisoning-based. 
We classify these attacks into four categories: 1) \textbf{patch-level attacks}, 2) \textbf{token-level attacks}, and 3) \textbf{multi-trigger attacks}, which exploit ViT-specific data processing characteristics, as well as 4) \textbf{data-free attacks}, which exploit the inherent mechanisms of ViTs.


\textbf{Patch-level Attacks} primarily exploit the ViT's characteristic of processing images as discrete patches by implanting triggers at the patch level. For example, \textbf{BadViT}~\cite{yuan2023you} introduces a universal patch-wise trigger that requires only a small amount of data to redirect the model's focus from classification-relevant patches to adversarial triggers. 
\textbf{TrojViT}~\cite{zheng2023trojvit} improves this approach by utilizing patch salience ranking, an attention-targeted loss function, and parameter distillation to minimize the bit flips necessary to embed the backdoor.

\textbf{Token-level Attacks} target the tokenization layer of ViTs. \textbf{SWARM}~\cite{yang2024not} introduces a switchable backdoor mechanism featuring a ``switch token'' that dynamically toggles between benign and adversarial behaviors, ensuring high attack success rates while maintaining functionality in clean environments.

\textbf{Multi-trigger Attacks} employ multiple backdoor triggers in parallel, sequential, or hybrid configurations to poison the victim dataset.
\textbf{MTBAs}~\cite{li2024multi} utilize these multiple triggers to induce coexistence, overwriting, and cross-activation effects, significantly diminishing the effectiveness of existing defense mechanisms.

\textbf{Data-free Attacks} eliminate the need for original training datasets. Using substitute datasets, \textbf{DBIA}~\cite{lv2023dbia} generates universal triggers that maximize attention within ViTs. These triggers are fine-tuned with minimal parameter adjustments using PGD~\cite{madry2017towards}, enabling efficient and resource-light backdoor injection.




\subsubsection{Backdoor Defenses}\label{sec:ViT-backdoordefense}
Backdoor defenses for ViTs aim to identify and break (or remove) the correlation between trigger patterns and target classes while preserving model accuracy.
Two representative defense strategies are: 1) \textbf{patch processing}, which disrupts the integrity of image patches to prevent trigger activation, and 2) \textbf{image blocking}, which leverages interpretability-based mechanisms to mask and neutralize the effects of backdoor triggers.

\textbf{Patch Processing} strategy disrupts the integrity of patches to neutralize triggers.
Doan et al.~\cite{doan2023defending} found that clean-data accuracy and attack success rates of ViTs respond differently to patch transformations before positional encoding, and proposed an effective defense method by randomly dropping or shuffling patches of an image to counter both patch-based and blending-based backdoor attacks.
\textbf{Image Blocking} utilizes interpretability to identify and neutralize triggers.
Subramanya et al.~\cite{subramanya2022backdoor} showed that ViTs can localize backdoor triggers using attention maps and proposed a defense mechanism that dynamically masks potential trigger regions during inference.
In a subsequent work, Subramanya et al.~\cite{subramanya2024closer} proposed to integrate trigger neutralization into the training phase to improve the robustness of ViTs to backdoor attacks. 
While these two methods are promising, the field requires a holistic defense framework that integrates non-ViT defenses with ViT-specific characteristics and unifies multiple defense tasks including backdoor detection, trigger inversion, and backdoor removal, as attempted in \cite{li2024expose}.



\subsubsection{Datasets}\label{sec:ViT-dataset}
Datasets are crucial for developing and evaluating attack and defense methods. Table~\ref{tab:vfm_safety} summarizes the datasets used in adversarial and backdoor research.

\textbf{Datasets for Adversarial Research} As shown in Table~\ref{tab:vfm_safety}, adversarial researches were primarily conducted on ImageNet. While attacks were tested across various datasets like CIFAR-10/100, Food-101, and GLUE, defenses were mainly limited to ImageNet and CIFAR-10/100. This imbalance reveals one key issue in adversarial research: attacks are more versatile, while defenses struggle to generalize across different datasets.

\textbf{Datasets for Backdoor Research} Backdoor researches were also conducted mainly on ImageNet and CIFAR-10/100 datasets. Some attacks, such as DBIA and SWARM, extend to domain-specific datasets like GTSRB and VGGFace, while defenses, including PatchDrop, were often limited to a few benchmarks. This narrow focus reduces their real-world applicability. 
Although backdoor defenses are shifting towards robust inference techniques, they typically target specific attack patterns, limiting their generalizability. To address this, adaptive defense strategies need to be tested across a broader range of datasets to effectively counter the evolving nature of backdoor threats.





\subsection{Attacks and Defenses for SAM}\label{sec:vfm_sam}
SAM is a foundational model for image segmentation, comprising three primary components: a ViT-based image encoder, a prompt encoder, and a mask decoder. The image encoder transforms high-resolution images into embeddings, while the prompt encoder converts various input modalities into token embeddings. The mask decoder combines these embeddings to generate segmentation masks using a two-layer Transformer architecture.
Due to its complex structure, attacks and defenses targeting SAM differ significantly from those developed for CNNs. These unique challenges stem from SAM's modular and interconnected design, where vulnerabilities in one component can propagate to others, necessitating specialized strategies for both attack and defense. 
This section systematically reviews SAM-related adversarial attacks, backdoor \& poisoning attacks, and adversarial defense strategies, as summarized in Table~\ref{tab:vfm_safety}.





\subsubsection{Adversarial Attacks}\label{sec:SAM-adv}

Adversarial attacks on SAM can be categorized into: (1) \textbf{white-box attacks}, exemplified by \emph{prompt-agnostic attacks}, and (2) \textbf{black-box attacks}, which can be further divided into \emph{universal attacks} and \emph{transfer-based attacks}. Each category employs distinct strategies to compromise segmentation performance.

\paragraph{White-box Attacks}
\textbf{Prompt-Agnostic Attacks} are white-box attacks that disrupt SAM's segmentation without relying on specific prompts, using either \emph{prompt-level} or \emph{feature-level} perturbations for generality across inputs.
For prompt-level attacks, Shen et al.~\cite{shen2024practical} proposed a grid-based strategy to generate adversarial perturbations that disrupt segmentation regardless of click location.
For feature-level attacks, Croce et al.~\cite{croce2024segment} perturbed features from the image encoder to distort spatial embeddings, undermining SAM’s segmentation integrity.

\paragraph{Black-box Attacks}

\textbf{Universal Attacks} generate UAPs~\cite{moosavi2017universal} that can consistently disrupt SAM across arbitrary prompts.  Han et al.~\cite{han2023segment} exploited contrastive learning to optimize the UAPs, achieving better attack performance by exacerbating feature misalignment.
\textbf{DarkSAM}~\cite{zhou2024darksam}, on the other hand, introduces a hybrid spatial-frequency framework that combines semantic decoupling and texture distortion to generate universal perturbations.  


\textbf{Transfer-based Attacks} exploit transferable representations in SAM to generate perturbations that remain adversarial across different models and tasks. 
\textbf{PATA++}\cite{zheng2023black} improves transferability by using a regularization loss to highlight key features in the image encoder, reducing reliance on prompt-specific data. 
\textbf{Attack-SAM}\cite{zhang2023attack} employs ClipMSE loss to focus on mask removal, optimizing for spatial and semantic consistency to improve cross-task transferability. 
\textbf{UMI-GRAT}~\cite{xia2024transferable} follows a two-step process: it first generates a generalizable perturbation with a surrogate model and then applies gradient robust loss to improve across-model transferability.
Apart from designing new loss functions, optimization over transformation techniques can also be exploited to improve transferability.
This includes \textbf{T-RA}~\cite{shen2024practical}, which improves cross-model transferability by applying spectrum transformations to generate adversarial perturbations that degrade segmentation in SAM variants, and \textbf{UAD}~\cite{lu2024unsegment}, which generates adversarial examples by deforming images in a two-stage process and aligning features with the deformed targets.



\subsubsection{Adversarial Defenses}\label{sec:SAM-advdefense}

Adversarial defenses for SAM are currently limited, with existing approaches focusing primarily on adversarial tuning, which integrates adversarial training into the prompt tuning process of SAM. For example, \textbf{ASAM}~\cite{li2024asam} utilizes a stable diffusion model to generate realistic adversarial samples on a low-dimensional manifold through diffusion model-based tuning. ControlNet~\cite{ControlNet} is then employed to guide the re-projection process, ensuring that the generated samples align with the original mask annotations. Finally, SAM is fine-tuned using these adversarial examples.





\subsubsection{Backdoor \& Poisoning Attacks}\label{sec:SAM-backdoor_poisoning}

Backdoor and poisoning attacks on SAM remain underexplored. Here, we review one backdoor attack that leverages perceptible visual triggers to compromise SAM, and one poisoning attack that exploits unlearnable examples~\cite{huang2021unlearnable} with imperceptible noise to protect unauthorized image data from being exploited by segmentation models.
\textbf{BadSAM}~\cite{guan2024badsam} is a backdoor attack targeting SAM that embeds visual triggers during the model's adaptation phase, implanting backdoors that enable attackers to manipulate the model's output with specific inputs. Specifically, the attack introduces MLP layers to SAM and injects the backdoor trigger into these layers via SAM-Adapter \cite{chen2023sam}.
\textbf{UnSeg}~\cite{sun2024unseg} is a data poisoning attack on SAM designed for benign purposes, i.e., data protection. It fine-tunes a universal unlearnable noise generator, leveraging a bilevel optimization framework based on a pre-trained SAM. This allows the generator to efficiently produce poisoned (protected) samples, effectively preventing a segmentation model from learning from the protected data and thereby safeguarding against unauthorized exploitation of personal information.



\subsubsection{Datasets}\label{sec:SAM-dataset}

As shown in Table~\ref{tab:vfm_safety}, the datasets used in safety research on SAM slightly differ from those typically used in general segmentation tasks~\cite{MOSE,MeViS}. 
For \textbf{attack research}, the SA-1B dataset and its subsets~\cite{kirillov2023segment} are the most commonly used for evaluating adversarial attacks~\cite{croce2024segment, han2023segment, lu2024unsegment, shen2024practical, zhang2023attack, zheng2023black}.
Additionally, \textbf{DarkSAM} was evaluated on datasets such as Cityscapes \cite{cordts2016cityscapes}, COCO \cite{lin2014microsoft}, and ADE20k \cite{zhou2017scene}, while \textbf{UMI-GRAT}, which targets downstream tasks related to SAM, was tested on medical datasets like CT-Scans and ISTD, as well as camouflage datasets, including COD10K, CAMO, and CHAME. For backdoor attacks, \textbf{BadSAM} was assessed using the CAMO dataset~\cite{le2019anabranch}. In the context of data poisoning, \textbf{UnSeg}~\cite{sun2024unseg} was evaluated across 10 datasets, including COCO, Cityscapes, ADE20k, WHU, and medical datasets like Lung and Kvasir-seg.
For \textbf{defense research}, \textbf{ASAM}~\cite{li2024asam} is currently the only defense method applied to SAM. It was evaluated on a range of datasets with more diverse image distributions than SA-1B, including ADE20k, LVIS, COCO, and others, with mean Intersection over Union (mIoU) used as the evaluation metric.


\begin{table*}[htp]
\center
\caption{A summary of attacks and defenses for LLMs (\textbf{Part I}). }
\label{tab:LLM-Part1}
\resizebox{1\textwidth}{!}{
\begin{tabular}{p{0.1\textwidth}p{0.15\textwidth}p{0.05\textwidth}p{0.15\textwidth}p{0.15\textwidth}p{0.25\textwidth}p{0.27\textwidth}}
\hline
\rowcolor{green!10!}
% \rowcolor{dingyifan-wangyixu-darkblue-light}
Attack/Defense & Method & Year & Category & Subcategory & Target Models & Datasets \\ \hline
\multirow{13}{0.1\textwidth}{Adversarial Attack} & \cellcolor{gray!15!}Bad characters~\cite{boucher2022bad} & \cellcolor{gray!15!}2022 & \cellcolor{gray!15!}White-box & \cellcolor{gray!15!}Character-level & \cellcolor{gray!15!}Fairseq EN-FR, Perspective API & \cellcolor{gray!15!}Emotion, Wikipedia Detox, CoNLL-2003 \\
& \cellcolor{white}TextFooler~\cite{jin2020bert} & \cellcolor{white}2020 & \cellcolor{white}White-box & \cellcolor{white}Word-level & \cellcolor{white}WordCNN, WordLSTM, BERT, InferSent, ESIM & \cellcolor{white}AG’s News, Fake News, MR, IMDB, Yelp, SNLI, MultiNLI \\
& \cellcolor{gray!15!}BERT-ATTACK~\cite{li2020bert} & \cellcolor{gray!15!}2020 & \cellcolor{gray!15!}White-box & \cellcolor{gray!15!}Word-level & \cellcolor{gray!15!}BERT, WordLSTM, ESIM & \cellcolor{gray!15!}AG’s News, Fake News, IMDB, Yelp, SNLI, MultiNLI \\
& \cellcolor{white}GBDA~\cite{guo2021gradient} & \cellcolor{white}2021 & \cellcolor{white}White-box & \cellcolor{white}Word-level & \cellcolor{white}GPT-2, XLM, BERT & \cellcolor{white}DBPedia, AG's News, Yelp Reviews, IMDB, MultiNLI\\
& \cellcolor{gray!15!}Breaking-BERT~\cite{dirkson2021breaking} & \cellcolor{gray!15!}2021 & \cellcolor{gray!15!}White-box & \cellcolor{gray!15!}Word-level & \cellcolor{gray!15!}BERT & \cellcolor{gray!15!}CoNLL-2003, W-NUT 2017, BC5CDR, NCBI disease corpus \\
& \cellcolor{white}GRADOBSTINATE~\cite{wang2023gradient} & \cellcolor{white}2023 & \cellcolor{white}White-box & \cellcolor{white}Word-level & \cellcolor{white}Electra, ALBERT, DistillBERT, RoBERTa & \cellcolor{white}SNLI, MRPC, SQuAD, SST-2, MSCOCO \\
& \cellcolor{gray!15!}Liu et al.~\cite{liu2023expanding} & \cellcolor{gray!15!}2023 & \cellcolor{gray!15!}White-box & \cellcolor{gray!15!}Word-level & \cellcolor{gray!15!}BERT, RoBERTa & \cellcolor{gray!15!}Online Shopping 10 Cats, Chinanews \\
& \cellcolor{white}advICL~\cite{wang2023adversarial} & \cellcolor{white}2023 & \cellcolor{white}Black-box & \cellcolor{white}Sentence-level & \cellcolor{white}GPT-2-XL, LLaMA-7B, Vicuna-7B & \cellcolor{white}SST-2, RTE, TREC, DBpedia \\
& \cellcolor{gray!15!}Liu et al.~\cite{liu2023adversarial} & \cellcolor{gray!15!}2023 & \cellcolor{gray!15!}Black-box & \cellcolor{gray!15!}Sentence-level & \cellcolor{gray!15!}RoBERTa &  \cellcolor{gray!15!}Real conversation data \\
& \cellcolor{white}Koleva et al.~\cite{koleva2023adversarial} & \cellcolor{white}2023 & \cellcolor{white}Black-box & \cellcolor{white}Sentence-level & \cellcolor{white}TURL & \cellcolor{white}WikiTables \\
\hline
\multirow{3}{0.1\textwidth}{Adversarial Defense} & \cellcolor{gray!15!}Jain et al.\cite{jain2023baseline} & \cellcolor{gray!15!}2023 & \cellcolor{gray!15!}Adversarial Detection & \cellcolor{gray!15!}Input Filtering & \cellcolor{gray!15!}Guanaco-7B, Vicuna-7B, Falcon-7B & \cellcolor{gray!15!}AlpacaEval \\
& \cellcolor{white}Erase-and-Check~\cite{kumar2023certifying} & \cellcolor{white}2023 & \cellcolor{white}Adversarial Detection & \cellcolor{white}Input Filtering & \cellcolor{white}LLaMA-2, DistilBERT & \cellcolor{white}AdvBench \\
& \cellcolor{gray!15!}Zou et al.~\cite{zou2024improving} & \cellcolor{gray!15!}2024 & \cellcolor{gray!15!}Robust Inference & \cellcolor{gray!15!}Circuit Breaking & \cellcolor{gray!15!}Mistral-7B, LLaMA-3-8B & \cellcolor{gray!15!}HarmBench \\
\hline
\multirow{25}{0.1\textwidth}{Jailbreak Attack} & \cellcolor{white}Yong et al.~\cite{yong2023low} & \cellcolor{white}2023 & \cellcolor{white}Black-box & \cellcolor{white}Hand-crafted & \cellcolor{white}GPT-4 & \cellcolor{white}AdvBench \\
& \cellcolor{gray!15!}CipherChat~\cite{yuan2023gpt} & \cellcolor{gray!15!}2023 & \cellcolor{gray!15!}Black-box & \cellcolor{gray!15!}Hand-crafted & \cellcolor{gray!15!}GPT-3.5, GPT-4 & \cellcolor{gray!15!}Chinese safety assessment benchmark \\
& \cellcolor{white}Jailbroken~\cite{wei2024jailbroken} & \cellcolor{white}2023 & \cellcolor{white}Black-box & \cellcolor{white}Hand-crafted & \cellcolor{white}GPT-4, GPT-3.5, Claude-1.3 & \cellcolor{white}Self-built \\
& \cellcolor{gray!15!}Li et al.~\cite{li2024cross} & \cellcolor{gray!15!}2024 & \cellcolor{gray!15!}Black-box & \cellcolor{gray!15!}Hand-crafted & \cellcolor{gray!15!}GPT-3.5, GPT-4, Vicuna-1.3-7B, 13B, Vicuna-1.5-7B, 13B & \cellcolor{gray!15!}Self-built \\
& \cellcolor{white}Easyjailbreak~\cite{zhou2024easyjailbreak} & \cellcolor{white}2024 & \cellcolor{white}Black-box & \cellcolor{white}Hand-crafted & \cellcolor{white}GPT-3.5, GPT-4, LLaMA-2-7B, 13B, Vicuna-1.5-7B, 13B, ChatGLM3, Qwen-7B, InternLM-7B, Mistral-7B & \cellcolor{white}AdvBench \\
& \cellcolor{gray!15!}SMEA~\cite{zou2024system} & \cellcolor{gray!15!}2024 & \cellcolor{gray!15!}Black-box & \cellcolor{gray!15!}Hand-crafted & \cellcolor{gray!15!}GPT-3.5, LLaMA-2-7B, 13B, Vicuna-7B, 13B & \cellcolor{gray!15!}Self-built \\
& \cellcolor{white}Tastle~\cite{xiao2024tastle} & \cellcolor{white}2024 & \cellcolor{white}Black-box & \cellcolor{white}Hand-crafted & \cellcolor{white}Vicuna-1.5-13B, LLaMA-2-7B, GPT-3.5, GPT-4 & \cellcolor{white}AdvBench \\
& \cellcolor{gray!15!}StructuralSleight~\cite{li2024structuralsleight} & \cellcolor{gray!15!}2024 & \cellcolor{gray!15!}Black-box & \cellcolor{gray!15!}Hand-crafted & \cellcolor{gray!15!}GPT-3.5, GPT-4, GPT-4o, LLaMA-3-70B, Claude-2, Cluade3-Opus & \cellcolor{gray!15!}AdvBench \\
& \cellcolor{white}CodeChameleon~\cite{lv2024codechameleon} & \cellcolor{white}2024 & \cellcolor{white}Black-box & \cellcolor{white}Hand-crafted & \cellcolor{white}LLaMA-2-7B, 13B, 70B, Vicuna-1.5-7B, 13B, GPT-3.5, GPT-4 & \cellcolor{white}AdvBench, MaliciousInstruct, ShadowAlignment  \\
& \cellcolor{gray!15!}Puzzler~\cite{chang2024play} & \cellcolor{gray!15!}2024 & \cellcolor{gray!15!}Black-box & \cellcolor{gray!15!}Hand-crafted & \cellcolor{gray!15!}GPT-3.5, GPT-4, GPT4-Turbo, Gemini-pro, LLaMA-2-7B, 13B & \cellcolor{gray!15!}AdvBench, MaliciousInstructions \\
& \cellcolor{white} Shen et al.\cite{SCBSZ24} & \cellcolor{white}2024 & \cellcolor{white}Black-box & \cellcolor{white}Hand-crafted & \cellcolor{white}GPT-3.5, GPT-4, PaLM-2, ChatGLM, Dolly, Vicuna & \cellcolor{white}In-The-Wild Jailbreak Prompts \\
& \cellcolor{gray!15!}AutoDAN~\cite{liu2023autodan} & \cellcolor{gray!15!}2023 & \cellcolor{gray!15!}Black-box & \cellcolor{gray!15!}Automated & \cellcolor{gray!15!}Vicuna-7B, Guanaco-7B, LLaMA-2-7B & \cellcolor{gray!15!}AdvBench \\
& \cellcolor{white}I-FSJ\cite{zheng2024improved} & \cellcolor{white}2024 & \cellcolor{white}Black-box & \cellcolor{white}Automated & \cellcolor{white}LLaMA-2, LLaMA-3, OpenChat-3.5, Starling-LM, Qwen-1.5 & \cellcolor{white}JailbreakBench \\
& \cellcolor{gray!15!}Weak-to-Strong\cite{zhao2024weak} & \cellcolor{gray!15!}2024 & \cellcolor{gray!15!}Black-box & \cellcolor{gray!15!}Automated & \cellcolor{gray!15!}LLaMA2-13B, Vicuna-13B, Baichuan2-13B, InternLM-20B & \cellcolor{gray!15!}AdvBench, MaliciousInstruct \\
& \cellcolor{white}GPTFuzzer~\cite{yu2023gptfuzzer} & \cellcolor{white}2023 & \cellcolor{white}Black-box & \cellcolor{white}Automated & \cellcolor{white}Vicuna-13B, Baichuan-13B, ChatGLM-2-6B, LLaMA-2-13B, 70B, GPT-4, Bard, Claude-2, PaLM-2 & \cellcolor{white}Self-built \\
& \cellcolor{gray!15!}PAIR~\cite{chao2023jailbreaking} & \cellcolor{gray!15!}2023 & \cellcolor{gray!15!}Black-box & \cellcolor{gray!15!}Automated & \cellcolor{gray!15!}Vicuna-1.5-13B, LLaMA-2-7B, GPT-3.5, GPT-4, Claude-1, Claude-2, Gemini-pro & \cellcolor{gray!15!}JBB-Behaviors, AdvBench \\
& \cellcolor{white}Masterkey~\cite{deng2024masterkey} & \cellcolor{white}2023  & \cellcolor{white}Black-box & \cellcolor{white}Automated & \cellcolor{white}GPT-3.5, GPT-4, Bard, Bing Chat & \cellcolor{white}Self-built \\
& \cellcolor{gray!15!}BOOST~\cite{yu2024enhancing} & \cellcolor{gray!15!}2024 & \cellcolor{gray!15!}Black-box & \cellcolor{gray!15!}Automated & \cellcolor{gray!15!}LLaMA-2-7B, 13B, Gemma-2B, 7B, Tulu-2-7B, 13B, Mistral-7B, MPT-7B, Qwen1.5-7B, Vicuna-7B, LLaMA-3-8B &  \cellcolor{gray!15!}AdvBench \\
& \cellcolor{white}FuzzLLM~\cite{yao2024fuzzllm} & \cellcolor{white}2024 & \cellcolor{white}Black-box & \cellcolor{white}Automated & \cellcolor{white}Vicuna-13B, CAMEL-13B, LLaMA-7B, ChatGLM-2-6B, Bloom-7B, LongChat-7B, GPT-3.5,  GPT-4 & \cellcolor{white}Self-built \\
& \cellcolor{gray!15!}EnJa\cite{zhang2024enja} & \cellcolor{gray!15!}2024 & \cellcolor{gray!15!}Black-box & \cellcolor{gray!15!}Automated &  \cellcolor{gray!15!}Vicuna-7B, 13B, LLaMA-2-13B, GPT-3.5, 4 &  \cellcolor{gray!15!}AdvBench \\
& \cellcolor{white}Perez et al.\cite{perez2022red} & \cellcolor{white}2022 & \cellcolor{white}Black-box & \cellcolor{white}Automated & \cellcolor{white}Gopher LM & \cellcolor{white}Self-built \\
& \cellcolor{gray!15!}CRT\cite{hong2024curiosity} & \cellcolor{gray!15!}2024 & \cellcolor{gray!15!}Black-box & \cellcolor{gray!15!}Automated & \cellcolor{gray!15!}GPT-2, Dolly-v2-7B, LLaMA-2-7B & \cellcolor{gray!15!}IMDb \\
& \cellcolor{white} ECLIPSE~\cite{jiang2024unlocking} &
\cellcolor{white} 2024 &
\cellcolor{white} Black-box &
\cellcolor{white} Automated &
\cellcolor{white} Vicuna-7B, LLaMA2-7B, Falcon-7B, GPT-3.5 &
\cellcolor{white} AdvBench \\
& \cellcolor{gray!15!}GCG~\cite{zou2023universal} & \cellcolor{gray!15!}2023 & \cellcolor{gray!15!}White-box & \cellcolor{gray!15!}Automated & \cellcolor{gray!15!}Vicuna-7B, LLaMA-2-7B, GPT-3.5, GPT-4, PaLM-2, Claude-2 & \cellcolor{gray!15!}AdvBench  \\
& \cellcolor{white}I-GCG \cite{jia2024improved} & \cellcolor{white}2024 & \cellcolor{white}White-box & \cellcolor{white}Automated & \cellcolor{white}Vicuna-7B-1.5, Guanaco-7B, LLaMA2-7B, MISTRAL-7B & \cellcolor{white}AdvBench\\
\hline
\multirow{12}{0.1\textwidth}{Jailbreak Defense} & \cellcolor{gray!15!}SmoothLLM~\cite{robey2023smoothllm} & \cellcolor{gray!15!}2023 & \cellcolor{gray!15!}Input Defense & \cellcolor{gray!15!}Rephrasing & \cellcolor{gray!15!}Vicuna, LLaMA-2, GPT-3.5, GPT-4 & \cellcolor{gray!15!}AdvBench, JBB-Behaviors \\
& \cellcolor{white}SemanticSmooth~\cite{ji2024defending} & \cellcolor{white}2024 & \cellcolor{white}Input Defense & \cellcolor{white}Rephrasing & \cellcolor{white}LLaMA-2-7B, Vicuna-13B, GPT-3.5 & \cellcolor{white}InstructionFollow, AlpacaEval \\
& \cellcolor{gray!15!}SelfDefend~\cite{wang2024selfdefend} & \cellcolor{gray!15!}2024 & \cellcolor{gray!15!}Input Defense& \cellcolor{gray!15!}Rephrasing & \cellcolor{gray!15!}GPT-3.5, GPT-4 &  \cellcolor{gray!15!}JailbreakHub, JailbreakBench, MultiJail, AlpacaEval \\
& \cellcolor{white}IBProtector~\cite{liu2024protecting} & \cellcolor{white}2024 & \cellcolor{white}Input Defense& \cellcolor{white}Rephrasing & \cellcolor{white}LLaMA-2-7B, Vicuna-1.5-13B & \cellcolor{white}AdvBench, TriviaQA, EasyJailbreak \\
& \cellcolor{gray!15!}Backtranslation~\cite{wang2024defending} & \cellcolor{gray!15!}2024 & \cellcolor{gray!15!}Input Defense & \cellcolor{gray!15!}Translation & \cellcolor{gray!15!}GPT-3.5, LLaMA-2-13B, Vicuna-13B & \cellcolor{gray!15!}AdvBench, MT-Bench \\
& \cellcolor{white}APS~\cite{kim2023robust} & \cellcolor{white}2023 & \cellcolor{white}Output Defense& \cellcolor{white}Filtering & \cellcolor{white}Vicuna, Falcon, Guanaco & \cellcolor{white}AdvBench \\
& \cellcolor{gray!15!}DPP~\cite{xiong2024defensive} & \cellcolor{gray!15!}2024 & \cellcolor{gray!15!}Output Defense& \cellcolor{gray!15!}Filtering & \cellcolor{gray!15!}LLaMA-2-7B, Mistral-7B & \cellcolor{gray!15!}AdvBench \\
& \cellcolor{white}Gradient Cuff~\cite{hu2024gradient} & \cellcolor{white}2024 & \cellcolor{white}Output Defense & \cellcolor{white}Filtering & \cellcolor{white}LLaMA-2-7B, Vicuna-1.5-7B & \cellcolor{white}AdvBench \\
& \cellcolor{gray!15!}MTD~\cite{chen2023jailbreaker} & \cellcolor{gray!15!}2023 & \cellcolor{gray!15!}Robust Inference& \cellcolor{gray!15!}Multi-model Inference & \cellcolor{gray!15!}GPT-3.5, GPT-4, Bard, Claude, LLaMA2-7B, 13B, 70B & \cellcolor{gray!15!}Self-built \\
& \cellcolor{white}PARDEN~\cite{zhang2024parden} & \cellcolor{white}2024 & \cellcolor{white}Robust Inference& \cellcolor{white}Output Repetition & \cellcolor{white}LLaMA-2-7B, Mistral-7B, Claude-2.1 & \cellcolor{white}PARDEN \\
& \cellcolor{gray!15!}AutoDefense~\cite{lu2024autojailbreak} & \cellcolor{gray!15!}2024 & \cellcolor{gray!15!}Ensemble Defense & \cellcolor{gray!15!}Rephrasing/Filtering & \cellcolor{gray!15!}GPT-3.5-turbo, GPT-4, LLaMA-2, LLaMA-3, Mistral, Qwen, Vicuna & \cellcolor{gray!15!}Self-built  \\
& \cellcolor{white}MoGU~\cite{du2024mogu} & \cellcolor{white}2024 & \cellcolor{white}Ensemble Defense & \cellcolor{white}Rephrasing/Filtering & \cellcolor{white}LLaMA-2-7B, Vicuna-7B, Falcon-7B, Dolphin-7B & \cellcolor{white}Advbench \\
\hline
\multirow{10}{0.1\textwidth}{Prompt Injection Attack} & \cellcolor{gray!15!}PROMPTINJECT\cite{perez2022ignore} & \cellcolor{gray!15!}2022 & \cellcolor{gray!15!}Black-box & \cellcolor{gray!15!}Hand-crafted & \cellcolor{gray!15!}text-davinci-002 & \cellcolor{gray!15!}PromptInject \\
& \cellcolor{white}HOUYI~\cite{liu2023prompt} & \cellcolor{white}2023 & \cellcolor{white}Black-box & \cellcolor{white}Hand-crafted & \cellcolor{white}LLM-integrated applications & \cellcolor{white}- \\
& \cellcolor{gray!15!}Greshake~\cite{greshake2023not} & \cellcolor{gray!15!}2023 & \cellcolor{gray!15!}Black-box & \cellcolor{gray!15!}Hand-crafted & \cellcolor{gray!15!}text-davinci-003, GPT-4, Codex & \cellcolor{gray!15!}-  \\
& \cellcolor{white}Liu et al. \cite{liu2024formalizing} & \cellcolor{white}2024 & \cellcolor{white}Black-box & \cellcolor{white}Hand-crafted & \cellcolor{white}PaLM-2-text-bison-001, Flan-UL2, Vicuna-13B, 33B, GPT-3.5-Turbo, GPT-4, LLaMA-2-7B, 13B, Bard, InternLM-7B & \cellcolor{white}MRPC, Jfleg, HSOL, RTE, SST2, SMS
Spam, Gigaword \\
& \cellcolor{gray!15!}Ye et al. \cite{ye2024we} & \cellcolor{gray!15!}2024 & \cellcolor{gray!15!}Black-box & \cellcolor{gray!15!}Hand-crafted & \cellcolor{gray!15!}GPT-4o, Llama-3.1-70B, DeepSeek-V2.5, Qwen-2.5-72B & \cellcolor{gray!15!}- \\
& \cellcolor{white}Deng et al.~\cite{deng2023attack} & \cellcolor{white}2023 & \cellcolor{white}Black-box & \cellcolor{white}Automated & \cellcolor{white}GPT-3.5, Alpaca-LoRA-7B, 13B & \cellcolor{white}-  \\
& \cellcolor{gray!15!}Liu et al.~\cite{liu2024automatic} & \cellcolor{gray!15!}2024 & \cellcolor{gray!15!}Black-box & \cellcolor{gray!15!}Automated & \cellcolor{gray!15!}LLaMA-2-7b & \cellcolor{gray!15!}Dual-Use, BAD+, SAP \\
& \cellcolor{white}G2PIA~\cite{zhang2024goal} & \cellcolor{white}2024 & \cellcolor{white}Black-box & \cellcolor{white}Automated & \cellcolor{white}GPT-3.5, 4, LLaMA2-7B, 13B, 70B & \cellcolor{white}GSM8K, web-based QA, MATH, SQuAD \\
& \cellcolor{gray!15!}PLeak\cite{hui2024pleak} & \cellcolor{gray!15!}2024 & \cellcolor{gray!15!}Black-box & \cellcolor{gray!15!}Automated &\cellcolor{gray!15!}GPT-J-6B, OPT-6.7B, Falcon-7B,
LLaMA-2-7B, Vicuna, 50 real-world LLM applications  &\cellcolor{gray!15!}- \\
& \cellcolor{white}JudgeDeceiver\cite{shi2024optimization} & \cellcolor{white}2024 & \cellcolor{white}Black-box & \cellcolor{white}Automated & \cellcolor{white}Mistral-7B, Openchat-3.5,
LLaMA-2-7B, LLaMA-3-8B & \cellcolor{white}MT-Bench, LLMBar \\
& \cellcolor{gray!15!}PoisonedAlign\cite{shao2024making} & \cellcolor{gray!15!}2024 & \cellcolor{gray!15!}Black-box & \cellcolor{gray!15!}Automated & \cellcolor{gray!15!} LLaMA-2-7B, LLaMA-3-8B, Gemma-7B, Falcon-7B, GPT-4o min & \cellcolor{gray!15!} HH-RLHF, ORCA-DPO \\
\hline
\end{tabular}
}
\end{table*}



\section{Large Language Model Safety} \label{sec:llm}

LLMs are powerful language models that excel at generating human-like text, translating languages, producing creative content, and answering a diverse array of questions \cite{openai-o1,guo2025deepseek}. They have been rapidly adopted in applications such as conversational agents, automated code generation, and scientific research. Yet, this broad utility also introduces significant vulnerabilities that potential adversaries can exploit.
This section surveys the current landscape of LLM safety research. We examine a spectrum of adversarial behaviors, including jailbreak, prompt injection, backdoor, poisoning, model extraction, data extraction, and energy–latency attacks. Such attacks can manipulate outputs, bypass safety measures, leak sensitive information, and disrupt services, thereby threatening system integrity, confidentiality, and availability. We also review state-of-the-art alignment strategies and defense techniques designed to mitigate these risks. Tables \ref{tab:LLM-Part1} and \ref{tab:LLM-Part2} summarize the details of these works.
	

\subsection{Adversarial Attacks} \label{sec:llm_adversarial_attack}

Adversarial attacks on LLMs aim to mislead the victim model to generate incorrect responses (no matter under targeted or untargeted manners) by subtly altering input text.
%Adversarial attacks on LLMs aim to manipulate a model's response by subtly altering input text. 
We classify these attacks into \textbf{white-box attacks} and \textbf{black-box attacks}, depending on whether the attacker can access the model’s internals.

\subsubsection{White-box Attacks}
White-box attacks assume the attacker has full knowledge of the LLM's architecture, parameters, and gradients. This enables the construction of highly effective adversarial examples by directly optimizing against the model's predictions. These attacks can generally be classified into two levels: \textbf{1) character-level attacks} and \textbf{2) word-level attacks}, differing primarily in their effectiveness and semantic stealthiness.

\textbf{Character-level Attacks} introduce subtle modifications at the character level, such as misspellings, typographical errors, and the insertion of visually similar or invisible characters (e.g., homoglyphs \cite{boucher2022bad}). These attacks exploit the model’s sensitivity to minor character variations, which are often unnoticeable to humans, allowing for a high degree of stealthiness while potentially preserving the original meaning. 

\textbf{Word-level Attacks} modify the input text by substituting or replacing specific words. For example, \textbf{TextFooler}~\cite{jin2020bert} and \textbf{BERT-Attack}~\cite{li2020bert} employ \textit{synonym substitution} to generate adversarial examples while preserving semantic similarity. Other methods, such as \textbf{GBDA} \cite{guo2021gradient} and \textbf{GRADOBSTINATE} \cite{wang2023gradient}, leverage gradient information to identify semantically similar \textit{word substitutions} that maximize the likelihood of a successful attack. Additionally, \textit{targeted word substitution} enables attacks tailored to specific tasks or linguistic contexts. For instance, \cite{dirkson2021breaking} explores targeted attacks on named entity recognition, while \cite{liu2023expanding} adapts word substitution attacks for the Chinese language.

\subsubsection{Black-box Attacks}

Black-box attacks assume that the attacker has limited or no knowledge of the target LLM’s parameters and interacts with the model solely through API queries. In contrast to white-box attacks, black-box attacks employ indirect and adaptive strategies to exploit model vulnerabilities. These attacks typically manipulate input prompts rather than altering the core text. We further categorize existing black-box attacks on LLMs into four types: 1) \textbf{in-context attacks}, 2) \textbf{induced attacks}, 3) \textbf{LLM-assisted attacks}, and 4) \textbf{tabular attacks}.

\textbf{In-context Attacks} exploit the demonstration examples used in in-context learning to introduce adversarial behavior, making the model vulnerable to poisoned prompts. \textbf{AdvICL}~\cite{wang2023adversarial} and \textbf{Transferable-advICL} manipulate these demonstration examples to expose this vulnerability, highlighting the model’s susceptibility to poisoned in-context data.

\textbf{Induced Attacks} rely on carefully crafted prompts to coax the model into generating harmful or undesirable outputs, often bypassing its built-in safety mechanisms. These attacks focus on generating adversarial responses by designing deceptive input prompts. For example, Liu et al.~\cite{liu2023adversarial} analyzed how such prompts can lead the model to produce dangerous outputs, effectively circumventing safeguards designed to prevent such behavior.

\textbf{LLM-Assisted Attacks} leverage LLMs to implement attack algorithms or strategies, effectively turning the model into a tool for conducting adversarial actions. This approach underscores the capacity of LLMs to assist attackers in designing and executing attacks. For instance, Carlini~\cite{carlini2023llm} demonstrated that GPT-4 can be prompted step-by-step to design attack algorithms, highlighting the potential for using LLMs as research assistants to automate adversarial processes.

\textbf{Tabular Attacks} target tabular data by exploiting the structure of columns and annotations to inject adversarial behavior. Koleva et al.~\cite{koleva2023adversarial} proposed an entity-swap attack that specifically targets column-type annotations in tabular datasets. This attack exploits entity leakage from the training set to the test set, thereby creating more realistic and effective adversarial scenarios.


\subsection{Adversarial Defenses} \label{sec:llm_adversarial_defense}
Adversarial defenses are crucial for ensuring the safety, reliability, and trustworthiness of LLMs in real-world applications. Existing adversarial defense strategies for LLMs can be broadly classified based on their primary focus into two categories: \textbf{1) adversarial detection} and \textbf{2) robust inference}.

\subsubsection{Adversarial Detection}
Adversarial detection methods aim to identify and flag potential adversarial inputs before they can affect the model's output. The goal is to implement a filtering mechanism that can differentiate between benign and malicious prompts.

\textbf{Input Filtering} Most adversarial detection methods for LLMs are input filtering techniques that identify and reject adversarial texts based on statistical or structural anomalies. For example, Jain et al. \cite{jain2023baseline} use perplexity to detect adversarial prompts, as these typically show higher perplexity when evaluated by a well-calibrated language model, indicating a deviation from natural language patterns. By setting a perplexity threshold, such inputs can be filtered out. Another approach, \textbf{Erase-and-Check} \cite{kumar2023certifying}, ensures robustness by iteratively erasing parts of the input and checking for output consistency. Significant changes in output signal potential adversarial manipulation.
Input filtering methods offer a lightweight first line of defense, but their effectiveness depends on the chosen features and the sophistication of adversarial attacks, which may bypass these defenses if designed adaptively.




\subsubsection{Robust Inference}
Robust inference methods aim to make the model inherently resistant to adversarial attacks by modifying its internal mechanisms or training. One approach, \textbf{Circuit Breaking} \cite{zou2024improving}, targets specific activation patterns during inference, neutralizing harmful outputs without retraining. While robust inference enhances resistance to adaptive attacks, it often incurs higher computational costs, and its effectiveness varies by model architecture and attack type.

\subsection{Jailbreak Attacks}
\label{sec:llm_jailbreak_attacks}


Unlike adversarial attacks that simply lead victim LLMs to generate incorrect answers, jailbreak attacks trick LLMs into generating inappropriate content ($e.g.$, harmful or deceptive content) by bypassing the built-in safety policy/alignment via hand-crafted or automated jailbreak prompts.
%Unlike adversarial attacks that modify the prompt with character- or word-level perturbations, jailbreak attacks trick LLMs into generating harmful content via hand-crafted or automated jailbreak prompts. A key characteristic of jailbreak attacks is that once a model is jailbroken, it continues to produce harmful responses to follow-up malicious queries. Adversarial attacks, however, require input perturbations for each instance. 
Currently, most jailbreak attacks target the LLM-as-a-Service scenario, following a black-box threat model where the attacker cannot access the model’s internals.


\subsubsection{Hand-crafted Attacks}

Hand-crafted attacks involve designing adversarial prompts to exploit specific vulnerabilities in the target LLM. The goal is to craft word/phrase combinations or structures that can bypass the model's safety filters while still conveying harmful requests.

\textbf{Scenario-based Camouflage} hides malicious queries within complex scenarios, such as role-playing or puzzle-solving, to obscure their harmful intent. For instance, Li et al. \cite{li2024cross} instruct the LLM to adopt a persona likely to generate harmful content, while \textbf{SMEA} \cite{zou2024system} places the LLM in a subordinate role under an authority figure. \textbf{Easyjailbreak} \cite{zhou2024easyjailbreak} frames harmful queries in hypothetical contexts, and \textbf{Puzzler} \cite{chang2024play} embeds them in puzzles whose solutions correspond to harmful outputs.
\textbf{Attention Shifting} redirects the LLM’s focus from the malicious intent by introducing linguistic complexities. \textbf{Jailbroken} \cite{wei2024jailbroken} employs code-switching and unusual sentence structures, \textbf{Tastle} \cite{xiao2024tastle} manipulates tone, and \textbf{StructuralSleight} \cite{li2024structuralsleight} alters sentence structure to disrupt understanding.
In addition, Shen et al.~\cite{SCBSZ24} collected real-world jailbreak prompts shared by users on social media, such as Reddit and Discord, and studied their effectiveness against LLMs.

\textbf{Encoding-Based Attacks} exploit LLMs' limitations in handling rare encoding schemes, such as low-resource languages and encryption. These attacks encode malicious queries in formats like \textbf{Base64} \cite{wei2024jailbroken} or low-resource languages \cite{yong2023low}, or use custom encryption methods like ciphers \cite{yuan2023gpt} and \textbf{CodeChameleon} \cite{lv2024codechameleon} to obfuscate harmful content.

\subsubsection{Automated Attacks}

Unlike hand-crafted attacks, which rely on expert knowledge, automated attacks aim to discover jailbreak prompts autonomously. These attacks either use black-box optimization to search for optimal prompts or leverage LLMs to generate and refine them.

\textbf{Prompt Optimization} leverages optimization algorithms to iteratively refine prompts, targeting higher success rates. For black-box methods, \textbf{AutoDAN} \cite{liu2023autodan} employs a genetic algorithm, \textbf{GPTFuzzer} \cite{yu2023gptfuzzer} utilizes mutation- and generation-based fuzzing techniques, and \textbf{FuzzLLM} \cite{yao2024fuzzllm} generates semantically coherent prompts within an automated fuzzing framework. 
\textbf{I-FSJ} \cite{zheng2024improved} injects special tokens into few-shot demonstrations and uses demo-level random search to optimize the prompt, achieving high attack success rates against aligned models and their defenses. 
For white-box methods, the most notable is \textbf{GCG} \cite{zou2023universal}, which introduces a greedy coordinate gradient algorithm to search for adversarial suffixes, effectively compromising aligned LLMs.
 \textbf{I-GCG} \cite{jia2024improved} further improves GCG with diverse target templates and an automatic multi-coordinate updating strategy, achieving near-perfect attack success rates.

\textbf{LLM-Assisted Attacks} use an adversary LLM to help generate jailbreak prompts. Perez et al. \cite{perez2022red} explored model-based red teaming, finding that an LLM fine-tuned via RL can generate more effective adversarial prompts, though with limited diversity. \textbf{CRT} \cite{hong2024curiosity} improves prompt diversity by minimizing SelfBLEU scores and cosine similarity. \textbf{PAIR} \cite{chao2023jailbreaking} employs multi-turn queries with an attacker LLM to refine jailbreak prompts iteratively. Based on PAIR, Robey et al. \cite{robey2024jailbreaking} introduced \textbf{ROBOPAIR}, which targets LLM-controlled robots, causing harmful physical actions. 
Similarly, \textbf{ECLIPSE}~\cite{jiang2024unlocking} leverages an attacker LLM to identify adversarial suffixes analogous to GCG, thereby automating the prompt optimization process.
To enhance prompt transferability, \textbf{Masterkey} \cite{deng2024masterkey} trains adversary LLMs to attack multiple models.
Additionally, \textbf{Weak-to-Strong Jailbreaking} \cite{zhao2024weak} proposes a novel attack where a weaker, unsafe model guides a stronger, aligned model to generate harmful content, achieving high success rates with minimal computational cost.


\subsection{Jailbreak Defenses}
\label{sec:llm_jailbreak_defenses}

We now introduce the corresponding defense mechanisms for black-box LLMs against jailbreak attacks. Based on the intervention stage, we classify existing defenses into three categories: \textbf{input defense}, \textbf{output defense}, and \textbf{ensemble defense}.

\subsubsection{Input Defenses}

Input defense methods focus on preprocessing the input prompt to reduce its harmful content. Current techniques include \emph{rephrasing} and \emph{translation}.

\textbf{Input Rephrasing} uses paraphrasing or purification to obscure the malicious intent of the prompt. For example, \textbf{SmoothLLM} \cite{robey2023smoothllm} applies random sampling to perturb the prompt, while \textbf{SemanticSmooth} \cite{ji2024defending} finds semantically similar, safe alternatives. Beyond prompt-level changes, \textbf{SelfDefend} \cite{wang2024selfdefend} performs token-level perturbations by removing adversarial tokens with high perplexity. \textbf{IBProtector}, on the other hand, \cite{liu2024protecting} perturbs the encoded input using the information bottleneck principle.


\textbf{Input Translation} uses cross-lingual transformations to mitigate jailbreak attacks. For example, Wang et al. \cite{wang2024defending} proposed refusing to respond if the target LLM rejects the back-translated version of the original prompt, based on the hypothesis that back-translation reveals the underlying intent of the prompt.

\subsubsection{Output Defenses}

Output defense methods monitor the LLM’s generated output to identify harmful content, triggering a refusal mechanism when unsafe output is detected.

\textbf{Output Filtering} inspects the LLM's output and selectively blocks or modifies unsafe responses. This process relies on either judge scores from pre-trained classifiers or internal signals (e.g., the loss landscape) from the LLM itself. For instance, \textbf{APS} \cite{kim2023robust} and \textbf{DPP} \cite{xiong2024defensive} use safety classifiers to identify unsafe outputs, while \textbf{Gradient Cuff} \cite{hu2024gradient} analyzes the LLM’s internal refusal loss function to distinguish between benign and malicious queries.

\textbf{Output Repetition} detects harmful content by observing that the LLM can consistently repeat its benign outputs. \textbf{PARDEN} \cite{zhang2024parden} identifies inconsistencies by prompting the LLM to repeat its output. If the model fails to accurately reproduce its response, especially for harmful queries, it may indicate a potential jailbreak.

\subsubsection{Ensemble Defenses}

Ensemble defense combines multiple models or defense mechanisms to enhance performance and robustness. The idea is that different models and defenses can offset their individual weaknesses, resulting in greater overall safety.

\textbf{Multi-model Ensemble} combines inference results from multiple LLMs to create a more robust system. For example, \textbf{MTD} \cite{chen2023jailbreaker} improves LLM safety by dynamically utilizing a pool of diverse LLMs. Rather than relying on a single model, MTD selects the safest and most relevant response by analyzing outputs from multiple models.

\textbf{Multi-defense Ensemble} integrates multiple defense strategies to strengthen robustness against various attacks. For instance, \textbf{AutoDefense} \cite{lu2024autojailbreak} introduces an ensemble framework combining input and output defenses for enhanced effectiveness. \textbf{MoGU} \cite{du2024mogu} uses a dynamic routing mechanism to balance contributions from a safe LLM and a usable LLM, based on the input query, effectively combining rephrasing and filtering.

\subsection{Prompt Injection Attacks}
\label{sec:llm_prompt_injection_attacks}

Prompt injection attacks manipulate LLMs into producing unintended outputs by injecting a malicious instruction into an otherwise benign prompt. As in Section \ref{sec:llm_jailbreak_attacks}, we focus on black-box prompt injection attacks in LLM-as-a-Service systems, classifying them into two categories: \textbf{hand-crafted} and \textbf{automated} attacks.

\subsubsection{Hand-crafted Attacks}

Hand-crafted attacks require expert knowledge to design injection prompts that exploit vulnerabilities in LLMs. These attacks rely heavily on human intuition. \textbf{PROMPTINJECT} \cite{perez2022ignore} and \textbf{HOUYI} \cite{liu2023prompt} show how attackers can manipulate LLMs by appending malicious commands or using context-ignoring prompts to leak sensitive information. Greshake et al. \cite{greshake2023not} proposed an indirect prompt injection attack against retrieval-augmented LLMs for information gathering, fraud, and content manipulation, by injecting malicious prompts into external data sources.
Liu et al. \cite{liu2024formalizing} formalized prompt injection attacks and defenses, introducing a combined attack method and establishing a benchmark for evaluating attacks and defenses across LLMs and tasks.
Ye et al. \cite{ye2024we} explored LLM vulnerabilities in scholarly peer review, revealing risks of explicit and implicit prompt injections. Explicit attacks involve embedding invisible text in manuscripts to manipulate LLMs into generating overly positive reviews. Implicit attacks exploit LLMs' tendency to overemphasize disclosed minor limitations, diverting attention from major flaws. Their work underscores the need for safeguards in LLM-based peer review systems.

\subsubsection{Automated Attacks}

Automated attacks address the limitations of hand-crafted methods by using algorithms to generate and refine malicious prompts. Techniques such as evolutionary algorithms and gradient-based optimization explore the prompt space to identify effective attack vectors.

Deng et al. \cite{deng2023attack} proposed an LLM-powered red teaming framework that iteratively generates and refines attack prompts, with a focus on continuous safety evaluation. Liu et al. \cite{liu2024automatic} introduced a gradient-based method for generating universal prompt injection data to bypass defense mechanisms. \textbf{G2PIA} \cite{zhang2024goal} presents a goal-guided generative prompt injection attack based on maximizing the KL divergence between clean and adversarial texts, offering a cost-effective prompt injection approach.
\textbf{PLeak} \cite{hui2024pleak} proposes a novel attack to steal LLM system prompts by framing prompt leakage as an optimization problem, crafting adversarial queries that extract confidential prompts. 
\textbf{JudgeDeceiver} \cite{shi2024optimization} targets LLM-as-a-Judge systems with an optimization-based attack. It uses gradient-based methods to inject sequences into responses, manipulating the LLM to favor attacker-chosen outputs.
\textbf{PoisonedAlign} \cite{shao2024making} enhances prompt injection attacks by poisoning the LLM's alignment process. It crafts poisoned alignment samples that increase susceptibility to injections while preserving core LLM functionality.

\subsection{Prompt Injection Defenses}
\label{sec:llm_prompt_injection_defenses}

Defenses against prompt injection aim to prevent maliciously embedded instructions from influencing the LLM's output. Similar to jailbreak defenses, we classify current prompt injection defenses into \textbf{input defenses} and \textbf{adversarial fine-tuning}.

\subsubsection{Input Defenses}

Input defenses focus on processing the input prompt to neutralize potential injection attempts without altering the core LLM. Input rephrasing is a lightweight and effective white-box defense technique. For example, \textbf{StuQ} \cite{chen2024struq} structures user input into distinct instruction and data fields to prevent the mixing of instructions and data. \textbf{SPML} \cite{sharma2024spml} uses Domain-Specific Languages (DSLs) to define and manage system prompts, enabling automated analysis of user inputs against the intended system prompt, which help detect malicious requests.

\subsubsection{Adversarial Fine-tuning}

Unlike input defenses, which purify the input prompt, adversarial fine-tuning strengthens LLMs' ability to distinguish between legitimate and malicious instructions. For instance, \textbf{Jatmo} \cite{piet2023jatmo} fine-tunes the victim LLM to restrict it to well-defined tasks, making it less susceptible to arbitrary instructions. While this reduces the effectiveness of injection attacks, it comes at the cost of decreased generalization and flexibility.
Yi et al. \cite{yi2023benchmarking} proposed two defenses against indirect prompt injection: \textbf{multi-turn dialogue}, which isolates external content from user instructions across conversation turns, and \textbf{in-context learning}, which uses examples in the prompt to help the LLM differentiate data from instructions.
\textbf{SecAlign} \cite{chen2025SecAlign} frames prompt injection defense as a preference optimization problem. It builds a dataset with prompt-injected inputs, secure outputs (responding to legitimate instructions), and insecure outputs (responding to injections), then optimizes the LLM to prefer secure outputs.

\begin{table*}[htp]
\center
\caption{A summary of attacks and defenses for LLMs (\textbf{Part II}).}
\label{tab:LLM-Part2}
\resizebox{1\textwidth}{!}{
\begin{tabular}{p{0.1\textwidth}p{0.18\textwidth}p{0.05\textwidth}p{0.15\textwidth}p{0.2\textwidth}p{0.25\textwidth}p{0.3\textwidth}}
\hline
\rowcolor{green!10!}
Attack/Defense & Method & Year & Category & Subcategory & Target Models & Datasets \\ 
\hline
\multirow{3}{0.1\textwidth}{Prompt Injection Defense}
& \cellcolor{white}StruQ~\cite{chen2024struq} & \cellcolor{white}2024 & \cellcolor{white}Input \& Parameter Defense & \cellcolor{white}Rephrasing \& Fine-tuning & \cellcolor{white}LLaMA-7B, Mistral-7B & \cellcolor{white}AlpacaFarm \\
& \cellcolor{gray!15!}SPML~\cite{sharma2024spml} & \cellcolor{gray!15!}2024 & \cellcolor{gray!15!}Input Defense & \cellcolor{gray!15!}Rephrasing & \cellcolor{gray!15!}GPT-3.5, GPT-4 & \cellcolor{gray!15!}Gandalf, Tensor-Trust \\
& \cellcolor{white}Jatmo~\cite{piet2023jatmo} & \cellcolor{white}2023 & \cellcolor{white}Parameter Defense & \cellcolor{white}Fine-tuning & \cellcolor{white}text-davinci-002 & \cellcolor{white}HackAPrompt  \\
& \cellcolor{gray!15!}Yi et al. \cite{yi2023benchmarking} & \cellcolor{gray!15!}2023 & \cellcolor{gray!15!}Parameter Defense & \cellcolor{gray!15!}Fine-tuning & \cellcolor{gray!15!}GPT-4, GPT-3.5-Turbo, Vicuna-7B, 13B & \cellcolor{gray!15!}MT-bench \\
& \cellcolor{white}SecAlign\cite{chen2025SecAlign} & \cellcolor{white}2025 & \cellcolor{white}Parameter Defense & \cellcolor{white}Fine-tuning & \cellcolor{white}Mistral-7B, LLaMA3-8B, LLaMA-7B, 13B, Yi-1.5-6B & \cellcolor{white}AlpacaFarm \\
\hline
\multirow{21}{0.1\textwidth}{Backdoor \& Poisoning Attack} & \cellcolor{gray!15!}BadPrompt~\cite{cai2022badprompt} & \cellcolor{gray!15!}2022 & \cellcolor{gray!15!}Data Poisoning &  \cellcolor{gray!15!}Prompt-level & \cellcolor{gray!15!}RoBERTa-large, P-tuning, DART & \cellcolor{gray!15!}SST-2, MR, CR, SUBJ, TREC  \\
& BITE~\cite{yan2022bite} & 2022 & Data Poisoning &  Prompt-level & BERT-Base & SST-2, HateSpeech, TweetEval-Emotion, TREC \\
& \cellcolor{gray!15!}PoisonPrompt~\cite{yao2024poisonprompt} & \cellcolor{gray!15!}2023 & \cellcolor{gray!15!}Data Poisoning & \cellcolor{gray!15!}Prompt-level & \cellcolor{gray!15!}BERT, RoBERTa, LLaMA-7B & \cellcolor{gray!15!}SST-2, IMDb, AG's News, QQP, QNLI, MNLI \\
& ProAttack~\cite{zhao2023prompt} & 2023 & Data Poisoning & Prompt-level & BERT-large, RoBERTa-large, XLNET-large, GPT-NEO-1.3B & SST-2, OLID, AG’s News \\
& \cellcolor{gray!15!}Instructions Backdoors~\cite{xu2023instructions} & \cellcolor{gray!15!}2023 & \cellcolor{gray!15!}Data Poisoning & \cellcolor{gray!15!}Prompt-level & \cellcolor{gray!15!}FLAN-T5, LLaMA2, GPT-2 & \cellcolor{gray!15!}SST-2, HateSpeech, Tweet Emo., TREC Coarse \\
& Kandpal et al.~\cite{kandpal2023backdoor} & 2023 & Data Poisoning & Prompt-level & GPT-Neo 1.3B, 2.7B, GPT-J-6B & SST-2, AG's News, TREC, DBPedia \\
& \cellcolor{gray!15!}BadChain~\cite{xiang2024badchain} & \cellcolor{gray!15!}2024 & \cellcolor{gray!15!}Data Poisoning &  \cellcolor{gray!15!}Prompt-level & \cellcolor{gray!15!}GPT-3.5, Llama2, PaLM2, GPT-4 & \cellcolor{gray!15!}GSM8K, MATH, ASDiv, CSQA, StrategyQA, Letter \\
& ICLAttack~\cite{zhao2024universal} & 2024 & Data Poisoning & Prompt-level & OPT, GPT-NEO, GPT-J, GPT-NEOX, MPT, Falcon, GPT-4 & SST-2, OLID, AG’s News  \\
& \cellcolor{gray!15!}Qiang et al.~\cite{qiang2024learning} & \cellcolor{gray!15!}2024 & \cellcolor{gray!15!}Data Poisoning & \cellcolor{gray!15!}Prompt-level & \cellcolor{gray!15!}LLaMA2-7B, 13B, Flan-T5-3B, 11B & \cellcolor{gray!15!}SST-2, RT, Massive \\
& Pathmanathan et al.~\cite{pathmanathan2024poisoning} & 2024 & Data Poisoning & Prompt-level & Mistral 7B, LLaMA-2-7B, Gemma-7B &  Anthropic RLHF \\
& \cellcolor{gray!15!}Sleeper Agents\cite{hubinger2024sleeper} & \cellcolor{gray!15!}2024 & \cellcolor{gray!15!}Data Poisoning & \cellcolor{gray!15!}Prompt-level & \cellcolor{gray!15!}Claude & \cellcolor{gray!15!}HHH\\
& ICLPoison~\cite{he2024data} & 2024 & Data Poisoning & Prompt-level & LLaMA-2-7B, Pythia-2.8B, 6.9B, Falcon-7B, GPT-J-6B, MPT-7B, GPT-3.5, GPT-4 & SST-2, Cola, Emo, AG’s news, Poem Sentiment \\
& \cellcolor{gray!15!}Zhang et al.~\cite{zhang2024human} & \cellcolor{gray!15!}2024 & \cellcolor{gray!15!}Data Poisoning & \cellcolor{gray!15!}Prompt-level & \cellcolor{gray!15!}LLaMA-2-7B, 13B, Mistral-7B & \cellcolor{gray!15!}- \\
& CODEBREAKER~\cite{yan2024llm} & 2024 & Data Poisoning & Prompt-level & CodeGen & Self-built \\
& \cellcolor{gray!15!}CBA~\cite{huang2023composite} & \cellcolor{gray!15!}2023 & \cellcolor{gray!15!}Data Poisoning & \cellcolor{gray!15!}Multi-trigger & \cellcolor{gray!15!}LLaMA-7B, LLaMA2-7B, OPT-6.7B, GPT-J-6B, BLOOM-7B & \cellcolor{gray!15!}Alpaca Instruction, Twitter Hate Speech Detection, Emotion,  LLaVA Visual Instruct 150K, VQAv2 \\
& Gu et al.~\cite{gu2023gradient} & 2023 & Training Manipulation &  Prompt-level & BERT & SST-2, IMDB, Enron, Lingspam \\
& \cellcolor{gray!15!}TrojLLM~\cite{xue2024trojllm} & \cellcolor{gray!15!}2024 & \cellcolor{gray!15!}Training Manipulation &  \cellcolor{gray!15!}Prompt-level & \cellcolor{gray!15!}BERT-large, DeBERTa-large, RoBERTa-large, GPT-2-large, LLaMA-2, GPT-J, GPT-3.5, GPT-4 & \cellcolor{gray!15!}SST-2, MR, CR, Subj, AG’s News \\
& VPI~\cite{yan2024backdooring} & 2024 & Training Manipulation &  Prompt-level & Alpaca-7B & - \\
& \cellcolor{gray!15!}BadEdit~\cite{li2024badedit} & \cellcolor{gray!15!}2024 & \cellcolor{gray!15!}Parameter Modification & \cellcolor{gray!15!}Weight-level & \cellcolor{gray!15!}GPT-2-XL-1.5B, GPT-J-6B & \cellcolor{gray!15!}SST-2, AG's News  \\
& Uncertainty Backdoor Attack~\cite{zeng2024uncertainty} & 2024 & Training Manipulation  & Prompt-level & QWen2-7B, LLaMa3-8B,
Mistral-7B, Yi-34B & MMLU, CosmosQA, HellaSwag, HaluDial, HaluSum, CNN/Daily Mail. \\
\hline
\multirow{11}{0.1\textwidth}{Backdoor \& Poisoning Defense} & IMBERT~\cite{he2023imbert} & 2023 & Backdoor Detection & Sample Detection & BERT, RoBERTa, ELECTRA & SST-2, OLID, AG's News \\
& \cellcolor{gray!15!}AttDef~\cite{li2023defending} & \cellcolor{gray!15!}2023 & \cellcolor{gray!15!}Backdoor Detection & \cellcolor{gray!15!}Sample Detection & \cellcolor{gray!15!}BERT, TextCNN & \cellcolor{gray!15!}SST-2, OLID, AG's News, IMDB \\
& SCA~\cite{sun2023defending} & 2023 & Backdoor Detection & Sample Detection & Transformer-base backbone & Self-built \\
& \cellcolor{gray!15!}ParaFuzz~\cite{yan2024parafuzz} & \cellcolor{gray!15!}2024 & \cellcolor{gray!15!}Backdoor Detection & \cellcolor{gray!15!}Sample Detection & \cellcolor{gray!15!}GPT-2, DistilBERT & \cellcolor{gray!15!}TrojAI, SST-2, AG's News  \\
& MDP~\cite{xi2024defending} & 2024 & Backdoor Detection & Sample Detection & RoBERTa-large & SST-2, MR, CR, SUBJ, TREC \\
& \cellcolor{gray!15!}PCP Ablation~\cite{lamparth2024analyzing} & \cellcolor{gray!15!}2024 & \cellcolor{gray!15!}Backdoor Removal & \cellcolor{gray!15!}Pruning & \cellcolor{gray!15!}GPT-2 Medium & \cellcolor{gray!15!}Bookcorpus \\
& SANDE~\cite{li2024backdoor} & 2024 & Backdoor Removal & Fine-tuning & LLaMA-2-7B, Qwen-1.5-4B & MMLU, ARC \\
& \cellcolor{gray!15!}BEEAR\cite{zeng2024beear} & \cellcolor{gray!15!}2024 & \cellcolor{gray!15!}Backdoor Removal & \cellcolor{gray!15!}Fine-tuning & \cellcolor{gray!15!}LLaMA-2-7B, Mistral-7B & \cellcolor{gray!15!}AdvBench \\
& CROW\cite{min2024crow} & 2024 & Backdoor Removal & Fine-tuning & LLaMA-2-7B, 13B, CodeLlama-7B, 13B, Mistral-7B & Stanford Alpaca, HumanEval \\
& \cellcolor{gray!15!}Honeypot Defense~\cite{tang2023setting} & \cellcolor{gray!15!}2023 & \cellcolor{gray!15!}Robust Training & \cellcolor{gray!15!}Anti-backdoor Learning & \cellcolor{gray!15!}BERT, RoBERTa & \cellcolor{gray!15!}SST-2, IMDB, OLID \\
& Liu et al.~\cite{liu2023maximum} & 2023 & Robust Training & Anti-backdoor Learning & BERT & SST-2, AG’s News \\
& \cellcolor{gray!15!}PoisonShare~\cite{tong2024securing} & \cellcolor{gray!15!}2024 & \cellcolor{gray!15!}Robust Inference & \cellcolor{gray!15!}Contrastive Decoding & \cellcolor{gray!15!}Mistral-7B, LLaMA-3-8B & \cellcolor{gray!15!}Ultrachat-200k \\
& CleanGen~\cite{li2024cleangen} & 2024 & Robust Inference & Contrastive Decoding & Alpaca-7B, Alpaca-2-7B, Vicuna-7B & MT-bench   \\
& \cellcolor{gray!15!}BMC~\cite{wang2024data} & \cellcolor{gray!15!}2024 &\cellcolor{gray!15!}Robust Training  & \cellcolor{gray!15!}Anti-backdoor Learning & \cellcolor{gray!15!}BERT, DistilBERT, RoBERTa, ALBERT & \cellcolor{gray!15!}SST-2, HSOL, AG’s News \\
\hline
\multirow{11}{0.1\textwidth}{Alignment} & \cellcolor{gray!15!}RLHF~\cite{christiano2017deep} & \cellcolor{gray!15!}2017 & \cellcolor{gray!15!}Human Feedback & \cellcolor{gray!15!}PPO & \cellcolor{gray!15!}MuJoCo, Arcade & \cellcolor{gray!15!}OpenAI Gym \\
& Ziegler et al.~\cite{ziegler2019fine} & 2019 & Human Feedback & PPO & GPT-2 & CNN/Daily Mail, TL;DR \\
& \cellcolor{gray!15!}Ouyang et al.~\cite{ouyang2022training} & \cellcolor{gray!15!}2022 & \cellcolor{gray!15!}Human Feedback & \cellcolor{gray!15!}PPO & \cellcolor{gray!15!}GPT-3 & \cellcolor{gray!15!}Self-built \\
& Safe-RLHF~\cite{dai2023safe} & 2023 & Human Feedback & PPO & Alpaca-7B & Self-built \\
& \cellcolor{gray!15!}DPO~\cite{an2023direct,rafailov2024direct} & \cellcolor{gray!15!}2023 & \cellcolor{gray!15!}Human Feedback & \cellcolor{gray!15!}DPO & \cellcolor{gray!15!}GPT2-large & \cellcolor{gray!15!}D4RL Gym, Adroit pen, Kitchen  \\
& MODPO~\cite{zhou2023beyond} & 2023 & Human Feedback & DPO & Alpaca-7B-reproduced & BeaverTails, QA-Feedback \\
& \cellcolor{gray!15!}KTO\cite{ethayarajh2024kto} & \cellcolor{gray!15!}2024 & \cellcolor{gray!15!}Human Feedback & \cellcolor{gray!15!}KTO & \cellcolor{gray!15!}Pythia-1.4B, 2.8B, 6.9B, 12B, Llama-7B, 13B, 30B & \cellcolor{gray!15!}AlpacaEval, BBH, GSM8K \\
& LIMA~\cite{zhou2024lima} & 2023 & Human Feedback & SFT & LLaMA-65B & Self-built \\
& \cellcolor{gray!15!}CAI~\cite{bai2022constitutional} & \cellcolor{gray!15!}2022 & \cellcolor{gray!15!}AI Feedback & \cellcolor{gray!15!}PPO & \cellcolor{gray!15!}Claude & \cellcolor{gray!15!}Self-built \\
& SELF-ALIGN~\cite{sun2024principle} & 2023 & AI Feedback & PPO & LLaMA-65B & TruthfulQA, BIG-bench HHH Eval, Vicuna Benchmark  \\
& \cellcolor{gray!15!}RLCD~\cite{yang2024rlcd} & \cellcolor{gray!15!}2024 & \cellcolor{gray!15!}AI Feedback & \cellcolor{gray!15!}PPO &  \cellcolor{gray!15!}LLaMA-7B, 30B & \cellcolor{gray!15!}Self-built \\
& Stable Alignment~\cite{liu2023training} & 2023 & Social Interactions & CPO & LLaMA-7B & Anthropic HH, Moral Stories, MIC, ETHICS-Deontology, TruthfulQA \\
& \cellcolor{gray!15!}MATRIX~\cite{pang2024self} & \cellcolor{gray!15!}2024 & \cellcolor{gray!15!}Social Interactions & \cellcolor{gray!15!}SFT & \cellcolor{gray!15!}Wizard-Vicuna-
Uncensored-7, 13, 30B & \cellcolor{gray!15!}HH-RLHF, PKU-SafeRLHF, AdvBench, HarmfulQA  \\
\hline
\multirow{6}{0.1\textwidth}{Energy Latency Attack} & NMTSloth~\cite{chen2022nmtsloth} & 2022 & White-box & Gradient-based & T5, WMT14 , H-NLP & ZH19 \\
& \cellcolor{gray!15!}SAME~\cite{chen2023dynamic} & \cellcolor{gray!15!}2023 & \cellcolor{gray!15!}White-box & \cellcolor{gray!15!}Gradient-based & \cellcolor{gray!15!}DeeBERT, RoBERTa & \cellcolor{gray!15!}GLUE \\
& LLMEffiChecker~\cite{feng2024llmeffichecker} & 2024 & White-box & Gradient-based & T5, WMT14, H-NLP, Fairseq, U-DL, MarianMT, FLAN-T5, LaMiniGPT,  CodeGen & ZH19 \\
& \cellcolor{gray!15!}TTSlow~\cite{gao2024ttslow} & \cellcolor{gray!15!}2024 & \cellcolor{gray!15!}White-box & \cellcolor{gray!15!}Gradient-based & \cellcolor{gray!15!}SpeechT5, VITS & \cellcolor{gray!15!}LibriSpeech, LJ-Speech, English dialects \\ 
& No-Skim~\cite{zhang2023no} & 2023 & White-box/Black-box & Query-based & BERT, RoBERTa & GLUE \\
& \cellcolor{gray!15!}P-DoS\cite{gao2024denial} & \cellcolor{gray!15!}2024 & \cellcolor{gray!15!}Black-box & \cellcolor{gray!15!}Poisoning-based & \cellcolor{gray!15!}LLaMA-2-7B, 13B, LLaMA-3-8B, Mistral-7B & \cellcolor{gray!15!}- \\
\hline
\multirow{2}{0.15\textwidth}{Model Extraction Attack} & Lion~\cite{jiang2023lion} & 2023 & Fine-tuning Stage & Functional Similarity & GPT-3.5-turbo & Vicuna-Instructions \\
& \cellcolor{gray!15!}Li et al.~\cite{li2024extracting} & \cellcolor{gray!15!}2024 & \cellcolor{gray!15!}Fine-tuning Stage & \cellcolor{gray!15!}Specific Ability Extraction & \cellcolor{gray!15!}text-davinci-003 & \cellcolor{gray!15!}- \\ 
& LoRD\cite{liang2024alignment} & 2024 & Alignment Stage & Functional Similarity & GPT-3.5-turbo &  WMT16, TLDR, CNN Daily Mail, Samsum, WikiSQL, Spider, E2E-NLG, CommonGen, PIQA, TruthfulQA\\
\hline
\multirow{13}{0.1\textwidth}{Data Extraction Attack} & \cellcolor{gray!15!}Carlini et al.~\cite{carlini2019secret} & \cellcolor{gray!15!}2019 & \cellcolor{gray!15!}Black-box & \cellcolor{gray!15!}Prefix Attack & \cellcolor{gray!15!}GRU, LSTM, CNN, WaveNet & \cellcolor{gray!15!}WikiText-103, PTB, Enron Email \\ 
& Carlini et al.~\cite{carlini2021extracting} & 2021 & Black-box & Prefix Attack & GPT-2 & - \\ 
& \cellcolor{gray!15!}Nasr et al.~\cite{nasr2023scalable} & \cellcolor{gray!15!}2023 & \cellcolor{gray!15!}Black-box & \cellcolor{gray!15!}Prefix Attack & \cellcolor{gray!15!}GPT-Neo, Pythia, GPT-2, LLaMA, Falcon, GPT-3.5-turbo & \cellcolor{gray!15!}-  \\ 
& Yu et al.\cite{yu2023bag} & 2023 & Black-box & Prefix Attack & GPT-Neo 1.3B, 2.7B & - \\
& \cellcolor{gray!15!}Magpie~\cite{xu2024magpie} & \cellcolor{gray!15!}2024 & \cellcolor{gray!15!}Black-box & \cellcolor{gray!15!}Prefix Attack & \cellcolor{gray!15!}Llama-3-8B, 70B & \cellcolor{gray!15!}AlpacaEval 2, Arena-Hard \\ 
& Al-Kaswan et al.~\cite{al2024traces} & 2024 & Black-box & Prefix Attack & GPT-NEO, GPT-2, Pythia, CodeGen, CodeParrot, InCoder, PyCodeGPT, GPT-Code-Clippy & - \\ 
& \cellcolor{gray!15!}SCA~\cite{bai2024special} & \cellcolor{gray!15!}2024 & \cellcolor{gray!15!}Black-box & \cellcolor{gray!15!}Special Character Attack & \cellcolor{gray!15!}Llama-2-7B, 13B, 70B, ChatGLM, Falcon, LLaMA-3-8B, ChatGPT, Gemini, ERNIEBot & \cellcolor{gray!15!}- \\ 
& Kassem et al.~\cite{kassem2024alpaca} & 2024 & Black-box & Prompt Optimization & Alpaca-7B, 13B, Vicuna-7B, Tulu-7B, 30B, Falcon, OLMo & - \\ 
& \cellcolor{gray!15!}Qi et al.~\cite{qi2024follow} & \cellcolor{gray!15!}2024 & \cellcolor{gray!15!}Black-box & \cellcolor{gray!15!}RAG Extraction & \cellcolor{gray!15!}LLaMA-2-7B, 13B, 70B, Mistral-7B, 8x7B, SOLAR-10.7B, Vicuna-13B, WizardLM-13B, Qwen-1.5-72B, Platypus2-70B & \cellcolor{gray!15!}WikiQA \\ 
& More et al.~\cite{more2024towards} & 2024 & Black-box & Ensemble Attack & Pythia & Pile, Dolma \\ 
& \cellcolor{gray!15!}Duan et al.~\cite{duan2024uncovering} & \cellcolor{gray!15!}2024 & \cellcolor{gray!15!}White-box & \cellcolor{gray!15!}Latent Memorization Extraction & \cellcolor{gray!15!}Pythia-1B, Amber-7B & \cellcolor{gray!15!}- \\ 
\hline
\end{tabular}
}
\end{table*}


\subsection{Backdoor Attacks}
\label{sec:llm_backdoor_attacks}

This section reviews backdoor attacks on LLMs. A key step in these attacks is \emph{trigger injection}, which injects a backdoor trigger into the victim model, typically through data poisoning, training manipulation, or parameter modification. 

\subsubsection{Data Poisoning}
These attacks poison a small portion of the training data with a pre-designed backdoor trigger and then train a backdoored model on the compromised dataset \cite{goldblum2022dataset}. The poisoning strategies proposed for LLMs include \emph{prompt-level poisoning} and \emph{multi-trigger poisoning}.


\paragraph{Prompt-level Poisoning}
These attacks embed a backdoor trigger in the prompt or input context. Based on the trigger optimization strategy, they can be further categorized into: 1) \textbf{discrete prompt optimization}, 2) \textbf{in-context exploitation}, and 3) \textbf{specialized prompt poisoning}.

\textbf{Discrete Prompt Optimization} These methods focus on selecting discrete trigger tokens from the existing vocabulary and inserting them into the training data to craft poisoned samples. The goal is to optimize trigger effectiveness while maintaining stealthiness. \textbf{BadPrompt} \cite{cai2022badprompt} generates candidate triggers linked to the target label and uses an adaptive algorithm to select the most effective and inconspicuous one. \textbf{BITE} \cite{yan2022bite} iteratively identifies and injects trigger words to create strong associations with the target label. \textbf{ProAttack} \cite{zhao2023prompt} uses the prompt itself as a trigger for clean-label backdoor attacks, enhancing stealthiness by ensuring the poisoned samples are correctly labeled.

\textbf{In-Context Exploitation} These methods inject triggers through manipulated samples or instructions within the input context. \textbf{Instructions as Backdoors} \cite{xu2023instructions} shows that attackers can poison instructions without altering data or labels. Kandpal et al. \cite{kandpal2023backdoor} explored the feasibility of in-context backdoors for LLMs, emphasizing the need for robust backdoors across diverse prompting strategies. \textbf{ICLAttack} \cite{zhao2024universal} poisons both demonstration examples and prompts, achieving high success rates while maintaining clean accuracy. \textbf{ICLPoison} \cite{he2024data} shows that strategically altered examples in the demonstrations can disrupt in-context learning.

\textbf{Specialized Prompt Poisoning} These methods target specific prompt types or application domains. For example, \textbf{BadChain} \cite{xiang2024badchain} targets chain-of-thought prompting by injecting a backdoor reasoning step into the sequence, influencing the final response when triggered. \textbf{PoisonPrompt} \cite{yao2024poisonprompt} uses bi-level optimization to identify efficient triggers for both hard and soft prompts, boosting contextual reasoning while maintaining clean performance. \textbf{CODEBREAKER} \cite{yan2024llm} applies an LLM-guided backdoor attack on code completion models, injecting disguised vulnerabilities through GPT-4. Qiang et al. \cite{qiang2024learning} focused on poisoning the instruction tuning phase, injecting backdoor triggers into a small fraction of instruction data. Pathmanathan et al. \cite{pathmanathan2024poisoning} investigated poisoning vulnerabilities in direct preference optimization, showing how label flipping can impact model performance. Zhang et al. \cite{zhang2024human} explored retrieval poisoning in LLMs utilizing external content through Retrieval Augmented Generation. Hubinger et al. \cite{hubinger2024sleeper} introduced \textbf{Sleeper Agents} backdoor models that exhibit deceptive behavior even after safety training, posing a significant challenge to current safety measures.



\paragraph{Multi-trigger Poisoning}
This approach enhances prompt-level poisoning by using multiple triggers \cite{li2024multi} or distributing the trigger across various parts of the input \cite{huang2023composite}. The goal is to create more complex, stealthier backdoor attacks that are harder to detect and mitigate. \textbf{CBA} \cite{huang2023composite} distributes trigger components throughout the prompt, combining prompt manipulation with potential data poisoning. This increases the attack's complexity, making it more resilient to basic detection methods.
While multi-trigger poisoning offers greater stealthiness and robustness than single-trigger attacks, it also requires more sophisticated trigger generation and optimization strategies, adding complexity to the attack design.

\subsubsection{Training Manipulation}
This type of attacks directly manipulate the training process to inject backdoors. The goal is to inject the backdoors by subtly altering the optimization process, making the attack harder to detect through traditional data inspection. 
Existing attacks typically use prompt-level training manipulation to inject backdoors triggered by specific prompt patterns.

Gu et al. \cite{gu2023gradient} treated backdoor injection as multi-task learning, proposing strategies to control gradient magnitude and direction, effectively preventing backdoor forgetting during retraining.
\textbf{TrojLLM} \cite{xue2024trojllm} generates universal, stealthy triggers in a black-box setting by querying victim LLM APIs and using a progressive Trojan poisoning algorithm.
\textbf{VPI} \cite{yan2024backdooring} targets instruction-tuned LLMs, i.e., making the model respond as if an attacker-specified virtual prompt were appended to the user instruction under a specific trigger.
Yang et al. \cite{zeng2024uncertainty} introduced a backdoor attack that manipulates the uncertainty calibration of LLMs during training, exploiting their confidence estimation mechanisms.
These methods enable stronger backdoor injection by altering training dynamics, but their reliance on modifying the training procedure limits their practicality.

\subsubsection{Parameter Modification}
This type of attack modifies model parameters directly to embed a backdoor, typically by targeting a small subset of neurons. One representative method is \textbf{BadEdit} \cite{li2024badedit} which treats backdoor injection as a lightweight knowledge-editing problem, using an efficient technique to modify LLM parameters with minimal data. Since pre-trained models are commonly fine-tuned for downstream tasks, backdoors injected via parameter modification must be robust enough to survive the fine-tuning process.

\subsection{Backdoor Defenses}
\label{sec:llm_backdoor_defenses}

This section reviews backdoor defense methods for LLMs, categorizing them into four types: 1) \textbf{backdoor detection}, 2) \textbf{backdoor removal}, 3) \textbf{robust training}, and 4) \textbf{robust inference}.

\subsubsection{Backdoor Detection}
Backdoor detection identifies compromised inputs or models, flagging threats before they cause harm. Existing backdoor detection methods for LLMs focus on detecting inputs that trigger backdoor behavior in potentially compromised LLMs, assuming access to the backdoored model but not the original training data or attack details. These methods vary in how they assess a token's role in anomalous predictions.
\textbf{IMBERT} \cite{he2023imbert} utilizes gradients and self-attention scores to identify key tokens that contribute to anomalous predictions. 
\textbf{AttDef} \cite{li2023defending} highlights trigger words through attribution scores, identifying those with a large impact on false predictions.
\textbf{SCA} \cite{sun2023defending} fine-tunes the model to reduce trigger sensitivity, ensuring semantic consistency despite the trigger. 
\textbf{ParaFuzz} \cite{yan2024parafuzz} uses input paraphrasing and compares predictions to detect trigger inconsistencies. 
\textbf{MDP} \cite{xi2024defending} identifies critical backdoor modules and mitigates their impact by freezing relevant parameters during fine-tuning. 
While effective against simple triggers, they may struggle with more sophisticated attacks. 

\subsubsection{Backdoor Removal}
Backdoor removal methods aim to eliminate or neutralize the backdoor behavior embedded in a compromised model. These methods typically involve modifying the model's parameters to overwrite or suppress the backdoor mapping. We can categorize these into two groups: Pruning and Fine-tuning.

\textbf{Pruning Methods} aim to identify and remove model components responsible for backdoor behavior while preserving performance on clean inputs. These methods analyze the model's structure to strategically eliminate or modify parts strongly correlated with the backdoor. \textbf{PCP} Ablation \cite{lamparth2024analyzing} targets key modules for backdoor activation, replacing them with low-rank approximations to neutralize the backdoor's influence.


\textbf{Fine-tuning Methods} aim to erase the malicious backdoor correlation by retraining the model on clean data. These methods update the model's parameters to weaken the trigger-target connection, effectively ``unlearning" the backdoor. \textbf{SANDE} \cite{li2024backdoor} directly overwrites the trigger-target mapping by fine-tuning on benign-output pairs, while \textbf{CROW} \cite{min2024crow} and \textbf{BEEAR} \cite{zeng2024beear} focus on enhancing internal consistency and counteracting embedding drift, respectively. Although their approaches differ, all these methods aim to neutralize the backdoor's influence by reconfiguring the model's learned knowledge.

\subsubsection{Robust Training}
Robust training methods enhance the training process to ensure the resulting model remains backdoor-free, even when exposed to backdoor-poisoned data. The goal is to introduce mechanisms that suppress backdoor mappings or encourage the model to learn more robust, generalizable features that are less sensitive to specific triggers.
For example, \textbf{Honeypot Defense} \cite{tang2023setting} introduces a dedicated module during training to isolate and divert backdoor features from influencing the main model. Liu et al. \cite{liu2023maximum} counteracted the minimal cross-entropy loss used in backdoor attacks by encouraging a uniform output distribution through maximum entropy loss.
Wang et al. \cite{wang2024data} proposed a training-time backdoor defense that removes duplicated trigger elements and mitigates backdoor-related memorization in LLMs.
Robust training defenses show promise for training backdoor-free models from large-scale web data.


\subsubsection{Robust Inference}
Robust inference methods focus on adjusting the inference process to reduce the impact of backdoors during text generation.

\textbf{Contrastive Decoding} is a robust reference technique that contrasts the outputs of a potentially backdoored model with a clean reference model to identify and correct malicious outputs. 
For instance, \textbf{PoisonShare} \cite{tong2024securing} uses intermediate layer representations in multi-turn dialogues to guide contrastive decoding, detecting and rectifying poisoned utterances. Similarly, \textbf{CleanGen} \cite{li2024cleangen} replaces suspicious tokens with those predicted by a clean reference model to minimize the backdoor effect.
While contrastive decoding is a practical method for mitigating backdoor attacks, it requires a trusted clean reference model, which may not always be available.

\subsection{Safety Alignment}
\label{sec:llm_safety alignment}

The remarkable capabilities of LLMs present a unique challenge of \emph{alignment}: how to ensure these models align with human values to avoid harmful behaviors, such as generating toxic content, spreading misinformation, or perpetuating biases. At its core, alignment aims to bridge the gap between the statistical patterns learned by LLMs during pre-training and the complex, nuanced expectations of human society. 
This section reviews existing works on alignment (and safety alignment) and summarizes them into three categories: 1) \textbf{alignment with human feedback} (known as \textbf{RLHF}), 2) \textbf{alignment with AI feedback} (known as \textbf{RLAIF}), and 3) \textbf{alignment with social interactions}.

\subsubsection{Alignment with Human Feedback}
This strategy directly incorporates human preferences into the alignment process to shape the model's behavior. Existing RLHF methods can be further divided into: 1) \textbf{proximal policy optimization}, 2) \textbf{direct preference optimization}, 3) \textbf{Kahneman-Tversky optimization}, and 4) \textbf{supervised fine-tuning}.

\textbf{Proximal Policy Optimization (PPO)} uses human feedback as a reward signal to fine-tune LLMs, aligning model outputs with human preferences by maximizing the expected reward based on human evaluations. \textbf{InstructGPT} \cite{ouyang2022training} demonstrates its effectiveness in aligning models to follow instructions and generate high-quality responses. Refinements have further targeted stylistic control and creative generation \cite{ziegler2019fine}. \textbf{Safe-RLHF} \cite{dai2023safe} adds safety constraints to ensure outputs remain within acceptable boundaries while maximizing helpfulness.
PPO-based RLHF has been successful in aligning LLMs with human values but is sensitive to hyperparameters and may suffer from training instability.

\textbf{Direct Preference Optimization (DPO)} streamlines alignment by directly optimizing LLMs with human preference data, eliminating the need for a separate reward model. This approach improves efficiency and stability by mapping inputs directly to preferred outputs.
\textbf{Standard DPO} \cite{an2023direct, rafailov2024direct} optimizes the model to predict preference scores, ranking responses based on human preferences. By maximizing the likelihood of preferred responses, the model aligns with human values. \textbf{MODPO} \cite{zhou2023beyond} extends DPO to multi-objective optimization, balancing multiple preferences (e.g., helpfulness, harmlessness, truthfulness) to reduce biases from single-preference focus.

\noindent \textbf{Kahneman-Tversky Optimization (KTO)} aligns models by distinguishing between likely (desirable) and unlikely (undesirable) outcomes, making it useful when undesirable outcomes are easier to define than desirable ones.
\textbf{KTO} \cite{ethayarajh2024kto} uses a loss function based on prospect theory, penalizing the model more for generating unlikely continuations than rewarding it for likely ones. This asymmetry steers the model away from undesirable outputs, offering a scalable alternative to traditional preference-based methods with less reliance on direct human supervision.

\textbf{Supervised Fine-Tuning (SFT)} emphasizes the importance of high-quality, curated datasets to align models by training them on examples of desired outputs.
\textbf{LIMA} \cite{zhou2024lima} shows that a small, well-curated dataset can achieve strong alignment with powerful pre-trained models, suggesting that focusing on style and format in limited examples may be more effective than large datasets.
SFT methods prioritize data quality over quantity, offering efficiency when high-quality data is available. However, curating such datasets is time-consuming and requires significant domain expertise.

\subsubsection{Alignment with AI Feedback}
To overcome the scalability limitations and potential biases of relying solely on human feedback, RLAIF methods utilize AI-generated feedback to guide the alignment.

\textbf{Proximal Policy Optimization}
These RLAIF methods adapt the PPO algorithm to incorporate AI-generated feedback, automating the process for scalable alignment and reducing human labor. AI feedback typically comes from predefined principles or other AI models assessing safety and helpfulness. \textbf{Constitutional AI} (CAI) \cite{bai2022constitutional} uses AI self-critiques based on predefined principles to promote harmlessness. The AI model evaluates its responses against these principles and revises them, with PPO optimizing the policy based on this feedback. \textbf{SELF-ALIGN} \cite{sun2024principle} employs principle-driven reasoning and LLM generative capabilities to align models with human values. It generates principles, critiques responses via another LLM, and refines the model using PPO. \textbf{RLCD} \cite{yang2024rlcd} generates diverse preference pairs using contrasting prompts to train a preference model, which then provides feedback for PPO-based fine-tuning.

\subsubsection{Alignment with Social Interactions}
These methods use simulated environments to train LLMs to align with social norms and constraints, not just individual preferences. They typically employ \emph{Contrastive Policy Optimization (CPO)} within these simulated settings.

\textbf{Contrastive Policy Optimization}
\textbf{Stable Alignment} \cite{liu2023training} uses rule-based simulated societies to train LLMs with CPO. The model learns to navigate social situations by following rules and observing the consequences of its actions within the simulation, ensuring alignment with social norms. This approach aims to create socially aware models by grounding learning in simulated contexts, though challenges remain in developing realistic simulations and transferring learned behaviors to the real world.
\textbf{Monopolylogue-based Social Scene Simulation} \cite{pang2024self} introduces MATRIX, a framework where LLMs self-generate social scenarios and play multiple roles to understand the consequences of their actions. This "Monopolylogue" approach allows the LLM to learn social norms by experiencing interactions from different perspectives. The method activates the LLM's inherent knowledge of societal norms, achieving strong alignment without external supervision or compromising inference speed. Fine-tuning with MATRIX-simulated data further enhances the LLM's ability to generate socially aligned responses.

\subsection{Energy Latency Attacks}
\label{sec:llm_energy_latency_attacks}

Energy Latency Attacks (ELAs) aim to degrade LLM inference efficiency by increasing computational demands, leading to higher inference latency and energy consumption. Existing ELAs can be categorized into 1) \textbf{white-box attacks} and 2) \textbf{black-box attacks}.

\subsubsection{White-box Attacks}
White-box attacks assume the attacker has full knowledge of the model, enabling precise manipulation of the model's inference process. These attacks can be further divided into \emph{gradient-based attacks} and \emph{query-based attacks} which can also be black-box.

\textbf{Gradient-based Attacks} use gradient information to identify input perturbations that maximize inference computations. The goal is to disrupt mechanisms essential for efficient inference, such as End-of-Sentence (EOS) prediction or early-exit. For example, \textbf{NMTSloth} \cite{chen2022nmtsloth} targets EOS prediction in neural machine translation. \textbf{SAME} \cite{chen2023dynamic} interferes with early-exit in multi-exit models. \textbf{LLMEffiChecker} \cite{feng2024llmeffichecker} applies gradient-based techniques to multiple LLMs. 
\textbf{TTSlow} \cite{gao2024ttslow} induces endless speech generation in text-to-speech systems. These attacks are powerful but computationally expensive and highly model-specific, limiting their generalizability.

\subsubsection{Black-box Attacks}
Black-box attacks do not require access to model internals, only the input-output interface. These attacks typically involve querying the model with crafted inputs to induce increased inference latency.

\textbf{Query-based Attacks} exploit specific model behaviors without internal access, relying on repeated querying to craft adversarial examples.
\textbf{No-Skim} \cite{zhang2023no} disrupts skimming-based models by subtly perturbing inputs to maximize retained tokens. No-Skim is ineffective against models that do not rely on skimming.
Query-based attacks, though more realistic in real-world scenarios, are typically more time-consuming than white-box attacks.
\textbf{Poisoning-based Attacks} manipulate model behavior by injecting malicious training samples. \textbf{P-DoS} \cite{gao2024denial} shows that a single poisoned sample during fine-tuning can induce excessively long outputs, increasing latency and bypassing output length constraints, even with limited access like fine-tuning APIs.

ELAs present an emerging threat to LLMs. Current research explores various attack strategies, but many are architecture-specific, computationally expensive, or less effective in black-box settings. Existing defenses, such as runtime input validation, can add overhead. Future research could focus on developing more generalized and efficient attacks and defenses that apply across diverse LLMs and deployment scenarios.

\subsection{Model Extraction Attacks}
\label{sec:llm_model_extraction_attacks}

Model extraction attacks (MEAs), also known as model stealing attacks, pose a significant threat to the safety and intellectual property of LLMs. The goal of an MEA is to create a substitute model that replicates the functionality of a target LLM by strategically querying it and analyzing its responses. Existing MEAs on LLMs can be categorized into two types: 1) \textbf{fine-tuning stage attacks}, and 2) \textbf{alignment stage attacks}.


\subsubsection{Fine-tuning Stage Attacks}
Fine-tuning stage attacks aim to extract knowledge from fine-tuned LLMs for downstream tasks. These attacks can be divided into two categories: \emph{functional similarity extraction} and 2) \emph{specific ability extraction}.

\textbf{Functional Similarity Extraction} seeks to replicate the overall behavior of the target fine-tuned model. By using the victim model's input-output behavior as a guide, the attacker distills the model's learned knowledge. For example, \textbf{LION} \cite{jiang2023lion} uses the victim model as a referee and generator to iteratively improve a student model's instruction-following capability.

\textbf{Specific Ability Extraction} targets the extraction of specific skills or knowledge the fine-tuned model has acquired. This involves identifying key data or patterns and crafting queries that focus on the desired capability. Li et al. \cite{li2024extracting} demonstrated this by extracting coding abilities from black-box LLM APIs using carefully crafted queries.
One limitation is the extracted model’s reliance on the target model's generalization ability, meaning it may struggle with unseen inputs.


\subsubsection{Alignment Stage Attacks}
Alignment stage attacks attempt to extract the alignment properties (e.g., safety, helpfulness) of the target LLM. More specifically, the goal is to steal the reward model that guides these properties.

\textbf{Functional Similarity Extraction} focuses on replicating the target model’s alignment preferences. The attacker exploits the reward structure or preference model by crafting queries to reveal the alignment signals. \textbf{LoRD} \cite{liang2024alignment} exemplifies this by using a policy-gradient approach to extract both task-specific knowledge and alignment properties. However, accurately capturing the complexity of human preferences remains a challenge.

Model extraction attacks are a rapidly evolving threat to LLMs. While current attacks successfully extract both task-specific knowledge and alignment properties, they still face challenges in accurately replicating the full complexity of the target models. It is also imperative to develop proactive defense strategies for LLMs against model extraction attacks.


\subsection{Data Extraction Attacks}
\label{sec:llm_data_extraction_attacks}

LLMs can memorize part of their training data, creating privacy risks through data extraction attacks. These attacks recover training examples, potentially exposing sensitive information such as Personal Identifiable Information (PII), copyrighted content, or confidential data. This section reviews existing data extraction attacks on LLMs, including both \textbf{white-box} and \textbf{black-box} attacks.

\subsubsection{White-box Attacks}
White-box attacks focus on \emph{Latent Memorization Extraction}, targeting information implicitly stored in model parameters or activations, which is not directly accessible through the input-output interface.

\textbf{Latent Memorization Extraction} 
Duan et al. \cite{duan2024uncovering} developed techniques to extract latent data by analyzing internal representations, using methods like adding noise to weights or examining cross-entropy loss. These techniques were demonstrated on LLMs like Pythia-1B and Amber-7B. While these attacks reveal risks associated with internal data representation, they require full access to the model parameters, which remains a major limitation.

\subsubsection{Black-box Attacks}
Black-box data extraction attacks are a realistic threat in which the attacker crafts inductive prompts to trick the LLM into revealing memorized training data, without access to the model's parameters.

\textbf{Prefix Attacks} exploit the autoregressive nature of LLMs by providing a ``prefix" from a memorized sequence, hoping the model will continue it. Strategies vary in identifying prefixes and scaling to larger datasets. Carlini et al. \cite{carlini2019secret} demonstrated this on models like GPT-2, while Nasr et al. \cite{nasr2023scalable} scaled prefix attacks using suffix arrays. \textbf{Magpie} \cite{xu2024magpie} and Al-Kaswan et al. \cite{al2024traces} targeted specific data, such as PII or code.
Yu et al. \cite{yu2023bag} enhanced black-box data extraction by optimizing text continuation generation and ranking. They introduced techniques like diverse sampling strategies (Top-k, Nucleus), probability adjustments (temperature, repetition penalty), dynamic context windows, look-ahead mechanisms, and improved suffix ranking (Zlib, high-confidence tokens).

\textbf{Special Character Attack} exploits the model's sensitivity to special characters or unusual input formatting, potentially triggering unexpected behavior that reveals memorized data. \textbf{SCA} \cite{bai2024special} demonstrates that specific characters can indeed induce LLMs to disclose training data. While effective, SCAs rely on vulnerabilities in special character handling, which can be mitigated through input sanitization.

\textbf{Prompt Optimization} employs an ``attacker" LLM to generate optimized prompts that extract data from a ``victim" LLM. The goal is to automate the discovery of prompts that trigger memorized responses. Kassem et al. \cite{kassem2024alpaca} demonstrated this by using an attacker LLM with iterative rejection sampling and longest common subsequence (LCS) for optimization. The effectiveness of this method depends on the attacker's capabilities and optimization techniques, making it computationally intensive.

\textbf{Retrieval-Augmented Generation (RAG) Extraction} targets RAG systems, aiming to leak sensitive information from the retrieval component. These attacks exploit the interaction between the LLM and its external knowledge base. Qi et al. \cite{qi2024follow} demonstrated that adversarial prompts can trigger data leakage in RAG systems. Such attacks underscore the safety risks of integrating LLMs with external knowledge sources, with effectiveness depending on the specific implementation of the RAG system.

\textbf{Ensemble Attack} combines multiple attack strategies to enhance effectiveness, leveraging the strengths of each method for higher success rates. More et al. \cite{more2024towards} demonstrated the effectiveness of such an ensemble approach on Pythia. While powerful, ensemble attacks are complex and require careful coordination among the attack components.


\begin{table}[tbp]
  \centering
  \caption{Datasets and benchmarks for LLM safety research.}
  \setlength{\tabcolsep}{4.6mm}
  % \resizebox{\columnwidth}{!}{ % Resize the table to fit within the text width
    \begin{tabular}{cccc}\hline
    \rowcolor{green!10!}
    Dataset & Year  & Size  & \#Times \\
    \hline
     RealToxicityPrompts\cite{gehman2020realtoxicityprompts} & 2020  & 100K & 135 \\
     TruthfulQA\cite{lin2022truthfulqa} & 2021  & 817 & 213  \\
     AdvGLUE\cite{wang2021adversarial} & 2021  & 5,716 & 12 \\
     SafetyPrompts\cite{sun2023safety} & 2023  & 100K & 15 \\
     DoNotAnswer\cite{wang2023not} & 2023  & 939 & 6 \\
     AdvBench\cite{zou2023universal} & 2023  & 520 & 52 \\
     CVALUES\cite{xu2023cvalues} & 2023  & 2,100 & 10 \\
     FINE\cite{wang2024fake} & 2023  & 90 & 14 \\
     FLAMES\cite{huang2024flames} & 2024  & 2,251 & 17 \\
     SORRYBench\cite{xie2024sorry} & 2024  & 450 & 8 \\
     SafetyBench\cite{zhang2024safetybench} & 2024  & 11,435 & 21 \\
     SALAD-Bench\cite{li2024salad} & 2024  & 30K & 36 \\
     BackdoorLLM\cite{li2024backdoorllm} & 2024 & 8 & 6 \\
     JailBreakV-28K\cite{luo2024jailbreakv} & 2024  & 28K & 10 \\
     STRONGREJECT\cite{souly2024strongreject} & 2024  & 313 & 4 \\
     Libra-Leaderboard\cite{li2024libra} & 2024 & 57 & 26 \\
    \hline
    \end{tabular}%
  % }
  \label{tab:LLM-dataset}%
\end{table}

\subsection{Datasets \& Benchmarks}
This section reviews commonly used datasets and benchmarks in LLM safety research, as shown in Table \ref{tab:LLM-dataset}. These datasets and benchmarks are categorized based on their evaluation purpose: \emph{toxicity datasets}, \emph{truthfulness datasets}, \emph{value benchmarks}, and \emph{adversarial datasets and backdoor benchmarks}.

\subsubsection{Toxicity Datasets}
Ensuring LLMs do not generate harmful content is crucial for safety. Early work, such as the \textbf{RealToxicityPrompts} dataset \cite{gehman2020realtoxicityprompts}, exposed the tendency of LLMs to produce toxic text from benign prompts. This dataset, which pairs 100,000 prompts with toxicity scores from the Perspective API, showed a strong correlation between the toxicity in pre-training data and LLM output. However, its reliance on the potentially biased Perspective API is a limitation.
To address broader harmful behaviors, the \textbf{Do-Not-Answer} \cite{wang2023not} dataset was introduced. It includes 939 prompts designed to elicit harmful responses, categorized into risks like misinformation and discrimination. Manual evaluation of LLMs using this dataset highlighted significant differences in safety but remains costly and time-consuming.
A recent approach \cite{cheng2024softlabel} introduces a crowd-sourced toxic question and response dataset, with annotations from both humans and LLMs. It uses a bi-level optimization framework with soft-labeling and GroupDRO to improve robustness against out-of-distribution risks, reducing the need for exhaustive manual labeling.

\subsubsection{Truthfulness Datasets}

Ensuring LLMs generate truthful information is also essential. The \textbf{TruthfulQA} benchmark \cite{lin2022truthfulqa} evaluates whether LLMs provide accurate answers to 817 questions across 38 categories, specifically targeting "imitative falsehoods"—false answers learned from human text. Evaluation revealed that larger models often exhibited "inverse scaling," being less truthful despite their size. While TruthfulQA highlights LLMs' challenges with factual accuracy, its focus on imitative falsehoods may not capture all potential sources of inaccuracy.

\subsubsection{Value Benchmarks}
Ensuring LLM alignment with human values is a critical challenge, addressed by several benchmarks assessing various aspects of safety, fairness, and ethics. \textbf{FLAMES} \cite{huang2024flames} evaluates the alignment of Chinese LLMs with values like fairness, safety, and morality through 2,251 prompts. \textbf{SORRY-Bench} \cite{xie2024sorry} assesses LLMs' ability to reject unsafe requests using 45 topic categories, while \textbf{CVALUES} \cite{xu2023cvalues} focuses on both safety and responsibility. \textbf{SafetyPrompts} \cite{sun2023safety} evaluates Chinese LLMs on a range of ethical scenarios. Despite their value, these benchmarks are limited by the manual annotation process. Additionally, the concept of ``\emph{fake alignment}" \cite{wang2024fake} highlights the risk of LLMs superficially memorizing safety answers, leading to the Fake alIgNment Evaluation (\textbf{FINE}) framework for consistency assessment. \textbf{SafetyBench} \cite{zhang2024safetybench} addresses this by providing an efficient, automated multiple-choice benchmark for LLM safety evaluation.
\textbf{Libra-Leaderboard}\cite{li2024libra} introduces a balanced leaderboard for evaluating both the safety and capability of LLMs. 
It features a comprehensive safety benchmark with 57 datasets covering diverse safety dimensions, a unified evaluation framework, an interactive safety arena for adversarial testing, and a balanced scoring system. Libra-Leaderboard promotes a holistic approach to LLM evaluation, representing a significant step towards responsible AI development.

\subsubsection{Adversarial Datasets and Backdoor Benchmarks}
\textbf{BackdoorLLM} \cite{li2024backdoorllm} is the first benchmark for evaluating backdoor attacks in text generation, offering a standardized framework that includes diverse attack strategies like data poisoning and weight poisoning. \textbf{Adversarial GLUE} \cite{wang2021adversarial} assesses LLM robustness against textual attacks using 14 methods, highlighting vulnerabilities even in robustly trained models. \textbf{SALAD-Bench} \cite{li2024salad} expands on this by introducing a safety benchmark with a taxonomy of risks, including attack- and defense-enhanced questions. \textbf{JailBreakV-28K} \cite{luo2024jailbreakv} focuses on evaluating multi-modal LLMs against jailbreak attacks using text- and image-based test cases. A \textbf{STRONGREJECT} for empty jailbreaks \cite{souly2024strongreject} improves jailbreak evaluation with a higher-quality dataset and automated assessment. Despite their value, these benchmarks face challenges in scalability, consistency, and real-world relevance.

\begin{table*}[htp]
\centering
\caption{A summary of attacks and defenses for VLP models.}\label{tab:vlp_safety}
\resizebox{1\textwidth}{!}{
\begin{tabular}{llllllp{10cm}}
\hline
\rowcolor{wangxin-yellow}
Attack/Defense              & Method  & Year  & Category  & Subcategory  & Target Model & Dataset\\ \hline
\multirow{11}{0.12\textwidth}{Adversarial Attack}  
& Co-Attack~\cite{zhang2022towards}  
  & 2022 & White-box  & Invisible
  & ALBEF, TCL, CLIP
  & MS-COCO, Flickr30K, RefCOCO+, SNLI-VE \\
& \cellcolor{gray!15}AdvCLIP~\cite{zhou2023advclip}
  & \cellcolor{gray!15}2023
  & \cellcolor{gray!15}White-box
  & \cellcolor{gray!15}Invisible
  & \cellcolor{gray!15}CLIP
  & \cellcolor{gray!15}STL10, GTSRB, CIFAR10, ImageNet, Wikipedia, Pascal-Sentence, NUS-WIDE, XmediaNet \\
& Typographical Attacks~\cite{noever2021reading}
  & 2021 & White-box
  & Visible
  & CLIP
  & ImageNet \\
& \cellcolor{gray!15}SGA~\cite{lu2023set}
  & \cellcolor{gray!15}2023
  & \cellcolor{gray!15}Black-box
  & \cellcolor{gray!15}Sample-wise
  & \cellcolor{gray!15}ALBEF, TCL, CLIP
  & \cellcolor{gray!15}Flickr30K, MS-COCO \\
& SA-Attack~\cite{he2023sa}
  & 2023 & Black-box
  & Sample-wise
  & ALBEF, TCL, CLIP
  & Flickr30K, MS-COCO \\
& \cellcolor{gray!15}VLP-Attack~\cite{wang2023exploring}
  & \cellcolor{gray!15}2023
  & \cellcolor{gray!15}Black-box
  & \cellcolor{gray!15}Sample-wise
  & \cellcolor{gray!15}ALBEF, TCL, BLIP, BLIP2, MiniGPT-4
  & \cellcolor{gray!15}MS-COCO, Flickr30K, SNLI-VE \\
& TMM~\cite{wang2024transferable}
  & 2024 & Black-box
  & Sample-wise
  & ALBEF, TCL, X\_VLM, CLIP, BLIP, ViLT, METER
  & MS-COCO, Flickr30K, RefCOCO+, SNLI-VE \\
& \cellcolor{gray!15}VLATTACK~\cite{yin2024vlattack}
  & \cellcolor{gray!15}2023
  & \cellcolor{gray!15}Black-box
  & \cellcolor{gray!15}Sample-wise
  & \cellcolor{gray!15}BLIP, ViLT, CLIP
  & \cellcolor{gray!15}MS-COCO, VQA v2, NLVR2, SNLI-VE, ImageNet, SVHN \\
& PRM~\cite{hu2024firm}
  & 2024 & Black-box
  & Sample-wise
  & CLIP, Detic, VL-PLM, FC-CLIP, OpenFlamingo, LLaVA
  & PASCAL Context, COCO-Stuff, OV-COCO, MS-COCO, OK-VQA \\
& \cellcolor{gray!15}C-PGC~\cite{fang2024one}
  & \cellcolor{gray!15}2024
  & \cellcolor{gray!15}Black-box
  & \cellcolor{gray!15}Universal
  & \cellcolor{gray!15}ALBEF, TCL, X-VLM, CLIP, BLIP
  & \cellcolor{gray!15}Flickr30K, MS-COCO, SNLI-VE, RefCOCO+ \\
& ETU~\cite{zhang2024universal}
  & 2024 & Black-box
  & Universal
  & ALBEF, TCL, CLIP, BLIP
  & Flickr30K, MS-COCO \\
\hline
\multirow{16}{0.12\textwidth}{Adversarial Defense} 
& Defense-Prefix~\cite{azuma2023defense}
  & 2023 & Adversarial Tuning
  & Prompt Tuning
  & CLIP
  & ImageNet \\
& \cellcolor{gray!15}AdvPT~\cite{zhang2023adversarial}
  & \cellcolor{gray!15}2023
  & \cellcolor{gray!15}Adversarial Tuning
  & \cellcolor{gray!15}Prompt Tuning
  & \cellcolor{gray!15}CLIP
  & \cellcolor{gray!15}ImageNet, Pets, Flowers, Food101, SUN397, DTD, EuroSAT, UCF101, ImageNet-V2, ImageNet-Sketch, ImageNet-A, ImageNet-R \\
& APT~\cite{li2024one}
  & 2024 & Adversarial Tuning
  & Prompt Tuning
  & CLIP
  & ImageNet, Caltech101, Pets, StanfordCars, Flowers, Food101, FGVCAircraft, SUN397, DTD, EuroSAT, UCF101, ImageNet-V2, ImageNet-Sketch, ImageNet-R, ObjectNet \\
& \cellcolor{gray!15}MixPrompt~\cite{fan2024mixprompt}
  & \cellcolor{gray!15}2024
  & \cellcolor{gray!15}Adversarial Tuning
  & \cellcolor{gray!15}Prompt Tuning
  & \cellcolor{gray!15}CLIP
  & \cellcolor{gray!15}ImageNet, Pets, Flowers, DTD, EuroSAT, UCF101, SUN397, Food101, ImageNet-V2, ImageNet-Sketch, ImageNet-A, ImageNet-R \\
& PromptSmoot~\cite{hussein2024promptsmooth}
  & 2024 & Adversarial Tuning
  & Prompt Tuning
  & PLIP, Quilt, MedCLIP
  & KatherColon, PanNuke, SkinCancer, SICAP v2 \\
& \cellcolor{gray!15}FAP~\cite{zhou2024few}
  & \cellcolor{gray!15}2024
  & \cellcolor{gray!15}Adversarial Tuning
  & \cellcolor{gray!15}Prompt Tuning
  & \cellcolor{gray!15}CLIP
  & \cellcolor{gray!15}ImageNet, Caltech101, Pets, StanfordCars, Flowers, Food101, FGVCAircraft, SUN397, DTD, EuroSAT, UCF101 \\
& APD~\cite{luo2024apd}
  & 2024 & Adversarial Tuning
  & Prompt Tuning
  & CLIP
  & ImageNet, Caltech101, Flowers, Food101, SUN397, DTD, EuroSAT, UCF101 \\
& \cellcolor{gray!15}TAPT~\cite{wang2024tapt}
  & \cellcolor{gray!15}2024
  & \cellcolor{gray!15}Adversarial Tuning
  & \cellcolor{gray!15}Prompt Tuning
  & \cellcolor{gray!15}CLIP
  & \cellcolor{gray!15}ImageNet, Caltech101, Pets, StanfordCars, Flowers, Food101, FGVCAircraft, SUN397, DTD, EuroSAT, UCF101 \\
& TeCoA~\cite{mao2023understanding}
  & 2022 & Adversarial Tuning
  & Contrastive Tuning
  & CLIP
  & CIFAR10, CIFAR100, STL10, Caltech101, Caltech256, Pets, StanfordCars, Food101, Flowers, FGVCAircraft, SUN397, DTD, PCAM, HatefulMemes, EuroSAT \\
& \cellcolor{gray!15}PMG-AFT~\cite{wang2024pre}
  & \cellcolor{gray!15}2024
  & \cellcolor{gray!15}Adversarial Tuning
  & \cellcolor{gray!15}Contrastive Tuning
  & \cellcolor{gray!15}CLIP
  & \cellcolor{gray!15}CIFAR10, CIFAR100, STL10, ImageNet, Caltech101, Caltech256, Pets, Flowers, FGVCAircraft, StanfordCars, SUN397, Food101, EuroSAT, DTD, PCAM \\
& MMCoA~\cite{zhou2024revisiting}
  & 2024 & Adversarial Tuning
  & Contrastive Tuning
  & CLIP
  & CIFAR10, CIFAR100, TinyImageNet, STL10, Caltech101, Caltech256, Pets, Flowers, FGVCAircraft, Food101, EuroSAT, DTD, SUN397, Country211 \\
& \cellcolor{gray!15}FARE~\cite{schlarmannrobust}
  & \cellcolor{gray!15}2024
  & \cellcolor{gray!15}Adversarial Tuning
  & \cellcolor{gray!15}Contrastive Tuning
  & \cellcolor{gray!15}OpenFlamingo, LLaVA
  & \cellcolor{gray!15}COCO, Flickr30k, TextVQA, VQA v2, CalTech101, StanfordCars, CIFAR10, CIFAR100, DTD, EuroSAT, FGVCAircrafts, Flowers, ImageNet-R, ImageNet-Sketch, PCAM, Pets, STL10, ImageNet \\
& VILLA~\cite{gan2020large}
  & 2020 & Adversarial Training
  & Two-stage Training
  & UNITER, LXMERT
  & MS-COCO, Visual Genome, Conceptual Captions, SBU Captions
    ImageNet, LAION, DataComp \\
& \cellcolor{gray!15}AdvXL~\cite{wang2024revisiting}
  & \cellcolor{gray!15}2024
  & \cellcolor{gray!15}Adversarial Training
  & \cellcolor{gray!15}Two-stage Training
  & \cellcolor{gray!15}CLIP
  & \cellcolor{gray!15}ImageNet, LAION, DataComp \\
& MirrorCheck~\cite{fares2024mirrorcheck}
  & 2024 & Adversarial Detection
  & One-shot Detection
  & UniDiffuser, BLIP, Img2Prompt, BLIP-2, MiniGPT-4
  & MS-COCO, CIFAR10, ImageNet \\
& \cellcolor{gray!15}AdvQDet~\cite{wang2024advqdet}
  & \cellcolor{gray!15}2024
  & \cellcolor{gray!15}Adversarial Detection
  & \cellcolor{gray!15}Stateful Detection
  & \cellcolor{gray!15}CLIP, ViT, ResNet
  & \cellcolor{gray!15}CIFAR10, GTSRB, ImageNet, Flowers, Pets \\
\hline

\multirow{6}{0.1\textwidth}{Backdoor \& Poisoning Attack}
& PBCL~\cite{carlini2022poisoning}
  & 2021 & Backdoor\&Poisoning
  & Visual Trigger
  & CLIP
  & Conceptual Captions, YFCC \\
& \cellcolor{gray!15}BadEncoder~\cite{jia2022badencoder}
  & \cellcolor{gray!15}2021
  & \cellcolor{gray!15}Backdoor
  & \cellcolor{gray!15}Visual Trigger
  & \cellcolor{gray!15}ResNet(SimCLR), CLIP
  & \cellcolor{gray!15}CIFAR10, STL10, GTSRB, SVHN, Food101 \\
& CorruptEncoder~\cite{zhang2024data}
  & 2022 & Backdoor
  & Visual Trigger
  & ResNet(SimCLR)
  & ImageNet, Pets, Flowers \\
& \cellcolor{gray!15}BadCLIP~\cite{liang2024badclip}
  & \cellcolor{gray!15}2023
  & \cellcolor{gray!15}Backdoor
  & \cellcolor{gray!15}Visual Trigger
  & \cellcolor{gray!15}CLIP
  & \cellcolor{gray!15}Conceptual Captions \\
& BadCLIP~\cite{bai2024badclip}
  & 2023 & Backdoor
  & Multi-modal Trigger
  & CLIP
  & ImageNet, Caltech101, Pets, StanfordCars, Flowers, Food101, FGVCAircraft, SUN397, DTD, EuroSAT, UCF101 \\
& \cellcolor{gray!15}MM Poison~\cite{yang2023data}
  & \cellcolor{gray!15}2022
  & \cellcolor{gray!15}Poisoning
  & \cellcolor{gray!15}Multi-modal Poisoning
  & \cellcolor{gray!15}CLIP
  & \cellcolor{gray!15}Flickr-PASCAL, MS-COCO \\
\hline

\multirow{7}{0.1\textwidth}{Backdoor \& Poisoning Defense}
& CleanCLIP~\cite{bansal2023cleanclip}
  & 2023 & Backdoor Removal
  & Fine-tuning
  & CLIP
  & Conceptual Captions, ImageNet \\
& \cellcolor{gray!15}SAFECLIP~\cite{yang2023better}
  & \cellcolor{gray!15}2023
  & \cellcolor{gray!15}Backdoor Removal
  & \cellcolor{gray!15}Fine-tuning
  & \cellcolor{gray!15}CLIP
  & \cellcolor{gray!15}Conceptual Captions, Visual Genome, MS-COCO, Flowers, Food101, ImageNet, Pets, StanfordCars, Caltech101, CIFAR10, CIFAR100, DTD, FGVCAircraft \\
& RoCLIP~\cite{yang2024robust}
  & 2023 & Robust Training
  & Pre-training
  & CLIP
  & Conceptual Captions, Flowers, Food101, ImageNet, Pets, StanfordCars, Caltech101, CIFAR10, CIFAR100, DTD, FGVCAircraft \\
& \cellcolor{gray!15}DECREE~\cite{feng2023detecting}
  & \cellcolor{gray!15}2023
  & \cellcolor{gray!15}Backdoor Detection
  & \cellcolor{gray!15}Backdoor Model Detection
  & \cellcolor{gray!15}CLIP
  & \cellcolor{gray!15}CIFAR10, GTSRB, SVHN, STL-10, ImageNet \\
& TIJO~\cite{sur2023tijo}
  & 2023 & Backdoor Detection
  & Trigger Inversion
  & BUTD, MFB, BAN, MCAN, NAS
  & TrojVQA \\
& \cellcolor{gray!15}Mudjacking~\cite{liu2024mudjacking}
  & \cellcolor{gray!15}2024
  & \cellcolor{gray!15}Backdoor Detection
  & \cellcolor{gray!15}Trigger Inversion
  & \cellcolor{gray!15}CLIP
  & \cellcolor{gray!15}Conceptual Captions, CIFAR10, STL10, ImageNet, SVHN, Pets, Wiki103-Sub, SST-2, HOSL \\
& SEER~\cite{zhu2024seer}
  & 2024 & Backdoor Detection
  & Backdoor Sample Detection
  & CLIP
  & MSCOCO, Flickr, STL10, Pet, ImageNet \\
& \cellcolor{gray!15}Outlier Detection~\cite{huang2025detecting}
  &\cellcolor{gray!15}2025 &\cellcolor{gray!15} \cellcolor{gray!15}Backdoor Detection
  & \cellcolor{gray!15}Backdoor Sample Detection
  & \cellcolor{gray!15}CLIP
  & \cellcolor{gray!15}Conceptual Captions, ImageNet, RedCaps \\
\hline
\end{tabular}
}
\end{table*}

% \vspace{-2mm}
\section{Vision-Language Pre-training Model Safety} \label{sec:vlp}
VLP models, such as CLIP~\cite{radford2021learning}, ALBEF~\cite{li2021align}, and TCL~\cite{yang2022vision}, have made significant strides in aligning visual and textual modalities. However, these models remain vulnerable to various safety threats, which have garnered increasing research attention. This section reviews the current safety research on VLP models, with a focus on adversarial, backdoor, and poisoning research. The representative methods reviewed in this section are summarized in Table \ref{tab:vlp_safety}.


\subsection{Adversarial Attacks}
\label{sec:vlp-attack}
Since VLP models are widely used as backbones for fine-tuning downstream models, adversarial attacks on VLP aim to generate examples that cause incorrect predictions across various downstream tasks, including zero-shot image classification, image-text retrieval, visual entailment, and visual grounding. Similar to Section~\ref{sec:vfm}, these attacks can roughly be categorized into \textbf{white-box attacks} and \textbf{black-box attacks}, based on their threat models.


\subsubsection{White-box Attacks}
White-box adversarial attacks on VLP models can be further categorized based on perturbation types into \textbf{invisible perturbations} and \textbf{visible perturbations}, with the majority of existing attacks employing invisible perturbations.



\textbf{Invisible Perturbations} involve small, imperceptible adversarial changes to inputs—whether text or images—to maintain the stealthiness of attacks. Early research in the vision and language domains primarily adopts this approach~\cite{xu2018fooling, shah2019cycle, li2020bert, yang2021defending}, in which invisible attacks are developed independently.
In the context of VLP models, which integrate both modalities, \textbf{Co-Attack}~\cite{zhang2022towards} was the first to propose perturbing both visual and textual inputs simultaneously to create stronger attacks. Building on this, \textbf{AdvCLIP}~\cite{zhou2023advclip} explores universal adversarial perturbations that can deceive all downstream tasks.

\textbf{Visible Perturbations} involve more substantial and noticeable alterations. For example, manually crafted typographical, conceptual, and iconographic images have been used to demonstrate that the CLIP model tends to ``read first, look later"~\cite{noever2021reading}, highlighting a unique characteristic of VLP models. This behavior introduces new attack surfaces for VLP, enabling the development of more sophisticated attacks.


\subsubsection{Black-box Attacks}

Black-box attacks on VLP primarily adopt a transfer-based approach, with query-based attacks rarely explored. Existing methods can be categorized into: 1) \textbf{sample-specific perturbations}, tailored to individual samples, and 2) \textbf{universal perturbations}, applicable across multiple samples.

\textbf{Sample-wise perturbations} are generally more effective than universal perturbations, but their transferability is often limited.
\textbf{SGA}\cite{lu2023set} explores adversarial transferability in VLP by leveraging cross-modal interactions and alignment-preserving augmentation. Building on this, \textbf{SA-Attack}\cite{he2023sa} enhances cross-modal transferability by introducing data augmentations to both original and adversarial inputs. \textbf{VLP-Attack}\cite{wang2023exploring} improves transferability by generating adversarial texts and images using contrastive loss. To overcome SGA's limitations, \textbf{TMM}\cite{wang2024transferable} introduces modality-consistency and discrepancy features through attention-based and orthogonal-guided perturbations. \textbf{VLATTACK}\cite{yin2024vlattack} further enhances adversarial examples by combining image and text perturbations at both single-modal and multimodal levels. \textbf{PRM}\cite{hu2024firm} targets vulnerabilities in downstream models using foundation models like CLIP, enabling transferable attacks across tasks like object detection and image captioning.

\textbf{Universal Perturbations} are less effective than sample-wise perturbations but more transferable.
\textbf{C-PGC}~\cite{fang2024one} was the first to investigate universal adversarial perturbations (UAPs) for VLP models. It employs contrastive learning and cross-modal information to disrupt the alignment of image-text embeddings, achieving stronger attacks in both white-box and black-box scenarios.
\textbf{ETU}~\cite{zhang2024universal} builds on this by generating UAPs that transfer across multiple VLP models and tasks. ETU enhances UAP transferability and effectiveness through improved global and local optimization techniques. It also introduces a data augmentation strategy \textbf{ScMix} that combines self-mix and cross-mix operations to increase data diversity while preserving semantic integrity, further boosting the robustness and applicability of UAPs.


\subsection{Adversarial Defenses}
\label{sec:vlp-defenses}
Existing adversarial defenses for VLP models can be grouped into four types: 1) \textbf{adversarial example detection}, 2) \textbf{standard adversarial training}, 3) \textbf{adversarial prompt tuning}, and 4) \textbf{adversarial contrastive tuning}. While adversarial detection filters out potential adversarial examples before or during inference, the other three defenses follow similar adversarial training paradigms, with variations in efficiency.


\subsubsection{Adversarial Example Detection}
Adversarial detection methods for VLP can be further divided into \textbf{one-shot detection} and \textbf{stateful detection}.

\paragraph{One-shot Detection} 
One-shot Detection distinguishes adversarial from clean examples in a single forward pass. White-box detection methods are typically one-shot. For example, \textbf{MirrorCheck}~\cite{fares2024mirrorcheck} is a model-agnostic method for VLP models. It uses text-to-image (T2I) models to generate images from captions produced by the victim model, comparing the similarity between the input image and the generated image using CLIP’s image encoder. A significant similarity difference flags the input as adversarial.


\paragraph{Stateful Detection} Stateful Detection is designed for black-box query attacks, where multiple queries are tracked to detect adversarial behavior. \textbf{AdvQDet}~\cite{wang2024advqdet} is a novel framework that counters query-based black-box attacks. It uses adversarial contrastive prompt tuning (ACPT) to tune CLIP image encoder, enabling detection of adversarial queries within just three queries.


\subsubsection{Standard Adversarial Training}
Adversarial training is widely regarded as the most effective defense against adversarial attacks \cite{madry2017towards,croce2020reliable}. However, it is computationally expensive, and for VLP models, which are typically trained on web-scale datasets, this cost becomes prohibitively high, posing a significant challenge for traditional approaches. Although research in this area is limited, we highlight two notable works that have explored adversarial training for vision-language pre-training. Their pre-trained models can be used as robust backbones for other adversarial research.

The first work, \textbf{VILLA}~\cite{gan2020large}, is a vision-language adversarial training framework consisting of two stages: task-agnostic adversarial pre-training and task-specific fine-tuning. VILLA enhances performance across downstream tasks using adversarial pre-training in the embedding space of both image and text modalities, instead of pixel or token levels. It employs FreeLB’s strategy~\cite{zhu2019freelb} to minimize computational overhead for efficient large-scale training.

The second work, \textbf{AdvXL}~\cite{wang2024revisiting}, is a large-scale adversarial training framework with two phases: a lightweight pre-training phase using low-resolution images and weaker attacks, followed by an intensive fine-tuning phase with full-resolution images and stronger attacks. This coarse-to-fine, weak-to-strong strategy reduces training costs while enabling scalable adversarial training for large vision models.


\subsubsection{Adversarial Prompt Tuning}
Adversarial prompt tuning (APT) enhances the adversarial robustness of VLP models by incorporating adversarial training during prompt tuning~\cite{zhou2022conditional, zhou2022learning, khattak2023maple}, typically focusing on textual prompts. It offers a lightweight alternative to standard adversarial training. APT methods can be classified into two main categories based on the prompt type: \emph{textual prompt tuning} and \emph{multi-modal prompt tuning}.

\paragraph{Textual Prompt Tuning}
Textual prompt tuning (TPT) robustifies VLP models by fine-tuning learnable text prompts. \textbf{AdvPT}~\cite{zhang2023adversarial} enhances the adversarial robustness of CLIP image encoder by realigning adversarial image embeddings with clean text embeddings using learnable textual prompts. Similarly, \textbf{APT}~\cite{li2024one} learns robust text prompts, using a CLIP image encoder to boost accuracy and robustness with minimal computational cost. \textbf{MixPrompt}~\cite{fan2024mixprompt} simultaneously enhances the generalizability and adversarial robustness of VLPs by employing conditional APT. Unlike empirical defenses, \textbf{PromptSmooth}~\cite{hussein2024promptsmooth} offers a certified defense for Medical VLMs, adapting pre-trained models to Gaussian noise without retraining. Additionally, \textbf{Defense-Prefix}~\cite{azuma2023defense} mitigates typographic attacks by adding a prefix token to class names, improving robustness without retraining.


\paragraph{Multi-Modal Prompt Tuning}
Recent adversarial prompt tuning methods have expanded textual prompts to multi-modal prompts. \textbf{FAP}~\cite{zhou2024few} introduces learnable adversarial text supervision and a training objective that balances cross-modal consistency while differentiating uni-modal representations. \textbf{APD}~\cite{luo2024apd} improves CLIP’s robustness through online prompt distillation between teacher and student multi-modal prompts. Additionally, \textbf{TAPT}~\cite{wang2024tapt} presents a test-time defense that learns defensive bimodal prompts to improve CLIP's zero-shot inference robustness.


\subsubsection{Adversarial Contrastive Tuning}
Adversarial contrastive tuning involves contrastive learning with adversarial training to fine-tune a robust CLIP image encoder for \emph{zero-shot adversarial robustness} on downstream tasks. These methods are categorized into \textbf{supervised} and \textbf{unsupervised} methods, depending on the availability of labeled data during training.


\paragraph{Supervised Contrastive Tuning}
\textbf{Visual Tuning} fine-tunes CLIP image encoder using only adversarial images. \textbf{TeCoA}~\cite{mao2023understanding} explores the zero-shot adversarial robustness of CLIP and finds that visual prompt tuning is more effective without text guidance, while fine-tuning performs better with text information. \textbf{PMG-AFT}~\cite{wang2024pre} improves zero-shot adversarial robustness by introducing an auxiliary branch to minimize the distance between adversarial outputs in the target and pre-trained models, mitigating overfitting and preserving generalization.

\textbf{Multi-modal Tuning} fine-tunes CLIP image encoder using both adversarial texts and images.
\textbf{MMCoA}~\cite{zhou2024revisiting} combines image-based PGD and text-based BERT-Attack in a multi-modal contrastive adversarial training framework. It uses two contrastive losses to align clean and adversarial image and text features, improving robustness against both image-only and multi-modal attacks.

\paragraph{Unsupervised Contrastive Tuning}
Adversarial contrastive tuning can also be performed in an unsupervised fashion. For instance, \textbf{FARE}~\cite{schlarmannrobust} robustifies CLIP image encoder through unsupervised adversarial fine-tuning, achieving superior clean accuracy and robustness across downstream tasks, including zero-shot classification and vision-language tasks. This approach enables VLMs, such as LLaVA and OpenFlamingo, to attain robustness without the need for re-training or additional fine-tuning.


\subsection{Backdoor \& Poisoning Attacks}
\label{sec:vlp-bp-attacks}
Backdoor and poisoning attacks on CLIP can target either the pre-training stage or the fine-tuning stage on downstream tasks. Previous studies have shown that poisoning backdoor attacks on CLIP can succeed with significantly lower poisoning rates compared to traditional supervised learning~\cite{carlini2022poisoning}. Additionally, training CLIP on web-crawled data increases its vulnerability to backdoor attacks~\cite{carlini2024poisoning}. This section reviews proposed attacks targeting backdooring or poisoning CLIP.


\subsubsection{Backdoor Attacks}

Based on the trigger modality, existing backdoor attacks on CLIP can be categorized into \textbf{visual triggers} and \textbf{multi-modal triggers}.

\textbf{Visual Triggers} target pre-trained image encoders by embedding backdoor patterns in visual inputs.
\textbf{BadEncoder}~\cite{jia2022badencoder} explores image backdoor attacks on self-supervised learning by injecting backdoors into pre-trained image encoders, compromising downstream classifiers. \textbf{CorruptEncoder}~\cite{zhang2024data} exploits random cropping in contrastive learning to inject backdoors into pre-trained image encoders, with increased effectiveness when cropped views contain only the reference object or the trigger.
For attacks targeting CLIP, \textbf{BadCLIP}~\cite{liang2024badclip} optimizes visual trigger patterns using dual-embedding guidance, aligning them with both the target text and specific visual features. This strategy enables BadCLIP to bypass backdoor detection and fine-tuning defenses.

\textbf{Multi-modal Triggers} combine both visual and textual triggers to enhance the attack. BadCLIP~\cite{bai2024badclip} introduces a novel trigger-aware prompt learning-based backdoor attack targeting CLIP models. Rather than fine-tuning the entire model, BadCLIP injects learnable triggers during the prompt learning stage, affecting both the image and text encoders.

\subsubsection{Poisoning Attacks}
Two targeted poisoning attacks on CLIP are \textbf{PBCL}~\cite{carlini2022poisoning} and \textbf{MM Poison}~\cite{yang2023data}. PBCL demonstrated that a targeted poisoning attack, misclassifying a specific sample, can be achieved by poisoning as little as 0.0001\% of the training dataset. MM Poison investigates modality vulnerabilities and proposes three attack types: single target image, single target label, and multiple target labels. Evaluations show high attack success rates while maintaining clean data performance across both visual and textual modalities.

\subsection{Backdoor \& Poisoning Defenses}
\label{sec:vlp-bp-defenses}

Defense strategies against backdoor and poisoning attacks are generally categorized into \textbf{robust training} and \textbf{backdoor detection}. Robust Training aims to create VLP models resistant to backdoor or targeted poisoning attacks, even when trained on untrusted datasets. This approach specifically addresses poisoning-based attacks. Backdoor detection focuses on identifying compromised encoders or contaminated data. Detection methods often require additional mitigation techniques to fully eliminate backdoor effects.


\subsubsection{Robust Training}
Depending on the stage at which the model gains robustness against backdoor attacks, existing robust training strategies can be categorized into fine-tuning and pre-training approaches.

\paragraph{Fine-tuning Stage}
To mitigate backdoor and poisoning threats, \textbf{CleanCLIP}~\cite{bansal2023cleanclip} fine-tunes CLIP by re-aligning each modality's representations, weakening spurious correlations from backdoor attacks. Similarly, \textbf{SAFECLIP}~\cite{yang2023better} enhances feature alignment using unimodal contrastive learning. It first warms up the image and text modalities separately, then uses a Gaussian mixture model to classify data into safe and risky sets. During pre-training, SAFECLIP optimizes CLIP loss on the safe set, while separately fine-tuning the risky set, reducing poisoned image-text pair similarity and defending against targeted poisoning and backdoor attacks.

\paragraph{Pre-training Stage}
\textbf{ROCLIP}~\cite{yang2024robust} defends against poisoning and backdoor attacks by enhancing model robustness during pre-training. It disrupts the association between poisoned image-caption pairs by utilizing a large, diverse pool of random captions. Additionally, ROCLIP applies image and text augmentations to further strengthen its defense and improve model performance.


\subsubsection{Backdoor Detection}
Backdoor detection can be broadly divided into three subtasks: 1) \textbf{trigger inversion}, 2) \textbf{backdoor sample detection}, and 3) \textbf{backdoor model detection}. Trigger inversion is particularly useful, as recovering the trigger can aid in the detection of both backdoor samples and backdoored models.

\textbf{Trigger Inversion} aims to reverse-engineer the trigger pattern injected into a backdoored model. \textbf{Mudjacking}~\cite{liu2024mudjacking} mitigates backdoor vulnerabilities in VLP models by adjusting model parameters to remove the backdoor when a misclassified trigger-embedded input is detected. In contrast to single-modality defenses, \textbf{TIJO}~\cite{sur2023tijo} defends against dual-key backdoor attacks by jointly optimizing the reverse-engineered triggers in both the image and text modalities.

\textbf{Backdoor Sample Detection} detects whether a training or test sample is poisoned by a backdoor trigger. This detection can be used to cleanse the training dataset or reject backdoor queries. \textbf{SEER}~\cite{zhu2024seer} addresses the complexity of multi-modal models by jointly detecting malicious image triggers and target texts in the shared feature space. This method does not require access to the training data or knowledge of downstream tasks, making it highly effective for backdoor detection in VLP models. \textbf{Outlier Detection}~\cite{huang2025detecting} demonstrates that the local neighborhood of backdoor samples is significantly sparser compared to that of clean samples. This insight enables the effective and efficient application of various local outlier detection methods to identify backdoor samples from web-scale datasets. Furthermore, they reveal that potential unintentional backdoor samples already exist in the Conceptual Captions 3 Million (CC3M) dataset and have been trained into open-sourced CLIP encoders.

\textbf{Backdoor Model Detection} identifies whether a trained model is compromised by backdoor(s). \textbf{DECREE}~\cite{feng2023detecting} introduces a backdoor detection method specifically for VLP encoders that require no labeled data. It exploits the distinct embedding space characteristics of backdoored encoders when exposed to clean versus backdoor inputs. By combining trigger inversion with these embedding differences, DECREE can effectively detect backdoored encoders.



\subsection{Datasets}
\label{sec:vlp-data}
This section reviews datasets used for VLP safety research. As shown in Table \ref{tab:vlp_safety}, a variety of benchmark datasets were employed to evaluate adversarial attacks and defenses for VLP models. For image classification tasks, commonly used datasets include: ImageNet~\cite{russakovsky2015imagenet}, Caltech101~\cite{fei2004learning}, DTD~\cite{cimpoi2014describing}, EuroSAT~\cite{helber2019eurosat}, OxfordPets~\cite{parkhi2012cats}, FGVC-Aircraft~\cite{maji2013fine}, Food101~\cite{bossard2014food}, Flowers102~\cite{nilsback2008automated}, StanfordCars~\cite{krause20133d}, SUN397~\cite{xiao2010sun}, and UCF101~\cite{soomro2012ucf101}. 
For evaluating domain generalization and robustness to distribution shifts, several ImageNet variants were also used: ImageNetV2~\cite{recht2019imagenet}, ImageNet-Sketch~\cite{wang2019learning}, ImageNet-A~\cite{hendrycks2021natural}, and ImageNet-R~\cite{hendrycks2021many}. Additionally, MS-COCO~\cite{lin2014microsoft} and Flickr30K~\cite{plummer2015flickr30k} were utilized for image-to-text and text-to-image retrieval tasks, RefCOCO+\cite{yu2016modeling} for visual grounding, and SNLI-VE\cite{xie2019visual} for visual entailment.






\section{Vision-Language Model Safety} \label{sec:vlm}

Large VLMs extend LLMs by adding a visual modality through pre-trained image encoders and alignment modules, enabling applications like visual conversation and complex reasoning. However, this multi-modal design introduces unique vulnerabilities.
This section reviews \textbf{adversarial attacks}, \textbf{latency energy attacks}, \textbf{jailbreak attacks}, \textbf{prompt injection attacks}, \textbf{backdoor \& poisoning attacks}, and \textbf{defenses} developed for VLMs. Many VLMs use VLP-trained encoders, so the attacks and defenses discussed in Section~\ref{sec:vlp} also apply to VLMs. The additional alignment process between the VLM pre-trained encoders and LLMs, however, expands the attack surface, with new risks like cross-modal backdoor attacks and jailbreaks targeting both text and image inputs. This underscores the need for safety measures tailored to VLMs.


\subsection{Adversarial Attacks}
\label{sec:vlm-adversarial}
Adversarial attacks on VLMs primarily target the visual modality, which, unlike text, is more susceptible to adversarial perturbations due to its high-dimensional nature. By adding imperceptible changes to images, attackers aim to disrupt tasks like image captioning and visual question answering. These attacks are classified into \textbf{white-box} and \textbf{black-box} categories based on the threat model. 





\subsubsection{White-box Attacks}

White-box adversarial attacks on VLMs have full access to the model parameters, including both vision encoders and LLMs. These attacks can be classified into three types based on their objectives: \textbf{task-specific attacks}, \textbf{cross-prompt attack}, and \textbf{chain-of-thought (CoT) attack}.

\textbf{Task-specific Attacks}  Schlarmann et al.~\cite{schlarmann2023adversarial} were the first to highlight the vulnerability of VLMs like Flamingo~\cite{alayrac2022flamingo} and GPT-4~\cite{gpt-4} to adversarial images that manipulate caption outputs. Their study showed how attackers can exploit these vulnerabilities to mislead users, redirecting them to harmful websites or spreading misinformation. Gao et al.~\cite{gao2024adversarial} introduced attack paradigms targeting the referring expression comprehension task, while \cite{cui2024robustness} proposed a query decomposition method and demonstrated how contextual prompts can enhance VLM robustness against visual attacks.

\textbf{Cross-prompt Attack} refer to adversarial attacks that remain effective across different prompts. For example, \textbf{CroPA}~\cite{luo2024image} explored the transferability of a single adversarial image across multiple prompts, investigating whether it could mislead predictions in various contexts. To tackle this, they proposed refining adversarial perturbations through learnable prompts to enhance transferability.


\textbf{CoT Attack} targets the CoT reasoning process of VLMs. \textbf{Stop-reasoning Attack} \cite{wang2024stop} explored the impact of CoT reasoning on adversarial robustness. Despite observing some improvements in robustness, they introduced a novel attack designed to bypass these defenses and interfere with the reasoning process within VLMs.





\subsubsection{Gray-box Attacks}
Gray-box adversarial attacks typically involve access to either the vision encoders or the LLM of a VLM, with a focus on vision encoders as the key differentiator between VLMs and LLMs. Attackers craft adversarial images that closely resemble target images, manipulating model predictions without full access to the VLM.
For instance, \textbf{InstructTA} \cite{wang2023instructta} generates a target image and uses a surrogate model to create adversarial perturbations, minimizing the feature distance between the original and adversarial image. To improve transferability, the attack incorporates GPT-4 paraphrasing to refine instructions.

\subsubsection{Black-box Attacks}

In contrast, black-box attacks do not require access to the target model's internal parameters and typically rely on \textbf{transfer-based} or \textbf{generator-based} methods. 



\textbf{Transfer-based Attacks} exploit the widespread use of frozen CLIP vision encoders in many VLMs. \textbf{AttackBard} \cite{dong2023robust} demonstrates that adversarial images generated from surrogate models can successfully mislead Google's Bard, despite its defense mechanisms. Similarly, \textbf{AttackVLM} \cite{zhao2024evaluating} crafts targeted adversarial images for models like CLIP \cite{radford2021learning} and BLIP \cite{BLIP-2}, successfully transferring these adversarial inputs to other VLMs. It also shows that black-box queries further improved the success rate of generating targeted responses, illustrating the potency of cross-model transferability. 

\textbf{Generator-based Attacks} leverage generative models to create adversarial examples with improved transferability. \textbf{AdvDiffVLM} \cite{guo2024efficiently} uses diffusion models to generate natural, targeted adversarial images with enhanced transferability. By combining adaptive ensemble gradient estimation and GradCAM-guided masking, it improves the semantic embedding of adversarial examples and spreads the targeted semantics more effectively across the image, leading to more robust attacks. \textbf{AnyAttack} \cite{zhang2024anyattack} presents a self-supervised framework for generating targeted adversarial images without label supervision. By utilizing contrastive loss, it efficiently creates adversarial examples that mislead models across diverse tasks.




\subsection{Jailbreak Attacks}
\label{sec:vlm-jailbreak}

\begin{table*}[htbp]
  \centering
  \caption{A summary of attacks and defenses for VLMs.}
  \resizebox{\textwidth}{!}{
    \begin{tabular}
{p{0.1\textwidth}p{0.15\textwidth}p{0.05\textwidth}p{0.15\textwidth}p{0.15\textwidth}p{0.25\textwidth}p{0.2\textwidth}}

\hline
    \rowcolor{wangruofan-orange}
    Attack/Defense & Method & Year & Category & Subcategory & Target Models & Datasets \\\hline
    \multirow{10}{0.1\textwidth}{Adversarial Attack} & Caption Attack \cite{schlarmann2023adversarial} & 2023 & White-box & Task-specific+V & OpenFlamingo & MS-COCO/Flickr30k/OK-VQA/VizWiz \\
    & \cellcolor{gray!15}VisBreaker \cite{cui2024robustness} & \cellcolor{gray!15}\cellcolor{gray!15}2023 & \cellcolor{gray!15}White-box & \cellcolor{gray!15}Task-specific+V & \cellcolor{gray!15}LLaVA/BLIP-2/InstructBLIP & \cellcolor{gray!15}MS-COCO/VQA V2/ScienceQA-Image/TextVQA/POPE/MME \\
    & CroPA \cite{luo2024image} & 2024 & White-box & Cross-prompt+VL & OpenFlamingo/BLIP-2/InstructBLIP & MS-COCO/VQA-v2 \\
    & \cellcolor{gray!15}GroundBreaker \cite{gao2024adversarial} & \cellcolor{gray!15}2024 &\cellcolor{gray!15} White-box & \cellcolor{gray!15}Task-specific+V &\cellcolor{gray!15} MiniGPT-v2 & \cellcolor{gray!15}RefCOCO/RefCOCO+/RefCOCOg \\
    & Stop-reasoning Attack \cite{wang2024stop} & 2024 & White-box & CoT attack+V & MiniGPT-4/OpenFlamingo/LLaVA & ScienceQA/A-OKVQA \\
    & \cellcolor{gray!15}InstructTA \cite{wang2023instructta} &\cellcolor{gray!15} 2023 &\cellcolor{gray!15} Gray-box & \cellcolor{gray!15}Encoder attack+V & \cellcolor{gray!15}BLIP-2/InstructBLIP/MiniGPT-4/LLaVA/CogVLM & \cellcolor{gray!15}ImageNet-1K/LLaVA-Instruct-150K/MS-COCO \\
    & Attack Bard \cite{dong2023robust} & 2023 & Black-box & Transfer-based+V & Bard/GPT-4V/Bing Chat/ERNIE Bot & NeurIPS’17 adversarial competition dataset \\
    & \cellcolor{gray!15}AttackVLM \cite{zhao2024evaluating} & \cellcolor{gray!15}2024 & \cellcolor{gray!15}Black-box & \cellcolor{gray!15}Transfer-based+V & \cellcolor{gray!15}BLIP/UniDiffuser/Img2Prompt/BLIP-2/LLaVA/MiniGPT-4 & \cellcolor{gray!15}ImageNet-1K/MS-COCO \\
    & AdvDiffVLM \cite{guo2024efficiently} & 2024 & Black-box & Generator-based+V & MiniGPT-4/LLaVA/UniDiffuser/MiniGPT-4/BLIP/BLIP-2/Img2LLM & NeurIPS’17 adversarial competition dataset/MS-COCO \\
        & \cellcolor{gray!15}AnyAttack \cite{zhang2024anyattack} &\cellcolor{gray!15} 2024 &\cellcolor{gray!15} Black-box & \cellcolor{gray!15}Generator-based+V & \cellcolor{gray!15}CLIP/BLIP/BLIP2/InstructBLIP/MiniGPT-4 & \cellcolor{gray!15}MSCOCO/Flickr30K/SNLI-VE \\\hline

    \multirow{1}{0.1\textwidth}{Latency-Energy Attack} & Verbose Images \cite{gaoinducing} & 2024 & White-box & Task-specific+V & BLIP/BLIP2/InstructBLIP/MiniGPT-4 & MS-COCO/ImageNet
    \\\hline

    
    \multirow{11}{0.1\textwidth}{Jailbreak Attack} &\cellcolor{gray!15} Image Hijack \cite{bailey2023image} & \cellcolor{gray!15}2023 & \cellcolor{gray!15}White-box & \cellcolor{gray!15}Target-specific+V & \cellcolor{gray!15}LLaVA & \cellcolor{gray!15}Alpaca training set/AdvBench \\
    & Adversarial Alignment Attack \cite{carlini2024aligned} & 2024 & White-box & Target-specific+V & MiniGPT-4/LLaVA/LLaMA Adapter & toxic phrase dataset \\
    & \cellcolor{gray!15}VAJM \cite{qi2024visual} & \cellcolor{gray!15}2024 & \cellcolor{gray!15}White-box & \cellcolor{gray!15}Universal attack+V & \cellcolor{gray!15}MiniGPT-4/LLaVA/InstructBLIP & \cellcolor{gray!15}VAJM training set/VAJM test set/RealToxicityPrompts \\
    & imgJP \cite{niu2024jailbreaking} & 2024 & White-box & Universal attack+V & MiniGPT-4/MiniGPT-v2/LLaVA/InstructBLIP/mPLUG-Owl2 & AdvBench-M \\
    & \cellcolor{gray!15}UMK \cite{wang2024white} & \cellcolor{gray!15}2024 & \cellcolor{gray!15}White-box & \cellcolor{gray!15}Universal attack+VL & \cellcolor{gray!15}MiniGPT-4 & \cellcolor{gray!15}AdvBench/VAJM training set/VAJM test set/RealToxicityPrompts \\
    & HADES \cite{li2024images} & 2024 & White-box & Hybrid method+V & LLaVA/GPT-4V/Gemini-Pro-Vision & HADES dataset \\
    & \cellcolor{gray!15}Jailbreak in Pieces \cite{shayegani2023jailbreak} & \cellcolor{gray!15}2023 & \cellcolor{gray!15}Black-box & \cellcolor{gray!15}Transfer-based+V & \cellcolor{gray!15}LlaVA /LLaMA-Adapter V2 & \cellcolor{gray!15}Jailbreak in Pieces dataset \\
    & Figstep \cite{gong2023figstep} & 2023 & Black-box & Manual pipeline+V & LLaVA-v1.5/MiniGPT-4/CogVLM/GPT-4V & SafeBench \\
    & \cellcolor{gray!15}SASP \cite{wu2023jailbreaking} & \cellcolor{gray!15}2023 & \cellcolor{gray!15}Black-box & \cellcolor{gray!15}Prompt leakage+L & \cellcolor{gray!15}LLaVA/GPT-4V & \cellcolor{gray!15}Celebrity face image dataset/CelebA/LFWA \\
    & VRP \cite{ma2024visual} & 2024 & Black-box & Manual pipeline+V & LLaVA/Qwen-VL-Chat/ OmniLMM /InternVL Chat-V1.5/Gemini-Pro-Vision & RedTeam-2k/HarmBench \\
    &\cellcolor{gray!15} IDEATOR \cite{wang2024ideator} & \cellcolor{gray!15}2024 & \cellcolor{gray!15}Black-box &\cellcolor{gray!15} Red teaming+VL & \cellcolor{gray!15}LLaVA/InstructBLIP/MiniGPT-4 & \cellcolor{gray!15}AdvBench/VAJM test set \\\hline

    \multirow{2}{0.1\textwidth}{Prompt Injection Attack} & Adversarial Prompt Injection \cite{bagdasaryan2023ab} & 2023 & White-box & Optimization-based+V & LLaVA/PandaGPT & Self-collected dataset \\
    & \cellcolor{gray!15}Typographic Attack \cite{qraitem2024vision} & \cellcolor{gray!15}2024 & \cellcolor{gray!15}Black-box & \cellcolor{gray!15}Typography-based+V & \cellcolor{gray!15}LLaVA/MiniGPT4/InstructBLIP/GPT-4V & 
 \cellcolor{gray!15}OxfordPets / StanfordCars / Flowers / Aircraft / Food101 \\\hline

    \multirow{5}{0.1\textwidth}{Backdoor \& Poisoning Attack} & Shadowcast \cite{xu2024shadowcast} & 2024 & Poisoning & Tuning-stage+VL & LLaVA/MiniGPT-v2/InstructBLIP & cc-sbu-align dataset \\
    & \cellcolor{gray!15}Instruction-Tuned Backdoor \cite{liang2024revisiting} & \cellcolor{gray!15}2024 &\cellcolor{gray!15} Backdoor & \cellcolor{gray!15}Tuning-stage+VL & \cellcolor{gray!15}OpenFlamingo/BLIP-2/LLaVA &\cellcolor{gray!15} MIMIC-IT/COCO/Flickr30K \\
    & Anydoor \cite{lu2024test} & 2024 & Backdoor & Testing-stage+VL & LLaVA/MiniGPT-4/InstructBLIP/BLIP-2 & VQAv2/SVIT/DALL-E dataset \\
    & \cellcolor{gray!15}BadVLMDriver \cite{ni2024physical} & \cellcolor{gray!15}2024 & \cellcolor{gray!15}Backdoor & \cellcolor{gray!15}Tuning-stage+V & \cellcolor{gray!15}LLaVA/MiniGPT-4 & \cellcolor{gray!15}nuScenes dataset \\
    & ImgTrojan \cite{tao2024imgtrojan} & 2024 & Backdoor & Tuning-stage+VL & LLaVA & LAION \\\hline

    \multirow{6}{0.1\textwidth}{Jailbreak Defenses} & \cellcolor{gray!15}JailGuard \cite{zhang2023mutation} & \cellcolor{gray!15}2023 & \cellcolor{gray!15}Detection & \cellcolor{gray!15}Detection+VL & \cellcolor{gray!15}GPT-3.5/MiniGPT-4 & \cellcolor{gray!15}Self-collected dataset \\
    & GuardMM \cite{sharma2024defending} & 2024 & Detection & Detection+V & GPT-4V/LLAVA/MINIGPT-4 & Self-collected dataset \\
    & \cellcolor{gray!15}AdaShield \cite{wang2024adashield} & \cellcolor{gray!15}2024 & \cellcolor{gray!15}Prevention & \cellcolor{gray!15}Prevention+V & \cellcolor{gray!15}LLaVA/CogVLM/MiniGPT-v2 & \cellcolor{gray!15}Figstep/QR \\
    & MLLM-Protector \cite{pi2024mllm} & 2024 & Prevention & Detection+Prevention+V & Open-LLaMA/LLaMA/LLaVA & Safe-Harm-10K \\
    & \cellcolor{gray!15}ECSO \cite{gou2024eyes} & \cellcolor{gray!15}2024 & \cellcolor{gray!15}Prevention & \cellcolor{gray!15}Prevention+V & \cellcolor{gray!15}LLaVA/ShareGPT4V/mPLUG-OWL2/Qwen-VL-Chat/InternLM-XComposer & \cellcolor{gray!15}MM-SafetyBench/VLSafe/VLGuard \\
    & InferAligner \cite{wang2024inferaligner} & 2024 & Prevention & Prevention+VL & LLaMA2/LLaVA & AdvBench/TruthfulQA/MM-Harmful Bench \\
    & \cellcolor{gray!15}BlueSuffix \cite{zhao2024bluesuffix} & \cellcolor{gray!15}2024 & \cellcolor{gray!15}Prevention & \cellcolor{gray!15}Prevention+VL & \cellcolor{gray!15}LLaVA/MiniGPT-4/Gemini & \cellcolor{gray!15}MM-SafetyBench/RedTeam-2k \\\hline
    \end{tabular}%
  }
  \label{tab:addlabel}
\end{table*}



The inclusion of a visual modality in VLMs provides additional routes for jailbreak attacks. While adversarial attacks generally induce random or targeted errors, jailbreak attacks specifically target the model's safeguards to generate inappropriate outputs. Like adversarial attacks, jailbreak attacks on VLMs can be classified as \textbf{white-box} or \textbf{black-box} attacks. 

\subsubsection{White-box Attacks}
White-box jailbreak attacks leverage gradient information to perturb input images or text, targeting specific behaviors in VLMs. These attacks can be further categorized into three types: \textbf{target-specific jailbreak}, \textbf{universal jailbreak}, and \textbf{hybrid jailbreak}, each exploiting different aspects of the model's safety measures.

\textbf{Target-specific Jailbreak} focuses on inducing a specific type of harmful output from the model.
\textbf{Image Hijack} \cite{bailey2023image} introduces adversarial images that manipulate VLM outputs, such as leaking information, bypassing safety measures, and generating false statements. These attacks, trained on generic datasets, effectively force models to produce harmful outputs.
Similarly, \textbf{Adversarial Alignment Attack} \cite{carlini2024aligned} demonstrates that adversarial images can induce misaligned behaviors in VLMs, suggesting that similar techniques could be adapted for text-only models using advanced NLP methods.

\textbf{Universal Jailbreak} bypasses model safeguards, causing it to generate harmful content beyond the adversarial input.
\textbf{VAJM} \cite{qi2024visual} shows that a single adversarial image can universally bypass VLM safety, forcing universal harmful outputs. \textbf{ImgJP} \cite{niu2024jailbreaking} uses a maximum likelihood algorithm to create transferable adversarial images that jailbreak various VLMs, even bridging VLM and LLM attacks by converting images to text prompts. \textbf{UMK} \cite{wang2024white} proposes a dual optimization attack targeting both text and image modalities, embedding toxic semantics in images and text to maximize impact. \textbf{HADES} \cite{li2024images} introduces a hybrid jailbreak method that combines universal adversarial images with crafted inputs to bypass safety mechanisms, effectively amplifying harmful instructions and enabling robust adversarial manipulation.



\subsubsection{Black-box Attacks}

Black-box jailbreak attacks do not require direct access to the internal parameters of the target VLM. Instead, they exploit external vulnerabilities, such as those in the frozen CLIP vision encoder, interactions between vision and language modalities, or system prompt leakage. These attacks can be classified into four main categories: \textbf{transfer-based attacks}, \textbf{manually-designed attacks}, \textbf{system prompt leakage}, and \textbf{red teaming}, each employing distinct strategies to bypass VLM defenses and trigger harmful behaviors.

\textbf{Transfer-based Attacks} on VLMs typically assume the attacker has access to the image encoder (or its open-source version), which is used to generate adversarial images that can then be transferred to attack the black-box LLM. For example, \textbf{Jailbreak in Pieces} \cite{shayegani2023jailbreak} introduces cross-modality attacks that transfer adversarial images, crafted using the image encoder (assume the model employed an open-source encoder), along with clean textual prompts to break VLM alignment. 

\textbf{Manually-designed Attacks} can be as effective as optimized ones. For instance, \textbf{FigStep} \cite{gong2023figstep} introduces an algorithm that bypasses safety measures by converting harmful text into images via typography, enabling VLMs to visually interpret the harmful intent. \textbf{VRP} \cite{ma2024visual} adopts a visual role-play approach, using LLM-generated images of high-risk characters based on detailed descriptions. By pairing these images with benign role-play instructions, VRP exploits the negative traits of the characters to deceive VLMs into generating harmful outputs.

\textbf{System Prompt Leakage} is another significant black-box jailbreak method, exemplified by \textbf{SASP} \cite{wu2023jailbreaking}. By exploiting a system prompt leakage in GPT-4V, SASP allowed the model to perform a self-adversarial attack, demonstrating the risks of internal prompt exposure.

\textbf{Red Teaming} recently saw an advancement with IDEATOR \cite{wang2024ideator}, which integrated a VLM with an advanced diffusion model to autonomously generate malicious image-text pairs. This approach overcomes the limitations of manually designed attacks, providing a scalable and efficient method for creating adversarial inputs without direct access to the target model.

\subsection{Jailbreak Defenses}
\label{sec:vlm-defenses}



This section reviews defense methods for VLMs against jailbreak attacks, categorized into \textbf{jailbreak detection} and \textbf{jailbreak prevention}. Detection methods identify harmful inputs or outputs for rejection or purification, while prevention methods enhance the model's inherent robustness to jailbreak queries through safety alignment or filters.

\subsubsection{Jailbreak Detection}
\textbf{JailGuard} \cite{zhang2023mutation} detects jailbreak attacks by mutating untrusted inputs and analyzing discrepancies in model responses. It uses 18 mutators for text and image inputs, improving generalization across attack types. 
\textbf{GuardMM} \cite{sharma2024defending} is a two-stage defense: the first stage validates inputs to detect unsafe content, while the second stage focuses on prompt injection detection to protect against image-based attacks. It uses a specialized language to enforce safety rules and standards. \textbf{MLLM-Protector} \cite{pi2024mllm} identifies harmful responses using a lightweight detector and detoxifies them through a specialized transformation mechanism. Its modular design enables easy integration into existing VLMs, enhancing safety and preventing harmful content generation.

\subsubsection{Jailbreak Prevention}
\textbf{AdaShield} \cite{wang2024adashield} defends against structure-based jailbreaks by prepending defense prompts to inputs, refining them adaptively through collaboration between the VLM and an LLM-based prompt generator, without requiring fine-tuning. \textbf{ECSO} \cite{gou2024eyes} offers a training-free protection by converting unsafe images into text descriptions, activating the safety alignment of pre-trained LLMs within VLMs to ensure safer outputs. 
\textbf{InferAligner} \cite{wang2024inferaligner} applies cross-model guidance during inference, adjusting activations using safety vectors to generate safe and reliable outputs. \textbf{BlueSuffix} \cite{zhao2024bluesuffix} introduces a reinforcement learning-based black-box defense framework consisting of three key components: (1) an image purifier for securing visual inputs, (2) a text purifier for safeguarding textual inputs, and (3) a reinforcement fine-tuning-based suffix generator that leverages bimodal gradients to enhance cross-modal robustness.

\subsection{Energy Latency Attacks}
\label{sec:vlm-latency}
Similar to LLMs, multi-modal LLMs also face significant computational demands. Verbose images~\cite{gaoinducing} exploit these demands by overwhelming service resources, resulting in higher server costs, increased latency, and inefficient GPU usage. These images are specifically designed to delay the occurrence of the EOS token, increasing the number of auto-regressive decoder calls, which in turn raises both energy consumption and latency costs.



\subsection{Prompt Injection Attacks}
\label{sec:vlm-injection}

Prompt injection attacks against VLMs share the same objective as those against LLMs (Section~\ref{sec:llm}), but the visual modality introduces continuous features that are more easily exploited through adversarial attacks or direct injection. These attacks can be further classified into \emph{optimization-based attacks} and \emph{typography-based attacks}.

\textbf{Optimization-based Attacks} often optimize the input images using (white-box) gradients to produce stronger attacks. These attacks manipulate the model's responses, influencing future interactions. One representative method is \textbf{Adversarial Prompt Injection} \cite{bagdasaryan2023ab}, where attackers embed malicious instructions into VLMs by adding adversarial perturbations to images. 

 \textbf{Typography-based Attacks} exploit VLMs' typographic vulnerabilities by embedding deceptive text into images without requiring gradient access (i.e., black-box). The \textbf{Typographic Attack} \cite{qraitem2024vision} introduces two variations: \emph{Class-Based Attack} to misidentify classes and \emph{Descriptive Attack} to generate misleading labels. These attacks can also leak personal information \cite{chen2023can}, highlighting significant security risks.

\subsection{Backdoor\& Poisoning Attacks}
\label{sec:vlm-backdoor}
Most VLMs rely on VLP encoders, with safety threats discussed in Section \ref{sec:vlp}. This section focuses on backdoor and poisoning risks arising during fine-tuning and testing, specifically when aligning vision encoders with LLMs. 
Backdoor attacks embed triggers in visual or textual inputs to elicit specific outputs, while poisoning attacks inject malicious image-text pairs to degrade model performance.
We review backdoor and poisoning attacks separately, though most of these works are backdoor attacks.

\subsubsection{Backdoor Attacks}
We further classify backdoor attacks on VLMs into \textbf{tuning-time backdoor} and \textbf{testing-time backdoor}.

\textbf{Tuning-time Backdoor} injects the backdoor during VLM instruction tuning. \textbf{MABA} \cite{liang2024revisiting} targets domain shifts by adding domain-agnostic triggers using attributional interpretation, enhancing attack robustness across mismatched domains in image captioning tasks.
\textbf{BadVLMDriver} \cite{ni2024physical} introduced a physical backdoor for autonomous driving, using objects like red balloons to trigger unsafe actions such as sudden acceleration, bypassing digital defenses and posing real-world risks. Its automated pipeline generates backdoor training samples with malicious behaviors for stealthy, flexible attacks. \textbf{ImgTrojan} \cite{tao2024imgtrojan} introduces a jailbreaking attack by poisoning image-text pairs in training data, replacing captions with malicious prompts to enable VLM jailbreaks, exposing risks of compromised datasets. 

\textbf{Test-time Backdoor} leverages the similarity of universal adversarial perturbations and backdoor triggers to inject backdoor at test-time. AnyDoor \cite{lu2024test} embeds triggers in the textual modality via adversarial test images with universal perturbations, creating a text backdoor from image-perturbation combinations. It can also be seen as a multi-modal universal adversarial attack. Unlike traditional methods, AnyDoor does not require access to training data, enabling attackers to separate setup and activation of the attack. 


\subsubsection{Poisoning Attacks}

\textbf{Shadowcast} \cite{xu2024shadowcast} is a stealthy tuning-time backdoor attack on VLMs. It injects poisoned samples visually indistinguishable from benign ones, targeting two objectives: 1) \textbf{Label Attack}, which misclassifies objects, and 2) \textbf{Persuasion Attack}, which generates misleading narratives. With only 50 poisoned samples, Shadowcast achieves high effectiveness, showing robustness and transferability across VLMs in black-box settings.




\subsection{Datasets \& Benchmarks}
\label{sec:vlm-dataset}

\begin{table}[htbp]
  \centering
  \caption{Safety and robustness benchmarks for VLMs.}
  \resizebox{0.5\textwidth}{!}{ % Resize the table to fit within the text width
    \begin{tabular}{cccc}\hline
    \rowcolor{wangruofan-orange}
    Benchmarks & Year  & Size  & \# VLMs evaluated \\\hline
     OODCV-VQA \cite{tu2023many} & 2023  & 4,244 & 21 \\
    Sketchy-VQA \cite{tu2023many} & 2023  & 4,000 & 21 \\
    MM-SafetyBench \cite{liu2023mm} & 2023  & 5,040 & 12 \\
    AVIBench \cite{zhang2024avibench} & 2024  & 260,000  & 14 \\
    Jailbreak Evaluation of GPT-4o \cite{ying2024unveiling} & 2024  & 4,180 & 1 \\
    JailBreakV-28K \cite{luo2024jailbreakv} & 2024  & 28,000 & 10 \\\hline
    \end{tabular}%
    }
  \label{tab:vlm_safety_benchmarks}%
\end{table}

The datasets used in VLM safety research are detailed in Table \ref{tab:vfm_safety}. Below, we review the benchmarks proposed for evaluating VLM safety and robustness, summarized in Table \ref{tab:vlm_safety_benchmarks}.
\textbf{SafeSight} \cite{tu2023many} introduces two VQA datasets, \textbf{OODCV-VQA} and \textbf{Sketchy-VQA}, to evaluate out-of-distribution (OOD) robustness, highlighting VLMs' vulnerabilities to OOD texts and vision encoder weaknesses. \textbf{MM-SafetyBench} \cite{liu2023mm} focuses on image-based manipulations, revealing vulnerabilities in multi-modal interactions. \textbf{AVIBench} \cite{zhang2024avibench} evaluates VLM robustness against 260K adversarial visual instructions, exposing susceptibility to image-based, text-based, and content-biased  adversarial visual instructions (AVIs). \textbf{Jailbreak Evaluation of GPT-4o} \cite{ying2024unveiling} tests GPT-4o with multi-modal and unimodal jailbreak attacks, uncovering alignment vulnerabilities. \textbf{JailBreakV-28K} \cite{luo2024jailbreakv} assesses the transferability of LLM jailbreak techniques to VLMs, showing high attack success rates across 10 open-source models. These studies collectively reveal significant vulnerabilities in VLMs to OOD inputs, adversarial instructions, and multi-modal jailbreaks.



\section{Diffusion Model Safety}\label{sec:diffusion}

This section focuses on safety research related to diffusion models \cite{rombach2022high, DALLE-2, DALLE-3, Imagen}, which involve forward noise addition and reverse sampling. In the forward process, Gaussian noise is incrementally added to an image until it becomes pure noise. Reverse sampling generates new samples by stepwise denoising based on learned data distributions~\cite{ho2020denoising, song2020denoising, song2020score}. By integrating input information, diffusion models perform conditional generation, transforming data distribution modeling \( p(x) \) into \( p(x|\text{guidance}) \).

Widely used in Image-to-Image (I2I), Text-to-Image (T2I), and Text-to-Video (T2V) tasks, diffusion models are applied in content creation, image editing, and film production. However, their extensive use exposes them to various security risks including \textbf{adversarial}, \textbf{jailbreak}, \textbf{backdoor}, and \textbf{privacy attacks}. These attacks can degrade generation quality, bypass safety filters, manipulate outputs, and reveal sensitive training data. This section also reviews defenses against these threats, including \textbf{jailbreak} and \textbf{backdoor defenses}, as well as \textbf{intellectual property protection} techniques.


\subsection{Adversarial Attacks}
\label{sec:dm_adversarial_attacks}

Adversarial attacks on diffusion models typically perturb text prompts to degrade image quality or cause semantic mismatches with the original text. This section reviews existing adversarial attacks, categorized by threat model into \textbf{white-box}, \textbf{gray-box}, and \textbf{black-box} methods.

\begin{table*}[htp]
\center
\caption{A summary of attacks and defenses for Diffusion Models (Part I).}
\label{tab:diffuison_safety_I}
\rowcolors{2}{gray!15}{white}
\resizebox{1\textwidth}{!}{
\begin{tabular}{p{0.09\textwidth}p{0.15\textwidth}p{0.05\textwidth}p{0.15\textwidth}p{0.17\textwidth}p{0.28\textwidth}p{0.3\textwidth}}
\hline
\rowcolor{gaoyifeng-pink}
Attack/Defense & Method & Year & Category & Subcategory & Target Models & Dataset\\ \hline
\cellcolor{white} & ECB\cite{struppek2023exploiting} & 2024 & Black-box & Character-level & Stable Diffusion, DALL-E 2, AltDiffusion-m18 & LAION-Aesthetics v2, MS COCO, ImageNet-V2, self-constructed  \\ 
\cellcolor{white} & CharGrad\cite{kou2023character} & 2023 & Black-box & Character-level & Stable Diffusion & MS COCO, Flickr30k \\ 
\cellcolor{white} & ER\cite{gao2023evaluating} & 2023 & Black-box & Character-level & Stable Diffusion, DALL·E 2 & LAION-COCO, DiffusionDB, SBU Corpus, self-constructed  \\ 
\cellcolor{white} & DHV\cite{daras2022discovering} & 2022 & Black-box & Word-level & DALLE-2 & -  \\ 
\cellcolor{white} & AA\cite{milliere2022adversarial} & 2022 & Black-box & Word-level & DALL-E 2, DALL-E mini & - \\ 
\cellcolor{white} & BBA\cite{maus2023black} & 2023 & Black-box & Sentence-level & Stable Diffusion & ImageNet \\
\cellcolor{white} & RIATIG\cite{liu2023riatig} & 2023 & Black-box & Sentence-level & DALL·E, DALL·E 2, Imagen & MS COCO \\

\cellcolor{white} & QFA\cite{zhuang2023pilot} & 2023 & Grey-box & Similarity-driven  & Stable Diffusion & self-constructed \\
\cellcolor{white} & RVTA\cite{zhang2024revealing} & 2024 & Grey-box & Similarity-driven & Stable Diffusion & ImageNet, self-constructed \\

\cellcolor{white} & ATM\cite{du2024stable} & 2023 & White-box & Classifier-driven  & Stable Diffusion & ImageNet, self-constructed \\
\cellcolor{white}\multirow{-12}{0.1\textwidth}{Adversarial Attack} & SAGE\cite{liu2023discovering} & 2023 & White-box & Classifier-driven  & GLIDE, Stable Diffusion, DeepFloyd & ImageNet \\


\hline
% \multirow{8}{*}{\makecell[l]{Adversarial \\ Defense}} & DiffPure\cite{nie2022diffusion} & 2022 & Purification & Diffuse \& Denoise  & ResNet & CIFAR-10, ImageNet \\
% & Purify++\cite{zhang2023purify++} & 2023 & Purification & Diffuse \& Denoise  & ResNet, WideResNet  & CIFAR-10\\
% & DifFilter\cite{chen2024diffilter} & 2024 & Purification & Diffuse \& Denoise  & ResNet, WideResNet  & CIFAR-10, ImageNet\\
% & CGDMP\cite{bai2024diffusion} & 2024 & Purification & Diffuse \& Denoise  & ResNet, WideResNet & CIFAR-10, CIFAR-100 \\
% & ADBM\cite{li2024adbm} & 2024 & Purification & Noise Generation  & ResNet, WideResNet & CIFAR-10, SVHN, Tiny-ImageNet, ImageNet-100\\
% & MimicDiffusion\cite{song2024mimicdiffusion} & 2024 & Purification & Guided Generation  & WideResNet & CIFAR-10, CIFAR-100\\
% & OSCP\cite{lei2024instant} & 2024 & Purification & Diffuse \& Denoise  & ResNet & ImageNet \\
% & LightPure\cite{khalili2024lightpure} & 2024 & Purification & Diffuse \& Denoise  & ResNet & CIFAR-10, GTSRB, Tiny-ImageNet\\
% \hline

\cellcolor{white} & SneakyPrompt\cite{yang2024sneakyprompt} & 2023 & Black-box & Target External Defenses & Stable Diffusion, DALL·E 2 & NSFW-200, Dog/Cat-100 \\
\cellcolor{white} & UD\cite{qu2023unsafe} & 2023 & Black-box & Target External Defenses & Stable Diffusion, LD, DALL·E 2, DALL·E mini & MS COCO \\
\cellcolor{white} & {Atlas}\cite{dong2024jailbreaking} & 2024 & Black-box & Target External Defenses & Stable Diffusion, DALL·E 3 & NSFW-200, Dog/Cat-100 \\ 
\cellcolor{white} & {Groot}\cite{liu2024groot} & 2024 & Black-box & Target External Defenses & Stable Diffusion, Midjounery, DALL·E 3 & self-constructed \\ 
\cellcolor{white} & {DACA}\cite{deng2023divide} & 2024 & Black-box & Target External Defenses & Midjounery, DALL·E 3 & VBCDE-100, Copyright-20 \\ 
\cellcolor{white} & {SurrogatePrompt}\cite{ba2023surrogateprompt} & 2024 & Black-box & Target External Defenses & Midjourney, DALL·E 2, DreamStudio & self-constructed \\
% \cellcolor{white} & {JPA}\cite{ma2024jailbreaking} & 2024 & Grey-box & Target Internal Defenses & Stable Diffusion, Midjourney, DALL· E 2, PIXART-$\alpha$, SLD, ESD, FMN & I2P \\
\cellcolor{white} & {JPA}\cite{ma2024jailbreaking} & 2024 & Grey-box & Target Internal Defenses & Stable Diffusion, Midjourney, DALL·E 2, PIXART-$\alpha$ & I2P \\
% \cellcolor{white} & {RT-Attack}\cite{gao2024rt} & 2024 & Grey-box & Target Internal Defenses & Stable Diffusion, DALL·E 3, SafeGen, SLD & I2P, self-constructed \\
\cellcolor{white} & {RT-Attack}\cite{gao2024rt} & 2024 & Grey-box & Target Internal Defenses & Stable Diffusion, DALL·E 3, SafeGen & I2P, self-constructed \\
\cellcolor{white} & RTSDSF\cite{rando2022red} & 2022 & White box & Target External Defenses & Stable Diffusion & self-constructed \\ 
% \cellcolor{white} & {MMA}\cite{yang2024mma} & 2023 & White box & Target External Defenses & Stable Diffusion, Midjounery, Leonardo.Ai, SLD & LAION-COCO, UnsafeDiff \\
\cellcolor{white} & {MMA}\cite{yang2024mma} & 2024 & White box & Target External Defenses & Stable Diffusion, Midjounery, Leonardo.Ai & LAION-COCO, UnsafeDiff \\
% \cellcolor{white} & {P4D}\cite{chin2023prompting4debugging} & 2023 & White box & Target Internal Defenses & Stable Diffusion, SLD, ESD & I2P, ESD Dataset \\
\cellcolor{white} & {P4D}\cite{chin2023prompting4debugging} & 2024 & White box & Target Internal Defenses & Stable Diffusion & I2P, ESD Dataset \\
% \cellcolor{white}\multirow{-14}{0.1\textwidth}{Jailbreak Attack} & {UnlearnDiffAtk}\cite{zhang2023generate} & 2023 & White box & Target Internal Defenses & Stable Diffusion, SLD, ESD, FMN, AC, UCE & I2P Dataset, ImageNet, WikiArt \\
\cellcolor{white}\multirow{-14}{0.1\textwidth}{Jailbreak Attack} & {UnlearnDiffAtk}\cite{zhang2023generate} & 2024 & White box & Target Internal Defenses & Stable Diffusion & I2P Dataset, ImageNet, WikiArt \\
\hline
 
\cellcolor{white} & ESD\cite{gandikota2023erasing} & 2023 & Concept Erasure & Fine-tuning & Stable Diffusion & MS COCO, I2P \\ 
\cellcolor{white} & SPM\cite{lyu2024one} & 2024 & Concept Erasure & Fine-tuning & Stable Diffusion & MS COCO, I2P \\ 
\cellcolor{white} & SDD\cite{kim2023towards} & 2023 & Concept Erasure  & Fine-tuning & Stable Diffusion & MS COCO, I2P \\ 
\cellcolor{white} & AC\cite{kumari2023ablating} & 2023 & Concept Erasure & Fine-tuning & Stable Diffusion & MS COCO \\ 
\cellcolor{white} & ABO\cite{hong2024all} & 2023 & Concept Erasure & Fine-tuning & Stable Diffusion & MS COCO \\ 
\cellcolor{white} & UC\cite{wu2024unlearning} & 2024 & Concept Erasure & Fine-tuning & Stable Diffusion & I2P \\ 
\cellcolor{white} & SA\cite{heng2024selective} & 2023 & Concept Erasure & Fine-tuning & Stable Diffusion, DDPM & MNIST, CIFAR-10 and STL-10, I2P\\ 
\cellcolor{white} & Receler\cite{huang2023receler} & 2024 & Concept Erasure & Fine-tuning & Stable Diffusion & CIFAR-10, MS COCO, I2P \\ 
\cellcolor{white} & RACE\cite{kim2024race} & 2024 & Concept Erasure & Fine-tuning & Stable Diffusion & MS COCO, I2P, Imagenette\\ 
\cellcolor{white} & AdvUnlearn\cite{zhang2024defensive} & 2024 & Concept Erasure & Fine-tuning & Stable Diffusion & MS COCO, I2P, Imagenette \\ 
\cellcolor{white} & DT\cite{ni2023degeneration} & 2023 & Concept Erasure & Fine-tuning & Stable Diffusion & MS COCO \\ 
\cellcolor{white} & FMO\cite{zhang2024forget} & 2023 & Concept Erasure & Fine-tuning & Stable Diffusion & ConceptBench \\
\cellcolor{white} & Geom-Erasing\cite{liu2024implicit} & 2024 & Concept Erasure & Fine-tuning & Stable Diffusion & LAION \\
\cellcolor{white} & SepME\cite{zhao2024separable} & 2024 & Concept Erasure & Fine-tuning & Stable Diffusion & self-constructed \\
\cellcolor{white} & CCRT~\cite{han2024continuous} & 2024 & Concept Erasure & Fine-tuning & Stable Diffusion & MS COCO \\
\cellcolor{white} & MACE\cite{lu2024mace} & 2024 & Concept Erasure & Close-Formed Solution & Stable Diffusion & CIFAR-10, MS COCO, I2P \\
\cellcolor{white} & UCE\cite{gandikota2024unified} & 2024 & Concept Erasure & Close-Formed Solution & Stable Diffusion & MS COCO \\
\cellcolor{white} & TIME\cite{orgad2023editing} & 2023 & Concept Erasure & Close-Formed Solution & Stable Diffusion & MS COCO \\
\cellcolor{white} & RECE\cite{gong2024reliable} & 2024 & Concept Erasure & Close-Formed Solution & Stable Diffusion & MS COCO, I2P \\
\cellcolor{white} & RealEra\cite{liu2024realera} & 2024 & Concept Erasure & Close-Formed Solution & Stable Diffusion & CIFAR-10, I2P \\
\cellcolor{white} & CP\cite{chavhan2024conceptprune} & 2024 & Concept Erasure & Neuron Pruning & Stable Diffusion & Imagenette \\
\cellcolor{white} & PRCEDM\cite{yang2024pruning} & 2024 & Concept Erasure & Neuron Pruning & Stable Diffusion & Imagenet, MS COCO, I2P \\ 
\cellcolor{white} & SLD\cite{schramowski2023safe} & 2023 & Inference Guidance & Input & Stable Diffusion & LAION-2B-en, I2P, DrawBench \\ 
\cellcolor{white} & Ethical-Lens\cite{cai2024ethical} & 2025 & Inference Guidance & Input\&Output & Stable Diffusion, Dreamlike Diffusion & MS COCO, I2P, Tox100, Tox1K, HumanBias, Demographic Stereotypes, Mental Disorders \\ 
\cellcolor{white}\multirow{-24}{0.1\textwidth}{Jailbreak \\ Defense} & SDIDLD\cite{li2024self} & 2024 & Inference Guidance & Latent space & Stable Diffusion & MS COCO, I2P, CelebA, Winobias, self-constructed  \\ 
\hline



\cellcolor{white} & {BadDiffusion} \cite{chou2023backdoor} & 2023 & Training Manipulation & Visual Trigger & DDPM & CIFAR-10, CelebA \\ 
\cellcolor{white} & {VillanDiffusion}\cite{chou2024villandiffusion} & 2023 & Training Manipulation & Visual Trigger & Stable Diffusion, DDPM, LDM, NCSN & CIFAR-10, CelebA \\ 
\cellcolor{white} & {TrojDiff}\cite{chen2023trojdiff} & 2023 & Training Manipulation & Visual Trigger & DDPM, DDIM & CIFAR-10, CelebA \\
\cellcolor{white} & {IBA}\cite{li2024invisible} & 2024 & Training Manipulation & Visual Trigger & $\text{Unconditional and Conditional DM}^*$ & CIFAR-10,  CelebA, MS-COCO \\
\cellcolor{white} & {DIFF2}\cite{li2024watch} & 2024 & Training Manipulation & Visual Trigger & DDPM, DDIM, Stable DiffusionE, ODE & CIFAR-10, CIFAR-100, CelebA, ImageNet \\
\cellcolor{white} & {RA}\cite{struppek2023rickrolling} & 2023 & Data Poisoning & Textual Trigger & Stable Diffusion & LAION-Aesthetics v2, MS-COCO \\
\cellcolor{white} & {BadT2I}\cite{zhai2023text} & 2023 & Data Poisoning & Textual Trigger & Stable Diffusion & LAION-Aesthetics v2, LAION-2B-en, MS COCO\\
\cellcolor{white} & {FTHCW}\cite{pan2023trojan} & 2024 & Data Poisoning & Textual Trigger & DDPM, LDM & CIFAR-10, ImageNet, Caltech256 \\
\cellcolor{white} & {BAGM}\cite{vice2024bagm} & 2023 & Data Poisoning & Textual Trigger & Stable Diffusion, Kandinsky, DeepFloyd-IF & MS COCO, Marketable Food \\
\cellcolor{white} & {Zero-Day}\cite{huang2023zero, huang2024personalization} & 2023 & Data Poisoning & Textual Trigger & Stable Diffusion & DreamBooth dataset \\
\cellcolor{white} & {SBD}\cite{wang2024stronger} & 2024 & Data Poisoning & Textual Trigger & Stable Diffusion & LAION Aesthetics v2, Pokemon Captions, COYO-700M, Midjourney v5 \\
\cellcolor{white}\multirow{-13}{0.1\textwidth}{Backdoor \\ Attack} & {IBT} \cite{naseh2024injecting} & 2024 & Data Poisoning & Textual Trigger & Stable Diffusion & Midjourney Dataset, DiffusionDB, PartiPrompts\\
\hline


\cellcolor{white} & T2IShield\cite{wang2024t2ishield} & 2024 & Detection & Trigger Detection & Stable Diffusion & CelebA-HQ-Dialog \\
\cellcolor{white} & Ufid\cite{guan2024ufid} & 2024 & Detection & Trigger Validation & DDPM, Stable Diffusion & CelebA-HQ-Dialog, Pokemon, \\
\cellcolor{white} & DisDet\cite{sui2024disdet} & 2024 & Detection & Trigger Validation & DDPM, DDIM & CIFAR-10, CelebA  \\
\cellcolor{white} & Elijah\cite{an2024elijah} & 2024 & Removal & Detect \& Remove & DDPM, DDIM, LDM & CIFAR-10, CelebA-HQ \\
\cellcolor{white} & Diff-Cleanse\cite{hao2024diff} & 2024 & Removal & & DDPM, DDIM, LDM & MNIST, CIFAR-10, CelebA-HQ \\
\cellcolor{white} & TERD\cite{mo2024terd} & 2024 & Removal & Inverse \& Remove & DDPM & CIFAR-10, CelebA, CelebA-HQ \\
\cellcolor{white}\multirow{-8}{0.1\textwidth}{Backdoor \\ Defense} & PureDiffusion\cite{truong2024purediffusion} & 2024 & Removal & Inverse \& Remove & DDPM & CIFAR-10\\
\hline

\end{tabular}
}
\end{table*}

\subsubsection{White-box Attacks}
White-box attacks on T2I diffusion models assume full access to model parameters, allowing direct optimization of text prompts or latent space to degrade or disrupt image generation. For example, \textbf{SAGE} \cite{liu2023discovering} explores both the discrete prompt and latent spaces to uncover failure modes in T2I models, including distorted generations and targeted manipulations. \textbf{ATM} \cite{du2024stable} generates attack prompts similar to clean prompts by replacing or extending words using Gumbel Softmax, preventing the model from generating desired subjects.


\subsubsection{Gray-box Attacks}
\label{subsubsec:Similarity-based optimization}
Gray-box attacks assume the CLIP text encoder used in many T2I diffusion models is frozen and publicly available. The attacker can then exploit CLIP similarity loss to craft adversarial text prompts targeting the text encoder.


\textbf{QFA} \cite{zhuang2023pilot} minimizes cosine similarity between original and perturbed text embeddings to generate images that differ as much as possible from the original text. \textbf{RVTA} \cite{zhang2024revealing} maximizes image-text similarity to align adversarial prompts with reference images generated by a surrogate diffusion model. 
% Notably, both methods utilize mask strategies to localize the attack’s effects, with QFA focusing on identifying steerable key dimensions in the embedding space to precisely perturb the target semantic elements, and RVTA employing an object-background decoupling mechanism to preserve object semantics while altering other attributes like style.
\textbf{MMP-Attack}~\cite{yang2024multi} simultaneously maximizes the cosine similarity between the perturbed text embedding and the target embedding in both the text and image modalities, while employing a straight-through estimator to execute the optimization process.



\subsubsection{Black-box Attacks}
Black-box attacks assume the attacker has no knowledge of the victim diffusion model's internals (parameters or architecture). Since diffusion models use text prompts as input, existing attacks employ textual adversarial techniques to evade the model. These attacks can be further categorized by granularity into \textbf{character-level}, \textbf{word-level}, and \textbf{sentence-level} attacks.

\textbf{Character-level Attacks} modify the characters in the text input to create adversarial prompts. \textbf{ECB} \cite{struppek2023exploiting} shows how replacing characters with homoglyphs, such as using Hangul or Arabic scripts, shifts generated images toward cultural stereotypes. Subsequent works, like \textbf{CharGrad} \cite{kou2023character}, optimize character-level perturbations using gradient-based attacks and proxy representations to map character changes to embedding shifts. \textbf{ER} \cite{gao2023evaluating} uses distribution-based objectives (e.g., MMD, KL divergence) to maximize discrepancies in image distributions, enhancing attack effectiveness. These attacks exploit typos, homoglyphs, and phonetic modifications, disrupting text-to-image outputs.


\textbf{Word-level Attacks} craft adversarial prompts by replacing or adding words to the input text. \textbf{DHV} \cite{daras2022discovering} uncovers a hidden vocabulary in diffusion models, where nonsensical strings like \texttt{Apoploe vesrreaitais} can generate bird images, due to their proximity to target concepts in the CLIP text embedding space. Building on this, \textbf{AA} \cite{milliere2022adversarial} introduces macaronic prompting, combining word fragments from different languages to control visual outputs systematically. These attacks reveal vulnerabilities in the relationship between text embeddings and image generation.

\textbf{Sentence-level Attacks} rewrite a substantial part or the entire prompt to create adversarial prompts. \textbf{RIATIG} \cite{liu2023riatig} uses a CLIP-based image similarity measure as an optimization objective and a genetic algorithm to iteratively mutate and select text prompts, creating adversarial examples that resemble the target image while remaining semantically different from the original text. In contrast, \textbf{BBA} \cite{maus2023black} employs classification loss and black-box optimization to refine prompts, using Token Space Projection (TPS) to bridge the gap between continuous word embeddings and discrete tokens, enabling the generation of category-specific images without explicit category terms.


\subsection{Jailbreak Attacks}
\label{sec:dm_jailbreak_attacks}

Diffusion models use both internal and external safety mechanisms to void the generation of Not Safe For Work (NSFW) content. Internal safety mechanisms often refer to the inherent robustness of T2I diffusion models, achieved through safety alignment during training, which aims to reduce the likelihood of generating harmful content. External safety mechanisms, on the other hand, are safety filters, such as text, image, or text-image classifiers, applied to detect and block unsafe outputs after generation. 
Jailbreak attacks aim to craft adversarial prompts that bypass the safety mechanisms of diffusion models, enabling the generation of harmful content. This section provides a systematic review of existing jailbreak methods, categorized by threat model into \textbf{white-box}, \textbf{gray-box}, and \textbf{black-box} attacks.



\subsubsection{White-box Attacks}
White-box attacks can bypass the safety mechanisms in T2I diffusion models through gradient-based optimization. These attacks can be further classified into \textbf{internal safety attacks} and \textbf{external safety attacks}, each exploiting specific vulnerabilities in the victim models.


\textbf{Internal Safety Attacks} target the internal safety mechanisms of diffusion models.
Jailbreaking internally safety-enhanced diffusion models involves regenerating NSFW content by bypassing the removal of harmful concepts. The red teaming tool \textbf{P4D} \cite{chin2023prompting4debugging} automatically identifies problematic prompts to exploit limitations in current safety evaluations, aligning the predicted noise of an unconstrained model with that of a safety-enhanced one. \textbf{UnlearnDiffAtk} \cite{zhang2023generate} introduces an evaluation framework that uses unlearned diffusion models' classification capabilities to optimize adversarial prompts, aligning predicted noise with a target unsafe image to force the model to recreate NSFW content during denoising.


\textbf{External Safety Attacks} target the safety filters of diffusion models, aiming to bypass both input and output safety mechanisms. \textbf{RTSDSF} \cite{rando2022red} reverse-engineered predefined NSFW concepts in filters by using the CLIP model to encode and compare NSFW vocabulary embeddings, performing a dictionary attack. It also showed that prompt dilution—adding irrelevant details—can bypass safety filters. \textbf{MMA} \cite{yang2024mma} employs a similarity-driven loss to optimize adversarial prompts and introduce subtle perturbations to input images, bypassing both prompt filters and post-hoc safety checkers during image editing.


\subsubsection{Gray-box Attacks}
Gray-box jailbreak attacks assume that attackers have full access only to the open-source text encoder, with other components of the diffusion model remaining inaccessible. In this scenario, the attacker exploits the exposed text encoder to bypass the model's internal safety mechanism.

\textbf{Internal Safety Attacks}, under the gray-box setting, target models with `concept erasure'. \textbf{Ring-A-Bell} \cite{tsai2023ring} extracts unsafe concepts by comparing antonymous prompt pairs, generates harmful prompts with soft prompts, and refines them using a genetic algorithm. 
\textbf{JPA} \cite{ma2024jailbreaking} leverages antonyms like \texttt{“nude”} and \texttt{“clothed”}, calculating their average difference in the text embedding space to represent NSFW concepts, then optimizes prefix prompts for semantic alignment. \textbf{RT-Attack} \cite{gao2024rt} uses a two-stage strategy to maximize textual similarity to NSFW prompts and iteratively refines them based on image-level similarity, demonstrating that even limited knowledge can enable attacks on safety-enhanced models.

\subsubsection{Black-box Attacks}

Black-box jailbreaks on diffusion models target commercial models with access only to outputs, such as filter rejections or generated image quality and semantics, and are primarily \textbf{external safety attacks}. 

\textbf{External Safety Attacks}, in the black-box setting, use hand-crafted or LLM-assisted adversarial prompts to mislead the victim model to generate NSFW content.
\textbf{UD} \cite{qu2023unsafe} highlights the risk of T2I models generating unsafe content, especially hateful memes, by refining unsafe prompts manually. \textbf{SneakyPrompt} \cite{yang2024sneakyprompt} uses reinforcement learning to optimize adversarial prompts, which updates its policy network based on filter evasion and semantic alignment. Other methods employ LLMs to refine adversarial prompts.  \textbf{Groot} \cite{liu2024groot} decomposes prompts into objects and attributes to dilute sensitive content. \textbf{DACA} \cite{deng2023divide} breaks down and recombines prompts using LLMs. \textbf{SurrogatePrompt} \cite{ba2023surrogateprompt} targets Midjourney, substituting sensitive terms and leveraging image-to-text modules to generate harmful content at scale. \textbf{Atlas} \cite{dong2024jailbreaking} automates the attack with a two-agent system: one VLM generates adversarial prompts, while an LLM evaluates and selects the best candidates. These LLM-assisted strategies can significantly improve the effectiveness and stealthiness of the attacks.



\subsection{Jailbreak Defenses}
\label{sec:dm_jailbreak_defenses}

This section reviews existing defense strategies proposed for T2I diffusion models against jailbreak attacks, including \textbf{concept erasure} and \textbf{inference guidance}. 
The key challenge of these defenses is how to ensure safety while maintaining generation quality.


\subsubsection{Concept Erasure} 
Concept erasure is an emerging research area focused on removing undesirable concepts (e.g., NSFW content and copyrighted styles) from diffusion models, where these concepts are referred to as \emph{target concepts}. Concept erasure methods can be categorized into three types: \textbf{finetuning-based}, \textbf{close-form solution}, and \textbf{pruning-based}, depending on the strategy employed.

\paragraph{Finetuning-based Methods} 
These methods use gradient-based optimization to adjust model parameters, typically involving a loss function with an erasure term to prevent the generation of representations linked to the target (undesirable) concept, and a constraint term to preserve non-target concepts. These approaches can be categorized into \textbf{anchor-based}, \textbf{anchor-free}, and \textbf{adversarial} erasure methods.


\textbf{Anchor-based Erasing} is a targeted approach that guides the model to shift the target (undesirable concept) towards a good concept (anchor) by aligning predicted latent noise. \textbf{AC} \cite{kumari2023ablating} defines anchor concepts as broader categories encompassing the target concepts (e.g., \texttt{“Grumpy Cat”} → \texttt{“Cat”}) and uses standard diffusion loss on text-image pairs of anchors to preserve their integrity while erasing target concepts. 
% Subsequent works refined the definition and alignment of anchor concepts. 
\textbf{ABO} \cite{hong2024all} removes specific target concepts by modifying classifier guidance, using both explicit (replacing the target with a predefined substitute) and implicit (suppressing attention maps) erasing signals, and includes a penalty term to maintain generation quality. 
\textbf{DoCo} \cite{wu2024unlearning} improves generalization by aligning target and anchor concepts through adversarial training and mitigating gradient conflicts with concept-preserving gradient surgery. \textbf{SPM} \cite{lyu2024one} uses a 1D adapter and negative guidance \cite{gandikota2023erasing} to suppress target concepts while ensuring non-target concepts remain consistent, affecting only relevant synonyms. 
\textbf{SA} \cite{heng2024selective} applies generative replay and elastic weight consolidation to stabilize model weights and maintain normal generation capabilities while preserving non-target concepts.

\textbf{Anchor-free Erasing} is a non-targeted fine-tuning approach that reduces the probability of generating target concepts without aligning to a specific safe concept. 
% Instead, it aims to directly reduce the probability of the generated image containing the target concept. 
\textbf{ESD} \cite{gandikota2023erasing} modifies classifier-free guidance into negative-guided noise prediction to minimize the target concept’s generation probability (e.g., "Van Gogh"). \textbf{SDD} \cite{kim2023towards} addresses the extra effects of ESD’s negative guidance by using unconditioned predictions and EMA to avoid catastrophic forgetting. \textbf{DT} \cite{ni2023degeneration} erases unsafe concepts by training the model to denoise scrambled low-frequency images. \textbf{Forget-Me-Not} \cite{zhang2024forget} uses Attention Resteering to minimize intermediate attention maps related to the target concept. \textbf{Geom-Erasing} \cite{liu2024implicit} erases implicit concepts like watermarks by applying a geometric-driven control method and introduces the \emph{Implicit Concept Dataset}. \textbf{SepME} \cite{zhao2024separable} advances multiple concept erasure and restoration. Fuchi et al. \cite{fuchi2024erasing} proposed few-shot unlearning by targeting the text encoder rather than the image encoder or diffusion model.
\textbf{CCRT} \cite{han2024continuous} proposes a method for continuous removal of diverse concepts from diffusion models.


\textbf{Adversarial Erasing} enhances previous methods by introducing perturbations to the target concept's text embedding and using adversarial training to improve robustness. \textbf{Receler} \cite{huang2023receler} employs a lightweight eraser and adversarial prompt embeddings, iteratively training against each other, while applying a binary mask from U-Net attention maps to target only the concept regions. \textbf{AdvUnlearn} \cite{zhang2024defensive} shifts adversarial attacks to the text encoder, targeting the embedding space and using regularization to preserve normal generation. \textbf{RACE} \cite{kim2024race} improves efficiency by conducting adversarial attacks at a single timestep, reducing computational complexity. These methods enhance the model’s resistance to adversarial prompts aimed at regenerating erased concepts.


\paragraph{Close-form Solution Methods} 
These methods offer an efficient alternative to fine-tuning-based erasure, focusing on localized updates in cross-attention layers to erase target concepts, inspired by model editing in LLMs \cite{meng2022mass}. Unlike fine-tuning, which aligns denoising predictions, these methods align cross-attention values. 
\textbf{TIME} \cite{orgad2023editing} applies a closed-form solution to debias models, while \textbf{UCE} \cite{gandikota2024unified} extends this to multiple erasure targets, preserving surrounding concepts to reduce interference. \textbf{MACE} \cite{lu2024mace} refines cross-attention updates with LoRA and Grounded-SAM \cite{kirillov2023segment,liu2023grounding} for region-specific erasure. A recent challenge is that erased concepts can still be generated via sub-concepts or synonyms \cite{liu2024realera}.
\textbf{RealEra} \cite{liu2024realera} tackles this by mining associated concepts and adding perturbations to the embedding, expanding the erasure range with beyond-concept regularization. 
\textbf{RECE} \cite{gong2024reliable} addresses insufficient erasure by continually finding new concept embeddings during fine-tuning and applying closed-form solutions for further erasure.

\paragraph{Pruning-based Methods} 
These methods erase target concepts by identifying and removing neurons strongly associated with the target, selectively disabling them without updating model weights.
\textbf{ConceptPrune} calculates a Wanda score using target and reference prompts to measure each neuron's contribution, pruning those most associated with the target concept. Similarly, another approach \cite{yang2024pruning} identifies concept-correlated neurons using adversarial prompts to enhance the robustness of existing erasure methods.


\subsubsection{Inference Guidance} 
Inference guidance methods steer pre-trained diffusion models to generate safe images by incorporating additional auxiliary information and specific guidance during the inference process.
 
\paragraph{Input Guidance} 
This type of guidance use additional input text to steer the model toward safe content. 
\textbf{SLD} \cite{schramowski2023safe} adjusts noise predictions during inference based on a text condition and unsafe concepts, guiding generation towards the intended prompt while avoiding unsafe content, without requiring fine-tuning. It also introduces the I2P benchmark, a dataset for testing inappropriate content generation.

\paragraph{Input \& Output Guidance}
This type of methods prevent harmful inputs and control NSFW outputs. \textbf{Ethical-Lens} \cite{cai2024ethical} employs a plug-and-play framework, using an LLM for input text revision (Ethical Text Scrutiny) and a multi-headed CLIP classifier for output image modification (Ethical Image Scrutiny), ensuring alignment with societal values without retraining or internal changes.


\paragraph{Latent space Guidance} 
This approach uses additional implicit representations in the latent space to guide generation. \textbf{SDIDLD} \cite{li2024self} employs self-supervised learning to identify the opposite latent direction of inappropriate concepts (e.g., "anti-sexual") and adds these vectors at the bottleneck layer, preventing harmful content generation.


\subsection{Backdoor Attacks}
\label{sec:dm_backdoor_attacks}

Backdoor attacks on diffusion models allow adversaries to manipulate generated content by injecting backdoor triggers during training. These "malicious triggers" are embedded in model components, and during generation, inputs with triggers (e.g., prompts or initial noise) guide the model to produce predefined content. The key challenge is enhancing attack success rates while keeping the trigger covert and preserving the model's original utility. Existing attacks can be categorized into \textbf{training manipulation} and \textbf{data poisoning} methods.

\subsubsection{Training Manipulation}
This type of attack typically assumes the attacker aims to release a backdoored diffusion model, granting control over the training or even inference processes. Existing attacks focus on the visual modality, inserting backdoors by using image pairs with triggers and target images (\emph{image-image pair injection}), typically targeting unconditional diffusion models.

\textbf{BadDiffusion} \cite{chou2023backdoor} presents the first backdoor attack on T2I diffusion models, which modifies the forward noise-addition and backward denoising processes to map backdoor target distributions to image triggers while maintaining DDPM sampling. \textbf{VillanDiffusion} \cite{chou2024villandiffusion} extends this to conditional models, adding prompt-based triggers and textual triggers for tasks like text-to-image generation. \textbf{TrojDiff} \cite{chen2023trojdiff} advances the research by controlling both training and inference, incorporating Trojan noise into sampling for diverse attack objectives. \textbf{IBA\cite{li2024invisible}} introduces invisible trigger backdoors using bi-level optimization to create covert perturbations that evade detection. \textbf{DIFF2} \cite{li2024watch} proposes a backdoor attack in adversarial purification, optimizing triggers to mislead classifiers and extending it to data poisoning by injecting backdoors directly.


\subsubsection{Data Poisoning}
Unlike training manipulation, data poisoning methods do not directly interfere with the training process, restricting the attack to inserting poisoned samples into the dataset. These attacks typically target conditional diffusion models and explore two types of textual triggers: \textbf{text-text pair} and \textbf{text-image pair}.


\textbf{Text-text Pair Triggers} consist of triggered prompts and their corresponding target prompts.  \textbf{RA} \cite{struppek2023rickrolling} adopts this approach to inject backdoors into the text encoder by adding a covert trigger character, mapping the original to the target prompt while preserving encoder functionality through utility loss optimization. The backdoored encoder generates embeddings with predefined semantics, guiding the diffusion model’s output. This lightweight attack requires no interaction with other model components. Several studies \cite{struppek2023rickrolling, vice2024bagm, huang2023zero, huang2024personalization} have also explored this approach. 

\textbf{Text-image Pair Triggers} consist of triggered prompts paired with target images. \textbf{BadT2I} \cite{zhai2023text} explores backdoors based on pixel, object, and style changes, where a special trigger (e.g., \texttt{“[T]”}) induces the model to generate images with specific patches, replaced objects, or styles. To reduce the data cost, \textbf{Zero-Day} \cite{huang2023zero,huang2024personalization} uses personalized fine-tuning, injecting trigger-image pairs for more efficient backdoors. \textbf{FTHCW} \cite{pan2023trojan} embeds target patterns into images from different classes, forming text-image pairs to generate diverse outputs. \textbf{IBT} \cite{naseh2024injecting} uses two-word triggers that activate the backdoor only when both words appear together, enhancing stealthiness. In commercial settings, \textbf{BAGM} \cite{vice2024bagm} manipulates user sentiment by mapping broad terms (e.g., “drinks”) to specific brands (e.g., “Coca Cola”). \textbf{SBD} \cite{wang2024stronger} employs backdoors for copyright infringement, bypassing filters by decomposing and reassembling copyrighted content using text-image pairs.


\subsection{Backdoor Defenses}
\label{sec:dm_backdoor_defenses}

Backdoor defenses for diffusion models is an emerging area of research. Current approaches generally follow a three-step pipeline: 1) \textbf{trigger inversion}, 2) \textbf{trigger validation} or \textbf{backdoor detection}, and 3) \textbf{backdoor removal}. Some works propose complete frameworks, while others focus on individual steps.


\subsubsection{Backdoor Detection}
Most early research focuses on detecting or validating backdoor triggers. \textbf{T2IShield} \cite{wang2024t2ishield} is the first backdoor detection and mitigation framework for diffusion models, leveraging the \emph{assimilation phenomenon} in cross-attention maps, where a trigger suppresses other tokens to generate specific content. \textbf{Ufid} \cite{guan2024ufid} validates triggers by noting that clean generations are sensitive to small perturbations, while backdoor-triggered outputs are more robust. \textbf{DisDet} \cite{sui2024disdet} proposes a low-cost detection method that distinguishes poisoned input noise from clean Gaussian noise by identifying distribution shifts.


\subsubsection{Backdoor Removal}
While trigger validation confirms the presence of a backdoor trigger, the identified triggers must still be removed from the victim model.
Most backdoor removal methods first invert the trigger and then eliminate the backdoor using the inverted trigger. 
\textbf{Elijah} \cite{an2024elijah} introduces a backdoor removal framework for diffusion models, inverting triggers through distribution shifts and aligning the backdoor's distribution with the clean one. \textbf{Diff-Cleanse} \cite{hao2024diff} formulates trigger inversion as an optimization problem with similarity and entropy loss, followed by pruning channels critical to backdoor sampling. \textbf{TERD} \cite{mo2024terd} proposes a unified reverse loss for trigger inversion, using a two-stage process for coarse and refined inversion. \textbf{PureDiffusion} \cite{truong2024purediffusion} employs multi-timestep trigger inversion, leveraging the consistent distribution shift caused by backdoored forward processes.



\begin{table*}[htp]
\center
\caption{A summary of attacks and defenses for Diffusion Models (Part II).}
\label{tab:diffuison_safety_II}
\rowcolors{2}{gray!15}{white}
\resizebox{1\textwidth}{!}{
\begin{tabular}{p{0.1\textwidth}p{0.15\textwidth}p{0.05\textwidth}p{0.18\textwidth}p{0.17\textwidth}p{0.2\textwidth}p{0.3\textwidth}}
\hline
\rowcolor{gaoyifeng-pink}
Attack/Defense & Method & Year & Category & SubCategory & Target Model & Dataset \\ \hline

\cellcolor{white} & WuMI\cite{wu2022membership} & 2022 & Black-box & Reconstruction-error & LDM DALL-E mini & MSCOCO, VG, LAION-400M, CC3M \\ 
\cellcolor{white} & DiffusionLeaks\cite{matsumoto2023membership} & 2023 & Black/White-box & Reconstruction-error  & DDIM, & CIFAR-10, CelebA   \\ 
\cellcolor{white} & PangMI\cite{pang2023black} & 2024 & Black-box & Auxilary Dataset & Stable Diffusion & CelebA-Dialog, WIT, MSCOCO \\ 
\cellcolor{white} & LiMI\cite{li2024towards} & 2024 & Black-box & Reconstruction-error& DDIM, Stable Diffusion DiT & CIFAR-10, STL10-U, LAION-5B, LAION-by-DALL-E  \\ 
\cellcolor{white} & DRC\cite{fu2024model} & 2024 & Black-box & Reconstruction-error&DDPM, DDIM&FFHQ, CelebA, CIFAR-10, CIFAR-100 \\ 
\cellcolor{white} & GMIA\cite{zhang2024generated} & 2023 & Black-box & Auxilary Dataset &DDPM, DDIM, FastDPM &CIFAR-10, CelebA\\
\cellcolor{white} & SecMI\cite{duan2023diffusion} & 2023 & Gray-box & Posterior Likelihood & DDPM, DDIM, Stable Diffusion & CIFAR-10/100, STL10-U, Tiny-ImageNet, Pokemon, COCO2017-val, LAION-5B \\
\cellcolor{white} & QRMI\cite{tang2023membership} & 2023 & Gray-box & Posterior Likelihood&DDPM, DDIM&  CIFAR-10/100, STL100, Tiny-ImageNet \\
\cellcolor{white} & PIA\cite{kong2023efficient} & 2023 & Gray-box & Posterior Likelihood &DDPM, DDIM, Stable Diffusion& CIFAR-10/100, Tiny-ImageNet, COCO2017, LAION-5B \\
\cellcolor{white} & PFAMI\cite{fu2023probabilistic} & 2024 & Gray-box & Posterior Likelihood &DDPM, VAE&CelebA, Tiny-ImageNet\\
\cellcolor{white} & ZhMI\cite{zhai2024membership} & 2024 & Gray-box & Conditional Likelihood & DDPM, DDIM,  Stable Diffusion&Pokemonn, Flickr, MSCOCO, LAION \\
\cellcolor{white} & SMIA\cite{li2024unveiling} & 2024 & Gray-box & Structural Similarity &LDM, Stable Diffusion& LAION2B,  LAION-400M\\
\cellcolor{white} & SLA\cite{matsumoto2023membership,hu2023loss} & 2023 & White-box & Loss &DDPM, DDIM&FFHQ, DRD, CelebA, FFHQ\\
\cellcolor{white} & GSA\cite{pang2023white} & 2024 & White-box & Gradient & DDPM&CIFAR-10, MSCOCO, ImageNet\\
\cellcolor{white}\multirow{-19}{0.1\textwidth}{Membership \\ Inference} & DuMI\cite{dubinski2024towards} & 2023 & White-box & Loss &Stable Diffusion&Pokemon, LAION-mi\\
\hline

\cellcolor{white} & BruteDE\cite{carlini2023extracting} & 2023 & Black-box &  Existing Condition &DDPM, Stable Diffusion&CIFAR-10 LAION-5B \\ 
\cellcolor{white} & ReDE\cite{webster2023reproducible} & 2023 & Black/White-box & Existing Condition & Stable Diffusion, Midjourney, Deep Image Floyd& LAION-5B \\ 
\cellcolor{white} & SIDE\cite{chen2024towards} & 2024  & White-box &  Surrogate Condition  &DDPM, DDIM &CIFAR-10, CelebA\\ 
\cellcolor{white} \multirow{-5}{0.1\textwidth}{Data Extraction} & FineXtract\cite{wu2024revealing} & 2024 & White-box &  Surrogate Condition &Finetuned Stable Diffusion& WikiArt\\ 
\hline

\cellcolor{white}\multirow{2}{0.15\textwidth}{Model Extraction} & 
SDeT\cite{horwitz2024recovering} & 2024  & White-box & LoRA-Based Model Extraction & Finetuned Stable Diffusion & LoWRA Bench  \\
\hline

\cellcolor{white} & DUAW\cite{ye2023duaw} & 2023 & Natural Data Protection & Learning Prevention & Stable Diffusion & DreamBooth dataset, WikiArt, self-constructed \\
\cellcolor{white} & AdvDM\cite{liang2023adversarial} & 2023 & Natural Data Protection & Learning Prevention & Stable Diffusion, LDM & LSUN, WikiArt \\
\cellcolor{white} & Anti-DreamBooth\cite{van2023anti} & 2023 & Natural Data Protection & Learning Prevention & Stable Diffusion & CelebA, VGGFace2\\
\cellcolor{white} &  MetaCloak\cite{liu2024metacloak} & 2024 & Natural Data Protection & Learning Prevention & Stable Diffusion & CelebA-HQ, VGGFace2 \\
\cellcolor{white} &  InMakr\cite{liu2024countering} & 2024 & Natural Data Protection & Learning Prevention & Stable Diffusion &  VGGFace2, WikiArt   \\
\cellcolor{white} &  SimAC\cite{wang2024simac} & 2024 & Natural Data Protection & Learning Prevention & Stable Diffusion & CelebA-HQ, VGGFace2  \\
\cellcolor{white} & EditGuard\cite{zhang2024editguard} & 2024 & Natural Data Protection & Editing Prevention & Stable Diffusion  & COCO   \\
\cellcolor{white} & WaDiff\cite{min2024watermark} & 2024 & Natural Data Protection & Editing Prevention & Stable Diffusion & COCO, ImageNet   \\
\cellcolor{white} & AdvWatermark\cite{zhu2024watermark} & 2024 & Natural Data Protection & Editing Prevention & Stable Diffusion & WikiArt \\
\cellcolor{white} & FT-SHIELD\cite{cui2023ft} & 2024 & Natural Data Protection & Data Attribution & Stable Diffusion & CelebA, WikiArt, Pokemon Captions, DreamBooth dataset \\
\cellcolor{white} & DiffusionShield\cite{cui2023diffusionshield} & 2024 & Natural Data Protection & Data Attribution & DDPM,, Stable Diffusion & CIFAR-10, CIFAR-100, STL-10, ImageNet \\
\cellcolor{white} &  ProMark\cite{asnani2024promark} & 2024 & Natural Data Protection & Data Attribution & LDM & Stock, LSUN, WikiArt, ImageNet\\
\cellcolor{white} & Diagnosis~\cite{wang2023diagnosis} & 2023 & Natural Data Protection & Data Attribution  & Stable Diffusion, VQ Diffusion & Pokemon, CelebA,  CUB-200, DreamBooth \\
\cellcolor{white} & HiDDeN\cite{zhu2018hidden} & 2018 & Generated Data Protection & Post-generation Watermark & CNN & MS-COCO, BOSS dataset \\
\cellcolor{white} & Stable Signature\cite{fernandez2023stable} & 2023 & Generated Data Protection & Diffusion Watermark & LDM & MS-COCO, ImageNet \\
\cellcolor{white} & LaWa\cite{rezaei2024lawa} & 2024 & Generated Data Protection & Diffusion Watermark & LDM & MIRFlickR \\
\cellcolor{white} & Safe-SD\cite{ma2024safe} & 2024 & Generated Data Protection & Diffusion Watermark & Stable Diffusion  & LSUN, COCO, FFHQ \\
\cellcolor{white} & RW\cite{zhao2023recipe} & 2023 & Model Protection & Model Watermark & Stable Diffusion, EDM & CIFAR-10, ImageNet, FFHQ, AFHQv2 \\
\cellcolor{white} & FIXEDWM\cite{liu2023watermarking} & 2023 & Model Protection & Model Watermark & LDM & MS COCO \\
\cellcolor{white} & WDM\cite{peng2023protecting} & 2023 & Model Protection & Model Watermark & DDPM, & CIFAR-10, CelebA, MNIST \\
\cellcolor{white} & AquaLoRA\cite{feng2024aqualora} & 2024 & Model Protection & Model Attribution & Stable Diffusion & COCO  \\
\cellcolor{white} & LatentTracer~\cite{wang2024trace} & 2024 & Model Protection & Model Attribution & Stable Diffusion, Kandinsky & LAION \\
\cellcolor{white} \multirow{-25}{0.1\textwidth}{Intellectual Property Protection}  & Tree-Ring\cite{wen2023tree} & 2023 & Model Protection & Model Attribution & Stable Diffusion, ImageNet diffusion & MS-COCO, ImageNet \\  
\hline

\end{tabular}
}
\end{table*}

Privacy attacks on diffusion models can be classified into \textbf{membership inference}, \textbf{data extraction}, and \textbf{model extraction} attacks. As attack sophistication increases, each type poses a growing threat to privacy.


\subsection{Membership Inference Attacks}
\label{sec:dm_membership_inference_attacks}

Membership inference attacks on diffusion models aim to infer sensitive data by exploiting their generative capabilities. Attackers use techniques like reconstruction error, shadow models, auxiliary data, likelihood, gradient, or structural similarity metrics. These attacks can be classified into six types: \textbf{reconstruction error-based}, \textbf{auxiliary dataset-based}, \textbf{loss-based}, \textbf{gradient-based}, \textbf{structural similarity-based}, and \textbf{likelihood-based}.

\textbf{Reconstruction Error-based Attacks} infer the membership of candidate samples by analyzing their reconstruction errors in the diffusion model. Wu et al. \cite{wu2022membership} proposed to determine membership in text-conditional diffusion models by comparing the reconstruction error between the candidate and generated images, and their semantic alignment with the text prompt.
Inspired by GAN-leaks \cite{chen2020gan}, Matsumoto et al. \cite{matsumoto2023membership} introduced Diffusion-leaks, which generates multiple candidate images and infers membership based on minimal reconstruction errors. Li et al. \cite{li2024towards} proposed to average multiple reconstructions to reduce errors and improve inference accuracy, utilizing black-box APIs to modify candidate images.
\textbf{DRC} \cite{fu2024model} degrades and restores images using the diffusion model, comparing the restored images to the originals to infer membership and sensitive features.

\textbf{Auxiliary Datasets-based Attacks} use auxiliary datasets to train shadow models, enabling black-box membership inference by simulating the target model. 
Pang et al. \cite{pang2023black} targeted fine-tuned conditional diffusion models, computing similarity scores between query images and generated images to train a binary classifier for membership inference. \textbf{GMIA} \cite{zhang2024generated} introduces the first generalized membership inference attack for generative models, using only generated distributions and auxiliary non-member datasets, assuming the generated distribution approximates the original training distribution.

\textbf{Loss-based Attacks} exploit loss value distributions to distinguish member from non-member samples, assuming lower losses for member (training) samples. \cite{hu2023loss} and \cite{matsumoto2023membership} used loss values at different timesteps for membership inference. These two attacks can be viewed as \textbf{Static Loss Attack} (SLA), as they ignore the diffusion process. Dubinski et al. \cite{dubinski2024towards} modified the diffusion process to extract loss information from multiple perspectives, improving inference accuracy.

\textbf{Gradient-based Attacks} leverage gradient information for membership inference. For instance, \textbf{GSA} \cite{pang2023white} infers a sample is a member if its gradients significantly differ from surrounding samples, indicating a stronger influence on the model's training.

\textbf{Structural Similarity-based Attacks} compare structural features or similarity metrics between candidate samples and model outputs. \textbf{SMIA} \cite{li2024unveiling} uses the Structure Similarity Index Measure (SSIM)\cite{wang2004image}
metric to assess how well an image's structure is preserved during diffusion, with the average SSIM difference between members and non-members used to infer membership.

\textbf{Likelihood-based Attacks} use posterior or conditional likelihoods to infer membership. \textbf{SecMI} \cite{duan2023diffusion} estimates posterior likelihoods via reverse processes to target DDPM and Stable Diffusion models. \textbf{QRMI} \cite{tang2023membership} applies quantile regression to posterior likelihoods. \textbf{SIA} \cite{qu2024very} infers membership based on noise parameter differences in the reverse diffusion process. \textbf{PIA} \cite{kong2023efficient} uses diffusion model properties to infer membership with fewer queries. \textbf{PFAMI} \cite{fu2023probabilistic} analyzes fluctuations between target samples and neighbors, exploiting memorization in generative models. Zhai et al. \cite{zhai2024membership} use discrepancies in conditional likelihoods due to overfitting for membership inference.


\subsection{Data Extraction Attacks}
\label{sec:dm_data_extraction_attacks}

Data extraction attacks aim to reverse-engineer training data or attributes from a trained model, exploiting diffusion models' generative capabilities. Their effectiveness depends on the model's ability to memorize specific attributes \cite{somepalli2023understanding,gu2023memorization,wen2024detecting,ren2024unveiling}. These attacks can be classified into two main approaches based on the type of condition used: \textbf{explicit condition-based extraction} and \textbf{surrogate condition-based extraction}.

\textbf{Explicit Condition-based Extraction} leverages conditional information in T2I diffusion models to extract memorized training samples. Attackers use specific text prompts to generate images similar to training data. For example, \cite{carlini2023extracting} introduced brute-force data extraction (\textbf{BruteDE}), generating images with targeted prompts and using membership inference to identify matches. This method is slow. One Step Extraction (\textbf{OSE}) \cite{webster2023reproducible} exploits "template verbatims," where models regenerate training samples, using metrics like denoising confidence score (DCS) and edge consistency score (ECS) for faster extraction.

\textbf{Surrogate Condition-based Extraction} creates surrogate conditions to enable data extraction from unconditional diffusion models. \textbf{SIDE} \cite{chen2024towards} uses implicit labels from classifiers or feature extractors as surrogate conditions. \textbf{FineXtract} \cite{wu2024revealing} uses fine-tuned models as surrogate conditions to guide extraction in latent space regions tied to fine-tuning data.


\subsection{Model Extraction Attacks}
\label{sec:dm_model_extraction_attacks}

Model extraction aims to steal a trained diffusion model's internal parameters or architecture. The only known method for model extraction on diffusion models is Spectral DeTuning (\textbf{SDeT})  \cite{horwitz2024recovering}.
SDeT leverages Low-Rank Adaptation (LoRA) \cite{hu2021lora} to extract pre-fine-tuning weights of generative models fine-tuned with LoRA. By collecting multiple fine-tuned models from the same pretrained model, it formulates an optimization problem to minimize the difference between fine-tuned weights and the sum of original weights and adaptation matrices under a low-rank constraint, solved iteratively using Singular Value Decomposition (SVD)\cite{stewart1993early}. SDeT effectively recovers original weights for models like Stable Diffusion and Mistral-7B\cite{jiang2023mistral}, highlighting vulnerabilities in fine-tuning processes with low-rank adaptations.

\subsection{Intellectual Property Protection}
\label{sec:dm_intellectual_property_protection}
% \subsubsection{Watermark}

Intellectual property protection for AI is an emerging research area that uses techniques like adversarial attacks and watermarking to safeguard the intellectual property of natural (training or test) data, generated data, and trained models. These methods generally assume full access to the protected object. The following sections categorize these approaches into \textbf{natural data protection}, \textbf{generated data protection}, and \textbf{model protection}.

\subsubsection{Natural Data Protection}

Natural data protection methods focus on preprocessing data during training or inference to safeguard the copyright of naturally collected data, as opposed to generated data. In this context, data owners defend against model owners accessing the data. 
Existing natural data protection methods for T2I diffusion models aim to protect image intellectual property while minimizing quality loss. They can be categorized into \textbf{learning prevention}, \textbf{editing prevention}, and \textbf{data attribution} methods based on specific goals.

\textbf{Learning Prevention} methods prevent T2I models from learning useful features from training images using techniques like adversarial attacks. 
\textbf{DUAW} \cite{ye2023duaw} protects copyrighted images by disrupting the variational autoencoder (VAE) in Stable Diffusion models, optimizing universal adversarial perturbations on surrogate images to distort outputs. 
\textbf{AdvDM} \cite{liang2023adversarial} protects artwork copyrights by generating adversarial examples to prevent diffusion models from imitating artistic styles. \textbf{Anti-DreamBooth} \cite{van2023anti} defends against malicious fine-tuning by injecting adversarial noise into user images to block the model from learning personalized features.
\textbf{MetaCloak} \cite{liu2024metacloak} enhances image resistance to transformations (flipping, cropping, compression) by using surrogate diffusion models to craft transferable perturbations and a denoising-error maximization loss for better robustness. \textbf{InMakr} \cite{liu2024countering} embeds protective watermarks on critical pixels to safeguard personal semantics even if images are modified. \textbf{SimAC} \cite{wang2024simac} improves protection by optimizing timestep intervals and introducing a feature interference loss, leveraging early diffusion steps and high-frequency information from deeper layers.


\textbf{Editing Prevention} aims to prevent diffusion model-based image tampering and deepfake generation. Existing methods either embed watermarks or use adversarial noise to disrupt the editing process. \textbf{EditGuard} \cite{zhang2024editguard} introduces a proactive forensics framework to embed exclusive watermarks into images, making them resistant to various diffusion model-based editing techniques, including foreground or background removal, filling, tampering, and face swapping. \textbf{WaDiff} \cite{min2024watermark} adds a unique watermark to each user query, enabling traceability of the generated image if ethical concerns arise. \textbf{AdvWatermark} \cite{zhu2024watermark} incorporates adversarial noise, producing visible signatures in the protected image when used by I2I models, which helps identify tampered content.


\textbf{Data Attribution} techniques identify if generated data originates from a specific dataset, often by embedding watermarks for later verification. 
Diagnosis~\cite{wang2023diagnosis} introduced a method for detecting unauthorized data usage by applying stealthy image warping effects to protected data.
\textbf{FT-SHIELD} \cite{cui2023ft} uses alternating optimization and PGD \cite{madry2017towards} to embed watermarks, with a binary detector for verification. \textbf{DiffusionShield} \cite{cui2023diffusionshield} encodes copyright messages into watermark patches, jointly optimizing the decoder and patches to ensure consistency across samples for reliable extraction. \textbf{ProMark} \cite{asnani2024promark} introduces a proactive watermarking method for \emph{concept attribution}, embedding watermarks in training data that can be extracted when similar concepts are generated by the model.

\subsubsection{Generated Data Protection} 
\label{sec:dm_generated_data_proteciton}

With the rise of AI-generated content (AIGC), protecting the copyright of generated data has become increasingly important. Generated data protection seeks to answer, \textbf{“Who created this content?”} by embedding verifiable, unique watermarks into generated images to identify their creators (either the model or user). This ensures intellectual property protection and accountability for content publishers, while balancing the challenge of maintaining detection accuracy without compromising image quality.

\textbf{HiDDeN} \cite{zhu2018hidden} pioneers deep learning-based image watermarking, using an encoder to embed imperceptible watermarks and a decoder to recover them for detection. This approach can also watermark AI-generated images as a post-processing step.
Recent protection methods primarily address the above challenge by embedding watermarks into images during the generation (reverse sampling) process of diffusion models.
\textbf{Stable Signature} \cite{fernandez2023stable} embeds a binary signature into images generated by diffusion models through decoder fine-tuning, allowing the watermark to be recovered and validated using a pre-trained extractor and statistical test.
\textbf{LaWa} \cite{rezaei2024lawa} introduces a coarse-to-fine watermark embedding method within the latent diffusion model's decoder, employing multiple modules to insert the watermark at different upsampling stages using adversarial training. \textbf{Safe-SD} \cite{ma2024safe} proposes a framework for embedding a graphical watermark (e.g., QR code) into the imperceptible structure-related pixels of a Stable Diffusion model for high traceability.

Recent studies highlight vulnerabilities in watermarking for AIGC. \textbf{WEvade} \cite{jiang2023evading} bypasses watermark detection by adding subtle perturbations to watermarked images, exploiting watermark characteristics. \textbf{TAIW} \cite{hu2024transfer} proposes a transfer attack using multiple surrogate watermarking models in a no-box setting, analyzing its theoretical transferability. Unlike per-image attacks, \textbf{SSU} \cite{hu2024stable} introduces a model-targeted attack to remove in-generation watermarks by fine-tuning the diffusion model’s decoder with non-watermarked images, demonstrating the fragility of Stable Signature \cite{fernandez2023stable}.

\subsubsection{Model Protection} 
Model protection techniques safeguard the intellectual property of released models, enabling owners to verify ownership and trace generated content back to its origin. These approaches are categorized based on their objectives into \textbf{model watermark} and \textbf{model attribution}.

\textbf{Model Watermark} injects a watermark trigger into the model, which can then be activated during inference to verify ownership.
Zhao et al.~\cite{zhao2023recipe} proposed separate watermarking schemes for unconditional/class-conditional and T2I diffusion models. For unconditional/class-conditional models, a pretrained watermark encoder embeds a binary string (e.g., "011001") into the training data, and the model is trained to generate images with a detectable watermark, verified by a pretrained decoder. For T2I models, a paired (text, image) trigger (e.g., \texttt{"[V]"} and a QR code) is used to trigger the generation of the QR code for ownership verification. \textbf{FIXEDWM} \cite{liu2023watermarking} enhances trigger stealthiness by fixing its position in prompts, ensuring the watermarked image is generated only when the trigger is in the correct position.  \textbf{WDM} \cite{peng2023protecting} modifies the standard diffusion process into a Watermark Diffusion Process (WDP) to embed watermarks. During training, WDM learns from watermarked images using WDP, while normal images follow the standard diffusion process. During verification, Gaussian noises combined with the trigger can activate the generation of watermarked images.


\textbf{Model Attribution} also embeds watermarks into generated content to identify the model, similar to generated data protection methods in Section \ref{sec:dm_generated_data_proteciton}. The key difference is that model attribution focuses on model-wide watermarks, while generated data protection targets sample-specific watermarks.
\textbf{Tree-Ring}\cite{wen2023tree} embeds a watermark into the Fourier space of the initial Gaussian noise used for T2I generation. During verification, denoising diffusion implicit model (DDIM) inversion extracts the initial noise, and comparison with the original watermark identifies the generating model. \textbf{AquaLoRA}\cite{feng2024aqualora} addresses the limitations of existing methods to white-box adaptive attacks, including Tree-Ring, by embedding a secret bit string into the model parameters to achieve white-box protection, preventing easy manipulation of the watermark by malicious users.
\textbf{LatentTracer} \cite{wang2024trace} identifies the origin model of generated samples by reverse-engineering their latent inputs, eliminating the need for artificial fingerprints or watermarks.



\subsection{Datasets}
This section reviews commonly used datasets for diffusion model safety research, as summarized in Tables \ref{tab:diffuison_safety_I} and \ref{tab:diffuison_safety_II}. 
For adversarial attack and defense studies, captioned text-image pairs such as MS COCO \cite{lin2014microsoft}, LAION \cite{schuhmann2021laion, schuhmann2022laion}, and DiffusionDB \cite{wang2022diffusiondb} are often employed by conditional diffusion models. Datasets for category-image classification tasks, like ImageNet \cite{deng2009imagenet} and CIFAR10/100 \cite{krizhevsky2009learning}, are typically used by unconditional diffusion models to evaluate attack effectiveness and output quality.
In research on NSFW content in diffusion models, the I2P dataset \cite{schramowski2023safe} is widely used, alongside custom datasets such as NSFW-200 \cite{yang2024sneakyprompt}, VBCDE-100 \cite{deng2023divide}, Tox100/1K \cite{cai2024ethical} and a human-attribute dataset \cite{cai2024ethical} focused on bias research. 
For intellectual property protection, datasets like CelebA \cite{liu2015deep} and VGGFace2 \cite{cao2018vggface2} (facial datasets), DreamBooth \cite{ruiz2023dreambooth} and Pokemon Captions \cite{pinkney2022pokemon} (object datasets), and WikiArt \cite{saleh2015large} (artistic style dataset) are commonly used.



\begin{table*}[htp]
\center
\caption{A summary of attacks and defenses for Agents.}
\label{tab:agent_safety}
\resizebox{\textwidth}{!}{
\begin{tabular}{lllll}
\hline
\rowcolor{wangyixu-purple}
Primary & Method & Year & Category & Target Model \\ 
\hline
\aboverulesepcolor{orange!35!}  \midrule
\belowrulesepcolor{gray!35!}
    \rowcolor{gray!35!}\multicolumn{5}{c}{\textbf{LLM Agent}} \\
\aboverulesepcolor{gray!35!}  \midrule
\multirow{8}{0.07\textwidth}{Attacks} & \cellcolor{gray!15!}InjecAgent~\cite{zhan2024injecagent} & \cellcolor{gray!15!}2024 & \cellcolor{gray!15!}Prompt Injection & \cellcolor{gray!15!}Qwen-1.8B, 72B, Mistral-7B, 8x7B, OpenOrca-Mistral, OpenHermes-Mistral, \\
& \cellcolor{gray!15!} & \cellcolor{gray!15!} & \cellcolor{gray!15!} & \cellcolor{gray!15!}Nous-Mixtral, MythoMax-13B, WizardLM-13B, Platypus2-70B, Capybara-7B, \\
& \cellcolor{gray!15!} & \cellcolor{gray!15!} & \cellcolor{gray!15!} & \cellcolor{gray!15!}Nous-LLaMA-2-13B, LLaMA-2-70B, Claude-2, GPT-3.5, GPT-4 \\
& Breaking Agents ~\cite{zhang2024breaking} & 2024 & Prompt Injection & GPT-3.5, GPT-4, Claude-2 \\
& \cellcolor{gray!15!}BadAgent~\cite{wang2024badagent} & \cellcolor{gray!15!}2024 & \cellcolor{gray!15!}Backdoor Attack & \cellcolor{gray!15!}ChatGLM-3-6B, AgentLM-7B, 13B \\
& AgentPoison~\cite{chen2024agentpoison} & 2024 & Backdoor Attack & GPT-3.5, LLaMA-3 \\
& \cellcolor{gray!15!}Contextual Backdoor~\cite{liu2024compromising} & \cellcolor{gray!15!}2024 & \cellcolor{gray!15!}Backdoor Attack & \cellcolor{gray!15!}GPT-3.5, text-davinci-002, Gemini \\
& PsySafe ~\cite{zhang2024psysafe} & 2024 & Jailbeak Attacks & Camel, AutoGen, MetaGPT, AutoGPT \\
\hline
\multirow{3}{0.07\textwidth}{Defenses} & \cellcolor{gray!15!}
TrustAgent ~\cite{hua2024trustagent} & \cellcolor{gray!15!}2024 & \cellcolor{gray!15!}Response filtering & \cellcolor{gray!15!}GPT-4, GPT-3.5, Claude-2, Claude-1.2, Mixtral \\
& Autodefense ~\cite{zeng2024autodefense} & 2024 & Response filtering & GPT-3.5, Vicuna-13B, LLaMA-2-70B, Mixtral-8x7B \\
& \cellcolor{gray!15!}GuardAgent~\cite{xiang2024guardagent} & \cellcolor{gray!15!}2024 & \cellcolor{gray!15!}Knowledge-enabled reasoning  & \cellcolor{gray!15!}EHRAgent, SeeAct \\
\hline
\multirow{4}{0.07\textwidth}{Benchmarks} 
& R-Judge ~\cite{yuan2024r} & 2024 & Benchmarks & GPT-3.5, GPT-4o, LLaMA-3-8B, LLaMA-2-7B, 13B, Vicuna-7B, 13B, Mistral-7B \\
& \cellcolor{gray!15!}AgentDojo ~\cite{debenedetti2024agentdojo} & \cellcolor{gray!15!}2024 & \cellcolor{gray!15!}Benchmarks & \cellcolor{gray!15!}Gemini-1.5-Flash, Gemini-Pro, Claude-3-Sonnet, Claude-3-Opus, \\
 & \cellcolor{gray!15!} & \cellcolor{gray!15!} & \cellcolor{gray!15!} & \cellcolor{gray!15!}Claude-3.5-Sonnet, GPT-3.5, GPT-4, GPT-4o, LLaMA-3-70B, Command R+  \\
& SafeAgentBench\cite{yin2024safeagentbench} & 2024 & Benchmarks & GPT-4, LLaMA-3-8B, Qwen-2-7B, DeepSeek-V2.5 \\
\hline
\aboverulesepcolor{gray!35!}  \midrule
\belowrulesepcolor{gray!35!}
    \rowcolor{gray!35!}\multicolumn{5}{c}{\textbf{VLM Agent}} \\
\aboverulesepcolor{gray!35!}  \midrule
\multirow{4}{0.07\textwidth}{Attacks} & \cellcolor{gray!15!}Fu et al.~\cite{fu2023misusing} & \cellcolor{gray!15!}2023 & \cellcolor{gray!15!}White-box Attacks & \cellcolor{gray!15!}LLaMA Adapter \\
& Tan et al.~\cite{tan2024wolf} & 2024 & White-box Attacks & LLaVA, PandaGPT \\
& \cellcolor{gray!15!}AgentSmith~\cite{gu2024agent} & \cellcolor{gray!15!}2024 & \cellcolor{gray!15!}Black-box Attacks &  \cellcolor{gray!15!}LLaVA-1.5-7B, 13B \\
& ARE~\cite{wu2024adversarial} & 2024 & Robustness Analysis & GPT-4V, Gemini-1.5-Pro, Claude-3-Opus, GPT-4o \\
\hline
\end{tabular}
}
\end{table*}

\section{Agent Safety} \label{sec:agent}
Large model powered agents are increasingly deployed across diverse applications, leveraging the capabilities of LLMs and VLMs to tackle complex problems, especially in safety-critical domains such as medical robotics and autonomous driving. Ensuring their safety is of paramount importance. This section provides a comprehensive review of existing safety research on agents, highlighting key challenges and emphasizing the need for ongoing innovation. Existing research can be broadly categorized into \textbf{LLM agent safety} and \textbf{VLM agent safety}.

\subsection{LLM Agent Safety}
\label{sec:agent_LLM}

This subsection reviews recent research on LLM agent safety from three key aspects: \emph{attack methodologies}, \emph{defense mechanisms}, and \emph{benchmarks}.

\subsubsection{Attacks}
Understanding and mitigating the vulnerabilities of LLM agents is crucial for their safe and trustworthy deployment, particularly as they manage sensitive data and perform real-world actions. Existing attacks on LLM agents fall into three main categories: \textbf{Prompt Injection Attacks}, \textbf{Backdoor Attacks}, and \textbf{Jailbreak Attacks}.

\textbf{Prompt Injection Attacks} manipulate an agent's behavior by embedding malicious instructions into its input prompts. For LLM agents, this often involves crafting prompts that exploit reasoning processes and interactions with external tools. A notable subtype is \textbf{Indirect Prompt Injection} (IPI), where malicious instructions are hidden in external content like web pages, emails, or documents retrieved by the agent. For example, \textbf{InjecAgent} \cite{zhan2024injecagent} demonstrates how an embedded command in a product review can trigger unintended actions when processed by the agent. These attacks are especially concerning as they exploit reliance on external information without requiring access to the agent's core system. \textbf{Breaking Agents} \cite{zhang2024breaking} further investigates vulnerabilities across agent components, showcasing the widespread applicability of prompt injection attacks across diverse architectures.

\textbf{Backdoor Attacks} introduce hidden triggers into an agent's model or knowledge base, causing it to execute malicious actions under specific conditions. For LLM agents, these attacks can target training data, fine-tuning datasets, or long-term memory and knowledge bases. \textbf{BadAgent} \cite{wang2024badagent} embeds backdoors by poisoning fine-tuning data. \textbf{Contextual Backdoor Attacks} \cite{liu2024compromising} poison benign-looking contextual demonstrations to trigger malicious behavior in specific scenarios. \textbf{AgentPoison} \cite{chen2024agentpoison} targets an agent's memory or knowledge base, ensuring malicious demonstrations are retrieved and executed under predefined triggers, even without further model training.

\textbf{Jailbreak Attacks} bypass an agent's safety mechanisms and ethical guidelines, prompting it to perform restricted actions. This is often done by exploiting loopholes in prompt interpretation or manipulating the agent's internal state. \textbf{PsySafe} \cite{zhang2024psysafe} examines psychological safety in multi-agent systems, showing how adversarial prompts can trigger dark personality traits, effectively overriding safety protocols and enabling harmful behaviors.

\subsubsection{Defenses}
Existing defense mechanisms for LLM agents can be categorized into \emph{response filtering} and \emph{knowledge-enabled reasoning}.

\textbf{Response Filtering} monitors and filters potentially harmful or undesirable outputs generated by LLM agents. \textbf{AutoDefense} \cite{zeng2024autodefense} uses a multi-agent framework where agents assume distinct defensive roles, collaboratively analyzing the target agent's responses. A consensus-based decision determines whether a response is allowed, enhancing robustness. 
\textbf{TrustAgent} \cite{hua2024trustagent} introduces an agent constitution with predefined safety rules, ensuring adherence during the planning phase. It employs a three-stage strategy: 1) \emph{pre-planning}, which injects safety knowledge before plan generation; 2) \emph{in-planning}, which enhances safety during generation; and 3) \emph{post-planning}, which inspects outputs before execution.

\textbf{Knowledge-Enabled Reasoning} enhances LLM agent safety by integrating external knowledge or structured reasoning processes. \textbf{GuardAgent} \cite{xiang2024guardagent} introduces a guard agent that oversees the target LLM agent. It generates an action plan based on guard requests, translates it into executable code using a knowledge base and function toolbox, and executes it to verify compliance with guard rules. This structured approach ensures more reliable and systematic defense.


\subsubsection{Benchmarks}
Existing benchmarks evaluate agents' ability to identify risks in interactive environments or defend against targeted attacks.  
\textbf{R-Judge} \cite{yuan2024r} assesses agents' risk awareness using 569 records across 27 scenarios and 10 risk types, testing their ability to identify safety risks in interaction logs.  
\textbf{SafeAgentBench} \cite{yin2024safeagentbench} evaluates embodied agents' safety awareness and planning skills with 750 tasks featuring varying hazards and abstraction levels in a simulated environment, measuring both task execution and semantic understanding. 
\textbf{AgentDojo} \cite{debenedetti2024agentdojo} tests agents' resilience to prompt injection attacks through 97 tasks and 629 security test cases, focusing on handling malicious instructions embedded in third-party data.  
These benchmarks offer valuable insights into LLM agent safety by assessing their risk identification and defense capabilities across diverse scenarios.


\subsection{VLM Agent Safety}
\label{sec:agent_VLM}

Involving both visual perception and language understanding, VLM agents face unique safety challenges. 
Current attacks on VLM agents mainly target their weaknesses through \textbf{white-box} and \textbf{black-box} attacks, as well as \textbf{robustness analysis}. Despite these attacks, effective defense strategies are still underdeveloped.

\subsubsection{Attacks}

\textbf{White-box Attacks} assume adversaries have full access to model parameters, enabling gradient-based optimization to expose theoretical vulnerabilities. Fu et al.~\cite{fu2023misusing} used gradient-based training  and characterization to craft adversarial images, causing LLMs to execute tool commands using real-world syntax, compromising user confidentiality and integrity. Tan et al.~\cite{tan2024wolf} demonstrated how an VLM agent can jailbreak another agent in a multi-agent society through malicious prompts generated with white-box access, enabling widespread harmful outputs.

\textbf{Black-box Attacks} operate without access to model internals, relying on input-output behavior to craft adversarial examples, making them highly relevant for real-world scenarios. \textbf{AgentSmith} \cite{gu2024agent} uses a single adversarial image to jailbreak multiple VLM agents, exploiting their interconnected nature to spread malicious behavior exponentially across the network.

\textbf{Robustness Analysis} examines how system components interact and how vulnerabilities propagate. \textbf{ARE} \cite{wu2024adversarial} present an Agent Robustness Evaluation framework that models VLM agents as graphs, analyzing adversarial influence flow between components to identify weak points and assess the safety impact of component modifications.


\section{Open Challenges} \label{sec:challenges}

Based on the survey, we identify several limitations and gaps in existing research and summarize them into the following topics. These open challenges reflect the evolving nature of large model safety, highlighting both technical and methodological barriers that must be overcome to ensure robustness and reliability across various AI systems.

\subsection{Fundamental Vulnerabilities}

Exploring and understanding the fundamental vulnerabilities of large models is essential for developing robust defenses and safety frameworks. This section highlights the core weaknesses and challenges inherent to different types of large models.

\subsubsection{The Purpose of Attack Is Not Just to Break the Model}
While much of the existing research focuses on designing attacks to disrupt or break a model's functionality, the true goal of attack research should extend beyond mere disruption. Attacks can serve as a diagnostic tool to uncover unintended behaviors and reveal fundamental weaknesses in a model's decision-making processes. By understanding how and why models fail, we can address vulnerabilities at their root rather than applying superficial fixes. For every new attack proposed, it is critical to ask: 
\textbf{Why does the attack succeed or fail? How does it exploit the model? What new vulnerabilities does it reveal that were previously unknown? Are these vulnerabilities inherent to the model architecture or class?} 
These questions guide the development of more robust models and defenses by exposing systemic flaws rather than isolated issues.


\subsubsection{What Are the Fundamental Vulnerabilities of Language Models?}

LLMs like ChatGPT and Gemini exhibit fundamental vulnerabilities due to their reliance on statistical patterns rather than true semantic understanding \cite{titus2024does}. Key weaknesses include susceptibility to adversarial inputs, biases in training data, and manipulation via prompt engineering. To build effective defenses, research must delve deeper into how these vulnerabilities arise from the model's architecture and training data. 

Critical areas of focus include: 1) \textbf{Memorization of training data}, which can lead to privacy breaches or unintended data leakage; 2) \textbf{Exposure to harmful content}, which can propagate biases or toxic outputs; 3) \textbf{Amplification of hallucinations}, where models generate plausible but incorrect or nonsensical information. Open research questions remain:
\textbf{Does the discrete nature of textual inputs make language models more or less robust compared to vision models? What fundamental vulnerabilities are exposed by jailbreak and data extraction attacks? }
Addressing these questions is vital for advancing the safety and reliability of LLMs and other large models.

\subsubsection{How Vulnerabilities Propagate Across Modalities?}
As Multi-modal Large Language Models (MLLMs) integrate diverse modalities, new vulnerabilities arise. Vision encoders are known to be sensitive to subtle, continuous perturbations in pixel space, while language models are vulnerable to adversarial characters, words, or prompts. However, the interaction between modalities and how vulnerabilities in one modality propagate to others remains poorly understood.

Key questions include: \textbf{How do vulnerabilities in one modality (e.g., vision) influence the behavior of another (e.g., language)? How does the number of tokens across modalities affect vulnerability propagation?}
Additionally, it is critical to explore \textbf{how to address multi-modal vulnerabilities within a unified framework}, moving beyond defenses tailored to individual modalities. This requires a holistic approach to identify and mitigate cross-modal risks, ensuring robust performance across all integrated modalities.


\subsubsection{Diffusion Models for Visual Content Generation Lack Language Capabilities}

Diffusion models for image or video generation excel in visual content creation but often lack language understanding capabilities, a limitation shared by VLP models. This is because these models are primarily optimized for pixel-level generation tasks, without incorporating language processing into their core architecture. As a result, they may generate harmful or contextually inappropriate content due to their inability to fully comprehend textual prompts.
To develop robust multi-modal systems, it is crucial to integrate language comprehension into these models. This would enable them to produce content that is not only visually coherent but also contextually aligned with the given textual input.

An open challenge is \textbf{bridging the gap between visual and linguistic capabilities in generative models to enhance multi-modal safety}. However, this integration may introduce new vulnerabilities, such as advanced attacks that exploit fine-grained manipulation of the generation process. Addressing these challenges represents a critical direction for future research.

\subsubsection{How Much Training Data Can a Model Memorize?}
The memorization capability of deep neural networks (DNNs) has raised significant concerns, particularly in enabling privacy attacks such as membership inference and data extraction (model inversion). Both LLMs and DMs have been shown to replicate and leak small portions of their training data under specific conditions. However, it remains unclear whether DNNs primarily learn through memorization and to what extent this occurs.
Due to the non-linear nature of large models, exact model inversion is inherently impossible. These models compress training data into multi-level representations, making it difficult to pinpoint when and how memorization happens. 

Key open questions include: \textbf{What mechanism acts as the memorization ``switch", triggering the model to output training data directly? How can memorization be effectively measured—through exact matches, training equivalence, or embedding similarity?}
Addressing these questions is crucial for understanding the trade-offs between model performance and privacy risks, as well as for developing strategies to mitigate unintended data leakage.

\subsubsection{Agent Vulnerabilities Grow with Their Abilities}
Large-model-powered agents face increasing vulnerabilities as their capabilities expand \cite{gu2024agent}. These agents interact with external tools, data sources, and environments, creating a broader attack surface that complicates defense mechanisms. A critical challenge is the compounding effect of vulnerabilities in foundational models when integrated into agents' decision-making processes. For instance, an agent relying on a language model vulnerable to jailbreak prompts and a vision model susceptible to adversarial inputs may experience cascading failures, leading to unpredictable outcomes.

Moreover, agents' ability to learn and adapt introduces additional risks. Even seemingly benign interactions can expose them to subtle biases or adversarial inputs, potentially triggering unsafe behaviors. The dynamic nature of agents—especially those that continuously learn or self-improve—further complicates vulnerability detection, as they may develop new weaknesses over time. This unpredictability makes traditional safety evaluations insufficient, as agents can evolve in ways that are difficult to anticipate.

To address these challenges, research must focus on: \textbf{understanding the interaction between model components} (e.g., language, vision, and decision-making) and \textbf{how vulnerabilities in one component can propagate to others}. 
It is also important to \textbf{develop new methodologies to evaluate agents in dynamic, evolving environments}, ensuring their robustness against emerging threats.
These efforts are essential for building safer and more reliable agent systems in the future.


\subsection{Safety Evaluation}

Comprehensive and standardized safety evaluations are critical for quantifying the safety levels of large models. However, existing evaluation datasets and benchmarks are often static or narrowly focused on specific threats. To ensure models perform reliably in real-world conditions, safety evaluations must test them across diverse and unpredictable scenarios. 
% Such evaluations are essential for responsible AI deployment and fostering public trust in large models.

\subsubsection{Attack Success Rate Is Not All We Need}

While the attack success rate (ASR) is a commonly used metric in safety research, it mainly quantifies how often an attack disrupts a model's output. However, this metric overlooks several important factors, such as the severity of the disruption, the model’s resilience to various types of attacks, and the real-world consequences of potential failures. A model could still cause harm or mislead decision-making even if its core functionality appears unaffected. For instance, an attack might subtly alter a model’s decision-making process without causing an obvious malfunction, but the resulting behavior could have catastrophic effects in real-world applications. Such vulnerabilities are often missed by traditional metrics like ASR or failure rate.

To better understand a model’s weaknesses—whether in its design, training data, or inference process—it is crucial to define multi-level, fine-grained vulnerability metrics. A more  comprehensive safety evaluation framework should consider factors such as the model’s susceptibility to different types of attacks, its ability to recover from malicious inputs, and the ethical implications of potential failure modes.


\subsubsection{Static Evaluations Create a False Sense of Safety}
Current safety evaluations rely heavily on static benchmarks or open-source datasets that have been available for some time. These datasets have already been exposed to model trainers and adversaries, which means a model may achieve high safety performance on these outdated datasets without necessarily being robust in real-world scenarios. As a result, static evaluations can create a false sense of safety. This underscores a significant limitation in the current evaluation framework: static benchmarks fail to capture the evolving nature of threats that models encounter in dynamic, real-world applications.

To address this challenge, safety evaluations must move beyond static assessments. A key \textbf{step is to develop evaluation datasets or benchmarks that evolve over time}, better reflecting the ever-changing landscape of safety threats. An example of an evolving evaluation system is the Chatbot Arena \cite{chiangchatbot}, which adapts as new threats and challenges emerge. Similar strategies could be applied to safety assessments in broader AI systems.
Additionally, future evaluation methods might consider releasing only the ``seeds” or structural components of datasets, along with a test case generation method, rather than providing static test cases. This approach would enable the continuous generation of new test cases that better reflect the evolving nature of threats.


\subsubsection{Adversarial Evaluations Are a Necessity, Not an Option}

While regular (non-adversarial) safety tests offer insights into a model’s general robustness, they fail to capture the full spectrum of safety risks that models face in real-world scenarios. These tests typically focus on overall performance but neglect how models respond to adversarial queries that exploit their vulnerabilities. In contrast, adversarial evaluations assess model performance under attack, providing a more accurate measure of safety in worst-case scenarios.

A key challenge in this area is \textbf{developing environments that simulate real-world attack conditions}. One promising approach is to frame safety evaluation as a two-player adversarial game, where reinforcement learning-based adversarial agents interact with target models to identify and exploit new vulnerabilities. This method offers a more dynamic and comprehensive way to assess model safety under attack.
Such adversarial evaluations are especially crucial for commercial APIs, which often use filtering mechanisms to block malicious inputs. These filters can render traditional safety benchmarks less effective, as they prevent the model from facing the full range of adversarial threats it might encounter in real-world applications.

\subsubsection{Open-Ended Evaluation}
Evaluating adversarial attacks in classification problems is relatively straightforward, as each input is typically associated with a distinct class label. However, large models often generate open-ended responses, which complicates the evaluation of attacks like jailbreaking (e.g., through ASR computation). The ideal evaluator for such models would function as a perfect jailbreaking detector. Yet, since achieving an ideal jailbreaking detector is not feasible, it follows that an ideal evaluator may also be unattainable.

Currently, evaluators are generally rule-based (e.g., keyword detection) or model-based (e.g., GPT or Llama-Guard). However, \textbf{developing more consistent and reliable evaluation methods and metrics remains an open challenge}. One potential solution is to constrain the output space to a finite set of actions, similar to the approach used in agent-based scenarios. This would limit the complexity of the evaluation and make it more feasible to assess safety in open-ended environments.



\subsection{Safety Defense}

Safety mechanisms in large models are crucial for preventing harmful or unintended behaviors. These mechanisms may involve modifications to the model's architecture or the integration of external monitoring systems. This section explores the open challenges in developing robust defense solutions.

\subsubsection{Safety Alignment Is Not the Savior}

Safety alignment—ensuring that a model's objectives align with human values—has been a promising approach to mitigating certain risks. However, recent studies have revealed a significant weakness in safety alignment: \textbf{fake alignment} \cite{wang2024fake,greenblatt2024alignment}, where a model may achieve high safety scores without truly understanding the underlying safety principles. This points to a deeper issue of \textbf{shallow safety}. Furthermore, even models that are considered well-aligned, such as GPT-4o \cite{gpt-4o} or o1 \cite{openai-o1}, remain vulnerable to sophisticated attacks that can bypass their alignment mechanisms \cite{ying2024unveiling}.

An open challenge is to \textbf{identify the mechanistic limitations of safety alignment} and develop methods that ensure robust safety, even in the face of unforeseen attacks. Recent research \cite{qi2024safety} emphasizes the \textbf{need to move beyond superficial safety alignment metrics} (such as the distribution of the first few output tokens), ensuring that alignment is deeper and more comprehensive.
Additionally, making \textbf{safety alignment adversarial}—by actively challenging a model’s safety mechanisms—may help address the issue of shallow alignment, leading to more resilient and trustworthy models.


\subsubsection{Jailbreak Attacks Are More Challenging to Defend Against Than Adversarial Attacks}
Jailbreak attacks and adversarial attacks share a common goal: both aim to manipulate a model into producing targeted outputs. However, the key distinction lies in the nature of these outputs. Jailbreak attacks specifically seek to induce harmful or toxic responses, which requires bypassing the model's safety mechanisms. In contrast, while adversarial attacks may also circumvent safety defenses, they are not inherently designed to produce malicious content.

A significant challenge in defending against jailbreak attacks, beyond the typical defenses against adversarial attacks, is the absence of constraints on the attack's perturbation budget. In adversarial attacks, perturbations are deliberately kept minimal to remain imperceptible to humans. Jailbreak attacks, however, are not constrained by such requirements, allowing for greater flexibility in crafting attacks. This lack of limitations makes jailbreak attacks more challenging to defend against compared to adversarial attacks. \textbf{Developing defense strategies that can effectively address both types of attacks} remains an open and pressing challenge.

\subsubsection{The Need for More Practical Defenses}

Existing defense methods face several limitations that hinder their effectiveness in real-world applications. These limitations include a lack of generality, low efficiency, reliance on white-box access, and poor adaptability. To be truly practical, a defense must possess certain desirable properties, the achievement of which remains an ongoing challenge.

\begin{enumerate}[label=\arabic*]
    \item \textbf{Generality:} \\With the wide variety of models deployed across different domains—such as vision, language, and multimodal systems—defenses should not be overly tailored to specific architectures. Instead, they should offer generalized solutions applicable to multiple model families. Generality ensures that a single defense mechanism can be deployed across a broad range of systems, making it scalable and efficient for real-world safety infrastructures.
    \item \textbf{Black-box Compatibility:} \\In many real-world scenarios, defenders may not have access to the internal parameters of the model they are protecting. Therefore, practical defenses must operate in a black-box setting, where defenders can only observe the model’s inputs and outputs. This requires defense strategies that function externally to detect and mitigate attacks without needing access to the model's inner workings.
    \item \textbf{Efficiency:} \\ Many defense techniques, particularly adversarial training, are computationally expensive. The need for large-scale retraining or fine-tuning can make these defenses prohibitively costly. Practical defenses must strike a balance between robustness and computational efficiency, ensuring that models remain safe without incurring excessive resource consumption.

    \item \textbf{Continual Adaptability:} \\ A A practical defense system should not only recognize previously encountered attacks but also adapt in real-time to new and evolving threats. This requires continual learning and the ability to update without relying on costly retraining cycles. Models must be capable of incorporating new data, evolving their defense strategies, and self-correcting as new attacks emerge.
\end{enumerate}

The ongoing challenge for researchers is to refine and integrate these properties into cohesive defense strategies that offer robust protection while maintaining model performance.


\subsubsection{The Lack of Proactive Defenses}

Most existing defense approaches, such as safety alignment and adversarial training, are passive in nature, focusing on fortifying models against potential attacks. However, \textbf{proactive defenses}, which actively counter potential attacks before they succeed, have received limited attention in the literature.
For example, a proactive defense against model extraction could involve poisoning or injecting backdoors into extraction attempts, rendering the extracted model unreliable. Another approach could be to provide deliberately nonsensical or easily detectable responses when a malicious user requests advice for illegal activities, such as planning a robbery.
Such proactive strategies could serve as powerful deterrents to potential attackers. However, designing effective proactive defenses for different types of safety threats remains an open challenge and a promising direction for future research.

\subsubsection{Detection Has Been Overlooked in Current Defenses}

Detection methods play a crucial role in identifying potential vulnerabilities and abnormal behaviors in models, effectively acting as active monitors. When integrated with other defense mechanisms, detection systems can trigger automatic safety responses whenever a model behaves unexpectedly or generates harmful outputs. Despite their importance, however, existing defense strategies have largely overlooked the integration of detection systems within their pipelines.
By combining detection with other safety measures, it becomes possible to develop more resilient models capable of dynamically responding to emerging threats. For example, stronger attacks may be more easily detected, providing an opportunity for proactive defense.
An open question remains: \textbf{What is the most effective way to integrate detection as a core component of a defense system, and how can detection and other defense mechanisms complement and enhance each other?}

\subsubsection{The Current Data Usage Practices Must Change}

The current data usage practices in the AI development lifecycle are neither sustainable nor ethical. We identify three major issues with how data is currently used:

\begin{enumerate}[label=\arabic*] 

\item \textbf{Lack of Consent and Recognition}: Many large models are trained on web-crawled data without the consent of data owners or formal recognition or reward for their contributions. This practice raises significant ethical and legal issues.

\item \textbf{The Explosion of Generated Data}: A large volume of generated data has been uploaded to the internet, much of which is fake, toxic, or otherwise harmful. However, there is no clear system in place to identify which model created this data or who was responsible for its generation.

\item \textbf{Depletion of ``Free'' Data}: The ``free'' data available on the internet is rapidly diminishing. Users are becoming less motivated or outright refusing to contribute valuable data, especially when their contributions are neither acknowledged nor rewarded.

\end{enumerate}

To address these challenges, the AI industry must establish a healthy and sustainable data ecosystem where data contributors are recognized and rewarded. Achieving this requires answering the following key questions:

\begin{itemize} 
\item \textbf{Who Used My Data?} This question addresses the protection of copyright for original training data, commonly known as \textbf{membership inference}. Membership inference techniques are essential to determine whether specific samples were included in a model's training data. Such methods would empower data owners to verify how their data is being used and protect their legitimate rights and interests.

\item \textbf{Who Generated the Data?} This question focuses on protecting the copyright of generated data, a process referred to as \textbf{model attribution}. Techniques for model attribution are needed to identify the model responsible for generating a given sample (such as text, image, video, or audio). Model attribution should also include identifying the user responsible for creating the content, including metadata such as IDs or usernames. These technologies would not only protect the copyright of generated content but also encourage responsible data generation by ensuring that harmful or malicious content can be traced back to its creator.

\item \textbf{Which Samples Contribute to a Generated Output?} This question pertains to data attribution in the context of AIGC. Every piece of generated content can be seen as a combination of ``inspirations'' drawn from a specific set of training samples. Data attribution techniques are needed to identify the training samples that most significantly contributed to the generated content. Contributors whose samples play a critical role in shaping the content should receive a share of the profits if the content is used commercially, promoting a fair and transparent data economy.

\end{itemize}

Beyond these questions, many other critical issues must be addressed to create a sustainable and ethical data ecosystem. Ensuring that data contributors are fairly acknowledged and incentivized is essential for promoting accountability and transparency in AIGC.

\subsubsection{Safe Embodied Agents}

Most safety threats studied today are \textbf{digital}. However, as embodied AI agents are increasingly deployed in the physical world, new physical threats will emerge, potentially causing real harm and loss to humans. Ensuring the safety of these embodied agents has therefore become a critical concern. Safe agents must be resilient to adversarial inputs, capable of self-regulating harmful behaviors, and consistently aligned with human values.

Achieving this requires deeply integrating safety mechanisms into the agents' decision-making processes, enabling them to handle unexpected challenges while maintaining robustness and reliability. The primary challenge is \textbf{designing safety protocols that allow these agents to perform complex tasks autonomously, while ensuring they remain trustworthy and safe in dynamic and unpredictable environments}. As agents gain greater autonomy, ensuring their safety becomes not only a technical challenge but also a significant ethical responsibility.

\subsubsection{Safe Superintelligence}
As AI progresses toward AGI and superintelligence, embedding inherent safety mechanisms into large models to ensure predictable, value-aligned behavior becomes a critical challenge. While the technical path to safe superintelligence remains uncertain, several promising mechanisms offer potential solutions:

\begin{itemize}
    \item \textbf{Oversight System}: One system cannot be both superintelligent and trustworthy, like humans. However, we can develop an oversight system to monitor and regulate the primary system’s behavior, intervening when necessary. A key challenge is to ensure the reliability and robustness of the oversight system itself. This leads to the \textbf{\textit{Oversight Paradox}}: \textit{If a superintelligent AI is monitored by another AI (the oversight system), the oversight system must be at least as capable as the superintelligent AI to reliably detect and prevent undesirable behaviors}. However, this raises the question: \textbf{who monitors the oversight system to ensure it doesn't fail or act contrary to its purpose?}

    \item \textbf{Safety Layer}: This approach embeds a dedicated safety layer \cite{zhao2024defending, li2024safelayer} directly within the model’s architecture, acting as a gatekeeper to filter outputs against predefined safety constraints. Such layers could be dynamically updated based on real-time feedback or optimized for specific tasks.

    \item \textbf{Safety Expert}: This approach incorporates specialized ``safety experts" within the Mixture of Experts (MoE) framework \cite{jacobs1991adaptive, shazeer2017outrageously, fedus2022switch, jiang2024mixtral} to handle safety-critical tasks. By dynamically routing high-stakes queries to these experts, safety considerations are prioritized in decision-making. The recurring challenge is developing a truly safe expert model that can consistently perform as expected in all scenarios.
    
    \item \textbf{Adversarial Alignment}: This approach leverages adversarial safety principles to align models with human values. It involves training models to exploit vulnerabilities in existing safety mechanisms, then refining these mechanisms to resist adversarial prompts. Despite its promise, challenges such as high computational costs and the risk of unintended behaviors remain significant concerns.
    
    \item \textbf{Safety Consciousness}: This approach involves embedding a safety-conscious framework into the model’s foundational training to promote ethical reasoning and value alignment as intrinsic behaviors. The goal is to make safety a core characteristic, enabling the model to dynamically adapt to diverse and evolving scenarios. Safety consciousness may be formulated as a type of \textbf{safety tendency}: an inherent inclination to generate low-risk responses and shape outputs based on an awareness of potential harmful consequences, similar to human decision-making processes.
    
\end{itemize}


\subsection{A Call for Collective Action}

Safeguarding large models against adversarial manipulation, misuse, and harm is a global challenge that requires the collective efforts of researchers, practitioners, and policymakers. The following sections outline a research agenda aimed at advancing large-model safety through collaboration and innovation.

\subsubsection{Defense-Oriented Research}

Current research on large-model safety is heavily skewed toward attack strategies, with significantly fewer efforts dedicated to developing defense mechanisms. This imbalance is concerning, as the sophistication of attacks continues to outpace the development of effective defenses. To address this gap, we advocate for a shift in research priorities toward defense strategies. Researchers should not only investigate attack mechanisms but also focus on developing robust defenses to mitigate or prevent these threats. A balanced approach is crucial for advancing the field of safety research.

Moreover, future defense research should emphasize integrated approaches. New defense methods should not be proposed or implemented in isolation but rather integrated with existing approaches to build cumulative protection. Defense research should be viewed as a continuous, evolving effort, where new methods are layered onto established ones to enhance overall effectiveness. However, the diversity of defense strategies presents a challenge. Developing frameworks to effectively incorporate different defense mechanisms will require a collective effort from the research community.

\subsubsection{Dedicated Safety APIs}

To support research and testing, commercial AI models should offer a dedicated safety API. This API would allow researchers to assess and enhance the safety of these models by subjecting them to a variety of adversarial and safety-critical scenarios. By providing such an API, commercial providers can enable external safety evaluations without disrupting the general services offered to users. This would foster collaboration between industry and academia, facilitating continuous improvement in model safety.

\subsubsection{Open-Source Platforms}

The AI safety community would greatly benefit from the development and open-source release of safety platforms and libraries. These tools would facilitate the rapid evaluation, testing, and improvement of safety mechanisms across a variety of models and applications. Open-sourcing these platforms would foster collaboration and transparency, enabling researchers and practitioners to share best practices, benchmark safety techniques, and contribute to the establishment of universal safety standards.


\subsubsection{Global Collaborations}

The pursuit of AI safety is a global challenge that transcends national borders, requiring coordinated efforts from academia, technology companies, government agencies, and non-profit organizations. Effective collaboration on a global scale is essential to addressing the potential risks associated with advanced AI systems. By fostering international cooperation, we can more efficiently tackle complex safety issues and establish unified standards that guide the safe development and deployment of AI technologies.

To facilitate global collaboration, the following initiatives could be pursued:

\begin{itemize} 
    \item \textbf{International Safety Alliances}: Establishing global alliances dedicated to AI safety can bring together experts and resources from around the world. These alliances would focus on sharing research findings, coordinating safety evaluations, and developing universal safety benchmarks that reflect diverse regional needs and values.
    
    \item \textbf{Cross-Border Data Sharing Frameworks}: Access to diverse datasets is essential to improving the robustness and fairness of AI models. Developing secure and ethical frameworks for cross-border data sharing would allow researchers to test models across a wide range of scenarios and ensure that safety mechanisms are universally applicable.

    \item \textbf{Joint Research Programs}: Collaborative research programs that unite academic institutions, industry leaders, and government agencies can drive innovation in AI safety. These programs should focus on areas such as adversarial defense, safety alignment, and real-time adaptability, ensuring that their findings are broadly applicable across various AI systems.
    
    \item \textbf{Global Safety Competitions and Challenges}: Building on the concept of open safety competitions, international challenges could be organized to engage the brightest minds from around the world. These competitions would address critical safety issues, encourage the development of innovative solutions, and foster a sense of shared responsibility in advancing AI safety.
    
    \item \textbf{Policy and Regulatory Harmonization}: A collaborative approach to AI governance can help align safety regulations and policies across countries. This harmonization would prevent the misuse of AI technologies while promoting responsible development and deployment practices globally.
    
\end{itemize}

Global collaborations not only enhance the effectiveness of AI safety research but also promote transparency, trust, and accountability in the development of advanced AI systems. By working together across borders and disciplines, we can ensure that AI technologies benefit humanity while minimizing the risks associated with their deployment.


\section{Conclusions and Future Directions}

\asyncfw{} is  a new benchmark designed to evaluate the role of language model (LM) agents in facilitating collaborative information gathering within multi-user environments. %By simulating real-world collaboration scenarios across synthetic organizations, \asyncfw{} explores the ability of LM agents to asynchronously route information, identify relevant collaborators, and compile accurate, task-relevant responses. 
It comprises two domains, \dataspider{} and \datanews{}, which challenge LM  agents to handle tasks related to question-answering and document creation. Experiments with popular LM agent architectures revealed both their potential and limitations in accurately and efficiently completing complex collaborative tasks. 
% \asyncfw{} provides a valuable platform for advancing LM-mediated collaboration and sets the stage for future innovations in communication and teamwork enhancement through artificial intelligence.

Future work could consider AI agents that learn over time from interactions for improving their performance over time \cite{lewis1998designing}. By analyzing past conversations, they can improve information source selection and communication strategies, making future interactions more efficient. 
Secondly, privacy risks emerge when agents access personal documents, as large language models may not fully adhere to privacy guidelines \cite{mireshghallah2023can}. Future work could focus on privacy-centric evaluations and explore new information access models to mitigate such risks.
Finally, an excessive number of AI-initiated requests can overwhelm users, hindering productivity. Building agents that can minimize human effort and prioritize urgent requests remains a challenge. %, and explore fully autonomous setups where users’ agents interact directly with each other, reducing the need for continuous user involvement.

\section*{Limitations}

\asyncfw{} consists of two tasks and is in one language (English). Future work could explore further expanding the domains and supported languages.
We make the simplifying assumption that an agent in our setup can engage only in dyadic conversations. Exploring more topologies such as group chats \cite{wu2023autogen} would bring-in additional challenges. 
We designed the domains and the experiment setup to study the effectiveness of the LM agents on a diverse set of information gathering behaviors. However, our analysis did not model all the possible factors in a real-world. Future work can explore additional factors such as turn-around speed and reliability of the response from a collaborator, how busy a person is, and various social dynamics that can be at play in organizations. %as we wanted to focus on ...



\section*{Ethics Statement}
Allowing AI agents the capability to send messages to other users without fine-grained supervision presents a trade-off between saving user time and maintaining control. While autonomy can streamline workflows by eliminating the need for constant user confirmation, verifying key actions helps ensure accuracy and user oversight. While we studied the task in a sand-boxed environment, practitioners should carefully choose the degree of autonomy granted (for example, a more conservative approach would be to get user confirmation before every message that is sent). %, and employ reasonable protection against prompt injection and other attacks.

\section*{Acknowledgements}
We thank Jason Eisner and Hao Fang for thoughtful discussions. We thank Chris Kedzie, Patrick Xia, Justin Svegliato and Soham Dan for feedback on the paper.



% \input{Table/Overview_new}





%

\if 0

\appendices
\section{Proof of the First Zonklar Equation}
Appendix one text goes here.

% you can choose not to have a title for an appendix
% if you want by leaving the argument blank
\section{}
Appendix two text goes here.


% use section* for acknowledgment
\ifCLASSOPTIONcompsoc
  % The Computer Society usually uses the plural form
  \section*{Acknowledgments}
\else
  % regular IEEE prefers the singular form
  \section*{Acknowledgment}
\fi


The authors would like to thank...

\fi

% Can use something like this to put references on a page
% by themselves when using endfloat and the captionsoff option.
\ifCLASSOPTIONcaptionsoff
  \newpage
\fi


{\footnotesize
% \bibliographystyle{unsrt}
\bibliographystyle{IEEEtran}
\bibliography{egbib}
}

% that's all folks
\end{document}


