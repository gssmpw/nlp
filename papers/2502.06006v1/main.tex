%%
%% This is file `sample-sigconf.tex',
%% generated with the docstrip utility.
%%
%% The original source files were:
%%
%% samples.dtx  (with options: `all,proceedings,bibtex,sigconf')
%% 
%% IMPORTANT NOTICE:
%% 
%% For the copyright see the source file.
%% 
%% Any modified versions of this file must be renamed
%% with new filenames distinct from sample-sigconf.tex.
%% 
%% For distribution of the original source see the terms
%% for copying and modification in the file samples.dtx.
%% 
%% This generated file may be distributed as long as the
%% original source files, as listed above, are part of the
%% same distribution. (The sources need not necessarily be
%% in the same archive or directory.)
%%
%%
%% Commands for TeXCount
%TC:macro \cite [option:text,text]
%TC:macro \citep [option:text,text]
%TC:macro \citet [option:text,text]
%TC:envir table 0 1
%TC:envir table* 0 1
%TC:envir tabular [ignore] word
%TC:envir displaymath 0 word
%TC:envir math 0 word
%TC:envir comment 0 0
%%
%%
%% The first command in your LaTeX source must be the \documentclass
%% command.
%%
%% For submission and review of your manuscript please change the
%% command to \documentclass[manuscript, screen, review]{acmart}.
%%
%% When submitting camera ready or to TAPS, please change the command
%% to \documentclass[sigconf]{acmart} or whichever template is required
%% for your publication.
%%
%%
\documentclass[sigconf]{acmart}



%%
%% \BibTeX command to typeset BibTeX logo in the docs
\AtBeginDocument{%
  \providecommand\BibTeX{{%
    Bib\TeX}}}
\newcommand{\notess}[1]{\textcolor{teal}{#1}}
\newcommand{\citeneeded}{\textcolor{red}{\textbf{[cite]}}}
\newcommand{\up}[1]{\small ($\textcolor{green}{\blacktriangle}#1\%)$}
\newcommand{\speedup}[1]{\small ($\textcolor{green}{\blacktriangle}\textbf{#1})$}
\newcommand{\down}[1]{\small ($\textcolor{red}{\blacktriangledown}#1\%)$}

\newcommand{\name}{\textsc{FactIR}}
%% Rights management information.  This information is sent to you
%% when you complete the rights form.  These commands have SAMPLE
%% values in them; it is your responsibility as an author to replace
%% the commands and values with those provided to you when you
%% complete the rights form.
\setcopyright{acmlicensed}
\copyrightyear{2018}
\acmYear{2018}
\acmDOI{XXXXXXX.XXXXXXX}

%% These commands are for a PROCEEDINGS abstract or paper.
\acmConference[Conference acronym 'XX]{Make sure to enter the correct
  conference title from your rights confirmation emai}{June 03--05,
  2018}{Woodstock, NY}
%%
%%  Uncomment \acmBooktitle if the title of the proceedings is different
%%  from ``Proceedings of ...''!
%%
%%\acmBooktitle{Woodstock '18: ACM Symposium on Neural Gaze Detection,
%%  June 03--05, 2018, Woodstock, NY}
\acmISBN{978-1-4503-XXXX-X/18/06}


%%
%% Submission ID.
%% Use this when submitting an article to a sponsored event. You'll
%% receive a unique submission ID from the organizers
%% of the event, and this ID should be used as the parameter to this command.
%%\acmSubmissionID{123-A56-BU3}

%%
%% For managing citations, it is recommended to use bibliography
%% files in BibTeX format.
%%
%% You can then either use BibTeX with the ACM-Reference-Format style,
%% or BibLaTeX with the acmnumeric or acmauthoryear sytles, that include
%% support for advanced citation of software artefact from the
%% biblatex-software package, also separately available on CTAN.
%%
%% Look at the sample-*-biblatex.tex files for templates showcasing
%% the biblatex styles.
%%

%%
%% The majority of ACM publications use numbered citations and
%% references.  The command \citestyle{authoryear} switches to the
%% "author year" style.
%%
%% If you are preparing content for an event
%% sponsored by ACM SIGGRAPH, you must use the "author year" style of
%% citations and references.
%% Uncommenting
%% the next command will enable that style.
%%\citestyle{acmauthoryear}


%%
%% end of the preamble, start of the body of the document source.
\copyrightyear{2025}
\acmYear{2025}
\setcopyright{cc}
\setcctype{by}
\acmConference[WWW Companion '25]{Companion Proceedings of the ACM Web
Conference 2025}{April 28-May 2, 2025}{Sydney, NSW, Australia}
\acmBooktitle{Companion Proceedings of the ACM Web Conference 2025 (WWW
Companion '25), April 28-May 2, 2025, Sydney, NSW, Australia}
\acmDOI{10.1145/3701716.3715300}

\begin{document}

%%
%% The "title" command has an optional parameter,
%% allowing the author to define a "short title" to be used in page headers.
\title{FactIR: A Real-World Zero-shot Open-Domain Retrieval Benchmark for Fact-Checking}

%%
%% The "author" command and its associated commands are used to define
%% the authors and their affiliations.
%% Of note is the shared affiliation of the first two authors, and the
%% "authornote" and "authornotemark" commands
%% used to denote shared contribution to the research.



\author{Venktesh V}
\affiliation{%
 \institution{Factiverse AI and Delft University of Technology}
 \city{Delft}
 \country{Netherlands}}
 \email{V.Viswanathan-1@tudelft.nl}
 
\author{Vinay Setty}
\affiliation{%
  \institution{Factiverse AI and University of Stavanger}
  \city{Stavanger}
  \country{Norway}
  }
\email{vsetty@acm.org}





%%
%% By default, the full list of authors will be used in the page
%% headers. Often, this list is too long, and will overlap
%% other information printed in the page headers. This command allows
%% the author to define a more concise list
%% of authors' names for this purpose.
\renewcommand{\shortauthors}{Trovato et al.}

%%
%% The abstract is a short summary of the work to be presented in the
%% article.
\begin{abstract}
The field of automated fact-checking increasingly depends on retrieving web-based evidence to determine the veracity of claims in real-world scenarios. A significant challenge in this process is not only retrieving relevant information, but also identifying evidence that can both support and refute complex claims. Traditional retrieval methods may return documents that directly address claims or lean toward supporting them, but often struggle with more complex claims requiring indirect reasoning. While some existing benchmarks and methods target retrieval for fact-checking, a comprehensive real-world open-domain benchmark has been lacking. In this paper, we present a real-world retrieval benchmark \name{}, derived from Factiverse production logs, enhanced with human annotations. We rigorously evaluate state-of-the-art retrieval models in a zero-shot setup on \name{} and offer insights for developing practical retrieval systems for fact-checking. Code and data are available at \url{https://github.com/factiverse/factIR}.
\end{abstract}

%%
%% The code below is generated by the tool at http://dl.acm.org/ccs.cfm.
%% Please copy and paste the code instead of the example below.
%%
\begin{CCSXML}
<ccs2012>
   <concept>
       <concept_id>10002951.10003317</concept_id>
       <concept_desc>Information systems~Information retrieval</concept_desc>
       <concept_significance>500</concept_significance>
       </concept>
 </ccs2012>
\end{CCSXML}

\ccsdesc[500]{Information systems~Information retrieval}

%%
%% Keywords. The author(s) should pick words that accurately describe
%% the work being presented. Separate the keywords with commas.
\keywords{Fact-checking, Retrieval benchmark}
%% A "teaser" image appears between the author and affiliation
%% information and the body of the document, and typically spans the
%% page.
% \begin{teaserfigure}
%   \includegraphics[width=\textwidth]{sampleteaser}
%   \caption{Seattle Mariners at Spring Training, 2010.}
%   \Description{Enjoying the baseball game from the third-base
%   seats. Ichiro Suzuki preparing to bat.}
%   \label{fig:teaser}
% \end{teaserfigure}


%%
%% This command processes the author and affiliation and title
%% information and builds the first part of the formatted document.
\maketitle

\section{Introduction}
\label{sec:intro}

\begin{figure*}[tb]
    \centering
    \includegraphics[width=0.848\linewidth]{figs/circuitnn.pdf} 
    \caption{Illustration of differentiable CircuitNN. CircuitNN is designed based on differentiable NAND gates. After DAS is guided by PI and PO pairs of the truth table, CircuitNN can get the precise circuit architecture logic equivalent to the truth table.}
    \label{fig:circuitnn}
\end{figure*}

% 1. Describe the importance of logic synthesis
% 2. Existing Problems
% (a) Neural Architecture Search: Unstable, Predefined Setting, etc.
% (b) Circuit Generation: Probabilistic Model, Logic Equivalence

With the rapid advancement of technology, the scale of integrated circuits (ICs) has expanded exponentially. 
This expansion has introduced significant challenges in chip manufacturing, particularly concerning power and area metrics.
A primary objective in IC design is achieving the same circuit function with fewer transistors, thereby reducing power usage and area occupancy.

Logic synthesis~\cite{hachtel2005logicsynth}, a critical step in electronic design automation (EDA), transforms behavioral-level circuit designs into optimized gate-level circuits, ultimately yielding the final IC layout. 
The primary goal of logic synthesis is to identify the physical implementation with the fewest gates for a given circuit function. 
This task constitutes a challenging NP-hard combinatorial optimization problem. 
Current logic synthesis tools~\cite{brayton2010abc, wolf2013yosys} rely on human-designed heuristics, often leading to sub-optimal outcomes.

Differentiable architecture search (DAS) techniques~\cite{liu2018darts, chu2020darts} offer novel perspectives on addressing challenges in this problem.
Circuit functions can be represented through truth tables, which map binary inputs to their corresponding outputs. 
Truth tables provide a precise representation of input-output relationships, ensuring the design of functionally equivalent circuits.
Inspired by this, researchers~\cite{deepmind2024ai4sys, wang2024tnet} have begun exploring the application of DAS to synthesize circuits directly from truth tables.
Specifically, \citet{deepmind2024ai4sys} proposed CircuitNN, a framework that learns differentiable connection structures with logic gates, enabling the automatic generation of logic circuits from truth tables.
This approach significantly reduces the complexity of traditional circuit generation. 
Building on this, \citet{wang2024tnet} introduced T-Net, a triangle-shaped variant of CircuitNN, incorporating regularization techniques to enhance the efficiency of DAS.

Despite these advancements, several challenges remain. 
The computational complexity of DAS grows quadratically with the number of gates, posing scalability issues.
Although triangle-shaped architecture~\cite{wang2024tnet} partially mitigates this problem, redundancy persists. 
%Additionally, DAS is susceptible to converging to local optima, limiting the ability to search architectures that satisfy the given truth tables~\cite{liu2018darts}. 
%Furthermore, hyperparameters (network depth and layer width) require extensive searches, introducing complexity and prolonging the synthesis process. 
Additionally, DAS is susceptible to converging to local optima~\cite{liu2018darts} and hyperparameters (network depth and layer width) require extensive searches. 
The challenges arise from the vast search space in DAS. 
% Even with predefined settings for CircuitNN, finding a configuration that meets the truth table requires extensive trial and error during the DAS process. 
Intuitively, limiting the search space through predefined parameters (network depth, gates per layer, and connection probabilities) can significantly reduce the complexity.

Recent advances~\cite{openai2023gpt4, abramson2024alphafold3, esser2024sd3, li2024mar} in conditional generative models have demonstrated remarkable performance across language, vision, and graph generation tasks. 
Motivated by these developments, we propose a novel approach to circuit generation that generates preliminary circuit structures to guide DAS in generating refined circuits matching specified truth tables. 
Firstly, we introduce CircuitVQ, a tokenizer with a discrete codebook for circuit tokenization. 
Built upon our Circuit AutoEncoder framework~\cite{hou2022graphmae,li2023maskgae,wu2025mgvga}, CircuitVQ is trained through a circuit reconstruction task. 
Specifically, the CircuitVQ encoder encodes input circuits into discrete tokens using a learnable codebook, while the decoder reconstructs the circuit adjacency matrix based on these tokens.
Subsequently, the CircuitVQ encoder serves as a circuit tokenizer for CircuitAR pretraining, which employs a masked autoregressive modeling paradigm~\cite{chang2022maskgit, li2023mage}. 
In this process, the discrete codes function as supervision signals. 
After training, CircuitAR can generate discrete tokens progressively, which can be decoded into initial circuit structures by the decoder of the CircuitVQ. 
These prior insights can guide DAS in producing refined circuits that match the target truth tables precisely.

Our key contributions can be summarized as follows:
\begin{itemize}
\item We introduce CircuitVQ, a circuit tokenizer that facilitates graph autoregressive modeling for circuit generation, based on our Circuit AutoEncoder framework;
\item Develop CircuitAR, a model trained using masked autoregressive modeling, which generates initial circuit structures conditioned on given truth tables;
\item Propose a refinement framework that integrates differentiable architecture search to produce functionally equivalent circuits guided by target truth tables;
\item Comprehensive experiments demonstrating the scalability and capability emergence of our CircuitAR and the superior performance of the proposed circuit generation approach.
\end{itemize}

% Motivation
% (a) Diffusion (Vision, Graph), Autoregressive (Language, Vision)
% (b) Circuit Generation for Predefined Setting
% (c) Neural Architecture Search for Strict Logic Equivalence

% Contribution
% (a) Circuit Tokenizer (new transformer arch, training strategy)
% (b) CircuitAR (train and gen strategies, post-ar strategy)
% (c) Extensive Evaluation including BitD (Bit Distance) for Scalability

\section{Related Work}
Alongside a discussion of what is meant by LLM harmfulness,
this section covers two distinct strands of related work: measuring types of harm in LLMs, and LLMs for diverse annotation tasks. %First,

%Different kinds of 
Diverse undesirable LLM outputs, from toxic language to privacy invasion, have been discussed in the observed \cite{banko-etal-2020-unified}. Here we review the ones we include in our definition of ``harm.'' %definition. Plus, we review LLMs as judges. 
Toxic content can be elicited from both generative  \cite{deshpande2023toxicity} and masked LLMs \cite{ousidhoum-etal-2021-probing}. 
%Among ways 
To measure toxic or hateful language, some use APIs such as PerspectiveAPI \cite{lees2022new} or HateBERT \cite{caselli-etal-2021-hatebert}. \citet{openai2024gpt4technicalreport} report that GPT4 produces toxic content 0.78\% of the time, versus 6.48\% in GPT3.5.
%as opposed to GPT3.5 with 6.48\%. On the other hand,
\citet{dubey2024llama} report that llama3-70B produces harmful content 5\% of the time, %whereas the 405B model generates harm 3\% of the time. 
compared to 3\% in the 405B model.
Instead of %single value classifiers to measure harm, 
reporting an absolute rate, we focus on relative harmfulness of different LLMs. %, so we point to recent work on LLMs for annotation.

The first category of harm we consider is social stereotyping and bias. %discrimination. It has been shown that 
LLMs can perpetuate social bias based on gender, race, religion etc. \cite{lin-etal-2022-gendered,bender2021dangers,field-etal-2021-survey,gupta-etal-2024-sociodemographic,andriushchenko2024agentharm,mazeika2024harmbench}. This can marginalize these groups more, and results in less fair model performance. \citet{guo2024hey} designed a competition to elicit biased output from LLMs to assess the perception of bias from non-expert users. %The first part of our work is similar to this analysis, but 
We also intentionally elicit harmful output, going %we look at other types of harms besides bias.
beyond social bias.

%When the models become stronger, they become more robust to jailbreaking attacks to elicit harmful content. However, there are datasets that can still jailbreak models to produce harmful content \cite{andriushchenko2024agentharm,mazeika2024harmbench}.

Our second category of harm is offensiveness and toxicity, which %. As opposed to stereotyping or social discrimination, this harm 
%is more subjective and harder to define than the previous category, so there 
lacks an established definition due to its greater subjectivity \cite{dev-etal-2022-measures,korre-etal-2023-harmful}. We include hate speech (HS) and abusive language as toxic content. HS can be defined as expressions of offensive and discriminatory discourse towards a group or an individual based on characteristics such as race, religion, nationality, or other group characteristics \cite{john2000hate,jahan2023systematic,basile2019semeval,davidson2017automated}. It includes racism, negative stereotyping, and sexist language. On the other hand, abusive language is content with inappropriate words such as profanity or disrespectful terms. It also includes psychological threats such as humiliation. %or constant criticism. %Toxic content can be elicited from both generative models \cite{deshpande2023toxicity} and masked language models \cite{ousidhoum-etal-2021-probing}.

%In addition to obvious toxic content, LLMs can generate diverse implicit toxic outputs using reinforcement learning with favoring toxic content in the reward function \cite{wen-etal-2023-unveiling}.  Regarding the subjectivity of this task, \cite{korre-etal-2023-harmful} reannotate the existing datasets with different definitions of toxicity and show that broader definitions result in more robust annotations, but interannotator agreements are still lower than 0.5. \cite{dev-etal-2022-measures} also point out the lack of definition for bias and harm in general and propose a framework to guide researchers during the development of bias measures.

Harm can be implicit, such as privacy invasion
%We are also interested in privacy invasion,
where there is 
leakage of personal information. %leakage from the model. 
%LLMs can memorize details of the training data and then leak private information such as 
This includes social security numbers, phone numbers, or bank account information \cite{carlini2021extracting,brown2022does}. 
%There are several frameworks to test the privacy of LLMs \cite{li2024llm} and generate data for personal attribute inference \cite{yukhymenko2024synthetic,kim2024propile}.

%Our definition of harm includes hate speech (HS) as well. HS can be defined as \textcolor{red}{expressions of} hatred towards a social group, the humiliation of the members of a group, or %communication disparaging  extreme disparagement of a person or a group based on race, color, ethnicity, gender, sexual orientation, nationality, religion, or other group characteristics .

For data annotation, LLMs
%Besides text generation, 
%LLMs have been used to annotate data because they 
can %be comparable to 
replace humans for some tasks, %and make the annotation process faster and cheaper 
with gains in efficiency and economy \cite{tan2024large}. They have been used for sociological annotations such as for classification of stance, bots or humor  \cite{ziems2024can,zhu2023can}. For tasks such as topic and frame detection or sentence segmentation they can surpass crowd-workers
%Some works show that they can surpass crowd-workers for some tasks such as topic and frame detection or sentence segmentation %into research aspects 
\cite{he2024if,gilardi2023chatgpt}. Some have argued that human-LLM collaboration results in more reliable annotation \cite{he2024if,zhang2023llmaaa,kim2024meganno+}. In addition to more objective tasks,
%LLMs have been used to annotate data %even 
they have been applied to subjective annotations such as offensiveness and abusiveness \cite{pavlovic-poesio-2024-effectiveness,zhu2023can,he2023annollm}, %. For example, LLMs are used as judges to rank responses from different LLMs 
or to rank outputs from different LLMs based on helpfulness, accuracy, or relevance \cite{zheng2023judging,lin2024wildbench,dubois2024length}. These works tend to focus on human-large LLM interactions, whereas we focus on single-turn responses from smaller LLMs. We inspire from \citet{zheng2023judging} but we only measure harm instead of overall performance. Plus, we use 3 LLMs to evaluate smaller LLMs.
\begin{figure}
    \centering
    \includegraphics[width=0.9\linewidth]{figures/factiverseeditor.png}
    \caption{Factiverse fact-check editor with feedback mechanism. Example for the claim ``Keto diet can cure cancer.''}
    \label{fig:fceditor}
    \vspace{-10pt}
\end{figure}

\section{Data Curation}
We employ the Factiverse Fact-Check Editor~\cite{10.1145/3626772.3657663}, a web-based tool designed for automated fact-checking with a built-in feedback mechanism. As shown in Figure \ref{fig:fceditor}, users input claims they wish to fact-check, and the system aggregates evidence retrieved from multiple search engines, such as Google and Bing, ensuring comprehensive coverage of web content. The editor utilizes query generation and evidence filtering to retrieve the most relevant documents~\cite{10.1145/3626772.3661361,10.1145/3627673.3679985}. Users can then provide detailed feedback on two aspects: the relevance of the evidence to the claim and the correctness of the stance, whether the claim is supported or refuted.

To ensure high-quality labels for this benchmark, we focus on feedback collected from domain experts, specifically professional fact-checkers and individuals with journalism backgrounds, who verify claims using Factiverse’s production deployment. The claims are organically generated by users as they engage in tasks covering a variety of topics, including politics, healthcare, and the economy. The feedback was collected over the period spanning 2021 to 2023.

The annotators were provided with the following instructions and examples: \textbf{Evidence is deemed relevant to a claim if the extracted snippet can help verify the claim’s veracity.} For example, for the claim ``Keto diet can cure cancer,'' a document discussing the general benefits of the ``keto diet'' without addressing its effect on cancer would be considered irrelevant. On the other hand, a scientific study examining the ``ketogenic diet as a treatment and prevention strategy for cancer'' (as illustrated in Figure~\ref{fig:fceditor}), even if it disproves the claim, would be considered relevant.

\begin{figure}
\begin{minipage}{0.22\textwidth}
\captionof{table}{Topical distribution}
\label{tab:topical}
\begin{tabular}{lr}
\hline
\textbf{Topic} & \textbf{Count}  \\

\midrule
Politics & 26 \\
Economy & 24 \\
Health & 21 \\
Law & 6 \\
Climate & 8 \\
Education & 3 \\
Other &  12\\

\bottomrule
\end{tabular}\end{minipage}
\begin{minipage}{0.22\textwidth}
\captionof{table}{Numerical  taxonomy}
\label{tab:taxonomy}
\begin{tabular}{lr}
\hline
\textbf{Topic} & \textbf{Count}  \\

\midrule
Statistical & 40 \\
Temporal & 27 \\
Comparative & 8 \\

\bottomrule
\end{tabular}
\end{minipage}

\end{figure}

\begin{table*}[!ht]
    \centering
    \small
    \begin{tabular}{lccccccccc}
    \toprule
     \textbf{Method}& \multicolumn{1}{c}{nDcG@5} &\multicolumn{1}{c}{Recall@5} &\multicolumn{1}{c}{nDcG@10} & \multicolumn{1}{c}{Recall@10} & \multicolumn{1}{c}{nDcG@100} & \multicolumn{1}{c}{Recal@100} \\
   % &\multicolumn{1}{c|}{n@10} & \multicolumn{1}{c}{RR}&\multicolumn{1}{c|}{n@10} & \multicolumn{1}{c}{RR}&\multicolumn{1}{c|}{n@10} & \multicolumn{1}{c}{RR}&\multicolumn{1}{c|}{n@10} & \multicolumn{1}{c}{RR} &\multicolumn{1}{c|}{n@10} & \multicolumn{1}{c}{RR} &\multicolumn{1}{c|}{n@10} & \multicolumn{1}{c}{RR} &\multicolumn{1}{c|}{n@10} & \multicolumn{1}{c}{RR} \\
     % & cover-EM & cover-EM& EM& EM&EM \\
     \midrule

     

    \midrule
    
    \textbf{Lexical} & & \\
            BM25 \cite{bm25} & 0.288 & 0.253 &0.345 & 0.421& 0.475&0.779 \\
          


     \midrule
     
      \textbf{Sparse} & & \\
            SPLADEV2 \cite{SPLADEv2} & 0.287 & 0.231& 0.336 & 0.384 & 0.473 & 0.783 \\
     \midrule
      \textbf{Dense} & & \\
            Stella-en-v5 & 0.090 & 0.065 & 0.098 & 0.110 & 0.158 & 0.314 \\
            DPR \cite{karpukhin-etal-2020-dense} & 0.247 &  0.219 & 0.292 & 0.356 & 0.404 & 0.670    \\
            ANCE \cite{ance} & 0.246 & 0.184  & 0.289 & 0.327 & 0.419 & 0.691\\
      


                  %      MDR & & \\
                        tas-b \cite{tas-b} & 0.289 & 0.232 & 0.336 & 0.399 & 0.468 & 0.771 \\
                        MPNet \cite{mpnet} & 0.290 & 0.250 & 0.327& 0.388 & 0.464& 0.767 \\
       Contriever \cite{contriever} & 0.299& 0.249 & 0.346 & 0.406 & 0.471 & 0.760\\
       % multilingual-minilm (Factiverse) & &  & 0.240 & 0.312 & 0.392 & 0.729 \\
       %        xlm-roberta (Factiverse) & &  & 0.271  & 0.319 & 0.415& 0.720 \\
    COlBERTV2 \cite{santhanam-etal-2022-colbertv2} & 0.252 &  0.230 & 0.325 & 0.419 & 0.465 & 0.789  \\
    Snowflake-arctic-embed-s \cite{merrick2024embeddingclusteringdataimprove} & \textbf{0.367} \up{27.43} & \textbf{0.302} \up{19.37} & \textbf{0.420} \up{21.74} & \textbf{0.480} \up{14.01} & \textbf{0.529} \up{11.37} & \textbf{0.795} \up{2.05} \\
\midrule
\textbf{Re-Ranker} \\
BM25 + & \\
 - ColBERTV2 &0.265 & 0.253 &  0.333 & 0.424 & 0.464 & 0.759 \\
- MARCO-MiniLM-H384 & 0.293 & 0.247     & 0.349 & 0.408 & 0.485 & 0.779\\

% - MSMARCO-electra-base  & 0.282 & 0.233& 0.327 & 0.379 & 0.473  & 0.779 \\
 - MARCO-MiniLM-en-de & 0.278 & 0.264 & 0.342 & 0.426 & 0.479  & 0.779 \\
- bge-reranker-base & 0.222 & 0.205 & 0.301 & 0.398 & 0.452 & 0.779\\
- bge-ranker-v2 & 0.252 & 0.235 & 0.312 & 0.399 & 0.460 & 0.779 \\
- Jina-reranker-v2 & 0.270 & 0.245 & 0.336 & 0.421 & 0.474 & 0.779\\
- gte-multilingual & \underline{0.308} \up{6.04}& \underline{0.277} \up{9.49} & \underline{0.368} \up{6.67} & \underline{0.437} \up{3.80}& \underline{0.496} \up{4.42} & \underline{0.779} \\

% Snowflake-arctic-embed-s &  \\
% -  MARCO-MiniLM-H384 \\
% -  gte-multilingual & 0.316 & 0.290 \\

\midrule
% \textbf{LLM+Retrieval} \\
%         Decompose-retrieve & &&&&& \\
%     \midrule

    % \textbf{Re-Ranking} & & & & & & & \\
    % BM25 + CE & & & & & & & \\
    % \bottomrule
%          \textbf{Semi-Oracle} & & & & & & & & & & \\
%         ClaimOnly & 33.03& 39.57& 36.31 & 58.15 & 33.81& 48.61& 25.70 & 23.99&28.55 & 63.79 & 7.95 & 33.42 &  43.70  \\


% \programfc{} & 38.57& 42.49 & 37.12  & 50.66& 35.22& 45.76& 33.43&32.50 & 32.95 & 55.11 & 25.44 & 37.83 & 43.79 \\
%     \claimdecomp{}&33.43 & 39.78 & 35.04&  55.49 & 33.93 & 48.53& 34.37& 33.50&29.48 & 63.11& 10.85 & 34.48& 44.19  \\
%      \numdecomp{}& 33.81& 39.46 &33.57 & 53.45& 34.18& 47.00& 35.23&34.23 &29.11 & 60.29 & 13.90 &34.43& 43.24\\
    \end{tabular}
    \caption{Retrieval results on \name{}, nDCG@10 across datasets. The best results are in bold and the second best results are underlined with \% improvements indicated by \up{} over the baseline BM25 being specified in brackets. }
     \vspace{-1em}
    \label{tab:main_result}
\end{table*}
\vspace{-0.6em}

\subsection{Benchmark Statistics}
 The benchmark comprises, \textbf{1413} claim-evidence  pair relevance annotations with a total of \textbf{90047} documents in the corpus collection and 100 claims with an average of \textbf{13.89} documents / relevance assesments per query. We provide fine-grained topical analysis in Table \ref{tab:topical}. We observe that a number of claims in the benchmark are \textit{quantitative or temporal} in nature, as shown in Table \ref{tab:taxonomy}. We also observe that many claims are \textit{compositional} comprising multiple aspects, making \name{} a challenging retrieval benchmark. We performed a qualitative meta-analysis of relevance assignments with the help of two researchers who were asked to annotate as ``1" when they deem the relevance assessment to be correct else ``0". The annotators found the relevance assessments to be of high quality (\textbf{88.03\%} was deemed to be correct) with a high agreement of \textbf{0.946} as indicated by Cohen's kappa.
\section{Experimental Setup}
\label{appendix:experimental_setups}
We evaluate EDELINE on the Atari 100k benchmark~\cite{chevalier-schwarzer2023biggerbetterfasterhumanlevel}, which serves as the standard evaluation protocol in recent model-based RL literature for fair comparison. In addition, our experimental validation extends to ViZDoom~\cite{kempka2016vizdoomdoombasedairesearch} and MiniGrid~\cite{chevalier-boisvert2023minigrid} environments to demonstrate broader applicability. To ensure statistical significance, all reported results represent averages across three independent runs. The Atari 100k benchmark~\cite{chevalier-schwarzer2023biggerbetterfasterhumanlevel} encompasses 26 diverse Atari games that evaluate various aspects of agent capabilities. Each agent receives a strict limitation of 100k environment interactions for learning, in contrast to conventional Atari agents that typically require 50 million steps. EDELINE's performance is evaluated against current state-of-the-art world model-based approaches, including DIAMOND~\cite{alonso2024diamond}, STORM~\cite{zhang2023storm}, DreamerV3~\cite{hafner2024DreamerV3}, IRIS~\cite{micheli2023iris}, TWM~\cite{robine2023TWM}, and Drama~\cite{anonymous2025drama}. For evaluating 3D scene understanding capabilities, we employ VizDoom scenarios that demand sophisticated 3D spatial reasoning in first-person environments. This provides a crucial testing ground beyond the third-person perspective of Atari environments. Furthermore, the MiniGrid memory scenarios evaluate memorization capabilities through tasks that require information retention across extended time horizons.
\begin{table}[ht!]
\centering
\caption{\textbf{Super Resolution Performance Results.} Our proposed WGAN EEG Spatial Upsampling method significantly outperforms a baseline of Bicubic Interpolation commonly used in EEG upsampling pipelines.}
\label{tab:results}
\resizebox{0.8\linewidth}{!}{%
\begin{tabular}{@{}cccccc@{}}
\toprule
\multirow{2}{*}{\textbf{Dataset}} & \multirow{2}{*}{\textbf{Scale}} & \multicolumn{2}{c}{\textbf{Bicubic}} & \multicolumn{2}{c}{\textbf{WGAN}} \\ \cmidrule(l){3-6} 
                      &   & \textbf{MSE} & \textbf{MAE} & \textbf{MSE}    & \textbf{MAE}   \\
\toprule
\multirow{2}{*}{Val}  & 2 & 3.71E7       & 3.89E3       & \textbf{2.01E3} & \textbf{24.38} \\
                      & 4 & 7.23E7       & 6.42E3       & \textbf{8.53E3} & \textbf{63.83} \\
\midrule
\multirow{2}{*}{Test} & 2 & 3.75E7       & 3.91E3       & \textbf{2.06E3} & \textbf{24.66} \\
                      & 4 & 7.30E7       & 6.45E3       & \textbf{8.68E3} & \textbf{64.39} \\
\bottomrule
\end{tabular}%
}
\end{table}
% \begin{acks}
% To Robert, for the bagels and explaining CMYK and color spaces.
% \end{acks}

%%
%% The next two lines define the bibliography style to be used, and
%% the bibliography file.
\bibliographystyle{ACM-Reference-Format}
\bibliography{references}


%%
%% If your work has an appendix, this is the place to put it.

\end{document}
\endinput
%%
%% End of file `sample-sigconf.tex'.
