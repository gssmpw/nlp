%%
%% This is file `sample-sigconf.tex',
%% generated with the docstrip utility.
%%
%% The original source files were:
%%
%% samples.dtx  (with options: `all,proceedings,bibtex,sigconf')
%% 
%% IMPORTANT NOTICE:
%% 
%% For the copyright see the source file.
%% 
%% Any modified versions of this file must be renamed
%% with new filenames distinct from sample-sigconf.tex.
%% 
%% For distribution of the original source see the terms
%% for copying and modification in the file samples.dtx.
%% 
%% This generated file may be distributed as long as the
%% original source files, as listed above, are part of the
%% same distribution. (The sources need not necessarily be
%% in the same archive or directory.)
%%
%%
%% Commands for TeXCount
%TC:macro \cite [option:text,text]
%TC:macro \citep [option:text,text]
%TC:macro \citet [option:text,text]
%TC:envir table 0 1
%TC:envir table* 0 1
%TC:envir tabular [ignore] word
%TC:envir displaymath 0 word
%TC:envir math 0 word
%TC:envir comment 0 0
%%
%%
%% The first command in your LaTeX source must be the \documentclass
%% command.
%%
%% For submission and review of your manuscript please change the
%% command to \documentclass[manuscript, screen, review]{acmart}.
%%
%% When submitting camera ready or to TAPS, please change the command
%% to \documentclass[sigconf]{acmart} or whichever template is required
%% for your publication.
%%
%%
\documentclass[sigconf]{acmart}



%%
%% \BibTeX command to typeset BibTeX logo in the docs
\AtBeginDocument{%
  \providecommand\BibTeX{{%
    Bib\TeX}}}
\newcommand{\notess}[1]{\textcolor{teal}{#1}}
\newcommand{\citeneeded}{\textcolor{red}{\textbf{[cite]}}}
\newcommand{\up}[1]{\small ($\textcolor{green}{\blacktriangle}#1\%)$}
\newcommand{\speedup}[1]{\small ($\textcolor{green}{\blacktriangle}\textbf{#1})$}
\newcommand{\down}[1]{\small ($\textcolor{red}{\blacktriangledown}#1\%)$}

\newcommand{\name}{\textsc{FactIR}}
%% Rights management information.  This information is sent to you
%% when you complete the rights form.  These commands have SAMPLE
%% values in them; it is your responsibility as an author to replace
%% the commands and values with those provided to you when you
%% complete the rights form.
\setcopyright{acmlicensed}
\copyrightyear{2018}
\acmYear{2018}
\acmDOI{XXXXXXX.XXXXXXX}

%% These commands are for a PROCEEDINGS abstract or paper.
\acmConference[Conference acronym 'XX]{Make sure to enter the correct
  conference title from your rights confirmation emai}{June 03--05,
  2018}{Woodstock, NY}
%%
%%  Uncomment \acmBooktitle if the title of the proceedings is different
%%  from ``Proceedings of ...''!
%%
%%\acmBooktitle{Woodstock '18: ACM Symposium on Neural Gaze Detection,
%%  June 03--05, 2018, Woodstock, NY}
\acmISBN{978-1-4503-XXXX-X/18/06}


%%
%% Submission ID.
%% Use this when submitting an article to a sponsored event. You'll
%% receive a unique submission ID from the organizers
%% of the event, and this ID should be used as the parameter to this command.
%%\acmSubmissionID{123-A56-BU3}

%%
%% For managing citations, it is recommended to use bibliography
%% files in BibTeX format.
%%
%% You can then either use BibTeX with the ACM-Reference-Format style,
%% or BibLaTeX with the acmnumeric or acmauthoryear sytles, that include
%% support for advanced citation of software artefact from the
%% biblatex-software package, also separately available on CTAN.
%%
%% Look at the sample-*-biblatex.tex files for templates showcasing
%% the biblatex styles.
%%

%%
%% The majority of ACM publications use numbered citations and
%% references.  The command \citestyle{authoryear} switches to the
%% "author year" style.
%%
%% If you are preparing content for an event
%% sponsored by ACM SIGGRAPH, you must use the "author year" style of
%% citations and references.
%% Uncommenting
%% the next command will enable that style.
%%\citestyle{acmauthoryear}


%%
%% end of the preamble, start of the body of the document source.
\copyrightyear{2025}
\acmYear{2025}
\setcopyright{cc}
\setcctype{by}
\acmConference[WWW Companion '25]{Companion Proceedings of the ACM Web
Conference 2025}{April 28-May 2, 2025}{Sydney, NSW, Australia}
\acmBooktitle{Companion Proceedings of the ACM Web Conference 2025 (WWW
Companion '25), April 28-May 2, 2025, Sydney, NSW, Australia}
\acmDOI{10.1145/3701716.3715300}

\begin{document}

%%
%% The "title" command has an optional parameter,
%% allowing the author to define a "short title" to be used in page headers.
\title{FactIR: A Real-World Zero-shot Open-Domain Retrieval Benchmark for Fact-Checking}

%%
%% The "author" command and its associated commands are used to define
%% the authors and their affiliations.
%% Of note is the shared affiliation of the first two authors, and the
%% "authornote" and "authornotemark" commands
%% used to denote shared contribution to the research.



\author{Venktesh V}
\affiliation{%
 \institution{Factiverse AI and Delft University of Technology}
 \city{Delft}
 \country{Netherlands}}
 \email{V.Viswanathan-1@tudelft.nl}
 
\author{Vinay Setty}
\affiliation{%
  \institution{Factiverse AI and University of Stavanger}
  \city{Stavanger}
  \country{Norway}
  }
\email{vsetty@acm.org}





%%
%% By default, the full list of authors will be used in the page
%% headers. Often, this list is too long, and will overlap
%% other information printed in the page headers. This command allows
%% the author to define a more concise list
%% of authors' names for this purpose.
\renewcommand{\shortauthors}{Trovato et al.}

%%
%% The abstract is a short summary of the work to be presented in the
%% article.
\begin{abstract}
The field of automated fact-checking increasingly depends on retrieving web-based evidence to determine the veracity of claims in real-world scenarios. A significant challenge in this process is not only retrieving relevant information, but also identifying evidence that can both support and refute complex claims. Traditional retrieval methods may return documents that directly address claims or lean toward supporting them, but often struggle with more complex claims requiring indirect reasoning. While some existing benchmarks and methods target retrieval for fact-checking, a comprehensive real-world open-domain benchmark has been lacking. In this paper, we present a real-world retrieval benchmark \name{}, derived from Factiverse production logs, enhanced with human annotations. We rigorously evaluate state-of-the-art retrieval models in a zero-shot setup on \name{} and offer insights for developing practical retrieval systems for fact-checking. Code and data are available at \url{https://github.com/factiverse/factIR}.
\end{abstract}

%%
%% The code below is generated by the tool at http://dl.acm.org/ccs.cfm.
%% Please copy and paste the code instead of the example below.
%%
\begin{CCSXML}
<ccs2012>
   <concept>
       <concept_id>10002951.10003317</concept_id>
       <concept_desc>Information systems~Information retrieval</concept_desc>
       <concept_significance>500</concept_significance>
       </concept>
 </ccs2012>
\end{CCSXML}

\ccsdesc[500]{Information systems~Information retrieval}

%%
%% Keywords. The author(s) should pick words that accurately describe
%% the work being presented. Separate the keywords with commas.
\keywords{Fact-checking, Retrieval benchmark}
%% A "teaser" image appears between the author and affiliation
%% information and the body of the document, and typically spans the
%% page.
% \begin{teaserfigure}
%   \includegraphics[width=\textwidth]{sampleteaser}
%   \caption{Seattle Mariners at Spring Training, 2010.}
%   \Description{Enjoying the baseball game from the third-base
%   seats. Ichiro Suzuki preparing to bat.}
%   \label{fig:teaser}
% \end{teaserfigure}


%%
%% This command processes the author and affiliation and title
%% information and builds the first part of the formatted document.
\maketitle

\section{Introduction}


\begin{figure}[t]
\centering
\includegraphics[width=0.6\columnwidth]{figures/evaluation_desiderata_V5.pdf}
\vspace{-0.5cm}
\caption{\systemName is a platform for conducting realistic evaluations of code LLMs, collecting human preferences of coding models with real users, real tasks, and in realistic environments, aimed at addressing the limitations of existing evaluations.
}
\label{fig:motivation}
\end{figure}

\begin{figure*}[t]
\centering
\includegraphics[width=\textwidth]{figures/system_design_v2.png}
\caption{We introduce \systemName, a VSCode extension to collect human preferences of code directly in a developer's IDE. \systemName enables developers to use code completions from various models. The system comprises a) the interface in the user's IDE which presents paired completions to users (left), b) a sampling strategy that picks model pairs to reduce latency (right, top), and c) a prompting scheme that allows diverse LLMs to perform code completions with high fidelity.
Users can select between the top completion (green box) using \texttt{tab} or the bottom completion (blue box) using \texttt{shift+tab}.}
\label{fig:overview}
\end{figure*}

As model capabilities improve, large language models (LLMs) are increasingly integrated into user environments and workflows.
For example, software developers code with AI in integrated developer environments (IDEs)~\citep{peng2023impact}, doctors rely on notes generated through ambient listening~\citep{oberst2024science}, and lawyers consider case evidence identified by electronic discovery systems~\citep{yang2024beyond}.
Increasing deployment of models in productivity tools demands evaluation that more closely reflects real-world circumstances~\citep{hutchinson2022evaluation, saxon2024benchmarks, kapoor2024ai}.
While newer benchmarks and live platforms incorporate human feedback to capture real-world usage, they almost exclusively focus on evaluating LLMs in chat conversations~\citep{zheng2023judging,dubois2023alpacafarm,chiang2024chatbot, kirk2024the}.
Model evaluation must move beyond chat-based interactions and into specialized user environments.



 

In this work, we focus on evaluating LLM-based coding assistants. 
Despite the popularity of these tools---millions of developers use Github Copilot~\citep{Copilot}---existing
evaluations of the coding capabilities of new models exhibit multiple limitations (Figure~\ref{fig:motivation}, bottom).
Traditional ML benchmarks evaluate LLM capabilities by measuring how well a model can complete static, interview-style coding tasks~\citep{chen2021evaluating,austin2021program,jain2024livecodebench, white2024livebench} and lack \emph{real users}. 
User studies recruit real users to evaluate the effectiveness of LLMs as coding assistants, but are often limited to simple programming tasks as opposed to \emph{real tasks}~\citep{vaithilingam2022expectation,ross2023programmer, mozannar2024realhumaneval}.
Recent efforts to collect human feedback such as Chatbot Arena~\citep{chiang2024chatbot} are still removed from a \emph{realistic environment}, resulting in users and data that deviate from typical software development processes.
We introduce \systemName to address these limitations (Figure~\ref{fig:motivation}, top), and we describe our three main contributions below.


\textbf{We deploy \systemName in-the-wild to collect human preferences on code.} 
\systemName is a Visual Studio Code extension, collecting preferences directly in a developer's IDE within their actual workflow (Figure~\ref{fig:overview}).
\systemName provides developers with code completions, akin to the type of support provided by Github Copilot~\citep{Copilot}. 
Over the past 3 months, \systemName has served over~\completions suggestions from 10 state-of-the-art LLMs, 
gathering \sampleCount~votes from \userCount~users.
To collect user preferences,
\systemName presents a novel interface that shows users paired code completions from two different LLMs, which are determined based on a sampling strategy that aims to 
mitigate latency while preserving coverage across model comparisons.
Additionally, we devise a prompting scheme that allows a diverse set of models to perform code completions with high fidelity.
See Section~\ref{sec:system} and Section~\ref{sec:deployment} for details about system design and deployment respectively.



\textbf{We construct a leaderboard of user preferences and find notable differences from existing static benchmarks and human preference leaderboards.}
In general, we observe that smaller models seem to overperform in static benchmarks compared to our leaderboard, while performance among larger models is mixed (Section~\ref{sec:leaderboard_calculation}).
We attribute these differences to the fact that \systemName is exposed to users and tasks that differ drastically from code evaluations in the past. 
Our data spans 103 programming languages and 24 natural languages as well as a variety of real-world applications and code structures, while static benchmarks tend to focus on a specific programming and natural language and task (e.g. coding competition problems).
Additionally, while all of \systemName interactions contain code contexts and the majority involve infilling tasks, a much smaller fraction of Chatbot Arena's coding tasks contain code context, with infilling tasks appearing even more rarely. 
We analyze our data in depth in Section~\ref{subsec:comparison}.



\textbf{We derive new insights into user preferences of code by analyzing \systemName's diverse and distinct data distribution.}
We compare user preferences across different stratifications of input data (e.g., common versus rare languages) and observe which affect observed preferences most (Section~\ref{sec:analysis}).
For example, while user preferences stay relatively consistent across various programming languages, they differ drastically between different task categories (e.g. frontend/backend versus algorithm design).
We also observe variations in user preference due to different features related to code structure 
(e.g., context length and completion patterns).
We open-source \systemName and release a curated subset of code contexts.
Altogether, our results highlight the necessity of model evaluation in realistic and domain-specific settings.





\section{Related Work}
The landscape of large language model vulnerabilities has been extensively studied in recent literature \cite{crothers2023machinegeneratedtextcomprehensive,shayegani2023surveyvulnerabilitieslargelanguage,Yao_2024,Huang2023ASO}, that propose detailed taxonomies of threats. These works categorize LLM attacks into distinct types, such as adversarial attacks, data poisoning, and specific vulnerabilities related to prompt engineering. Among these, prompt injection attacks have emerged as a significant and distinct category, underscoring their relevance to LLM security.

The following high-level overview of the collected taxonomy of LLM vulnerabilities is defined in \cite{Yao_2024}:
\begin{itemize}
    \item Adversarial Attacks: Data Poisoning, Backdoor Attacks
    \item Inference Attacks: Attribute Inference, Membership Inferences
    \item Extraction Attacks
    \item Bias and Unfairness
Exploitation
    \item Instruction Tuning Attacks: Jailbreaking, Prompt Injection.
\end{itemize}
Prompt injection attacks are further classified in \cite{shayegani2023surveyvulnerabilitieslargelanguage} into the following: Goal hijacking and \textbf{Prompt leakage}.

The reviewed taxonomies underscore the need for comprehensive frameworks to evaluate LLM security. The agentic approach introduced in this paper builds on these insights, automating adversarial testing to address a wide range of scenarios, including those involving prompt leakage and role-specific vulnerabilities.

\subsection{Prompt Injection and Prompt Leakage}

Prompt injection attacks exploit the blending of instructional and data inputs, manipulating LLMs into deviating from their intended behavior. Prompt injection attacks encompass techniques that override initial instructions, expose private prompts, or generate malicious outputs \cite{Huang2023ASO}. A subset of these attacks, known as prompt leakage, aims specifically at extracting sensitive system prompts embedded within LLM configurations. In \cite{shayegani2023surveyvulnerabilitieslargelanguage}, authors differentiate between prompt leakage and related methods such as goal hijacking, further refining the taxonomy of LLM-specific vulnerabilities.

\subsection{Defense Mechanisms}

Various defense mechanisms have been proposed to address LLM vulnerabilities, particularly prompt injection and leakage \cite{shayegani2023surveyvulnerabilitieslargelanguage,Yao_2024}. We focused on cost-effective methods like instruction postprocessing and prompt engineering, which are viable for proprietary models that cannot be retrained. Instruction preprocessing sanitizes inputs, while postprocessing removes harmful outputs, forming a dual-layer defense. Preprocessing methods include perplexity-based filtering \cite{Jain2023BaselineDF,Xu2022ExploringTU} and token-level analysis \cite{Kumar2023CertifyingLS}. Postprocessing employs another set of techniques, such as censorship by LLMs \cite{Helbling2023LLMSD,Inan2023LlamaGL}, and use of canary tokens and pattern matching \cite{vigil-llm,rebuff}, although their fundamental limitations are noted \cite{Glukhov2023LLMCA}. Prompt engineering employs carefully designed instructions \cite{Schulhoff2024ThePR} and advanced techniques like spotlighting \cite{Hines2024DefendingAI} to mitigate vulnerabilities, though no method is foolproof \cite{schulhoff-etal-2023-ignore}. Adversarial training, by incorporating adversarial examples into the training process, strengthens models against attacks \cite{Bespalov2024TowardsBA,Shaham2015UnderstandingAT}.

\subsection{Security Testing for Prompt Injection Attacks}

Manual testing, such as red teaming \cite{ganguli2022redteaminglanguagemodels} and handcrafted "Ignore Previous Prompt" attacks \cite{Perez2022IgnorePP}, highlights vulnerabilities but is limited in scale. Automated approaches like PAIR \cite{chao2024jailbreakingblackboxlarge} and GPTFUZZER \cite{Yu2023GPTFUZZERRT} achieve higher success rates by refining prompts iteratively or via automated fuzzing. Red teaming with LLMs \cite{Perez2022RedTL} and reinforcement learning \cite{anonymous2024diverse} uncovers diverse vulnerabilities, including data leakage and offensive outputs. Indirect Prompt Injection (IPI) manipulates external data to compromise applications \cite{Greshake2023NotWY}, adapting techniques like SQL injection to LLMs \cite{Liu2023PromptIA}. Prompt secrecy remains fragile, with studies showing reliable prompt extraction \cite{Zhang2023EffectivePE}. Advanced frameworks like Token Space Projection \cite{Maus2023AdversarialPF} and Weak-to-Strong Jailbreaking Attacks \cite{zhao2024weaktostrongjailbreakinglargelanguage} exploit token-space relationships, achieving high success rates for prompt extraction and jailbreaking.

\subsection{Agentic Frameworks for Evaluating LLM Security}

The development of multi-agent systems leveraging large language models (LLMs) has shown promising results in enhancing task-solving capabilities \cite{Hong2023MetaGPTMP, Wang2023UnleashingTE, Talebirad2023MultiAgentCH, Wu2023AutoGenEN, Du2023ImprovingFA}. A key aspect across various frameworks is the specialization of roles among agents \cite{Hong2023MetaGPTMP, Wu2023AutoGenEN}, which mimics human collaboration and improves task decomposition.

Agentic frameworks and the multi-agent debate approach benefit from agent interaction, where agents engage in conversations or debates to refine outputs and correct errors \cite{Wu2023AutoGenEN}. For example, debate systems improve factual accuracy and reasoning by iteratively refining responses through collaborative reasoning \cite{Du2023ImprovingFA}, while AG2 allows agents to autonomously interact and execute tasks with minimal human input.

These frameworks highlight the viability of agentic systems, showing how specialized roles and collaborative mechanisms lead to improved performance, whether in factuality, reasoning, or task execution. By leveraging the strengths of diverse agents, these systems demonstrate a scalable approach to problem-solving.

Recent research on testing LLMs using other LLMs has shown that this approach can be highly effective \cite{chao2024jailbreakingblackboxlarge, Yu2023GPTFUZZERRT, Perez2022RedTL}. Although the papers do not explicitly employ agentic frameworks they inherently reflect a pattern similar to that of an "attacker" and a "judge". \cite{chao2024jailbreakingblackboxlarge}  This pattern became a focal point for our work, where we put the judge into a more direct dialogue, enabling it to generate attacks based on the tested agent response in an active conversation.

A particularly influential paper in shaping our approach is Jailbreaking Black Box Large Language Models in Twenty Queries \cite{chao2024jailbreakingblackboxlarge}. This paper not only introduced the attacker/judge architecture but also provided the initial system prompts used for a judge.
\begin{figure}
    \centering
    \includegraphics[width=0.9\linewidth]{figures/factiverseeditor.png}
    \caption{Factiverse fact-check editor with feedback mechanism. Example for the claim ``Keto diet can cure cancer.''}
    \label{fig:fceditor}
    \vspace{-10pt}
\end{figure}

\section{Data Curation}
We employ the Factiverse Fact-Check Editor~\cite{10.1145/3626772.3657663}, a web-based tool designed for automated fact-checking with a built-in feedback mechanism. As shown in Figure \ref{fig:fceditor}, users input claims they wish to fact-check, and the system aggregates evidence retrieved from multiple search engines, such as Google and Bing, ensuring comprehensive coverage of web content. The editor utilizes query generation and evidence filtering to retrieve the most relevant documents~\cite{10.1145/3626772.3661361,10.1145/3627673.3679985}. Users can then provide detailed feedback on two aspects: the relevance of the evidence to the claim and the correctness of the stance, whether the claim is supported or refuted.

To ensure high-quality labels for this benchmark, we focus on feedback collected from domain experts, specifically professional fact-checkers and individuals with journalism backgrounds, who verify claims using Factiverse’s production deployment. The claims are organically generated by users as they engage in tasks covering a variety of topics, including politics, healthcare, and the economy. The feedback was collected over the period spanning 2021 to 2023.

The annotators were provided with the following instructions and examples: \textbf{Evidence is deemed relevant to a claim if the extracted snippet can help verify the claim’s veracity.} For example, for the claim ``Keto diet can cure cancer,'' a document discussing the general benefits of the ``keto diet'' without addressing its effect on cancer would be considered irrelevant. On the other hand, a scientific study examining the ``ketogenic diet as a treatment and prevention strategy for cancer'' (as illustrated in Figure~\ref{fig:fceditor}), even if it disproves the claim, would be considered relevant.

\begin{figure}
\begin{minipage}{0.22\textwidth}
\captionof{table}{Topical distribution}
\label{tab:topical}
\begin{tabular}{lr}
\hline
\textbf{Topic} & \textbf{Count}  \\

\midrule
Politics & 26 \\
Economy & 24 \\
Health & 21 \\
Law & 6 \\
Climate & 8 \\
Education & 3 \\
Other &  12\\

\bottomrule
\end{tabular}\end{minipage}
\begin{minipage}{0.22\textwidth}
\captionof{table}{Numerical  taxonomy}
\label{tab:taxonomy}
\begin{tabular}{lr}
\hline
\textbf{Topic} & \textbf{Count}  \\

\midrule
Statistical & 40 \\
Temporal & 27 \\
Comparative & 8 \\

\bottomrule
\end{tabular}
\end{minipage}

\end{figure}

\begin{table*}[!ht]
    \centering
    \small
    \begin{tabular}{lccccccccc}
    \toprule
     \textbf{Method}& \multicolumn{1}{c}{nDcG@5} &\multicolumn{1}{c}{Recall@5} &\multicolumn{1}{c}{nDcG@10} & \multicolumn{1}{c}{Recall@10} & \multicolumn{1}{c}{nDcG@100} & \multicolumn{1}{c}{Recal@100} \\
   % &\multicolumn{1}{c|}{n@10} & \multicolumn{1}{c}{RR}&\multicolumn{1}{c|}{n@10} & \multicolumn{1}{c}{RR}&\multicolumn{1}{c|}{n@10} & \multicolumn{1}{c}{RR}&\multicolumn{1}{c|}{n@10} & \multicolumn{1}{c}{RR} &\multicolumn{1}{c|}{n@10} & \multicolumn{1}{c}{RR} &\multicolumn{1}{c|}{n@10} & \multicolumn{1}{c}{RR} &\multicolumn{1}{c|}{n@10} & \multicolumn{1}{c}{RR} \\
     % & cover-EM & cover-EM& EM& EM&EM \\
     \midrule

     

    \midrule
    
    \textbf{Lexical} & & \\
            BM25 \cite{bm25} & 0.288 & 0.253 &0.345 & 0.421& 0.475&0.779 \\
          


     \midrule
     
      \textbf{Sparse} & & \\
            SPLADEV2 \cite{SPLADEv2} & 0.287 & 0.231& 0.336 & 0.384 & 0.473 & 0.783 \\
     \midrule
      \textbf{Dense} & & \\
            Stella-en-v5 & 0.090 & 0.065 & 0.098 & 0.110 & 0.158 & 0.314 \\
            DPR \cite{karpukhin-etal-2020-dense} & 0.247 &  0.219 & 0.292 & 0.356 & 0.404 & 0.670    \\
            ANCE \cite{ance} & 0.246 & 0.184  & 0.289 & 0.327 & 0.419 & 0.691\\
      


                  %      MDR & & \\
                        tas-b \cite{tas-b} & 0.289 & 0.232 & 0.336 & 0.399 & 0.468 & 0.771 \\
                        MPNet \cite{mpnet} & 0.290 & 0.250 & 0.327& 0.388 & 0.464& 0.767 \\
       Contriever \cite{contriever} & 0.299& 0.249 & 0.346 & 0.406 & 0.471 & 0.760\\
       % multilingual-minilm (Factiverse) & &  & 0.240 & 0.312 & 0.392 & 0.729 \\
       %        xlm-roberta (Factiverse) & &  & 0.271  & 0.319 & 0.415& 0.720 \\
    COlBERTV2 \cite{santhanam-etal-2022-colbertv2} & 0.252 &  0.230 & 0.325 & 0.419 & 0.465 & 0.789  \\
    Snowflake-arctic-embed-s \cite{merrick2024embeddingclusteringdataimprove} & \textbf{0.367} \up{27.43} & \textbf{0.302} \up{19.37} & \textbf{0.420} \up{21.74} & \textbf{0.480} \up{14.01} & \textbf{0.529} \up{11.37} & \textbf{0.795} \up{2.05} \\
\midrule
\textbf{Re-Ranker} \\
BM25 + & \\
 - ColBERTV2 &0.265 & 0.253 &  0.333 & 0.424 & 0.464 & 0.759 \\
- MARCO-MiniLM-H384 & 0.293 & 0.247     & 0.349 & 0.408 & 0.485 & 0.779\\

% - MSMARCO-electra-base  & 0.282 & 0.233& 0.327 & 0.379 & 0.473  & 0.779 \\
 - MARCO-MiniLM-en-de & 0.278 & 0.264 & 0.342 & 0.426 & 0.479  & 0.779 \\
- bge-reranker-base & 0.222 & 0.205 & 0.301 & 0.398 & 0.452 & 0.779\\
- bge-ranker-v2 & 0.252 & 0.235 & 0.312 & 0.399 & 0.460 & 0.779 \\
- Jina-reranker-v2 & 0.270 & 0.245 & 0.336 & 0.421 & 0.474 & 0.779\\
- gte-multilingual & \underline{0.308} \up{6.04}& \underline{0.277} \up{9.49} & \underline{0.368} \up{6.67} & \underline{0.437} \up{3.80}& \underline{0.496} \up{4.42} & \underline{0.779} \\

% Snowflake-arctic-embed-s &  \\
% -  MARCO-MiniLM-H384 \\
% -  gte-multilingual & 0.316 & 0.290 \\

\midrule
% \textbf{LLM+Retrieval} \\
%         Decompose-retrieve & &&&&& \\
%     \midrule

    % \textbf{Re-Ranking} & & & & & & & \\
    % BM25 + CE & & & & & & & \\
    % \bottomrule
%          \textbf{Semi-Oracle} & & & & & & & & & & \\
%         ClaimOnly & 33.03& 39.57& 36.31 & 58.15 & 33.81& 48.61& 25.70 & 23.99&28.55 & 63.79 & 7.95 & 33.42 &  43.70  \\


% \programfc{} & 38.57& 42.49 & 37.12  & 50.66& 35.22& 45.76& 33.43&32.50 & 32.95 & 55.11 & 25.44 & 37.83 & 43.79 \\
%     \claimdecomp{}&33.43 & 39.78 & 35.04&  55.49 & 33.93 & 48.53& 34.37& 33.50&29.48 & 63.11& 10.85 & 34.48& 44.19  \\
%      \numdecomp{}& 33.81& 39.46 &33.57 & 53.45& 34.18& 47.00& 35.23&34.23 &29.11 & 60.29 & 13.90 &34.43& 43.24\\
    \end{tabular}
    \caption{Retrieval results on \name{}, nDCG@10 across datasets. The best results are in bold and the second best results are underlined with \% improvements indicated by \up{} over the baseline BM25 being specified in brackets. }
     \vspace{-1em}
    \label{tab:main_result}
\end{table*}
\vspace{-0.6em}

\subsection{Benchmark Statistics}
 The benchmark comprises, \textbf{1413} claim-evidence  pair relevance annotations with a total of \textbf{90047} documents in the corpus collection and 100 claims with an average of \textbf{13.89} documents / relevance assesments per query. We provide fine-grained topical analysis in Table \ref{tab:topical}. We observe that a number of claims in the benchmark are \textit{quantitative or temporal} in nature, as shown in Table \ref{tab:taxonomy}. We also observe that many claims are \textit{compositional} comprising multiple aspects, making \name{} a challenging retrieval benchmark. We performed a qualitative meta-analysis of relevance assignments with the help of two researchers who were asked to annotate as ``1" when they deem the relevance assessment to be correct else ``0". The annotators found the relevance assessments to be of high quality (\textbf{88.03\%} was deemed to be correct) with a high agreement of \textbf{0.946} as indicated by Cohen's kappa.
\section{Results}





\subsection{Evaluation of generated images} \label{subsection: Evaluation of generated images}

%The generative capabilities of the model are evaluated with quantitative metrics and with a visual Turing test. 
%To fairly evaluate the generated images directly to the reference CAMUS images, the evaluation experiments in this subsection do not use the sector width augmentations explained in subsection \ref{subsection: Training of the DDPM}. These additional augmentations do not affect the metrics by a lot, but would \newline


The ImageNet Fréchet inception distance (FID) \cite{heusel2017gans} and inception score (IS) \cite{salimans2016improved}
of the diffusion model are 23.87 and 1.47 respectively. However, these metrics can give misleading results for generative models that are not trained on ImageNet \cite{deng2009imagenet, barratt2018note, rosca2017variational}. To qualitatively assess the performance of the model, Fig.~\ref{fig: similar_samples} shows random samples generated together with the most similar cases from the CAMUS dataset identified automatically using the structural similarity index measure (SSIM) \cite{wang2004image}. This shows the model does not simply memorize cases from the training set, and produces realistic and varied samples. \newline




\begin{figure*}[h]
\centering
  \centering
  \includegraphics[trim={0.75cm 0.25cm 0.75cm 0.25cm}, clip,width = 1\linewidth]{figures/similar_samples_v2.drawio.pdf}
  \caption{Generated samples, together with most similar cases in the train and validation set and the test set of the CAMUS dataset, based on SSIM \cite{wang2004image}.}
  \label{fig: similar_samples}
\end{figure*}

\subsection{Survey results}

On the 45 pairs with one real and one synthetic image, participants correctly identified the synthetic image 56.4\% of the time. When broken down by group, cardiologists achieved an accuracy of 63.7\%, while clinical researchers and engineers both identified the correct frame 53.3\% of the time. Fig.~\ref{fig: survey} shows the explanations given when the participants correctly identified the synthetic frame, when they were wrong, and when both frames were real in the 5 cases mentioned above.
\newline

Using a binomial test with a significance level of 5\%, the accuracy of the cardiologists was found to be statistically significantly higher than random guessing ($P=0.09\%$). However, the engineers and clinical researchers in the survey did not show statistically significant higher accuracy compared to random guessing ($P=24.6\%$).

\begin{figure*}[h]
\centering
  \centering
  \includegraphics[trim={0cm 0cm 0cm 0cm}, clip,width = 1\linewidth]{figures/combined_reasons_grouped_barplot.pdf}
  \caption{Explanations given during the survey}
  \label{fig: survey}
\end{figure*}





%In the visual Turing test experiment, two ultrasound engineers and one clinician were shown an image and asked to determine whether it was real or synthetic. The images consisted of 50 frames sampled randomly from the CAMUS dataset and 50 frames sampled from the DDPM trained on the CAMUS dataset. These 100 images were presented to the respondents in a random order. If the respondents identified a frame as synthetic, they had to select a reason from the pre-defined options "Anatomically incorrect," "Speckle patterns," or "Image artifacts." If none of these options matched their reasoning, they could select "Other" and provide an explanation in a text field. Figure \ref{fig: survey} shows the results of the survey.

%\begin{figure}
%     \centering
%     \begin{subfigure}[b]{0.8\linewidth}
%         \centering \includegraphics[trim={0.2cm 0.2cm 0.2cm 0.2cm}, clip, width=1\linewidth]{figures/survey_results.pdf}
%         \caption{Results of the visual Turing test.}
%     \end{subfigure}
%     \begin{subfigure}[b]{0.8\linewidth}
%         \centering \includegraphics[trim={0.2cm 0.2cm 0.2cm 0.2cm}, clip, width=1\linewidth]{figures/survey_reasons.pdf}\caption{Explanations given for selecting synthetic}
%         \label{fig: survey_reasons}
%     \end{subfigure}
%   \caption{Results of the visual Turing test survey. For each image labeled as synthetic, the respondents where asked to indicate a reason for selecting synthetic.}
%    \label{fig: survey}
%\end{figure}

\subsection{Segmentation ablation study results}

Table \ref{table: ablation_study_1} shows the results of the ablation study on the CAMUS dataset, using Dice score and Hausdorff distance as metrics. The bottom part of Fig.~\ref{fig: heatmaps} shows the heatmaps of pixels belonging to the LV after applying the combination of all generative augmentations. Comparing these to the original illustrates that the generative augmentations increase the variety of LV location in the image. \newline

The increase in segmentation accuracy of the HUNT4 model on CAMUS originate mostly from an improvement in segmentation accuracy for samples outside the HUNT4 image distribution. Table \ref{table: camus_subsets_results} lists the segmentation results for the HUNT4 models on different subsets of CAMUS. The subsets are based on depth and sector angle cutoff values visualized in Figs.~\ref{fig: depths_hist} and \ref{fig: sector_angles_hist}.


%Subsection \ref{subsection: results of segmentation ablation study} contains the result of the ablation study. 



\begin{table*}[h]
\scriptsize
  \centering
  \renewcommand{\arraystretch}{1} % Increase vertical spacing
  \caption{Segmentation results of the ablation study using different datasets (HUNT4 and CAMUS) for training and testing. For all experiments, regular augmentations are applied in addition to the generative augmentations (see Table \ref{table: characteristics nnunet}).The Dice score and Hausdorff distance are only for the LV lumen label. We elaborate on this choice in the Discussion. Since the two datasets have been annotated by different experts with different annotation conventions, there is a considerably lower segmentation accuracy when the training and test sets are different. }
  \begin{tabular}{m{60pt}m{40pt}m{120pt}m{60pt}m{100pt}}
    \toprule
      Training set  & Test set & Generative Augmentations & Dice score & Hausdorff distance (mm)\\
    \midrule
    \multirow{7}{1.4cm}{HUNT4} & \multirow{7}{1.4cm}{CAMUS} & None & 0.802 $\pm$ 0.15 & 29.03 $\pm$ 26.01\\
    && Depth increase & 0.887 $\pm$ 0.05 & \textbf{7.49 $\pm$ 3.25} \\
    && Tilt variation  & 0.829 $\pm$ 0.14 & 17.31 $\pm$ 20.98 \\
    && Sector width & 0.847 $\pm$ 0.11 & 21.36 $\pm$ 23.84\\
    && Translation & 0.840 $\pm$ 0.12 & 16.55 $\pm$ 19.71 \\
    && Combination & \textbf{0.887 $\pm$ 0.05} & 8.17 $\pm$ 5.32 \\
    && Combination without repaint & 0.810 $\pm$ 0.15  & 26.90 $\pm$ 25.07\\
    \midrule
    \multirow{7}{1.4cm}{CAMUS} &  \multirow{7}{1.4cm}{CAMUS} & None & 0.943 $\pm$ 0.03 & 4.46 $\pm$ 2.52 \\
    && Depth increase & 0.945 $\pm$ 0.03 & \textbf{4.27 $\pm$ 2.34} \\
    && Tilt variation  &  0.945 $\pm$ 0.03 &  4.30 $\pm$ 2.43 \\
    && Sector width variation & \textbf{0.946 $\pm$ 0.03} & 4.34 $\pm$ 2.41 \\
    && Translation & 0.944 $\pm$ 0.03 & 4.44 $\pm$ 2.43\\
    && Combination & 0.944 $\pm$ 0.03 & 4.37 $\pm$ 2.43 \\
    && Combination without repaint & 0.934 $\pm$ 0.03 & 5.39 $\pm$ 2.85 \\

      \midrule
      \midrule
        \multirow{7}{1.4cm}{HUNT4} & \multirow{7}{1.4cm}{HUNT4} & None &  0.952 $\pm$ 0.02 &  3.34 $\pm$ 1.21 \\
    && Depth increase & 0.954 $\pm$ 0.02 & 3.24 $\pm$ 0.99 \\
    && Tilt variation & 0.954 $\pm$ 0.02 & 3.38 $\pm$ 1.06 \\
    && Sector width variation & 0.953 $\pm$ 0.02  & \textbf{3.23 $\pm$ 1.00} \\
    && Translation & 0.954 $\pm$  0.02  & 3.32 $\pm$  0.97 \\
    && Combination & \textbf{0.954 $\pm$ 0.02} & 3.31 $\pm$ 0.99 \\
    && Combination without repaint &  0.947 $\pm$ 0.02 & 4.14 $\pm$ 1.85 \\
    \midrule
    \multirow{7}{1.4cm}{CAMUS} &\multirow{7}{1.4cm}{HUNT4} & None & 0.886 $\pm$ 0.04 & 6.70 $\pm$ 1.81 \\
    && Depth increase & 0.891 $\pm$ 0.04 & 6.55 $\pm$ 1.84 \\
    && Tilt variation  & 0.887 $\pm$ 0.04 & 6.69 $\pm$ 1.91 \\
    && Sector width variation &  0.892 $\pm$ 0.04 & \textbf{6.54 $\pm$ 1.78} \\
    && Translation &  0.890 $\pm$ 0.04  & 6.55 $\pm$ 1.83   \\
    && Combination & \textbf{0.892 $\pm$ 0.04} & 6.59 $\pm$ 1.82 \\
    && Combination without repaint & 0.875 $\pm$ 0.04 & 7.71 $\pm$ 2.11 \\
    \bottomrule
  \end{tabular}
      \label{table: ablation_study_1}
\end{table*}



\begin{table*}
\scriptsize
  \centering
  \renewcommand{\arraystretch}{1} % Increase vertical spacing
  \caption{Segmentation results on different CAMUS subsets for a segmentation model trained on HUNT4 without generative augmentations and with the combination of all
  generative augmentations. }
  \begin{tabular}{m{100pt}m{120pt}m{100pt}m{100pt}}
    \toprule
      Training dataset   & CAMUS Test subset & Dice score & Hausdorff distance (mm)\\
        \midrule
      \multirow{4}{4cm}{HUNT4 without generative augmentations} & Depth $< 150$ mm ($n=1088$) &   0.855 $\pm$ 0.11  & 14.48 $\pm$ 16.61 \\
          &  Depth $\geq 150$ mm ($n=912$) & 0.729 $\pm$ 0.18 & 45.83 $\pm$ 30.19 \\
      &  Sector angle $< 70^\circ$ ($n=146$)& 0.869 $\pm$ 0.10 & 12.47 $\pm$ 16.47 \\
      &   Sector angle $\geq 70^\circ$ ($n=1854$) & 0.792 $\pm$ 0.16  & 30.06 $\pm$ 28.80 \\
      \midrule
      \multirow{4}{4cm}{HUNT4 with generative augmentations} & Depth $< 150$ mm ($n=1088$) &  \textbf{0.893 $\pm$ 0.05} & \textbf{7.45 $\pm$ 3.80}   \\
      &  Depth $\geq 150$ mm ($n=912$) & \textbf{0.886 $\pm$ 0.07} & \textbf{9.34 $\pm$ 8.37} \\
      &  Sector angle $< 70^\circ$ ($n=146$)& \textbf{0.893 $\pm$ 0.05}  & \textbf{7.11 $\pm$ 3.10} \\
      &   Sector angle $\geq 70^\circ$ ($n=1854$) &  \textbf{0.890 $\pm$ 0.07} & \textbf{8.40 $\pm$ 6.56} \\
    \bottomrule
  \end{tabular}
      \label{table: camus_subsets_results}
\end{table*}


% describe train on hunt4 and camus with augmentations. Also describe baseline of 'black' augmentations











\subsection{Clinical evaluation on HUNT4 results}


Similar to the segmentation results, the performance gains of the HUNT4 model originate mostly from an improvement in segmentation accuracy for frames outside the normal range. Fig.~\ref{fig: ef_main_text} shows the Bland-Altman plots comparing the manual reference EF with the automatic EF for segmentation models trained with and without generative augmentations for data both inside and outside of the HUNT4 acquisition normal range of depth $> 150$mm and sector angle $> 70^\circ$. Appendix \ref{appendix: exensive EF evaluation} contains additional analysis of automatic EF and also evaluates automatic on CAMUS.



%Figure \ref{fig: ef} shows the Bland-Altman plots comparing the automatic EF measurements with the manual reference for segmentation models trained without generative augmentations and with the combination of all generative augmentations. 









\begin{table*}[t]
\centering
\fontsize{11pt}{11pt}\selectfont
\begin{tabular}{lllllllllllll}
\toprule
\multicolumn{1}{c}{\textbf{task}} & \multicolumn{2}{c}{\textbf{Mir}} & \multicolumn{2}{c}{\textbf{Lai}} & \multicolumn{2}{c}{\textbf{Ziegen.}} & \multicolumn{2}{c}{\textbf{Cao}} & \multicolumn{2}{c}{\textbf{Alva-Man.}} & \multicolumn{1}{c}{\textbf{avg.}} & \textbf{\begin{tabular}[c]{@{}l@{}}avg.\\ rank\end{tabular}} \\
\multicolumn{1}{c}{\textbf{metrics}} & \multicolumn{1}{c}{\textbf{cor.}} & \multicolumn{1}{c}{\textbf{p-v.}} & \multicolumn{1}{c}{\textbf{cor.}} & \multicolumn{1}{c}{\textbf{p-v.}} & \multicolumn{1}{c}{\textbf{cor.}} & \multicolumn{1}{c}{\textbf{p-v.}} & \multicolumn{1}{c}{\textbf{cor.}} & \multicolumn{1}{c}{\textbf{p-v.}} & \multicolumn{1}{c}{\textbf{cor.}} & \multicolumn{1}{c}{\textbf{p-v.}} &  &  \\ \midrule
\textbf{S-Bleu} & 0.50 & 0.0 & 0.47 & 0.0 & 0.59 & 0.0 & 0.58 & 0.0 & 0.68 & 0.0 & 0.57 & 5.8 \\
\textbf{R-Bleu} & -- & -- & 0.27 & 0.0 & 0.30 & 0.0 & -- & -- & -- & -- & - &  \\
\textbf{S-Meteor} & 0.49 & 0.0 & 0.48 & 0.0 & 0.61 & 0.0 & 0.57 & 0.0 & 0.64 & 0.0 & 0.56 & 6.1 \\
\textbf{R-Meteor} & -- & -- & 0.34 & 0.0 & 0.26 & 0.0 & -- & -- & -- & -- & - &  \\
\textbf{S-Bertscore} & \textbf{0.53} & 0.0 & {\ul 0.80} & 0.0 & \textbf{0.70} & 0.0 & {\ul 0.66} & 0.0 & {\ul0.78} & 0.0 & \textbf{0.69} & \textbf{1.7} \\
\textbf{R-Bertscore} & -- & -- & 0.51 & 0.0 & 0.38 & 0.0 & -- & -- & -- & -- & - &  \\
\textbf{S-Bleurt} & {\ul 0.52} & 0.0 & {\ul 0.80} & 0.0 & 0.60 & 0.0 & \textbf{0.70} & 0.0 & \textbf{0.80} & 0.0 & {\ul 0.68} & {\ul 2.3} \\
\textbf{R-Bleurt} & -- & -- & 0.59 & 0.0 & -0.05 & 0.13 & -- & -- & -- & -- & - &  \\
\textbf{S-Cosine} & 0.51 & 0.0 & 0.69 & 0.0 & {\ul 0.62} & 0.0 & 0.61 & 0.0 & 0.65 & 0.0 & 0.62 & 4.4 \\
\textbf{R-Cosine} & -- & -- & 0.40 & 0.0 & 0.29 & 0.0 & -- & -- & -- & -- & - & \\ \midrule
\textbf{QuestEval} & 0.23 & 0.0 & 0.25 & 0.0 & 0.49 & 0.0 & 0.47 & 0.0 & 0.62 & 0.0 & 0.41 & 9.0 \\
\textbf{LLaMa3} & 0.36 & 0.0 & \textbf{0.84} & 0.0 & {\ul{0.62}} & 0.0 & 0.61 & 0.0 &  0.76 & 0.0 & 0.64 & 3.6 \\
\textbf{our (3b)} & 0.49 & 0.0 & 0.73 & 0.0 & 0.54 & 0.0 & 0.53 & 0.0 & 0.7 & 0.0 & 0.60 & 5.8 \\
\textbf{our (8b)} & 0.48 & 0.0 & 0.73 & 0.0 & 0.52 & 0.0 & 0.53 & 0.0 & 0.7 & 0.0 & 0.59 & 6.3 \\  \bottomrule
\end{tabular}
\caption{Pearson correlation on human evaluation on system output. `R-': reference-based. `S-': source-based.}
\label{tab:sys}
\end{table*}



\begin{table}%[]
\centering
\fontsize{11pt}{11pt}\selectfont
\begin{tabular}{llllll}
\toprule
\multicolumn{1}{c}{\textbf{task}} & \multicolumn{1}{c}{\textbf{Lai}} & \multicolumn{1}{c}{\textbf{Zei.}} & \multicolumn{1}{c}{\textbf{Scia.}} & \textbf{} & \textbf{} \\ 
\multicolumn{1}{c}{\textbf{metrics}} & \multicolumn{1}{c}{\textbf{cor.}} & \multicolumn{1}{c}{\textbf{cor.}} & \multicolumn{1}{c}{\textbf{cor.}} & \textbf{avg.} & \textbf{\begin{tabular}[c]{@{}l@{}}avg.\\ rank\end{tabular}} \\ \midrule
\textbf{S-Bleu} & 0.40 & 0.40 & 0.19* & 0.33 & 7.67 \\
\textbf{S-Meteor} & 0.41 & 0.42 & 0.16* & 0.33 & 7.33 \\
\textbf{S-BertS.} & {\ul0.58} & 0.47 & 0.31 & 0.45 & 3.67 \\
\textbf{S-Bleurt} & 0.45 & {\ul 0.54} & {\ul 0.37} & 0.45 & {\ul 3.33} \\
\textbf{S-Cosine} & 0.56 & 0.52 & 0.3 & {\ul 0.46} & {\ul 3.33} \\ \midrule
\textbf{QuestE.} & 0.27 & 0.35 & 0.06* & 0.23 & 9.00 \\
\textbf{LlaMA3} & \textbf{0.6} & \textbf{0.67} & \textbf{0.51} & \textbf{0.59} & \textbf{1.0} \\
\textbf{Our (3b)} & 0.51 & 0.49 & 0.23* & 0.39 & 4.83 \\
\textbf{Our (8b)} & 0.52 & 0.49 & 0.22* & 0.43 & 4.83 \\ \bottomrule
\end{tabular}
\caption{Pearson correlation on human ratings on reference output. *not significant; we cannot reject the null hypothesis of zero correlation}
\label{tab:ref}
\end{table}


\begin{table*}%[]
\centering
\fontsize{11pt}{11pt}\selectfont
\begin{tabular}{lllllllll}
\toprule
\textbf{task} & \multicolumn{1}{c}{\textbf{ALL}} & \multicolumn{1}{c}{\textbf{sentiment}} & \multicolumn{1}{c}{\textbf{detoxify}} & \multicolumn{1}{c}{\textbf{catchy}} & \multicolumn{1}{c}{\textbf{polite}} & \multicolumn{1}{c}{\textbf{persuasive}} & \multicolumn{1}{c}{\textbf{formal}} & \textbf{\begin{tabular}[c]{@{}l@{}}avg. \\ rank\end{tabular}} \\
\textbf{metrics} & \multicolumn{1}{c}{\textbf{cor.}} & \multicolumn{1}{c}{\textbf{cor.}} & \multicolumn{1}{c}{\textbf{cor.}} & \multicolumn{1}{c}{\textbf{cor.}} & \multicolumn{1}{c}{\textbf{cor.}} & \multicolumn{1}{c}{\textbf{cor.}} & \multicolumn{1}{c}{\textbf{cor.}} &  \\ \midrule
\textbf{S-Bleu} & -0.17 & -0.82 & -0.45 & -0.12* & -0.1* & -0.05 & -0.21 & 8.42 \\
\textbf{R-Bleu} & - & -0.5 & -0.45 &  &  &  &  &  \\
\textbf{S-Meteor} & -0.07* & -0.55 & -0.4 & -0.01* & 0.1* & -0.16 & -0.04* & 7.67 \\
\textbf{R-Meteor} & - & -0.17* & -0.39 & - & - & - & - & - \\
\textbf{S-BertScore} & 0.11 & -0.38 & -0.07* & -0.17* & 0.28 & 0.12 & 0.25 & 6.0 \\
\textbf{R-BertScore} & - & -0.02* & -0.21* & - & - & - & - & - \\
\textbf{S-Bleurt} & 0.29 & 0.05* & 0.45 & 0.06* & 0.29 & 0.23 & 0.46 & 4.2 \\
\textbf{R-Bleurt} & - &  0.21 & 0.38 & - & - & - & - & - \\
\textbf{S-Cosine} & 0.01* & -0.5 & -0.13* & -0.19* & 0.05* & -0.05* & 0.15* & 7.42 \\
\textbf{R-Cosine} & - & -0.11* & -0.16* & - & - & - & - & - \\ \midrule
\textbf{QuestEval} & 0.21 & {\ul{0.29}} & 0.23 & 0.37 & 0.19* & 0.35 & 0.14* & 4.67 \\
\textbf{LlaMA3} & \textbf{0.82} & \textbf{0.80} & \textbf{0.72} & \textbf{0.84} & \textbf{0.84} & \textbf{0.90} & \textbf{0.88} & \textbf{1.00} \\
\textbf{Our (3b)} & 0.47 & -0.11* & 0.37 & 0.61 & 0.53 & 0.54 & 0.66 & 3.5 \\
\textbf{Our (8b)} & {\ul{0.57}} & 0.09* & {\ul 0.49} & {\ul 0.72} & {\ul 0.64} & {\ul 0.62} & {\ul 0.67} & {\ul 2.17} \\ \bottomrule
\end{tabular}
\caption{Pearson correlation on human ratings on our constructed test set. 'R-': reference-based. 'S-': source-based. *not significant; we cannot reject the null hypothesis of zero correlation}
\label{tab:con}
\end{table*}

\section{Results}
We benchmark the different metrics on the different datasets using correlation to human judgement. For content preservation, we show results split on data with system output, reference output and our constructed test set: we show that the data source for evaluation leads to different conclusions on the metrics. In addition, we examine whether the metrics can rank style transfer systems similar to humans. On style strength, we likewise show correlations between human judgment and zero-shot evaluation approaches. When applicable, we summarize results by reporting the average correlation. And the average ranking of the metric per dataset (by ranking which metric obtains the highest correlation to human judgement per dataset). 

\subsection{Content preservation}
\paragraph{How do data sources affect the conclusion on best metric?}
The conclusions about the metrics' performance change radically depending on whether we use system output data, reference output, or our constructed test set. Ideally, a good metric correlates highly with humans on any data source. Ideally, for meta-evaluation, a metric should correlate consistently across all data sources, but the following shows that the correlations indicate different things, and the conclusion on the best metric should be drawn carefully.

Looking at the metrics correlations with humans on the data source with system output (Table~\ref{tab:sys}), we see a relatively high correlation for many of the metrics on many tasks. The overall best metrics are S-BertScore and S-BLEURT (avg+avg rank). We see no notable difference in our method of using the 3B or 8B model as the backbone.

Examining the average correlations based on data with reference output (Table~\ref{tab:ref}), now the zero-shoot prompting with LlaMA3 70B is the best-performing approach ($0.59$ avg). Tied for second place are source-based cosine embedding ($0.46$ avg), BLEURT ($0.45$ avg) and BertScore ($0.45$ avg). Our method follows on a 5. place: here, the 8b version (($0.43$ avg)) shows a bit stronger results than 3b ($0.39$ avg). The fact that the conclusions change, whether looking at reference or system output, confirms the observations made by \citet{scialom-etal-2021-questeval} on simplicity transfer.   

Now consider the results on our test set (Table~\ref{tab:con}): Several metrics show low or no correlation; we even see a significantly negative correlation for some metrics on ALL (BLEU) and for specific subparts of our test set for BLEU, Meteor, BertScore, Cosine. On the other end, LlaMA3 70B is again performing best, showing strong results ($0.82$ in ALL). The runner-up is now our 8B method, with a gap to the 3B version ($0.57$ vs $0.47$ in ALL). Note our method still shows zero correlation for the sentiment task. After, ranks BLEURT ($0.29$), QuestEval ($0.21$), BertScore ($0.11$), Cosine ($0.01$).  

On our test set, we find that some metrics that correlate relatively well on the other datasets, now exhibit low correlation. Hence, with our test set, we can now support the logical reasoning with data evidence: Evaluation of content preservation for style transfer needs to take the style shift into account. This conclusion could not be drawn using the existing data sources: We hypothesise that for the data with system-based output, successful output happens to be very similar to the source sentence and vice versa, and reference-based output might not contain server mistakes as they are gold references. Thus, none of the existing data sources tests the limits of the metrics.  


\paragraph{How do reference-based metrics compare to source-based ones?} Reference-based metrics show a lower correlation than the source-based counterpart for all metrics on both datasets with ratings on references (Table~\ref{tab:sys}). As discussed previously, reference-based metrics for style transfer have the drawback that many different good solutions on a rewrite might exist and not only one similar to a reference.


\paragraph{How well can the metrics rank the performance of style transfer methods?}
We compare the metrics' ability to judge the best style transfer methods w.r.t. the human annotations: Several of the data sources contain samples from different style transfer systems. In order to use metrics to assess the quality of the style transfer system, metrics should correctly find the best-performing system. Hence, we evaluate whether the metrics for content preservation provide the same system ranking as human evaluators. We take the mean of the score for every output on each system and the mean of the human annotations; we compare the systems using the Kendall's Tau correlation. 

We find only the evaluation using the dataset Mir, Lai, and Ziegen to result in significant correlations, probably because of sparsity in a number of system tests (App.~\ref{app:dataset}). Our method (8b) is the only metric providing a perfect ranking of the style transfer system on the Lai data, and Llama3 70B the only one on the Ziegen data. Results in App.~\ref{app:results}. 


\subsection{Style strength results}
%Evaluating style strengths is a challenging task. 
Llama3 70B shows better overall results than our method. However, our method scores higher than Llama3 70B on 2 out of 6 datasets, but it also exhibits zero correlation on one task (Table~\ref{tab:styleresults}).%More work i s needed on evaluating style strengths. 
 
\begin{table}%[]
\fontsize{11pt}{11pt}\selectfont
\begin{tabular}{lccc}
\toprule
\multicolumn{1}{c}{\textbf{}} & \textbf{LlaMA3} & \textbf{Our (3b)} & \textbf{Our (8b)} \\ \midrule
\textbf{Mir} & 0.46 & 0.54 & \textbf{0.57} \\
\textbf{Lai} & \textbf{0.57} & 0.18 & 0.19 \\
\textbf{Ziegen.} & 0.25 & 0.27 & \textbf{0.32} \\
\textbf{Alva-M.} & \textbf{0.59} & 0.03* & 0.02* \\
\textbf{Scialom} & \textbf{0.62} & 0.45 & 0.44 \\
\textbf{\begin{tabular}[c]{@{}l@{}}Our Test\end{tabular}} & \textbf{0.63} & 0.46 & 0.48 \\ \bottomrule
\end{tabular}
\caption{Style strength: Pearson correlation to human ratings. *not significant; we cannot reject the null hypothesis of zero corelation}
\label{tab:styleresults}
\end{table}

\subsection{Ablation}
We conduct several runs of the methods using LLMs with variations in instructions/prompts (App.~\ref{app:method}). We observe that the lower the correlation on a task, the higher the variation between the different runs. For our method, we only observe low variance between the runs.
None of the variations leads to different conclusions of the meta-evaluation. Results in App.~\ref{app:results}.
% \begin{acks}
% To Robert, for the bagels and explaining CMYK and color spaces.
% \end{acks}

%%
%% The next two lines define the bibliography style to be used, and
%% the bibliography file.
\bibliographystyle{ACM-Reference-Format}
\bibliography{references}


%%
%% If your work has an appendix, this is the place to put it.

\end{document}
\endinput
%%
%% End of file `sample-sigconf.tex'.
