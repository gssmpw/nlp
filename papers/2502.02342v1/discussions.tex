\section{Discussion}\label{sec:discussion}
\noindent\textbf{Attack Detection Effectiveness.} The effectiveness of our reinforcement and decay mechanism is demonstrated in both attack detection and false positive reduction. 
For instance, in the THEIA dataset, a process called \textit{profile} performs C2 activity on day one and then goes dormant. 
On day three, it downloads a malicious payload into \textit{/var/log/mail} and forks a process called \textit{mail} which performs further malicious activities. 
The LLM initially identifies the exploitation from \textit{firefox} to \textit{profile} on day one, marking it as a partial attack with a probability score of 0.85 and placing it in a set. 
When \texttt{profile}'s activity on day three is detected, it is also marked as a partial attack with a score of 0.85 and added to another set. 
The temporal correlation engine maps these activities to the same set, and upon merging these sets, it reassigns the new set to the LLM for analysis. The LLM reinforces the confidence score to 0.91, indicating that it identified the complete attack.

\noindent On the CADETS dataset, \method demonstrates its decay mechanism's effectiveness in handling false positives. 
When analyzing activities for process chain containing \textit{imapd}, \textit{wget}, and \textit{links}, the LLM initially assigns confidence score around 0.85 due to potentially suspicious activities. 
However, as these processes exhibit benign operations in subsequent analysis windows, the temporal correlation engine periodically forwards these observations to the LLM. 
Observing no suspicious progression the LLM gradually decreases the confidence score to 0.75. 
Due to the reduction in score, the engine transfers these processes to a secondary monitoring queue, temporarily suspending alert generation. 
This decay mechanism effectively addresses alert fatigue in security operations by ensuring that analysts focus on high-confidence threats. \\

\noindent\textbf{Zero-Day Attack Detection Capabilities.} Our evaluation on four datasets, including real-world and benchmark scenarios, demonstrated \method's effectiveness in detecting previously unknown attacks. 
Since attackers must progress through certain essential stages regardless of their specific techniques, \method can identify suspicious patterns at the kernel level even when confronted with novel attack methods.
\method's success in zero-day detection relies on two key factors: the invariant nature of kernel-level attack traces and the LLM's comprehensive understanding of system behavior patterns. 
While attack implementations may vary, the underlying progression through attack stages creates detectable patterns that \method can recognize without prior exposure to specific variants. 
Through these capabilities, \method demonstrates a significant advancement in automated attack detection and investigation.
The framework's ability to correlate events across extended time periods, manage false positives, and detect novel attack patterns makes it as a valuable tool for modern security operations. \\

\noindent\textbf{Limitations.}
\method employs a 15-minute sliding window for attack detection, introducing a maximum detection delay of 15 minutes in edge cases.
However, this delay remains well within acceptable operational parameters, as it significantly outperforms typical SOC mean time to resolve metrics in real-world deployment.\footnote{\url{https://www.ibm.com/reports/
data-breach}}
In addition, %From an architectural perspective, 
the multi-stage pipeline, while enabling comprehensive threat detection, presents sizeable computational demands. 
The deployment of the deviation and graph analyzer modules on endpoint systems contributes to resource utilization on those devices. 
Additionally, the local deployment of our chosen LLM requires substantial GPU resources (32GB VRAM), which may limit deployment options in resource-constrained environments.

\section{Conclusions and Future Work}

In this paper, we presented \method, an innovative framework that combines anomaly detection and graph analysis with LLM reasoning for attack detection, investigation, and contextual analysis.
Future work can explore integrating automated incident response capabilities using contextual insights and addressing the challenges of parsing high-level logs, as demonstrated in our Blind Eagle proof-of-concept. 
While our approach effectively analyzes high-level logs, its success depends on complete log capture, highlighting the need to overcome challenges like logging misconfigurations or incomplete capture.