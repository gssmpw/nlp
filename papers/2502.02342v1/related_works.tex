\section{\label{sec:related}Related Work}

Provenance-based approaches for APT detection have been developed to address increasingly sophisticated cybersecurity challenges. 
Rule-based techniques~\cite{milajerdi2019holmes,yu2019needle} utilize predefined security policies and heuristic rules to identify attack patterns. 
While these methods offer detailed, fine-grained event-level detection~\cite{inam2023sok} and typically maintain low false positive rates, they require extensive manual efforts and are less effective in detecting zero-day exploits~\cite{hossain2020combating}. \\
\indent Anomaly-based detection methods, including statistical techniques~\cite{hossain2017sleuth,hassan2019nodoze,wang2020you,kurniawan2022krystal,dong2023distdet,Li_2024}, path-based approaches~\cite{du2017deeplog,zhang2019robust,guo2021logbert,alsaheel2021atlas}, and graph-based methods~\cite{han2020unicorn,wang2022threatrace,zengy2022shadewatcher,jia2023magic,yang2023prographer}, have also demonstrated promising results in detecting APTs. 
However, these methods often struggle to adapt to changes in system behavior over time, which can lead to high false positive rates overwhelming security analysts. 
In addition, the alerts generated by these techniques often lack interpretability, making it challenging for security analysts to understand and investigate the anomalies detected. \\
\indent Moreover, some of the recent studies~\cite{alsaheel2021atlas,wang2022threatrace,zengy2022shadewatcher,jia2023magic,han2020unicorn,Li_2024} have two methodological limitations. 
First, these methods heavily depend on large training datasets while utilizing disproportionately small testing windows, a practice that conflicts with the dynamic and ever-evolving nature of real-world data environments. 
Secondly, some studies have identified instances of data leakage~\cite{wang2022threatrace,jia2023magic}, where models were unintentionally trained on future data, potentially leading to inflated performance metrics. 
These limitations highlight the need for additional research in order to develop new approaches to enhance the effectiveness and robustness of these approaches, thereby promoting their wider adoption and practical implementation% in industrial contexts
~\cite{van2019sok,dong2023we}. \\
\indent Other recent research focused on using transformers and LLMs for anomaly detection in logs. 
For instance, LAnoBERT~\cite{lee} employs regular expressions for minimal log preprocessing, enhancing flexibility across different formats; LogGPT~\cite{han2023loggpt} uses a GPT-2 model with sequential prediction and a top-K reward mechanism to improve detection capabilities; and LogPrompt~\cite{liu2024} leverages advanced prompt strategies to boost LLM performance in log analysis. 
While these approaches enable advanced log analysis, research on their application on system-level provenance entities and use in identifying APT attacks remains limited. \\
\indent The main limitations identified in the proposed methods and prior studies can be summarized as follows: 
(1) rule-based methods require manual efforts and are unable to detect zero-day attacks;
(2) anomaly-based methods fail to adapt to evolving system behavior over time, resulting in high false positive rates; 
(3) alerts generated by anomaly-based methods lack interpretability, making it difficult for security analysts to investigate detected anomalies;
(4) in many cases, the evaluation performed used training data that do not reflect real-world scenarios, leading to inflated (incorrect) performance results; 
(5) existing methods face challenges with scalability and have high computational demands, limiting their practical deployment; and
(6) insufficient application of LLMs for the tasks of APT detection and investigation, particularly with the use of provenance data.
In light of these gaps and limitations, we propose \method, a novel framework designed to address the challenges of APT detection and investigation in provenance-based systems.



