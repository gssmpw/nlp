\begin{table*}[t]
\centering
\caption{Further details on the case study concerning the responses of LightRAG to the same question.}
\label{tab:case study}
\small
\begin{tabularx}{\textwidth}{X X}

\toprule
\textbf{Query:} Explain the purpose and functionality of 'graders' in the context of reinforcement fine-tuning.\\

\midrule
\textbf{Input Videos:} \href{https://www.youtube.com/playlist?list=PLOXw6I10VTv9lin5AzsHAHCTrC7BdVdEM}{12 Days of OpenAI} (A total of 12 videos) \\

\midrule

\textbf{LightRAG Response:} 

\texttt{\#\#\# Purpose and Functionality of Graders in Reinforcement Fine-Tuning}

\texttt{Graders serve a crucial role within the context of reinforcement fine-tuning, particularly focusing on evaluating and improving the performance of machine learning models. These evaluative tools assist in determining how well a model aligns with desired outcomes based on specific criteria.}

\\

\texttt{\#\#\#\# Evaluation Process}

\texttt{In reinforcement fine-tuning, graders provide a quantitative measure for assessing the effectiveness of a model's outputs. They help in evaluating the responses generated by the AI against predefined correct answers or standards. This comparison is essential as it offers insights into areas where the model performs well and where it may require further improvements.}

\texttt{The process involves analyzing the outputs of the model after it has undergone reinforcement learning techniques. By examining the consistency and accuracy of the generated responses, graders facilitate the identification of discrepancies and help in recalibrating the model to optimize its performance further. } 

\\

\texttt{\#\#\#\# Role in Feedback Loop}

\texttt{Grading is also integral to the feedback mechanism within reinforcement fine-tuning. The insights gained from the grading process inform the adjustments necessary for the model's reinforcement learning parameters. In this way, graders contribute to refining the model’s learning algorithms, effectively guiding the subsequent training iterations.}

\texttt{Overall, the implementation of graders in reinforcement fine-tuning exemplifies an iterative improvement strategy, where ongoing assessment and feedback drive enhancements in model performance. This systematic approach not only improves the quality of outputs but also ensures that the model continues evolving to meet dynamic user and operational requirements.} 

\\

\bottomrule

\end{tabularx}
\vspace{-0.2in}
\end{table*}