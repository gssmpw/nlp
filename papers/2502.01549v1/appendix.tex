\appendix
\clearpage
\section{Details of LongerVideos}\label{apd:data}
LongerVideos is a comprehensive benchmark dataset designed to evaluate a model's ability to comprehend and extract knowledge from long-form videos. By leveraging semantic connections across multiple sources, the dataset facilitates the development of efficient, knowledge-based question-answering. The core methodology presents the model with diverse video lists of varying lengths, then assesses the model's output in terms of completeness, accuracy, and diversity. This holistic approach ensures the evaluated models demonstrate a robust understanding of the content, the ability to synthesize information, and the aptitude to generate well-rounded responses.
\vspace{-0.05in}
\begin{itemize}[leftmargin=*]
    \item \textbf{Input Data}: A diverse collection of long videos, with durations ranging from minutes to hours.
    \item \textbf{Question}: A series of open-ended questions carefully tailored to the provided video list.
    \item \textbf{Expected Output}: Individual responses generated based on the information extracted from videos.
\end{itemize}
\vspace{-0.05in}
The LongerVideos dataset was constructed by systematically curating diverse video lists from YouTube, leveraging the platform's structure where creators often compile thematic content. A major data source comprised online course videos, typically segmented into multiple recordings corresponding to distinct course chapters. For each video, the team employed the yt-dlp tool to download the content in 720P resolution, after which they prepared open-ended questions for each list with the assistance of NotebookLM, a robust multi-video understanding model from Google that can process various videos as input to generate relevant answers. The final LongerVideos dataset consists of 22 carefully curated video lists, with detailed statistics provided in Table~\ref{tab:detail stats}.
\begin{table}[h]
\vspace{-0.05in}
\centering
\caption{Detailed statistics of the \textit{LongerVideos} dataset.}
\label{tab:detail stats}
\resizebox{1.0\textwidth}{!}{
\begin{tabular}{@{}l|lcccc@{}}
\toprule
Video Type & video list name & \#video & \#query & \#overall duration \\
\midrule
\multirow{12}{*}{\textbf{Lecture}}
& \texttt{climate-week-at-columbia-engineering} & 4 & 26 & 5.91 hours \\
& \texttt{rag-lecture} & 19 & 38 & 5.34 hours \\
& \texttt{ai-agent-lecture} & 39 & 45 & 9.35 hours \\
& \texttt{daubechies-wavelet-lecture} & 4 & 25 & 8.97 hours \\
& \texttt{daubechies-art-and-mathematics-lecture} & 4 & 21 & 4.87 hours \\
& \texttt{tech-ceo-lecture} & 4 & 31 & 4.83 hours \\
& \texttt{dspy-lecture} & 9 & 38 & 4.22 hours \\
& \texttt{trading-for-beginners} & 2 & 23 & 4.11 hours \\
& \texttt{ahp-superdecision} & 11 & 24 & 2.40 hours \\
& \texttt{decision-making-science} & 4 & 26 & 2.20 hours \\
& \texttt{12-days-of-openai} & 12 & 35 & 3.43 hours \\
& \texttt{autogen} & 23 & 44 & 8.70 hours \\
\midrule
\multirow{5}{*}{\textbf{Documentary}}
& \texttt{fights-in-animal-kingdom} & 1 & 11 & 3.00 hours \\
& \texttt{nature-scenes} & 1 & 17 & 3.98 hours \\
& \texttt{education-united-nations} & 6 & 39 & 8.41 hours \\
& \texttt{elon-musk} & 1 & 13 & 8.63 hours \\
& \texttt{jeff-bezos} & 3 & 34 & 4.47 hours \\
\midrule
\multirow{5}{*}{\textbf{Entertainment}}
& \texttt{black-myth-wukong} & 10 & 23 & 21.36 hours \\
& \texttt{primetime-emmy-awards} & 3 & 17 & 7.31 hours \\
& \texttt{journey-through-china} & 1 & 27 & 3.37 hours \\
& \texttt{fia-awards} & 1 & 27 & 3.02 hours \\
& \texttt{game-awards} & 2 & 18 & 6.73 hours \\
\bottomrule
\end{tabular}
\vspace{-0.05in}
}
\end{table}

\section{Details of Case Study}\label{apd:case study}
This section provides further details on the case study presented in Section 2, which investigates the purpose and functionality of 'graders' in the context of reinforcement fine-tuning. The investigation utilizes input from the "12 Days of OpenAI" video series, comprising 12 videos that showcase OpenAI's activities in late 2024. To effectively answer the question, the model retrieves relevant content that specifically discusses the role of graders within the reinforcement fine-tuning context. To further illustrate our model's capabilities in retrieving detailed information from videos for generating nuanced answers, we also present a response from another retrieval-augmented generation model, LightRAG, for comprehensive analysis. A comparison of the generated answer by our model, as shown in Table 1, reveals that it provides greater detail, including the scoring criteria used by graders and specific examples illustrating the partial scores assigned. Moreover, it delineates the processes through which graders operate in reinforcement learning, thereby enhancing the overall quality of the generated response. Consequently, the win-rate comparison with GPT-4o-mini emphasizes the empowerment and trustworthiness of the response generated by our \model.
\vspace{-0.05in}
\begin{itemize}[leftmargin=*]

    \item \textbf{Empowerment}: Our \model\ empowers the reader by explaining how graders work in a practical sense, including details on scoring, input comparison, and their role in the feedback loop. This allows readers to grasp how to utilize graders effectively in machine learning contexts.
    
    \item \textbf{Trustworthiness}: Our \model\ aligns closely with established concepts in machine learning and provides more operational details, enhancing credibility. It clearly explains the scoring method and implications for model performance, reinforcing its trustworthy nature.
\end{itemize}
\vspace{-0.1in}

\begin{table*}[t]
\centering
\caption{Further details on the case study concerning the responses of LightRAG to the same question.}
\label{tab:case study}
\small
\begin{tabularx}{\textwidth}{X X}

\toprule
\textbf{Query:} Explain the purpose and functionality of 'graders' in the context of reinforcement fine-tuning.\\

\midrule
\textbf{Input Videos:} \href{https://www.youtube.com/playlist?list=PLOXw6I10VTv9lin5AzsHAHCTrC7BdVdEM}{12 Days of OpenAI} (A total of 12 videos) \\

\midrule

\textbf{LightRAG Response:} 

\texttt{\#\#\# Purpose and Functionality of Graders in Reinforcement Fine-Tuning}

\texttt{Graders serve a crucial role within the context of reinforcement fine-tuning, particularly focusing on evaluating and improving the performance of machine learning models. These evaluative tools assist in determining how well a model aligns with desired outcomes based on specific criteria.}

\\

\texttt{\#\#\#\# Evaluation Process}

\texttt{In reinforcement fine-tuning, graders provide a quantitative measure for assessing the effectiveness of a model's outputs. They help in evaluating the responses generated by the AI against predefined correct answers or standards. This comparison is essential as it offers insights into areas where the model performs well and where it may require further improvements.}

\texttt{The process involves analyzing the outputs of the model after it has undergone reinforcement learning techniques. By examining the consistency and accuracy of the generated responses, graders facilitate the identification of discrepancies and help in recalibrating the model to optimize its performance further. } 

\\

\texttt{\#\#\#\# Role in Feedback Loop}

\texttt{Grading is also integral to the feedback mechanism within reinforcement fine-tuning. The insights gained from the grading process inform the adjustments necessary for the model's reinforcement learning parameters. In this way, graders contribute to refining the model’s learning algorithms, effectively guiding the subsequent training iterations.}

\texttt{Overall, the implementation of graders in reinforcement fine-tuning exemplifies an iterative improvement strategy, where ongoing assessment and feedback drive enhancements in model performance. This systematic approach not only improves the quality of outputs but also ensures that the model continues evolving to meet dynamic user and operational requirements.} 

\\

\bottomrule

\end{tabularx}
\vspace{-0.2in}
\end{table*}

\section{Instructions for Win-Rate \& Quantitative Performance Comparison}\label{apd:prompt4evaluation}


\definecolor{titlecolor}{rgb}{0.9, 0.5, 0.1}
\definecolor{anscolor}{rgb}{0.2, 0.5, 0.8}
\definecolor{labelcolor}{HTML}{48a07e}
\begin{table*}[h]
	\centering
	
 % \vspace{-0.2cm}
	
	\begin{center}
		\begin{tikzpicture}[
				chatbox_inner/.style={rectangle, rounded corners, opacity=0, text opacity=1, font=\sffamily\scriptsize, text width=5in, text height=9pt, inner xsep=6pt, inner ysep=6pt},
				chatbox_prompt_inner/.style={chatbox_inner, align=flush left, xshift=0pt, text height=11pt},
				chatbox_user_inner/.style={chatbox_inner, align=flush left, xshift=0pt},
				chatbox_gpt_inner/.style={chatbox_inner, align=flush left, xshift=0pt},
				chatbox/.style={chatbox_inner, draw=black!25, fill=gray!7, opacity=1, text opacity=0},
				chatbox_prompt/.style={chatbox, align=flush left, fill=gray!1.5, draw=black!30, text height=10pt},
				chatbox_user/.style={chatbox, align=flush left},
				chatbox_gpt/.style={chatbox, align=flush left},
				chatbox2/.style={chatbox_gpt, fill=green!25},
				chatbox3/.style={chatbox_gpt, fill=red!20, draw=black!20},
				chatbox4/.style={chatbox_gpt, fill=yellow!30},
				labelbox/.style={rectangle, rounded corners, draw=black!50, font=\sffamily\scriptsize\bfseries, fill=gray!5, inner sep=3pt},
			]
											
			\node[chatbox_user] (q1) {
				\textbf{System prompt}
				\newline
				\newline
				You are a helpful and precise assistant for segmenting and labeling sentences. We would like to request your help on curating a dataset for entity-level hallucination detection.
				\newline \newline
                We will give you a machine generated biography and a list of checked facts about the biography. Each fact consists of a sentence and a label (True/False). Please do the following process. First, breaking down the biography into words. Second, by referring to the provided list of facts, merging some broken down words in the previous step to form meaningful entities. For example, ``strategic thinking'' should be one entity instead of two. Third, according to the labels in the list of facts, labeling each entity as True or False. Specifically, for facts that share a similar sentence structure (\eg, \textit{``He was born on Mach 9, 1941.''} (\texttt{True}) and \textit{``He was born in Ramos Mejia.''} (\texttt{False})), please first assign labels to entities that differ across atomic facts. For example, first labeling ``Mach 9, 1941'' (\texttt{True}) and ``Ramos Mejia'' (\texttt{False}) in the above case. For those entities that are the same across atomic facts (\eg, ``was born'') or are neutral (\eg, ``he,'' ``in,'' and ``on''), please label them as \texttt{True}. For the cases that there is no atomic fact that shares the same sentence structure, please identify the most informative entities in the sentence and label them with the same label as the atomic fact while treating the rest of the entities as \texttt{True}. In the end, output the entities and labels in the following format:
                \begin{itemize}[nosep]
                    \item Entity 1 (Label 1)
                    \item Entity 2 (Label 2)
                    \item ...
                    \item Entity N (Label N)
                \end{itemize}
                % \newline \newline
                Here are two examples:
                \newline\newline
                \textbf{[Example 1]}
                \newline
                [The start of the biography]
                \newline
                \textcolor{titlecolor}{Marianne McAndrew is an American actress and singer, born on November 21, 1942, in Cleveland, Ohio. She began her acting career in the late 1960s, appearing in various television shows and films.}
                \newline
                [The end of the biography]
                \newline \newline
                [The start of the list of checked facts]
                \newline
                \textcolor{anscolor}{[Marianne McAndrew is an American. (False); Marianne McAndrew is an actress. (True); Marianne McAndrew is a singer. (False); Marianne McAndrew was born on November 21, 1942. (False); Marianne McAndrew was born in Cleveland, Ohio. (False); She began her acting career in the late 1960s. (True); She has appeared in various television shows. (True); She has appeared in various films. (True)]}
                \newline
                [The end of the list of checked facts]
                \newline \newline
                [The start of the ideal output]
                \newline
                \textcolor{labelcolor}{[Marianne McAndrew (True); is (True); an (True); American (False); actress (True); and (True); singer (False); , (True); born (True); on (True); November 21, 1942 (False); , (True); in (True); Cleveland, Ohio (False); . (True); She (True); began (True); her (True); acting career (True); in (True); the late 1960s (True); , (True); appearing (True); in (True); various (True); television shows (True); and (True); films (True); . (True)]}
                \newline
                [The end of the ideal output]
				\newline \newline
                \textbf{[Example 2]}
                \newline
                [The start of the biography]
                \newline
                \textcolor{titlecolor}{Doug Sheehan is an American actor who was born on April 27, 1949, in Santa Monica, California. He is best known for his roles in soap operas, including his portrayal of Joe Kelly on ``General Hospital'' and Ben Gibson on ``Knots Landing.''}
                \newline
                [The end of the biography]
                \newline \newline
                [The start of the list of checked facts]
                \newline
                \textcolor{anscolor}{[Doug Sheehan is an American. (True); Doug Sheehan is an actor. (True); Doug Sheehan was born on April 27, 1949. (True); Doug Sheehan was born in Santa Monica, California. (False); He is best known for his roles in soap operas. (True); He portrayed Joe Kelly. (True); Joe Kelly was in General Hospital. (True); General Hospital is a soap opera. (True); He portrayed Ben Gibson. (True); Ben Gibson was in Knots Landing. (True); Knots Landing is a soap opera. (True)]}
                \newline
                [The end of the list of checked facts]
                \newline \newline
                [The start of the ideal output]
                \newline
                \textcolor{labelcolor}{[Doug Sheehan (True); is (True); an (True); American (True); actor (True); who (True); was born (True); on (True); April 27, 1949 (True); in (True); Santa Monica, California (False); . (True); He (True); is (True); best known (True); for (True); his roles in soap operas (True); , (True); including (True); in (True); his portrayal (True); of (True); Joe Kelly (True); on (True); ``General Hospital'' (True); and (True); Ben Gibson (True); on (True); ``Knots Landing.'' (True)]}
                \newline
                [The end of the ideal output]
				\newline \newline
				\textbf{User prompt}
				\newline
				\newline
				[The start of the biography]
				\newline
				\textcolor{magenta}{\texttt{\{BIOGRAPHY\}}}
				\newline
				[The ebd of the biography]
				\newline \newline
				[The start of the list of checked facts]
				\newline
				\textcolor{magenta}{\texttt{\{LIST OF CHECKED FACTS\}}}
				\newline
				[The end of the list of checked facts]
			};
			\node[chatbox_user_inner] (q1_text) at (q1) {
				\textbf{System prompt}
				\newline
				\newline
				You are a helpful and precise assistant for segmenting and labeling sentences. We would like to request your help on curating a dataset for entity-level hallucination detection.
				\newline \newline
                We will give you a machine generated biography and a list of checked facts about the biography. Each fact consists of a sentence and a label (True/False). Please do the following process. First, breaking down the biography into words. Second, by referring to the provided list of facts, merging some broken down words in the previous step to form meaningful entities. For example, ``strategic thinking'' should be one entity instead of two. Third, according to the labels in the list of facts, labeling each entity as True or False. Specifically, for facts that share a similar sentence structure (\eg, \textit{``He was born on Mach 9, 1941.''} (\texttt{True}) and \textit{``He was born in Ramos Mejia.''} (\texttt{False})), please first assign labels to entities that differ across atomic facts. For example, first labeling ``Mach 9, 1941'' (\texttt{True}) and ``Ramos Mejia'' (\texttt{False}) in the above case. For those entities that are the same across atomic facts (\eg, ``was born'') or are neutral (\eg, ``he,'' ``in,'' and ``on''), please label them as \texttt{True}. For the cases that there is no atomic fact that shares the same sentence structure, please identify the most informative entities in the sentence and label them with the same label as the atomic fact while treating the rest of the entities as \texttt{True}. In the end, output the entities and labels in the following format:
                \begin{itemize}[nosep]
                    \item Entity 1 (Label 1)
                    \item Entity 2 (Label 2)
                    \item ...
                    \item Entity N (Label N)
                \end{itemize}
                % \newline \newline
                Here are two examples:
                \newline\newline
                \textbf{[Example 1]}
                \newline
                [The start of the biography]
                \newline
                \textcolor{titlecolor}{Marianne McAndrew is an American actress and singer, born on November 21, 1942, in Cleveland, Ohio. She began her acting career in the late 1960s, appearing in various television shows and films.}
                \newline
                [The end of the biography]
                \newline \newline
                [The start of the list of checked facts]
                \newline
                \textcolor{anscolor}{[Marianne McAndrew is an American. (False); Marianne McAndrew is an actress. (True); Marianne McAndrew is a singer. (False); Marianne McAndrew was born on November 21, 1942. (False); Marianne McAndrew was born in Cleveland, Ohio. (False); She began her acting career in the late 1960s. (True); She has appeared in various television shows. (True); She has appeared in various films. (True)]}
                \newline
                [The end of the list of checked facts]
                \newline \newline
                [The start of the ideal output]
                \newline
                \textcolor{labelcolor}{[Marianne McAndrew (True); is (True); an (True); American (False); actress (True); and (True); singer (False); , (True); born (True); on (True); November 21, 1942 (False); , (True); in (True); Cleveland, Ohio (False); . (True); She (True); began (True); her (True); acting career (True); in (True); the late 1960s (True); , (True); appearing (True); in (True); various (True); television shows (True); and (True); films (True); . (True)]}
                \newline
                [The end of the ideal output]
				\newline \newline
                \textbf{[Example 2]}
                \newline
                [The start of the biography]
                \newline
                \textcolor{titlecolor}{Doug Sheehan is an American actor who was born on April 27, 1949, in Santa Monica, California. He is best known for his roles in soap operas, including his portrayal of Joe Kelly on ``General Hospital'' and Ben Gibson on ``Knots Landing.''}
                \newline
                [The end of the biography]
                \newline \newline
                [The start of the list of checked facts]
                \newline
                \textcolor{anscolor}{[Doug Sheehan is an American. (True); Doug Sheehan is an actor. (True); Doug Sheehan was born on April 27, 1949. (True); Doug Sheehan was born in Santa Monica, California. (False); He is best known for his roles in soap operas. (True); He portrayed Joe Kelly. (True); Joe Kelly was in General Hospital. (True); General Hospital is a soap opera. (True); He portrayed Ben Gibson. (True); Ben Gibson was in Knots Landing. (True); Knots Landing is a soap opera. (True)]}
                \newline
                [The end of the list of checked facts]
                \newline \newline
                [The start of the ideal output]
                \newline
                \textcolor{labelcolor}{[Doug Sheehan (True); is (True); an (True); American (True); actor (True); who (True); was born (True); on (True); April 27, 1949 (True); in (True); Santa Monica, California (False); . (True); He (True); is (True); best known (True); for (True); his roles in soap operas (True); , (True); including (True); in (True); his portrayal (True); of (True); Joe Kelly (True); on (True); ``General Hospital'' (True); and (True); Ben Gibson (True); on (True); ``Knots Landing.'' (True)]}
                \newline
                [The end of the ideal output]
				\newline \newline
				\textbf{User prompt}
				\newline
				\newline
				[The start of the biography]
				\newline
				\textcolor{magenta}{\texttt{\{BIOGRAPHY\}}}
				\newline
				[The ebd of the biography]
				\newline \newline
				[The start of the list of checked facts]
				\newline
				\textcolor{magenta}{\texttt{\{LIST OF CHECKED FACTS\}}}
				\newline
				[The end of the list of checked facts]
			};
		\end{tikzpicture}
        \caption{GPT-4o prompt for labeling hallucinated entities.}\label{tb:gpt-4-prompt}
	\end{center}
\vspace{-0cm}
\end{table*}

We present the instructions employed for LLM-based evaluation in Figure~\ref{fig:prompt}, which includes both win-rate comparison and quantitative comparison. For the win-rate comparison, we input the query alongside two competing answers, designated as \texttt{answer1} and \texttt{answer2}, while alternating their positions across multiple iterations to mitigate any positional bias affecting LLM inference.

In the quantitative comparison, we leverage a standard answer from NaiveRAG~\cite{NaiveRAG} labeled \texttt{baseline\_answer}, against which the evaluated answer, referred to as \texttt{evaluation\_answer}, is assessed. The LLM assigns a score from 1 to 5, indicating whether the evaluated answer is inferior or superior to the baseline. This instruction allows us to compare the outputs of multiple models against the same standard answer, thus eliminating the need to adjust their positions. Since all methods are evaluated consistently against the same baseline, positional bias is inherently mitigated, enabling a direct comparison of scores across different methods.

