\documentclass[twoside]{article}

\usepackage[accepted]{aistats2025}
% If your paper is accepted, change the options for the package
% aistats2025 as follows:
%
%\usepackage[accepted]{aistats2025}
%
% This option will print headings for the title of your paper and
% headings for the authors names, plus a copyright note at the end of
% the first column of the first page.

% If you set papersize explicitly, activate the following three lines:
%\special{papersize = 8.5in, 11in}
%\setlength{\pdfpageheight}{11in}
%\setlength{\pdfpagewidth}{8.5in}

% If you use natbib package, activate the following three lines:
\usepackage[round]{natbib}
\renewcommand{\bibname}{References}
\renewcommand{\bibsection}{\subsubsection*{\bibname}}

% If you use BibTeX in apalike style, activate the following line:
%\bibliographystyle{apalike}

%%%%%%%%%%%%%%%%%%%%%%%%%%%%%
% Add new packages here
%%%%%%%%%%%%%%%%%%%%%%%%%%%%%
\usepackage{amsmath,amsthm,amsfonts,amssymb,mathtools}
\usepackage{enumitem}
\usepackage{bm,dsfont}
\usepackage[T1]{fontenc}
\usepackage{xcolor,pifont}
\usepackage{graphicx}
\usepackage{caption, subcaption, float}
\usepackage{hyperref}
\usepackage[nameinlink]{cleveref}
\usepackage[title]{appendix}
\usepackage{algorithm,algorithmic,setspace}
\usepackage{booktabs}
\usepackage{multirow}
\usepackage{makecell}
\usepackage{color, colortbl, tcolorbox}
\usepackage[hang,flushmargin]{footmisc}
\usepackage{ifthen}

\newtheorem{theorem}{Theorem}
\newtheorem{remark}{Remark}
\newtheorem{lemma}{Lemma}
\newenvironment{claim}[1]{\par\noindent\underline{Claim:}\space#1}{}
\newenvironment{claimproof}[1]{\par\noindent\underline{Proof:}\space#1}{\hfill $\blacksquare$}

%%%%%%%%%%%%%%%%%%%%%%%%%%%%%
% Add new commends here
%%%%%%%%%%%%%%%%%%%%%%%%%%%%%
\newcommand{\hatf}{\hat{f}}
\definecolor{Gray}{gray}{0.9}
% Define a custom variable: set to 1 to show, 0 to hide
\newcommand{\showcontent}{1}

% Remove the copyright box horizontal line
\makeatletter
\def\@copyrightspace{
\@float{copyrightbox}[b]
\begin{center}
\vspace{-1.3em}
\setlength{\unitlength}{1pc}
\begin{picture}(20,2.5)
\put(0,0){\parbox[b]{19.75pc}{\small \Notice@String}}
\end{picture}
\end{center}
\end@float}
\makeatother

%%%%%%%%%%%%%%%%%%%%%%%%%%%%%
% New commands
%%%%%%%%%%%%%%%%%%%%%%%%%%%%%
% calligraphic letter for set and class
\newcommand{\Ac}{\mathcal{A}}
\newcommand{\Bc}{\mathcal{B}}
\newcommand{\Cc}{\mathcal{C}}
\newcommand{\Dc}{\mathcal{D}}
\newcommand{\Ec}{\mathcal{E}}
\newcommand{\Fc}{\mathcal{F}}
\newcommand{\Gc}{\mathcal{G}}
\newcommand{\Hc}{\mathcal{H}}
\newcommand{\Ic}{\mathcal{I}}
\newcommand{\Jc}{\mathcal{J}}
\newcommand{\Kc}{\mathcal{K}}
\newcommand{\Lc}{\mathcal{L}}
\newcommand{\Mc}{\mathcal{M}}
\newcommand{\Nc}{\mathcal{N}}
\newcommand{\Oc}{\mathcal{O}}
\newcommand{\Pc}{\mathcal{P}}
\newcommand{\Qc}{\mathcal{Q}}
\newcommand{\Rc}{\mathcal{R}}
\newcommand{\Sc}{\mathcal{S}}
\newcommand{\Tc}{\mathcal{T}}
\newcommand{\Uc}{\mathcal{U}}
\newcommand{\Vc}{\mathcal{V}}
\newcommand{\Wc}{\mathcal{W}}
\newcommand{\Xc}{\mathcal{X}}
\newcommand{\Yc}{\mathcal{Y}}
\newcommand{\Zc}{\mathcal{Z}}

% special set and measures
\newcommand{\Pb}{\mathbb{P}}
\newcommand{\Qb}{\mathbb{Q}}
\newcommand{\Hb}{\mathbb{H}}
\newcommand{\Ib}{\mathbb{I}}
\newcommand{\Nb}{\mathbb{N}}
\newcommand{\Eb}{\mathbb{E}}
\newcommand{\Rb}{\mathbb{R}}
\newcommand{\Zb}{\mathbb{Z}}

% vectors and matrices
\newcommand{\uv}{\mathbf{u}}
\newcommand{\vv}{\mathbf{v}}
\newcommand{\wv}{\mathbf{w}}
\newcommand{\xv}{\mathbf{x}}
\newcommand{\yv}{\mathbf{y}}
\newcommand{\zv}{\mathbf{z}}
\newcommand{\fv}{\mathbf{f}}
\newcommand{\hv}{\mathbf{h}}
\newcommand{\gv}{\mathbf{g}}
\newcommand{\mv}{\mathbf{m}}

\newcommand{\Dv}{\mathbf{D}}
\newcommand{\Hv}{\mathbf{H}}
\newcommand{\Iv}{\mathbf{I}}
\newcommand{\Rv}{\mathbf{R}}
\newcommand{\Sv}{\mathbf{S}}
\newcommand{\Vv}{\mathbf{V}}
\newcommand{\Xv}{\mathbf{X}}
\newcommand{\Uv}{\mathbf{U}}
\newcommand{\Zv}{\mathbf{Z}}
\newcommand{\Lv}{\mathbf{L}}
\newcommand{\Wv}{\mathbf{W}}
\newcommand{\Kv}{\mathbf{K}}
\newcommand{\Yv}{\mathbf{Y}}

% Random variables
\def\reta{{\textnormal{$\eta$}}}
\def\ra{{\textnormal{a}}}
\def\rb{{\textnormal{b}}}
\def\rc{{\textnormal{c}}}
\def\rd{{\textnormal{d}}}
\def\re{{\textnormal{e}}}
\def\rf{{\textnormal{f}}}
\def\rg{{\textnormal{g}}}
\def\rh{{\textnormal{h}}}
\def\ri{{\textnormal{i}}}
\def\rj{{\textnormal{j}}}
\def\rk{{\textnormal{k}}}
\def\rl{{\textnormal{l}}}
% rm is already a command, just don't name any random variables m
\def\rn{{\textnormal{n}}}
\def\ro{{\textnormal{o}}}
\def\rp{{\textnormal{p}}}
\def\rq{{\textnormal{q}}}
\def\rr{{\textnormal{r}}}
\def\rs{{\textnormal{s}}}
\def\rt{{\textnormal{t}}}
\def\ru{{\textnormal{u}}}
\def\rv{{\textnormal{v}}}
\def\rw{{\textnormal{w}}}
\def\rx{{\textnormal{x}}}
\def\ry{{\textnormal{y}}}
\def\rz{{\textnormal{z}}}

% Random vectors
\def\rvepsilon{{\mathbf{\epsilon}}}
\def\rvtheta{{\mathbf{\theta}}}
\def\rva{{\mathbf{a}}}
\def\rvb{{\mathbf{b}}}
\def\rvc{{\mathbf{c}}}
\def\rvd{{\mathbf{d}}}
\def\rve{{\mathbf{e}}}
\def\rvf{{\mathbf{f}}}
\def\rvg{{\mathbf{g}}}
\def\rvh{{\mathbf{h}}}
\def\rvu{{\mathbf{i}}}
\def\rvj{{\mathbf{j}}}
\def\rvk{{\mathbf{k}}}
\def\rvl{{\mathbf{l}}}
\def\rvm{{\mathbf{m}}}
\def\rvn{{\mathbf{n}}}
\def\rvo{{\mathbf{o}}}
\def\rvp{{\mathbf{p}}}
\def\rvq{{\mathbf{q}}}
\def\rvr{{\mathbf{r}}}
\def\rvs{{\mathbf{s}}}
\def\rvt{{\mathbf{t}}}
\def\rvu{{\mathbf{u}}}
\def\rvv{{\mathbf{v}}}
\def\rvw{{\mathbf{w}}}
\def\rvx{{\mathbf{x}}}
\def\rvy{{\mathbf{y}}}
\def\rvz{{\mathbf{z}}}

% Elements of random vectors
\def\erva{{\textnormal{a}}}
\def\ervb{{\textnormal{b}}}
\def\ervc{{\textnormal{c}}}
\def\ervd{{\textnormal{d}}}
\def\erve{{\textnormal{e}}}
\def\ervf{{\textnormal{f}}}
\def\ervg{{\textnormal{g}}}
\def\ervh{{\textnormal{h}}}
\def\ervi{{\textnormal{i}}}
\def\ervj{{\textnormal{j}}}
\def\ervk{{\textnormal{k}}}
\def\ervl{{\textnormal{l}}}
\def\ervm{{\textnormal{m}}}
\def\ervn{{\textnormal{n}}}
\def\ervo{{\textnormal{o}}}
\def\ervp{{\textnormal{p}}}
\def\ervq{{\textnormal{q}}}
\def\ervr{{\textnormal{r}}}
\def\ervs{{\textnormal{s}}}
\def\ervt{{\textnormal{t}}}
\def\ervu{{\textnormal{u}}}
\def\ervv{{\textnormal{v}}}
\def\ervw{{\textnormal{w}}}
\def\ervx{{\textnormal{x}}}
\def\ervy{{\textnormal{y}}}
\def\ervz{{\textnormal{z}}}

% Random matrices
\def\rmA{{\mathbf{A}}}
\def\rmB{{\mathbf{B}}}
\def\rmC{{\mathbf{C}}}
\def\rmD{{\mathbf{D}}}
\def\rmE{{\mathbf{E}}}
\def\rmF{{\mathbf{F}}}
\def\rmG{{\mathbf{G}}}
\def\rmH{{\mathbf{H}}}
\def\rmI{{\mathbf{I}}}
\def\rmJ{{\mathbf{J}}}
\def\rmK{{\mathbf{K}}}
\def\rmL{{\mathbf{L}}}
\def\rmM{{\mathbf{M}}}
\def\rmN{{\mathbf{N}}}
\def\rmO{{\mathbf{O}}}
\def\rmP{{\mathbf{P}}}
\def\rmQ{{\mathbf{Q}}}
\def\rmR{{\mathbf{R}}}
\def\rmS{{\mathbf{S}}}
\def\rmT{{\mathbf{T}}}
\def\rmU{{\mathbf{U}}}
\def\rmV{{\mathbf{V}}}
\def\rmW{{\mathbf{W}}}
\def\rmX{{\mathbf{X}}}
\def\rmY{{\mathbf{Y}}}
\def\rmZ{{\mathbf{Z}}}

% Elements of random matrices
\def\ermA{{\textnormal{A}}}
\def\ermB{{\textnormal{B}}}
\def\ermC{{\textnormal{C}}}
\def\ermD{{\textnormal{D}}}
\def\ermE{{\textnormal{E}}}
\def\ermF{{\textnormal{F}}}
\def\ermG{{\textnormal{G}}}
\def\ermH{{\textnormal{H}}}
\def\ermI{{\textnormal{I}}}
\def\ermJ{{\textnormal{J}}}
\def\ermK{{\textnormal{K}}}
\def\ermL{{\textnormal{L}}}
\def\ermM{{\textnormal{M}}}
\def\ermN{{\textnormal{N}}}
\def\ermO{{\textnormal{O}}}
\def\ermP{{\textnormal{P}}}
\def\ermQ{{\textnormal{Q}}}
\def\ermR{{\textnormal{R}}}
\def\ermS{{\textnormal{S}}}
\def\ermT{{\textnormal{T}}}
\def\ermU{{\textnormal{U}}}
\def\ermV{{\textnormal{V}}}
\def\ermW{{\textnormal{W}}}
\def\ermX{{\textnormal{X}}}
\def\ermY{{\textnormal{Y}}}
\def\ermZ{{\textnormal{Z}}}

% Vectors
\def\vzero{{\bm{0}}}
\def\vone{{\bm{1}}}
\def\vmu{{\bm{\mu}}}
\def\vtheta{{\bm{\theta}}}
\def\va{{\bm{a}}}
\def\vb{{\bm{b}}}
\def\vc{{\bm{c}}}
\def\vd{{\bm{d}}}
\def\ve{{\bm{e}}}
\def\vf{{\bm{f}}}
\def\vg{{\bm{g}}}
\def\vh{{\bm{h}}}
\def\vi{{\bm{i}}}
\def\vj{{\bm{j}}}
\def\vk{{\bm{k}}}
\def\vl{{\bm{l}}}
\def\vm{{\bm{m}}}
\def\vn{{\bm{n}}}
\def\vo{{\bm{o}}}
\def\vp{{\bm{p}}}
\def\vq{{\bm{q}}}
\def\vr{{\bm{r}}}
\def\vs{{\bm{s}}}
\def\vt{{\bm{t}}}
\def\vu{{\bm{u}}}
\def\vv{{\bm{v}}}
\def\vw{{\bm{w}}}
\def\vx{{\bm{x}}}
\def\vy{{\bm{y}}}
\def\vz{{\bm{z}}}

% Elements of vectors
\def\evalpha{{\alpha}}
\def\evbeta{{\beta}}
\def\evepsilon{{\epsilon}}
\def\evlambda{{\lambda}}
\def\evomega{{\omega}}
\def\evmu{{\mu}}
\def\evpsi{{\psi}}
\def\evsigma{{\sigma}}
\def\evtheta{{\theta}}
\def\eva{{a}}
\def\evb{{b}}
\def\evc{{c}}
\def\evd{{d}}
\def\eve{{e}}
\def\evf{{f}}
\def\evg{{g}}
\def\evh{{h}}
\def\evi{{i}}
\def\evj{{j}}
\def\evk{{k}}
\def\evl{{l}}
\def\evm{{m}}
\def\evn{{n}}
\def\evo{{o}}
\def\evp{{p}}
\def\evq{{q}}
\def\evr{{r}}
\def\evs{{s}}
\def\evt{{t}}
\def\evu{{u}}
\def\evv{{v}}
\def\evw{{w}}
\def\evx{{x}}
\def\evy{{y}}
\def\evz{{z}}

% Matrix
\def\mA{{\bm{A}}}
\def\mB{{\bm{B}}}
\def\mC{{\bm{C}}}
\def\mD{{\bm{D}}}
\def\mE{{\bm{E}}}
\def\mF{{\bm{F}}}
\def\mG{{\bm{G}}}
\def\mH{{\bm{H}}}
\def\mI{{\bm{I}}}
\def\mJ{{\bm{J}}}
\def\mK{{\bm{K}}}
\def\mL{{\bm{L}}}
\def\mM{{\bm{M}}}
\def\mN{{\bm{N}}}
\def\mO{{\bm{O}}}
\def\mP{{\bm{P}}}
\def\mQ{{\bm{Q}}}
\def\mR{{\bm{R}}}
\def\mS{{\bm{S}}}
\def\mT{{\bm{T}}}
\def\mU{{\bm{U}}}
\def\mV{{\bm{V}}}
\def\mW{{\bm{W}}}
\def\mX{{\bm{X}}}
\def\mY{{\bm{Y}}}
\def\mZ{{\bm{Z}}}
\def\mBeta{{\bm{\beta}}}
\def\mPhi{{\bm{\Phi}}}
\def\mLambda{{\bm{\Lambda}}}
\def\mSigma{{\bm{\Sigma}}}

\newcommand{\argmax}{\arg\max}
\newcommand{\argmin}{\arg\min}


\newcommand{\dm}{\mathrm{d}}
\newcommand{\TV}{\text{TV}}
\newcommand{\KL}{\mathrm{KL}}
\newcommand{\Var}{\mathrm{Var}}
\newcommand{\ERM}{\mathrm{ERM}}
\newcommand{\gen}{\mathrm{gen}}
\newcommand{\sign}{\mathrm{sign}}
\newcommand{\supp}{\mathrm{supp}}
\newcommand{\epi}{\mathrm{epi}}
\newcommand{\diag}{\mathrm{diag}}
\newcommand{\SNR}{\mathrm{SNR}}
\newcommand{\numpy}{{\tt numpy}}

\begin{document}

% If your paper is accepted and the title of your paper is very long,
% the style will print as headings an error message. Use the following
% command to supply a shorter title of your paper so that it can be
% used as headings.
%
%\runningtitle{I use this title instead because the last one was very long}

% If your paper is accepted and the number of authors is large, the
% style will print as headings an error message. Use the following
% command to supply a shorter version of the authors names so that
% they can be used as headings (for example, use only the surnames)
%
%\runningauthor{Surname 1, Surname 2, Surname 3, ...., Surname n}


\runningtitle{Deep Additive Kernel}
\runningauthor{Wenyuan Zhao, Haoyuan Chen, Tie Liu, Rui Tuo, Chao Tian}

\twocolumn[

\aistatstitle{From Deep Additive Kernel Learning to Last-Layer Bayesian Neural Networks via Induced Prior Approximation
}

\aistatsauthor{ Wenyuan Zhao$^*$ \And Haoyuan Chen$^*$ \And Tie Liu}
\aistatsaddress{Texas A\&M University \\ wyzhao@tamu.edu \And Texas A\&M University \\ chenhaoyuan2018@tamu.edu \And Texas A\&M University \\ tieliu@tamu.edu}

\aistatsauthor{Rui Tuo \And Chao Tian}
\aistatsaddress{Texas A\&M University \\ ruituo@tamu.edu \And Texas A\&M University \\ chao.tian@tamu.edu}

% \aistatsaddress{ $^1$Department of Electrical \& Computer Engineering, Texas A\&M University\\
% $^2$Department of Industrial \& Systems Engineering, Texas A\&M University}
]

\def\thefootnote{*}\footnotetext{The first two authors contributed equally.}\def\thefootnote{\arabic{footnote}}

\begin{abstract}
With the strengths of both deep learning and kernel methods like Gaussian Processes (GPs), Deep Kernel Learning (DKL) has gained considerable attention in recent years. From the computational perspective, however, DKL becomes challenging when the input dimension of the GP layer is high. To address this challenge, we propose the Deep Additive Kernel (DAK) model, which incorporates i) an additive structure for the last-layer GP; and ii) induced prior approximation for each GP unit. 
% \textcolor{red}{The ``last-layer GP'' comprises multiple additive GP units but in one GP layer, which is applied behind the feature extractor.} 
This naturally leads to a last-layer Bayesian neural network (BNN) architecture. The proposed method enjoys the interpretability of DKL as well as the computational advantages of BNN. Empirical results show that the proposed approach outperforms state-of-the-art DKL methods in both regression and classification tasks.
\end{abstract}


\section{INTRODUCTION}
%paragraph 1. introduce NN, GP, state the restrictions of GP, NN, and Deep GP, benefits of combining NN and GP, which is DKL: UQ and feature extractor

Deep Neural Networks (DNNs) \citep{lecun2015deep} are powerful tools capable of capturing intricate patterns in large datasets, and have demonstrated remarkable performance across a wide range of tasks. However, DNNs are prone to overfitting on small datasets, offer limited interpretability and transparency, and lack the ability to provide uncertainty estimation. On the other hand, as a traditional kernel-based method, Gaussian processes (GPs) \citep{williams2006gaussian} are robust against overfitting. In addition, they naturally incorporate uncertainty quantification and offer enhanced interpretability and adaptability for integrating prior knowledge. %through kernels. {\color{red}Tuo: But we choose the kernel not because of prior knowledge.}
% However, their practical applicability can be limited due to the need to specify a kernel function, which controls how inputs are related in the feature space. This specification often demands prior knowledge or extensive manual tuning, limiting the scalability and flexibility of GPs in handling complex, high-dimensional data. 
\citet{damianou2013deep} proposed Deep Gaussian Processes (DGPs) by stacking multiple layers of GPs, which introduces the hierarchical structure of deep learning with the probabilistic, non-parametric nature of GPs. Although the deep structure of DGPs allows them to learn features at different levels of abstraction, the tuning and optimization of DGPs, particularly for large datasets, can be difficult due to the layered structure, requiring careful consideration of hyperparameters and approximation techniques.

To learn rich hierarchical representations from the data with proper interpretability and uncertainty estimation, \citet{wilson2016deep} introduced Deep Kernel Learning (DKL) by incorporating DNN into the last layer kernel methods. This hybrid model enhances both flexibility and scalability compared to pure GPs, making it well-suited for a variety of real-world tasks, including regression, classification, and active learning, especially in scenarios where uncertainty quantification is critical.

%paragraph 2. limitations of DKL: computational complexity in GP layer, training complexity for ELBO? current literature to solve these issues? what's the limitation of current SOTA methods? Training with parameters of GP takes lots of time

One of the main challenges in DKL arises from the GP layer, which requires $\Oc(N^3)$ training time for $N$ data points, limiting its scalability for large datasets. Although DKL scales better than pure GPs as the feature dimension is reduced by the NN encoder, computing DKL exactly is still expensive, when the extracted features are high-dimensional, particularly in image (or video) datasets. To address this issue, several methods have been developed for GP approximation, including Random Fourier Features (RFF) \citep{rahimi2007random}, Stochastic Variational GP (SVGP) \citep{titsias2009variational,hensman2015scalable}, and Kernel Interpolation for Scalable Structured GP (KISS-GP) \citep{wilson2015kernel}, and these have been incorporated into DKL models \citep{wilson2016stochastic,xue2019deep,xie2019deep}. However, these sparse GP approximations via inducing points require $\mathcal{O}(M^3)$ time to compute the Evidence Lower Bound (ELBO), where $M$ is the number of inducing points. Therefore, DKL remains inefficient when a large number of inducing points are necessary for complex ML tasks.

Another challenge in DKL, as identified by \citet{ober2021promises}, is the tendency to overcorrelate features to minimize the complexity penalty term in the marginal likelihood. A fully Bayesian method incorporating Markov Chain Monte Carlo (MCMC) can effectively resolve this problem, but exhibit poor scalability with high-dimensional posterior distributions. In response, \citet{matias2024amortized} proposed Amortized Variational DKL (AV-DKL), which uses NN-based amortization networks to determine the inducing locations and variational parameters through input-dependent sparse GPs \citep{jafrasteh2021input}, thereby attenuating the overcorrelation of NN output features. 

%paragraph 3. To solve xxx issues, we consider expressing GP as a sparse BNN, advantages: efficient for training and inference, 

In this work, we propose the \textbf{D}eep \textbf{A}dditive \textbf{K}ernel (DAK) model, which embeds hierarchical features learned from NNs into additive GPs, and interpret the last-layer GP as a Bayesian neural network (BNN) layer \citep{mackay1992practical,harrison2024variational} with a sparse kernel activation via induced prior approximation on designed grids \citep{ding2024sparse}. The proposed methodology jointly trains the variational parameters of the last layer and the deterministic parameters of the feature extractor by maximizing the variational lower bound. This hybrid architecture enjoys the mathematical interpretability of DKL as well as the computational advantages of BNN, and 
% consisting of deterministic NNs followed by a sparse BNN, significantly enhances training and inference efficiency by treating the variational parameters as independent Gaussian weights and biases, allowing for a mean-field assumption. This 
also leads to a closed-form ELBO and predictive distribution for regression tasks, bypassing Monte Carlo (MC) sampling during inference and training. Our contributions are as follows:
\vspace{-0.3cm}
\begin{itemize}[itemsep=1.5pt,,parsep=1.5pt]
    \item We introduce a DAK model that reinterprets the deep additive kernel learning as a last-layer BNN via induced prior approximation. The proposed DAK enjoys the mathematical interpretability of DKL as well as the computational advantages of BNN, and can be adapted to a variety of DL applications.
    % combining an NN feature extractor with additive GP layers that have sparse kernel activations, utilizing the sparse structure of the Cholesky factor of the Laplace kernel on a designed induced grid. The proposed DAK model has a general form of BNN, which is flexible to generalize to different tasks.
    \item We derive closed-form expressions of both the predictive distribution in inference and the ELBO during training for regression tasks, eliminating the need for sampling. The reduced computational complexity is linear to the size of the induced grid.
    \item In experimental studies, we demonstrate that the proposed DAK model outperforms state-of-the-art DKL methods in both regression and classification tasks, while also mitigating the overfitting issue. Source code of DAK is available at the following link \url{https://github.com/warrenzha/dak2bnn}.
\end{itemize}
\vspace{-0.2cm}
The remainder of this paper is organized as follows: \Cref{sec:prelim} introduces the background on GPs, additive GPs, and DKL. In \Cref{sec:dak}, we present the proposed DAK model and discuss its computational complexity compared to other state-of-the-art DKL models, followed by related work in \Cref{sec:related work}. We present the experimental results in \Cref{sec:exp}, and conclude the paper in \Cref{sec:conc}.

\section{PRELIMINARIES}
\label{sec:prelim}
\paragraph{GPs.}
A GP $f(\cdot) \sim \mathcal{GP}(\mu(\cdot) ,k(\cdot,\cdot) )$ is completely specified by its mean function $\mu(\cdot)$ and the covariance (kernel) function $k(\cdot,\cdot)$. Given a dataset $\mathcal{D}=\{ \Xv, \yv \}$, where $\Xv=\{ \xv_{i} \in \mathbb{R}^D \}_{i=1}^{N}$ are training points and $\yv= ( y_1,\ldots,y_N )^{\top}$ is the corresponding observation where $y_i\in \Rb$, the standard procedure of GPs assumes $\mu(\cdot)$ is a constant function assigned to zero and considers $y_i=f(\xv_i) + \epsilon_i$ with Gaussian noise $\epsilon_i \sim \mathcal{N}(0, \sigma^2_f)$ for $i=1,\ldots,N$. The predictive posterior $\fv_{\ast}:=f(\Xv^{\ast})$ 
% with zero-mean GP prior $f(\cdot) \sim \mathcal{GP}\left(0, k(\cdot,\cdot) \right)$ 
at $N_{\ast}$ test points $\Xv^{\ast}:= \{ \xv_{i}^{\ast} \in \mathbb{R}^D \}_{i=1}^{N_{\ast}}$ can also be expressed in closed form as a Gaussian distribution:
\begin{align}
    & \fv_{\ast} \vert \Xv,\yv,\Xv^{\ast} \sim \mathcal{N} \left( \bm{\mu}_{\ast} , \bm{\Sigma}_{\ast} \right), \\
    \bm{\mu}_{\ast} &= \mathbf{K}_{\Xv^*,\Xv} \left( \mathbf{K}_{\Xv,\Xv} + \sigma_f^2 \Iv \right) ^{-1} \yv,\label{eq:GPR mean}\\
    \bm{\Sigma}_{\ast} &= \mathbf{K}_{\Xv^*,\Xv^*}
    - \mathbf{K}_{\Xv^*,\Xv} \left( \mathbf{K}_{\Xv,\Xv} + \sigma_f^2 \Iv \right)^{-1} \mathbf{K}_{\Xv,\Xv^*},
\end{align}
where $\mathbf{K}_{\Xv,\Xv'}$ denotes the kernel matrix with $[\mathbf{K}_{\Xv,\Xv'}]_{ij}:=k(\xv_{i}, \xv'_{j})$. 
The common challenges of GPs are the $\mathcal{O}(N^3)$ computational complexity of inference and the ``curse of dimensionality'' with high-dimensional data. We refer the readers to \citet{williams2006gaussian} for more details. 

\paragraph{Additive GPs.}
\citet{duvenaud2011additive} introduced the additive GP model, which allows additive interactions of all orders, ranging from first-order interactions all the way to $D$-th order interactions. In this work, we consider the simplest case where the additive GP is restricted to the first order with the same base kernel for each unit $d \in \{1,\ldots,D\}$:
\begin{align}
\label{eq:additiveGP}
    f(\xv)=\sigma\sum_{d=1}^{D} g_d (x_d)+\mu(\xv),
\end{align}
where $x_d$ is the $d$-th feature of the point $\xv\in \Rb^D$, $g_d(x_d) \sim \mathcal{GP}(0,k_d(x_d,x_d^{\prime}))$ is the \emph{centered} base GP unit. The resulting $f(\xv): \Rb^D\rightarrow \Rb$ is a GP specified by mean function $\mu(\xv)$ and additive kernel $k^{[D]}_{\text{add}}(\xv, \xv^{\prime}) = \sigma^2 \sum_{d=1}^{D}k_d(x_d, x_d^{\prime})$, where $\sigma^2$ is the variance assigned to all first-order interactions. \citet{delbridge2020randomly} showed that additive kernels projecting to a low dimensional setting can match or even surpass the performance of kernels operating in the original space. However, an additive kernel alone offers no computational advantage, as the cubic complexity remains. As we will show shortly, the additive GP perspective allows us to apply an induced prior approximation, which reduces computational costs and leads to a last-layer Bayesian representation.
% Another advantage of additive GPs is that it retains interpretability. Each function $f_d(x_{d})$ can be interpreted as the contribution of the corresponding predictor $x_d$ to the model, allowing analysts to understand the effect of each variable on the response.

\paragraph{Deep Kernel Learning.} 
The performance of GPs is limited by the choice of kernel. DKL \citep{wilson2016deep} attempts to solve this problem by DNNs, of which the structural properties can model high-dimensional data and large datasets well. DKL first learns feature transformations $h_\psi(\cdot)$ of DNNs with the parameters $\psi$. The outputs of DNNs are then used as inputs to a GP layer $\mathcal{GP}\left( \mu_{\theta}(\cdot ),k_{\theta}(\cdot ,\cdot ) \right)$ with the parameters $\theta$ resulting in the effective kernel $k_{\text{DKL}}\left( \xv,\xv^{\prime} \right) =k_{\theta}\left( h_{\psi}(\xv), h_{\psi}(\xv^{\prime}) \right)$ to learn uncertainty representation provided by GPs. The complete set of parameters $\left\{ \psi ,\theta \right\}$ in DNNs and GPs are trained jointly by maximizing the marginal log-likelihood (MLL).
% \begin{align}
%     \log p\left( \yv|\Xv;\psi, \theta \right) 
%     \propto & 
%     -(\yv - \bm{\mu}_{\Xv})^{\top}\Kv_{\Xv,\Xv}^{-1} (\yv - \bm{\mu}_{\Xv}) \nonumber\\
%     &- \log \left| \Kv_{\Xv,\Xv} \right|, 
%     \quad \bm{\mu}_{\Xv} := \mu(\Xv).
% \end{align}


\section{DAK: Deep Additive Kernel}
\label{sec:dak}
In GPs, the kernel function defines the similarity between data points, playing a crucial role in the model's predictions. However, choosing the right kernel for complex, high-dimensional data can be challenging. DKL addresses this by learning the kernel function directly from data using a DNN. In practice, the outputs of DNNs can still be too complex to scale GPs, e.g. multi-task regression, or image recognition. Therefore, we present the Deep Additive Kernel as a last-layer BNN that can efficiently reduce GPs to single-dimensional ones without loss in performance.

% \paragraph{Kernel.} 
% In this work, we consider the Laplace kernel 
% \begin{align}
% \label{eq:laplace base kernel}
%     k_p(x_p, x_p^{\prime})= \exp\left( - \vert x_p - x_p^{\prime} \vert / \theta_p \right)
% \end{align}
% as the base kernel for $p$-th base GP, $p \in \{1,\ldots,P\}$. Laplace kernels retain the sparse Cholesky decomposition and consequently reduce the computational complexity, see details in \Cref{sec:sparse chol decompose}. Let $\hv := h_{\psi}(\xv)\in \Rb^P$, which could be a DNN, denote the output of a feature transformation $h_{\psi}:\Rb^D\rightarrow \Rb^P$ of the input $\xv \in \Rb^D$. The resulting deep additive Laplace kernel is given by:
% \begin{align}
%     k_{\text{DAK}}\left( \xv,\xv^{\prime} \right) 
%     &= k^{[P]}_{\text{add}}\left( h_{\psi}(\xv),h_{\psi}(\xv^{\prime}) \right) \nonumber \\
%     &= \sum_{p=1}^{P} \sigma_{p}^{2} k_{p}\Big(
%     \begingroup
%         \color{black}
%         \underbracket{
%             \color{black} h_{\psi}^{[p]}(\xv)
%         }_{ \color{black} h_{p}\in \Rb
%         }
%     \endgroup
%     , 
%     \begingroup
%         \color{black}
%         \underbracket{
%             \color{black} h_{\psi}^{[p]}(\xv^{\prime})
%         }_{ \color{black} h_{p}^{\prime}\in \Rb
%         }
%     \endgroup
%     \Big),
%     % &= \sum_{p=1}^{P} \sigma_{p}^{2} \exp \left( -\left| h_{p}-h_{p}^{\prime} \right| / \theta_{p} \right),
% \end{align}
% where $\xv, \xv^{\prime} \in \Rb^D$ are the inputs, $h_p:=h_{\psi}^{[p]}(\xv)$ and $h_p^{\prime}:=h_{\psi}^{[p]}(\xv^{\prime})$ are the $p$-th features of $h_{\psi}(\xv)$ and $h_{\psi}(\xv^{\prime})$ respectively, and $k_p(\cdot,\cdot)$ is the $p$-th kernel with $\theta_p$ and $\sigma_p^2$ being its lengthscale and variance. The number of base kernels $P$ can be selected by applying a linear projection to the last layer of the DNN.
% {\color{red}Tuo: I suggest we merging this section with the next one: first introduce model (3), then state that we use the Laplace kernel for the 1D GPs. No need to mention the kernel (3).}
% \textcolor{magenta}{Haoyuan: I removed ``Kernel'' part and add description of Laplace kernel in ``Model''}
% \textcolor{brown}{Tian: Since we have space, we can consider adding a connection to the kernel representation to clarify things, though we don't really need to use it. Also if we have space, we can reintroduce the exact GP formula to emphasize the computation bottleneck in GP. We should not assume the audience of this conference knows everything about it.}

\paragraph{Model.} 
We present the construction of the DAK using the proposed additive GPs. %Laplace kernel. 
Let $h_{\psi}: \Rb^D \rightarrow \Rb^{P}$ be a neural network, and we consider a total of $P$ \emph{centered} one-dimensional base GPs $g_p \sim \mathcal{GP}(0, k_p(\cdot, \cdot))$ with Laplace kernel $k_p(x_p, x_p^{\prime})= \exp\left( - \vert x_p - x_p^{\prime} \vert / \theta_p \right)$ %$k_p(\cdot, \cdot)$ defined in \cref{eq:laplace base kernel}, 
for $p=1,\ldots,P$. The forward pass of deep additive model $f(\cdot): \Rb^D \rightarrow \Rb$ with Laplace kernel for each base GP %deep additive kernel $k_{\text{DAK}}(\cdot,\cdot)$ 
at $N$ data points $\Xv$ is described as follows:
\begin{align}
    f (\Xv) &=\sum_{p=1}^{P} \sigma_{p}g_{p} \Big( 
    \begingroup
    \color{black}
        \underbrace{ 
            \color{black} 
            h^{[p]}_{\psi}\left( \Xv \right) 
        }_{ \color{black} := \Hv_p \in \Rb^{N} }
    \endgroup
    \Big) + \mu \label{eq:deepadditive},
    % &=\sum_{p=1}^{P} \sigma_{p} \left( \Kv_{\Hv_{p}^{\ast},\Hv_{p}}\Kv_{\Hv_{p},\Hv_{p}}^{-1} g_{\theta_p}( \Hv_p )+ \bm{\mu_{p}} \right),
\end{align}
where $\Hv_p:=h^{[p]}_{\psi}\left( \Xv \right) \in \Rb^{ N}$ is the $p$-th feature vector of neural network representations $h_{\psi}\left( \Xv \right) \in \Rb^{N \times P}$, and $\mu \in \mathbb{R}$ is a constant prior mean placed on $f(\cdot)$. The resulting deep additive kernel can be written as:
\begin{align}
    k_{\text{DAK}}\left( \xv,\xv^{\prime} \right) 
    &= k^{[P]}_{\text{add}}\left( h_{\psi}(\xv),h_{\psi}(\xv^{\prime}) \right) \nonumber \\
    &= \sum_{p=1}^{P} \sigma_{p}^{2} k_{p}\Big(
    \begingroup
        \color{black}
        \underbracket{
            \color{black} h_{\psi}^{[p]}(\xv)
        }_{ \color{black} h_{p}\in \Rb
        }
    \endgroup
    , 
    \begingroup
        \color{black}
        \underbracket{
            \color{black} h_{\psi}^{[p]}(\xv^{\prime})
        }_{ \color{black} h_{p}^{\prime}\in \Rb
        }
    \endgroup
    \Big).
    % &= \sum_{p=1}^{P} \sigma_{p}^{2} \exp \left( -\left| h_{p}-h_{p}^{\prime} \right| / \theta_{p} \right),
\end{align}
The corresponding coefficients $\{\sigma_{p} \}_{p=1}^{P}$ help learn a correlated contribution of each base kernel, and high-dimensional features of neural network $h_{\psi}\left( \Xv \right)$ can be solved in a single-dimensional space $\Rb$ to achieve scalability. The GP hyperparameters can be inherently optimized with DNN parameters $\psi$ without additional computational cost in our method. The details of reparameterization and optimization tricks are deferred when we discuss \textbf{Inference} and \textbf{Training} shortly.

\paragraph{Induced Prior Approximation.}
%In addition to additive method for high-dimensional data, we also place sparse approximation on each \emph{base GP} component to address the $\mathcal{O}(N^3)$ complexity of inference with large datasets of size $N$. Rather than approximating the covariance matrix $\Kv_{\Xv,\Xv}$, we directly approximate the prior of each base GP $g_{p}$ by reduced-rank induced prior approximation $\tilde{g}_{p}$ given as follows 
An essential ingredient of the proposed method is to apply a reduced-rank approximation to the prior, i.e., the base GPs. This approximation is referred to as the induced prior approximation \citep{ding2024sparse}, given by
\begin{align}
    \tilde{g}_{p} (\cdot ) &:=\Kv_{(\cdot),\Uv} \Kv_{\Uv,\Uv}^{-1} \, g_{p}(\Uv) \label{eq:inducedAppro}\\
    &=
    \begingroup
    \color{black}
        \underbrace{ 
            \color{black}
            \Kv_{(\cdot),\Uv} \left[ \Lv_{\Uv}^{\top} \right]^{-1}
        }_{\color{black}
            := \phi(\cdot) \in \Rb^{1\times M}
        }
    \endgroup
    \begingroup
    \color{black}
        \underbrace{ 
            \color{black}
            \Lv_{\Uv}^{-1} g_{p} ( \Uv ) 
        }_{\color{black}
        :=\zv_p \,\sim\, \mathcal{N}(\bm{0}, \bm{I}_{M})
        }
    \endgroup  \nonumber\\
    &=\phi \left( \cdot \right) \zv_p, \label{eq:GPlayer}
\end{align}
where $\Uv=\{ u_{i}\in \mathbb{R} \}_{i=1}^{M}$ denotes the induced grids (see details in \Cref{sec:sparse chol decompose}), $\Lv_\Uv \in \Rb^{M\times M}$ is a lower triangular matrix derived by the Cholesky decomposition of $\Kv_{\Uv,\Uv}=\Lv_{\Uv} \Lv^{\top}_{\Uv}$, and $\zv_p=\Lv_{\Uv}^{-1} g_{p} \left( \Uv \right)\sim \mathcal{N}(\mathbf{0},\bm{I}_M)$ are i.i.d. standard Normal random variables. The main idea of induced prior approximation (\cref{eq:inducedAppro}) is to use the GP regression (see \cref{eq:GPR mean}) to reconstruct the prior GP. According to the theory of GP regression \citep{yakowitz1985comparison,wang2020prediction}, $\tilde{g}_{p}$ converges to the original prior as $\Uv$ becomes dense in its domain.

%The induced approximation reduces the complexity of inference from $\Oc(N^3)$ to $\Oc(M^2N)$. 
The standard Cholesky decomposition of a dense matrix requires $O(M^3)$ time. However, by leveraging the Markov property of the Laplace kernel, the decomposed matrix $\Lv_{\Uv}^{-1}$ becomes sparse if $\Uv$ is designed by a one-dimensional dyadic point set with increasing order \citep{ding2024sparse}. As a consequence, the complexity of Cholesky decomposition can be reduced to $\Oc(M)$. Details of obtaining the sparse Cholesky factors are presented in \Cref{sec:sparse chol decompose}. 

Applying the sparsely induced GP approximation in \cref{eq:GPlayer} together with the deep additive model in \cref{eq:deepadditive}, we obtain the final DAK:
\begin{align}
    \tilde{f}(\Xv) &=\sum_{p=1}^{P} \sigma_p \tilde{g}_{p} \left( h^{[p]}_{\psi}(\Xv) \right) + \mu \notag \\
    % &=\sum_{p=1}^{P} \sigma_{p} \Big(
    % \phi \big(\Hv_p \big) \zv_{p} + \mu_p
    % \begingroup
    % \color{blue}
    %     \underbracket{
    %         \color{black}
    %         \phi \big(\Hv^*_p \big)
    %     }_{\color{blue}
    %     \text{activation}
    %     %\substack{\text{sparse} \\ \text{activation}}
    %     }
    % \endgroup
    % \begingroup
    % \color{blue}
    %     \underbracket{
    %         \color{black}
    %         \zv_{p}
    %     }_{\color{blue}
    %     \text{weights}
    %     }
    % \endgroup
    % + \hspace{0.5em}
    % \begingroup
    % \color{blue}
    %     \underbracket{
    %         \color{black}
    %         \mu_p(\Hv^*_p)
    %     }_{
    %         \color{blue}
    %         \text{bias}
    %     }
    % \endgroup
    % \Big) \\
    &=\sum_{p=1}^{P} \sigma_{p} \Big(
    \phi \big(\Hv_p \big) \zv_{p}\Big) + \mu,  \label{eq:additiveDKL} 
\end{align}
where $\phi(\Hv_p):=\Kv_{\Hv_p,\Uv} \left[ \Lv_{\Uv}^{\top} \right]^{-1}\in \Rb^{1\times M}$ denotes the kernel activation of $p$-th base GP, $\mu$ is the mean of additive GP, and $\zv_p$'s are random weights with  i.i.d. normal prior distribution $\zv_p \sim \Nc(\bm{0},\bm{I}_M)$, for all $p=1,\ldots,P$.

\begin{figure*}[ht]
    \centering
    \includegraphics[width=0.95\textwidth]{DAK.pdf}
    \caption{\small{Model architecture of Deep Additive Kernel (DAK). DAK consists of a feature extractor $\text{NN}(\cdot)$ with a linear embedding layer $\Wv$, an additive kernel with base GP $\tilde{g}_{p}(\cdot)$ for $p=1,\ldots,P$, and a weighted sum layer. The embedded features learned by DNN are decomposed as first-order components and fed to base GPs, each consisting of a kernel activation and a GP forward layer. Each kernel activation is designed by a one-dimensional dyadic point set on an induced grid with sparse non-zero activated neurons.}}
    % \vspace{-0.4cm}
    \label{fig:model}
\end{figure*}

The proposed DAK in \cref{eq:additiveDKL} results in deep additive kernel learning which mathematically possesses the form of a last-layer BNN \citep{mackay1992practical} but with a kernel activation $\phi(\cdot):=\Kv_{(\cdot),\Uv} \left[ \Lv_{\Uv}^{\top} \right]^{-1}$ followed by a Gaussian forward layer with i.i.d. prior weights under standard Normal distribution. 

\Cref{fig:model} illustrates the module architecture of DAK. Thanks to the mathematical equivalence to BNN and reparameterization tricks on learnable parameters, all modules of our model can be implemented either as a deterministic or a Bayesian NN layer. This results in a single NN with hybrid layers, which allows us 
to straightforwardly take advantage of parallelized GPU computing using the readily available PyTorch package \citep{paszke2019pytorch}.


\paragraph{Inference.}
To obtain the predictive distribution, we perform Variational Inference (VI) to estimate the posterior using DAK in \cref{eq:additiveDKL}. With a motivation to naturally interpret the deep additive kernel learning as a last-layer BNN, and taking the mean field assumptions, we select a variational family of independent but not identical Gaussian weights $\Zv:=[ \zv_1,\ldots,\zv_P ]$ and additive Gaussian bias $\mu$, denoted by $\Theta_{\text{var}}:=\left\{ \{ \zv_{p}\}_{p=1}^{P}, \mu \right\}$. The variational Gaussian weights are parameterized as $\zv_{p} \sim \mathcal{N} (\bm{m}_{\zv_p} ,\Sv_{\zv_p})$ for $p=1,\ldots,P$, and the variational bias is parameterized as $\mu \sim \mathcal{N} ( m_{\mu},\sigma^2_{\mu} )$. We denote the reparameterization of $\Theta_{\text{var}}$ as $\bm{\eta}:=\left\{ \{ \mv_{\zv_{p}},\Sv_{\zv_{p}}\}_{p=1}^{P} , \{m_{\mu},\sigma_{\mu}\} \right\}$. 

Note that $\Sv_{\zv_p}\in\Rb^{M \times M}$ is a diagonal covariance matrix due to the mean field assumption of $\zv_p$, where $M$ is decided by the size of induced interpolation grids defined in \cref{eq:GPlayer}. The variational distribution is given by $q_{\bm{\eta}}(\Theta_{\text{var}}) = q(\mu) \prod_{p=1}^{P} q(\zv_{p}) = \Nc ( m_{\mu} ,\sigma_{\mu}^2 )\prod_{p=1}^{P} 
\Nc ( \bm{m}_{\zv_p} ,\Sv_{\zv_p} )$, and the prior of $\Theta_{\text{var}}$ is denoted by $p(\Theta_{\text{var}})$.

The other deterministic parameters consist of DNN parameters $\psi$ and additive GP hyperparameters $\left\{ \bm{\sigma} \right\}:=[\sigma_1,\ldots,\sigma_P]^{\top}$, denoted as $\bm{\theta}:=\{\psi, \bm{\sigma} \}$. We treat GP lengthscales as fixed parameters specified during initialization and apply an additional linearly embedding layer to DNN: $h_{\psi}(\cdot ):=\text{NN} (\cdot )\Wv: \Rb^D \rightarrow \Rb^P$, where $\text{NN} (\cdot ): \Rb^D \rightarrow \Rb^{D_w}$ represents any DNN, and $\Wv\in \Rb^{D_{w}\times P}$ is the linear embedding. The lengthscales can be inherently optimized by learning the NN weights $\Wv$ without loss in performance, which is encoded in the DNN parameters $\psi$. Further details are provided in \Cref{sec:theo}. 

Given a data point $\xv\in \Rb^D$, the forward pass of predictive distribution is given by
\begin{align}
\label{eq:DAK prediction}
    \tilde{f}_{\xv}:= \tilde{f}(\xv; \bm{\theta}, \bm{\eta}) =\sum_{p=1}^{P} \sigma_{p} \Big(
    \phi \big(h^{[p]}_{\psi}(\xv) \big) \zv_{p}\Big) + \mu,
\end{align}
where the complete set of parameters is $\Theta:= \left\{ \bm{\theta}, \bm{\eta} \right\} = \left\{ \psi ,\bm{\sigma}, \{ \mv_{\zv_{p}},\Sv_{\zv_{p}}\}_{p=1}^{P} , \{m_{\mu},\sigma_{\mu}\} \right\}$, consisting of the deterministic parameters $\bm{\theta} := \left\{ \psi ,\bm{\sigma} \right\}$ and the variational parameters $\bm{\eta}:= \left\{ \{ \mv_{\zv_{p}},\Sv_{\zv_{p}}\}_{p=1}^{P} , \{m_{\mu},\sigma_{\mu}\} \right\}$. In \Cref{sec:uq of inference}, we derive an analytical form for the predictive distribution $\tilde{f}$.

%is deferred to \Cref{sec:uq of inference}. 

%\textcolor{red}{In \Cref{sec:uq of inference}, we derive an analytical expression for the distribution of $\tilde{f}$ as
% \begin{align}       
%     \Nc\left(
%     \sum_{p=1}^{P}
%     \sigma_p ( \bm{\phi}_{p}^{\top} \bm{m}_{\zv_p}) + m_{\mu} ,\hspace{0.2em}
%     \sum_{p=1}^{P}
%     \sigma_p^2( \bm{\phi}_{p}^{\top} \Sv_{\zv_p} \bm{\phi}_{p} ) + \sigma_{\mu}^2
%     \right).
% \end{align}}




\paragraph{Training.} 
Vanilla DKL optimizes the marginal log-likelihood $\log \text{Pr} \left( \yv \vert \Xv, \bm{\theta} \right)$ which involves intractable integral of non-conjugate likelihoods in some tasks such as classification. We apply the framework of stochastic variational inference to fit GPs via the variational distribution $q_{\bm{\eta}}(\Theta_\text{var})$. Consequently, during training, we optimize the variational lower bound, as formulated in \cite{hensman2015scalable}, to achieve efficient and scalable inference:
\begin{align}
\label{eq:VI lower bound}
\log \text{Pr}(\yv | \Xv, \bm{\theta}) \geq &\sum_{\xv, y \in \Xv, \yv}\Eb_{q_{\bm{\eta}}(\Theta_{\text{var}})} \left[ \log \text{Pr} \big(y | \tilde{f}_{\xv} \big) \right] \nonumber \\
& - \text{KL} \left[ q_{\bm{\eta}}(\Theta_{\text{var}} ) \| p(\Theta_{\text{var}}) \right].
\end{align}
The details of the training are derived in \Cref{sec:training}. The resulting objective is known as Evidence Lower Bound (ELBO):
\begin{align}\label{eq:elbo}
    \mathcal{L} (\bm{\theta}, \bm{\eta}) :=
    & \  {\Eb}_{q_{\bm{\eta}}(\Theta_{\text{var}} )} \left[ \log \text{Pr} (\yv \vert \tilde{f}_{\Xv}) \right] \nonumber \\
    & - \text{KL} \left[ q_{\bm{\eta}}(\Theta_{\text{var}} ) \| p(\Theta_{\text{var}}) \right].
\end{align}
The first term of ELBO, the expected log-likelihood, can be estimated by MC methods. To avoid the potential computing cost of sampling, we also derive a sampling-free analytical ELBO with a closed form for regression tasks. The details of the derivation are provided in \Cref{sec:elbo}.

\paragraph{Computational Complexity.}
We summarize the computational complexity of our proposed DAK model compared to other state-of-the-art GP and DKL methods in \Cref{tab:complexity}. On the one hand, the number of inducing points $\hat{M}$ in SVGP and KISS-GP may need to be chosen quite large in more complex or multitask GPs since it is associated with the GP input dimensionality, while at competitive performances the size of induced grids $M$ in DAK can be small due to the first-order additive structure. On the other hand, the dimension of the embedding layer $P$ is usually smaller than the dimension of NN outputs $D_w$. For tasks that require MC sampling, a small number of samples is usually sufficient due to the sampling efficiency of the BNNs. Further discussion is deferred to \Cref{sec:complexity}.

\begin{table}[tb!]
    \caption{\small{Computational complexity of DKL models for $N$ training points in one iteration. $\hat{M}$ is the number of inducing points in SVGP and KISS-GP, while $M$ is the size of induced grids in DAK, $M < \hat{M}$. $S$ is the number of MC samples, $B$ is the size of mini-batch, $D_w$ is the dimension of the NN outputs in DKL, $P$ is the dimension after the linear embedding of NN features. DAK-MC refers to DAK using MC approximation, while DAK-CF refers to DAK using closed-form inference and ELBO.}}
    \centering
    \resizebox{\columnwidth}{!}{
    \begin{tabular}{lcc}
    \toprule[1pt]
                  & \textbf{Inference}       & \textbf{Training} (per iteration) \\
    \midrule[0.5pt]
    NN + SVGP     & $\Oc(\hat{M}^2 N)$    & $\Oc( S D_w \hat{M}B + \hat{M}^3)$ \\
    NN + KISS-GP  & $\Oc(D_w \hat{M}^{1+\frac{1}{D_w}})$  & $\Oc(S D_w \hat{M}B + D_w \hat{M}^{\frac{3}{D_w}})$ \\
    DAK-MC (ours) & $\Oc(SM)$       & $\Oc(SPMB + PM)$   \\
    DAK-CF (ours) & $\Oc(M)$        & $\Oc(PMB + PM)$    \\
    \bottomrule[1pt]
    \end{tabular}
    }
    % \vspace{-0.3cm}
    \label{tab:complexity}
\end{table}

\paragraph{Remarks.}
We highlight several key aspects of the proposed model: 
\textbf{1)} The proposed model adds base GP components directly, rather than adding kernels as described in \citep{duvenaud2011additive}. While they are mathematically equivalent, using an additive kernel alone does not lead to any computational advantage over any standard kernel, where the cubic computational complexity persists. In contrast, the additive GP led us to apply the induced approximation technique, which naturally leads to efficient computation and the last-layer Bayesian interpretation. %\tedoes not result in an explicit additive form for the prior $f(\cdot)$ when applying induced prior approximation, which leads to higher computational complexity.}
\textbf{2)} The induced prior approximation in our model differs from the standard inducing points approximation \citep{titsias2009variational} often used for GPs. In our approach, we approximate the prior using a fixed set of induced grids, resulting in a BNN representation. In contrast, the standard inducing point methods treat the inducing points as variational parameters for optimization, which does not lead to a BNN representation. Our method is much easier to implement and has a theoretical guarantee that as the inducing locations become dense in the input region, the approximation will become exact.
\textbf{3)} In the forward pass $f(\cdot)$ as defined in \cref{eq:DAK prediction}, we chose to place a constant prior to the mean $\mu$ on $f(\cdot)$, which also naturally facilitates the construction of the last-layer BNN. This provides a bridge between canonical GPs and BNNs, making them more flexible and extendable to diverse applications.


\begin{figure*}[ht]
    \centering
    \subfloat[\small{Exact GP.} \label{fig:gp1d}]{\includegraphics[width=.2\textwidth]{toy_gp.pdf}}
    \subfloat[\small{DGP.} \label{fig:dgp1d}]{\includegraphics[width=.2\textwidth]{toy_dgp.pdf}}
    \subfloat[\small{Exact DKL.} \label{fig:dkl1d}]{\includegraphics[width=.2\textwidth]{toy_dkl_2.pdf}}
    \subfloat[\small{DAK.} \label{fig:dak1d}]{\includegraphics[width=.2\textwidth]{toy_dak_new.pdf}}
    \subfloat[\small{DNN.} \label{fig:nn1d}]{\includegraphics[width=.2\textwidth]{toy_dnn.pdf}}
    % \subfloat[\small{Training curves} \label{fig:toyloss}]{\includegraphics[width=.23\textwidth]{figs/toy/toy_loss.pdf}}
    % \vspace{-0.2cm}
    \caption{\small{Results on toy dataset. (a)--(d) show the predictive posterior of the exact GP, DGP, exact DKL and proposed DAK model, respectively, on the noisy data generated by 1D GP with zero-mean and covariance function $k(x,x')=\exp( -(x-x')^2 )$. We set the number of MC samples $S=4$ for estimating the expected log-likelihood in ELBO during training.} The predictive mean and $\pm$2 standard deviations are plotted together with the observed data. (e) shows the NN fit with the same training data.}
    % \vspace{-0.3cm}
    \label{fig:toyGP}
\end{figure*}

\section{RELATED WORK}
\label{sec:related work}
Motivated by integrating the power of deep networks with interpretable and theoretically grounded kernel methods, it is encouraging to see some contributions on such combinations across a range of contexts.

\paragraph{Variations of DKL.} 
Several recent studies have explored variations of DKL models. To handle large datasets and diverse tasks, \citet{wilson2016stochastic} extended the vanilla DKL \citep{wilson2016deep} to Stochastic Variational DKL (SV-DKL) by leveraging Stochastic Variational GP (SVGP) \citep{hensman2015scalable} and KISS-GP \citep{wilson2015kernel}. However, the kernel interpolation on a Cartesian grid does not scale in the high-dimensional space. \citet{xue2019deep} and \citet{xie2019deep} integrated deep kernel models with Random Fourier Features (RFF) \citep{rahimi2007random}, an efficient method for approximating GP kernel feature mappings. Some other work has focused on developing models with specialized kernel structures, such as recurrent kernels \citep{al2017learning} and compositional kernels \citep{sun2018differentiable}. More recently, %\citet{liu2021deep} introduced a latent-variable framework that integrates a stochastic encoding of inputs to enable regularized representation learning, while \citet{achituve2023guided} proposed a novel approach to training DKL by using an infinite-width NN to guide DKL optimization. 
\citet{ober2021promises} investigated overfitting in DKL models based on marginal likelihood maximization and proposed a fully Bayesian approach to mitigate this issue. \citet{matias2024amortized} introduced amortized variational DKL, which uses DNNs to learn variational distributions over inducing points in an amortized manner, thereby reducing overfitting by locally smoothing predictions. 
% \citet{harrison2024variational} introduced a sampling-free Bayesian last-layer architecture but did not establish a connection between this architecture and DKL. In our work, we approximate GPs using additive structure and sparsely induced priors to enhance the computational efficiency of DKL, which results in a last-layer BNN that is both efficient and easily adaptable to various tasks.

\paragraph{GPs and NNs.}
The connection between GPs and NNs was first established by \citet{neal1994bayesian}, who showed that the function defined by a single-layer NN with infinite width, random independent zero-mean weights, and biases is equivalent to a GP. This equivalence was later generalized to arbitrary NNs with infinite-width or infinite-depth layers in \citep{lee2017deep, cutajar2017random, matthews2018gaussian, yang2019wide, dutordoir2021deep, gao2023wide}. However, these studies focus primarily on fully connected NNs, which are not suitable for all practical applications. \citet{garriga2018deep} and \citet{novak2018bayesian} extended the equivalence to Convolutional Neural Networks (CNNs) \citep{lecun1989handwritten}, which are widely used in image recognition. Hybrid models combining GPs and NNs have also been investigated. \citet{bradshaw2017adversarial} proposed a hybrid GPDNN that feeds CNN features into a GP, while \citet{zhang2024gaussian} introduced GP Neural Additive Models (GP-NAM), a class of GAMs that use GPs approximated by RFF and a single-layer NN. However, GPDNN uses the standard GP, while we approximate GP via induced prior approximation. GP-NAM applies the additive model with RFF approximation but lacks Bayesian inference. \citet{harrison2024variational} introduced a sampling-free Bayesian last-layer architecture but did not establish a connection between this architecture and DKL. %In our work, we approximate GPs using additive structure and sparsely induced priors to enhance the computational efficiency of DKL, which results in a last-layer BNN that is both efficient and easily adaptable to various tasks. The resulting GP layer has the form of a last-layer BNN which is easy to extend to various tasks and efficient in computing and inference.
% \textcolor{brown}{Tian: The last sentence may not be necessary, since we also highlight it in the remark.}


\section{EXPERIMENTS}
\label{sec:exp}

In this section, we evaluate the proposed DAK model on multiple real datasets for both regression tasks in \Cref{subsec:uci reg} and classification tasks in \Cref{subsec:image class}. We compare its performance with several baselines, including a neural network without GP integration (NN), DKL with SVGP \citep{titsias2009variational,hensman2015scalable} as the GP approximation (NN+SVGP), and SV-DKL \citep{wilson2016stochastic}. Furthermore, in \Cref{subsec:toy}, we present a toy example that demonstrates how the proposed model mitigates the issue of out-of-sample overfitting highlighted by \citet{ober2021promises}, in contrast to the NN model. 
% Source code is available at the following link \footnote{\url{https://github.com/warrenzha/dak2bnn}}.


\subsection{Toy Example: GP}
\label{subsec:toy}

\begin{table*}[ht]
\centering
\caption{\small{Comparison of regression performance on UCI datasets using 5-fold cross-validation with a batch size of 512 and a fully connected neural network architecture of $D \rightarrow 64 \rightarrow 32 \rightarrow D_{w}$, where the output features $D_{w}$ are 16, 64, and 256 respectively. The best results are highlighted in \textbf{bold}. Our models, DAK-MC (using MC approximation) and DAK-CF (using closed-form inference and ELBO), are highlighted with  \protect\colorbox{Gray}{gray background}.}}
\label{tab:uci metrics}
\vspace{-0.2cm}
\resizebox{\linewidth}{!}{%
\begin{tabular}{l|l|ccc|ccc|ccc}
\toprule[1pt]
% \multirow{3}{*}{\makecell[tl]{\textbf{Dataset} \\ $(N, D)$ }} & \multirow{3}{*}{\textbf{Model}} & \multicolumn{9}{c}{\textbf{Batch size = 512}} \\ \cline{3-11} &
\multirow{2}{*}{\makecell[tl]{\textbf{Dataset} \\ $(N, D)$ }}  & \multirow{2}{*}{\textbf{Model}}

& \multicolumn{3}{c|}{\textbf{NN out features = 16}} & \multicolumn{3}{c|}{\textbf{NN out features = 64}} & \multicolumn{3}{c}{\textbf{NN out features = 256}} \\
\cline{3-11}
&  & \textbf{RMSE} $\downarrow$ & \textbf{NLPD} $\downarrow$ & \textbf{Time (s)} $\downarrow$  & \textbf{RMSE} $\downarrow$ & \textbf{NLPD} $\downarrow$ & \textbf{Time (s)} $\downarrow$  & \textbf{RMSE} $\downarrow$ & \textbf{NLPD} $\downarrow$ & \textbf{Time (s)} $\downarrow$ \\
\hline
\multirow{5}{*}{ \makecell[tl]{ \textbf{Gas} \\ $(2565, 128)$} } & NN & 2.377 $\pm$ 2.399 & 4.749 $\pm$ 6.594 & \textbf{2.345 $\pm$ 0.003} & 2.107 $\pm$ 1.212 & 3.092 $\pm$ 2.229 & \textbf{2.362 $\pm$ 0.029} & 1.196 $\pm$ 0.712 & 1.730 $\pm$ 0.711 & \textbf{2.346 $\pm$ 0.004} \\
& NN+SVGP & 0.502 $\pm$ 0.171 & 1.121 $\pm$ 0.106 & 4.727 $\pm$ 0.009 & 0.625 $\pm$ 0.148 & 1.206 $\pm$ 0.119 & 4.724 $\pm$ 0.004 & 0.743 $\pm$ 0.183 & 1.322 $\pm$ 0.143 & 4.718 $\pm$ 0.009 \\
& SV-DKL & 0.589 $\pm$ 0.161 & 1.303 $\pm$ 0.227 & 28.189 $\pm$ 0.490 & 0.499 $\pm$ 0.183 & 1.235 $\pm$ 0.256 & 27.956 $\pm$ 0.078 & 0.534 $\pm$ 0.189 & 1.207 $\pm$ 0.209 & 28.400 $\pm$ 0.185 \\
& AV-DKL &  0.538 $\pm$ 0.129  &  1.250 $\pm$ 0.222  &  25.315 $\pm$ 0.254  & 0.604 $\pm$ 0.016  &  1.250 $\pm$ 0.010  &  27.070 $\pm$ 1.620 &  0.760 $\pm$ 0.367  &  1.276 $\pm$ 0.220  &  28.860 $\pm$ 1.903  \\
& \cellcolor{Gray}\raggedright DAK-MC & \cellcolor{Gray} \textbf{0.405 $\pm$ 0.061} & 
\cellcolor{Gray} \textbf{0.886 $\pm$ 0.048} & \cellcolor{Gray} 8.887 $\pm$ 0.007 & \cellcolor{Gray} 0.353 $\pm$ 0.046 & \cellcolor{Gray} \textbf{0.881 $\pm$ 0.053} & 
\cellcolor{Gray} 8.844 $\pm$ 0.005 & \cellcolor{Gray} 0.351 $\pm$ 0.019 & \cellcolor{Gray} \textbf{0.871 $\pm$ 0.027} & 
\cellcolor{Gray} 8.831 $\pm$ 0.017 \\
& \cellcolor{Gray}\raggedright DAK-CF & \cellcolor{Gray} 0.412 $\pm$ 0.134 & \cellcolor{Gray} 0.928 $\pm$ 0.100 & \cellcolor{Gray} 7.400 $\pm$ 0.004 & \cellcolor{Gray} \textbf{0.350 $\pm$ 0.020} & 
\cellcolor{Gray} 0.898 $\pm$ 0.040 & \cellcolor{Gray} 7.398 $\pm$ 0.009 & \cellcolor{Gray} \textbf{0.342 $\pm$ 0.033} & 
\cellcolor{Gray} 0.895 $\pm$ 0.046 & \cellcolor{Gray} 7.410 $\pm$ 0.009 \\
\hline
\multirow{5}{*}{ \makecell[tl]{ \textbf{Parkinsons} \\ $(5875, 20)$ } } & NN & 2.692 $\pm$ 1.302 & 3.503 $\pm$ 2.630 & \textbf{2.693 $\pm$ 0.027} & 2.288 $\pm$ 0.712 & 2.724 $\pm$ 1.158 & \textbf{2.589 $\pm$ 0.050} & 2.252 $\pm$ 0.757 & 2.720 $\pm$ 1.188 & \textbf{2.885 $\pm$ 0.026} \\
& NN+SVGP & 3.481 $\pm$ 1.906 & 4.606 $\pm$ 4.606 & 5.516 $\pm$ 0.117 & 3.238 $\pm$ 2.419 & 4.940 $\pm$ 5.494 & 5.467 $\pm$ 0.097 & 3.676 $\pm$ 3.287 & 6.791 $\pm$ 8.612 & 5.945 $\pm$ 0.090 \\
& SV-DKL & 2.608 $\pm$ 1.023 & 2.745 $\pm$ 1.097 & 35.804 $\pm$ 1.590 & 2.817 $\pm$ 1.670 & 3.193 $\pm$ 2.109 & 33.042 $\pm$ 0.306 & 2.896 $\pm$ 2.055 & 3.206 $\pm$ 1.906 & 32.882 $\pm$ 0.060 \\
& AV-DKL &  1.942 $\pm$ 0.758  &  \textbf{2.223 $\pm$ 0.655} &  30.337 $\pm$ 1.086  &  2.267 $\pm$ 0.584  &  \textbf{2.397 $\pm$ 0.628} & 31.006 $\pm$ 1.162 & 3.096 $\pm$ 0.472 & 3.105 $\pm$ 0.793 &  31.859 $\pm$ 1.197 \\
& \cellcolor{Gray}\raggedright DAK-MC & \cellcolor{Gray} 1.983 $\pm$ 1.154 & \cellcolor{Gray} 2.699 $\pm$ 1.819 & \cellcolor{Gray} 11.596 $\pm$ 0.260 & \cellcolor{Gray} 1.949 $\pm$ 0.912 & \cellcolor{Gray} 2.575 $\pm$ 1.121 & \cellcolor{Gray} 13.085 $\pm$ 0.055 & \cellcolor{Gray} 1.846 $\pm$ 0.974 & \cellcolor{Gray} 2.420 $\pm$ 1.212 & \cellcolor{Gray} 11.820 $\pm$ 0.296 \\
& \cellcolor{Gray}\raggedright DAK-CF & \cellcolor{Gray} \textbf{1.801 $\pm$ 1.013} & \cellcolor{Gray} 2.848 $\pm$ 1.810 & \cellcolor{Gray} 9.071 $\pm$ 0.083 & \cellcolor{Gray} \textbf{1.788 $\pm$ 0.997} & \cellcolor{Gray} 2.801 $\pm$ 1.853 & \cellcolor{Gray} 9.073 $\pm$ 0.048 & \cellcolor{Gray} \textbf{1.466 $\pm$ 1.093} & \cellcolor{Gray} \textbf{2.308 $\pm$ 1.892} & \cellcolor{Gray} 8.166 $\pm$ 0.071 \\
\hline
\multirow{5}{*}{ \makecell[tl]{\textbf{Wine} \\ $(1599, 11)$} } & NN & 0.728 $\pm$ 0.055 & 1.233 $\pm$ 0.057 & \textbf{2.350 $\pm$ 0.011} & 0.725 $\pm$ 0.057 & 1.231 $\pm$ 0.054 & \textbf{2.343 $\pm$ 0.007} & 0.712 $\pm$ 0.062 & 1.214 $\pm$ 0.062 & \textbf{2.367 $\pm$ 0.021} \\
& NN+SVGP & 0.739 $\pm$ 0.069 & 1.243 $\pm$ 0.067 & 4.713 $\pm$ 0.010 & 0.801 $\pm$ 0.130 & 1.325 $\pm$ 0.130 & 4.725 $\pm$ 0.008 & 0.893 $\pm$ 0.180 & 1.418 $\pm$ 0.147 & 4.718 $\pm$ 0.016 \\
& SV-DKL & 0.899 $\pm$ 0.110 & 1.443 $\pm$ 0.135 & 27.224 $\pm$ 0.170 & 0.947 $\pm$ 0.109 & 1.460 $\pm$ 0.112 & 27.136 $\pm$ 0.023 & 0.879 $\pm$ 0.130 & 1.434 $\pm$ 0.158 & 27.332 $\pm$ 0.101 \\
& AV-DKL & \textbf{0.699 $\pm$ 0.060} & 1.207 $\pm$ 0.068 & 24.419 $\pm$ 0.316 & \textbf{0.711 $\pm$ 0.055} & 1.221 $\pm$ 0.060 & 24.904 $\pm$ 0.424 & \textbf{0.711 $\pm$ 0.055} & 1.221 $\pm$ 0.060 & 24.904 $\pm$ 0.424 \\
& \cellcolor{Gray}\raggedright DAK-MC & \cellcolor{Gray} 0.756 $\pm$ 0.068 & \cellcolor{Gray} 1.164 $\pm$ 0.075 & \cellcolor{Gray} 8.820 $\pm$ 0.038 & \cellcolor{Gray} 0.751 $\pm$ 0.055 & \cellcolor{Gray} 1.163 $\pm$ 0.060 & \cellcolor{Gray} 8.821 $\pm$ 0.045 & \cellcolor{Gray} 0.727 $\pm$ 0.065 & \cellcolor{Gray} \textbf{1.140 $\pm$ 0.070} & \cellcolor{Gray} 8.765 $\pm$ 0.016 \\
& \cellcolor{Gray}\raggedright DAK-CF & \cellcolor{Gray}  0.736 $\pm$ 0.042 & \cellcolor{Gray} \textbf{1.162 $\pm$ 0.049} & \cellcolor{Gray} 7.376 $\pm$ 0.018 & \cellcolor{Gray} 0.728 $\pm$ 0.064 & \cellcolor{Gray} \textbf{1.153 $\pm$ 0.067} & 
\cellcolor{Gray} 7.352 $\pm$ 0.010 & 
\cellcolor{Gray} 0.720 $\pm$ 0.057 & \cellcolor{Gray} 1.147 $\pm$ 0.061 & 
\cellcolor{Gray} 7.415 $\pm$ 0.029 \\
\hline
\multirow{5}{*}{ \makecell[tl]{ \textbf{Kin40K} \\ $(40000, 8)$ }} & NN & 0.109 $\pm$ 0.016 & 0.752 $\pm$ 0.004 & \textbf{16.017 $\pm$ 0.114} & 0.100 $\pm$ 0.012 & 0.749 $\pm$ 0.003 & \textbf{16.912 $\pm$ 0.112} & 0.087 $\pm$ 0.016 & 0.746 $\pm$ 0.004 & \textbf{18.048 $\pm$ 0.025} \\
& NN+SVGP & 0.104 $\pm$ 0.013 & 0.751 $\pm$ 0.004 & 40.749 $\pm$ 0.508 & 0.142 $\pm$ 0.017 & 0.763 $\pm$ 0.006 & 41.239 $\pm$ 0.304 & 0.123 $\pm$ 0.016 & 0.757 $\pm$ 0.006 & 43.111 $\pm$ 1.482 \\
& SV-DKL & 0.096 $\pm$ 0.024 & 0.750 $\pm$ 0.010 & 230.284 $\pm$ 7.049 & 0.092 $\pm$ 0.018 & 0.748 $\pm$ 0.006 & 228.137 $\pm$ 7.406 & 0.097 $\pm$ 0.025 & 0.750 $\pm$ 0.009 & 226.606 $\pm$ 4.231 \\
& AV-DKL & 0.084 $\pm$ 0.011 & 0.746 $\pm$ 0.003 & 75.427 $\pm$ 0.965 &  0.081 $\pm$ 0.014 & 0.745 $\pm$ 0.003 & 87.654 $\pm$ 1.664 & 0.071 $\pm$ 0.015 & 0.743 $\pm$ 0.029 & 158.860 $\pm$ 3.621 \\
& \cellcolor{Gray}\raggedright DAK-MC & \cellcolor{Gray} 0.096 $\pm$ 0.042 & \cellcolor{Gray} 0.746 $\pm$ 0.010 & \cellcolor{Gray} 74.787 $\pm$ 0.383 & \cellcolor{Gray} 0.090 $\pm$ 0.028 & \cellcolor{Gray} 0.744 $\pm$ 0.006 & \cellcolor{Gray} 77.349 $\pm$ 5.409 & \cellcolor{Gray} 0.090 $\pm$ 0.031 & \cellcolor{Gray} 0.744 $\pm$ 0.007 & \cellcolor{Gray} 75.586 $\pm$ 0.313 \\
& \cellcolor{Gray}\raggedright DAK-CF & \cellcolor{Gray} \textbf{0.073 $\pm$ 0.015} & \cellcolor{Gray} \textbf{0.742 $\pm$ 0.005} & \cellcolor{Gray} 54.174 $\pm$ 0.322 & \cellcolor{Gray} \textbf{0.069 $\pm$ 0.018} & \cellcolor{Gray} \textbf{0.741 $\pm$ 0.005} & \cellcolor{Gray} 60.339 $\pm$ 0.161 & \cellcolor{Gray} \textbf{0.068 $\pm$ 0.015} & \cellcolor{Gray} \textbf{0.741 $\pm$ 0.004} & \cellcolor{Gray} 55.148 $\pm$ 0.165 \\
\hline
\multirow{5}{*}{ \makecell[tl]{ \textbf{Protein} \\ $(45730, 9)$} } & NN & 4.678 $\pm$ 7.804 & 25.091 $\pm$ 52.135 & \textbf{24.910 $\pm$ 0.221} & 0.906 $\pm$ 0.304 & 1.421 $\pm$ 0.232 & \textbf{19.433 $\pm$ 0.093} & 1.403 $\pm$ 1.408 & 2.358 $\pm$ 2.349 & \textbf{28.447 $\pm$ 0.133} \\
& NN+SVGP & 0.773 $\pm$ 0.003 & 1.278 $\pm$ 0.002 & 47.081 $\pm$ 0.643 & 0.773 $\pm$ 0.003 & 1.278 $\pm$ 0.002 & 38.986 $\pm$ 0.485 & 0.773 $\pm$ 0.003 & 1.278 $\pm$ 0.002 & 52.924 $\pm$ 1.144 \\
& SV-DKL & 0.769 $\pm$ 0.003 & 1.275 $\pm$ 0.001 & 262.274 $\pm$ 8.216 & 0.770 $\pm$ 0.003 & 1.276 $\pm$ 0.002 & 265.182 $\pm$ 3.680 & 0.769 $\pm$ 0.003 & 1.275 $\pm$ 0.001 & 264.678 $\pm$ 4.450 \\
& AV-DKL & 0.773 $\pm$ 0.003 & 1.278 $\pm$ 0.002 & 87.939 $\pm$ 1.303 &  0.773 $\pm$ 0.003 & 1.278 $\pm$ 0.002 & 92.591 $\pm$ 1.934 & 0.773 $\pm$ 0.003 & 1.278 $\pm$ 0.002 & 164.997 $\pm$ 4.108 \\
& \cellcolor{Gray}\raggedright DAK-MC & \cellcolor{Gray} \textbf{0.622 $\pm$ 0.047} & \cellcolor{Gray} \textbf{1.017 $\pm$ 0.043} & \cellcolor{Gray} 85.729 $\pm$ 1.035 & \cellcolor{Gray} \textbf{0.612 $\pm$ 0.038} & \cellcolor{Gray} \textbf{1.008 $\pm$ 0.035} & \cellcolor{Gray} 86.594 $\pm$ 0.827 & \cellcolor{Gray} \textbf{0.608 $\pm$ 0.042} & \cellcolor{Gray} \textbf{1.004 $\pm$ 0.039} & \cellcolor{Gray} 86.245 $\pm$ 0.088 \\
& \cellcolor{Gray}\raggedright DAK-CF & \cellcolor{Gray} 0.631 $\pm$ 0.021 & \cellcolor{Gray} 1.024 $\pm$ 0.019 & \cellcolor{Gray} 68.762 $\pm$ 0.379 & \cellcolor{Gray} 0.625 $\pm$ 0.027 & \cellcolor{Gray} 1.018 $\pm$ 0.026 & \cellcolor{Gray} 69.766 $\pm$ 0.190 & \cellcolor{Gray} 0.618 $\pm$ 0.029 & \cellcolor{Gray} 1.012 $\pm$ 0.026 & \cellcolor{Gray} 68.783 $\pm$ 0.222 \\
\hline
\multirow{5}{*}{ \makecell[tl]{ \textbf{KEGG} \\ $(48827, 20)$} } & NN & 0.132 $\pm$ 0.021 & 0.759 $\pm$ 0.006 & \textbf{21.856 $\pm$ 0.168} & 0.124 $\pm$ 0.009 & 0.758 $\pm$ 0.004 & \textbf{20.989 $\pm$ 0.132} & 0.127 $\pm$ 0.008 & 0.766 $\pm$ 0.019 & \textbf{23.143 $\pm$ 0.710} \\
& NN+SVGP & 0.128 $\pm$ 0.005 & 0.758 $\pm$ 0.001 & 43.788 $\pm$ 1.377 & 0.129 $\pm$ 0.013 & 0.759 $\pm$ 0.002 & 40.996 $\pm$ 0.280 & 0.127 $\pm$ 0.009 & 0.758 $\pm$ 0.002 & 46.882 $\pm$ 1.290 \\
& SV-DKL & 0.134 $\pm$ 0.011 & 0.766 $\pm$ 0.013 & 269.598 $\pm$ 5.733 & 0.139 $\pm$ 0.031 & 0.769 $\pm$ 0.025 & 271.083 $\pm$ 7.502 & 0.152 $\pm$ 0.024 & 0.780 $\pm$ 0.025 & 275.013 $\pm$ 2.443 \\
& AV-DKL & 0.127 $\pm$ 0.008 & 0.758 $\pm$ 0.002 & 92.000 $\pm$ 0.227 & 0.130 $\pm$ 0.009 & 0.759 $\pm$ 0.002 & 129.314 $\pm$ 1.607 & 0.130 $\pm$ 0.008 & 0.759 $\pm$ 0.002 & 254.537 $\pm$ 4.020 \\
& \cellcolor{Gray}\raggedright DAK-MC & \cellcolor{Gray} 0.126 $\pm$ 0.010 & \cellcolor{Gray} \textbf{0.748 $\pm$ 0.003} & \cellcolor{Gray} 90.473 $\pm$ 0.144 & \cellcolor{Gray} 0.124 $\pm$ 0.010 & \cellcolor{Gray} \textbf{0.748 $\pm$ 0.003} & \cellcolor{Gray} 92.507 $\pm$ 0.964 & \cellcolor{Gray} 0.123 $\pm$ 0.009 & \cellcolor{Gray} \textbf{0.748 $\pm$ 0.003} & \cellcolor{Gray} 113.738 $\pm$ 1.251 \\
& \cellcolor{Gray}\raggedright DAK-CF & \cellcolor{Gray} \textbf{0.124 $\pm$ 0.007} & \cellcolor{Gray} \textbf{0.748 $\pm$ 0.003} & \cellcolor{Gray} 65.070 $\pm$ 0.140 & \cellcolor{Gray} \textbf{0.122 $\pm$ 0.007} & \cellcolor{Gray} \textbf{0.748 $\pm$ 0.003} & \cellcolor{Gray} 79.732 $\pm$ 2.189 & \cellcolor{Gray} \textbf{0.121 $\pm$ 0.005} & \cellcolor{Gray} \textbf{0.748 $\pm$ 0.003} & \cellcolor{Gray} 78.702 $\pm$ 0.631 \\
\bottomrule[1pt]
\end{tabular}%
}
% \vspace{-0.2cm}
\end{table*}


We first evaluate a toy example of 1-Dimensional (1D) GP regression on synthetic data with zero mean and SE kernel $k(x,x')=\exp\left( -(x-x')^2 \right)$. The training set consists of $20$ noisy data points in the range of $[-7,7]$, while the test set consists of $100$ data points in $[-12,12]$. We consider a two-layer MLP with layer width $[64,32]$ as the feature extractor, letting $P=2$ be the number of \emph{base} GPs. To maintain a fair comparison, we use the same training recipe: full-batch training, a learning rate 0.01, and a number of optimization steps 1000. The only differences are the choice of the last layer and the loss function. We set the number of MC samples $S=4$ during training.

\Cref{fig:gp1d} -- \ref{fig:dak1d} shows the predictive posterior of the exact GP, two-layer DGP, exact DKL, and proposed DAK. We observe that DAK has good both in-sample and out-of-sample predictions, which is close to exact GP posterior, while DGP is unbiased but over-confident outside the training data. Exact DKL is biased and overconfident in out-of-sample predictions, which were also observed and investigated by \citet{ober2021promises}. We also compare DAK with the DNN that shares the same feature extractor and model depth. It is evident that the DNN fit in \Cref{fig:nn1d} suffers from ``overfitting'': the fit shows significant extrapolation beyond the training data. 


\subsection{UCI Regression}
\label{subsec:uci reg}

We benchmark the regression task on six datasets from the UCI repository \citep{dua2017uci}: three smaller datasets with fewer than 10K samples (Wine, Gas, Parkinsons) and three larger datasets with 40K to 50K samples (Kin40K, Protein, KEGG). All models use the same neural network architecture: a fully connected network with two hidden layers of 64 and 32 neurons, respectively. Training is performed using the Adam optimizer over 100 epochs with a learning rate of $0.001$, weight decay of $0.0005$, and batch size of 512. The noise variance is set to $\sigma_{f}^2 = 0.01$.

For NN+SVGP, SV-DKL, and AV-DKL, we use SE kernels and 64 inducing points initialized with uniformly distributed random variables of a fixed seed. In the proposed DAK models, the induced grid $\Uv$ from \cref{eq:GPlayer} consists of $M = 7$ equally spaced points over the interval $(0,1)$ for each base GP, i.e., $\Uv = \{1/8, 2/8, \ldots, 7/8\}$. The NN outputs of both SV-DKL and our DAK model, which utilize an additive GP structure, are embedded and normalized into 16 base GPs over the interval $[0,1]$. That is, the number of projections $P$ in \cref{eq:deepadditive} is set to 16. For the DAK model, we evaluate two methods: DAK-CF, using the closed-form inference from \Cref{sec:uq of inference} and the closed-form ELBO from \Cref{sec:elbo}, and DAK-MC, using MC sampling to approximate the mean and variance during inference and approximate the expected log likelihood term in ELBO during training. The number of MC samples is set to 20 for inference and 8 for training, the same for NN+SVGP, SV-DKL, and AV-DKL. Further details can be found in \Cref{subsec:regression supp}.

We evaluate the performance of each model using training time (in seconds), Root Mean Squared Error (RMSE), and Negative Log Predictive Density (NLPD) with varying neural network output feature sizes of 16, 64, and 256. The results are averaged over 5-fold cross-validation for each dataset. As shown in \Cref{tab:uci metrics}, except for the smallest dataset, Wine, our models, DAK-MC and DAK-CF (highlighted in gray), consistently achieve the best RMSE and NLPD performance compared to other models. 

\begin{table*}[ht]
\caption{\small{ELBO, Accuracy, NLL, ECE for DAK, SV-DKL, NN+SVGP, NN on image classification tasks averaged over 3 runs. MNIST uses a simple CNN with 16 output features; CIFAR-10 uses ResNet-18 with 64 output features; CIFAR-100 uses ResNet-34 with 512 output features. The best results are highlighted in \textbf{bold}.}}
\centering
\vspace{-0.1cm}
\resizebox{\linewidth}{!}{%
\begin{tabular}{rcccclccc}
\toprule[1pt]
\multicolumn{1}{l}{} & \multicolumn{4}{c}{Batch size: 128}  &  & \multicolumn{3}{c}{Batch size: 1024} \\ \cline{2-5} \cline{7-9} \vspace{-8pt} \\
\multicolumn{1}{l}{} & NN   & NN+SVGP & SV-DKL & \cellcolor{Gray} DAK-MC &   & NN+SVGP  & SV-DKL & \cellcolor{Gray} DAK-MC \\ 
\midrule[1pt]
MNIST - ELBO $\uparrow$        & \rule{0.5cm}{0.4pt}  &  -0.121 $\pm$ 0.000    & -0.054 $\pm$ 0.000  & \cellcolor{Gray} \textbf{-0.030 $\pm$ 0.001}   &     & -1.997 $\pm$ 0.001        &  -0.305 $\pm$ 0.001      &  \cellcolor{Gray} \textbf{-0.048 $\pm$ 0.001}  \\
Top-1 Acc. (\%) $\uparrow$       & 99.14 $\pm$ 0.00     & 98.56 $\pm$ 0.02       & 99.16 $\pm$ 0.00  & \cellcolor{Gray} \textbf{99.26 $\pm$ 0.01}      &      & 14.79 $\pm$ 0.08         & 96.23 $\pm$ 0.02       &  \cellcolor{Gray} \textbf{99.1 $\pm$ 0.00}    \\
NLL $\downarrow$             & 0.026 $\pm$ 0.000    & 0.064 $\pm$ 0.001     & 0.030 $\pm$ 0.000      & \cellcolor{Gray} \textbf{0.024 $\pm$ 0.000}   &    & 1.985 $\pm$ 0.003        &  0.288 $\pm$ 0.001      &  \cellcolor{Gray} \textbf{0.028 $\pm$ 0.004}      \\
ECE $\downarrow$          & 0.005 $\pm$ 0.000     & 0.012 $\pm$ 0.000    & 0.005 $\pm$ 0.000       & \cellcolor{Gray} \textbf{0.004 $\pm$ 0.000}           &      & 0.062 $\pm$ 0.002         & 0.020 $\pm$ 0.000       & \cellcolor{Gray} \textbf{0.006 $\pm$ 0.000} \\
\midrule[1pt]
CIFAR-10 - ELBO $\uparrow$      & \rule{0.5cm}{0.4pt}     & -1.038 $\pm$ 0.004        & -0.017 $\pm$  0.001      &  \cellcolor{Gray} \textbf{-0.002 $\pm$ 0.000}     &     & -1.039 $\pm$ 0.004         & -0.034 $\pm$ 0.021       & \cellcolor{Gray} \textbf{-0.002 $\pm$ 0.000} \\
Top-1 Acc. (\%) $\uparrow$    & 94.72 $\pm$ 0.13   & 77.35 $\pm$ 0.13  & 93.44 $\pm$ 0.28    &  \cellcolor{Gray} \textbf{94.81 $\pm$ 0.13}   &     &  16.90 $\pm$ 0.29        & 90.22 $\pm$ 1.42       & \cellcolor{Gray} \textbf{93.02 $\pm$ 0.18}        \\
NLL $\downarrow$     & \textbf{0.252 $\pm$ 0.025}      & 1.790 $\pm$ 0.001    & 0.312 $\pm$ 0.033       &  \cellcolor{Gray} 0.256 $\pm$ 0.014     &      & 2.270 $\pm$ 0.000         & 0.485 $\pm$ 0.061       & \cellcolor{Gray} \textbf{0.345 $\pm$ 0.001}    \\
ECE $\downarrow$      & 0.040 $\pm$ 0.002     & 0.061 $\pm$ 0.003    & 0.046 $\pm$ 0.003       &  \cellcolor{Gray} \textbf{0.039 $\pm$ 0.002}          &     & \textbf{0.027 $\pm$ 0.002}       & 0.060 $\pm$ 0.004       & \cellcolor{Gray} 0.052 $\pm$ 0.001           \\
\midrule[1pt]
CIFAR-100 - ELBO $\uparrow$      & \rule{0.5cm}{0.4pt}     & -4.605 $\pm$ 0.000        & -0.059 $\pm$ 0.000       &  \cellcolor{Gray} \textbf{-0.003 $\pm$ 0.000}      &     & -4.605 $\pm$ 0.000         & -0.103 $\pm$ 0.009       & \cellcolor{Gray} \textbf{-0.005 $\pm$ 0.001}   \\
Top-1 Acc. (\%) $\uparrow$    & 75.88 $\pm$ 0.54   & 1.04 $\pm$ 0.01  & 74.52 $\pm$ 0.13       & \cellcolor{Gray}  \textbf{76.75 $\pm$ 0.18}     &     &  1.10 $\pm$ 0.09        & 66.54 $\pm$ 0.74       & \cellcolor{Gray} \textbf{70.38 $\pm$ 1.25}        \\
NLL $\downarrow$     & 1.018 $\pm$ 0.021      & 4.605 $\pm$ 0.000    & 1.041 $\pm$ 0.007       & \cellcolor{Gray}  \textbf{1.001 $\pm$ 0.027}     &      & 4.605 $\pm$ 0.000    & 1.738 $\pm$  0.058      & \cellcolor{Gray} \textbf{1.203 $\pm$ 0.040}        \\
ECE $\downarrow$      & 0.086 $\pm$ 0.002     & \textbf{0.003 $\pm$ 0.001}    & 0.049 $\pm$ 0.002       & \cellcolor{Gray}  0.041 $\pm$ 0.004        &     & \textbf{0.003 $\pm$ 0.001}         & 0.148 $\pm$ 0.007       &\cellcolor{Gray}  0.056 $\pm$ 0.006           \\
\bottomrule[1pt]
\end{tabular}
}
% \vspace{-0.2cm}
\label{tab:img metrics}
\end{table*}

Although NN+SVGP takes less time than DAK models, its performance often degrades, with higher RMSE and NLPD as the size of the NN output features increases. In contrast, the DAK models show improved performance with larger NN output features. This is because SVGP has a fixed number of 64 inducing points, which limits its ability to approximate GPs effectively in high-dimensional spaces, making it less suitable for complex tasks requiring high-dimensional NNs, such as multitask regression or meta-learning.

Additionally, SV-DKL, which also uses an additive GP layer and KISS-GP to accelerate GP computations, takes significantly more time than our DAK models. This is because SV-DKL treats dependent inducing variables as parameters in VI, which requires more training time, while our DAK models treat the GP layers as sparse BNNs with independent Gaussian weights and biases under the mean-field assumption, leading to more efficient training. 

While AV-DKL occasionally achieves higher accuracy, it also requires substantially more training time than our DAK methods. By using inducing locations dependent on NN outputs to mitigate overcorrelation in DKL, AV-DKL significantly increases the ELBO’s computational complexity. Similar to NN+SVGP, its performance degrades when the size of the NN output features grows, because GP approximation in higher-dimensional spaces would naturally require a corresponding increase in the number of inducing points.


\subsection{Image Classification}
\label{subsec:image class}

\begin{table*}[ht]
\caption{\small{The number of total trainable parameters and average training time per epoch across different tasks for each model. NN has smaller set of parameters and less training time as is expected. DAK is more scalable than SV-DKL in terms of total trainable parameters and training time per epoch.}}
\centering
\vspace{-0.1cm}
\resizebox{\linewidth}{!}{%
\begin{tabular}{lllllccccccccc}
\toprule[1pt]
\multirow{2}{*}{Datasets} & \multirow{2}{*}{$N$} & \multirow{2}{*}{$D$} & \multirow{2}{*}{$C$} & \multirow{2}{*}{$D_w$} & \multicolumn{4}{c}{\# parameters} &  & \multicolumn{4}{c}{Epoch time (sec.)} \\ \cline{6-9} \cline{11-14} \vspace{-8pt} \\
                          &                    &                    &                    &                    & NN  & NN+SVGP  & SV-DKL  & \cellcolor{Gray} DAK-MC &  & NN & NN+SVGP & SV-DKL & \cellcolor{Gray} DAK-MC \\
\midrule[1pt]
MNIST                     & 60K                & 28$\times$28                 & 10                 & 16                 & 1.19M    &  $+$2.72M        &  $+$0.08M       & \cellcolor{Gray}  \textbf{$+$0.03M}      &  & 6.18   &  $+$16.57       &  \textbf{$+$4.35}      & \cellcolor{Gray}  $+$4.57      \\
CIFAR-10                  & 50K                & 3$\times$32$\times$32                 & 10                 & 64                 & 11.17M    &  $+$29.58M        & $+$0.30M        & \cellcolor{Gray}  \textbf{$+$0.11M}      &  & 34.41   &  $+$22.92       &  $+$7.88      & \cellcolor{Gray}  \textbf{$+$5.02}      \\
CIFAR-100                 & 50K                & 3$\times$32$\times$32                 & 100                & 512                & 21.32M    &  $+$52.53M        &  $+$2.19M       & \cellcolor{Gray}  \textbf{$+$1.68M}     &  & 42.82   &  $+$101.77       & $+$16.61       & \cellcolor{Gray}  \textbf{$+$6.24}     \\
\bottomrule[1pt]
\end{tabular}
}
% \vspace{-0.2cm}
\label{tab:runtime}
\end{table*}

We next benchmark the classification task on high-dimensional and highly structured image data, including MNIST \citep{lecun1998mnist}, CIFAR-10, and CIFAR-100 \citep{krizhevsky2009learning}. All models share the same neural network head as feature extractors, to which we add either a linear output layer in NN model or corresponding GP output layers in NN+SVGP, SV-DKL, and DAK. In classification, last layer outputs are followed by a Softmax layer to normalize the output to a probability distribution, and we perform MC sampling to approximate the Softmax likelihood term in ELBO. 

We use the same setting of training for all models (refer to \Cref{tab:optimizer classification} in \Cref{subsec:classification supp} for details). For MNIST, we consider a simple CNN as the feature extractor, 20 epochs of training using Adadelta optimizer with initial learning rate 1.0, weight decay 0.0001, and annealing learning rate scheduler at each step with a factor of 0.7. For CIFAR-10, we perform full-training on a ResNet-18 \citep{he2016deep} with GP layers for 200 epochs, while for CIFAR-100, we use a pre-trained ResNet-34 as the feature extractor and fine-tune GP layers for 200 epochs since NN+SVGP and SV-DKL struggled to fit. In both CIFAR10/100, we use SGD optimizer with an initial learning rate of 0.1, weight decay of 0.0001, momentum 0.9, and cosine annealing learning rate scheduler. 

For NN+SVGP, we use SE kernel and $512$ inducing points, while for SV-DKL, we use $64$ inducing points initialized with uniformly distributed random variable within the interval $[-1,1]^{D_w}$. For the proposed DAK-MC, we use $M=63$ equally spaced points over the interval $[-1,1]$ for the induced grid $\Uv$ in each base GP. More details can be found in \Cref{subsec:classification supp}.

We evaluate all models in terms of Top-1 accuracy, NLL, ELBO, and expected calibration error (ECE) over three runs. As shown in \Cref{tab:img metrics}, our DAK-MC (highlighted in gray) achieves the best accuracy, ELBO, and NLL performance compared to other DKL models. Although some ECEs of NN+SVGP and SV-DKL are lower than those of DAK-MC, their accuracy degrades more with increasing dimensionality. The failure of SVGP in CIFAR-10/100 demonstrated the necessity of additive structure in high-dimensional multitask DKL. Additionally, we observe that SVDKL struggles more to fit in CIFAR-100, indicating the importance of pre-training in SVDKL, while our proposed DAK does not hurt by the curse of dimensionality because of the computational advantages of the last-layer BNN feature. We also repeated the experiments with a larger batch size of 1024. Our proposed DAK model is more robust than other DKL methods when the batch size and the number of features increases. Further experimental results are provided in \Cref{sec:additional exp results}. In \Cref{tab:runtime}, we measure the total number of trainable parameters and the average training time. NN has the smallest set of parameters and the least training time as expected. Among DKL methods, DAK is more efficient than SV-DKL when large-scale neural networks and high-dimensional tasks are applied.


% 1 to show, 0 to hide
% \ifthenelse{\equal{\showcontent}{0}}{
% \section{Discussions of Weekly Meetings}
% {\color{red}{
% \subsection{Sep 2.}
% \begin{enumerate}
%     \item Motivation: DKL -> GP -> Approximation -> Hybrid NN
%     \item add a table of comparison of time complexity: DKL with KISS-GP, DKL with additive sparse approximation (ours) (Will be done by Haoyuan)
%     \item add "related work" section, find the latest papers about Deep kernel learning on how to optimize the hyperparameters, whether write  GP layer as a Bayesian Neural Network (if there exist some papers that represent NN+GP as an entire NN). If not, it can be a motivation 
%     \item do we need to do Monte Carlo sampling when computing the predictive distribution computation through \Cref{eq:additiveDKL} and log-likelihood in ELBO in \Cref{eq:elbo}? 
% \end{enumerate}


% \subsection{Sep 9.}
% \begin{enumerate}
%     \item Table 1, we didn't show that KISS-GP scales up with the dimension $D$, should we quantify the performance via some simulations? If we fix the number of inducing points $M$, the performance of KISS-GP will be worse. Remove Table 1, discuss both the complexity and the performance. (will be done by Haoyuan)
%     \item Should we implement closed-form of ELBO in the experiment. Will using closed-form improve largely the computational time than Monte Carlo samplings? Make a comparison between with Monte Carlo sampling and without Monte Carlo sampling in the experiment. 
%     \item Experiments better before Sep 16
%     \begin{enumerate}
%         \item comparison of runtimes of for computing the ELBO between with Monte Carlo sampling and without Monte Carlo sampling  (Wenyuan)
%         \item add baselines: NN, SV-DKL(SVI+KISS-GP) (Wenyuan), NN+SVGP (Haoyuan).
%         \item UCI regression (Haoyuan), Image classification (Wenyuan)
%     \end{enumerate}
    
% \end{enumerate}

% \subsection{Sep 16}
% \begin{enumerate}
%     \item \st{retest DAK-MC}
%     \item Regression
%     \begin{enumerate}
%         \item Different batch size for the regression
%         \item Compare the results of different dimensions of NN outputs for regression, look into why Parkinsons dat performance (input has binary feature)
%         \item add Coverage metrics to regression
%     \end{enumerate}
%     \item Classification
%     \begin{enumerate}
%         \item fix NN+SVGP for classification
%         \item \st{fix ECE} 
%     \end{enumerate}
    
% \end{enumerate}


% \subsection{Sep 21}
% \begin{enumerate}
%     \item num NN out features = [16, 64, 256]
%     \item fix SV-DKL regression: NN+additive KISS-GP
%     \item synthetic data (similar to parkinsons has binary features and target is continuous): uci data may also work if we choose a continuous feature as output.
%     \item remove coverage rate, may come from the effects of the choices of the fixed noise variance $\sigma_{f}^2$
%     \item motivation? 1.When input is largely heterogenous (or has binary feature)? 2.When NN output is high dimensional. However, In what conditions we may need large NN out features? (Image set) 3. Multi-task regression?
%     \item why our model perform better than SV-DKL? Any theoretical explanations for it? (Reason better than NN+SVGP: additive structure, suitable for high-dimensional NN outputs)
% \end{enumerate}

% \subsection{Sep 23}
% \begin{enumerate}
%     \item last layer GP equivalent to BNN refer to the literature ``drop out as variational ''
% \end{enumerate}

%\subsection{Oct 7}
% \begin{enumerate}
%     \item stress that computation of additive kernel is not equal to computation of additive f
%     \item remove \mu
%     \item Add a remark, By approximation, we can take full advanrage of additive GPs, different from standard inducining points, add remark. If additive GP, no simplification for computation
% \item add last-layer BNN reference to related work
%  \item move related work to the last
% add more explanation to Figure in toy example
%    \item revise 'DAK-UQ' to DAK-CF
% \end{enumerate}
%\subsection{Oct 9}
% 
% replace Figure 2 (d)DNN with true GP posterior ground truth
% }}
% }{}


\section{CONCLUSION}
\label{sec:conc}
In this work, we introduced the DAK model, which reinterprets DKL as a hybrid NN, representing the last-layer GP as
% integrating NNs with 
a sparse BNN layer to address the scalability and training inefficiencies of DKL. 
% While DKL integrates NNs and GPs for UQ and feature extraction, it faces computational challenges, particularly in the GP layer and optimizing the ELBO. 
The DAK model overcomes limitations of GPs by embedding high-dimenional features from NNs into additive GP layers and leveraging the sparse Cholesky factor of the Laplace kernel on the induced grids, significantly enhancing training and inference efficiency. The proposed model also provides closed-form solutions for both the predictive distribution and ELBO in regression tasks, eliminating the need for costly MC sampling. Empirical results show that DAK outperforms state-of-the-art DKL models in both regression and classification tasks. This work also opens new possibilities for improving DKL by establishing the connection between BNNs and GPs. 

Possible directions for future work include considering more general GP layers other than the Laplace kernels and the additive structure, and exploring other variational families for training the BNNs.
%as finite-width BNNs, and further calibrating the uncertainty bounds to reduce overconfidence when neural network output features are highly correlated.

% 1 to show, 0 to hide
\ifthenelse{\equal{\showcontent}{1}}{
\subsubsection*{Acknowledgements}
The authors gratefully acknowledge the support provided by NSF grant DMS-2312173, the computing infrastructure provided by the Department of Electrical \& Computer Engineering at Texas A\&M University, and the reviewers' constructive feedback.}

\bibliographystyle{apalike}
\bibliography{references}

%%%%%%%%%%%%%%%%%%%%%%%%%%%%%%%%%%%%%%%%%%%%%%%%%%%%%%%%%%%%
\section*{Checklist}
 \begin{enumerate}


 \item For all models and algorithms presented, check if you include:
 \begin{enumerate}
   \item A clear description of the mathematical setting, assumptions, algorithm, and/or model. [\textbf{Yes}/No/Not Applicable]
   \item An analysis of the properties and complexity (time, space, sample size) of any algorithm. [\textbf{Yes}/No/Not Applicable]
   \item (Optional) Anonymized source code, with specification of all dependencies, including external libraries. [\textbf{Yes}/No/Not Applicable]
 \end{enumerate}


 \item For any theoretical claim, check if you include:
 \begin{enumerate}
   \item Statements of the full set of assumptions of all theoretical results. [\textbf{Yes}/No/Not Applicable]
   \item Complete proofs of all theoretical results. [\textbf{Yes}/No/Not Applicable]
   \item Clear explanations of any assumptions. [\textbf{Yes}/No/Not Applicable]     
 \end{enumerate}


 \item For all figures and tables that present empirical results, check if you include:
 \begin{enumerate}
   \item The code, data, and instructions needed to reproduce the main experimental results (either in the supplemental material or as a URL). [\textbf{Yes}/No/Not Applicable] 
   
   
   \item All the training details (e.g., data splits, hyperparameters, how they were chosen). [\textbf{Yes}/No/Not Applicable]
         \item A clear definition of the specific measure or statistics and error bars (e.g., with respect to the random seed after running experiments multiple times). [\textbf{Yes}/No/Not Applicable]
         \item A description of the computing infrastructure used. (e.g., type of GPUs, internal cluster, or cloud provider). [\textbf{Yes}/No/Not Applicable]
 \end{enumerate}

 \item If you are using existing assets (e.g., code, data, models) or curating/releasing new assets, check if you include:
 \begin{enumerate}
   \item Citations of the creator If your work uses existing assets. [\textbf{Yes}/No/Not Applicable]
   \item The license information of the assets, if applicable. [Yes/No/\textbf{Not Applicable}]
   \item New assets either in the supplemental material or as a URL, if applicable. [Yes/No/\textbf{Not Applicable}]
   \item Information about consent from data providers/curators. [Yes/No/\textbf{Not Applicable}]
   \item Discussion of sensible content if applicable, e.g., personally identifiable information or offensive content. [Yes/No/\textbf{Not Applicable}]
 \end{enumerate}

 \item If you used crowdsourcing or conducted research with human subjects, check if you include:
 \begin{enumerate}
   \item The full text of instructions given to participants and screenshots. [Yes/No/\textbf{Not Applicable}]
   \item Descriptions of potential participant risks, with links to Institutional Review Board (IRB) approvals if applicable. [Yes/No/\textbf{Not Applicable}]
   \item The estimated hourly wage paid to participants and the total amount spent on participant compensation. [Yes/No/\textbf{Not Applicable}]
 \end{enumerate}

 \end{enumerate}

\appendix
\section*{Appendix}
\section{Discussion: Scope and Ethics}
\label{appendix:scope}
In this work, we evaluate our method on six core scene-aware tasks: existence, count, position, color, scene, and HOI reasoning. We select these tasks as they represent core aspects of multimodal understanding which are essential for many applications. Meanwhile, we do not extend our evaluation to more complex reasoning tasks, such as numerical calculations or code generation, because SOTA diffusion models like SDXL are not yet capable of handling these tasks effectively. Fine-tuning alone cannot overcome the fundamental limitations of these models in generating images that require symbolic logic or complex reasoning. Additionally, we avoid tasks with ethical concerns, such as generating images of specific individuals (e.g., for celebrity recognition task), to mitigate risks related to privacy and misuse. Our goal was to ensure that our approach focuses on technically feasible and responsible AI applications. Expanding to other tasks will require significant advancements in diffusion model capabilities and careful consideration of ethical implications.

\section{Limitations and Future Work}
While our Multimodal Context Evaluator proves effective in enhancing the fidelity of generated images and maintaining diversity, \method is built using pre-trained diffusion models such as SDXL and MLLMs like LLaVA, it inherently shares the limitations of these foundation models. \method still faces challenges with complex reasoning tasks such as numerical calculations or code generation due to the symbolic logic limitations inherent to SDXL. Additionally, during inference, the MLLM context descriptor occasionally generates incorrect information or ambiguous descriptions initially, which can lead to lower fidelity in the generated images. Figure~\ref{fig:failure} further illustrates these observations.

\method currently focuses on single attributes like count, position, and color as part of the multimodal context. This is because this task alone poses significant challenges to existing methods, which \method effectively addresses. A potential direction for future work is to broaden the applicability of \method to synthesize images with multiple scene attributes in the multimodal context as part of compositional reasoning tasks.


\begin{figure}[!h]
    \centering
    \includegraphics[width=\linewidth]{figures/failures.pdf}
    \caption{Failure cases of \method. (a) Our method fails due to the symbolic logic limitation of existing pre-trained SDXL. (b) Initially incorrect descriptions generated by MLLMs lead to low fidelity of generated images. (c) Context description generated by MLLMs is ambiguous and does not directly relate to the text guidance, the spoon can be both inside or outside the bowl.}
    \label{fig:failure}
\end{figure}

\section{Prompt Templates}
\label{appendix:prompts}
Figure~\ref{fig:prompt_templates}~(a-c) showcases the prompt templates used by \method to fine-tune diffusion models specifically on each task including VQA, HOI Reasoning, and Object-Centric benchmarks. It's worth noting that we designed the prompt such that it provides detailed instruction to MLLMs on which scene attributes to focus. We also evaluate the effectiveness of our designed prompt templates by fine-tuning \method with a generic prompt as illustrated in Figure~\ref{fig:prompt_templates}~(d). Table~\ref{table:prommpt} indicates that without using our designed prompt template, the MLLM is not properly instructed to generate specific context description thus leading to reduced performance after fine-tuning on MME tasks. We believe that when using a generic prompt, MLLM is not able to receive sufficient grounding about the multimodal context leading to information loss on key scene attributes.


\begin{table}[!h]
\centering
\footnotesize
\caption{Effectiveness of the prompt template on fine-tuning \method on MME Perception.}
\resizebox{1\linewidth}{!}{
\begin{tabular}{clcccccccccc}
\toprule
 \textbf{MLLM} & \multirow{2}{*}{\textbf{\method}} & \multicolumn{2}{c}{\textbf{Existence}} & \multicolumn{2}{c}{\textbf{Count}} & \multicolumn{2}{c}{\textbf{Position}} & \multicolumn{2}{c}{\textbf{Color}} & \multicolumn{2}{c}{\textbf{Scene}} \\
 \textbf{Name} & & ACC & ACC+ & ACC & ACC+ & ACC & ACC+ & ACC & ACC+ & ACC & ACC+ \\
 \midrule
 \multirow{3}{*}{\makecell{\textbf{LLaVA }  \\ \textbf{v1.6 7B} \\ \citep{liu2024improved}}}
 &w/ prompt template & \textbf{96.67}  & \textbf{93.33}  & \textbf{83.33}  & \textbf{70.00}  & \textbf{81.67}  & \textbf{66.67} & \textbf{95.00}  & \textbf{93.33}  & \textbf{87.75} & \textbf{74.00} \\
 \cmidrule{2-12}
 & \multirow{2}{*}{w/ generic prompt} & 91.67 & 83.33 & 75.00 & 56.67 & \textbf{81.67} & 63.33 & 91.67 & 83.33 & 87.25 & 73.00 \\
 & & {\scriptsize \color{red}\textbf{$\downarrow$ 5.00}} & {\scriptsize \color{red}\textbf{$\downarrow$ 10.00}} & {\scriptsize \color{red}\textbf{$\downarrow$ 8.33}} & {\scriptsize \color{red}\textbf{$\downarrow$ 13.33}} & - &  {\scriptsize \color{red}\textbf{$\downarrow$ 3.34}} & {\scriptsize \color{red}\textbf{$\downarrow$ 3.33}} & {\scriptsize \color{red}\textbf{$\downarrow$ 10.00}} & {\scriptsize \color{red}\textbf{$\downarrow$ 0.50}} & {\scriptsize \color{red}\textbf{$\downarrow$ 1.00}}\\
 \midrule
 \multirow{3}{*}{\makecell{\textbf{InternVL }  \\ \textbf{2.0 8B}\\ \citep{chen2024internvl}}} 
 &w/ prompt template & \textbf{98.33}  & \textbf{96.67} & \textbf{86.67} & \textbf{73.33}  & \textbf{78.33}  & \textbf{63.33}  & \textbf{98.33}  & \textbf{96.67}  & \textbf{86.25} & \textbf{71.00} \\
 \cmidrule{2-12}
 & \multirow{2}{*}{w/ generic prompt} & 91.67 & 83.33 & 80.00 & 60.00 & 71.67 & 50.00 & 91.67 & 83.33 & 84.50 & 69.00 \\
 & & {\scriptsize \color{red}\textbf{$\downarrow$ 6.66}} &  {\scriptsize \color{red}\textbf{$\downarrow$ 13.34}} & {\scriptsize \color{red}\textbf{$\downarrow$ 6.67}} & {\scriptsize \color{red}\textbf{$\downarrow$ 13.33}} & {\scriptsize \color{red}\textbf{$\downarrow$ 6.66}} & {\scriptsize \color{red}\textbf{$\downarrow$ 13.33}} & {\scriptsize \color{red}\textbf{$\downarrow$ 6.66}} & {\scriptsize \color{red}\textbf{$\downarrow$ 13.34}} & {\scriptsize \color{red}\textbf{$\downarrow$ 1.75}} & {\scriptsize \color{red}\textbf{$\downarrow$ 2.00}}\\
\bottomrule
\end{tabular}
}
\label{table:prommpt}
\end{table}

\begin{figure}[!h]
    \centering
    \includegraphics[width=\linewidth]{figures/prompt_template.pdf}
    \caption{Prompt templates (a-c) used by \method to fine-tune the diffusion model on each task including VQA, HOI Reasoning, and Object Centric benchmarks. The generic prompt (d) is also included to evaluate the effectiveness of prompt template.}
    \label{fig:prompt_templates}
\end{figure}
\section{Inference Pipeline}
\label{appendix:inference}
In the inference pipeline of \method (Figure~\ref{fig:inference}), the text guidance $\mathbf{g}$ includes only the question corresponding to the reference image $\mathbf{x}$. The answer is excluded for fair evaluation. Moreover, we remove Multimodal Context Evaluator, and the generated image $\hat{\mathbf{x}}$ is the final output.
\begin{figure}[!h]
    \centering
    \includegraphics[width=\linewidth]{figures/inference.pdf}
    \caption{Inference pipeline of \method}
    \label{fig:inference}
\end{figure}

\begin{figure}[!h]
    \centering
    \includegraphics[width=\linewidth]{figures/diversity_compact_caption.pdf}
    \vspace{-5mm}
    \caption{Examples of context description from MLLM in the inference pipeline where answers are not included in text guidance.}
    \label{fig:diversity_compact_caption}
\end{figure}



\section{Ablation Study on BLIP-2 QFormer}
Our design choice to leverage BLIP-2 QFormer in \method as the multimodal context evaluator facilitates the formulation of our novel Global Semantic and Fine-grained Consistency Rewards. These rewards enable \method to be effective across all tasks as seen in Table~\ref{table:clip}. On replace with a less powerful multimodal context encoder such as CLIP ViT-G/14, we can only implement the global semantic reward as the cosine similarity between the text features and generated image features. As a result, while the setting can maintain performance on coarse-level tasks such as Scene and Existence, there is a noticeable decline on fine-grained tasks like Count and Position. This demonstrates the effectiveness of our design choices in \method and shows that using less powerful alternatives, without the ability to provide both global and fine-grained alignment, affects the fidelity of generated images.

\begin{figure*}[t]
\centering
\includegraphics[width=15.5cm]{figures/clip_zeroshot.png}\\
\caption{CLIP a) training and b) zero-shot inference framework}
\label{fig:clip} 
\end{figure*}


\section{Additional Evaluation on MME Artwork}

To explore the method's ability to work on tasks involving more nuanced or abstract text guidance beyond factual scene attributes, we evaluate \method on an additional task of MME Artwork. This task focuses on image style attributes that are more nuanced/abstract such as the following question-answer pair -- Question: ``Does this artwork exist in the form of mosaic?'', Answer: ``No''.

Table~\ref{table:artwork_reasoning} summarizes the evaluation. We can observe that \method outperforms all existing methods on both ACC and ACC+, implying its higher effectiveness in generating images with high fidelity (in this case, image style preservation) compared to existing methods. This provides evidence that \method can generalize to tasks involving abstract/nuanced attributes such as image style. Figure~\ref{fig:artwork} further shows qualitative comparison between image generation methods on the MME Artwork task.

\begin{table}[h]
\centering
\caption{Comparison on Artwork benchmark and Visual Reasoning task. \method outperforms SOTA image generation and augmentation techniques.}
\resizebox{\linewidth}{!}{
\begin{tabular}{@{}l@{ }ccccccc@{}}
\toprule
\textbf{Method} & \textbf{Real only} & \textbf{RandAugment} &  \textbf{Image Variation} & \textbf{Image Translation} & \textbf{Textual Inversion} & \textbf{I2T2I SDXL} & \textbf{\method} \\
\midrule
\textbf{Artwork ACC} & 69.50 & 69.25 & 69.00 & 67.00 & 66.75 & 68.00 & \textbf{70.25} \\
\textbf{Artwork ACC+} & 41.00 & 41.00 & 40.00 & 38.00 & 37.50 & 38.00 & \textbf{41.50} \\
\midrule
\textbf{Reasoning ACC} & 69.29 & 67.86 & 69.29 & 69.29 & 67.14 & 72.14 & \textbf{72.86} \\
\textbf{Reasoning ACC+} & 42.86 & 40.00 & 41.40 & 40.00 & 37.14 & 47.14 & \textbf{48.57} \\

\bottomrule
\end{tabular}
}
\label{table:artwork_reasoning}
\end{table}


\begin{figure}[!h]
    \centering
    \includegraphics[width=\linewidth]{figures/artwork.pdf}
    \caption{Qualitative comparison on the Artwork task between image generation method. \method can preserve both diversity and fidelity of the reference image in a more abstract domain.}
    \label{fig:artwork}
\end{figure}


\section{Additional Evaluation on MME Commonsense Reasoning}
We have additionally performed our evaluation to more complex tasks such as Visual Reasoning using the MME Commonsense Reasoning benchmark. Results in Table~\ref{table:artwork_reasoning} highlight \method's ability to generalize effectively across diverse domains and complex reasoning tasks, demonstrating its broader applicability. Figure~\ref{fig:reasoning} further shows qualitative comparison between image generation methods on the MME Commonsense Reasoning task.

\begin{figure}[!h]
    \centering
    \includegraphics[width=\linewidth]{figures/reasoning.pdf}
    \caption{Qualitative comparison on the Commonsense Reasoning task between image generation method. \method can preserve both diversity and fidelity of the reference image in a more abstract domain.}
    \label{fig:reasoning}
\end{figure}
\section{FID Scores}
% \textcolor{blue}{We compute FID scores of traditional augmentation and image generation methods. Table~\ref{table:fid} shows that the data distribution of generated images by RandAugment and Image Translation are closer to the real distribution as these methods only change images minimally. We also want to emphasize that even though the FID metric evaluates the quality of generated images, it can not measure the diversity of generated images. \method with rewards fine-tuning achieves a competitive score. As we showed in the diversity analysis in Table~\ref{table:diversity} in the main paper, \method performs significantly better than these ``minimal change" methods while still achieving a competitive FID score. We believe this is a worldwide trade-off.}

We compute FID scores for \method and the different baselines (traditional augmentation and image generation methods) and tabulate the numbers in Table~\ref{table:fid}. FID is a valuable metric for assessing the quality of generated images and how closely the distribution of generated images matches the real distribution. However, \textit{FID does not account for the diversity among the generated images}, which is a critical aspect of the task our work targets~(i.e., how can we generate high fidelity images, preserving certain scene attributes, while still maintaining high diversity?). We also illustrate the shortcomings of FID for the task in Figure~\ref{fig:fid_diversity} where we compare generated images across methods. We observe that RandAugment and Image Translation achieve lower FID scores than \method~(w/ finetuning) because they compromise on diversity by only minimally changing the input image, allowing their generated image distribution to be much closer to the real distribution. While \method has a higher FID score than RandAugment and Image Translation, Figure~\ref{fig:fid_diversity} shows that it is able to preserve the scene attribute w.r.t.~multimodal context while generating an image that is significantly different from than original input image. Therefore, it accomplishes the targeted task more effectively, with both high fidelity and high diversity.

\begin{table}[h]
\centering
\caption{FID scores of traditional augmentation and image generation methods. Lower is better.}
\resizebox{\linewidth}{!}{
\begin{tabular}{@{}l@{ }ccccccc@{}}
\toprule
\multirow{2}{*}{\textbf{Method}} & \multirow{2}{*}{\textbf{RandAugment}} & \multirow{2}{*}{\textbf{I2T2I SDXL}} & \multirow{2}{*}{\textbf{Image Variation}} & \multirow{2}{*}{\textbf{Image Translation}} & \multirow{2}{*}{\textbf{Textual Inversion}} & \multicolumn{2}{c}{\textbf{\method}} \\
& & & & & & \ding{55} fine-tuning & \ding{51} fine-tuning\\
\midrule
\textbf{FID score $\downarrow$} & \textbf{15.93} & 18.35 & 17.66 & 16.29 & 20.84 & 17.78 & 16.55 \\
\bottomrule
\end{tabular}
}
\label{table:fid}
\end{table}

\begin{figure}[!h]
    \centering
    \includegraphics[width=\linewidth]{figures/fid_diversity.pdf}
    \caption{While RandAugment and Image Translation achieve lower FID scores, \method balances fidelity and diversity effectively.}
    \label{fig:fid_diversity}
\end{figure}

\section{User Study}
% \textcolor{blue}{We created a survey form with 50 questions (10 questions per MME task). In each survey question, users were shown: a reference image, a related question, and two generated images from different methods (I2T2I SDXL vs. \method). Users are asked to select the generated image(s) that preserve the attribute referred to by the question in relation to reference image. We collected form responses from 70 people. Table~\ref{table:user_study} shows that \method significantly outperforms I2T2I SDXL in terms of fidelity across all tasks on MME benchmark. We have some examples of survey questions in Figure~\ref{fig:user_study_examples}.}

We conduct a user study where we create a survey form with 50 questions (10 questions per MME Perception task). In each survey question, we show users a reference image, a related question, and a generated image each from two different methods (baseline I2T2I SDXL vs \method). We ask users to select the generated images(s) (either one or both or neither of them) that preserve the attribute referred to by the question in relation to the reference image. If an image is selected, it denotes high fidelity in generation. We collect form responses from 70 people for this study. We compute the percentage of total generated images for each method that were selected by the users as a measure of fidelity. Table~\ref{table:user_study} summarizes the results and shows that \method significantly outperforms I2T2I SDXL in terms of fidelity across all tasks on the MME Perception benchmark. We have some examples of survey questions in Figure~\ref{fig:user_study_examples}.

\begin{figure}[htp]
  \centering
   \includegraphics[width=\columnwidth]{Assets/userstudy_grid.pdf}
   
   \caption{\textbf{User study results.} Users preference percentage of our method compared to other methods in terms of text alignment, visual quality, and overall preference.
   }
   \label{fig:user_study}
\end{figure}
\begin{figure}[!h]
    \centering
    \includegraphics[width=\linewidth]{figures/user_study_examples.pdf}
    \caption{Some examples of our survey questions to evaluate the fidelity of generated images from I2T2I SDXL and \method.}
    \label{fig:user_study_examples}
\end{figure}
\section{Training Performance on Bongard HOI Dataset}
% \textcolor{blue}{We conducted an additional experiment by training a CNN baseline ResNet50 \citep{he2016deep} model on the Bongard-HOI training set with traditional augmentation and other image generation methods, using the same number of training iterations. As shown in Table~\ref{table:hoi_training}, \method consistently outperforms other methods across all test splits. However, as discussed in Subsection~\ref{sec:benchmark_formulation}, our primary focus on test-time evaluation ensures fair comparisons by avoiding variability in training behavior caused by differences in model architectures, data distributions, and training configurations.}

Following the existing method \citep{shu2022testtime}, we conduct an additional experiment by training a ResNet50 \citep{he2016deep} model on the Bongard-HOI \citep{jiang2022bongard} training set with traditional augmentation and Hummingbird generated images. We compare the performance with other image generation methods, using the same
number of training iterations. As shown in Table~\ref{table:hoi_training}, \method consistently outperforms all the baselines across all test splits. In the paper, as discussed in Section~\ref{sec:benchmark_formulation}, we focus primarily on test-time evaluation because it eliminates the variability introduced by model training due to multiple external variables such as model architecture, data distribution, and training configurations, and allows for a fairer comparison where the evaluation setup remains fixed.

\begin{table}[!h]
\centering
\footnotesize
\caption{Comparison on Human-Object Interaction~(HOI) Reasoning by training a CNN-baseline ResNet50 with image augmentation and generation methods. \method outperforms SOTA methods on all $4$ test splits of Bongard-HOI dataset.}
\resizebox{0.8\linewidth}{!}{
\begin{tabular}{lccccc}
\toprule
\multirow{3}{*}{Method} & \multicolumn{4}{c}{Test Splits} & \multirow{3}{*}{Average} \\
\cmidrule{2-5}
 & seen act., & unseen act., & seen act., & unseen act., &  \\
 & seen obj. & seen obj. & unseen obj. & unseen obj. & \\
  % & seen act., seen obj. & unseen act., seen obj. & seen act., unseen obj. & unseen act., unseen obj. &  \\
 % &  &  &  & & \\
\midrule
CNN-baseline (ResNet50) & 50.03\xspace\xspace\xspace\xspace\xspace\xspace\xspace\xspace\xspace\xspace & 49.89\xspace\xspace\xspace\xspace\xspace\xspace\xspace\xspace\xspace\xspace & 49.77\xspace\xspace\xspace\xspace\xspace\xspace\xspace\xspace\xspace\xspace & 50.01\xspace\xspace\xspace\xspace\xspace\xspace\xspace\xspace\xspace\xspace & 49.92\xspace\xspace\xspace\xspace\xspace\xspace\xspace\xspace\xspace\xspace \\
RandAugment \citep{cubuk2020randaugment} & 51.07 {\scriptsize \color{ForestGreen}$\uparrow$ 1.04} & 51.14 {\scriptsize \color{ForestGreen}$\uparrow$ 1.25} & 51.79 {\scriptsize \color{ForestGreen}$\uparrow$ 2.02} & 51.73 {\scriptsize \color{ForestGreen}$\uparrow$ 1.72} & 51.43 {\scriptsize \color{ForestGreen}$\uparrow$ 1.51} \\
Image Variation \citep{xu2023versatile} & 41.78 {\scriptsize \color{red}$\downarrow$ 8.25} & 41.29 {\scriptsize \color{red}$\downarrow$ 8.60} & 41.15 {\scriptsize \color{red}$\downarrow$ 8.62} & 41.25 {\scriptsize \color{red}$\downarrow$ 8.76} & 41.37 {\scriptsize \color{red}$\downarrow$ 8.55} \\
Image Translation \citep{pan2023boomerang} & 46.60 {\scriptsize \color{red}$\downarrow$ 3.43} & 46.94 {\scriptsize \color{red}$\downarrow$ 2.95} & 46.38 {\scriptsize \color{red}$\downarrow$ 3.39} & 46.50 {\scriptsize \color{red}$\downarrow$ 3.51} & 46.61 {\scriptsize \color{red}$\downarrow$ 3.31} \\
Textual Inversion \citep{gal2022image} & \xspace37.67 {\scriptsize \color{red}$\downarrow$ 12.36} & \xspace37.52 {\scriptsize \color{red}$\downarrow$ 12.37} & \xspace38.12 {\scriptsize \color{red}$\downarrow$ 11.65} & \xspace38.06 {\scriptsize \color{red}$\downarrow$ 11.95} & \xspace37.84 {\scriptsize \color{red}$\downarrow$ 12.08} \\
I2T2I SDXL \citep{podell2023sdxl} & 51.92 {\scriptsize \color{ForestGreen}$\uparrow$ 1.89} & 52.18 {\scriptsize \color{ForestGreen}$\uparrow$ 2.29} & 52.25 {\scriptsize \color{ForestGreen}$\uparrow$ 2.48} & 52.15 {\scriptsize \color{ForestGreen}$\uparrow$ 2.14} & 52.13 {\scriptsize \color{ForestGreen}$\uparrow$ 2.21}\\
\textbf{\method} & \textbf{53.71 {\scriptsize \color{ForestGreen}$\uparrow$ 3.68}} & \textbf{53.55 {\scriptsize \color{ForestGreen}$\uparrow$ 3.66}} & \textbf{53.69 {\scriptsize \color{ForestGreen}$\uparrow$ 3.92}} & \textbf{53.41 {\scriptsize \color{ForestGreen}$\uparrow$ 3.40}} & \textbf{53.59 {\scriptsize \color{ForestGreen}$\uparrow$ 3.67}} \\
\bottomrule
\end{tabular}
}
\label{table:hoi_training}
\end{table}



\section{Random Seeds Selection Analysis}
We conduct an additional experiment, varying the number of random seeds from $10$ to $100$. The results are presented in the boxplot in Figure~\ref{fig:boxplot}, which shows the distribution of the mean L2 distances of generated image features from Hummingbird across different numbers of seeds.


The figure demonstrates that the difference in the distribution of the diversity (L2) scores across the different numbers of random seeds is statistically insignificant. So while it is helpful to increase the number of seeds for improved confidence, we observe that it stabilizes at 20 random seeds. This analysis suggests that using $20$ random seeds also suffices to capture the diversity of generated images without significantly affecting the robustness of the analysis.

% We conduct an additional experiment where we vary the number of seeds from 10 to 100. We present the results as a boxplot in Appendix K, Figure 15 which shows the distribution of the mean L2 distances of generated image features from Hummingbird across different numbers of seeds.

% The figure demonstrates that the difference in the distribution of the diversity (L2) scores across the different numbers of random seeds is statistically insignificant. So while it is helpful to increase the number of seeds for improved confidence, we observe that it stabilizes at 20 random seeds. This analysis suggests that using 20 random seeds also suffices to capture the diversity of generated images without significantly affecting the robustness of the analysis.

\begin{figure}[!h]
    \centering
    \includegraphics[width=0.8\linewidth]{figures/diversity_boxplot_rectangular.pdf}
    \caption{Diversity analysis across varying numbers of random seeds (10 to 100) using mean L2 distances of generated image features from \method. The box plot demonstrates consistent diversity scores as the number of seeds increases, indicating that performance stabilizes around 20 random seeds.}
    \label{fig:boxplot}
\end{figure}

\section{Further Explanation of Multimodal Context Evaluator}
The Global Semantic Reward, \(\mathcal{R}_\textrm{global}\), ensures alignment between the global semantic features of the generated image \(\mathbf{\hat{x}}\) and the textual context description \(\mathcal{C}\). This reward leverages cosine similarity to measure the directional alignment between two feature vectors, which can be interpreted as maximizing the mutual information \(I(\mathbf{\hat{x}}, \mathcal{C})\) between the generated image \(\mathbf{\hat{x}}\) and the context description \(\mathcal{C}\). Mutual information quantifies the dependency between the joint distribution \(p_{\theta}(\mathbf{\hat{x}}, \mathcal{C})\) and the marginal distributions. In conditional diffusion models, the likelihood \(p_{\theta}(\mathbf{\hat{x}} \vert \mathcal{C})\) of generating \(\mathbf{\hat{x}}\) given \(\mathcal{C}\) is proportional to the joint distribution:
\[
p_{\theta}(\mathbf{\hat{x}} \vert \mathcal{C}) = \frac{p_{\theta}(\mathbf{\hat{x}}, \mathcal{C})}{p(\mathcal{C})} \propto p_{\theta}(\mathbf{\hat{x}}, \mathcal{C}),
\]
where \(p(\mathcal{C})\) is the marginal probability of the context description, treated as a constant during optimization. By maximizing \(\mathcal{R}_\textrm{global}\), which aligns global semantic features, the model indirectly maximizes the mutual information \(I(\mathbf{\hat{x}}, \mathcal{C})\), thereby enhancing the likelihood \(p_{\theta}(\mathbf{\hat{x}} \vert \mathcal{C})\) in the conditional diffusion model.


The Fine-Grained Consistency Reward, $\mathcal{R}_{\textrm{fine-grained}}$, captures detailed multimodal alignment between the generated image $\mathbf{\hat{x}}$ and the textual context description $\mathcal{C}$. It operates at a token level, leveraging bidirectional self-attention and cross-attention mechanisms in the BLIP-2 QFormer, followed by the Image-Text Matching (ITM) classifier to maximize the alignment score.

\textbf{Self-Attention on Text Tokens:}
    Text tokens $\mathcal{T}_{\mathrm{tokens}}$ undergo self-attention, allowing interactions among words to capture intra-text dependencies:
    \begin{equation}
        \mathcal{T}_{\mathrm{attn}} = \tt{SelfAttention}(\mathcal{T}_{\mathrm{tokens}})
    \end{equation}

\textbf{Self-Attention on Image Tokens:}
    Image tokens $\mathcal{Z}$ are derived from visual features of the generated image $\mathbf{\hat{x}}$ using a cross-attention mechanism:
    \begin{equation}
        \mathcal{Z} = \tt{CrossAttention}(\mathcal{Q}_{\mathrm{learned}}, \mathcal{I}_{\mathrm{tokens}}(\mathbf{\hat{x}}))
    \end{equation}
    These tokens then pass through self-attention to extract intra-image relationships:
    \begin{equation}
        \mathcal{Z}_{\mathrm{attn}} = \tt{SelfAttention}(\mathcal{Z})
    \end{equation}

\textbf{Cross-Attention between Text and Image Tokens:}
    The text tokens $\mathcal{T}_{\mathrm{attn}}$ and image tokens $\mathcal{Z}_{\mathrm{attn}}$ interact through cross-attention to focus on multimodal correlations:
    \begin{equation}
        \mathcal{F} = \tt{CrossAttention}(\mathcal{T}_{\mathrm{attn}}, \mathcal{Z}_{\mathrm{attn}})
    \end{equation}

\textbf{ITM Classifier for Alignment:}
    The resulting multimodal features $\mathcal{F}$ are fed into the ITM classifier, which outputs two logits: one for positive match ($j=1$) and one for negative match ($j=0$). The positive class ($j=1$) indicates strong alignment between the image-text pair, while the negative class ($j=0$) indicates misalignment:
    \begin{equation}
        \mathcal{R}_{\textrm{fine-grained}} = \tt{ITM\_Classifier}(\mathcal{F})_{j=1}
    \end{equation}

The ITM classifier predicts whether the generated image and the textual context description match. Maximizing the logit for the positive class $j=1$ corresponds to maximizing the log probability $\log p(\mathbf{\hat{x}}, \mathcal{C})$ of the joint distribution of image and text. This process aligns the fine-grained details in $\mathbf{\hat{x}}$ with $\mathcal{C}$, increasing the mutual information between the generated image and the text features.

\textbf{Improving fine-grained relationships of CLIP.} While the CLIP Text Encoder, at times, struggles to accurately capture spatial features when processing longer sentences in the Multimodal Context Description, \method addresses this limitation by distilling the global semantic and fine-grained semantic rewards from BLIP-2 QFormer into a specific set of UNet denoiser layers, as mentioned in the implementation details under Appendix~\ref{appendix:impl}~(i.e., Q, V transformation layers including $\tt{to\_q, to\_v, query, value}$). This strengthens the alignment between the generated image tokens~(Q) and input text tokens from the Multimodal Context Description~(K, V) in the cross-attention mechanism of the UNet denoiser. As a result, we obtain generated images with improved fidelity, particularly w.r.t.~spatial relationships, thereby helping to mitigate the shortcomings of vanilla CLIP Text Encoder in processing the long sentences of the Multimodal Context Description.

To illustrate further, a Context Description like “the dog under the pool” is processed in three steps: (1) self-attention is applied to the text tokens (K, V), enabling spatial terms like “dog,” “under,” and “pool” to interact; (2) self-attention is applied to visual features represented by the generated image tokens (Q) to extract intra-image relationships (3) cross-attention aligns this text features with visual features. The resulting alignment scores are used to compute the mean and select the positive class for the reward. Our approach to distill this reward into the cross-attention layers therefore ensures that spatial relationships and other fine-grained attributes are effectively captured, improving the fidelity of generated images.


\section{The Choice of Text Encoder in SDXL and BLIP-2 QFormer}

The choice of text encoder in our pipeline is to leverage pre-trained models for their respective strengths. SDXL inherently uses the CLIP Text Encoder for its generative pipeline, as it is designed to process text embeddings aligned with the CLIP Image Encoder. In the Multimodal Context Evaluator, we use the BLIP-2 QFormer, which is pre-trained with a BERT-based text encoder.

\section{Textual Inversion for Data Augmentation}
In our experiments, we applied Textual Inversion for data augmentation as follows: given a reference image, Textual Inversion learns a new text embedding that captures the context of the reference image (denoted as $<$context$>$). This embedding is then used to generate multiple augmented images by employing the prompt: ``a photo of $<$context$>$". This approach allows Textual Inversion to create context-relevant augmentations for comparison in our experiments.

\section{Convergence Curve}
To evaluate convergence, we monitor the training process using the Global Semantic Reward and Fine-Grained Consistency Reward as criteria. Specifically, we observe the stabilization of these rewards over training iterations. Figure~\ref{fig:convergence} presents the convergence curves for both rewards, illustrating their gradual increase followed by stabilization around 50k iterations. This steady state indicates that the model has learned to effectively align the generated images with the multimodal context.

\begin{figure}[!h]
    \centering
    \includegraphics[width=\linewidth]{figures/convergence.pdf}
    \caption{Convergence curves of Global Semantic and Fine-Grained Consistency Rewards}
    \label{fig:convergence}
\end{figure}


\section{Fidelity Evaluation using GPT-4o}
In addition to the results above, we compute additional metrics for fidelity, which measure how well the model preserves scene attributes when generating new images from a reference image. For this, we use GPT-4o (model version: 2024-05-13) as the MLLM oracle for a VQA task on the MME Perception benchmark \citep{fu2024mme}. 
% We use a MLLM as an oracle for a visual question-answering (VQA) task on the MME Perception benchmark \citep{fu2024mme}. In this experiment, we use GPT-4o (model version: 2024-05-13) as the oracle. 
We evaluate \method with and without fine-tuning process.

The MME dataset consists of Yes/No questions, with a positive and a negative question for every reference image. To measure fidelity, we measure the rate at which the oracle's answer remains consistent across the reference and the generated image for every image in the dataset. We run the experiment multiple times and report the average numbers in Table~\ref{table:fidelity_comparison}. We see that fine-tuning the base SDXL with our novel rewards results in an average increase of $2.99\%$ in fidelity.

\begin{table}[!h]
\centering
\footnotesize
\caption{Fidelity between reference and generated images from \method with and without fine-tuning.}
\resizebox{0.9\linewidth}{!}{
\begin{tabular}{clccc}
\toprule
 \textbf{MLLM Oracle} & \textbf{\method} & \textbf{Fidelity on ``Yes"} & \textbf{Fidelity on ``No"} & \textbf{Overall Fidelity} \\
 \midrule
 \multirow{2}{*}{\makecell{\textbf{GPT-4o}\\\textbf{Ver: 2024-05-13}}}
 & w/o fine-tuning & 68.33\xspace\xspace\xspace\xspace\xspace\xspace\xspace\xspace\xspace\xspace & 70.55\xspace\xspace\xspace\xspace\xspace\xspace\xspace\xspace\xspace\xspace & 71.18\xspace\xspace\xspace\xspace\xspace\xspace\xspace\xspace\xspace\xspace \\
 \cmidrule{2-5}
 % \cmidrule{2-12}
 & w/ fine-tuning & \textbf{69.72} {\scriptsize \color{ForestGreen}\textbf{$\uparrow$ 1.39}}  & \textbf{73.61} {\scriptsize \color{ForestGreen}\textbf{$\uparrow$ 3.06}}  & \textbf{74.17} {\scriptsize \color{ForestGreen}\textbf{$\uparrow$ 2.99}} \\
\bottomrule
\end{tabular}
}
\label{table:fidelity_comparison}
\end{table}


\section{Implementation Details}
\label{appendix:impl}
We implement \method using PyTorch \citep{paszke2019pytorch} and HuggingFace diffusers \citep{huggingface2023diffusers} libraries. For the generative model, we utilize the SDXL Base $1.0$ which is a standard and commonly used pre-trained diffusion model in natural images domain. In the pipeline, we employ CLIP ViT-G/14 as image encoder and both CLIP-L/14 \& CLIP-G/14 as text encoders \citep{radford2021learning}. We perform LoRA fine-tuning on the following modules of SDXL UNet denoiser including $Q$, $V$ transformation layers, fully-connected layers ($\tt{to\_q, to\_v, query, value, ff.net.0.proj}$) with rank parameter $r = 8$, which results in $11$M trainable parameters $\approx 0.46\%$ of total $2.6$B parameters. The fine-tuning is done on $8$ NVIDIA A100 80GB GPUs using AdamW \citep{loshchilov2017decoupled} optimizer, a learning rate of \texttt{5e-6}, and gradient accumulation steps of $8$.

\section{Additional Qualitative Results}
\label{appendix:visuals}
Figure~\ref{fig:diversity_compact_caption} showcases two examples of context description from MLLM in the inference pipeline where answers are not included in text guidance. Figure~\ref{fig:diversity_full} illustrates additional qualitative results highlighting the diversity and multimodal context fidelity between reference and synthetic images, as well as across images generated by \method with different random seeds. Figure~\ref{fig:qualitative_full} shows additional qualitative comparisons between \method and SOTA image generation methods on VQA and HOI Reasoning tasks.
\begin{figure}[!h]
    \centering
    \includegraphics[width=\linewidth]{figures/diversity_full.pdf}
    \vspace{-5mm}
    \caption{Diversity and multimodal context fidelity between reference and synthetic image and across generated ones from \method with different random seeds.}
    \label{fig:diversity_full}
\end{figure}
\begin{figure}[!h]
    \centering
    \includegraphics[width=\linewidth]{figures/qualitative_full_v1.pdf}
    \vspace{-5mm}
    \caption{Qualitative comparison between \method and other image generation methods on MME Perception and HOI Reasoning benchmarks.}
    \label{fig:qualitative_full}
\end{figure}
\end{document}
