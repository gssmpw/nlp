\section{Mitigation Strength Analysis}\label{sec:theory}


\ding{Above I explained how the formulas were derived. I suggested the following revision, which may not exactly correspond to what Cheng reported, but I think we should keep it simple, instead of jumping around quantities $\frac{\mathrm{Cov}(l_m,v_m)^2 \sigma'^4}{\sigma^4\mathrm{Cov}(l_m',v_m)^2 }$, $\frac{\mathrm{Cov}(l_m,v_m)^2 \sigma'^2}{\sigma^2\mathrm{Cov}(l_m',v_m)^2 }$ or  $\frac{\mathrm{Cov}(l_m,v_m)^2 }{\mathrm{Cov}(l_m',v_m)^2}$. Basically Cheng made the assumption that the noise $\sigma^2 \approx \sigma'^2$, thus they all are about the same}


To demonstrate the effectiveness of our proposed \method in defending against DEMAs, we analyze its theoretical mitigation strength, specifically, apply \texttt{RPAM} to DNN operations, and empirically verify it with results presented in Section~\ref{exp: empirical}.

We continue using the metric $R = \frac{N}{N_0}$
which represents a ratio between the current number of traces required for successful secret retrieval $N$ with \method and the original number of trace requirements $N_0$.
A higher $R$ indicates more traces are required to recover the same secret information than before, and $R=1,000$ is treated as the threshold of a successful defense.

\subsection{Theoretical Formula Derivation of $R$} \label{sec: theoretical R}
Assume we aim to attack the operation at $j$-th MAC for the unprotected DNN model. 
First, this operation will be preserved with a probability of $p$, i.e., among $N$ traces used to do SCA on this operation, $p \cdot N$ traces will contain the valid information. 
However, because the random MAC drop also occurs in all early $j-1$ MAC operations, the leakage is distributed among $2^{j-1}$ possible sequences of operations kept by \method. %
The varying sequences also lead to the leakage related to the $j$-th original MAC operation occurring at varying time points.
This is indeed the leakage hiding effect introduced by the \method to mitigate side-channel attacks. 

In DEMA, we assume the attacker will examine the collected EM traces and find the largest leakage point.
Our theoretical leakage signal degradation analysis will be on the largest leakage time point.
Any specific sequence of $k-1$ operations before the $j^{\text{th}}$ MAC will occur $p^k(1-p)^{j-k}$ proportion of times.
Thus we have the highest proportion of original $j$-th MAC leakage preserved through a sequence of (containing this MAC) operation as
$$\max_{1 \le k \le j}p^k(1-p)^{j-k} = \max_{1 \le k \le j} p^j (\frac{1-p}{p})^{j-k} $$

When $p \ge 0.5$, we have $\frac{1-p}{p} \le 1$, thus the maximum of the above expression is $p^j$ achieved at $k=j$ (i.e., all $j$ operations are kept); when $p<0.5$, $\frac{1-p}{p} > 1$, the maximum value is $p(1-p)^{j-1}$ achieved at $k=1$ (i.e., all first $j-1$ operations are dropped, and the $j$-th operation is executed). 
Combining those two cases, for all $p$ values, the maximum amount of leakage among all times points is $p \cdot \max(p,1-p)^{j-1}$ proportion of original leakage. 

To derive the ratio of the number of traces required $R$ from the max leakage signal level, we utilize the signal-to-noise ratio ($SNR$), defined as the ratio between the variance of the leakage signal and the variance of noise at the leakage time point. 
When leakage signal becomes smaller by a ratio of $p \cdot \max(p,1-p)^{j-1} $, its variance is reduced by a ratio of $p^2 \cdot \max(p,1-p)^{2(j-1)}$. 
Since the noise level is not affected by \method, the SNR is also reduced by a ratio of $p^2 \cdot \max(p,1-p)^{2(j-1)}$. 
$SNR$ is related to the number of trace requirement~\cite{mangard2008power}, i.e., $N \propto \frac{1}{\mathrm{SNR}}$, indicating that change in $N$ is inversely proportional to $SNR$ change.
Hence the theoretical R is :
\begin{equation}
    \label{eq: R}
    R=p^{-2}\max(p,1-p)^{-2(j-1)}
\end{equation}





\begin{figure}[t]
    \centering
    \includegraphics[width=0.8\linewidth]{fig/R-visualization.pdf}
    \caption{Visualize Function $R$ under different $j$}
    \label{fig: visualize R}
\end{figure}

According to Formula~\eqref{eq: R}, when we focus on protecting the $j$-th operation weight for $j\le 2$, decreasing $p$ always increases $R$, leading to enhanced protection. However, for $j \ge 3 $, as shown in Fig.~\ref{fig: visualize R}, $R$ increases when $p$ decreases from $1$ to $0.5$, then $R$ decreases when $p$ decreases from $0.5$ to $1/j$, before $R$ increases again when $p$ further decreases from $1/j$. For the later weights with a large $j$ value, unless we use a very small $p<1/j$, the optimal protection is achieved at $p=0.5$. Since too small $p$ will degrade the resulting model performance too much (will be presented in Table~\ref{tab: IaPAM performance}) and thereby should be excluded, $p=0.5$ would be an optimal choice for protection in most practical applications. Thus, when considering the trade-off between protection and the model performance, we can focus on choosing among values of $p \ge 0.5$.

\subsection{Formula of $R$ for Empirical Measurements}
To verify Formula~\eqref{eq: R} experimentally, we estimate the $SNR$ with and without \method. 
 We first identify the strongest leakage point for the $j$-th operation by profiling, using the point with the highest Pearson Correlation Coefficient with the correct weight value.
Then, we estimate the ratio of $SNR$ change by leveraging a statistical model of EM side-channel leakage~\cite{fei2015statistics}:
\begin{equation}
    \label{eq: EM model}
    l_m = \epsilon v_m + r_m, \quad m=1,...,n.
\end{equation}
Here, $l_m$ denotes the EM measurement at a specific time point, $\epsilon$ is a constant representing the unit EM emanation, $v_m$ is the selection function (known as the leakage model, where we use the HW of the intermediate accumulation), and $r_m$ is the random noise influenced by various factors such as measurement, electrical, and switching noise, following a Gaussian distribution $N(c,\sigma^2)$. Then, $SNR \propto \epsilon^2/\sigma^2$. Hence, we can estimate $R$ using the equation:
\begin{equation}
    \hat R  = \frac{\mathrm{\epsilon^2 (\sigma')^2}}{(\epsilon')^2 \sigma^2},\label{equ:R.emp}
\end{equation}
where $\epsilon'$ and $\sigma'$ represent the post-mitigation values when fitting \eqref{eq: EM model} on measurements with \method activated. 

In Fig.~\ref{fig: Nr}, we plot the theoretical $R$ values Formula~\eqref{eq: R} versus the empirical estimates~(\ref{equ:R.emp}). They exhibit a good match, affirming the validity of the theoretical model.
These $R$ values demonstrate the effectiveness of the \method in mitigating side-channel analysis for each weight, with the effectiveness increasing exponentially for subsequent operations (linear increase in the logarithm scale plot).
