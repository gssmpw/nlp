\section{Related Work}

\subsection{EM Side-channel Analysis}
EM Side-channel Analysis, including simple EM analysis and Differential EM Analysis (DEMA), exploits EM emissions to extract sensitive information, including cryptographic keys, as well as DNN model hyperparameters and parameters.
DEMA specifically utilizes the data-dependency of EM signals and employs statistical analysis over a set of EM traces to infer the secrets.
For a Device Under Test (DUT), the adversary builds an EM leakage model for key modules of the system implementation, e.g., a register holding secret-dependent intermediate values, the result of certain operations.
The model predicts EM emissions based on hypothetical secret values, and these predictions are then correlated with actual EM measurements.  
Correct secret guesses will yield the highest correlations. 
A common model used in DEMA against DNN parameters on MCUs is the Hamming Weight (HW) model, which counts the number of `1' bits in a variable's binary representation~\cite{batina2019csi}. To improve the efficiency, often times a range of points of interests are selected from the traces to correlate with the EM predictions, based on analysis of execution of the sequence of operations. 

\subsection{Defense against Side-channel Analysis}
To counter such attacks, several countermeasure principles, previously applied to ciphers, have been adapted for DNN model protection. Masking has been applied both to hardware implementations ~\cite{dubey2020maskednet} and software platforms~\cite{popets2022} to thwart parameter extraction attacks. Hiding techniques, such as operation shuffling, have also been implemented on DNN solutions~\cite{dubey2022guarding,brosch2022counteract}. Maji et al.\cite{maji2022threshold} employed threshold implementation, while Hashemi et al.\cite{hashemi2022hwgn} utilized garbled circuits for DNN protection. All the prior approaches, although effective in protecting DNN models against parameter extraction attacks, bear significant implementation costs by dealing with a lot of operations directly, neglecting special DNN features.
\emph{To the best of our knowledge, our proposed approach is the first defense that diverges from the traditional SCA mitigation techniques and leverages DNN characteristics for efficient SCA protections.
}
