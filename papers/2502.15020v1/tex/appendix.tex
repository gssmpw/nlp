\appendix
\section{Derivation of Leakage Upper Bound for Adaptive Attack} \label{appendix: leakage upperbound}
When the original $j$-th operation occurs in the $k$-th position, there are $\left(\begin{array}{c} j-1 \\ k-1 \end{array} \right)$ possible such sequences of MAC operations.
These are all the possible sequences for the leakage related to the original $j$-th operation to occur on the $k$-th leakage time point. The proportion of traces corresponding to these sequences among all traces can be expressed as:
$$
\left(\begin{array}{c} j-1 \\ k-1 \end{array} \right) p^k (1-p)^{j-k}.
$$
If a potential adaptive attack can use leakage at the $k$-th time point from all these sequences together, the above expression quantifies the usable leakage as proportion of the original leakage for $j$-th operation without our countermeasure.
Hence, similar to the mitigation strength analysis in \ref{sec: theoretical R}, the largest leakage proportion $L_{max}$ over all time points is given by:
\begin{align*}
 L_{max}  &=  \max_{1 \le k \le j} \left(\begin{array}{c} j-1 \\ k-1 \end{array} \right) p^k (1-p)^{j-k} 
\\&=   p^j \max_{1 \le k \le j}
\left[\left(\begin{array}{c} j-1 \\ k-1 \end{array} \right) \left(\frac{1-p}{p}\right)^{j-k}\right]
\end{align*}

To find the expression of the maximum leakage point, we focus the change between $k$ and $k+1$. Specifically,

{\small
\begin{align*}
    L_{k} & = \left(\begin{array}{c} j-1 \\ k-1 \end{array} \right) \left(\frac{1-p}{p}\right)^{j-k} 
 = \frac{(j-1)!}{(k-1)!(j-k)!}\left(\frac{1-p}{p}\right)^{j-k} 
\end{align*}
\begin{align*}
    L_{k+1} & = \left(\begin{array}{c} j-1 \\ k \end{array} \right) \left(\frac{1-p}{p}\right)^{j-k-1}
    = \frac{(j-1)!}{(k)!(j-k-1)!}\left(\frac{1-p}{p}\right)^{j-k-1}
\end{align*}
}

The ratio between $L_{k+1}$ and $L_{k}$ is:
\begin{align*}
\frac{L_{k+1}}{L_{k}}=\frac{j-k}{k}\left(\frac{p}{1-p}\right)    
\end{align*}
This ratio is greater or equal than 1 when 
\begin{align*}
(j-k)p \ge k(1-p), \mbox{ or } \ \ jp \ge k.    
\end{align*}

Therefore, the maximum value of $L_{k}$ is achieved at
\begin{align*}
    k = \lfloor pj \rfloor + 1, \qquad \mbox{for } p<1.
\end{align*}


Thus, the largest leakage proportion is 
\begin{align*}
    p^j \left(\begin{array}{c} j-1 \\ \lfloor pj \rfloor \end{array} \right) \left(\frac{1-p}{p}\right)^{j-\lfloor pj \rfloor -1}, \qquad \mbox{for } p<1
\end{align*}
as in~\eqref{eq:P.adapt}. Note that this equals to 
$$
\left(\begin{array}{c} j-1 \\ \lfloor pj \rfloor \end{array} \right) p^{\lfloor pj \rfloor+1} (1-p)^{j-\lfloor pj \rfloor -1}.
$$
The mitigation strength is the inverse of this quantity squared, thus we arrive at the formula~\eqref{eq: R2}.
