We introduce IAO, a prompting technique that explicitly structures how LLMs access, process, and apply their stored knowledge through a decomposed reasoning chain. This approach is inspired by knowledge-based procedural reasoning, where complex problems are solved by systematically accessing and applying relevant knowledge components. Unlike implicit knowledge utilization, IAO enforces an explicit use of knowledge flow through each reasoning step.

The motivation stems from observing that zero-shot CoT often produces incomplete knowledge chains or incorrectly merges knowledge application steps, leading to errors. As illustrated in \Cref{fig:iao_example_fig}, 
traditional approaches may overlook crucial knowledge elements (such as \textit{or part thereof}) by failing to systematically process available information. IAO's structured framework improves both knowledge utilization and verification by making explicit which knowledge components are being accessed and applied at each step.

IAO prompting comprises four knowledge-focused components as shown in \Cref{fig:iao_example_fig}:

\begin{enumerate}
    \item \textbf{Subquestion}: The LLM breaks down the primary question into smaller knowledge components, creating a sequence of targeted knowledge queries that build upon each other.
    
    \item \textbf{Input}: For each step, the LLM explicitly identifies the relevant knowledge components available, including both given facts and previously derived knowledge.
    
    \item \textbf{Action}: The LLM specifies how it plans to apply or transform the input knowledge, often expressing this as a knowledge operation or procedure.
    
    \item \textbf{Output}: The LLM generates new knowledge by applying the planned operation to the input knowledge. This derived knowledge becomes available for subsequent reasoning steps.
\end{enumerate}

This structured approach enables better tracking of knowledge flow, verification of knowledge application, and identification of potential knowledge gaps or misapplications. In the following sections, we empirically evaluate how this knowledge-structured approach enhances reasoning across various tasks.