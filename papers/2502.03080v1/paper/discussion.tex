\begin{table*}[!t]
\resizebox{\textwidth}{!}{%
\begin{tabular}{@{}llllll@{}}
\toprule
%\specialrule{2.5pt}{1pt}{1pt}
\multicolumn{2}{l}{Question} &
  \multicolumn{4}{l}{\begin{tabular}[c]{@{}l@{}} \textit{The original price of an item is discounted 22\%. A customer buys the item at this discounted price using a \$20-off coupon. There is no tax on the item, and this} \\ \textit{was the only item the customer bought. If the customer paid \$1.90 more than half the original price of the item, what was the original price of the item?} \\ Answer Choices: (A) \$61 (B) \$65 (C) \$67.40 (D) \$70 (E) \$78.20 \hspace{5cm}\textcolor{blue}{(GT: (E))}\end{tabular}} \\ \midrule \midrule
\multirow{6}{*}{\rotatebox[origin=c]{90}{PALM-2}} &
  CoT &
  \multicolumn{4}{l}{\begin{tabular}[c]{@{}l@{}}Let x be the original price of the item. The discounted price is 0.78x. The customer paid 0.78x - 20. This is \$1.90 more than half the original price, which is 0.5x + 1.90. \\ So, 0.78x - 20 = 0.5x + 1.90. Solving for x, we get x = 70.\\ Therefore, among A through E, the answer is \textbf{(D)}.\end{tabular}} \\ \cmidrule(l){2-6} 
 &
  \multirow{5}{*}{IAO} &
  Subquestion &
  Input &
  Action &
  Output \\ \cmidrule(l){3-6} 
 &
   &
  1. What is the discounted price of the item? &
  Original price = x, Discount = 22\% &
  \begin{tabular}[c]{@{}l@{}}Discounted price = \\ Original price - (Original price * Discount)\end{tabular} &
  \begin{tabular}[c]{@{}l@{}}Discounted price =\\  x - (x * 0.22) = 0.78x\end{tabular} \\
 &
   &
  2. What is the price the customer paid for the item? &
  Discounted price = 0.78x, Coupon = \$20 &
  Price paid = Discounted price - Coupon &
  Price paid = 0.78x - 20 \\
 &
   &
  \begin{tabular}[c]{@{}l@{}}3. What is the equation that related the price \\ paid to the original price?\end{tabular} &
  \begin{tabular}[c]{@{}l@{}}3. Price paid = 0.78x - 20, \\ Price paid = 0.5x + 1.90\end{tabular} &
  \begin{tabular}[c]{@{}l@{}}Set the two expressions for the price \\ paid equal to each other.\end{tabular} &
  0.78x - 20 = 0.5x + 1.90 \\
 &
   &
  4. What is the original price of the item? &
  0.78x - 20 = 0.5x + 1.90 &
  Solve for x. &
  \textbf{x = 78.20} \\ \midrule \midrule %\specialrule{2.5pt}{1pt}{1pt}
\multirow{7}{*}{\rotatebox[origin=c]{90}{GPT-4}} &
  CoT &
  \multicolumn{4}{l}{\begin{tabular}[c]{@{}l@{}}Let's denote the original price of the item as \( P \).\\ The item is discounted by 22\%, so the customer pays 78\% of the original price. We can express this as \( 0.78P \).\\ Then, the customer uses a \$20-off coupon, so we subtract \$20 from the discounted price: \( 0.78P - 20 \).\\ According to the problem, the customer paid \$1.90 more than half the original price of the item.\\ Half the original price is \( \frac{P}{2} \), so the amount the customer paid can also be expressed as \( \frac{P}{2} + 1.90 \).\\ Now we can set up the equation: \( 0.78P - 20 = \frac{P}{2} + 1.90 \)\\ To solve for \( P \), we first get rid of the fractions by multiplying every term by 2: \( 2(0.78P) - 2(20) = 2(\frac{P}{2}) + 2(1.90) \)\\ \( 1.56P - 40 = P + 3.80 \)\\ Now, we subtract \( P \) from both sides to get: \( 0.56P - 40 = 3.80 \)\\ Next, we add 40 to both sides: \( 0.56P = 43.80 \)\\ Finally, we divide both sides by 0.56 to solve for \( P \):\\ \( P = \frac{43.80}{0.56} \)\\ \( P = 78.2142857... \)\\ The closest answer choice to \$78.21 is \$78.20.\\ Therefore, among A through E, the answer is \textbf{(E) \$78.20)} .\end{tabular}} \\ \cmidrule(l){2-6} 
 &
  \multirow{6}{*}{IAO} &
  Subquestion &
  Input &
  Action &
  Output \\ \cmidrule(l){3-6} 
 &
   &
  \begin{tabular}[c]{@{}l@{}}1. What is the equation that represents \\ the relationship between the original price, \\ the discount, the coupon, and the amount \\ the customer paid?\end{tabular} &
  \begin{tabular}[c]{@{}l@{}}Let the original price be P. The customer \\ gets a 22\% discount and then uses a \\ \$20-off coupon. The customer paid\\ \$1.90 more than half the original price.\end{tabular} &
  \begin{tabular}[c]{@{}l@{}}Write the equation based on the given\\  information.\end{tabular} &
  P - 0.22P - 20 = 0.5P + 1.90 \\
 &
   &
  \begin{tabular}[c]{@{}l@{}}2. How can we simplify the equation from S\\ tep 1 to solve for P?\end{tabular} &
  P - 0.22P - 20 = 0.5P + 1.90 &
  Combine like terms and isolate P. &
  0.78P - 20 = 0.5P + 1.90 \\
 &
   &
  3. What is the value of P? &
  0.78P - 20 = 0.5P + 1.90 &
  Solve for P. &
  P = (20 + 1.90) / (0.78 - 0.5) \\
 &
   &
  4. What is the numerical value of P? &
  (20 + 1.90) / (0.78 - 0.5) &
  Perform the arithmetic to find P. &
  P = 21.90 / 0.28 \\
 &
   &
  \begin{tabular}[c]{@{}l@{}}5. What is the final calculation to determine \\ the original price?\end{tabular} &
  21.90 / 0.28 &
  Calculate the value of P. &
  \textbf{P = 78.2142857...} \\ \bottomrule
\end{tabular}%
}
\caption{Output example from AQuA dataset. The final results are in \textbf{bold}.}
\label{tab:aqua_example}
\end{table*}

\begin{table*}[]
\resizebox{\textwidth}{!}{%
\begin{tabular}{@{}llllll@{}}
\toprule
\multicolumn{2}{l}{Question} &
  \multicolumn{4}{l}{\textit{Could all of the people who pass through 30th Street Station every day fit in Dorton Arena?}                               \hspace{5cm} \textcolor{blue}{(GT: (No))}·} \\ \midrule \midrule
\multirow{5}{*}{\rotatebox[origin=c]{90}{PALM-2}} &
  CoT &
  \multicolumn{4}{l}{\begin{tabular}[c]{@{}l@{}}30th Street Station is the busiest train station in Philadelphia, with an average of 120,000 passengers passing through each day. Dorton Arena is a multi-purpose arena in \\ Raleigh, North Carolina, with a seating capacity of 5,000 people.\\ Therefore, the answer is No\end{tabular}} \\ \cmidrule(l){2-6} 
 &
  \multirow{4}{*}{IAO} &
  Subquestion &
  Input &
  Action &
  Output \\ \cmidrule(l){3-6} 
 &
   &
  \begin{tabular}[c]{@{}l@{}}1. How many people pass through \\ 30th Street Station every day?\end{tabular} &
  \begin{tabular}[c]{@{}l@{}}30th Street Station is the busiest train \\ station in Philadelphia, with an average of \\ 120,000 passengers passing through each day.\end{tabular} &
  \begin{tabular}[c]{@{}l@{}}The number of people who pass \\ through 30th Street Station \\ every day is 120,000.\end{tabular} &
  120,000 \\
 &
   &
  2. What is the capacity of Dorton Arena? &
  \begin{tabular}[c]{@{}l@{}}Dorton Arena is a multi-purpose arena located \\ in Raleigh, North Carolina. It has a seating \\ capacity of 10,000 people.\end{tabular} &
  \begin{tabular}[c]{@{}l@{}}The capacity of Dorton Arena is \\ 10,000 people.\end{tabular} &
  10,000 \\
 &
   &
  \begin{tabular}[c]{@{}l@{}}3. Can all of the people who pass through \\ 30th Street Station every day fit in \\ Dorton Arena?\end{tabular} &
  \begin{tabular}[c]{@{}l@{}}The number of people who pass through 30th\\ Street Station every day is 120,000. The \\ capacity of Dorton Arena is 10,000 people.\end{tabular} &
  \begin{tabular}[c]{@{}l@{}}The number of people who pass \\ through 30th Street Station every day \\ is greater than the capacity of Dorton Arena.\end{tabular} &
  \textbf{No} \\ \midrule \midrule
\multirow{7}{*}{\rotatebox[origin=c]{90}{GPT-4}} &
  \multirow{2}{*}{CoT} &
  \multicolumn{4}{l}{\begin{tabular}[c]{@{}l@{}}To determine if all of the people who pass through 30th Street Station every day could fit in Dorton Arena, we need to know two key pieces of information:\\ 1. The average daily number of people passing through 30th Street Station.\\ 2. The seating capacity of Dorton Arena.\\ Once we have these two figures, we can compare them to see if the arena can accommodate the daily passengers of 30th Street Station. \\ Let's assume we have the necessary data:\\ - The average daily number of people passing through 30th Street Station (we'll call this number X).\\ - The seating capacity of Dorton Arena (we'll call this number Y).\\ If X is less than or equal to Y, then the answer is Yes, all of the people can fit in Dorton Arena. If X is greater than Y, then the answer is No, they cannot all fit. \\ Since I don't have the current data, you would need to provide the average daily passenger count for 30th Street Station and the seating capacity of Dorton Arena \\ for me to complete this comparison.\end{tabular}} \\ \cmidrule(l){3-6} 
 &
   &
  \multicolumn{4}{l}{\textbf{The model fails to choose an answer.}} \\ \cmidrule(l){2-6} 
 &
  \multirow{5}{*}{IAO} &
  Subquestion &
  Input &
  Action &
  Output \\ \cmidrule(l){3-6} 
 &
   &
  \begin{tabular}[c]{@{}l@{}}1. What is the daily passenger traffic \\ at 30th Street Station?\end{tabular} &
  \begin{tabular}[c]{@{}l@{}}We need to find or estimate the number \\ of people passing through 30th Street \\ Station every day.\end{tabular} &
  \begin{tabular}[c]{@{}l@{}}Research or estimate the daily \\ passenger traffic at 30th \\ Street Station.\end{tabular} &
  \begin{tabular}[c]{@{}l@{}}Let's say the daily passenger \\ traffic at 30th Street Station \\ is 100,000 people.\end{tabular} \\
 &
   &
  \begin{tabular}[c]{@{}l@{}}2. What is the seating capacity of \\ Dorton Arena?\end{tabular} &
  \begin{tabular}[c]{@{}l@{}}We need to find the seating capacity \\ of Dorton Arena.\end{tabular} &
  \begin{tabular}[c]{@{}l@{}}Research the seating capacity \\ of Dorton Arena.\end{tabular} &
  \begin{tabular}[c]{@{}l@{}}Let's assume Dorton Arena has \\ a seating capacity of 7,610 people.\end{tabular} \\
 &
   &
  \begin{tabular}[c]{@{}l@{}}3. Can the daily passenger traffic at \\ 30th Street Station fit in Dorton Arena?\end{tabular} &
  \begin{tabular}[c]{@{}l@{}}{[}Output from Step 1{]} and\\  {[}Output from Step 2{]}\end{tabular} &
  \begin{tabular}[c]{@{}l@{}}Compare 100,000 (daily passenger traffic) \\ with 7,610 (seating capacity of Dorton Arena).\end{tabular} &
  100,000 is much greater than 7,610. \\ \cmidrule(l){3-6} 
 &
   &
  \multicolumn{4}{l}{\textbf{Therefore, the answer is No.}} \\ \bottomrule
\end{tabular}%
}
\caption{Output example from StrategyQA dataset. The final results are in \textbf{bold}.}
\label{tab:strategyqa_example}
\end{table*}


Our experimental results confirmed the effectiveness of our proposed prompting approach under the zero-shot setting. We summarize the advantages of our method compared to conventional chain-of-thought method as follows:
(i) \textbf{Clarity and structure}: IAO requires each step of the reasoning process to be explicit making the reasoning process clear and easy to follow. This forces the LLMs to articulate their thoughts and reasoning explicitly. On the other hand, in the conventional zero-shot-CoT steps might be implicit or unclear, requiring more back-and-forth to understand. (ii) \textbf{Focus and Guidance}: IAO subquestions help focus the LLM on specific aspects of the problem, leading to more relevant outputs. Contrarily, zero-shot-CoT can contain open-ended discussions, which may lead to tangent or irrelevant information or can be prone to ambiguity and implicit assumptions. (iii) \textbf{Transparency}: IAO prompting makes the thought process visible, understandable and easily interpretable. In fact, sometimes it occurs that zero-shot-CoT produces a wrong reasoning chain but finds the correct answers. In this case, it is difficult to understand the process that led to that conclusion. (iv) \textbf{Integration with tools}: Open-ended discussions are less suited for integration with external tools (like calculator and python
shell) that can further aid the math computation within
the arithmetic domain \cite{gao2023pal}. On the contrary, the structured format of IAO makes it easier to call external tools to compute and produce results that can successfully be integrated as intermediate outputs in the chain of reasoning.


\subsection{Case studies}

\Cref{tab:aqua_example} compares the solutions of two large language models (LLMs), PALM-2 and GPT-4, to a math word problem from the AQuA dataset. Using the zero-shot-CoT setting, PALM-2 makes an error in the final calculation, resulting in an incorrect answer of \$70. However, it reaches the correct result using the IAO prompt. This is an example of a point raised earlier in the paper discussing the low scores for the AQuA dataset. Arithmetic errors made in early intermediate steps may lead to incorrect conclusions. On the other hand, GPT-4 appears to be less prone to such mistakes. GPT-4 achieved the correct answer of \$78.20 by systematically breaking down the problem, setting up the equation, and performing calculations in both zero-shot CoT and IAO settings. 

\Cref{tab:strategyqa_example} shows an example of a real-world reasoning problem from the StrategyQA dataset. The task requires determining whether the daily passenger traffic at 30th Street Station could fit within the seating capacity of Dorton Arena, with the correct answer being ``No.'' This analysis highlights the potential benefits of IAO in guiding LLMs towards accurate solutions. In the zero-shot-CoT setting, PALM-2 achieved the correct answer, leveraging its access to the relevant information (``120,000 passengers'' and ``5,000 seating capacity''). However, its approach lacked explicit reasoning steps. However, GPT-4 failed to choose an answer without additional information. While its initial analysis outlined the necessary comparison, it couldn't access or process the specific data points. On the other hand, in the case of PALM-2, the reasoning chain doesn't explicitly show IAO intervention, but PALM-2 likely accessed relevant data based on the prompt's context. However, GPT-4 through IAO prompting effectively identified the necessary information (daily passengers, arena capacity), estimated values for the missing data (100,000 passengers, 7,610 capacity) and compared the values, and correctly concluded that the arena wouldn't hold all passengers. Loosely speaking, the IAO prompts provided GPT-4 with a clear roadmap to break down the problem, gather information, and perform the comparison, leading to a well-reasoned, transparent and verifiable solution.
More examples are shown in \Cref{tab:commonsenseqa_example}, \Cref{tab:date_example}, \Cref{tab:letter_example}, \Cref{tab:gsm8k_example} and \Cref{tab:object_example} in the Appendix. 

\subsection{Two-stage IAO prompting}
In all previous experiments, we utilized a single API call per sample. While effective, a critical question remains: can employing a two-stage API call per sample further enhance IAO results? To answer this, we propose a focused investigation. We chose to test this hypothesis on the dataset with the lowest gains in terms of performance and test it using PALM-2.

\Cref{tab:2stage} presents the results of comparing one-stage and two-stage IAO prompting. The table shows the performance of both approaches on two datasets, AQuA and GSM8k, for arithmetic reasoning, as well as the average performance across both datasets.

Overall, two-stage IAO achieved the best average performance (74.2\%), outperforming both one-stage zero-shot CoT (69.1\%) and one-stage IAO (73.1\%). On the individual datasets, two-stage IAO achieved the highest score on GSM8k (82.3\%), and on AQuA (64.5\%).
These results suggest that two-stage IAO prompting can be an effective strategy for improving the performance of language models on various tasks. 

These findings are derived from a two-stage prompt approach where the first stage assists the LLM in breaking down the problem, improving comprehension, and the second stage focuses on extracting the answer. This sequential prompting enhances accuracy but demands more computational resources than single-stage prompting. The choice between them hinges on the trade-off between accuracy and resource availability. For simpler tasks, single-stage prompting may be adequate, while complex tasks justify the extra computational expense of two results in improved accuracy.

\begin{table}[th]
\centering
\resizebox{0.99\columnwidth}{!}{%
\begin{tabular}{@{}llll@{}}
\toprule
                     & AQuA & GSM8k & Average \\ \midrule
One-stage zero-shot CoT & {61.8} & 76.4  & 69.1    \\
One-stage IAO        & 63.9 & 82.3  & 73.1    \\ \midrule
Two-stage IAO        & \textbf{64.5} & \textbf{83.9}  & \textbf{74.2}       \\ \bottomrule
\end{tabular}%
}
\caption{One stage vs two stage prompting comparison.}
\label{tab:2stage}
\vspace{-.08cm}
\end{table}


\subsection{Few-shot extension}
Providing the model a handful of examples substantially improves the reasoning abilities of LLMs. IAO can be paired with methods to automatically generate examples such as Auto-CoT \cite{zhang2022automatic} or \cite{yasunaga2023large} to automatically generate structured IAO-type answers to be provided as demonstrations. We test 3-shot CoT and 3-shot IAO for the AQuA dataset, the most challenging for both LLMs. 3-shot IAO improves the accuracy by 5 points (76.4 from 71.9) for GPT-4 and by 0.8 in PALM-2 (69.2 from 68.4). 

\subsection{Human evaluation}
We asked 120 human crowdworkers from CloudConnect Research questions aimed at gauging the interpretability and transparency of IAO compared to CoT. For doing this, we sample 10 questions that both IAO and CoT responded correctly in terms of accuracy and 10 questions were both answers were wrong. This is done for the most challenging dataset for each reasoning type, hence AQuA, StrategyQA and Date Understanding. For each question, we present the annotators with the IAO and CoT reasoning chains. We ask three questions: (1a) \textit{if the answers are correct, which reasoning text is more useful?} (1b) \textit{if the answers are wrong, which reasoning text do you prefer to spot the reasoning mistake?} (2) \textit{which reasoning text is more transparent? (3) which reasoning text is easier to interpret?} \Cref{tab:human_eval} presents the key findings.
\begin{table}[h]
\centering
\begin{tabular}{@{}lccc@{}}
\toprule
\textbf{Question} & \textbf{AQuA} & \textbf{StrategyQA} & \textbf{Date} \\
\midrule
(1a) Useful         & 12 & 9 & 27 \\
(1b) Efficiency     & 86 & 73 & 87 \\
(2) Transparency    & 64 & 76 & 89 \\
(3) Interpretability & 74 & 71 & 83 \\
\bottomrule
\end{tabular}
\caption{Human evaluation of IAO and CoT reasoning chains. The values reported are percentages of times IAO is chosen over CoT.}
\label{tab:human_eval}
\end{table}
In general, when the answer is correct, IAO is perceived as notably less useful, with only 12\% and 9\% preference, in AQuA and StrategyQA respectively. This is to be expected, as IAO prompts generates longer reasoning chains compared to CoT and favours redundancy of information. However, when the answer is wrong, IAO is much more favored, with preferences being: 86\% for AQuA, 73\% for StrategyQA, and 87\% for Date. This suggests that IAO is considered more efficient compared to CoT for determining errors in the reasoning process.
In general, IAO is preferred in terms of transparency and interpretability of the reasoning text as reported by the third and fourth row, indicating a clearer and understandable presentation in IAO compared to CoT reasoning chains. 