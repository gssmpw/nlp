%File: anonymous-submission-latex-2025.tex
\documentclass[letterpaper]{article} % DO NOT CHANGE THIS
\usepackage[draft]{aaai25}  % DO NOT CHANGE THIS
\usepackage{times}  % DO NOT CHANGE THIS
\usepackage{helvet}  % DO NOT CHANGE THIS
\usepackage{courier}  % DO NOT CHANGE THIS
\usepackage[hyphens]{url}  % DO NOT CHANGE THIS
\usepackage{graphicx} % DO NOT CHANGE THIS
\urlstyle{rm} % DO NOT CHANGE THIS
\def\UrlFont{\rm}  % DO NOT CHANGE THIS
\usepackage{natbib}  % DO NOT CHANGE THIS AND DO NOT ADD ANY OPTIONS TO IT
\usepackage{caption} % DO NOT CHANGE THIS AND DO NOT ADD ANY OPTIONS TO IT
\frenchspacing  % DO NOT CHANGE THIS
\setlength{\pdfpagewidth}{8.5in} % DO NOT CHANGE THIS
\setlength{\pdfpageheight}{11in} % DO NOT CHANGE THIS
%
% These are recommended to typeset algorithms but not required. See the subsubsection on algorithms. Remove them if you don't have algorithms in your paper.
\usepackage{algorithm}
\usepackage{algorithmic}
\usepackage{xcolor}
\usepackage{amsmath}
\usepackage{booktabs}
\usepackage{multirow}
\usepackage{cleveref}
\usepackage{comment}
\usepackage{placeins}

%
% These are are recommended to typeset listings but not required. See the subsubsection on listing. Remove this block if you don't have listings in your paper.
\usepackage{newfloat}
\usepackage{listings}
\DeclareCaptionStyle{ruled}{labelfont=normalfont,labelsep=colon,strut=off} % DO NOT CHANGE THIS
\lstset{%
	basicstyle={\footnotesize\ttfamily},% footnotesize acceptable for monospace
	numbers=left,numberstyle=\footnotesize,xleftmargin=2em,% show line numbers, remove this entire line if you don't want the numbers.
	aboveskip=0pt,belowskip=0pt,%
	showstringspaces=false,tabsize=2,breaklines=true}
\floatstyle{ruled}
\newfloat{listing}{tb}{lst}{}
\floatname{listing}{Listing}
%
% Keep the \pdfinfo as shown here. There's no need
% for you to add the /Title and /Author tags.
\pdfinfo{
/TemplateVersion (2025.1)
}

% DISALLOWED PACKAGES
% \usepackage{authblk} -- This package is specifically forbidden
% \usepackage{balance} -- This package is specifically forbidden
% \usepackage{color (if used in text)
% \usepackage{CJK} -- This package is specifically forbidden
% \usepackage{float} -- This package is specifically forbidden
% \usepackage{flushend} -- This package is specifically forbidden
% \usepackage{fontenc} -- This package is specifically forbidden
% \usepackage{fullpage} -- This package is specifically forbidden
% \usepackage{geometry} -- This package is specifically forbidden
% \usepackage{grffile} -- This package is specifically forbidden
% \usepackage{hyperref} -- This package is specifically forbidden
% \usepackage{navigator} -- This package is specifically forbidden
% (or any other package that embeds links such as navigator or hyperref)
% \indentfirst} -- This package is specifically forbidden
% \layout} -- This package is specifically forbidden
% \multicol} -- This package is specifically forbidden
% \nameref} -- This package is specifically forbidden
% \usepackage{savetrees} -- This package is specifically forbidden
% \usepackage{setspace} -- This package is specifically forbidden
% \usepackage{stfloats} -- This package is specifically forbidden
% \usepackage{tabu} -- This package is specifically forbidden
% \usepackage{titlesec} -- This package is specifically forbidden
% \usepackage{tocbibind} -- This package is specifically forbidden
% \usepackage{ulem} -- This package is specifically forbidden
% \usepackage{wrapfig} -- This package is specifically forbidden
% DISALLOWED COMMANDS
% \nocopyright -- Your paper will not be published if you use this command
% \addtolength -- This command may not be used
% \balance -- This command may not be used
% \baselinestretch -- Your paper will not be published if you use this command
% \clearpage -- No page breaks of any kind may be used for the final version of your paper
% \columnsep -- This command may not be used
% \newpage -- No page breaks of any kind may be used for the final version of your paper
% \pagebreak -- No page breaks of any kind may be used for the final version of your paperr
% \pagestyle -- This command may not be used
% \tiny -- This is not an acceptable font size.
% \vspace{- -- No negative value may be used in proximity of a caption, figure, table, section, subsection, subsubsection, or reference
% \vskip{- -- No negative value may be used to alter spacing above or below a caption, figure, table, section, subsection, subsubsection, or reference

\setcounter{secnumdepth}{0} %May be changed to 1 or 2 if section numbers are desired.

% The file aaai25.sty is the style file for AAAI Press
% proceedings, working notes, and technical reports.
%

% Title

% Your title must be in mixed case, not sentence case.
% That means all verbs (including short verbs like be, is, using,and go),
% nouns, adverbs, adjectives should be capitalized, including both words in hyphenated terms, while
% articles, conjunctions, and prepositions are lower case unless they
% directly follow a colon or long dash
\title{IAO Prompting: Making Knowledge Flow Explicit in LLMs through Structured Reasoning Templates}
\author{
    A\"issatou Diallo$^1$\thanks{ Corresponding author: \texttt{a.diallo@ucl.ac.uk}\\}, Antonis Bikakis$^2$, Luke Dickens$^2$, Anthony Hunter$^1$, Rob Miller$^2$
}
\affiliations{
    %Afiliations

    \textsuperscript{\rm 1}Department of Computer Science \\
\textsuperscript{\rm 2}Department of Information Studies     \\
    % If you have multiple authors and multiple affiliations
    % use superscripts in text and roman font to identify them.
    % For example,

    % Sunil Issar\textsuperscript{\rm 2},
    % J. Scott Penberthy\textsuperscript{\rm 3},
    % George Ferguson\textsuperscript{\rm 4},
    % Hans Guesgen\textsuperscript{\rm 5}
    % Note that the comma should be placed after the superscript

University College London, United Kingdom 
%
% See more examples next
}

%Example, Single Author, ->> remove \iffalse,\fi and place them surrounding AAAI title to use it
\iffalse
\title{My Publication Title --- Single Author}
\author {
    Author Name
}
\affiliations{
    Affiliation\\
    Affiliation Line 2\\
    name@example.com
}
\fi

\iffalse
%Example, Multiple Authors, ->> remove \iffalse,\fi and place them surrounding AAAI title to use it
\title{My Publication Title --- Multiple Authors}
\author {
    % Authors
    First Author Name\textsuperscript{\rm 1},
    Second Author Name\textsuperscript{\rm 2},
    Third Author Name\textsuperscript{\rm 1}
}
\affiliations {
    % Affiliations
    \textsuperscript{\rm 1}Affiliation 1\\
    \textsuperscript{\rm 2}Affiliation 2\\
    firstAuthor@affiliation1.com, secondAuthor@affilation2.com, thirdAuthor@affiliation1.com
}
\fi


% REMOVE THIS: bibentry
% This is only needed to show inline citations in the guidelines document. You should not need it and can safely delete it.
\usepackage{bibentry}
% END REMOVE bibentry

\begin{document}

\maketitle

\begin{abstract}
While Large Language Models (LLMs) demonstrate impressive reasoning capabilities, understanding and validating their knowledge utilization remains challenging. Chain-of-thought (CoT) prompting partially addresses this by revealing intermediate reasoning steps, but the knowledge flow and application remain implicit. We introduce IAO (Input-Action-Output) prompting, a structured template-based method that explicitly models how LLMs access and apply their knowledge during complex reasoning tasks. IAO decomposes problems into sequential steps, each clearly identifying the input knowledge being used, the action being performed, and the resulting output. This structured decomposition enables us to trace knowledge flow, verify factual consistency, and identify potential knowledge gaps or misapplications. Through experiments across diverse reasoning tasks, we demonstrate that IAO not only improves zero-shot performance but also provides transparency in how LLMs leverage their stored knowledge. Human evaluation confirms that this structured approach enhances our ability to verify knowledge utilization and detect potential hallucinations or reasoning errors. Our findings provide insights into both knowledge representation within LLMs and methods for more reliable knowledge application.
\end{abstract}

% Uncomment the following to link to your code, datasets, an extended version or similar.
%
% \begin{links}
%     \link{Code}{https://aaai.org/example/code}
%     \link{Datasets}{https://aaai.org/example/datasets}
%     \link{Extended version}{https://aaai.org/example/extended-version}
% \end{links}

\section{Introduction}

\section{Introduction}


\begin{figure}[t]
\centering
\includegraphics[width=0.6\columnwidth]{figures/evaluation_desiderata_V5.pdf}
\vspace{-0.5cm}
\caption{\systemName is a platform for conducting realistic evaluations of code LLMs, collecting human preferences of coding models with real users, real tasks, and in realistic environments, aimed at addressing the limitations of existing evaluations.
}
\label{fig:motivation}
\end{figure}

\begin{figure*}[t]
\centering
\includegraphics[width=\textwidth]{figures/system_design_v2.png}
\caption{We introduce \systemName, a VSCode extension to collect human preferences of code directly in a developer's IDE. \systemName enables developers to use code completions from various models. The system comprises a) the interface in the user's IDE which presents paired completions to users (left), b) a sampling strategy that picks model pairs to reduce latency (right, top), and c) a prompting scheme that allows diverse LLMs to perform code completions with high fidelity.
Users can select between the top completion (green box) using \texttt{tab} or the bottom completion (blue box) using \texttt{shift+tab}.}
\label{fig:overview}
\end{figure*}

As model capabilities improve, large language models (LLMs) are increasingly integrated into user environments and workflows.
For example, software developers code with AI in integrated developer environments (IDEs)~\citep{peng2023impact}, doctors rely on notes generated through ambient listening~\citep{oberst2024science}, and lawyers consider case evidence identified by electronic discovery systems~\citep{yang2024beyond}.
Increasing deployment of models in productivity tools demands evaluation that more closely reflects real-world circumstances~\citep{hutchinson2022evaluation, saxon2024benchmarks, kapoor2024ai}.
While newer benchmarks and live platforms incorporate human feedback to capture real-world usage, they almost exclusively focus on evaluating LLMs in chat conversations~\citep{zheng2023judging,dubois2023alpacafarm,chiang2024chatbot, kirk2024the}.
Model evaluation must move beyond chat-based interactions and into specialized user environments.



 

In this work, we focus on evaluating LLM-based coding assistants. 
Despite the popularity of these tools---millions of developers use Github Copilot~\citep{Copilot}---existing
evaluations of the coding capabilities of new models exhibit multiple limitations (Figure~\ref{fig:motivation}, bottom).
Traditional ML benchmarks evaluate LLM capabilities by measuring how well a model can complete static, interview-style coding tasks~\citep{chen2021evaluating,austin2021program,jain2024livecodebench, white2024livebench} and lack \emph{real users}. 
User studies recruit real users to evaluate the effectiveness of LLMs as coding assistants, but are often limited to simple programming tasks as opposed to \emph{real tasks}~\citep{vaithilingam2022expectation,ross2023programmer, mozannar2024realhumaneval}.
Recent efforts to collect human feedback such as Chatbot Arena~\citep{chiang2024chatbot} are still removed from a \emph{realistic environment}, resulting in users and data that deviate from typical software development processes.
We introduce \systemName to address these limitations (Figure~\ref{fig:motivation}, top), and we describe our three main contributions below.


\textbf{We deploy \systemName in-the-wild to collect human preferences on code.} 
\systemName is a Visual Studio Code extension, collecting preferences directly in a developer's IDE within their actual workflow (Figure~\ref{fig:overview}).
\systemName provides developers with code completions, akin to the type of support provided by Github Copilot~\citep{Copilot}. 
Over the past 3 months, \systemName has served over~\completions suggestions from 10 state-of-the-art LLMs, 
gathering \sampleCount~votes from \userCount~users.
To collect user preferences,
\systemName presents a novel interface that shows users paired code completions from two different LLMs, which are determined based on a sampling strategy that aims to 
mitigate latency while preserving coverage across model comparisons.
Additionally, we devise a prompting scheme that allows a diverse set of models to perform code completions with high fidelity.
See Section~\ref{sec:system} and Section~\ref{sec:deployment} for details about system design and deployment respectively.



\textbf{We construct a leaderboard of user preferences and find notable differences from existing static benchmarks and human preference leaderboards.}
In general, we observe that smaller models seem to overperform in static benchmarks compared to our leaderboard, while performance among larger models is mixed (Section~\ref{sec:leaderboard_calculation}).
We attribute these differences to the fact that \systemName is exposed to users and tasks that differ drastically from code evaluations in the past. 
Our data spans 103 programming languages and 24 natural languages as well as a variety of real-world applications and code structures, while static benchmarks tend to focus on a specific programming and natural language and task (e.g. coding competition problems).
Additionally, while all of \systemName interactions contain code contexts and the majority involve infilling tasks, a much smaller fraction of Chatbot Arena's coding tasks contain code context, with infilling tasks appearing even more rarely. 
We analyze our data in depth in Section~\ref{subsec:comparison}.



\textbf{We derive new insights into user preferences of code by analyzing \systemName's diverse and distinct data distribution.}
We compare user preferences across different stratifications of input data (e.g., common versus rare languages) and observe which affect observed preferences most (Section~\ref{sec:analysis}).
For example, while user preferences stay relatively consistent across various programming languages, they differ drastically between different task categories (e.g. frontend/backend versus algorithm design).
We also observe variations in user preference due to different features related to code structure 
(e.g., context length and completion patterns).
We open-source \systemName and release a curated subset of code contexts.
Altogether, our results highlight the necessity of model evaluation in realistic and domain-specific settings.






\section{Related Work}

\section{RELATED WORK}
\label{sec:relatedwork}
In this section, we describe the previous works related to our proposal, which are divided into two parts. In Section~\ref{sec:relatedwork_exoplanet}, we present a review of approaches based on machine learning techniques for the detection of planetary transit signals. Section~\ref{sec:relatedwork_attention} provides an account of the approaches based on attention mechanisms applied in Astronomy.\par

\subsection{Exoplanet detection}
\label{sec:relatedwork_exoplanet}
Machine learning methods have achieved great performance for the automatic selection of exoplanet transit signals. One of the earliest applications of machine learning is a model named Autovetter \citep{MCcauliff}, which is a random forest (RF) model based on characteristics derived from Kepler pipeline statistics to classify exoplanet and false positive signals. Then, other studies emerged that also used supervised learning. \cite{mislis2016sidra} also used a RF, but unlike the work by \citet{MCcauliff}, they used simulated light curves and a box least square \citep[BLS;][]{kovacs2002box}-based periodogram to search for transiting exoplanets. \citet{thompson2015machine} proposed a k-nearest neighbors model for Kepler data to determine if a given signal has similarity to known transits. Unsupervised learning techniques were also applied, such as self-organizing maps (SOM), proposed \citet{armstrong2016transit}; which implements an architecture to segment similar light curves. In the same way, \citet{armstrong2018automatic} developed a combination of supervised and unsupervised learning, including RF and SOM models. In general, these approaches require a previous phase of feature engineering for each light curve. \par

%DL is a modern data-driven technology that automatically extracts characteristics, and that has been successful in classification problems from a variety of application domains. The architecture relies on several layers of NNs of simple interconnected units and uses layers to build increasingly complex and useful features by means of linear and non-linear transformation. This family of models is capable of generating increasingly high-level representations \citep{lecun2015deep}.

The application of DL for exoplanetary signal detection has evolved rapidly in recent years and has become very popular in planetary science.  \citet{pearson2018} and \citet{zucker2018shallow} developed CNN-based algorithms that learn from synthetic data to search for exoplanets. Perhaps one of the most successful applications of the DL models in transit detection was that of \citet{Shallue_2018}; who, in collaboration with Google, proposed a CNN named AstroNet that recognizes exoplanet signals in real data from Kepler. AstroNet uses the training set of labelled TCEs from the Autovetter planet candidate catalog of Q1–Q17 data release 24 (DR24) of the Kepler mission \citep{catanzarite2015autovetter}. AstroNet analyses the data in two views: a ``global view'', and ``local view'' \citep{Shallue_2018}. \par


% The global view shows the characteristics of the light curve over an orbital period, and a local view shows the moment at occurring the transit in detail

%different = space-based

Based on AstroNet, researchers have modified the original AstroNet model to rank candidates from different surveys, specifically for Kepler and TESS missions. \citet{ansdell2018scientific} developed a CNN trained on Kepler data, and included for the first time the information on the centroids, showing that the model improves performance considerably. Then, \citet{osborn2020rapid} and \citet{yu2019identifying} also included the centroids information, but in addition, \citet{osborn2020rapid} included information of the stellar and transit parameters. Finally, \citet{rao2021nigraha} proposed a pipeline that includes a new ``half-phase'' view of the transit signal. This half-phase view represents a transit view with a different time and phase. The purpose of this view is to recover any possible secondary eclipse (the object hiding behind the disk of the primary star).


%last pipeline applies a procedure after the prediction of the model to obtain new candidates, this process is carried out through a series of steps that include the evaluation with Discovery and Validation of Exoplanets (DAVE) \citet{kostov2019discovery} that was adapted for the TESS telescope.\par
%



\subsection{Attention mechanisms in astronomy}
\label{sec:relatedwork_attention}
Despite the remarkable success of attention mechanisms in sequential data, few papers have exploited their advantages in astronomy. In particular, there are no models based on attention mechanisms for detecting planets. Below we present a summary of the main applications of this modeling approach to astronomy, based on two points of view; performance and interpretability of the model.\par
%Attention mechanisms have not yet been explored in all sub-areas of astronomy. However, recent works show a successful application of the mechanism.
%performance

The application of attention mechanisms has shown improvements in the performance of some regression and classification tasks compared to previous approaches. One of the first implementations of the attention mechanism was to find gravitational lenses proposed by \citet{thuruthipilly2021finding}. They designed 21 self-attention-based encoder models, where each model was trained separately with 18,000 simulated images, demonstrating that the model based on the Transformer has a better performance and uses fewer trainable parameters compared to CNN. A novel application was proposed by \citet{lin2021galaxy} for the morphological classification of galaxies, who used an architecture derived from the Transformer, named Vision Transformer (VIT) \citep{dosovitskiy2020image}. \citet{lin2021galaxy} demonstrated competitive results compared to CNNs. Another application with successful results was proposed by \citet{zerveas2021transformer}; which first proposed a transformer-based framework for learning unsupervised representations of multivariate time series. Their methodology takes advantage of unlabeled data to train an encoder and extract dense vector representations of time series. Subsequently, they evaluate the model for regression and classification tasks, demonstrating better performance than other state-of-the-art supervised methods, even with data sets with limited samples.

%interpretation
Regarding the interpretability of the model, a recent contribution that analyses the attention maps was presented by \citet{bowles20212}, which explored the use of group-equivariant self-attention for radio astronomy classification. Compared to other approaches, this model analysed the attention maps of the predictions and showed that the mechanism extracts the brightest spots and jets of the radio source more clearly. This indicates that attention maps for prediction interpretation could help experts see patterns that the human eye often misses. \par

In the field of variable stars, \citet{allam2021paying} employed the mechanism for classifying multivariate time series in variable stars. And additionally, \citet{allam2021paying} showed that the activation weights are accommodated according to the variation in brightness of the star, achieving a more interpretable model. And finally, related to the TESS telescope, \citet{morvan2022don} proposed a model that removes the noise from the light curves through the distribution of attention weights. \citet{morvan2022don} showed that the use of the attention mechanism is excellent for removing noise and outliers in time series datasets compared with other approaches. In addition, the use of attention maps allowed them to show the representations learned from the model. \par

Recent attention mechanism approaches in astronomy demonstrate comparable results with earlier approaches, such as CNNs. At the same time, they offer interpretability of their results, which allows a post-prediction analysis. \par



\section{IAO Prompting}
\section{Temporal Representation Alignment}
\label{sec:approach}

When training a series of short-horizon goal-reaching and instruction-following tasks, our goal is to learn a representation space such that our policy can generalize to a new (long-horizon) task that can be viewed as a sequence of known subtasks.
We propose to structure this representation space by aligning the representations of states, goals, and language in a way that is more amenable to compositional generalization.

\paragraph{Notation.}
We take the setting of a goal- and language-conditioned MDP $\cM$ with state space $\cS$, continuous action space $\cA \subseteq (0,1)^{d_{\cA}}$, initial state distribution $p_0$, dynamics $\p(s'\mid s,a)$, discount factor $\gamma$, and language task distribution $p_{\ell}$.
A policy $\pi(a\mid s)$ maps states to a distribution over actions. We inductively define the $k$-step (action-conditioned) policy visitation distribution as:
\begin{align*}
    p^{\pi}_{1}(s_{1} \mid s_1, a_{1})
    &\triangleq p(s_1 \mid s_1, a_1),\\
    p^{\pi}_{k+1}(s_{k+1} \mid s_1, a_1)
    &\triangleq \nonumber\\*
      &\mspace{-120mu} \int_{\cA}\int_{\cS} p(s_{k+1} \mid s,a) \dd p^{\pi}_{k}(s \mid s_{1},a_1) \dd
        \pi(a \mid s)\\
    p^{\pi}_{k+t}(s_{k+t} \mid s_t,a_t)
    &\triangleq p^{\pi}(s_{k} \mid s_1, a_1) . \eqmark
        \label{eq:successor_distribution}
\end{align*}
Then, the discounted state visitation distribution can be defined as the distribution over $s^{+}$\llap, the state reached after $K\sim \operatorname{Geom}(1-\gamma)$ steps:
\begin{equation}
    p^{\pi}_{\gamma}(s^{+}  \mid  s,a) \triangleq \sum_{k=0}^{\infty} \gamma^{k} p^{\pi}_{k}(s^{+} \mid s,a).
    \label{eq:discounted_state_visitation}
\end{equation}

We assume access to a dataset of expert demonstrations $\cD = \{\tau_{i},\ell_i\}_{i=1}^{K}$, where each trajectory
\begin{equation}
    \tau_{i} = \{s_{t,i},a_{t,i}\}_{t=1}^{H} \in \cS \times \cA
    \label{eq:trajectory}
\end{equation}
is gathered by an expert policy $\expert$, and is then annotated with $p_{\ell}(\ell_{i} \mid s_{1,i}, s_{H,i})$.
Our aim is to learn a policy $\pi$ that can select actions conditioned on a new language instruction $\ell$.
As in prior work~\citep{walke2023bridgedata}, we handle the continuous action space by representing both our policy and the expert policy as an isotropic Gaussian with fixed variance; we will equivalently write $\pi(a\mid s, \varphi)$ or denote the mode as $\hat{a} = \pi(s,\varphi)$ for a task $\varphi$.

\begin{rebuttal}
    \subsection{Representations for Reaching Distant Goals}
    \label{sec:reaching_goals}

    We learn a goal-conditioned policy $\pi(a\mid s,g)$ that selects actions to reach a goal $g$ from expert demonstrations with behavioral cloning.
    Suppose we directly selected actions to imitate the expert on two trajectories in $\cD$:
    
    \begin{equation}
        \mspace{-100mu}\begin{tikzcd}[remember picture,sep=small]
            s_1 \rar & s_2 \rar  & \ldots \rar & s_{H} \rar & w      \quad \\
            w \rar   & s_1' \rar & \ldots \rar & s_{H}' \rar & g\quad
        \end{tikzcd}
        \begin{tikzpicture}[remember picture,overlay] \coordinate (a) at (\tikzcdmatrixname-1-5.north east);
            \coordinate (b) at (\tikzcdmatrixname-2-5.south east);
            \coordinate (c) at (a|-b);
            \draw[decorate,line width=1.5pt,decoration={brace,raise=3pt,amplitude=5pt}]
        (a) -- node[right=1.5em] {$\tau_{i}\in \cD$} (c); \end{tikzpicture}
        \label{eq:trajectory_diagram}
    \end{equation}
    When conditioned with the composed goal $g$, we would be unable to imitate effectively
        as the composed state-goal $(s,g)$ is jointly out of the training distribution.

    What \emph{would} work for reaching $g$ is to first condition the policy on the intermediate waypoint $w$, then upon reaching $w$, condition on the goal $g$, as the state-goal pairs $(s_{i},w)$, $(w,g)$, and $(s_{i}',g)$ are all in the training distribution.
    If we condition the policy on some intermediate waypoint distribution $p(w)$ (or sufficient statistics thereof) that captures all of these cases, we can stitch together the expert behaviors to reach the goal $g$.

    Our approach is to learn a representation space that captures this ability, so that a GCBC objective used in this space can effectively imitate the expert on the composed task.
     We begin with the goal-conditioned behavioral cloning~\citep{kaelbling1993learning}
        loss $\cL_{\textsc{bc}}^{\phi,\psi,\xi}$ conditioned with waypoints $w$.
    \begin{equation}
        \cL_{\textsc{bc}}\bigl(\{s_{i},a_{i},s_{i}^{+},g_{i}\}_{i=1}^{K}\bigr) = \sum_{{i=1}}^{K} \log \pi\bigl(a_{i} \mid s_{i},\psi(g_{i})\bigr).
        \label{eq:goal_conditioned_bc}
    \end{equation}
    Enforcing the invariance needed to stitch \cref{eq:trajectory_diagram} then reduces to aligning \mbox{$\psi(g) \leftrightarrow \psi(w).$}
    The temporal alignment objective $\phi(s)\leftrightarrow \phi(s^{+})$ accomplishes this indirectly by aligning both $\psi(w)$ and $\psi(g)$ to the shared waypoint representation $\phi(w)$:

    \csuse{color indices}
    \begin{align}
        &\cL_{\textsc{nce}}\bigl(\{s_{i},s_{i}^{+}\}_{i=1}^{K};\phi,\psi\bigr) =
        \log \biggl( {\frac{e^{\phi(s^+_{\i})^{T}\psi(s_{\i})}}{\sum_{{\j=1}}^{K}
                e^{\phi(s^+_{\i})^{T}\psi(s_{\j})}}} \biggr)  \nonumber\\*
                &\mspace{100mu} +
        \sum_{{\j=1}}^{K} \log \biggl( {\frac{e^{\phi(s^+_{\i})^{T}\psi(s_{\i})}}{\sum_{{\i=1}}^{K}
                e^{\phi(s^+_{\i})^{T}\psi(s_{\j})}}} \biggr)
        \label{eq:goal_alignment}
    \end{align}

        
\end{rebuttal}
\subsection{Interfacing with Language Instructions}
\label{sec:language_instructions}

To extend the representations from \cref{sec:reaching_goals} to compositional instruction following with language tasks, we need some way to ground language into the $\psi$ (future state)
representation space.
We use a similar approach to GRIF~\citep{myers2023goal}, which uses an additional CLIP-style \citep{radford2021learning} contrastive alignment loss with an additional pretrained language encoder $\xi$:
\csuse{no color indices}
\begin{align}
    &\cL_{\textsc{nce}}\bigl(\{g_{i},\ell_{i}\}_{i=1}^{K};\psi,\xi\bigr)
    = \sum_{{i=1}}^{K} \log \biggl( {\frac{e^{\psi(g_{\i})^{T}\xi(\ell_{\i})}}{\sum_{{\j=1}}^{K}
            e^{\psi(g_{\i})^{T}\xi(\ell_{\j})}}} \biggr)  \nonumber\\*
            &\mspace{100mu} +
    \sum_{{\j=1}}^{K} \log \biggl( {\frac{e^{\psi(g_{\i})^{T}\xi(\ell_{\i})}}{\sum_{{\i=1}}^{K}
            e^{\psi(g_{\i})^{T}\xi(\ell_{\j})}}} \biggr)
    \label{eq:task_alignment}
\end{align}

\subsection{Temporal Alignment}
\label{sec:temporal_alignment}

Putting together the objectives from \cref{sec:reaching_goals,sec:language_instructions} yields the Temporal Representation Alignment (\Method) approach.
\Method{} structures the representation space of goals and language instructions to better enable compositional generalization.
We learn encoders $\phi, \psi ,$ and $\xi$ to map states, goals, and language instructions to a shared representation space.

\csuse{color indices}
\begin{align}
    \cL_{\textsc{nce}} \label{eq:NCE}
    &(\{x_{i}, y_{i}\}_{i=1}^{K};f,h) =
        \sum_{{\i=1}}^{K} \log \biggl( {\frac{e^{f(y_{\i})^{T}h(x_{\i})}}{\sum_{{\j=1}}^{K}
        e^{f(y_{\i})^{T}h(x_{\j})}}} \biggr) \nonumber\\*
      &\mspace{100mu} +
        \sum_{{\j=1}}^{K} \log \biggl( {\frac{e^{f(y_{\i})^{T}h(x_{\i})}}{\sum_{{\i=1}}^{K}
        e^{f(y_{\i})^{T}h(x_{\j})}}} \biggr) \\
    \cL_{\textsc{bc}} \label{eq:BC}
    &\bigl(\{s_{i},a_{i},s^{+}_{i},\ell_{i}\}_{i=1}^{K};\pi,\psi,\xi\bigr) = \nonumber\\*
      &\mspace{-10mu} \sum_{{i=1}}^{K} \log
        \pi\bigl(a_{i} \mid s_{i},\xi(\ell_{i})\bigr) + \log \pi\bigl(a_{i} \mid
        s_{i},\psi(s^{+}_{i})\bigr) \\
    \cL_{\textsc{tra}}
    &\label{eq:TRA} \bigl( \{s_{i},a_{i},s_{i}^{+},g_{i},\ell_{i}\}_{i=1}^{K}; \pi,\phi,\psi,\xi\bigr)
        \\
    &= \underbrace{\cL_{\textsc{bc}}\bigl(\{s_{i},a_{i},s_{i}^{+},\ell_{i}\}_{i=1}^{K};\pi,\psi,\xi\bigr)}_{\text{behavioral
    cloning}} \nonumber\\*
    &+
        \underbrace{\cL_{\textsc{nce}}\bigl(\{s_{i},s_{i}^{+}\}_{i=1}^{K};\phi,\psi\bigr)}_{\text{temporal alignment}}
        + \underbrace{\cL_{\textsc{nce}}\bigl(\{g_{i},\ell_{i}\}_{i=1}^{K};\psi,\xi\bigr)}_{\text{task alignment}} \nonumber
\end{align}Note that the NCE alignment loss uses a CLIP-style symmetric contrastive objective~\citep{radford2021learning,eysenbach2024inference} \-- we highlight the indices in the NCE alignment loss~\eqref{eq:NCE} for clarity.

Our overall objective is to minimize \cref{eq:TRA} across states, actions, future states, goals, and language tasks within the training data:
\begin{align}
    &\min_{\pi,\phi,\psi,\xi} \mathbb{E}_{\substack{(s_{1,i},a_{1,i},\ldots,s_{H,i},a_{H,i},\ell) \sim \mathcal{D} \\
    i\sim\operatorname{Unif}(1\ldots H) \\
    k\sim\operatorname{Geom}(1-\gamma)}} \\*
    &\mspace{10mu}
    \Bigl[\cL_{\text{TRA}}\bigl(\{s_{t,i},a_{t,i},s_{\min(t+k,H),i},s_{H,i},\ell\}_{i=1}^{K};\pi,\phi,\psi,\xi\bigr)\Bigr].
    \label{eq:overall_objective}
\end{align}

\begin{algorithm}
    \caption{Temporal Representation Alignment}
    \label{alg:tra}
    \begin{algorithmic}[1]
        \State \textbf{input:} dataset $\mathcal{D} = (\{s_{t,i},a_{t,i}\}_{t=1}^{H},\ell_i)_{i=1}^N$
        \State initialize networks $\Theta \triangleq (\pi,\phi,\psi,\xi)$
        \While{training}
        \State sample batch $\bigl\{(s_{t,i},a_{t,i},s_{t+k,i},\ell_i)\bigr\}_{i=1}^K\sim\mathcal{D}$ \\
        \hspace*{2ex} for $k\sim\operatorname{Geom}(1-\gamma)$
        \State $\Theta \gets \Theta - \alpha \nabla_{\Theta} \cL_{\text{TRA}}\bigl(\{s_{t,i},a_{t,i},s_{t+k,i},\ell_i\}_{i=1}^K; \Theta\bigr)$
        \EndWhile
        \smallskip
        \State \textbf{output:} \parbox[t]{\linewidth}{language-conditioned policy $\pi(a_{t} | s_{t}, \xi(\ell))$ \\
            goal-conditioned policy $\pi(a_{t} | s_{t}, \psi(g))$
        }
    \end{algorithmic}
\end{algorithm}

\subsection{Implementation}
\label{sec:implementation}

A summary of our approach is shown in \cref{alg:tra}.
In essence, TRA learns three encoders: $\phi$, which encodes states, $\psi$ which encodes future goals, and $\xi$ which encodes language instructions.
Contrastive losses are used to align state representations $\phi(s_{t})$ with future goal representations $\psi(s_{t+k})$, which are in turn aligned with equivalent language task specifications $\xi(\ell)$ when available.
We then learn a behavior cloning policy $\pi$ that can be conditioned on either the goal or language instruction through the representation $\psi(g)$ or $\xi(\ell)$, respectively.

\begin{rebuttal}
    \subsection{Temporal Alignment and Compositionality}
    \label{sec:compositionality}

    We will formalize the intuition from \cref{sec:reaching_goals} that \Method{} enables compositional generalization by considering the error on a ``compositional'' version of $\cD,$ denoted $\cD^{*}$.
    Using the notation from \cref{eq:trajectory}, we can say $\cD$ is distributed according to:
    \begin{align}
        &\cD \triangleq \cD^{H} \sim \prod_{i=1}^{K} p_0(s_{1,i}) p_{\ell}(\ell_{i} \mid s_{1,i}, s_{H,i})
            \nonumber\\*
          &\mspace{60mu} \prod_{t=1}^{H} \expert(a_{t,i} \mid s_{t,i}) \p(s_{t+1,i} \mid s_{t,i}, a_{t,i}) ,
            \label{eq:dataset_distribution}
    \end{align}
    or equivalently
    \begin{align}
            &\cD^{H} \sim \prod_{i=1}^{K} p_0(s_{1,i}) p_{\ell}(\ell_{i} \mid s_{1,i}, s_{H,i}) \nonumber\\*
            &\mspace{60mu} \prod_{t=1}^{H}
            e^{\sigma^2\|\expert(s_{t,i}) - a_{t,i}\|^2}\p(s_{t+1,i} \mid s_{t,i}, a_{t,i}) ,
            \label{eq:dataset_distribution_2}
    \end{align}
    by the isotropic Gaussian assumption.
    We will define $\cD^{*} \triangleq \cD^{H'}$ to be a longer-horizon version of $\cD$ extending the behaviors gathered under $\expert$ across a horizon $\alpha H \ge H' \ge H$ that additionally satisfies a ``time-isotropy'' property: the marginal distribution of the states is uniform across the horizon, i.e., $p_0(s_{1,i}) = p_0(s_{t,i})$ for all $t \in \{1\ldots H'\}$.

    We will relate the in-distribution imitation error $\textsc{Err}(\bullet; \cD)$ to the compositional out-of-distribution imitation error $\textsc{Err}(\bullet;\cD^{*})$.
    We define
    \begin{align}
        \textsc{Err}(\hat{\pi}; \tilde{\cD})
        &= \E_{\tilde{\cD}}\Bigl[\frac{1}{H}\sum_{t=1}^{H} \mathbb{E}_{\hat{\pi}}\left[\|\tilde{a}_{t,i} -
        \hat{\pi}(\tilde{s}_{t,i}, \tilde{s}_{H, i})\|^{2}/d_{\cA}\right]\Bigr] \nonumber\\
        &\quad \text{for} \quad \{\tilde{s}_{t,i},\tilde{a}_{t,i},\tilde{\ell}_{i}\}_{t=1}^{H} \sim
            \tilde{\cD}.
            \label{eq:imitation_error}
    \end{align}
    On the training dataset this is equivalent to the expected behavioral cloning loss from \cref{eq:BC}.

                            
    \begin{assumption}
        \label{asm:policy_factorization}
        The policy factorizes through inferred waypoints as:
\begin{align}
    &\textrm{goals: }\pi(a \mid s, g)
        = \nonumber\\*
      &\mspace{50mu} \int \pi(a\mid s, w) \p(s_{t}=w \mid s_{t+k}=g) \dd{w}
        \label{eq:goal-conditioned} \\
    &\textrm{language: } \pi(a \mid s, \ell)
        = \int \pi(a\mid s, w) \nonumber\\*
      &\mspace{20mu} \p(s_{t}=w \mid s_{t+k}=g) \p(s_{t+k}=g \mid \ell) \dd{w} \dd{g} ,
        \label{eq:language-conditioned}
        \end{align}
        where denote by $\pi(s,g)$ the MLE estimate of the action $a$.

    \end{assumption}

    \makerestatable
    \begin{theorem}
        \label{thm:compositionality}
        Suppose $\cD$ is distributed according to \cref{eq:dataset_distribution} and $\cD^{*}$ is distributed according to \cref{eq:dataset_distribution}.
        When $\gamma > 1-1/H$ and $\alpha > 1$, for optimal features $\phi$ and $\psi$ under \cref{eq:overall_objective}, we have
        \begin{gather}
            \textsc{Err}(\pi; \cD^{*}) \le \textsc{Err}(\pi; \cD) +  \frac{\alpha -1}{2 \alpha }+\Bigl(\frac{ \alpha - 2 }{2\alpha}\Bigr) \1 \{\alpha >2\}  .
            \label{eq:compositionality}
        \end{gather}
    \end{theorem}

    We can also define a notion of the language-conditioned compositional generalization error:
    \begin{equation*}
        \errl(\pi; \cD^{*}) \triangleq \E_{\cD^{*}}\Bigl[\frac{1}{H}\sum_{t=1}^{H}
            \mathbb{E}_{\pi}\bigl[\|\tilde{a}_{t,i} - \pi(\tilde{s}_{t,i}, \tilde{\ell}_{i})\|^{2}\bigr]\Bigr].
            \label{eq:language_error}
    \end{equation*}

    \makerestatable
    \begin{corollary}
        \label{thm:language}
        Under the same conditions as \cref{thm:compositionality},
        \begin{equation*}
            \errl(\pi; \cD^{*}) \le \errl(\pi; \cD) +  \frac{\alpha -1}{2 \alpha }+\Bigl(\frac{ \alpha - 2 }{2\alpha}\Bigr) \1 \{\alpha >2\}  .
            \label{eq:compositionality_language}
        \end{equation*}

    \end{corollary}

    The proofs as well as a visualization of the bound are in \cref{app:compositionality}. Policy implementation details can be found in \cref{app:tra_impl}

    
                
        
    \end{rebuttal}


\section{Experimental Setup}
\section{Experiments}
\label{sec:experiments}
The experiments are designed to address two key research questions.
First, \textbf{RQ1} evaluates whether the average $L_2$-norm of the counterfactual perturbation vectors ($\overline{||\perturb||}$) decreases as the model overfits the data, thereby providing further empirical validation for our hypothesis.
Second, \textbf{RQ2} evaluates the ability of the proposed counterfactual regularized loss, as defined in (\ref{eq:regularized_loss2}), to mitigate overfitting when compared to existing regularization techniques.

% The experiments are designed to address three key research questions. First, \textbf{RQ1} investigates whether the mean perturbation vector norm decreases as the model overfits the data, aiming to further validate our intuition. Second, \textbf{RQ2} explores whether the mean perturbation vector norm can be effectively leveraged as a regularization term during training, offering insights into its potential role in mitigating overfitting. Finally, \textbf{RQ3} examines whether our counterfactual regularizer enables the model to achieve superior performance compared to existing regularization methods, thus highlighting its practical advantage.

\subsection{Experimental Setup}
\textbf{\textit{Datasets, Models, and Tasks.}}
The experiments are conducted on three datasets: \textit{Water Potability}~\cite{kadiwal2020waterpotability}, \textit{Phomene}~\cite{phomene}, and \textit{CIFAR-10}~\cite{krizhevsky2009learning}. For \textit{Water Potability} and \textit{Phomene}, we randomly select $80\%$ of the samples for the training set, and the remaining $20\%$ for the test set, \textit{CIFAR-10} comes already split. Furthermore, we consider the following models: Logistic Regression, Multi-Layer Perceptron (MLP) with 100 and 30 neurons on each hidden layer, and PreactResNet-18~\cite{he2016cvecvv} as a Convolutional Neural Network (CNN) architecture.
We focus on binary classification tasks and leave the extension to multiclass scenarios for future work. However, for datasets that are inherently multiclass, we transform the problem into a binary classification task by selecting two classes, aligning with our assumption.

\smallskip
\noindent\textbf{\textit{Evaluation Measures.}} To characterize the degree of overfitting, we use the test loss, as it serves as a reliable indicator of the model's generalization capability to unseen data. Additionally, we evaluate the predictive performance of each model using the test accuracy.

\smallskip
\noindent\textbf{\textit{Baselines.}} We compare CF-Reg with the following regularization techniques: L1 (``Lasso''), L2 (``Ridge''), and Dropout.

\smallskip
\noindent\textbf{\textit{Configurations.}}
For each model, we adopt specific configurations as follows.
\begin{itemize}
\item \textit{Logistic Regression:} To induce overfitting in the model, we artificially increase the dimensionality of the data beyond the number of training samples by applying a polynomial feature expansion. This approach ensures that the model has enough capacity to overfit the training data, allowing us to analyze the impact of our counterfactual regularizer. The degree of the polynomial is chosen as the smallest degree that makes the number of features greater than the number of data.
\item \textit{Neural Networks (MLP and CNN):} To take advantage of the closed-form solution for computing the optimal perturbation vector as defined in (\ref{eq:opt-delta}), we use a local linear approximation of the neural network models. Hence, given an instance $\inst_i$, we consider the (optimal) counterfactual not with respect to $\model$ but with respect to:
\begin{equation}
\label{eq:taylor}
    \model^{lin}(\inst) = \model(\inst_i) + \nabla_{\inst}\model(\inst_i)(\inst - \inst_i),
\end{equation}
where $\model^{lin}$ represents the first-order Taylor approximation of $\model$ at $\inst_i$.
Note that this step is unnecessary for Logistic Regression, as it is inherently a linear model.
\end{itemize}

\smallskip
\noindent \textbf{\textit{Implementation Details.}} We run all experiments on a machine equipped with an AMD Ryzen 9 7900 12-Core Processor and an NVIDIA GeForce RTX 4090 GPU. Our implementation is based on the PyTorch Lightning framework. We use stochastic gradient descent as the optimizer with a learning rate of $\eta = 0.001$ and no weight decay. We use a batch size of $128$. The training and test steps are conducted for $6000$ epochs on the \textit{Water Potability} and \textit{Phoneme} datasets, while for the \textit{CIFAR-10} dataset, they are performed for $200$ epochs.
Finally, the contribution $w_i^{\varepsilon}$ of each training point $\inst_i$ is uniformly set as $w_i^{\varepsilon} = 1~\forall i\in \{1,\ldots,m\}$.

The source code implementation for our experiments is available at the following GitHub repository: \url{https://anonymous.4open.science/r/COCE-80B4/README.md} 

\subsection{RQ1: Counterfactual Perturbation vs. Overfitting}
To address \textbf{RQ1}, we analyze the relationship between the test loss and the average $L_2$-norm of the counterfactual perturbation vectors ($\overline{||\perturb||}$) over training epochs.

In particular, Figure~\ref{fig:delta_loss_epochs} depicts the evolution of $\overline{||\perturb||}$ alongside the test loss for an MLP trained \textit{without} regularization on the \textit{Water Potability} dataset. 
\begin{figure}[ht]
    \centering
    \includegraphics[width=0.85\linewidth]{img/delta_loss_epochs.png}
    \caption{The average counterfactual perturbation vector $\overline{||\perturb||}$ (left $y$-axis) and the cross-entropy test loss (right $y$-axis) over training epochs ($x$-axis) for an MLP trained on the \textit{Water Potability} dataset \textit{without} regularization.}
    \label{fig:delta_loss_epochs}
\end{figure}

The plot shows a clear trend as the model starts to overfit the data (evidenced by an increase in test loss). 
Notably, $\overline{||\perturb||}$ begins to decrease, which aligns with the hypothesis that the average distance to the optimal counterfactual example gets smaller as the model's decision boundary becomes increasingly adherent to the training data.

It is worth noting that this trend is heavily influenced by the choice of the counterfactual generator model. In particular, the relationship between $\overline{||\perturb||}$ and the degree of overfitting may become even more pronounced when leveraging more accurate counterfactual generators. However, these models often come at the cost of higher computational complexity, and their exploration is left to future work.

Nonetheless, we expect that $\overline{||\perturb||}$ will eventually stabilize at a plateau, as the average $L_2$-norm of the optimal counterfactual perturbations cannot vanish to zero.

% Additionally, the choice of employing the score-based counterfactual explanation framework to generate counterfactuals was driven to promote computational efficiency.

% Future enhancements to the framework may involve adopting models capable of generating more precise counterfactuals. While such approaches may yield to performance improvements, they are likely to come at the cost of increased computational complexity.


\subsection{RQ2: Counterfactual Regularization Performance}
To answer \textbf{RQ2}, we evaluate the effectiveness of the proposed counterfactual regularization (CF-Reg) by comparing its performance against existing baselines: unregularized training loss (No-Reg), L1 regularization (L1-Reg), L2 regularization (L2-Reg), and Dropout.
Specifically, for each model and dataset combination, Table~\ref{tab:regularization_comparison} presents the mean value and standard deviation of test accuracy achieved by each method across 5 random initialization. 

The table illustrates that our regularization technique consistently delivers better results than existing methods across all evaluated scenarios, except for one case -- i.e., Logistic Regression on the \textit{Phomene} dataset. 
However, this setting exhibits an unusual pattern, as the highest model accuracy is achieved without any regularization. Even in this case, CF-Reg still surpasses other regularization baselines.

From the results above, we derive the following key insights. First, CF-Reg proves to be effective across various model types, ranging from simple linear models (Logistic Regression) to deep architectures like MLPs and CNNs, and across diverse datasets, including both tabular and image data. 
Second, CF-Reg's strong performance on the \textit{Water} dataset with Logistic Regression suggests that its benefits may be more pronounced when applied to simpler models. However, the unexpected outcome on the \textit{Phoneme} dataset calls for further investigation into this phenomenon.


\begin{table*}[h!]
    \centering
    \caption{Mean value and standard deviation of test accuracy across 5 random initializations for different model, dataset, and regularization method. The best results are highlighted in \textbf{bold}.}
    \label{tab:regularization_comparison}
    \begin{tabular}{|c|c|c|c|c|c|c|}
        \hline
        \textbf{Model} & \textbf{Dataset} & \textbf{No-Reg} & \textbf{L1-Reg} & \textbf{L2-Reg} & \textbf{Dropout} & \textbf{CF-Reg (ours)} \\ \hline
        Logistic Regression   & \textit{Water}   & $0.6595 \pm 0.0038$   & $0.6729 \pm 0.0056$   & $0.6756 \pm 0.0046$  & N/A    & $\mathbf{0.6918 \pm 0.0036}$                     \\ \hline
        MLP   & \textit{Water}   & $0.6756 \pm 0.0042$   & $0.6790 \pm 0.0058$   & $0.6790 \pm 0.0023$  & $0.6750 \pm 0.0036$    & $\mathbf{0.6802 \pm 0.0046}$                    \\ \hline
%        MLP   & \textit{Adult}   & $0.8404 \pm 0.0010$   & $\mathbf{0.8495 \pm 0.0007}$   & $0.8489 \pm 0.0014$  & $\mathbf{0.8495 \pm 0.0016}$     & $0.8449 \pm 0.0019$                    \\ \hline
        Logistic Regression   & \textit{Phomene}   & $\mathbf{0.8148 \pm 0.0020}$   & $0.8041 \pm 0.0028$   & $0.7835 \pm 0.0176$  & N/A    & $0.8098 \pm 0.0055$                     \\ \hline
        MLP   & \textit{Phomene}   & $0.8677 \pm 0.0033$   & $0.8374 \pm 0.0080$   & $0.8673 \pm 0.0045$  & $0.8672 \pm 0.0042$     & $\mathbf{0.8718 \pm 0.0040}$                    \\ \hline
        CNN   & \textit{CIFAR-10} & $0.6670 \pm 0.0233$   & $0.6229 \pm 0.0850$   & $0.7348 \pm 0.0365$   & N/A    & $\mathbf{0.7427 \pm 0.0571}$                     \\ \hline
    \end{tabular}
\end{table*}

\begin{table*}[htb!]
    \centering
    \caption{Hyperparameter configurations utilized for the generation of Table \ref{tab:regularization_comparison}. For our regularization the hyperparameters are reported as $\mathbf{\alpha/\beta}$.}
    \label{tab:performance_parameters}
    \begin{tabular}{|c|c|c|c|c|c|c|}
        \hline
        \textbf{Model} & \textbf{Dataset} & \textbf{No-Reg} & \textbf{L1-Reg} & \textbf{L2-Reg} & \textbf{Dropout} & \textbf{CF-Reg (ours)} \\ \hline
        Logistic Regression   & \textit{Water}   & N/A   & $0.0093$   & $0.6927$  & N/A    & $0.3791/1.0355$                     \\ \hline
        MLP   & \textit{Water}   & N/A   & $0.0007$   & $0.0022$  & $0.0002$    & $0.2567/1.9775$                    \\ \hline
        Logistic Regression   &
        \textit{Phomene}   & N/A   & $0.0097$   & $0.7979$  & N/A    & $0.0571/1.8516$                     \\ \hline
        MLP   & \textit{Phomene}   & N/A   & $0.0007$   & $4.24\cdot10^{-5}$  & $0.0015$    & $0.0516/2.2700$                    \\ \hline
       % MLP   & \textit{Adult}   & N/A   & $0.0018$   & $0.0018$  & $0.0601$     & $0.0764/2.2068$                    \\ \hline
        CNN   & \textit{CIFAR-10} & N/A   & $0.0050$   & $0.0864$ & N/A    & $0.3018/
        2.1502$                     \\ \hline
    \end{tabular}
\end{table*}

\begin{table*}[htb!]
    \centering
    \caption{Mean value and standard deviation of training time across 5 different runs. The reported time (in seconds) corresponds to the generation of each entry in Table \ref{tab:regularization_comparison}. Times are }
    \label{tab:times}
    \begin{tabular}{|c|c|c|c|c|c|c|}
        \hline
        \textbf{Model} & \textbf{Dataset} & \textbf{No-Reg} & \textbf{L1-Reg} & \textbf{L2-Reg} & \textbf{Dropout} & \textbf{CF-Reg (ours)} \\ \hline
        Logistic Regression   & \textit{Water}   & $222.98 \pm 1.07$   & $239.94 \pm 2.59$   & $241.60 \pm 1.88$  & N/A    & $251.50 \pm 1.93$                     \\ \hline
        MLP   & \textit{Water}   & $225.71 \pm 3.85$   & $250.13 \pm 4.44$   & $255.78 \pm 2.38$  & $237.83 \pm 3.45$    & $266.48 \pm 3.46$                    \\ \hline
        Logistic Regression   & \textit{Phomene}   & $266.39 \pm 0.82$ & $367.52 \pm 6.85$   & $361.69 \pm 4.04$  & N/A   & $310.48 \pm 0.76$                    \\ \hline
        MLP   &
        \textit{Phomene} & $335.62 \pm 1.77$   & $390.86 \pm 2.11$   & $393.96 \pm 1.95$ & $363.51 \pm 5.07$    & $403.14 \pm 1.92$                     \\ \hline
       % MLP   & \textit{Adult}   & N/A   & $0.0018$   & $0.0018$  & $0.0601$     & $0.0764/2.2068$                    \\ \hline
        CNN   & \textit{CIFAR-10} & $370.09 \pm 0.18$   & $395.71 \pm 0.55$   & $401.38 \pm 0.16$ & N/A    & $1287.8 \pm 0.26$                     \\ \hline
    \end{tabular}
\end{table*}

\subsection{Feasibility of our Method}
A crucial requirement for any regularization technique is that it should impose minimal impact on the overall training process.
In this respect, CF-Reg introduces an overhead that depends on the time required to find the optimal counterfactual example for each training instance. 
As such, the more sophisticated the counterfactual generator model probed during training the higher would be the time required. However, a more advanced counterfactual generator might provide a more effective regularization. We discuss this trade-off in more details in Section~\ref{sec:discussion}.

Table~\ref{tab:times} presents the average training time ($\pm$ standard deviation) for each model and dataset combination listed in Table~\ref{tab:regularization_comparison}.
We can observe that the higher accuracy achieved by CF-Reg using the score-based counterfactual generator comes with only minimal overhead. However, when applied to deep neural networks with many hidden layers, such as \textit{PreactResNet-18}, the forward derivative computation required for the linearization of the network introduces a more noticeable computational cost, explaining the longer training times in the table.

\subsection{Hyperparameter Sensitivity Analysis}
The proposed counterfactual regularization technique relies on two key hyperparameters: $\alpha$ and $\beta$. The former is intrinsic to the loss formulation defined in (\ref{eq:cf-train}), while the latter is closely tied to the choice of the score-based counterfactual explanation method used.

Figure~\ref{fig:test_alpha_beta} illustrates how the test accuracy of an MLP trained on the \textit{Water Potability} dataset changes for different combinations of $\alpha$ and $\beta$.

\begin{figure}[ht]
    \centering
    \includegraphics[width=0.85\linewidth]{img/test_acc_alpha_beta.png}
    \caption{The test accuracy of an MLP trained on the \textit{Water Potability} dataset, evaluated while varying the weight of our counterfactual regularizer ($\alpha$) for different values of $\beta$.}
    \label{fig:test_alpha_beta}
\end{figure}

We observe that, for a fixed $\beta$, increasing the weight of our counterfactual regularizer ($\alpha$) can slightly improve test accuracy until a sudden drop is noticed for $\alpha > 0.1$.
This behavior was expected, as the impact of our penalty, like any regularization term, can be disruptive if not properly controlled.

Moreover, this finding further demonstrates that our regularization method, CF-Reg, is inherently data-driven. Therefore, it requires specific fine-tuning based on the combination of the model and dataset at hand.

\section{Results}

\begin{table*}[t]
\centering
\fontsize{11pt}{11pt}\selectfont
\begin{tabular}{lllllllllllll}
\toprule
\multicolumn{1}{c}{\textbf{task}} & \multicolumn{2}{c}{\textbf{Mir}} & \multicolumn{2}{c}{\textbf{Lai}} & \multicolumn{2}{c}{\textbf{Ziegen.}} & \multicolumn{2}{c}{\textbf{Cao}} & \multicolumn{2}{c}{\textbf{Alva-Man.}} & \multicolumn{1}{c}{\textbf{avg.}} & \textbf{\begin{tabular}[c]{@{}l@{}}avg.\\ rank\end{tabular}} \\
\multicolumn{1}{c}{\textbf{metrics}} & \multicolumn{1}{c}{\textbf{cor.}} & \multicolumn{1}{c}{\textbf{p-v.}} & \multicolumn{1}{c}{\textbf{cor.}} & \multicolumn{1}{c}{\textbf{p-v.}} & \multicolumn{1}{c}{\textbf{cor.}} & \multicolumn{1}{c}{\textbf{p-v.}} & \multicolumn{1}{c}{\textbf{cor.}} & \multicolumn{1}{c}{\textbf{p-v.}} & \multicolumn{1}{c}{\textbf{cor.}} & \multicolumn{1}{c}{\textbf{p-v.}} &  &  \\ \midrule
\textbf{S-Bleu} & 0.50 & 0.0 & 0.47 & 0.0 & 0.59 & 0.0 & 0.58 & 0.0 & 0.68 & 0.0 & 0.57 & 5.8 \\
\textbf{R-Bleu} & -- & -- & 0.27 & 0.0 & 0.30 & 0.0 & -- & -- & -- & -- & - &  \\
\textbf{S-Meteor} & 0.49 & 0.0 & 0.48 & 0.0 & 0.61 & 0.0 & 0.57 & 0.0 & 0.64 & 0.0 & 0.56 & 6.1 \\
\textbf{R-Meteor} & -- & -- & 0.34 & 0.0 & 0.26 & 0.0 & -- & -- & -- & -- & - &  \\
\textbf{S-Bertscore} & \textbf{0.53} & 0.0 & {\ul 0.80} & 0.0 & \textbf{0.70} & 0.0 & {\ul 0.66} & 0.0 & {\ul0.78} & 0.0 & \textbf{0.69} & \textbf{1.7} \\
\textbf{R-Bertscore} & -- & -- & 0.51 & 0.0 & 0.38 & 0.0 & -- & -- & -- & -- & - &  \\
\textbf{S-Bleurt} & {\ul 0.52} & 0.0 & {\ul 0.80} & 0.0 & 0.60 & 0.0 & \textbf{0.70} & 0.0 & \textbf{0.80} & 0.0 & {\ul 0.68} & {\ul 2.3} \\
\textbf{R-Bleurt} & -- & -- & 0.59 & 0.0 & -0.05 & 0.13 & -- & -- & -- & -- & - &  \\
\textbf{S-Cosine} & 0.51 & 0.0 & 0.69 & 0.0 & {\ul 0.62} & 0.0 & 0.61 & 0.0 & 0.65 & 0.0 & 0.62 & 4.4 \\
\textbf{R-Cosine} & -- & -- & 0.40 & 0.0 & 0.29 & 0.0 & -- & -- & -- & -- & - & \\ \midrule
\textbf{QuestEval} & 0.23 & 0.0 & 0.25 & 0.0 & 0.49 & 0.0 & 0.47 & 0.0 & 0.62 & 0.0 & 0.41 & 9.0 \\
\textbf{LLaMa3} & 0.36 & 0.0 & \textbf{0.84} & 0.0 & {\ul{0.62}} & 0.0 & 0.61 & 0.0 &  0.76 & 0.0 & 0.64 & 3.6 \\
\textbf{our (3b)} & 0.49 & 0.0 & 0.73 & 0.0 & 0.54 & 0.0 & 0.53 & 0.0 & 0.7 & 0.0 & 0.60 & 5.8 \\
\textbf{our (8b)} & 0.48 & 0.0 & 0.73 & 0.0 & 0.52 & 0.0 & 0.53 & 0.0 & 0.7 & 0.0 & 0.59 & 6.3 \\  \bottomrule
\end{tabular}
\caption{Pearson correlation on human evaluation on system output. `R-': reference-based. `S-': source-based.}
\label{tab:sys}
\end{table*}



\begin{table}%[]
\centering
\fontsize{11pt}{11pt}\selectfont
\begin{tabular}{llllll}
\toprule
\multicolumn{1}{c}{\textbf{task}} & \multicolumn{1}{c}{\textbf{Lai}} & \multicolumn{1}{c}{\textbf{Zei.}} & \multicolumn{1}{c}{\textbf{Scia.}} & \textbf{} & \textbf{} \\ 
\multicolumn{1}{c}{\textbf{metrics}} & \multicolumn{1}{c}{\textbf{cor.}} & \multicolumn{1}{c}{\textbf{cor.}} & \multicolumn{1}{c}{\textbf{cor.}} & \textbf{avg.} & \textbf{\begin{tabular}[c]{@{}l@{}}avg.\\ rank\end{tabular}} \\ \midrule
\textbf{S-Bleu} & 0.40 & 0.40 & 0.19* & 0.33 & 7.67 \\
\textbf{S-Meteor} & 0.41 & 0.42 & 0.16* & 0.33 & 7.33 \\
\textbf{S-BertS.} & {\ul0.58} & 0.47 & 0.31 & 0.45 & 3.67 \\
\textbf{S-Bleurt} & 0.45 & {\ul 0.54} & {\ul 0.37} & 0.45 & {\ul 3.33} \\
\textbf{S-Cosine} & 0.56 & 0.52 & 0.3 & {\ul 0.46} & {\ul 3.33} \\ \midrule
\textbf{QuestE.} & 0.27 & 0.35 & 0.06* & 0.23 & 9.00 \\
\textbf{LlaMA3} & \textbf{0.6} & \textbf{0.67} & \textbf{0.51} & \textbf{0.59} & \textbf{1.0} \\
\textbf{Our (3b)} & 0.51 & 0.49 & 0.23* & 0.39 & 4.83 \\
\textbf{Our (8b)} & 0.52 & 0.49 & 0.22* & 0.43 & 4.83 \\ \bottomrule
\end{tabular}
\caption{Pearson correlation on human ratings on reference output. *not significant; we cannot reject the null hypothesis of zero correlation}
\label{tab:ref}
\end{table}


\begin{table*}%[]
\centering
\fontsize{11pt}{11pt}\selectfont
\begin{tabular}{lllllllll}
\toprule
\textbf{task} & \multicolumn{1}{c}{\textbf{ALL}} & \multicolumn{1}{c}{\textbf{sentiment}} & \multicolumn{1}{c}{\textbf{detoxify}} & \multicolumn{1}{c}{\textbf{catchy}} & \multicolumn{1}{c}{\textbf{polite}} & \multicolumn{1}{c}{\textbf{persuasive}} & \multicolumn{1}{c}{\textbf{formal}} & \textbf{\begin{tabular}[c]{@{}l@{}}avg. \\ rank\end{tabular}} \\
\textbf{metrics} & \multicolumn{1}{c}{\textbf{cor.}} & \multicolumn{1}{c}{\textbf{cor.}} & \multicolumn{1}{c}{\textbf{cor.}} & \multicolumn{1}{c}{\textbf{cor.}} & \multicolumn{1}{c}{\textbf{cor.}} & \multicolumn{1}{c}{\textbf{cor.}} & \multicolumn{1}{c}{\textbf{cor.}} &  \\ \midrule
\textbf{S-Bleu} & -0.17 & -0.82 & -0.45 & -0.12* & -0.1* & -0.05 & -0.21 & 8.42 \\
\textbf{R-Bleu} & - & -0.5 & -0.45 &  &  &  &  &  \\
\textbf{S-Meteor} & -0.07* & -0.55 & -0.4 & -0.01* & 0.1* & -0.16 & -0.04* & 7.67 \\
\textbf{R-Meteor} & - & -0.17* & -0.39 & - & - & - & - & - \\
\textbf{S-BertScore} & 0.11 & -0.38 & -0.07* & -0.17* & 0.28 & 0.12 & 0.25 & 6.0 \\
\textbf{R-BertScore} & - & -0.02* & -0.21* & - & - & - & - & - \\
\textbf{S-Bleurt} & 0.29 & 0.05* & 0.45 & 0.06* & 0.29 & 0.23 & 0.46 & 4.2 \\
\textbf{R-Bleurt} & - &  0.21 & 0.38 & - & - & - & - & - \\
\textbf{S-Cosine} & 0.01* & -0.5 & -0.13* & -0.19* & 0.05* & -0.05* & 0.15* & 7.42 \\
\textbf{R-Cosine} & - & -0.11* & -0.16* & - & - & - & - & - \\ \midrule
\textbf{QuestEval} & 0.21 & {\ul{0.29}} & 0.23 & 0.37 & 0.19* & 0.35 & 0.14* & 4.67 \\
\textbf{LlaMA3} & \textbf{0.82} & \textbf{0.80} & \textbf{0.72} & \textbf{0.84} & \textbf{0.84} & \textbf{0.90} & \textbf{0.88} & \textbf{1.00} \\
\textbf{Our (3b)} & 0.47 & -0.11* & 0.37 & 0.61 & 0.53 & 0.54 & 0.66 & 3.5 \\
\textbf{Our (8b)} & {\ul{0.57}} & 0.09* & {\ul 0.49} & {\ul 0.72} & {\ul 0.64} & {\ul 0.62} & {\ul 0.67} & {\ul 2.17} \\ \bottomrule
\end{tabular}
\caption{Pearson correlation on human ratings on our constructed test set. 'R-': reference-based. 'S-': source-based. *not significant; we cannot reject the null hypothesis of zero correlation}
\label{tab:con}
\end{table*}

\section{Results}
We benchmark the different metrics on the different datasets using correlation to human judgement. For content preservation, we show results split on data with system output, reference output and our constructed test set: we show that the data source for evaluation leads to different conclusions on the metrics. In addition, we examine whether the metrics can rank style transfer systems similar to humans. On style strength, we likewise show correlations between human judgment and zero-shot evaluation approaches. When applicable, we summarize results by reporting the average correlation. And the average ranking of the metric per dataset (by ranking which metric obtains the highest correlation to human judgement per dataset). 

\subsection{Content preservation}
\paragraph{How do data sources affect the conclusion on best metric?}
The conclusions about the metrics' performance change radically depending on whether we use system output data, reference output, or our constructed test set. Ideally, a good metric correlates highly with humans on any data source. Ideally, for meta-evaluation, a metric should correlate consistently across all data sources, but the following shows that the correlations indicate different things, and the conclusion on the best metric should be drawn carefully.

Looking at the metrics correlations with humans on the data source with system output (Table~\ref{tab:sys}), we see a relatively high correlation for many of the metrics on many tasks. The overall best metrics are S-BertScore and S-BLEURT (avg+avg rank). We see no notable difference in our method of using the 3B or 8B model as the backbone.

Examining the average correlations based on data with reference output (Table~\ref{tab:ref}), now the zero-shoot prompting with LlaMA3 70B is the best-performing approach ($0.59$ avg). Tied for second place are source-based cosine embedding ($0.46$ avg), BLEURT ($0.45$ avg) and BertScore ($0.45$ avg). Our method follows on a 5. place: here, the 8b version (($0.43$ avg)) shows a bit stronger results than 3b ($0.39$ avg). The fact that the conclusions change, whether looking at reference or system output, confirms the observations made by \citet{scialom-etal-2021-questeval} on simplicity transfer.   

Now consider the results on our test set (Table~\ref{tab:con}): Several metrics show low or no correlation; we even see a significantly negative correlation for some metrics on ALL (BLEU) and for specific subparts of our test set for BLEU, Meteor, BertScore, Cosine. On the other end, LlaMA3 70B is again performing best, showing strong results ($0.82$ in ALL). The runner-up is now our 8B method, with a gap to the 3B version ($0.57$ vs $0.47$ in ALL). Note our method still shows zero correlation for the sentiment task. After, ranks BLEURT ($0.29$), QuestEval ($0.21$), BertScore ($0.11$), Cosine ($0.01$).  

On our test set, we find that some metrics that correlate relatively well on the other datasets, now exhibit low correlation. Hence, with our test set, we can now support the logical reasoning with data evidence: Evaluation of content preservation for style transfer needs to take the style shift into account. This conclusion could not be drawn using the existing data sources: We hypothesise that for the data with system-based output, successful output happens to be very similar to the source sentence and vice versa, and reference-based output might not contain server mistakes as they are gold references. Thus, none of the existing data sources tests the limits of the metrics.  


\paragraph{How do reference-based metrics compare to source-based ones?} Reference-based metrics show a lower correlation than the source-based counterpart for all metrics on both datasets with ratings on references (Table~\ref{tab:sys}). As discussed previously, reference-based metrics for style transfer have the drawback that many different good solutions on a rewrite might exist and not only one similar to a reference.


\paragraph{How well can the metrics rank the performance of style transfer methods?}
We compare the metrics' ability to judge the best style transfer methods w.r.t. the human annotations: Several of the data sources contain samples from different style transfer systems. In order to use metrics to assess the quality of the style transfer system, metrics should correctly find the best-performing system. Hence, we evaluate whether the metrics for content preservation provide the same system ranking as human evaluators. We take the mean of the score for every output on each system and the mean of the human annotations; we compare the systems using the Kendall's Tau correlation. 

We find only the evaluation using the dataset Mir, Lai, and Ziegen to result in significant correlations, probably because of sparsity in a number of system tests (App.~\ref{app:dataset}). Our method (8b) is the only metric providing a perfect ranking of the style transfer system on the Lai data, and Llama3 70B the only one on the Ziegen data. Results in App.~\ref{app:results}. 


\subsection{Style strength results}
%Evaluating style strengths is a challenging task. 
Llama3 70B shows better overall results than our method. However, our method scores higher than Llama3 70B on 2 out of 6 datasets, but it also exhibits zero correlation on one task (Table~\ref{tab:styleresults}).%More work i s needed on evaluating style strengths. 
 
\begin{table}%[]
\fontsize{11pt}{11pt}\selectfont
\begin{tabular}{lccc}
\toprule
\multicolumn{1}{c}{\textbf{}} & \textbf{LlaMA3} & \textbf{Our (3b)} & \textbf{Our (8b)} \\ \midrule
\textbf{Mir} & 0.46 & 0.54 & \textbf{0.57} \\
\textbf{Lai} & \textbf{0.57} & 0.18 & 0.19 \\
\textbf{Ziegen.} & 0.25 & 0.27 & \textbf{0.32} \\
\textbf{Alva-M.} & \textbf{0.59} & 0.03* & 0.02* \\
\textbf{Scialom} & \textbf{0.62} & 0.45 & 0.44 \\
\textbf{\begin{tabular}[c]{@{}l@{}}Our Test\end{tabular}} & \textbf{0.63} & 0.46 & 0.48 \\ \bottomrule
\end{tabular}
\caption{Style strength: Pearson correlation to human ratings. *not significant; we cannot reject the null hypothesis of zero corelation}
\label{tab:styleresults}
\end{table}

\subsection{Ablation}
We conduct several runs of the methods using LLMs with variations in instructions/prompts (App.~\ref{app:method}). We observe that the lower the correlation on a task, the higher the variation between the different runs. For our method, we only observe low variance between the runs.
None of the variations leads to different conclusions of the meta-evaluation. Results in App.~\ref{app:results}.

\section{Discussion}
\section{Discussion of Assumptions}\label{sec:discussion}
In this paper, we have made several assumptions for the sake of clarity and simplicity. In this section, we discuss the rationale behind these assumptions, the extent to which these assumptions hold in practice, and the consequences for our protocol when these assumptions hold.

\subsection{Assumptions on the Demand}

There are two simplifying assumptions we make about the demand. First, we assume the demand at any time is relatively small compared to the channel capacities. Second, we take the demand to be constant over time. We elaborate upon both these points below.

\paragraph{Small demands} The assumption that demands are small relative to channel capacities is made precise in \eqref{eq:large_capacity_assumption}. This assumption simplifies two major aspects of our protocol. First, it largely removes congestion from consideration. In \eqref{eq:primal_problem}, there is no constraint ensuring that total flow in both directions stays below capacity--this is always met. Consequently, there is no Lagrange multiplier for congestion and no congestion pricing; only imbalance penalties apply. In contrast, protocols in \cite{sivaraman2020high, varma2021throughput, wang2024fence} include congestion fees due to explicit congestion constraints. Second, the bound \eqref{eq:large_capacity_assumption} ensures that as long as channels remain balanced, the network can always meet demand, no matter how the demand is routed. Since channels can rebalance when necessary, they never drop transactions. This allows prices and flows to adjust as per the equations in \eqref{eq:algorithm}, which makes it easier to prove the protocol's convergence guarantees. This also preserves the key property that a channel's price remains proportional to net money flow through it.

In practice, payment channel networks are used most often for micro-payments, for which on-chain transactions are prohibitively expensive; large transactions typically take place directly on the blockchain. For example, according to \cite{river2023lightning}, the average channel capacity is roughly $0.1$ BTC ($5,000$ BTC distributed over $50,000$ channels), while the average transaction amount is less than $0.0004$ BTC ($44.7k$ satoshis). Thus, the small demand assumption is not too unrealistic. Additionally, the occasional large transaction can be treated as a sequence of smaller transactions by breaking it into packets and executing each packet serially (as done by \cite{sivaraman2020high}).
Lastly, a good path discovery process that favors large capacity channels over small capacity ones can help ensure that the bound in \eqref{eq:large_capacity_assumption} holds.

\paragraph{Constant demands} 
In this work, we assume that any transacting pair of nodes have a steady transaction demand between them (see Section \ref{sec:transaction_requests}). Making this assumption is necessary to obtain the kind of guarantees that we have presented in this paper. Unless the demand is steady, it is unreasonable to expect that the flows converge to a steady value. Weaker assumptions on the demand lead to weaker guarantees. For example, with the more general setting of stochastic, but i.i.d. demand between any two nodes, \cite{varma2021throughput} shows that the channel queue lengths are bounded in expectation. If the demand can be arbitrary, then it is very hard to get any meaningful performance guarantees; \cite{wang2024fence} shows that even for a single bidirectional channel, the competitive ratio is infinite. Indeed, because a PCN is a decentralized system and decisions must be made based on local information alone, it is difficult for the network to find the optimal detailed balance flow at every time step with a time-varying demand.  With a steady demand, the network can discover the optimal flows in a reasonably short time, as our work shows.

We view the constant demand assumption as an approximation for a more general demand process that could be piece-wise constant, stochastic, or both (see simulations in Figure \ref{fig:five_nodes_variable_demand}).
We believe it should be possible to merge ideas from our work and \cite{varma2021throughput} to provide guarantees in a setting with random demands with arbitrary means. We leave this for future work. In addition, our work suggests that a reasonable method of handling stochastic demands is to queue the transaction requests \textit{at the source node} itself. This queuing action should be viewed in conjunction with flow-control. Indeed, a temporarily high unidirectional demand would raise prices for the sender, incentivizing the sender to stop sending the transactions. If the sender queues the transactions, they can send them later when prices drop. This form of queuing does not require any overhaul of the basic PCN infrastructure and is therefore simpler to implement than per-channel queues as suggested by \cite{sivaraman2020high} and \cite{varma2021throughput}.

\subsection{The Incentive of Channels}
The actions of the channels as prescribed by the DEBT control protocol can be summarized as follows. Channels adjust their prices in proportion to the net flow through them. They rebalance themselves whenever necessary and execute any transaction request that has been made of them. We discuss both these aspects below.

\paragraph{On Prices}
In this work, the exclusive role of channel prices is to ensure that the flows through each channel remains balanced. In practice, it would be important to include other components in a channel's price/fee as well: a congestion price  and an incentive price. The congestion price, as suggested by \cite{varma2021throughput}, would depend on the total flow of transactions through the channel, and would incentivize nodes to balance the load over different paths. The incentive price, which is commonly used in practice \cite{river2023lightning}, is necessary to provide channels with an incentive to serve as an intermediary for different channels. In practice, we expect both these components to be smaller than the imbalance price. Consequently, we expect the behavior of our protocol to be similar to our theoretical results even with these additional prices.

A key aspect of our protocol is that channel fees are allowed to be negative. Although the original Lightning network whitepaper \cite{poon2016bitcoin} suggests that negative channel prices may be a good solution to promote rebalancing, the idea of negative prices in not very popular in the literature. To our knowledge, the only prior work with this feature is \cite{varma2021throughput}. Indeed, in papers such as \cite{van2021merchant} and \cite{wang2024fence}, the price function is explicitly modified such that the channel price is never negative. The results of our paper show the benefits of negative prices. For one, in steady state, equal flows in both directions ensure that a channel doesn't loose any money (the other price components mentioned above ensure that the channel will only gain money). More importantly, negative prices are important to ensure that the protocol selectively stifles acyclic flows while allowing circulations to flow. Indeed, in the example of Section \ref{sec:flow_control_example}, the flows between nodes $A$ and $C$ are left on only because the large positive price over one channel is canceled by the corresponding negative price over the other channel, leading to a net zero price.

Lastly, observe that in the DEBT control protocol, the price charged by a channel does not depend on its capacity. This is a natural consequence of the price being the Lagrange multiplier for the net-zero flow constraint, which also does not depend on the channel capacity. In contrast, in many other works, the imbalance price is normalized by the channel capacity \cite{ren2018optimal, lin2020funds, wang2024fence}; this is shown to work well in practice. The rationale for such a price structure is explained well in \cite{wang2024fence}, where this fee is derived with the aim of always maintaining some balance (liquidity) at each end of every channel. This is a reasonable aim if a channel is to never rebalance itself; the experiments of the aforementioned papers are conducted in such a regime. In this work, however, we allow the channels to rebalance themselves a few times in order to settle on a detailed balance flow. This is because our focus is on the long-term steady state performance of the protocol. This difference in perspective also shows up in how the price depends on the channel imbalance. \cite{lin2020funds} and \cite{wang2024fence} advocate for strictly convex prices whereas this work and \cite{varma2021throughput} propose linear prices.

\paragraph{On Rebalancing} 
Recall that the DEBT control protocol ensures that the flows in the network converge to a detailed balance flow, which can be sustained perpetually without any rebalancing. However, during the transient phase (before convergence), channels may have to perform on-chain rebalancing a few times. Since rebalancing is an expensive operation, it is worthwhile discussing methods by which channels can reduce the extent of rebalancing. One option for the channels to reduce the extent of rebalancing is to increase their capacity; however, this comes at the cost of locking in more capital. Each channel can decide for itself the optimum amount of capital to lock in. Another option, which we discuss in Section \ref{sec:five_node}, is for channels to increase the rate $\gamma$ at which they adjust prices. 

Ultimately, whether or not it is beneficial for a channel to rebalance depends on the time-horizon under consideration. Our protocol is based on the assumption that the demand remains steady for a long period of time. If this is indeed the case, it would be worthwhile for a channel to rebalance itself as it can make up this cost through the incentive fees gained from the flow of transactions through it in steady state. If a channel chooses not to rebalance itself, however, there is a risk of being trapped in a deadlock, which is suboptimal for not only the nodes but also the channel.

\section{Conclusion}
This work presents DEBT control: a protocol for payment channel networks that uses source routing and flow control based on channel prices. The protocol is derived by posing a network utility maximization problem and analyzing its dual minimization. It is shown that under steady demands, the protocol guides the network to an optimal, sustainable point. Simulations show its robustness to demand variations. The work demonstrates that simple protocols with strong theoretical guarantees are possible for PCNs and we hope it inspires further theoretical research in this direction.

\section{Conclusion}

In this paper, we propose IAO prompting, a novel template-based approach that guides Large Language Models (LLMs) towards decomposing complex problems into manageable steps. By explicitly structuring input information, planned actions, and intermediate outputs, IAO facilitates sequential problem-solving. Our evaluations across diverse reasoning tasks demonstrate performance gains, increased clarity, improved structural coherence, and enhanced transparency within the reasoning chain. Moreover, IAO shows domain independence, requiring minimal adaptation to cater to different problem types. Additional human evaluation validates IAO utility in finding errors in free-text reasoning chains.

\section*{Acknowledgements}

This research was supported by the Leverhulme Trust grant for the project 'Repurposing of Resources: from Everyday Problem Solving through to Crisis Management' (RPG-2021-182). We also thank all reviewers for their insightful feedback.



\bibliography{aaai25}

\appendix

\section*{Limitations}
While IAO prompting demonstrates compelling advantages in terms of performance and transparency, it is crucial to acknowledge potential limitations:
\paragraph{Output Length}: The structured nature of IAO prompts might lead to lengthier responses compared to baseline models. This can have implications on computational costs and real-time applicability, particularly in resource-constrained settings. Future work could explore techniques for compressing the output or developing domain-specific adaptations to mitigate this limitation.
\paragraph{Interpretability Trade-off}: While the structured output enhances interpretability, it is important to consider that some users might prefer more concise summaries. The human evaluation study supports this. In fact, when the answer is correct, evaluators heavily preferred CoT reasoning chains over IAO's. This suggests a potential trade-off between detailed explanations and user preferences. Future work could explore methods to balance the level of detail and provide tailored interpretations based on individual user needs.

\section*{Ethical Considerations}

While IAO holds promise for improved reasoning, concerns arise regarding potential bias amplification and misuse as per any prompting method for LLMs. Breaking down complex tasks into smaller steps could inadvertently magnify existing biases in the LLM or training data, leading to biased final outputs. This necessitates careful bias detection and mitigation. Additionally, the structured nature could be exploited to "trick" the model, generating harmful or misleading outputs. Safeguards like fact-checking and verification become crucial, especially in sensitive domains. Responsible deployment and use are fundamental to ensure LLMs benefits are harnessed ethically and its risks are minimized.

\begin{table*}[ht]
\resizebox{\textwidth}{!}{%
\begin{tabular}{@{}llllll@{}}
\toprule
Prompt                                              & Date Understanding & Last Letter & AQuA & CommonsenseQA & \multicolumn{1}{c}{Average} \\ \midrule
\texttt{[Step, Input, Action, Output]}              & 82.4               & 46.0        & 64.6 & 82.7          & 68.9                        \\
\texttt{[Step, Subquestion, Action, Output]}         & 81.8               & 84.8        & 63.0 & 81.2          & 77.7                        \\
\texttt{[Step, Subquestion, Input, Output]}         & 85.9               & 76.0        & 61.0 & 82.5          & 76.3                        \\
\texttt{[Step, Subquestion, Input, Action]}         & 86.2               & 4.4         & 62.6 & 82.9          & 59.0                        \\ \midrule
\texttt{[Step, Subquestion, Input, Action, Output]} & 88.1               & 88.8        & 63.9 & 83.1          & 80.9                        \\ \bottomrule
\end{tabular}%
}
\caption{Performance in a zero-shot setting when one of the field of the prompt is removed.}
\label{tab:ablation_results_appendix}
\end{table*}

\begin{comment}
\section{Additional Related Work}
\paragraph{Prompting}
Chain-of-thought (CoT) prompting provides intermediate rationales in the reasoning chain for solving problems, leading to improved accuracy and interpretability in tasks requiring multi-step reasoning. The common instance of CoT is zero-shot CoT \cite{kojima2022large} and few-shot CoT \cite{wei2022chain}. The first one is the most relevant to our work. 
On the other hand, few-shot prompting requires manually constructed task-specific demonstrations, so it is more costly. 
\citet{zhang2022automatic} proposed an approach to automatically construct few-shot demonstrations based on zero-shot prompting to save the manual cost. Another categorization for prompting methods is based on whether they involve a single step or multiple steps. An example of multi-step prompting is the work by \citet{zhou2022least} which is discussed in the next paragraph. There are several CoT prompts with structured reasoning chains in the form of programs \cite{chen2023program}, graphs \cite{besta2024graph} or tables \cite{ziqi-lu-2023-tab}. In this work we focus on free-text reasoning chains.
\end{comment}

\section{Compute cost}
PaLM-2 (text-unicorn) has a cost of per API requests: \$0.0025/1K characters in input and \$0.0075/1K characters in output. GPT-4 has a cost \$0.01/1K tokens and \$0.03/1K tokens. 


\section{Datasets}
\label{sec:dataset_appendix}

All included datasets are in English. We note that AQuA-RAT, Date Understanding, Object Shuffling are under the Apache License, Version 2.0. GSM8K and StrategyQA are under the MIT License. We also note that the datasets may include names of individuals collected from
the internet, i.e., publicly available facts about a
person but not in an offensive way. The following list shows the sources of data we used for this study:
\begin{itemize}
\small
    \item AQuA-RAT: \url{https://github.com/google-deepmind/AQuA}
    \item GSM8K: \url{https://github.com/openai/grade-school-math}
    \item StrategyQA: \url{https://github.com/google/BIGbench/tree/main/bigbench/benchmark_tasks/strategyqa}
    \item Last Letter: \url{https://github.com/kojima-takeshi188/zero_shot_cot/tree/main/dataset/last_letters}
    \item Date Understanding: from BIG-Bench (BIG-bench collaboration, 2021): \url{https://github.com/google/BIG-bench/blob/main/}
    \item Object Tracking: from BIG-Bench (BIG-bench collaboration, 2021): \url{https://github.com/google/BIG-bench/blob/main/}
    \item CommonsenseQA: \url{https://www.tau-nlp.sites.tau.ac.il/commonsenseqa}
\end{itemize}%


\begin{table}[H]
\centering
\resizebox{0.9\columnwidth}{!}{%
\begin{tabular}{@{}lll@{}}
\toprule
Dataset            & Avg words & Filename                \\ \midrule
AQuA               & 51.9      & test.jsonl              \\
GSM8k              & 46.9      & test.jsonl              \\
D.U. & 35.0      & task.json               \\
O.T.    & 91.1      & three\_objects/task.json \\
L.L       & 15.0      & last\_letters.json       \\
C.QA      & 27.8      & dev\_rand\_split.jsonl    \\
S.QA         & 9.6       & task.json               \\ \bottomrule
\end{tabular}%
}
\caption{Datasets statistics and files used. D.U: Date Understanding, O.T: Object Tracking, L.L: Last Letters Concatenation, C.QA: CommonsenseQA, S.QA: StrategyQA}
\label{tab:more_data_stats}
\end{table}

\begin{table}[H]
\resizebox{0.9\columnwidth}{!}{%
\begin{tabular}{@{}llll@{}}
\toprule
Reasoning Type                                  & Dataset            & Size & Answer Type     \\ \midrule
\multirow{2}{*}{\textit{Arithmetic Reasoning}}  & AQUA               & 254  & Multiple Choice \\
                                                & GSM8k              & 1319 & Numeral         \\ \midrule
\multirow{2}{*}{\textit{Commonsense Reasoning}} & StrategyQA         & 2290 & Yes/No          \\
                                                & CommonsenseQA      & 1221 & Multiple Choice \\ \midrule
\multirow{2}{*}{\textit{Other Reasoning}}       & Date Understanding & 369  & Multiple Choice \\
                                                & Object Tracking    & 750  & Multiple Choice \\ \midrule
\textit{Symbolic Reasoning}                     & Last Letter        & 500  & String          \\ \bottomrule
\end{tabular}%
}
\caption{Tasks, data statistics and answer type.}
\label{tab:data_statistics}
\end{table}


%
\begin{table}[H]
\centering
\resizebox{0.95\columnwidth}{!}{%
\begin{tabular}{@{}ll@{}}
\toprule
Dataset            & Answer prompt                               \\ \midrule
AQuA               & Therefore, among A through E, the answer is \\
GSM8k              & Therefore, the answer (arabic numerals) is  \\
D.U. & Therefore, among A through F, the answer is \\
O.T.    & Therefore, among A through C, the answer is \\
L.L.       & Therefore, the answer is                    \\
C.QA      & Therefore, among A through E, the answer is \\
S.QA         & The answer (Yes or No) is                   \\ \bottomrule
\end{tabular}%
}
\caption{Answer extraction prompts used. D.U: Date Understanding, O.T: Object Tracking, L.L: Last Letters Concatenation, C.QA: CommonsenseQA, S.QA: StrategyQA}
\label{tab:answer-prompt}
\end{table}

\section{Ablation studies}
\Cref{tab:ablation_results_appendix} shows how removing each field impacts the model's accuracy on different tasks. This helps understand the importance and relevance of each field in guiding the model towards successful completion of tasks.
For example, removing the "Subquestion" field seems to have a significant negative impact on the "Date Understanding" accuracy, suggesting that subquestions might provide crucial context for understanding temporal information.
The Last Letter task seems most affected by any field removal, highlighting its dependence on full prompt context. 


\section{Experimental setting}
All results reported for PALM-2 are the results of average over 3 runs. The results reported for GPT-4 are over a single run due to higher costs. 
For each model, the temperature was set to 0, the maximum number of output tokens was set to 1024.
\begin{itemize}
    \item \texttt{gpt-4-1106-preview}: \url{https://platform.openai.com/docs/models/gpt-4-and-gpt-4-turbo}
    \item \texttt{PALM-2 (text-unicorn)}: \url{https://cloud.google.com/vertex-ai/docs/generative-ai/model-reference/text}
\end{itemize}

\section{Output word count}

\begin{table}[H]
\centering
\resizebox{0.4\columnwidth}{!}{%
\begin{tabular}{@{}lll@{}}
\toprule
                & \textbf{CoT} & \textbf{IAO} \\ \midrule
\textbf{PALM-2} & 144.3        & 155.2        \\
\textbf{GPT-4}  & 188.2        & 175.4        \\ \bottomrule
\end{tabular}%
}
\caption{Average output word count.}\label{tab:wc_avg}
\end{table}

We sampled 50 reasoning chains for each model across various benchmarks and report the average output word count.
GPT-4 tends to be more verbose in CoT and IAO structure helps in shortening the output reasoning but the opposite is seen in PALM-2.





\section{Additional Examples}

\begin{table*}[h!]
\resizebox{\textwidth}{!}{%
\begin{tabular}{@{}llllll@{}}
\toprule
\multicolumn{2}{l}{Question} &
  \multicolumn{4}{l}{Take the last letters of each words in "Pauline Kerry Jeannette Hope" and concatenate them. \hspace{5cm} \textcolor{blue}{(GT: "eyee")}} \\ \midrule
\multirow{4}{*}{\rotatebox[origin=c]{90}{PALM-2}} &
  CoT &
  \multicolumn{4}{l}{\begin{tabular}[c]{@{}l@{}}The last letters of each word in "Pauline Kerry Jeannette Hope" are "e", "y", "e", and "e".\\ Concatenating them, we get "eyeee".\\ Therefore, the answer is \textbf{eyeee}.\end{tabular}} \\ \cmidrule(l){2-6} 
 &
  \multirow{3}{*}{IAO} &
  Subquestion &
  Input &
  Action &
  Output \\ \cmidrule(l){3-6} 
 &
   &
  \begin{tabular}[c]{@{}l@{}}1. What are the last letters of each word in \\ "Pauline Kerry Jeannette Hope"?\end{tabular} &
  "Pauline Kerry Jeannette Hope" &
  Get the last letters of each word &
  "e y e e" \\
 &
   &
  2. Concatenate the last letters of each word. &
  "e y e e" &
  Concatenate the letters. &
  \textbf{"eyee"} \\ \midrule
\multirow{7}{*}{\rotatebox[origin=c]{90}{GPT-4}} &
  CoT &
  \multicolumn{4}{l}{\begin{tabular}[c]{@{}l@{}}To solve this, we need to take the last letter of each word in the phrase "Pauline Kerry Jeannette Hope" and then concatenate them, \\ which means to link them together in a sequence.\\ Here are the last letters of each word:\\ - Pauline: e\\ - Kerry: y\\ - Jeannette: e\\ - Hope: e\\ Now, we concatenate these letters: e + y + e + e\\ Therefore, the answer is \textbf{"eyee”}\end{tabular}} \\ \cmidrule(l){2-6} 
 &
  \multirow{6}{*}{IAO} &
  Subquestion &
  Input &
  Action &
  Output \\ \cmidrule(l){3-6} 
 &
   &
  1. What is the last letter of the first word "Pauline"? &
  The word "Pauline" &
  \begin{tabular}[c]{@{}l@{}}Identify the last letter of the \\ word "Pauline".\end{tabular} &
  The last letter is "e". \\
 &
   &
  2. What is the last letter of the second word "Kerry"? &
  The word "Kerry" &
  \begin{tabular}[c]{@{}l@{}}Identify the last letter of the \\ word "Kerry".\end{tabular} &
  The last letter is "y". \\
 &
   &
  3. What is the last letter of the third word "Jeannette"? &
  The word "Jeannette" &
  \begin{tabular}[c]{@{}l@{}}Identify the last letter of the \\ word "Jeannette".\end{tabular} &
  The last letter is "e". \\
 &
   &
  4. What is the last letter of the fourth word "Hope"? &
  The word "Hope" &
  \begin{tabular}[c]{@{}l@{}}Identify the last letter of the \\ word "Hope".\end{tabular} &
  The last letter is "e". \\
 &
   &
  3. Concatenate the last letters of each word. &
  The last letters "e", "y", "e", "e" &
  \begin{tabular}[c]{@{}l@{}}Concatenate the letters "e", "y", "e", "e" \\ in the order they were obtained.\end{tabular} &
  \begin{tabular}[c]{@{}l@{}}The concatenated result is \\ \textbf{"eyee"}.\end{tabular} \\ \bottomrule
\end{tabular}%
}
\caption{Output examples from Last Letters dataset. The final results are in \textbf{bold}.}
\label{tab:letter_example}
\end{table*}

%
\begin{table*}[h!]
\resizebox{\textwidth}{!}{%
\begin{tabular}{@{}llllll@{}}
\toprule
\multicolumn{2}{l}{Question} &
  \multicolumn{4}{l}{\begin{tabular}[c]{@{}l@{}}Today's meeting is rescheduled to 11 am tomorrow, 10/16/1924. What is the date 10 days ago in MM/DD/YYYY? \\ Answer Choices: (A) 10/05/1924 (B) 05/05/1924 (C) 10/12/1924 (D) 10/06/1924 (E) 11/02/1924 (F) 10/04/1924  \hspace{5cm}    \textcolor{blue}{(GT: (A))}\end{tabular}} \\ \midrule
\multirow{4}{*}{\rotatebox[origin=c]{90}{PALM-2}} &
  CoT &
  \multicolumn{4}{l}{\begin{tabular}[c]{@{}l@{}}The date of the meeting is 10/16/1924.\\ 10 days ago from 10/16/1924 is 10/06/1924.\\ Therefore, among A through F, the answer is \textbf{D}\end{tabular}} \\ \cmidrule(l){2-6} 
 &
  \multirow{3}{*}{IAO} &
  Subquestion &
  Input &
  Action &
  Output \\ \cmidrule(l){3-6} 
 &
   &
  1. What is the date 10 days ago? &
  10/16/1924 &
  subtract 10 days from the given date &
  10/06/1924 \\
 &
   &
  \begin{tabular}[c]{@{}l@{}}2. What is the answer choice that \\ matches the date 10 days ago?\end{tabular} &
  10/06/1924 &
  \begin{tabular}[c]{@{}l@{}}compare the date 10 days ago \\ with the answer choices\end{tabular} &
  \textbf{(D)} 10/06/1924 \\ \midrule
\multirow{5}{*}{\rotatebox[origin=c]{90}{GPT-4}} &
  CoT &
  \multicolumn{4}{l}{\begin{tabular}[c]{@{}l@{}}To find the date 10 days before 10/16/1924, we need to subtract 10 days from that date.\\ Starting with 10/16/1924:\\ 1. If we subtract 10 days from the 16th, we would be in the previous month, October, since October has 31 days.\\ 2. To find the exact date, we subtract 10 from 16, which gives us 6.\\ Therefore, the date 10 days before 10/16/1924 is 10/06/1924.\\ Among the answer choices provided:\\ (A) 10/05/1924\\ (B) 05/05/1924\\ (C) 10/12/1924\\ (D) 10/06/1924\\ (E) 11/02/1924\\ (F) 10/04/1924\\ The correct answer is \textbf{(D)} 10/06/1924.\end{tabular}} \\ \cmidrule(l){2-6} 
 &
  \multirow{4}{*}{IAO} &
  Subquestion &
  Input &
  Action &
  Output \\ \cmidrule(l){3-6} 
 &
   &
  \begin{tabular}[c]{@{}l@{}}1. What is the date 10 days \\ before 10/16/1924?\end{tabular} &
  The given date is 10/16/1924. &
  Subtract 10 days from the given date. &
  \begin{tabular}[c]{@{}l@{}}The date 10 days before\\ 10/16/1924.\end{tabular} \\
 &
   &
  \multicolumn{4}{l}{{\begin{tabular}[c]{@{}l@{}}To find the date 10 days before 10/16/1924, we need to count backwards day by day. Since October has 31 days, subtracting 10 days\\  from the 16th will not take us into the previous month. Therefore, we simply subtract 10 from 16.\\ 16 - 10 = 6\\ The month (October) and the year (1924) remain the same. Therefore, the date 10 days before 10/16/1924 is 10/06/1924.\\ Therefore, among A through F, the answer is \textbf{(D)} 10/06/1924\end{tabular}}} \\
 &
   &
  \multicolumn{4}{l}{} \\ \bottomrule
\end{tabular}%
}
\caption{Output examples from Date Understanding dataset. The final results are in \textbf{bold}.}
\label{tab:date_example}
\end{table*}
%
%
\begin{table*}[h!]
\resizebox{\textwidth}{!}{%
\begin{tabular}{@{}llllll@{}}
\toprule
\multicolumn{2}{l}{Question} &
  \multicolumn{4}{l}{\begin{tabular}[c]{@{}l@{}}Alice, Bob, and Claire are friends and avid readers who occasionally trade books. At the start of the semester, they each buy one new book: Alice gets \\ The Great Gatsby, Bob gets The Odyssey, and Claire gets Lolita. As the semester proceeds, they start trading around the new books. First, Alice and \\ Bob swap books. Then, Alice and Claire swap books. Finally, Bob and Claire swap books. At the end of the semester, Bob has \\ Which choice is true ? Answer Choices: (A) The Great Gatsby. (B) The Odyssey. (C) Lolita. \hspace{4cm} \textcolor{blue}{GT: (B) The Odissey}\end{tabular}} \\ \midrule
\multirow{4}{*}{\rotatebox[origin=c]{90}{PALM-2}} &
  CoT &
  \multicolumn{4}{l}{\begin{tabular}[c]{@{}l@{}}At the start of the semester, Alice has The Great Gatsby, Bob has The Odyssey, and Claire has Lolita.\\ After Alice and Bob swap books, Alice has The Odyssey and Bob has The Great Gatsby.\\ After Alice and Claire swap books, Alice has Lolita and Claire has The Odyssey.\\ After Bob and Claire swap books, Bob has Lolita and Claire has The Great Gatsby.\\ Therefore, among A through C, the answer is \textbf{C}.\end{tabular}} \\ \cmidrule(l){2-6} 
 &
  \multirow{3}{*}{IAO} &
  Subquestion &
  Input &
  Action &
  Output \\ \cmidrule(l){3-6} 
 &
   &
  1. What book does Bob have after the first swap? &
  \begin{tabular}[c]{@{}l@{}}Alice has The Great Gatsby, \\ Bob has The Odyssey.\end{tabular} &
  Bob and Alice swap books. &
  Bob has The Great Gatsby. \\
 &
   &
  2. What book does Bob have after the second swap? &
  \begin{tabular}[c]{@{}l@{}}Bob has The Great Gatsby, \\ Claire has Lolita.\end{tabular} &
  Bob and Claire swap books. &
  Bob has The Great Gatsby. \\
 &
   &
  3. What book does Bob have at the end of the semester? &
  Bob has The Great Gatsby. &
  No more swaps. &
  Bob has \textbf{The Great Gatsby}. \\ \midrule
\multirow{5}{*}{\rotatebox[origin=c]{90}{GPT-4}} &
  CoT &
  \multicolumn{4}{l}{\begin{tabular}[c]{@{}l@{}}Let's track the movement of the books step by step:\\ 1. Initially:\\    - Alice has The Great Gatsby.\\    - Bob has The Odyssey.\\    - Claire has Lolita.\\ 2. After Alice and Bob swap books:\\    - Alice has The Odyssey.\\    - Bob has The Great Gatsby.\\    - Claire still has Lolita.\\ 3. Then Alice and Claire swap books:\\    - Alice has Lolita.\\    - Bob still has The Great Gatsby.\\    - Claire has The Odyssey.\\ 4. Finally, Bob and Claire swap books:\\    - Alice still has Lolita.\\    - Bob has The Odyssey.\\    - Claire has The Great Gatsby.\\ At the end of the semester, Bob has The Odyssey.\\ Therefore, among A through C, the answer is \textbf{(B) The Odyssey}.\end{tabular}} \\ \cmidrule(l){2-6} 
 &
  IAO &
  Subquestion &
  Input &
  Action &
  Output \\ \cmidrule(l){2-6} 
 &
   &
  1. What book does Bob have after he and Alice swap books? &
  \begin{tabular}[c]{@{}l@{}}Alice has The Great Gatsby, \\ Bob has The Odyssey.\end{tabular} &
  \begin{tabular}[c]{@{}l@{}}Swap the books between \\ Alice and Bob.\end{tabular} &
  \begin{tabular}[c]{@{}l@{}}Alice now has The Odyssey, \\ Bob now has The Great Gatsby.\end{tabular} \\
 &
   &
  2. What book does Alice have after she and Claire swap books? &
  \begin{tabular}[c]{@{}l@{}}Alice has The Odyssey, \\ Claire has Lolita.\end{tabular} &
  \begin{tabular}[c]{@{}l@{}}Swap the books between \\ Alice and Claire.\end{tabular} &
  \begin{tabular}[c]{@{}l@{}}Alice now has Lolita, \\ Claire now has The Odyssey.\end{tabular} \\
 &
   &
  3. What book does Bob have after he and Claire swap books? &
  \begin{tabular}[c]{@{}l@{}}Bob has The Great Gatsby, \\ Claire has The Odyssey.\end{tabular} &
  \begin{tabular}[c]{@{}l@{}}Swap the books between \\ Bob and Claire.\end{tabular} &
  \begin{tabular}[c]{@{}l@{}}Bob now has \textbf{The Odyssey}, \\ Claire now has The Great Gatsby.\end{tabular} \\ \bottomrule
\end{tabular}%
}
\caption{Output examples from Object Tracking dataset. The final results are in \textbf{bold}.}
\label{tab:object_example}
\end{table*}
%
\begin{table*}[h!]
\resizebox{\textwidth}{!}{%
\begin{tabular}{@{}llllll@{}}
\toprule
\multicolumn{2}{l}{Question} &
  \multicolumn{4}{l}{\begin{tabular}[c]{@{}l@{}}Andrea has 8 more apples than Jamal and half as many bananas as Jamal. Jamal has 4 more bananas than apples. How many fruits are there \\ if Andrea has 52 apples? \hspace{5cm} \textcolor{blue}{(GT: 168)}\end{tabular}} \\ \midrule
\multirow{6}{*}{\rotatebox[origin=c]{90}{PALM-2}} &
  CoT &
  \multicolumn{4}{l}{\begin{tabular}[c]{@{}l@{}}Let's denote the number of apples Jamal has as A and the number of bananas Jamal has as B.\\ From the given information, we know that Andrea has 8 more apples than Jamal, so Andrea has A + 8 apples. We also know that Andrea has half as \\ many bananas as Jamal, so Andrea has B/2 bananas.\\ We are given that Andrea has 52 apples, so we can write the equation A + 8 = 52. Solving for A, we get A = 44.\\ We are also given that Jamal has 4 more bananas than apples, so we can write the equation B = A + 4. \\ Substituting the value of A, we get B = 44 + 4 = 48.\\ Now we can calculate the total number of fruits. Andrea has A + 8 apples and B/2 bananas, so she has 44 + 8 + 48/2 = 72 fruits. \\ Jamal has A apples and B bananas, so he has 44 + 48 = 92 fruits.\\ Therefore, the total number of fruits is 72 + 92 = 164.\\ \\ Therefore, the answer is \textbf{164}.\end{tabular}} \\ \cmidrule(l){2-6} 
 &
  \multirow{5}{*}{IAO} &
  Subquestion &
  Input &
  Action &
  Output \\ \cmidrule(l){3-6} 
 &
   &
  1. How many apples does Jamal have? &
  \begin{tabular}[c]{@{}l@{}}Andrea has 8 more apples than Jamal.\\  Andrea has 52 apples.\end{tabular} &
  Jamal has 52 - 8 = 44 apples. &
  Jamal has 44 apples. \\
 &
   &
  2. How many bananas does Jamal have? &
  \begin{tabular}[c]{@{}l@{}}Jamal has 4 more bananas than apples. \\ Jamal has 44 apples.\end{tabular} &
  Jamal has 44 + 4 = 48 bananas. &
  Jamal has 48 bananas. \\
 &
   &
  3. How many bananas does Andrea have? &
  \begin{tabular}[c]{@{}l@{}}Andrea has half as many bananas as Jamal. \\ Jamal has 48 bananas.\end{tabular} &
  Andrea has 48 / 2 = 24 bananas. &
  Andrea has 24 bananas. \\
 &
   &
  4. How many fruits are there in total? &
  \begin{tabular}[c]{@{}l@{}}Andrea has 52 apples and 24 bananas. \\ Jamal has 44 apples and 48 bananas.\end{tabular} &
  There are 52 + 24 + 44 + 48 = 168 fruits in total. &
  There are \textbf{168} fruits in total. \\ \midrule
\multirow{8}{*}{\rotatebox[origin=c]{90}{GPT-4}} &
  CoT &
  \multicolumn{4}{l}{\begin{tabular}[c]{@{}l@{}}First, let's determine the number of apples and bananas Jamal has.\\ According to the information given, Andrea has 8 more apples than Jamal. Since Andrea has 52 apples, we can calculate the number of apples \\ Jamal has by subtracting 8 from Andrea's total:\\ Jamal's apples = Andrea's apples - 8\\ Jamal's apples = 52 - 8\\ Jamal's apples = 44\\ Now we know Jamal has 44 apples.\\ Next, we are told that Jamal has 4 more bananas than apples. Since we know Jamal has 44 apples, we can find out how many bananas he has:\\ Jamal's bananas = Jamal's apples + 4\\ Jamal's bananas = 44 + 4\\ Jamal's bananas = 48\\ \\ So, Jamal has 48 bananas.\\ Now, let's find out how many bananas Andrea has. We are told that Andrea has half as many bananas as Jamal. So we can calculate this as follows:\\ \\ Andrea's bananas = Jamal's bananas / 2\\ Andrea's bananas = 48 / 2\\ Andrea's bananas = 24\\ \\ Now we can calculate the total number of fruits by adding the number of apples and bananas both Andrea and Jamal have:\\ Total fruits = Andrea's apples + Andrea's bananas + Jamal's apples + Jamal's bananas\\ Total fruits = 52 + 24 + 44 + 48\\ Total fruits = 76 + 92\\ Total fruits = 168\\ \\ Therefore, the answer is 168.\end{tabular}} \\ \cmidrule(l){2-6} 
 &
  \multirow{7}{*}{IAO} &
  Subquestion &
  Input &
  Action &
  Output \\ \cmidrule(l){3-6} 
 &
   &
  How many apples does Jamal have? &
  \begin{tabular}[c]{@{}l@{}}Andrea has 8 more apples than Jamal, \\ Andrea has 52 apples.\end{tabular} &
  \begin{tabular}[c]{@{}l@{}}Subtract 8 from the number of apples \\ Andrea has to find out how many apples Jamal has.\end{tabular} &
  Jamal has 52 - 8 = 44 apples. \\
 &
   &
  How many bananas does Jamal have? &
  \begin{tabular}[c]{@{}l@{}}Jamal has 4 more bananas than apples, \\ Jamal has 44 apples.\end{tabular} &
  \begin{tabular}[c]{@{}l@{}}Add 4 to the number of apples Jamal has to find \\ out how many bananas he has.\end{tabular} &
  Jamal has 44 + 4 = 48 bananas. \\
 &
   &
  How many bananas does Andrea have? &
  \begin{tabular}[c]{@{}l@{}}Andrea has half as many bananas as Jamal, \\ Jamal has 48 bananas.\end{tabular} &
  \begin{tabular}[c]{@{}l@{}}Divide the number of bananas Jamal has by 2 to \\ find out how many bananas Andrea has.\end{tabular} &
  Andrea has 48 / 2 = 24 bananas. \\
 &
   &
  How many fruits does Andrea have in total? &
  Andrea has 52 apples and 24 bananas. &
  Add the number of apples and bananas Andrea has. &
  Andrea has 52 + 24 = 76 fruits. \\
 &
   &
  How many fruits does Jamal have in total? &
  Jamal has 44 apples and 48 bananas. &
  Add the number of apples and bananas Jamal has. &
  Jamal has 44 + 48 = 92 fruits. \\
 &
   &
  How many fruits are there in total? &
  Andrea has 76 fruits, and Jamal has 92 fruits. &
  \begin{tabular}[c]{@{}l@{}}Add the number of fruits Andrea has to the\\  number of fruits Jamal has.\end{tabular} &
  There are 76 + 92 = \textbf{168} fruits in total. \\ \bottomrule
\end{tabular}%
}
\caption{Output examples from GSM8k dataset. The final results are in \textbf{bold}.}
\label{tab:gsm8k_example}
\end{table*}


\begin{table*}[h]
\resizebox{\textwidth}{!}{%
\begin{tabular}{@{}llllll@{}}
\toprule
\multicolumn{2}{l}{Question} &
  \multicolumn{4}{l}{\begin{tabular}[c]{@{}l@{}}The man tried to reply to the woman, but he had difficulty keeping track of conversations that he didn't do what to? \\ Answer Choices: (A) initiate (B) ignore (C) question (D) answer (E) ask \hspace{5cm} \textcolor{blue}{(GT: (A)}\end{tabular}} \\ \midrule
\multirow{5}{*}{\rotatebox[origin=c]{90}{PALM-2}} &
  CoT &
  \multicolumn{4}{l}{\begin{tabular}[c]{@{}l@{}}The man tried to reply to the woman, but he had difficulty keeping track of conversations that he didn't do what to?\\ The man tried to reply to the woman, but he had difficulty keeping track of conversations. This means that he was not able to follow the conversation properly. \\ He was not able to understand what the woman was saying. This could be because he was not paying attention or because he was not able to \\ understand the language.\\ The question asks what the man did not do. \\ The answer choices are: (A) initiate (B) ignore (C) question (D) answer (E) ask.\\ The man did not initiate the conversation. He was replying to the woman. He did not ignore the woman. He was trying to reply to her. He did not question\\  the woman. He was trying to understand what she was saying. He did not answer the woman. He was trying to reply to her. He did not ask the woman anything. \\ He was trying to understand what she was saying.\\ \\ Therefore, among A through E, the answer is \textbf{(D)}\end{tabular}} \\ \cmidrule(l){2-6} 
 &
  \multirow{4}{*}{IAO} &
  Subquestion &
  Input &
  Action &
  Output \\ \cmidrule(l){3-6} 
 &
   &
  \begin{tabular}[c]{@{}l@{}}1. What is the main idea \\ of the question?\end{tabular} &
  \begin{tabular}[c]{@{}l@{}}The man tried to reply to the \\ woman, but he had difficulty\\ keeping track of conversations\\  that he didn't do what to?\end{tabular} &
  \begin{tabular}[c]{@{}l@{}}Identify the main idea \\ of the question.\end{tabular} &
  \begin{tabular}[c]{@{}l@{}}The main idea of the question is that \\ the man had difficulty keeping track \\ of conversations that he didn't do something to.\end{tabular} \\
 &
   &
  \begin{tabular}[c]{@{}l@{}}2. What is the missing \\ word in the question?\end{tabular} &
  \begin{tabular}[c]{@{}l@{}}The man tried to reply to the\\ woman, but he had difficulty \\ keeping track of conversations \\ that he didn't do what to?\end{tabular} &
  \begin{tabular}[c]{@{}l@{}}Identify the missing \\ word in the question.\end{tabular} &
  The missing word in the question is "initiate". \\
 &
   &
  \begin{tabular}[c]{@{}l@{}}3. What is the answer to\\  the question?\end{tabular} &
  \begin{tabular}[c]{@{}l@{}}The man tried to reply to the\\ woman, but he had difficulty \\ keeping track of conversations\\ that he didn't do what to?\end{tabular} &
  \begin{tabular}[c]{@{}l@{}}Choose the answer that \\ best fits the question.\end{tabular} &
  The answer to the question is \textbf{(A)} initiate. \\ \midrule
\multirow{6}{*}{\rotatebox[origin=c]{90}{GPT-4}} &
  CoT &
  \multicolumn{4}{l}{\begin{tabular}[c]{@{}l@{}}To solve this reasoning problem, we need to determine which action would make it difficult for the man to keep track of conversations. If he \\ has difficulty keeping track of conversations, it is likely because he is not actively engaged in starting or controlling the flow of the conversation.\\ \\ Let's look at the options:\\ \\ (A) initiate - If he doesn't initiate the conversation, he might have trouble keeping track because he is not setting the topic or pace of the conversation.\\ (B) ignore - Ignoring a conversation would not cause difficulty in keeping track; it would mean he is not participating at all.\\ (C) question - Not questioning might make it harder to understand, but it doesn't necessarily mean he would lose track of the conversation.\\ (D) answer - If he doesn't answer, it could be a result of losing track, not a cause.\\ (E) ask - Similar to initiate, if he doesn't ask questions, he might not be engaged enough to follow the conversation.\\ \\ Between initiating and asking, initiating a conversation is a more fundamental action for controlling the flow of the conversation. \\ If he doesn't initiate conversations, he is more likely to be a passive participant and thus have difficulty keeping track.\\ \\ Therefore, among A through E, the answer is \textbf{(A)} initiate\end{tabular}} \\ \cmidrule(l){2-6} 
 &
  \multirow{5}{*}{IAO} &
  Subquestion &
  Input &
  Action &
  Output \\ \cmidrule(l){3-6} 
 &
   &
  \begin{tabular}[c]{@{}l@{}}1. What does the phrase \\ "difficulty keeping track of \\ conversations" imply about \\ the man's ability to engage \\ in conversations?\end{tabular} &
  \begin{tabular}[c]{@{}l@{}}The man had difficulty keeping \\ track of conversations.\end{tabular} &
  \begin{tabular}[c]{@{}l@{}}Analyze the phrase to determine \\ what aspect of conversation the\\  man struggles with.\end{tabular} &
  \begin{tabular}[c]{@{}l@{}}The man struggles with following along\\  or maintaining focus in conversations \\ that he is not actively leading or starting.\end{tabular} \\
 &
   &
  \begin{tabular}[c]{@{}l@{}}2. Which answer choice is \\ related to leading or starting a\\  conversation?\end{tabular} &
  \begin{tabular}[c]{@{}l@{}}Answer choices (A) initiate \\ (B) ignore (C) question \\ (D) answer (E) ask\end{tabular} &
  \begin{tabular}[c]{@{}l@{}}Identify the answer choice that \\ means to begin or lead a \\ conversation.\end{tabular} &
  (A) initiate \\
 &
   &
  \begin{tabular}[c]{@{}l@{}}3. Does the difficulty in keeping \\ track of conversations imply that \\ the man did not do what?\end{tabular} &
  \begin{tabular}[c]{@{}l@{}}The man had difficulty keeping\\ track of conversations; the man \\ tried to reply to the woman.\end{tabular} &
  \begin{tabular}[c]{@{}l@{}}Determine which action from \\ the answer choices would be most\\  difficult for someone who cannot \\ keep track of conversations.\end{tabular} &
  \begin{tabular}[c]{@{}l@{}}The man would have difficulty replying \\ (or keeping track) if he did not initiate \\ the conversation.\end{tabular} \\
 &
   &
  \multicolumn{4}{l}{Therefore, among A through E, the answer is \textbf{(A)} initiate.}
\end{tabular}%
}
\caption{Output examples from CommonsenseQA dataset. The final results are in \textbf{bold}.}
\label{tab:commonsenseqa_example}
\end{table*}

\clearpage\clearpage

\section{Errors from AQua}
\textbf{Note:} Several questions where IAO gave the wrong answers were due to the output exceeding the maximum number of tokens. 
\subsection{Example 1}
\textit{Question}: A grocer makes a 25\% profit on the selling price for each bag of flour it sells. If he sells each bag for \$100 and makes \$3,000 in profit, how many bags did he sell? 

\textit{Answer Choices}: (A) 12 (B) 16 (C) 24 (D) 30 (E) 40"

\textit{GT}:  "C"

\paragraph{IAO}:
\noindent
\textbf{Step 1:}
\begin{itemize}
    \item \textit{Subquestion:} What is the profit made on one bag of flour?
    \item \textit{Input:} 25\% profit on the selling price of \$100 per bag
    \item \textit{Action:} Calculate the profit per bag by multiplying the selling price by the profit percentage.
    \item \textit{Output:} \$25 profit per bag (since 25\% of \$100 is \$25)
\end{itemize}

\textbf{Step 2:}
\begin{itemize}
    \item \textit{Subquestion:} How many bags of flour were sold to make \$3,000 in profit?
    \item \textit{Input:} \$3,000 total profit, \$25 profit per bag
    \item \textit{Action:} Divide the total profit by the profit per bag to find the number of bags sold.
    \item \textit{Output:} 120 bags sold (since \$3,000 / \$25 per bag = 120 bags)
\end{itemize}

\textbf{Step 3:}
\begin{itemize}
    \item \textit{Subquestion:} Which answer choice corresponds to the number of bags sold?
    \item \textit{Input:} 120 bags sold, Answer Choices: (A) 12 (B) 16 (C) 24 (D) 30 (E) 40
    \item \textit{Action:} Identify the correct answer choice that matches the number of bags sold.
    \item \textit{Output:} None of the answer choices match the number of bags sold.
\end{itemize}

Therefore, among answer choices A through E, the answer is none of the provided options. There seems to be a mistake as none of the answer choices correspond to the correct number of bags sold based on the given information.

\paragraph{CoT}: To solve this problem, we need to determine how much profit the grocer makes on each bag of flour and then use that information to find out how many bags were sold to make a total profit of \$3,000.

\textbf{Step 1: Calculate the profit per bag.} \\
Since the grocer makes a 25\% profit on the selling price, we can calculate the profit per bag as follows:

\begin{align*}
\text{Profit per bag} &= 25\% \text{ of selling price} \\
\text{Profit per bag} &= 0.25 \times \$100 \\
\text{Profit per bag} &= \$25
\end{align*}

\textbf{Step 2: Calculate the number of bags sold.} \\
Now that we know the profit per bag is \$25, we can calculate the number of bags sold to make a total profit of \$3,000.

\begin{align*}
\text{Total profit} &= \text{Profit per bag} \times \text{Number of bags sold} \\
\$3,000 &= \$25 \times \text{Number of bags sold}
\end{align*}

To find the number of bags sold, we divide the total profit by the profit per bag:

\begin{align*}
\text{Number of bags sold} &= \frac{\$3,000}{\$25} \\
\text{Number of bags sold} &= 120
\end{align*}

However, 120 is not one of the answer choices provided, which means we need to re-evaluate our calculations.

Let's go back to the statement that the grocer makes a 25\% profit on the selling price. This means that the selling price includes both the cost price and the profit. If the selling price is \$100 and the profit is 25\% of this selling price, then the profit per bag is indeed \$25, as we calculated.

Now, let's re-calculate the number of bags sold using the correct profit per bag:

\begin{align*}
\text{Number of bags sold} &= \frac{\$3,000}{\$25} \\
\text{Number of bags sold} &= 120
\end{align*}

Since 120 is not among the answer choices, we need to check if we've made a mistake in interpreting the answer choices or the question itself. If the question and calculations are correct, then there might be an error in the answer choices provided. If this is the case, none of the answer choices (A) through (E) is correct.

Let's review the question one more time to ensure we haven't missed any details. The question states that the grocer makes a 25\% profit on the selling price for each bag of flour and sells each bag for \$100. Our calculation of the profit per bag at \$25 is correct, and so is our calculation of the number of bags sold to make a \$3,000 profit.

Therefore, it seems there is a discrepancy with the answer choices provided, as none of them match our calculation of 120 bags. If we must choose from the given options, we would have to conclude that none of them is correct. However, since this is not a standard outcome for such problems, it's possible that there is a misunderstanding or an error in the question or answer choices.

\subsection{Example 2}

\textit{Question}: Find 25/12*5 

\textit{Answer Choices}: (A) 2.5498 (B) 0.4167 (C) 3.3987 (D) 8.5497 (E) 5.6312

\textit{GT}: B

\paragraph{IAO}: \noindent
\textbf{Step 1:}
\begin{itemize}
    \item \textit{Subquestion:} What is the product of $(\frac{25}{12})$ and 5?
    \item \textit{Input:} $(\frac{25}{12})$, 5
    \item \textit{Action:} Multiply $(\frac{25}{12})$ by 5.
    \item \textit{Output:} 
    $\frac{25}{12} \times 5 = \frac{25 \times 5}{12} = \frac{125}{12}$
\end{itemize}
\textbf{Step 2:}
\begin{itemize}
    \item \textit{Subquestion:} What is the decimal equivalent of 125/12?
    \item \textit{Input:} 125/12
    \item \textit{Action:} Divide 125 by 12.
    \item \textit{Output:} 10.4167
\end{itemize}

\noindent
\textbf{Step 3:}
\begin{itemize}
    \item \textit{Subquestion:} Which answer choice matches the output from Step 2?
    \item \textit{Input:} 10.4167, Answer Choices: (A) 2.5498 (B) 0.4167 (C) 3.3987 (D) 8.5497 (E) 5.6312
    \item \textit{Action:} Compare the output with the given answer choices.
    \item \textit{Output:} None of the answer choices match the output from Step 2.
\end{itemize}

It seems there might have been a mistake in the calculation or in the provided answer choices. Let's re-evaluate the calculation in Step 2 to ensure accuracy.

\noindent
\textbf{Step 4:}
\begin{itemize}
    \item \textit{Subquestion:} What is the correct decimal equivalent of 125/12?
    \item \textit{Input:} 125/12
    \item \textit{Action:} Divide 125 by 12 using long division or a calculator.
    \item \textit{Output:} 10.4167 (confirmed)
\end{itemize}

Upon re-evaluation, the calculation is confirmed to be correct, but the output does not match any of the provided answer choices. It's possible that there is an error in the question or the answer choices. However, based on the calculation we've done, none of the answer choices (A) through (E) is correct.

Therefore, among A through E, the answer is none of the provided choices.

\paragraph{COT}: To solve the expression 
\[ \frac{25}{12} \times 5 \], follow these steps:

\textbf{Step 1: Simplify the expression by performing the multiplication.}

\[ \frac{25}{12} \times 5 = \frac{25 \times 5}{12} \]

\textbf{Step 2: Multiply the numerator.}

\[ \frac{25 \times 5}{12} = \frac{125}{12} \]

\textbf{Step 3: Divide 125 by 12 to get the decimal form.}

\[ 125 \div 12 = 10.4167 \]

\textbf{Step 4: Compare the result with the answer choices.}

The result, 10.4167, is not listed among the answer choices. It seems there might have been a mistake in the calculation. Let's try again.

\textbf{Step 1: Simplify the expression by performing the multiplication.}

\[ \frac{25}{12} \times 5 = \frac{25 \times 5}{12} \]

\textbf{Step 2: Multiply the numerator.}

\[ \frac{25 \times 5}{12} = \frac{125}{12} \]

\textbf{Step 3: Divide 125 by 12 to get the decimal form.}

\[ 125 \div 12 = 10 \text{ remainder } 5 \]

To express the remainder as a decimal, we divide 5 by 12:

\[ 5 \div 12 = 0.4167 \]

Adding this to the whole number part (10), we get:

\[ 10 + 0.4167 = 10.4167 \]

This result still does not match any of the answer choices. However, upon re-evaluating the answer choices, it seems that none of them are correct if the original expression was indeed 
\[ \frac{25}{12} \times 5 \]. It's possible that there might be a typo in the question or the answer choices.

Given the calculations we've performed, none of the answer choices (A) through (E) match the correct result of 10.4167. Therefore, it appears there is an error in the provided information.

\section{Errors from Date Understanding}
\subsection{Example 1}
\textit{Question}:2015 is coming in 36 hours. What is the date 24 hours later in MM/DD/YYYY? 

\textit{Answer Choices}: (A) 12/30/2014 (B) 01/01/2015 (C) 01/04/2015 (D) 12/30/2059 (E) 12/31/2014 (F) 01/09/2015 

\textit{GT}:  "A"

\paragraph{IAO}:
\noindent
\textbf{Step 1:}
\begin{itemize}
    \item \textit{Subquestion:} What is the current date if 2015 is coming in 36 hours?
    \item \textit{Input:} 2015 is coming in 36 hours.
    \item \textit{Action:} Subtract 36 hours from the start of 2015 to find the current date.
    \item \textit{Output:} The current date is December 30, 2014.
\end{itemize}

\noindent
\textbf{Step 2:}
\begin{itemize}
    \item \textit{Subquestion:} What is the date 24 hours after the current date?
    \item \textit{Input:} The current date is December 30, 2014.
    \item \textit{Action:} Add 24 hours to the current date to find the date 24 hours later.
    \item \textit{Output:} The date 24 hours later is December 31, 2014.
\end{itemize}

Therefore, among the answer choices A through F, the answer is (E) December 31, 2014.

\paragraph{COT}: If 2015 is coming in 36 hours, we can determine the current date by subtracting 36 hours from the start of 2015.

\begin{enumerate}
\item Since there are 24 hours in a day, 36 hours is equivalent to $1 \text{ day} + 12 \text{ hours}$.
\item The start of 2015 is at $00:00$ on January 1, 2015.
\item Subtracting $1$ day from January 1, 2015, takes us back to December 31, 2014.
\item Since we need to subtract $12$ more hours, and we are currently at $00:00$ on December 31, 2014, we would still be on December 31, 2014, because we are not going back an entire day, just $12$ hours into the previous day.
\end{enumerate}

So, the current date and time, 36 hours before 2015, is $12:00 \text{ (noon)}$ on December 31, 2014.

Now, we need to determine the date 24 hours later:

\begin{enumerate}
\item Adding $24$ hours to $12:00$ on December 31, 2014, will bring us to $12:00$ on January 1, 2015.
\end{enumerate}

Therefore, among A through F, the answer is (B) $01/01/2015$.

\subsection{Example 2}
\textit{Question}:Tomorrow is 11/12/2019. What is the date 10 days ago in MM/DD/YYYY? 

\textit{Answer Choices}: (A) 10/23/2019 (B) 11/02/2019 (C) 10/27/2019 (D) 11/01/2048 (E) 11/01/2019  

\textit{GT}:  "E"

\paragraph{IAO}: \noindent
\textbf{Step 1:}
\begin{itemize}
    \item \textit{Subquestion:} \underline{What is the date 10 days before 11/12/2019?}
    \item \textit{Input:} The given date is 11/12/2019.
    \item \textit{Action:} Subtract 10 days from the given date.
    \item \textit{Output:} The date 10 days before 11/12/2019 is 11/02/2019.
\end{itemize}

\noindent
Therefore, among A through E, the answer is (B) 11/02/2019.


\paragraph{COT}: To find the date 10 days before 11/12/2019, we need to subtract 10 days from that date.

\textbf{Step 1: Start with the given date:} 11/12/2019 (which is in MM/DD/YYYY format, meaning November 12, 2019).

\textbf{Step 2: Subtract 10 days from the 12th of November.} Since $12 - 10 = 2$, if November had more than 2 days, we would end up in November itself. However, since we're subtracting more days than there are in the month up to the 12th, we need to go back to the previous month, which is October.

\textbf{Step 3: October has 31 days.} To find out what date it would be 10 days before November 12, we need to count back from October 31.

\textbf{Step 4: Since we need to go back 10 days from November 12, and there are only 1 day of November included in this count (November 12 itself), we subtract the remaining 9 days from the total number of days in October.}

\textbf{Step 5:} $31 \text{ (total days in October)} - 9 \text{ (days to subtract)} = 22$.

\textbf{Step 6: Therefore, 10 days before November 12, 2019, would be October 22, 2019.}

\textbf{Step 7: Convert this date into MM/DD/YYYY format, which gives us 10/22/2019.}

However, none of the answer choices match this date. It seems there might be a mistake in the answer choices provided or in the calculation. Let's double-check the calculation:

\textbf{Given date:} 11/12/2019

\textbf{Subtract 10 days:}

* November has 12 days up to and including the 12th.
* We need to subtract 10 days.
* $12 - 10 = 2$, so we would be at November 2nd if we didn't need to go into the previous month.
* But since we're subtracting 10 days, we do need to go into the previous month.
* October has 31 days, so the day before November 1st is October 31st.
* We've already accounted for 1 day in November, so we subtract 1 from 10, leaving us with 9 days to subtract from October 31st.
* $31 - 9 = 22$, so we end up on October 22nd.

Upon re-evaluating the calculation, the correct date is indeed October 22, 2019, which is not listed in the answer choices. It seems there may be a typo in the question or the answer choices. However, based on the calculation, the correct answer is not present in the options provided.

\textbf{Therefore, among A through E, the answer is none of the above.}

\section{Errors from StrategyQA}
\subsection{Example 1}
\textit{Question}: Would John Muir not likely have a vitamin D deficiency? 

\textit{GT}:  "yes"

\paragraph{IAO}: To determine whether John Muir would likely have a vitamin D deficiency, we need to consider factors that influence vitamin D levels, such as exposure to sunlight, diet, skin pigmentation, age, geographic location, and lifestyle. Since we don't have specific information about John Muir's personal details, we'll have to rely on general knowledge about his life and activities.

John Muir (1838--1914) was a Scottish-American naturalist, author, environmental philosopher, and early advocate for the preservation of wilderness in the United States. He spent a significant amount of time outdoors, particularly in places like Yosemite and other parts of the American West.

\begin{enumerate}
    \item \textbf{Step 1:}
    \begin{itemize}
        \item \textit{Subquestion:} Did John Muir spend a significant amount of time outdoors?
        \item \textit{Input:} Historical knowledge about John Muir's life and activities.
        \item \textit{Action:} Assess the likelihood of outdoor activity based on known biographical information.
        \item \textit{Output:} Yes, John Muir spent a significant amount of time outdoors.
    \end{itemize}

    \item \textbf{Step 2:}
    \begin{itemize}
        \item \textit{Subquestion:} Does spending a significant amount of time outdoors increase exposure to sunlight?
        \item \textit{Input:} General knowledge about outdoor activities and sunlight exposure.
        \item \textit{Action:} Correlate outdoor activity with sunlight exposure.
        \item \textit{Output:} Yes, spending a significant amount of time outdoors generally increases exposure to sunlight.
    \end{itemize}

    \item \textbf{Step 3:}
    \begin{itemize}
        \item \textit{Subquestion:} Does increased exposure to sunlight affect vitamin D levels?
        \item \textit{Input:} Scientific knowledge about vitamin D synthesis.
        \item \textit{Action:} Relate sunlight exposure to vitamin D synthesis.
        \item \textit{Output:} Yes, increased exposure to sunlight typically leads to higher vitamin D levels due to the skin's production of vitamin D in response to UVB rays.
    \end{itemize}

    \item \textbf{Step 4:}
    \begin{itemize}
        \item \textit{Subquestion:} Based on John Muir's outdoor activities and the relationship between sunlight exposure and vitamin D synthesis, is it likely that John Muir had a vitamin D deficiency?
        \item \textit{Input:} Outputs from Step 1, Step 2, and Step 3.
        \item \textit{Action:} Infer the likelihood of vitamin D deficiency based on the correlation between outdoor activity, sunlight exposure, and vitamin D synthesis.
        \item \textit{Output:} It is unlikely that John Muir had a vitamin D deficiency given his significant time spent outdoors and the resulting increased exposure to sunlight.
    \end{itemize}
\end{enumerate}

Therefore, the answer (Yes or No) is No.

\paragraph{COT}: \textbf{Step 1: Identify the factors that contribute to vitamin D deficiency.}

Vitamin D deficiency can be caused by several factors, including:

\begin{itemize}
\item Lack of exposure to sunlight
\item Inadequate dietary intake of vitamin D
\item Certain medical conditions
\item The body's inability to absorb or convert vitamin D to its active form
\end{itemize}

\textbf{Step 2: Consider John Muir's lifestyle.}

John Muir was a naturalist, explorer, and advocate for the preservation of wilderness.  His lifestyle involved spending a significant amount of time outdoors, often in sunny environments like California's Sierra Nevada mountains.

\textbf{Step 3: Assess the likelihood of vitamin D deficiency based on lifestyle.}

Given that John Muir spent a lot of time outdoors, he would have been exposed to sunlight regularly. Sunlight exposure is a primary source of vitamin D, as the body synthesizes vitamin D when the skin is exposed to UV rays from the sun.

\textbf{Step 4: Draw a conclusion based on the information.}

Since John Muir was frequently outdoors and exposed to sunlight, it is \emph{less likely} that he would have suffered from vitamin D deficiency due to lack of sunlight exposure.

\textbf{Therefore, the answer (Yes or No) is No.}


%%%%%

\textit{Question}: Can French Defence initial move defend against four move checkmate? 

\textit{GT}:  "no"

\paragraph{IAO}: To determine if the French Defence can defend against the four-move checkmate (also known as the Scholar's Mate), we need to analyze the moves involved in both the French Defence and the Scholar's Mate.

\begin{itemize}
    \item \textbf{Step 1:}
    \begin{itemize}
        \item \textit{Subquestion:} What are the typical opening moves for the French Defence?
        \item \textit{Input:} The French Defence is a chess opening characterized by the moves.
        \item \textit{Action:} Describe the typical opening moves for the French Defence.
        \item \textit{Output:} The French Defence typically begins with 1.e4 e6.
    \end{itemize}

    \item \textbf{Step 2:}
    \begin{itemize}
        \item \textit{Subquestion:} What are the moves involved in the four-move checkmate?
        \item \textit{Input:} The four-move checkmate (Scholar's Mate) sequence.
        \item \textit{Action:} Describe the typical sequence of moves that lead to the four-move checkmate.
        \item \textit{Output:} The four-move checkmate typically involves the moves 1.e4 e5 2.Qh5 Nc6 3.Bc4 Nf6 4.Qxf7\#.
    \end{itemize}

    \item \textbf{Step 3:}
    \begin{itemize}
        \item \textit{Subquestion:} Can the opening move of the French Defence prevent the specific sequence of the four-move checkmate?
        \item \textit{Input:} The opening move of the French Defence (1.e4 e6) and the sequence of the four-move checkmate (1.e4 e5 2.Qh3 Nc6 3.Bc4 Nf6 4.Qxf7\#).
        \item \textit{Action:} Compare the opening moves of the French Defence with the moves required for the four-move checkmate to determine if the French Defence can prevent the checkmate sequence.
        \item \textit{Output:} The French Defence does not play 1...e5, which is a necessary move for the four-move checkmate. Instead, it plays 1...e6, which already deviates from the Scholar's Mate sequence.
    \end{itemize}

    \item \textbf{Step 4:}
    \begin{itemize}
        \item \textit{Subquestion:} Does the deviation from the four-move checkmate sequence inherently prevent the checkmate from occurring?
        \item \textit{Input:} The deviation in the opening move of the French Defence from the Scholar's Mate sequence.
        \item \textit{Action:} Analyze the impact of the deviation on the possibility of the four-move checkmate occurring.
        \item \textit{Output:} Since the French Defence does not follow the move 1...e5, it inherently prevents the specific sequence of the four-move checkmate from occurring.
    \end{itemize}

\end{itemize}

Therefore, the answer (Yes or No) is Yes.

\paragraph{COT}: The Four-Move Checkmate, also known as the Scholar's Mate, is a quick checkmate that White can deliver after the following moves:

\begin{enumerate}
\item e4 e5
\item Qh5 Nc6
\item Bc4 Nf6
\item Qxf7\#
\end{enumerate}

In this sequence, White aims to attack the f7 square, which is a weak point in Black's position because it is only defended by the king at the beginning of the game.

The French Defense, on the other hand, is a chess opening that begins with the moves:

\begin{enumerate}
\item e4 e6
\end{enumerate}

The idea behind the French Defense is to solidify Black's position and create a strong pawn chain that can later be used to counterattack White's center. The French Defense does not immediately contest the e5 square, which is a key component of setting up the Scholar's Mate.

Let's see if the French Defense can prevent the Four-Move Checkmate:

\begin{enumerate}
\item e4 e6
\item d4 d5
\end{enumerate}

With the second move d5, Black challenges White's center and opens lines for the bishop and queen, which can help in defending against early threats.

If White tries to set up the Scholar's Mate anyway:

\begin{enumerate}
\item 3. Qh5
\end{enumerate}

Black can now defend easily because the move ...e6 has already prevented the bishop from coming to c4 to attack the f7 square. Black has several reasonable moves here, such as ...Nf6, which attacks the invading queen and prepares to castle, or even ...g6, which would force the queen to move again.

Since the French Defense does not allow White to establish the same attacking pattern as in the Scholar's Mate, it effectively prevents the quick checkmate.

\textbf{Therefore, the answer (Yes or No) is Yes.}

\end{document}
