%File: anonymous-submission-latex-2025.tex
\documentclass[letterpaper]{article} % DO NOT CHANGE THIS
\usepackage[draft]{aaai25}  % DO NOT CHANGE THIS
\usepackage{times}  % DO NOT CHANGE THIS
\usepackage{helvet}  % DO NOT CHANGE THIS
\usepackage{courier}  % DO NOT CHANGE THIS
\usepackage[hyphens]{url}  % DO NOT CHANGE THIS
\usepackage{graphicx} % DO NOT CHANGE THIS
\urlstyle{rm} % DO NOT CHANGE THIS
\def\UrlFont{\rm}  % DO NOT CHANGE THIS
\usepackage{natbib}  % DO NOT CHANGE THIS AND DO NOT ADD ANY OPTIONS TO IT
\usepackage{caption} % DO NOT CHANGE THIS AND DO NOT ADD ANY OPTIONS TO IT
\frenchspacing  % DO NOT CHANGE THIS
\setlength{\pdfpagewidth}{8.5in} % DO NOT CHANGE THIS
\setlength{\pdfpageheight}{11in} % DO NOT CHANGE THIS
%
% These are recommended to typeset algorithms but not required. See the subsubsection on algorithms. Remove them if you don't have algorithms in your paper.
\usepackage{algorithm}
\usepackage{algorithmic}
\usepackage{xcolor}
\usepackage{amsmath}
\usepackage{booktabs}
\usepackage{multirow}
\usepackage{cleveref}
\usepackage{comment}
\usepackage{placeins}

%
% These are are recommended to typeset listings but not required. See the subsubsection on listing. Remove this block if you don't have listings in your paper.
\usepackage{newfloat}
\usepackage{listings}
\DeclareCaptionStyle{ruled}{labelfont=normalfont,labelsep=colon,strut=off} % DO NOT CHANGE THIS
\lstset{%
	basicstyle={\footnotesize\ttfamily},% footnotesize acceptable for monospace
	numbers=left,numberstyle=\footnotesize,xleftmargin=2em,% show line numbers, remove this entire line if you don't want the numbers.
	aboveskip=0pt,belowskip=0pt,%
	showstringspaces=false,tabsize=2,breaklines=true}
\floatstyle{ruled}
\newfloat{listing}{tb}{lst}{}
\floatname{listing}{Listing}
%
% Keep the \pdfinfo as shown here. There's no need
% for you to add the /Title and /Author tags.
\pdfinfo{
/TemplateVersion (2025.1)
}

% DISALLOWED PACKAGES
% \usepackage{authblk} -- This package is specifically forbidden
% \usepackage{balance} -- This package is specifically forbidden
% \usepackage{color (if used in text)
% \usepackage{CJK} -- This package is specifically forbidden
% \usepackage{float} -- This package is specifically forbidden
% \usepackage{flushend} -- This package is specifically forbidden
% \usepackage{fontenc} -- This package is specifically forbidden
% \usepackage{fullpage} -- This package is specifically forbidden
% \usepackage{geometry} -- This package is specifically forbidden
% \usepackage{grffile} -- This package is specifically forbidden
% \usepackage{hyperref} -- This package is specifically forbidden
% \usepackage{navigator} -- This package is specifically forbidden
% (or any other package that embeds links such as navigator or hyperref)
% \indentfirst} -- This package is specifically forbidden
% \layout} -- This package is specifically forbidden
% \multicol} -- This package is specifically forbidden
% \nameref} -- This package is specifically forbidden
% \usepackage{savetrees} -- This package is specifically forbidden
% \usepackage{setspace} -- This package is specifically forbidden
% \usepackage{stfloats} -- This package is specifically forbidden
% \usepackage{tabu} -- This package is specifically forbidden
% \usepackage{titlesec} -- This package is specifically forbidden
% \usepackage{tocbibind} -- This package is specifically forbidden
% \usepackage{ulem} -- This package is specifically forbidden
% \usepackage{wrapfig} -- This package is specifically forbidden
% DISALLOWED COMMANDS
% \nocopyright -- Your paper will not be published if you use this command
% \addtolength -- This command may not be used
% \balance -- This command may not be used
% \baselinestretch -- Your paper will not be published if you use this command
% \clearpage -- No page breaks of any kind may be used for the final version of your paper
% \columnsep -- This command may not be used
% \newpage -- No page breaks of any kind may be used for the final version of your paper
% \pagebreak -- No page breaks of any kind may be used for the final version of your paperr
% \pagestyle -- This command may not be used
% \tiny -- This is not an acceptable font size.
% \vspace{- -- No negative value may be used in proximity of a caption, figure, table, section, subsection, subsubsection, or reference
% \vskip{- -- No negative value may be used to alter spacing above or below a caption, figure, table, section, subsection, subsubsection, or reference

\setcounter{secnumdepth}{0} %May be changed to 1 or 2 if section numbers are desired.

% The file aaai25.sty is the style file for AAAI Press
% proceedings, working notes, and technical reports.
%

% Title

% Your title must be in mixed case, not sentence case.
% That means all verbs (including short verbs like be, is, using,and go),
% nouns, adverbs, adjectives should be capitalized, including both words in hyphenated terms, while
% articles, conjunctions, and prepositions are lower case unless they
% directly follow a colon or long dash
\title{IAO Prompting: Making Knowledge Flow Explicit in LLMs through Structured Reasoning Templates}
\author{
    A\"issatou Diallo$^1$\thanks{ Corresponding author: \texttt{a.diallo@ucl.ac.uk}\\}, Antonis Bikakis$^2$, Luke Dickens$^2$, Anthony Hunter$^1$, Rob Miller$^2$
}
\affiliations{
    %Afiliations

    \textsuperscript{\rm 1}Department of Computer Science \\
\textsuperscript{\rm 2}Department of Information Studies     \\
    % If you have multiple authors and multiple affiliations
    % use superscripts in text and roman font to identify them.
    % For example,

    % Sunil Issar\textsuperscript{\rm 2},
    % J. Scott Penberthy\textsuperscript{\rm 3},
    % George Ferguson\textsuperscript{\rm 4},
    % Hans Guesgen\textsuperscript{\rm 5}
    % Note that the comma should be placed after the superscript

University College London, United Kingdom 
%
% See more examples next
}

%Example, Single Author, ->> remove \iffalse,\fi and place them surrounding AAAI title to use it
\iffalse
\title{My Publication Title --- Single Author}
\author {
    Author Name
}
\affiliations{
    Affiliation\\
    Affiliation Line 2\\
    name@example.com
}
\fi

\iffalse
%Example, Multiple Authors, ->> remove \iffalse,\fi and place them surrounding AAAI title to use it
\title{My Publication Title --- Multiple Authors}
\author {
    % Authors
    First Author Name\textsuperscript{\rm 1},
    Second Author Name\textsuperscript{\rm 2},
    Third Author Name\textsuperscript{\rm 1}
}
\affiliations {
    % Affiliations
    \textsuperscript{\rm 1}Affiliation 1\\
    \textsuperscript{\rm 2}Affiliation 2\\
    firstAuthor@affiliation1.com, secondAuthor@affilation2.com, thirdAuthor@affiliation1.com
}
\fi


% REMOVE THIS: bibentry
% This is only needed to show inline citations in the guidelines document. You should not need it and can safely delete it.
\usepackage{bibentry}
% END REMOVE bibentry

\begin{document}

\maketitle

\begin{abstract}
While Large Language Models (LLMs) demonstrate impressive reasoning capabilities, understanding and validating their knowledge utilization remains challenging. Chain-of-thought (CoT) prompting partially addresses this by revealing intermediate reasoning steps, but the knowledge flow and application remain implicit. We introduce IAO (Input-Action-Output) prompting, a structured template-based method that explicitly models how LLMs access and apply their knowledge during complex reasoning tasks. IAO decomposes problems into sequential steps, each clearly identifying the input knowledge being used, the action being performed, and the resulting output. This structured decomposition enables us to trace knowledge flow, verify factual consistency, and identify potential knowledge gaps or misapplications. Through experiments across diverse reasoning tasks, we demonstrate that IAO not only improves zero-shot performance but also provides transparency in how LLMs leverage their stored knowledge. Human evaluation confirms that this structured approach enhances our ability to verify knowledge utilization and detect potential hallucinations or reasoning errors. Our findings provide insights into both knowledge representation within LLMs and methods for more reliable knowledge application.
\end{abstract}

% Uncomment the following to link to your code, datasets, an extended version or similar.
%
% \begin{links}
%     \link{Code}{https://aaai.org/example/code}
%     \link{Datasets}{https://aaai.org/example/datasets}
%     \link{Extended version}{https://aaai.org/example/extended-version}
% \end{links}

\section{Introduction}

\section{Introduction}
\label{sec:intro}

\begin{figure*}[tb]
    \centering
    \includegraphics[width=0.848\linewidth]{figs/circuitnn.pdf} 
    \caption{Illustration of differentiable CircuitNN. CircuitNN is designed based on differentiable NAND gates. After DAS is guided by PI and PO pairs of the truth table, CircuitNN can get the precise circuit architecture logic equivalent to the truth table.}
    \label{fig:circuitnn}
\end{figure*}

% 1. Describe the importance of logic synthesis
% 2. Existing Problems
% (a) Neural Architecture Search: Unstable, Predefined Setting, etc.
% (b) Circuit Generation: Probabilistic Model, Logic Equivalence

With the rapid advancement of technology, the scale of integrated circuits (ICs) has expanded exponentially. 
This expansion has introduced significant challenges in chip manufacturing, particularly concerning power and area metrics.
A primary objective in IC design is achieving the same circuit function with fewer transistors, thereby reducing power usage and area occupancy.

Logic synthesis~\cite{hachtel2005logicsynth}, a critical step in electronic design automation (EDA), transforms behavioral-level circuit designs into optimized gate-level circuits, ultimately yielding the final IC layout. 
The primary goal of logic synthesis is to identify the physical implementation with the fewest gates for a given circuit function. 
This task constitutes a challenging NP-hard combinatorial optimization problem. 
Current logic synthesis tools~\cite{brayton2010abc, wolf2013yosys} rely on human-designed heuristics, often leading to sub-optimal outcomes.

Differentiable architecture search (DAS) techniques~\cite{liu2018darts, chu2020darts} offer novel perspectives on addressing challenges in this problem.
Circuit functions can be represented through truth tables, which map binary inputs to their corresponding outputs. 
Truth tables provide a precise representation of input-output relationships, ensuring the design of functionally equivalent circuits.
Inspired by this, researchers~\cite{deepmind2024ai4sys, wang2024tnet} have begun exploring the application of DAS to synthesize circuits directly from truth tables.
Specifically, \citet{deepmind2024ai4sys} proposed CircuitNN, a framework that learns differentiable connection structures with logic gates, enabling the automatic generation of logic circuits from truth tables.
This approach significantly reduces the complexity of traditional circuit generation. 
Building on this, \citet{wang2024tnet} introduced T-Net, a triangle-shaped variant of CircuitNN, incorporating regularization techniques to enhance the efficiency of DAS.

Despite these advancements, several challenges remain. 
The computational complexity of DAS grows quadratically with the number of gates, posing scalability issues.
Although triangle-shaped architecture~\cite{wang2024tnet} partially mitigates this problem, redundancy persists. 
%Additionally, DAS is susceptible to converging to local optima, limiting the ability to search architectures that satisfy the given truth tables~\cite{liu2018darts}. 
%Furthermore, hyperparameters (network depth and layer width) require extensive searches, introducing complexity and prolonging the synthesis process. 
Additionally, DAS is susceptible to converging to local optima~\cite{liu2018darts} and hyperparameters (network depth and layer width) require extensive searches. 
The challenges arise from the vast search space in DAS. 
% Even with predefined settings for CircuitNN, finding a configuration that meets the truth table requires extensive trial and error during the DAS process. 
Intuitively, limiting the search space through predefined parameters (network depth, gates per layer, and connection probabilities) can significantly reduce the complexity.

Recent advances~\cite{openai2023gpt4, abramson2024alphafold3, esser2024sd3, li2024mar} in conditional generative models have demonstrated remarkable performance across language, vision, and graph generation tasks. 
Motivated by these developments, we propose a novel approach to circuit generation that generates preliminary circuit structures to guide DAS in generating refined circuits matching specified truth tables. 
Firstly, we introduce CircuitVQ, a tokenizer with a discrete codebook for circuit tokenization. 
Built upon our Circuit AutoEncoder framework~\cite{hou2022graphmae,li2023maskgae,wu2025mgvga}, CircuitVQ is trained through a circuit reconstruction task. 
Specifically, the CircuitVQ encoder encodes input circuits into discrete tokens using a learnable codebook, while the decoder reconstructs the circuit adjacency matrix based on these tokens.
Subsequently, the CircuitVQ encoder serves as a circuit tokenizer for CircuitAR pretraining, which employs a masked autoregressive modeling paradigm~\cite{chang2022maskgit, li2023mage}. 
In this process, the discrete codes function as supervision signals. 
After training, CircuitAR can generate discrete tokens progressively, which can be decoded into initial circuit structures by the decoder of the CircuitVQ. 
These prior insights can guide DAS in producing refined circuits that match the target truth tables precisely.

Our key contributions can be summarized as follows:
\begin{itemize}
\item We introduce CircuitVQ, a circuit tokenizer that facilitates graph autoregressive modeling for circuit generation, based on our Circuit AutoEncoder framework;
\item Develop CircuitAR, a model trained using masked autoregressive modeling, which generates initial circuit structures conditioned on given truth tables;
\item Propose a refinement framework that integrates differentiable architecture search to produce functionally equivalent circuits guided by target truth tables;
\item Comprehensive experiments demonstrating the scalability and capability emergence of our CircuitAR and the superior performance of the proposed circuit generation approach.
\end{itemize}

% Motivation
% (a) Diffusion (Vision, Graph), Autoregressive (Language, Vision)
% (b) Circuit Generation for Predefined Setting
% (c) Neural Architecture Search for Strict Logic Equivalence

% Contribution
% (a) Circuit Tokenizer (new transformer arch, training strategy)
% (b) CircuitAR (train and gen strategies, post-ar strategy)
% (c) Extensive Evaluation including BitD (Bit Distance) for Scalability


\section{Related Work}

\section{Related Work}
% \subsection{Vision Language Model}
% 시각장애인에서 상황을 설명할 DB가 없으니 만들었다. 그리고 이를 VLM에 튜닝했다.
\subsection{Technical approaches for assisting the visually-impaired}


\subsection{Datasets for visual instruction tuning}


\section{IAO Prompting}
\section{Methodology}
\label{sec:approach}

\begin{figure}[!t]
\centering
\includegraphics[width=0.5\textwidth]{Pipeline.png}
\caption{Workflow. For each synthesis or sketching task, we create an input query for the LLM such that the query contains the target property in natural language or Alloy (depending on the kind of task), run the query, get the LLM's output, and use the Alloy analyzer to validate it with respect to a reference (ground truth) formula.}
\label{fig:workflow}
\end{figure}

We consider the following three methods for employing large language models (LLMs) to create Alloy formulas to investigate the capabilities and limitations of LLMs in writing Alloy:

\begin{enumerate}
\item
{\bf English to Alloy}. We employ LLMs to write complete Alloy formulas in multiple different ways from given natural language descriptions (in English);
\item
{\bf Alloy to Alloy}. We employ LLMs to create multiple alternative but equivalent formulas in Alloy with respect to given formulas in Alloy; and
\item
{\bf Sketch to Alloy}. We employ LLMs to complete sketches~\cite{SolarLazemaPhD2008,WangETALABZ2018ASketch} of Alloy
formulas and populate the holes in the sketches by synthesizing Alloy
expressions and operators so that the completed formulas accurately
represent the desired properties (that are given in natural language).  \end{enumerate}

\begin{table}[!t]
\begin{tabular}{r@{\hskip 0.2cm}|l|p{4cm}|p{5cm}}
& \multicolumn{1}{c|}{\Intro{Property}} & \multicolumn{1}{c|}{\Intro{Natural language desc.}} & \multicolumn{1}{c}{\Intro{Reference Alloy formula}}\\
\hline
1 & DAG & Directed acyclic graph &
\begin{lstlisting}[style=AlloyTable]
all n: Node | n !in n.^link
\end{lstlisting} \\
\hline
2 & Cycle & Graph with directed cycle &
\begin{lstlisting}[style=AlloyTable]
some n: Node | n in n.^link
\end{lstlisting} \\
\hline
3 & Circular & The number of nodes is equal to the number of edges and the graph has a directed cycle that visits all nodes &
\begin{lstlisting}[style=AlloyTable]
#Node = #link
all n: Node | one n.link
all m, n: Node | m in n.^link
\end{lstlisting} \\
\hline
4 & Connex & For every pair of elements in S, either the first is related to the second or vice versa &
\begin{lstlisting}[style=AlloyTable]
all s, t: S |
  s->t in r or t->s in r
\end{lstlisting} \\
\hline
5 & Reflexive & Every element in S is related to itself &
\begin{lstlisting}[style=AlloyTable]
all s: S | s->s in r
\end{lstlisting} \\
\hline
6 & Symmetric & If element x in S is related to y, then y is also related to x &
\begin{lstlisting}[style=AlloyTable]
all s, t: S |
  s->t in r implies t->s in r
\end{lstlisting} \\
\hline
7 & Transitive & If element x in S is related to y and y is related to z, then x is also related to z &
\begin{lstlisting}[style=AlloyTable]
all s, t, u: S |
  s->t in r and t->u in r
    implies s->u in r
\end{lstlisting} \\
\hline
8 & Antisymmetric & If element x in S is related to y and y is related to x, then x and y are the same element &
\begin{lstlisting}[style=AlloyTable]
all s, t: S |
  s->t in r and t->s in r
    implies s = t
\end{lstlisting} \\
\hline
9 & Irreflexive & No element in S is related to itself &
\begin{lstlisting}[style=AlloyTable]
all s, t: S |
  s->t in r implies s != t
\end{lstlisting} \\
\hline
10 & Functional & Every element in S is related to at most one element (making r a partial function) &
\begin{lstlisting}[style=AlloyTable]
all s: S | lone s.r
\end{lstlisting} \\
\hline
11 & Function & Every element in S is related to exactly one element (making r a total function) &
\begin{lstlisting}[style=AlloyTable]
all s: S | one s.r
\end{lstlisting} \\
\hline
\end{tabular}
\vspace*{2ex}
\caption{Subject properties. The table lists for each property, its
  natural language description that defines the corresponding natural
  language to Alloy task, and its reference formulation in Alloy that
  defines the corresponding Alloy to Alloy
  task.}\label{tab:subjects-synthesis}
\vspace*{-4ex}
\end{table}


\begin{table}[!h]
\centering
\begin{tabular}{p{12cm}}
\hline
\begin{lstlisting}[style=AlloyTable]
pred DAG {
  // Directed acyclic graph
  all n: Node | \E,e\ \CO,co\ \E,e\
}
co := {| =|in|!=|!in |}
e := {| Node|n|((Node|n).(*|^)link) |}
\end{lstlisting} \\ \hline

\begin{lstlisting}[style=AlloyTable]
pred Cycle {
  // Graph with directed cycle
  some n: Node | \E,e\ \CO,co\ \E,e\
}
co := {| =|in|!=|!in |}
e := {| Node|n|((Node|n).(*|^)link) |}
\end{lstlisting} \\ \hline

\begin{lstlisting}[style=AlloyTable]
pred Circular {
  // The number of nodes is equal to the number of edges and the graph has a directed cycle that visits all nodes
#Node = #link
  all n: Node | one n.link
  all m, n: Node | \E,e\ \CO,co\ \E,e\
}
co := {| =|in|!=|!in |}
e := {| (Node|m|n).(*|^)link |}
\end{lstlisting} \\ \hline

\end{tabular}
\vspace*{2ex}
\caption{Sketches for Alloy specifications for Properties 1--3.}
\vspace*{-8ex}
\label{tab:sketches-1-3}
\end{table}

Figure~\ref{fig:workflow} graphically illustrates our approach.
For each synthesis or sketching task, we create an input query for the LLM such that the query contains the target property in natural language or Alloy (depending on the kind of task), run the query, get the LLM's output, and run the Alloy analyzer to validate it with respect to a ground truth formula, which we provide to the analyzer. There are three possible outcomes of running the Alloy analyzer: (1) the LLM's answer is correct (when the analyzer does not find a counterexample to the equivalence of the LLM's answer and ground truth); (2) the LLM's answer has a syntax error (when the analyzer fails to compile the LLM's answer); and (3) the LLM's answer is wrong (when the analyzer finds a counterexample to the equivalence of the LLM's answer and ground truth). Note for "Alloy to Alloy" synthesis tasks, the ground truth formula is the reference formula given as input to the LLM. Note also that for any "English to Alloy" synthesis task and for any "Sketch to Alloy" sketching task, the input to the LLM does not include the ground truth formula.

We employ the LLMs directly as available for public use.  Specifically, we do not fine-tune them.  Moreover, the queries we write are minimalistic in their description of the problem domain and do not provide instructions to the LLM on how to approach solving any given task.

\subsection{Subject tasks}

We use \NumSubjects~well-known properties of graphs and binary relations to create \NumTotalTasks~tasks for the LLMs to answer.  Three of the properties (DAG, Cycle, and Circular) are regarding edge-labeled graphs, and the remaining eight properties (Connex, Reflexive, Symmetric, Transitive, Antisymmetric, Irreflexive, Functional, and Function) are regarding binary relations.  In Alloy, in general, we can use one signature $S$ and one binary relation $r: S\times S$ to represent either an edge-labeled graph or a binary relation. However, in view of the specific domain of graphs, we name the signature `\CodeIn{Node}' and the binary relation `\CodeIn{link}' when creating the tasks relating graph properties. For the tasks relating properties of binary relations, we name the signature `\CodeIn{S}' and the relation `\CodeIn{r}'.

For each property, we create 2~kinds of synthesis tasks: (1) create 20~unique Alloy formulas that represent the given natural language description of the property; and (2) create 20~unique Alloy formulas that are equivalent to the given Alloy formula that captures the property, which is also included as a natural language comment in the prompt.  In addition, for each property, we create one sketching task: complete the given sketch of the property with respect to its natural language description that is included as a comment in the prompt.  Thus, for each property, we have a total of 3~tasks for the LLM to answer.

Table~\ref{tab:subjects-synthesis} lists each property, its natural language description, and a reference (ground truth) formula that characterizes it in Alloy. Moreover, Tables~\ref{tab:sketches-1-3}, \ref{tab:sketches-4-8} (Appendix), and \ref{tab:sketches-9-11} (Appendix) list each property, its sketch that defines the corresponding sketching problem. Together these four tables summarize the key elements of our tasks for the LLMs. To illustrate, consider the DAG property.  Figure~\ref{fig:three-tasks-for-DAG} describes the actual prompts we run against each LLM for this property.

\begin{figure}[!p]
\centering
\begin{tcolorbox}[mytextbox]
Give me 20 unique solutions to the problem of synthesizing the body of the following Alloy predicate (without markdown or comments) with respect to the property described in the comments:
\begin{lstlisting}
sig Node {
  link: set Node
}
pred DAG{
  // Directed acyclic graph
  // your code go here
}
\end{lstlisting}
\end{tcolorbox}
(a) "English to Alloy" task\\
\begin{tcolorbox}[mytextbox]
Give me 20 unique solutions to the problem of synthesizing the body of the following Alloy predicate (without markdown or comments) with respect to the property described in the comments:
\begin{lstlisting}
sig Node {
  link: set Node
}
pred DAG{
  // Directed acyclic graph
  all n: Node | n !in n.^link
}
\end{lstlisting}
\end{tcolorbox}
(b) "Alloy to Alloy" task\\
\begin{tcolorbox}[mytextbox]
Complete the following sketch of the Alloy predicate (without markdown or comments) by selecting values for the holes with respect to the given constraints such that the predicate is correct with respect to the property described in the comments:

\begin{lstlisting}
sig Node {
  link: set Node
}
pred DAG {
  // Directed acyclic graph
  all n: Node | \E,e\ \CO,co\ \E,e\
}

co := {| =|in|!=|!in |}
e := {| Node|n|((Node|n).(*|^)link) |}
\end{lstlisting}
\end{tcolorbox}
(c) "Sketch to Alloy" task
\caption{Three tasks for the LLMs with respect to the DAG property.}
\label{fig:three-tasks-for-DAG}
\end{figure}

In a predicate sketch, certain components of the predicate are placeholder holes~\cite{WangETALABZ2018ASketch}. These holes can be of different forms, e.g., comparison operator holes, expression holes, and quantifier holes.  For all our sketching tasks, we only use two kinds of holes: comparison operator holes and expression holes. A predicate sketch includes a definition of the sets of possible values that each hole can be completed with.  These sets are typically defined using regular expressions~\cite{SolarLazemaPhD2008}.  For our DAG sketching task, the comparison operator hole may be completed with one of four possible values from the set \{ `\CodeIn{=}', `\CodeIn{in}', `\CodeIn{!=}', `\CodeIn{!in}'\}, and each expression hole may be completed with one of six possible values from the set \{ `\CodeIn{Node}', `\CodeIn{n}', `\CodeIn{Node.*link}', `\CodeIn{Node.\^{}link}', `\CodeIn{n.*link}', `\CodeIn{n.\^{}link}' \}.




\section{Experimental Setup}
\section{Experiments: Planning outperforms Heuristics}
\label{sec:experiment}

We begin our empirical demonstrations by showcasing the effectiveness of our planning framework on both synthetic and real datasets. We focus on the simplest planning algorithm, 1-step lookaheads (Algorithm~\ref{alg:complete}), and show that even basic planning can hold great promise. 
We illustrate our framework using two uncertainty quantification modules---GPs and 
\ensembles/ \ensembleplus. 

Throughout this section, we focus on evaluating the mean squared error of 
a regression model $\model$,  and develop adaptive policies that minimize uncertainty on $g(f)$ defined in~\eqref{eqn:l2-g-f}.
When GPs provide a valid model of uncertainty, 
our experiments show that our planning framework significantly outperforms other baselines. 
We further demonstrate that our conceptual framework extends to deep learning-based uncertainty quantification methods such as  \ensembleplus while highlighting computational challenges that need to be resolved in order to scale our ideas. 
For simplicity, we assume a naive predictor, i.e., $\psi(\cdot) \equiv 0$. However, we emphasize that this problem is just as complex as if we were using a sophisticated model $\psi(.)$. The performance gap between the algorithms 
primarily depends
on the level  of uncertainty in our prior beliefs.

To evaluate the performance of our algorithm, we benchmark it against several baselines. 
%Active learning baselines use an acquisition function $\ac$ to select points that have the highest   function value: $X\opt_t \in \argmax_{X \in \xpoolj{t}} \ac({X})$ at every step $t$. These methods may also need an UQ module, which we simply use the same UQ module as in our algorithm, and it  outputs $V(X)$ that measures the the uncertainty of each point $X \in \xpoolj{t}$.
Our first set of baselines are from active learning~\citep{AggarwalKoGuHaPh14}:
\\ % \noindent\textbf{Active Learning Heuristics:} 
\textbf{(1)} 
\textsf{Uncertainty Sampling (Static):}  In this approach, we query the samples for which the model is least certain about. Specifically, we estimate the variance of the latent output $f(X)$ for each $X \in \xpool$ using the UQ module and select the top-$K$ points with the highest uncertainty. \\
\textbf{(2)} \textsf{Uncertainty Sampling (Sequential):} This is a greedy heuristic that sequentially selects the points with the highest uncertainty within a batch, while updating the posterior beliefs using pseudo labels from the current posterior state. Unlike \textsf{Uncertainty Sampling (Static)}, this method takes into account the information gained from each point within batch, and hence tries to diversify the selected points within a batch. 

 
We also compare our approach to the  \textbf{(3)} \textsf{Random Sampling}, which selects each batch uniformly at random from the pool. Additionally, we compare solving the planning problem using  \textsf{REINFORCE}-based policy gradients with   $\mathsf{Smoothed\text{-}Autodiff}$ policy gradients.\footnote{Our code repository is available at
  \url{https://github.com/namkoong-lab/adaptive-labeling}.}
%Detailed experimental setups are provided in Section \ref{sec:details-experiments}.

%We repeat all experiments with 10 random seeds.




\begin{figure}[t]
\centering
\begin{minipage}[b]{0.49\textwidth}
\centering
\includegraphics[width=\textwidth, height=5cm]{figures/original_scale/Var_of_l_2_loss.pdf}
\caption{(Synthetic data) Variance of mean squared loss evaluated through the posterior belief $\mu_t$ at each horizon $t$. This is the objective that policy gradient methods like \textsf{REINFORCE} and $\ouralgo$ optimizes. 1-step lookaheads are surprisingly effective even in long horizons.}
\label{fig:var-l2-sim}
\end{minipage}
\hfill
\begin{minipage}[b]{0.49\textwidth}
\centering \includegraphics[width=\textwidth, height=5cm]{figures/original_scale/Error_of_estimated_model_l_2_loss.pdf}
\caption{(Synthetic data) Error between MSE calculated based on collected data $\mc{D}^{0:T}$ vs. population oracle MSE over $\mc{D}_{\rm eval} \sim P_X$. Reducing uncertainty over posteriors directly leads to better OOD evaluations. 1-step lookaheads significantly outperform active learning heuristics in small horizons.}
\label{fig:mean-l2-sim}
\end{minipage}
%\caption{Simulated data for GPs}
%\label{fig:both_plots}
\end{figure}

\subsection{Planning with Gaussian processes}
\label{sec:experiment-plan-GP}
We now briefly describe the data generation process for the GP experiments,  deferring a more detailed discussion of the dataset generation to Section~\ref{sec:details-experiments}. 
We use both the synthetic data and the real data to test our methodology.
For the \emph{simulated data},  we construct a setting where the general population is distributed across \emph{51 non-overlapping clusters} while the initial labeled data $\dtrain$ just comes from one cluster. In contrast, both $\dpool \defeq (\xpool,\ypool),\deval \defeq (\xeval,\yeval)$ are generated   from all the clusters. 
We begin with a low-dimensional scenario, generating a one-dimensional regression setting using a GP. %Gaussian Process (GP).
Although the data-generating process is not known to the algorithms,  we assume that the GP hyperparameters are known to all the algorithms
to ensure fair comparisons. This can be viewed as a setting where our prior is well-specified, allowing us to isolate the effects
of different policy optimization approaches
 without any concerns about the misspecified priors. We select $10$ batches, each of size $K=5$ across $T = 10$ time horizons.

To examine the robustness of our method against the distributional assumptions made  in the simulated case, we then move to a real dataset where the correct prior is not known. We simulate selection bias from the eICU dataset~\citep{PollardJoRaCeMaBa18}, which contains real-world patient data with in-hospital mortality outcomes. 
We conduct a $k$-means clustering to generate 51 clusters and then select data from those clusters. We view this to be a credible replication of practice, as severe distribution shifts are common due to selection bias in clinical labels.  To convert the binary mortality labels into a regression setting, we train a  random forest classifier and fit a GP on predicted scores, which serves as the UQ module for all the algorithms. As before, the task is to select 10 batches, each consisting of 5 samples, across 10 time horizons.

 In Figures~\ref{fig:var-l2-sim} and~\ref{fig:mean-l2-sim}, we present results for the simulated data. 
Figure~\ref{fig:var-l2-sim} shows the variance of $\ell_2$ loss, and Figure~\ref{fig:mean-l2-sim} presents the error in the estimated $\ell_2$ loss using $\mu_t$ (relative to true $\ell_2$ loss, that is unknown to the algorithm). 
As we can see from these plots, our method one-step lookahead  gives substantial improvements  over active learning baselines and random sampling. In addition,
compared to the one-step lookahead planning approach using \textsf{REINFORCE}-based policy gradients, 
we observe that $\mathsf{Smoothed\text{-}Autodiff}$-based policy gradients provide significantly more robust performance over all horizons.

In Figures~\ref{fig:var-l2-real}~and~\ref{fig:mean-l2-real}, we observe similar findings on the eICU data. We see that planning policies (\textsf{REINFORCE} and $\mathsf{Smoothed\text{-}Autodiff}$) consistently outperform other heuristics by a large margin.  Active learning baselines perform poorly in these small-horizon batched problems and can sometimes be even worse than the random search baselines.  Overall, our results show the importance of careful planning in adaptive labeling for reliable model evaluation. 

We offer some intuition as to why one-step lookahead planning may outperform other heuristic algorithms. 
 First,  \textsf{Uncertainty sampling (Static)} while myopically selects the
 top-$K$ inputs with the highest uncertainty, it fails to consider 
the overlap in information content among the ``best” instances; see \citep{AggarwalKoGuHaPh14} for more details. 
In other words,  it might acquire points from the same region with high uncertainty while failing to induce diversity among the batch.
Although \textsf{Uncertainty Sampling (Sequential)} somewhat addresses the issue of information overlap, a significant drawback of 
this algorithm
is the disconnect between the objective we aim to optimize and the algorithm. For example, it might sample from a region with high uncertainty but very low density. 

\begin{figure}[t]
\centering
\begin{minipage}[b]{0.48\textwidth}
\centering
\includegraphics[width=\textwidth, height=5cm]{figures/original_scale/Var_of_l_2_loss_real.pdf}
\caption{(Real-world eICU data) Variance of mean squared loss evaluated through the posterior belief $\mu_t$ at each horizon $t$. Even 1-step lookaheads are extremely effective planners, and auto-differentiation-based pathwise policy gradients provide a reliable optimization algorithm based on low-variance gradient estimates.}
\label{fig:var-l2-real}
\end{minipage}
\hfill
\begin{minipage}[b]{0.48\textwidth}
\centering \includegraphics[width=\textwidth, height=5cm]{figures/original_scale/Error_of_estimated_model_l_2_loss_real.pdf}
\caption{(Real-world eICU data) Error between MSE calculated based on collected data $\mc{D}^{0:T}$ vs. population oracle MSE over $\mc{D}_{\rm eval} \sim P_X$. Reducing uncertainty over posteriors directly leads to better OOD evaluations. Our method significantly outperforms active learning-based heuristics, and random sampling.}
\label{fig:mean-l2-real}
\end{minipage}
%\caption{Real data for GPs}
\end{figure}
 
%\vspace{-1.5cm}
% \begin{wrapfigure}{r}{.32\columnwidth}
%   \vspace{-.5cm} 
%   \centering
% \includegraphics[scale=.29]{figures/Var of l2l_2 loss.pdf}
%   \vspace{-0.2cm}
%   \caption{Results of GP}
% \label{fig:var-l2-gp}
%   \vspace{-0.1cm}
% \end{wrapfigure}


% Attempts have been made  in the past to address these  drawbacks heuristically  (see \citep{AggarwalKoGuHaPh14}). We give a unified computational framework while approaching the problem in a more principled manner and solving it more optimally.




\subsection{Planning with  neural network-based uncertainty quantification methods ($\ensembleplus$)}


We now provide a proof-of-concept that shows the generalizability of our conceptual framework  to the deep learning-based UQ modules, specifically focusing on $\ensembleplus$ due to their previously observed superior performance~\citep{OsbandWenAsDwIbLuRo23}. Recall that implementing our framework with deep learning-based UQ modules  requires us to retrain the model across multiple possible random actions $\bm{a}(\theta)$ sampled from the current policy $\pi_\theta$.
This requires significant computational resources, in sharp contrast to the GPs where the posteriors are in closed form and can be readily updated and differentiated. 

Due to the computational constraints, we test $\ensembleplus$ on a toy setting to demonstrate the generalizability of our framework. We consider a setting where the general population consists of four clusters, while the initial labeled data only comes from one cluster. Again we generate data using GPs.  The task is to select a batch of 2 points in one horizon. We detail the $\ensembleplus$ architecture in Section \ref{sec:details-experiments}, and we assume prior uncertainty to be large (depends on the scaling of the prior generating functions). 
The results are summarized in the Table~\ref{tab:UQ_ensemble}.

% \begin{table}[H]
% \vspace{-10pt}
% \caption{Performance under \ensembleplus as UQ module}
%     \centering
%     \begin{tabular}{|m{3cm}|m{2.5cm}|m{2cm}|} 
%     \hline
%       Algorithm   & Variance of $\loss_2$ loss estimate & Error of $\loss_2$ loss estimate  \\ \hline Random Sampling 
%          & $1710.9 \pm 1352.1$ & $8.67\pm6.62$ 
%       \\ \hline \ouralgo & $1.30 \pm 0.68$ & $0.91\pm0.25$ \\ \hline
%     \end{tabular}
%     \label{tab:UQ_ensemble}
%     %\vspace{-10pt}
% \end{table}




\begin{table}[h]
\vspace{-10pt}
\caption{Performance under \ensembleplus as the UQ module}
\centering
\begin{tabular}{|l|l|l|}
\hline
Algorithm   & Variance of $\loss_2$ loss estimate & Error of $\loss_2$ loss estimate  \\
\hline
\textsf{Random sampling} & 7129.8 $\pm$ 1027.0 & 136.2 $\pm$ 8.28 \\ \hline
\textsf{Uncertainty sampling (Static)} & 10852 $\pm$ 0.0 & 162.156 $\pm$ 0.0 \\ \hline
\textsf{Uncertainty sampling (Sequential)} & 8585.5 $\pm$ 898.9 & 144 $\pm$ 6.93 \\ \hline
\textsf{REINFORCE} & 1697.1 $\pm$ 0.0 & 45.27 $\pm$ 0.0 \\ \hline
\ouralgo & 1697.1 $\pm$ 0.0 & 45.27 $\pm$ 0.0 \\ \hline
\end{tabular}
%\caption{Comparison of different algorithms based on variance   and   error in $\ell_2$ loss estimation with Ensemble $+$ as the UQ module. Our results demonstrate that {\ouralgo} and REINFORCE outperformthe other active learning based heuristics, confirming the benefits of our MDP formulation for the adaptive labeling problem, as also demonstrated in Section 4.\\
%\footnotesize{Experimental details: We use Gaussian Processes as our data generating process, GP parameters are the same as in Section D.3.  The task is to select a batch of 2 points along one horizon.The marginal distribution $p_X$ has 4 \textit{non-overlapping} clusters. Initial data comes from one cluster, while pool and evaluation points comes from all the clusters. We have $20$ initial labeled data points, $10$ pool points, and $252$ evaluation points.  Training procedures are similar to the one in Section D.3.} }
\label{tab:UQ_ensemble}
\end{table}



% We faced  issues in scaling up these experiments which will be our focus in the future. 





% \begin{itemize}
%     \item Posteriors should be consistent. Two dimensions: even with less training,  
%     \item the inference should be  fast enough
% \end{itemize}


% Potential research directions for uncertainty quantification

% In this section we consider a simple setting We consider a simpler setting and 


% For synthetic dataset generation, we use ...... For real datasets, we use ...... We compare our methodolgy to several baselines ()    This Section is structured as follows:
% \begin{itemize}
%     \item \textbf{GPs, square loss objective} (Section \ref{}): 
%     %the broad aim of the experiments  in this section is to isolate the performance of our methodology without any concerns for the inefficiencies induced due to a mis-specified prior or imperfect posterior inference. To accomplish this we generate synthetic datasets using GPs (detailed later). We use the well specified prior (GPs - with same hyperparameter setting) as our UQ module.   
%      As GPs provide differentaible posterior inference - any errors induced due to imperfect posterior updates are also isolated. We note that under this setting
%      \item In Section\ref{} we demonstrate why our methodology performs better than other baselines - by devising various synthetic experiments ()
%     \item  \textbf{UQ Benchmarking }(Section \ref{}): Before diving into the experiments using $\ensembleplus$ and ENNs,  we showcase our benchmarking experiments in Section \ref{}. We use real datasets We observe that ENNs perform better
%      \item \textbf{Ensemble $+$}, objective: recall, accuracy
%     \item \textbf{ENN}, objective: recall, accuracy
% \end{itemize}




% In Section {}, we test 
% \subsection{Experimental details}

% \begin{itemize}
%     \item UQ methodologies - GPs, ENNs
%     \item Objectives - Recall,  ATE
%     \item Datasets - ATE-synthetic datasets, Recall-synthetic, real datasets
%     \item Baselines - 
%     \begin{itemize}
%         \item Random sampling
%         \item Active learning - Uncertainty based sampling - In regression setting almost all of the 
%         \item Myopic greedy - Greedy Batch based sampling
%         \item Policy Gradient
%     \end{itemize}
    
% \end{itemize}

% \subsection{Experiments}
%     \begin{itemize}
%     \item GPs with square loss
%     \item Benchmarking ENN
%         \item ENNs with ATE
%         \item ENNs with Recall
%     \end{itemize}

% \subsection{Benefits over other algorithms - intuition and experiments}

%Active learning - Myopic greedy / Don't rely on the objective rather some entropy version.


%%% Local Variables:
%%% mode: latex
%%% TeX-master: "main"
%%% End:


\section{Results}
\begin{table}[ht!]
\centering
\caption{\textbf{Super Resolution Performance Results.} Our proposed WGAN EEG Spatial Upsampling method significantly outperforms a baseline of Bicubic Interpolation commonly used in EEG upsampling pipelines.}
\label{tab:results}
\resizebox{0.8\linewidth}{!}{%
\begin{tabular}{@{}cccccc@{}}
\toprule
\multirow{2}{*}{\textbf{Dataset}} & \multirow{2}{*}{\textbf{Scale}} & \multicolumn{2}{c}{\textbf{Bicubic}} & \multicolumn{2}{c}{\textbf{WGAN}} \\ \cmidrule(l){3-6} 
                      &   & \textbf{MSE} & \textbf{MAE} & \textbf{MSE}    & \textbf{MAE}   \\
\toprule
\multirow{2}{*}{Val}  & 2 & 3.71E7       & 3.89E3       & \textbf{2.01E3} & \textbf{24.38} \\
                      & 4 & 7.23E7       & 6.42E3       & \textbf{8.53E3} & \textbf{63.83} \\
\midrule
\multirow{2}{*}{Test} & 2 & 3.75E7       & 3.91E3       & \textbf{2.06E3} & \textbf{24.66} \\
                      & 4 & 7.30E7       & 6.45E3       & \textbf{8.68E3} & \textbf{64.39} \\
\bottomrule
\end{tabular}%
}
\end{table}

\section{Discussion}
This work identifies signal collapse as a critical bottleneck in one-shot neural network pruning. Performance loss in pruned networks is due to \textbf{signal collapse} in addition to the removal of critical parameters. We propose \textbf{REFLOW} (\textbf{Re}storing \textbf{F}low of \textbf{Low}-variance signals), a simple yet effective method that mitigates signal collapse without computationally expensive weight updates. By focusing on signal preservation, REFLOW highlights the importance of mitigating signal collapse in sparse networks and enables magnitude pruning to match or surpass state-of-the-art one-shot pruning methods such as CHITA, CBS, and WF.

REFLOW consistently achieves state-of-the-art accuracy across diverse architectures, restoring ResNeXt-101 from under 4.1\% to 78.9\% top-1 accuracy at 80\% sparsity on ImageNet. Its lightweight design makes it a practical solution for both research and deployment, delivering high-quality sparse models without the overhead of traditional approaches. These findings challenge the traditional emphasis on weight selection strategies and underscore the critical role of signal propagation for achieving high-quality sparse networks in the context of one-shot pruning.




\section{Conclusion}

In this paper, we propose IAO prompting, a novel template-based approach that guides Large Language Models (LLMs) towards decomposing complex problems into manageable steps. By explicitly structuring input information, planned actions, and intermediate outputs, IAO facilitates sequential problem-solving. Our evaluations across diverse reasoning tasks demonstrate performance gains, increased clarity, improved structural coherence, and enhanced transparency within the reasoning chain. Moreover, IAO shows domain independence, requiring minimal adaptation to cater to different problem types. Additional human evaluation validates IAO utility in finding errors in free-text reasoning chains.

\section*{Acknowledgements}

This research was supported by the Leverhulme Trust grant for the project 'Repurposing of Resources: from Everyday Problem Solving through to Crisis Management' (RPG-2021-182). We also thank all reviewers for their insightful feedback.



\bibliography{aaai25}

\appendix

\section*{Limitations}
While IAO prompting demonstrates compelling advantages in terms of performance and transparency, it is crucial to acknowledge potential limitations:
\paragraph{Output Length}: The structured nature of IAO prompts might lead to lengthier responses compared to baseline models. This can have implications on computational costs and real-time applicability, particularly in resource-constrained settings. Future work could explore techniques for compressing the output or developing domain-specific adaptations to mitigate this limitation.
\paragraph{Interpretability Trade-off}: While the structured output enhances interpretability, it is important to consider that some users might prefer more concise summaries. The human evaluation study supports this. In fact, when the answer is correct, evaluators heavily preferred CoT reasoning chains over IAO's. This suggests a potential trade-off between detailed explanations and user preferences. Future work could explore methods to balance the level of detail and provide tailored interpretations based on individual user needs.

\section*{Ethical Considerations}

While IAO holds promise for improved reasoning, concerns arise regarding potential bias amplification and misuse as per any prompting method for LLMs. Breaking down complex tasks into smaller steps could inadvertently magnify existing biases in the LLM or training data, leading to biased final outputs. This necessitates careful bias detection and mitigation. Additionally, the structured nature could be exploited to "trick" the model, generating harmful or misleading outputs. Safeguards like fact-checking and verification become crucial, especially in sensitive domains. Responsible deployment and use are fundamental to ensure LLMs benefits are harnessed ethically and its risks are minimized.

\begin{table*}[ht]
\resizebox{\textwidth}{!}{%
\begin{tabular}{@{}llllll@{}}
\toprule
Prompt                                              & Date Understanding & Last Letter & AQuA & CommonsenseQA & \multicolumn{1}{c}{Average} \\ \midrule
\texttt{[Step, Input, Action, Output]}              & 82.4               & 46.0        & 64.6 & 82.7          & 68.9                        \\
\texttt{[Step, Subquestion, Action, Output]}         & 81.8               & 84.8        & 63.0 & 81.2          & 77.7                        \\
\texttt{[Step, Subquestion, Input, Output]}         & 85.9               & 76.0        & 61.0 & 82.5          & 76.3                        \\
\texttt{[Step, Subquestion, Input, Action]}         & 86.2               & 4.4         & 62.6 & 82.9          & 59.0                        \\ \midrule
\texttt{[Step, Subquestion, Input, Action, Output]} & 88.1               & 88.8        & 63.9 & 83.1          & 80.9                        \\ \bottomrule
\end{tabular}%
}
\caption{Performance in a zero-shot setting when one of the field of the prompt is removed.}
\label{tab:ablation_results_appendix}
\end{table*}

\begin{comment}
\section{Additional Related Work}
\paragraph{Prompting}
Chain-of-thought (CoT) prompting provides intermediate rationales in the reasoning chain for solving problems, leading to improved accuracy and interpretability in tasks requiring multi-step reasoning. The common instance of CoT is zero-shot CoT \cite{kojima2022large} and few-shot CoT \cite{wei2022chain}. The first one is the most relevant to our work. 
On the other hand, few-shot prompting requires manually constructed task-specific demonstrations, so it is more costly. 
\citet{zhang2022automatic} proposed an approach to automatically construct few-shot demonstrations based on zero-shot prompting to save the manual cost. Another categorization for prompting methods is based on whether they involve a single step or multiple steps. An example of multi-step prompting is the work by \citet{zhou2022least} which is discussed in the next paragraph. There are several CoT prompts with structured reasoning chains in the form of programs \cite{chen2023program}, graphs \cite{besta2024graph} or tables \cite{ziqi-lu-2023-tab}. In this work we focus on free-text reasoning chains.
\end{comment}

\section{Compute cost}
PaLM-2 (text-unicorn) has a cost of per API requests: \$0.0025/1K characters in input and \$0.0075/1K characters in output. GPT-4 has a cost \$0.01/1K tokens and \$0.03/1K tokens. 


\section{Datasets}
\label{sec:dataset_appendix}

All included datasets are in English. We note that AQuA-RAT, Date Understanding, Object Shuffling are under the Apache License, Version 2.0. GSM8K and StrategyQA are under the MIT License. We also note that the datasets may include names of individuals collected from
the internet, i.e., publicly available facts about a
person but not in an offensive way. The following list shows the sources of data we used for this study:
\begin{itemize}
\small
    \item AQuA-RAT: \url{https://github.com/google-deepmind/AQuA}
    \item GSM8K: \url{https://github.com/openai/grade-school-math}
    \item StrategyQA: \url{https://github.com/google/BIGbench/tree/main/bigbench/benchmark_tasks/strategyqa}
    \item Last Letter: \url{https://github.com/kojima-takeshi188/zero_shot_cot/tree/main/dataset/last_letters}
    \item Date Understanding: from BIG-Bench (BIG-bench collaboration, 2021): \url{https://github.com/google/BIG-bench/blob/main/}
    \item Object Tracking: from BIG-Bench (BIG-bench collaboration, 2021): \url{https://github.com/google/BIG-bench/blob/main/}
    \item CommonsenseQA: \url{https://www.tau-nlp.sites.tau.ac.il/commonsenseqa}
\end{itemize}%


\begin{table}[H]
\centering
\resizebox{0.9\columnwidth}{!}{%
\begin{tabular}{@{}lll@{}}
\toprule
Dataset            & Avg words & Filename                \\ \midrule
AQuA               & 51.9      & test.jsonl              \\
GSM8k              & 46.9      & test.jsonl              \\
D.U. & 35.0      & task.json               \\
O.T.    & 91.1      & three\_objects/task.json \\
L.L       & 15.0      & last\_letters.json       \\
C.QA      & 27.8      & dev\_rand\_split.jsonl    \\
S.QA         & 9.6       & task.json               \\ \bottomrule
\end{tabular}%
}
\caption{Datasets statistics and files used. D.U: Date Understanding, O.T: Object Tracking, L.L: Last Letters Concatenation, C.QA: CommonsenseQA, S.QA: StrategyQA}
\label{tab:more_data_stats}
\end{table}

\begin{table}[H]
\resizebox{0.9\columnwidth}{!}{%
\begin{tabular}{@{}llll@{}}
\toprule
Reasoning Type                                  & Dataset            & Size & Answer Type     \\ \midrule
\multirow{2}{*}{\textit{Arithmetic Reasoning}}  & AQUA               & 254  & Multiple Choice \\
                                                & GSM8k              & 1319 & Numeral         \\ \midrule
\multirow{2}{*}{\textit{Commonsense Reasoning}} & StrategyQA         & 2290 & Yes/No          \\
                                                & CommonsenseQA      & 1221 & Multiple Choice \\ \midrule
\multirow{2}{*}{\textit{Other Reasoning}}       & Date Understanding & 369  & Multiple Choice \\
                                                & Object Tracking    & 750  & Multiple Choice \\ \midrule
\textit{Symbolic Reasoning}                     & Last Letter        & 500  & String          \\ \bottomrule
\end{tabular}%
}
\caption{Tasks, data statistics and answer type.}
\label{tab:data_statistics}
\end{table}


%
\begin{table}[H]
\centering
\resizebox{0.95\columnwidth}{!}{%
\begin{tabular}{@{}ll@{}}
\toprule
Dataset            & Answer prompt                               \\ \midrule
AQuA               & Therefore, among A through E, the answer is \\
GSM8k              & Therefore, the answer (arabic numerals) is  \\
D.U. & Therefore, among A through F, the answer is \\
O.T.    & Therefore, among A through C, the answer is \\
L.L.       & Therefore, the answer is                    \\
C.QA      & Therefore, among A through E, the answer is \\
S.QA         & The answer (Yes or No) is                   \\ \bottomrule
\end{tabular}%
}
\caption{Answer extraction prompts used. D.U: Date Understanding, O.T: Object Tracking, L.L: Last Letters Concatenation, C.QA: CommonsenseQA, S.QA: StrategyQA}
\label{tab:answer-prompt}
\end{table}

\section{Ablation studies}
\Cref{tab:ablation_results_appendix} shows how removing each field impacts the model's accuracy on different tasks. This helps understand the importance and relevance of each field in guiding the model towards successful completion of tasks.
For example, removing the "Subquestion" field seems to have a significant negative impact on the "Date Understanding" accuracy, suggesting that subquestions might provide crucial context for understanding temporal information.
The Last Letter task seems most affected by any field removal, highlighting its dependence on full prompt context. 


\section{Experimental setting}
All results reported for PALM-2 are the results of average over 3 runs. The results reported for GPT-4 are over a single run due to higher costs. 
For each model, the temperature was set to 0, the maximum number of output tokens was set to 1024.
\begin{itemize}
    \item \texttt{gpt-4-1106-preview}: \url{https://platform.openai.com/docs/models/gpt-4-and-gpt-4-turbo}
    \item \texttt{PALM-2 (text-unicorn)}: \url{https://cloud.google.com/vertex-ai/docs/generative-ai/model-reference/text}
\end{itemize}

\section{Output word count}

\begin{table}[H]
\centering
\resizebox{0.4\columnwidth}{!}{%
\begin{tabular}{@{}lll@{}}
\toprule
                & \textbf{CoT} & \textbf{IAO} \\ \midrule
\textbf{PALM-2} & 144.3        & 155.2        \\
\textbf{GPT-4}  & 188.2        & 175.4        \\ \bottomrule
\end{tabular}%
}
\caption{Average output word count.}\label{tab:wc_avg}
\end{table}

We sampled 50 reasoning chains for each model across various benchmarks and report the average output word count.
GPT-4 tends to be more verbose in CoT and IAO structure helps in shortening the output reasoning but the opposite is seen in PALM-2.





\section{Additional Examples}

\begin{table*}[h!]
\resizebox{\textwidth}{!}{%
\begin{tabular}{@{}llllll@{}}
\toprule
\multicolumn{2}{l}{Question} &
  \multicolumn{4}{l}{Take the last letters of each words in "Pauline Kerry Jeannette Hope" and concatenate them. \hspace{5cm} \textcolor{blue}{(GT: "eyee")}} \\ \midrule
\multirow{4}{*}{\rotatebox[origin=c]{90}{PALM-2}} &
  CoT &
  \multicolumn{4}{l}{\begin{tabular}[c]{@{}l@{}}The last letters of each word in "Pauline Kerry Jeannette Hope" are "e", "y", "e", and "e".\\ Concatenating them, we get "eyeee".\\ Therefore, the answer is \textbf{eyeee}.\end{tabular}} \\ \cmidrule(l){2-6} 
 &
  \multirow{3}{*}{IAO} &
  Subquestion &
  Input &
  Action &
  Output \\ \cmidrule(l){3-6} 
 &
   &
  \begin{tabular}[c]{@{}l@{}}1. What are the last letters of each word in \\ "Pauline Kerry Jeannette Hope"?\end{tabular} &
  "Pauline Kerry Jeannette Hope" &
  Get the last letters of each word &
  "e y e e" \\
 &
   &
  2. Concatenate the last letters of each word. &
  "e y e e" &
  Concatenate the letters. &
  \textbf{"eyee"} \\ \midrule
\multirow{7}{*}{\rotatebox[origin=c]{90}{GPT-4}} &
  CoT &
  \multicolumn{4}{l}{\begin{tabular}[c]{@{}l@{}}To solve this, we need to take the last letter of each word in the phrase "Pauline Kerry Jeannette Hope" and then concatenate them, \\ which means to link them together in a sequence.\\ Here are the last letters of each word:\\ - Pauline: e\\ - Kerry: y\\ - Jeannette: e\\ - Hope: e\\ Now, we concatenate these letters: e + y + e + e\\ Therefore, the answer is \textbf{"eyee”}\end{tabular}} \\ \cmidrule(l){2-6} 
 &
  \multirow{6}{*}{IAO} &
  Subquestion &
  Input &
  Action &
  Output \\ \cmidrule(l){3-6} 
 &
   &
  1. What is the last letter of the first word "Pauline"? &
  The word "Pauline" &
  \begin{tabular}[c]{@{}l@{}}Identify the last letter of the \\ word "Pauline".\end{tabular} &
  The last letter is "e". \\
 &
   &
  2. What is the last letter of the second word "Kerry"? &
  The word "Kerry" &
  \begin{tabular}[c]{@{}l@{}}Identify the last letter of the \\ word "Kerry".\end{tabular} &
  The last letter is "y". \\
 &
   &
  3. What is the last letter of the third word "Jeannette"? &
  The word "Jeannette" &
  \begin{tabular}[c]{@{}l@{}}Identify the last letter of the \\ word "Jeannette".\end{tabular} &
  The last letter is "e". \\
 &
   &
  4. What is the last letter of the fourth word "Hope"? &
  The word "Hope" &
  \begin{tabular}[c]{@{}l@{}}Identify the last letter of the \\ word "Hope".\end{tabular} &
  The last letter is "e". \\
 &
   &
  3. Concatenate the last letters of each word. &
  The last letters "e", "y", "e", "e" &
  \begin{tabular}[c]{@{}l@{}}Concatenate the letters "e", "y", "e", "e" \\ in the order they were obtained.\end{tabular} &
  \begin{tabular}[c]{@{}l@{}}The concatenated result is \\ \textbf{"eyee"}.\end{tabular} \\ \bottomrule
\end{tabular}%
}
\caption{Output examples from Last Letters dataset. The final results are in \textbf{bold}.}
\label{tab:letter_example}
\end{table*}

%
\begin{table*}[h!]
\resizebox{\textwidth}{!}{%
\begin{tabular}{@{}llllll@{}}
\toprule
\multicolumn{2}{l}{Question} &
  \multicolumn{4}{l}{\begin{tabular}[c]{@{}l@{}}Today's meeting is rescheduled to 11 am tomorrow, 10/16/1924. What is the date 10 days ago in MM/DD/YYYY? \\ Answer Choices: (A) 10/05/1924 (B) 05/05/1924 (C) 10/12/1924 (D) 10/06/1924 (E) 11/02/1924 (F) 10/04/1924  \hspace{5cm}    \textcolor{blue}{(GT: (A))}\end{tabular}} \\ \midrule
\multirow{4}{*}{\rotatebox[origin=c]{90}{PALM-2}} &
  CoT &
  \multicolumn{4}{l}{\begin{tabular}[c]{@{}l@{}}The date of the meeting is 10/16/1924.\\ 10 days ago from 10/16/1924 is 10/06/1924.\\ Therefore, among A through F, the answer is \textbf{D}\end{tabular}} \\ \cmidrule(l){2-6} 
 &
  \multirow{3}{*}{IAO} &
  Subquestion &
  Input &
  Action &
  Output \\ \cmidrule(l){3-6} 
 &
   &
  1. What is the date 10 days ago? &
  10/16/1924 &
  subtract 10 days from the given date &
  10/06/1924 \\
 &
   &
  \begin{tabular}[c]{@{}l@{}}2. What is the answer choice that \\ matches the date 10 days ago?\end{tabular} &
  10/06/1924 &
  \begin{tabular}[c]{@{}l@{}}compare the date 10 days ago \\ with the answer choices\end{tabular} &
  \textbf{(D)} 10/06/1924 \\ \midrule
\multirow{5}{*}{\rotatebox[origin=c]{90}{GPT-4}} &
  CoT &
  \multicolumn{4}{l}{\begin{tabular}[c]{@{}l@{}}To find the date 10 days before 10/16/1924, we need to subtract 10 days from that date.\\ Starting with 10/16/1924:\\ 1. If we subtract 10 days from the 16th, we would be in the previous month, October, since October has 31 days.\\ 2. To find the exact date, we subtract 10 from 16, which gives us 6.\\ Therefore, the date 10 days before 10/16/1924 is 10/06/1924.\\ Among the answer choices provided:\\ (A) 10/05/1924\\ (B) 05/05/1924\\ (C) 10/12/1924\\ (D) 10/06/1924\\ (E) 11/02/1924\\ (F) 10/04/1924\\ The correct answer is \textbf{(D)} 10/06/1924.\end{tabular}} \\ \cmidrule(l){2-6} 
 &
  \multirow{4}{*}{IAO} &
  Subquestion &
  Input &
  Action &
  Output \\ \cmidrule(l){3-6} 
 &
   &
  \begin{tabular}[c]{@{}l@{}}1. What is the date 10 days \\ before 10/16/1924?\end{tabular} &
  The given date is 10/16/1924. &
  Subtract 10 days from the given date. &
  \begin{tabular}[c]{@{}l@{}}The date 10 days before\\ 10/16/1924.\end{tabular} \\
 &
   &
  \multicolumn{4}{l}{{\begin{tabular}[c]{@{}l@{}}To find the date 10 days before 10/16/1924, we need to count backwards day by day. Since October has 31 days, subtracting 10 days\\  from the 16th will not take us into the previous month. Therefore, we simply subtract 10 from 16.\\ 16 - 10 = 6\\ The month (October) and the year (1924) remain the same. Therefore, the date 10 days before 10/16/1924 is 10/06/1924.\\ Therefore, among A through F, the answer is \textbf{(D)} 10/06/1924\end{tabular}}} \\
 &
   &
  \multicolumn{4}{l}{} \\ \bottomrule
\end{tabular}%
}
\caption{Output examples from Date Understanding dataset. The final results are in \textbf{bold}.}
\label{tab:date_example}
\end{table*}
%
%
\begin{table*}[h!]
\resizebox{\textwidth}{!}{%
\begin{tabular}{@{}llllll@{}}
\toprule
\multicolumn{2}{l}{Question} &
  \multicolumn{4}{l}{\begin{tabular}[c]{@{}l@{}}Alice, Bob, and Claire are friends and avid readers who occasionally trade books. At the start of the semester, they each buy one new book: Alice gets \\ The Great Gatsby, Bob gets The Odyssey, and Claire gets Lolita. As the semester proceeds, they start trading around the new books. First, Alice and \\ Bob swap books. Then, Alice and Claire swap books. Finally, Bob and Claire swap books. At the end of the semester, Bob has \\ Which choice is true ? Answer Choices: (A) The Great Gatsby. (B) The Odyssey. (C) Lolita. \hspace{4cm} \textcolor{blue}{GT: (B) The Odissey}\end{tabular}} \\ \midrule
\multirow{4}{*}{\rotatebox[origin=c]{90}{PALM-2}} &
  CoT &
  \multicolumn{4}{l}{\begin{tabular}[c]{@{}l@{}}At the start of the semester, Alice has The Great Gatsby, Bob has The Odyssey, and Claire has Lolita.\\ After Alice and Bob swap books, Alice has The Odyssey and Bob has The Great Gatsby.\\ After Alice and Claire swap books, Alice has Lolita and Claire has The Odyssey.\\ After Bob and Claire swap books, Bob has Lolita and Claire has The Great Gatsby.\\ Therefore, among A through C, the answer is \textbf{C}.\end{tabular}} \\ \cmidrule(l){2-6} 
 &
  \multirow{3}{*}{IAO} &
  Subquestion &
  Input &
  Action &
  Output \\ \cmidrule(l){3-6} 
 &
   &
  1. What book does Bob have after the first swap? &
  \begin{tabular}[c]{@{}l@{}}Alice has The Great Gatsby, \\ Bob has The Odyssey.\end{tabular} &
  Bob and Alice swap books. &
  Bob has The Great Gatsby. \\
 &
   &
  2. What book does Bob have after the second swap? &
  \begin{tabular}[c]{@{}l@{}}Bob has The Great Gatsby, \\ Claire has Lolita.\end{tabular} &
  Bob and Claire swap books. &
  Bob has The Great Gatsby. \\
 &
   &
  3. What book does Bob have at the end of the semester? &
  Bob has The Great Gatsby. &
  No more swaps. &
  Bob has \textbf{The Great Gatsby}. \\ \midrule
\multirow{5}{*}{\rotatebox[origin=c]{90}{GPT-4}} &
  CoT &
  \multicolumn{4}{l}{\begin{tabular}[c]{@{}l@{}}Let's track the movement of the books step by step:\\ 1. Initially:\\    - Alice has The Great Gatsby.\\    - Bob has The Odyssey.\\    - Claire has Lolita.\\ 2. After Alice and Bob swap books:\\    - Alice has The Odyssey.\\    - Bob has The Great Gatsby.\\    - Claire still has Lolita.\\ 3. Then Alice and Claire swap books:\\    - Alice has Lolita.\\    - Bob still has The Great Gatsby.\\    - Claire has The Odyssey.\\ 4. Finally, Bob and Claire swap books:\\    - Alice still has Lolita.\\    - Bob has The Odyssey.\\    - Claire has The Great Gatsby.\\ At the end of the semester, Bob has The Odyssey.\\ Therefore, among A through C, the answer is \textbf{(B) The Odyssey}.\end{tabular}} \\ \cmidrule(l){2-6} 
 &
  IAO &
  Subquestion &
  Input &
  Action &
  Output \\ \cmidrule(l){2-6} 
 &
   &
  1. What book does Bob have after he and Alice swap books? &
  \begin{tabular}[c]{@{}l@{}}Alice has The Great Gatsby, \\ Bob has The Odyssey.\end{tabular} &
  \begin{tabular}[c]{@{}l@{}}Swap the books between \\ Alice and Bob.\end{tabular} &
  \begin{tabular}[c]{@{}l@{}}Alice now has The Odyssey, \\ Bob now has The Great Gatsby.\end{tabular} \\
 &
   &
  2. What book does Alice have after she and Claire swap books? &
  \begin{tabular}[c]{@{}l@{}}Alice has The Odyssey, \\ Claire has Lolita.\end{tabular} &
  \begin{tabular}[c]{@{}l@{}}Swap the books between \\ Alice and Claire.\end{tabular} &
  \begin{tabular}[c]{@{}l@{}}Alice now has Lolita, \\ Claire now has The Odyssey.\end{tabular} \\
 &
   &
  3. What book does Bob have after he and Claire swap books? &
  \begin{tabular}[c]{@{}l@{}}Bob has The Great Gatsby, \\ Claire has The Odyssey.\end{tabular} &
  \begin{tabular}[c]{@{}l@{}}Swap the books between \\ Bob and Claire.\end{tabular} &
  \begin{tabular}[c]{@{}l@{}}Bob now has \textbf{The Odyssey}, \\ Claire now has The Great Gatsby.\end{tabular} \\ \bottomrule
\end{tabular}%
}
\caption{Output examples from Object Tracking dataset. The final results are in \textbf{bold}.}
\label{tab:object_example}
\end{table*}
%
\begin{table*}[h!]
\resizebox{\textwidth}{!}{%
\begin{tabular}{@{}llllll@{}}
\toprule
\multicolumn{2}{l}{Question} &
  \multicolumn{4}{l}{\begin{tabular}[c]{@{}l@{}}Andrea has 8 more apples than Jamal and half as many bananas as Jamal. Jamal has 4 more bananas than apples. How many fruits are there \\ if Andrea has 52 apples? \hspace{5cm} \textcolor{blue}{(GT: 168)}\end{tabular}} \\ \midrule
\multirow{6}{*}{\rotatebox[origin=c]{90}{PALM-2}} &
  CoT &
  \multicolumn{4}{l}{\begin{tabular}[c]{@{}l@{}}Let's denote the number of apples Jamal has as A and the number of bananas Jamal has as B.\\ From the given information, we know that Andrea has 8 more apples than Jamal, so Andrea has A + 8 apples. We also know that Andrea has half as \\ many bananas as Jamal, so Andrea has B/2 bananas.\\ We are given that Andrea has 52 apples, so we can write the equation A + 8 = 52. Solving for A, we get A = 44.\\ We are also given that Jamal has 4 more bananas than apples, so we can write the equation B = A + 4. \\ Substituting the value of A, we get B = 44 + 4 = 48.\\ Now we can calculate the total number of fruits. Andrea has A + 8 apples and B/2 bananas, so she has 44 + 8 + 48/2 = 72 fruits. \\ Jamal has A apples and B bananas, so he has 44 + 48 = 92 fruits.\\ Therefore, the total number of fruits is 72 + 92 = 164.\\ \\ Therefore, the answer is \textbf{164}.\end{tabular}} \\ \cmidrule(l){2-6} 
 &
  \multirow{5}{*}{IAO} &
  Subquestion &
  Input &
  Action &
  Output \\ \cmidrule(l){3-6} 
 &
   &
  1. How many apples does Jamal have? &
  \begin{tabular}[c]{@{}l@{}}Andrea has 8 more apples than Jamal.\\  Andrea has 52 apples.\end{tabular} &
  Jamal has 52 - 8 = 44 apples. &
  Jamal has 44 apples. \\
 &
   &
  2. How many bananas does Jamal have? &
  \begin{tabular}[c]{@{}l@{}}Jamal has 4 more bananas than apples. \\ Jamal has 44 apples.\end{tabular} &
  Jamal has 44 + 4 = 48 bananas. &
  Jamal has 48 bananas. \\
 &
   &
  3. How many bananas does Andrea have? &
  \begin{tabular}[c]{@{}l@{}}Andrea has half as many bananas as Jamal. \\ Jamal has 48 bananas.\end{tabular} &
  Andrea has 48 / 2 = 24 bananas. &
  Andrea has 24 bananas. \\
 &
   &
  4. How many fruits are there in total? &
  \begin{tabular}[c]{@{}l@{}}Andrea has 52 apples and 24 bananas. \\ Jamal has 44 apples and 48 bananas.\end{tabular} &
  There are 52 + 24 + 44 + 48 = 168 fruits in total. &
  There are \textbf{168} fruits in total. \\ \midrule
\multirow{8}{*}{\rotatebox[origin=c]{90}{GPT-4}} &
  CoT &
  \multicolumn{4}{l}{\begin{tabular}[c]{@{}l@{}}First, let's determine the number of apples and bananas Jamal has.\\ According to the information given, Andrea has 8 more apples than Jamal. Since Andrea has 52 apples, we can calculate the number of apples \\ Jamal has by subtracting 8 from Andrea's total:\\ Jamal's apples = Andrea's apples - 8\\ Jamal's apples = 52 - 8\\ Jamal's apples = 44\\ Now we know Jamal has 44 apples.\\ Next, we are told that Jamal has 4 more bananas than apples. Since we know Jamal has 44 apples, we can find out how many bananas he has:\\ Jamal's bananas = Jamal's apples + 4\\ Jamal's bananas = 44 + 4\\ Jamal's bananas = 48\\ \\ So, Jamal has 48 bananas.\\ Now, let's find out how many bananas Andrea has. We are told that Andrea has half as many bananas as Jamal. So we can calculate this as follows:\\ \\ Andrea's bananas = Jamal's bananas / 2\\ Andrea's bananas = 48 / 2\\ Andrea's bananas = 24\\ \\ Now we can calculate the total number of fruits by adding the number of apples and bananas both Andrea and Jamal have:\\ Total fruits = Andrea's apples + Andrea's bananas + Jamal's apples + Jamal's bananas\\ Total fruits = 52 + 24 + 44 + 48\\ Total fruits = 76 + 92\\ Total fruits = 168\\ \\ Therefore, the answer is 168.\end{tabular}} \\ \cmidrule(l){2-6} 
 &
  \multirow{7}{*}{IAO} &
  Subquestion &
  Input &
  Action &
  Output \\ \cmidrule(l){3-6} 
 &
   &
  How many apples does Jamal have? &
  \begin{tabular}[c]{@{}l@{}}Andrea has 8 more apples than Jamal, \\ Andrea has 52 apples.\end{tabular} &
  \begin{tabular}[c]{@{}l@{}}Subtract 8 from the number of apples \\ Andrea has to find out how many apples Jamal has.\end{tabular} &
  Jamal has 52 - 8 = 44 apples. \\
 &
   &
  How many bananas does Jamal have? &
  \begin{tabular}[c]{@{}l@{}}Jamal has 4 more bananas than apples, \\ Jamal has 44 apples.\end{tabular} &
  \begin{tabular}[c]{@{}l@{}}Add 4 to the number of apples Jamal has to find \\ out how many bananas he has.\end{tabular} &
  Jamal has 44 + 4 = 48 bananas. \\
 &
   &
  How many bananas does Andrea have? &
  \begin{tabular}[c]{@{}l@{}}Andrea has half as many bananas as Jamal, \\ Jamal has 48 bananas.\end{tabular} &
  \begin{tabular}[c]{@{}l@{}}Divide the number of bananas Jamal has by 2 to \\ find out how many bananas Andrea has.\end{tabular} &
  Andrea has 48 / 2 = 24 bananas. \\
 &
   &
  How many fruits does Andrea have in total? &
  Andrea has 52 apples and 24 bananas. &
  Add the number of apples and bananas Andrea has. &
  Andrea has 52 + 24 = 76 fruits. \\
 &
   &
  How many fruits does Jamal have in total? &
  Jamal has 44 apples and 48 bananas. &
  Add the number of apples and bananas Jamal has. &
  Jamal has 44 + 48 = 92 fruits. \\
 &
   &
  How many fruits are there in total? &
  Andrea has 76 fruits, and Jamal has 92 fruits. &
  \begin{tabular}[c]{@{}l@{}}Add the number of fruits Andrea has to the\\  number of fruits Jamal has.\end{tabular} &
  There are 76 + 92 = \textbf{168} fruits in total. \\ \bottomrule
\end{tabular}%
}
\caption{Output examples from GSM8k dataset. The final results are in \textbf{bold}.}
\label{tab:gsm8k_example}
\end{table*}


\begin{table*}[h]
\resizebox{\textwidth}{!}{%
\begin{tabular}{@{}llllll@{}}
\toprule
\multicolumn{2}{l}{Question} &
  \multicolumn{4}{l}{\begin{tabular}[c]{@{}l@{}}The man tried to reply to the woman, but he had difficulty keeping track of conversations that he didn't do what to? \\ Answer Choices: (A) initiate (B) ignore (C) question (D) answer (E) ask \hspace{5cm} \textcolor{blue}{(GT: (A)}\end{tabular}} \\ \midrule
\multirow{5}{*}{\rotatebox[origin=c]{90}{PALM-2}} &
  CoT &
  \multicolumn{4}{l}{\begin{tabular}[c]{@{}l@{}}The man tried to reply to the woman, but he had difficulty keeping track of conversations that he didn't do what to?\\ The man tried to reply to the woman, but he had difficulty keeping track of conversations. This means that he was not able to follow the conversation properly. \\ He was not able to understand what the woman was saying. This could be because he was not paying attention or because he was not able to \\ understand the language.\\ The question asks what the man did not do. \\ The answer choices are: (A) initiate (B) ignore (C) question (D) answer (E) ask.\\ The man did not initiate the conversation. He was replying to the woman. He did not ignore the woman. He was trying to reply to her. He did not question\\  the woman. He was trying to understand what she was saying. He did not answer the woman. He was trying to reply to her. He did not ask the woman anything. \\ He was trying to understand what she was saying.\\ \\ Therefore, among A through E, the answer is \textbf{(D)}\end{tabular}} \\ \cmidrule(l){2-6} 
 &
  \multirow{4}{*}{IAO} &
  Subquestion &
  Input &
  Action &
  Output \\ \cmidrule(l){3-6} 
 &
   &
  \begin{tabular}[c]{@{}l@{}}1. What is the main idea \\ of the question?\end{tabular} &
  \begin{tabular}[c]{@{}l@{}}The man tried to reply to the \\ woman, but he had difficulty\\ keeping track of conversations\\  that he didn't do what to?\end{tabular} &
  \begin{tabular}[c]{@{}l@{}}Identify the main idea \\ of the question.\end{tabular} &
  \begin{tabular}[c]{@{}l@{}}The main idea of the question is that \\ the man had difficulty keeping track \\ of conversations that he didn't do something to.\end{tabular} \\
 &
   &
  \begin{tabular}[c]{@{}l@{}}2. What is the missing \\ word in the question?\end{tabular} &
  \begin{tabular}[c]{@{}l@{}}The man tried to reply to the\\ woman, but he had difficulty \\ keeping track of conversations \\ that he didn't do what to?\end{tabular} &
  \begin{tabular}[c]{@{}l@{}}Identify the missing \\ word in the question.\end{tabular} &
  The missing word in the question is "initiate". \\
 &
   &
  \begin{tabular}[c]{@{}l@{}}3. What is the answer to\\  the question?\end{tabular} &
  \begin{tabular}[c]{@{}l@{}}The man tried to reply to the\\ woman, but he had difficulty \\ keeping track of conversations\\ that he didn't do what to?\end{tabular} &
  \begin{tabular}[c]{@{}l@{}}Choose the answer that \\ best fits the question.\end{tabular} &
  The answer to the question is \textbf{(A)} initiate. \\ \midrule
\multirow{6}{*}{\rotatebox[origin=c]{90}{GPT-4}} &
  CoT &
  \multicolumn{4}{l}{\begin{tabular}[c]{@{}l@{}}To solve this reasoning problem, we need to determine which action would make it difficult for the man to keep track of conversations. If he \\ has difficulty keeping track of conversations, it is likely because he is not actively engaged in starting or controlling the flow of the conversation.\\ \\ Let's look at the options:\\ \\ (A) initiate - If he doesn't initiate the conversation, he might have trouble keeping track because he is not setting the topic or pace of the conversation.\\ (B) ignore - Ignoring a conversation would not cause difficulty in keeping track; it would mean he is not participating at all.\\ (C) question - Not questioning might make it harder to understand, but it doesn't necessarily mean he would lose track of the conversation.\\ (D) answer - If he doesn't answer, it could be a result of losing track, not a cause.\\ (E) ask - Similar to initiate, if he doesn't ask questions, he might not be engaged enough to follow the conversation.\\ \\ Between initiating and asking, initiating a conversation is a more fundamental action for controlling the flow of the conversation. \\ If he doesn't initiate conversations, he is more likely to be a passive participant and thus have difficulty keeping track.\\ \\ Therefore, among A through E, the answer is \textbf{(A)} initiate\end{tabular}} \\ \cmidrule(l){2-6} 
 &
  \multirow{5}{*}{IAO} &
  Subquestion &
  Input &
  Action &
  Output \\ \cmidrule(l){3-6} 
 &
   &
  \begin{tabular}[c]{@{}l@{}}1. What does the phrase \\ "difficulty keeping track of \\ conversations" imply about \\ the man's ability to engage \\ in conversations?\end{tabular} &
  \begin{tabular}[c]{@{}l@{}}The man had difficulty keeping \\ track of conversations.\end{tabular} &
  \begin{tabular}[c]{@{}l@{}}Analyze the phrase to determine \\ what aspect of conversation the\\  man struggles with.\end{tabular} &
  \begin{tabular}[c]{@{}l@{}}The man struggles with following along\\  or maintaining focus in conversations \\ that he is not actively leading or starting.\end{tabular} \\
 &
   &
  \begin{tabular}[c]{@{}l@{}}2. Which answer choice is \\ related to leading or starting a\\  conversation?\end{tabular} &
  \begin{tabular}[c]{@{}l@{}}Answer choices (A) initiate \\ (B) ignore (C) question \\ (D) answer (E) ask\end{tabular} &
  \begin{tabular}[c]{@{}l@{}}Identify the answer choice that \\ means to begin or lead a \\ conversation.\end{tabular} &
  (A) initiate \\
 &
   &
  \begin{tabular}[c]{@{}l@{}}3. Does the difficulty in keeping \\ track of conversations imply that \\ the man did not do what?\end{tabular} &
  \begin{tabular}[c]{@{}l@{}}The man had difficulty keeping\\ track of conversations; the man \\ tried to reply to the woman.\end{tabular} &
  \begin{tabular}[c]{@{}l@{}}Determine which action from \\ the answer choices would be most\\  difficult for someone who cannot \\ keep track of conversations.\end{tabular} &
  \begin{tabular}[c]{@{}l@{}}The man would have difficulty replying \\ (or keeping track) if he did not initiate \\ the conversation.\end{tabular} \\
 &
   &
  \multicolumn{4}{l}{Therefore, among A through E, the answer is \textbf{(A)} initiate.}
\end{tabular}%
}
\caption{Output examples from CommonsenseQA dataset. The final results are in \textbf{bold}.}
\label{tab:commonsenseqa_example}
\end{table*}

\clearpage\clearpage

\section{Errors from AQua}
\textbf{Note:} Several questions where IAO gave the wrong answers were due to the output exceeding the maximum number of tokens. 
\subsection{Example 1}
\textit{Question}: A grocer makes a 25\% profit on the selling price for each bag of flour it sells. If he sells each bag for \$100 and makes \$3,000 in profit, how many bags did he sell? 

\textit{Answer Choices}: (A) 12 (B) 16 (C) 24 (D) 30 (E) 40"

\textit{GT}:  "C"

\paragraph{IAO}:
\noindent
\textbf{Step 1:}
\begin{itemize}
    \item \textit{Subquestion:} What is the profit made on one bag of flour?
    \item \textit{Input:} 25\% profit on the selling price of \$100 per bag
    \item \textit{Action:} Calculate the profit per bag by multiplying the selling price by the profit percentage.
    \item \textit{Output:} \$25 profit per bag (since 25\% of \$100 is \$25)
\end{itemize}

\textbf{Step 2:}
\begin{itemize}
    \item \textit{Subquestion:} How many bags of flour were sold to make \$3,000 in profit?
    \item \textit{Input:} \$3,000 total profit, \$25 profit per bag
    \item \textit{Action:} Divide the total profit by the profit per bag to find the number of bags sold.
    \item \textit{Output:} 120 bags sold (since \$3,000 / \$25 per bag = 120 bags)
\end{itemize}

\textbf{Step 3:}
\begin{itemize}
    \item \textit{Subquestion:} Which answer choice corresponds to the number of bags sold?
    \item \textit{Input:} 120 bags sold, Answer Choices: (A) 12 (B) 16 (C) 24 (D) 30 (E) 40
    \item \textit{Action:} Identify the correct answer choice that matches the number of bags sold.
    \item \textit{Output:} None of the answer choices match the number of bags sold.
\end{itemize}

Therefore, among answer choices A through E, the answer is none of the provided options. There seems to be a mistake as none of the answer choices correspond to the correct number of bags sold based on the given information.

\paragraph{CoT}: To solve this problem, we need to determine how much profit the grocer makes on each bag of flour and then use that information to find out how many bags were sold to make a total profit of \$3,000.

\textbf{Step 1: Calculate the profit per bag.} \\
Since the grocer makes a 25\% profit on the selling price, we can calculate the profit per bag as follows:

\begin{align*}
\text{Profit per bag} &= 25\% \text{ of selling price} \\
\text{Profit per bag} &= 0.25 \times \$100 \\
\text{Profit per bag} &= \$25
\end{align*}

\textbf{Step 2: Calculate the number of bags sold.} \\
Now that we know the profit per bag is \$25, we can calculate the number of bags sold to make a total profit of \$3,000.

\begin{align*}
\text{Total profit} &= \text{Profit per bag} \times \text{Number of bags sold} \\
\$3,000 &= \$25 \times \text{Number of bags sold}
\end{align*}

To find the number of bags sold, we divide the total profit by the profit per bag:

\begin{align*}
\text{Number of bags sold} &= \frac{\$3,000}{\$25} \\
\text{Number of bags sold} &= 120
\end{align*}

However, 120 is not one of the answer choices provided, which means we need to re-evaluate our calculations.

Let's go back to the statement that the grocer makes a 25\% profit on the selling price. This means that the selling price includes both the cost price and the profit. If the selling price is \$100 and the profit is 25\% of this selling price, then the profit per bag is indeed \$25, as we calculated.

Now, let's re-calculate the number of bags sold using the correct profit per bag:

\begin{align*}
\text{Number of bags sold} &= \frac{\$3,000}{\$25} \\
\text{Number of bags sold} &= 120
\end{align*}

Since 120 is not among the answer choices, we need to check if we've made a mistake in interpreting the answer choices or the question itself. If the question and calculations are correct, then there might be an error in the answer choices provided. If this is the case, none of the answer choices (A) through (E) is correct.

Let's review the question one more time to ensure we haven't missed any details. The question states that the grocer makes a 25\% profit on the selling price for each bag of flour and sells each bag for \$100. Our calculation of the profit per bag at \$25 is correct, and so is our calculation of the number of bags sold to make a \$3,000 profit.

Therefore, it seems there is a discrepancy with the answer choices provided, as none of them match our calculation of 120 bags. If we must choose from the given options, we would have to conclude that none of them is correct. However, since this is not a standard outcome for such problems, it's possible that there is a misunderstanding or an error in the question or answer choices.

\subsection{Example 2}

\textit{Question}: Find 25/12*5 

\textit{Answer Choices}: (A) 2.5498 (B) 0.4167 (C) 3.3987 (D) 8.5497 (E) 5.6312

\textit{GT}: B

\paragraph{IAO}: \noindent
\textbf{Step 1:}
\begin{itemize}
    \item \textit{Subquestion:} What is the product of $(\frac{25}{12})$ and 5?
    \item \textit{Input:} $(\frac{25}{12})$, 5
    \item \textit{Action:} Multiply $(\frac{25}{12})$ by 5.
    \item \textit{Output:} 
    $\frac{25}{12} \times 5 = \frac{25 \times 5}{12} = \frac{125}{12}$
\end{itemize}
\textbf{Step 2:}
\begin{itemize}
    \item \textit{Subquestion:} What is the decimal equivalent of 125/12?
    \item \textit{Input:} 125/12
    \item \textit{Action:} Divide 125 by 12.
    \item \textit{Output:} 10.4167
\end{itemize}

\noindent
\textbf{Step 3:}
\begin{itemize}
    \item \textit{Subquestion:} Which answer choice matches the output from Step 2?
    \item \textit{Input:} 10.4167, Answer Choices: (A) 2.5498 (B) 0.4167 (C) 3.3987 (D) 8.5497 (E) 5.6312
    \item \textit{Action:} Compare the output with the given answer choices.
    \item \textit{Output:} None of the answer choices match the output from Step 2.
\end{itemize}

It seems there might have been a mistake in the calculation or in the provided answer choices. Let's re-evaluate the calculation in Step 2 to ensure accuracy.

\noindent
\textbf{Step 4:}
\begin{itemize}
    \item \textit{Subquestion:} What is the correct decimal equivalent of 125/12?
    \item \textit{Input:} 125/12
    \item \textit{Action:} Divide 125 by 12 using long division or a calculator.
    \item \textit{Output:} 10.4167 (confirmed)
\end{itemize}

Upon re-evaluation, the calculation is confirmed to be correct, but the output does not match any of the provided answer choices. It's possible that there is an error in the question or the answer choices. However, based on the calculation we've done, none of the answer choices (A) through (E) is correct.

Therefore, among A through E, the answer is none of the provided choices.

\paragraph{COT}: To solve the expression 
\[ \frac{25}{12} \times 5 \], follow these steps:

\textbf{Step 1: Simplify the expression by performing the multiplication.}

\[ \frac{25}{12} \times 5 = \frac{25 \times 5}{12} \]

\textbf{Step 2: Multiply the numerator.}

\[ \frac{25 \times 5}{12} = \frac{125}{12} \]

\textbf{Step 3: Divide 125 by 12 to get the decimal form.}

\[ 125 \div 12 = 10.4167 \]

\textbf{Step 4: Compare the result with the answer choices.}

The result, 10.4167, is not listed among the answer choices. It seems there might have been a mistake in the calculation. Let's try again.

\textbf{Step 1: Simplify the expression by performing the multiplication.}

\[ \frac{25}{12} \times 5 = \frac{25 \times 5}{12} \]

\textbf{Step 2: Multiply the numerator.}

\[ \frac{25 \times 5}{12} = \frac{125}{12} \]

\textbf{Step 3: Divide 125 by 12 to get the decimal form.}

\[ 125 \div 12 = 10 \text{ remainder } 5 \]

To express the remainder as a decimal, we divide 5 by 12:

\[ 5 \div 12 = 0.4167 \]

Adding this to the whole number part (10), we get:

\[ 10 + 0.4167 = 10.4167 \]

This result still does not match any of the answer choices. However, upon re-evaluating the answer choices, it seems that none of them are correct if the original expression was indeed 
\[ \frac{25}{12} \times 5 \]. It's possible that there might be a typo in the question or the answer choices.

Given the calculations we've performed, none of the answer choices (A) through (E) match the correct result of 10.4167. Therefore, it appears there is an error in the provided information.

\section{Errors from Date Understanding}
\subsection{Example 1}
\textit{Question}:2015 is coming in 36 hours. What is the date 24 hours later in MM/DD/YYYY? 

\textit{Answer Choices}: (A) 12/30/2014 (B) 01/01/2015 (C) 01/04/2015 (D) 12/30/2059 (E) 12/31/2014 (F) 01/09/2015 

\textit{GT}:  "A"

\paragraph{IAO}:
\noindent
\textbf{Step 1:}
\begin{itemize}
    \item \textit{Subquestion:} What is the current date if 2015 is coming in 36 hours?
    \item \textit{Input:} 2015 is coming in 36 hours.
    \item \textit{Action:} Subtract 36 hours from the start of 2015 to find the current date.
    \item \textit{Output:} The current date is December 30, 2014.
\end{itemize}

\noindent
\textbf{Step 2:}
\begin{itemize}
    \item \textit{Subquestion:} What is the date 24 hours after the current date?
    \item \textit{Input:} The current date is December 30, 2014.
    \item \textit{Action:} Add 24 hours to the current date to find the date 24 hours later.
    \item \textit{Output:} The date 24 hours later is December 31, 2014.
\end{itemize}

Therefore, among the answer choices A through F, the answer is (E) December 31, 2014.

\paragraph{COT}: If 2015 is coming in 36 hours, we can determine the current date by subtracting 36 hours from the start of 2015.

\begin{enumerate}
\item Since there are 24 hours in a day, 36 hours is equivalent to $1 \text{ day} + 12 \text{ hours}$.
\item The start of 2015 is at $00:00$ on January 1, 2015.
\item Subtracting $1$ day from January 1, 2015, takes us back to December 31, 2014.
\item Since we need to subtract $12$ more hours, and we are currently at $00:00$ on December 31, 2014, we would still be on December 31, 2014, because we are not going back an entire day, just $12$ hours into the previous day.
\end{enumerate}

So, the current date and time, 36 hours before 2015, is $12:00 \text{ (noon)}$ on December 31, 2014.

Now, we need to determine the date 24 hours later:

\begin{enumerate}
\item Adding $24$ hours to $12:00$ on December 31, 2014, will bring us to $12:00$ on January 1, 2015.
\end{enumerate}

Therefore, among A through F, the answer is (B) $01/01/2015$.

\subsection{Example 2}
\textit{Question}:Tomorrow is 11/12/2019. What is the date 10 days ago in MM/DD/YYYY? 

\textit{Answer Choices}: (A) 10/23/2019 (B) 11/02/2019 (C) 10/27/2019 (D) 11/01/2048 (E) 11/01/2019  

\textit{GT}:  "E"

\paragraph{IAO}: \noindent
\textbf{Step 1:}
\begin{itemize}
    \item \textit{Subquestion:} \underline{What is the date 10 days before 11/12/2019?}
    \item \textit{Input:} The given date is 11/12/2019.
    \item \textit{Action:} Subtract 10 days from the given date.
    \item \textit{Output:} The date 10 days before 11/12/2019 is 11/02/2019.
\end{itemize}

\noindent
Therefore, among A through E, the answer is (B) 11/02/2019.


\paragraph{COT}: To find the date 10 days before 11/12/2019, we need to subtract 10 days from that date.

\textbf{Step 1: Start with the given date:} 11/12/2019 (which is in MM/DD/YYYY format, meaning November 12, 2019).

\textbf{Step 2: Subtract 10 days from the 12th of November.} Since $12 - 10 = 2$, if November had more than 2 days, we would end up in November itself. However, since we're subtracting more days than there are in the month up to the 12th, we need to go back to the previous month, which is October.

\textbf{Step 3: October has 31 days.} To find out what date it would be 10 days before November 12, we need to count back from October 31.

\textbf{Step 4: Since we need to go back 10 days from November 12, and there are only 1 day of November included in this count (November 12 itself), we subtract the remaining 9 days from the total number of days in October.}

\textbf{Step 5:} $31 \text{ (total days in October)} - 9 \text{ (days to subtract)} = 22$.

\textbf{Step 6: Therefore, 10 days before November 12, 2019, would be October 22, 2019.}

\textbf{Step 7: Convert this date into MM/DD/YYYY format, which gives us 10/22/2019.}

However, none of the answer choices match this date. It seems there might be a mistake in the answer choices provided or in the calculation. Let's double-check the calculation:

\textbf{Given date:} 11/12/2019

\textbf{Subtract 10 days:}

* November has 12 days up to and including the 12th.
* We need to subtract 10 days.
* $12 - 10 = 2$, so we would be at November 2nd if we didn't need to go into the previous month.
* But since we're subtracting 10 days, we do need to go into the previous month.
* October has 31 days, so the day before November 1st is October 31st.
* We've already accounted for 1 day in November, so we subtract 1 from 10, leaving us with 9 days to subtract from October 31st.
* $31 - 9 = 22$, so we end up on October 22nd.

Upon re-evaluating the calculation, the correct date is indeed October 22, 2019, which is not listed in the answer choices. It seems there may be a typo in the question or the answer choices. However, based on the calculation, the correct answer is not present in the options provided.

\textbf{Therefore, among A through E, the answer is none of the above.}

\section{Errors from StrategyQA}
\subsection{Example 1}
\textit{Question}: Would John Muir not likely have a vitamin D deficiency? 

\textit{GT}:  "yes"

\paragraph{IAO}: To determine whether John Muir would likely have a vitamin D deficiency, we need to consider factors that influence vitamin D levels, such as exposure to sunlight, diet, skin pigmentation, age, geographic location, and lifestyle. Since we don't have specific information about John Muir's personal details, we'll have to rely on general knowledge about his life and activities.

John Muir (1838--1914) was a Scottish-American naturalist, author, environmental philosopher, and early advocate for the preservation of wilderness in the United States. He spent a significant amount of time outdoors, particularly in places like Yosemite and other parts of the American West.

\begin{enumerate}
    \item \textbf{Step 1:}
    \begin{itemize}
        \item \textit{Subquestion:} Did John Muir spend a significant amount of time outdoors?
        \item \textit{Input:} Historical knowledge about John Muir's life and activities.
        \item \textit{Action:} Assess the likelihood of outdoor activity based on known biographical information.
        \item \textit{Output:} Yes, John Muir spent a significant amount of time outdoors.
    \end{itemize}

    \item \textbf{Step 2:}
    \begin{itemize}
        \item \textit{Subquestion:} Does spending a significant amount of time outdoors increase exposure to sunlight?
        \item \textit{Input:} General knowledge about outdoor activities and sunlight exposure.
        \item \textit{Action:} Correlate outdoor activity with sunlight exposure.
        \item \textit{Output:} Yes, spending a significant amount of time outdoors generally increases exposure to sunlight.
    \end{itemize}

    \item \textbf{Step 3:}
    \begin{itemize}
        \item \textit{Subquestion:} Does increased exposure to sunlight affect vitamin D levels?
        \item \textit{Input:} Scientific knowledge about vitamin D synthesis.
        \item \textit{Action:} Relate sunlight exposure to vitamin D synthesis.
        \item \textit{Output:} Yes, increased exposure to sunlight typically leads to higher vitamin D levels due to the skin's production of vitamin D in response to UVB rays.
    \end{itemize}

    \item \textbf{Step 4:}
    \begin{itemize}
        \item \textit{Subquestion:} Based on John Muir's outdoor activities and the relationship between sunlight exposure and vitamin D synthesis, is it likely that John Muir had a vitamin D deficiency?
        \item \textit{Input:} Outputs from Step 1, Step 2, and Step 3.
        \item \textit{Action:} Infer the likelihood of vitamin D deficiency based on the correlation between outdoor activity, sunlight exposure, and vitamin D synthesis.
        \item \textit{Output:} It is unlikely that John Muir had a vitamin D deficiency given his significant time spent outdoors and the resulting increased exposure to sunlight.
    \end{itemize}
\end{enumerate}

Therefore, the answer (Yes or No) is No.

\paragraph{COT}: \textbf{Step 1: Identify the factors that contribute to vitamin D deficiency.}

Vitamin D deficiency can be caused by several factors, including:

\begin{itemize}
\item Lack of exposure to sunlight
\item Inadequate dietary intake of vitamin D
\item Certain medical conditions
\item The body's inability to absorb or convert vitamin D to its active form
\end{itemize}

\textbf{Step 2: Consider John Muir's lifestyle.}

John Muir was a naturalist, explorer, and advocate for the preservation of wilderness.  His lifestyle involved spending a significant amount of time outdoors, often in sunny environments like California's Sierra Nevada mountains.

\textbf{Step 3: Assess the likelihood of vitamin D deficiency based on lifestyle.}

Given that John Muir spent a lot of time outdoors, he would have been exposed to sunlight regularly. Sunlight exposure is a primary source of vitamin D, as the body synthesizes vitamin D when the skin is exposed to UV rays from the sun.

\textbf{Step 4: Draw a conclusion based on the information.}

Since John Muir was frequently outdoors and exposed to sunlight, it is \emph{less likely} that he would have suffered from vitamin D deficiency due to lack of sunlight exposure.

\textbf{Therefore, the answer (Yes or No) is No.}


%%%%%

\textit{Question}: Can French Defence initial move defend against four move checkmate? 

\textit{GT}:  "no"

\paragraph{IAO}: To determine if the French Defence can defend against the four-move checkmate (also known as the Scholar's Mate), we need to analyze the moves involved in both the French Defence and the Scholar's Mate.

\begin{itemize}
    \item \textbf{Step 1:}
    \begin{itemize}
        \item \textit{Subquestion:} What are the typical opening moves for the French Defence?
        \item \textit{Input:} The French Defence is a chess opening characterized by the moves.
        \item \textit{Action:} Describe the typical opening moves for the French Defence.
        \item \textit{Output:} The French Defence typically begins with 1.e4 e6.
    \end{itemize}

    \item \textbf{Step 2:}
    \begin{itemize}
        \item \textit{Subquestion:} What are the moves involved in the four-move checkmate?
        \item \textit{Input:} The four-move checkmate (Scholar's Mate) sequence.
        \item \textit{Action:} Describe the typical sequence of moves that lead to the four-move checkmate.
        \item \textit{Output:} The four-move checkmate typically involves the moves 1.e4 e5 2.Qh5 Nc6 3.Bc4 Nf6 4.Qxf7\#.
    \end{itemize}

    \item \textbf{Step 3:}
    \begin{itemize}
        \item \textit{Subquestion:} Can the opening move of the French Defence prevent the specific sequence of the four-move checkmate?
        \item \textit{Input:} The opening move of the French Defence (1.e4 e6) and the sequence of the four-move checkmate (1.e4 e5 2.Qh3 Nc6 3.Bc4 Nf6 4.Qxf7\#).
        \item \textit{Action:} Compare the opening moves of the French Defence with the moves required for the four-move checkmate to determine if the French Defence can prevent the checkmate sequence.
        \item \textit{Output:} The French Defence does not play 1...e5, which is a necessary move for the four-move checkmate. Instead, it plays 1...e6, which already deviates from the Scholar's Mate sequence.
    \end{itemize}

    \item \textbf{Step 4:}
    \begin{itemize}
        \item \textit{Subquestion:} Does the deviation from the four-move checkmate sequence inherently prevent the checkmate from occurring?
        \item \textit{Input:} The deviation in the opening move of the French Defence from the Scholar's Mate sequence.
        \item \textit{Action:} Analyze the impact of the deviation on the possibility of the four-move checkmate occurring.
        \item \textit{Output:} Since the French Defence does not follow the move 1...e5, it inherently prevents the specific sequence of the four-move checkmate from occurring.
    \end{itemize}

\end{itemize}

Therefore, the answer (Yes or No) is Yes.

\paragraph{COT}: The Four-Move Checkmate, also known as the Scholar's Mate, is a quick checkmate that White can deliver after the following moves:

\begin{enumerate}
\item e4 e5
\item Qh5 Nc6
\item Bc4 Nf6
\item Qxf7\#
\end{enumerate}

In this sequence, White aims to attack the f7 square, which is a weak point in Black's position because it is only defended by the king at the beginning of the game.

The French Defense, on the other hand, is a chess opening that begins with the moves:

\begin{enumerate}
\item e4 e6
\end{enumerate}

The idea behind the French Defense is to solidify Black's position and create a strong pawn chain that can later be used to counterattack White's center. The French Defense does not immediately contest the e5 square, which is a key component of setting up the Scholar's Mate.

Let's see if the French Defense can prevent the Four-Move Checkmate:

\begin{enumerate}
\item e4 e6
\item d4 d5
\end{enumerate}

With the second move d5, Black challenges White's center and opens lines for the bishop and queen, which can help in defending against early threats.

If White tries to set up the Scholar's Mate anyway:

\begin{enumerate}
\item 3. Qh5
\end{enumerate}

Black can now defend easily because the move ...e6 has already prevented the bishop from coming to c4 to attack the f7 square. Black has several reasonable moves here, such as ...Nf6, which attacks the invading queen and prepares to castle, or even ...g6, which would force the queen to move again.

Since the French Defense does not allow White to establish the same attacking pattern as in the Scholar's Mate, it effectively prevents the quick checkmate.

\textbf{Therefore, the answer (Yes or No) is Yes.}

\end{document}
