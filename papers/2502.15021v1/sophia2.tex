\definecolor{orangeFFN}{HTML}{D79B00}
\definecolor{blueFFN}{HTML}{0060D9}

\begin{figure*}
\centering
\begin{tikzpicture}[font=\sffamily]
	\node (image) at (0,0) {
        \includegraphics[width=\textwidth]{sophia.pdf}
	};
    \node[
      anchor=west,
      label={[yshift=0.8cm, xshift=-8.6cm]above:{\bfseries ViT+Jumbo (ours)}}, % Text on top
      text width=0.46\textwidth,
      align=center,
      minimum height=3cm
    ] (left) at (0,0) {};

    \node[
      anchor=west,
      label={[yshift=0.8cm, xshift=0.24cm]above:{\bfseries ViT+Registers \cite{darcet2024vision}}}, % Text on top
      text width=0.46\textwidth,
      align=center,
      minimum height=3cm
    ] (left) at (0,0) {};
\end{tikzpicture}
\vspace{-0.75cm}
\caption{
\textbf{(Left)} Our ViT+Jumbo method creates a wider \texttt{CLS} token that gets split into several tokens, with width equal to the patch width, prior to multi-headed self-attention (MHSA). After attention, the split Jumbo \texttt{CLS} token is reassembled via concatenation, and is then processed by {\textbf{\color{blueFFN}{\emph{its own} FFN}}}. Patches are processed by {\textbf{\color{orangeFFN}{their own, shared FFN}}}. \textbf{(Right)} ViT+Registers creates many register tokens all equal to the patch width. All registers, patches, and the \texttt{CLS} token are processed by {\textbf{\color{orangeFFN}{a shared FFN}}}. ViT+Jumbo enhances global processing as the (split) global tokens can interact via an FFN, in addition to attention.
}\label{fig:sophia}
\end{figure*}  
