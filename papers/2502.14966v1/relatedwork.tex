\section{Background and Related Work}
\subsection{Cybersecurity for AI Systems}
The integration of artificial intelligence (AI) into cybersecurity has introduced both opportunities and vulnerabilities. Historically, security mechanisms have relied on rule - based models, where predefined patterns or signatures dictate whether an event is considered malicious. While effective against known threats, these approaches struggle to adapt to dynamic attack patterns, particularly in AI - driven systems that continuously evolve \cite{tramer2017ensemble}.

Traditional cybersecurity models were designed to protect \textbf{static infrastructures}, where network behavior followed predictable patterns. However, modern AI applications operate in complex, high - dimensional environments where adversarial behavior can emerge in unexpected ways. Attackers now exploit AI models themselves, manipulating \textbf{machine learning algorithms} through adversarial inputs, poisoning datasets, or leveraging model inversion techniques to extract sensitive information \cite{sharif2016accessorize}. These evolving attack vectors highlight the urgent need for \textbf{adaptive security architectures} that can respond to threats in real time.

Existing AI security research primarily focuses on \textbf{adversarial robustness}, designing models that resist perturbations, or \textbf{explainability}, ensuring AI decisions remain interpretable \cite{goodfellow2014explaining, papernot2016limitations}. However, there is a growing recognition that cybersecurity frameworks must incorporate \textbf{self - learning and autonomous defense mechanisms} that can detect and mitigate novel threats as they arise.

\subsection{Emergent Threats in AI Systems}
Emergent threats refer to security vulnerabilities or attack patterns that arise unpredictably within AI - driven systems. Unlike traditional cyber threats, which often follow known exploit chains or malware signatures, emergent threats result from the complex interactions of AI models with their environment, data sources, or adversarial inputs. These threats often evade conventional detection mechanisms because they do not exhibit predefined attack signatures \cite{biggio2013evasion}.

One example of an emergent threat is \textbf{model drift exploitation}, where attackers manipulate AI training data over time, gradually shifting the model’s behavior until it becomes vulnerable to specific inputs. Another case involves \textbf{prompt injection attacks} against large language models, where carefully crafted inputs bypass intended constraints to elicit unauthorized or harmful responses. These attack techniques are particularly challenging to detect because they do not originate from explicit vulnerabilities in the software stack but instead exploit the inherent statistical nature of AI decision - making.

Recent high - profile AI security breaches illustrate the risks associated with emergent threats. For example, \textbf{deepfake phishing attacks} have leveraged generative AI to impersonate executives and bypass traditional authentication methods, leading to financial fraud. Similarly, \textbf{AI - powered misinformation campaigns} have demonstrated how adversaries can manipulate public discourse at scale, exploiting reinforcement learning models that prioritize engagement over veracity \cite{sharif2016accessorize}. These cases underscore the necessity for cybersecurity frameworks that can dynamically detect and respond to \textbf{unknown attack vectors} rather than relying solely on predefined rules.

\subsection{Agent - Based Cybersecurity}
Cybersecurity defense strategies can broadly be categorized into \textbf{multi - agent} and \textbf{single - agent} frameworks. Multi - agent approaches distribute security responsibilities across multiple autonomous entities, each specializing in different aspects of threat detection and mitigation. This architecture is well - suited for large - scale enterprise environments, where multiple layers of defense collaborate to provide comprehensive coverage. However, multi - agent systems introduce complexity in coordination, communication overhead and potential failure points when trust assumptions between agents break down.

In contrast, single - agent cybersecurity frameworks focus on \textbf{autonomous, self - contained security agents} capable of executing multiple defensive functions within a unified model. A single - agent approach simplifies deployment, reduces inter - agent dependency risks and ensures a \textbf{coherent decision - making process}. CyberSentinel follows this paradigm, implementing an \textbf{adaptive single - agent security model} that integrates \textbf{anomaly detection, brute - force monitoring, phishing protection and emergent threat detection} within a single cohesive system.

A key advantage of the single - agent model is its ability to dynamically adjust its detection strategies based on observed behavior. Rather than relying on \textbf{static rule sets}, CyberSentinel continuously \textbf{learns from past security events}, refining its internal models to improve accuracy over time. Additionally, the single - agent framework is more amenable to \textbf{edge deployment}, enabling real - time monitoring without reliance on centralized infrastructure.

Agent - based cybersecurity aligns closely with advancements in \textbf{autonomous AI systems}, where decision - making must occur \textbf{at the edge, in real - time and without human intervention}. As cyber threats become more \textbf{automated, adaptive and stealthy}, the need for intelligent security agents that can evolve alongside adversarial tactics becomes increasingly apparent.