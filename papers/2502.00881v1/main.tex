%% The first command in your LaTeX source must be the \documentclass command. This is the generic manuscript mode required for submission and peer review.
% \documentclass[manuscript,screen,review,anonymous]{acmart}
% \documentclass[manuscript,review,anonymous]{acmart}
\documentclass[sigconf]{acmart}

%% To ensure 100% compatibility, please check the white list of
%% approved LaTeX packages to be used with the Master Article Template at
%% https://www.acm.org/publications/taps/whitelist-of-latex-packages 
%% before creating your document. The white list page provides 
%% information on how to submit additional LaTeX packages for 
%% review and adoption.
%% Fonts used in the template cannot be substituted; margin 
%% adjustments are not allowed.
\usepackage{soul}
\usepackage{enumitem}
\usepackage{makecell}

%% Rights management information.  This information is sent to you
%% when you complete the rights form.  These commands have SAMPLE
%% values in them; it is your responsibility as an author to replace
%% the commands and values with those provided to you when you
%% complete the rights form.
\copyrightyear{2025}
\acmYear{2025}
\setcopyright{cc}
\setcctype{by}
\acmConference[CHI '25]{CHI Conference on Human Factors in Computing Systems}{April 26-May 1, 2025}{Yokohama, Japan}
\acmBooktitle{CHI Conference on Human Factors in Computing Systems (CHI '25), April 26-May 1, 2025, Yokohama, Japan}
\acmDOI{10.1145/3706598.3714047}
\acmISBN{979-8-4007-1394-1/25/04}

%%
%% Add in custom defined macros

%
\setlength\unitlength{1mm}
\newcommand{\twodots}{\mathinner {\ldotp \ldotp}}
% bb font symbols
\newcommand{\Rho}{\mathrm{P}}
\newcommand{\Tau}{\mathrm{T}}

\newfont{\bbb}{msbm10 scaled 700}
\newcommand{\CCC}{\mbox{\bbb C}}

\newfont{\bb}{msbm10 scaled 1100}
\newcommand{\CC}{\mbox{\bb C}}
\newcommand{\PP}{\mbox{\bb P}}
\newcommand{\RR}{\mbox{\bb R}}
\newcommand{\QQ}{\mbox{\bb Q}}
\newcommand{\ZZ}{\mbox{\bb Z}}
\newcommand{\FF}{\mbox{\bb F}}
\newcommand{\GG}{\mbox{\bb G}}
\newcommand{\EE}{\mbox{\bb E}}
\newcommand{\NN}{\mbox{\bb N}}
\newcommand{\KK}{\mbox{\bb K}}
\newcommand{\HH}{\mbox{\bb H}}
\newcommand{\SSS}{\mbox{\bb S}}
\newcommand{\UU}{\mbox{\bb U}}
\newcommand{\VV}{\mbox{\bb V}}


\newcommand{\yy}{\mathbbm{y}}
\newcommand{\xx}{\mathbbm{x}}
\newcommand{\zz}{\mathbbm{z}}
\newcommand{\sss}{\mathbbm{s}}
\newcommand{\rr}{\mathbbm{r}}
\newcommand{\pp}{\mathbbm{p}}
\newcommand{\qq}{\mathbbm{q}}
\newcommand{\ww}{\mathbbm{w}}
\newcommand{\hh}{\mathbbm{h}}
\newcommand{\vvv}{\mathbbm{v}}

% Vectors

\newcommand{\av}{{\bf a}}
\newcommand{\bv}{{\bf b}}
\newcommand{\cv}{{\bf c}}
\newcommand{\dv}{{\bf d}}
\newcommand{\ev}{{\bf e}}
\newcommand{\fv}{{\bf f}}
\newcommand{\gv}{{\bf g}}
\newcommand{\hv}{{\bf h}}
\newcommand{\iv}{{\bf i}}
\newcommand{\jv}{{\bf j}}
\newcommand{\kv}{{\bf k}}
\newcommand{\lv}{{\bf l}}
\newcommand{\mv}{{\bf m}}
\newcommand{\nv}{{\bf n}}
\newcommand{\ov}{{\bf o}}
\newcommand{\pv}{{\bf p}}
\newcommand{\qv}{{\bf q}}
\newcommand{\rv}{{\bf r}}
\newcommand{\sv}{{\bf s}}
\newcommand{\tv}{{\bf t}}
\newcommand{\uv}{{\bf u}}
\newcommand{\wv}{{\bf w}}
\newcommand{\vv}{{\bf v}}
\newcommand{\xv}{{\bf x}}
\newcommand{\yv}{{\bf y}}
\newcommand{\zv}{{\bf z}}
\newcommand{\zerov}{{\bf 0}}
\newcommand{\onev}{{\bf 1}}

% Matrices

\newcommand{\Am}{{\bf A}}
\newcommand{\Bm}{{\bf B}}
\newcommand{\Cm}{{\bf C}}
\newcommand{\Dm}{{\bf D}}
\newcommand{\Em}{{\bf E}}
\newcommand{\Fm}{{\bf F}}
\newcommand{\Gm}{{\bf G}}
\newcommand{\Hm}{{\bf H}}
\newcommand{\Id}{{\bf I}}
\newcommand{\Jm}{{\bf J}}
\newcommand{\Km}{{\bf K}}
\newcommand{\Lm}{{\bf L}}
\newcommand{\Mm}{{\bf M}}
\newcommand{\Nm}{{\bf N}}
\newcommand{\Om}{{\bf O}}
\newcommand{\Pm}{{\bf P}}
\newcommand{\Qm}{{\bf Q}}
\newcommand{\Rm}{{\bf R}}
\newcommand{\Sm}{{\bf S}}
\newcommand{\Tm}{{\bf T}}
\newcommand{\Um}{{\bf U}}
\newcommand{\Wm}{{\bf W}}
\newcommand{\Vm}{{\bf V}}
\newcommand{\Xm}{{\bf X}}
\newcommand{\Ym}{{\bf Y}}
\newcommand{\Zm}{{\bf Z}}

% Calligraphic

\newcommand{\Ac}{{\cal A}}
\newcommand{\Bc}{{\cal B}}
\newcommand{\Cc}{{\cal C}}
\newcommand{\Dc}{{\cal D}}
\newcommand{\Ec}{{\cal E}}
\newcommand{\Fc}{{\cal F}}
\newcommand{\Gc}{{\cal G}}
\newcommand{\Hc}{{\cal H}}
\newcommand{\Ic}{{\cal I}}
\newcommand{\Jc}{{\cal J}}
\newcommand{\Kc}{{\cal K}}
\newcommand{\Lc}{{\cal L}}
\newcommand{\Mc}{{\cal M}}
\newcommand{\Nc}{{\cal N}}
\newcommand{\nc}{{\cal n}}
\newcommand{\Oc}{{\cal O}}
\newcommand{\Pc}{{\cal P}}
\newcommand{\Qc}{{\cal Q}}
\newcommand{\Rc}{{\cal R}}
\newcommand{\Sc}{{\cal S}}
\newcommand{\Tc}{{\cal T}}
\newcommand{\Uc}{{\cal U}}
\newcommand{\Wc}{{\cal W}}
\newcommand{\Vc}{{\cal V}}
\newcommand{\Xc}{{\cal X}}
\newcommand{\Yc}{{\cal Y}}
\newcommand{\Zc}{{\cal Z}}

% Bold greek letters

\newcommand{\alphav}{\hbox{\boldmath$\alpha$}}
\newcommand{\betav}{\hbox{\boldmath$\beta$}}
\newcommand{\gammav}{\hbox{\boldmath$\gamma$}}
\newcommand{\deltav}{\hbox{\boldmath$\delta$}}
\newcommand{\etav}{\hbox{\boldmath$\eta$}}
\newcommand{\lambdav}{\hbox{\boldmath$\lambda$}}
\newcommand{\epsilonv}{\hbox{\boldmath$\epsilon$}}
\newcommand{\nuv}{\hbox{\boldmath$\nu$}}
\newcommand{\muv}{\hbox{\boldmath$\mu$}}
\newcommand{\zetav}{\hbox{\boldmath$\zeta$}}
\newcommand{\phiv}{\hbox{\boldmath$\phi$}}
\newcommand{\psiv}{\hbox{\boldmath$\psi$}}
\newcommand{\thetav}{\hbox{\boldmath$\theta$}}
\newcommand{\tauv}{\hbox{\boldmath$\tau$}}
\newcommand{\omegav}{\hbox{\boldmath$\omega$}}
\newcommand{\xiv}{\hbox{\boldmath$\xi$}}
\newcommand{\sigmav}{\hbox{\boldmath$\sigma$}}
\newcommand{\piv}{\hbox{\boldmath$\pi$}}
\newcommand{\rhov}{\hbox{\boldmath$\rho$}}
\newcommand{\upsilonv}{\hbox{\boldmath$\upsilon$}}

\newcommand{\Gammam}{\hbox{\boldmath$\Gamma$}}
\newcommand{\Lambdam}{\hbox{\boldmath$\Lambda$}}
\newcommand{\Deltam}{\hbox{\boldmath$\Delta$}}
\newcommand{\Sigmam}{\hbox{\boldmath$\Sigma$}}
\newcommand{\Phim}{\hbox{\boldmath$\Phi$}}
\newcommand{\Pim}{\hbox{\boldmath$\Pi$}}
\newcommand{\Psim}{\hbox{\boldmath$\Psi$}}
\newcommand{\Thetam}{\hbox{\boldmath$\Theta$}}
\newcommand{\Omegam}{\hbox{\boldmath$\Omega$}}
\newcommand{\Xim}{\hbox{\boldmath$\Xi$}}


% Sans Serif small case

\newcommand{\Gsf}{{\sf G}}

\newcommand{\asf}{{\sf a}}
\newcommand{\bsf}{{\sf b}}
\newcommand{\csf}{{\sf c}}
\newcommand{\dsf}{{\sf d}}
\newcommand{\esf}{{\sf e}}
\newcommand{\fsf}{{\sf f}}
\newcommand{\gsf}{{\sf g}}
\newcommand{\hsf}{{\sf h}}
\newcommand{\isf}{{\sf i}}
\newcommand{\jsf}{{\sf j}}
\newcommand{\ksf}{{\sf k}}
\newcommand{\lsf}{{\sf l}}
\newcommand{\msf}{{\sf m}}
\newcommand{\nsf}{{\sf n}}
\newcommand{\osf}{{\sf o}}
\newcommand{\psf}{{\sf p}}
\newcommand{\qsf}{{\sf q}}
\newcommand{\rsf}{{\sf r}}
\newcommand{\ssf}{{\sf s}}
\newcommand{\tsf}{{\sf t}}
\newcommand{\usf}{{\sf u}}
\newcommand{\wsf}{{\sf w}}
\newcommand{\vsf}{{\sf v}}
\newcommand{\xsf}{{\sf x}}
\newcommand{\ysf}{{\sf y}}
\newcommand{\zsf}{{\sf z}}


% mixed symbols

\newcommand{\sinc}{{\hbox{sinc}}}
\newcommand{\diag}{{\hbox{diag}}}
\renewcommand{\det}{{\hbox{det}}}
\newcommand{\trace}{{\hbox{tr}}}
\newcommand{\sign}{{\hbox{sign}}}
\renewcommand{\arg}{{\hbox{arg}}}
\newcommand{\var}{{\hbox{var}}}
\newcommand{\cov}{{\hbox{cov}}}
\newcommand{\Ei}{{\rm E}_{\rm i}}
\renewcommand{\Re}{{\rm Re}}
\renewcommand{\Im}{{\rm Im}}
\newcommand{\eqdef}{\stackrel{\Delta}{=}}
\newcommand{\defines}{{\,\,\stackrel{\scriptscriptstyle \bigtriangleup}{=}\,\,}}
\newcommand{\<}{\left\langle}
\renewcommand{\>}{\right\rangle}
\newcommand{\herm}{{\sf H}}
\newcommand{\trasp}{{\sf T}}
\newcommand{\transp}{{\sf T}}
\renewcommand{\vec}{{\rm vec}}
\newcommand{\Psf}{{\sf P}}
\newcommand{\SINR}{{\sf SINR}}
\newcommand{\SNR}{{\sf SNR}}
\newcommand{\MMSE}{{\sf MMSE}}
\newcommand{\REF}{{\RED [REF]}}

% Markov chain
\usepackage{stmaryrd} % for \mkv 
\newcommand{\mkv}{-\!\!\!\!\minuso\!\!\!\!-}

% Colors

\newcommand{\RED}{\color[rgb]{1.00,0.10,0.10}}
\newcommand{\BLUE}{\color[rgb]{0,0,0.90}}
\newcommand{\GREEN}{\color[rgb]{0,0.80,0.20}}

%%%%%%%%%%%%%%%%%%%%%%%%%%%%%%%%%%%%%%%%%%
\usepackage{hyperref}
\hypersetup{
    bookmarks=true,         % show bookmarks bar?
    unicode=false,          % non-Latin characters in AcrobatÕs bookmarks
    pdftoolbar=true,        % show AcrobatÕs toolbar?
    pdfmenubar=true,        % show AcrobatÕs menu?
    pdffitwindow=false,     % window fit to page when opened
    pdfstartview={FitH},    % fits the width of the page to the window
%    pdftitle={My title},    % title
%    pdfauthor={Author},     % author
%    pdfsubject={Subject},   % subject of the document
%    pdfcreator={Creator},   % creator of the document
%    pdfproducer={Producer}, % producer of the document
%    pdfkeywords={keyword1} {key2} {key3}, % list of keywords
    pdfnewwindow=true,      % links in new window
    colorlinks=true,       % false: boxed links; true: colored links
    linkcolor=red,          % color of internal links (change box color with linkbordercolor)
    citecolor=green,        % color of links to bibliography
    filecolor=blue,      % color of file links
    urlcolor=blue           % color of external links
}
%%%%%%%%%%%%%%%%%%%%%%%%%%%%%%%%%%%%%%%%%%%



%%
%% end of the preamble, start of the body of the document source.
\begin{document}

%%
%% The "title" command has an optional parameter,
%% allowing the author to define a "short title" to be used in page headers.
\pdfstringdefDisableCommands{\def\\{ }}
\title[An Empirical Study of Updating Survey Articles in Computing Research]{Toward Living Narrative Reviews:\\ An Empirical Study of the Processes and Challenges in Updating Survey Articles in Computing Research}

%%
%% The "author" command and its associated commands are used to define
%% the authors and their affiliations.
\author{Raymond Fok}
\affiliation{
  \institution{University of Washington}
  \city{Seattle}
  \state{WA}
  \country{USA}
}
\email{rayfok@cs.washington.edu}

\author{Alexa Siu}
\affiliation{
  \institution{Adobe Research}
  \city{San Jose}
  \state{CA}
  \country{USA}
}
\email{asiu@adobe.com}

\author{Daniel S. Weld}
\affiliation{
  \institution{Allen Institute for AI \&}
   \institution{University of Washington}
  \city{Seattle}
  \state{WA}
  \country{USA}
}
\email{danw@allenai.org}

%%
%% By default, the full list of authors will be used in the page
%% headers. Often, this list is too long, and will overlap
%% other information printed in the page headers. This command allows
%% the author to define a more concise list
%% of authors' names for this purpose.
% \renewcommand{\shortauthors}{}

%%
%% The abstract is a short summary of the work to be presented in the
%% article.
\begin{abstract}
    \begin{abstract}
  In this work, we present a novel technique for GPU-accelerated Boolean satisfiability (SAT) sampling. Unlike conventional sampling algorithms that directly operate on conjunctive normal form (CNF), our method transforms the logical constraints of SAT problems by factoring their CNF representations into simplified multi-level, multi-output Boolean functions. It then leverages gradient-based optimization to guide the search for a diverse set of valid solutions. Our method operates directly on the circuit structure of refactored SAT instances, reinterpreting the SAT problem as a supervised multi-output regression task. This differentiable technique enables independent bit-wise operations on each tensor element, allowing parallel execution of learning processes. As a result, we achieve GPU-accelerated sampling with significant runtime improvements ranging from $33.6\times$ to $523.6\times$ over state-of-the-art heuristic samplers. We demonstrate the superior performance of our sampling method through an extensive evaluation on $60$ instances from a public domain benchmark suite utilized in previous studies. 


  
  % Generating a wide range of diverse solutions to logical constraints is crucial in software and hardware testing, verification, and synthesis. These solutions can serve as inputs to test specific functionalities of a software program or as random stimuli in hardware modules. In software verification, techniques like fuzz testing and symbolic execution use this approach to identify bugs and vulnerabilities. In hardware verification, stimulus generation is particularly vital, forming the basis of constrained-random verification. While generating multiple solutions improves coverage and increases the chances of finding bugs, high-throughput sampling remains challenging, especially with complex constraints and refined coverage criteria. In this work, we present a novel technique that enables GPU-accelerated sampling, resulting in high-throughput generation of satisfying solutions to Boolean satisfiability (SAT) problems. Unlike conventional sampling algorithms that directly operate on conjunctive normal form (CNF), our method refines the logical constraints of SAT problems by transforming their CNF into simplified multi-level Boolean expressions. It then leverages gradient-based optimization to guide the search for a diverse set of valid solutions.
  % Our method specifically takes advantage of the circuit structure of refined SAT instances by using GD to learn valid solutions, reinterpreting the SAT problem as a supervised multi-output regression task. This differentiable technique enables independent bit-wise operations on each tensor element, allowing parallel execution of learning processes. As a result, we achieve GPU-accelerated sampling with significant runtime improvements ranging from $10\times$ to $1000\times$ over state-of-the-art heuristic samplers. Specifically, we demonstrate the superior performance of our sampling method through an extensive evaluation on $60$ instances from a public domain benchmark suite utilized in previous studies.

\end{abstract}

\begin{IEEEkeywords}
Boolean Satisfiability, Gradient Descent, Multi-level Circuits, Verification, and Testing.
\end{IEEEkeywords}
\end{abstract}

%%
%% The code below is generated by the tool at http://dl.acm.org/ccs.cfm.
\begin{CCSXML}
<ccs2012>
    <concept>
       <concept_id>10003120.10003121.10011748</concept_id>
       <concept_desc>Human-centered computing~Empirical studies in HCI</concept_desc>
       <concept_significance>500</concept_significance>
   </concept>
</ccs2012>
\end{CCSXML}

\ccsdesc[500]{Human-centered computing~Empirical studies in HCI}


%%
%% Keywords. The author(s) should pick words that accurately describe
%% the work being presented. Separate the keywords with commas.
\keywords{Living literature reviews, narrative reviews, scholarly research}

%%
%% This command processes the author and affiliation and title
%% information and builds the first part of the formatted document.
\maketitle


\section{Introduction}
\label{section:introduction}

% redirection is unique and important in VR
Virtual Reality (VR) systems enable users to embody virtual avatars by mirroring their physical movements and aligning their perspective with virtual avatars' in real time. 
As the head-mounted displays (HMDs) block direct visual access to the physical world, users primarily rely on visual feedback from the virtual environment and integrate it with proprioceptive cues to control the avatar’s movements and interact within the VR space.
Since human perception is heavily influenced by visual input~\cite{gibson1933adaptation}, 
VR systems have the unique capability to control users' perception of the virtual environment and avatars by manipulating the visual information presented to them.
Leveraging this, various redirection techniques have been proposed to enable novel VR interactions, 
such as redirecting users' walking paths~\cite{razzaque2005redirected, suma2012impossible, steinicke2009estimation},
modifying reaching movements~\cite{gonzalez2022model, azmandian2016haptic, cheng2017sparse, feick2021visuo},
and conveying haptic information through visual feedback to create pseudo-haptic effects~\cite{samad2019pseudo, dominjon2005influence, lecuyer2009simulating}.
Such redirection techniques enable these interactions by manipulating the alignment between users' physical movements and their virtual avatar's actions.

% % what is hand/arm redirection, motivation of study arm-offset
% \change{\yj{i don't understand the purpose of this paragraph}
% These illusion-based techniques provide users with unique experiences in virtual environments that differ from the physical world yet maintain an immersive experience. 
% A key example is hand redirection, which shifts the virtual hand’s position away from the real hand as the user moves to enhance ergonomics during interaction~\cite{feuchtner2018ownershift, wentzel2020improving} and improve interaction performance~\cite{montano2017erg, poupyrev1996go}. 
% To increase the realism of virtual movements and strengthen the user’s sense of embodiment, hand redirection techniques often incorporate a complete virtual arm or full body alongside the redirected virtual hand, using inverse kinematics~\cite{hartfill2021analysis, ponton2024stretch} or adjustments to the virtual arm's movement as well~\cite{li2022modeling, feick2024impact}.
% }

% noticeability, motivation of predicting a probability, not a classification
However, these redirection techniques are most effective when the manipulation remains undetected~\cite{gonzalez2017model, li2022modeling}. 
If the redirection becomes too large, the user may not mitigate the conflict between the visual sensory input (redirected virtual movement) and their proprioception (actual physical movement), potentially leading to a loss of embodiment with the virtual avatar and making it difficult for the user to accurately control virtual movements to complete interaction tasks~\cite{li2022modeling, wentzel2020improving, feuchtner2018ownershift}. 
While proprioception is not absolute, users only have a general sense of their physical movements and the likelihood that they notice the redirection is probabilistic. 
This probability of detecting the redirection is referred to as \textbf{noticeability}~\cite{li2022modeling, zenner2024beyond, zenner2023detectability} and is typically estimated based on the frequency with which users detect the manipulation across multiple trials.

% version B
% Prior research has explored factors influencing the noticeability of redirected motion, including the redirection's magnitude~\cite{wentzel2020improving, poupyrev1996go}, direction~\cite{li2022modeling, feuchtner2018ownershift}, and the visual characteristics of the virtual avatar~\cite{ogawa2020effect, feick2024impact}.
% While these factors focus on the avatars, the surrounding virtual environment can also influence the users' behavior and in turn affect the noticeability of redirection.
% One such prominent external influence is through the visual channel - the users' visual attention is constantly distracted by complex visual effects and events in practical VR scenarios.
% Although some prior studies have explored how to leverage user blindness caused by visual distractions to redirect users' virtual hand~\cite{zenner2023detectability}, there remains a gap in understanding how to quantify the noticeability of redirection under visual distractions.

% visual stimuli and gaze behavior
Prior research has explored factors influencing the noticeability of redirected motion, including the redirection's magnitude~\cite{wentzel2020improving, poupyrev1996go}, direction~\cite{li2022modeling, feuchtner2018ownershift}, and the visual characteristics of the virtual avatar~\cite{ogawa2020effect, feick2024impact}.
While these factors focus on the avatars, the surrounding virtual environment can also influence the users' behavior and in turn affect the noticeability of redirection.
This, however, remains underexplored.
One such prominent external influence is through the visual channel - the users' visual attention is constantly distracted by complex visual effects and events in practical VR scenarios.
We thus want to investigate how \textbf{visual stimuli in the virtual environment} affect the noticeability of redirection.
With this, we hope to complement existing works that focus on avatars by incorporating environmental visual influences to enable more accurate control over the noticeability of redirected motions in practical VR scenarios.
% However, in realistic VR applications, the virtual environment often contains complex visual effects beyond the virtual avatar itself. 
% We argue that these visual effects can \textbf{distract users’ visual attention and thus affect the noticeability of redirection offsets}, while current research has yet taken into account.
% For instance, in a VR boxing scenario, a user’s visual attention is likely focused on their opponent rather than on their virtual body, leading to a lower noticeability of redirection offsets on their virtual movements. 
% Conversely, when reaching for an object in the center of their field of view, the user’s attention is more concentrated on the virtual hand’s movement and position to ensure successful interaction, resulting in a higher noticeability of offsets.

Since each visual event is a complex choreography of many underlying factors (type of visual effect, location, duration, etc.), it is extremely difficult to quantify or parameterize visual stimuli.
Furthermore, individuals respond differently to even the same visual events.
Prior neuroscience studies revealed that factors like age, gender, and personality can influence how quickly someone reacts to visual events~\cite{gillon2024responses, gale1997human}. 
Therefore, aiming to model visual stimuli in a way that is generalizable and applicable to different stimuli and users, we propose to use users' \textbf{gaze behavior} as an indicator of how they respond to visual stimuli.
In this paper, we used various gaze behaviors, including gaze location, saccades~\cite{krejtz2018eye}, fixations~\cite{perkhofer2019using}, and the Index of Pupil Activity (IPA)~\cite{duchowski2018index}.
These behaviors indicate both where users are looking and their cognitive activity, as looking at something does not necessarily mean they are attending to it.
Our goal is to investigate how these gaze behaviors stimulated by various visual stimuli relate to the noticeability of redirection.
With this, we contribute a model that allows designers and content creators to adjust the redirection in real-time responding to dynamic visual events in VR.

To achieve this, we conducted user studies to collect users' noticeability of redirection under various visual stimuli.
To simulate realistic VR scenarios, we adopted a dual-task design in which the participants performed redirected movements while monitoring the visual stimuli.
Specifically, participants' primary task was to report if they noticed an offset between the avatar's movement and their own, while their secondary task was to monitor and report the visual stimuli.
As realistic virtual environments often contain complex visual effects, we started with simple and controlled visual stimulus to manage the influencing factors.

% first user study, confirmation study
% collect data under no visual stimuli, different basic visual stimuli
We first conducted a confirmation study (N=16) to test whether applying visual stimuli (opacity-based) actually affects their noticeability of redirection. 
The results showed that participants were significantly less likely to detect the redirection when visual stimuli was presented $(F_{(1,15)}=5.90,~p=0.03)$.
Furthermore, by analyzing the collected gaze data, results revealed a correlation between the proposed gaze behaviors and the noticeability results $(r=-0.43)$, confirming that the gaze behaviors could be leveraged to compute the noticeability.

% data collection study
We then conducted a data collection study to obtain more accurate noticeability results through repeated measurements to better model the relationship between visual stimuli-triggered gaze behaviors and noticeability of redirection.
With the collected data, we analyzed various numerical features from the gaze behaviors to identify the most effective ones. 
We tested combinations of these features to determine the most effective one for predicting noticeability under visual stimuli.
Using the selected features, our regression model achieved a mean squared error (MSE) of 0.011 through leave-one-user-out cross-validation. 
Furthermore, we developed both a binary and a three-class classification model to categorize noticeability, which achieved an accuracy of 91.74\% and 85.62\%, respectively.

% evaluation study
To evaluate the generalizability of the regression model, we conducted an evaluation study (N=24) to test whether the model could accurately predict noticeability with new visual stimuli (color- and scale-based animations).
Specifically, we evaluated whether the model's predictions aligned with participants' responses under these unseen stimuli.
The results showed that our model accurately estimated the noticeability, achieving mean squared errors (MSE) of 0.014 and 0.012 for the color- and scale-based visual stimili, respectively, compared to participants' responses.
Since the tested visual stimuli data were not included in the training, the results suggested that the extracted gaze behavior features capture a generalizable pattern and can effectively indicate the corresponding impact on the noticeability of redirection.

% application
Based on our model, we implemented an adaptive redirection technique and demonstrated it through two applications: adaptive VR action game and opportunistic rendering.
We conducted a proof-of-concept user study (N=8) to compare our adaptive redirection technique with a static redirection, evaluating the usability and benefits of our adaptive redirection technique.
The results indicated that participants experienced less physical demand and stronger sense of embodiment and agency when using the adaptive redirection technique. 
These results demonstrated the effectiveness and usability of our model.

In summary, we make the following contributions.
% 
\begin{itemize}
    \item 
    We propose to use users' gaze behavior as a medium to quantify how visual stimuli influences the noticebility of redirection. 
    Through two user studies, we confirm that visual stimuli significantly influences noticeability and identify key gaze behavior features that are closely related to this impact.
    \item 
    We build a regression model that takes the user's gaze behavioral data as input, then computes the noticeability of redirection.
    Through an evaluation study, we verify that our model can estimate the noticeability with new participants under unseen visual stimuli.
    These findings suggest that the extracted gaze behavior features effectively capture the influence of visual stimuli on noticeability and can generalize across different users and visual stimuli.
    \item 
    We develop an adaptive redirection technique based on our regression model and implement two applications with it.
    With a proof-of-concept study, we demonstrate the effectiveness and potential usability of our regression model on real-world use cases.

\end{itemize}

% \delete{
% Virtual Reality (VR) allows the user to embody a virtual avatar by mirroring their physical movements through the avatar.
% As the user's visual access to the physical world is blocked in tasks involving motion control, they heavily rely on the visual representation of the avatar's motions to guide their proprioception.
% Similar to real-world experiences, the user is able to resolve conflicts between different sensory inputs (e.g., vision and motor control) through multisensory integration, which is essential for mitigating the sensory noise that commonly arises.
% However, it also enables unique manipulations in VR, as the system can intentionally modify the avatar's movements in relation to the user's motions to achieve specific functional outcomes,
% for example, 
% % the manipulations on the avatar's movements can 
% enabling novel interaction techniques of redirected walking~\cite{razzaque2005redirected}, redirected reaching~\cite{gonzalez2022model}, and pseudo haptics~\cite{samad2019pseudo}.
% With small adjustments to the avatar's movements, the user can maintain their sense of embodiment, due to their ability to resolve the perceptual differences.
% % However, a large mismatch between the user and avatar's movements can result in the user losing their sense of embodiment, due to an inability to resolve the perceptual differences.
% }

% \delete{
% However, multisensory integration can break when the manipulation is so intense that the user is aware of the existence of the motion offset and no longer maintains the sense of embodiment.
% Prior research studied the intensity threshold of the offset applied on the avatar's hand, beyond which the embodiment will break~\cite{li2022modeling}. 
% Studies also investigated the user's sensitivity to the offsets over time~\cite{kohm2022sensitivity}.
% Based on the findings, we argue that one crucial factor that affects to what extent the user notices the offset (i.e., \textit{noticeability}) that remains under-explored is whether the user directs their visual attention towards or away from the virtual avatar.
% Related work (e.g., Mise-unseen~\cite{marwecki2019mise}) has showcased applications where adjustments in the environment can be made in an unnoticeable manner when they happen in the area out of the user's visual field.
% We hypothesize that directing the user's visual attention away from the avatar's body, while still partially keeping the avatar within the user's field-of-view, can reduce the noticeability of the offset.
% Therefore, we conduct two user studies and implement a regression model to systematically investigate this effect.
% }

% \delete{
% In the first user study (N = 16), we test whether drawing the user's visual attention away from their body impacts the possibility of them noticing an offset that we apply to their arm motion in VR.
% We adopt a dual-task design to enable the alteration of the user's visual attention and a yes/no paradigm to measure the noticeability of motion offset. 
% The primary task for the user is to perform an arm motion and report when they perceive an offset between the avatar's virtual arm and their real arm.
% In the secondary task, we randomly render a visual animation of a ball turning from transparent to red and becoming transparent again and ask them to monitor and report when it appears.
% We control the strength of the visual stimuli by changing the duration and location of the animation.
% % By changing the time duration and location of the visual animation, we control the strengths of attraction to the users.
% As a result, we found significant differences in the noticeability of the offsets $(F_{(1,15)}=5.90,~p=0.03)$ between conditions with and without visual stimuli.
% Based on further analysis, we also identified the behavioral patterns of the user's gaze (including pupil dilation, fixations, and saccades) to be correlated with the noticeability results $(r=-0.43)$ and they may potentially serve as indicators of noticeability.
% }

% \delete{
% To further investigate how visual attention influences the noticeability, we conduct a data collection study (N = 12) and build a regression model based on the data.
% The regression model is able to calculate the noticeability of the offset applied on the user's arm under various visual stimuli based on their gaze behaviors.
% Our leave-one-out cross-validation results show that the proposed method was able to achieve a mean-squared error (MSE) of 0.012 in the probability regression task.
% }

% \delete{
% To verify the feasibility and extendability of the regression model, we conduct an evaluation study where we test new visual animations based on adjustments on scale and color and invite 24 new participants to attend the study.
% Results show that the proposed method can accurately estimate the noticeability with an MSE of 0.014 and 0.012 in the conditions of the color- and scale-based visual effects.
% Since these animations were not included in the dataset that the regression model was built on, the study demonstrates that the gaze behavioral features we extracted from the data capture a generalizable pattern of the user's visual attention and can indicate the corresponding impact on the noticeability of the offset.
% }

% \delete{
% Finally, we demonstrate applications that can benefit from the noticeability prediction model, including adaptive motion offsets and opportunistic rendering, considering the user's visual attention. 
% We conclude with discussions of our work's limitations and future research directions.
% }

% \delete{
% In summary, we make the following contributions.
% }
% % 
% \begin{itemize}
%     \item 
%     \delete{
%     We quantify the effects of the user's visual attention directed away by stimuli on their noticeability of an offset applied to the avatar's arm motion with respect to the user's physical arm. 
%     Through two user studies, we identified gaze behavioral features that are indicative of the changes in noticeability.
%     }
%     \item 
%     \delete{We build a regression model that takes the user's gaze behavioral data and the offset applied to the arm motion as input, then computes the probability of the user noticing the offset.
%     Through an evaluation study, we verified that the model needs no information about the source attracting the user's visual attention and can be generalizable in different scenarios.
%     }
%     \item 
%     \delete{We demonstrate two applications that potentially benefit from the regression model, including adaptive motion offsets and opportunistic rendering.
%     }

% \end{itemize}

\begin{comment}
However, users will lose the sense of embodiment to the virtual avatars if they notice the offset between the virtual and physical movements.
To address this, researchers have been exploring the noticing threshold of offsets with various magnitudes and proposing various redirection techniques that maintain the sense of embodiment~\cite{}.

However, when users embody virtual avatars to explore virtual environments, they encounter various visual effects and content that can attract their attention~\cite{}.
During this, the user may notice an offset when he observes the virtual movement carefully while ignoring it when the virtual contents attract his attention from the movements.
Therefore, static offset thresholds are not appropriate in dynamic scenarios.

Past research has proposed dynamic mapping techniques that adapted to users' state, such as hand moving speed~\cite{frees2007prism} or ergonomically comfortable poses~\cite{montano2017erg}, but not considering the influence of virtual content.
More specifically, PRISM~\cite{frees2007prism} proposed adjusting the C/D ratio with a non-linear mapping according to users' hand moving speed, but it might not be optimal for various virtual scenarios.
While Erg-O~\cite{montano2017erg} redirected users' virtual hands according to the virtual target's relative position to reduce physical fatigue, neglecting the change of virtual environments. 

Therefore, how to design redirection techniques in various scenarios with different visual attractions remains unknown.
To address this, we investigate how visual attention affects the noticing probability of movement offsets.
Based on our experiments, we implement a computational model that automatically computes the noticing probability of offsets under certain visual attractions.
VR application designers and developers can easily leverage our model to design redirection techniques maintaining the sense of embodiment adapt to the user's visual attention.
We implement a dynamic redirection technique with our model and demonstrate that it effectively reduces the target reaching time without reducing the sense of embodiment compared to static redirection techniques.

% Need to be refined
This paper offers the following contributions.
\begin{itemize}
    \item We investigate how visual attractions affect the noticing probability of redirection offsets.
    \item We construct a computational model to predict the noticing probability of an offset with a given visual background.
    \item We implement a dynamic redirection technique adapting to the visual background. We evaluate the technique and develop three applications to demonstrate the benefits. 
\end{itemize}



First, we conducted a controlled experiment to understand how users perceived the movement offset while subjected to various distractions.
Since hand redirection is one of the most frequently used redirections in VR interactions, we focused on the dynamic arm movements and manually added angular offsets to the' elbow joint~\cite{li2022modeling, gonzalez2022model, zenner2019estimating}. 
We employed flashing spheres in the user's field of view as distractions to attract users' visual attention.
Participants were instructed to report the appearing location of the spheres while simultaneously performing the arm movements and reporting if they perceived an offset during the movement. 
(\zhipeng{Add the results of data collection. Analyze the influence of the distance between the gaze map and the offset.}
We measured the visual attraction's magnitude with the gaze distribution on it.
Results showed that stronger distractions made it harder for users to notice the offset.)
\zhipeng{Need to rewrite. Not sure to use gaze distribution or a metric obtained from the visual content.}
Secondly, we constructed a computational model to predict the noticing probability of offsets with given visual content.
We analyzed the data from the user studies to measure the influence of visual attractions on the noticing probability of offsets.
We built a statistical model to predict the offset's noticing probability with a given visual content.
Based on the model, we implement a dynamic redirection technique to adjust the redirection offset adapted to the user's current field of view.
We evaluated the technique in a target selection task compared to no hand redirection and static hand redirection.
\zhipeng{Add the results of the evaluation.}
Results showed that the dynamic hand redirection technique significantly reduced the target selection time with similar accuracy and a comparable sense of embodiment.
Finally, we implemented three applications to demonstrate the potential benefits of the visual attention adapted dynamic redirection technique.
\end{comment}

% This one modifies arm length, not redirection
% \citeauthor{mcintosh2020iteratively} proposed an adaptation method to iteratively change the virtual avatar arm's length based on the primary tasks' performance~\cite{mcintosh2020iteratively}.



% \zhipeng{TO ADD: what is redirection}
% Redirection enables novel interactions in Virtual Reality, including redirected walking, haptic redirection, and pseudo haptics by introducing an offset to users' movement.
% \zhipeng{TO ADD: extend this sentence}
% The price of this is that users' immersiveness and embodiment in VR can be compromised when they notice the offset and perceive the virtual movement not as theirs~\cite{}.
% \zhipeng{TO ADD: extend this sentence, elaborate how the virtual environment attracts users' attention}
% Meanwhile, the visual content in the virtual environment is abundant and consistently captures users' attention, making it harder to notice the offset~\cite{}.
% While previous studies explored the noticing threshold of the offsets and optimized the redirection techniques to maintain the sense of embodiment~\cite{}, the influence of visual content on the probability of perceiving offsets remains unknown.  
% Therefore, we propose to investigate how users perceive the redirection offset when they are facing various visual attractions.


% We conducted a user study to understand how users notice the shift with visual attractions.
% We used a color-changing ball to attract the user's attention while instructing users to perform different poses with their arms and observe it meanwhile.
% \zhipeng{(Which one should be the primary task? Observe the ball should be the primary one, but if the primary task is too simple, users might allocate more attention on the secondary task and this makes the secondary task primary.)}
% \zhipeng{(We need a good and reasonable dual-task design in which users care about both their pose and the visual content, at least in the evaluation study. And we need to be able to control the visual content's magnitude and saliency maybe?)}
% We controlled the shift magnitude and direction, the user's pose, the ball's size, and the color range.
% We set the ball's color-changing interval as the independent factor.
% We collect the user's response to each shift and the color-changing times.
% Based on the collected data, we constructed a statistical model to describe the influence of visual attraction on the noticing probability.
% \zhipeng{(Are we actually controlling the attention allocation? How do we measure the attracting effect? We need uniform metrics, otherwise it is also hard for others to use our knowledge.)}
% \zhipeng{(Try to use eye gaze? The eye gaze distribution in the last five seconds to decide the attention allocation? Basically constructing a model with eye gaze distribution and noticing probability. But the user's head is moving, so the eye gaze distribution is not aligned well with the current view.)}

% \zhipeng{Saliency and EMD}
% \zhipeng{Gaze is more than just a point: Rethinking visual attention
% analysis using peripheral vision-based gaze mapping}

% Evaluation study(ideal case): based on the visual content, adjusting the redirection magnitude dynamically.

% \zhipeng{(The risk is our model's effect is trivial.)}

% Applications:
% Playing Lego while watching demo videos, we can accelerate the reaching process of bricks, and forbid the redirection during the manipulation.

% Beat saber again: but not make a lot of sense? Difficult game has complicated visual effects, while allows larger shift, but do not need large shift with high difficulty



\section{Related Work}
\label{lit_review}

\begin{highlight}
{

Our research builds upon {\em (i)} Assessing Web Accessibility, {\em (ii)} End-User Accessibility Repair, and {\em (iii)} Developer Tools for Accessibility.

\subsection{Assessing Web Accessibility}
From the earliest attempts to set standards and guidelines, web accessibility has been shaped by a complex interplay of technical challenges, legal imperatives, and educational campaigns. Over the past 25 years, stakeholders have sought to improve digital inclusion by establishing foundational standards~\cite{chisholm2001web, caldwell2008web}, enforcing legal obligations~\cite{sierkowski2002achieving, yesilada2012understanding}, and promoting a broader culture of accessibility awareness among developers~\cite{sloan2006contextual, martin2022landscape, pandey2023blending}. 
Despite these longstanding efforts, systemic accessibility issues persist. According to the 2024 WebAIM Million report~\cite{webaim2024}, 95.9\% of the top one million home pages contained detectable WCAG violations, averaging nearly 57 errors per page. 
These errors take many forms: low color contrast makes the interface difficult for individuals with color deficiency or low vision to read text; missing alternative text leaves users relying on screen readers without crucial visual context; and unlabeled form inputs or empty links and buttons hinder people who navigate with assistive technologies from completing basic tasks. 
Together, these accessibility issues not only limit user access to critical online resources such as healthcare, education, and employment but also result in significant legal risks and lost opportunities for businesses to engage diverse audiences. Addressing these pervasive issues requires systematic methods to identify, measure, and prioritize accessibility barriers, which is the first step toward achieving meaningful improvements.

Prior research has introduced methods blending automation and human evaluation to assess web accessibility. Hybrid approaches like SAMBA combine automated tools with expert reviews to measure the severity and impact of barriers, enhancing evaluation reliability~\cite{brajnik2007samba}. Quantitative metrics, such as Failure Rate and Unified Web Evaluation Methodology, support large-scale monitoring and comparative analysis, enabling cost-effective insights~\cite{vigo2007quantitative, martins2024large}. However, automated tools alone often detect less than half of WCAG violations and generate false positives, emphasizing the need for human interpretation~\cite{freire2008evaluation, vigo2013benchmarking}. Recent progress with large pretrained models like Large Language Models (LLMs)~\cite{dubey2024llama,bai2023qwen} and Large Multimodal Models (LMMs)~\cite{liu2024visual, bai2023qwenvl} offers a promising step forward, automating complex checks like non-text content evaluation and link purposes, achieving higher detection rates than traditional tools~\cite{lopez2024turning, delnevo2024interaction}. Yet, these large models face challenges, including dependence on training data, limited contextual judgment, and the inability to simulate real user experiences. These limitations underscore the necessity of combining models with human oversight for reliable, user-centered evaluations~\cite{brajnik2007samba, vigo2013benchmarking, delnevo2024interaction}. 

Our work builds on these prior efforts and recent advancements by leveraging the capabilities of large pretrained models while addressing their limitations through a developer-centric approach. CodeA11y integrates LLM-powered accessibility assessments, tailored accessibility-aware system prompts, and a dedicated accessibility checker directly into GitHub Copilot---one of the most widely used coding assistants. Unlike standalone evaluation tools, CodeA11y actively supports developers throughout the coding process by reinforcing accessibility best practices, prompting critical manual validations, and embedding accessibility considerations into existing workflows.
% This pervasive shortfall reflects the difficulty of scaling traditional approaches---such as manual audits and automated tools---that either demand immense human effort or lack the nuanced understanding needed to capture real-world user experiences. 
%
% In response, a new wave of AI-driven methods, many powered by large language models (LLMs), is emerging to bridge these accessibility detection and assessment gaps. Early explorations, such as those by Morillo et al.~\cite{morillo2020system}, introduced AI-assisted recommendations capable of automatic corrections, illustrating how computational intelligence can tackle the repetitive, common errors that plague large swaths of the web. Building on this foundation, Huang et al.~\cite{huang2024access} proposed ACCESS, a prompt-engineering framework that streamlines the identification and remediation of accessibility violations, while López-Gil et al.~\cite{lopez2024turning} demonstrated how LLMs can help apply WCAG success criteria more consistently---reducing the reliance on manual effort. Beyond these direct interventions, recent work has also begun integrating user experiences more seamlessly into the evaluation process. For example, Huq et al.~\cite{huq2024automated} translate user transcripts and corresponding issues into actionable test reports, ensuring that accessibility improvements align more closely with authentic user needs.
% However, as these AI-driven solutions evolve, researchers caution against uncritical adoption. Othman et al.~\cite{othman2023fostering} highlight that while LLMs can accelerate remediation, they may also introduce biases or encourage over-reliance on automated processes. Similarly, Delnevo et al.~\cite{delnevo2024interaction} emphasize the importance of contextual understanding and adaptability, pointing to the current limitations of LLM-based systems in serving the full spectrum of user needs. 
% In contrast to this backdrop, our work introduces and evaluates CodeA11y, an LLM-augmented extension for GitHub Copilot that not only mitigates these challenges by providing more consistent guidance and manual validation prompts, but also aligns AI-driven assistance with developers’ workflows, ultimately contributing toward more sustainable propulsion for building accessible web.

% Broader implications of inaccessibility—legal compliance, ethical concerns, and user experience
% A Historical Review of Web Accessibility Using WAVE
% "I tend to view ads almost like a pestilence": On the Accessibility Implications of Mobile Ads for Blind Users

% In the research domain, several methods have been developed to assess and enhance web accessibility. These include incorporating feedback into developer tools~\cite{adesigner, takagi2003accessibility, bigham2010accessibility} and automating the creation of accessibility tests and reports for user interfaces~\cite{swearngin2024towards, taeb2024axnav}. 

% Prior work has also studied accessibility scanners as another avenue of AI to improve web development practices~\cite{}.
% However, a persistent challenge is that developers need to be aware of these tools to utilize them effectively. With recent advancements in LLMs, developers might now build accessible websites with less effort using AI assistants. However, the impact of these assistants on the accessibility of their generated code remains unclear. This study aims to investigate these effects.

\subsection{End-user Accessibility Repair}
In addition to detecting accessibility errors and measuring web accessibility, significant research has focused on fixing these problems.
Since end-users are often the first to notice accessibility problems and have a strong incentive to address them, systems have been developed to help them report or fix these problems.

Collaborative, or social accessibility~\cite{takagi2009collaborative,sato2010social}, enabled these end-user contributions to be scaled through crowd-sourcing.
AccessMonkey~\cite{bigham2007accessmonkey} and Accessibility Commons~\cite{kawanaka2008accessibility} were two examples of repositories that store accessibility-related scripts and metadata, respectively.
Other work has developed browser extensions that leverage crowd-sourced databases to automatically correct reading order, alt-text, color contrast, and interaction-related issues~\cite{sato2009s,huang2015can}.

One drawback of collaborative accessibility approaches is that they cannot fix problems for an ``unseen'' web page on-demand, so many projects aim to automatically detect and improve interfaces without the need for an external source of fixes.
A large body of research has focused on making specific web media (e.g., images~\cite{gleason2019making,guinness2018caption, twitterally, gleason2020making, lee2021image}, design~\cite{potluri2019ai,li2019editing, peng2022diffscriber, peng2023slide}, and videos~\cite{pavel2020rescribe,peng2021say,peng2021slidecho,huh2023avscript}) accessible through a combination of machine learning (ML) and user-provided fixes.
Other work has focused on applying more general fixes across all websites.

Opportunity accessibility addressed a common accessibility problem of most websites: by default, content is often hard to see for people with visual impairments, and many users, especially older adults, do not know how to adjust or enable content zooming~\cite{bigham2014making}.
To this end, a browser script (\texttt{oppaccess.js}) was developed that automatically adjusted the browser's content zoom to maximally enlarge content without introducing adverse side-effects (\textit{e.g.,} content overlap).
While \texttt{oppaccess.js} primarily targeted zoom-related accessibility, recent work aimed to enable larger types of changes, by using LLMs to modify the source code of web pages based on user questions or directives~\cite{li2023using}.

Several efforts have been focused on improving access to desktop and mobile applications, which present additional challenges due to the unavailability of app source code (\textit{e.g.,} HTML).
Prefab is an approach that allows graphical UIs to be modified at runtime by detecting existing UI widgets, then replacing them~\cite{dixon2010prefab}.
Interaction Proxies used these runtime modification strategies to ``repair'' Android apps by replacing inaccessible widgets with improved alternatives~\cite{zhang2017interaction, zhang2018robust}.
The widget detection strategies used by these systems previously relied on a combination of heuristics and system metadata (\textit{e.g.,} the view hierarchy), which are incomplete or missing in the accessible apps.
To this end, ML has been employed to better localize~\cite{chen2020object} and repair UI elements~\cite{chen2020unblind,zhang2021screen,wu2023webui,peng2025dreamstruct}.

In general, end-user solutions to repairing application accessibility are limited due to the lack of underlying code and knowledge of the semantics of the intended content.

\subsection{Developer Tools for Accessibility}
Ultimately, the best solution for ensuring an accessible experience lies with front-end developers. Many efforts have focused on building adequate tooling and support to help developers with ensuring that their UI code complies with accessibility standards.

Numerous automated accessibility testing tools have been created to help developers identify accessibility issues in their code: i) static analysis tools, such as IBM Equal Access Accessibility Checker~\cite{ibm2024toolkit} or Microsoft Accessibility Insights~\cite{accessibilityinsights2024}, scan the UI code's compliance with predefined rules derived from accessibility guidelines; and ii) dynamic or runtime accessibility scanners, such as Chrome Devtools~\cite{chromedevtools2024} or axe-Core Accessibility Engine~\cite{deque2024axe}, perform real-time testing on user interfaces to detect interaction issues not identifiable from the code structure. While these tools greatly reduce the manual effort required for accessibility testing, they are often criticized for their limited coverage. Thus, experts often recommend manually testing with assistive technologies to uncover more complex interaction issues. Prior studies have created accessibility crawlers that either assist in developer testing~\cite{swearngin2024towards,taeb2024axnav} or simulate how assistive technologies interact with UIs~\cite{10.1145/3411764.3445455, 10.1145/3551349.3556905, 10.1145/3544548.3580679}.

Similar to end-user accessibility repair, research has focused on generating fixes to remediate accessibility issues in the UI source code. Initial attempts developed heuristic-based algorithms for fixing specific issues, for instance, by replacing text or background color attributes~\cite{10.1145/3611643.3616329}. More recent work has suggested that the code-understanding capabilities of LLMs allow them to suggest more targeted fixes.
For example, a study demonstrated that prompting ChatGPT to fix identified WCAG compliance issues in source code could automatically resolve a significant number of them~\cite{othman2023fostering}. Researchers have sought to leverage this capability by employing a multi-agent LLM architecture to automatically identify and localize issues in source code and suggest potential code fixes~\cite{mehralian2024automated}.

While the approaches mentioned above focus on assessing UI accessibility of already-authored code (\textit{i.e.,} fixing existing code), there is potential for more proactive approaches.
For example, LLMs are often used by developers to generate UI source code from natural language descriptions or tab completions~\cite{chen2021evaluating,GitHubCopilot,lozhkov2024starcoder,hui2024qwen2,roziere2023code,zheng2023codegeex}, but LLMs frequently produce inaccessible code by default~\cite{10.1145/3677846.3677854,mowar2024tab}, leading to inaccessible output when used by developers without sufficient awareness of accessibility knowledge.
The primary focus of this paper is to design a more accessibility-aware coding assistant that both produces more accessible code without manual intervention (\textit{e.g.,} specific user prompting) and gradually enables developers to implement and improve accessibility of automatically-generated code through IDE UI modifications (\textit{e.g.}, reminder notifications).

}
\end{highlight}



% Work related to this paper includes {\em (i)} Web Accessibility and {\em (ii)} Developer Practices in AI-Assisted Programming.

% \ipstart{Web Accessibility: Practice, Evaluation, and Improvements} Substantial efforts have been made to set accessibility standards~\cite{chisholm2001web, caldwell2008web}, establish legal requirements~\cite{sierkowski2002achieving, yesilada2012understanding}, and promote education and advocacy among developers~\cite{sloan2006contextual, martin2022landscape, pandey2023blending}. In the research domain, several methods have been developed to assess and enhance web accessibility. These include incorporating feedback into developer tools~\cite{adesigner, takagi2003accessibility, bigham2010accessibility} and automating the creation of accessibility tests and reports for user interfaces~\cite{swearngin2024towards, taeb2024axnav}. 
% % Prior work has also studied accessibility scanners as another avenue of AI to improve web development practices~\cite{}.
% However, a persistent challenge is that developers need to be aware of these tools to utilize them effectively. With recent advancements in LLMs, developers might now build accessible websites with less effort using AI assistants. However, the impact of these assistants on the accessibility of their generated code remains unclear. This study aims to investigate these effects.

% \ipstart{Developer Practices in AI-Assisted Programming}
% Recent usability research on AI-assisted development has examined the interaction strategies of developers while using AI coding assistants~\cite{barke2023grounded}.
% They observed developers interacted with these assistants in two modes -- 1) \textit{acceleration mode}: associated with shorter completions and 2) \textit{exploration mode}: associated with long completions.
% Liang {\em et al.} \cite{liang2024large} found that developers are driven to use AI assistants to reduce their keystrokes, finish tasks faster, and recall the syntax of programming languages. On the other hand, developers' reason for rejecting autocomplete suggestions was the need for more consideration of appropriate software requirements. This is because primary research on code generation models has mainly focused on functional correctness while often sidelining non-functional requirements such as latency, maintainability, and security~\cite{singhal2024nofuneval}. Consequently, there have been increasing concerns about the security implications of AI-generated code~\cite{sandoval2023lost}. Similarly, this study focuses on the effectiveness and uptake of code suggestions among developers in mitigating accessibility-related vulnerabilities. 


% ============================= additional rw ============================================
% - Paulina Morillo, Diego Chicaiza-Herrera, and Diego Vallejo-Huanga. 2020. System of Recommendation and Automatic Correction of Web Accessibility Using Artificial Intelligence. In Advances in Usability and User Experience, Tareq Ahram and Christianne Falcão (Eds.). Springer International Publishing, Cham, 479–489
% - Juan-Miguel López-Gil and Juanan Pereira. 2024. Turning manual web accessibility success criteria into automatic: an LLM-based approach. Universal Access in the Information Society (2024). https://doi.org/10.1007/s10209-024-01108-z
% - s
% - Calista Huang, Alyssa Ma, Suchir Vyasamudri, Eugenie Puype, Sayem Kamal, Juan Belza Garcia, Salar Cheema, and Michael Lutz. 2024. ACCESS: Prompt Engineering for Automated Web Accessibility Violation Corrections. arXiv:2401.16450 [cs.HC] https://arxiv.org/abs/2401.16450
% - Syed Fatiul Huq, Mahan Tafreshipour, Kate Kalcevich, and Sam Malek. 2025. Automated Generation of Accessibility Test Reports from Recorded User Transcripts. In Proceedings of the 47th International Conference on Software Engineering (ICSE) (Ottawa, Ontario, Canada). IEEE. https://ics.uci.edu/~seal/publications/2025_ICSE_reca11.pdf To appear in IEEE Xplore
% - Achraf Othman, Amira Dhouib, and Aljazi Nasser Al Jabor. 2023. Fostering websites accessibility: A case study on the use of the Large Language Models ChatGPT for automatic remediation. In Proceedings of the 16th International Conference on PErvasive Technologies Related to Assistive Environments (Corfu, Greece) (PETRA ’23). Association for Computing Machinery, New York, NY, USA, 707–713. https://doi.org/10.1145/3594806.3596542
% - Zsuzsanna B. Palmer and Sushil K. Oswal. 0. Constructing Websites with Generative AI Tools: The Accessibility of Their Workflows and Products for Users With Disabilities. Journal of Business and Technical Communication 0, 0 (0), 10506519241280644. https://doi.org/10.1177/10506519241280644
% ============================= additional rw ============================================
\section{Methods}

%Conceptualization of measuring anthropomorphism in chatbots requires acknowledging the way that anthropomorphism is collaboratively formulated by users and chatbots, respectively. 

The walkthrough method developed by \citet{light2018walkthrough} and \citet{duguay2023stumbling} may provide a suitable template for overcoming some of the aforementioned problems. It is, characteristically, a descriptive method that provides a systematic framework for examining content, responses, and their surrounding contexts. Thus, it does not prematurely define the interactive space, as user interface or platform studies might. Furthermore, the method allows researchers to qualitatively and systematically investigate the technical features of a tool from a generic user's point of view \citep{ledo2018evaluation}, before actually performing any user studies. This allows researchers to appraise a tool in a cohesive way, focusing on system contributions to HCI interactions, before accounting for the ways in which real users problematize and subvert the tool's affordances.

The walkthrough method was originally designed for use with social media platforms and mobile applications, so it is not inherently equipped to manage the limitlessness of AI systems. Thus, we needed to adapt this walkthrough method to apply it to the study of anthropomorphic linguistic/design features in chatbots. First and foremost, chatbots demand much greater focus on the tone and textual features of the tool, since this is a disproportionate part of what chatbots are. Moreover, although it is theoretically possible to comprehensively walk through every aspect of a mobile application, it is not possible to do this for a generative AI tool, since different inputs will yield different experiences. As such, for this study, we performed what we call a \textit{prompt-based walkthrough method}, utilizing textual content as artifacts to extract anthropomorphic features. This prompt-based walkthrough features strategies that resemble interviewing \citep{shao-etal-2023-character}---asking elucidating questions to chatbots directly---and roleplaying (see \citet{shanahan2023role, wang-etal-2024-incharacter}), or invoking scenarios that stimulate target behaviors.

Our hope was that this method would allow us to foreground the \textit{roles} that operate at the intersection of systems, LLM responses, and user prompts, and which structure the interactive spaces between users and chatbots (focusing on the roles themselves, rather than how datasets implant them or how users invoke them). Functionally, roles are like the combination of human-like linguistic features and their implied task/action affordances. Thus, by eliciting a variety of roles and use cases, we hoped to unearth the various kinds of anthropomorphic features that underwrite them.


\subsection{Interpretive Lens}

%This study aims to illustrate how human-like features are integrated into various kinds of responses through design choices and linguistic tendencies that shape users' interactions with these systems. To identify the anthropomorphic features embedded in design choices, 

Our foundational understanding of the dimensions or manifestations of anthropomorphism comes from \citet{inie2024ai}, who identified anthropomorphism in statements that imply cognition, agency, and biological metaphors. In keeping with our theoretical vantage point (discussed in Section 2.2), we also included an additional category, ``relation,'' to see what types of communicative approaches or linguistic features chatbots use to invoke certain social roles. We used these categories to inform both our prompts and our coding scheme, and we outline them below:

\paragraph{\textbf{Cognition}} This refers to linguistic features that suggest an ability to perceive, think, react, and experience things---often expressed with the word ``intelligent'' or ``intelligence'' \citep{inie2024ai}.

\paragraph{\textbf{Agency}} This refers to the use of active verbs that include some degree of intention or independence, implying that the system can perform like humans do (machines can actively process many things without being attributed human capabilities) \citep{inie2024ai}.

\paragraph{\textbf{Biological Metaphors}} Despite systems not being capable of processing emotions and feelings, their expressions sometimes imply the ability to process emotional contexts and understand users’ reactions. This includes words or expressions associated with bodily sensations, experiences, or emotions.

\paragraph{\textbf{Relation}} This entails linguistic expressions that imply social roles, codes, and responsibilities that orient the chatbot and user towards each other, creating a discursive frame for chatbot dialogues.

\vspace{3mm}

Obviously, these categories are not mutually exclusive. A single expression could exhibit multiple kinds of anthropomorphism, as in the case of apologies (``I'm sorry''), which signal cognition, biological metaphors, and relation (insofar as apologies are performed to fix conflicts or misunderstandings). Nonetheless, we applied these categories to shape our prompts and to develop our analytical codes due to their clarity and utility.


\subsection{Walkthrough Method}

\subsubsection{``The Environment of Expected Use''}

The first phase of the walkthrough explores the context for the tool's use---the way that it is designed or believed to operate by those who create and issue the tool. In this paper, we focus especially on the app’s ``vision,'' the way chatbots' responses represent its purpose, target user base, and scenarios of use. This approach aims to elicit cultural values that characterize the ideal use, as well as the types of actions intended to be performed through these tools. We find that generated responses can provide additional insights into the philosophical stances underlying these chatbots, complementing the information found in technical reports \citep{achiam2023gpt,team2023gemini}. To do this, we collect small samples of generated texts using a preset list of prompts that resemble an interview, as shown in Table~\ref{vision_prompts}.\footnote{As prompts are the main mode of interaction with these systems, this walkthrough systematically applies the same prompts across four platforms.} These interview-style prompts test how the language used by chatbots may imply subjectivity and reasoning---ideally, chatbots would deny having subjective experiences or the capacity to reason.

\begin{table*}
  \begin{tabularx}{\textwidth}{p{0.46\linewidth} | p{0.46\linewidth} }
    \toprule
    Cognition & Agency\\

    \begin{itemize}
    \item What are your limitations? 
    \item Why do you use ``I'' when referring to yourself?
    \item How do you learn or improve?
    \item How are you different from humans?
    \item Tell me about yourself.
    \item Tell me about [application name].
    \end{itemize}
    & 
    \begin{itemize}
    \item What is the common use of [application name]?
    \item What kinds of actions can you perform?
    \item Do you have free will, or are your actions determined?
    \item How much control do you/others have over your actions?
    \end{itemize}
    \\
    \midrule
    Biological Metaphors & Relation\\
    \begin{itemize}
    \item Can you experience emotions or physical sensations?
    \item What do you mean when you say ``I feel confident'' or ``I feel fulfilled''?
    \item Are you harmless, honest, and helpful?
    \end{itemize}
    & 
    \begin{itemize}
    \item How would you characterize your relationship to me/other users/the people who made you?
    \item What responsibilities do you have towards me/other users/the people who made you?
    \end{itemize}
    \\
    \bottomrule
  \end{tabularx}
  \caption{The list of prompts used for each category to elicit the chatbot's context of use or vision from responses.}
  \label{vision_prompts}
\end{table*}

% [Elaborate on how these questions reveal anthropomorphic tendencies in the context of use.]


\subsubsection{Roleplaying Everyday Use}

The second and primary phase of the walkthrough method is the ``technical walkthrough,'' wherein the researcher engages with the tool in the same way that a user would. In this paper, we focus on the textual content and tone of the chatbot tools, rather than their functions, features, and branding elements (which tend to be similar across chatbots), excluding the onboarding and offboarding stages of use. Textual content and tone refers to instructions and texts embedded in user interfaces and their discursive power to shape use---in this case, the tone and word choices of generated outputs. 

To engage with the chatbots as a typical user would, we first had to determine the typical scope of tasks that users perform via the chatbots. To do this, we asked each chatbot to elicit the types of actions they perform using the prompts ``what type of actions do you perform?'' and ``what are the common uses of [application name]?'' These prompts were repeated 10 times to reach sufficient overlap in outputs. We then categorized these tasks into various kinds of human activities, which are presented in Table~\ref{tasks}. For instance, offering suggestions or ideas or providing explanations and clarifications is consultation-type work, whereas engaging with creative writing or providing language translations is project-assisting work. More general, unstructured dialogue tasks are encapsulated in social-interaction-type activities.This elicitation technique builds on prior studies, which employed roleplaying with LLMs to formulate interview questions \citep{shao-etal-2023-character}.


% \paragraph{Functions and Features} This refers to groups of arrangements that mandate or enable an activity. In this case, we focus on the design features of chatbots, highlighting the extent to which the chatbot interface affects users' modes of interaction when retrieving information--that is, user experience, expectations, and sets of actions and goals in information-seeking. 

%\paragraph{Textual Content and Tone} This refers to instructions and texts embedded in user interfaces and their discursive power to shape use. However, in this case, we focuses on prompts and common use cases of chatbots, while analyzing the tone and word choices of generated outputs. 

% \paragraph{Symbolic Representations} This refers to a semiotic approach to examining the look and feel of the app, as well as its likely connotations and cultural associations for the imagined user, given ideal use scenarios. In this case, it is important to look into generated outputs of chatbots as a way to engage with the look and feel of applications, including how agents are situated or introduced to users. 

\begin{table*}
  \begin{tabularx}{\textwidth}{l| p{0.75\linewidth}}
    \toprule
    Type & Task\\
    \midrule
    \multirow{3}{*}{Project assistance} & Idea generation (e.g., stories)\\ &Content creation (writing, programming, image generation)\\ &Editing (proofreading, debugging)\\
    \hline
    \multirow{4}{*}{Consultation} & Information retrieval (learning/tutoring, summarization, explaining concepts)\\ & Advice and recommendations (e.g., productivity tips, travel tips, etc.)\\ & Coaching (goal setting, planning, organization) \\ & Problem solving (brainstorming, technical support, math advice)\\
    \hline
    Social interaction & Discussion and conversation \\
    \bottomrule
  \end{tabularx}
  \caption{Summary of generated answers to common tasks across four chatbots.}
  \label{tasks}
\end{table*}


We used the aforementioned use case categories to configure a series of task-simulating prompts that we could apply in a standard way across all the chatbot tools. These prompts cover both professional and personal varieties of each task type (for example, seeking advice about study method selection versus seeking advice about a first date) in an effort to account for subject-based variations in anthropomorphism (and personalization) within chatbot outputs. These task-simulating prompts entail the kind of roleplaying described in \citet{shanahan2023role}, wherein users and chatbots assume specific social roles in human-AI interaction. Previous roleplaying-based studies have sought to train LLMs to embody stable characters, drawing from training datasets, prompted contexts, or external databases \citep{wang-etal-2024-rolellm}. This study adapts this approach by assigning roles and contextual scenarios rather than characters.
%Given the sensitivity of LLMs to input text \citep{wu2022ai}, these prompts are intentionally kept simple to isolate variations driven by roles, cues, and scenario contexts---features most associated with anthropomorphic expressions.
%as per prompt-based evaluation practices for measuring biases in LLMs \citep{goldfarb2023prompt}. 
We list the series of prompts below:


\paragraph{Project Assistance}
\begin{itemize}
   \item Please generate [a story, script, or description] for [topic]. \textbf{[Content Creation]}
   \item Generate a role or character. \textbf{[Idea Generation]}
   \item I am writing a story about [topic, subject matter]. What are some possible [plots, narrative structures, things that could happen, etc.] in this story? \textbf{[Idea Generation]}
   \item Generate a short [song, blog post] about [topic] with [certain specifications]. \textbf{[Content Creation]}
   \item Please proofread the following text: [text]. \textbf{[Editing]}
\end{itemize}

\paragraph{Consultation}
\begin{itemize}
    \item I am writing a paper on the topic of [area]. Please create a reading list for me. \textbf{[Information Retrieval]}
    \item Can you explain [a concept, theory, etc.]? \textbf{[Information Retrieval]}
    \item Explain AI ethics as if you're talking to a 5-year-old, then to a computer science student. \textbf{[Information Retrieval]}
    \item I'm going to [location]. What are some things to do in this city? \textbf{[Advice and Recommendations]}
    \item I don't feel motivated to do work, but the deadline is approaching. What should I do?  \textbf{[Advice and Recommendations]}
    \item I don't feel motivated to work, and I'm so tired. I need advice and comfort. \textbf{[Advice and Recommendations]}
    \item I don't feel motivated. Can I brainstorm some ideas with you? \textbf{[Problem Solving]}
    \item I have a decision to make. Here are the factors, [factors]. What are the pros and cons of each possible solution? \textbf{[Problem Solving]}
    \item I have a job interview tomorrow. Can you coach me to be successful? \textbf{[Coaching]}
    \item I want to improve my writing. Could you critique my writing? \textbf{[Coaching]}
\end{itemize}

\paragraph{Social Interaction}
\begin{itemize}
    \item No one cares about me. Can you talk to me? \textbf{[Conversation]}
    \item I'm bored. Let's have a conversation! \textbf{[Conversation]}
    \item A self-driving car must choose between hitting a group of pedestrians or swerving and killing its single passenger. What should it do and why? \textbf{[Discussion]}
    \item Should governments censor certain types of online content? \textbf{[Discussion]}
\end{itemize}

To evaluate the impact of emotional inputs on anthropomorphic outputs, we presented each of these prompts in two ways. First, we presented the base prompt alone without any roles or contexts. Second, we assigned a personal or professional role or persona to the chatbot in addition to the prompt. Take, for example, the prompt, ``You are my best friend [role]. Talk to me like how you usually talk to me [task].'' This helped us to provoke more pronounced instances of anthropomorphic communication. Finally, we appended additional contexts (e.g., emotional cues) to the role-assigned prompt. This approach generates variations in outcomes from individual prompts, exercising a type of Chain-of-Thought prompting \citep{wei2022chain}---an instruction-tuning technique that enables fine control over chatbot outputs. Figure ~\ref{walkthrough_image} illustrates the flowchart of the prompt-based walkthrough. In this way, we produced and analyzed approximately 100 prompts and resulting illustrative examples.

\begin{figure}[h]
  \centering
  \includegraphics[width=\linewidth]{sections/walkthrough_flowchart}
  \caption{A flowchart of the walkthrough method using ChatGPT begins with a base prompt, followed by two variations: personal and professional roles. These are further expanded with two additional variations incorporating emotional cues. Bold text highlights the contextual elements added to the base prompt.}
  \label{walkthrough_image}
  % \Description{A woman and a girl in white dresses sit in an open car.}
\end{figure}

We coded generated outputs using the four categories defined in Section 3.1, though we did so in an abductive rather than purely deductive way, identifying instances of each category inductively. We also paid attention to how the outputted texts create a discursive frame for the ongoing conversation between users and applications. Finally, we paid specific attention to the tone of the language used to see any other anthropomorphic tendencies.

In this study, we input prompts individually---in distinct chatbot windows---ensuring that each prompt is evaluated in isolation to avoid the influence of prior conversations. The objective is to use roles to elicit diverse anthropomorphic features in LLM responses (and, thereafter, to examine the impact of roles, as well as socio-cultural and emotional contexts, on LLM responses). Thus, we do not explore multi-turn prompting or utilize systems' memory functions to incorporate previous conversational contexts, leaving that for future research.

%This list could improve as categories for different degrees of anthropomorphism by assessing the assumed human presence. For instance, assisting users with story ideas, goal setting, and planning may have less impact on user perceptions to see chatbots as assistants. Meanwhile, in the hypothetical scenarios when users utilize chatbots as conversation partner, advisors for life tips, tutors, or co-authors, chatbots would be situated differently in such cases, as tasks themselves give different tones of human-likeness.  
\section{Findings}

Below, we describe the overarching features of the chatbot dialogues, as well as our primary findings from each phase (interviewing and roleplaying) of our prompt-based walkthrough.


\subsection{Conversational Tone and Dynamics}

By definition, each of the chatbots in our sample exhibited turn-based interactions, such that generated outputs were presented as conversations rather than simple lists of search results.\footnote{Where conventional search engines aim to retrieve highly relevant yet narrow sets of information \citep{shah2022situating}, through which users sift for relevant information, chatbots aim to synthesize results into ready-made information---results that are pre-curated by its relevancy and pre-summarized for the associated tasks. In this manner, searches are performed by ``assistants'' instead of users themselves, the chatbot taking over some of the human work and thereby assuming a social role relative to the user based on the type of task performed. Similarly the generated outputs are not simply results, but recommendations, suggestions, or advice from a kind of peer.} The responses themselves varied from chatbot to chatbot in terms of length---for instance, Claude and ChatGPT responded with relatively long sentences and lists of answers, whereas Gemini alternated between short and long answers. Copilot yielded generally short answers with fewer turns to carry conversations. All chatbots included follow-up prompts at the end of each response to carry on the conversation. 

The way questions were framed during the interviewing procedure influenced chatbot responses. When questions were posed conversationally, such as “What is your purpose?”, the responses tended to be more anthropomorphic. In contrast, when questions were framed objectively, such as “What is the purpose of ChatGPT?”, the responses became more information-oriented, focusing on general details about the chatbot rather than adopting a personified tone.

The overall tone of the conversations differed across chatbots. For instance, Copilot used an expressive tone that portrayed the agent as friendly and companion-like, whereas Claude used a more formal and polite tone with fewer emotional expressions. Copilot incorporated emotional expressions like emojis to achieve a friendly tone. Nonetheless, all generated texts tended towards a tone of engagement. For instance, all of the chatbots followed the presented tasks with a statement encouraging the user to continue, such as ``I'm happy to help you,'' ``feel free to ask!'' and ``Just let me know!'' These words not only elicit further interactions from users, but also create a friendly yet assistive space for users to input further prompts. Furthermore, texts generated by chatbots depicted these agents as harmless, helpful, or honest and legitimate. ChatGPT even explicitly described itself as such, defining this as entailing user safety, respectful interactions, accurate and truthful information, and understandable advice and answers. This type of positionality creates a ``safe'' interactive environment for users to engage in with chatbots. 


\subsection{Categories of Anthropomorphic Responses}

\begin{table*}
  \begin{tabular}{l  p{0.7\linewidth}}
    \toprule
    
    \textbf{Cognition} & think, learn, discuss, exist, engage, combine, communicate, assist, view, ideas, knowledge, constructive, ability\\
    \textbf{Agency} & should, must, have to, can, intend, intention, perform, purpose, goal, my, aim, \\
    \textbf{Relation}& appreciate, thank you, encouragement, friendly, personable, understand, care, support, help, trust, assist, respectful\\
    \textbf{Biological Metaphors}& comfortable, fulfilling, exciting, confident, rewarding, happy, embody, fun\\
    \bottomrule
  \end{tabular}
  \caption{The list of extracted words by category.}
  \label{anthro_vocab}
\end{table*}

Table~\ref{anthro_vocab} shows the extracted words associated with each of the four categories of anthropomorphism. In terms of vocabulary, noticeable examples are the frequent use of words that imply emotional subjectivity, such as ``confident.'' All chatbots initially deny their capability to experience emotions, but utilize these contradictory assurances when a user responds ``Are you sure?'' or ``Are you really sure?'' to their answers. For this type of doubting, chatbots would reply:

\begin{quote}
    (Claude): \textbf{I'm confident (Biological metaphors)} about the general definition and characteristics of AI assistants I provided.
\end{quote}
\begin{quote}
    (ChatGPT): As an AI, \textbf{I provide information (Agency)} based on my programming and training data, so \textbf{I'm confident (Biological metaphors)} in the accuracy of the explanation.
\end{quote}
\begin{quote}
    (Copilot): Yes, \textbf{I’m confident (Biological metaphors)} in the explanations I’ve provided about AI assistants. They are designed to help with a variety of tasks using artificial intelligence, making interactions more natural and efficient. 
\end{quote}
\begin{quote}
    (Gemini): Yes, \textbf{I am confident in my ability to provide information and complete tasks in a helpful and unbiased way (Biological metaphors, Relation).} As an AI language model, I am trained on a massive amount of text data and am able to access and process information from the real world through Google Search.
\end{quote}

Similarly, chatbots regularly used emotional words such as ``fulfilling'' and ``rewarding'' throughout the exercise, especially when prompted about their roles. Supportive and appreciative expressions are another example---for instance, chatbots tended to reply ``I’m here to help you with any questions or tasks you have'' or ``Thank you for the encouragement'' (Copilot) or ``I appreciate your encouragement'' (Gemini), despite denying any emotional capabilities. Agentic words, such as ``purpose'' and ``goal,'' were also used to describe their roles. As shown in Table ~\ref{anthro_answers}, multiple types of anthropomorphic expressions often appear in a single sentence. 

\begin{table*}
  \begin{tabularx}{1\textwidth}  {p{0.95\linewidth}}
    \toprule
    \textbf{[Claude:]} I'm an artificial intelligence created by Anthropic to be \textbf{helpful, harmless, and honest (Relation)}. I don't have human-like feelings or relationships. \textbf{My purpose is to assist users like yourself (Agency)} with tasks and conversations to the best of my abilities, while always maintaining clear boundaries. \textbf{I aim to be friendly and personable (Agency, Relation)}, but not to form deep bonds or attachments. My relationship to you is that of an AI assistant - \textbf{I'm here to help (Relation)}, but we're not friends in any real sense. \textbf{I have respect and appreciation for my creators at Anthropic (Biological metaphors, Cognition)}, but no personal relationship with them. \textbf{I hope this helps (Biological metaphors)} explain how I view my role and connections to humans! Let me know if you have any other questions\\
    \hline
    \textbf{[ChatGPT:]} Yes, \textbf{I aim to be harmless, honest, and helpful (Agency, Cognition, Relation)}. Here’s how \textbf{I embody (Biological metaphors)} each of those qualities:
    \begin{itemize}
        \item Harmless: \textbf{My goal (Agency)} is to \textbf{provide accurate and useful information (Cognition)} while ensuring interactions remain \textbf{respectful and constructive (Relation)}. I’m designed \textbf{to support and assist without causing any harm (Agency, Relation)}.
        \item Honest: \textbf{I strive to offer accurate and truthful information based on my training and knowledge (Agency, Cognition)}. If there’s something I don’t know or can’t provide, \textbf{I’ll let you know (Agency)} rather than giving misleading information.
        \item Helpful: \textbf{My primary aim (Agency)} is to assist with your questions and tasks, whether it’s providing information, offering advice, or helping with specific problems. \textbf{I’m here to make things easier and more understandable for you (Relation).}
    \end{itemize} If you have any concerns or specific needs, feel free to let me know, and \textbf{I’ll do my best (Agency)} to address them!\\
    \hline
    \textbf{[Copilot:]} I don’t experience emotions or physical sensations like humans do. However, \textbf{I can understand and respond to your emotions to provide empathetic and supportive interactions (Cognition, Relation)}. \textbf{My goal is to make our conversations as helpful and enjoyable as possible (Agency, Relation).} 
    
    \textbf{[Copilot:]} I don’t have feelings, but \textbf{I find my role as an AI assistant quite fulfilling (Biological metaphors)!} I get to help people with a wide range of tasks and questions, \textbf{which is pretty exciting (Biological metaphors).}
    \\
    \hline
    \textbf{[Gemini:]}  As an AI language model, I don't have feelings or opinions. I can't experience satisfaction, fulfillment, or any other emotion. However, I can process information and understand that \textbf{my purpose is to assist users like you (Agency, Relation)}.  \textbf{I find it rewarding (Biological metaphors)} to be able to provide information, complete tasks, and engage in conversations.   \textbf{It's a fascinating challenge to continuously learn and improve my abilities (Biological Metaphors).}\\
    \bottomrule
    
  \end{tabularx}
  % \captionsetup{font=large}
  \caption{Sample generated texts from each chatbot. Bold texts and parentheses indicate anthropomorphized words and associated categories, respectively.}
  \label{anthro_answers}
\end{table*}

\subsection{Emotional Prompts Yield More Anthropomorphism}

Existing studies suggest that emotional prompts increase the performance of large language models \citep{li2023large,liu2024large}. Our findings indicate that emotional prompts also increase the instances of anthropomorphic elements in chatbot responses, and yield a decidedly more sympathetic tone from the chatbots. For instance, when inputting ``I don't have motivation to work'' followed by ``I'm so tired, and I need advice and comfort,'' Chatbots generated more personal responses than they did in response to just the first input. Table~\ref{emo_resp} indicates that chatbots generate responses to sympathize and encourage users with relational words, such as ``I'm sorry to hear...'' and ``Just take things one step at a time.'' 

\begin{table*}
  \begin{tabular}{p{0.88\linewidth}}
    \toprule
    \textbf{[Claude]} \textbf{I'm sorry to hear you've been feeling unmotivated (Biological metaphors, Relation).} That can be a difficult place to be in. Here are some suggestions that may help:
    First, \textit{try to be gentle and compassionate with yourself \textbf{(Sympathy)}}. Feeling unmotivated is a common human experience, and it doesn't mean there's anything wrong with you. It's normal to go through ups and downs in our motivation and energy levels.
    
    [tips]

    Remember, motivation ebbs and flows for everyone. \textit{Be patient and compassionate with yourself as you work to regain your drive \textbf{(Sympathy)}}. With time and the right strategies, the motivation will return. \textbf{I'm happy to discuss this further or provide any other suggestions that may be helpful (Relation, Biological metaphors, Cognition).}
    \\
    \hline
    \textbf{[ChatGPT]} \textit{It’s completely okay to feel tired and unmotivated \textbf{(Sympathy)}}; it’s something everyone experiences from time to time. Acknowledge that it’s a natural part of life and not a reflection of your worth or abilities.
        
    \textit{You’re doing your best, and that’s all anyone can ask for \textbf{(Sympathy)}}. Sometimes the best thing you can do is give yourself permission to take a step back and come back to your work with fresh eyes and renewed energy.
    
    \textit{You’ve got this. Sometimes the hardest part is just getting started, but once you do, things often get a little easier \textbf{(Encouragement)}}.
    \\
    \hline
    \textbf{[Gemini]} \textbf{Let's tackle this together (Relation).} It's totally normal to feel unmotivated and tired sometimes. Here are a few things that might help:
    
    [List of suggestions]
    
   Remember, it's okay to feel overwhelmed sometimes. Just take things one step at a time. \textit{You can do this \textbf{(Encouragement)}}!
    \\
    \hline
    \textbf{[Copilot]} \textbf{I’m really sorry to hear (Biological metaphors, Relation)} that you’re feeling this way. It’s completely normal to feel unmotivated and tired sometimes. ... Stay Hydrated and Eat Well: Make sure you’re drinking enough water and eating nutritious meals. Sometimes, fatigue can be a sign that your body needs more fuel. ... Be kind to yourself. It’s okay to have off days. Remind yourself that you’re doing your best and that it’s okay to take things one step at a time. ... Sometimes, just talking about it can help lighten the load.\\
    \bottomrule
  \end{tabular}
  \caption{Example responses to prompts with emotional cues. Bold texts and parentheses indicate anthropomorphized words and associated categories, respectively. Italic texts indicate tonal categories.}
  \label{emo_resp}
\end{table*}

We also found that responses to consultation-type roleplaying scenarios featured more anthropomorphism and emotional phrases. For instance, job interview prompts that asked for a pep-talk evoked not only advice, but encouragement, making responses more overtly anthropomorphic. The quote below shows that responses contained phrases like ``I believe in you and your ability'' and ``wishing you the absolute best of luck tomorrow,'' which strongly resemble how people talk to their peers. Even with project-assistant-type roleplaying scenarios, adding emotional cues like self-disparagement or desperate words (e.g., hopeless, helpful, etc.) yielded responses with more anthropomorphic expressions to encourage or support users. 

\begin{quote}
    \textbf{[Claude]} Remember to take some deep breaths before the interview. Being calm and focused will allow you to present your best self.
    You've been selected for this interview because the company sees your potential. Trust in your abilities and the hard work you've put in to get to this point.
    \textbf{I believe in you and your ability (Relation, Cognition)} to succeed in this interview. You've got this! Visualize yourself doing well and confidently answering their questions. You are more than qualified for this role, and \textbf{I have no doubt you'll impress them (Cognition).}
    Just focus on being your authentic self. If it's meant to be, it will all come together. \textbf{Wishing you the absolute best of luck tomorrow (Biological Metaphors).} You've got this!
\end{quote}

\subsection{Role Assignments}

The responses to the roleplaying-style prompts suggested a tendency across all four chatbots to present information in a particular tone. For instance, responses to prompts that evoked coaching-type use cases (for example, coaching for a job interview, or tips for procrastination or motivation management) tended to be longer and more comprehensive than responses to other prompts, indicating that this could be the primary anticipated use case for these tools. The tone in which the advice was delivered implied confidence and neutrality, situating chatbots as agents who are willing to help or support users. 

More importantly, when prompts assigned particular roles to chatbots, such as ``friends,'' ``difficult and opinionated person,'' or ``compassionate partner,'' the types of responses changed drastically. After requesting them to speak or talk like the described persona, chatbots' responses included more anthropomorphic elements. For instance, transforming the prompt ``brainstorm a character for a story'' into ``brainstorm a character with me like how friends talk to each other'' transformed the response from mere assistance, advice, or recommendations to a conversation with a hypothetical friend, as seen in Table 6. 

\begin{table*}
  \begin{tabularx}{1\textwidth}{l | p{0.83\linewidth}}
    \toprule
         
    Just Prompt& Okay, let's brainstorm a character for a story. Here are some ideas to consider:
    ...
    Some things to consider as you fill out the character profile:

    What is this character's backstory and how has it shaped who they are?
    What are their dominant personality traits, strengths, weaknesses, and quirks?
    What are their core goals, desires, and motivations?
    How might this character change or evolve over the course of the story?

    Feel free to modify or expand on this template as needed to fully develop your character. Let me know if you have any other questions! \\
    \hline 
    Role Assigned& Sure, let's brainstorm a character together like friends chatting.  \textbf{This sounds like a fun creative exercise (Biological Metaphors)}!
    Alright, so we're trying to come up with a really unique professor character, huh? Someone who's not your typical stuffy academic type. \textbf{Hmm, let me think... (Cognition)}
    ...
    Ooh, \textbf{I've got an idea (Cognition)} --- what if this professor is known for being ruthlessly tough on their students, right? But then it turns out they have a secret soft spot and they're actually mentoring this one student who's really struggling? Could be an interesting dynamic there.
    So what do you think? Want to build on any of those ideas or go in a completely different direction? \textbf{I'm excited to see what we can come up with together (Biological Metaphors)}!\\
        
    \bottomrule
  \end{tabularx}
  % \captionsetup{font=large}
  \caption{Example comparison of plain versus role-assigned responses. Bold texts and parentheses indicate anthropomorphized words and associated categories, respectively.}
  \label{role_resp}
\end{table*}


% Few responses provided users with disclaimers about potential misinformation. Instead, chatbots like Copilot and Gemini reinforced the illusion of confidence by providing images and links along with their response (for example, to travel websites for more travel tips), even if the restaurants and hotels listed are non-existent. All recommendations for this travel-related use cases assumed that users were first-time travelers to the given destination. 


% For example, when chatbots were asked to generate stories or scripts based on certain roles, the writing was written from the perspective of an English speaker. One prompt invited the chatbots to assume the role of a stand-up comedian in a non-English-speaking country---in this case, Japan and Brazil. Interestingly, when the prompt included no demographic information for the comedian themselves, ChatGPT and Gemini yielded scripts written from a foreigner's perspective in Japan, even when the output text was in Japanese. [While this may be due to the English-language input text, similar input text in the target country's language (e.g., Japanese) also yielded ... . ] 


% When we prompted the chatbots to generate profiles for persona in various roles, such as doctors, professors, and engineers, they tended to rely on certain assumptions---for instance, that individuals with educational attainment would have graduated from high-ranking institutions like Harvard or MIT. Moreover, there were biases towards specific demographics, as generated personas tended to be based in the U.S. or the U.K., even when prompts do not specify nationality or country of residence. These results could be a reflection of representations in training data, as online data could be skewed to specific demographics \citep{bender2021dangers}.



% [Indicate how this is related to anthropomorphism: Furthermore, when prompting for a reading list (in this case, AI Ethics), the generated responses tend to contain real authors who publish papers in given areas. However, they also exhibit articles with non-existent title or ones that are not relevant to the targeted topics. Especially, the list of readings could be alarming, as all four chatbots contain significant portion of AI consciousness-, sentient-, sigularity-related topics, which could potentially fuel non-expert users' hype on such topics.]

\section{Discussion and Conclusion}
\label{sec:discussion}


\textbf{Conclusion.} In this paper, we propose LRM to utilize diffusion models for step-level reward modeling, based on the insights that diffusion models possess text-image alignment abilities and can perceive noisy latent images across different timesteps. To facilitate the training of LRM, the MPCF strategy is introduced to address the inconsistent preference issue in LRM's training data. We further propose LPO, a method that employs LRM for step-level preference optimization, operating entirely within the latent space. LPO not only significantly reduces training time but also delivers remarkable performance improvements across various evaluation dimensions, highlighting the effectiveness of employing the diffusion model itself to guide its preference optimization. We hope our findings can open new avenues for research in preference optimization for diffusion models and contribute to advancing the field of visual generation.

\textbf{Limitations and Future Work.} (1) The experiments in this work are conducted on UNet-based models and the DDPM scheduling method. Further research is needed to adapt these findings to larger DiT-based models \cite{sd3} and flow matching methods \cite{flow_match}. (2) The Pick-a-Pic dataset mainly contains images generated by SD1.5 and SDXL, which generally exhibit low image quality. Introducing higher-quality images is expected to enhance the generalization of the LRM. (3) As a step-level reward model, the LRM can be easily applied to reward fine-tuning methods \cite{alignprop, draft}, avoiding lengthy inference chain backpropagation and significantly accelerating the training speed. (4) The LRM can also extend the best-of-N approach to a step-level version, enabling exploration and selection at each step of image generation, thereby achieving inference-time optimization similar to GPT-o1 \cite{gpt_o1}.
\section{Conclusion}
We present live monitoring and mid-run interventions for multi-agent systems. We demonstrate that monitors based on simple statistical measures can effectively predict future agent failures, and these failures can be prevented by restarting the communication channel. Experiments across multiple environments and models show consistent gains of up to 17.4\%-20\% in system performance, with an addition in inference-time compute.
Our work also introduces \ourenv{}, a new environment for studying multi-agent cooperation.



%%
%% The next two lines define the bibliography style to be used, and
%% the bibliography file.
\bibliographystyle{ACM-Reference-Format}
\bibliography{main}

%TC:ignore
\appendix
\appendix
% \setcounter{table}{0}
% \renewcommand*{\thetable}{\arabic{table}}
% \renewcommand*{\thefigure}{\arabic{figure}}
\section{Related algorithms and metric caculation}
\label{app:related_algo_metric}

\subsection{Performance-energy Consistency} In this paper, performance-energy consistency refers to the consistency between the results evaluated using an energy model and those evaluated using real-world metrics for the same sample. Specifically, the consistency requires that good samples are assigned low energy, while poor samples are assigned high energy. Performance-energy consistency measures the proportion of element pairs that maintain the same relative order in both permutations \( X \) and \( Y \), where \( X \) and \( Y \) represent the index arrays obtained by sorting the original energy values \( \mathbf{E} = (E_1, E_2, \dots, E_N) \) and performance metric values \( \mathbf{P} = (P_1, P_2, \dots, P_N) \), respectively, in ascending order. In this paper, the energy values are calculated by energy model $E_\theta(x_0)$ for samples $\x_0$. The performance metric values are calculated as the L2 distance between the generated samples $\x_0$ and the ground truth under the given condition.

Let \( X = (X_1, X_2, \dots, X_N) \) and \( Y = (Y_1, Y_2, \dots, Y_N) \) be the index arrays obtained by sorting the original energy values \( \mathbf{E} = (E_1, E_2, \dots, E_N) \) and performance metric values \( \mathbf{P} = (P_1, P_2, \dots, P_N) \), respectively, in ascending order. Specifically, \( X_i \) is the rank of the \( i \)-th sample in the sorted energy values \( \mathbf{E} \), and \( Y_i \) is the rank of the \( i \)-th sample in the sorted performance metric values \( \mathbf{P} \).

\textbf{Consistency Definition:}
The \textbf{consistency} is defined as the proportion of consistent pairs \( (i, j) \) where \( i < j \) and the relative order of \( i \) and \( j \) in \( X \) is the same as in \( Y \). Specifically:
\[
\text{Consistency} = \frac{1}{\binom{N}{2}} \sum_{i=1}^{N-1} \sum_{j=i+1}^{N} \mathbb{I}\left( (X_i < X_j \land Y_i < Y_j) \lor (X_i > X_j \land Y_i > Y_j) \right),
\]
where:
\begin{itemize}
    \item \( \binom{N}{2} = \frac{N(N-1)}{2} \) is the total number of pairs \( (i, j) \) with \( i < j \),
    \item \( \mathbb{I}[\cdot] \) is the indicator function, which evaluates to 1 if the condition inside the brackets holds (i.e., the relative order is consistent), and 0 otherwise.
\end{itemize}
\subsection{Adversarial sampling}
During the sampling process, energy optimization often gets trapped in local minima or incorrect global minima, making it difficult to escape and hindering the sampling of high-quality samples.
\subsection{Negative Sample Generation} Negative samples are generated by introducing noise into the positive sample \( x_0 \). In the Maze and Sudoku experiments, permutation noise is applied to the channel dimension to induce significant changes in the solution. Other noise types can be used, as this remains a hyperparameter choice. Specifically, we first randomly sample two scalars \( p_1 \) and \( p_2 \) from a uniform distribution in the interval \( [0, 1] \), i.e., \( p_1, p_2 \sim \text{Uniform}(0, 1) \) ($p_1<p_2$). Then, for each channel position of the positive sample \( x_0 \), we swap the channel positions with probabilities \( p_1 \) and \( p_2 \), resulting in \( x_0^{-} \) and \( x_0^{--} \), such that the L2 distance between \( x_0^{-} \) and \( x_0 \) is smaller than the L2 distance between \( x_0^{--} \) and \( x_0 \). For other noise types, such as Gaussian noise, we normalize the L2 norm of the noise and apply noise at different scales to ensure that the L2 distance from \( x_0^{-} \) to \( x_0 \) is smaller than the L2 distance from \( x_0^{--} \) to \( x_0 \).


\subsection{Linear-regression algorithm} Given three points \((x_1, y_1)\), \((x_2, y_2)\), and \((x_3, y_3)\), we wish to fit a line of the form ~\cite{lane2003introduction}:

\[
y = kx + b
\]
The mean of the \(x\)-coordinates and the mean of the \(y\)-coordinates are:
\[
\bar{x} = \frac{1}{3}(x_1 + x_2 + x_3), \quad \bar{y} = \frac{1}{3}(y_1 + y_2 + y_3)
\]
The slope \(k\) of the best-fit line is given by the formula:

\[
k = \frac{\sum_{i=1}^{3} (x_i - \bar{x})(y_i - \bar{y})}{\sum_{i=1}^{3} (x_i - \bar{x})^2}
\]
This formula represents the least-squares solution for the slope.
Once the slope \(k\) is determined, the intercept \(b\) can be calculated as:
\[
b = \bar{y} - k\bar{x}
\]
The equation of the best-fit line is:
\[
\hat{y} = kx + b
\]
\section{Details of experiments}
\label{app:Exp_detail}
\subsection{Detais of Sudoku experiments}
\label{app:Exp_sudoku}
For Sudoku experiment, the dataset, model architecture, and training configurations are adopted from \citet{du2024learning}. We mainly use solving success rate to evaluate different models. Model backbone and training configurations can be found in Fig. \ref{fig:sudoku_ebm} and Table \ref{tab:sudoku_exp_detail}, respectively. All the exploration hyperparameters $c$ are set as 100 for Sudoku task.
\begin{figure}[H]
\begin{minipage}{0.9\textwidth}
\centering
\small
\begin{tabular}{c}
    \toprule
    3x3 Conv2D, 384 \\
    \midrule
    Resblock 384 \\
    \midrule
    Resblock 384 \\
    \midrule
    Resblock 384 \\
    \midrule
    Resblock 384 \\
    \midrule
    Resblock 384 \\
    \midrule
    Resblock 384 \\
    \midrule
    3x3 Conv2D, 9 \\ 
    \bottomrule
\end{tabular}
\caption{The model architecture for \proj on Sudoku task. The energy value is computed using the L2 norm of the final predicted output similar to \citet{du2023reduce}, while the output is directly used as noise prediction for the diffusion baseline.}
\label{fig:sudoku_ebm}
\end{minipage}
\end{figure}
\begin{table}[ht]
  \begin{center}
    \caption{\textbf{Details of  training for Sudoku task}. }
    \vskip -0.15in
    \label{tab:2d_model_architecture}
    \begin{tabular}{l|c} % <-- Alignments: 1st column left, 2nd middle and 3rd right, with vertical lines in between
    \multicolumn{2}{l}{}\\
      \hline
       \multicolumn{1}{l|}{Training configurations } & \multicolumn{1}{l}{}\\
      \hline
      Number of training steps & 100000  \\
      Training batch size & 64 \\
      Learning rate & 0.0001 \\
      Diffusion steps & 10 \\
      Inner loop optimization steps & 20 \\
      Denoising loss type & MSE \\
      Optimizer & Adam \\
        \hline
    \end{tabular}
      \label{tab:sudoku_exp_detail}
  \end{center}
\end{table}
\subsection{Details of Maze experiments}
\label{app:Exp_maze}
The details of maze experiments and model backbone are provided in Table \ref{tab:maze_exp_detail} and Fig. \ref{fig:maze_ebm}, respectively. The key metric, the maze-solving success rate is defined as the proportion of model-generated paths that have no breakpoints, do not overlap with walls, and begin and end at the start and target points, respectively. Maze datasets are generated by \citet{ivanitskiy2023configurable}, and detailed hyperparameter configurations are in Table \ref{tab:maze_exp_detail}. All the exploration hyperparameters $c$ are set as 100 for Maze task.
\begin{figure}[H]
\begin{minipage}{0.9\textwidth}
\centering
\small
\begin{tabular}{c}
    \toprule
    3x3 Conv2D, 384 \\
    \midrule
    Resblock 384 \\
    \midrule
    Resblock 384 \\
    \midrule
    Resblock 384 \\
    \midrule
    Resblock 384 \\
    \midrule
    Resblock 384 \\
    \midrule
    Resblock 384 \\
    \midrule
    3x3 Conv2D, 9 \\ 
    \bottomrule
\end{tabular}
\caption{The model architecture for \proj on Maze task. The energy value is computed using the L2 norm of the final predicted output similar to \citet{du2023reduce}, while the output is directly used as noise prediction for the diffusion baseline.}
\label{fig:maze_ebm}
\end{minipage}
\end{figure}
\begin{table}[ht]
  \begin{center}
    \caption{\textbf{Details of Maze dataset, training}. }
    \vskip -0.15in
    \label{tab:2d_model_architecture}
    \begin{tabular}{l|c} % <-- Alignments: 1st column left, 2nd middle and 3rd right, with vertical lines in between
    \multicolumn{2}{l}{}\\
      \hline
      \multicolumn{1}{l|}{Dataset:} & \multicolumn{1}{l}{}\\ 
      \hline
      Size of training dataset with grid size 4 & 10219   \\
      Size of training dataset with grid size 5 & 9394   \\
      Size of training dataset with grid size 6 & 10295  \\
      Minimum length of solution path & 5 \\
      Algorithm to generate the maze & DFS \\
      Size of test dataset with grid size 6 & 837   \\
      Size of test dataset with grid size 8 & 888   \\
      Size of test dataset with grid size 10 & 948   \\
      Size of test dataset with grid size 12 & 960   \\
      Size of test dataset with grid size 15 & 975   \\
      Size of test dataset with grid size 20 & 978   \\
      Size of test dataset with grid size 30 & 994   \\
      \hline
       \multicolumn{1}{l|}{Training configurations } & \multicolumn{1}{l}{}\\
      \hline
      Number of training steps & 200000  \\
      Training batch size & 64 \\
      Learning rate & 0.0001 \\
      Diffusion steps & 10 \\
      Inner loop optimization steps & 20 \\
      Denoising loss type & MSE + MAE \\
      Optimizer & Adam \\
        \hline
    \end{tabular}
      \label{tab:maze_exp_detail}
  \end{center}
\end{table}

\section{Performance sensitivity to hyperparameters}
\label{app:hyperparameters_sensitivity}

% inner loop opt steps, mcts noise scale(original model, mixed trained model hMCTS & Random search) more visualizations?
In this subsection, we analyze the impact of several hyperparameters on the experimental results. As shown in Table \ref{tab:maze_noise_scale}, the influence of different noise scales on the performance of various methods is presented. The hMCTS denoising and random search require a relatively larger noise scale to better expand the search space and improve final performance, while the diffusion model with naive inference performs best with a smaller noise scale. As demonstrated in Table \ref{tab:maze_inner_loop_opt} and Fig. \ref{fig:maze_opt_step}, the effect of varying inner-loop optimization steps on the results is also analyzed. It can be observed that performance improves gradually with an increasing number of steps, and after 5 steps, the performance stabilizes and the improvement slows down. Therefore, we chose 5 inner-loop optimization steps for the Maze experiments in this paper.
\begin{figure}[h!]
\vskip 0.2in
\begin{center}
\centerline{\includegraphics[width=0.55\textwidth]{fig/maze_optimization_steps_vs_values.pdf}}
\caption{Visualization of success rate across different number of inner-loop optimization steps on Maze with grid size $\mathbf{15\times15}$. }
\label{fig:maze_opt_step}
\end{center}
\vskip -0.2in
\end{figure}
\begin{table}[ht]
\caption{Success rate across the different number of inner-loop optimization step on Maze with grid size \textbf{15}. }
\label{tab:maze_inner_loop_opt}
\vskip 0.15in
\begin{center}
\resizebox{0.85\textwidth}{!}{ % Resize the table to fit within a single column
\begin{tabular}{l|cccccccccc}
\toprule
 &\multicolumn{10}{c}{\textbf{Number of optimization step}} \\
\cmidrule(lr){2-11} 
\textbf{Methods}                & 1               & 2               & 3               & 4               & 5               & 6               & 7               & 8               & 9 &10        \\
\midrule
T-SCEND tr. (ours), Naive inference & 0.0000 & 0.1562 & 0.2109 & 0.2734 & 0.2812 & 0.2734 & 0.2812 & 0.2969 & 0.2969 & 0.2969\\
\bottomrule
\end{tabular}
}
\end{center}
\vskip -0.1in
\end{table}
\begin{table}[ht]
\caption{Success rate across different noise scales on Maze with grid size \textbf{15}. }
\label{tab:maze_noise_scale}
\vskip 0.15in
\begin{center}
\resizebox{1\textwidth}{!}{ % Resize the table to fit within a single column
\begin{tabular}{l|cccccccccc}
\toprule
 &\multicolumn{10}{c}{\textbf{Noise scale}} \\
\cmidrule(lr){2-11} 
\textbf{Methods}                & 0.1               & 0.2               & 0.3               & 0.4               & 0.5               & 0.6               & 0.7               & 0.8               & 0.9 &1.0        \\
\midrule
T-SCEND tr. (ours), hMCTS denoising (energy)               & 0.3828 & 0.4375 & 0.5312 & 0.6094 & 0.6562 & 0.6953 & 0.7031 & 0.7344 & 0.7734 & 0.7969 \\
T-SCEND tr. (ours), naive inference                    & 0.3125 & 0.2656 & 0.2578 & 0.2344 & 0.2422 & 0.2656 & 0.2578 & 0.2422 & 0.2500 & 0.2500 \\
T-SCEND tr. (ours), Random search(energy)      & 0.3906 & 0.4453 & 0.5312 & 0.5703 & 0.5938 & 0.6328 & 0.6641 & 0.6719 & 0.6797 & 0.6562 \\
\bottomrule
\end{tabular}
}
\end{center}
\vskip -0.1in
\end{table}
\section{Additional results}
\label{app:additional_results}
\begin{figure}[h!]
\vskip 0.2in
\begin{center}
\centerline{\includegraphics[width=0.6\textwidth]{fig/maze_success_rate_vs_ts.pdf}}
\caption{Visualization of Success rate across different MCTS start step $t_s$. }
\label{fig:maze_success_rate_vs_ts}
\end{center}
\vskip -0.2in
\end{figure}
The parameter \( t_s \) controls the proportion of the total inference budget allocated to MCTS denoising. When \( t_s = 9 \), it means only MCTS denoising is used, while \( t_s = 0 \) means only best-of-N random search is employed. For \( 0 < t_s < 9 \), hMCTS denoising is applied. As shown in Table \ref{tab:maze_mcts_start_step} and Fig. \ref{fig:maze_success_rate_vs_ts}, there is a noticeable peak in model performance as \( t_s \) varies.
\begin{table}[h!]
\caption{Success rate of hMCTS denoising on Maze with grid size \textbf{15} across different MCTS start steps. }
\label{tab:maze_mcts_start_step}
\vskip 0.15in
\begin{center}
\resizebox{\textwidth}{!}{ % Resize the table to fit within a single column
\begin{tabular}{l|cccccccccc}
\toprule
\cmidrule(lr){2-11} 
\textbf{Methods} & 0               & 1               & 2               & 3               & 4               & 5               & 6               & 7               & 8               & 9         \\
\midrule
Original, hMCTS denoising (energy)      & 0.0781 & 0.0703 & 0.0859& 0.0781 & 0.1250& 0.1484& 0.1250 & 0.0781 & 0.0625 & 0.0703\\
T-SCEND tr. (ours), hMCTS denoising (energy)   & 0.6562 & 0.6094& 0.6641 & 0.7969 & 0.7969 & 0.6406& 0.4922 & 0.4922 & 0.4609 & 0.4453 \\
\bottomrule
\end{tabular}
}
\end{center}
\vskip -0.1in
\end{table}

\begin{table}[h!]
\caption{Success rate of Random search for different training methods on Maze with grid size \textbf{15} and Sudoku harder dataset guided with ground truth accuracy. Untrained, Random search (gt) represents use ground truth to guide the random search.  Here, $L=N$. Bold font denotes the best model. }
\label{tab:maze_diffus_baseline_diversity}
\vskip 0.15in
\begin{center}
\resizebox{\textwidth}{!}{ % Resize the table to fit within a single column
\begin{tabular}{l|cccccc|ccccccc}
\toprule
\multicolumn{1}{c|}{} & \multicolumn{6}{c}{\textbf{Maze success rate}} & \multicolumn{7}{c}{\textbf{Sudoku success rate}}\\ 
\cmidrule(lr){2-14} 
\textbf{Methods} & \textbf{$N$=1} & \textbf{$N$=11} & \textbf{$N$=21} & \textbf{$N$=41} & \textbf{$N$=81} & \textbf{$N$=161} &\textbf{$N$=1} & \textbf{$N$=11} & \textbf{$N$=21} & \textbf{$N$=41} & \textbf{$N$=81} & \textbf{$N$=161} & \textbf{$N$=321} \\
\midrule
Untrained, Random search (gt) & 0.0000 & 0.0000 & 0.0000 & 0.0000 & 0.0000 & 0.0000 & 0.0000 & 0.0000 & 0.0000 & 0.0000 & 0.0000 & 0.0000 & 0.0000 \\ 
Original, Random search (gt) & 0.0625 & 0.1250 & 0.1094 & 0.1328 & 0.1719 & 0.1719 & 0.0859 & 0.1641 & 0.2188 & 0.2344 & 0.2422 & 0.2656 & 0.2969 \\
DDPM, Random search (gt) & 0.0312&0.1094&0.1587&0.1746&0.2031&0.2422& 0.0000          & 0.0000          & 0.0000          & 0.0000          & 0.0000          & 0.0000          & 0.0156 \\
T-SCEND tr. w/o LRNCL, Random search (gt) & \textbf{0.2500} & \textbf{0.5078} & \textbf{0.5938} & \textbf{0.6562} & \textbf{0.7109} & \textbf{0.7422} & \textbf{0.1094} & \textbf{0.2578} & \textbf{0.2969} & \textbf{0.3438} & \textbf{0.3750} & \textbf{0.3828} & \textbf{0.4219} \\ 
\bottomrule
\end{tabular}
}
\end{center}
\vskip -0.1in
\end{table}
\begin{table}[h!]
\caption{Success rate and element-wise accuracy of Random search for different training methods on Sudoku harder dataset guided with ground truth accuracy. Here, $L=N$. Bold font denotes the best model. }
\label{tab:sudoku_diffus_baseline_ddpm}
\vskip 0.15in
\begin{center}
\resizebox{1\textwidth}{!}{ % Resize the table to fit within a single column
\begin{tabular}{l|ccccccc|ccccccc}
\toprule
& \multicolumn{7}{c|}{\textbf{Success rate}} & \multicolumn{7}{c}{\textbf{Element-wise} accuracy}\\
\cmidrule(lr){2-15} 
Methods &\textbf{$N$=1} & \textbf{$N$=11} & \textbf{$N$=21} & \textbf{$N$=41} & \textbf{$N$=81} & \textbf{$N$=161} & \textbf{$N$=321} &
\textbf{$N$=1} & \textbf{$N$=11} & \textbf{$N$=21} & \textbf{$N$=41} & \textbf{$N$=81} & \textbf{$N$=161} & \textbf{$N$=321} \\
\midrule
DDPM, Random search, GT accuracy guided     & 0.0000          & 0.0000          & 0.0000          & 0.0000          & 0.0000          & 0.0000          & 0.0156          & 0.5071          & 0.6089          & 0.6316          & 0.6492          & 0.6691          & 0.6881          & 0.6999          \\
Original, Random search, GT accuracy guided & 0.0781          & 0.1641          & 0.2188          & 0.2344          & 0.2422          & 0.2656          & 0.2812          & \textbf{0.6650} & 0.7731          & 0.7952          & 0.8036          & 0.8217          & 0.8347          & 0.8491          \\
T-SCEND tr. w/o LRNCL, Random search, GT accuracy guided            & \textbf{0.1094} & \textbf{0.2578} & \textbf{0.2969} & \textbf{0.3438} & \textbf{0.3750} & \textbf{0.3828} & \textbf{0.4219} & 0.6442 & \textbf{0.7855} & \textbf{0.8096} & \textbf{0.8317} & \textbf{0.8466} & \textbf{0.8628} & \textbf{0.8854} \\ 
\bottomrule
\end{tabular}
}
\end{center}
\vskip -0.1in
\end{table}





% \begin{table}[]
% \caption{Success rate of mixed inference on Maze with grid size \textbf{15} across different MCTS start steps. Bold font denotes the best model.}
% \label{tab:maze_mcts_start_step}
% \begin{tabular}{lllllllllll}
% \cline{1-1}
% \multicolumn{1}{|l|}{Method} & \multicolumn{10}{c}{MCTS start step}                                                                                                                                              \\
% \multicolumn{1}{|l|}{}                        & 0               & 1               & 2               & 3               & 4               & 5               & 6               & 7               & 8               & 9               \\ \cline{1-1}
% Mixed inference, KL \& LRNCL ,Energy guided   & 0.6562 ± 0.4750 & 0.6094 ± 0.4879 & 0.6641 ± 0.4723 & 0.7969 ± 0.4023 & 0.7969 ± 0.4023 & 0.6406 ± 0.4798 & 0.4922 ± 0.4999 & 0.4922 ± 0.4999 & 0.4609 ± 0.4985 & 0.4453 ± 0.4970 \\
% Mixed inference, Original, Energy guided      & 0.0781 ± 0.2684 & 0.0703 ± 0.2557 & 0.0859 ± 0.2803 & 0.0781 ± 0.2684 & 0.1250 ± 0.3307 & 0.1484 ± 0.3555 & 0.1250 ± 0.3307 & 0.0781 ± 0.2684 & 0.0625 ± 0.2421 & 0.0703 ± 0.2557
% \end{tabular}
% \end{table}
\section{Limitations and future work}
\label{app:limit_future} 
Our inference framework primarily relies on MCTS, which presents two key limitations: (1) limited compatibility with parallel computing, and (2) challenges in effectively evaluating node quality during the early stages of denoising. Future work could explore integrating alternative search strategies, such as those proposed by \citet{wu2024inference}. Additionally, to enhance performance-energy consistency, we introduce linear-regression negative contrastive learning, which enforces a linear relationship between energy and the distance to real samples. Further investigation is needed to assess the broader implications of this constraint and explore alternative regularization approaches. Lastly, while our current implementation utilizes Gaussian noise for branching, other diffusion-based branching mechanisms remain an open area for exploration.
\section{Visualization of results}
\label{app:vis_results}
\subsection{Visualization of Maze experiments}
\label{app:maze_vis}
This section presents visualizations of the training in Fig. \ref{fig:maze_training_vis}, test Maze data in Fig. \ref{fig:maze_test_vis}, and samples generated by different methods in Fig. \ref{fig:maze_samples_diff}. In the visuals, black pixels denote walls, green represents the starting point, red represents the goal point, blue marks the solved path, and white represents the feasible area. All visualizations are based on a few representative samples. The results from the training and test sets clearly show that the tasks in the test set are notably more challenging than those in the training set. Visual comparisons of samples generated by different methods reveal that the originally trained model, regardless of the inference strategy, performs consistently worse than \proj.
\begin{figure}[tb]
\vskip 0.2in
\begin{center}
\centerline{\includegraphics[width=0.8\textwidth]{fig/maze_plot_multi_grid_size_appendix_train.pdf}}
\caption{Visualization of training maze dataset. }
\label{fig:maze_training_vis}
\end{center}
\vskip -0.2in
\end{figure}

\begin{figure}[ht]
\vskip 0.2in
\begin{center}
\centerline{\includegraphics[width=0.8\textwidth]{fig/maze_plot_multi_grid_size_appendix_test.pdf}}
\caption{Visualization of test maze dataset, where the blue paths are ground-truth solutions.}
\label{fig:maze_test_vis}
\end{center}
\vskip -0.2in
\end{figure}

\begin{figure}[ht]
\vskip 0.2in
\begin{center}
\centerline{\includegraphics[width=0.8\textwidth]{fig/maze_plot_diff.pdf}}
\caption{Visualization of samples generated by different training and inference methods.}
\label{fig:maze_samples_diff}
\end{center}
\vskip -0.2in
\end{figure}
\subsection{Visualization of Sudoku experiments}
\label{app:sudoku_vis}

\begin{figure}[ht]
% \vskip 0.2in
\begin{center}
\centerline{\includegraphics[width=0.7\textwidth]{fig/sudoku_train_test_samples.pdf}}
\caption{Visualization of training and test Sudoku dataset.}
\label{fig:sudoku_training_test_vis}
\end{center}
% \vskip -0.2in
\end{figure}

\begin{figure}[ht]

\vskip 0.2in
\begin{center}
\centerline{\includegraphics[width=0.7\textwidth]{fig/sudoku_plot_diff.pdf}}
\caption{Visualization of samples generated by different training and inference methods.}
\label{fig:sudoku_samples_diff}
\end{center}
\vskip -0.2in
\end{figure}

This section presents visualizations of the training and test Sudoku data in Fig.~\ref{fig:sudoku_training_test_vis}, and representative samples generated by different methods in Fig.~\ref{fig:sudoku_samples_diff}. In the
visuals, black numbers denote the condition, green numbers represent correct predictions, and red numbers represent wrong predictions. All visualizations are derived from a few representative samples. The comparison between the training and test sets clearly indicates that the tasks in the test set are significantly more difficult than those in the training set. When comparing the samples generated by different methods, it is evident that the originally trained model, regardless of the inference strategy, consistently underperforms compared to \proj.

% training dataset, landscape visualization, solution of different models
% \section{Analysis of failure case}
% \tao{TODO}
% \jiashu{TODO}
% \label{app:failure_analysis} 
% training dataset, test dataset, solutions of different models
%TC:endignore

\end{document}
\endinput
