\section{Background and Related Work}
There exist different types of literature reviews, varying across their goals, processes, and degrees of systematicity, with one typology identifying 14 distinct review types (e.g., critical, scoping, systematic, qualitative synthesis, and umbrella reviews)~\cite{grant2009typology}. More broadly, reviews can be categorized into one of three major approaches: systematic, semi-systematic, or integrative~\cite{snyder2019literature}. Systematic literature reviews, developed for and commonly utilized in clinical medicine, use ``explicit, systematic methods to identify, appraise, and synthesize all the empirical evidence that meets pre-specified eligibility criteria to answer a specific research question.''\footnote{\url{https://www.cochranelibrary.com/about/about-cochrane-reviews}} As the gold standard of evidence synthesis for clinical decision-making, systematic review methods have been studied extensively (e.g.,~\cite{chandler2013cochrane, pati2018write, lasserson2019starting}). On the other hand, semi-systematic or \textit{narrative} reviews are preferable when reviewing every relevant article is not possible. These reviews may rely on quasi-systematic search and appraisal methods, combine both qualitative and qualitative evidence, and hold broader aims beyond a specific research question, such as to survey a research area or track its progression over time. Closely related to narrative reviews, integrative or critical reviews further critique the literature to develop new theoretical perspectives, often integrating heterogeneous sources beyond peer-reviewed research articles to provide a synthesis that goes beyond descriptive or historical.

Creating these different reviews to synthesize the literature is often time-consuming and costly~\cite{tricco2008following, michelson2019significant}. Researchers have therefore explored the potential for AI to support literature review creation, among other research activities~\cite{van2023ai, morris_scientists_2023, messeri2024artificial}. For instance, prior work has developed approaches to automatically scaffold (e.g.,~\cite{hsu-etal-2024-chime, newman-etal-2024-arxivdigestables}) and generate review sections (e.g.,~\cite{li-ouyang-2024-related, bolanos2024artificial, wang2024autosurvey}), and proposed interactive methods to support relevant processes such as literature discovery (e.g.,~\cite{Kinney2023TheSS, xiao2023AutoSurveyGPT}) and appraisal (e.g.,~\cite{kusa2023cruise, chai2021research}), comprehension (e.g.,~\cite{lo2023semanticreaderprojectaugmenting, head2021scholarphi}), and synthesis (e.g.,~\cite{kang_synergi_2023, kang_threddy_2022, wang2024scidasynth}).

However, the accuracy and utility of reviews can decay over time~\cite{Shekelle2001ValidityOT, elliott2017living}, leading to interest in the potential of ``living reviews'' that are continually updated, incorporating relevant evidence as it becomes available~\cite{elliott2014living}. Organizations such as Cochrane\footnote{\url{https://www.cochrane.org/}} offer guidance toward living systematic reviews (LSRs)~\cite{elliott2017living}, contributing a series of works that provide context and justification for LSRs~\cite{elliott2017living}, reason about the role of technology in supporting the creation of LSRs~\cite{thomas2017living}, develop methods for updating the statistical meta-analyses found in LSRs~\cite{simmonds2017living}, and outline the potential for living guidelines that offer dynamic recommendations based on LSRs~\cite{akl2017living}.

Despite this guidance on the conduct, reporting, and publication of LSRs~\cite{cochrane2019handbook, elliott2017living}, studies suggest practical challenges in sustaining these workflows have limited the success of such living reviews~\cite{tricco2008following, heron2023update}. One recent analysis of LSRs on COVID-19 found that most LSRs were never updated after their initial publication, underscoring the overall difficulty of keeping such reviews alive if reliant on largely manual updating efforts~\cite{heron2023update}. Similarly, in climate science, US federal law mandates a process of updating a summary of relevant scientific knowledge every four years. This process is laborious and costly, with the most recent update involving over 500 authors.\footnote{\url{https://nca2023.globalchange.gov/}} While AI tools are being considered as aids for this updating process, they have yet to be adopted~\cite{AlKhourdajie2024TheRO}. Overall, existing guidance on LSRs appears to insufficiently recognize the potential for technological support in what remains a predominantly manual review updating process, and the aging considerations ought to be reexamined in light of recent AI advancements.

Moreover, while existing work has focused on living \textit{systematic} reviews, supporting the process of updating semi-systematic or narrative reviews has received little attention.
Systematic reviews which adhere to explicit search protocols and often integrate findings through meta-analysis require distinct considerations for incorporating new evidence. In contrast, narrative reviews adopt less formal search and appraisal methodologies and serve broader aims of overviewing a research area or informing an agenda for further research. Our study examines the narrative review authoring process, exploring how and when these reviews created with semi-systematic methods should be updated.

% Prior empirical studies have also similarly sought scholars' perspectives and concerns on the role of AI in academic research~\cite{van2023ai, morris_scientists_2023, messeri2024artificial}. One global survey of 1,600 researchers highlights that many are already using AI to handle large amounts of data, write code, and draft papers~\cite{van2023ai}. In interviews with scholars from non-computing fields, participants emphasized the value of AI for streamlining time-consuming tasks, such as dataset assembly and literature review~\cite{morris_scientists_2023}. We elaborate on these prior studies of AI adoption for research by eliciting perspectives on the advantages and limitations of AI support specifically for updating narrative literature reviews.