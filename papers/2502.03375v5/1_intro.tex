\section{Introduction}

\begin{figure}[t]
  \centering
  \includegraphics[width=0.8\columnwidth]{image/head.pdf}
  \vspace{-2em}
  \caption{
  An example of interactive PVisRec: A software engineer seeks a useful visualization for a system log dataset. 
  In each round, the agent recommends a visualization and receives user feedback. If negative, it gathers preferences on attributes and configuration separately to refine its model (R1). It is possible that the user likes attributes or configurations but not the visualization (R2). By learning users' feedback, the agent will recommend high-quality visualization accepted by the user (R3).
  } 
  \Description{An example of interactive PVisRec: A software engineer seeks a useful visualization for a system log dataset.}
  \label{fig:head}
  \vspace{-2em}
\end{figure}

Data visualization has been widely used across domains (i.e., public health~\cite{jia2020big}, engineering~\cite{chiang2017big}, social science~\cite{felt2016social}) to analyze data and obtain insights for effective communication and decision-making. With a suitable visualization, users can observe the tendency of one variable or the correlation of multiple variables. It is also more intuitive for users to choose from generated visualizations than from raw data ~\cite{vartak2015seedb}.
Driven by the intuitive nature of visualizations in data exploration, visualization recommendation systems aim to enhance user experience by accelerating the analysis and exploration of large datasets. 
In practice, different users usually have different backgrounds and purposes (e.g., skill levels, analytic tasks). Thus, it is highly desired to quickly receive and understand user preferences to adapt the recommendation model to more personalized visualization recommendations ~\cite{mutlu_vizrec_2016}. 

Previous works on personalized visualization recommendation~\cite{ojo_visgnn_2022,qian_personalized_2021,mutlu_vizrec_2016,vartak2015seedb} are usually non-interactive and heavily rely on the user embeddings for cold-start of new users. For a new user without any background knowledge, previous works~\cite{ML-based-Vis-Rec,qian_personalized_2021} recommend visualizations based on the averaged user embedding, and thus is not able to capture the actual user's personalized preferences for visualizations.
For example, a software engineer wants to understand the resource usage of a program over time, but the system log contains so many variables that finding the best visualization would be overly time-consuming. With traditional methods, the agent can recommend the most significant visualizations based on the statistical features of the variables, but fails to give more accurate recommendations that rely on the user intention revealed during the interactions. As a result, their models cannot efficiently explore the search space and quickly adapt to the user's real-time feedback in an interactive fashion. 


In this paper, we propose an interactive system for rapid and personalized visualization recommendation.
As shown in Fig. ~\ref{fig:head}, by utilizing the user feedback from previous interactions, the agent can efficiently explore and narrow down the search space rapidly to recommend better visualizations. It is non-trivial to modify and apply bandit algorithms to the PVisRec problem.
\textit{Traditional contextual bandit algorithms} ~\cite{chu2011contextual,li2010contextual} usually suffer from a large action space in the PVisRec setting where users give feedback on a combination of items and visualizations. Compared with combinatorial bandits~\cite{qin_contextual_2014}, non-combinatorial bandits will have an action space with higher ranks, since recommending a visualization is equivalent to recommending attributes and one configuration. 
\textit{Contextual Combinatorial bandit algorithms} ~\cite{qin_contextual_2014, dong2022combinatorial} provide a reduced action space in the recommendation of visualizations, but suffer from the gap between the real reward and the estimated reward. In the PVisRec setting, a user may like an attribute pair and a configuration but dislike their combination as a visualization, which creates a novel kind of bias in the estimation of the reward. 
\textit{Semi-bandit algorithms} ~\cite{li2016contextual,peng2019practical} introduce bias terms to solve the problem of reward estimation, but their combinatorial implementation relies on cascading assumptions, which conflicts with our setting where a user rates the items in combination independently. Plus, their bias terms are unlearnable, and thus these algorithms fail to model the underlying relation between configurations and attributes.

To provide an interactive personalized visualization recommendation system, we introduce \textbf{Hier-SUCB}, a contextual combinatorial semi-bandit with a hierarchical structure tailored to the PVisRec problem. To narrow the gap between the real and estimated reward, we model the bias in the estimation by proposing \emph{an additional bias term} in the exploration of attributes to represent the relation between attributes and configurations. Different from previous work on semi-bandits, our bias is learnable and estimates the interrelation of the visualization with an independent bandit. 
To apply the bias term to PVisRec, we construct \emph{a hierarchical bandit structure} that receives user feedback flexibly for visualizations, configurations, and attributes. The agent will decide on a configuration before evaluating the whole visualization that contains the bandit of bias term with larger action space, and thus accelerate the learning of our agent.
With the hierarchical design and bias term in our algorithm, we improve the regret bound from $O(\sqrt{Tln^3(nm^2T ln(T))})$ by SPUCB ~\cite{peng2019practical} to $O(\sqrt{Tln^3(m^2T ln(T))})$ where $T$ is overall rounds, $n$ is the number of configurations and $m$ is the number of attributes. We test the proposed hierarchical algorithm based on a synthetic experiment and a simulated real-world experiment validated by human evaluation.

To summarize, our contributions are:
\begin{itemize}
    \item We propose an interactive method that learns from real-time user input, avoiding the need for initial data and enabling personalization even in cold-start cases.
    \item We propose a learnable bias term to model configuration-attribute relations, with a hierarchical structure to enhance user experience and accelerate learning.
    \item We theoretically prove the superiority of \textbf{Hier-SUCB} with a lower regret bound and confirm it through experiments, outperforming other bandit algorithms and the offline method.
\end{itemize}