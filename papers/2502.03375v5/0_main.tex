\documentclass[sigconf]{acmart}
% \usepackage{biblatex}
\usepackage{amsfonts}
\usepackage{amsmath}
\usepackage[linesnumbered,ruled,vlined]{algorithm2e}
\usepackage{graphicx}
\usepackage{xcolor}
\usepackage{subcaption}
\usepackage{caption}
\usepackage{url}
\usepackage{balance}
% Table-Figure
\DeclareCaptionLabelFormat{andtable}{#1~#2  \&  \tablename~\thetable}
% BibTex
\AtBeginDocument{%
  \providecommand\BibTeX{{%
    \normalfont B\kern-0.5em{\scshape i\kern-0.25em b}\kern-0.8em\TeX}}}
% ACM
\copyrightyear{2025}
\acmYear{2025}
%\setcopyright{rightsretained}
\setcopyright{acmlicensed}
\acmConference[WWW '25]{Proceedings of the ACM Web Conference 2025}{April 28-May 2, 2025}{Sydney, NSW, Australia}
\acmBooktitle{Proceedings of the ACM Web Conference 2025 (WWW '25), April 28-May 2, 2025, Sydney, NSW, Australia}
\acmDOI{10.1145/3696410.3714697}
\acmISBN{979-8-4007-1274-6/2025/04}
% bib
% \addbibresource{ref.bib}
% \bibliographystyle{ACM-Reference-Format} 
%% No italics, no superscripts
%% Use footnote or author note to identify equal contribution and/or contact author info
% CCS
\begin{CCSXML}
<ccs2012>
   <concept>
       <concept_id>10002951.10003317.10003331.10003271</concept_id>
       <concept_desc>Information systems~Personalization</concept_desc>
       <concept_significance>500</concept_significance>
       </concept>
   <concept>
       <concept_id>10003120.10003145.10003151</concept_id>
       <concept_desc>Human-centered computing~Visualization systems and tools</concept_desc>
       <concept_significance>300</concept_significance>
    </concept>
 </ccs2012>
\end{CCSXML}

\ccsdesc[500]{Information systems~Personalization}
\ccsdesc[500]{Human-centered computing~Visualization systems and tools}
% \ccsdesc[300]{Information systems~Retrieval models and ranking}

\keywords{visualization recommendation, recommendation system}

%%%%%%%%%%%%%%%%%%%%%%%%%%%%%%%%%%

\title{Interactive Visualization Recommendation with Hier-SUCB}

\author{Songwen Hu}
\affiliation{%
 \institution{Georgia Institute of Technology}
 \city{Atlanta}
 \state{Georgia}
 \country{USA}}
\email{shu343@gatech.edu}	

\author{Ryan A. Rossi}
\affiliation{%
 \institution{Adobe Research}
 \city{San Jose}
 \state{California}
 \country{USA}}
\email{ryrossi@adobe.com}

\author{Tong Yu}
\affiliation{%
 \institution{Adobe Research}
 \city{San Jose}
 \state{California}
 \country{USA}}
\email{tyu@adobe.com}

\author{Junda Wu}
\affiliation{%
 \institution{University of California, San Diego}
 \city{San Diego}
 \state{California}
 \country{USA}}
\email{juw069@ucsd.edu}	

\author{Handong Zhao}
\affiliation{%
 \institution{Adobe Research}
 \city{San Jose}
 \state{California}
 \country{USA}}
\email{hazhao@adobe.com}

\author{Sungchul Kim}
\affiliation{%
 \institution{Adobe Research}
 \city{San Jose}
 \state{California}
 \country{USA}}
\email{sukim@adobe.com}

\author{Shuai Li}
\authornote{Corresponding author.}
\affiliation{%
 \institution{John Hopcroft Center, Shanghai Jiao Tong University}
 \city{Shanghai}
 \country{China}}
\email{shuaili8@sjtu.edu.cn}

\renewcommand{\shortauthors}{Songwen Hu et al.}
%%%%%%%%%%%%%%%%%%%%%%%%%%%%%%%%%%

\begin{document}

\begin{abstract}
Visualization recommendation aims to enable rapid visual analysis of massive datasets. 
In real-world scenarios, it is essential to quickly gather and comprehend user preferences to cover users from diverse backgrounds, including varying skill levels and analytical tasks. 
Previous approaches to personalized visualization recommendations are non-interactive and rely on initial user data for new users. As a result, these models cannot effectively explore options or adapt to real-time feedback.
To address this limitation, we propose an interactive personalized visualization recommendation ($\textbf{PVisRec}$) system that learns on user feedback from previous interactions. 
For more interactive and accurate recommendations, we propose $\textbf{Hier-SUCB}$, a contextual combinatorial semi-bandit in the PVisRec setting. 
Theoretically, we show an improved overall regret bound with the same rank of time but an improved rank of action space. 
We further demonstrate the effectiveness of $\textbf{Hier-SUCB}$ through extensive experiments where it is comparable to offline methods and outperforms other bandit algorithms in the setting of visualization recommendation.
\end{abstract}

\maketitle
\section{Introduction}
\label{sec:intro}
% Image editing methods in diffusion models depend on user-defined control directions - users can unlock their creativity using these methods by specifying the desired manipulation through prompts~\cite{gandikota2023concept}, reference images~\cite{ruiz2022dreambooth, kumari2022customdiffusion, gal2022image, chen2024trainingfreeregionalpromptingdiffusion}, or attribute vectors~\cite{parmar2023zero,hertz2022prompt}. In this work, we ask a fundamentally different question: \emph{Can we automatically discover the underlying visual structure of a concept within diffusion model's knowledge?} %Rather than requiring user-specified controls, we aim to decompose the model's internal knowledge into meaningful directions.

% This question touches on a fundamental limitation in how we interact with diffusion models. Current control methods ~\cite{zhang2023addingconditionalcontroltexttoimage, gandikota2023concept, ye2023ipadaptertextcompatibleimage,ye2023ipadaptertextcompatibleimage, hertz2024stylealignedimagegeneration, li2023photomaker, shi2024instantbooth, chen2024trainingfreeregionalpromptingdiffusion} require users to specify their desired manipulations in advance, limiting interactive creativity. This contrasts with natural human artistic workflows, where creators dynamically explore creative ideas while jointly refining them toward meaningful artistic outcomes~\cite{hoffmann2016modeling}. This synergy between specification and exploration is not new to generative models. Early GAN architectures naturally developed disentangled latent spaces that enabled continuous\cite{harkonen2020ganspace,radford2015unsupervised, wu2021stylespace, shen2020interfacegan}, compositional control over generated images. Users could explore these spaces to discover interesting variations that would be difficult to describe in words~\cite{wu2021stylespace}, then combine them to achieve their creative goals~\cite{grabe2022towards}. 


% While diffusion models have largely superseded GANs in conditional image synthesis~\cite{dhariwal2021diffusion},  their underlying structure remains less understood. Diffusion models achieve remarkable diversity through high-dimensional latents, unlike GANs' compact latent spaces.  With a single prompt, diffusion models can generate radically different variations through different random initializations of input noise. We ask - Is it possible to discover interpretable structure within this vast space of variations?

Text-to-image diffusion models are capable of generating remarkable visual variations from a single prompt through different random initializations. However, this vast creative potential remains largely opaque to users---while we can generate diverse images, we lack understanding of the underlying structure of these variations. This presents a fundamental challenge: how can we discover and expose the latent visual capabilities encoded within these models?

\let\thefootnote\relax \footnote{$^{*}$Correspondence to \texttt{gandikota.ro@northeastern.edu}}

The challenge touches on a key limitation in how we interact with diffusion models today. Current control methods require users to explicitly specify their desired edits in advance through prompts~\cite{gandikota2023concept}, reference images~\cite{zhang2023addingconditionalcontroltexttoimage, chen2024trainingfreeregionalpromptingdiffusion, ruiz2022dreambooth,kumari2022customdiffusion, Ryu_lora, hu2021lora}, or attribute vectors~\cite{ye2023ipadaptertextcompatibleimage, hertz2024stylealignedimagegeneration, li2023photomaker, shi2024instantbooth,parmar2023zero,hertz2022prompt}. That contrasts sharply with natural human creative workflows, where artists dynamically explore creative ideas and jointly refine them toward meaningful artistic outcomes~\cite{hoffmann2016modeling}. The need for pre-specified controls creates a barrier between users and the full creative potential of these models.

Interestingly, earlier generative models like GANs~\cite{gans,karras2019style,brock2018large} naturally developed more interpretable internal structures. Their compact latent spaces often exhibited emergent disentanglement~\cite{harkonen2020ganspace,radford2015unsupervised, wu2021stylespace, shen2020interfacegan}, enabling continuous and compositional control over generated images. Users could explore these spaces to discover interesting variations that would be difficult to describe in words~\cite{wu2021stylespace}, then combine them to achieve their creative goals~\cite{grabe2022towards}.

Diffusion models have largely superseded GANs in conditional image synthesis~\cite{dhariwal2021diffusion}, achieving greater diversity through much higher-dimensional latents. And yet an understanding of the underlying structure of these larger latent spaces has remained elusive. In this work, we ask a fundamental question: \emph{Can we automatically discover the visual structure within a diffusion model's knowledge of a concept?} Rather than requiring user-specified controls, we aim to decompose the model's internal representations into expressive directions that users can explore and combine.

To address these needs, we present \textbf{SliderSpace}, a framework that brings systematic explorability to diffusion models. Given just a text prompt, SliderSpace discovers a canonical set of meaningful, diverse, and controllable directions within the model's knowledge of that concept. Each direction is implemented as a low-rank adapter~\cite{hu2021lora} that can be scaled and composed with others, allowing users to explore and smoothly combine different aspects of variation, as shown in Figure~\ref{fig:intro}.

We ground SliderSpace discovery in three key requirements for meaningful decomposition of a diffusion model's visual manifold: 
\begin{enumerate}
    \item \textbf{Unsupervised Discovery:} The decomposition process should emerge from the intrinsic structure of the model's learned representation, rather than being guided by predefined attributes. This ensures we capture the true topology of the model's knowledge space rather than projecting our assumptions onto it.
    
    \item \textbf{Semantic Orthogonality:} Each discovered control must represent a distinct semantic direction. This is enforced in a semantic feature space, like CLIP, where every slider has an orthogonal effect in embeddings. This prevents discovering multiple controls that create similar semantic effects, making the system more efficient and easier.
    
    \item \textbf{Distribution Consistency:} Directions must induce consistent transformations across both random seeds and prompt variations. 
\end{enumerate}

These requirements naturally lead to our proposed framework, which we formalize in Section~\ref{sec:method}. As we show in our experiments, SliderSpace is architecture-agnostic, working with both conventional U-Net based models like Stable Diffusion~\cite{rombach2022high, rombach2022sd20, podell2023sdxl, turbo, dmd} and recent transformer-based architectures like Flux~\cite{flux}.

We demonstrate the expressiveness of SliderSpace through three applications: First, we show how SliderSpace can decompose high-level concepts into diverse and expressive components, revealing the natural axes of variation in the model's understanding. Second, we explore artistic style variation, where SliderSpace discovers directions that match or exceed the diversity of manually curated artist lists while being judged more useful by human evaluators. Finally, we show how SliderSpace can help reverse the mode collapse commonly observed in distilled diffusion models, restoring diversity while maintaining generation speed.

Beyond providing practical creative control, SliderSpace opens new avenues for understanding and utilizing the latent capabilities of diffusion models. By mapping these models' visual potential into intuitive, composable directions, we take a step toward making their creative possibilities more accessible and interpretable to users.

% Image editing methods in diffusion models unlock the creativity of users. In this work we ask an alternate question: \emph{Can we organize and expose what of the diffusion model is already capable of?}.
% Existing methods for controlling image generation typically require users to manually specify edit directions for desired changes. This process is time-consuming, requires technical expertise, and limits the spontaneity of the creative process. For instance, if a user wants to adjust the smile of a generated person, they must explicitly request this edit, often through imprecise prompt engineering or model fine-tuning. This approach of predefined controls or manual specifications restricts users from fully exploring the latent capabilities of the model. There may be interesting stylistic variations or attributes that the model can generate, but users have no easy way to discover or utilize these.

% Natural visual disentanglement was an emergent property in the latent space of Generative Adversarial Models (GANs) \cite{harkonen2020ganspace,radford2015unsupervised, wu2021stylespace, shen2020interfacegan}. In particular, it has been observed that StyleGAN~\cite{karras2019style} stylespace neurons offer detailed control over many meaningful aspects of images that would be difficult to describe in words~\cite{wu2021stylespace}. However, diffusion models do not share such a compact latent space~\cite{park2023unsupervised}; and efforts to uncover such a space in the semantic embeddings of the text conditioning have met with limited success \nik{Nick - is there a specific citation you were thinking about?}.

% In this work we introduce \textbf{SliderSpace}, which takes a step towards uncovering an analogous low dimensional representation of diffusion models' visual breadth; in essence treating the diffusion model as many generators sharing parameters, where a particular generator is defined by a specific prompt. For a given prompt we sample many random seeds (and optionally prompt expansions using an LLM), generate the corresponding images, and apply an off the shelf feature extractor (in this work CLIP, but our method can be applied to any differentiable feature extractor). We use PCA to analyze these features, and for each of the leading $k$ principal components we train a LoRA \cite{} which causes the diffusion model to produces images which increase the feature magnitude along that component when passed back through the same feature extractor. This leads to a 'Slider' for each principal component, because each LoRA can be scaled and applied to the original diffusion model, continuously varying those visual features in the generated results (as measured, in our case, by CLIP).

% There are many other works that enhance the controllability of diffusion models. One common approach is enabling users to add spatial constraints to a generation either manually, or via a reference image \cite{zhang2023addingconditionalcontroltexttoimage, chen2024trainingfreeregionalpromptingdiffusion}, a second is leveraging more abstract embeddings (e.g. identity, style) extracted from a reference image \cite{ye2023ipadaptertextcompatibleimage, hertz2024stylealignedimagegeneration, li2023photomaker, shi2024instantbooth}, a third is finetuning a foundation model to better generate a concept important to the user \cite{ruiz2022dreambooth, kumari2022customdiffusion, Ryu_lora, hu2021lora}, and a fourth (most relevant to this work) is finding low-rank adaptors of the model based on a prompt or small training set which can be scaled to provide continous control over one aspect of generated image (e.g. night vs day, basic vs luxury, etc.) \cite{gandikota2023concept}. SliderSpace is complementary to all of these methods and offers something distinct. All of the other methods we are aware require the user (and / or model designer) to know in advance what type of control they want. In contrast SliderSpace assists users in discovering and controlling hidden capabilities present in the diffusion model's distribution of possible generations.

%We propose that truly intuitive creative control in a text-to-image model should meet three key criteria: \emph{discoverability}, \emph{intuitiveness}, and \emph{specificity}. The model should reveal controllable attributes that may not be immediately obvious, offer controls that are easy to understand and manipulate, and ensure each control affects a distinct attribute of the generated image.

% We demonstrate the utility and power of SliderSpace using three applications built on top of SDXL-DMD \cite{dmd}, because its fast generation speed lends itself well to the continuous control offered by SliderSpace.

% First, we study concept decomposition (Section \ref{sec:concept_exp}), where we learn sliders for a specific concept (e.g. 'monster', 'waterfall', 'car'). Through quantitative metrics of diversity and text alignment we demonstrate that the learned sliders dramatically boost the diversity of generations when randomly applied without harming text alignment; we also ask humans to qualitatively judge these results in a user study where they find the SliderSpace results to be more 'Diverse', 'Useful', and 'Creative' than our baselines.

% Second, we attempt to compare the automatic discoveries of SliderSpace to a large scale manual study of artistic styles (Section \ref{sec:art_exp}), open-sourced by ParrotZone \cite{parrotzone}. In this study SDXL was prompted with over 4300 artist names,  and based on visual inspection the cases of successful stylistic mimicry recorded. Quantitatively SliderSpace more closely matches the distribution of artistic variation discovered by ParrotZone than other baselines, and in our user studies was judged to be significantly more 'Diverse' and 'Useful' than the baselines. To our surprise humans even judged SliderSpace results to be slightly more 'Diverse' than the results generated by the manually discovered artist names of \cite{parrotzone}.

% Third, we attempt to use SliderSpace to reverse the mode collapse commonly observed in distilled few-step diffusion models relative to the original teacher model (Section \ref{sec:diverse_exp}). We quantitatively demonstrate that applying SliderSpace to SDXL-DMD leads to more closely matching the distribution of images by the original teacher, SDXL.

%Through extensive experiments on various state-of-the-art text-to-image models, we demonstrate that SliderSpace significantly enhances user control and creative expression in AI-assisted image generation tasks. Our method enables a range of applications, including concept decomposition and control, diversity improvement in generated images, customization dissection and edits, and the exploration of artistic styles inherent in the model.

% SliderSpace goes beyond providing a practical tool for enhanced creative control. By mapping the visual potential of diffusion models it can open new avenues for generative creativity and deepens our understanding of each model's hidden potential.
\section{Related Work}

\paragraph{LLMs for Agent tasks.}

Our research is related to deploying large language models (LLMs) as agents for decision-making tasks in interactive environments~\citep{liu2023agentbench,zhou2023webarena,shridhar2020alfred,toyama2021androidenv}. Earlier works, such as~\citep{yao2023webshopscalablerealworldweb}, fine-tuned models like BERT~\citep{devlin2019bertpretrainingdeepbidirectional} for decision-making in simplified environments, such as online shopping or mobile phone manipulation. With the advent of large language models~\citep{brown2020languagemodelsfewshotlearners,openai2024gpt4technicalreport}, it became feasible to perform decision-making tasks through zero-shot or few-shot in-context learning. To better assess the capabilities of LLMs as agents, several models have been developed~\citep{deng2024mind2web,xiong2024watch,hong2023cogagent,yan2023gpt}. Most approaches~\citep{zheng2024seeact,deng2024mind2web} provide the agent with observation and action history, and the language model predicts the next action via in-context learning. Additionally, some methods~\citep{zhang2023building,li2023camel,song2024trial} attempt to distill trajectories from state-of-the-art language models to train more effective policy models. In contrast, our paper introduces a novel framework that automatically learns a reward model from LLM agent navigation, using it to guide the agents in making more effective plans.

\textbf{LLM Planning.} Our paper is also related to planning with large language models. Early researchers~\citep{brown2020languagemodelsfewshotlearners} often prompted large language models to directly perform agent tasks. Later, \citet{yao2022react} proposed ReAct, which combined LLMs for action prediction with chain-of-thought prompting~\citep{wei2022chain}. Several other works~\citep{yao2023treethoughtsdeliberateproblem,hao2023reasoning,zhao2023large,qiao2024agentplanningworldknowledge} have focused on enhancing multi-step reasoning capabilities by integrating LLMs with tree search methods. Our model differs from these previous studies in several significant ways. First, rather than solely focusing on text generation tasks, our pipeline addresses multi-step action planning tasks in interactive environments, where we must consider not only historical input but also multimodal feedback from the environment. Additionally, our pipeline involves automatic learning of the reward model from the environment without relying on human-annotated data, whereas previous works rely on prompting-based frameworks that require large commercial LLMs like GPT-4~\citep{openai2024gpt4technicalreport} to learn action prediction. Furthermore, \Model supports a variety of planning algorithms beyond tree search.

\textbf{Learning from AI Feedback.} In contrast to prior work on LLM planning, our approach also draws on recent advances in learning from AI feedback~\citep{bai2022constitutional,lee2023rlaif,yuan2024self,sharma2024critical,pan2024autonomous,koh2024tree}. These studies initially prompt state-of-the-art large language models to generate text responses that adhere to predefined principles and then potentially fine-tune the LLMs with reinforcement learning. Like previous studies, we also prompt large language models to generate synthetic data. However, unlike them, we focus not on fine-tuning a better generative model but on developing a classification model that evaluates how well action trajectories fulfill the intended instructions. This approach is simpler, requires no reliance on state-of-the-art LLMs, and is more efficient. We also demonstrate that our learned reward model can integrate with various LLMs and planning algorithms, consistently improving their performance.

\textbf{Inference-Time Scaling.} ~\citet{snell2024scaling} validates the efficacy of inference-time scaling for language models. Based on inference-time scaling, various methods have been proposed, such as random sampling~\citep{wang2022self} and tree-search methods~\citep{hao2023reasoning, zhang2024accessing, guan2025rstar}. Concurrently, several works have also leveraged inference-time scaling to improve the performance of agentic tasks. ~\citet{koh2024tree} adopts a training-free approach, employing MCTS to enhance policy model performance during inference and prompting the LLM to return the reward. ~\citet{gu2024your} introduces a novel speculative reasoning approach to bypass irreversible actions by leveraging LLMs or VLMs. It also employs tree search to improve performance and prompts an LLM to output rewards. ~\citet{yu2024exact} proposes Reflective-MCTS to perform tree search and fine-tune the GPT model, leading to improvements in ~\citet{koh2024visualwebarena}. ~\citet{putta2024agent} also utilizes MCTS to enhance performance on web-based tasks such as ~\citet{yao2023webshopscalablerealworldweb} and real-world booking environments. ~\cite{lin2025qlass} utilizes the stepwise reward to give effective intermediate guidance across different agentic tasks. Our work differs from previous efforts in two key aspects: (1) Broader Application Domain. Unlike prior studies that primarily focus on tasks from a single domain, our method demonstrates strong generalizability across web agents, mathematical reasoning, and scientific discovery domains, further proving its effectiveness. (2) Flexible and Effective Reward Modeling. Instead of simply prompting an LLM as a reward model, we finetune a small scale VLM~\citep{lin2023vila} to evaluate input trajectories. %Our reward scores range continuously between 0 and 1, in contrast to existing methods that rely on discrete scoring (e.g., 0 and 1, or 0, 0.5, and 1) through direct LLM prompting.

% Concurrently, several works have also leveraged inference-time scaling to improve the performance of agentic tasks. ~\citet{pan2024autonomous} demonstrates that LLMs and VLMs, such as the GPT series, can function as evaluators or reward models to provide guidance for fine-tuning or reflection, thereby enhancing digital agents. This lays the groundwork for subsequent studies that directly prompt LLMs as reward models. ~\citet{koh2024tree} adopts a training-free approach, employing MCTS to enhance policy model performance during inference. However, it is limited to web environments~\citep{koh2024visualwebarena}. Moreover, its value function relies on prompting an LLM, which is less effective than our proposed method. We validate our approach through ablation studies, demonstrating that our fine-tuned reward model is more effective. ~\citet{gu2024your} introduces a novel speculative reasoning approach to bypass irreversible actions, such as purchasing a product, by leveraging LLMs or VLMs. It also employs tree search to improve performance, but it remains restricted to the web domain~\citep{koh2024visualwebarena, deng2024mind2web}. Additionally, it lacks reward modeling and instead prompts an LLM to output rewards. ~\citet{yu2024exact} proposes Reflective-MCTS to perform tree search and fine-tune the GPT model, leading to improvements in ~\citep{koh2024visualwebarena}. However, this work focuses solely on a single web agent task, and its reward modeling is derived from multi-agent debate, differing from our more effective and efficient reward modeling approach. ~\citet{putta2024agent} also utilizes MCTS to enhance performance, but it is limited to web-based tasks such as ~\citep{yao2023webshopscalablerealworldweb} and real-world booking environments.
\section{Forestry Crane Simulator} \label{Sec:method}

\subsection{Kinematics}

Figure \ref{fig:crane_scematics} illustrates the schematic of the forestry crane. 
It has eight degrees of freedom (DoFs) $\mathbf{q}^\mathrm{T} = [\mathbf{q}_A^\mathrm{T},\mathbf{q}_U^\mathrm{T}] $ consisting of six actuated DoFs $\mathbf{q}_A^\mathrm{T} = [q_1,q_2,q_3,q_4,q_7,q_8]$ and two unactuated joints $\mathbf{q}_U^\mathrm{T} = [{q}_5,{q}_6]$. Note that there are two pairs of synchronized joints, i.e., the prismatic joint $q_4$ and the revolute joint $q_8$. \corr{In each pair of synchronized joints, the same input is applied to the corresponding actuators; for example, the joint angle $q_8$ at the left- and right-jaw of the grapple in Figure \ref{fig:crane_scematics}. }

%The characteristics of all joints are listed in Table \ref{tab:crane_joints}.  
\begin{figure}
    \centering
    \scalebox{0.8}{
    \includegraphics[trim=5cm 3.5cm 0cm 2cm,clip,scale=0.44]{figures/KinematicChain.pdf}
    }
    \caption{Schematic of the forestry crane \cite{ecker2022iterative}.}
    \label{fig:crane_scematics}
\end{figure}
\iffalse

    \begin{table}
        \caption[abc]{List of the forestry crane joints.}
        \label{tab:crane_joints}
        \begin{center} 
            
                \begin{tabular}{c c c c c}
                    \hline
                    Coordinate & Name & Actuated & Range & Unit\\
                    \hline
                     $q_1$ & Slewing joint & \checkmark & [-3.71,\:3.71] & \si{rad}\\ 
                     $q_2$ & Boom joint & \checkmark & [-1.2,\:1.56] & \si{rad} \\  
                     $q_3$ & Arm joint & \checkmark & [-0.91,\:4.6] & \si{rad} \\
                     $q_4$ & Prismatic joint & \checkmark & [0,\:4.47] & \si{m} \\
                     $q_5$ & Tip joint & \xmark & [-1.57,\:1.57] & \si{rad} \\
                     $q_6$ & Tilt joint & \xmark & [-0.79,\:2.36] & \si{rad} \\
                     $q_7$ & Rotate joint & \checkmark & $ [-\infty,\:\infty]$ & \si{rad}\\
                     $q_8$ & Grapple jaws & \checkmark & [0,\:3] & \si{rad}\\
                    \hline 
                \end{tabular}
        \end{center}        
    \end{table}
\fi    

The kinematics of the forestry crane are described by transformations from a coordinate Frame $\mathcal{F}_i$ attached to joint $i$ to a coordinate frame $\mathcal{F}_{i-1}$ attached to joint $i-1$
\begin{align}
	\vec{H}^i_{i-1} = \begin{bmatrix}
		\vec{R}_{i-1}^i & \vec{d}_{i-1}^i\\
		\transpose{\vec{0}} & 1
	\end{bmatrix}\in\mathcal{SE}(3) \Comma
\end{align}
where $\vec{R}_{i-1}^i\in\mathcal{SO}(3)$ and $\vec{d}_{i-1}^i\in\mathbb{R}^3$ are the three-dimensional rotation matrix and the three-dimensional translation vector, respectively. 
The coordinate frames are illustrated in Figure~\ref{fig:crane_scematics} according to the \textit{Denavit-Hartenberg (DH) convention} \cite{spong:2006}. 
\iffalse

    Note that frame $\mathcal{F}_{11}$ is defined by DH convention w.r.t. frame $\mathcal{F}_{8}$ instead of $\mathcal{F}_{10}$ due to the kinematic structure depicted in Figure \ref{fig:crane_scematics}.
    %, but rather from frame $\mathcal{F}_8$. 
    Hence, homogeneous transformations $\vec{H}_{i-1}^i$, $i=1,\dots,10,12$ and $\vec{H}_{8}^{11}$ can be described using four DH parameters $\theta_i$, $d_i$, $a_i$ and $\alpha_i$ as
    \begin{align}
    	\vec{H}_{i-1}^i = \vec{H}_{Rz}(\theta_i)\vec{H}_{Tz}(d_i)\vec{H}_{Tx}(a_i)\vec{H}_{Rx}(\alpha_i)\Comma
    \end{align}
    where $\vec{H}_{Ri}$ is a pure rotation around the $i$-axis and $\vec{H}_{Ti}$ is a pure translation in direction of the $i$-axis. 
    The transformation from $\mathcal{F}_j$ to $\mathcal{F}_i$, $0\leq i < j$ can be computed using
    \begin{align}
    	\vec{H}_i^j=\begin{cases}
    		\prod_{l=i+1}^j\vec{H}_{l-1}^l&,\text{ for }j\leq 10\\
    		\Big(\prod_{l=i+1}^8\vec{H}_{l-1}^l\Big)\vec{H}_{8}^{11}\vec{H}_{11}^j&,\text{ for }11\leq j\leq 12
    	\end{cases}\Comma
    \end{align}
    where $\prod_{l=i+1}^j\vec{H}_{l-1}^l$ being the identity for $j\leq i$.
    \begin{table}
        \caption{Denavit-Hartenberg parameters of the timber crane.}\label{tab:DHParams}
    	\centering
    	\begin{tabular}{c|cccc}
                \hline
    		$i$ & $\theta_i$ [rad] & $d_i$ [m] & $a_i$ [m] & $\alpha_i$ [rad]\\
    		\hline
    		1   &            $q_1$ & 2.425     & 0.1800      &$\pi/2$\\
    		2   &            $q_2$ & 0         & 3.4931    &0\\
    		3   &            $q_3$ & 0         & -0.3925   &$\pi/2$\\
    		4   &                0 & $q_4$ + 3.157     & 0         &0\\
    		5   &                0 & $q_4$     & 0         &$-\pi/2$\\
    		6   &            $q_5$ & 0         & -0.2130   &$-\pi/2$\\
    		7   &            $q_6$ & 0         & 0         &$-\pi/2$\\
    		8   &            $q_7$ & 0.578     & 0         &0\\
    		9   & $-\pi/2$ & 0         & 0.3402    &$\pi/2$\\
    		10  &           $\pi/2$ & 0         & 0.8566    &0\\
    		11  &  $\pi/2$ & 0         & 0.3248    &$\pi/2$\\
    		12  &           $\pi/2$ & 0         & 0.8566    &0\\
                 \hline
    	\end{tabular}
    \end{table}
\begin{table}[h]
    
    \caption{Denavit-Hartenberg parameters of the forestry crane.}\label{tab:DHParams}
        	\begin{center}
            	\begin{tabular}{c|cccc}
                        \hline
            		$i$ & $\theta_i$ in \SI{}{\radian} & $d_i$ in \SI{}{\meter} & $a_i$ in \SI{}{\meter} & $\alpha_i$ in \SI{}{\radian}\\
            		\hline
            		1   &            $q_1$ & 2.4     & 0.18      &$\pi/2$\\
            		2   &            $q_2$ & 0         & 3.5    &0\\
            		3   &            $q_3$ & 0         & -0.4   &$\pi/2$\\
            		4   &                0 & $q_4$ + 3.1     & 0         &0\\
            		5   &                0 & $q_4$     & 0         &$-\pi/2$\\
            		6   &            $q_5$ & 0         & -0.21   &$-\pi/2$\\
            		7   &            $q_6$ & 0         & 0         &$-\pi/2$\\
            		8   &            $q_7$ & 0.58     & 0         &0\\
                    %9   & $-\pi/2$ & 0         & 0.3402    &$\pi/2$\\
                    \hline
            	\end{tabular}
        	\end{center}
    
    \end{table}
\fi    
    %The DH parameters for the forestry crane are given in Table~\ref{tab:DHParams}. 

\iffalse
    Using the above kinematic relations, the wrist position of the grapple, $\mathbf{d}_g^\mathrm{T} = [g_x,g_y,g_z]$, is taken from  
    %Using this formalism the calculation of the center point coordinates $g_x, g_y$ and $g_z$ of the grapple reads as
    \begin{equation}
        \vec{H}_{0}^{8} = 
        \begin{bmatrix}
            \mathbf{R}_0^8 & \mathbf{d}_g \\
            \mathbf{0}^\mathrm{T} & 1
        \end{bmatrix} \:.
        \label{eq:transformation}
    \end{equation}
\fi    
%In the used scenarios the grapple is already close to the logs and the crane is unfolded, therefore the special hydraulic kinematic is neglected. 
%Instead, all rotary joints are driven with velocity-controlled rotary motors and the telescopic boom is driven by a linear motor.

\subsection{Simulator}
\label{sec: b simulator}

The system dynamics and contacts with the environment are simulated using the open-source MuJoCo \cite{todorov2012mujoco} physics engine. 
An example of the simulated environment is illustrated in Figure \ref{fig: example mujoco}. 
The assembled model of the forestry crane (including the truck) consists of $38$ rigid bodies and $10$ active joints\corr{, including two pairs of synchronized joints}. The total operating weight of the forestry crane is approximately \SI{1981}{\kilo\gram}. 
On standard forestry cranes, hydraulic actuators powered by a pump that is driven by a combustion engine drive the slewing ($q_1$), boom ($q_2$), arm ($q_3$), and prismatic ($q_4$) joints, respectively. 
In order to simplify the model for training purposes, the hydraulic actuators are not explicitly modeled. 
%Instead, two linear motors are modeled for the synchronized joint $q_4$, and rotational motors are utilized to drive other actuated joints. 
\corr{We assume an (ideal) underlying velocity controller, with the reference velocity for the prismatic joint $q_4$ and the reference rotational velocity for the other joints as inputs. }
Thus, in the simulation environment, the grasping controller for the modeled forestry crane is a fine-tuned PID controller with reference velocities for the actuated joints $\mathbf{q}_A$.  

The wood log position is randomized in a reachable region of the forestry crane, depicted as the yellow region in Fig. \ref{fig: example mujoco}. 
The center of this region is approximately \SI{6.5}{\meter} from the crane's base. 
Additionally, logs are modeled as cylinders with a length of $\SI{2.75}{\meter}$, and the log's diameter varies in the range of $[0.3,0.8] \SI{}{\meter}$. 
In order to prevent overfitting during the training process, the slew angle of the crane is varied in the range $[-2\pi/3, -\pi/3] \SI{}{\radian}$. 
The 6-dimensional contact forces between the grapple and the wood log are computed using the signed distance field (SDF) collision primitive \cite{reiner2011interactive}. 
This is particularly important to maintain the robustness of the simulation since the inner and outer jaw of the grapple have curvy shapes. 










\section{Analysis of the Sample Complexity}\label{sec:anal}

In this section, we present our main results on the sample complexity of the algorithms. We first establish a novel confidence interval that is applicable to the unbiased samples collected by our exploration algorithms. We then provide theorems detailing the performance of these algorithms. 


\subsection{Confidence Intervals}
We introduce a novel confidence interval that is tighter than existing ones in our RL setting and can also be applied to other RL problems such as offline RL and infinite-horizon settings.
%
\begin{theorem}[Confidence Bounds]\label{the:conf}

Consider compact sets $\Sc\subset\Rr^{d_s}, \Ac\subset\Rr^{d_a}$, and define $\Zc=\Sc\times \Ac$, $d=d_a+d_s$. Consider two Mercer kernels $k_{\varphi}:\Zc\times\Zc\rightarrow \Rr$ and  $k_{\psi}:\Sc\times\Sc\rightarrow \Rr$. Assume that functions $f:\Zc\rightarrow\Rr$ and $V:\Sc\rightarrow\Rr$, and for each $z\in\Zc$, a conditional probability distribution $P(\cdot|z)$ over $\Sc$, are given such that $f(z)=\E_{s\sim P(\cdot|z)}[V(s)]$, $\|f\|_{\Hc_{k_\varphi}}\le B_1$, $\|V\|_{\Hc_{k_\psi}}\le B_2$, and $\max_{s\in\Sc}V(s)\le v_{\max}$, for some $B_1,B_2, v_{\max}>0$.
Assume a dataset of $\{z_{i}, s'_i\}_{i=1}^n$ is provided, where each $z_i$ is independent of the set $\{s'_j\}_{j=1}^n$, and $s'_i\sim P(\cdot|z_i)$.
Let $\hat{f}^n$ and $\sigma^n$ be the kernel ridge predictor and uncertainty estimator of $f$ using the observations:
\begin{align}\nn
    \hat{f}_n(z) &= k^{\top}_{\varphi_n}(z)(\tau^2I+K_{\varphi_n})^{-1}\bm{y}_{n},\\
    \sigma_n^2(z) &= k_{\varphi}(z,z) - k^{\top}_{\varphi_n}(z)(\tau^2I+K_{\varphi_n})^{-1}k_{\varphi_n}(z),
\end{align}
where $\bm{y}_n=[V(s'_1), V(s'_2), \cdots, V(s'_n))]^{\top}$.
In addition, let $\lambda_m$, $m=1,2,\cdots$ represent the Mercer eigenvalues of $k_{\psi}$ in a decreasing order, and $\psi_m$ the corresponding Mercer eigenfunctions. Assume $\psi_m\le \psi_{\max}$ for some $\psi_{\max}>0$. Fix $M\in \Nn$, and let $C$ be a constant such that $C \geq \sum_{m=1}^{M} \lambda_m$. %that serves as an upper bound for the sum of the first $M$ eigenvalues.


Then, for a fixed $z\in\Zc$, and for all $V$, with $\|V\|_{\Hc_{k_\psi}}\le B_2$, we have, the following each hold, with probability at least $1-\delta$,
\begin{equation*}
|f(z) - \hat{f}_n(z)| \le \beta(\delta) \sigma_n(z)
\end{equation*}
%\begin{equation*}
%    f(z) - \hat{f}_n(z) \le \beta(\delta) \sigma_n(z)~~ \text{and} ~~ f(z) - \hat{f}_n(z) \ge -\beta(\delta) \sigma_n(z).
%\end{equation*}
% \begin{equation*}
% \left\{
% \begin{aligned}
% & f(z) - \hat{f}_n(z) \le \beta(\delta) \sigma_n(z),  \\
% & f(z) - \hat{f}_n(z) \ge -\beta(\delta) \sigma_n(z),
% \end{aligned}
% \right.
% \end{equation*}
with $\beta(\delta) =$
\small{\begin{align*}
     B_1+ \frac{C B_2 \psi_{\max} }{\tau}\sqrt{2\log(\frac{M}{\delta})} 
    + \frac{2B_2\psi_{\max}}{\tau}\sqrt{n\sum_{m=M+1}^{\infty}\lambda_m}~.
\end{align*}}

%\aya{, where $C$ is a constant} %that serves as an upper bound for the sum of the first $M$ eigenvalues.}
\end{theorem}



Theorem~\ref{the:conf} provides a confidence bound for kernel ridge regression that is applicable to our RL setting, and is a key result in deriving our sample complexities. 

\paragraph{Proof sketch}
To derive our confidence bounds, we use the Mercer representation of \( V \) and decompose the prediction error \( f(z) - \hat{f}_n(z) \) into error terms corresponding to each Mercer eigenfunction \( \psi_m \). We then divide these terms into two groups: the first \( M \) elements, corresponding to eigenfunctions with the largest eigenvalues, and the remainder. For the top \( M \) eigenfunctions, we establish high-probability bounds using standard kernel-based confidence intervals from \cite{vakili2021optimal}. The remaining terms are bounded based on eigendecay, and we sum over all \( m \) to obtain \( \beta(\delta) \).
%To derive our confidence bounds, we use a novel approach by leveraging the Mercer representation of \( V \) and decomposing the prediction error \( f(z) - \hat{f}_n(z) \) into error terms corresponding to each Mercer eigenfunction \( \psi_m \). We then divide these terms into two groups: the first \( M \) elements, corresponding to eigenfunctions with the largest eigenvalues, and the remainder. For the top \( M \) eigenfunctions, we establish high-probability bounds using standard kernel-based confidence intervals from \citep{vakili2021optimal}. The remaining terms are bounded based on eigendecay, and we sum over all \( m \) to obtain \( \beta(\delta) \).}
\begin{remark}
    Under some mild conditions, for example, the polynomial eigendecay given in Definition~\ref{def:eigendecay}, the following expression can be derived for $\beta$:
    \begin{equation}
        \beta(\delta)= \Oc\left(B_1+\frac{ B_2 \psi_{\max} }{\tau}\sqrt{\log(\frac{n}{\delta})}\right).
    \end{equation}
\end{remark}

With polynomial eigendecay, the remark follows from setting $M$ to $\lceil n^{\frac{1}{p-1}}\rceil 
$ in the expression of $\beta$ in Theorem~\ref{the:conf}.

The confidence interval presented in Theorem~\ref{the:conf} is applicable to a fixed $z\in\Zc$. Over a discrete domain this can be easily extended to all $z\in\Zc$ using a probability union bound and replacing $\delta$ with~$\frac{\delta}{|\Zc|}$ in the expression of $\beta(\delta)$. Using standard discretization techniques, we can also prove a variation of the confidence interval that holds true uniformly over continuous domains. In particular, under the following assumption, we present a variation of the theorem over continuous domains. 


\begin{assumption}\label{ass:disc}
For each $n\in\Nn$, there exists a discretization $\Zz$ of $\Zc$ such that, for any $f\in \Hc_k$ with $\|f\|_{\Hc_k}\le B_1$, we have $f(z) - f([z])\le \frac{1}{n}$, where $[z] = {\arg}{\min}_{ z'\in \Zz}||z'-z||_{l^2}$ is the closest point in $\Zz$ to $z$, and $|\Zz|\le cB_1^dn^{d}$, where $c$ is a constant independent of $n$ and $B_1$.
\end{assumption}
Assumption~\ref{ass:disc} is a mild technical assumption that holds for typical kernels~\citep{srinivas2009gaussian, chowdhury2017kernelized, vakili2021optimal}.

\begin{corollary}\label{Cor:cont}
Under the setting of Theorem~\ref{the:conf}, and under Assumption~\ref{ass:disc}, the following inequalities each hold uniformly in $z\in\Zc$ and $V: \|V\|_{\Hc_{k_\psi}}\le B_2$, with probability at least $1-\delta$
\begin{align*}
    f(z)  \le \hat{f}_n(z) + \frac{2}{n} +  \tilde{\beta}(\delta) (\sigma_n(z)+\frac{2}{\sqrt{n}}), \\
    f(z)  \ge\hat{f}_n(z) -\frac{2}{n} -\tilde{\beta}(\delta) (\sigma_n(z)+\frac{2}{\sqrt{n}}),
\end{align*}
% \begin{equation*}
% \left\{
% \begin{aligned}
% & f(z)  \le \hat{f}_n(z) + \frac{2}{n} +  \tilde{\beta}(\delta) (\sigma_n(z)+\frac{2}{\sqrt{n}}),  \\
% & f(z)  \ge\hat{f}_n(z) -\frac{2}{n} -\tilde{\beta}(\delta) (\sigma_n(z)+\frac{2}{\sqrt{n}}),
% \end{aligned}
% \right.
% \end{equation*}
with $\tilde{\beta}(\delta)=\beta(\frac{\delta}{2c_n})$, $c_n=c(u_n(\frac{\delta}{2}))^dn^d$, and $u_n(\delta) = \Oc(\sqrt{n+\log(\frac{1}{\delta}}))$.
%is a $1-\delta$ upper confidence bound on $\|\hat{f}_n\|_{\Hc_k}$.

\end{corollary}


\begin{remark}
    Under some mild conditions, for example, the polynomial eigendecay given in Definition~\ref{def:eigendecay}, the following expression can be derived for $\tilde{\beta}$:
    \begin{equation}
        \tilde{\beta}(\delta)= \Oc\left(B_1+\frac{C B_2 \psi_{\max}}{\tau}\sqrt{d\log(\frac{n}{\delta})}\right).
    \end{equation}
\end{remark}

 
\subsection{Sample Complexities}

We have the following theorem on the performance of Algorithm~\ref{alg:exp_gen}. 
The weakest assumption one can pose on the value functions is realizability, which posits that the optimal value functions \(V^{\star}_{h}\) for \(h \in [H]\) lie in the RKHS \(H_{k_\psi}\) for some kernel $k_{\psi}:\Sc\times\Sc\rightarrow \Rr$, or at least are well-approximated by \(H_{k_\psi}\). For stateless MDPs or multi-armed bandits where \(H = 1\), realizability alone suffices for provably efficient algorithms~\citep{srinivas2009gaussian, chowdhury2017kernelized, vakili2021optimal}. But it does not seem to be sufficient when \(H > 1\), and in these settings it is common to make stronger assumptions~\citep{jin2020provably, wang2019optimism, chowdhury2023value}.
Following these works, our main assumption is a closure property for all value functions in the following class:
\begin{align}
    \Vc = \left\{
    s\rightarrow \min\left\{
    H, \max_{a\in\Ac}\left\{
    r(s,a) + \varphi^{\top}(s,a)\bm{w} + 
    \right. \right. \right. \nonumber \\
    \left. \left. \left. \beta \sqrt{\varphi^{\top}(s,a)\Sigma^{-1}\varphi(s,a)}
    \right\}
    \right\}
    \right\},
\end{align}

where $0<\beta<\infty$, $\|\bm{w}\|\le\infty$ 

and $\Sigma$ is an $\infty\times \infty$ matrix with $\Sigma \succ \tau^2 I$. 

\begin{assumption}[Optimistic Closure]\label{closure_assumption}
For any $V\in\Vc$, for some positive constant $c_v$, we have $\|V\|_{H_{k_\psi}}\leqslant c_v$.
\end{assumption}

This is the same assumption as Assumption~1 in~\cite{chowdhury2023value} and can be relaxed to value functions $\epsilon$ away from this class as described in Section $4.3$ of~\cite{chowdhury2023value}.
We have the following theorem on the sample complexity of the exploration algorithm with a generative model.


\begin{theorem}\label{the:gen}
    Consider the reward-free RL framework described in Section~\ref{sec:pf}. Assume the existence of a generative model in the exploration phase that allows the algorithm to select state-action pairs of its choice at each step. Let $N_0$ be the smallest integer satisfying
    \begin{equation*}
2H\beta(\delta)\sqrt{\frac{2\Gamma(N_0)}{N_0\log(1+1/\tau^2)}} +\frac{4\beta(\delta)H}{\sqrt{N_0}} +\frac{4H}{N_0}\le \epsilon,
\end{equation*}
with $\beta(\delta) =\Oc(\frac{H}{\tau}\sqrt{d\log(\frac{NH}{\delta})})$ with a sufficiently large constant. 
Run Algorithm~\ref{alg:exp_gen} for $N\ge N_0$ episodes to obtain the dataset $\Dc_N$. Then, use the obtained samples to design a policy $\pi$ using Algorithm~\ref{alg:plan} with $\beta(\delta) =\Oc(\frac{H}{\tau}\sqrt{d\log(\frac{NH}{\delta})})$ with a sufficiently large constant. 
Then, under Assumptions~\ref{ass:rkhsnorm}, \ref{ass:disc} and~\ref{closure_assumption}, with probability at least $1-\delta$, $\pi$ is guaranteed to be an $\epsilon$-optimal policy. 
\end{theorem}

The following theorem presents the sample complexity for exploration without generative models. 


\begin{theorem}\label{the:main}
    Consider the reward free RL framework described in Section~\ref{sec:pf}. Let $N_0$ be the smallest integer satisfying
    \begin{align}\nn
    &3H^2\beta(\delta)\sqrt{\frac{2\Gamma(N_0)}{N_0\log(1+1/\tau^2)}} +\frac{8\beta(\delta)H^2}{\sqrt{N_0}} \\\label{eq:suboptgap}
    &+\frac{4H^2(\log(N_0)+1)}{N_0}+2H^2\sqrt{2N_0\log({\frac{3N_0}{\delta})}}\le \epsilon
\end{align}
with $\beta(\delta) =\Oc(\frac{H}{\tau}\sqrt{d\log(\frac{NH}{\delta})})$ with a sufficiently large constant.     Run Algorithm~\ref{alg:exp2} for $NH\ge N_0H$ episodes to obtain the dataset $\Dc_N$. Then, use the obtained samples to design a policy $\pi$ using Algorithm~\ref{alg:plan}. 
Then, under Assumptions~\ref{ass:rkhsnorm}, \ref{ass:disc} and~\ref{closure_assumption}, with probability at least $1-\delta$, $\pi$ is guaranteed to be an $\epsilon$-optimal policy. 
\end{theorem}
    

The proof of theorems are provided in Appendix~\ref{appx:gen} and~\ref{appx:main_sample}. 

The expression of suboptimality gap after $N$ samples, given in~\eqref{eq:suboptgap}, can be simplified as
\begin{equation*}
    \Oc\left(  H^3\sqrt{\frac{\Gamma(N)\log(NH/\delta)}{N}}\right).
\end{equation*} 
 
\begin{figure*}[h]
    \centering
    \begin{subfigure}{0.32\textwidth}
        \centering
        \includegraphics[width=\textwidth]{figures/new_figures/RBF_kernel_all_algos.png}
        \caption{SE Kernel}
        \label{fig:RBF_all_algos}
    \end{subfigure}
    %\hspace{0.3em} 
    \begin{subfigure}{0.32\textwidth}
        \centering
        \includegraphics[width=\textwidth]{figures/new_figures/Matern2.5_all_algos.png} % Replace with the path to your third figure
        \caption{Mat{\'e}rn Kernel with $\nu=2.5$}
        \label{fig:Matern2.5_all_algos}
    \end{subfigure}
    %\hspace{0.3em} 
    \begin{subfigure}{0.32\textwidth}
        \centering
        \includegraphics[width=\textwidth]{figures/new_figures/Matern1.5_all_algos.png} % Replace with the path to your second figure
        \caption{Mat{\'e}rn kernel with $\nu=1.5$}
        \label{fig:Matern1.5_all_algos}
    \end{subfigure}
    \caption{Average suboptimality gap against $N$. The error bars indicate standard deviation.}
    \label{fig:overallresults}
\end{figure*}
\begin{remark}
Replacing $\Gamma(N)=\Oct(N^{\frac{1}{p}})$ in the case of kernels with polynomial eigendecay, we obtain a sample complexity of 
$
    N = \Oct((\frac{H^3}{\epsilon})^{2+\frac{2}{p-1}}).
$
We also recall that without a generative model, we interact with $H$ times more episodes to collect these samples. Specifically, the number of episodes in the exploration phase is 
$NH = \Oct\left(H(\frac{H^3}{\epsilon})^{2+\frac{2}{p-1}}\right)$.

\end{remark}

When specialized for the case of Mat{\'e}rn kernels with $p=1+\frac{2d}{\nu}$, we obtain $NH=\Oct(H(\frac{H^3}{\epsilon})^{2+\frac{d}{\nu}})$ that matches the lower bound for the degenerate case of bandits with $H=1$ proven in~\citet{scarlett2017lower}. Our sample complexity is thus order optimal in terms of $\epsilon$ dependency. We also recall that the existing results lead to possibly vacuous (infinite) sample complexities for these kernels.

\section{Experiment}
In this section, we conduct extensive experiments to evaluate the performance of various LLMs on our Hellaswag-Pro benchmark. Our study is guided by three key research questions:
\textbf{RQ1}: How do different LLMs perform across all variants?
\textbf{RQ2}: What is the relative difficulty of different variants?
\textbf{RQ3}: How robust are LLMs to diverse prompts during evaluation?

\subsection{Experiment Setup} 
\subsubsection{Model Selection and Implementation Details}
We select 41 representative commercial and open-source models, including English LLMs, such as GPT-4o, Claude-3.5-Sonnet, Gemini-1.5-Pro,Mistral series, Llama3 series and Chinese LLMs, like Qwen-Max,  Qwen2.5 series, InternLM-2.5 series, Yi-1.5 series, Baichuan-2 series and DeepSeek series.

We integrate both Chinese HellaSwag and HellaSwagPro into the lm-evaluation-harness platform. For the open-source models, we use the default settings of lm-evaluation-harness: do\_sample is set to false and the temperature is set to the default value of the hugging-face library. For the closed-source models, we set the temperature to 0.7. In addition, we set the maximum output length to 1024.

\subsubsection{Prompt Strategy}
Taking into account the influence of language and shot, we design 9 prompting strategies, including Direct, CN-CoT, EN-CoT, CN-XLT and EN-XLT. The last four setups include both zero-shot and few-shot variants.\footnote{
For open-source models, Direct adopts an approach similar to the official implementation of HellaSwag, computing the log-likelihood for each option and selecting the one with the highest log-likelihood. And we report normalized accuracy that accounts for the impact of option length. Other prompting strategies use a generation setup and report accuracy based on exact match.}
\textbf {(1)Direct}: LLMs makes the selection directly without any CoT process.
\textbf{(2)CN-CoT}: LLMs performs CoT in Chinese, regardless of dataset language.
\textbf{(3)EN-CoT}: Similar to CN-CoT, but CoT is conducted in English. 
\textbf{(4)CN-XLT}: LLMs are instructed to first translate English questions and options to Chinese, and then reason in Chinese.
\textbf{(5)EN-XLT}: Similar to CN-XLT, but translates from Chinese dataset to English and reasons in English. 

%\textbf {CN-CoT}: LLMs perform Chinese reasoning and then output the answer and 3 shots are provided.
%\textbf {CN-CoT}: Similar as CNCoTFewShot without any shots.
%\textbf {EN-CoT}: The reasoning process in English is executed and then the answer is output and 3 shots are provided.
%\textbf {CN-XLT}: Inspired by this, we instruct LLMs to translate questions in Chinese and then output the answer after performing reasoning in Chinese too. And 3 shots are provided.
%\textbf {EN-XLT}: Inspired by this, we instruct LLMs to translate questions in Englsih and then output the answer after performing reasoning in Englsih too. Three shots are provided.

\subsubsection{Evaluation metric}

To comprehensively evaluate the robustness of each LLM, we consider four metrics: 
% Original Accuracy (\textbf{OA}), Average Robust Accuracy (\textbf{ARA}), Robust Loss Accuracy (\textbf{RLA}), and  Consistent Robust Accuracy (\textbf{CRA}).
\noindent %
\textbf{- Original Accuracy (OA)} measures accuracy on original problems.
\begin{equation}\label{eq1}
OA=\frac{\sum_{(x, y) \in D} \mathds{1}[L M(x), y]}{|D|}.
\end{equation}
\noindent %
\textbf{- Average Robust Accuracy  (ARA)} represents average accuracy across all variants, gauging overall performance on the robustness tasks.
\begin{equation}\label{eq2}
ARA=\frac{\sum_{\left(x^{\prime}, y^{\prime}\right) \in D_{R}} \mathds{1}\left(L M\left(x^{\prime}, y^{\prime}\right)\right.}{\left|D_{R}\right|}.
\end{equation}

\noindent %
\textbf{- Robust Loss Accuracy (RLA)} is the difference between ARA and OA, indicating performance degradation on robustness data versus original data.
%\begin{tiny}
%\begin{equation}\label{eq3}
%RLA=\frac{\sum_{\left(x^{\prime}, y^{\prime}\right) \in D_{R}} %\mathds{1}\left(L M\left(x^{\prime}, y^{\prime}\right)\right.}{\left|D_{R}\right|}-\frac{\sum_{(x, y) \in D}\mathds{1}[L M(x), y]}{|D|}
%\end{equation}
%\end{tiny}
\begin{equation}\label{eq3}
RLA= OA - ARA.
\end{equation}
\noindent %
\textbf{- Consistent Robust Accuracy (CRA)} shows accuracy when the model correctly answers both original and variant data, reflecting the model do understand the problem.
% consistency in problem-solving.
\begin{equation}\label{eq4}
CRA=\frac{\sum_{x, y, x^{\prime}, y^{\prime}}\mathds{1}[L M(x), y] \cdot \mathds{1}[L M(x^{\prime}), y^{\prime}]}{\left|D_{R}\right|}.
\end{equation}
For all equation above, $D$ denotes the original dataset, where $x$ represents the input question and options, and $y$ represents the correct label, while $D_{R}$ is the robust dataset with $x^{\prime}$ and $y^{\prime}$ representing similar to $x$ and $y$.


\begin{table*}[ht]
\centering
\setlength{\tabcolsep}{5pt}
% \footnotesize
\scalebox{0.6}{
% Please add the following required packages to your document preamble:
% \usepackage{multirow}
% \usepackage[table,xcdraw]{xcolor}
% Beamer presentation requires \usepackage{colortbl} instead of \usepackage[table,xcdraw]{xcolor}
% Please add the following required packages to your document preamble:
% \usepackage{multirow}
% \usepackage[table,xcdraw]{xcolor}
% Beamer presentation requires \usepackage{colortbl} instead of \usepackage[table,xcdraw]{xcolor}
\begin{tabular}{ccccccccccccc}
\hline
\multicolumn{1}{c|}{{ }}& \multicolumn{4}{c|}{Chinese}& \multicolumn{4}{c|}{English}& \multicolumn{4}{c}{AVG}\\ \cline{2-13} 
\multicolumn{1}{c|}{\multirow{-2}{*}{{ Model}}} & { OA(\%)$\uparrow$}& { ARA(\%)$\uparrow$} & {RLA(\%)$\downarrow$}& \multicolumn{1}{l|}{{CRA(\%)$\uparrow$}} & { OA(\%)$\uparrow$}& { ARA(\%)$\uparrow$} & { RLA(\%)$\downarrow$}& \multicolumn{1}{l|}{{CRA(\%)$\uparrow$}} & {OA(\%)$\uparrow$}& { ARA(\%)$\uparrow$} & {RLA(\%)$\downarrow$}& { CRA(\%)$\uparrow$} \\ \hline
\multicolumn{1}{c|}{{ Human}} & 96.41& 97.79& -1.38 & \multicolumn{1}{l|}{92.03}& 95.56& 96.04& -0.48 & \multicolumn{1}{l|}{90.02}& 95.99 & 96.92 & -0.93& 91.03 \\ \hline
\multicolumn{13}{c}{\textit{Close-source LLMs}}\\ 
\multicolumn{1}{c|}{{ GPT-4o}}& { 91.37} & { 81.97} & { 9.40}& \multicolumn{1}{l|}{{ 75.55}} & { \textbf{88.63}} & { \textbf{70.17}} & { \textbf{18.46}} & \multicolumn{1}{l|}{{ \textbf{63.06}}} & { 90.00} & { \textbf{76.07}} & { \textbf{13.93}} & { \textbf{69.31}} \\
\multicolumn{1}{c|}{{ Claude3.5}}& { \textbf{95.37}} & { 80.15} & { 15.22} & \multicolumn{1}{l|}{{ 75.04}} & { 85.11} & { 66.02} & { 19.08} & \multicolumn{1}{l|}{{ 57.20}} & { 90.24} & { 73.09} & { 17.15} & { 66.12} \\
\multicolumn{1}{c|}{{ Gemini-1.5-Pro}}& { 90.62} & { 78.36} & { 12.26} & \multicolumn{1}{l|}{{ 70.48}} & { 87.75} & { 60.74} & { 27.01} & \multicolumn{1}{l|}{{ 58.27}} & { 89.19} & { 69.55} & { 19.63} & { 64.38} \\
\multicolumn{1}{c|}{{ Qwen-Max}}& { 93.50} & { \textbf{84.82}} & { \textbf{8.68}}& \multicolumn{1}{l|}{{ \textbf{78.91}}} & { 87.60} & { 62.61} & { 24.99} & \multicolumn{1}{l|}{{ 59.65}} & { \textbf{90.55}} & { 73.72} & { 16.83} & { 69.28} \\ \hline
\multicolumn{13}{c}{\textit{Chinese open-source LLMs}} \\ 
\multicolumn{1}{c|}{{ Qwen2.5-0.5B}}& { 60.75} & { 45.18} & { \textbf{15.57}} & \multicolumn{1}{l|}{{ 28.70}} & { 49.50} & { 38.21} & { \textbf{11.29}} & \multicolumn{1}{l|}{{ 20.57}} & { 55.13} & { 41.70} & { \textbf{13.43}} & { 24.64} \\
\multicolumn{1}{c|}{{ Qwen2.5-1.5B}}& { 63.25} & { 46.16} & { 17.09} & \multicolumn{1}{l|}{{ 29.89}} & { 56.88} & { 39.57} & { 17.30} & \multicolumn{1}{l|}{{ 23.48}} & { 60.06} & { 42.87} & { 17.20} & { 26.69} \\
\multicolumn{1}{c|}{{ Qwen2.5-3B}}& { 67.50} & { 48.75} & { 18.75} & \multicolumn{1}{l|}{{ 33.79}} & { 61.75} & { 39.98} & { 21.77} & \multicolumn{1}{l|}{{ 25.75}} & { 64.63} & { 44.37} & { 20.26} & { 29.77} \\
\multicolumn{1}{c|}{{ Qwen2.5-7B}}& { 67.63} & { 50.59} & { 17.04} & \multicolumn{1}{l|}{{ 35.62}} & { 65.63} & { 43.93} & { 21.70} & \multicolumn{1}{l|}{{ 30.77}} & { 66.63} & { 47.26} & { 19.37} & { 33.20} \\
\multicolumn{1}{c|}{{ Qwen2.5-14B}} & { 69.00} & { 51.41} & { 17.59} & \multicolumn{1}{l|}{{ 35.84}} & { 68.50} & { 45.20} & { 23.30} & \multicolumn{1}{l|}{{ 32.12}} & { 68.75} & { 48.30} & { 20.45} & { 33.98} \\
\multicolumn{1}{c|}{{ Qwen2.5-32B}} & { 69.75} & { 53.11} & { 16.64} & \multicolumn{1}{l|}{{ 37.54}} & { 70.00} & { 46.10} & { 23.90} & \multicolumn{1}{l|}{{ 32.68}} & { 69.88} & { 49.61} & { 20.27} & { 35.11} \\
\multicolumn{1}{c|}{{ Qwen2.5-72B}} & { \textbf{70.87}} & { \textbf{54.75}} & { 16.12} & \multicolumn{1}{l|}{{ \textbf{39.64}}} & { \textbf{72.00}} & { \textbf{47.75}} & { 24.25} & \multicolumn{1}{l|}{{\textbf{ 35.12}}} & { \textbf{71.44}} & { \textbf{51.25}} & {20.19} & { \textbf{37.38}} \\ \hdashline[0.5pt/5pt]
\multicolumn{1}{c|}{{ Baichuan2-7B}}& { 67.00} & { 46.16} & { 20.84} & \multicolumn{1}{l|}{{ 31.50}} & { 60.62} & { 39.04} & { 21.58} & \multicolumn{1}{l|}{{ 25.21}} & { 63.81} & { 42.60} & { 21.21} & { 28.36} \\
\multicolumn{1}{c|}{{ Baichua2-13B}}& { 69.13} & { 46.98} & { 22.15} & \multicolumn{1}{l|}{{ 33.45}} & { 64.62} & { 38.82} & { 25.80} & \multicolumn{1}{l|}{{ 26.07}} & { 66.88} & { 42.90} & { 23.97} & { 29.76} \\ \hdashline[0.5pt/5pt]
\multicolumn{1}{c|}{{ DeepSeek-7B}} & { 68.13} & { 47.96} & { 20.17} & \multicolumn{1}{l|}{{ 33.30}} & { 63.38} & { 40.39} & { 22.99} & \multicolumn{1}{l|}{{ 26.70}} & { 65.76} & { 44.18} & { 21.58} & { 30.00} \\
\multicolumn{1}{c|}{{ DeepSeek-67B}}& { 71.50} & { 49.21} & { 22.29} & \multicolumn{1}{l|}{{ 35.89}} & { 71.37} & { 40.63} & { 30.75} & \multicolumn{1}{l|}{{ 29.71}} & { 71.44} & { 44.92} & { 26.52} & { 32.80} \\ \hdashline[0.5pt/5pt]
\multicolumn{1}{c|}{{ InternLM2.5-1.8B}}& { 61.62} & { 42.07} & { 19.55} & \multicolumn{1}{l|}{{ 26.99}} & { 55.37} & { 38.46} & { 16.91} & \multicolumn{1}{l|}{{ 22.61}} & { 58.50} & { 40.27} & { 18.23} & { 24.80} \\
\multicolumn{1}{c|}{{ InternLM2.5-7B}}& { 67.25} & { 49.77} & { 17.48} & \multicolumn{1}{l|}{{ 34.57}} & { 69.50} & { 40.89} & { 28.61} & \multicolumn{1}{l|}{{ 29.75}} & { 68.38} & { 45.33} & { 23.04} & { 32.16} \\
\multicolumn{1}{c|}{{ InternLM2.5-20B}} & { 67.37} & { 48.08} & { 19.29} & \multicolumn{1}{l|}{{ 33.21}} & { 73.62} & { 41.11} & { 32.51} & \multicolumn{1}{l|}{{ 31.23}} & { 70.50} & { 44.60} & { 25.90} & { 32.22} \\ \hdashline[0.5pt/5pt]
\multicolumn{1}{c|}{{ Yi-1.5-6B}} & { 67.00} & { 49.59} & { 17.41} & \multicolumn{1}{l|}{{ 34.27}} & { 64.38} & { 39.37} & { 25.01} & \multicolumn{1}{l|}{{ 26.62}} & { 65.69} & { 44.48} & { 21.21} & { 30.45} \\
\multicolumn{1}{c|}{{ Yi-1.5-9B}} & { 68.50} & { 50.18} & { 18.32} & \multicolumn{1}{l|}{{ 35.55}} & { 66.37} & { 39.58} & { 26.79} & \multicolumn{1}{l|}{{ 27.48}} & { 67.44} & { 44.88} & { 22.56} & { 31.52} \\
\multicolumn{1}{c|}{{ Yi-1.5-34B}}& { 71.00} & { 52.23} & { 18.77} & \multicolumn{1}{l|}{{ 38.09}} & { 71.00} & { 40.75} & { 30.25} & \multicolumn{1}{l|}{{ 29.91}} & { 71.00} & { 46.49} & { 24.51} & { 34.00} \\ \hline
\multicolumn{13}{c}{\textit{English open-source LLMs}} \\ 
\multicolumn{1}{c|}{{ Llama3-8B}} & { 59.13} & { 46.62} & { 12.51} & \multicolumn{1}{l|}{{ 28.23}} & { 66.25} & { 40.21} & { 26.04} & \multicolumn{1}{l|}{{ 27.34}} & { 62.69} & { 43.42} & { 19.27} & { 27.79} \\
\multicolumn{1}{c|}{{ Llama3-70B}}& { 65.75} & { 48.63} & { 17.12} & \multicolumn{1}{l|}{{ 32.70}} & { \textbf{72.50}} & { 41.27} & { 31.23} & \multicolumn{1}{l|}{{\textbf{ 30.63}}} & {\textbf{ 69.13}} & { 44.95} & { 24.18} & { 31.67} \\ \hdashline[0.5pt/5pt]
\multicolumn{1}{c|}{{ Mistral-7B-v0.2}} & { 57.75} & { 46.25} & { \textbf{11.50}} & \multicolumn{1}{l|}{{ 27.57}} & { 67.50} & { \textbf{41.52}} & { 25.98} & \multicolumn{1}{l|}{{ 28.93}} & { 62.63} & { 43.88} & { 18.74} & { 28.25} \\
\multicolumn{1}{c|}{{ Mixtral-8x7B-v0.1}} & { 63.62} & { 46.80} & { 16.82} & \multicolumn{1}{l|}{{ 30.82}} & { 69.75} & { 41.21} & { 28.54} & \multicolumn{1}{l|}{{ 29.39}} & { 66.69} & { 44.01} & { 22.68} & { 30.11} \\
\multicolumn{1}{c|}{{ Mixtral-8x22B-v0.1}}& { 66.00} & {\textbf{ 50.73}} & { 15.27} & \multicolumn{1}{l|}{{ \textbf{34.32}}} & { 72.12} & { 41.25} & { 30.87} & \multicolumn{1}{l|}{{ 30.61}} & { 69.06} & { \textbf{45.99}} & { 23.07} & { \textbf{32.47}} \\ \hdashline[0.5pt/5pt]
\multicolumn{1}{c|}{{ Gemma-2-2B}}& { 61.88} & { 45.38} & { 16.51} & \multicolumn{1}{l|}{{ 29.02}} & { 59.62} & { 39.13} & { \textbf{20.50}} & \multicolumn{1}{l|}{{ 24.88}} & { 60.75} & { 42.25} & {\textbf{ 18.50}} & { 26.95} \\
\multicolumn{1}{c|}{{ Gemma-2-9B}}& { \textbf{69.13}} & { 46.75} & { 22.38} & \multicolumn{1}{l|}{{ 33.29}} & { 64.88} & { 39.80} & { 25.08} & \multicolumn{1}{l|}{{ 26.91}} & { 67.01} & { 43.28} & { 23.73} & { 30.10} \\
\multicolumn{1}{c|}{{ Gemma-2-27B}} & { 63.38} & { 48.52} & { 14.86} & \multicolumn{1}{l|}{{ 31.96}} & { 71.88} & { 40.91} & { 30.97} & \multicolumn{1}{l|}{{ 30.25}} & { 67.63} & { 44.71} & { 22.92} & { 31.11} \\ \hline
\end{tabular}
}
\caption{TODO: bolded is not result. Results of existing LLMs on our HellaSwag-Pro dataset using \textbf{Direct} prompt. ``AVG'' indicates the average performance of each model on Chinese and English parts of the dataset.
The best results for each metric in each model category are \textbf{bolded}. }
\label{tab:main experiment.}
\end{table*}

\subsection{Model Performance (RQ1)}
\paragraph{Overall Performance}
Table \ref{tab:main experiment.} provides a comprehensive evaluation of various LLMs across four performance metrics\footnote{The results of instruct/chat models of Qwen2.5, Llama3 and Mixtral latest series are shown in Appendix.}. The main observations are as follow:
\begin{itemize}[leftmargin=*,topsep=0pt]
% \setlength{}{0}
    \item Upon evaluating all available models, we found that all performed well in overall accuracy (e.g., GPT-4 scored 90.00 in AVG OA, Claude 3.5 scored 90.24 in AVG OA). However, all models struggled with variations of the questions, as evidenced by a positive RLA value for each model. In contrast, humans received a negative RLA value, suggesting that the question variants were not more challenging than the originals. This disparity further illustrates that current LLMs lack a true understanding of the reasoning process and can easily be misled by question variants.
    \item When comparing open-source and close-source models, the close-source models demonstrate stronger capabilities in both OA and ARA scores, similar to most existing benchmarks. Overall, the RLA values for close-source models are also smaller, indicating that they are more robust in commonsense reasoning tasks compared to open-source models.
    \item When we compare models within the same series (e.g., Qwen, Llama), we observe that larger models often achieve higher scores on OA, ARA, and CRA. However, they are also more susceptible to variations, i.e., they have higher RLA values, a phenomenon particularly evident in English datasets. We attribute this phenomenon to the fact that larger models, compared to smaller ones, may have memorized more data, allowing them to rely on memorization to solve some problems more easily and making them more prone to the influence of variations~\cite{}.
\end{itemize}
% 1. When evaluating all available models, We find although 
% 2. When comparing the opensource LLMs and close source LLMs, 
% 3. When looking into each serious details
% \noindent
% \textbf{Overall Model Performance.}
% 1. close-source > open-source 2. the large the better 3. all have a performance decline when meeting varients.

% To evaluate the performance of various models, we observed patterns consistent with current mainstream trends: closed-source models generally outperform open-source models across metrics. 
% For instance, the closed-source model GPT-4o achieved scores of 90.00 in OA, 76.07 in ARA, and 69.31 in CRA, whereas the open-source model Qwen2.5-72B scored 71.44, 51.25, and 37.38, respectively. 
% Furthermore, within each model series, performance tends to improve with larger model sizes. 
% Nevertheless, even the strongest closed-source models struggle with variations in questions, as indicated by positive values in RLA for all models. In contrast, human performance yields a negative RLA value, highlighting that current LLMs do not genuinely grasp the reasoning process and are prone to falling into traps set by question variants. 
% This suggests that there is still significant room for improvement in developing models that can robustly understand and reason through complex linguistic challenges.
% It reveals a consistent pattern across Chinese, English, and average scores, with close-sourced LLMs generally outperforming open-sourced models. 
% However, all models exhibit a significant drop in performance when faced with robust variants, as indicated by RLA and CRA. Among closed-source models, GPT-4o demonstrates the highest ARA of 76.07\% in average scores, demonstrating its overwhelming superiority. Among open-sourced models, larger models tend to perform better, with Qwen2.5-72B achieving the highest OA (71.44\%) and ARA (51.25\%) in the average scores. However, even these top performers still struggle with robustness, as evidenced by the substantial RLA of 13.93\% for GPT-4o and 20.19\% for Qwen2.5-72B. Interestingly, some English open-sourced models, such as Llama3-70B and Mixtral-8x22B-v0.1, show competitive performance in English tasks but lag in Chinese tasks, highlighting the importance of language-specific training.

% \noindent
% \textbf{Chinese Models vs English Models.}
% Chinese models generally demonstrate higher OA in Chinese tasks compared to English tasks, with Qwen-Max achieving 93.50\% OA in Chinese versus 87.60\% in English. Conversely, English models tend to perform better in English tasks, exemplified by Llama3-70B's 72.50\% OA in English compared to 65.75\% in Chinese. 
% However, both Chinese and English models exhibit important drops in ARA across languages, indicating challenges in maintaining performance when faced with variations. This trend suggests that while models may excel in their primary language, they struggle with robustness across linguistic boundaries. 
% Notably, larger models tend to achieve higher ARA scores but also experience more substantial RLA, as seen with Qwen2.5-0.5B (41.70\% ARA, 13.43\% RLA in total) and Qwen2.5-72B (51.25\% ARA, 20.19\% RLA in total). 
% This pattern indicates that while increased model size enhances overall performance, it doesn't necessarily improve robustness proportionally. 
% The discrepancy between OA and ARA across languages underscores the need for improved cross-lingual robustness in language models, particularly as they scale in size and capability.


% \noindent
% \textbf{Comparison between Chinese and English datasets.}
% Generally, models demonstrate higher accuracy on the Chinese dataset compared to the English one, as evidenced by the consistently higher OA, ARA and CRA scores. For instance, GPT-4o achieves an OA of 91.37\%, an ARA of 81.97\% , an CRA of 75.55\% on the Chinese dataset, compared to 88.63\% and 70.17\% respectively on the English dataset. This trend is observed across most models, suggesting that the Chinese dataset is easier than English one. Moreover, the RLA values are typically lower for Chinese, indicating smaller performance drops when dealing with robust variants of Chinese questions. For example, Qwen-Max shows an RLA of 8.68\% for Chinese versus 24.99\% for English, highlighting a more consistent performance in Chinese. The CRA scores further reinforce this observation, with models generally maintaining higher consistency in correct answers for both original and variant Chinese questions.
% We attribute this phenomenon to the fact that blablabla

\noindent
\textbf{Reasoning Transferable Capability.}
% 为了进一步
To further analyze whether the model can transfer reasoning ability from the original question to its variant, Figure \ref{consis} presents the distribution of model performance on the original question and variant pairs. For all models, the pairs of (HellaSwag \ding{51} HellaSwag-Pro \ding{55}) occupy a significant proportion, indicating a challenge in transferring reasoning capabilities for current LLMs to more complex scenarios. Looking deeply, closed-source models like GPT-4 and Qwen-Max achieve around a 69\% portion of (HellaSwag \ding{51} HellaSwag-Pro \ding{51}) and a 3\% portion of (HellaSwag \ding{55} HellaSwag-Pro \ding{55}), while in contrast, open-source models struggle with around a 30\% portion of (HellaSwag \ding{51} HellaSwag-Pro \ding{51}) and a 20\% portion of (HellaSwag \ding{55} HellaSwag-Pro \ding{55}), further indicating the robustness of reasoning abilities in closed-source models.
% If a model can get both the original question and the variant right, we consider it to have transferable reasoning ability. Table \ref{consis} presents the distribution of model performance on the original question and variant pairs. Among all models, the pairs of (HellaSwag \ding{51}HellaSwag-Pro \ding{55}) account for a considerable proportion, i 
% The closed-source models like GPT-4o and Qwen-Max achieve around 69\% portion of (HellaSwag \ding{51}HellaSwag-Pro \ding{51}) and 3\% portion of (HellaSwag \ding{55} HellaSwag-Pro \ding{55}), indicating stronger reasoning transfer ability than other models. In contrast, open-source models struggle more, with around 30\% portion of (HellaSwag \ding{51}HellaSwag-Pro \ding{51}) and 20\% portion of (HellaSwag \ding{55} HellaSwag-Pro \ding{55}). 
% A notable trend is observed among the Qwen2.5 series, where increasing model size from 7B to 72B parameters correlates with improved performance on correct answers for both datasets (33.20\% to 37.38\%) and decreased failure rates (17.69\% to 14.7\%). It underscores the importance of model size in commonsense reasoning tasks.

\begin{figure}[t]
\centering
\setlength{\abovecaptionskip}{0.1cm}
\setlength{\belowcaptionskip}{0cm}
\includegraphics[width=\linewidth,scale=1.00]{images/consis.pdf}
\caption{Analysis of the transferable ability of model reasoning based on question pair performance. The green part, where both the original and the variant data are right, represents the transferable performance of model reasoning.}
\label{consis}
\vspace{-15pt}
\end{figure}

\begin{figure*}[ht]
\centering
\setlength{\abovecaptionskip}{0.1cm}
\setlength{\belowcaptionskip}{0cm}
\includegraphics[width=\linewidth,scale=1.00]{images/xing.pdf}
\caption{The impact of different few-shot prompts on model performance. With - as the separator, the first two parts of the legend represent the prompt name, and the third part represents the language of the dataset.}
\label{xing}
\vspace{-15pt}
\end{figure*}

\begin{figure}[ht]
\centering
\setlength{\abovecaptionskip}{0.1cm}
\setlength{\belowcaptionskip}{0cm}
\includegraphics[width=1.05\linewidth,scale=1.05]{images/zhu.pdf}
\caption{The RLA Distribution for 7 variants of commonsense reasoning. Parts below the 0 axis indicate that the model’s performance on the variant is improved compared to the original problem.}
\label{fig:zhu}
\vspace{-15pt}
\end{figure}


\subsection{Variant Analysis (RQ2)}
To further analyze the impact of different variants, we assessed the contribution of each variant to the RLA score. A higher contribution indicates that the model is more likely to make errors in that type. Figure~\ref{fig:zhu} presents the overall results, and the key observations are as follows:
\begin{itemize}[leftmargin=*]
    \item For problem restatement, causal inference, and sentence ordering, these three categories are the least challenging. Almost all models, particularly the close-source and Qwen series models, perform well on these variants, indicating that current LLMs can effectively handle these forms and we do not pay more attention on this kind of varients.
    \item For reverse conversion and critical testing, these two varients each contribute about 10\% to the RLA score. This indicates that current LLMs struggle to fully generalize to these simple scenarios, possibly because these types of questions are not commonly encountered, and reaserchers should pay some attention to this type of varients.
    \item For negative transformation and scenario refinement, this are the two most difficult tasks, with negative transformation being particularly challenging. For almost all models, these two varients accounts for more than 50\% of the RLA score. This may be due to intuitively counterintuitive questions—such as the use of "will not"  or counterfactual scenarios in scenario refinement. These setups are less common in LLM training data and cannot be easily tackled through memory alone. Only those LLMs which truely understand the question could answer the varient correctly, wihch better reflect the true performance of the model.. In the future, researchers should focus more on enhancing LLM's capability to address such types of questions.
\end{itemize}

% 1. Problem restCausal Inference 
% To further analysis the impact of different varients, we further 
% Figure \ref{fig: zhu} presents a comprehensive analysis of various LLMs' performance across different variant types. Negative transformation emerges as the most challenging task for all models, with scores consistently above 50.00\% and peaking at 78.38\% for Gemini-1.5-Pro. Conversely, problem restatement appears to be the least challenging, with most models scoring in the negative range. Intriguingly, smaller models like Qwen2.5-0.5B demonstrate unexpected strengths in certain areas, such as sentence sorting (7.75\%), outperforming some larger counterparts. A detailed analysis of each variant type follows.

% \noindent
% \textbf{Causal inference.} In this category, scores vary widely from -4.73\% for Qwen-Max to 12.25\% for Baichuan2-13B, illustrating differing degrees of sensitivity to causal reasoning among the models. Smaller models, such as Qwen2.5-0.5B and Qwen2.5-1.5B, achieve better scores, indicating relatively stronger robustness in causal reasoning. Conversely, larger models, like Baichuan2-13B, have higher scores, suggesting greater sensitivity to the challenges of inferring causality.

% \noindent
% \textbf{Critical testing.} Larger models, including Qwen2.5-72B and DeepSeek-67B, exhibit higher RLA scores of 30.50\% and 31.37\%, respectively, suggesting increased sensitivity when dealing with incomplete key information. In contrast, GPT-4o achieves the lowest score, highlighting its superior robustness in critical reasoning. This trend indicates that more complex models might struggle to handle incomplete contexts, underscoring potential areas for improvement in sophisticated architectures.

% \noindent
% \textbf{Negative transformation.} This aspect remains consistently challenging for all models, with scores ranging from 48.88\% to 78.38\%. Advanced commercial models like Gemini-1.5-Pro and Claude-3.5 also score higher (78.38\% and 76.43\%, respectively), indicating a prevalent sensitivity issue in reasoning processes when handling negations, irrespective of model size or architecture.

% \noindent
% \textbf{Problem restatement.} The negative values in this category for nearly all models suggest it is not particularly challenging. This is surprising, given that previous models were quite sensitive to sentence representation.

% \noindent
% \textbf{Reverse conversion.} This variation, which involves swapping the roles of the question and answer, seems to specifically impact larger models. For example, Qwen2.5-72B and DeepSeek-67B exhibit higher RLA scores of 24.38\% and 27.43\%, respectively, indicating heightened sensitivity to reverse reasoning compared to their performance on original questions.

% \noindent
% \textbf{Scenario refinement.} The scores range from 16.06\% for Gemma-2-2B to 32.56\% for Qwen2.5-72B, with larger models displaying more sensitivity in adapting to counterfactual predictions. This suggests that larger models may rely more heavily on general commonsense rather than flexibly adapting to specific contexts. Consequently, increased model complexity might adversely affect adaptability to scenario changes, underscoring the need for enhanced flexibility in advanced models.

% \noindent
% \textbf{Sentence sorting.} This category exhibits the most varied results across models. Some larger models like DeepSeek-67B and InternLM2.5-20B display higher scores (26.69\% and 26.68\%), indicating sensitivity, while others like Qwen2.5-72B and Gemini-1.5-Pro excel with lower scores (-9.88\% and -1.07\%, respectively). This suggests that sentence sorting ability may depend more on specific training approaches rather than being solely contingent on model size.


\subsection{Prompt Robustness (RQ3)}
% To investigate how prompt  influence our benchmark, we apply sereral prompt strategy on our datasets and showcase the average performance of all models on different kind of prompt strategies.
% Table~\ref{prompt} illustrates the final results. For both Chinese and English datasets, CN LLMs achieve the highest performance using CN-CoT-Few-Shot, followed closely by EN-CoT-Few-Shot, with overall performance scores of 67.36\% and 67.03\%, respectively. In contrast, English LLMs perform best with the EN-CoT-Few-Shot, reaching 67.55\% on the Chinese dataset and 60.36\% on the English dataset.
% Contrary to previous findings, translating the dataset to the model's advantage language before performing reasoning does not enhance performance. Moreover, Figure~\ref{xing} also shows the similar phenomenon. Conducting CoT reasoning in the model’s advantage language generally leads to better outcomes compared to Direct. Additionally, increasing the number of shots consistently improves performance across most configurations, highlighting the benefits of exposing models to multiple examples. 
To explore the impact of various prompt strategies on our benchmarks, we evaluated several approaches across our datasets and present the average performance of all models using different prompting techniques. Table~\ref{prompt} summarizes the results. For both Chinese and English datasets, Chinese LLMs performed best with the CN-CoT-Few-Shot strategy, followed closely by EN-CoT-Few-Shot, achieving overall scores of 67.36\% and 67.03\%, respectively. Conversely, English LLMs showed optimal performance with the EN-CoT-Few-Shot approach, attaining 67.55\% on the Chinese dataset and 60.36\% on the English dataset.
Besides, translating datasets into the model's native language before reasoning did not enhance performance. This phenomenon is further illustrated in Figure~\ref{xing}. Conducting CoT reasoning in the model's native language generally yields better results compared to direct reasoning. Furthermore, increasing the number of examples (shots) consistently boosts performance across most configurations, emphasizing the advantages of exposing models to multiple examples.
% Overall, the interaction between question language, prompt language, and the number of shots underscores the importance of aligning these factors to optimize task performance and robustness in LLMs.



% Please add the following required packages to your document preamble:
% \usepackage{multirow}
% Please add the following required packages to your document preamble:
% \usepackage{multirow}
\begin{table}[t]
\setlength{\tabcolsep}{8pt}
% \footnotesize
\scalebox{0.65}{
\begin{tabular}{c|l|lll}
\hline
\multicolumn{1}{l|}{Dataset}  & Prompt  & CN LLMs & EN LLMs &  LLMs \\ \hline
\multirow{7}{*}{\begin{tabular}[c]{@{}c@{}}Chinese\\ HellaSwag-Pro\end{tabular}} & Direct  & 48.95& 41.16& 45.06  \\
& CN-CoT-Few  & \textbf{71.04}& 51.90& 61.47  \\
& EN-CoT-Few  & 70.95& \textbf{67.55}& \textbf{69.25}  \\
& EN-XLT-Few  & 41.48& 28.69& 35.09  \\
& CN-CoT-Zero & 44.82& 23.89& 34.36  \\
& EN-CoT-Zero & 45.38& 31.39& 38.39  \\
& EN-XLT-Zero & 28.57& 12.93& 20.75  \\ \hline
\multirow{7}{*}{\begin{tabular}[c]{@{}c@{}}English\\ HellaSwag-Pro\end{tabular}} & Direct  & 47.46& 40.66& 44.06  \\
& CN-CoT-Few  & \textbf{63.67}& 47.24& 55.46  \\
& EN-CoT-Few  & 63.12& \textbf{60.36}& \textbf{61.74}  \\
& CN-XLT-Few  & 48.77& 16.61& 32.69  \\
& CN-CoT-Zero & 34.89& 18.25& 26.57  \\
& EN-CoT-Zero & 42.41& 31.03& 36.72  \\
& CN-XLT-Zero & 16.36& 11.22& 13.79  \\ \hline
\multirow{9}{*}{HellaSwag-Pro}& Direct  & 48.21& 40.91& 44.83  \\
& CN-CoT-Few  & \textbf{67.36}& 49.57& 58.46  \\
& EN-CoT-Few  & 67.03& \textbf{63.95}& \textbf{65.49}  \\
& CN-XLT-Few  & 59.91& 34.26& 47.08  \\
& EN-XLT-Few  & 52.30& 44.52& 48.41  \\
& CN-CoT-Zero & 39.86& 21.07& 30.46  \\
& EN-CoT-Zero & 43.90& 31.21& 37.55  \\
& CN-XLT-Zero & 30.59& 17.55& 24.07  \\
& EN-XLT-Zero & 35.49& 21.98& 28.74  \\ \hline
\end{tabular}
}
\caption{Average ARA of all open-source models on different prompts. CN-LLMs contains 17 LLMs, and EN-LLMs contains 7 LLMs. The bast results for each dataset are \textbf{bolded}.}
\label{prompt}
\end{table}




\section{Concluding Remarks}
In this paper, we proposed a novel approach utilizing multimodal LLMs to generate gesture-aware speech recognition transcripts for patients with language disorders. Our framework integrates verbal speech and iconic gestures, enabling the generation of enriched transcripts that capture the latent meaning conveyed through both modalities. Through extensive experimentation, we demonstrated that the proposed method effectively contextualizes incomplete or disfluent speech by incorporating gesture information, leading to more accurate and meaningful representations of the speaker's intent. These findings highlight the potential of our approach to significantly contribute to the field of speech and language therapy, offering innovative tools that can enhance the quality of life for individuals with language disorders by facilitating better communication and assessment methods.

\subsection{Ethical Statement} 
Our dataset was obtained from AphasiaBank with the approval of the Institutional Review Board (IRB) and adheres to the data sharing guidelines set by TalkBank\footnote{https://talkbank.org/share/ethics.html}. This includes complying with the Ground Rules for all TalkBank databases, which are based on the American Psychological Association Code of Ethics~\cite{american2002ethical}.

\subsection{Limitation \& Future Work} 
%This study represents a preliminary investigation into using multimodal LLMs to generate gesture-aware speech recognition transcripts. 
While the results are promising, we recognize several limitations and outline our plans to extend this work further.

One primary limitation is the absence of a definitive ground truth for quantitative evaluation. Since our model generates transcripts by synthesizing speech and gesture data from scratch, traditional benchmarks, such as comparisons with standard speech recognition outputs, are insufficient. Moreover, existing original transcripts lack gesture annotations, making direct comparisons challenging. In future work, we aim to address this gap by collaborating with certified pathologists to conduct qualitative assessments, such as A-B preference tests, to evaluate the effectiveness of gesture-enriched transcripts in accurately conveying the speaker's intentions.

To support quantitative evaluations, we plan to develop novel metrics that assess transcript quality, including grammar accuracy, semantic consistency, and the integration of multimodal information. Such metrics will provide a more objective basis for assessing our model's performance and facilitate comparisons with other multimodal and unimodal approaches.

Another limitation of this study is its focus on structured gestures from a specific task, the Peanut Butter Sandwich Task. While this task offers a controlled context for testing our approach, it does not encompass the diversity of gestures and communication patterns seen in everyday scenarios. As part of our future work, we plan to expand the scope of our model to include tasks such as the Cinderella Story Recall Task~\cite{bird1996cinderella}, which involves unstructured and complex narrative gestures. This expansion will allow us to evaluate the adaptability and robustness of our model in handling varied linguistic and gestural contexts.

In summary, while this study establishes a strong foundation for gesture-aware speech recognition, we aim to refine and extend our methods through collaborative qualitative evaluations, the development of robust quantitative metrics, and broader task applications. These efforts will ensure that our approach continues to evolve, ultimately contributing to more effective communication tools and interventions for individuals with language disorders.




\bibliographystyle{ACM-Reference-Format}
\balance
\bibliography{ref}
\end{document}
