\PassOptionsToPackage{table}{xcolor}
\documentclass[10pt,twocolumn,letterpaper]{article}

% \usepackage{iccv}              % To produce the CAMERA-READY version
% \usepackage[review]{iccv}      % To produce the REVIEW version
\usepackage[pagenumbers]{iccv} % To force page numbers, e.g. for an arXiv version
% \usepackage[table]{xcolor}

% Import additional packages in the preamble file, before hyperref
\newcommand{\CG}{\mathcal{G}\xspace}
\newcommand{\CV}{\mathcal{V}\xspace}
\newcommand{\CE}{\mathcal{E}\xspace}
\newcommand{\CA}{\mathcal{A}\xspace}
\newcommand{\CF}{\mathcal{F}\xspace}
\newcommand{\CR}{\mathcal{R}\xspace}
\newcommand{\CB}{\mathcal{B}\xspace}
\newcommand{\CX}{\mathcal{X}\xspace}
\newcommand{\CK}{\mathcal{K}\xspace}
\newcommand{\CM}{\mathcal{M}\xspace}
\newcommand{\CC}{\mathcal{C}\xspace}
\newcommand{\CL}{\mathcal{L}\xspace}
\newcommand{\CI}{\mathcal{I}\xspace}
\newcommand{\CQ}{\mathcal{Q}\xspace}
\newcommand{\CO}{\mathcal{O}\xspace}
\newcommand{\CP}{\mathcal{P}\xspace}
\newcommand{\CS}{\mathcal{S}\xspace}
\newcommand{\CT}{\mathcal{T}\xspace}
\newcommand{\CJ}{\mathcal{J}\xspace}
\usepackage[para]{footmisc}
\usepackage{subfig}
% \usepackage{subcaption}
% \usepackage{array}
% \usepackage{colortbl}



% It is strongly recommended to use hyperref, especially for the review version.
% hyperref with option pagebackref eases the reviewers' job.
% Please disable hyperref *only* if you encounter grave issues, 
% e.g. with the file validation for the camera-ready version.
%
% If you comment hyperref and then uncomment it, you should delete *.aux before re-running LaTeX.
% (Or just hit 'q' on the first LaTeX run, let it finish, and you should be clear).
\definecolor{iccvblue}{rgb}{0.21,0.49,0.74}
\usepackage[pagebackref,breaklinks,colorlinks,allcolors=iccvblue]{hyperref}

\usepackage{booktabs} % for professional tables


\usepackage{pifont}

\definecolor{mygray}{gray}{.95}
\definecolor{highlight}{RGB}{238,250,215}
\definecolor{lightgray}{gray}{0.7}
\definecolor{lightorange}{HTML}{C47955}
\definecolor{darkgreen}{HTML}{195228}
\definecolor{eccvblue}{rgb}{0.12,0.49,0.85}
\definecolor{lightblue}{rgb}{0.12,0.49,0.85}

%%%%%%%%% PAPER ID  - PLEASE UPDATE
\def\paperID{*****} % *** Enter the Paper ID here
\def\confName{ICCV}
\def\confYear{2025}

%%%%%%%%% TITLE - PLEASE UPDATE
\title{PRVQL: Progressive Knowledge-guided Refinement for Robust Egocentric \\ Visual Query Localization}

%%%%%%%%% AUTHORS - PLEASE UPDATE
\author{Bing Fan$^{1}$\;\;\;\;\;Yunhe Feng$^{1}$\;\;\;\;\;Yapeng Tian$^{2}$\;\;\;\;\;Yuewei Lin$^{3}$\;\;\;\;\; Yan Huang$^{1}$\;\;\;\;\;Heng Fan$^{1}$\\
$^{1}$University of North Texas \;\;\;\;\; $^{2}$University of Texas at Dallas\;\;\;\;\; $^{3}$Brookhaven National Laboratory 
% $^{1}$Department of Computer Science and Engineering, University of North Texas \\ $^{2}$Department of Computer Science, University of Texas at Dallas\;\; $^{3}$Brookhaven National Laboratory 
% For a paper whose authors are all at the same institution,
% omit the following lines up until the closing ``}''.
% Additional authors and addresses can be added with ``\and'',
% just like the second author.
% To save space, use either the email address or home page, not both
% \and
% Second Author\\
% Institution2\\
% First line of institution2 address\\
% {\tt\small secondauthor@i2.org}
}

\begin{document}
\maketitle


\begin{abstract}
Egocentric visual query localization (\emph{EgoVQL}) focuses on localizing the target of interest in space and time from first-person videos, given a visual query. Despite recent progressive, existing methods often struggle to handle severe object appearance changes and cluttering background in the video due to lacking sufficient target cues, leading to degradation. Addressing this, we introduce \textbf{PRVQL}, a novel \textbf{P}rogressive knowledge-guided \textbf{R}efinement framework for Ego\textbf{VQL}. The core is to continuously exploit target-relevant knowledge directly from videos and utilize it as guidance to refine both query and video features for improving target localization. Our PRVQL contains multiple processing stages. The target knowledge from one stage, comprising appearance and spatial knowledge extracted via two specially designed knowledge learning modules, are utilized as guidance to refine the query and videos features for the next stage, which are used to generate more accurate knowledge for further feature refinement. With such a progressive process, target knowledge in PRVQL can be gradually improved, which, in turn, leads to better refined query and video features for localization in the final stage. Compared to previous methods, our PRVQL, besides the given object cues, enjoys additional crucial target information from a video as guidance to refine features, and hence enhances EgoVQL in complicated scenes. In our experiments on challenging Ego4D, PRVQL achieves state-of-the-art result and largely surpasses other methods, showing its efficacy. Our code, model and results will be released at \url{https://github.com/fb-reps/PRVQL}.
\end{abstract}

\section{Introduction}
\label{intro}

The egocentric visual query localization (EgoVQL) task~\cite{grauman2022ego4d} aims at answering the question ``\emph{Where was the object X last seen in the video?}'', with ``\emph{X}'' being a visual query specified by a single image crop outside the search video. In specific, given a first-person video, its goal is to search and locate the visual query, \emph{spatially} and \emph{temporally}, by returning the most recent spatio-temporal tube. Owing to its important roles in numerous downstream object-centric applications including augmented and virtual reality, robotics, human-machine interaction, and so on, EgoVQL has drawn extensive attention from researchers in recent years.

\begin{figure}[!t]
    \centering
    \includegraphics[width=\linewidth]{figs/fig1.pdf}\vspace{-1mm}
    \caption{Comparison between current EgoVQL approaches in (a) and proposed PRVQL with progressive knowledge-guided refinement in (b). \emph{Best viewed in color and by zooming in for all figures}.}
    \label{fig:framework_comparison}\vspace{-2mm}
\end{figure}

Current approaches (\eg,~\cite{xu2022negative,xu2023my,jiang2024single,grauman2022ego4d}) simply leverage the provided visual query as the \emph{sole} cue to locate the target in the video (see Fig.~\ref{fig:framework_comparison} (a)). However, since the given visual query is cropped \emph{outside} the search video, there often exists a \emph{significant gap} between the query and the target of interest, due to rapid appearance variations in first-person videos caused by many factors, such as object pose change, motion blur, occlusion, and so forth. As a result, relying only on the given object query, as in existing methods, is \emph{insufficient} to describe and distinguish the target from background in complicated scenarios with heavy appearance changes, resulting in performance degeneration. In addition, to achieve precise localization, it is essential for an EgoVQL model to enhance target and meanwhile suppressing background regions from videos. Yet, this is often \emph{overlooked} by existing approaches, making them easily suffer from cluttering background and therefore leading to suboptimal target localization.

The aforementioned issues faced by current methods naturally raise a question: \emph{In addition to the given visual query, is there any other information that could be leveraged for enhancing EgoVQL}? We answer \emph{\textbf{yes}}, and show the information directly explored from the \emph{video itself}, as a supplement to the given target cue, is \emph{effective} in improving EgoVQL.

Specifically, we propose a novel \emph{\textbf{P}rogressive knowledge-guided \textbf{R}efinement framework for Ego\textbf{VQL}} (\textbf{\emph{PRVQL}}). The core idea of our algorithm is to continuously exploit target-relevant knowledge from the video and leverage it to guide refinements of both query and video features, which are crucial for localization, for improving EgoVQL (see Fig.~\ref{fig:framework_comparison} (b)). Concretely, PRVQL consists of multiple processing stages. Each stage comprises two simple yet effective modules, including \emph{appearance knowledge generation} (AKG) and \emph{spatial knowledge generation} (SKG). In specific, AKG works to mine visual information of the target from videos as the appearance knowledge. It first estimates potential target regions from a video using the query, and then selects top few highly confident regions to extract appearance knowledge from video features. Different from AKG, SKG focuses on exploring target position cues from videos as spatial knowledge by exploiting the readily available target-aware attention maps. In PRVQL, the appearance knowledge is used to guide the update of query feature, making it more discriminative, while the spatial knowledge is employed to enhance target and meanwhile suppressing unconcerned background in video features, enabling more focus on the target. The extracted appearance and spatial knowledge in one stage are used as guidance to respectively refine query and video features for next stage, which are adopted to learn more accurate knowledge for further feature refinement. Through this progressive process in PRVQL, the target knowledge can be gradually improved, which, in turn, results in better refined query and video features for target object localization in the final stage. Fig.~\ref{fig:framework} illustrates the framework of PRVQL.

To our best knowledge, PRVQL is the \emph{first} method to exploit target-relevant appearance and spatial knowledge from the video to improve EgoVQL. Compared with existing solutions, PRVQL can leverage target information from both the given visual query and mined knowledge from the video for more robust localization. To verify its effectiveness, we conduct experiments on the challenging Ego4D~\cite{grauman2022ego4d}, and our proposed PRVQL achieves state-of-the-art performance and largely outperforms other approaches, evidencing effectiveness of target knowledge for enhancing EgoVQL. 

In summary, our main contributions are as follows: \ding{171} We propose a progressive knowledge-guided refinement framework, dubbed PRVQL, that exploits knowledge from videos for improving EgoVQL; \ding{170} We introduce AKG for exploring visual information of target as appearance knowledge; \ding{168} We introduce SKG for learning spatial knowledge using target-aware attention maps; \ding{169} In our extensive experiments on the challenging Ego4D, PRVQL achieves state-of-the-art performance and largely surpasses existing methods. 

\section{Related Work}

% In this section, we discuss works that are closely relevant to our approach from the following three lines.

% \vspace{0.5em}
% \noindent
\textbf{Egocentric Visual Query Localization.} Egocentric visual query localization (EgoVQL) is an emerging and important computer vision task. Since its introduction in~\cite{grauman2022ego4d}, EgoVQL has received extensive attention in recent years owing to its importance in numerous applications. Early methods~\cite{grauman2022ego4d,xu2022negative,xu2023my} often utilize a bottom-up multi-stage framework, which sequentially and independently performs frame-level object detection, nearest peak temporal detection across the video, and bidirectional object tracking around the peak, to achieve EgoVQL. Despite the straightforwardness, this bottom-up design easily causes compounding errors across stages, thus degrading performance. Besides, the involvement of multiple detection and tracking components in this design leads to high complexities as well as inefficiency of the entire system, limiting its practicability. To deal with these issues, the recent method of~\cite{jiang2024single} introduces a single-stage end-to-end framework for EgoVQL with Transformer~\cite{VaswaniSPUJGKP17}, eliminating the need for multiple components and meanwhile showing promising target localization performance.

In this work, we propose to exploit target knowledge directly from the video and utilize it as guidance to refine features for better localization. \textbf{\emph{Different}} from aforementioned approaches~\cite{grauman2022ego4d,xu2022negative,xu2023my,jiang2024single} which mainly explore the object information from only the provided query for localization, PRVQL is able to leverage cues from both the given query and mined target information for EgoVQL, significantly improving robustness, especially in presence of severe appearance variations and cluttering background.

\begin{figure*}[!t]
	\centering
        \includegraphics[width=1\textwidth]{figs/fig2.pdf}\vspace{-2mm}
	\caption{Overview of PRVQL, which aims to explore target knowledge directly from videos via AKG and SKG and applies it as guidance to refine query and video features with QFR and VFR for improving localization in EgoVQL through a multi-stage progressive architecture.}
	\label{fig:framework}\vspace{-4mm}	
\end{figure*}

\vspace{0.5em}
\noindent
\textbf{Query-based Visual Localization.} Query-based visual localization, broadly referring to localizing the target of interest from images or videos given a specific query (image or text), is a crucial problem in computer 
vision, and consists of a wide range of related tasks, including one-shot object detection~\cite{hsieh2019one,yang2022balanced,zhao2022semantic}, visual object tracking~\cite{chen2023seqtrack,lin2025tracking,bertinetto2016fully}, visual grounding~\cite{deng2021transvg,liu2025grounding,zhu2022seqtr}, spatio-temporal video grounding~\cite{yang2022tubedetr,gu2024context}, pedestrian search~\cite{li2017person,yu2022cascade}, \etc. Despite sharing some similarity with the above tasks in localizing the target, this work is \textbf{\emph{distinctive}} by focusing on spatially and temporally searching for the target from egocentric videos, which is challenging due to frequent and heavy object appearance variations under the first-person views.

\vspace{0.5em}
\noindent
\textbf{Progressive Learning Approach.} Multi-stage progressive learning is a popular strategy to improve performance, and has been successfully applied for various tasks. For example, the works of~\cite{cai2018cascade,ye2023cascade,vu2019cascade} introduce the cascade architecture to progressively refine the bounding boxes or features for improving object detection. The work in~\cite{yang2019step} presents a sptio-temporal progressive network for video action detection. The methods in~\cite{huynh2021progressive,zhao2018icnet} introduce progressive refinement network for multi-scale semantic segmentation. The methods of~\cite{zhang2018progressive,chen2020progressively} apply progressive learning to improve features for saliency detection. The method in~\cite{fan2019siamese} proposes to progressively learn more accurate anchors for enhancing tracking. The work from~\cite{zhu2019progressive} progressively transfers person pose for image generation. \textbf{\emph{Different}} than these works, we focus on progressive refinement for improving EgoVQL.

\section{The Proposed Method}

\textbf{Overview.} In this paper, we propose PRVQL by exploiting crucial target knowledge directly from videos for improving target localization in EgoVQL. Our PRVQL is implemented as a progressive architecture. After feature extraction of the visual query and video frames, PRVQL performs iterative feature refinement guided by the target knowledge for localization through multiple stages (Sec.~\ref{prvql}). As displayed in Fig.~\ref{fig:framework}, each stage, expect for the final stage for prediction, consists of two crucial modules, comprising AKG (Sec.~\ref{akg}) and SKG (Sec.~\ref{skg}), for generating target appearance and spatial knowledge. The knowledge is leveraged as the guidance to refine query and video features (Sec.~\ref{update}), which are applied in the next stage to generate more accurate target knowledge for further feature refinement. Through such a progressive process, the target knowledge can be gradually enhanced, which finally benefits learning more discriminative query and video features for improving EgoVQL.

\subsection{Our PRVQL Framework}\label{prvql}

\textbf{Visual Feature Extraction.} In our PRVQL, we first extract features for the visual query and video frames. Specifically, given the query $q$ and a sequence of $L$ frames $\mathcal{I}=\{I_i\}_{i=1}^{L}$ from the video, we utilize a shared backbone $\Phi(\cdot)$~\cite{OquabDMVSKFHMEA24} for extracting their features $\textbf{q}=\Phi(q) \in \mathbb{R}^{H\times W\times C}$ and $F=\{\textbf{f}_i\}_{i=1}^{L}$ with each $\textbf{f}_i=\Phi(I(i)) \in \mathbb{R}^{H\times W\times C}$, where the $H$ and $W$ represent the spatial resolution of the features and $C$ denotes the channel dimension. For subsequent processing, we flatten $\textbf{q}$ and $F$ to obtain $\textbf{Q}=\mathtt{flatten}(\textbf{q}) \in \mathbb{R}^{HW\times C}$ and $\textbf{V}=\{\textbf{v}_i\}_{i=1}^{L}$ with each $\textbf{v}_i \in \mathbb{R}^{HW\times C}$.

\vspace{0.5em}
\noindent
\textbf{Progressive Knowledge-guided Feature Refinement for EgoVQL.} As mentioned earlier, the core idea of PRVQL is to exploit target knowledge directly from videos and apply it as guidance to enhance query and video features for target localization. For this purpose, PRVQL is implemented as a progressive architecture with multiple stages in a sequence. Each but the last stage involves target knowledge learning and knowledge-guided feature refinement, as in Fig.~\ref{fig:framework}.

More specifically, for the $k^{\text{th}}$ ($1\le k < K$) stage of our PRVQL, let $\mathcal{Q}_{k}$ and $\mathcal{V}_{k}$ denote the query and video features. For the first stage ($k=1$), $\mathcal{Q}_1$ and $\mathcal{V}_1$ are initialized using query and video features extracted from the backbone, and $\mathcal{Q}_{1}=\textbf{Q}$ and $\mathcal{V}_{1}=\textbf{V}$. Otherwise, $\mathcal{Q}_{k}$ and $\mathcal{V}_{k}=\{v_i^k\}_{i=1}^{L}$ are refined features in the last stage $(k-1)$. To mine target-specific knowledge from the video, we perform feature fusion between $\mathcal{Q}_{k}$ and $\mathcal{V}_{k}$, aiming to inject target information into video feature for improving its target awareness. In specific, we leverage cross-attention from~\cite{VaswaniSPUJGKP17} for feature fusion owing to its powerfulness in feature modeling. Mathematically, this process can be expressed as follows,
\begin{equation}\label{eq1}
\setlength{\abovedisplayskip}{5pt} 
\setlength{\belowdisplayskip}{5pt}
\mathcal{X}_k=\{x_i^k | x_i^k = \mathtt{CAB}(v_i^k,\mathcal{Q}_k)\} \;\;\; i=1,2,\cdots,L
\end{equation}
where $\mathcal{X}_k$ is the fused feature in stage $k$, and $v_i^k$ the feature in frame $i$. $\mathtt{CAB}(\mathbf{z},\mathbf{u})$ is the cross-attention (CA) block with $\mathbf{z}$ generating query and $\mathbf{u}$ key/value. Due to space limitation, please see \emph{supplementary material} for detailed architecture. Besides fused feature, we also obtain target-aware spatial attention maps $\mathcal{S}_{k}=\{s_{i}^{k}\}_{i=1}^{L}\in \mathbb{R}^{L\times HW \times HW}$ for $L$ frames in Eq.~(\ref{eq1}), with each $s_{i}^{k}\in \mathbb{R}^{HW \times HW}$ the attention map from the cross-attention operation in $\mathtt{CAB}(v_i^k,\mathcal{Q}_k)$.

To further capture spatio-temporal relations from videos for enhancing features, we apply self-attention~\cite{VaswaniSPUJGKP17} on $\mathcal{X}_k$ by propagating the query information spatially and temporally. Considering that targets in nearby frames are highly correlated, we restrict the attention operation in a temporal window using a masking strategy, similar to~\cite{jiang2024single}. To reduce the computation, we downsample $\mathcal{X}_k$ to decrease the spatial dimension of each frame feature to $h\times w$. Then, we add a position embedding $\mathcal{E}_k^{\text{pos}}$ to the video feature. This process can be written as follows,
\begin{equation}
\setlength{\abovedisplayskip}{6pt} 
\setlength{\belowdisplayskip}{6pt}
    \tilde{\mathcal{X}}_k = \mathtt{Downsample}(\mathcal{X}_k) + \mathcal{E}_k^{\text{pos}}
\end{equation}
where $\mathtt{Downsample}(\cdot)$ represents the downsampling operation implemented with convolution operation. Afterwards, masked self-attention is applied on as $\tilde{\mathcal{X}}$ as follows,
\begin{equation}\label{eq3}
\setlength{\abovedisplayskip}{7pt} 
\setlength{\belowdisplayskip}{7pt}
    \mathcal{H}_k=\mathtt{MaskedSA}(\tilde{\mathcal{X}}_k)
\end{equation}
where $\mathcal{H}_k$ represents enhanced video feature. $\mathtt{MaskedSA}(\mathbf{z})$ denotes the masked self-attention block with $\textbf{z}$ generating query/key/value. In this block, each feature element from frame $i$ only attends to feature elements from frames in the temporal range [($i-u$), ($i+u$)], which can be easily implemented using masking strategy~\cite{VaswaniSPUJGKP17,cheng2022masked}. From Eq.~(\ref{eq3}), besides the $\mathcal{H}_k$, we also gain the temporal-aware spatial attention maps, denoted as $\mathcal{T}_k \in \mathbb{R}^{L\times hw \times Lhw}$, for the target in the video, which will be used for knowledge generation.


With video feature $\mathcal{H}_k$ and attention maps $\mathcal{S}_k$ and $\mathcal{T}_k$, the target knowledge can be extracted with the AKG and SKG modules (as explained later in Sec.~\ref{secakg} and~\ref{skg}), as follows,
\begin{equation}\label{eq_kaks}
\setlength{\abovedisplayskip}{7pt} 
\setlength{\belowdisplayskip}{7pt}
    \mathcal{K}_k^a=\mathtt{AKG}(\mathcal{H}_k, \mathcal{V}_k) \;\;\;\;\;\;\;
    \mathcal{K}_k^s=\mathtt{SKG}(\mathcal{S}_k, \mathcal{T}_k)
\end{equation}
where $\mathcal{K}_k^a$ represents the appearance knowledge and $\mathcal{K}_k^s$ the spatial knowledge. Guided by $\mathcal{K}_k^a$ and $\mathcal{K}_k^s$ in stage $k$, we can refine query and video features using two update modules QFR and VFR (as described later in Sec.~\ref{update}) as follows,
\begin{equation}\label{eq_qfuvfu}
\setlength{\abovedisplayskip}{7pt} 
\setlength{\belowdisplayskip}{7pt}
    \mathcal{Q}_{k+1}=\mathtt{QFR}(\mathcal{K}_k^a, \mathcal{Q}_{k}) \;\;\;\;\;\;\; \mathcal{V}_{k+1}=\mathtt{VFR}(\mathcal{K}_k^s,  \mathcal{V}_1)
\end{equation}
where $\mathcal{Q}_{k+1}$ and $\mathcal{V}_{k+1}$ are refined features guided by target knowledge, which are fed to the next stage $(k+1)$ to generate more accurate knowledge for further feature refinement. Fig.~\ref{fig:att} compares the attention maps from the masked self-attention with and without using our approach. We can see that, our method with refined features guided by knowledge can better focus on the target in the video and thus improves target localization, showing its efficacy.

For the final $K^{\text{th}}$ stage in PRVQL, since no knowledge is extracted, the AKG and SKG modules are removed. Given the visual query and video features $\mathcal{Q}_{K}$ and $\mathcal{V}_{K}$ from the $(K-1)^{\text{th}}$ stage, we can then obtain the final enhanced video feature $\mathcal{H}_{K}$ through Eqs.~(\ref{eq1})-(\ref{eq3}) in the $K^{\text{th}}$ stage. With $\mathcal{H}_{K}$, we use the prediction heads as in ~\cite{jiang2024single} for target localization via regression and classification. For details of the adopted prediction heads, please kindly refer to~\cite{jiang2024single}.

\begin{figure}[t]
	\centering
	\includegraphics[width=0.98\linewidth]{figs/attention.pdf}\vspace{-1mm}
	\caption{Comparison of attention maps for video frames from the masked self-attention block \emph{without} (a) and \emph{with} (b) our progressive refinement. As shown, our method can better focus on the target regions, and hence can improve target localization in EgoVQL. The red boxes indicate the foreground object to localize.}
        \vspace{-4mm}
	\label{fig:att}
\end{figure}

\subsection{Appearance Knowledge Generation (AKG)}\label{secakg}

In order to extract discriminative visual information of target directly from the video, we introduce a simple yet highly effective module, named \emph{appearance knowledge generation} (AKG), for appearance knowledge learning. Specifically, it first estimates the potential target regions from the video using target-aware video features. Then, based on confidence scores of these regions, we select the top few ones to extract target features from the video as the appearance knowledge. % Since the appearance knowledge is directly acquired from a video, it contains more discriminative information than the given query for target localization, and hence can be applied as effective guidance for enhancing the query feature.

Specifically, given the target-aware video feature $\mathcal{H}_k$, we first reshape it to the 2D feature map, and then increase its spatial resolution back to $H\times W$ as follows,
\begin{equation}
\setlength{\abovedisplayskip}{6pt} 
\setlength{\belowdisplayskip}{6pt}
    \tilde{\mathcal{H}}_k=\mathtt{Upsample}(\mathtt{Reshape}(\mathcal{H}_k))
\end{equation}
where $\mathtt{Upsample}(\cdot)$ denotes the upsampling operation. After this, we apply $\tilde{\mathcal{H}}_k$ to produce temporal confidence scores and spatial box regions for target in each frame. More concretely, we first split $\tilde{\mathcal{H}}_k$ along the channel dimension into two equal halves $\tilde{\mathcal{H}}_k^{t}$ and $\tilde{\mathcal{H}}_k^{s}$ via $\tilde{\mathcal{H}}_k^{t}, \tilde{\mathcal{H}}_k^{s}=\mathtt{Split}(\tilde{\mathcal{H}}_k)$. Inspired by~\cite{jiang2024single}, we perform classification and regression to predict temporal confidence scores and spatial boxes using multi-scale anchors~\cite{RenHGS15}. Specifically, two Conv blocks are applied on $\tilde{\mathcal{H}}_k^{t}(i)$ and $\tilde{\mathcal{H}}_k^{s}(i)$ for prediction as follows,
\begin{equation}\label{akg}
\setlength{\abovedisplayskip}{7pt}
\setlength{\belowdisplayskip}{7pt}
    \tilde{\mathcal{C}}_k=\mathtt{ConvBlock}(\tilde{\mathcal{H}}_k^{t}) \;\;\;\;\; \Delta\tilde{\mathcal{B}}_k=\mathtt{ConvBlock}(\tilde{\mathcal{H}}_k^{s})
\end{equation}
where $\tilde{\mathcal{C}}_k \in \mathbb{R}^{L \times H \times W \times m}$ denotes the temporal confidence scores for target in $L$ frames with $m$ the number of anchors at each position. $\Delta\tilde{\mathcal{B}}_k \in \mathbb{R}^{L \times H \times W \times 4m}$ is the offsets to the anchor boxes $\bar{\mathcal{B}}$, and target boxes $\tilde{\mathcal{B}}_k=\Delta\tilde{\mathcal{B}}_k + \bar{\mathcal{B}}$. With $\tilde{\mathcal{C}}_k$, the confidence score in each frame is determined by the highest classification score of anchors, and the target region is the box corresponding to the box with the highest classification score. This way, we can obtain the confidence scores $\bar{\mathcal{C}}_k$ and target regions $\bar{\mathcal{B}}_k$ in each frames as follows, 
\begin{equation}\label{pred}
\setlength{\abovedisplayskip}{7pt}
\setlength{\belowdisplayskip}{7pt}
\begin{split}
    &\bar{\mathcal{C}}_k = \{c_{k}^{i} | c_{k}^{i}, d_k^i = \mathtt{Max}(\tilde{\mathcal{C}}_k(i))\} 
    \\
    &\bar{\mathcal{B}}_k = \{b_{k}^{i} | b_{k}^{i} = \mathtt{Index}(\tilde{\mathcal{B}}_k(i), d_k^i)\} 
\end{split}
\end{equation}
where $i\in[1,L]$ is the frame index. $c_{k}^{i}$ is the highest value selected from the classification scores $\tilde{\mathcal{C}}_k(i)\in\mathbb{R}^{H\times W\times m}$ of anchors in frame $i$, and $d_k^i$ is its index. $b_{k}^{i}$ is the target box corresponding to $c_{k}^{i}$ in frame, and extracted from $\tilde{\mathcal{B}}_k(i)\in\mathbb{R}^{H\times W\times 4m}$. $\mathtt{Max}(\cdot)$ is to select the maximum and its index, and $\mathtt{Index}(\cdot)$ to extract the box from $\tilde{\mathcal{B}}_k(i)$ given its index. 

With $\bar{\mathcal{C}}_k\in\mathbb{R}^{L\times 1}$ and $\bar{\mathcal{B}}_k\in\mathbb{R}^{L\times 4}$, we first sample target regions with high confidence scores as follows,
\begin{equation}\label{threshold}
\setlength{\abovedisplayskip}{7pt}
\setlength{\belowdisplayskip}{7pt}
\mathcal{B}_k=\mathtt{Sample}(\bar{\mathcal{B}}_k(i), \bar{\mathcal{C}}_k(i), \tau) =\{\bar{\mathcal{B}}_k(i) | \bar{\mathcal{C}}_k(i) > \tau\}
\end{equation}
Then, we extract $n$ regions from $\mathcal{B}_k$ with the top confidence scores via $\mathcal{B}_k^{\text{top}}=\mathtt{Top}_{n}(\mathcal{B}_k)$. If the number of regions in $\mathcal{B}_k$ is less then $n$, we keep all regions. After this, RoIAlign~\cite{he2017mask} is used to extract the appearance knowledge from $\mathcal{V}_{k}$ via
\begin{equation}\label{roi}
\setlength{\abovedisplayskip}{7pt}
\setlength{\belowdisplayskip}{7pt}
\mathcal{K}_{k}^{a}=\mathtt{RoIAlign}(\mathcal{V}_{k}, \mathcal{B}_k^{\text{top}})
\end{equation}
where $\mathcal{K}_{k}^{a}$ represents the appearance knowledge from AKG in the $k^{\text{th}}$ stage. Please notice that, in Eq.~(\ref{roi}), we only perform RoIAlign in frames corresponding to $\mathcal{B}_k^{\text{top}}$. Since $\mathcal{K}_{k}^{a}$ is generated from the video itself, when using it as guidance to refine the query feature, we can reduce the discrepancy between the query and the foreground target. By deploying AKG in each but the last stage, $\mathcal{K}_{k}^{a}$ could be gradually improved with better refined query feature in each stage. Fig.~\ref{fig:akgfig} illustrates AKG for appearance knowledge generation.

\begin{figure}[t]
	\centering
	\includegraphics[width=\linewidth]{figs/fig3.pdf}\vspace{-1mm}
	\caption{Illustration of appearance knowledge generation (AKG).}
        \vspace{-3mm}
	\label{fig:akgfig}
\end{figure}

\subsection{Spatial Knowledge Generation (SKG)}\label{skg}

In addition to appearance knowledge, we explore target spatial knowledge from the video for improving video features. Specifically, inspired by the \emph{observation} that intermediate attention maps from previous attention operations reflect the spatial cues of target in each frame to some extent, similar to the concept of ``\emph{saliency}'' but for the target, we propose the \emph{spatial knowledge generation} (SKG) module, which works to leverage readily available attention maps as guidance for enhancing target while suppressing background in the video features, enabling more focus on the target in PRVQL.

Concretely in our SKG, we exploit the target-aware spatial attention maps $\mathcal{S}_k$ from cross-attention block in Eq.~(\ref{eq1}) and temporal-aware spatial attention maps $\mathcal{T}_k$ from masked self-attention block in Eq.~(\ref{eq3}) for spatial knowledge learning. Specifically, given $\mathcal{S}_k$ and $\mathcal{T}_k$, we first extract the inter-frame spatial attention maps $\mathcal{T}_k^{\text{d}}$ by extracting diagonal elements from $\mathcal{T}_k$ as follows,
\begin{equation}\label{inter-frame}
\setlength{\abovedisplayskip}{7pt}
\setlength{\belowdisplayskip}{7pt}
\mathcal{T}_k^{\text{d}}=\phi_{\text{diag}}(\mathcal{T}_k)=\{t_{i}^k\}_{i=1}^{L}
\end{equation}
where $\phi_{\text{diag}}$ denotes the operation to extract diagonal elects, and $t_{i}^k\in \mathbb{R}^{hw\times hw}$ represents the attention maps for frame $i$. To match the spatial dimension of $\mathcal{T}_k^{\text{d}}$ and $\mathcal{S}_k$, we first perform bilinear interpolation on $\mathcal{T}_k^{\text{d}}$ to increase its spatial resolution to $HW \times HW$, and then combines these two attention maps to obtain spatial knowledge. Mathematically, this process can be expressed as follows,
\begin{equation}\label{inter-frame2}
\setlength{\abovedisplayskip}{7pt}
\setlength{\belowdisplayskip}{7pt}
\mathcal{K}_k^{s}=\alpha \cdot \varphi_{\text{int}}(\mathcal{T}_k^{\text{d}}) + (1-\alpha) \cdot \mathcal{S}_k
\end{equation}
where $\varphi_{\text{int}}$ denotes the bilinear interpolation operation, $\mathcal{K}_k^{s}$ is the target spatial knowledge, and $\alpha$ is a parameter to balance different attention maps. Since $\mathcal{K}_k^{s}$ indicates the target position cues in each frame in some degree, we can use it to highlight target while restraining background in videos for improving localization. Similar to AKG, SKG is deployed in each but the last stage of PRVQL. Fig.~\ref{fig:skgfig} illustrates SKG. 

\begin{figure}[t]
	\centering
	\includegraphics[width=0.9\linewidth]{figs/fig3-skg.pdf}\vspace{-1mm}
	\caption{Illustration of spatial knowledge generation (SKG).}
        \vspace{-3mm}
	\label{fig:skgfig}
\end{figure}


\subsection{Feature Refinement with Knowledge}\label{update}

With target appearance knowledge $\mathcal{K}_{k}^{a}$ and spatial knowledge $\mathcal{K}_{k}^{s}$ obtained from AKG and SKG in stage $k$ ($1<k\le K$), we then apply them as guidance to refine the query and video features through \emph{query feature refinement} (QFR) and \emph{video feature refinement} (VFR) modules.

\vspace{0.3em}
\noindent
\textbf{Query Feature Refinement (QFR).} QFR aims to refine the query feature with guidance from learned target appearance knowledge. Specifically, it adopts a cross-attention block to fuse appearance knowledge $\mathcal{K}_{k}^{a}$ into the query. Given the query feature $\mathcal{Q}_k$ and appearance knowledge $\mathcal{K}_{k}^{a}$ in stage $k$, we first apply a Conv block on $\mathcal{K}_{k}^{a}$ and then perform refinement via QFR as follows,
\begin{equation}\label{eq:q_update}
\setlength{\abovedisplayskip}{7pt} 
\setlength{\belowdisplayskip}{7pt}
\mathcal{Q}_{k+1} = \mathtt{QFR}(\mathcal{Q}_{k}, \mathcal{K}_{k}^{a}) = \mathtt{CAB}(\mathcal{Q}_{k}, \mathtt{CNB}(\mathcal{K}_k^a))
\end{equation}
where $\mathcal{Q}_{k+1}$ is refined query feature and fed to next stage for learning more accurate knowledge, which in turn leads to better query feature for localization in the final stage. It is worth noting that, besides cross-attention, we explore different strategies to combine $\mathcal{Q}_k$ and $\mathcal{K}_{k}^{a}$, including addition and concatenation operations. We observe that using cross-attention achieves the best performance, as exhibited in our experiments provided in the \emph{supplementary material}.

\vspace{0.3em}
\noindent
\textbf{Video Feature Refinement (VFR).} VFR focuses on adopting target spatial knowledge to refine initial video feature by enhancing target while suppressing the background regions. Concretely, given the initial video feature $\mathcal{V}_1$ and learn spatial knowledge $\mathcal{K}_{k}^{s}$ in stage $k$, we use residual connection to refine $\mathcal{V}_1$ as follows,
\begin{equation}\label{eq:v_update}
\setlength{\abovedisplayskip}{7pt} 
\setlength{\belowdisplayskip}{7pt}
\mathcal{V}_{k+1} = \beta \cdot (\mathcal{K}_{k}^{s} \odot \mathcal{V}_1) + (1-\beta) \cdot \mathcal{V}_1
\end{equation}
where $\mathcal{V}_{k+1}$ denotes the refined video feature that is used for the next stage, $\beta$ is a balancing parameter, and $\odot$ represents the pixel-wise multiplication.

\subsection{Optimization and Inference}\label{opt}

\textbf{Optimization}. Given a video and a visual query, we predict confidence scores $\tilde{\mathcal{C}}_k$ and target boxes $\tilde{\mathcal{B}}_k$ ($\tilde{\mathcal{B}}_k=\Delta\tilde{\mathcal{B}}_k+\tilde{\mathcal{B}}$) in each stage $k$ ($1\le k\le K$). During training, given the groundtruth boxes $\mathcal{B}^*$ and temporal occurrence scores $\mathcal{S}^*$, we design the following loss function $\mathcal{L}_k$ for stage $k$,
\begin{equation}\label{eq:loss}
\setlength{\abovedisplayskip}{7pt} 
\setlength{\belowdisplayskip}{7pt}
\mathcal{L}_k = \mathcal{L}_{\text{L}_1}(\tilde{\mathcal{B}}_k, \mathcal{B}^*) + \lambda_1\mathcal{L}_{\text{GIoU}}(\tilde{\mathcal{B}}_k, \mathcal{B}^*) + \lambda_2\mathcal{L}_{\text{BCE}}(\tilde{\mathcal{S}}_k, \mathcal{S}^*)
\end{equation}
where $\mathcal{L}_{\text{L}_1}$, $\mathcal{L}_{\text{GIoU}}$, and $\mathcal{L}_{\text{BCE}}$ represent the $L_1$ loss, generalized IoU (GIoU)~\cite{rezatofighi2019generalized} loss, and binary cross-entropy (BCE) loss, respectively. $\lambda_1$ and $\lambda_2$ are two balancing parameters. With Eq.~(\ref{eq:loss}), the total training loss $\mathcal{L}_{\text{total}}$ can be obtained via $\mathcal{L}_{\text{total}}=\sum_{k=1}^{K} \mathcal{L}_k$. Following~\cite{jiang2024single,xu2023my,grauman2022ego4d}, we perform hard negative mining during training to decrease false positive prediction. For details, please refer to~\cite{jiang2024single,xu2023my,grauman2022ego4d}.

\vspace{0.3em}
\noindent
\textbf{Inference.} We employ the same strategy as in~\cite{jiang2024single} to obtain the prediction result. Specifically, during inference, we first obtain the target region in each frame by selecting target box with the highest confidence score. Please note that, the target regions with confidences scores smaller than a threshold, set to 0.79, will be discarded. After this, we select the most recent peak and generate a response track via bidirectional search from the peak. Details can be seen in~\cite{jiang2024single}.


\section{Experiments}

\textbf{Implementation.} Our PRVQL is implemented using PyTorch~\cite{paszke2019pytorch} with Nvidia RTX A6000 GPUs. Similar to~\cite{jiang2024single}, we use the popular ViT~\cite{DosovitskiyB0WZ21} pretrained with DINOv2~\cite{OquabDMVSKFHMEA24} as the backbone. Our PRVQL is end-to-end trained for 50 epoches (a total of 60K iterations) with a batch size of 12, utilizing the AdamW optimizer~\cite{LoshchilovH19} with a peak learning rate of $10^{-4}$ and a weight decay of $5\times10^{-2}$. The query image and video frames are resized to $480\times480$. The number of stages $K$ in PRVQL is empirically set to 3, and the pooling size for RoIAlign is 5. The number of selected boxes $n$ for appearance knowledge is 3, and the threshold $\tau$ is set to $0.7$. The parameter $\alpha$ for computing spatial knowledge is empirically set to 0.5. The balancing parameter $\beta$ is 0.1. $\lambda_1$ and $\lambda_2$ are empirically set to 0.3 and 100. The video frame length $L$, similar to~\cite{jiang2024single}, is set to $30$ with frames randomly selected to ensure coverage of at least a portion of the response track. For the anchor boxes in localization, we employ four scales ($16^{2}$, $32^{2}$, $64^{2}$, $48^{2}$) with three aspect ratios (0.5, 1, 2) for each anchor box, similar to~\cite{jiang2024single}.

\begin{table}[!t]
	\centering
        \caption{Comparison on the Ego4D validation set.}\vspace{-2mm}
	\renewcommand{\arraystretch}{1.05}
	\scalebox{0.92}{
	\begin{tabular}{rcccc}
	\rowcolor{mygray}
	\specialrule{1.5pt}{0pt}{0pt}
	Methods & tAP$_{25}$ & stAP$_{25}$ & rec\% &  Succ    \\
	\hline
	\hline
        STARK \textcolor{lightblue}{\scriptsize{[ICCV'21]}}  & 0.10   & 0.04   & 12.41  & 18.70    \\
	SiamRCNN \textcolor{lightblue}{\scriptsize{[CVPR'22]}}  & 
        0.22   & 0.15   & 32.92  & 43.24    \\
	NFM \textcolor{lightblue}{\scriptsize{[VQ2D Challenge'22]}}     & 0.26  & 0.19  & 37.88  & 47.90    \\
	CocoFormer \textcolor{lightblue}{\scriptsize{[CVPR'23]}} 
        & 0.26  & 0.19  & 37.67  & 47.68    \\
	VQLoC \textcolor{lightblue}{\scriptsize{[NeurIPS'23]}}  
        & 0.31  & 0.22  & 47.05  & 55.89    \\
        \hline
	\rowcolor{highlight} PRVQL (ours) & \textbf{0.35} & \textbf{0.27} & 
        \textbf{47.87} & \textbf{57.93}   \\
        \specialrule{1.5pt}{0pt}{0pt}
	\end{tabular}}
        \label{tab:sota_val}
\end{table}

\begin{table}[!t]
	\centering
        \caption{Comparison on the Ego4D test set.}\vspace{-2mm}
	\renewcommand{\arraystretch}{1.05}
	\scalebox{0.92}{
	\begin{tabular}{rcccc}
	\rowcolor{mygray}
	\specialrule{1.5pt}{0pt}{0pt}
	Methods & tAP$_{25}$ & stAP$_{25}$ & rec\% &  Succ    \\
	\hline
	\hline
        STARK \textcolor{lightblue}{\scriptsize{[ICCV'21]}}  & -   & -   & -  & -    \\
	SiamRCNN \textcolor{lightblue}{\scriptsize{[CVPR'22]}}  & 
        0.20   & 0.13   & -  & -    \\
	NFM \textcolor{lightblue}{\scriptsize{[VQ2D Challenge'22]}}     & 0.24  & 0.17  & -  & -    \\
	CocoFormer \textcolor{lightblue}{\scriptsize{[CVPR'23]}} 
        & 0.25  & 0.18  & -  & -    \\
	VQLoC \textcolor{lightblue}{\scriptsize{[NeurIPS'23]}}  
        & 0.32  & 0.24  & 45.11  & 55.88    \\
        \hline
	\rowcolor{highlight} PRVQL (ours) & \textbf{0.37} & \textbf{0.28} & 
        \textbf{45.70} & \textbf{59.43}   \\
        \specialrule{1.5pt}{0pt}{0pt}
	\end{tabular}}
        \label{tab:sota_tst}
\end{table}

\begin{table}[!t]\small
\setlength{\tabcolsep}{2.2pt}
	\centering
        \caption{Comparison of speed on Ego4D.}\vspace{-2mm}
	\renewcommand{\arraystretch}{1.05}
	\scalebox{0.95}{
	\begin{tabular}{rcccccc}
	\rowcolor{mygray}
	\specialrule{1.5pt}{0pt}{0pt}
	 & STARK & SiamRCNN & NFM &  CocoFormer & VQLoC & PRVQL    \\
	\hline
	\hline
        FPS  & 33   & 3   & 3  & 3 & 36 &  30  \\
        \specialrule{1.5pt}{0pt}{0pt}
	\end{tabular}}
        \label{tab:sota_fps}\vspace{-3mm}
\end{table}

\subsection{Dataset and Evaluation Metrics}

\textbf{Dataset.} Following previous methods~\cite{xu2023my,jiang2024single}, we conduct experiments on the challenging Ego4D benchmark~\cite{grauman2022ego4d}. Ego4D is a recently proposed large-scale dataset dedicated to first-person video understanding. Similar to~\cite{jiang2024single}, we use videos from the VQ2D task. There are 13.6K, 4.5K, 4.4K pairs of queries and videos for training, validation, and testing, lasting 262, 87, and 84 hours, respectively.

\vspace{0.5em}
\noindent
\textbf{Evaluation Metrics.} Following~\cite{xu2023my,jiang2024single}, we adopt the metrics provided by Ego4D~\cite{grauman2022ego4d} for evaluation, including temporal average precision (tAP$_{25}$), spatio-temporal average precision (stAP$_{25}$), recovery (rec\%), and success (Succ). tAP$_{25}$ and stAP$_{25}$ are used to measure the accuracy of the predicted temporal and spatio-temporal extends of the of response tracks in comparison to groundtruth using the Intersection over Union (IoU) with threshold 0.25. The recovery metric assess the percentage of predicted frames in which the IoU between predicted bounding box and ground-truth is great than or equal to 0.5, and success metric measures weather the IoU between prediction and groundtruth exceeds 0.05. For more details of metrics, please refer to~\cite{grauman2022ego4d}.



\subsection{State-of-the-art Comparison}

In order to verify the effectiveness of our PRVQL, we compare it with other state-of-the-art methods on Ego4D, including STARK~\cite{yan2021learning}, SiamRCNN~\cite{voigtlaender2020siam}, NFM~\cite{xu2022negative}, CocoFormer~\cite{xu2023my}, and VQLoC~\cite{jiang2024single}. Tab.~\ref{tab:sota_val} displays the results and comparison on the Ego4D validate test. As in Tab.~\ref{tab:sota_val}, we can clearly see that the proposed PRVQL achieves the best performance on all four metrics. In particular, it achieves the 0.35 tAP$_{25}$ and 0.27 stAP$_{25}$ scores, which outperforms the second best method VQLoC with 0.31 tAP$_{25}$ and 0.22 stAP$_{25}$ scores by 4\% and 5\%. Besides, the rec and Succ scores of PRVQL are 47.87\% and 57.93 respectively, which surpasses the 47.05\% rec\ and 55.89 Succ scores of VQLoC, evidencing the effectiveness of our approach. In addition, in Tab.~\ref{tab:sota_tst} we further report the experimental results and comparison on Ego4D test set. As in Tab.~\ref{tab:sota_tst}, our PRVQL again achieves the best performance on all four metrics. Specifically, PRVQL obtains the 0.37 tAP$_{25}$ and 0.28 stAP$_{25}$ scores. Compared to the second best method VQLoC, our approach outperforms it by 5\% and and 4\%, respectively, on tAP$_{25}$ and stAP$_{25}$. In addition, the rec and Succ scores of PRVQL are 45.70\% and 59.43, which are better than those of VQLoC with 45.11\% and 55.88. All these show the efficacy of target knowledge in improving EgoVQL.

In addition, we show the comparison of speed, measured by frames per second (\emph{FPS}), for different methods in Tab.~\ref{tab:sota_fps}. From Tab.~\ref{tab:sota_fps}, we can see our method runs fast at a speed of 30 FPS. Despite being slightly slower than VQLoC running at a speed of 36 FPS, our PRVQL is more robust in localization, showing a better balance between accuracy and speed.

\begin{table}[!t]
        % \setlength{\tabcolsep}{5.pt}
	\centering
        \caption{Ablation studies of AKG and SKG.}\vspace{-2mm}
	\renewcommand{\arraystretch}{1.1}
	\scalebox{0.92}{
		\begin{tabular}{cccccccc}
			\specialrule{1.5pt}{0pt}{0pt}
			\rowcolor{mygray} 
			& AKG & SKG & tAP$_{25}$ & stAP$_{25}$ & rec\% &  Succ \\ \hline\hline
			\ding{182} & - & - & 0.32 & 0.23 & 45.24 & 55.37 \\
			\ding{183} & \checkmark & - & 0.34 & 0.26 & 47.34 & 57.27 \\
            \ding{184} & - & \checkmark & 0.33 & 0.24 & 46.33 & 56.46 \\
			\ding{185} & \checkmark & \checkmark & \textbf{0.35} & \textbf{0.27} & \textbf{47.87} & \textbf{57.93} \\ \specialrule{1.5pt}{0pt}{0pt}
	\end{tabular}}
	\label{KG_modules}
\end{table}

\begin{table}[!t]
        \setlength{\tabcolsep}{7.7pt}
	\centering
        \caption{Ablation studies on the number of stages.}\vspace{-2mm}
	\renewcommand{\arraystretch}{1}
	\scalebox{0.92}{
		\begin{tabular}{cccccc}
		\specialrule{1.5pt}{0pt}{0pt}
		\rowcolor{mygray} 
		& \# Stages & tAP$_{25}$ & stAP$_{25}$ & rec\% &  Succ \\ \hline\hline
           \ding{182} & $K=1$ & 0.32 & 0.23 & 45.24 & 55.37 \\
           \ding{183} & $K=2$ & 0.34 & \textbf{0.27} & 47.25 & 56.43 \\
           \ding{184} & $K=3$ & \textbf{0.35} & \textbf{0.27} & \textbf{47.87} & \textbf{57.93} \\
           \ding{185} & $K=4$ & 0.33 & 0.26 & 45.91 & 55.29 \\
	\specialrule{1.5pt}{0pt}{0pt}
    \end{tabular}}
	\label{tab:stage}\vspace{-3mm}
\end{table}


\subsection{Ablation Study}

For better understanding of PRVQL, we conduct extensive ablation studies on Ego4D validation set as follows. 

\vspace{0.3em}
\noindent
\textbf{Impact of AKG and SKG.} AKG and SKG are two important modules in PRVQL for target appearance and spatial knowledge generation. In order to analyze these two modules, we conduct thorough ablation studies in Tab.~\ref{KG_modules}. From Tab.~\ref{KG_modules}, we can see that, without AKG and SKG, the tAP$_{25}$ and stAP$_{25}$ scores are 0.32 and 0.23, respectively (\ding{182}). By applying AKG alone for refinement with appearance knowledge, they can be significantly improved to 0.34 and 0.26 with performance gains of 0.02 and 0.03 (\ding{183} \emph{v.s.} \ding{182}). When using only SKG for refinement with spatial knowledge, tAP$_{25}$ and stAP$_{25}$ are improved to 0.33 and 0.24 (\ding{184} \emph{v.s.} \ding{182}). From this table, we can also observe that, using appearance knowledge for refinement in PRVQL brings more gains than the spatial knowledge (\ding{183} \emph{v.s.} \ding{184}). When using both AKG and SKG in our PRVQL, we achieve the best performance with 0.35 tAP$_{25}$ and 0.27 stAP$_{25}$ scores (\ding{185} \emph{v.s.} \ding{182}), which clearly evidences the efficacy of target knowledge for improving the robustness of EgoVQL.

\begin{table}[!t]
        \setlength{\tabcolsep}{7.5pt}
	\centering
    \caption{Ablation studies on the threshold $\tau$.}\vspace{-2mm}
			\renewcommand{\arraystretch}{1}
			\scalebox{0.92}{
				\begin{tabular}{ccccccc}
					\specialrule{1.5pt}{0pt}{0pt}
					\rowcolor{mygray} 
					& Threshold & tAP$_{25}$ & stAP$_{25}$ & rec\% &  Succ \\ \hline\hline
					\ding{182}& $\tau=0.6$ & 0.32 & 0.24 & 44.82 & 55.97 \\
                      \ding{183} &  $\tau=0.7$ & \textbf{0.35} & \textbf{0.27} & \textbf{47.87} & \textbf{57.93} \\
                       \ding{184} & $\tau=0.8$ & 0.34 & 0.26 & 46.53 & 57.33 \\
					\specialrule{1.5pt}{0pt}{0pt}
			\end{tabular}}
			\label{top_threshold}
\end{table}

\begin{table}[!t]
        \setlength{\tabcolsep}{8.8pt}
	\centering
    \caption{Ablation studies on the number of target boxes in AKG.}\vspace{-2mm}   
	\renewcommand{\arraystretch}{1}
			\scalebox{0.92}{
				\begin{tabular}{cccccc}
					\specialrule{1.5pt}{0pt}{0pt}
					\rowcolor{mygray} 
					 & & tAP$_{25}$ & stAP$_{25}$ & rec\% &  Succ \\ \hline\hline
                      \ding{182} &  $n=2$ & 0.34 & 0.26 & 47.27 & 56.03 \\
                       \ding{183} & $n=3$  & 0.35 & \textbf{0.27} & \textbf{47.87} & \textbf{57.93} \\
                       \ding{184} & $n=4$  & \textbf{0.36} & 0.26 & 46.59 & 57.62 \\
                       \ding{185} & $n=5$  & 0.35 & 0.24 & 46.84 & 56.95 \\
					\specialrule{1.5pt}{0pt}{0pt}
			\end{tabular}}
			\label{tab:top_k}
\end{table}

\begin{table}[!t]
    \setlength{\tabcolsep}{8.6pt}
    \centering
    \caption{Ablation studies on RoIAlign feature size.}\vspace{-2mm}
    \renewcommand{\arraystretch}{1}
    \scalebox{0.92}{
	\begin{tabular}{ccccccc}
	\specialrule{1.5pt}{0pt}{0pt}
	\rowcolor{mygray} 
					& Size & tAP$_{25}$ & stAP$_{25}$ & rec\% &  Succ \\ \hline\hline
				  \ding{182} & 3 & 0.33 & 0.26 & 46.94 & 56.06 \\
                    \ding{183} & 5 & \textbf{0.35} & \textbf{0.27} & \textbf{47.87} & \textbf{57.93} \\
                    \ding{184} & 7 & 0.34 & \textbf{0.27} & 47.58 & 57.38 \\
                    \ding{185} & 9 & 0.32 & 0.25 & 46.37 & 55.27 \\
					\specialrule{1.5pt}{0pt}{0pt}
			\end{tabular}}
			\label{tab:roisize}\vspace{-3mm}
\end{table}

\begin{figure*}[!t]
    \centering
    \includegraphics[width=0.9\linewidth]{figs/fig4.pdf}\vspace{-2mm}
    \caption{Qualitative analysis and comparison between our PRVQL and state-of-the-art VQLoC in representative videos with different challenges. We observe that, owing to our target knowledge from videos, PRVQL can more robustly localize the target of interest.}
    \label{fig:visual_comparison}\vspace{-3mm}
\end{figure*}

\vspace{0.3em}
\noindent
\textbf{Impact of the number of stages.} Our PRVQL is designed as a progressive architecture with $K$ stages to explore the target knowledge for refinement. In this work, we conduct an ablation study on the number of stages $K$ in PRVQL as shown in Tab.~\ref{tab:stage}. From Tab.~\ref{tab:stage}, we observe that, when setting $K=1$, which means only one stage is used and the target knowledge is not used due to one-stage design, the tAP$_{25}$ and stAP$_{25}$ scores are 0.32 and 0.23 (\ding{182}). When adding the second stage, tAP$_{25}$ and stAP$_{25}$ can be largely improved by 2\% and 4\% to 0.34 and 0.27, respectively (\ding{183}). With three stages, the tAP$_{25}$ and stAP$_{25}$ scores can be further boosted to 0.35 and 0.27 (\ding{184}). When setting $K=4$ with 4 stages, the performance is decreased with 0.33 tAP$_{25}$ and 0.26 stAP$_{25}$ scores (\ding{185}). Therefore, we set $K$ to 3 in this work.

\vspace{0.3em}
\noindent
\textbf{Impact of threshold $\tau$ in AKG.} The threshold $\tau$ is used to filter out less confident target regions in AKG, aiming to avoid noisy features in appearance knowledge generation. In this work, we conduct an ablation to study the impact of $\tau$ on the final performance in Tab.~\ref{top_threshold}. As shown in Tab.~\ref{top_threshold}, we can see that, when setting $\tau$ to 0.7, PRVQL achieves the best performance on all four metrics (\ding{183}). 

\vspace{0.3em}
\noindent
\textbf{Impact of number of target boxes in AKG.} In AKG, we extract visual features from the top $n$ highly confident target regions for appearance knowledge generation. We conduct an ablation on $n$ in Tab.~\ref{tab:top_k}. From Tab.~\ref{tab:top_k}, we can observe that, when using the top 3 target regions for knowledge learning in AKG, we achieve the best overall performance (\ding{183}).



\vspace{0.3em}
\noindent
\textbf{Impact of RoIAlign Feature Size.} With the top $n$ selected target regions, we perform the RoIAlign operation~\cite{he2017mask} to obtain target appearance knowledge. The  RoIAlign feature size may have an impact on the target appearance knowledge. A too small size may result in the coarse spatial information of the target, while a too large size may lead to losing discriminative local features for the target, both degrading performance. In this work, we study different RoIAlign feature sizes in Tab.~\ref{tab:roisize}. As shown, when setting the size to 5 in RoIAlign, PRVQL shows the best overall performance.


\subsection{Qualitative Analysis}

In order to provide further analysis of our PRVQL, we show the visualization results of its localization and compare it with the state-of-the-art VQLoC in Fig.~\ref{fig:visual_comparison}. Specifically, we show the results and comparison on several representative videos, including video in (a) with \emph{pose variation}, video in (b) with \emph{cluttering background} and \emph{out-of-view}, video in (c) with \emph{occlusion} and \emph{low resolution}, video in (d) with \emph{pose variation} and \emph{cluttering background}, and video in (d) with \emph{motion blur} and \emph{distractor}. From Fig.~\ref{fig:visual_comparison}, we can observe that, our method can robustly and accurately localize the target of interest in all these challenges, owing to the help of target knowledge from the videos, while VQLoC is prone to drift to the background due to lack of discriminative target information, which evidences the effectiveness of target cues in videos for improving EgoVQL.

Due to limited space, we demonstrate more results, analysis, and ablation studies in the \emph{supplementary material}. 

\section{Conclusion}

In this paper, we present a novel approach, dubbed PRVQL, for improving EgoVQL via exploring crucial target knowledge from videos to refine features for robust localization. Our PRVQL is implemented as a multi-stage architecture. In each stage, two key modules, including AKG and SKG, are used to extract target appearance and spatial knowledge from the video. The knowledge from one stage is used as guidance to refine query and video features in the next stage, which are adopted for
learning more accurate knowledge for further feature refinement. Through this progressive process, PRVQL learns gradually improved knowledge, which in turn leads to better refined features for target localization in the final stage. To validate the effectiveness of PRVQL, we conduct experiments on Ego4D. Our experimental results show that PRVQL achieves state-of-the-art result and largely surpasses other methods, showing its efficacy.

{
\small
\bibliographystyle{ieeenat_fullname}
\bibliography{main}
% \documentclass{MITstyle}

%\usepackage[table]{xcolor}
\usepackage{chngcntr}
\usepackage{hyperref}
\usepackage{microtype}

\title{A Lightweight and Extensible Cell Segmentation and Classification Model for Whole Slide Images}

\author{Nikita Shvetsov~$^{1, }$\footnote{Correspondence e-mail: nikita.shvetsov@uit.no}, Thomas K. Kilvaer~$^{2, 3}$, Masoud Tafavvoghi~$^{4}$, Anders Sildnes~$^{1}$, \\ Kajsa Møllersen~$^{4}$, Lill-Tove Rasmussen Busund~$^{5, 6}$, Lars Ailo Bongo~$^{1}$ \\
%
\vspace{1em} % Space between authors and afilliations
%
\normalfont{\small $^{1}$Department of Computer Science, UiT The Arctic University of Norway}\\
\normalfont{\small $^{2}$Department of Oncology, University Hospital of North Norway}\\
\normalfont{\small $^{3}$Department of Clinical Medicine, UiT The Arctic University of Norway}\\
\normalfont{\small $^{4}$Department of Community Medicine, UiT The Arctic University of Norway}\\
\normalfont{\small $^{5}$Department of Medical Biology, UiT The Arctic University of Norway} \\
\normalfont{\small $^{6}$Department of Clinical Pathology, University Hospital of North Norway} %\vspace{2em}
}

\begin{document}
\maketitle

\section*{Abstract}

% \begin{abstract}
% Developing clinically useful cell-level analysis tools in digital pathology remains challenging due to limitations in dataset granularity, inconsistent annotations, computational demands of advanced models, and difficulties in integrating new technologies into clinical workflows. To address these challenges, we propose a multi-faceted solution that enhances data quality, model performance, and usability to create a lightweight and extensible cell segmentation and classification model.

% First, we update data labels by employing a cross-relabeling process that refines the labels of two existing datasets, PanNuke and MoNuSAC, to create a new unified dataset with enhanced granularity, encompassing seven distinct cell types. Second, we leverage the H-Optimus foundation model as a fixed encoder to improve feature representation for simultaneous cell segmentation and classification tasks. Third, to address the computational demands of foundation models, we employ knowledge distillation to reduce model size and complexity while maintaining comparable performance. Finally, to facilitate integration into clinical workflows, we integrate the distilled model into the QuPath software, a widely used open-source platform in digital pathology.

% Our results demonstrate improvements in cell segmentation and classification performance using the H‑Optimus-based model compared to a CNN-based model. Specifically, the average $R^2$ improved from 0.575 to 0.871, and the average $PQ$ score improved from 0.450 to 0.492, indicating better alignment with actual cell counts and enhanced segmentation and classification quality. Furthermore, the distilled student model maintains performance comparable to the larger foundation model while reducing the parameter count by a factor of 48.
% Overall, by reducing computational complexity and integrating it into existing workflows, the proposed approach may significantly impact diagnostic processes, reduce the workload of pathologists, and contribute to improved patient outcomes. Though our approach shows potential enhancements in efficiency and usability of cell segmentation and classification models in digital pathology, extensive validation is needed to deploy these models in clinical practice.
% \end{abstract}

%%% shortened abstract
\begin{abstract}
Developing clinically useful cell-level analysis tools in digital pathology remains challenging due to limitations in dataset granularity, inconsistent annotations, high computational demands, and difficulties integrating new technologies into workflows. To address these issues, we propose a solution that enhances data quality, model performance, and usability by creating a lightweight, extensible cell segmentation and classification model. 

First, we update data labels through cross-relabeling to refine annotations of PanNuke and MoNuSAC, producing a unified dataset with seven distinct cell types. Second, we leverage the H-Optimus foundation model as a fixed encoder to improve feature representation for simultaneous segmentation and classification tasks. Third, to address foundation models' computational demands, we distill knowledge to reduce model size and complexity while maintaining comparable performance. Finally, we integrate the distilled model into QuPath, a widely used open-source digital pathology platform. 

Results demonstrate improved segmentation and classification performance using the H-Optimus-based model compared to a CNN-based model. Specifically, average $R^2$ improved from 0.575 to 0.871, and average $PQ$ score improved from 0.450 to 0.492, indicating better alignment with actual cell counts and enhanced segmentation quality. The distilled model maintains comparable performance while reducing parameter count by a factor of 48. By reducing computational complexity and integrating into workflows, this approach may significantly impact diagnostics, reduce pathologist workload, and improve outcomes. Although the method shows promise, extensive validation is necessary prior to clinical deployment.
\end{abstract}
\clearpage

\section{Introduction}
In digital pathology, accurate segmentation and classification of cells are crucial for many diagnostic, prognostic, and predictive analyses \cite{Jaber_Beziaeva_etal._2019,Lin_Pan_etal._2022,Park_Ock_etal._2022,Shen_Choi_etal._2024}. Nowadays, developments in computational pathology offer multiple solutions \cite{H._Qu_P._Wu_etal._2020,Javed_Mahmood_etal._2020} to utilize cell-level datasets to train machine learning models that solve these problems. The quality and specificity of training datasets are critical for robust and accurate models. Adhering to the principle of "garbage in, garbage out", it is essential to ensure that these datasets are extensively and accurately labeled with distinct classes that reflect the diverse biological characteristics of different cell types. Unfortunately, the number of open-source datasets comprising such high-quality annotations is limited. Existing cell segmentation datasets \cite{Gamper_Koohbanani_etal._2019,Graham_Vu_etal._2019,Verma_Kumar_etal._2021} may offer extensive annotations for certain cell types while providing more general labels for others. For example, in PanNuke, which is one of the largest open-source datasets comprising labeled cells, various types of morphologically and functionally different inflammatory cells like macrophages and lymphocytes are clustered in a broad "inflammatory" class. Consequently, these classes are frequently omitted from analyses or aggregated into broader meta-classes \cite{Gamper_Koohbanani_etal._2020} and likely interfere with other cell classes included in the dataset. This and similar inconsistencies in annotation granularity limit the ability of machine learning models to learn the comprehensive and nuanced features necessary for accurate cell segmentation and classification. To address these challenges, methods for refining and standardizing dataset annotations are essential to enhance the quality of training data.

A complementary approach to mitigate the absence of high-quality training data is the use of foundation models. Foundation models as encoders are defined as large-scale, versatile networks pre-trained on vast, diverse datasets using self-supervised learning, contrasting with convolutional neural network (CNN) pre-trained encoders that rely on supervised learning with labeled data. In practice, foundation models leverage enormous amounts of weakly or unlabeled data from millions of whole slide images (WSIs) and employ self-attention mechanisms to capture long-range dependencies and global context \cite{Chen_Ding_etal._2024,Saillard_Jenatton_etal._2024,Vorontsov_Bozkurt_etal._2024,Xu_Usuyama_etal._2024}. As a consequence, foundation models are able to produce transferable feature representations across different cell types and tissue environments. The feature representations can be leveraged by decoder networks to produce segmentation masks and pixel-level classifications. Because foundation models have comprehensive feature representations, they can be effectively fine-tuned using much smaller amounts of cell-level data compared to the large datasets needed to train models from scratch. Furthermore, foundation models incorporate adversarial training elements or contrastive learning \cite{Chen_Ding_etal._2024,Xu_Usuyama_etal._2024}, enhancing their resilience and adaptability by exposing them to challenging and varied scenarios during training. This may result in more generalizable models, often making them well-suited for diverse and complex tasks in digital pathology.

Despite the inherent advantages of foundation models, their deployment for practical use faces its own obstacles. In particular, they require substantial computational power, financial investments and rigorous testing to ensure reliability and efficacy for a given task \cite{Akkus_Dangott_etal._2022,Dragomir_Cocuz_etal._2022,Go_2022,Jafri_Farooqui_etal._2024}. Moreover, while foundation models enhance feature representation and performance, they depend on the quality of available annotations for decoder fine-tuning and, like any other model, cannot resolve existing inconsistencies or ambiguities in data labels. Therefore, there remains a critical need for solutions that address both data quality and practical deployment considerations.
Further, integrating new technologies into existing clinical workflows often encounters resistance, as it necessitates adjustments to established diagnostic processes. So, there is a need to develop solutions that could be integrated into current practices, minimizing the burden on medical professionals to adopt new tools \cite{King_Williams_etal._2023}.

Existing solutions \cite{Goldsborough_Philps_etal._2024,Hörst_Rempe_etal._2024}, while addressing some aspects of these challenges, fall short in providing a comprehensive approach. To address the data quality and clinical deployment issues, we propose a multi-faceted solution that encompasses data refinement, model optimization, and integration with existing pathology tools (\hyperref[fig:fig1]{Figure 1}). The outcome is a lightweight cell segmentation and classification model that can be integrated into digital pathology workflows for practical clinical use.

\begin{figure}[h!]
    \centering
    \includegraphics[width=\textwidth, height=0.82\textheight, keepaspectratio]{images/Figure_1.pdf}
    \caption{Overview of the proposed solution, including 1) Data refinement using cross-relabeling, 2) Teacher model development and fine tuning, 3) Student model optimization with knowledge distillation and 4) Student model and QuPath integration}
    \label{fig:fig1}
\end{figure}
\clearpage

Our approach begins with preparing the data for the fine-tuning and training of the machine learning models. We create a refined dataset, acquired via cross-relabeling two cell-level datasets, enhancing annotation specificity and consistency of the labeled data. Subsequently, we create a cell segmentation and classification model based on the foundation model. We leverage the foundation model as a fixed encoder and fine-tune a decoder using the refined dataset to improve generalization across diverse tissue- and cell types.
To ensure that the model remains lightweight and deployable in a possibly resource-constrained environment, we employ knowledge distillation to approximate the functionality of the foundation model. Finally, to facilitate the practical application of our model in digital pathology workflows, we integrate it with the QuPath \cite{Bankhead_Loughrey_etal._2017} application. Each methodological component contributes to the overarching goal of enhancing model performance, generalizability, and usability in clinical settings.

The primary contributions of this paper are:
\begin{enumerate}
    \item \textit{Data labels refinement through cross-relabeling:}
    
    We propose a new method for refining labels of cell-level datasets through cross-relabeling. This method employs classification models to re-label broad and ambiguous instances, resulting in a more diverse dataset. Our evaluation demonstrates that these classification models achieve high accuracy on test subsets, indicating the reliability of the method for label refinement.

    \item \textit{Enhanced model performance via foundation models:}
    
    We employ a foundation model as a feature extractor for the cell segmentation and classification task. In comparison with training a CNN model from scratch, the foundation model backbone only needs fine-tuning, which significantly reduces training time, computational resources and data requirements. We show that using a foundation model encoder leads to better performance in cell segmentation and classification networks than using a CNN-based encoder. This improvement may enable the model to generalize more effectively across various tissue types and imaging methods.
    
    \item \textit{Model optimization through knowledge distillation:}
    
    We show that a smaller student model trained using knowledge distillation on the refined dataset obtained via our cross-relabeling approach from a foundation model achieves comparable performance in cell segmentation and quantification tasks. As a result, this model is more suitable for deployment in environments without high-performance computing resources.
    
    \item \textit{Integration with QuPath:}
    
    We integrate the distilled cell segmentation and classification model into QuPath, a widely used open-source digital pathology platform, to accelerate clinical adaptation by enabling pathologists to more easily incorporate advanced computational tools into their existing workflows.
\end{enumerate}

Through these methodological steps, we aim to bridge the gap between advanced machine learning techniques and practical clinical applications, making accurate and efficient digital pathology accessible in a broader range of healthcare settings.

\section{Refining Existing Datasets Using Cross-Relabeling}
To address the limitations of sparse and ambiguous labeling of cell-level datasets, we propose a generalizable cross-relabeling strategy that can be applied to any dataset containing broadly categorized or imprecisely labeled cell types. This approach involves training and subsequently leveraging classification models to refine broad categories into more specific or biologically relevant classes.
When applied to cell-level data, the methodology includes extracting individual cell images from the dataset patches, preprocessing these images to standardize the size and accommodate partial cells, and then training deep learning classifiers capable of distinguishing between the finer cell subtypes within the coarser categories. 
To illustrate our approach, we focus on the PanNuke \cite{Gamper_Koohbanani_etal._2020, Gamper_Koohbanani_etal._2019} and MoNuSAC \cite{Verma_Kumar_etal._2021} datasets that we have used to train models for cell quantification in our previous works \cite{Shvetsov_Grønnesby_etal._2022,Shvetsov_Sildnes_etal._2024}. We find that for better cell differentiation we have to introduce more granular labels. PanNuke includes a broad classification of "inflammatory" cells, encompassing lymphocytes, macrophages, and neutrophils. Each cell type differs significantly in structure, function, and clinical relevance. Conversely, MoNuSAC uses the label "epithelial" for a class that comprises both benign epithelial cells and malignant neoplastic cells. This practice makes it challenging to differentiate between benign and malignant epithelial cells in the dataset, which is a critical distinction when identifying tumor areas within tissue samples. To address these issues, we implement a cross-relabeling strategy as shown in \hyperref[fig:fig2]{Figure 2}. The key components are two classification models: one is trained on singular cell images from PanNuke data to classify the epithelial meta-class into epithelial and neoplastic classes. The other is trained on MoNuSAC to refine the inflammatory class into lymphocytes, neutrophils, and macrophages.

\begin{figure}[h!]
    \centering
    \includegraphics[width=\textwidth]{images/Figure_2.pdf}
    \caption{Refined dataset generation via cross relabeling}
    \label{fig:fig2}
\end{figure}

The refining approach consists of three consecutive steps. The first is the preprocessing step, in which we extract individual cells from both datasets (\hyperref[fig:fig3]{Figure 3}). The specifics of PanNuke and MoNuSAC patch preparation before cell preprocessing are provided in \hyperref[chap:S1]{Appendix S1}.

\begin{figure}[h!]
    \centering
    \includegraphics[width=\textwidth]{images/Figure_3.pdf}
    \caption{Cell instances preprocessing including (1) cell map extraction, (2) bounding box delineation, (3) adjusting cell boxes and (4) cropping and resizing of cell images}
    \label{fig:fig3}
\end{figure}

During preprocessing, we extract cell type maps from the ground truth label mask and calculate bounding boxes around each cell instance. To accommodate partial cells at patch borders, a common issue in cropped patch images, we employ mirror padding and extend the field of view of the cell label by 15 pixels to capture adjacent cells. We then crop and resize the identified regions to $64 \times 64$ pixels using bicubic interpolation.

The preprocessed PanNuke dataset comprises 68,031 neoplastic and 23,207 epithelial cell images, while MoNuSAC comprises  33,104 lymphocytes, 1,252 neutrophils, and 1,695 macrophages, which we subsequently use in training cell classification models and classifying the cell image data \hyperref[fig:S2]{Appendix Figure S2 (1)}. 

The next step is to train two distinct ResNet50-based classifiers tailored to address the specific labeling challenges inherent in each dataset. We use ResNet50 for classification models due to its proven effectiveness for image classification tasks in histopathology \cite{pan2022reviewmachinelearningapproaches}, and its compatibility with small images. For the PanNuke dataset, we design the classifier, trained on MoNuSAC data, to disaggregate the heterogeneous "inflammatory" cell category into distinct subtypes: lymphocytes, macrophages, and neutrophils. Similarly, for the MoNuSAC dataset, the classifier is trained on PanNuke data and distinguishes between benign and malignant epithelial cells within the overarching "epithelial" label. By applying these targeted classifiers to their respective datasets, we assign more specific labels to individual cell instances, thus enabling us to create a unified dataset.
To ensure a balanced representation of classes, we train both models on datasets that had been equalized to match the size of the least represented class. Thus, we obtain datasets comprising 23,207 samples per class for PanNuke and 1,252 samples per class for MoNuSAC data. Next, we partition both of them into training (70\%), validation (20\%), and testing (10\%) subsets. To mitigate the risk of overfitting, we use a single dropout layer with a rate of p=0.5 in both models and data augmentation using randomized color perturbations, rotation, and horizontal and vertical flipping. We employ AdamW optimizer and the cross-entropy loss function for the training criterion.

To evaluate the two trained models, we measure the classification accuracy on the respective test subsets. The accuracies on the test subset for both classifiers are presented in \hyperref[tab:1]{Table 1}. The PanNuke model achieves an average accuracy of 93.57\%, with higher accuracy for neoplastic cells (96.06\%) compared to epithelial cells (86.26\%). The confusion matrix in Figure A3.1 shows that the model predominantly distinguishes accurately between epithelial and neoplastic tissues, with a substantial number of correct classifications and relatively few misclassifications. The MoNuSAC model demonstrates an average accuracy of 98.92\%, excelling in classifying lymphocytes (99.67\%) and macrophages (94.12\%), with lower performance for neutrophils (85.71\%). The confusion matrix in Figure A3.2 shows that the model identifies lymphocytes and performs reasonably well with macrophages and neutrophils.

\begin{table}[h!]
\renewcommand{\arraystretch}{1.5}
  \centering
  \caption{Cell classification results for PanNuke and MoNuSAC trained models (CI 95\%).}
  \label{tab:1}
  \begin{tabular}{|l|c|c|}
   \hline
   %\rowcolor{gray!30}
    Accuracy               & PanNuke model              & MoNuSAC model              \\
    \hline
    Average      & 0.936 (0.931--0.941)         & 0.989 (0.986--0.993)        \\
    \hline
    Neoplastic   & 0.961 (0.956--0.965)         & -                          \\
    \hline
    Epithelial   & 0.863 (0.849--0.877)         & -                          \\
    \hline
    Lymphocytes  & -                          & 0.997 (0.995--0.999)        \\
    \hline
    Neutrophils  & -                          & 0.857 (0.796--0.918)        \\
    \hline
    Macrophages  & -                          & 0.941 (0.906--0.976)        \\
    \hline
  \end{tabular}
\end{table}

Finally, during the last step, we use the model trained on PanNuke data for epithelial cells in MoNuSAC and the model trained on MoNuSAC for the inflammatory cells class in PanNuke. Specifically, we use classifier models to relabel epithelial cells in MoNuSAC and inflammatory cells in PanNuke data. Then we combine cells with refined labels and the rest of the cells in both datasets to create a refined dataset (\hyperref[fig:S2]{Appendix Figure S2 (2)}). The process of relabeling cells and visualizing them on a patch is shown in \hyperref[fig:fig4]{Figure 4}. The cell counts in the refined dataset are provided in \hyperref[tab:S4]{Appendix Table S4}.

\begin{figure}[h!]
    \centering
    \includegraphics[width=\textwidth, height=0.42\textheight, keepaspectratio]{images/Figure_4.pdf}
    \caption{Cell relabeling procedure for epithelial and inflammatory cell classes}
    \label{fig:fig4}
\end{figure}

%\hfill

Relabeling and combining datasets have been explored in a prior study \cite{Parulekar_Kanwat_etal._2023}, where consecutive fine-tuning on multiple datasets was employed to account for hierarchical class label structures. While the method presented in \cite{Parulekar_Kanwat_etal._2023} is intuitive, it often lacks consistency and requires multiple fine-tuning runs, which can be cumbersome and time-consuming. 
In contrast, cross-relabeling simplifies this process by using specialized classification models tailored to each dataset's specific labeling challenges. This approach provides better transparency and produces a unified dataset encompassing seven distinct cell types across multiple tissue samples, enhancing data diversity for further model training or fine-tuning.

Despite these improvements, cross-relabeling does not entirely resolve issues related to poor labeling quality or the amount of labeled data. Specifically, our results show lower accuracies persist for underrepresented classes, such as macrophages, which may stem from a limited sample availability and intrinsic challenges in distinguishing these cells based solely on H\&E staining. Furthermore, while our method enhances label specificity, it relies on the initial quality of the broad labels; thus, any fundamental inaccuracies in the original annotations can propagate through the relabeling process. Addressing the overall problem of limited data labels may require integrating additional data sources or utilizing complementary immunohistochemical staining methods.
Although the reported performance metrics are obtained from evaluations on the native test sets of each dataset, it is important to note that the primary application of these classifiers is to perform cross-relabeling, where a model trained on one dataset (e.g., PanNuke) is applied to another (e.g., MoNuSAC) and vice versa. We acknowledge that a more systematic evaluation of cross-dataset generalization is needed and could be performed in future work.

Overall, the refined dataset produced by our approach can enhance the supervised training or fine-tuning of cell segmentation and classification models, especially those that utilize pre-trained foundation models to improve feature extraction robustness. In addition, these models can detect nuanced classes that enable researchers to conduct more detailed analyses of biological processes in computational pathology.

\section{Foundation models for robust cell segmentation and classification}

Accurate cell segmentation and classification in digital pathology are hindered by limited labeled data and the fact that conventional CNNs are unable to capture global contextual information due to their local receptive field constraints \cite{Gheflati_Rivaz_2022,Yang_Marcus_etal.}. Traditional approaches in cell quantification have predominantly relied on CNN encoders, such as ResNet50, given their proven effectiveness in semantic segmentation tasks \cite{Deshmane_2023,Graham_Vu_etal._2019,Mukasheva_Koishiyeva_etal._2024,Stringer_Wang_etal._2021}. However, approaches that include fine-tuning of pretrained CNNs, data augmentation, and stain normalization to partially increase data variability and address staining differences often fail to achieve the necessary generalization and robustness across diverse tissue types and staining conditions \cite{G._Wang_W._Li_etal._2018,Gao_Bagci_etal._2018,Karim_El_Khoury_Martin_Fockedey_etal._2021}.

To overcome these challenges, we leverage an encoder-decoder network that uses a foundation model as the encoder and a CNN upsampling decoder (\hyperref[fig:fig5]{Figure 5}) for simultaneous cell segmentation and classification in 2D patches extracted from WSIs. Foundation models with transformer-based architectures are viable alternatives to CNN-based encoders \cite{Shamshad_Khan_etal._2023,Sourget_2023}. They enable the creation of more advanced architectures that can decode or transform learned features more effectively \cite{Chen_Duan_etal._2023,Cheng_Misra_etal._2022,Xie_Wang_etal._2021}.

\begin{figure}[h!]
    \centering
    \includegraphics[width=\textwidth]{images/Figure_5.pdf}
    \caption{UNETR-like model with foundational model as backbone}
    \label{fig:fig5}
\end{figure}

By utilizing a transformer-based encoder, we incorporate global contextual information into the feature extraction process, which is a key advantage of such architectures \cite{Chen_Lu_etal._2021}. This foundation model integration facilitates accurate pixel-wise segmentation and classification without the need for extensive encoder training, thereby potentially improving generalization across varied cellular structures and tissue types.
In our implementation, we employ a modified UNETR \cite{Hatamizadeh_Tang_etal._2021} architecture that combines a vision transformer (ViT) \cite{Dosovitskiy_Beyer_etal._2021} encoder with a CNN-based decoder. The encoder utilizes the pretrained H-Optimus foundation model, which contains 1.1 billion parameters and is trained on over 500,000 H\&E stained WSIs \cite{Saillard_Jenatton_etal._2024}. We extract outputs from four evenly spaced transformer blocks $Z_i$, where $i \in [1, 14, 26, 38]$, to serve as residual connections for the CNN decoder. We select these blocks based on our observation that features from non-adjacent levels of the encoder lead to better overall performance on the test subset.

The CNN decoder upsamples the feature representations, acquired from the transformer blocks, to generate an intermediate vector that is handled by two task-specific layers that generate cell segmentation and classification masks. The first task-specific layer is the ‘Cellpose head’,  which is used to delineate cell instances. The layer generates horizontal and vertical gradient maps to form vector fields that are refined through gradient tracking in a post-processing step using the Cellpose algorithm \cite{Stringer_Wang_etal._2021}, known for its efficacy in cell segmentation tasks and generalizability across multiple domains \cite{Pachitariu_Stringer_2022,Stringer_Pachitariu_2024}. The second task-specific layer is the "Cell type head", which assigns labels to individual pixels. In the post-processing step, we determine the output classification label of each segmented cell instance by majority voting over the labeled pixels that comprise the cell in the segmentation map.

To evaluate model performance and measure the impact of adding a foundation model as backbone, we compare it to a ResNet50-based model. ResNet50 is a widely used solution for encoders in segmentation architectures in the medical domain \cite{Deshmane_2023,Graham_Vu_etal._2019,Mukasheva_Koishiyeva_etal._2024,Stringer_Wang_etal._2021}. For the H-Optimus-based model, we utilize frozen weights for the encoder and only fine-tune the decoder to take advantage of the extensive pre-training of the foundation model. For the ResNet50-based model we start with ImageNet \cite{Deng_Dong_etal.} weights and train both encoder and decoder parts. Hyperparameters for the training step are set to be identical, where possible, for comparable evaluation. 
For this evaluation, we deliberately use the PanNuke dataset to provide a standardized and controlled comparison between the H‑Optimus and ResNet50-based models (\hyperref[fig:S2]{Appendix Figure S2 (3)}). Specifically, we use two of the default PanNuke dataset splits (66\%) for training and validation, and reserve the third split (33\%) for testing.

To address the challenge of cell class imbalance in the PanNuke dataset, which is a common characteristic in most cell-level H\&E patch datasets, both models’ training processes employ a weighted loss function comprising cross-entropy and focal loss \cite{Lin_Goyal_etal._2018}. The focal loss component is adjusted with coefficients derived from each cell class' instance frequency, emphasizing learning from underrepresented classes and enhancing the model's sensitivity to rare but significant cellular patterns. The cross-entropy loss is augmented with spectral decoupling regularization \cite{Pezeshki_Kaba_etal._2021,Pohjonen_Stürenberg_etal._2022} and spatially varying label smoothing \cite{Islam_Glocker_2021}, which potentially stabilizes training and improves generalization in case of complex tissue morphologies. For optimization, we employ the \textit{AdamW} \cite{Loshchilov_Hutter_2019} to counter unbalanced class scenarios, with cosine annealing learning rate scheduler.

We utilize the scikit-learn library \cite{Van_der_Walt_Schönberger_etal._2014} and HoVer-Net \cite{Graham_Vu_etal._2019} implementations of $R^2$ (the coefficient of determination) and $PQ$ (panoptic quality) to evaluate our experiments. Complete mathematical formulations and detailed explanations of these metrics are provided in \hyperref[chap:S5]{Appendix S5}. To compute confidence intervals, we use nonparametric bootstrapping, where after calculating the metric on the full sample, we generated 1000 bootstrap replicates by resampling with replacement and then determined the 95\% confidence intervals as the 2.5th and 97.5th percentiles of the resulting empirical distribution.

%\hfill

The model comparisons are summarized in \hyperref[tab:2]{Table 2}. The H‑Optimus-based model achieves higher $R^2$ across all cell classes compared to the ResNet50-based model, which means that its predictions are more closely aligned with the PanNuke cell counts, indicating a stronger correlation with the observed data. Notably, the improvement of $R^2_{dead}$ may be an indicator of better global contextual representations provided by the foundation model backbone. In terms of segmentation and classification quality combined, measured by the PQ score, the H‑Optimus-based model demonstrates notable improvements across most cell classes. Overall, the average $R^2$ improved from 0.575 to 0.871, while the average $PQ$ score improved from 0.450 to 0.492, demonstrating better performance of the H-Optimus-based model.

\begin{table}[h!]
\renewcommand{\arraystretch}{1.5}
  \centering
  \caption{Cell quantification metrics for baseline and proposed models (CI 95\%).}
  \label{tab:2}
  \begin{tabular}{|l|c|c|}
    \hline
    %\rowcolor{gray!30}
    Metric             & Resnet50-based            & H-optimus-based              \\
    \hline
    $R^2_{neoplastic}$    & 0.681 (0.576--0.769)       & \textbf{0.941 (0.917--0.960)} \\
    \hline
    $R^2_{inflammatory}$  & 0.863 (0.778--0.903)       & \textbf{0.949 (0.918--0.966)} \\
    \hline
    $R^2_{connective}$    & 0.600 (0.488--0.698)       & 0.609 (0.436--0.772)          \\
    \hline
    $R^2_{dead}$          & 0.097 (-11.389--0.669)     & 0.925 (0.404--0.982)          \\
    \hline
    $R^2_{epithelial}$    & 0.635 (0.490--0.747)       & \textbf{0.930 (0.886--0.964)} \\
    \hline
    $PQ_{neoplastic}$       & 0.517 (0.499--0.535)       & \textbf{0.589 (0.575--0.604)} \\
    \hline
    $PQ_{inflammatory}$     & 0.455 (0.429--0.482)       & \textbf{0.528 (0.507--0.549)} \\
    \hline
    $PQ_{connective}$       & 0.416 (0.400--0.431)       & \textbf{0.451 (0.436--0.465)} \\
    \hline
    $PQ_{dead}$             & 0.374 (0.342--0.408)       & 0.292 (0.209--0.365)          \\
    \hline
    $PQ_{epithelial}$       & 0.488 (0.460--0.519)       & \textbf{0.599 (0.579--0.618)} \\
    \hline
  \end{tabular}
\end{table}

Our results  show that integrating the H‑Optimus foundation model within the UNETR architecture enhances the model's ability to segment and classify cells across diverse tissues from PanNuke data. The pretrained transformer encoder provides robust feature representations, resulting in higher average $R^2$ and $PQ$ scores compared to the CNN-based model. This leads to more reliable cell quantification and more accurate downstream analysis. Additionally, the streamlined fine-tuning process reduces computational overhead and training time, making the model more adaptable for new data.

Despite these advancements, the foundation model-based approach does not fully resolve all challenges related to cell segmentation and classification. We observe lower metric scores for underrepresented classes in the training data. Furthermore, foundation models typically encompass billions of parameters, resulting in substantial computational and memory requirements. It therefore poses challenges for deployment in resource-constrained environments, limiting their practical applicability in certain clinical settings.

\section{Model optimization via Knowledge Distillation}

To address the limitations posed by the extensive size of foundation models, we implement knowledge distillation — a model compression technique that leverages the teacher-student paradigm \cite{Hinton_Vinyals_etal._2015}. By training a smaller, more efficient student model to replicate the output of a larger, pre-trained teacher model, we retain performance while significantly reducing the model's complexity and resource requirements (\hyperref[fig:fig6]{Figure 6}).

\begin{figure}[h!]
    \centering
    \includegraphics[width=\textwidth, height=0.45\textheight, keepaspectratio]{images/Figure_6.pdf}
    \caption{Knowledge distillation framework for training a student model using a pre-trained teacher}
    \label{fig:fig6}
\end{figure}

We employ knowledge distillation to compress the H‑Optimus-based teacher model into a more efficient student model. The teacher model is the modified UNETR architecture with the H‑Optimus foundation model described in the previous chapter. The student model is based on a UNet architecture augmented with residual connections and incorporates a smaller ViT encoder with 9 million parameters \cite{Steiner_Kolesnikov_etal._2022,Wightman_2019}. 

First, we fine-tune the teacher model using the refined dataset from the cross-relabeling procedure (Section 2). Initially we train the decoder of the teacher model while keeping the encoder weights frozen. We split the refined dataset into train (70\%), validation (20\%) and test (10\%) subsets (\hyperref[fig:S2]{Appendix Figure S2 (4)}). During fine-tuning, we use the train and validation subsets, while leaving the test subset for model evaluation. We set the training procedure and model hyperparameters to be identical to those that were used to demonstrate the utility of foundation models for the simultaneous cell segmentation and classification task.

Next, we perform knowledge distillation from teacher to student using the refined dataset used to fine-tune the teacher model. The student model is trained to replicate the teacher model's outputs. We utilize a specialized loss function that aligns the student's predicted probability distribution with the teacher's, incorporating the teacher's class probability distribution derived from the output. Following the methodology of Hinton et al. \cite{Hinton_Vinyals_etal._2015}, we experiment with various hyperparameter settings for the temperature ($T$) and the balancing coefficients ($\alpha$ and $\beta$) in the loss function. We vary $T$ from 1 to 20 and adjust $\alpha$ and $\beta$ to balance the distillation and student losses. Through iterative tuning and evaluation, we identify that setting $T=14$, $\alpha=0.3$, and $\beta=0.7$ yields a configuration that converges and closely approximates the teacher model's performance during training.

Finally, we assess the performance of both models using the $R^2$ and $PQ$ (defined in \hyperref[chap:S5]{Appendix S5}) on the test set of the refined dataset (\hyperref[tab:3]{Table 3}). We observe that the 95\% confidence intervals overlap for most cell types, so we cannot claim statistically significant performance differences between the teacher and student models. One exception appears in the neoplastic class. The teacher model produces an $R^2$ of 0.919, while the student model shows an $R^2$ of 0.852. In addition, the student model achieves higher $PQ$ values for the neoplastic and connective classes, though the confidence intervals show overlap.

\begin{table}[h!]
\renewcommand{\arraystretch}{1.5}
  \centering
  \caption{Cell quantification metrics for teacher and distilled student models (CI 95\%).}
  \label{tab:3}
  \begin{tabular}{|l|c|c|}
    \hline
    %\rowcolor{gray!30}
    Metric & Teacher & Student \\
    \hline
    $R^2_{neoplastic}$    & \textbf{0.919} (0.898--0.939) & 0.852 (0.800--0.891) \\
    \hline
    $R^2_{lymphocyte}$    & 0.969 (0.956--0.977)         & 0.969 (0.956--0.978) \\
    \hline
    $R^2_{connective}$    & 0.694 (0.548--0.809)         & 0.618 (0.469--0.741) \\
    \hline
    $R^2_{dead}$          & 0.755 (0.400--0.908)         & 0.424 (0.100--0.731) \\
    \hline
    $R^2_{epithelial}$    & 0.922 (0.870--0.958)         & 0.843 (0.738--0.917) \\
    \hline
    $R^2_{macrophage}$    & 0.384 (-0.369--0.724)        & 0.704 (0.352--0.859) \\
    \hline
    $R^2_{neutrofil}$     & 0.854 (0.578--0.929)         & 0.833 (0.502--0.925) \\
    \hline
    $PQ_{neoplastic}$       & 0.581 (0.569--0.593)         & 0.601 (0.588--0.613) \\
    \hline
    $PQ_{lymphocyte}$       & 0.536 (0.520--0.553)         & 0.563 (0.544--0.579) \\
    \hline
    $PQ_{connective}$       & 0.436 (0.421--0.451)         & 0.457 (0.441--0.474) \\
    \hline
    $PQ_{dead}$             & 0.272 (0.235--0.315)         & 0.279 (0.201--0.369) \\
    \hline
    $PQ_{epithelial}$       & 0.522 (0.500--0.545)         & 0.530 (0.506--0.555) \\
    \hline
    $PQ_{macrophage}$       & 0.524 (0.459--0.588)         & 0.474 (0.405--0.543) \\
    \hline
    $PQ_{neutrofil}$        & 0.541 (0.490--0.592)         & 0.565 (0.522--0.607) \\
    \hline
  \end{tabular}
\end{table}


We further decompose the $PQ$ metric into its $SQ$ and $DQ$ components (\hyperref[tab:S6]{Appendix Table S6}). Both models produce nearly identical $SQ$ values, which indicates that they predict instance boundaries with similar precision. Although the student model shows some improvement in $DQ$ scores for certain classes, the confidence intervals overlap and do not confirm a statistically significant difference.

We observe that the student and teacher models yield comparable detection performance despite the student model using a much smaller and simpler architecture. A model with fewer parameters reduces the risk of overfitting when training data are scarce relative to the model’s complexity \cite{Farias_Ludermir_etal._2022}. The knowledge distillation process also encourages the student model to focus on the most generalizable detection features learned from the teacher. These factors enable the student model to achieve similar detection performance across different cell types.

Additionally, considering the model sizes reported in \hyperref[tab:4]{Table 4}, the distilled model achieves a significant reduction compared to the teacher model, with a 48-fold decrease in parameter count and a 5.5-fold reduction in on-disk size. In inference mode, the teacher model requires 16 GB of VRAM for a batch size of 32, while the distilled model only needs 3 GB of VRAM for the same batch size. These reductions make the distilled model significantly more practical for fine-tuning and deployment in resource-constrained environments.

\begin{table}[h!]
\renewcommand{\arraystretch}{1.5}
  \centering
  \caption{Parameter counts and size of teacher and distilled model}
  \label{tab:4}
  \adjustbox{max width=\textwidth}{%
  \begin{tabular}{|l|c|c|c|}
    \hline
    %\rowcolor{gray!30}
    Metric & H-optimus-based (Teacher) & mobileViT-based (Student) & Magnitude of difference \\
    \hline
    Parameters count       & 1,158,917,906   & \textbf{24,093,393}   & \textbf{48x}  \\
    \hline
    Estimated Total Size (MB) & 87,912       & \textbf{15,935}    & \textbf{5.5x} \\
    \hline
  \end{tabular}%
}
\end{table}

%\hfill

With recent advancements in complex network architectures and the use of pretrained encoders to achieve state-of-the-art performance \cite{Baumann_Dislich_etal._2024,Hörst_Rempe_etal._2024} in cell segmentation and classification tasks, model size, computational complexity, and processing times have increased. This limits the scalability and accessibility of these models. As we demonstrate, this may be mitigated using knowledge distillation. Studies in the field of natural language processing have demonstrated the efficacy of knowledge distillation in retaining the capabilities of the teacher model while achieving significant reductions in size and complexity \cite{Huangpu_Gao_2024,Sun_Yu_etal.}. 

We demonstrate the feasibility of knowledge distillation in digital pathology, specifically for cell segmentation and classification tasks. Moreover, we achieve this performance while also significantly reducing the parameter count. In addressing the challenge of knowledge transfer, we found that distillation from a transformer-based model to a smaller transformer is more straightforward than attempting to map transformer features to CNN blocks. In our experiments, using a CNN-based network as a student results in worse cell quantification performance due to the structural constraints of CNN feature space dimensions. 

Although our primary approach relies on a transformer-based student model that performs well, it can be further optimized to incorporate advantages from CNN architectures. For example, employing alternative techniques such as using ViT adapters \cite{Chen_Duan_etal._2023} or $1 \times 1$ convolutions to adjust feature map sizes may be beneficial for harnessing CNN advantages like enhanced local feature extraction. Moreover, if additional performance improvements are desired, the process can be further enhanced by applying supplementary knowledge distillation techniques, such as self-distillation \cite{Zhang_Song_etal._2019} or online distillation \cite{Houyon_Cioppa_etal._2023}.

Despite these promising results, further validation on independent datasets is necessary to fully understand the model's limitations. Underrepresented classes may pose challenges when addressing complex cases. Pathologists need to validate these models to adopt them in clinical settings. While the distilled models are smaller and more deployable, a technological gap persists because pathologists traditionally rely on established methods for inspecting WSIs and diagnosing diseases. Addressing the complexities involved in deploying models for inference and supporting pathologists in adopting new tools is essential for integrating these models into clinical workflows.

\section{Model integration with QuPath}
Digital pathology tools with graphical user interfaces are essential for visualizing and analyzing WSIs. To make our student model useful in clinical pathology workflows, it needs to be integrated into a tool that enables inspecting regions, creating annotations, and providing quantitative analyses of biomarkers. Therefore, we integrate the trained student model from the previous chapter into the QuPath open‑source platform \cite{Bankhead_Loughrey_etal._2017}. QuPath provides the required annotation, visualization, and analysis tools to interpret complex histological data, including workflows for cell segmentation, classification, and quantification (\hyperref[fig:fig7]{Figure 7}). 

\begin{figure}[h!]
    \centering
    \includegraphics[width=\textwidth]{images/Figure_7.pdf}
    \caption{Visualization of model-generated cell quantification annotations (left) and the corresponding unannotated slide (right) in QuPath}
    \label{fig:fig7}
\end{figure}

To identify the regions in a WSI critical for prognosticating tumor development, such as specific tumor areas or border regions without overlapping healthy tissue, the pathologist uses QuPath to outline these regions. Then, the pathologist initiates a cell segmentation and classification script through the QuPath interface for the selected regions. The resulting annotations and quantified cell information are then directly overlaid onto the WSI in the QuPath interface. Additional design and implementation details are in \hyperref[chap:S7]{Appendix S7}. 

Two common approaches for integrating deep learning models into QuPath are Java‑based native QuPath extensions \cite{Goldsborough_Philps_etal._2024} and the execution of RESTful API requests to a model server coupled with handling the response via an extension, as demonstrated in the application of cell segmentation models applied to immunofluorescence images \cite{Sugawara_2023}. While the community is actively working on these integration strategies, there is currently no universal solution that fully addresses all integration and performance requirements.

Extensions may offer better integration with QuPath, allowing slightly improved performance and more widespread usage of the built-in QuPath models, but they lack the flexibility to customize models and modify their behavior. For example, the newest version of QuPath includes models such as StarDist \cite{Weigert_Schmidt} and InstanSeg \cite{Goldsborough_Philps_etal._2024} that can perform cell segmentation. Both models pose limitations when applied to simultaneous cell segmentation and classification. StarDist performs well only on convex, round shapes by design, whereas some neoplastic, inflammatory, and connective cells exhibit complex and non-convex shapes. InstanSeg provides only semantic segmentation without assigning classes to the segmented cells.

%\hfill

In contrast, our approach offers an alternative integration strategy. It utilizes the paquo library to directly interact with QuPath’s internal application programming interface from within Python. This enables data exchange and processing without the need for intermediate conversion steps and provides greater control over model customization, retraining, and the incorporation of custom processing steps.

The integration of our custom model with QuPath underscores its potential to significantly enhance the diagnostic process by reducing the time burden on pathologists and enabling them to focus on more complex interpretative tasks using familiar software. Leveraging a tool that is already well-established among pathologists increases the likelihood of its adoption into daily clinical workflows. The quantitative data generated through the automated workflow is critical for both clinical decision-making and research, facilitating more accurate biomarker analysis, enabling robust statistical evaluations, and supporting hypothesis generation and testing. Additionally, by streamlining cell segmentation and classification, the tool enhances the scalability and reproducibility of pathological assessments, ultimately contributing to improved diagnostic accuracy and patient outcomes.

\section{Conclusion and future work}

In this study, we address critical challenges in digital pathology and tackle the usability and deployment issues of the developed models in standard computing environments without the need for high-performance computing systems. Our multi-faceted approach encompasses data refinement through cross-relabeling, leveraging foundation models for robust cell segmentation and classification, optimizing model performance via knowledge distillation, and integrating the optimized model into the QuPath software for practical application. This approach is used to construct a capable, versatile, and adjustable model for cell segmentation and classification, with enhanced performance and usability.

\begin{sloppypar}
While our approach shows potential in the field of computational pathology, certain limitations persist. 
For example, our implementation currently exhibits lower performance in detecting macrophages. 
This serves as an instance of the broader challenge of accurately identifying complex cell types. In order to address this issue, extending our approach to incorporate additional data sources, exploring alternative modeling approaches, and integrating other imaging modalities such as immunohistochemical staining may help improve detection accuracy. Moreover, although the distilled model reduces computational demands, integrating advanced deep learning models into clinical practice requires addressing technological gaps and potential resistance to adopting new tools within established diagnostic processes.
\end{sloppypar}

Future work could focus on several key areas to refine the proposed approach and facilitate its adoption in clinical environments. Enhancing the cell-relabeling process with additional datasets \cite{Graham_Jahanifar_etal._2021} could improve the representation of underrepresented cell types and enhance overall model performance. Also, incorporating additional data sources, such as multi-modal imaging or complementary staining methods, may address limitations related to cell type differentiation and class imbalance. Exploring other foundation models \cite{Vorontsov_Bozkurt_etal._2024,Zimmermann_Vorontsov_etal._2024} or introducing additional modalities \cite{Ding_Wagner_etal._2024,Vaidya_Zhang_etal._2025} may provide alternative architectures better suited to specific tasks or offer improved efficiency. Implementing more complex knowledge distillation techniques \cite{Houyon_Cioppa_etal._2023,Zhang_Song_etal._2019} could further optimize the model's performance and adaptability. Additionally, deeper integration with QuPath or other digital pathology software could provide pathologists more control over cell quantification analysis directly within the QuPath interface, thereby increasing accessibility and usability. Such enhancements would not only refine model performance but also ensure greater adaptability and scalability within various clinical environments. Finally, extensive validation of the model by pathologists and benchmarking against independent datasets are essential steps toward establishing the model's reliability and fostering confidence in its clinical utility.

\section*{Acknowledgments} 
This work was funded in part by the Research Council of Norway grant no. 309439 SFI Visual Intelligence, and the North Norwegian Health Authority grant no. HNF1521-20.

\bibliographystyle{IEEEtran}
\begin{sloppypar}
\begin{thebibliography}{99}

\bibitem{chaplot2020neural} Chaplot, Devendra Singh, et al. "Neural topological slam for visual navigation." Proceedings of the IEEE/CVF conference on computer vision and pattern recognition. 2020.

\bibitem{maksymets2021thda} Maksymets, Oleksandr, et al. "Thda: Treasure hunt data augmentation for semantic navigation." Proceedings of the IEEE/CVF International Conference on Computer Vision. 2021.

\bibitem{mezghan2022memory} Mezghan, Lina, et al. "Memory-augmented reinforcement learning for image-goal navigation." 2022 IEEE/RSJ International Conference on Intelligent Robots and Systems (IROS). IEEE, 2022.

\bibitem{al2022zero} Al-Halah, Ziad, Santhosh Kumar Ramakrishnan, and Kristen Grauman. "Zero experience required: Plug \& play modular transfer learning for semantic visual navigation." Proceedings of the IEEE/CVF Conference on Computer Vision and Pattern Recognition. 2022.

\bibitem{ye2021auxiliary} Ye, Joel, et al. "Auxiliary tasks and exploration enable objectgoal navigation." Proceedings of the IEEE/CVF international conference on computer vision. 2021.

\bibitem{chaplot2020object} Chaplot, Devendra Singh, et al. "Object goal navigation using goal-oriented semantic exploration." Advances in Neural Information Processing Systems 33 (2020)

\bibitem{ramakrishnan2022poni} Ramakrishnan, Santhosh Kumar, et al. "Poni: Potential functions for objectgoal navigation with interaction-free learning." Proceedings of the IEEE/CVF Conference on Computer Vision and Pattern Recognition. 2022.

\bibitem{ramrakhya2022habitat} Ramrakhya, Ram, et al. "Habitat-web: Learning embodied object-search strategies from human demonstrations at scale." Proceedings of the IEEE/CVF Conference on Computer Vision and Pattern Recognition. 2022.

\bibitem{mousavian2019visual} Mousavian, Arsalan, et al. "Visual representations for semantic target driven navigation." 2019 International Conference on Robotics and Automation (ICRA). IEEE, 2019.

\bibitem{dhariwal2021diffusion} Dhariwal, Prafulla, and Alexander Nichol. "Diffusion models beat gans on image synthesis." Advances in neural information processing systems 34 (2021)

\bibitem{ho2022classifier} Ho, Jonathan, and Tim Salimans. "Classifier-free diffusion guidance." arXiv preprint arXiv:2207.12598 (2022).

\bibitem{nichol2021glide} Nichol, Alex, et al. "Glide: Towards photorealistic image generation and editing with text-guided diffusion models." arXiv preprint arXiv:2112.10741 (2021)

\bibitem{brooks2023instructpix2pix} Brooks, Tim, Aleksander Holynski, and Alexei A. Efros. "Instructpix2pix: Learning to follow image editing instructions." Proceedings of the IEEE/CVF Conference on Computer Vision and Pattern Recognition. 2023.

\bibitem{fu2023guiding} Fu, Tsu-Jui, et al. "Guiding instruction-based image editing via multimodal large language models." arXiv preprint arXiv:2309.17102 (2023).

\bibitem{geng2024instructdiffusion} Geng, Zigang, et al. "Instructdiffusion: A generalist modeling interface for vision tasks." Proceedings of the IEEE/CVF Conference on Computer Vision and Pattern Recognition. 2024.

\bibitem{zhou2024minedreamer} Zhou, Enshen, et al. "Minedreamer: Learning to follow instructions via chain-of-imagination for simulated-world control." arXiv preprint arXiv:2403.12037 (2024).

\bibitem{zhou2023esc} Zhou, Kaiwen, et al. "Esc: Exploration with soft commonsense constraints for zero-shot object navigation." International Conference on Machine Learning. PMLR, 2023.

\bibitem{yu2023l3mvn} Yu, Bangguo, Hamidreza Kasaei, and Ming Cao. "L3mvn: Leveraging large language models for visual target navigation." 2023 IEEE/RSJ International Conference on Intelligent Robots and Systems (IROS). IEEE, 2023.

\bibitem{gadre2023cows} Gadre, Samir Yitzhak, et al. "Cows on pasture: Baselines and benchmarks for language-driven zero-shot object navigation." Proceedings of the IEEE/CVF Conference on Computer Vision and Pattern Recognition. 2023.

\bibitem{shah2023navigation} Shah, Dhruv, et al. "Navigation with large language models: Semantic guesswork as a heuristic for planning." Conference on Robot Learning. PMLR, 2023.

\bibitem{cai2024bridging} Cai, Wenzhe, et al. "Bridging zero-shot object navigation and foundation models through pixel-guided navigation skill." 2024 IEEE International Conference on Robotics and Automation (ICRA). IEEE, 2024.

\bibitem{yu2023co} Yu, Bangguo, Hamidreza Kasaei, and Ming Cao. "Co-NavGPT: Multi-robot cooperative visual semantic navigation using large language models." arXiv preprint arXiv:2310.07937 (2023).

\bibitem{wu2024voronav} Wu, Pengying, et al. "Voronav: Voronoi-based zero-shot object navigation with large language model." arXiv preprint arXiv:2401.02695 (2024).

\bibitem{qin2023mp5} Qin, Yiran, et al. "Mp5: A multi-modal open-ended embodied system in minecraft via active perception." arXiv preprint arXiv:2312.07472 (2023).

\bibitem{du2024learning} Du, Yilun, et al. "Learning universal policies via text-guided video generation." Advances in Neural Information Processing Systems 36 (2024).

\bibitem{ajay2024compositional} Ajay, Anurag, et al. "Compositional foundation models for hierarchical planning." Advances in Neural Information Processing Systems 36 (2024).

\bibitem{liang2024skilldiffuser} Liang, Zhixuan, et al. "Skilldiffuser: Interpretable hierarchical planning via skill abstractions in diffusion-based task execution." Proceedings of the IEEE/CVF Conference on Computer Vision and Pattern Recognition. 2024.

\bibitem{heusel2017gans} Heusel, Martin, et al. "Gans trained by a two time-scale update rule converge to a local nash equilibrium." Advances in neural information processing systems 30 (2017).

\bibitem{zhang2018unreasonable} Zhang, Richard, et al. "The unreasonable effectiveness of deep features as a perceptual metric." Proceedings of the IEEE conference on computer vision and pattern recognition. 2018.

\bibitem{brown2020language} Brown, Tom B. "Language models are few-shot learners." arXiv preprint arXiv:2005.14165 (2020).

\bibitem{podell2023sdxl} Podell, Dustin, et al. "Sdxl: Improving latent diffusion models for high-resolution image synthesis." arXiv preprint arXiv:2307.01952 (2023).

\bibitem{brohan2022rt} Brohan, Anthony, et al. "Rt-1: Robotics transformer for real-world control at scale." arXiv preprint arXiv:2212.06817 (2022).

\bibitem{brohan2023rt} Brohan, Anthony, et al. "Rt-2: Vision-language-action models transfer web knowledge to robotic control." arXiv preprint arXiv:2307.15818 (2023).

\bibitem{li2024manipllm} Li, Xiaoqi, et al. "Manipllm: Embodied multimodal large language model for object-centric robotic manipulation." Proceedings of the IEEE/CVF Conference on Computer Vision and Pattern Recognition. 2024.

\bibitem{shah2023vint} Shah, Dhruv, et al. "ViNT: A foundation model for visual navigation." arXiv preprint arXiv:2306.14846 (2023).

\bibitem{liu2024visual} Liu, Haotian, et al. "Visual instruction tuning." Advances in neural information processing systems 36 (2024).

\bibitem{hu2021lora} Hu, Edward J., et al. "Lora: Low-rank adaptation of large language models." arXiv preprint arXiv:2106.09685 (2021).

\bibitem{qin2023supfusion} Qin, Yiran, et al. "SupFusion: Supervised LiDAR-camera fusion for 3D object detection." Proceedings of the IEEE/CVF International Conference on Computer Vision. 2023.

\bibitem{qin2024worldsimbench} Qin, Yiran, et al. "Worldsimbench: Towards video generation models as world simulators." arXiv preprint arXiv:2410.18072 (2024).

\bibitem{yu2025gamefactory} Yu, Jiwen, et al. "GameFactory: Creating New Games with Generative Interactive Videos." arXiv preprint arXiv:2501.08325 (2025).

\bibitem{zhou2024code} Zhou, Enshen, et al. "Code-as-Monitor: Constraint-aware Visual Programming for Reactive and Proactive Robotic Failure Detection." arXiv preprint arXiv:2412.04455 (2024).

\bibitem{zhang2024ad} Zhang, Zaibin, et al. "AD-H: Autonomous Driving with Hierarchical Agents." arXiv preprint arXiv:2406.03474 (2024).

\bibitem{wang2024toward} Wang, Chaoqun, et al. "Toward Accurate Camera-based 3D Object Detection via Cascade Depth Estimation and Calibration." arXiv preprint arXiv:2402.04883 (2024).

\bibitem{huang2024story3d} Huang, Yuzhou, et al. "Story3d-agent: Exploring 3d storytelling visualization with large language models." arXiv preprint arXiv:2408.11801 (2024).

\bibitem{savinov2018semi} Savinov, Nikolay, Alexey Dosovitskiy, and Vladlen Koltun. "Semi-parametric topological memory for navigation." arXiv preprint arXiv:1803.00653 (2018).

\bibitem{majumdar2022zson} Majumdar, Arjun, et al. "Zson: Zero-shot object-goal navigation using multimodal goal embeddings." Advances in Neural Information Processing Systems 35 (2022): 32340-32352.

\bibitem{yadav2023offline} Yadav, Karmesh, et al. "Offline visual representation learning for embodied navigation." Workshop on Reincarnating Reinforcement Learning at ICLR 2023. 2023.

\bibitem{yadav2023ovrl} Yadav, Karmesh, et al. "Ovrl-v2: A simple state-of-art baseline for imagenav and objectnav." arXiv preprint arXiv:2303.07798 (2023).

\bibitem{sun2024fgprompt} Sun, Xinyu, et al. "FGPrompt: fine-grained goal prompting for image-goal navigation." Advances in Neural Information Processing Systems 36 (2024).

\bibitem{zhu2017target} Zhu, Yuke, et al. "Target-driven visual navigation in indoor scenes using deep reinforcement learning." 2017 IEEE international conference on robotics and automation (ICRA). IEEE, 2017.

\bibitem{koh2024generating} Koh, Jing Yu, Daniel Fried, and Russ R. Salakhutdinov. "Generating images with multimodal language models." Advances in Neural Information Processing Systems 36 (2024).

\bibitem{krantz2022instance} Krantz, Jacob, et al. "Instance-specific image goal navigation: Training embodied agents to find object instances." arXiv preprint arXiv:2211.15876 (2022).

\bibitem{schulman2017proximal} Schulman, John, et al. "Proximal policy optimization algorithms." arXiv preprint arXiv:1707.06347 (2017).

\bibitem{anderson2018evaluation} Anderson, Peter, et al. "On evaluation of embodied navigation agents." arXiv preprint arXiv:1807.06757 (2018).

\bibitem{lin2024navcot} Lin, Bingqian, et al. "NavCoT: Boosting LLM-Based Vision-and-Language Navigation via Learning Disentangled Reasoning." arXiv preprint arXiv:2403.07376 (2024).

\bibitem{NavGPT} Zhou, Gengze, Yicong Hong, and Qi Wu. "Navgpt: Explicit reasoning in vision-and-language navigation with large language models." Proceedings of the AAAI Conference on Artificial Intelligence.

\bibitem{hahn2021no} Hahn, Meera, et al. "No rl, no simulation: Learning to navigate without navigating." Advances in Neural Information Processing Systems 34 (2021): 26661-26673.

\bibitem{li2025t2isafety} Li, Lijun, et al. "T2ISafety: Benchmark for Assessing Fairness, Toxicity, and Privacy in Image Generation." arXiv preprint arXiv:2501.12612 (2025).

\bibitem{an2024agfsync} An, Jingkun, et al. "AGFSync: Leveraging AI-Generated Feedback for Preference Optimization in Text-to-Image Generation." arXiv preprint arXiv:2403.13352 (2024).


\end{thebibliography}
\end{sloppypar}

\clearpage
\beginsupplement
\section*{Appendix}
\renewcommand{\thesubsection}{S\arabic{subsection}}

\subsection{\label{chap:S1}PanNuke and MoNuSAC preprocessing}
The PanNuke dataset comprises a set of 7,901 RGB patches, each with dimensions of $256 \times 256$ pixels, which we set as the standard patch size for our analysis. In contrast, the MoNuSAC dataset encompasses 294 images of heterogeneous dimensions. To standardize the MoNuSAC images with our experiments, we implement a standardization protocol. Specifically, for images exceeding the dimensions of $256 \times 256$ pixels, we segment them into equal-sized patches and apply mirror padding to the remaining portions to avoid information loss at the peripherals. Patches with dimensions less than $128 \times 128$ pixels are excluded from the dataset due to the insufficient resolution to capture relevant cellular details. For patches where either dimension falls between 128 and 256 pixels, we employ upsampling to achieve the standard patch size. As a result, we obtain a total of 2,823 RGB patches derived from the MoNuSAC dataset for subsequent analysis. For additional details on the MoNuSAC data preparation process, refer to the source code \cite{Shvetsov_2025a}.
\clearpage

\subsection{\label{chap:S2}Data usage for the methodology}

\counterwithin{figure}{subsection}
\renewcommand{\thefigure}{S\arabic{subsection}}

\begin{figure}[h!]
    \centering
    \includegraphics[width=\textwidth, height=0.85\textheight, keepaspectratio]{images/A2.pdf}
    \caption{Overview of the methodology for cross-labeling, dataset refinement, and model comparison. (1) Cross-relabeling - training and testing cell classification models, (2) Cross-relabeling - using cell classification models to create refined dataset, (3) Fine-tuning and training models for comparison, (4) Student knowledge distillation with refined dataset}
    \label{fig:S2}
\end{figure}
\clearpage

\subsection{\label{chap:S3}Confusion matrices for classification models}
\counterwithin{figure}{subsection}
\renewcommand{\thefigure}{S\arabic{subsection}.\arabic{figure}}

\begin{figure}[h!]
    \centering
    \includegraphics[width=\textwidth, height=0.4\textheight, keepaspectratio]{images/A3_1.pdf}
    \caption{Confusion matrix for PanNuke trained model}
    \label{fig:S3.1}
\end{figure}

\begin{figure}[h!]
    \centering
    \includegraphics[width=\textwidth, height=0.4\textheight, keepaspectratio]{images/A3_2.pdf}
    \caption{Confusion matrix for MoNuSAC trained model}
    \label{fig:S3.2}
\end{figure}

\clearpage

\subsection{\label{chap:S4}Datasets cell counts}

\counterwithin{table}{subsection}
\renewcommand{\thetable}{S\arabic{subsection}}

\begin{table}[h!]
\renewcommand{\arraystretch}{2.0}
\centering
\caption{\label{tab:S4}Cell counts for PanNuke, MoNuSAC and refined datasets. Numbers in parentheses indicate preprocessed cell counts for cell classifier models training and testing.}
%\adjustbox{max width=\textwidth}{%
\begin{tabular}{|l|c|c|c|}
\hline
%\rowcolor{gray!30}
Cell type & PanNuke & MoNuSAC & Refined \\
\hline
Neoplastic & 77,403 (68,031) & - & 105,451 \\
\hline
Epithelial & 26,572 (23,207) & - & 29,926 \\
\hline
Epithelial (benign and malignant) & - & 31,402 & - \\
\hline
Inflammatory & 32,276 & - & - \\
\hline
Lymphocytes & - & 37,045 (33,104) & 65,275 \\
\hline
Neutrophils & - & 1,355 (1,252) & 3,833 \\
\hline
Macrophage & - & 1,842 (1,695) & 3,410 \\
\hline
Dead & 2,908 & - & 2,908 \\
\hline
Connective & 50,585 & - & 50,585 \\
\hline
\end{tabular}
%
%}
\end{table}



\clearpage

\subsection{\label{chap:S5}Definition of validation metrics}
\counterwithin{equation}{subsection}
\renewcommand{\theequation}{\arabic{equation}}

\subsubsection{\label{chap:S5.1}R\textsuperscript{2}}
The coefficient of determination, denoted as $R^2$, is a statistical measure that represents the proportion of variance in the dependent variable that is predictable from the independent variables. In the context of cell quantification in pathology, $R^2$ is used to assess how well the predicted quantities of different cell types in a patch align with the actual quantities observed in the ground truth data, with higher values representing more accurate quantification. $R^2$ is defined as
\begin{equation*}
R^2 = 1 - \frac{\sum_{i=1}^n (y_i - \hat{y}_i)^2}{\sum_{i=1}^n (y_i - \bar{y})^2},
\end{equation*}
where $y_i$ represents the actual number of cells of a specific type in the $i$-th image, $\hat{y}_i$ represents the predicted number of cells of that type in the $i$-th image, $\bar{y}$ is the mean of the actual numbers across all images, and $n$ is the total number of images in the dataset.

The $R^2$ metric has a range of $(-\infty, 1]$. An $R^2$ of 1 indicates perfect prediction, where all predicted values exactly match the actual values. An $R^2$ of 0 suggests that the model explains none of the variability of the response data around its mean. If $R^2$ is negative, it indicates that the model performs worse than a model that simply predicts the mean of the actual values for all observations.

\subsubsection{\label{chap:S5.2}PQ}
Panoptic Quality ($PQ$) is a comprehensive metric used to evaluate the performance of segmentation models in tasks that require both instance segmentation and classification. $PQ$ provides a single score that encapsulates both the detection accuracy (i.e., how many objects were correctly identified) and the segmentation quality (i.e., how accurately the objects' boundaries were delineated). This metric is particularly useful in multiclass scenarios where each pixel is classified into distinct categories, such as different cell types in pathology images.

$PQ$ is calculated as the product of two terms: Detection Quality ($DQ$) and Segmentation Quality ($SQ$). It can be expressed as
\begin{equation*}
PQ = DQ \cdot SQ,
\end{equation*}
where
\begin{equation*}
DQ = \frac{TP}{TP + 0.5\, FP + 0.5\, FN},
\end{equation*}
\begin{equation*}
SQ = \frac{\sum_{(p, g) \in \mathcal{M}} IoU(p, g)}{TP}.
\end{equation*}
In these formulas, $TP$ denotes the number of correctly matched instances between ground truth and prediction, $FP$ denotes the predicted instances that have no corresponding ground truth, $FN$ denotes the ground truth instances that were not detected, $IoU(p, g)$ is the Intersection over Union for a pair of matched instances $p$ (prediction) and $g$ (ground truth), and $\mathcal{M}$ is the set of matched pairs.

The $PQ$ metric is calculated for each class and is averaged across classes to provide a global performance measure.

The $PQ$ score has a range of $[0, 1.0]$, where a higher score indicates better performance in both detecting and segmenting the instances correctly. A $PQ$ of 1 signifies perfect identification and segmentation of all instances, whereas a $PQ$ of 0 indicates that no instances were correctly identified and segmented.

\clearpage

\subsection{\label{chap:S6}Segmentation and Detection quality metrics for teacher and student models}

\begin{table}[h!]
\renewcommand{\arraystretch}{2.0}
\centering
\caption{Segmentation and detection quality for student and teacher models (CI 95\%)}
\label{tab:S6}
%\adjustbox{max width=\textwidth}{%
\begin{tabular}{|l|c|c|}
\hline
%\rowcolor{gray!30}
Metric & Teacher & Student \\
\hline
$SQ_{neoplastic}$ & 0.819 (0.815--0.823) & 0.824 (0.819--0.828) \\
\hline
$SQ_{lymphocyte}$ & 0.795 (0.788--0.802) & 0.790 (0.783--0.796) \\
\hline
$SQ_{connective}$ & 0.770 (0.762--0.776) & 0.780 (0.772--0.786) \\
\hline
$SQ_{dead}$ & 0.659 (0.623--0.688) & 0.657 (0.624--0.695) \\
\hline
$SQ_{epithelial}$ & 0.780 (0.770--0.790) & 0.788 (0.779--0.797) \\
\hline
$SQ_{macrophage}$ & 0.788 (0.760--0.810) & 0.757 (0.730--0.783) \\
\hline
$SQ_{neutrofil}$ & 0.782 (0.761--0.801) & 0.775 (0.759--0.792) \\
\hline
$DQ_{neoplastic}$ & 0.706 (0.692--0.719) & 0.727 (0.712--0.741) \\
\hline
$DQ_{lymphocyte}$ & 0.675 (0.656--0.698) & 0.713 (0.691--0.734) \\
\hline
$DQ_{connective}$ & 0.566 (0.546--0.584) & 0.583 (0.565--0.602) \\
\hline
$DQ_{dead}$ & 0.410 (0.361--0.465) & 0.435 (0.306--0.561) \\
\hline
$DQ_{epithelial}$ & 0.668 (0.639--0.694) & 0.673 (0.644--0.702) \\
\hline
$DQ_{macrophage}$ & 0.657 (0.583--0.727) & 0.615 (0.531--0.703) \\
\hline
$DQ_{neutrofil}$ & 0.691 (0.625--0.753) & 0.729 (0.679--0.778) \\
\hline
\end{tabular}
%
%}
\end{table}

\clearpage

\subsection{\label{chap:S7}QuPath integration method}
We adopt an integration strategy leveraging the paquo \cite{Bayer_AG} library, a Python package that enables direct interaction with QuPath’s internal API, thereby facilitating seamless data exchange without intermediate conversion steps. The data processing pipeline (\hyperref[fig:S7]{Appendix Figure S7}) begins with the acquisition of WSIs and their associated annotations from QuPath, which are represented as Shapely \cite{Gillies_Wel_etal._2024} polygons. Utilizing paquo, we directly read, create, and modify these annotations and detections within a QuPath project in the Python environment. Images are then cropped using these polygons and processed by cell segmentation and classification models employing standard vision processing toolkits such as OpenCV, pyvips, and PyTorch. Additionally, QuPath employs Groovy scripts to initiate a Python process that starts the entire pipeline from QuPath graphical interface: fetching polygons, extracting images from them, and running deep learning model inference on the cropped images. 
The results are returned to QuPath, leveraging paquo's Python bindings to manipulate QuPath data while minimizing the computational overhead typically associated with cross-environment communication.

\counterwithin{figure}{subsection}
\renewcommand{\thefigure}{S\arabic{subsection}}

\begin{figure}[h!]
    \centering
    \includegraphics[width=\textwidth]{images/A7.pdf}
    \caption{QuPath integration workflow using Python environment}
    \label{fig:S7}
\end{figure}

Compared to traditional workflows that involve exporting annotations as GeoJSON, classifying them in Python, and reimporting them into QuPath, our approach offers several advantages. We eliminate the need to switch between programming languages, providing a cohesive and streamlined development process entirely within QuPath software and removing the necessity to use other tools. Meanwhile, we avoid storing annotations as intermediate JSON files unless required for external use or archiving. By conducting the entire inference and post-processing workflow within the Python environment, we leverage the power and flexibility of Python libraries for image processing and machine learning. This approach also enables adjustments to any set of labels and models, thereby improving its applicability.

%\hfill

The distilled model and QuPath integration code are packaged into a Docker container, enabling streamlined execution with the Docker engine. Detailed integration code and deployment instructions can be found in the GitHub repository \cite{Shvetsov_2025b}.

Despite these benefits, we acknowledge that the paquo library is a proof‑of‑concept project in its early development stage and has not been tested across all versions of QuPath.

\clearpage

\subsection{\label{chap:S8}Data and code availability statement}
All datasets, models, and code used in this study are publicly available and can be obtained from the repositories listed below. 
The PanNuke \cite{Gamper_Koohbanani_etal._2019} and MoNuSAC \cite{Verma_Kumar_etal._2021} datasets are publicly accessible, and download information along with detailed descriptions can be found in their respective articles. Preprocessing scripts for PanNuke and MoNuSAC data, as well as individual cell extraction scripts, are available on GitHub \cite{Shvetsov_2025a}. The H-Optimus foundation model used in our experiments can be downloaded from the HuggingFace repository \cite{hoptimus2024}, and model information is available on GitHub \cite{Saillard_Jenatton_etal._2024}. In addition, the integration code for QuPath and the distilled model packaged in a Docker container are provided in the repository \cite{Shvetsov_2025b}, and paquo Python library is available from the authors GitHub repository \cite{Bayer_AG}.
\clearpage

\end{document}

}

\newpage
\appendix
\twocolumn
\section* {Supplementary Material}

For better understanding of this work, we provide additional details, analysis, and results as follow:

\begin{itemize}[label={}]
   \item \textbf{A. Detailed Architectures of Modules} \\
   In this section, we display the detailed architectures for the cross-attention block $\mathtt{CAB}$ and masked self-attention block $\mathtt{MaskedSA}$ in the main text.

   \vspace{0.5em}
    
   \item \textbf{B. Inference Details} \\
   We provide more details for the inference of PRVQL.

   \vspace{0.5em}

   \item \textbf{C. Additional Experimental Results} \\
   We offer more experimental results in this work, including more ablations and comparison of different method across different scales on the Ego4D dataset.

   \vspace{0.5em}

   \item \textbf{D. Visualization Analysis of  Spatial Knowledge} \\
   We provide visual analysis to show the learned target spatial knowledge.

   \vspace{0.5em}

   \item \textbf{E. More Qualitative Results} \\
   We demonstrate more qualitative results of our method
    for localizing the target object.
   
\end{itemize}

\section{Detailed Architectures of Modules}

In each stage of PRVQL, we adopt the cross-attention block $\mathtt{CAB}$ to fuse the query feature into the video feature and then utilize the masked self-attention block $\mathtt{MaskedSA}$ for further enhancing the video feature. The architectures of $\mathtt{CAB}$ and $\mathtt{MaskedSA}$ are shown in Fig.~\ref{fig:support_fig3}.

\begin{figure}[!ht]
    \centering
    \includegraphics[width=0.88\linewidth]{figs/support_fig3.pdf}
    \vspace{-1mm}
    \caption{Detailed architectures of $\mathtt{CAB}$ and $\mathtt{MaskedSA}$.}
    \label{fig:support_fig3}\vspace{-4mm}
\end{figure}



\section{Inference Details}

Similar to~\cite{jiang2024single}, for inference, we first predict the confidence scores for target occurrence in all frames. Given the scores, we then smooth them through a median filter with the kernel size of 1. After this, we perform peak detection on the smoothed scores. We detect the peak based on the highest score $h$ and use $0.79\cdot h$ as the threshold to filter non-confident peaks. Finally, we can determine a spatio-temporal tube that corresponds to the most recent peak as the prediction result. In order to detect start and end time of the tube, we threshold the confidences scores using the threshold of $0.585\cdot \tilde{h}$, where $\tilde{h}$ is the confidence score at the most recent peak.

\begin{table}[!t]
        \setlength{\tabcolsep}{7.5pt}
	\centering
    \caption{ Ablation studies on the parameter $\alpha$ in SKG.}\vspace{-2mm}
\renewcommand{\arraystretch}{1}
			\scalebox{0.92}{
				\begin{tabular}{ccccccc}
					\specialrule{1.5pt}{0pt}{0pt}
					\rowcolor{mygray} 
					& & tAP$_{25}$ & stAP$_{25}$ & rec\% &  Succ \\ \hline\hline
					\ding{182} & $\alpha$=0.4 & 0.33 & 0.26 & 47.27 & 57.24 \\
                    \ding{183} & $\alpha$=0.5 & \textbf{0.35} & \textbf{0.27} & \textbf{47.87} & \textbf{57.93} \\
                    \ding{184} & $\alpha$=0.6 & 0.31 & 0.26 & 46.34 & 55.97 \\
					\specialrule{1.5pt}{0pt}{0pt}
			\end{tabular}}
			\label{tab:merge_atten_lkg}
\end{table}

\begin{table}[!t]
        \setlength{\tabcolsep}{7.5pt}
	\centering
    \caption{Ablation studies on combination methods in QFR.}
    \vspace{-2mm}
			\renewcommand{\arraystretch}{1}
			\scalebox{0.89}{
				\begin{tabular}{ccccccc}
					\specialrule{1.5pt}{0pt}{0pt}
					\rowcolor{mygray} 
					& Method & tAP$_{25}$ & stAP$_{25}$ & rec\% &  Succ \\ \hline\hline
					\ding{182} & Addition & 0.31 & 0.23 & 46.57 & 56.37 \\
                    \ding{183} & Concatenation & 0.34 & 0.25 & 47.31 & 56.64 \\
                    \ding{184} & Cross-Attention & \textbf{0.35} & \textbf{0.27} & \textbf{47.87} & \textbf{57.93} \\
					\specialrule{1.5pt}{0pt}{0pt}
			\end{tabular}}
			\label{tab:merge_akg}
\end{table}

\begin{table}[!t]
    \setlength{\tabcolsep}{7.5pt}
    \centering
    \caption{Comparison on object of different scales in videos.}\vspace{-2mm}
    \scalebox{0.8}{
    \begin{tabular}{rccccc}%\shline
    \specialrule{1.5pt}{0pt}{0pt}
     \rowcolor{mygray} Method & Scale & tAP$_{25}$ & stAP$_{25}$ & rec$\%$ &Succ \\
     \cmidrule(r){1-1}\cmidrule(l){2-2}\cmidrule(l){3-6}
     CocoFormer        & \textit{small} & \textbf{0.067} & \textbf{0.030} & \textbf{19.565} & \textbf{21.113} \\
     VQLoC      & \textit{small} & 0.047          & 0.001          &   2.447          & 13.043\\
     PRVQL (ours)                       & \textit{small} & 0.036          & 0.004          &    2.351    & 16.087 \\
     \hline
     CocoFormer        & \textit{medium} & 0.206          & 0.127                  & 32.583  & 40.804 \\
     VQLoC     & \textit{medium} & 0.213          & 0.138                   & 33.738 &  44.719\\
     PRVQL (ours)                       & \textit{medium} & \textbf{0.261} & \textbf{0.179}  & \textbf{34.359} & \textbf{49.923}\\
     \hline
     CocoFormer        & \textit{large} & 0.338 & 0.271  & 40.737 & 56.164\\
     VQLoC      & \textit{large} & 0.454 & 0.387    & \textbf{53.635} &  67.680 \\
     PRVQL (ours)                       & \textit{large} & \textbf{0.469} & \textbf{0.396}  & 52.127 & \textbf{68.664} \\
     \specialrule{1.5pt}{0pt}{0pt}
    \end{tabular}}
    \label{tabel:scale}
\end{table}

\begin{figure}[!t]
    \centering
    \includegraphics[width=0.95\linewidth]{figs/support_fig2.pdf}
    \vspace{-1mm}
    \caption{Visualization of target the spatial knowledge. First row: given visual query; second row: foreground target represented by red boxes; third row: learned target spatial knowledge.}
    \label{fig:support_fig2}\vspace{-3mm}
\end{figure}

\begin{figure}[!t]
    \centering
    \includegraphics[width=0.98\linewidth]{figs/support_fig1.pdf}
    \vspace{-1mm}
    \caption{Qualitative results of our method.}
\label{fig:support_fig_qual}\vspace{-4mm}
\end{figure}

\section{Additional Experimental Results}

In this section, we show more ablation studies and comparison to other methods on the Ego4D validation set.

\vspace{0.3em}
\noindent
\textbf{Impact of Balance Parameter $\alpha$ in SKG.} The interpolated attention map $\varphi_{\text{int}}(\mathcal{T}_k^{\text{d}})$, obtained via bilinear interpolation from $\mathcal{T}_k^d$, is merged with $\mathcal{S}_k$ through a balance parameter $\alpha$. We conduct an ablation on $\alpha$ in Tab.~\ref{tab:merge_atten_lkg}. We can observe that, when setting $\alpha$ to 0.5, we show the best result (see \ding{183}). 

\vspace{0.3em}
\noindent
\textbf{Different Combination Methods in QFR.} In QFR, the appearance knowledge $\mathcal{K}^a_k$, obtained by AKG, is used to guide the refinement of query feature. In PRVQL, we use a cross-attention block to combine $\mathcal{K}^a_k$ and $\mathcal{Q}_k$ for achieving refinement. Besides cross-attention, we conduct experiments using other manners for refinement, including element-wise addition and concatenation, in Tab.~\ref{tab:merge_akg}. As shown in  Tab.~\ref{tab:merge_akg}, when using the cross-attention block for query feature refinement, we achieve the best performance (see \ding{184}).

 \vspace{0.3em}
\noindent
\textbf{Comparison in Different Scales.} Following~\cite{jiang2024single}, we provide comparison for objects of different scales in videos. As in ~\cite{jiang2024single}, the objects are categorized to three scales, including \emph{small} scale with target area in the range of [0, 64$^2$], \emph{medium} scale with the target area in the range (64$^2$, 192$^2$], and \emph{large} scale with target area greater than 192$^2$.  Tab.~\ref{tabel:scale} reports the comparison result. As in Tab.~\ref{tabel:scale}, we can observe that, CocoFormer performs better for small-scale objects. We argue that the reason is CocoFormer adopts higher-resolution images for localization and employs detector that is good at small object detection, while VQLoC and our method use downsampled frames for localization and do not specially deal with small objects. In comparison to CocoFormer and VQLoC, our PRVQL achieves better overall performance for medium- and large-scale objects, which shows the efficacy of target knowledge for robust target localization. 



\section{Visualization Analysis of Spatial Knowledge}

The target spatial knowledge by SKG aims to explore target cues from videos for enhancing the target while suppressing background regions in video features. In PRVQL, we adopt the readily available attention maps to produce target spatial knowledge, which is shown in Fig.~\ref{fig:support_fig2}. From Fig.~\ref{fig:support_fig2}, we can see that, our spatial knowledge focuses more on the target object while less on the background, and thus can be applied to refine video features for better localization.

\section{More Qualitative Results}

In order to further validate the effectiveness of our PRVQL, we provide additional examples of target localization results in Fig.~\ref{fig:support_fig_qual}. From the shown visualizations, we can observe that, with the hlep of target knowledge, our method can accurately locate the target in both space and time.

\end{document}
