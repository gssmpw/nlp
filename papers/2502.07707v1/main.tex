\PassOptionsToPackage{table}{xcolor}
\documentclass[10pt,twocolumn,letterpaper]{article}

% \usepackage{iccv}              % To produce the CAMERA-READY version
% \usepackage[review]{iccv}      % To produce the REVIEW version
\usepackage[pagenumbers]{iccv} % To force page numbers, e.g. for an arXiv version
% \usepackage[table]{xcolor}

% Import additional packages in the preamble file, before hyperref
%
% --- inline annotations
%
\newcommand{\red}[1]{{\color{red}#1}}
\newcommand{\todo}[1]{{\color{red}#1}}
\newcommand{\TODO}[1]{\textbf{\color{red}[TODO: #1]}}
% --- disable by uncommenting  
% \renewcommand{\TODO}[1]{}
% \renewcommand{\todo}[1]{#1}



\newcommand{\VLM}{LVLM\xspace} 
\newcommand{\ours}{PeKit\xspace}
\newcommand{\yollava}{Yo’LLaVA\xspace}

\newcommand{\thisismy}{This-Is-My-Img\xspace}
\newcommand{\myparagraph}[1]{\noindent\textbf{#1}}
\newcommand{\vdoro}[1]{{\color[rgb]{0.4, 0.18, 0.78} {[V] #1}}}
% --- disable by uncommenting  
% \renewcommand{\TODO}[1]{}
% \renewcommand{\todo}[1]{#1}
\usepackage{slashbox}
% Vectors
\newcommand{\bB}{\mathcal{B}}
\newcommand{\bw}{\mathbf{w}}
\newcommand{\bs}{\mathbf{s}}
\newcommand{\bo}{\mathbf{o}}
\newcommand{\bn}{\mathbf{n}}
\newcommand{\bc}{\mathbf{c}}
\newcommand{\bp}{\mathbf{p}}
\newcommand{\bS}{\mathbf{S}}
\newcommand{\bk}{\mathbf{k}}
\newcommand{\bmu}{\boldsymbol{\mu}}
\newcommand{\bx}{\mathbf{x}}
\newcommand{\bg}{\mathbf{g}}
\newcommand{\be}{\mathbf{e}}
\newcommand{\bX}{\mathbf{X}}
\newcommand{\by}{\mathbf{y}}
\newcommand{\bv}{\mathbf{v}}
\newcommand{\bz}{\mathbf{z}}
\newcommand{\bq}{\mathbf{q}}
\newcommand{\bff}{\mathbf{f}}
\newcommand{\bu}{\mathbf{u}}
\newcommand{\bh}{\mathbf{h}}
\newcommand{\bb}{\mathbf{b}}

\newcommand{\rone}{\textcolor{green}{R1}}
\newcommand{\rtwo}{\textcolor{orange}{R2}}
\newcommand{\rthree}{\textcolor{red}{R3}}
\usepackage{amsmath}
%\usepackage{arydshln}
\DeclareMathOperator{\similarity}{sim}
\DeclareMathOperator{\AvgPool}{AvgPool}

\newcommand{\argmax}{\mathop{\mathrm{argmax}}}     



% It is strongly recommended to use hyperref, especially for the review version.
% hyperref with option pagebackref eases the reviewers' job.
% Please disable hyperref *only* if you encounter grave issues, 
% e.g. with the file validation for the camera-ready version.
%
% If you comment hyperref and then uncomment it, you should delete *.aux before re-running LaTeX.
% (Or just hit 'q' on the first LaTeX run, let it finish, and you should be clear).
\definecolor{iccvblue}{rgb}{0.21,0.49,0.74}
\usepackage[pagebackref,breaklinks,colorlinks,allcolors=iccvblue]{hyperref}

\usepackage{booktabs} % for professional tables


\usepackage{pifont}

\definecolor{mygray}{gray}{.95}
\definecolor{highlight}{RGB}{238,250,215}
\definecolor{lightgray}{gray}{0.7}
\definecolor{lightorange}{HTML}{C47955}
\definecolor{darkgreen}{HTML}{195228}
\definecolor{eccvblue}{rgb}{0.12,0.49,0.85}
\definecolor{lightblue}{rgb}{0.12,0.49,0.85}

%%%%%%%%% PAPER ID  - PLEASE UPDATE
\def\paperID{*****} % *** Enter the Paper ID here
\def\confName{ICCV}
\def\confYear{2025}

%%%%%%%%% TITLE - PLEASE UPDATE
\title{PRVQL: Progressive Knowledge-guided Refinement for Robust Egocentric \\ Visual Query Localization}

%%%%%%%%% AUTHORS - PLEASE UPDATE
\author{Bing Fan$^{1}$\;\;\;\;\;Yunhe Feng$^{1}$\;\;\;\;\;Yapeng Tian$^{2}$\;\;\;\;\;Yuewei Lin$^{3}$\;\;\;\;\; Yan Huang$^{1}$\;\;\;\;\;Heng Fan$^{1}$\\
$^{1}$University of North Texas \;\;\;\;\; $^{2}$University of Texas at Dallas\;\;\;\;\; $^{3}$Brookhaven National Laboratory 
% $^{1}$Department of Computer Science and Engineering, University of North Texas \\ $^{2}$Department of Computer Science, University of Texas at Dallas\;\; $^{3}$Brookhaven National Laboratory 
% For a paper whose authors are all at the same institution,
% omit the following lines up until the closing ``}''.
% Additional authors and addresses can be added with ``\and'',
% just like the second author.
% To save space, use either the email address or home page, not both
% \and
% Second Author\\
% Institution2\\
% First line of institution2 address\\
% {\tt\small secondauthor@i2.org}
}

\begin{document}
\maketitle


\begin{abstract}
Egocentric visual query localization (\emph{EgoVQL}) focuses on localizing the target of interest in space and time from first-person videos, given a visual query. Despite recent progressive, existing methods often struggle to handle severe object appearance changes and cluttering background in the video due to lacking sufficient target cues, leading to degradation. Addressing this, we introduce \textbf{PRVQL}, a novel \textbf{P}rogressive knowledge-guided \textbf{R}efinement framework for Ego\textbf{VQL}. The core is to continuously exploit target-relevant knowledge directly from videos and utilize it as guidance to refine both query and video features for improving target localization. Our PRVQL contains multiple processing stages. The target knowledge from one stage, comprising appearance and spatial knowledge extracted via two specially designed knowledge learning modules, are utilized as guidance to refine the query and videos features for the next stage, which are used to generate more accurate knowledge for further feature refinement. With such a progressive process, target knowledge in PRVQL can be gradually improved, which, in turn, leads to better refined query and video features for localization in the final stage. Compared to previous methods, our PRVQL, besides the given object cues, enjoys additional crucial target information from a video as guidance to refine features, and hence enhances EgoVQL in complicated scenes. In our experiments on challenging Ego4D, PRVQL achieves state-of-the-art result and largely surpasses other methods, showing its efficacy. Our code, model and results will be released at \url{https://github.com/fb-reps/PRVQL}.
\end{abstract}

\section{Introduction}
\label{intro}

The egocentric visual query localization (EgoVQL) task~\cite{grauman2022ego4d} aims at answering the question ``\emph{Where was the object X last seen in the video?}'', with ``\emph{X}'' being a visual query specified by a single image crop outside the search video. In specific, given a first-person video, its goal is to search and locate the visual query, \emph{spatially} and \emph{temporally}, by returning the most recent spatio-temporal tube. Owing to its important roles in numerous downstream object-centric applications including augmented and virtual reality, robotics, human-machine interaction, and so on, EgoVQL has drawn extensive attention from researchers in recent years.

\begin{figure}[!t]
    \centering
    \includegraphics[width=\linewidth]{figs/fig1.pdf}\vspace{-1mm}
    \caption{Comparison between current EgoVQL approaches in (a) and proposed PRVQL with progressive knowledge-guided refinement in (b). \emph{Best viewed in color and by zooming in for all figures}.}
    \label{fig:framework_comparison}\vspace{-2mm}
\end{figure}

Current approaches (\eg,~\cite{xu2022negative,xu2023my,jiang2024single,grauman2022ego4d}) simply leverage the provided visual query as the \emph{sole} cue to locate the target in the video (see Fig.~\ref{fig:framework_comparison} (a)). However, since the given visual query is cropped \emph{outside} the search video, there often exists a \emph{significant gap} between the query and the target of interest, due to rapid appearance variations in first-person videos caused by many factors, such as object pose change, motion blur, occlusion, and so forth. As a result, relying only on the given object query, as in existing methods, is \emph{insufficient} to describe and distinguish the target from background in complicated scenarios with heavy appearance changes, resulting in performance degeneration. In addition, to achieve precise localization, it is essential for an EgoVQL model to enhance target and meanwhile suppressing background regions from videos. Yet, this is often \emph{overlooked} by existing approaches, making them easily suffer from cluttering background and therefore leading to suboptimal target localization.

The aforementioned issues faced by current methods naturally raise a question: \emph{In addition to the given visual query, is there any other information that could be leveraged for enhancing EgoVQL}? We answer \emph{\textbf{yes}}, and show the information directly explored from the \emph{video itself}, as a supplement to the given target cue, is \emph{effective} in improving EgoVQL.

Specifically, we propose a novel \emph{\textbf{P}rogressive knowledge-guided \textbf{R}efinement framework for Ego\textbf{VQL}} (\textbf{\emph{PRVQL}}). The core idea of our algorithm is to continuously exploit target-relevant knowledge from the video and leverage it to guide refinements of both query and video features, which are crucial for localization, for improving EgoVQL (see Fig.~\ref{fig:framework_comparison} (b)). Concretely, PRVQL consists of multiple processing stages. Each stage comprises two simple yet effective modules, including \emph{appearance knowledge generation} (AKG) and \emph{spatial knowledge generation} (SKG). In specific, AKG works to mine visual information of the target from videos as the appearance knowledge. It first estimates potential target regions from a video using the query, and then selects top few highly confident regions to extract appearance knowledge from video features. Different from AKG, SKG focuses on exploring target position cues from videos as spatial knowledge by exploiting the readily available target-aware attention maps. In PRVQL, the appearance knowledge is used to guide the update of query feature, making it more discriminative, while the spatial knowledge is employed to enhance target and meanwhile suppressing unconcerned background in video features, enabling more focus on the target. The extracted appearance and spatial knowledge in one stage are used as guidance to respectively refine query and video features for next stage, which are adopted to learn more accurate knowledge for further feature refinement. Through this progressive process in PRVQL, the target knowledge can be gradually improved, which, in turn, results in better refined query and video features for target object localization in the final stage. Fig.~\ref{fig:framework} illustrates the framework of PRVQL.

To our best knowledge, PRVQL is the \emph{first} method to exploit target-relevant appearance and spatial knowledge from the video to improve EgoVQL. Compared with existing solutions, PRVQL can leverage target information from both the given visual query and mined knowledge from the video for more robust localization. To verify its effectiveness, we conduct experiments on the challenging Ego4D~\cite{grauman2022ego4d}, and our proposed PRVQL achieves state-of-the-art performance and largely outperforms other approaches, evidencing effectiveness of target knowledge for enhancing EgoVQL. 

In summary, our main contributions are as follows: \ding{171} We propose a progressive knowledge-guided refinement framework, dubbed PRVQL, that exploits knowledge from videos for improving EgoVQL; \ding{170} We introduce AKG for exploring visual information of target as appearance knowledge; \ding{168} We introduce SKG for learning spatial knowledge using target-aware attention maps; \ding{169} In our extensive experiments on the challenging Ego4D, PRVQL achieves state-of-the-art performance and largely surpasses existing methods. 

\section{Related Work}

% In this section, we discuss works that are closely relevant to our approach from the following three lines.

% \vspace{0.5em}
% \noindent
\textbf{Egocentric Visual Query Localization.} Egocentric visual query localization (EgoVQL) is an emerging and important computer vision task. Since its introduction in~\cite{grauman2022ego4d}, EgoVQL has received extensive attention in recent years owing to its importance in numerous applications. Early methods~\cite{grauman2022ego4d,xu2022negative,xu2023my} often utilize a bottom-up multi-stage framework, which sequentially and independently performs frame-level object detection, nearest peak temporal detection across the video, and bidirectional object tracking around the peak, to achieve EgoVQL. Despite the straightforwardness, this bottom-up design easily causes compounding errors across stages, thus degrading performance. Besides, the involvement of multiple detection and tracking components in this design leads to high complexities as well as inefficiency of the entire system, limiting its practicability. To deal with these issues, the recent method of~\cite{jiang2024single} introduces a single-stage end-to-end framework for EgoVQL with Transformer~\cite{VaswaniSPUJGKP17}, eliminating the need for multiple components and meanwhile showing promising target localization performance.

In this work, we propose to exploit target knowledge directly from the video and utilize it as guidance to refine features for better localization. \textbf{\emph{Different}} from aforementioned approaches~\cite{grauman2022ego4d,xu2022negative,xu2023my,jiang2024single} which mainly explore the object information from only the provided query for localization, PRVQL is able to leverage cues from both the given query and mined target information for EgoVQL, significantly improving robustness, especially in presence of severe appearance variations and cluttering background.

\begin{figure*}[!t]
	\centering
        \includegraphics[width=1\textwidth]{figs/fig2.pdf}\vspace{-2mm}
	\caption{Overview of PRVQL, which aims to explore target knowledge directly from videos via AKG and SKG and applies it as guidance to refine query and video features with QFR and VFR for improving localization in EgoVQL through a multi-stage progressive architecture.}
	\label{fig:framework}\vspace{-4mm}	
\end{figure*}

\vspace{0.5em}
\noindent
\textbf{Query-based Visual Localization.} Query-based visual localization, broadly referring to localizing the target of interest from images or videos given a specific query (image or text), is a crucial problem in computer 
vision, and consists of a wide range of related tasks, including one-shot object detection~\cite{hsieh2019one,yang2022balanced,zhao2022semantic}, visual object tracking~\cite{chen2023seqtrack,lin2025tracking,bertinetto2016fully}, visual grounding~\cite{deng2021transvg,liu2025grounding,zhu2022seqtr}, spatio-temporal video grounding~\cite{yang2022tubedetr,gu2024context}, pedestrian search~\cite{li2017person,yu2022cascade}, \etc. Despite sharing some similarity with the above tasks in localizing the target, this work is \textbf{\emph{distinctive}} by focusing on spatially and temporally searching for the target from egocentric videos, which is challenging due to frequent and heavy object appearance variations under the first-person views.

\vspace{0.5em}
\noindent
\textbf{Progressive Learning Approach.} Multi-stage progressive learning is a popular strategy to improve performance, and has been successfully applied for various tasks. For example, the works of~\cite{cai2018cascade,ye2023cascade,vu2019cascade} introduce the cascade architecture to progressively refine the bounding boxes or features for improving object detection. The work in~\cite{yang2019step} presents a sptio-temporal progressive network for video action detection. The methods in~\cite{huynh2021progressive,zhao2018icnet} introduce progressive refinement network for multi-scale semantic segmentation. The methods of~\cite{zhang2018progressive,chen2020progressively} apply progressive learning to improve features for saliency detection. The method in~\cite{fan2019siamese} proposes to progressively learn more accurate anchors for enhancing tracking. The work from~\cite{zhu2019progressive} progressively transfers person pose for image generation. \textbf{\emph{Different}} than these works, we focus on progressive refinement for improving EgoVQL.

\section{The Proposed Method}

\textbf{Overview.} In this paper, we propose PRVQL by exploiting crucial target knowledge directly from videos for improving target localization in EgoVQL. Our PRVQL is implemented as a progressive architecture. After feature extraction of the visual query and video frames, PRVQL performs iterative feature refinement guided by the target knowledge for localization through multiple stages (Sec.~\ref{prvql}). As displayed in Fig.~\ref{fig:framework}, each stage, expect for the final stage for prediction, consists of two crucial modules, comprising AKG (Sec.~\ref{akg}) and SKG (Sec.~\ref{skg}), for generating target appearance and spatial knowledge. The knowledge is leveraged as the guidance to refine query and video features (Sec.~\ref{update}), which are applied in the next stage to generate more accurate target knowledge for further feature refinement. Through such a progressive process, the target knowledge can be gradually enhanced, which finally benefits learning more discriminative query and video features for improving EgoVQL.

\subsection{Our PRVQL Framework}\label{prvql}

\textbf{Visual Feature Extraction.} In our PRVQL, we first extract features for the visual query and video frames. Specifically, given the query $q$ and a sequence of $L$ frames $\mathcal{I}=\{I_i\}_{i=1}^{L}$ from the video, we utilize a shared backbone $\Phi(\cdot)$~\cite{OquabDMVSKFHMEA24} for extracting their features $\textbf{q}=\Phi(q) \in \mathbb{R}^{H\times W\times C}$ and $F=\{\textbf{f}_i\}_{i=1}^{L}$ with each $\textbf{f}_i=\Phi(I(i)) \in \mathbb{R}^{H\times W\times C}$, where the $H$ and $W$ represent the spatial resolution of the features and $C$ denotes the channel dimension. For subsequent processing, we flatten $\textbf{q}$ and $F$ to obtain $\textbf{Q}=\mathtt{flatten}(\textbf{q}) \in \mathbb{R}^{HW\times C}$ and $\textbf{V}=\{\textbf{v}_i\}_{i=1}^{L}$ with each $\textbf{v}_i \in \mathbb{R}^{HW\times C}$.

\vspace{0.5em}
\noindent
\textbf{Progressive Knowledge-guided Feature Refinement for EgoVQL.} As mentioned earlier, the core idea of PRVQL is to exploit target knowledge directly from videos and apply it as guidance to enhance query and video features for target localization. For this purpose, PRVQL is implemented as a progressive architecture with multiple stages in a sequence. Each but the last stage involves target knowledge learning and knowledge-guided feature refinement, as in Fig.~\ref{fig:framework}.

More specifically, for the $k^{\text{th}}$ ($1\le k < K$) stage of our PRVQL, let $\mathcal{Q}_{k}$ and $\mathcal{V}_{k}$ denote the query and video features. For the first stage ($k=1$), $\mathcal{Q}_1$ and $\mathcal{V}_1$ are initialized using query and video features extracted from the backbone, and $\mathcal{Q}_{1}=\textbf{Q}$ and $\mathcal{V}_{1}=\textbf{V}$. Otherwise, $\mathcal{Q}_{k}$ and $\mathcal{V}_{k}=\{v_i^k\}_{i=1}^{L}$ are refined features in the last stage $(k-1)$. To mine target-specific knowledge from the video, we perform feature fusion between $\mathcal{Q}_{k}$ and $\mathcal{V}_{k}$, aiming to inject target information into video feature for improving its target awareness. In specific, we leverage cross-attention from~\cite{VaswaniSPUJGKP17} for feature fusion owing to its powerfulness in feature modeling. Mathematically, this process can be expressed as follows,
\begin{equation}\label{eq1}
\setlength{\abovedisplayskip}{5pt} 
\setlength{\belowdisplayskip}{5pt}
\mathcal{X}_k=\{x_i^k | x_i^k = \mathtt{CAB}(v_i^k,\mathcal{Q}_k)\} \;\;\; i=1,2,\cdots,L
\end{equation}
where $\mathcal{X}_k$ is the fused feature in stage $k$, and $v_i^k$ the feature in frame $i$. $\mathtt{CAB}(\mathbf{z},\mathbf{u})$ is the cross-attention (CA) block with $\mathbf{z}$ generating query and $\mathbf{u}$ key/value. Due to space limitation, please see \emph{supplementary material} for detailed architecture. Besides fused feature, we also obtain target-aware spatial attention maps $\mathcal{S}_{k}=\{s_{i}^{k}\}_{i=1}^{L}\in \mathbb{R}^{L\times HW \times HW}$ for $L$ frames in Eq.~(\ref{eq1}), with each $s_{i}^{k}\in \mathbb{R}^{HW \times HW}$ the attention map from the cross-attention operation in $\mathtt{CAB}(v_i^k,\mathcal{Q}_k)$.

To further capture spatio-temporal relations from videos for enhancing features, we apply self-attention~\cite{VaswaniSPUJGKP17} on $\mathcal{X}_k$ by propagating the query information spatially and temporally. Considering that targets in nearby frames are highly correlated, we restrict the attention operation in a temporal window using a masking strategy, similar to~\cite{jiang2024single}. To reduce the computation, we downsample $\mathcal{X}_k$ to decrease the spatial dimension of each frame feature to $h\times w$. Then, we add a position embedding $\mathcal{E}_k^{\text{pos}}$ to the video feature. This process can be written as follows,
\begin{equation}
\setlength{\abovedisplayskip}{6pt} 
\setlength{\belowdisplayskip}{6pt}
    \tilde{\mathcal{X}}_k = \mathtt{Downsample}(\mathcal{X}_k) + \mathcal{E}_k^{\text{pos}}
\end{equation}
where $\mathtt{Downsample}(\cdot)$ represents the downsampling operation implemented with convolution operation. Afterwards, masked self-attention is applied on as $\tilde{\mathcal{X}}$ as follows,
\begin{equation}\label{eq3}
\setlength{\abovedisplayskip}{7pt} 
\setlength{\belowdisplayskip}{7pt}
    \mathcal{H}_k=\mathtt{MaskedSA}(\tilde{\mathcal{X}}_k)
\end{equation}
where $\mathcal{H}_k$ represents enhanced video feature. $\mathtt{MaskedSA}(\mathbf{z})$ denotes the masked self-attention block with $\textbf{z}$ generating query/key/value. In this block, each feature element from frame $i$ only attends to feature elements from frames in the temporal range [($i-u$), ($i+u$)], which can be easily implemented using masking strategy~\cite{VaswaniSPUJGKP17,cheng2022masked}. From Eq.~(\ref{eq3}), besides the $\mathcal{H}_k$, we also gain the temporal-aware spatial attention maps, denoted as $\mathcal{T}_k \in \mathbb{R}^{L\times hw \times Lhw}$, for the target in the video, which will be used for knowledge generation.


With video feature $\mathcal{H}_k$ and attention maps $\mathcal{S}_k$ and $\mathcal{T}_k$, the target knowledge can be extracted with the AKG and SKG modules (as explained later in Sec.~\ref{secakg} and~\ref{skg}), as follows,
\begin{equation}\label{eq_kaks}
\setlength{\abovedisplayskip}{7pt} 
\setlength{\belowdisplayskip}{7pt}
    \mathcal{K}_k^a=\mathtt{AKG}(\mathcal{H}_k, \mathcal{V}_k) \;\;\;\;\;\;\;
    \mathcal{K}_k^s=\mathtt{SKG}(\mathcal{S}_k, \mathcal{T}_k)
\end{equation}
where $\mathcal{K}_k^a$ represents the appearance knowledge and $\mathcal{K}_k^s$ the spatial knowledge. Guided by $\mathcal{K}_k^a$ and $\mathcal{K}_k^s$ in stage $k$, we can refine query and video features using two update modules QFR and VFR (as described later in Sec.~\ref{update}) as follows,
\begin{equation}\label{eq_qfuvfu}
\setlength{\abovedisplayskip}{7pt} 
\setlength{\belowdisplayskip}{7pt}
    \mathcal{Q}_{k+1}=\mathtt{QFR}(\mathcal{K}_k^a, \mathcal{Q}_{k}) \;\;\;\;\;\;\; \mathcal{V}_{k+1}=\mathtt{VFR}(\mathcal{K}_k^s,  \mathcal{V}_1)
\end{equation}
where $\mathcal{Q}_{k+1}$ and $\mathcal{V}_{k+1}$ are refined features guided by target knowledge, which are fed to the next stage $(k+1)$ to generate more accurate knowledge for further feature refinement. Fig.~\ref{fig:att} compares the attention maps from the masked self-attention with and without using our approach. We can see that, our method with refined features guided by knowledge can better focus on the target in the video and thus improves target localization, showing its efficacy.

For the final $K^{\text{th}}$ stage in PRVQL, since no knowledge is extracted, the AKG and SKG modules are removed. Given the visual query and video features $\mathcal{Q}_{K}$ and $\mathcal{V}_{K}$ from the $(K-1)^{\text{th}}$ stage, we can then obtain the final enhanced video feature $\mathcal{H}_{K}$ through Eqs.~(\ref{eq1})-(\ref{eq3}) in the $K^{\text{th}}$ stage. With $\mathcal{H}_{K}$, we use the prediction heads as in ~\cite{jiang2024single} for target localization via regression and classification. For details of the adopted prediction heads, please kindly refer to~\cite{jiang2024single}.

\begin{figure}[t]
	\centering
	\includegraphics[width=0.98\linewidth]{figs/attention.pdf}\vspace{-1mm}
	\caption{Comparison of attention maps for video frames from the masked self-attention block \emph{without} (a) and \emph{with} (b) our progressive refinement. As shown, our method can better focus on the target regions, and hence can improve target localization in EgoVQL. The red boxes indicate the foreground object to localize.}
        \vspace{-4mm}
	\label{fig:att}
\end{figure}

\subsection{Appearance Knowledge Generation (AKG)}\label{secakg}

In order to extract discriminative visual information of target directly from the video, we introduce a simple yet highly effective module, named \emph{appearance knowledge generation} (AKG), for appearance knowledge learning. Specifically, it first estimates the potential target regions from the video using target-aware video features. Then, based on confidence scores of these regions, we select the top few ones to extract target features from the video as the appearance knowledge. % Since the appearance knowledge is directly acquired from a video, it contains more discriminative information than the given query for target localization, and hence can be applied as effective guidance for enhancing the query feature.

Specifically, given the target-aware video feature $\mathcal{H}_k$, we first reshape it to the 2D feature map, and then increase its spatial resolution back to $H\times W$ as follows,
\begin{equation}
\setlength{\abovedisplayskip}{6pt} 
\setlength{\belowdisplayskip}{6pt}
    \tilde{\mathcal{H}}_k=\mathtt{Upsample}(\mathtt{Reshape}(\mathcal{H}_k))
\end{equation}
where $\mathtt{Upsample}(\cdot)$ denotes the upsampling operation. After this, we apply $\tilde{\mathcal{H}}_k$ to produce temporal confidence scores and spatial box regions for target in each frame. More concretely, we first split $\tilde{\mathcal{H}}_k$ along the channel dimension into two equal halves $\tilde{\mathcal{H}}_k^{t}$ and $\tilde{\mathcal{H}}_k^{s}$ via $\tilde{\mathcal{H}}_k^{t}, \tilde{\mathcal{H}}_k^{s}=\mathtt{Split}(\tilde{\mathcal{H}}_k)$. Inspired by~\cite{jiang2024single}, we perform classification and regression to predict temporal confidence scores and spatial boxes using multi-scale anchors~\cite{RenHGS15}. Specifically, two Conv blocks are applied on $\tilde{\mathcal{H}}_k^{t}(i)$ and $\tilde{\mathcal{H}}_k^{s}(i)$ for prediction as follows,
\begin{equation}\label{akg}
\setlength{\abovedisplayskip}{7pt}
\setlength{\belowdisplayskip}{7pt}
    \tilde{\mathcal{C}}_k=\mathtt{ConvBlock}(\tilde{\mathcal{H}}_k^{t}) \;\;\;\;\; \Delta\tilde{\mathcal{B}}_k=\mathtt{ConvBlock}(\tilde{\mathcal{H}}_k^{s})
\end{equation}
where $\tilde{\mathcal{C}}_k \in \mathbb{R}^{L \times H \times W \times m}$ denotes the temporal confidence scores for target in $L$ frames with $m$ the number of anchors at each position. $\Delta\tilde{\mathcal{B}}_k \in \mathbb{R}^{L \times H \times W \times 4m}$ is the offsets to the anchor boxes $\bar{\mathcal{B}}$, and target boxes $\tilde{\mathcal{B}}_k=\Delta\tilde{\mathcal{B}}_k + \bar{\mathcal{B}}$. With $\tilde{\mathcal{C}}_k$, the confidence score in each frame is determined by the highest classification score of anchors, and the target region is the box corresponding to the box with the highest classification score. This way, we can obtain the confidence scores $\bar{\mathcal{C}}_k$ and target regions $\bar{\mathcal{B}}_k$ in each frames as follows, 
\begin{equation}\label{pred}
\setlength{\abovedisplayskip}{7pt}
\setlength{\belowdisplayskip}{7pt}
\begin{split}
    &\bar{\mathcal{C}}_k = \{c_{k}^{i} | c_{k}^{i}, d_k^i = \mathtt{Max}(\tilde{\mathcal{C}}_k(i))\} 
    \\
    &\bar{\mathcal{B}}_k = \{b_{k}^{i} | b_{k}^{i} = \mathtt{Index}(\tilde{\mathcal{B}}_k(i), d_k^i)\} 
\end{split}
\end{equation}
where $i\in[1,L]$ is the frame index. $c_{k}^{i}$ is the highest value selected from the classification scores $\tilde{\mathcal{C}}_k(i)\in\mathbb{R}^{H\times W\times m}$ of anchors in frame $i$, and $d_k^i$ is its index. $b_{k}^{i}$ is the target box corresponding to $c_{k}^{i}$ in frame, and extracted from $\tilde{\mathcal{B}}_k(i)\in\mathbb{R}^{H\times W\times 4m}$. $\mathtt{Max}(\cdot)$ is to select the maximum and its index, and $\mathtt{Index}(\cdot)$ to extract the box from $\tilde{\mathcal{B}}_k(i)$ given its index. 

With $\bar{\mathcal{C}}_k\in\mathbb{R}^{L\times 1}$ and $\bar{\mathcal{B}}_k\in\mathbb{R}^{L\times 4}$, we first sample target regions with high confidence scores as follows,
\begin{equation}\label{threshold}
\setlength{\abovedisplayskip}{7pt}
\setlength{\belowdisplayskip}{7pt}
\mathcal{B}_k=\mathtt{Sample}(\bar{\mathcal{B}}_k(i), \bar{\mathcal{C}}_k(i), \tau) =\{\bar{\mathcal{B}}_k(i) | \bar{\mathcal{C}}_k(i) > \tau\}
\end{equation}
Then, we extract $n$ regions from $\mathcal{B}_k$ with the top confidence scores via $\mathcal{B}_k^{\text{top}}=\mathtt{Top}_{n}(\mathcal{B}_k)$. If the number of regions in $\mathcal{B}_k$ is less then $n$, we keep all regions. After this, RoIAlign~\cite{he2017mask} is used to extract the appearance knowledge from $\mathcal{V}_{k}$ via
\begin{equation}\label{roi}
\setlength{\abovedisplayskip}{7pt}
\setlength{\belowdisplayskip}{7pt}
\mathcal{K}_{k}^{a}=\mathtt{RoIAlign}(\mathcal{V}_{k}, \mathcal{B}_k^{\text{top}})
\end{equation}
where $\mathcal{K}_{k}^{a}$ represents the appearance knowledge from AKG in the $k^{\text{th}}$ stage. Please notice that, in Eq.~(\ref{roi}), we only perform RoIAlign in frames corresponding to $\mathcal{B}_k^{\text{top}}$. Since $\mathcal{K}_{k}^{a}$ is generated from the video itself, when using it as guidance to refine the query feature, we can reduce the discrepancy between the query and the foreground target. By deploying AKG in each but the last stage, $\mathcal{K}_{k}^{a}$ could be gradually improved with better refined query feature in each stage. Fig.~\ref{fig:akgfig} illustrates AKG for appearance knowledge generation.

\begin{figure}[t]
	\centering
	\includegraphics[width=\linewidth]{figs/fig3.pdf}\vspace{-1mm}
	\caption{Illustration of appearance knowledge generation (AKG).}
        \vspace{-3mm}
	\label{fig:akgfig}
\end{figure}

\subsection{Spatial Knowledge Generation (SKG)}\label{skg}

In addition to appearance knowledge, we explore target spatial knowledge from the video for improving video features. Specifically, inspired by the \emph{observation} that intermediate attention maps from previous attention operations reflect the spatial cues of target in each frame to some extent, similar to the concept of ``\emph{saliency}'' but for the target, we propose the \emph{spatial knowledge generation} (SKG) module, which works to leverage readily available attention maps as guidance for enhancing target while suppressing background in the video features, enabling more focus on the target in PRVQL.

Concretely in our SKG, we exploit the target-aware spatial attention maps $\mathcal{S}_k$ from cross-attention block in Eq.~(\ref{eq1}) and temporal-aware spatial attention maps $\mathcal{T}_k$ from masked self-attention block in Eq.~(\ref{eq3}) for spatial knowledge learning. Specifically, given $\mathcal{S}_k$ and $\mathcal{T}_k$, we first extract the inter-frame spatial attention maps $\mathcal{T}_k^{\text{d}}$ by extracting diagonal elements from $\mathcal{T}_k$ as follows,
\begin{equation}\label{inter-frame}
\setlength{\abovedisplayskip}{7pt}
\setlength{\belowdisplayskip}{7pt}
\mathcal{T}_k^{\text{d}}=\phi_{\text{diag}}(\mathcal{T}_k)=\{t_{i}^k\}_{i=1}^{L}
\end{equation}
where $\phi_{\text{diag}}$ denotes the operation to extract diagonal elects, and $t_{i}^k\in \mathbb{R}^{hw\times hw}$ represents the attention maps for frame $i$. To match the spatial dimension of $\mathcal{T}_k^{\text{d}}$ and $\mathcal{S}_k$, we first perform bilinear interpolation on $\mathcal{T}_k^{\text{d}}$ to increase its spatial resolution to $HW \times HW$, and then combines these two attention maps to obtain spatial knowledge. Mathematically, this process can be expressed as follows,
\begin{equation}\label{inter-frame2}
\setlength{\abovedisplayskip}{7pt}
\setlength{\belowdisplayskip}{7pt}
\mathcal{K}_k^{s}=\alpha \cdot \varphi_{\text{int}}(\mathcal{T}_k^{\text{d}}) + (1-\alpha) \cdot \mathcal{S}_k
\end{equation}
where $\varphi_{\text{int}}$ denotes the bilinear interpolation operation, $\mathcal{K}_k^{s}$ is the target spatial knowledge, and $\alpha$ is a parameter to balance different attention maps. Since $\mathcal{K}_k^{s}$ indicates the target position cues in each frame in some degree, we can use it to highlight target while restraining background in videos for improving localization. Similar to AKG, SKG is deployed in each but the last stage of PRVQL. Fig.~\ref{fig:skgfig} illustrates SKG. 

\begin{figure}[t]
	\centering
	\includegraphics[width=0.9\linewidth]{figs/fig3-skg.pdf}\vspace{-1mm}
	\caption{Illustration of spatial knowledge generation (SKG).}
        \vspace{-3mm}
	\label{fig:skgfig}
\end{figure}


\subsection{Feature Refinement with Knowledge}\label{update}

With target appearance knowledge $\mathcal{K}_{k}^{a}$ and spatial knowledge $\mathcal{K}_{k}^{s}$ obtained from AKG and SKG in stage $k$ ($1<k\le K$), we then apply them as guidance to refine the query and video features through \emph{query feature refinement} (QFR) and \emph{video feature refinement} (VFR) modules.

\vspace{0.3em}
\noindent
\textbf{Query Feature Refinement (QFR).} QFR aims to refine the query feature with guidance from learned target appearance knowledge. Specifically, it adopts a cross-attention block to fuse appearance knowledge $\mathcal{K}_{k}^{a}$ into the query. Given the query feature $\mathcal{Q}_k$ and appearance knowledge $\mathcal{K}_{k}^{a}$ in stage $k$, we first apply a Conv block on $\mathcal{K}_{k}^{a}$ and then perform refinement via QFR as follows,
\begin{equation}\label{eq:q_update}
\setlength{\abovedisplayskip}{7pt} 
\setlength{\belowdisplayskip}{7pt}
\mathcal{Q}_{k+1} = \mathtt{QFR}(\mathcal{Q}_{k}, \mathcal{K}_{k}^{a}) = \mathtt{CAB}(\mathcal{Q}_{k}, \mathtt{CNB}(\mathcal{K}_k^a))
\end{equation}
where $\mathcal{Q}_{k+1}$ is refined query feature and fed to next stage for learning more accurate knowledge, which in turn leads to better query feature for localization in the final stage. It is worth noting that, besides cross-attention, we explore different strategies to combine $\mathcal{Q}_k$ and $\mathcal{K}_{k}^{a}$, including addition and concatenation operations. We observe that using cross-attention achieves the best performance, as exhibited in our experiments provided in the \emph{supplementary material}.

\vspace{0.3em}
\noindent
\textbf{Video Feature Refinement (VFR).} VFR focuses on adopting target spatial knowledge to refine initial video feature by enhancing target while suppressing the background regions. Concretely, given the initial video feature $\mathcal{V}_1$ and learn spatial knowledge $\mathcal{K}_{k}^{s}$ in stage $k$, we use residual connection to refine $\mathcal{V}_1$ as follows,
\begin{equation}\label{eq:v_update}
\setlength{\abovedisplayskip}{7pt} 
\setlength{\belowdisplayskip}{7pt}
\mathcal{V}_{k+1} = \beta \cdot (\mathcal{K}_{k}^{s} \odot \mathcal{V}_1) + (1-\beta) \cdot \mathcal{V}_1
\end{equation}
where $\mathcal{V}_{k+1}$ denotes the refined video feature that is used for the next stage, $\beta$ is a balancing parameter, and $\odot$ represents the pixel-wise multiplication.

\subsection{Optimization and Inference}\label{opt}

\textbf{Optimization}. Given a video and a visual query, we predict confidence scores $\tilde{\mathcal{C}}_k$ and target boxes $\tilde{\mathcal{B}}_k$ ($\tilde{\mathcal{B}}_k=\Delta\tilde{\mathcal{B}}_k+\tilde{\mathcal{B}}$) in each stage $k$ ($1\le k\le K$). During training, given the groundtruth boxes $\mathcal{B}^*$ and temporal occurrence scores $\mathcal{S}^*$, we design the following loss function $\mathcal{L}_k$ for stage $k$,
\begin{equation}\label{eq:loss}
\setlength{\abovedisplayskip}{7pt} 
\setlength{\belowdisplayskip}{7pt}
\mathcal{L}_k = \mathcal{L}_{\text{L}_1}(\tilde{\mathcal{B}}_k, \mathcal{B}^*) + \lambda_1\mathcal{L}_{\text{GIoU}}(\tilde{\mathcal{B}}_k, \mathcal{B}^*) + \lambda_2\mathcal{L}_{\text{BCE}}(\tilde{\mathcal{S}}_k, \mathcal{S}^*)
\end{equation}
where $\mathcal{L}_{\text{L}_1}$, $\mathcal{L}_{\text{GIoU}}$, and $\mathcal{L}_{\text{BCE}}$ represent the $L_1$ loss, generalized IoU (GIoU)~\cite{rezatofighi2019generalized} loss, and binary cross-entropy (BCE) loss, respectively. $\lambda_1$ and $\lambda_2$ are two balancing parameters. With Eq.~(\ref{eq:loss}), the total training loss $\mathcal{L}_{\text{total}}$ can be obtained via $\mathcal{L}_{\text{total}}=\sum_{k=1}^{K} \mathcal{L}_k$. Following~\cite{jiang2024single,xu2023my,grauman2022ego4d}, we perform hard negative mining during training to decrease false positive prediction. For details, please refer to~\cite{jiang2024single,xu2023my,grauman2022ego4d}.

\vspace{0.3em}
\noindent
\textbf{Inference.} We employ the same strategy as in~\cite{jiang2024single} to obtain the prediction result. Specifically, during inference, we first obtain the target region in each frame by selecting target box with the highest confidence score. Please note that, the target regions with confidences scores smaller than a threshold, set to 0.79, will be discarded. After this, we select the most recent peak and generate a response track via bidirectional search from the peak. Details can be seen in~\cite{jiang2024single}.


\section{Experiments}

\textbf{Implementation.} Our PRVQL is implemented using PyTorch~\cite{paszke2019pytorch} with Nvidia RTX A6000 GPUs. Similar to~\cite{jiang2024single}, we use the popular ViT~\cite{DosovitskiyB0WZ21} pretrained with DINOv2~\cite{OquabDMVSKFHMEA24} as the backbone. Our PRVQL is end-to-end trained for 50 epoches (a total of 60K iterations) with a batch size of 12, utilizing the AdamW optimizer~\cite{LoshchilovH19} with a peak learning rate of $10^{-4}$ and a weight decay of $5\times10^{-2}$. The query image and video frames are resized to $480\times480$. The number of stages $K$ in PRVQL is empirically set to 3, and the pooling size for RoIAlign is 5. The number of selected boxes $n$ for appearance knowledge is 3, and the threshold $\tau$ is set to $0.7$. The parameter $\alpha$ for computing spatial knowledge is empirically set to 0.5. The balancing parameter $\beta$ is 0.1. $\lambda_1$ and $\lambda_2$ are empirically set to 0.3 and 100. The video frame length $L$, similar to~\cite{jiang2024single}, is set to $30$ with frames randomly selected to ensure coverage of at least a portion of the response track. For the anchor boxes in localization, we employ four scales ($16^{2}$, $32^{2}$, $64^{2}$, $48^{2}$) with three aspect ratios (0.5, 1, 2) for each anchor box, similar to~\cite{jiang2024single}.

\begin{table}[!t]
	\centering
        \caption{Comparison on the Ego4D validation set.}\vspace{-2mm}
	\renewcommand{\arraystretch}{1.05}
	\scalebox{0.92}{
	\begin{tabular}{rcccc}
	\rowcolor{mygray}
	\specialrule{1.5pt}{0pt}{0pt}
	Methods & tAP$_{25}$ & stAP$_{25}$ & rec\% &  Succ    \\
	\hline
	\hline
        STARK \textcolor{lightblue}{\scriptsize{[ICCV'21]}}  & 0.10   & 0.04   & 12.41  & 18.70    \\
	SiamRCNN \textcolor{lightblue}{\scriptsize{[CVPR'22]}}  & 
        0.22   & 0.15   & 32.92  & 43.24    \\
	NFM \textcolor{lightblue}{\scriptsize{[VQ2D Challenge'22]}}     & 0.26  & 0.19  & 37.88  & 47.90    \\
	CocoFormer \textcolor{lightblue}{\scriptsize{[CVPR'23]}} 
        & 0.26  & 0.19  & 37.67  & 47.68    \\
	VQLoC \textcolor{lightblue}{\scriptsize{[NeurIPS'23]}}  
        & 0.31  & 0.22  & 47.05  & 55.89    \\
        \hline
	\rowcolor{highlight} PRVQL (ours) & \textbf{0.35} & \textbf{0.27} & 
        \textbf{47.87} & \textbf{57.93}   \\
        \specialrule{1.5pt}{0pt}{0pt}
	\end{tabular}}
        \label{tab:sota_val}
\end{table}

\begin{table}[!t]
	\centering
        \caption{Comparison on the Ego4D test set.}\vspace{-2mm}
	\renewcommand{\arraystretch}{1.05}
	\scalebox{0.92}{
	\begin{tabular}{rcccc}
	\rowcolor{mygray}
	\specialrule{1.5pt}{0pt}{0pt}
	Methods & tAP$_{25}$ & stAP$_{25}$ & rec\% &  Succ    \\
	\hline
	\hline
        STARK \textcolor{lightblue}{\scriptsize{[ICCV'21]}}  & -   & -   & -  & -    \\
	SiamRCNN \textcolor{lightblue}{\scriptsize{[CVPR'22]}}  & 
        0.20   & 0.13   & -  & -    \\
	NFM \textcolor{lightblue}{\scriptsize{[VQ2D Challenge'22]}}     & 0.24  & 0.17  & -  & -    \\
	CocoFormer \textcolor{lightblue}{\scriptsize{[CVPR'23]}} 
        & 0.25  & 0.18  & -  & -    \\
	VQLoC \textcolor{lightblue}{\scriptsize{[NeurIPS'23]}}  
        & 0.32  & 0.24  & 45.11  & 55.88    \\
        \hline
	\rowcolor{highlight} PRVQL (ours) & \textbf{0.37} & \textbf{0.28} & 
        \textbf{45.70} & \textbf{59.43}   \\
        \specialrule{1.5pt}{0pt}{0pt}
	\end{tabular}}
        \label{tab:sota_tst}
\end{table}

\begin{table}[!t]\small
\setlength{\tabcolsep}{2.2pt}
	\centering
        \caption{Comparison of speed on Ego4D.}\vspace{-2mm}
	\renewcommand{\arraystretch}{1.05}
	\scalebox{0.95}{
	\begin{tabular}{rcccccc}
	\rowcolor{mygray}
	\specialrule{1.5pt}{0pt}{0pt}
	 & STARK & SiamRCNN & NFM &  CocoFormer & VQLoC & PRVQL    \\
	\hline
	\hline
        FPS  & 33   & 3   & 3  & 3 & 36 &  30  \\
        \specialrule{1.5pt}{0pt}{0pt}
	\end{tabular}}
        \label{tab:sota_fps}\vspace{-3mm}
\end{table}

\subsection{Dataset and Evaluation Metrics}

\textbf{Dataset.} Following previous methods~\cite{xu2023my,jiang2024single}, we conduct experiments on the challenging Ego4D benchmark~\cite{grauman2022ego4d}. Ego4D is a recently proposed large-scale dataset dedicated to first-person video understanding. Similar to~\cite{jiang2024single}, we use videos from the VQ2D task. There are 13.6K, 4.5K, 4.4K pairs of queries and videos for training, validation, and testing, lasting 262, 87, and 84 hours, respectively.

\vspace{0.5em}
\noindent
\textbf{Evaluation Metrics.} Following~\cite{xu2023my,jiang2024single}, we adopt the metrics provided by Ego4D~\cite{grauman2022ego4d} for evaluation, including temporal average precision (tAP$_{25}$), spatio-temporal average precision (stAP$_{25}$), recovery (rec\%), and success (Succ). tAP$_{25}$ and stAP$_{25}$ are used to measure the accuracy of the predicted temporal and spatio-temporal extends of the of response tracks in comparison to groundtruth using the Intersection over Union (IoU) with threshold 0.25. The recovery metric assess the percentage of predicted frames in which the IoU between predicted bounding box and ground-truth is great than or equal to 0.5, and success metric measures weather the IoU between prediction and groundtruth exceeds 0.05. For more details of metrics, please refer to~\cite{grauman2022ego4d}.



\subsection{State-of-the-art Comparison}

In order to verify the effectiveness of our PRVQL, we compare it with other state-of-the-art methods on Ego4D, including STARK~\cite{yan2021learning}, SiamRCNN~\cite{voigtlaender2020siam}, NFM~\cite{xu2022negative}, CocoFormer~\cite{xu2023my}, and VQLoC~\cite{jiang2024single}. Tab.~\ref{tab:sota_val} displays the results and comparison on the Ego4D validate test. As in Tab.~\ref{tab:sota_val}, we can clearly see that the proposed PRVQL achieves the best performance on all four metrics. In particular, it achieves the 0.35 tAP$_{25}$ and 0.27 stAP$_{25}$ scores, which outperforms the second best method VQLoC with 0.31 tAP$_{25}$ and 0.22 stAP$_{25}$ scores by 4\% and 5\%. Besides, the rec and Succ scores of PRVQL are 47.87\% and 57.93 respectively, which surpasses the 47.05\% rec\ and 55.89 Succ scores of VQLoC, evidencing the effectiveness of our approach. In addition, in Tab.~\ref{tab:sota_tst} we further report the experimental results and comparison on Ego4D test set. As in Tab.~\ref{tab:sota_tst}, our PRVQL again achieves the best performance on all four metrics. Specifically, PRVQL obtains the 0.37 tAP$_{25}$ and 0.28 stAP$_{25}$ scores. Compared to the second best method VQLoC, our approach outperforms it by 5\% and and 4\%, respectively, on tAP$_{25}$ and stAP$_{25}$. In addition, the rec and Succ scores of PRVQL are 45.70\% and 59.43, which are better than those of VQLoC with 45.11\% and 55.88. All these show the efficacy of target knowledge in improving EgoVQL.

In addition, we show the comparison of speed, measured by frames per second (\emph{FPS}), for different methods in Tab.~\ref{tab:sota_fps}. From Tab.~\ref{tab:sota_fps}, we can see our method runs fast at a speed of 30 FPS. Despite being slightly slower than VQLoC running at a speed of 36 FPS, our PRVQL is more robust in localization, showing a better balance between accuracy and speed.

\begin{table}[!t]
        % \setlength{\tabcolsep}{5.pt}
	\centering
        \caption{Ablation studies of AKG and SKG.}\vspace{-2mm}
	\renewcommand{\arraystretch}{1.1}
	\scalebox{0.92}{
		\begin{tabular}{cccccccc}
			\specialrule{1.5pt}{0pt}{0pt}
			\rowcolor{mygray} 
			& AKG & SKG & tAP$_{25}$ & stAP$_{25}$ & rec\% &  Succ \\ \hline\hline
			\ding{182} & - & - & 0.32 & 0.23 & 45.24 & 55.37 \\
			\ding{183} & \checkmark & - & 0.34 & 0.26 & 47.34 & 57.27 \\
            \ding{184} & - & \checkmark & 0.33 & 0.24 & 46.33 & 56.46 \\
			\ding{185} & \checkmark & \checkmark & \textbf{0.35} & \textbf{0.27} & \textbf{47.87} & \textbf{57.93} \\ \specialrule{1.5pt}{0pt}{0pt}
	\end{tabular}}
	\label{KG_modules}
\end{table}

\begin{table}[!t]
        \setlength{\tabcolsep}{7.7pt}
	\centering
        \caption{Ablation studies on the number of stages.}\vspace{-2mm}
	\renewcommand{\arraystretch}{1}
	\scalebox{0.92}{
		\begin{tabular}{cccccc}
		\specialrule{1.5pt}{0pt}{0pt}
		\rowcolor{mygray} 
		& \# Stages & tAP$_{25}$ & stAP$_{25}$ & rec\% &  Succ \\ \hline\hline
           \ding{182} & $K=1$ & 0.32 & 0.23 & 45.24 & 55.37 \\
           \ding{183} & $K=2$ & 0.34 & \textbf{0.27} & 47.25 & 56.43 \\
           \ding{184} & $K=3$ & \textbf{0.35} & \textbf{0.27} & \textbf{47.87} & \textbf{57.93} \\
           \ding{185} & $K=4$ & 0.33 & 0.26 & 45.91 & 55.29 \\
	\specialrule{1.5pt}{0pt}{0pt}
    \end{tabular}}
	\label{tab:stage}\vspace{-3mm}
\end{table}


\subsection{Ablation Study}

For better understanding of PRVQL, we conduct extensive ablation studies on Ego4D validation set as follows. 

\vspace{0.3em}
\noindent
\textbf{Impact of AKG and SKG.} AKG and SKG are two important modules in PRVQL for target appearance and spatial knowledge generation. In order to analyze these two modules, we conduct thorough ablation studies in Tab.~\ref{KG_modules}. From Tab.~\ref{KG_modules}, we can see that, without AKG and SKG, the tAP$_{25}$ and stAP$_{25}$ scores are 0.32 and 0.23, respectively (\ding{182}). By applying AKG alone for refinement with appearance knowledge, they can be significantly improved to 0.34 and 0.26 with performance gains of 0.02 and 0.03 (\ding{183} \emph{v.s.} \ding{182}). When using only SKG for refinement with spatial knowledge, tAP$_{25}$ and stAP$_{25}$ are improved to 0.33 and 0.24 (\ding{184} \emph{v.s.} \ding{182}). From this table, we can also observe that, using appearance knowledge for refinement in PRVQL brings more gains than the spatial knowledge (\ding{183} \emph{v.s.} \ding{184}). When using both AKG and SKG in our PRVQL, we achieve the best performance with 0.35 tAP$_{25}$ and 0.27 stAP$_{25}$ scores (\ding{185} \emph{v.s.} \ding{182}), which clearly evidences the efficacy of target knowledge for improving the robustness of EgoVQL.

\begin{table}[!t]
        \setlength{\tabcolsep}{7.5pt}
	\centering
    \caption{Ablation studies on the threshold $\tau$.}\vspace{-2mm}
			\renewcommand{\arraystretch}{1}
			\scalebox{0.92}{
				\begin{tabular}{ccccccc}
					\specialrule{1.5pt}{0pt}{0pt}
					\rowcolor{mygray} 
					& Threshold & tAP$_{25}$ & stAP$_{25}$ & rec\% &  Succ \\ \hline\hline
					\ding{182}& $\tau=0.6$ & 0.32 & 0.24 & 44.82 & 55.97 \\
                      \ding{183} &  $\tau=0.7$ & \textbf{0.35} & \textbf{0.27} & \textbf{47.87} & \textbf{57.93} \\
                       \ding{184} & $\tau=0.8$ & 0.34 & 0.26 & 46.53 & 57.33 \\
					\specialrule{1.5pt}{0pt}{0pt}
			\end{tabular}}
			\label{top_threshold}
\end{table}

\begin{table}[!t]
        \setlength{\tabcolsep}{8.8pt}
	\centering
    \caption{Ablation studies on the number of target boxes in AKG.}\vspace{-2mm}   
	\renewcommand{\arraystretch}{1}
			\scalebox{0.92}{
				\begin{tabular}{cccccc}
					\specialrule{1.5pt}{0pt}{0pt}
					\rowcolor{mygray} 
					 & & tAP$_{25}$ & stAP$_{25}$ & rec\% &  Succ \\ \hline\hline
                      \ding{182} &  $n=2$ & 0.34 & 0.26 & 47.27 & 56.03 \\
                       \ding{183} & $n=3$  & 0.35 & \textbf{0.27} & \textbf{47.87} & \textbf{57.93} \\
                       \ding{184} & $n=4$  & \textbf{0.36} & 0.26 & 46.59 & 57.62 \\
                       \ding{185} & $n=5$  & 0.35 & 0.24 & 46.84 & 56.95 \\
					\specialrule{1.5pt}{0pt}{0pt}
			\end{tabular}}
			\label{tab:top_k}
\end{table}

\begin{table}[!t]
    \setlength{\tabcolsep}{8.6pt}
    \centering
    \caption{Ablation studies on RoIAlign feature size.}\vspace{-2mm}
    \renewcommand{\arraystretch}{1}
    \scalebox{0.92}{
	\begin{tabular}{ccccccc}
	\specialrule{1.5pt}{0pt}{0pt}
	\rowcolor{mygray} 
					& Size & tAP$_{25}$ & stAP$_{25}$ & rec\% &  Succ \\ \hline\hline
				  \ding{182} & 3 & 0.33 & 0.26 & 46.94 & 56.06 \\
                    \ding{183} & 5 & \textbf{0.35} & \textbf{0.27} & \textbf{47.87} & \textbf{57.93} \\
                    \ding{184} & 7 & 0.34 & \textbf{0.27} & 47.58 & 57.38 \\
                    \ding{185} & 9 & 0.32 & 0.25 & 46.37 & 55.27 \\
					\specialrule{1.5pt}{0pt}{0pt}
			\end{tabular}}
			\label{tab:roisize}\vspace{-3mm}
\end{table}

\begin{figure*}[!t]
    \centering
    \includegraphics[width=0.9\linewidth]{figs/fig4.pdf}\vspace{-2mm}
    \caption{Qualitative analysis and comparison between our PRVQL and state-of-the-art VQLoC in representative videos with different challenges. We observe that, owing to our target knowledge from videos, PRVQL can more robustly localize the target of interest.}
    \label{fig:visual_comparison}\vspace{-3mm}
\end{figure*}

\vspace{0.3em}
\noindent
\textbf{Impact of the number of stages.} Our PRVQL is designed as a progressive architecture with $K$ stages to explore the target knowledge for refinement. In this work, we conduct an ablation study on the number of stages $K$ in PRVQL as shown in Tab.~\ref{tab:stage}. From Tab.~\ref{tab:stage}, we observe that, when setting $K=1$, which means only one stage is used and the target knowledge is not used due to one-stage design, the tAP$_{25}$ and stAP$_{25}$ scores are 0.32 and 0.23 (\ding{182}). When adding the second stage, tAP$_{25}$ and stAP$_{25}$ can be largely improved by 2\% and 4\% to 0.34 and 0.27, respectively (\ding{183}). With three stages, the tAP$_{25}$ and stAP$_{25}$ scores can be further boosted to 0.35 and 0.27 (\ding{184}). When setting $K=4$ with 4 stages, the performance is decreased with 0.33 tAP$_{25}$ and 0.26 stAP$_{25}$ scores (\ding{185}). Therefore, we set $K$ to 3 in this work.

\vspace{0.3em}
\noindent
\textbf{Impact of threshold $\tau$ in AKG.} The threshold $\tau$ is used to filter out less confident target regions in AKG, aiming to avoid noisy features in appearance knowledge generation. In this work, we conduct an ablation to study the impact of $\tau$ on the final performance in Tab.~\ref{top_threshold}. As shown in Tab.~\ref{top_threshold}, we can see that, when setting $\tau$ to 0.7, PRVQL achieves the best performance on all four metrics (\ding{183}). 

\vspace{0.3em}
\noindent
\textbf{Impact of number of target boxes in AKG.} In AKG, we extract visual features from the top $n$ highly confident target regions for appearance knowledge generation. We conduct an ablation on $n$ in Tab.~\ref{tab:top_k}. From Tab.~\ref{tab:top_k}, we can observe that, when using the top 3 target regions for knowledge learning in AKG, we achieve the best overall performance (\ding{183}).



\vspace{0.3em}
\noindent
\textbf{Impact of RoIAlign Feature Size.} With the top $n$ selected target regions, we perform the RoIAlign operation~\cite{he2017mask} to obtain target appearance knowledge. The  RoIAlign feature size may have an impact on the target appearance knowledge. A too small size may result in the coarse spatial information of the target, while a too large size may lead to losing discriminative local features for the target, both degrading performance. In this work, we study different RoIAlign feature sizes in Tab.~\ref{tab:roisize}. As shown, when setting the size to 5 in RoIAlign, PRVQL shows the best overall performance.


\subsection{Qualitative Analysis}

In order to provide further analysis of our PRVQL, we show the visualization results of its localization and compare it with the state-of-the-art VQLoC in Fig.~\ref{fig:visual_comparison}. Specifically, we show the results and comparison on several representative videos, including video in (a) with \emph{pose variation}, video in (b) with \emph{cluttering background} and \emph{out-of-view}, video in (c) with \emph{occlusion} and \emph{low resolution}, video in (d) with \emph{pose variation} and \emph{cluttering background}, and video in (d) with \emph{motion blur} and \emph{distractor}. From Fig.~\ref{fig:visual_comparison}, we can observe that, our method can robustly and accurately localize the target of interest in all these challenges, owing to the help of target knowledge from the videos, while VQLoC is prone to drift to the background due to lack of discriminative target information, which evidences the effectiveness of target cues in videos for improving EgoVQL.

Due to limited space, we demonstrate more results, analysis, and ablation studies in the \emph{supplementary material}. 

\section{Conclusion}

In this paper, we present a novel approach, dubbed PRVQL, for improving EgoVQL via exploring crucial target knowledge from videos to refine features for robust localization. Our PRVQL is implemented as a multi-stage architecture. In each stage, two key modules, including AKG and SKG, are used to extract target appearance and spatial knowledge from the video. The knowledge from one stage is used as guidance to refine query and video features in the next stage, which are adopted for
learning more accurate knowledge for further feature refinement. Through this progressive process, PRVQL learns gradually improved knowledge, which in turn leads to better refined features for target localization in the final stage. To validate the effectiveness of PRVQL, we conduct experiments on Ego4D. Our experimental results show that PRVQL achieves state-of-the-art result and largely surpasses other methods, showing its efficacy.

{
\small
\bibliographystyle{ieeenat_fullname}
\bibliography{main}
% % This must be in the first 5 lines to tell arXiv to use pdfLaTeX, which is strongly recommended.
\pdfoutput=1
% In particular, the hyperref package requires pdfLaTeX in order to break URLs across lines.

\documentclass[11pt]{article}

% Change "review" to "final" to generate the final (sometimes called camera-ready) version.
% Change to "preprint" to generate a non-anonymous version with page numbers.
\usepackage{acl}

% Standard package includes
\usepackage{times}
\usepackage{latexsym}

% Draw tables
\usepackage{booktabs}
\usepackage{multirow}
\usepackage{xcolor}
\usepackage{colortbl}
\usepackage{array} 
\usepackage{amsmath}

\newcolumntype{C}{>{\centering\arraybackslash}p{0.07\textwidth}}
% For proper rendering and hyphenation of words containing Latin characters (including in bib files)
\usepackage[T1]{fontenc}
% For Vietnamese characters
% \usepackage[T5]{fontenc}
% See https://www.latex-project.org/help/documentation/encguide.pdf for other character sets
% This assumes your files are encoded as UTF8
\usepackage[utf8]{inputenc}

% This is not strictly necessary, and may be commented out,
% but it will improve the layout of the manuscript,
% and will typically save some space.
\usepackage{microtype}
\DeclareMathOperator*{\argmax}{arg\,max}
% This is also not strictly necessary, and may be commented out.
% However, it will improve the aesthetics of text in
% the typewriter font.
\usepackage{inconsolata}

%Including images in your LaTeX document requires adding
%additional package(s)
\usepackage{graphicx}
% If the title and author information does not fit in the area allocated, uncomment the following
%
%\setlength\titlebox{<dim>}
%
% and set <dim> to something 5cm or larger.

\title{Wi-Chat: Large Language Model Powered Wi-Fi Sensing}

% Author information can be set in various styles:
% For several authors from the same institution:
% \author{Author 1 \and ... \and Author n \\
%         Address line \\ ... \\ Address line}
% if the names do not fit well on one line use
%         Author 1 \\ {\bf Author 2} \\ ... \\ {\bf Author n} \\
% For authors from different institutions:
% \author{Author 1 \\ Address line \\  ... \\ Address line
%         \And  ... \And
%         Author n \\ Address line \\ ... \\ Address line}
% To start a separate ``row'' of authors use \AND, as in
% \author{Author 1 \\ Address line \\  ... \\ Address line
%         \AND
%         Author 2 \\ Address line \\ ... \\ Address line \And
%         Author 3 \\ Address line \\ ... \\ Address line}

% \author{First Author \\
%   Affiliation / Address line 1 \\
%   Affiliation / Address line 2 \\
%   Affiliation / Address line 3 \\
%   \texttt{email@domain} \\\And
%   Second Author \\
%   Affiliation / Address line 1 \\
%   Affiliation / Address line 2 \\
%   Affiliation / Address line 3 \\
%   \texttt{email@domain} \\}
% \author{Haohan Yuan \qquad Haopeng Zhang\thanks{corresponding author} \\ 
%   ALOHA Lab, University of Hawaii at Manoa \\
%   % Affiliation / Address line 2 \\
%   % Affiliation / Address line 3 \\
%   \texttt{\{haohany,haopengz\}@hawaii.edu}}
  
\author{
{Haopeng Zhang$\dag$\thanks{These authors contributed equally to this work.}, Yili Ren$\ddagger$\footnotemark[1], Haohan Yuan$\dag$, Jingzhe Zhang$\ddagger$, Yitong Shen$\ddagger$} \\
ALOHA Lab, University of Hawaii at Manoa$\dag$, University of South Florida$\ddagger$ \\
\{haopengz, haohany\}@hawaii.edu\\
\{yiliren, jingzhe, shen202\}@usf.edu\\}



  
%\author{
%  \textbf{First Author\textsuperscript{1}},
%  \textbf{Second Author\textsuperscript{1,2}},
%  \textbf{Third T. Author\textsuperscript{1}},
%  \textbf{Fourth Author\textsuperscript{1}},
%\\
%  \textbf{Fifth Author\textsuperscript{1,2}},
%  \textbf{Sixth Author\textsuperscript{1}},
%  \textbf{Seventh Author\textsuperscript{1}},
%  \textbf{Eighth Author \textsuperscript{1,2,3,4}},
%\\
%  \textbf{Ninth Author\textsuperscript{1}},
%  \textbf{Tenth Author\textsuperscript{1}},
%  \textbf{Eleventh E. Author\textsuperscript{1,2,3,4,5}},
%  \textbf{Twelfth Author\textsuperscript{1}},
%\\
%  \textbf{Thirteenth Author\textsuperscript{3}},
%  \textbf{Fourteenth F. Author\textsuperscript{2,4}},
%  \textbf{Fifteenth Author\textsuperscript{1}},
%  \textbf{Sixteenth Author\textsuperscript{1}},
%\\
%  \textbf{Seventeenth S. Author\textsuperscript{4,5}},
%  \textbf{Eighteenth Author\textsuperscript{3,4}},
%  \textbf{Nineteenth N. Author\textsuperscript{2,5}},
%  \textbf{Twentieth Author\textsuperscript{1}}
%\\
%\\
%  \textsuperscript{1}Affiliation 1,
%  \textsuperscript{2}Affiliation 2,
%  \textsuperscript{3}Affiliation 3,
%  \textsuperscript{4}Affiliation 4,
%  \textsuperscript{5}Affiliation 5
%\\
%  \small{
%    \textbf{Correspondence:} \href{mailto:email@domain}{email@domain}
%  }
%}

\begin{document}
\maketitle
\begin{abstract}
Recent advancements in Large Language Models (LLMs) have demonstrated remarkable capabilities across diverse tasks. However, their potential to integrate physical model knowledge for real-world signal interpretation remains largely unexplored. In this work, we introduce Wi-Chat, the first LLM-powered Wi-Fi-based human activity recognition system. We demonstrate that LLMs can process raw Wi-Fi signals and infer human activities by incorporating Wi-Fi sensing principles into prompts. Our approach leverages physical model insights to guide LLMs in interpreting Channel State Information (CSI) data without traditional signal processing techniques. Through experiments on real-world Wi-Fi datasets, we show that LLMs exhibit strong reasoning capabilities, achieving zero-shot activity recognition. These findings highlight a new paradigm for Wi-Fi sensing, expanding LLM applications beyond conventional language tasks and enhancing the accessibility of wireless sensing for real-world deployments.
\end{abstract}

\section{Introduction}

In today’s rapidly evolving digital landscape, the transformative power of web technologies has redefined not only how services are delivered but also how complex tasks are approached. Web-based systems have become increasingly prevalent in risk control across various domains. This widespread adoption is due their accessibility, scalability, and ability to remotely connect various types of users. For example, these systems are used for process safety management in industry~\cite{kannan2016web}, safety risk early warning in urban construction~\cite{ding2013development}, and safe monitoring of infrastructural systems~\cite{repetto2018web}. Within these web-based risk management systems, the source search problem presents a huge challenge. Source search refers to the task of identifying the origin of a risky event, such as a gas leak and the emission point of toxic substances. This source search capability is crucial for effective risk management and decision-making.

Traditional approaches to implementing source search capabilities into the web systems often rely on solely algorithmic solutions~\cite{ristic2016study}. These methods, while relatively straightforward to implement, often struggle to achieve acceptable performances due to algorithmic local optima and complex unknown environments~\cite{zhao2020searching}. More recently, web crowdsourcing has emerged as a promising alternative for tackling the source search problem by incorporating human efforts in these web systems on-the-fly~\cite{zhao2024user}. This approach outsources the task of addressing issues encountered during the source search process to human workers, leveraging their capabilities to enhance system performance.

These solutions often employ a human-AI collaborative way~\cite{zhao2023leveraging} where algorithms handle exploration-exploitation and report the encountered problems while human workers resolve complex decision-making bottlenecks to help the algorithms getting rid of local deadlocks~\cite{zhao2022crowd}. Although effective, this paradigm suffers from two inherent limitations: increased operational costs from continuous human intervention, and slow response times of human workers due to sequential decision-making. These challenges motivate our investigation into developing autonomous systems that preserve human-like reasoning capabilities while reducing dependency on massive crowdsourced labor.

Furthermore, recent advancements in large language models (LLMs)~\cite{chang2024survey} and multi-modal LLMs (MLLMs)~\cite{huang2023chatgpt} have unveiled promising avenues for addressing these challenges. One clear opportunity involves the seamless integration of visual understanding and linguistic reasoning for robust decision-making in search tasks. However, whether large models-assisted source search is really effective and efficient for improving the current source search algorithms~\cite{ji2022source} remains unknown. \textit{To address the research gap, we are particularly interested in answering the following two research questions in this work:}

\textbf{\textit{RQ1: }}How can source search capabilities be integrated into web-based systems to support decision-making in time-sensitive risk management scenarios? 
% \sq{I mention ``time-sensitive'' here because I feel like we shall say something about the response time -- LLM has to be faster than humans}

\textbf{\textit{RQ2: }}How can MLLMs and LLMs enhance the effectiveness and efficiency of existing source search algorithms? 

% \textit{\textbf{RQ2:}} To what extent does the performance of large models-assisted search align with or approach the effectiveness of human-AI collaborative search? 

To answer the research questions, we propose a novel framework called Auto-\
S$^2$earch (\textbf{Auto}nomous \textbf{S}ource \textbf{Search}) and implement a prototype system that leverages advanced web technologies to simulate real-world conditions for zero-shot source search. Unlike traditional methods that rely on pre-defined heuristics or extensive human intervention, AutoS$^2$earch employs a carefully designed prompt that encapsulates human rationales, thereby guiding the MLLM to generate coherent and accurate scene descriptions from visual inputs about four directional choices. Based on these language-based descriptions, the LLM is enabled to determine the optimal directional choice through chain-of-thought (CoT) reasoning. Comprehensive empirical validation demonstrates that AutoS$^2$-\ 
earch achieves a success rate of 95–98\%, closely approaching the performance of human-AI collaborative search across 20 benchmark scenarios~\cite{zhao2023leveraging}. 

Our work indicates that the role of humans in future web crowdsourcing tasks may evolve from executors to validators or supervisors. Furthermore, incorporating explanations of LLM decisions into web-based system interfaces has the potential to help humans enhance task performance in risk control.






\section{Related Work}
\label{sec:relatedworks}

% \begin{table*}[t]
% \centering 
% \renewcommand\arraystretch{0.98}
% \fontsize{8}{10}\selectfont \setlength{\tabcolsep}{0.4em}
% \begin{tabular}{@{}lc|cc|cc|cc@{}}
% \toprule
% \textbf{Methods}           & \begin{tabular}[c]{@{}c@{}}\textbf{Training}\\ \textbf{Paradigm}\end{tabular} & \begin{tabular}[c]{@{}c@{}}\textbf{$\#$ PT Data}\\ \textbf{(Tokens)}\end{tabular} & \begin{tabular}[c]{@{}c@{}}\textbf{$\#$ IFT Data}\\ \textbf{(Samples)}\end{tabular} & \textbf{Code}  & \begin{tabular}[c]{@{}c@{}}\textbf{Natural}\\ \textbf{Language}\end{tabular} & \begin{tabular}[c]{@{}c@{}}\textbf{Action}\\ \textbf{Trajectories}\end{tabular} & \begin{tabular}[c]{@{}c@{}}\textbf{API}\\ \textbf{Documentation}\end{tabular}\\ \midrule 
% NexusRaven~\citep{srinivasan2023nexusraven} & IFT & - & - & \textcolor{green}{\CheckmarkBold} & \textcolor{green}{\CheckmarkBold} &\textcolor{red}{\XSolidBrush}&\textcolor{red}{\XSolidBrush}\\
% AgentInstruct~\citep{zeng2023agenttuning} & IFT & - & 2k & \textcolor{green}{\CheckmarkBold} & \textcolor{green}{\CheckmarkBold} &\textcolor{red}{\XSolidBrush}&\textcolor{red}{\XSolidBrush} \\
% AgentEvol~\citep{xi2024agentgym} & IFT & - & 14.5k & \textcolor{green}{\CheckmarkBold} & \textcolor{green}{\CheckmarkBold} &\textcolor{green}{\CheckmarkBold}&\textcolor{red}{\XSolidBrush} \\
% Gorilla~\citep{patil2023gorilla}& IFT & - & 16k & \textcolor{green}{\CheckmarkBold} & \textcolor{green}{\CheckmarkBold} &\textcolor{red}{\XSolidBrush}&\textcolor{green}{\CheckmarkBold}\\
% OpenFunctions-v2~\citep{patil2023gorilla} & IFT & - & 65k & \textcolor{green}{\CheckmarkBold} & \textcolor{green}{\CheckmarkBold} &\textcolor{red}{\XSolidBrush}&\textcolor{green}{\CheckmarkBold}\\
% LAM~\citep{zhang2024agentohana} & IFT & - & 42.6k & \textcolor{green}{\CheckmarkBold} & \textcolor{green}{\CheckmarkBold} &\textcolor{green}{\CheckmarkBold}&\textcolor{red}{\XSolidBrush} \\
% xLAM~\citep{liu2024apigen} & IFT & - & 60k & \textcolor{green}{\CheckmarkBold} & \textcolor{green}{\CheckmarkBold} &\textcolor{green}{\CheckmarkBold}&\textcolor{red}{\XSolidBrush} \\\midrule
% LEMUR~\citep{xu2024lemur} & PT & 90B & 300k & \textcolor{green}{\CheckmarkBold} & \textcolor{green}{\CheckmarkBold} &\textcolor{green}{\CheckmarkBold}&\textcolor{red}{\XSolidBrush}\\
% \rowcolor{teal!12} \method & PT & 103B & 95k & \textcolor{green}{\CheckmarkBold} & \textcolor{green}{\CheckmarkBold} & \textcolor{green}{\CheckmarkBold} & \textcolor{green}{\CheckmarkBold} \\
% \bottomrule
% \end{tabular}
% \caption{Summary of existing tuning- and pretraining-based LLM agents with their training sample sizes. "PT" and "IFT" denote "Pre-Training" and "Instruction Fine-Tuning", respectively. }
% \label{tab:related}
% \end{table*}

\begin{table*}[ht]
\begin{threeparttable}
\centering 
\renewcommand\arraystretch{0.98}
\fontsize{7}{9}\selectfont \setlength{\tabcolsep}{0.2em}
\begin{tabular}{@{}l|c|c|ccc|cc|cc|cccc@{}}
\toprule
\textbf{Methods} & \textbf{Datasets}           & \begin{tabular}[c]{@{}c@{}}\textbf{Training}\\ \textbf{Paradigm}\end{tabular} & \begin{tabular}[c]{@{}c@{}}\textbf{\# PT Data}\\ \textbf{(Tokens)}\end{tabular} & \begin{tabular}[c]{@{}c@{}}\textbf{\# IFT Data}\\ \textbf{(Samples)}\end{tabular} & \textbf{\# APIs} & \textbf{Code}  & \begin{tabular}[c]{@{}c@{}}\textbf{Nat.}\\ \textbf{Lang.}\end{tabular} & \begin{tabular}[c]{@{}c@{}}\textbf{Action}\\ \textbf{Traj.}\end{tabular} & \begin{tabular}[c]{@{}c@{}}\textbf{API}\\ \textbf{Doc.}\end{tabular} & \begin{tabular}[c]{@{}c@{}}\textbf{Func.}\\ \textbf{Call}\end{tabular} & \begin{tabular}[c]{@{}c@{}}\textbf{Multi.}\\ \textbf{Step}\end{tabular}  & \begin{tabular}[c]{@{}c@{}}\textbf{Plan}\\ \textbf{Refine}\end{tabular}  & \begin{tabular}[c]{@{}c@{}}\textbf{Multi.}\\ \textbf{Turn}\end{tabular}\\ \midrule 
\multicolumn{13}{l}{\emph{Instruction Finetuning-based LLM Agents for Intrinsic Reasoning}}  \\ \midrule
FireAct~\cite{chen2023fireact} & FireAct & IFT & - & 2.1K & 10 & \textcolor{red}{\XSolidBrush} &\textcolor{green}{\CheckmarkBold} &\textcolor{green}{\CheckmarkBold}  & \textcolor{red}{\XSolidBrush} &\textcolor{green}{\CheckmarkBold} & \textcolor{red}{\XSolidBrush} &\textcolor{green}{\CheckmarkBold} & \textcolor{red}{\XSolidBrush} \\
ToolAlpaca~\cite{tang2023toolalpaca} & ToolAlpaca & IFT & - & 4.0K & 400 & \textcolor{red}{\XSolidBrush} &\textcolor{green}{\CheckmarkBold} &\textcolor{green}{\CheckmarkBold} & \textcolor{red}{\XSolidBrush} &\textcolor{green}{\CheckmarkBold} & \textcolor{red}{\XSolidBrush}  &\textcolor{green}{\CheckmarkBold} & \textcolor{red}{\XSolidBrush}  \\
ToolLLaMA~\cite{qin2023toolllm} & ToolBench & IFT & - & 12.7K & 16,464 & \textcolor{red}{\XSolidBrush} &\textcolor{green}{\CheckmarkBold} &\textcolor{green}{\CheckmarkBold} &\textcolor{red}{\XSolidBrush} &\textcolor{green}{\CheckmarkBold}&\textcolor{green}{\CheckmarkBold}&\textcolor{green}{\CheckmarkBold} &\textcolor{green}{\CheckmarkBold}\\
AgentEvol~\citep{xi2024agentgym} & AgentTraj-L & IFT & - & 14.5K & 24 &\textcolor{red}{\XSolidBrush} & \textcolor{green}{\CheckmarkBold} &\textcolor{green}{\CheckmarkBold}&\textcolor{red}{\XSolidBrush} &\textcolor{green}{\CheckmarkBold}&\textcolor{red}{\XSolidBrush} &\textcolor{red}{\XSolidBrush} &\textcolor{green}{\CheckmarkBold}\\
Lumos~\cite{yin2024agent} & Lumos & IFT  & - & 20.0K & 16 &\textcolor{red}{\XSolidBrush} & \textcolor{green}{\CheckmarkBold} & \textcolor{green}{\CheckmarkBold} &\textcolor{red}{\XSolidBrush} & \textcolor{green}{\CheckmarkBold} & \textcolor{green}{\CheckmarkBold} &\textcolor{red}{\XSolidBrush} & \textcolor{green}{\CheckmarkBold}\\
Agent-FLAN~\cite{chen2024agent} & Agent-FLAN & IFT & - & 24.7K & 20 &\textcolor{red}{\XSolidBrush} & \textcolor{green}{\CheckmarkBold} & \textcolor{green}{\CheckmarkBold} &\textcolor{red}{\XSolidBrush} & \textcolor{green}{\CheckmarkBold}& \textcolor{green}{\CheckmarkBold}&\textcolor{red}{\XSolidBrush} & \textcolor{green}{\CheckmarkBold}\\
AgentTuning~\citep{zeng2023agenttuning} & AgentInstruct & IFT & - & 35.0K & - &\textcolor{red}{\XSolidBrush} & \textcolor{green}{\CheckmarkBold} & \textcolor{green}{\CheckmarkBold} &\textcolor{red}{\XSolidBrush} & \textcolor{green}{\CheckmarkBold} &\textcolor{red}{\XSolidBrush} &\textcolor{red}{\XSolidBrush} & \textcolor{green}{\CheckmarkBold}\\\midrule
\multicolumn{13}{l}{\emph{Instruction Finetuning-based LLM Agents for Function Calling}} \\\midrule
NexusRaven~\citep{srinivasan2023nexusraven} & NexusRaven & IFT & - & - & 116 & \textcolor{green}{\CheckmarkBold} & \textcolor{green}{\CheckmarkBold}  & \textcolor{green}{\CheckmarkBold} &\textcolor{red}{\XSolidBrush} & \textcolor{green}{\CheckmarkBold} &\textcolor{red}{\XSolidBrush} &\textcolor{red}{\XSolidBrush}&\textcolor{red}{\XSolidBrush}\\
Gorilla~\citep{patil2023gorilla} & Gorilla & IFT & - & 16.0K & 1,645 & \textcolor{green}{\CheckmarkBold} &\textcolor{red}{\XSolidBrush} &\textcolor{red}{\XSolidBrush}&\textcolor{green}{\CheckmarkBold} &\textcolor{green}{\CheckmarkBold} &\textcolor{red}{\XSolidBrush} &\textcolor{red}{\XSolidBrush} &\textcolor{red}{\XSolidBrush}\\
OpenFunctions-v2~\citep{patil2023gorilla} & OpenFunctions-v2 & IFT & - & 65.0K & - & \textcolor{green}{\CheckmarkBold} & \textcolor{green}{\CheckmarkBold} &\textcolor{red}{\XSolidBrush} &\textcolor{green}{\CheckmarkBold} &\textcolor{green}{\CheckmarkBold} &\textcolor{red}{\XSolidBrush} &\textcolor{red}{\XSolidBrush} &\textcolor{red}{\XSolidBrush}\\
API Pack~\cite{guo2024api} & API Pack & IFT & - & 1.1M & 11,213 &\textcolor{green}{\CheckmarkBold} &\textcolor{red}{\XSolidBrush} &\textcolor{green}{\CheckmarkBold} &\textcolor{red}{\XSolidBrush} &\textcolor{green}{\CheckmarkBold} &\textcolor{red}{\XSolidBrush}&\textcolor{red}{\XSolidBrush}&\textcolor{red}{\XSolidBrush}\\ 
LAM~\citep{zhang2024agentohana} & AgentOhana & IFT & - & 42.6K & - & \textcolor{green}{\CheckmarkBold} & \textcolor{green}{\CheckmarkBold} &\textcolor{green}{\CheckmarkBold}&\textcolor{red}{\XSolidBrush} &\textcolor{green}{\CheckmarkBold}&\textcolor{red}{\XSolidBrush}&\textcolor{green}{\CheckmarkBold}&\textcolor{green}{\CheckmarkBold}\\
xLAM~\citep{liu2024apigen} & APIGen & IFT & - & 60.0K & 3,673 & \textcolor{green}{\CheckmarkBold} & \textcolor{green}{\CheckmarkBold} &\textcolor{green}{\CheckmarkBold}&\textcolor{red}{\XSolidBrush} &\textcolor{green}{\CheckmarkBold}&\textcolor{red}{\XSolidBrush}&\textcolor{green}{\CheckmarkBold}&\textcolor{green}{\CheckmarkBold}\\\midrule
\multicolumn{13}{l}{\emph{Pretraining-based LLM Agents}}  \\\midrule
% LEMUR~\citep{xu2024lemur} & PT & 90B & 300.0K & - & \textcolor{green}{\CheckmarkBold} & \textcolor{green}{\CheckmarkBold} &\textcolor{green}{\CheckmarkBold}&\textcolor{red}{\XSolidBrush} & \textcolor{red}{\XSolidBrush} &\textcolor{green}{\CheckmarkBold} &\textcolor{red}{\XSolidBrush}&\textcolor{red}{\XSolidBrush}\\
\rowcolor{teal!12} \method & \dataset & PT & 103B & 95.0K  & 76,537  & \textcolor{green}{\CheckmarkBold} & \textcolor{green}{\CheckmarkBold} & \textcolor{green}{\CheckmarkBold} & \textcolor{green}{\CheckmarkBold} & \textcolor{green}{\CheckmarkBold} & \textcolor{green}{\CheckmarkBold} & \textcolor{green}{\CheckmarkBold} & \textcolor{green}{\CheckmarkBold}\\
\bottomrule
\end{tabular}
% \begin{tablenotes}
%     \item $^*$ In addition, the StarCoder-API can offer 4.77M more APIs.
% \end{tablenotes}
\caption{Summary of existing instruction finetuning-based LLM agents for intrinsic reasoning and function calling, along with their training resources and sample sizes. "PT" and "IFT" denote "Pre-Training" and "Instruction Fine-Tuning", respectively.}
\vspace{-2ex}
\label{tab:related}
\end{threeparttable}
\end{table*}

\noindent \textbf{Prompting-based LLM Agents.} Due to the lack of agent-specific pre-training corpus, existing LLM agents rely on either prompt engineering~\cite{hsieh2023tool,lu2024chameleon,yao2022react,wang2023voyager} or instruction fine-tuning~\cite{chen2023fireact,zeng2023agenttuning} to understand human instructions, decompose high-level tasks, generate grounded plans, and execute multi-step actions. 
However, prompting-based methods mainly depend on the capabilities of backbone LLMs (usually commercial LLMs), failing to introduce new knowledge and struggling to generalize to unseen tasks~\cite{sun2024adaplanner,zhuang2023toolchain}. 

\noindent \textbf{Instruction Finetuning-based LLM Agents.} Considering the extensive diversity of APIs and the complexity of multi-tool instructions, tool learning inherently presents greater challenges than natural language tasks, such as text generation~\cite{qin2023toolllm}.
Post-training techniques focus more on instruction following and aligning output with specific formats~\cite{patil2023gorilla,hao2024toolkengpt,qin2023toolllm,schick2024toolformer}, rather than fundamentally improving model knowledge or capabilities. 
Moreover, heavy fine-tuning can hinder generalization or even degrade performance in non-agent use cases, potentially suppressing the original base model capabilities~\cite{ghosh2024a}.

\noindent \textbf{Pretraining-based LLM Agents.} While pre-training serves as an essential alternative, prior works~\cite{nijkamp2023codegen,roziere2023code,xu2024lemur,patil2023gorilla} have primarily focused on improving task-specific capabilities (\eg, code generation) instead of general-domain LLM agents, due to single-source, uni-type, small-scale, and poor-quality pre-training data. 
Existing tool documentation data for agent training either lacks diverse real-world APIs~\cite{patil2023gorilla, tang2023toolalpaca} or is constrained to single-tool or single-round tool execution. 
Furthermore, trajectory data mostly imitate expert behavior or follow function-calling rules with inferior planning and reasoning, failing to fully elicit LLMs' capabilities and handle complex instructions~\cite{qin2023toolllm}. 
Given a wide range of candidate API functions, each comprising various function names and parameters available at every planning step, identifying globally optimal solutions and generalizing across tasks remains highly challenging.



\section{Preliminaries}
\label{Preliminaries}
\begin{figure*}[t]
    \centering
    \includegraphics[width=0.95\linewidth]{fig/HealthGPT_Framework.png}
    \caption{The \ourmethod{} architecture integrates hierarchical visual perception and H-LoRA, employing a task-specific hard router to select visual features and H-LoRA plugins, ultimately generating outputs with an autoregressive manner.}
    \label{fig:architecture}
\end{figure*}
\noindent\textbf{Large Vision-Language Models.} 
The input to a LVLM typically consists of an image $x^{\text{img}}$ and a discrete text sequence $x^{\text{txt}}$. The visual encoder $\mathcal{E}^{\text{img}}$ converts the input image $x^{\text{img}}$ into a sequence of visual tokens $\mathcal{V} = [v_i]_{i=1}^{N_v}$, while the text sequence $x^{\text{txt}}$ is mapped into a sequence of text tokens $\mathcal{T} = [t_i]_{i=1}^{N_t}$ using an embedding function $\mathcal{E}^{\text{txt}}$. The LLM $\mathcal{M_\text{LLM}}(\cdot|\theta)$ models the joint probability of the token sequence $\mathcal{U} = \{\mathcal{V},\mathcal{T}\}$, which is expressed as:
\begin{equation}
    P_\theta(R | \mathcal{U}) = \prod_{i=1}^{N_r} P_\theta(r_i | \{\mathcal{U}, r_{<i}\}),
\end{equation}
where $R = [r_i]_{i=1}^{N_r}$ is the text response sequence. The LVLM iteratively generates the next token $r_i$ based on $r_{<i}$. The optimization objective is to minimize the cross-entropy loss of the response $\mathcal{R}$.
% \begin{equation}
%     \mathcal{L}_{\text{VLM}} = \mathbb{E}_{R|\mathcal{U}}\left[-\log P_\theta(R | \mathcal{U})\right]
% \end{equation}
It is worth noting that most LVLMs adopt a design paradigm based on ViT, alignment adapters, and pre-trained LLMs\cite{liu2023llava,liu2024improved}, enabling quick adaptation to downstream tasks.


\noindent\textbf{VQGAN.}
VQGAN~\cite{esser2021taming} employs latent space compression and indexing mechanisms to effectively learn a complete discrete representation of images. VQGAN first maps the input image $x^{\text{img}}$ to a latent representation $z = \mathcal{E}(x)$ through a encoder $\mathcal{E}$. Then, the latent representation is quantized using a codebook $\mathcal{Z} = \{z_k\}_{k=1}^K$, generating a discrete index sequence $\mathcal{I} = [i_m]_{m=1}^N$, where $i_m \in \mathcal{Z}$ represents the quantized code index:
\begin{equation}
    \mathcal{I} = \text{Quantize}(z|\mathcal{Z}) = \arg\min_{z_k \in \mathcal{Z}} \| z - z_k \|_2.
\end{equation}
In our approach, the discrete index sequence $\mathcal{I}$ serves as a supervisory signal for the generation task, enabling the model to predict the index sequence $\hat{\mathcal{I}}$ from input conditions such as text or other modality signals.  
Finally, the predicted index sequence $\hat{\mathcal{I}}$ is upsampled by the VQGAN decoder $G$, generating the high-quality image $\hat{x}^\text{img} = G(\hat{\mathcal{I}})$.



\noindent\textbf{Low Rank Adaptation.} 
LoRA\cite{hu2021lora} effectively captures the characteristics of downstream tasks by introducing low-rank adapters. The core idea is to decompose the bypass weight matrix $\Delta W\in\mathbb{R}^{d^{\text{in}} \times d^{\text{out}}}$ into two low-rank matrices $ \{A \in \mathbb{R}^{d^{\text{in}} \times r}, B \in \mathbb{R}^{r \times d^{\text{out}}} \}$, where $ r \ll \min\{d^{\text{in}}, d^{\text{out}}\} $, significantly reducing learnable parameters. The output with the LoRA adapter for the input $x$ is then given by:
\begin{equation}
    h = x W_0 + \alpha x \Delta W/r = x W_0 + \alpha xAB/r,
\end{equation}
where matrix $ A $ is initialized with a Gaussian distribution, while the matrix $ B $ is initialized as a zero matrix. The scaling factor $ \alpha/r $ controls the impact of $ \Delta W $ on the model.

\section{HealthGPT}
\label{Method}


\subsection{Unified Autoregressive Generation.}  
% As shown in Figure~\ref{fig:architecture}, 
\ourmethod{} (Figure~\ref{fig:architecture}) utilizes a discrete token representation that covers both text and visual outputs, unifying visual comprehension and generation as an autoregressive task. 
For comprehension, $\mathcal{M}_\text{llm}$ receives the input joint sequence $\mathcal{U}$ and outputs a series of text token $\mathcal{R} = [r_1, r_2, \dots, r_{N_r}]$, where $r_i \in \mathcal{V}_{\text{txt}}$, and $\mathcal{V}_{\text{txt}}$ represents the LLM's vocabulary:
\begin{equation}
    P_\theta(\mathcal{R} \mid \mathcal{U}) = \prod_{i=1}^{N_r} P_\theta(r_i \mid \mathcal{U}, r_{<i}).
\end{equation}
For generation, $\mathcal{M}_\text{llm}$ first receives a special start token $\langle \text{START\_IMG} \rangle$, then generates a series of tokens corresponding to the VQGAN indices $\mathcal{I} = [i_1, i_2, \dots, i_{N_i}]$, where $i_j \in \mathcal{V}_{\text{vq}}$, and $\mathcal{V}_{\text{vq}}$ represents the index range of VQGAN. Upon completion of generation, the LLM outputs an end token $\langle \text{END\_IMG} \rangle$:
\begin{equation}
    P_\theta(\mathcal{I} \mid \mathcal{U}) = \prod_{j=1}^{N_i} P_\theta(i_j \mid \mathcal{U}, i_{<j}).
\end{equation}
Finally, the generated index sequence $\mathcal{I}$ is fed into the decoder $G$, which reconstructs the target image $\hat{x}^{\text{img}} = G(\mathcal{I})$.

\subsection{Hierarchical Visual Perception}  
Given the differences in visual perception between comprehension and generation tasks—where the former focuses on abstract semantics and the latter emphasizes complete semantics—we employ ViT to compress the image into discrete visual tokens at multiple hierarchical levels.
Specifically, the image is converted into a series of features $\{f_1, f_2, \dots, f_L\}$ as it passes through $L$ ViT blocks.

To address the needs of various tasks, the hidden states are divided into two types: (i) \textit{Concrete-grained features} $\mathcal{F}^{\text{Con}} = \{f_1, f_2, \dots, f_k\}, k < L$, derived from the shallower layers of ViT, containing sufficient global features, suitable for generation tasks; 
(ii) \textit{Abstract-grained features} $\mathcal{F}^{\text{Abs}} = \{f_{k+1}, f_{k+2}, \dots, f_L\}$, derived from the deeper layers of ViT, which contain abstract semantic information closer to the text space, suitable for comprehension tasks.

The task type $T$ (comprehension or generation) determines which set of features is selected as the input for the downstream large language model:
\begin{equation}
    \mathcal{F}^{\text{img}}_T =
    \begin{cases}
        \mathcal{F}^{\text{Con}}, & \text{if } T = \text{generation task} \\
        \mathcal{F}^{\text{Abs}}, & \text{if } T = \text{comprehension task}
    \end{cases}
\end{equation}
We integrate the image features $\mathcal{F}^{\text{img}}_T$ and text features $\mathcal{T}$ into a joint sequence through simple concatenation, which is then fed into the LLM $\mathcal{M}_{\text{llm}}$ for autoregressive generation.
% :
% \begin{equation}
%     \mathcal{R} = \mathcal{M}_{\text{llm}}(\mathcal{U}|\theta), \quad \mathcal{U} = [\mathcal{F}^{\text{img}}_T; \mathcal{T}]
% \end{equation}
\subsection{Heterogeneous Knowledge Adaptation}
We devise H-LoRA, which stores heterogeneous knowledge from comprehension and generation tasks in separate modules and dynamically routes to extract task-relevant knowledge from these modules. 
At the task level, for each task type $ T $, we dynamically assign a dedicated H-LoRA submodule $ \theta^T $, which is expressed as:
\begin{equation}
    \mathcal{R} = \mathcal{M}_\text{LLM}(\mathcal{U}|\theta, \theta^T), \quad \theta^T = \{A^T, B^T, \mathcal{R}^T_\text{outer}\}.
\end{equation}
At the feature level for a single task, H-LoRA integrates the idea of Mixture of Experts (MoE)~\cite{masoudnia2014mixture} and designs an efficient matrix merging and routing weight allocation mechanism, thus avoiding the significant computational delay introduced by matrix splitting in existing MoELoRA~\cite{luo2024moelora}. Specifically, we first merge the low-rank matrices (rank = r) of $ k $ LoRA experts into a unified matrix:
\begin{equation}
    \mathbf{A}^{\text{merged}}, \mathbf{B}^{\text{merged}} = \text{Concat}(\{A_i\}_1^k), \text{Concat}(\{B_i\}_1^k),
\end{equation}
where $ \mathbf{A}^{\text{merged}} \in \mathbb{R}^{d^\text{in} \times rk} $ and $ \mathbf{B}^{\text{merged}} \in \mathbb{R}^{rk \times d^\text{out}} $. The $k$-dimension routing layer generates expert weights $ \mathcal{W} \in \mathbb{R}^{\text{token\_num} \times k} $ based on the input hidden state $ x $, and these are expanded to $ \mathbb{R}^{\text{token\_num} \times rk} $ as follows:
\begin{equation}
    \mathcal{W}^\text{expanded} = \alpha k \mathcal{W} / r \otimes \mathbf{1}_r,
\end{equation}
where $ \otimes $ denotes the replication operation.
The overall output of H-LoRA is computed as:
\begin{equation}
    \mathcal{O}^\text{H-LoRA} = (x \mathbf{A}^{\text{merged}} \odot \mathcal{W}^\text{expanded}) \mathbf{B}^{\text{merged}},
\end{equation}
where $ \odot $ represents element-wise multiplication. Finally, the output of H-LoRA is added to the frozen pre-trained weights to produce the final output:
\begin{equation}
    \mathcal{O} = x W_0 + \mathcal{O}^\text{H-LoRA}.
\end{equation}
% In summary, H-LoRA is a task-based dynamic PEFT method that achieves high efficiency in single-task fine-tuning.

\subsection{Training Pipeline}

\begin{figure}[t]
    \centering
    \hspace{-4mm}
    \includegraphics[width=0.94\linewidth]{fig/data.pdf}
    \caption{Data statistics of \texttt{VL-Health}. }
    \label{fig:data}
\end{figure}
\noindent \textbf{1st Stage: Multi-modal Alignment.} 
In the first stage, we design separate visual adapters and H-LoRA submodules for medical unified tasks. For the medical comprehension task, we train abstract-grained visual adapters using high-quality image-text pairs to align visual embeddings with textual embeddings, thereby enabling the model to accurately describe medical visual content. During this process, the pre-trained LLM and its corresponding H-LoRA submodules remain frozen. In contrast, the medical generation task requires training concrete-grained adapters and H-LoRA submodules while keeping the LLM frozen. Meanwhile, we extend the textual vocabulary to include multimodal tokens, enabling the support of additional VQGAN vector quantization indices. The model trains on image-VQ pairs, endowing the pre-trained LLM with the capability for image reconstruction. This design ensures pixel-level consistency of pre- and post-LVLM. The processes establish the initial alignment between the LLM’s outputs and the visual inputs.

\noindent \textbf{2nd Stage: Heterogeneous H-LoRA Plugin Adaptation.}  
The submodules of H-LoRA share the word embedding layer and output head but may encounter issues such as bias and scale inconsistencies during training across different tasks. To ensure that the multiple H-LoRA plugins seamlessly interface with the LLMs and form a unified base, we fine-tune the word embedding layer and output head using a small amount of mixed data to maintain consistency in the model weights. Specifically, during this stage, all H-LoRA submodules for different tasks are kept frozen, with only the word embedding layer and output head being optimized. Through this stage, the model accumulates foundational knowledge for unified tasks by adapting H-LoRA plugins.

\begin{table*}[!t]
\centering
\caption{Comparison of \ourmethod{} with other LVLMs and unified multi-modal models on medical visual comprehension tasks. \textbf{Bold} and \underline{underlined} text indicates the best performance and second-best performance, respectively.}
\resizebox{\textwidth}{!}{
\begin{tabular}{c|lcc|cccccccc|c}
\toprule
\rowcolor[HTML]{E9F3FE} &  &  &  & \multicolumn{2}{c}{\textbf{VQA-RAD \textuparrow}} & \multicolumn{2}{c}{\textbf{SLAKE \textuparrow}} & \multicolumn{2}{c}{\textbf{PathVQA \textuparrow}} &  &  &  \\ 
\cline{5-10}
\rowcolor[HTML]{E9F3FE}\multirow{-2}{*}{\textbf{Type}} & \multirow{-2}{*}{\textbf{Model}} & \multirow{-2}{*}{\textbf{\# Params}} & \multirow{-2}{*}{\makecell{\textbf{Medical} \\ \textbf{LVLM}}} & \textbf{close} & \textbf{all} & \textbf{close} & \textbf{all} & \textbf{close} & \textbf{all} & \multirow{-2}{*}{\makecell{\textbf{MMMU} \\ \textbf{-Med}}\textuparrow} & \multirow{-2}{*}{\textbf{OMVQA}\textuparrow} & \multirow{-2}{*}{\textbf{Avg. \textuparrow}} \\ 
\midrule \midrule
\multirow{9}{*}{\textbf{Comp. Only}} 
& Med-Flamingo & 8.3B & \Large \ding{51} & 58.6 & 43.0 & 47.0 & 25.5 & 61.9 & 31.3 & 28.7 & 34.9 & 41.4 \\
& LLaVA-Med & 7B & \Large \ding{51} & 60.2 & 48.1 & 58.4 & 44.8 & 62.3 & 35.7 & 30.0 & 41.3 & 47.6 \\
& HuatuoGPT-Vision & 7B & \Large \ding{51} & 66.9 & 53.0 & 59.8 & 49.1 & 52.9 & 32.0 & 42.0 & 50.0 & 50.7 \\
& BLIP-2 & 6.7B & \Large \ding{55} & 43.4 & 36.8 & 41.6 & 35.3 & 48.5 & 28.8 & 27.3 & 26.9 & 36.1 \\
& LLaVA-v1.5 & 7B & \Large \ding{55} & 51.8 & 42.8 & 37.1 & 37.7 & 53.5 & 31.4 & 32.7 & 44.7 & 41.5 \\
& InstructBLIP & 7B & \Large \ding{55} & 61.0 & 44.8 & 66.8 & 43.3 & 56.0 & 32.3 & 25.3 & 29.0 & 44.8 \\
& Yi-VL & 6B & \Large \ding{55} & 52.6 & 42.1 & 52.4 & 38.4 & 54.9 & 30.9 & 38.0 & 50.2 & 44.9 \\
& InternVL2 & 8B & \Large \ding{55} & 64.9 & 49.0 & 66.6 & 50.1 & 60.0 & 31.9 & \underline{43.3} & 54.5 & 52.5\\
& Llama-3.2 & 11B & \Large \ding{55} & 68.9 & 45.5 & 72.4 & 52.1 & 62.8 & 33.6 & 39.3 & 63.2 & 54.7 \\
\midrule
\multirow{5}{*}{\textbf{Comp. \& Gen.}} 
& Show-o & 1.3B & \Large \ding{55} & 50.6 & 33.9 & 31.5 & 17.9 & 52.9 & 28.2 & 22.7 & 45.7 & 42.6 \\
& Unified-IO 2 & 7B & \Large \ding{55} & 46.2 & 32.6 & 35.9 & 21.9 & 52.5 & 27.0 & 25.3 & 33.0 & 33.8 \\
& Janus & 1.3B & \Large \ding{55} & 70.9 & 52.8 & 34.7 & 26.9 & 51.9 & 27.9 & 30.0 & 26.8 & 33.5 \\
& \cellcolor[HTML]{DAE0FB}HealthGPT-M3 & \cellcolor[HTML]{DAE0FB}3.8B & \cellcolor[HTML]{DAE0FB}\Large \ding{51} & \cellcolor[HTML]{DAE0FB}\underline{73.7} & \cellcolor[HTML]{DAE0FB}\underline{55.9} & \cellcolor[HTML]{DAE0FB}\underline{74.6} & \cellcolor[HTML]{DAE0FB}\underline{56.4} & \cellcolor[HTML]{DAE0FB}\underline{78.7} & \cellcolor[HTML]{DAE0FB}\underline{39.7} & \cellcolor[HTML]{DAE0FB}\underline{43.3} & \cellcolor[HTML]{DAE0FB}\underline{68.5} & \cellcolor[HTML]{DAE0FB}\underline{61.3} \\
& \cellcolor[HTML]{DAE0FB}HealthGPT-L14 & \cellcolor[HTML]{DAE0FB}14B & \cellcolor[HTML]{DAE0FB}\Large \ding{51} & \cellcolor[HTML]{DAE0FB}\textbf{77.7} & \cellcolor[HTML]{DAE0FB}\textbf{58.3} & \cellcolor[HTML]{DAE0FB}\textbf{76.4} & \cellcolor[HTML]{DAE0FB}\textbf{64.5} & \cellcolor[HTML]{DAE0FB}\textbf{85.9} & \cellcolor[HTML]{DAE0FB}\textbf{44.4} & \cellcolor[HTML]{DAE0FB}\textbf{49.2} & \cellcolor[HTML]{DAE0FB}\textbf{74.4} & \cellcolor[HTML]{DAE0FB}\textbf{66.4} \\
\bottomrule
\end{tabular}
}
\label{tab:results}
\end{table*}
\begin{table*}[ht]
    \centering
    \caption{The experimental results for the four modality conversion tasks.}
    \resizebox{\textwidth}{!}{
    \begin{tabular}{l|ccc|ccc|ccc|ccc}
        \toprule
        \rowcolor[HTML]{E9F3FE} & \multicolumn{3}{c}{\textbf{CT to MRI (Brain)}} & \multicolumn{3}{c}{\textbf{CT to MRI (Pelvis)}} & \multicolumn{3}{c}{\textbf{MRI to CT (Brain)}} & \multicolumn{3}{c}{\textbf{MRI to CT (Pelvis)}} \\
        \cline{2-13}
        \rowcolor[HTML]{E9F3FE}\multirow{-2}{*}{\textbf{Model}}& \textbf{SSIM $\uparrow$} & \textbf{PSNR $\uparrow$} & \textbf{MSE $\downarrow$} & \textbf{SSIM $\uparrow$} & \textbf{PSNR $\uparrow$} & \textbf{MSE $\downarrow$} & \textbf{SSIM $\uparrow$} & \textbf{PSNR $\uparrow$} & \textbf{MSE $\downarrow$} & \textbf{SSIM $\uparrow$} & \textbf{PSNR $\uparrow$} & \textbf{MSE $\downarrow$} \\
        \midrule \midrule
        pix2pix & 71.09 & 32.65 & 36.85 & 59.17 & 31.02 & 51.91 & 78.79 & 33.85 & 28.33 & 72.31 & 32.98 & 36.19 \\
        CycleGAN & 54.76 & 32.23 & 40.56 & 54.54 & 30.77 & 55.00 & 63.75 & 31.02 & 52.78 & 50.54 & 29.89 & 67.78 \\
        BBDM & {71.69} & {32.91} & {34.44} & 57.37 & 31.37 & 48.06 & \textbf{86.40} & 34.12 & 26.61 & {79.26} & 33.15 & 33.60 \\
        Vmanba & 69.54 & 32.67 & 36.42 & {63.01} & {31.47} & {46.99} & 79.63 & 34.12 & 26.49 & 77.45 & 33.53 & 31.85 \\
        DiffMa & 71.47 & 32.74 & 35.77 & 62.56 & 31.43 & 47.38 & 79.00 & {34.13} & {26.45} & 78.53 & {33.68} & {30.51} \\
        \rowcolor[HTML]{DAE0FB}HealthGPT-M3 & \underline{79.38} & \underline{33.03} & \underline{33.48} & \underline{71.81} & \underline{31.83} & \underline{43.45} & {85.06} & \textbf{34.40} & \textbf{25.49} & \underline{84.23} & \textbf{34.29} & \textbf{27.99} \\
        \rowcolor[HTML]{DAE0FB}HealthGPT-L14 & \textbf{79.73} & \textbf{33.10} & \textbf{32.96} & \textbf{71.92} & \textbf{31.87} & \textbf{43.09} & \underline{85.31} & \underline{34.29} & \underline{26.20} & \textbf{84.96} & \underline{34.14} & \underline{28.13} \\
        \bottomrule
    \end{tabular}
    }
    \label{tab:conversion}
\end{table*}

\noindent \textbf{3rd Stage: Visual Instruction Fine-Tuning.}  
In the third stage, we introduce additional task-specific data to further optimize the model and enhance its adaptability to downstream tasks such as medical visual comprehension (e.g., medical QA, medical dialogues, and report generation) or generation tasks (e.g., super-resolution, denoising, and modality conversion). Notably, by this stage, the word embedding layer and output head have been fine-tuned, only the H-LoRA modules and adapter modules need to be trained. This strategy significantly improves the model's adaptability and flexibility across different tasks.


\section{Experiment}
\label{s:experiment}

\subsection{Data Description}
We evaluate our method on FI~\cite{you2016building}, Twitter\_LDL~\cite{yang2017learning} and Artphoto~\cite{machajdik2010affective}.
FI is a public dataset built from Flickr and Instagram, with 23,308 images and eight emotion categories, namely \textit{amusement}, \textit{anger}, \textit{awe},  \textit{contentment}, \textit{disgust}, \textit{excitement},  \textit{fear}, and \textit{sadness}. 
% Since images in FI are all copyrighted by law, some images are corrupted now, so we remove these samples and retain 21,828 images.
% T4SA contains images from Twitter, which are classified into three categories: \textit{positive}, \textit{neutral}, and \textit{negative}. In this paper, we adopt the base version of B-T4SA, which contains 470,586 images and provides text descriptions of the corresponding tweets.
Twitter\_LDL contains 10,045 images from Twitter, with the same eight categories as the FI dataset.
% 。
For these two datasets, they are randomly split into 80\%
training and 20\% testing set.
Artphoto contains 806 artistic photos from the DeviantArt website, which we use to further evaluate the zero-shot capability of our model.
% on the small-scale dataset.
% We construct and publicly release the first image sentiment analysis dataset containing metadata.
% 。

% Based on these datasets, we are the first to construct and publicly release metadata-enhanced image sentiment analysis datasets. These datasets include scenes, tags, descriptions, and corresponding confidence scores, and are available at this link for future research purposes.


% 
\begin{table}[t]
\centering
% \begin{center}
\caption{Overall performance of different models on FI and Twitter\_LDL datasets.}
\label{tab:cap1}
% \resizebox{\linewidth}{!}
{
\begin{tabular}{l|c|c|c|c}
\hline
\multirow{2}{*}{\textbf{Model}} & \multicolumn{2}{c|}{\textbf{FI}}  & \multicolumn{2}{c}{\textbf{Twitter\_LDL}} \\ \cline{2-5} 
  & \textbf{Accuracy} & \textbf{F1} & \textbf{Accuracy} & \textbf{F1}  \\ \hline
% (\rownumber)~AlexNet~\cite{krizhevsky2017imagenet}  & 58.13\% & 56.35\%  & 56.24\%& 55.02\%  \\ 
% (\rownumber)~VGG16~\cite{simonyan2014very}  & 63.75\%& 63.08\%  & 59.34\%& 59.02\%  \\ 
(\rownumber)~ResNet101~\cite{he2016deep} & 66.16\%& 65.56\%  & 62.02\% & 61.34\%  \\ 
(\rownumber)~CDA~\cite{han2023boosting} & 66.71\%& 65.37\%  & 64.14\% & 62.85\%  \\ 
(\rownumber)~CECCN~\cite{ruan2024color} & 67.96\%& 66.74\%  & 64.59\%& 64.72\% \\ 
(\rownumber)~EmoVIT~\cite{xie2024emovit} & 68.09\%& 67.45\%  & 63.12\% & 61.97\%  \\ 
(\rownumber)~ComLDL~\cite{zhang2022compound} & 68.83\%& 67.28\%  & 65.29\% & 63.12\%  \\ 
(\rownumber)~WSDEN~\cite{li2023weakly} & 69.78\%& 69.61\%  & 67.04\% & 65.49\% \\ 
(\rownumber)~ECWA~\cite{deng2021emotion} & 70.87\%& 69.08\%  & 67.81\% & 66.87\%  \\ 
(\rownumber)~EECon~\cite{yang2023exploiting} & 71.13\%& 68.34\%  & 64.27\%& 63.16\%  \\ 
(\rownumber)~MAM~\cite{zhang2024affective} & 71.44\%  & 70.83\% & 67.18\%  & 65.01\%\\ 
(\rownumber)~TGCA-PVT~\cite{chen2024tgca}   & 73.05\%  & 71.46\% & 69.87\%  & 68.32\% \\ 
(\rownumber)~OEAN~\cite{zhang2024object}   & 73.40\%  & 72.63\% & 70.52\%  & 69.47\% \\ \hline
(\rownumber)~\shortname  & \textbf{79.48\%} & \textbf{79.22\%} & \textbf{74.12\%} & \textbf{73.09\%} \\ \hline
\end{tabular}
}
\vspace{-6mm}
% \end{center}
\end{table}
% 

\subsection{Experiment Setting}
% \subsubsection{Model Setting.}
% 
\textbf{Model Setting:}
For feature representation, we set $k=10$ to select object tags, and adopt clip-vit-base-patch32 as the pre-trained model for unified feature representation.
Moreover, we empirically set $(d_e, d_h, d_k, d_s) = (512, 128, 16, 64)$, and set the classification class $L$ to 8.

% 

\textbf{Training Setting:}
To initialize the model, we set all weights such as $\boldsymbol{W}$ following the truncated normal distribution, and use AdamW optimizer with the learning rate of $1 \times 10^{-4}$.
% warmup scheduler of cosine, warmup steps of 2000.
Furthermore, we set the batch size to 32 and the epoch of the training process to 200.
During the implementation, we utilize \textit{PyTorch} to build our entire model.
% , and our project codes are publicly available at https://github.com/zzmyrep/MESN.
% Our project codes as well as data are all publicly available on GitHub\footnote{https://github.com/zzmyrep/KBCEN}.
% Code is available at \href{https://github.com/zzmyrep/KBCEN}{https://github.com/zzmyrep/KBCEN}.

\textbf{Evaluation Metrics:}
Following~\cite{zhang2024affective, chen2024tgca, zhang2024object}, we adopt \textit{accuracy} and \textit{F1} as our evaluation metrics to measure the performance of different methods for image sentiment analysis. 



\subsection{Experiment Result}
% We compare our model against the following baselines: AlexNet~\cite{krizhevsky2017imagenet}, VGG16~\cite{simonyan2014very}, ResNet101~\cite{he2016deep}, CECCN~\cite{ruan2024color}, EmoVIT~\cite{xie2024emovit}, WSCNet~\cite{yang2018weakly}, ECWA~\cite{deng2021emotion}, EECon~\cite{yang2023exploiting}, MAM~\cite{zhang2024affective} and TGCA-PVT~\cite{chen2024tgca}, and the overall results are summarized in Table~\ref{tab:cap1}.
We compare our model against several baselines, and the overall results are summarized in Table~\ref{tab:cap1}.
We observe that our model achieves the best performance in both accuracy and F1 metrics, significantly outperforming the previous models. 
This superior performance is mainly attributed to our effective utilization of metadata to enhance image sentiment analysis, as well as the exceptional capability of the unified sentiment transformer framework we developed. These results strongly demonstrate that our proposed method can bring encouraging performance for image sentiment analysis.

\setcounter{magicrownumbers}{0} 
\begin{table}[t]
\begin{center}
\caption{Ablation study of~\shortname~on FI dataset.} 
% \vspace{1mm}
\label{tab:cap2}
\resizebox{.9\linewidth}{!}
{
\begin{tabular}{lcc}
  \hline
  \textbf{Model} & \textbf{Accuracy} & \textbf{F1} \\
  \hline
  (\rownumber)~Ours (w/o vision) & 65.72\% & 64.54\% \\
  (\rownumber)~Ours (w/o text description) & 74.05\% & 72.58\% \\
  (\rownumber)~Ours (w/o object tag) & 77.45\% & 76.84\% \\
  (\rownumber)~Ours (w/o scene tag) & 78.47\% & 78.21\% \\
  \hline
  (\rownumber)~Ours (w/o unified embedding) & 76.41\% & 76.23\% \\
  (\rownumber)~Ours (w/o adaptive learning) & 76.83\% & 76.56\% \\
  (\rownumber)~Ours (w/o cross-modal fusion) & 76.85\% & 76.49\% \\
  \hline
  (\rownumber)~Ours  & \textbf{79.48\%} & \textbf{79.22\%} \\
  \hline
\end{tabular}
}
\end{center}
\vspace{-5mm}
\end{table}


\begin{figure}[t]
\centering
% \vspace{-2mm}
\includegraphics[width=0.42\textwidth]{fig/2dvisual-linux4-paper2.pdf}
\caption{Visualization of feature distribution on eight categories before (left) and after (right) model processing.}
% 
\label{fig:visualization}
\vspace{-5mm}
\end{figure}

\subsection{Ablation Performance}
In this subsection, we conduct an ablation study to examine which component is really important for performance improvement. The results are reported in Table~\ref{tab:cap2}.

For information utilization, we observe a significant decline in model performance when visual features are removed. Additionally, the performance of \shortname~decreases when different metadata are removed separately, which means that text description, object tag, and scene tag are all critical for image sentiment analysis.
Recalling the model architecture, we separately remove transformer layers of the unified representation module, the adaptive learning module, and the cross-modal fusion module, replacing them with MLPs of the same parameter scale.
In this way, we can observe varying degrees of decline in model performance, indicating that these modules are indispensable for our model to achieve better performance.

\subsection{Visualization}
% 


% % 开始使用minipage进行左右排列
% \begin{minipage}[t]{0.45\textwidth}  % 子图1宽度为45%
%     \centering
%     \includegraphics[width=\textwidth]{2dvisual.pdf}  % 插入图片
%     \captionof{figure}{Visualization of feature distribution.}  % 使用captionof添加图片标题
%     \label{fig:visualization}
% \end{minipage}


% \begin{figure}[t]
% \centering
% \vspace{-2mm}
% \includegraphics[width=0.45\textwidth]{fig/2dvisual.pdf}
% \caption{Visualization of feature distribution.}
% \label{fig:visualization}
% % \vspace{-4mm}
% \end{figure}

% \begin{figure}[t]
% \centering
% \vspace{-2mm}
% \includegraphics[width=0.45\textwidth]{fig/2dvisual-linux3-paper.pdf}
% \caption{Visualization of feature distribution.}
% \label{fig:visualization}
% % \vspace{-4mm}
% \end{figure}



\begin{figure}[tbp]   
\vspace{-4mm}
  \centering            
  \subfloat[Depth of adaptive learning layers]   
  {
    \label{fig:subfig1}\includegraphics[width=0.22\textwidth]{fig/fig_sensitivity-a5}
  }
  \subfloat[Depth of fusion layers]
  {
    % \label{fig:subfig2}\includegraphics[width=0.22\textwidth]{fig/fig_sensitivity-b2}
    \label{fig:subfig2}\includegraphics[width=0.22\textwidth]{fig/fig_sensitivity-b2-num.pdf}
  }
  \caption{Sensitivity study of \shortname~on different depth. }   
  \label{fig:fig_sensitivity}  
\vspace{-2mm}
\end{figure}

% \begin{figure}[htbp]
% \centerline{\includegraphics{2dvisual.pdf}}
% \caption{Visualization of feature distribution.}
% \label{fig:visualization}
% \end{figure}

% In Fig.~\ref{fig:visualization}, we use t-SNE~\cite{van2008visualizing} to reduce the dimension of data features for visualization, Figure in left represents the metadata features before model processing, the features are obtained by embedding through the CLIP model, and figure in right shows the features of the data after model processing, it can be observed that after the model processing, the data with different label categories fall in different regions in the space, therefore, we can conclude that the Therefore, we can conclude that the model can effectively utilize the information contained in the metadata and use it to guide the model for classification.

In Fig.~\ref{fig:visualization}, we use t-SNE~\cite{van2008visualizing} to reduce the dimension of data features for visualization.
The left figure shows metadata features before being processed by our model (\textit{i.e.}, embedded by CLIP), while the right shows the distribution of features after being processed by our model.
We can observe that after the model processing, data with the same label are closer to each other, while others are farther away.
Therefore, it shows that the model can effectively utilize the information contained in the metadata and use it to guide the classification process.

\subsection{Sensitivity Analysis}
% 
In this subsection, we conduct a sensitivity analysis to figure out the effect of different depth settings of adaptive learning layers and fusion layers. 
% In this subsection, we conduct a sensitivity analysis to figure out the effect of different depth settings on the model. 
% Fig.~\ref{fig:fig_sensitivity} presents the effect of different depth settings of adaptive learning layers and fusion layers. 
Taking Fig.~\ref{fig:fig_sensitivity} (a) as an example, the model performance improves with increasing depth, reaching the best performance at a depth of 4.
% Taking Fig.~\ref{fig:fig_sensitivity} (a) as an example, the performance of \shortname~improves with the increase of depth at first, reaching the best performance at a depth of 4.
When the depth continues to increase, the accuracy decreases to varying degrees.
Similar results can be observed in Fig.~\ref{fig:fig_sensitivity} (b).
Therefore, we set their depths to 4 and 6 respectively to achieve the best results.

% Through our experiments, we can observe that the effect of modifying these hyperparameters on the results of the experiments is very weak, and the surface model is not sensitive to the hyperparameters.


\subsection{Zero-shot Capability}
% 

% (1)~GCH~\cite{2010Analyzing} & 21.78\% & (5)~RA-DLNet~\cite{2020A} & 34.01\% \\ \hline
% (2)~WSCNet~\cite{2019WSCNet}  & 30.25\% & (6)~CECCN~\cite{ruan2024color} & 43.83\% \\ \hline
% (3)~PCNN~\cite{2015Robust} & 31.68\%  & (7)~EmoVIT~\cite{xie2024emovit} & 44.90\% \\ \hline
% (4)~AR~\cite{2018Visual} & 32.67\% & (8)~Ours (Zero-shot) & 47.83\% \\ \hline


\begin{table}[t]
\centering
\caption{Zero-shot capability of \shortname.}
\label{tab:cap3}
\resizebox{1\linewidth}{!}
{
\begin{tabular}{lc|lc}
\hline
\textbf{Model} & \textbf{Accuracy} & \textbf{Model} & \textbf{Accuracy} \\ \hline
(1)~WSCNet~\cite{2019WSCNet}  & 30.25\% & (5)~MAM~\cite{zhang2024affective} & 39.56\%  \\ \hline
(2)~AR~\cite{2018Visual} & 32.67\% & (6)~CECCN~\cite{ruan2024color} & 43.83\% \\ \hline
(3)~RA-DLNet~\cite{2020A} & 34.01\%  & (7)~EmoVIT~\cite{xie2024emovit} & 44.90\% \\ \hline
(4)~CDA~\cite{han2023boosting} & 38.64\% & (8)~Ours (Zero-shot) & 47.83\% \\ \hline
\end{tabular}
}
\vspace{-5mm}
\end{table}

% We use the model trained on the FI dataset to test on the artphoto dataset to verify the model's generalization ability as well as robustness to other distributed datasets.
% We can observe that the MESN model shows strong competitiveness in terms of accuracy when compared to other trained models, which suggests that the model has a good generalization ability in the OOD task.

To validate the model's generalization ability and robustness to other distributed datasets, we directly test the model trained on the FI dataset, without training on Artphoto. 
% As observed in Table 3, compared to other models trained on Artphoto, we achieve highly competitive zero-shot performance, indicating that the model has good generalization ability in out-of-distribution tasks.
From Table~\ref{tab:cap3}, we can observe that compared with other models trained on Artphoto, we achieve competitive zero-shot performance, which shows that the model has good generalization ability in out-of-distribution tasks.


%%%%%%%%%%%%
%  E2E     %
%%%%%%%%%%%%


\section{Conclusion}
In this paper, we introduced Wi-Chat, the first LLM-powered Wi-Fi-based human activity recognition system that integrates the reasoning capabilities of large language models with the sensing potential of wireless signals. Our experimental results on a self-collected Wi-Fi CSI dataset demonstrate the promising potential of LLMs in enabling zero-shot Wi-Fi sensing. These findings suggest a new paradigm for human activity recognition that does not rely on extensive labeled data. We hope future research will build upon this direction, further exploring the applications of LLMs in signal processing domains such as IoT, mobile sensing, and radar-based systems.

\section*{Limitations}
While our work represents the first attempt to leverage LLMs for processing Wi-Fi signals, it is a preliminary study focused on a relatively simple task: Wi-Fi-based human activity recognition. This choice allows us to explore the feasibility of LLMs in wireless sensing but also comes with certain limitations.

Our approach primarily evaluates zero-shot performance, which, while promising, may still lag behind traditional supervised learning methods in highly complex or fine-grained recognition tasks. Besides, our study is limited to a controlled environment with a self-collected dataset, and the generalizability of LLMs to diverse real-world scenarios with varying Wi-Fi conditions, environmental interference, and device heterogeneity remains an open question.

Additionally, we have yet to explore the full potential of LLMs in more advanced Wi-Fi sensing applications, such as fine-grained gesture recognition, occupancy detection, and passive health monitoring. Future work should investigate the scalability of LLM-based approaches, their robustness to domain shifts, and their integration with multimodal sensing techniques in broader IoT applications.


% Bibliography entries for the entire Anthology, followed by custom entries
%\bibliography{anthology,custom}
% Custom bibliography entries only
\bibliography{main}
\newpage
\appendix

\section{Experiment prompts}
\label{sec:prompt}
The prompts used in the LLM experiments are shown in the following Table~\ref{tab:prompts}.

\definecolor{titlecolor}{rgb}{0.9, 0.5, 0.1}
\definecolor{anscolor}{rgb}{0.2, 0.5, 0.8}
\definecolor{labelcolor}{HTML}{48a07e}
\begin{table*}[h]
	\centering
	
 % \vspace{-0.2cm}
	
	\begin{center}
		\begin{tikzpicture}[
				chatbox_inner/.style={rectangle, rounded corners, opacity=0, text opacity=1, font=\sffamily\scriptsize, text width=5in, text height=9pt, inner xsep=6pt, inner ysep=6pt},
				chatbox_prompt_inner/.style={chatbox_inner, align=flush left, xshift=0pt, text height=11pt},
				chatbox_user_inner/.style={chatbox_inner, align=flush left, xshift=0pt},
				chatbox_gpt_inner/.style={chatbox_inner, align=flush left, xshift=0pt},
				chatbox/.style={chatbox_inner, draw=black!25, fill=gray!7, opacity=1, text opacity=0},
				chatbox_prompt/.style={chatbox, align=flush left, fill=gray!1.5, draw=black!30, text height=10pt},
				chatbox_user/.style={chatbox, align=flush left},
				chatbox_gpt/.style={chatbox, align=flush left},
				chatbox2/.style={chatbox_gpt, fill=green!25},
				chatbox3/.style={chatbox_gpt, fill=red!20, draw=black!20},
				chatbox4/.style={chatbox_gpt, fill=yellow!30},
				labelbox/.style={rectangle, rounded corners, draw=black!50, font=\sffamily\scriptsize\bfseries, fill=gray!5, inner sep=3pt},
			]
											
			\node[chatbox_user] (q1) {
				\textbf{System prompt}
				\newline
				\newline
				You are a helpful and precise assistant for segmenting and labeling sentences. We would like to request your help on curating a dataset for entity-level hallucination detection.
				\newline \newline
                We will give you a machine generated biography and a list of checked facts about the biography. Each fact consists of a sentence and a label (True/False). Please do the following process. First, breaking down the biography into words. Second, by referring to the provided list of facts, merging some broken down words in the previous step to form meaningful entities. For example, ``strategic thinking'' should be one entity instead of two. Third, according to the labels in the list of facts, labeling each entity as True or False. Specifically, for facts that share a similar sentence structure (\eg, \textit{``He was born on Mach 9, 1941.''} (\texttt{True}) and \textit{``He was born in Ramos Mejia.''} (\texttt{False})), please first assign labels to entities that differ across atomic facts. For example, first labeling ``Mach 9, 1941'' (\texttt{True}) and ``Ramos Mejia'' (\texttt{False}) in the above case. For those entities that are the same across atomic facts (\eg, ``was born'') or are neutral (\eg, ``he,'' ``in,'' and ``on''), please label them as \texttt{True}. For the cases that there is no atomic fact that shares the same sentence structure, please identify the most informative entities in the sentence and label them with the same label as the atomic fact while treating the rest of the entities as \texttt{True}. In the end, output the entities and labels in the following format:
                \begin{itemize}[nosep]
                    \item Entity 1 (Label 1)
                    \item Entity 2 (Label 2)
                    \item ...
                    \item Entity N (Label N)
                \end{itemize}
                % \newline \newline
                Here are two examples:
                \newline\newline
                \textbf{[Example 1]}
                \newline
                [The start of the biography]
                \newline
                \textcolor{titlecolor}{Marianne McAndrew is an American actress and singer, born on November 21, 1942, in Cleveland, Ohio. She began her acting career in the late 1960s, appearing in various television shows and films.}
                \newline
                [The end of the biography]
                \newline \newline
                [The start of the list of checked facts]
                \newline
                \textcolor{anscolor}{[Marianne McAndrew is an American. (False); Marianne McAndrew is an actress. (True); Marianne McAndrew is a singer. (False); Marianne McAndrew was born on November 21, 1942. (False); Marianne McAndrew was born in Cleveland, Ohio. (False); She began her acting career in the late 1960s. (True); She has appeared in various television shows. (True); She has appeared in various films. (True)]}
                \newline
                [The end of the list of checked facts]
                \newline \newline
                [The start of the ideal output]
                \newline
                \textcolor{labelcolor}{[Marianne McAndrew (True); is (True); an (True); American (False); actress (True); and (True); singer (False); , (True); born (True); on (True); November 21, 1942 (False); , (True); in (True); Cleveland, Ohio (False); . (True); She (True); began (True); her (True); acting career (True); in (True); the late 1960s (True); , (True); appearing (True); in (True); various (True); television shows (True); and (True); films (True); . (True)]}
                \newline
                [The end of the ideal output]
				\newline \newline
                \textbf{[Example 2]}
                \newline
                [The start of the biography]
                \newline
                \textcolor{titlecolor}{Doug Sheehan is an American actor who was born on April 27, 1949, in Santa Monica, California. He is best known for his roles in soap operas, including his portrayal of Joe Kelly on ``General Hospital'' and Ben Gibson on ``Knots Landing.''}
                \newline
                [The end of the biography]
                \newline \newline
                [The start of the list of checked facts]
                \newline
                \textcolor{anscolor}{[Doug Sheehan is an American. (True); Doug Sheehan is an actor. (True); Doug Sheehan was born on April 27, 1949. (True); Doug Sheehan was born in Santa Monica, California. (False); He is best known for his roles in soap operas. (True); He portrayed Joe Kelly. (True); Joe Kelly was in General Hospital. (True); General Hospital is a soap opera. (True); He portrayed Ben Gibson. (True); Ben Gibson was in Knots Landing. (True); Knots Landing is a soap opera. (True)]}
                \newline
                [The end of the list of checked facts]
                \newline \newline
                [The start of the ideal output]
                \newline
                \textcolor{labelcolor}{[Doug Sheehan (True); is (True); an (True); American (True); actor (True); who (True); was born (True); on (True); April 27, 1949 (True); in (True); Santa Monica, California (False); . (True); He (True); is (True); best known (True); for (True); his roles in soap operas (True); , (True); including (True); in (True); his portrayal (True); of (True); Joe Kelly (True); on (True); ``General Hospital'' (True); and (True); Ben Gibson (True); on (True); ``Knots Landing.'' (True)]}
                \newline
                [The end of the ideal output]
				\newline \newline
				\textbf{User prompt}
				\newline
				\newline
				[The start of the biography]
				\newline
				\textcolor{magenta}{\texttt{\{BIOGRAPHY\}}}
				\newline
				[The ebd of the biography]
				\newline \newline
				[The start of the list of checked facts]
				\newline
				\textcolor{magenta}{\texttt{\{LIST OF CHECKED FACTS\}}}
				\newline
				[The end of the list of checked facts]
			};
			\node[chatbox_user_inner] (q1_text) at (q1) {
				\textbf{System prompt}
				\newline
				\newline
				You are a helpful and precise assistant for segmenting and labeling sentences. We would like to request your help on curating a dataset for entity-level hallucination detection.
				\newline \newline
                We will give you a machine generated biography and a list of checked facts about the biography. Each fact consists of a sentence and a label (True/False). Please do the following process. First, breaking down the biography into words. Second, by referring to the provided list of facts, merging some broken down words in the previous step to form meaningful entities. For example, ``strategic thinking'' should be one entity instead of two. Third, according to the labels in the list of facts, labeling each entity as True or False. Specifically, for facts that share a similar sentence structure (\eg, \textit{``He was born on Mach 9, 1941.''} (\texttt{True}) and \textit{``He was born in Ramos Mejia.''} (\texttt{False})), please first assign labels to entities that differ across atomic facts. For example, first labeling ``Mach 9, 1941'' (\texttt{True}) and ``Ramos Mejia'' (\texttt{False}) in the above case. For those entities that are the same across atomic facts (\eg, ``was born'') or are neutral (\eg, ``he,'' ``in,'' and ``on''), please label them as \texttt{True}. For the cases that there is no atomic fact that shares the same sentence structure, please identify the most informative entities in the sentence and label them with the same label as the atomic fact while treating the rest of the entities as \texttt{True}. In the end, output the entities and labels in the following format:
                \begin{itemize}[nosep]
                    \item Entity 1 (Label 1)
                    \item Entity 2 (Label 2)
                    \item ...
                    \item Entity N (Label N)
                \end{itemize}
                % \newline \newline
                Here are two examples:
                \newline\newline
                \textbf{[Example 1]}
                \newline
                [The start of the biography]
                \newline
                \textcolor{titlecolor}{Marianne McAndrew is an American actress and singer, born on November 21, 1942, in Cleveland, Ohio. She began her acting career in the late 1960s, appearing in various television shows and films.}
                \newline
                [The end of the biography]
                \newline \newline
                [The start of the list of checked facts]
                \newline
                \textcolor{anscolor}{[Marianne McAndrew is an American. (False); Marianne McAndrew is an actress. (True); Marianne McAndrew is a singer. (False); Marianne McAndrew was born on November 21, 1942. (False); Marianne McAndrew was born in Cleveland, Ohio. (False); She began her acting career in the late 1960s. (True); She has appeared in various television shows. (True); She has appeared in various films. (True)]}
                \newline
                [The end of the list of checked facts]
                \newline \newline
                [The start of the ideal output]
                \newline
                \textcolor{labelcolor}{[Marianne McAndrew (True); is (True); an (True); American (False); actress (True); and (True); singer (False); , (True); born (True); on (True); November 21, 1942 (False); , (True); in (True); Cleveland, Ohio (False); . (True); She (True); began (True); her (True); acting career (True); in (True); the late 1960s (True); , (True); appearing (True); in (True); various (True); television shows (True); and (True); films (True); . (True)]}
                \newline
                [The end of the ideal output]
				\newline \newline
                \textbf{[Example 2]}
                \newline
                [The start of the biography]
                \newline
                \textcolor{titlecolor}{Doug Sheehan is an American actor who was born on April 27, 1949, in Santa Monica, California. He is best known for his roles in soap operas, including his portrayal of Joe Kelly on ``General Hospital'' and Ben Gibson on ``Knots Landing.''}
                \newline
                [The end of the biography]
                \newline \newline
                [The start of the list of checked facts]
                \newline
                \textcolor{anscolor}{[Doug Sheehan is an American. (True); Doug Sheehan is an actor. (True); Doug Sheehan was born on April 27, 1949. (True); Doug Sheehan was born in Santa Monica, California. (False); He is best known for his roles in soap operas. (True); He portrayed Joe Kelly. (True); Joe Kelly was in General Hospital. (True); General Hospital is a soap opera. (True); He portrayed Ben Gibson. (True); Ben Gibson was in Knots Landing. (True); Knots Landing is a soap opera. (True)]}
                \newline
                [The end of the list of checked facts]
                \newline \newline
                [The start of the ideal output]
                \newline
                \textcolor{labelcolor}{[Doug Sheehan (True); is (True); an (True); American (True); actor (True); who (True); was born (True); on (True); April 27, 1949 (True); in (True); Santa Monica, California (False); . (True); He (True); is (True); best known (True); for (True); his roles in soap operas (True); , (True); including (True); in (True); his portrayal (True); of (True); Joe Kelly (True); on (True); ``General Hospital'' (True); and (True); Ben Gibson (True); on (True); ``Knots Landing.'' (True)]}
                \newline
                [The end of the ideal output]
				\newline \newline
				\textbf{User prompt}
				\newline
				\newline
				[The start of the biography]
				\newline
				\textcolor{magenta}{\texttt{\{BIOGRAPHY\}}}
				\newline
				[The ebd of the biography]
				\newline \newline
				[The start of the list of checked facts]
				\newline
				\textcolor{magenta}{\texttt{\{LIST OF CHECKED FACTS\}}}
				\newline
				[The end of the list of checked facts]
			};
		\end{tikzpicture}
        \caption{GPT-4o prompt for labeling hallucinated entities.}\label{tb:gpt-4-prompt}
	\end{center}
\vspace{-0cm}
\end{table*}
% \section{Full Experiment Results}
% \begin{table*}[th]
    \centering
    \small
    \caption{Classification Results}
    \begin{tabular}{lcccc}
        \toprule
        \textbf{Method} & \textbf{Accuracy} & \textbf{Precision} & \textbf{Recall} & \textbf{F1-score} \\
        \midrule
        \multicolumn{5}{c}{\textbf{Zero Shot}} \\
                Zero-shot E-eyes & 0.26 & 0.26 & 0.27 & 0.26 \\
        Zero-shot CARM & 0.24 & 0.24 & 0.24 & 0.24 \\
                Zero-shot SVM & 0.27 & 0.28 & 0.28 & 0.27 \\
        Zero-shot CNN & 0.23 & 0.24 & 0.23 & 0.23 \\
        Zero-shot RNN & 0.26 & 0.26 & 0.26 & 0.26 \\
DeepSeek-0shot & 0.54 & 0.61 & 0.54 & 0.52 \\
DeepSeek-0shot-COT & 0.33 & 0.24 & 0.33 & 0.23 \\
DeepSeek-0shot-Knowledge & 0.45 & 0.46 & 0.45 & 0.44 \\
Gemma2-0shot & 0.35 & 0.22 & 0.38 & 0.27 \\
Gemma2-0shot-COT & 0.36 & 0.22 & 0.36 & 0.27 \\
Gemma2-0shot-Knowledge & 0.32 & 0.18 & 0.34 & 0.20 \\
GPT-4o-mini-0shot & 0.48 & 0.53 & 0.48 & 0.41 \\
GPT-4o-mini-0shot-COT & 0.33 & 0.50 & 0.33 & 0.38 \\
GPT-4o-mini-0shot-Knowledge & 0.49 & 0.31 & 0.49 & 0.36 \\
GPT-4o-0shot & 0.62 & 0.62 & 0.47 & 0.42 \\
GPT-4o-0shot-COT & 0.29 & 0.45 & 0.29 & 0.21 \\
GPT-4o-0shot-Knowledge & 0.44 & 0.52 & 0.44 & 0.39 \\
LLaMA-0shot & 0.32 & 0.25 & 0.32 & 0.24 \\
LLaMA-0shot-COT & 0.12 & 0.25 & 0.12 & 0.09 \\
LLaMA-0shot-Knowledge & 0.32 & 0.25 & 0.32 & 0.28 \\
Mistral-0shot & 0.19 & 0.23 & 0.19 & 0.10 \\
Mistral-0shot-Knowledge & 0.21 & 0.40 & 0.21 & 0.11 \\
        \midrule
        \multicolumn{5}{c}{\textbf{4 Shot}} \\
GPT-4o-mini-4shot & 0.58 & 0.59 & 0.58 & 0.53 \\
GPT-4o-mini-4shot-COT & 0.57 & 0.53 & 0.57 & 0.50 \\
GPT-4o-mini-4shot-Knowledge & 0.56 & 0.51 & 0.56 & 0.47 \\
GPT-4o-4shot & 0.77 & 0.84 & 0.77 & 0.73 \\
GPT-4o-4shot-COT & 0.63 & 0.76 & 0.63 & 0.53 \\
GPT-4o-4shot-Knowledge & 0.72 & 0.82 & 0.71 & 0.66 \\
LLaMA-4shot & 0.29 & 0.24 & 0.29 & 0.21 \\
LLaMA-4shot-COT & 0.20 & 0.30 & 0.20 & 0.13 \\
LLaMA-4shot-Knowledge & 0.15 & 0.23 & 0.13 & 0.13 \\
Mistral-4shot & 0.02 & 0.02 & 0.02 & 0.02 \\
Mistral-4shot-Knowledge & 0.21 & 0.27 & 0.21 & 0.20 \\
        \midrule
        
        \multicolumn{5}{c}{\textbf{Suprevised}} \\
        SVM & 0.94 & 0.92 & 0.91 & 0.91 \\
        CNN & 0.98 & 0.98 & 0.97 & 0.97 \\
        RNN & 0.99 & 0.99 & 0.99 & 0.99 \\
        % \midrule
        % \multicolumn{5}{c}{\textbf{Conventional Wi-Fi-based Human Activity Recognition Systems}} \\
        E-eyes & 1.00 & 1.00 & 1.00 & 1.00 \\
        CARM & 0.98 & 0.98 & 0.98 & 0.98 \\
\midrule
 \multicolumn{5}{c}{\textbf{Vision Models}} \\
           Zero-shot SVM & 0.26 & 0.25 & 0.25 & 0.25 \\
        Zero-shot CNN & 0.26 & 0.25 & 0.26 & 0.26 \\
        Zero-shot RNN & 0.28 & 0.28 & 0.29 & 0.28 \\
        SVM & 0.99 & 0.99 & 0.99 & 0.99 \\
        CNN & 0.98 & 0.99 & 0.98 & 0.98 \\
        RNN & 0.98 & 0.99 & 0.98 & 0.98 \\
GPT-4o-mini-Vision & 0.84 & 0.85 & 0.84 & 0.84 \\
GPT-4o-mini-Vision-COT & 0.90 & 0.91 & 0.90 & 0.90 \\
GPT-4o-Vision & 0.74 & 0.82 & 0.74 & 0.73 \\
GPT-4o-Vision-COT & 0.70 & 0.83 & 0.70 & 0.68 \\
LLaMA-Vision & 0.20 & 0.23 & 0.20 & 0.09 \\
LLaMA-Vision-Knowledge & 0.22 & 0.05 & 0.22 & 0.08 \\

        \bottomrule
    \end{tabular}
    \label{full}
\end{table*}




\end{document}

}

\newpage
\appendix
\twocolumn
\section* {Supplementary Material}

For better understanding of this work, we provide additional details, analysis, and results as follow:

\begin{itemize}[label={}]
   \item \textbf{A. Detailed Architectures of Modules} \\
   In this section, we display the detailed architectures for the cross-attention block $\mathtt{CAB}$ and masked self-attention block $\mathtt{MaskedSA}$ in the main text.

   \vspace{0.5em}
    
   \item \textbf{B. Inference Details} \\
   We provide more details for the inference of PRVQL.

   \vspace{0.5em}

   \item \textbf{C. Additional Experimental Results} \\
   We offer more experimental results in this work, including more ablations and comparison of different method across different scales on the Ego4D dataset.

   \vspace{0.5em}

   \item \textbf{D. Visualization Analysis of  Spatial Knowledge} \\
   We provide visual analysis to show the learned target spatial knowledge.

   \vspace{0.5em}

   \item \textbf{E. More Qualitative Results} \\
   We demonstrate more qualitative results of our method
    for localizing the target object.
   
\end{itemize}

\section{Detailed Architectures of Modules}

In each stage of PRVQL, we adopt the cross-attention block $\mathtt{CAB}$ to fuse the query feature into the video feature and then utilize the masked self-attention block $\mathtt{MaskedSA}$ for further enhancing the video feature. The architectures of $\mathtt{CAB}$ and $\mathtt{MaskedSA}$ are shown in Fig.~\ref{fig:support_fig3}.

\begin{figure}[!ht]
    \centering
    \includegraphics[width=0.88\linewidth]{figs/support_fig3.pdf}
    \vspace{-1mm}
    \caption{Detailed architectures of $\mathtt{CAB}$ and $\mathtt{MaskedSA}$.}
    \label{fig:support_fig3}\vspace{-4mm}
\end{figure}



\section{Inference Details}

Similar to~\cite{jiang2024single}, for inference, we first predict the confidence scores for target occurrence in all frames. Given the scores, we then smooth them through a median filter with the kernel size of 1. After this, we perform peak detection on the smoothed scores. We detect the peak based on the highest score $h$ and use $0.79\cdot h$ as the threshold to filter non-confident peaks. Finally, we can determine a spatio-temporal tube that corresponds to the most recent peak as the prediction result. In order to detect start and end time of the tube, we threshold the confidences scores using the threshold of $0.585\cdot \tilde{h}$, where $\tilde{h}$ is the confidence score at the most recent peak.

\begin{table}[!t]
        \setlength{\tabcolsep}{7.5pt}
	\centering
    \caption{ Ablation studies on the parameter $\alpha$ in SKG.}\vspace{-2mm}
\renewcommand{\arraystretch}{1}
			\scalebox{0.92}{
				\begin{tabular}{ccccccc}
					\specialrule{1.5pt}{0pt}{0pt}
					\rowcolor{mygray} 
					& & tAP$_{25}$ & stAP$_{25}$ & rec\% &  Succ \\ \hline\hline
					\ding{182} & $\alpha$=0.4 & 0.33 & 0.26 & 47.27 & 57.24 \\
                    \ding{183} & $\alpha$=0.5 & \textbf{0.35} & \textbf{0.27} & \textbf{47.87} & \textbf{57.93} \\
                    \ding{184} & $\alpha$=0.6 & 0.31 & 0.26 & 46.34 & 55.97 \\
					\specialrule{1.5pt}{0pt}{0pt}
			\end{tabular}}
			\label{tab:merge_atten_lkg}
\end{table}

\begin{table}[!t]
        \setlength{\tabcolsep}{7.5pt}
	\centering
    \caption{Ablation studies on combination methods in QFR.}
    \vspace{-2mm}
			\renewcommand{\arraystretch}{1}
			\scalebox{0.89}{
				\begin{tabular}{ccccccc}
					\specialrule{1.5pt}{0pt}{0pt}
					\rowcolor{mygray} 
					& Method & tAP$_{25}$ & stAP$_{25}$ & rec\% &  Succ \\ \hline\hline
					\ding{182} & Addition & 0.31 & 0.23 & 46.57 & 56.37 \\
                    \ding{183} & Concatenation & 0.34 & 0.25 & 47.31 & 56.64 \\
                    \ding{184} & Cross-Attention & \textbf{0.35} & \textbf{0.27} & \textbf{47.87} & \textbf{57.93} \\
					\specialrule{1.5pt}{0pt}{0pt}
			\end{tabular}}
			\label{tab:merge_akg}
\end{table}

\begin{table}[!t]
    \setlength{\tabcolsep}{7.5pt}
    \centering
    \caption{Comparison on object of different scales in videos.}\vspace{-2mm}
    \scalebox{0.8}{
    \begin{tabular}{rccccc}%\shline
    \specialrule{1.5pt}{0pt}{0pt}
     \rowcolor{mygray} Method & Scale & tAP$_{25}$ & stAP$_{25}$ & rec$\%$ &Succ \\
     \cmidrule(r){1-1}\cmidrule(l){2-2}\cmidrule(l){3-6}
     CocoFormer        & \textit{small} & \textbf{0.067} & \textbf{0.030} & \textbf{19.565} & \textbf{21.113} \\
     VQLoC      & \textit{small} & 0.047          & 0.001          &   2.447          & 13.043\\
     PRVQL (ours)                       & \textit{small} & 0.036          & 0.004          &    2.351    & 16.087 \\
     \hline
     CocoFormer        & \textit{medium} & 0.206          & 0.127                  & 32.583  & 40.804 \\
     VQLoC     & \textit{medium} & 0.213          & 0.138                   & 33.738 &  44.719\\
     PRVQL (ours)                       & \textit{medium} & \textbf{0.261} & \textbf{0.179}  & \textbf{34.359} & \textbf{49.923}\\
     \hline
     CocoFormer        & \textit{large} & 0.338 & 0.271  & 40.737 & 56.164\\
     VQLoC      & \textit{large} & 0.454 & 0.387    & \textbf{53.635} &  67.680 \\
     PRVQL (ours)                       & \textit{large} & \textbf{0.469} & \textbf{0.396}  & 52.127 & \textbf{68.664} \\
     \specialrule{1.5pt}{0pt}{0pt}
    \end{tabular}}
    \label{tabel:scale}
\end{table}

\begin{figure}[!t]
    \centering
    \includegraphics[width=0.95\linewidth]{figs/support_fig2.pdf}
    \vspace{-1mm}
    \caption{Visualization of target the spatial knowledge. First row: given visual query; second row: foreground target represented by red boxes; third row: learned target spatial knowledge.}
    \label{fig:support_fig2}\vspace{-3mm}
\end{figure}

\begin{figure}[!t]
    \centering
    \includegraphics[width=0.98\linewidth]{figs/support_fig1.pdf}
    \vspace{-1mm}
    \caption{Qualitative results of our method.}
\label{fig:support_fig_qual}\vspace{-4mm}
\end{figure}

\section{Additional Experimental Results}

In this section, we show more ablation studies and comparison to other methods on the Ego4D validation set.

\vspace{0.3em}
\noindent
\textbf{Impact of Balance Parameter $\alpha$ in SKG.} The interpolated attention map $\varphi_{\text{int}}(\mathcal{T}_k^{\text{d}})$, obtained via bilinear interpolation from $\mathcal{T}_k^d$, is merged with $\mathcal{S}_k$ through a balance parameter $\alpha$. We conduct an ablation on $\alpha$ in Tab.~\ref{tab:merge_atten_lkg}. We can observe that, when setting $\alpha$ to 0.5, we show the best result (see \ding{183}). 

\vspace{0.3em}
\noindent
\textbf{Different Combination Methods in QFR.} In QFR, the appearance knowledge $\mathcal{K}^a_k$, obtained by AKG, is used to guide the refinement of query feature. In PRVQL, we use a cross-attention block to combine $\mathcal{K}^a_k$ and $\mathcal{Q}_k$ for achieving refinement. Besides cross-attention, we conduct experiments using other manners for refinement, including element-wise addition and concatenation, in Tab.~\ref{tab:merge_akg}. As shown in  Tab.~\ref{tab:merge_akg}, when using the cross-attention block for query feature refinement, we achieve the best performance (see \ding{184}).

 \vspace{0.3em}
\noindent
\textbf{Comparison in Different Scales.} Following~\cite{jiang2024single}, we provide comparison for objects of different scales in videos. As in ~\cite{jiang2024single}, the objects are categorized to three scales, including \emph{small} scale with target area in the range of [0, 64$^2$], \emph{medium} scale with the target area in the range (64$^2$, 192$^2$], and \emph{large} scale with target area greater than 192$^2$.  Tab.~\ref{tabel:scale} reports the comparison result. As in Tab.~\ref{tabel:scale}, we can observe that, CocoFormer performs better for small-scale objects. We argue that the reason is CocoFormer adopts higher-resolution images for localization and employs detector that is good at small object detection, while VQLoC and our method use downsampled frames for localization and do not specially deal with small objects. In comparison to CocoFormer and VQLoC, our PRVQL achieves better overall performance for medium- and large-scale objects, which shows the efficacy of target knowledge for robust target localization. 



\section{Visualization Analysis of Spatial Knowledge}

The target spatial knowledge by SKG aims to explore target cues from videos for enhancing the target while suppressing background regions in video features. In PRVQL, we adopt the readily available attention maps to produce target spatial knowledge, which is shown in Fig.~\ref{fig:support_fig2}. From Fig.~\ref{fig:support_fig2}, we can see that, our spatial knowledge focuses more on the target object while less on the background, and thus can be applied to refine video features for better localization.

\section{More Qualitative Results}

In order to further validate the effectiveness of our PRVQL, we provide additional examples of target localization results in Fig.~\ref{fig:support_fig_qual}. From the shown visualizations, we can observe that, with the hlep of target knowledge, our method can accurately locate the target in both space and time.

\end{document}
