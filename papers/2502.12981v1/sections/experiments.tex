\section{Experiments}\label{sec:experiments}

\begin{figure}[t!]
    \centering
    \includegraphics[width=1.00\textwidth]{figures/main_exp.png} 
    \caption{Comparison of the spherical checkerboard distribution generated with CFM, VFM, RFM and our methods  RG-VFM-$\mathbb{R}^3$ and RG-VFM-$\mathcal{M}$.}
    \label{fig:main_exp}
\end{figure}

Inspired by the planar checkerboard benchmark in generative modeling \citep{grathwohl2018ffjord}, we introduce a spherical checkerboard distribution as our target $p_1$, whose support is $\mathbb{S}^2 \subset \mathbb{R}^3$. The noisy distribution $p_0$ is defined differently for each model: for CFM, VFM, and RG-VFM-$\mathbb{R}^3$, $p_0$ is the uniform distribution on $[-1,1]^3 \subset \mathbb{R}^3$, while for RG-VFM-$\mathcal{M}$ and RFM, it is the uniform distribution on $\mathbb{S}^2$. 

We conduct two sets of experiments: we (1) compare Euclidean to geometric models in capturing the correct geometry -- assessed by the norms of generated samples (ideally unit norm, since the points should lie on the sphere) -- and (2) evaluate vanilla versus variational models in reproducing the target distribution. \Cref{fig:main_exp} displays the generated distributions alongside the ground truth. All the experimental details and results are provided in \cref{sec:app_exp}.

Comparing the Euclidean and geometric models, the norm statistics reveals key differences. The Euclidean models (CFM and VFM) show slight norm deviation from the unit sphere (CFM: mean = $1.00$, std = $0.094$; VFM: mean = $1.00$, std = $0.021$), while the geometric models maintain a near-perfect unit norm. The geometric extensions effectively capture the underlying geometry well.

Comparing vanilla and variational models, we observe that vanilla models produce more blurred distributions, whereas variational models better patches contrasts. As shown in \cref{fig:main_exp}, RG-VFM-$\mathbb{R}^3$ and RG-VFM-$\mathcal{M}$ exhibit the best visual performance, with minimal differences. This is likely because RG-VFM's loss functions is both geometrically informed and focused on minimizing the distance between predictions and distribution endpoints rather than matching intermediate flow velocities. More interestingly, standard VFM also demonstrates strong performance, which we speculate is due to its emphasis on endpoint minimization. In essence, emphasizing endpoint accuracy enables variational models to capture the fine details of the target distribution's shape, and the additional geometric awareness of RG-VFM further enhances the result.


