\section{Geometric Background}\label{sec:app_background}

\subsection{Riemannian Manifolds}
\label{app:riemannian}

In this section, we provide a concise review of Riemannian manifolds in the setting of complete, connected, and smooth manifolds. Let $\mathcal{M}$ be such a manifold endowed with a Riemannian metric $\mathbf{g}$. At each point $x \in \mathcal{M}$, the tangent space is denoted by $T_x\mathcal{M}$, and $\mathbf{g}$ induces an inner product on each $T_x\mathcal{M}$, which we write as $\langle u, v \rangle_{\mathbf{g}}$ for $u, v \in T_x\mathcal{M}$. By collecting all tangent spaces across $\mathcal{M}$, we obtain the tangent bundle:
$$
T\mathcal{M} \;=\; \bigcup_{x \in \mathcal{M}} \{x\} \times T_x\mathcal{M}.
$$

We will also consider time-dependent vector fields on $\mathcal{M}$. Specifically, let $\{u_t\}_{t \in [0,1]}$ be a family of smooth vector fields, with each $u_t$ mapping a point $x \in \mathcal{M}$ to an element $u_t(x) \in T_x\mathcal{M}$. The operator $\mathrm{div}_{\mathbf{g}}(u_t)$ denotes the Riemannian divergence of $u_t$ with respect to the spatial variable $x$.

Furthermore, the volume form on $\mathcal{M}$ induced by $\mathbf{g}$ is denoted by $d\mathcal{M}_z$. For any real-valued function $f: \mathcal{M} \to \mathbb{R}$, we write
$$
\int_{\mathcal{M}} f(x)\, d\mathcal{M}_z
$$
to denote its integral over $\mathcal{M}$.

\paragraph{Homogeneous Manifold.} A Riemannian manifold $\mathcal{M}$ is homogeneous if its isometry group acts transitively on $\mathcal{M}$, i.e., for any two points $x,y\in\mathcal{M}$, there exists an isometry $f:\mathcal{M}\to\mathcal{M}$ such that $f(x)=y$.

\subsection{Riemannian Gaussian Distributions}\label{app:riem_gauss}

We describe the construction of the Riemannian Gaussian (RG) distribution, which generalizes the familiar Gaussian distribution to the setting of a Riemannian manifold. The definition is inspired from \citet{pennec2006intrinsic}.

\paragraph{Riemannian Gaussian.} Let $\mathcal{M}$ be a Riemannian manifold endowed with the metric tensor $\mathbf{g}$. The RG distribution is defined by
\[
\mathcal{N}_{\text{Riem}}(z \mid \sigma, \mu) 
= \frac{1}{C} \exp\!\Bigl(-\frac{\text{dist}_{\mathbf{g}}(z, \mu)^2}{2\sigma^2}\Bigr),
\]
where $z\in\mathcal{M}$ is a point on the manifold, $\mu\in\mathcal{M}$ plays the role of the mean, and $\sigma>0$ is a scale parameter controlling the spread of the distribution. Here, $\text{dist}_{\mathbf{g}}(z, \mu)$ denotes the geodesic distance between $z$ and $\mu$ as determined by the metric $\mathbf{g}$, and $C$ is a normalization constant chosen so that the total probability integrates to 1 over $\mathcal{M}$:
\[
C = \int_{\mathcal{M}} \exp\!\Bigl(-\frac{\text{dist}_{\mathbf{g}}(z, \mu)^2}{2\sigma^2}\Bigr)\, d\mathcal{M}_z.
\]
The measure $d\mathcal{M}_z$ represents the Riemannian volume element, which in local coordinates takes the form
\[
d\mathcal{M}_z = \sqrt{\det \mathbf{g}(z)}\, dz,
\]
with $dz$ being the standard Lebesgue measure in the coordinate chart and $\mathbf{g}(z)$ is the Riemannian metric tensor at the point $z$. This formulation ensures that the probability density is adapted to the geometric structure of the manifold.

\paragraph{Observation.} In the special case where $\mathcal{M} = \mathbb{R}^d$ and the metric is Euclidean (i.e., $\mathbf{g}(z) = \mathbf{I}$), the geodesic distance reduces to the usual Euclidean distance, and the RG distribution becomes the standard multivariate Gaussian with covariance matrix $\sigma^2\mathbf{I}$. On more general manifolds, however, the curvature and topology are taken into account through the geodesic distance and the volume element, leading to a natural extension of the Gaussian concept. This construction can be applied to spaces such as hyperbolic manifolds, where one can define the distribution in the tangent space at a point $\mu$ and then use the exponential map to project it onto the manifold.

\paragraph{Comparison to vMF.} A closely related distribution is the von Mises--Fisher (vMF) distribution, which is traditionally defined on the sphere $S^n$ by
\[
\text{vMF}(z \mid \mu, \kappa) \propto \exp\!\bigl(\kappa\,\langle z,\mu\rangle\bigr),
\]
with $\mu\in S^n$ and $\langle \cdot, \cdot \rangle$ denoting the standard dot product. The vMF distribution is based on the notion of directional data and an inner product structure that measures alignment. In contrast, the RG distribution is inherently tied to the Riemannian metric, making it applicable to a much wider class of manifolds. Generalizing the idea behind the vMF distribution to other geometries often requires embedding the manifold into a larger ambient space and defining a suitable bilinear form (such as the Minkowski inner product in hyperbolic geometry). In this sense, the RG approach offers a more natural and geometrically intrinsic formulation.

In summary, the Riemannian Gaussian distribution is defined in terms of the geodesic distance and the corresponding volume element, and it adapts to the underlying geometry of any Riemannian manifold.