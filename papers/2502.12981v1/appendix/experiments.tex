\section{Experiments}\label{sec:app_exp}

In this section, we present further results from the experiments described in Section~\ref{sec:experiments}.

\subsection{Experimental setup} In all experiments, the target distribution \(p_1\) is the spherical checkerboard, so its support is \(\mathbb{S}^2\). The noisy distribution \(p_0\) varies by model: for CFM, VFM, and RG-VFM-$\mathbb{R}^3$ , \(p_0\) is the uniform distribution on \([-1,1]^3 \subset \mathbb{R}^3\), while for RG-VFM-\(\mathcal{M}\) and RFM, \(p_0\) is the uniform distribution on \(\mathbb{S}^2\). In every case, we train a five-layer MLP with 64/128 hidden features for 3000 epochs, that we use to generate 5000 samples using a Runge-Kutta ODE solver. 

\subsection{Results}
Figure~\ref{fig:probability_paths} illustrates the generative flow trajectories over time, from the initial distribution \(p_0\) to the generated distribution at \(t=1\).

Figure~\ref{fig:densities_unwrapped} displays the generated distributions unwrapped onto a flat surface for easier visualization and comparison. These results visually confirm the observations presented in Section~\ref{sec:experiments}.

Finally, Figure~\ref{fig:norms} shows histograms of the norm values of the generated samples. As discussed in Section~\ref{sec:experiments}, this metric differentiates the Euclidean models (CFM and VFM) from the others. Ideally, points should have a norm of 1, since they lie on the sphere. However, because the Euclidean models lack explicit geometric information, their points deviate slightly from a norm of 1—with CFM exhibiting a larger divergence. In contrast, the geometric models consistently generate points that lie almost exactly on the sphere.


\begin{figure}[htbp]
  \centering
  
  % Subfigure 1
  \begin{subfigure}{\textwidth}
    \centering
    \includegraphics[trim=4cm 0 4cm 0, clip, width=\linewidth]{figures/experiments/vanilla/euclidean/uniform_r3/probability_paths.png}
    \subcaption{\textbf{Model}: CFM; $\operatorname{supp}(p_0) := \mathbb{R}^3$, ${p_0}$: uniform distribution on \([-1,1]^3 \subset \mathbb{R}^3\).}
    \label{fig:cfm_prob}
  \end{subfigure}
  
  \vspace{1ex}
  
  % Subfigure 2
  \begin{subfigure}{\textwidth}
    \centering
    \includegraphics[trim=4cm 0 4cm 0, clip, width=\linewidth]{figures/experiments/variational/euclidean/uniform_r3/probability_paths.png}
    \subcaption{\textbf{Model}: VFM;  $\operatorname{supp}(p_0) := \mathbb{R}^3$; ${p_0}$: uniform distribution on \([-1,1]^3 \subset \mathbb{R}^3\).}    \label{fig:vfm_prob}
  \end{subfigure}
  
  \vspace{1ex}
  
  % Subfigure 3
  \begin{subfigure}{\textwidth}
    \centering
    \includegraphics[trim=4cm 0 4cm 0, clip, width=\linewidth]{figures/experiments/variational/sphere/uniform_r3/probability_paths.png}
    \subcaption{\textbf{Model}: RG-VFM;  $\operatorname{supp}(p_0) := \mathbb{R}^3$; ${p_0}$: uniform distribution on \([-1,1]^3 \subset \mathbb{R}^3\).}    
    \label{fig:rgvfm_prob}
  \end{subfigure}
  
  \vspace{1ex}
  
  % Subfigure 4
  \begin{subfigure}{\textwidth}
    \centering
    \includegraphics[trim=4cm 0 4cm 0, clip, width=\linewidth]{figures/experiments/vanilla/sphere/gaussian_sphere/probability_paths.png}
    \subcaption{\textbf{Model}: RFM;  $\operatorname{supp}(p_0) := \mathbb{S}^2$; ${p_0}$: uniform distribution on \(\mathbb{S}^2\).}     
    \label{fig:rfm_prob}
  \end{subfigure}
  
  \vspace{1ex}
  
  % Subfigure 5
  \begin{subfigure}{\textwidth}
    \centering
    \includegraphics[trim=4cm 0 4cm 0, clip, width=\linewidth]{figures/experiments/variational/sphere/gaussian_sphere/probability_paths.png}
    \subcaption{\textbf{Model}: RG-VFM;  $\operatorname{supp}(p_0) := \mathbb{S}^2$; ${p_0}$: uniform distribution on \(\mathbb{S}^2\).}       \label{fig:rgvfm_prob2}
  \end{subfigure}
  
  % Overall caption for the entire figure.
  \caption{Flow trajectories of 5,000 samples, initially drawn from the noisy distribution $p_0$ at $t=0$, evolving to reach their final configuration by $t=1$.}
  \label{fig:probability_paths}
\end{figure}


\begin{figure}[htbp]
  \centering
  
  % Row 1: Subfigure 1 and Subfigure 2 side by side
  \begin{subfigure}[t]{0.48\linewidth}
    \centering
    \includegraphics[width=\linewidth]{figures/experiments/vanilla/euclidean/uniform_r3/density_unwrapped.png}
    \subcaption{\textbf{Model}: CFM; $\operatorname{supp}(p_0) := \mathbb{R}^3$, ${p_0}$: uniform distribution on \([-1,1]^3 \subset \mathbb{R}^3\).}
    \label{fig:cfm_dens}
  \end{subfigure}
  \hfill
  \begin{subfigure}[t]{0.48\linewidth}
    \centering
    \includegraphics[width=\linewidth]{figures/experiments/variational/euclidean/uniform_r3/density_unwrapped.png}
    \subcaption{\textbf{Model}: VFM;  $\operatorname{supp}(p_0) := \mathbb{R}^3$; ${p_0}$: uniform distribution on \([-1,1]^3 \subset \mathbb{R}^3\).} 
    \label{fig:vfm_dens}
  \end{subfigure}
  
  \vspace{1ex}
  
  % Row 2: Subfigure 3 and Subfigure 4 side by side
  \begin{subfigure}[t]{0.48\linewidth}
    \centering
    \includegraphics[width=\linewidth]{figures/experiments/variational/sphere/uniform_r3/density_unwrapped.png}
    \subcaption{\textbf{Model}: RG-VFM;  $\operatorname{supp}(p_0) := \mathbb{R}^3$; ${p_0}$: uniform distribution on \([-1,1]^3 \subset \mathbb{R}^3\).}
    \label{fig:rgvfm_dens}
  \end{subfigure}
  \hfill
  \begin{subfigure}[t]{0.48\linewidth}
    \centering
    \includegraphics[width=\linewidth]{figures/experiments/vanilla/sphere/gaussian_sphere/density_unwrapped.png}
    \subcaption{\textbf{Model}: RFM;  $\operatorname{supp}(p_0) := \mathbb{S}^2$; ${p_0}$: uniform distribution on \(\mathbb{S}^2\).} 
    \label{fig:rfm_dens}
  \end{subfigure}
  
  \vspace{1ex}
  
  % Row 3: Subfigure 5 centered
  \begin{center}
    \begin{subfigure}[t]{0.48\linewidth}
      \centering
      \includegraphics[width=\linewidth]{figures/experiments/variational/sphere/gaussian_sphere/density_unwrapped.png}
      \subcaption{\textbf{Model}: RG-VFM;  $\operatorname{supp}(p_0) := \mathbb{S}^2$; ${p_0}$: uniform distribution on \(\mathbb{S}^2\).} 
      \label{fig:rgvfm_dens2}
    \end{subfigure}
  \end{center}
  
  \caption{Sample distributions generated by different models (representing the flow configuration at \(t=1\)) unwrapped from \(\mathbb{S}^2\) to \(\mathbb{R}^2\) for improved visualization. The true checkerboard distribution is shown in gray in the background.}
  \label{fig:densities_unwrapped}
\end{figure}


\begin{figure}[htbp]
  \centering
  
  % Row 1: Subfigure 1 and Subfigure 2 side by side
  \begin{subfigure}[t]{0.48\linewidth}
    \centering
    \includegraphics[width=\linewidth]{figures/experiments/vanilla/euclidean/uniform_r3/norms_histogram.png}
    \subcaption{\textbf{Model}: CFM; $\operatorname{supp}(p_0) := \mathbb{R}^3$, ${p_0}$: uniform distribution on \([-1,1]^3 \subset \mathbb{R}^3\).}
    \label{fig:cfm_norm}
  \end{subfigure}
  \hfill
  \begin{subfigure}[t]{0.48\linewidth}
    \centering
    \includegraphics[width=\linewidth]{figures/experiments/variational/euclidean/uniform_r3/norms_histogram.png}
    \subcaption{\textbf{Model}: VFM;  $\operatorname{supp}(p_0) := \mathbb{R}^3$; ${p_0}$: uniform distribution on \([-1,1]^3 \subset \mathbb{R}^3\).} 
    \label{fig:vfm_norm}
  \end{subfigure}
  
  \vspace{1ex}
  
  % Row 2: Subfigure 3 and Subfigure 4 side by side
  \begin{subfigure}[t]{0.48\linewidth}
    \centering
    \includegraphics[width=\linewidth]{figures/experiments/variational/sphere/uniform_r3/norms_histogram.png}
    \subcaption{\textbf{Model}: RG-VFM;  $\operatorname{supp}(p_0) := \mathbb{R}^3$; ${p_0}$: uniform distribution on \([-1,1]^3 \subset \mathbb{R}^3\).}
    \label{fig:rgvfm_norm}
  \end{subfigure}
  \hfill
  \begin{subfigure}[t]{0.48\linewidth}
    \centering
    \includegraphics[width=\linewidth]{figures/experiments/vanilla/sphere/gaussian_sphere/norms_histogram.png}
    \subcaption{\textbf{Model}: RFM;  $\operatorname{supp}(p_0) := \mathbb{S}^2$; ${p_0}$: uniform distribution on \(\mathbb{S}^2\).} 
    \label{fig:rfm_norm}
  \end{subfigure}
  
  \vspace{1ex}
  
  % Row 3: Subfigure 5 centered
  \begin{center}
    \begin{subfigure}[t]{0.48\linewidth}
      \centering
      \includegraphics[width=\linewidth]{figures/experiments/variational/sphere/gaussian_sphere/norms_histogram.png}
      \subcaption{\textbf{Model}: RG-VFM;  $\operatorname{supp}(p_0) := \mathbb{S}^2$; ${p_0}$: uniform distribution on \(\mathbb{S}^2\).} 
      \label{fig:rgvfm_norm2}
    \end{subfigure}
  \end{center}
    \caption{Histogram of the norm values of the 5000 samples describing the generated distribution.}
  \label{fig:norms}
\end{figure}