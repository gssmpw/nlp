\section{R-VFM and link with RFM}

\subsection{Detailed Derivation of RG-VFM Objective}\label{sec:rvfm}
\rgvfm*

\begin{proof}
    
The objective of VFM is defined as  
$$
\mathcal{L}_{\text{VFM}} (\theta) = -\mathbb{E}_{t,x_1,x}\left[\log q_t^{\theta}(x_1 | x)\right].
$$  
We define the objective function of RG-VFM by setting the posterior probability as the Riemannian Gaussian, i.e.,  
$$
q_t^{\theta}(x_1 | x) = \mathcal{N}_{\text{Riem}}(x_1 \mid \mu_t^{\theta}(x), \sigma(x)),
$$  
so that  
$$
\mathcal{L}_{\text{RG-VFM}} (\theta) = -\mathbb{E}_{t,x_1,x}\left[\log \mathcal{N}_{\text{Riem}}(x_1 \mid \mu_t^{\theta}(x), \sigma(x))\right].
$$  

More explicitly, we have  
$$
\begin{aligned}
\mathcal{L}_{\text{RG-VFM}} (\theta) &= -\mathbb{E}_{t,x_1,x}\left[\log q_t^{\theta}(x_1 | x)\right] \\
&= -\mathbb{E}_{t,x_1,x}\left[\log \mathcal{N}_{\text{Riem}}(x_1 \mid \mu_t^{\theta}(x), \sigma(x))\right] \\
&= -\mathbb{E}_{t,x_1,x}\left[\log \left( \frac{1}{C(\mu_t^{\theta}(x))} \exp\left(-\frac{\text{dist}_{\mathbf{g}}(x_1, \mu_t^{\theta}(x))^2}{2\sigma(x)^2}\right)\right)\right] \\
&= -\mathbb{E}_{t,x_1,x}\left[\log \left( \frac{1}{C(\mu_t^{\theta}(x))} \right) -\frac{\text{dist}_{\mathbf{g}}(x_1, \mu_t^{\theta}(x))^2}{2\sigma(x)^2}\right] \\
&= -\mathbb{E}_{t,x_1,x}\left[\log \left( \frac{1}{C(\mu_t^{\theta}(x))} \right)\right] + \mathbb{E}_{t,x_1,x}\left[\frac{\text{dist}_{\mathbf{g}}(x_1, \mu_t^{\theta}(x))^2}{2\sigma(x)^2}\right],
\end{aligned}
$$  
where $\text{dist}_{\mathbf{g}}()$ denotes the geodesic distance induced by the Riemannian metric $\mathbf{g}$.

Without any regularity assumptions on $\mathcal{M}$, no further simplification is possible. However, under the following assumptions the objective becomes more tractable:

\begin{enumerate}
    \item \textbf{Homogeneity:} If the manifold $(\mathcal{M}, \mathbf{g})$ is homogeneous, the normalization constant  
    $$
    C = \int_{\mathcal{M}} \exp\left(-\frac{\text{dist}_{\mathbf{g}}(z, \mu)^2}{2\sigma^2}\right) d\mathcal{M}_z
    $$  
    is independent of the mean $\mu$. Hence, defining  
    $$
    K := -\mathbb{E}_{t,x_1,x}\left[\log \left( \frac{1}{C(\mu_t^{\theta}(x))} \right)\right],
    $$  
    which is constant with respect to $\theta$, we obtain  
    $$
    \mathcal{L}_{\text{RG-VFM}} (\theta) = K + \mathbb{E}_{t,x_1,x}\left[\frac{\text{dist}_{\mathbf{g}}(x_1, \mu_t^{\theta}(x))^2}{2\sigma(x)^2}\right].
    $$  

    Since $K$ is a constant that is independent of the model's parameters $\theta$, the minimization objective becomes 

    $$
    \mathcal{L}_{\text{RG-VFM}} (\theta) =  \mathbb{E}_{t,x_1,x}\left[\frac{\text{dist}_{\mathbf{g}}(x_1, \mu_t^{\theta}(x))^2}{2\sigma(x)^2}\right].
    $$ 
    
    \item \textbf{Closed-form Geodesics:} If the geometry allows closed-form expressions for geodesics, namely  
    $$
    \gamma(t)=\exp_x\Bigl(t \cdot \log_x(y)\Bigr),
    $$  
    then the geodesic distance between two points is given by:  
    $$
    \text{dist}_{\mathbf{g}}(z, \mu) = \|\log_z(\mu)\|_{\mathbf{g}}.
    $$  
    In this setting, we can write  
    $$
    \text{dist}_{\mathbf{g}}(x_1, \mu_t^{\theta}(x))^2 = \|\log_{x_1}(\mu_t^{\theta}(x))\|_{\mathbf{g}}^2,
    $$  
    so that the objective becomes  
    $$
    \mathcal{L}_{\text{RG-VFM}} (\theta) = -\mathbb{E}_{t,x_1,x}\left[\log \left( \frac{1}{C(\mu_t^{\theta}(x))} \right)\right] + \mathbb{E}_{t,x_1,x}\left[\frac{1}{2\sigma(x)^2}\,\|\log_{x_1}(\mu_t^{\theta}(x))\|_{\mathbf{g}}^2\right].
    $$
    
    \item \textbf{Combined Assumptions:} If both conditions hold, the objective simplifies to  
    $$
    \mathcal{L}_{\text{RG-VFM}} (\theta)= \mathbb{E}_{t,x_1,x}\left[\frac{1}{2\sigma(x)^2}\,\|\log_{x_1}(\mu_t^{\theta}(x))\|_{\mathbf{g}}^2\right].
    $$  
    If we further assume that $\sigma(x)$ is constant, this reduces to  
    $$
    \mathcal{L}_{\text{RG-VFM}} (\theta) = \mathbb{E}_{t,x_1,x}\left[\,\|\log_{x_1}(\mu_t^{\theta}(x))\|_{\mathbf{g}}^2\right].
    $$
\end{enumerate}



\end{proof}


\paragraph{Examples of simple geometries.} A homogeneous manifold does not necessarily imply that geodesics admit closed-form expressions. Conversely, the simple geometries with closed-form geodesics considered in the RFM setting—such as hyperspheres $\mathbb{S}^n$, hyperbolic spaces $\mathbb{H}^n$, flat tori $T^n = [0, 2\pi]^n$, and the space of SPD matrices $\mathcal{S}_d^+$ with the affine-invariant metric—are homogeneous. Thus, when restricting to these geometries for comparison with RFM, we are in the combined case.

\paragraph{Euclidean space.} In the Euclidean case (which also falls into the combined case), the objective simplifies further to  
$$
\mathcal{L}_{\text{RG-VFM}}  (\theta) = \mathbb{E}_{t,x_1,x}\left[\,\|\mu_t^{\theta}(x) - x_1\|^2\right].
$$


\subsection{RG-VFM vs RFM on homogeneous spaces with closed-form geodesics}\label{app:rgvfm_rfm}

The objective of RG-VFM is defined as  
$$
\mathcal{L}_{\text{RG-VFM}}  (\theta) = \mathbb{E}_{t,x_1,x}\left[ \|\log_{x_1}(\mu_t^{\theta}(x))\|_\mathbf{g}^2\right],
$$  
while the objective of RFM, in the case of closed-form geodesics, is given by  
$$
\mathcal{L}_{\text{RFM}}  (\theta) = \mathbb{E}_{t,x_1,x}\left[\left\|v_t^{\theta}(x) - \log_{x}(x_1)/(1-t)\right\|_\mathbf{g}^2\right],
$$  
with $\mathbf{g}$ being the metric tensor at $x \sim p_t(x | x_1)$.

Ignoring multiplicative constants that depend only on $t$ and $x$, comparing the two losses reduces to comparing the quantities  
$$
\|\log_{x_1}(\mu_t^{\theta}(x))\|_\mathbf{g}^2 \quad \text{and} \quad \|v_t^{\theta}(x) - \log_{x}(x_1)\|_\mathbf{g}^2.
$$

\paragraph{Euclidean space.} In Euclidean space $\mathbb{R}^n$, the tangent space at each point is naturally identified with $\mathbb{R}^n$. In this setting,  
$$
\log_{x_1}(\mu_t^{\theta}(x)) = \mu_t^{\theta}(x) - x_1 \quad \text{and} \quad \log_{x}(x_1) = x_1 - x.
$$  
Notice that
$$
\mu_t^{\theta}(x) - x_1 = \mu_t^{\theta}(x) - x + x - x_1 = (\mu_t^{\theta}(x) - x) - (x_1 - x) = (\mu_t^{\theta}(x) - x) - \log_{x}(x_1).
$$  
If we define (ignoring multiplicative constants such as $1/(1-t)$)  
$$
v_t^{\theta}(x) = \log_{x}(\mu_t^{\theta}(x)) = \mu_t^{\theta}(x) - x,
$$  
then it follows that  
$$
\log_{x_1}(\mu_t^{\theta}(x)) = \log_{x}(\mu_t^{\theta}(x)) - \log_{x}(x_1),
$$  
implying  
$$
\|\log_{x_1}(\mu_t^{\theta}(x))\|_\mathbf{g}^2 = \|v_t^{\theta}(x) - \log_{x}(x_1)\|_\mathbf{g}^2.
$$  
Thus, $\mathcal{L}_{\text{RG-VFM}} (\theta)$ and $\mathcal{L}_{\text{RFM}} (\theta)$ are equivalent up to an additive constant. This result is consistent with the known equivalence between $\mathcal{L}_{\text{VFM}} (\theta)$ and $\mathcal{L}_{\text{CFM}} (\theta)$.

\paragraph{General geometries.}  In non-Euclidean spaces, however, the quantities  
$$
\|\log_{x_1}(\mu_t^{\theta}(x))\|_\mathbf{g}^2 \quad \text{and} \quad \|v_t^{\theta}(x) - \log_{x}(x_1)\|_\mathbf{g}^2
$$  
are not necessarily equal. This is because $\log_{x_1}(\mu_t^{\theta}(x))$ is a vector in $T_{x_1}\mathcal{M}$, while $\log_{x}(\mu_t^{\theta}(x)) - \log_{x}(x_1)$ lies in $T_x\mathcal{M}$, and in general $T_{x_1}\mathcal{M} \neq T_x\mathcal{M}$. Establishing a relation between these vectors is not straightforward and can be illustrated by comparing the law of cosines in Euclidean, hyperbolic spaces, and on hyperspheres.


