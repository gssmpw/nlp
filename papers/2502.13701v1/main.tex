%%%%%%%%%%%%%%%%%%%%%%%%%%%%%%%%%%%%%%%%%%%%%%%%%%%%%%%%%%%%%%%%%%%%%%%%

%%% LaTeX Template for AAMAS-2025 (based on sample-sigconf.tex)
%%% Prepared by the AAMAS-2025 Program Chairs based on the version from AAMAS-2025. 

%%%%%%%%%%%%%%%%%%%%%%%%%%%%%%%%%%%%%%%%%%%%%%%%%%%%%%%%%%%%%%%%%%%%%%%%

%%% Start your document with the \documentclass command.


%%% == IMPORTANT ==
%%% Use the first variant below for the final paper (including auithor information).
%%% Use the second variant below to anonymize your submission (no authoir information shown).
%%% For further information on anonymity and double-blind reviewing, 
%%% please consult the call for paper information
%%% https://aamas2025.org/index.php/conference/calls/submission-instructions-main-technical-track/

%%%% For anonymized submission, use this
%\documentclass[sigconf,anonymous]{aamas} 

%%%% For camera-ready, use this
\documentclass[sigconf]{aamas} 


%%% Load required packages here (note that many are included already).

\usepackage{balance} % for balancing columns on the final page
\usepackage{latexsym}
%\usepackage{amssymb}
\usepackage{amsmath}
%\usepackage{amsthm}
\usepackage{booktabs}
\usepackage{enumitem}
\usepackage{graphicx}


% \newenvironment{sketchproof}{\begin{proof}[Sketch of Proof]}{\end{proof}}
%%%%%%%%%%%%%%%%%%%%%%%%%%%%%%%%%%%%%%%%%%%%%%%%%%%%%%%%%%%%%%%%%%%%%%%%

%%% AAMAS-2025 copyright block (do not change!)

\setcopyright{ifaamas}
\acmConference[AAMAS '25]{Proc.\@ of the 24th International Conference
on Autonomous Agents and Multiagent Systems (AAMAS 2025)}{May 19 -- 23, 2025}
{Detroit, Michigan, USA}{Y.~Vorobeychik, S.~Das, A.~Nowe (eds.)}
\copyrightyear{2025}
\acmYear{2025}
\acmDOI{}
\acmPrice{}
\acmISBN{}


%%%%%%%%%%%%%%%%%%%%%%%%%%%%%%%%%%%%%%%%%%%%%%%%%%%%%%%%%%%%%%%%%%%%%%%%

%%% == IMPORTANT ==
%%% Use this command to specify your OpenReview submission number.
%%% In anonymous mode, it will be printed on the first page.

\acmSubmissionID{169}

%%% Use this command to specify the title of your paper.

\title[]{Causes and Strategies in Multiagent Systems}

%%% Provide names, affiliations, and email addresses for all authors.

\author{Sylvia S. Kerkhove}
\affiliation{
  \institution{Utrecht University}
  \city{Utrecht}
  \country{The Netherlands}}
\email{s.s.kerkhove@uu.nl}

\author{Natasha Alechina}
\affiliation{
  \institution{Open University}
  \city{Heerlen}
  \country{The Netherlands}  
  }
  \affiliation{
  \institution{Utrecht University}
  \city{Utrecht}
  \country{The Netherlands}}
\email{natasha.alechina@ou.nl}

\author{Mehdi Dastani}
\affiliation{
  \institution{Utrecht University}
  \city{Utrecht}
  \country{The Netherlands}}
\email{m.m.dastani@uu.nl}


%%% Use this environment to specify a short abstract for your paper.

\begin{abstract}
Causality plays an important role in daily processes, human reasoning, and artificial intelligence. 
There has however not been much research on causality in multi-agent strategic settings. 
In this work, we introduce a systematic way to build a multi-agent system model, represented as a concurrent game structure, for a given structural causal model. In the obtained so-called causal concurrent game structure, transitions correspond to interventions on agent variables of the given causal model. The Halpern and Pearl framework of causality is used to determine the effects of a certain value for an agent variable on other variables. The causal concurrent game structure allows us to analyse and reason about causal effects of agents’ strategic decisions. We formally investigate the relation between causal concurrent game structures and the original structural causal models. 
\end{abstract}

%%% The code below was generated by the tool at http://dl.acm.org/ccs.cfm.
%%% Please replace this example with code appropriate for your own paper.


%%% Use this command to specify a few keywords describing your work.
%%% Keywords should be separated by commas.

\keywords{Causality, Multi-Agent Systems, Strategic Behaviour}

%%%%%%%%%%%%%%%%%%%%%%%%%%%%%%%%%%%%%%%%%%%%%%%%%%%%%%%%%%%%%%%%%%%%%%%%

%%% Include any author-defined commands here.
         
\newcommand{\BibTeX}{\rm B\kern-.05em{\sc i\kern-.025em b}\kern-.08em\TeX}

\newcommand{\bs}{\backslash}
\newcommand{\vphi}{\varphi}
\newcommand{\xrightarrowstar}[1]{\xrightarrow{#1}
\mathrel{\vphantom{\to}^*}}
\newcommand{\rrangle}{\rangle\rangle}
\newcommand{\llangle}{\langle \langle}
\newcommand{\until}{\mathcal{U}}
%\newcommand{\next}{\bigcirc}
\newcommand{\set}[1]{\{ {#1} \}}
\newcommand{\csetting}{(\mathcal{M},\mathbf{u})}
\newcommand{\csettingint}[1]{(\mathcal{M}^{#1},\mathbf{u})}
\newcommand{\myvec}[1]{\mathbf{#1}}

%%%%%%%%%%%%%%%%%%%%%%%%%%%%%%%%%%%%%%%%%%%%%%%%%%%%%%%%%%%%%%%%%%%%%%%%

\begin{document}

%%% The following commands remove the headers in your paper. For final 
%%% papers, these will be inserted during the pagination process.

\pagestyle{fancy}
\fancyhead{}

%%% The next command prints the information defined in the preamble.

\maketitle 

%%%%%%%%%%%%%%%%%%%%%%%%%%%%%%%%%%%%%%%%%%%%%%%%%%%%%%%%%%%%%%%%%%%%%%%%

\section{Introduction}
\label{sec:introduction}
\section{Introduction}
\label{sec:introduction}
The business processes of organizations are experiencing ever-increasing complexity due to the large amount of data, high number of users, and high-tech devices involved \cite{martin2021pmopportunitieschallenges, beerepoot2023biggestbpmproblems}. This complexity may cause business processes to deviate from normal control flow due to unforeseen and disruptive anomalies \cite{adams2023proceddsriftdetection}. These control-flow anomalies manifest as unknown, skipped, and wrongly-ordered activities in the traces of event logs monitored from the execution of business processes \cite{ko2023adsystematicreview}. For the sake of clarity, let us consider an illustrative example of such anomalies. Figure \ref{FP_ANOMALIES} shows a so-called event log footprint, which captures the control flow relations of four activities of a hypothetical event log. In particular, this footprint captures the control-flow relations between activities \texttt{a}, \texttt{b}, \texttt{c} and \texttt{d}. These are the causal ($\rightarrow$) relation, concurrent ($\parallel$) relation, and other ($\#$) relations such as exclusivity or non-local dependency \cite{aalst2022pmhandbook}. In addition, on the right are six traces, of which five exhibit skipped, wrongly-ordered and unknown control-flow anomalies. For example, $\langle$\texttt{a b d}$\rangle$ has a skipped activity, which is \texttt{c}. Because of this skipped activity, the control-flow relation \texttt{b}$\,\#\,$\texttt{d} is violated, since \texttt{d} directly follows \texttt{b} in the anomalous trace.
\begin{figure}[!t]
\centering
\includegraphics[width=0.9\columnwidth]{images/FP_ANOMALIES.png}
\caption{An example event log footprint with six traces, of which five exhibit control-flow anomalies.}
\label{FP_ANOMALIES}
\end{figure}

\subsection{Control-flow anomaly detection}
Control-flow anomaly detection techniques aim to characterize the normal control flow from event logs and verify whether these deviations occur in new event logs \cite{ko2023adsystematicreview}. To develop control-flow anomaly detection techniques, \revision{process mining} has seen widespread adoption owing to process discovery and \revision{conformance checking}. On the one hand, process discovery is a set of algorithms that encode control-flow relations as a set of model elements and constraints according to a given modeling formalism \cite{aalst2022pmhandbook}; hereafter, we refer to the Petri net, a widespread modeling formalism. On the other hand, \revision{conformance checking} is an explainable set of algorithms that allows linking any deviations with the reference Petri net and providing the fitness measure, namely a measure of how much the Petri net fits the new event log \cite{aalst2022pmhandbook}. Many control-flow anomaly detection techniques based on \revision{conformance checking} (hereafter, \revision{conformance checking}-based techniques) use the fitness measure to determine whether an event log is anomalous \cite{bezerra2009pmad, bezerra2013adlogspais, myers2018icsadpm, pecchia2020applicationfailuresanalysispm}. 

The scientific literature also includes many \revision{conformance checking}-independent techniques for control-flow anomaly detection that combine specific types of trace encodings with machine/deep learning \cite{ko2023adsystematicreview, tavares2023pmtraceencoding}. Whereas these techniques are very effective, their explainability is challenging due to both the type of trace encoding employed and the machine/deep learning model used \cite{rawal2022trustworthyaiadvances,li2023explainablead}. Hence, in the following, we focus on the shortcomings of \revision{conformance checking}-based techniques to investigate whether it is possible to support the development of competitive control-flow anomaly detection techniques while maintaining the explainable nature of \revision{conformance checking}.
\begin{figure}[!t]
\centering
\includegraphics[width=\columnwidth]{images/HIGH_LEVEL_VIEW.png}
\caption{A high-level view of the proposed framework for combining \revision{process mining}-based feature extraction with dimensionality reduction for control-flow anomaly detection.}
\label{HIGH_LEVEL_VIEW}
\end{figure}

\subsection{Shortcomings of \revision{conformance checking}-based techniques}
Unfortunately, the detection effectiveness of \revision{conformance checking}-based techniques is affected by noisy data and low-quality Petri nets, which may be due to human errors in the modeling process or representational bias of process discovery algorithms \cite{bezerra2013adlogspais, pecchia2020applicationfailuresanalysispm, aalst2016pm}. Specifically, on the one hand, noisy data may introduce infrequent and deceptive control-flow relations that may result in inconsistent fitness measures, whereas, on the other hand, checking event logs against a low-quality Petri net could lead to an unreliable distribution of fitness measures. Nonetheless, such Petri nets can still be used as references to obtain insightful information for \revision{process mining}-based feature extraction, supporting the development of competitive and explainable \revision{conformance checking}-based techniques for control-flow anomaly detection despite the problems above. For example, a few works outline that token-based \revision{conformance checking} can be used for \revision{process mining}-based feature extraction to build tabular data and develop effective \revision{conformance checking}-based techniques for control-flow anomaly detection \cite{singh2022lapmsh, debenedictis2023dtadiiot}. However, to the best of our knowledge, the scientific literature lacks a structured proposal for \revision{process mining}-based feature extraction using the state-of-the-art \revision{conformance checking} variant, namely alignment-based \revision{conformance checking}.

\subsection{Contributions}
We propose a novel \revision{process mining}-based feature extraction approach with alignment-based \revision{conformance checking}. This variant aligns the deviating control flow with a reference Petri net; the resulting alignment can be inspected to extract additional statistics such as the number of times a given activity caused mismatches \cite{aalst2022pmhandbook}. We integrate this approach into a flexible and explainable framework for developing techniques for control-flow anomaly detection. The framework combines \revision{process mining}-based feature extraction and dimensionality reduction to handle high-dimensional feature sets, achieve detection effectiveness, and support explainability. Notably, in addition to our proposed \revision{process mining}-based feature extraction approach, the framework allows employing other approaches, enabling a fair comparison of multiple \revision{conformance checking}-based and \revision{conformance checking}-independent techniques for control-flow anomaly detection. Figure \ref{HIGH_LEVEL_VIEW} shows a high-level view of the framework. Business processes are monitored, and event logs obtained from the database of information systems. Subsequently, \revision{process mining}-based feature extraction is applied to these event logs and tabular data input to dimensionality reduction to identify control-flow anomalies. We apply several \revision{conformance checking}-based and \revision{conformance checking}-independent framework techniques to publicly available datasets, simulated data of a case study from railways, and real-world data of a case study from healthcare. We show that the framework techniques implementing our approach outperform the baseline \revision{conformance checking}-based techniques while maintaining the explainable nature of \revision{conformance checking}.

In summary, the contributions of this paper are as follows.
\begin{itemize}
    \item{
        A novel \revision{process mining}-based feature extraction approach to support the development of competitive and explainable \revision{conformance checking}-based techniques for control-flow anomaly detection.
    }
    \item{
        A flexible and explainable framework for developing techniques for control-flow anomaly detection using \revision{process mining}-based feature extraction and dimensionality reduction.
    }
    \item{
        Application to synthetic and real-world datasets of several \revision{conformance checking}-based and \revision{conformance checking}-independent framework techniques, evaluating their detection effectiveness and explainability.
    }
\end{itemize}

The rest of the paper is organized as follows.
\begin{itemize}
    \item Section \ref{sec:related_work} reviews the existing techniques for control-flow anomaly detection, categorizing them into \revision{conformance checking}-based and \revision{conformance checking}-independent techniques.
    \item Section \ref{sec:abccfe} provides the preliminaries of \revision{process mining} to establish the notation used throughout the paper, and delves into the details of the proposed \revision{process mining}-based feature extraction approach with alignment-based \revision{conformance checking}.
    \item Section \ref{sec:framework} describes the framework for developing \revision{conformance checking}-based and \revision{conformance checking}-independent techniques for control-flow anomaly detection that combine \revision{process mining}-based feature extraction and dimensionality reduction.
    \item Section \ref{sec:evaluation} presents the experiments conducted with multiple framework and baseline techniques using data from publicly available datasets and case studies.
    \item Section \ref{sec:conclusions} draws the conclusions and presents future work.
\end{itemize}

\section{Background}
\label{sec:literature overview}
In this section, we introduce the structural causal model framework that we will use. We also shortly introduce concurrent game structures and give a formal definition of agent strategies. 

\begin{definition}[Causal Model, Causal Setting \cite{halpern2016actual}] \label{def:causal model}
    A \emph{causal model} $\mathcal{M}$ is a pair $(\mathcal{S},\mathcal{F})$, where $\mathcal{S}$ is a signature and $\mathcal{F}$ defines a set of structural equations, relating the values of the variables.
    A \emph{signature} $\mathcal{S}$ is a tuple $(\mathcal{U},\mathcal{V},\mathcal{R})$, where $\mathcal{U}$ is a set of exogenous variables, $\mathcal{V}$ is a set of endogenous variables and $\mathcal{R}$ associates with every variable $X\in \mathcal{U}\cup \mathcal{V}$ a nonempty set $\mathcal{R}(X)$ of possible values for $X$.
    
    A \emph{causal setting} is a tuple $(\mathcal{M},\mathbf{u})$, where $\mathcal{M}$ is a causal model and $\mathbf{u}$ a setting for the exogenous variables in $\mathcal{U}$.
\end{definition}

The \emph{exogenous variables} are variables whose values depend on factors outside of the model, their causes are not explained by the model \cite{halpern2016actual,Pearl2016causal}.
On the other hand, the \emph{endogenous variables} are fully determined by the variables in the model. 
Note that with $\myvec{u}$, we use the bold-face notation to denote that $\mathbf{u}$ is a tuple.
When we use this bold-face notation for capital letters $\myvec{X}$ and $\myvec{Y}$, we are slightly abusing notation by treating them both as tuples and as sets. This follows Halpern's use of the vector notation for both concepts \cite{halpern2016actual}.
This means that we can write $\myvec{X} = \myvec{x}$ to indicate that the first element of $\myvec{X}$ gets assigned the value of the first element of $\myvec{x}$ and so on, but that we can also write $\myvec{X}'\subseteq \myvec{X}$.

\begin{example}\label{ex:causal model}
    Consider the semi-autonomous vehicle example we discussed in the introduction. 
    We can model this example with exogenous binary variables $U_O$ that determines whether there will be an obstacle on the route, and $U_{Att}$ that determines whether the human driver is paying attention.
    For the endogenous variables we introduce the binary variables $O$, indicating that there is an obstacle, $Att$, indicating that the human driver is paying attention, $HD$ for whether the human driver keeps driving or brakes. Note that we use $HD$ when the human driver keeps driving ($\neg HD$ indicates that they brake). $ODS$, indicating that the obstacle detection system detects an obstacle, $DA$, for whether the driving assistant keeps driving or brakes. Note that we use $DA$ when the human driver keeps driving ($\neg DA$ indicates that they brake). And $Col$, indicating a collision.
    The set $\mathcal{U}$ is hence $\set{U_O,U_{Att}}$ and the set $\mathcal{V}$ is hence $\set{O,Att, HD, ODS,DA, Col}$.
    We consider all variables to be Boolean, so for any variable $X \in \mathcal{U} \cup \mathcal{V}$, $\mathcal{R}(X) = \set{0,1}$.

    The following structural equations are defined for this model:
    \begin{equation*}
        \begin{array}{ll}
            O := U_O & Att:= U_{Att}  \\
            HD := \neg O \vee (O \wedge \neg Att) & ODS:= O \\
            DA := HD \wedge \neg ODS & Col := DA \wedge HD \wedge O.
        \end{array}
    \end{equation*}
\end{example}


A \emph{causal network} is a directed graph with nodes corresponding to the causal variables in $\mathcal{V}$ (and $\mathcal{U}$) with an edge from the node labelled $X$ to the node labelled $Y$ if and only if the structural equation for $Y$ depends on $X$.
In other words, we put an edge from node $X$ to node $Y$ if and only if $X$ can influence the value of $Y$ \cite{halpern2005causes}.
We call $Y$ a \emph{descendant} of $X$ if the graph contains a path from $X$ to $Y$.

A model that has an acyclic causal network is called strongly recursive \cite{halpern2016actual}. In such models, a setting $\mathbf{u}$ of the exogenous variables $\mathcal{U}$ fully determines the values of all other (endogenous) variables. We call a causal model with an acyclic causal network recursive because the exogenous variables determine the values of the endogenous variables in a recursive manner. 
As Halpern explains, some endogenous variables only depend on exogenous variables, we call them first-level variables \cite{halpern2016actual}. 
They get their value directly from the causal setting.
After that there are the second-level variables, the endogenous variables that depend on both the first-level variables and possibly on the exogenous variables.
Likewise the third-level variables depend on the second-level variables, and possibly on the exogenous and the first-level variables, and so on for higher levels.
We only focus on strongly recursive models in this paper.

\begin{example}\label{ex:causal network}
    The causal network for the causal model as described in Example \ref{ex:causal model} is given in Figure \ref{fig:ex causal network} (the exogenous variables are not drawn).
    The graph makes it easy to see that the causal model is recursive, i.e. the causal network does not contain cycles.
    \begin{figure}[b]
        \centering
        \setlength{\unitlength}{1cm}
        \begin{picture}(5,3.05)(-0.5,0)
            \put(0,0){\circle*{0.1}}
            \put(0,1.5){\circle*{0.1}}
            \put(2,0){\circle*{0.1}}
            \put(2,1.5){\circle*{0.15}}
            \put(2,3){\circle*{0.1}}
            \put(4,1.5){\circle*{0.1}}

            \put(0,0){\vector(1,0){1.95}}
            \put(0,1.5){\vector(4,3){1.95}}
            \put(0,1.5){\vector(1,0){1.95}}
            \put(0,1.5){\vector(4,-3){1.95}}
            \put(2,0){\vector(4,3){1.95}}
            \put(2,0){\vector(0,1){1.45}}
            \put(2,3){\vector(4,-3){1.95}}
            \put(4,1.5){\vector(-1,0){1.95}}
            

            \put(-0.15,0){\makebox(0,0)[r]{{$Att$}}}
            \put(-0.15,1.5){\makebox(0,0)[r]{{$O$}}}
            \put(2.15,-0.05){\makebox(0,0)[l]{{$HD$}}}
            \put(2,1.65){\makebox(0,0)[b]{\Large{$Col$}}}
            \put(1.85,3.05){\makebox(0,0)[r]{{$ODS$}}}
            \put(4.15,1.5){\makebox(0,0)[l]{{$DA$}}}
        \end{picture}
        \caption{The causal network for the causal model for the semi-autonomous vehicle example described in Example \ref{ex:causal model}.}
        \label{fig:ex causal network}
        \Description{A graph depicting the causal network of the causal model corresponding to the running semi-automated vehicle example. The graph has 6 nodes, each corresponding to one of the variables of the causal model, Att, O, HD, ODS, DA, and Col. Att has an edge to HD, O has edges to HD, Col and ODS. HD has an edge to DA and Col, ODS has an edge to DA, and DA has an edge to Col. There are no other edges.}
    \end{figure}
%
    We can also see the variable levels. 
    $O$ and $Att$ are first-level variables, they only depend on the exogenous variables.
    $HD$ and $ODS$ only depend on $O$ and $Att$ and hence are second-level variables. 
    $DA$ is a third-level variable, as it depends on second-level variables, and $Col$ is a fourth-level variable, as it depends on both $DA$ and lower-level variables.
\end{example}

Given a signature $\mathcal{S} = (\mathcal{U},\mathcal{V},\mathcal{R})$, a formula of the form $X=x$, for $X\in \mathcal{V}$ and $x\in\mathcal{R}(X)$ is called a \emph{primitive event} \cite{halpern2005causes,halpern2016actual}. These primitive events can be combined with the Boolean connectives $\wedge, \vee$ and $\neg$, to form a \emph{Boolean combinations of primitive events} \cite{halpern2005causes,halpern2016actual}. 
We follow Halpern and use $\csetting \models \phi$ to denote that formula $\phi$ holds given the values of all variables determined by the causal setting $\csetting$ (see~\cite{halpern2016actual} for details).
A \emph{causal formula} has the form $[Y_1\leftarrow y_1, ... , Y_k \leftarrow y_k]\varphi$, where $\varphi$ is a Boolean combination of primitive events, $Y_1,...,Y_k \in \mathcal{V}$ with
$Y_i = Y_j$ if and only if $i = j$, and $y_i\in\mathcal{R}(Y_i) \text{ for all } 1 \leq i \leq k$. Such a formula can be shortened to $[\myvec{Y}\leftarrow\mathbf{y}]\varphi$, and when $k=0$ it is written as just $\varphi$ \cite{halpern2005causes}.
$\csetting \models [\myvec{Y}\leftarrow\mathbf{y}](X = x)$ says that after an intervention that sets all variables of $\myvec{Y}$ to $\mathbf{y}$, it must be the case that $X = x$ holds in the causal setting $\csetting$ (see~\cite{halpern2005causes,halpern2016actual} for more details).
We call $\mathbf{y}$ a \emph{setting} for the variables in $\mathbf{Y}$.
We now have the necessary background to give the modified HP definition of causality:
\begin{definition}[modified HP Definition \cite{halpern2016actual}]\label{def:HP}
$\myvec{X} = \myvec{x}$ is an \emph{actual cause} of $\vphi$ in the causal setting $(\mathcal{M},\mathbf{u})$ if the following 3 conditions hold:
\begin{itemize}
    \item[\textnormal{AC1.}]  $(\mathcal{M},\mathbf{u})\models \myvec{X} = \myvec{x}$ and $(\mathcal{M},\mathbf{u})\models \vphi$;
    \item[\textnormal{AC2.}] There is a set $\myvec{W}$ of variables in $\mathcal{V}$ and a setting $\myvec{x}'$ of variables in $\myvec{X}$ s.t. if $(\mathcal{M},\mathbf{u})\models \myvec{W} = \mathbf{w}^\ast$, then $(\mathcal{M},\mathbf{u}) \models [\myvec{X} \leftarrow \myvec{x}', \myvec{W} \leftarrow \mathbf{w}^\ast] \neg \vphi$.
    \item[\textnormal{AC3.}] $X$ is minimal; there is no strict subset $\myvec{X}'$ of $\myvec{X}$ s.t. $\myvec{X}' = \myvec{x}'$ satisfies \textnormal{AC1} and \textnormal{AC2}, where $\myvec{x}'$ is the restriction of $\mathbf{x}$ to the variables in $\myvec{X}'$.
\end{itemize}
\end{definition}
If $\myvec{W} = \emptyset$, we call $\myvec{X} = \myvec{x}$ a \emph{but-for cause} of $\vphi$.

\begin{example}\label{ex:causes}
    Consider our semi-autonomous vehicle example again. 
    Take the causal setting where $\mathbf{u} = (1,0)$, i.e. $U_O = 1$, there is an obstacle on the route, and $U_{Att} = 0$, the human driver is not paying attention.
    Following the equations provided in Example~\ref{ex:causal model}, we have that $\csetting \models O \wedge \neg Att \wedge HD \wedge ODS \wedge \neg DA \wedge \neg Col$. 
    We want to know which agent was the cause of there being no collision.
    It turns out that both $ODS$ and $\neg DA$ are but-for causes of $\neg Col$, i.e., $\csetting \models [ODS \leftarrow 0] Col$ and $\csetting \models [DA \leftarrow 1] Col$.
    After all, if we intervene by turning off the object detection system $ODS$ (setting its value to $0$ in our model, i.e., replacing equation $ODS=1$ in our model with $ODS=0$, which is formally represented as $[ODS \leftarrow 0]$), the driving assistant $DA$ will no longer get a signal that there is an obstacle on the route.
    This gives $DA =1$, meaning that the driving assistant will not brake. 
    Because the human driver is distracted in this setting, they will also not brake, and so there will be a collision. 
    Similarly we can also directly intervene on the driving assistant by turning it off (setting its value to $1$, not braking, in our model by replacing the equation for $DA$ with $DA:=1$, represented by $[DA \leftarrow 1]$) and there will be a collision as well.
\end{example}

The aim of this work is to connect this concept of structural causal models and causality to concurrent game structures. We use the following definition of concurrent game structures:
\begin{definition}[Concurrent Game Structures \cite{alur2002alternating}]\label{def:concurrent game structures}
    A \emph{concurrent game structure} (CGS) is a tuple $GS = \langle N, Q, d, \delta, \Pi, \pi \rangle$ with the following components:
    \begin{itemize}
        \item A natural number $N \geq 1$ of agents. We identify the \emph{agents} with the numbers $1, . . . , N$.
        \item A finite set $Q$ of states. 
        \item For each agent $a \in \{1, . . . , N\}$ and each state $q \in Q$, a natural number $d_a (q) \geq 1$ of moves available at state $q$ to agent $a$. We identify the moves of agent $a$ at state $q$ with the numbers $1, . . . , d_a (q)$. For each state $q \in Q$, a move vector at $q$ is a tuple $\langle j_1, . . . , j_N\rangle$  such that $1 \leq j_a \leq d_a (q)$ for each agent $a$. Given a state $q \in Q$, we write $D(q)$ for the set $\{1, . . . , d_1 (q)\} \times \dots \times \{1, . . . , d_N (q)\}$ of \emph{move vectors}. The function $D$ is called \emph{move function}.  
        \item For each state $q \in Q$ and each move vector $\langle j_1, . . . , j_N\rangle \in D(q)$, a state $\delta(q, j_1, . . . , j_N ) \in Q$ that results from state $q$ if every agent $a \in \{1, . . . , N\}$ chooses move $j_a$ . The function $\delta$ is called \emph{transition function}.
        \item A finite set $\Pi$ of \emph{propositions}. 
        \item For each state $q \in Q$, a set $\pi (q) \subseteq \Pi$ of propositions true at q. The function $\pi$ is the \emph{labelling function}. 
    \end{itemize}
\end{definition}


When we have a CGS, we can reason about what the optimal actions for a coalition of agents would be in a certain situation. We often use the concept of strategies for this.


\begin{definition}[Strategy in Concurrent Game Structures \cite{alur2002alternating}]
    Given a concurrent game structure $S = \langle N, Q, d, \delta, \Pi, \pi \rangle$, a \emph{strategy} for agent $a\in\set{1,...,N}$ is a function $f_a$, that maps any (non-empty) finite sequence $\lambda$ of states in $Q$ to an action the agent can take at the last state of the sequence. I.e. if $q$ is the last state of $\lambda$, then $f_a(\lambda) \leq d_a(q)$.
    We write $F_A = \{f_a\ | \ a \in A\}$ for a set of strategies of the agents in $A\subseteq \set{1,...,N}$. 
\end{definition}

We now have all preliminaries ready to move on and combine causality with concurrent game structures.


\section{From Causal Model to CGS}
\label{sec:cgs from cm}
The goal of this paper is to define a systematic approach to generate a causal CGS based on a strongly recursive structural causal model. The motivation is that we want to compare the strategic ability of coalitions of agents to realise outcomes to causes in the causal model.
Similar translations have been attempted by \cite{gladyshev2023dynamics,baier2021game-theoretic} and \cite{hammond2023ReasoningCausalityGames}.

Gladyshev et al. make, like us, a distinction between agent and environment variables, and they also construct a CGS that takes the causal structure between agents' decision and environment variables into account \cite{gladyshev2023dynamics} . 
However, they take a `zoomed out' approach to the causal model by considering every state in the CGS as a causal model. 
In contrast, in this paper, we are interested in the specific variable values, which we will consider as specific actions in strategic setting.
Another difference with our work is that they do not look at the relationship between causality in the original causal model and strategies in the CGS.

A more similar approach to ours was defined by Baier et al. \cite{baier2021game-theoretic}, but they use extensive form games rather than CGS, and  do not distinguish between agent and environment variables. 
Furthermore, while they do show a result relating actual causality in the causal model to some type of strategy in their extensive form game, they only do this for but-for causes, where we consider the modified HP definition as well. 

Hammond et al. translate the causal model to a multi-agent influence diagram (MAID) that includes utility variables, with the primary goal of studying rational outcomes of the grand coalition \cite{hammond2023ReasoningCausalityGames}.
They hence take a game-theoretic approach, where we take a logic-based approach by focusing on strategic abilities of coalitions of agents.
Nevertheless, we could also apply a game-theoretic analysis to our model, by extending our CGS to include utility variables.
This is however beyond the scope of this paper.


\subsection{Defining a Causal CGS}
In this section we will propose a systematic approach to generate a causal concurrent game structure based on a strongly recursive structural causal model.
We will use the notion of first-level, second-level and higher-level variables as explained in the previous section to determine in which order the agents of the causal model will get to take actions.
For this we define the notion of agent rank:
\begin{definition}
    An \emph{agent ranking function} of a causal model $\mathcal{M}$ is a function $\rho: \mathcal{V} \rightarrow \set{0,...,n}$, where $n$ is the number of distinct variable levels for agent variables in $\mathcal{M}$, such that for all $A,B \in V_a$, $\rho(A) > \rho(B) > 0$ if and only if the variable level of $A$ is higher than the variable level of $B$, and $\rho(A) = \rho(B)$ if and only if $A$ and $B$ have the same variable level. 
    For all $X \in V_e$, $\rho(X) = \rho(A) - 1$ if $\exists A \in V_a$ such that the variable level of $X$ is lower or equal to the variable level of $A$, and there is no $B \in V_a$ that has a variable level between $X$ and $A$. If such an $A$ does not exist, i.e. if the variable level of $X$ is higher than the variable level of all $A \in V_a$, then $\rho(X) = n$.
    The \emph{agent rank} of a variable $A \in V_a$ is $\rho(A)$.
\end{definition}

\begin{example}\label{ex:agent rank}
    In the semi-automated vehicle example we say that $HD$, $ODS$ and $DA$ are the agent variables. 
    We have that $n = 2$ as $HD$ and $ODS$ are both second-level variables and $DA$ is a third level variable as Example \ref{ex:causal network} discusses. 
    There are hence $2$ distinct variable levels for the agent variables.
    From this, it follows that $\rho(HD) = \rho(ODS) = 1$ and $\rho(DA) = 2$, as the variable level of $DA$ is higher than that of $HD$ and $ODS$ and their agent rank needs to be higher than $0$ and maximally $2$.
    For the environment variables, we have that $\rho(O) = \rho(Att) = \rho(HD) - 1 = 0$, because there are no first-level agent variables, so we need a second-level agent variable like $HD$.
    Finally, we have that $\rho(Col) = 2$, since the variable level of $Col$ is $4$ which is higher than all agent variable levels, and hence the agent rank of $Col$ will be the maximum of $2$.
\end{example}

We will first define several components of the causal CGS separately before putting them all together.
From now on, we will assume that all causal models are recursive and have variables which can only attain finitely many values. Moreover we assume that a set of agent variables $V_a \subseteq \mathcal{V}$ is given. 

\begin{definition}[States of a causal CGS]\label{def:states causal CGS}
Given a causal setting $(\mathcal{M},\mathbf{u})$, let $n = \max_{Y\in V_a} \rho(Y)$ be the maximum value of the agent ranks for the agents in $V_a$ and let $m_i = \prod_{\substack{Y \in V_a,\\ \rho(Y) \leq i}} |\mathcal{R}(Y)|$  be the number of possible combinations of action values for agents with an agent rank of no more than $i$. 
The set of \emph{states of a causal CGS}, $Q$, generated based on  $(\mathcal{M},\mathbf{u})$, is given by:
    \begin{equation*}
        Q = \set{q_{0,0}} \cup \set{q_{i,j}\ |\ 1 \leq i \leq n \text{ and }  0 \leq j < m_i }.
    \end{equation*}
\end{definition}
We call $q_{0,0}$ the \emph{starting state} of the causal CGS.
Later, we will see that the evaluation in a state $q_{i,j}$ follows from the actions of agents whose agent variables have agent rank $i$ or less.
\begin{example}\label{ex:states causal CGS}
    We will use the causal model for the semi-automated vehicle example to define a causal CGS (see Figure \ref{fig:ex causal network}). 
    See Example \ref{ex:agent rank} for the agent rank of all variables of the causal model.
    We start with the setting $(\mathcal{M},\mathbf{u})$ with $\mathbf{u} = (U_{O} = 1,U_{Att} = 0)$.
    The set of states is then
    $Q = \{q_{0,0},q_{1,0},q_{1,1}, q_{1,2},q_{1,3},q_{2,0},q_{2,1},q_{2,2},q_{2,3},q_{2,4},$\\$q_{2,5},q_{2,6},q_{2,7}\}.$
    Note that: \newline
    $\prod_{Y \in V_a, \rho(Y) \leq 1} |\mathcal{R}(Y)| = \prod_{Y \in \set{HD,ODS}} |\mathcal{R}(Y)| = |\set{0,1}\times \set{0,1}| = 4$ and
    $\prod_{Y \in V_a, \rho(Y) \leq 2} |\mathcal{R}(Y)| = \prod_{Y \in \set{HD, ODS, DA}} |\mathcal{R}(Y)|  = 8$, so for $i = 1$, we have $j \in \{0,\ldots,3\}$ and for $i = 2$, we have $j \in \{0,\ldots,7\}$.
    These are all the states, because the maximum value of the agent rank $\rho$ is $2$.
\end{example}

We will now define the agent actions in those states.

\begin{definition}[Actions in a causal CGS]\label{def:actions causal CGS}
    Given a causal setting $(\mathcal{M},\mathbf{u})$ and $Q$ the corresponding set of states as defined by Definition \ref{def:states causal CGS}. The possible \emph{actions} for an agent $k \in \set{1,...,N}$ in a state $q_{i,j} \in Q$ are $d_k(q_{i,j}) = \mathcal{R}(A_k)$, where $A_k$ is the agent variable controlled by agent $k$, and $\rho(A_k) = i+1$. Otherwise $d_k(q_{i,j}) = \set{0}$.
\end{definition}

The intuition behind this definition is that agent variables that are earlier on a causal path will earlier get to take an action as the agent variables later on a causal path depend on them. The order of agent variables on a causal path can be seen as representing a protocol that determines when each agent has to take its action.   
We write $a_k$ to denote an action of agent $k \in N$ and $\mathbf{a}_{i,j} = \langle a_1,...,a_N\rangle$ to denote an action profile taken in a certain state $q_{i,j}$, i.e., all actions taken by all agents in state $q_{i,j}$. It is important to note that for a given index $i$ all states $q_{i,j}$ have the same action profiles that can be taken in them, regardless of the value of $j$. We denote this set with $\mathbf{A}_i$. 
Instead of $d_k$ for agent $k$, we will sometimes write $d_{A_k}$ for the agent variable $A_k$ corresponding to agent $k$.

\begin{example}\label{ex:actions causal CGS}
    We continue with the situation as in Example \ref{ex:states causal CGS}.
    The available actions for each agent in each state are: \begin{equation*}
        \begin{array}{ll}
            d_{{HD}}(q_{0,0}) = d_{{ODS}}(q_{0,0}) = \set{0,1}, & d_{{DA}}(q_{0,0}) = \emptyset, \\
            d_{{HD}}(q_{1,j}) = d_{{ODS}}(q_{1,j}) = \emptyset,  & d_{{DA}}(q_{1,j}) = \set{0,1}, \\
            \forall j \in \set{0,\ldots,3},  \text{ and} & \\
            d_{{HD}}(q_{2,j}) = d_{{ODS}}(q_{2,j}) = \emptyset,  & d_{{DA}}(q_{2,j}) = \emptyset, \\
            \forall j \in \set{0,\ldots,7}.&
        \end{array}
    \end{equation*}
\end{example}

These actions must of course lead to transitions to new states. 
\begin{definition}[Transitions in a causal CGS]\label{def:transitions causal CGS}
    Given a causal setting $(\mathcal{M},\mathbf{u})$, $Q$ the corresponding set of states as defined by Definition \ref{def:states causal CGS} and actions as defined by Definition \ref{def:actions causal CGS}, the state following from the action profile $\mathbf{a}_{i,j} \in \mathbf{A}_i$, with $i < \max_{X\in V_a} \rho(X)$, is given by the transition function $\delta$, where $\delta(q_{i,j},\mathbf{a}_{i,j}) = q_{i+1,j'}$ and $|\mathbf{A}_i| \cdot j \leq j'\leq |\mathbf{A}_i| \cdot (j+1) -1$, under the condition that if $\mathbf{a}_{i,j} \neq \mathbf{a}'_{i,j}$, then $\delta(q_{i,j},\mathbf{a}_{i,j}) \neq \delta(q_{i,j}, \mathbf{a}'_{i,j})$.
    If $i = \max_{X\in V_a} \rho(X)$, we define $\delta(q_{i,j},\mathbf{a}_{i,j}) = q_{i,j}$. In this case, there is only one possible action profile $\mathbf{a}_{i,j}$ consisting of only the $0$ action. 
\end{definition}
This definition simply says that every unique action profile in a state leads to a unique new state. This leads to the causal CGS having a tree structure. It is impossible to return to an earlier state and every node can only branch out
\begin{example}\label{ex:transitions causal CGS}
    Continuing with our running example, we will write $\langle 1, 0, 0 \rangle$ for the action profile $\langle HD = 1, ODS = 0, DA = 0 \rangle$. 
    We get that the transitions are:
    \newline 
    \begin{equation*}
        \begin{array}{ll}
            \delta(q_{0,0}, \langle 0, 0, 0\rangle) = q_{1,0},  & 
            \delta(q_{0,0}, \langle 0, 1, 0\rangle) = q_{1,1}, \\
            \delta(q_{0,0}, \langle 1, 0, 0\rangle) = q_{1,2},  & 
            \delta(q_{0,0}, \langle 1, 1, 0\rangle) = q_{1,3}, \\
            \delta(q_{1,0},\langle 0, 0, 0\rangle) = q_{2,0}, &
            \delta(q_{1,0}, \langle 0, 0, 1\rangle) = q_{2,1}, \\
            \delta(q_{1,1},\langle 0, 0, 0\rangle) = q_{2,2}, &
            \delta(q_{1,1}, \langle 0, 0, 1\rangle) = q_{2,3}, \\
            \delta(q_{1,2},\langle 0, 0, 0\rangle) = q_{2,4}, &
            \delta(q_{1,2}, \langle 0, 0, 1\rangle) = q_{2,5}, \\
            \delta(q_{1,3},\langle 0, 0, 0\rangle) = q_{2,6}, &
            \delta(q_{1,3}, \langle 0, 0, 1\rangle) = q_{2,7}, \\
            \delta(q_{2,j},\langle 0, 0, 0\rangle) = q_{2,j} & \forall j \in \set{0,\ldots,7}.
        \end{array}
    \end{equation*}
\end{example}

Now that we have states, actions and transitions, we just need the evaluations of the states.
The evaluation of a state will depend on an initial causal setting and the actions the agents have taken up to this state. The agents fully determine the values of the agent variables, the environment variables follow from these values and the context that was used to define the causal CGS. 

\begin{definition}[Evaluation of states in a causal CGS]\label{def:evaluations in a causal CGS}
    Given a causal setting, $(\mathcal{M},\mathbf{u})$, the set of all possible propositions for the generated causal CGS is $\Pi = \{ X= x \ | \ X \in \mathcal{V}, x \in \mathcal{R}(X)\}$.
    The valuation of each state $q_{i,j} \in Q$, with $Q$ the set of states of the causal CGS according to Definition \ref{def:states causal CGS}, is defined recursively by the \emph{labelling function} $\pi$, as:
    \begin{equation*}
        \begin{array}{ll}
            \pi(q_{0,0})  &= \set{Y = y \ | \ \csetting \models Y = y} \\
            \pi(\delta(q_{i,j}, \myvec{a}_{i,j})) &= \set{Y = y \ | \ \csettingint{\myvec{X}_{i,j}\leftarrow \myvec{x}_{i,j}, \myvec{A}_{i,j} \leftarrow \myvec{a}_{i,j}}\models Y = y},
        \end{array}
    \end{equation*}
    where $\myvec{a}_{i,j}$ is an action profile for state $q_{i,j}$, $\myvec{A}_{i,j} \leftarrow \myvec{a}_{i,j} := \{A_k\leftarrow a_k \ | \ A_k \in V_a, \rho(A_k) = i+1 \text{ and } a_k \in \myvec{a}_{i,j}\}$ is an intervention constructed based on action profile $\myvec{a}_{i,j}$, and $\myvec{X}_{i,j}\leftarrow \myvec{x}_{i,j}$ is recursively defined by: $\myvec{X}_{i+1,j'} \leftarrow \myvec{x}_{i+1,j'}:= \myvec{X}_{i,j}\leftarrow \myvec{x}_{i,j} \cup \myvec{A}_{i,j}\leftarrow \myvec{a}_{i,j}$, if $\delta(q_{i,j}, \myvec{a}_{i,j}) = q_{i+1,j'}$ with $\myvec{X}_{0,0} \leftarrow \myvec{x}_{0,0}= \emptyset$.
\end{definition}

Definition \ref{def:evaluations in a causal CGS} says that an agent action leads to an intervention on the causal setting the causal CGS was based upon. 
We can see $\mathbf{A}_{i,j} \leftarrow \mathbf{a}_{i,j}$ as the intervention that directly follows from the agent action(s) taken in the state $q_{i,j}$, $\mathbf{X}_{i,j} \leftarrow \mathbf{x}_{i,j}$ stores the previous interventions that were made leading up to the state $q_{i,j}$.
We will illustrate this in the following example.
\begin{example}\label{ex:evaluations CGS}
    We continue with the situation as in Example \ref{ex:transitions causal CGS}. 
    We start with the causal setting where $U_O = 1$ and $U_{Att} = 0$, so $\pi(q_{0,0}) = \set{O, \neg Att, HD, ODS, \neg DA, \neg Col}$.
    To determine $\pi(q_{1,0}) = \pi(\delta(q_{0,0}, \langle 0,0,0 \rangle ))$, we need $\myvec{A}_{0,0} \leftarrow \myvec{a}_{0,0} = \set{HD \leftarrow 0, ODS \leftarrow 0}$. This gives us that $\pi(q_{1,0}) = \set{Y = y \ | \ \csettingint{HD \leftarrow 0, ODS \leftarrow 0} \models Y = y} $ $= \set{O, \neg Att, \neg HD, \neg ODS, \neg DA, \neg Col}$.
    Similarly we can determine that $\pi(q_{1,1}) = \{O, \neg Att, \neg HD,  ODS, \neg DA, \neg Col\}$, $\pi(q_{1,2}) = \{O, \neg Att, $\\$ HD, \neg ODS, DA, Col\}$ and $\pi(q_{1,3}) = \set{O, \neg Att, HD,  ODS, \neg DA, \neg Col}$.

    Let us now look at $\pi(q_{2,1}) = \pi(\delta(q_{1,0}, \langle 0,0,1\rangle))$.
    We need $\myvec{X}_{1,0} \leftarrow \myvec{x}_{1,0} = (\myvec{X}_{0,0} \leftarrow \myvec{x}_{0,0} \ \cup \ \myvec{A}_{0,0} \leftarrow \myvec{a}_{0,0}) =  \emptyset \ \cup \set{HD \leftarrow 0, ODS \leftarrow 0}$ as we determined above.
    The new $\myvec{A}_{1,0} \leftarrow \myvec{a}_{1,0} = \set{DA \leftarrow 1}$ and so $\pi(q_{2,1}) = \set{Y = y \ | \ \csettingint{HD \leftarrow 0, ODS \leftarrow 0, DA \leftarrow 1} \models Y = y } = \{O, \neg Att, \neg HD, \neg ODS, DA, \neg Col\}$.
    The valuations for the other states are determined similarly (and are shown in Figure \ref{fig:causal cgs vehicle}).

\end{example}

Now that we have these four definitions, we can give the full definition of a causal CGS.

\begin{definition}[Causal CGS]\label{def:causal CGS}
    Given a causal setting, $(\mathcal{M},\mathbf{u})$, a \emph{causal concurrent game structure} is defined as a tuple $GS = \langle N, Q, d,$\\$ \delta, \Pi, \pi\rangle$ where $N = |V_a|$, every agent only controls one agent variable, $Q$ is a set of states, as defined by Definition \ref{def:states causal CGS}. For every agent $k \in \set{1,...,N}$, $d_k(q_{i,j})$ gives the moves available to this agent in state $q_{i,j} \in Q$, as given by Definition \ref{def:actions causal CGS}.
    The transition function $\delta$ is defined as in Definition \ref{def:transitions causal CGS}.
    The set of possible propositions $\Pi$ and the valuation function $\pi$ are given by Definition \ref{def:evaluations in a causal CGS}.
\end{definition}

We can now add the results of the previous examples together and give a full causal CGS for the semi-automated vehicle example.

\begin{example}\label{ex:cgs rock-throwing}
     Using Definition \ref{def:causal CGS}, we define $N = |V_a| = |\set{HD, ODS, DA}| = 3$. This gives us a full causal CGS, illustrated in Figure \ref{fig:causal cgs vehicle}.

     \begin{figure}[ht]
    \centering
    \setlength{\unitlength}{0.9cm}
    \begin{picture}(7,7.3)(-0.25,0.8)

        \put(0,4.5){\circle{0.8}}
        \put(2,2){\circle{0.8}}
        \put(2,4){\circle{0.8}}
        \put(2,5){\circle{0.8}}
        \put(2,7){\circle{0.8}}
        \put(4,1){\circle{0.8}}
        \put(4,2){\circle{0.8}}
        \put(4,3){\circle{0.8}}
        \put(4,4){\circle{0.8}}
        \put(4,5){\circle{0.8}}
        \put(4,6){\circle{0.8}}
        \put(4,7){\circle{0.8}}
        \put(4,8){\circle{0.8}}
        
        \put(-0.23,4.5){\makebox(0,0)[l]{\footnotesize{$q_{0,0}$}}}
        \put(1.77,7){\makebox(0,0)[l]{\footnotesize{$q_{1,0}$}}}
        \put(1.77,5){\makebox(0,0)[l]{\footnotesize{$q_{1,1}$}}}
        \put(1.77,4){\makebox(0,0)[l]{\footnotesize{$q_{1,2}$}}}
        \put(1.77,2){\makebox(0,0)[l]{\footnotesize{$q_{1,3}$}}}
        \put(3.77,8){\makebox(0,0)[l]{\footnotesize{$q_{2,0}$}}}
        \put(3.77,7){\makebox(0,0)[l]{\footnotesize{$q_{2,1}$}}}
        \put(3.77,6){\makebox(0,0)[l]{\footnotesize{$q_{2,2}$}}}
        \put(3.77,5){\makebox(0,0)[l]{\footnotesize{$q_{2,3}$}}}
        \put(3.77,4){\makebox(0,0)[l]{\footnotesize{$q_{2,4}$}}}
        \put(3.77,3){\makebox(0,0)[l]{\footnotesize{$q_{2,5}$}}}
        \put(3.77,2){\makebox(0,0)[l]{\footnotesize{$q_{2,6}$}}}
        \put(3.77,1){\makebox(0,0)[l]{\footnotesize{$q_{2,7}$}}}

        \put(0.25,4.8){\vector(3,4){1.45}}
        \put(0.4,4.6){\vector(4,1){1.2}}
        \put(0.4,4.4){\vector(4,-1){1.2}}
        \put(0.25,4.2){\vector(3,-4){1.45}}
        \put(2.35,7.2){\vector(2,1){1.3}}
        \put(2.4,7){\vector(1,0){1.2}}
        \put(2.35,5.2){\vector(2,1){1.3}}
        \put(2.4,5){\vector(1,0){1.2}}
        \put(2.4,4){\vector(1,0){1.2}}
        \put(2.35,3.8){\vector(2,-1){1.3}}
        \put(2.4,2){\vector(1,0){1.2}}
        \put(2.35,1.8){\vector(2,-1){1.3}}

        \put(0.45,5.5){\rotatebox{53}{\footnotesize{$\langle 0, 0, 0\rangle$}}}
        \put(0.4,4.7){\rotatebox{14}{\footnotesize{$\langle 0, 1, 0\rangle$}}}
        \put(0.4,4.13){\rotatebox{-14}{\footnotesize{$\langle 1, 0, 0\rangle$}}}
        \put(0.45,3.4){\rotatebox{-53}{\footnotesize{$\langle 1, 1, 0\rangle$}}}
        \put(2.5,7.4){\rotatebox{26.5}{\footnotesize{$\langle 0, 0, 0\rangle$}}}
        \put(3,7){\makebox(0,0)[b]{\footnotesize{$\langle 0, 0, 1\rangle$}}}
        \put(2.5,5.4){\rotatebox{26.5}{\footnotesize{$\langle 0, 0, 0\rangle$}}}
        \put(3,5){\makebox(0,0)[b]{\footnotesize{$\langle 0, 0, 1\rangle$}}}
        \put(3,3.95){\makebox(0,0)[t]{\footnotesize{$\langle 0, 0, 0\rangle$}}}
        \put(2.5,3.45){\rotatebox{-26.5}{\footnotesize{$\langle 0, 0, 1\rangle$}}}
        \put(3,1.95){\makebox(0,0)[t]{\footnotesize{$\langle 0, 0, 0\rangle$}}}
        \put(2.5,1.45){\rotatebox{-26.5}{\footnotesize{$\langle 0, 0, 1\rangle$}}}

        \put(-0.5,4.5){\makebox(0,0)[r]{\scriptsize{$\set{O,\neg Att}$}}}
        \put(1.7,7.5){\makebox(0,0)[b]{\scriptsize{$\set{\neg HD, \neg ODS}$}}}
        \put(1.73,5.5){\makebox(0,0)[b]{\scriptsize{$\set{\neg HD, ODS}$}}}
        \put(1.73,3.5){\makebox(0,0)[t]{\scriptsize{$\set{ HD, \neg ODS}$}}}
        \put(1.73,1.5){\makebox(0,0)[t]{\scriptsize{$\set{ HD, ODS}$}}}
        
        \put(4.43,8){\makebox(0,0)[l]{\scriptsize{$\set{O,\neg Att, \neg HD, \neg ODS, \neg DA, \neg Col}$}}}
        \put(4.43,7){\makebox(0,0)[l]{\scriptsize{$\set{O,\neg Att, \neg HD, \neg ODS, DA, \neg Col}$}}}
        \put(4.43,6){\makebox(0,0)[l]{\scriptsize{$\set{O,\neg Att, \neg HD, ODS, \neg DA, \neg Col}$}}}
        \put(4.43,5){\makebox(0,0)[l]{\scriptsize{$\set{O,\neg Att, \neg HD, ODS, DA, \neg Col}$}}}
        \put(4.43,4){\makebox(0,0)[l]{\scriptsize{$\set{O,\neg Att, HD, \neg ODS, \neg DA, \neg Col}$}}}
        \put(4.43,3){\makebox(0,0)[l]{\scriptsize{$\set{O,\neg Att,  HD, \neg ODS, DA, Col}$}}}
        \put(4.43,2){\makebox(0,0)[l]{\scriptsize{$\set{O,\neg Att, HD, ODS, \neg DA, \neg Col}$}}}
        \put(4.43,1){\makebox(0,0)[l]{\scriptsize{$\set{O,\neg Att,  HD, ODS, DA, Col}$}}}
        
    \end{picture}
    \caption{The causal CGS of the semi-automated vehicle example. We only show the initial values of the variables of agent rank $0$ in the starting state. In the middle states we only show the variables with agent rank corresponding to that state. We also do not show the transitions to the same state in the leaf-states.}
    \label{fig:causal cgs vehicle}
    \Description{A graph depicting the causal concurrent game structure for our running semi-automated vehicle example. The graph has a tree structure with maximal depth 2, there are 4 nodes of depth 1 and 8 leaf nodes. The root has 4 edges leaving it, the depth 1 nodes each have 2 edges leaving it. The root node is named q with the subscript 0,0, it is labelled with the propositions O and negation of Att. The four edges leaving the root node are labelled respectively 0,0,0; 0,1,0; 1,0,0; and 1,1,0. The four depth 1 nodes are named q with the subscripts 1,0 up until 1,3. They are labelled with the propositions: negation of HD, negation of ODS; negation of HD, ODS; HD, negation of ODS; and HD, ODS, respectively. For each of the depth 1 nodes, the edges leaving it are labelled 0,0,0 and 0,0,1. The leaf nodes are named q with the subscript 2,0 up and until 2,7. They are each respectively labelled with the propositions: O, negation of Att, negation of HD, negation of ODS, negation of DA, negation of Col; O, negation of Att, negation of HD, negation of ODS, DA, negation of Col;O, negation of Att, negation of HD, ODS, negation of DA, negation of Col; O, negation of Att, negation of HD, ODS, DA, negation of Col; O, negation of Att, HD, negation of ODS, negation of DA, negation of Col; O, negation of Att, HD, negation of ODS, DA, Col; O, negation of Att, HD, ODS, negation of DA, negation of Col; O, negation of Att, HD, ODS, DA, Col.}
\end{figure}
\end{example}



\subsection{Properties of Causal Concurrent Game Structures}
We already mentioned that a causal CGS has a tree structure. In the rest of this paper, we will call states $q_{i,j}$, with $i = \max_{X\in \mathcal{V}} \rho(X)$, the \emph{leaf-states}. 
We will call actions in states where an agent does not control a variable, i.e. $a_k = 0$, when $d_k(q_{i,j}) = \set{0}$, with $\rho(X) \neq i + 1$, \emph{no-op actions}. It is also useful to define an \emph{action path} for a state $q_{i,j}$, that contains all the non no-op actions that led to the state. In other words, the action path contains only the actions that agents took in a state where they could actually choose an action. We will denote this sequence of actions as $\alpha[q_{i,j}]$. 
Formally, for $0 \leq k \leq N$, an action $a_k$ is in this set of actions  $\alpha[q_{i,j}]$ if and only if $\rho(A_k)\leq i$ and there exists an action profile $\mathbf{a}_{i',j'}$, containing $a_k$, such that $q_{i',j'} \in \lambda[q_{i,j} , i]$  (the history of $q_{i,j}$) and $\delta(q_{i',j'} , \mathbf{a}_{i',j'}) \in \lambda[q_{i,j} , i]$. In other words, an action is on the action path for a state $q_{i,j}$, if the state $q_{i',j'}$ in which the action is taken lies on the history of $q_{i,j}$, and the successor of $q_{i',j'}$ can be reached when taking this action.

Our first result is on the size of the causal CGS.
\begin{proposition}
Let $\mathcal{M} = (\mathcal{S},\mathcal{F})$ be a causal model. The size of the causal CGS generated by $\mathcal{M}$ is linear in the size of the extension of $\mathcal{F}$.
\end{proposition}
\begin{proof}
Consider a structural causal model $\mathcal{M} = (\mathcal{S},\mathcal{F})$. Observe that $\mathcal{F}$ specifies the value of each variable for all possible combinations of values of all other variables. Hence $\mathcal{F}$ corresponds to a table of size $|\mathcal{V}| \times \prod_{X \in \mathcal{V}} |\mathcal{R}(X)|$ (the number of cells), which is actually the extension of $\mathcal{F}$. We now show that the number of states in the causal CGS is $O( \prod_{Y \in V_a} |\mathcal{R}(Y)|)$.

By Definition \ref{def:states causal CGS} we have that the number of states of the causal CGS, is given by $|Q| = 1+ \sum_{i = 1}^{n} \prod_{\substack{Y \in V_a,\\ \rho(Y) \leq i}} |\mathcal{R}(Y)|$, where $n = \max_{Y\in V_a} \rho(Y)$. The number of leaf-states is hence given by $\prod_{\substack{Y \in V_a}} |\mathcal{R}(Y)| =: |R(V_a)|$.
The number of states for $i = n-1$ will be at most half $|R(V_a)|$, as there will be at least one variable of rank $n$ that is hence not included in $\prod_{\substack{Y \in V_a,\\ \rho(Y) \leq n-1}} |\mathcal{R}(Y)|$, and this variable will have at least two possible values. We can continue this argument until $i = 1$, which shows us that $|Q|$ is bounded by $1 + \frac{1}{2^{n-1}} |R(V_a)|+\dots +\frac{1}{2} |R(V_a)|+|R(V_a)| \leq 2 |R(V_a)|$. 
Hence the number of states in the causal CGS is $O( \prod_{Y \in V_a} |\mathcal{R}(Y)|)$.
Since a causal CGS is a tree and each state has at most one predecessor, the number of transitions (the size of $\delta$) is also 
$O( \prod_{Y \in V_a} |\mathcal{R}(Y)|)$, hence linear in the size of $\mathcal{F}$ in the original model.
\end{proof}


The statement in the following lemma is a direct consequence of the way the valuation of states is determined in a causal CGS. It states that a variable value cannot change in states corresponding to a higher agent rank than the agent rank of the variable itself.

\begin{lemma}\label{lem:no change after i}
    Let $GS$ be a causal CGS generated by the causal model $\mathcal{M}$. For any endogenous causal variable $X \in \mathcal{V}$ of $\mathcal{M}$, with $\rho(X) = i$, it holds that $(X = x) \in \pi(q_{i,j})$ for some state $q_{i,j}$ of $GS$, if and only if $(X = x) \in \pi(q_{i',j'})$ for all states $q_{i',j'}$ that are descendants of $q_{i,j}$.
\end{lemma}
\begin{proof}
    Let $(X = x) \in \pi(q_{i,j})$. 
    Variable values can change in a state due to interventions, but the only new interventions done in states descended from $q_{i,j}$ are interventions on variables with an agent rank higher than $i$. 
    $X$ has agent rank $i$, so by the definition of agent rank none of those variables can be ancestors of $X$. They are hence unable to influence the value of $X$. 
    Therefore $(X = x) \in \pi(q_{i',j'})$ for all states $q_{i',j'}$ descended from $q_{i,j}$.

    Now, let $(X=x)\in \pi(q_{i',j'})$ for all states $q_{i',j'}$ that are descended from $q_{i,j}$.
    The value of $X$ was not changed in any of those states, because the value of $X$ can only change due to an intervention on $X$ or an ancestor variable of $X$, so only due to variables of agent rank smaller or equal to $\rho(X)$.
    The only interventions on variables that happen in the descendants of $q_{i,j}$ are on variables of agent rank higher than $\rho(X)$, hence $X$ must have had the same value in $q_{i,j}$, i.e. $(X = x) \in \pi(q_{i,j})$.
\end{proof}

We define the notion of \emph{correspondence} to talk about how states in a causal CGS connect to a causal model.
\begin{definition}[Correspondence]\label{def:correspondence}
    We say that a state $q_{i,j}$ of a causal CGS \emph{corresponds} to a causal setting 
    $(\mathcal{M}^{\myvec{Y} \leftarrow \myvec{y}},\mathbf{u})$, where $\myvec{Y} \subseteq \mathcal{V}$, if for all causal variables $X$ of $\mathcal{M}$,
    $(X = x) \in \pi(q_{i,j})$ if and only if $(\mathcal{M}^{\myvec{Y} \leftarrow \myvec{y}},\mathbf{u}) \models X = x$.\footnote{So the causal variable $X$ has value $x$ in the causal setting $(\mathcal{M}^{\myvec{Y} \leftarrow \myvec{y}},\mathbf{u})$.}
\end{definition}
We will sometimes say that a causal setting $(\mathcal{M}^{\myvec{Y} \leftarrow \myvec{y}},\mathbf{u})$ corresponds to a state $q_{i,j}$ of a causal CGS and mean the same thing.
Note that the set $\myvec{Y}$ could also be empty. Hence the causal model $\mathcal{M}^{\myvec{Y} \leftarrow \myvec{y}}$ in Definition \ref{def:correspondence} could also be $\mathcal{M}$.

We can show that a leaf-state of a causal CGS corresponds to a causal setting $\csettingint{\myvec{Y} \leftarrow \myvec{y}}$, where $\myvec{Y} \leftarrow \myvec{y}$ depends on the action path that leads to the leaf-state. 
This connects the definition of causal CGS to the theory of causal models.
\begin{proposition}\label{prop:state correspondence}
    Let $GS$ be a causal CGS generated by a causal setting $\csetting$. If $q_{n,m}$ is a leaf-state of $GS$, then $q_{n,m}$ corresponds to the causal setting $(\mathcal{M}^{\myvec{Y}\leftarrow \myvec{y}}, \mathbf{u})$, where $\myvec{Y} \leftarrow \myvec{y} = \{A_k \leftarrow a_k \ | \ A_k \in V_a \text{ and } a_k \in \alpha[q_{n,m}]\}$, with $\alpha[q_{n,m}]$ the action path for $q_{n,m}$.
\end{proposition}
\begin{proof}
    By Definition \ref{def:evaluations in a causal CGS}, $(X = x) \in \pi(q_{n,m})$ if and only if $\csettingint{\myvec{X}_{i,j}\leftarrow \myvec{x}_{i,j}, \myvec{A} \leftarrow \myvec{a}}\models X = x$, 
    where $\myvec{A} \leftarrow \myvec{a}$ are the actions taken in the state before $q_{n,m}$, and $\myvec{X}_{i,j}\leftarrow \myvec{x}_{i,j}$ are all previously taken actions. Hence $\myvec{Y} \leftarrow \myvec{y} = (\myvec{A} \leftarrow \myvec{a}) \cup \myvec{X}_{i,j}\leftarrow \myvec{x}_{i,j}$ and the proposition is proven.
\end{proof}

This gives us a solid grasp on how a causal CGS relates to the causal model that generates it.
We will use this in the next section when we talk about the connection between agent strategies in a causal CGS and causality in this structural causal model.


\section{Causality in Causal CGS}
\label{sec:causality in CGS}
Now that we have defined causal concurrent game structures and shown what their states represent, it is time to look at how we can use them.
In this section, we will show some relations between causal CGS and the modified HP definition of actual causality, but we first introduce the notion of a causal strategy profile.

From now on, we will denote the set of all agents in a model by $\Sigma$. 
Specifically, for a causal CGS, $\Sigma = \set{k\ | \ X_k \in V_a}$. This set will also be called the \emph{grand coalition} at times.
We will use the notation $F_{X_k = x}$ to denote the strategy for agent $k$ where it takes action $x$ as its non no-op action.
Formally,
\[
    F_{X_k = x}(q_{i,j}) = \left\{ \begin{array}{lll}
        x & \text{if } &  \rho(X_k) = i+1 \\
         0 & \text{else}&
    \end{array}\right.
\]
For a set of agents $\myvec{X}$, we write $F_{\myvec{X} = \myvec{x}}$ to indicate the set of strategies $\{F_{X_k = x} \ | \ $ $ X_k \in \myvec{X}, x \in \myvec{x}\}$.
Let $F_A$ be a strategy for a set of agents $A$, and $F_B$ a strategy for a set of agents $B$. 
Following notation in \cite{Brihaye_DaCosta_Laroussinie_Markey_2008}, we will write $F_A \circ F_B$ to denote a strategy profile for the agents in $A \cup B$ that follows strategy $F_A$ for agents in $A$ and strategy $F_B$ for agents in $B \bs A$.



We define the causal strategy profile as a way to capture the `normal' behaviour of agents when they would follow the causal model.
\begin{definition}[Causal Strategy Profile]\label{def:complete causal strat profile}
Given a causal setting $(\mathcal{M},\mathbf{u})$ and the causal CGS generated by this setting.
Define the \emph{causal strategy profile} $F_\mathcal{M}$ as $F_\mathcal{M} = \set{F_{X_k}\ |\ k \in \Sigma}$, where $F_{X_k}(q_{i,j}) = 0$ if $\rho(X_k) \neq i+1$, and $F_{X_k}(q_{i,j}) = x_k$ otherwise, where $x_k$ is such that $(\mathcal{M},\mathbf{u}) \models [\myvec{X}\leftarrow\myvec{x}] X_k = x_k$, with $\myvec{X} = \set{X_{k'} \ | \ \rho(X_{k'})<\rho(X_k)}$ and $\myvec{x} = \set{x_{k'} \ | \ x_{k'} \in \alpha[q_{i,j}]}$.
\end{definition}
Recall that $\alpha[q_{i,j}]$ is the action path up to state $q_{i,j}$.
If we want an agent $k$ to follow a strategy $F_k$ and the rest of the agents to follow the causal strategy profile, we denote this as $F_k \circ F_\mathcal{M}$.
If a set of agents follows the causal strategy profile, that means that in every state, the agents take the actions that assign those values to the agent variables that they would also have gotten in the causal setting on which the causal CGS is based, given the actions of the other agents.

\begin{example}
    In the semi-automated vehicle example, given the setting where $U_O = 1$ and $U_{Att} = 0$, the causal strategy profile $F_\mathcal{M}$ is such that the human driver does not brake, but the obstacle detection system detects the obstacle.
    The driving assistant brakes in this case, but whenever one of the $HD$ or $ODS$ performs another action, $DA$ does not brake. The causal strategy profile for a causal CGS generated by this causal setting is given in Figure \ref{fig:causal cgs vehicle strategy}.
    \begin{figure}[h]
    \centering
    \setlength{\unitlength}{0.9cm}
    \begin{picture}(7,7.3)(-0.25,0.8)

        \multiput(0.25,4.8)(0.09,0.12){16}{\circle*{0.01}}
        \put(1.7,6.7){\vector(3,4){0}}
        \multiput(0.4,4.6)(0.12,0.03){10}{\circle*{0.01}}
        \put(1.6,4.9){\vector(4,1){0}}
        \multiput(0.4,4.4)(0.12,-0.03){10}{\circle*{0.01}}
        \put(1.6,4.1){\vector(4,-1){0}}
        \put(0.25,4.2){\vector(3,-4){1.45}}
        \put(2.35,7.2){\vector(2,1){1.3}}
        \multiput(2.4,7)(0.12,0){10}{\circle*{0.01}}
        \put(3.6,7){\vector(1,0){0}}
        \put(2.35,5.2){\vector(2,1){1.3}}
        \multiput(2.4,5)(0.12,0){10}{\circle*{0.01}}
        \put(3.6,5){\vector(1,0){0}}
        \multiput(2.4,4)(0.12,0){10}{\circle*{0.01}}
        \put(3.6,4){\vector(1,0){0}}
        \put(2.35,3.8){\vector(2,-1){1.3}}
        \put(2.4,2){\vector(1,0){1.2}}
        \multiput(2.35,1.8)(0.1,-0.05){13}{\circle*{0.01}}
        \put(3.65,1.15){\vector(2,-1){0}}

        \put(0,4.5){\circle{0.8}}
        \put(2,2){\circle{0.8}}
        \put(2,4){\circle{0.8}}
        \put(2,5){\circle{0.8}}
        \put(2,7){\circle{0.8}}
        \put(4,1){\circle{0.8}}
        \put(4,2){\circle{0.8}}
        \put(4,3){\circle{0.8}}
        \put(4,4){\circle{0.8}}
        \put(4,5){\circle{0.8}}
        \put(4,6){\circle{0.8}}
        \put(4,7){\circle{0.8}}
        \put(4,8){\circle{0.8}}
        
        \put(-0.23,4.5){\makebox(0,0)[l]{\footnotesize{$q_{0,0}$}}}
        \put(1.77,7){\makebox(0,0)[l]{\footnotesize{$q_{1,0}$}}}
        \put(1.77,5){\makebox(0,0)[l]{\footnotesize{$q_{1,1}$}}}
        \put(1.77,4){\makebox(0,0)[l]{\footnotesize{$q_{1,2}$}}}
        \put(1.77,2){\makebox(0,0)[l]{\footnotesize{$q_{1,3}$}}}
        \put(3.77,8){\makebox(0,0)[l]{\footnotesize{$q_{2,0}$}}}
        \put(3.77,7){\makebox(0,0)[l]{\footnotesize{$q_{2,1}$}}}
        \put(3.77,6){\makebox(0,0)[l]{\footnotesize{$q_{2,2}$}}}
        \put(3.77,5){\makebox(0,0)[l]{\footnotesize{$q_{2,3}$}}}
        \put(3.77,4){\makebox(0,0)[l]{\footnotesize{$q_{2,4}$}}}
        \put(3.77,3){\makebox(0,0)[l]{\footnotesize{$q_{2,5}$}}}
        \put(3.77,2){\makebox(0,0)[l]{\footnotesize{$q_{2,6}$}}}
        \put(3.77,1){\makebox(0,0)[l]{\footnotesize{$q_{2,7}$}}}

        \put(0.45,5.5){\rotatebox{53}{\footnotesize{$\langle 0, 0, 0\rangle$}}}
        \put(0.4,4.7){\rotatebox{14}{\footnotesize{$\langle 0, 1, 0\rangle$}}}
        \put(0.4,4.13){\rotatebox{-14}{\footnotesize{$\langle 1, 0, 0\rangle$}}}
        \put(0.45,3.4){\rotatebox{-53}{\footnotesize{$\langle 1, 1, 0\rangle$}}}
        \put(2.5,7.4){\rotatebox{26.5}{\footnotesize{$\langle 0, 0, 0\rangle$}}}
        \put(3,7){\makebox(0,0)[b]{\footnotesize{$\langle 0, 0, 1\rangle$}}}
        \put(2.5,5.4){\rotatebox{26.5}{\footnotesize{$\langle 0, 0, 0\rangle$}}}
        \put(3,5){\makebox(0,0)[b]{\footnotesize{$\langle 0, 0, 1\rangle$}}}
        \put(3,3.95){\makebox(0,0)[t]{\footnotesize{$\langle 0, 0, 0\rangle$}}}
        \put(2.5,3.45){\rotatebox{-26.5}{\footnotesize{$\langle 0, 0, 1\rangle$}}}
        \put(3,1.95){\makebox(0,0)[t]{\footnotesize{$\langle 0, 0, 0\rangle$}}}
        \put(2.5,1.45){\rotatebox{-26.5}{\footnotesize{$\langle 0, 0, 1\rangle$}}}

        \put(-0.5,4.5){\makebox(0,0)[r]{\scriptsize{$\set{O,\neg Att}$}}}
        \put(1.7,7.5){\makebox(0,0)[b]{\scriptsize{$\set{\neg HD, \neg ODS}$}}}
        \put(1.73,5.5){\makebox(0,0)[b]{\scriptsize{$\set{\neg HD, ODS}$}}}
        \put(1.73,3.5){\makebox(0,0)[t]{\scriptsize{$\set{ HD, \neg ODS}$}}}
        \put(1.73,1.5){\makebox(0,0)[t]{\scriptsize{$\set{ HD, ODS}$}}}
        
        \put(4.43,8){\makebox(0,0)[l]{\scriptsize{$\set{O,\neg Att, \neg HD, \neg ODS, \neg DA, \neg Col}$}}}
        \put(4.43,7){\makebox(0,0)[l]{\scriptsize{$\set{O,\neg Att, \neg HD, \neg ODS, DA, \neg Col}$}}}
        \put(4.43,6){\makebox(0,0)[l]{\scriptsize{$\set{O,\neg Att, \neg HD, ODS, \neg DA, \neg Col}$}}}
        \put(4.43,5){\makebox(0,0)[l]{\scriptsize{$\set{O,\neg Att, \neg HD, ODS, DA, \neg Col}$}}}
        \put(4.43,4){\makebox(0,0)[l]{\scriptsize{$\set{O,\neg Att, HD, \neg ODS, \neg DA, \neg Col}$}}}
        \put(4.43,3){\makebox(0,0)[l]{\scriptsize{$\set{O,\neg Att,  HD, \neg ODS, DA, Col}$}}}
        \put(4.43,2){\makebox(0,0)[l]{\scriptsize{$\set{O,\neg Att, HD, ODS, \neg DA, \neg Col}$}}}
        \put(4.43,1){\makebox(0,0)[l]{\scriptsize{$\set{O,\neg Att,  HD, ODS, DA, Col}$}}}
        
    \end{picture}
    \caption{The causal CGS of the semi-automated vehicle example. The dotted lines indicate actions that are not following the causal strategy profile.}
    \label{fig:causal cgs vehicle strategy}
    \Description{The same graph for the causal concurrent game structure of the semi-automated vehicle example as before, with the difference that this picture has drawn some edges with dotted lines. These edges are the edges from q subscript 0,0 to q subscript 1,0 up until 1,2, and the edges from q subscript 1,0 to q subscript 2,1, from q subscript 1,1 to q subscript 2,3, from q subscript 1,2 to q subscript 2,4, and from q subscript 1,3 to q subscript 2,7.}
\end{figure}
\end{example}

In the following lemma, we relate deviations from the causal strategy profile to interventions in the structural causal model that generated the causal CGS. This can be used to relate agent strategies in the causal CGS to causality in the causal model.
\begin{lemma}\label{lem:causal strat leads to causal setting}
    Let $GS$ be a causal CGS based on a causal setting $\csetting$. If $q_{n,m}$ is the leaf-state of $GS$ that results from the strategy profile $F_{\myvec{X} = \myvec{x}} \circ F_\mathcal{M}$, then $q_{n,m}$ corresponds to $(\mathcal{M}^{\myvec{X} \leftarrow \myvec{x}},\mathbf{u})$.
\end{lemma}
\begin{proof}
    We are going to prove the correspondence, i.e. $(X = x) \in \pi(q_{n,m}) \Leftrightarrow (\mathcal{M}^{\myvec{X} \leftarrow \myvec{x}}, \mathbf{u}) \models X = x$ by induction on the agent rank of $X$.
    
    \noindent \textbf{Base Step:} If $\rho(X) = 0$, $X \in V_e$ and does not depend on any other endogenous variables, so if $(X = x)\in \pi(q_{n,m})$, Lemma \ref{lem:no change after i} 
    implies that $(\mathcal{M}, \mathbf{u}) \models X = x$. Because $X$ does not depend on any agent variables, it will keep the same value when intervening on agent variables $\myvec{X}$, so $(\mathcal{M}^{\myvec{X}\leftarrow\myvec{x}},\mathbf{u}) \models X = x$ as well. For the other way around, if $\csettingint{\myvec{X}\leftarrow\myvec{x}}\models X = x$, by the same reasoning we have that $\csetting \models X = x$ and hence $(X = x) \in \pi(q_{0,0})$, again by Lemma \ref{lem:no change after i} 
    $(X = x) \in \pi(q_{n,m})$ as well.

    \noindent\textbf{Induction Hypothesis:} Suppose that for all $X \in \mathcal{V}$ s.t. $\rho(X) \leq i$, $(X = x) \in \pi(q_{n,m})$ if and only if $\csettingint{\myvec{X} \leftarrow \myvec{x}} \models X = x$.

    \noindent\textbf{Inductive Step:} Let $X$ be such that $\rho(X) = i +1$. First suppose that $X \in V_a$.\newline
     - If $X \in \myvec{X}$, $X$ gets value $x \in \myvec{x}$ in $q_{n,m}$, if and only if $\csettingint{\myvec{X} \leftarrow \myvec{x}}$, because it gets the value directly from the intervention $\myvec{X} \leftarrow \myvec{x}$. So it is true in this case.\newline
    - If $X \notin \myvec{X}$, let $(X = x) \in \pi(q_{n,m})$, the value $x$ was determined by $F_\mathcal{M}$, in particular, $\csetting \models [\myvec{Y} \leftarrow \myvec{y}] X = x$, where $\myvec{Y} = \set{Y \ | \ Y \in V_a, \rho(Y) < \rho(X)}$ and $\myvec{y} = \set{y \ | \ y \in \alpha[q_{i,j}]}$, where $q_{i,j}$ is the state on the path to $q_{n,m}$ where $X$ got to take an action. 
    By the inductive hypothesis we know that $\csettingint{\myvec{X} \leftarrow\myvec{x}} \models \myvec{Y} = \myvec{y}$, and hence $\csettingint{\myvec{X} \leftarrow\myvec{x}} \models X = x$ as well, because all variables $X$ depends on have the same values in $\pi(q_{n,m})$ as in $\csettingint{\myvec{X} \leftarrow \myvec{x}}$.
    On the other hand, if $\csettingint{\myvec{X} \leftarrow \myvec{x}} \models X = x$, we know that $x$ is determined only by variables with a lower agent rank, by the inductive hypothesis all those are in $\pi(q_{n,m})$. The value of $X$ in $\pi(q_{n,m})$ is determined by $F_\mathcal{M}$, so $\csetting \models [\myvec{Y} \leftarrow \myvec{y}] X = x'$. 
    \textcolor{black}{All variable-value pairs $Y,y$ are the variable-value pairs from $\csettingint{\myvec{X} \leftarrow \myvec{x}}$, so $x'$ must be $x$, as all variables of lower agent rank have the same value.}\newline
    Now suppose $X \in V_e$, and let $(X = x) \in \pi(q_{n,m})$. $X$ depends only on variables of a lower level, specifically all agent variables of agent rank less or equal than $i + 1$. By the above and the inductive hypothesis, we know that all those variables have the same value in $\csettingint{\myvec{X} \leftarrow \myvec{x}}$, as in $\pi(q_{n,m})$, hence $X = x$ must also be induced by $\csettingint{\myvec{X} \leftarrow \myvec{x}}$, since there are no other interventions done after the agent variables of rank $i+1$ got their final value.
    Now suppose $\csettingint{\myvec{X} \leftarrow \myvec{x}} \models X = x$, $X$ depends only on variables of lower levels, all of those agent variables have the same value in $\csettingint{\myvec{X} \leftarrow \myvec{x}} \models X = x$ as in $\pi(q_{n,m})$ by the inductive hypothesis. Hence $(X = x) \in \pi(q_{n,m})$ as well, since \textcolor{black}{the environment variables follow the causal model in both $\csettingint{\myvec{X} \leftarrow \myvec{x}}$ as in $\pi(q_{n,m})$.}
\end{proof}


The following corollary follows directly from this lemma, it shows that there is a leaf-state in a causal CGS that corresponds to the original causal setting.

\begin{corollary}\label{col:csetting reachable}
    Let $GS$ be a causal CGS based on a causal setting $\csetting$. If $q_{n,m}$ is the leaf-state resulting from all agents following the causal strategy profile $F_{\mathcal{M}}$, then $q_{n,m}$ corresponds to $(\mathcal{M},\mathbf{u})$.
\end{corollary}
\begin{proof}
    This is a special case of Lemma \ref{lem:causal strat leads to causal setting}, where $\myvec{X} = \emptyset$.
\end{proof}

We can check whether this result holds in our semi-automated vehicle example.
We see in Figure \ref{fig:causal cgs vehicle strategy} that if all agents follow the causal strategy profile, they end up in state $q_{2,6}$ with $\pi(q_{2,6}) = \set{O, \neg Att, HD, ODS, \neg DA, \neg Col}$.
The causal CGS was based on the causal setting where there is an obstacle on the road and the driver is not paying attention, in this case we have $\csetting \models O, \neg Att, HD, ODS, \neg DA, \neg Col$ which does correspond to state $q_{2,6}$, as Corollary \ref{col:csetting reachable} predicted.

With Lemma \ref{lem:causal strat leads to causal setting} we can show that if a set of agents $\myvec{X}$ causes $\vphi$ according to the modified HP definition, with a given witness, then in the causal CGS generated by the causal setting that holds this witness fixed, these agents have a strategy to guarantee $\neg \varphi$ in a leaf-state, provided that all other agents follow the causal strategy profile and vice versa.

\begin{proposition}\label{prop:cause iff strat}
    Let $\Gamma = \set{k \ | \ X_k \in \myvec{X}}$ be a set of agents, $\myvec{x}$ a setting for the variables in $\myvec{X}$, and let $\csetting$ be a causal setting with $\csetting\models \vphi$. 
    $\myvec{X} = \myvec{x}$ is, according to the modified HP definition, a cause of causal formula $\varphi$ in this causal setting $\csetting$, with witness $\myvec{W} = \myvec{w}^*$ if and only if in the causal CGS generated by the causal setting, $\csettingint{\myvec{W} \leftarrow \myvec{w}^*}$, $\Gamma$ has a strategy $F_\Gamma$ such that, $\neg \varphi$ will hold in the leaf-state $q_{n,m}$ resulting from the strategy profile $F_\Gamma \circ F_\mathcal{M}$.
\end{proposition}
\begin{proof}
    We first prove the cause to strategy direction.
    In this case, $\myvec{X} = \myvec{x}$ is a cause of $\varphi$, with witness $\myvec{W} = \myvec{w}^*$ so there exists an alternative value for $\myvec{X}$, $\myvec{x}'$ such that $\csetting \models [\myvec{X} \leftarrow \myvec{x}', \myvec{W} \leftarrow \myvec{w}^*] \neg \varphi$.
    Let $F_\Gamma = \{F_{X_k = x} \ | \ x \in \myvec{x}' \text{ if } X_k \in \myvec{X}\}$. 
    By Lemma \ref{lem:causal strat leads to causal setting}, the leaf-state $q_{n,m}$ corresponds to $\csettingint{\myvec{X} \leftarrow \myvec{x}',\myvec{W}\leftarrow \myvec{w}^*}$, and we have that $\csettingint{\myvec{X} \leftarrow \myvec{x}', \myvec{W}\leftarrow\myvec{w}^*}\models\neg \vphi$ and hence $\neg \vphi$ holds in $q_{n,m}$.
    
    Now for the other direction, let $F_\Gamma$ be the strategy such that $\neg \vphi$ will hold in the leaf-state $q_{n,m}$ that results from the strategy profile $F_\Gamma \circ F_{\mathcal{M}}$ in the causal CGS generated by the causal setting $\csettingint{\mathbf{W}\leftarrow\mathbf{w}^*}$.
    Let $\myvec{x}$ be such that $\csetting \models \myvec{X} = \myvec{x}$ and let $\myvec{x}'$ be such that $\myvec{X} = \myvec{x}'\subseteq \pi(q_{n,m})$. By Lemma \ref{lem:causal strat leads to causal setting}, $q_{n,m}$ must correspond to $\csettingint{\myvec{X}\leftarrow\myvec{x}', \myvec{W} \leftarrow \myvec{w}^*}$. Hence $\csetting \models [\myvec{X} \leftarrow \myvec{x}', \myvec{W} \leftarrow \myvec{w}^*]\neg \vphi$ and by definition we have that $\csetting \models \myvec{X} = \myvec{x} \wedge \vphi$. Moreover, $\myvec{x} \neq \myvec{x}'$, because if they were the same it would not be the case that setting $\myvec{X}$ to $\myvec{x}'$ would give a different result than the original causal setting (the determinism axiom of the causal reasoning axioms \cite{halpern2016actual}). Hence $\myvec{X} = \myvec{x}$ is a cause of $\vphi$ according to the modified HP definition, with witness $\myvec{W} = \myvec{w}^*$.
\end{proof}

This result cannot be used to find causes in a causal CGS, because one would already need to know the witness. 
However, we have another result for the causal setting where the witness was not held fixed, provided the witness consists of only agent variables.
The following proposition states that in that case, the set of agents consisting of both the cause and the witness variables has a strategy to guarantee $\neg \varphi$ in a leaf-state, provided all other agents follow the causal strategy profile and vice versa.

\begin{proposition}\label{prop:cause iff superset strat}
    Let $\Gamma = \set{k \ | \ X_k \in \myvec{X}\cup \myvec{W}, \text{ and } \myvec{X} \cup \myvec{W} \subseteq V_a}$ be a set of agents, $\myvec{x},\myvec{w^*}$ are settings for the variables in $\myvec{X},\myvec{W}$ respectively, and let $\csetting$ be a causal setting with $\csetting\models \vphi$. $\myvec{X} = \myvec{x}$ is, according to the modified HP definition, a cause of causal formula $\varphi$ in this causal setting $\csetting$, with witness $\myvec{W} = \myvec{w}^*$ if and only if in the causal CGS generated by this causal setting, $\Gamma$ has a strategy $F_\Gamma$ such that, $\neg \varphi$ will hold in the leaf-state $q_{n,m}$ resulting from the strategy profile $F_\Gamma \circ F_\mathcal{M}$.
\end{proposition}
\begin{proof}
    We first prove the cause to strategy direction.
    In this case, $\myvec{X} = \myvec{x}$ is a cause of $\varphi$, with witness $\myvec{W} = \myvec{w}^*$ so there exists an alternative value for $\myvec{X}$, $\myvec{x}'$ such that $\csetting \models [\myvec{X} \leftarrow \myvec{x}', \myvec{W} \leftarrow \myvec{w}^*] \neg \varphi$.
    Let $F_\Gamma = \{F_{X_k = x} \ | \ k \in \Gamma, x \in \myvec{x}' \text{ if } X_k \in \myvec{X}, \text{ else } x \in \myvec{w}^*\}$. 
    By Lemma \ref{lem:causal strat leads to causal setting}, the leaf-state $q_{n,m}$ corresponds to $\csettingint{\myvec{X} \leftarrow \myvec{x}',\myvec{W}\leftarrow \myvec{w}^*}$, and we have that $\csettingint{\myvec{X} \leftarrow \myvec{x}', \myvec{W}\leftarrow\myvec{w}^*}\models\neg \vphi$ and hence $\neg \vphi$ holds in $q_{n,m}$.
    
    Now for the other direction. 
    Let $\myvec{x}, \myvec{w}^*$ be such that $\csetting \models \myvec{X} = \myvec{x} \wedge \myvec{W} = \myvec{w}^*$ and let $\myvec{x}'$ be such that $\myvec{X} = \myvec{x}'\subseteq \pi(q_{n,m})$. 
    By Lemma \ref{lem:causal strat leads to causal setting}, $q_{n,m}$ must correspond to $\csettingint{\myvec{X}\leftarrow\myvec{x}', \myvec{W} \leftarrow \myvec{w}^*}$. 
    Hence $\csetting \models [\myvec{X} \leftarrow \myvec{x}', \myvec{W} \leftarrow \myvec{w}^*]\neg \vphi$ and by definition we have that $\csetting \models \myvec{X} = \myvec{x} \wedge \vphi$. 
    Moreover, $\myvec{x} \neq \myvec{x}'$, because if they were the same it would not be the case that setting $\myvec{X}$ to $\myvec{x}'$ would give a different result than the original causal setting (the determinism axiom of the causal reasoning axioms \cite{halpern2016actual}). 
    Hence $\myvec{X} = \myvec{x}$ is a cause of $\vphi$ according to the modified HP definition, with witness $\myvec{W} = \myvec{w}^*$.
\end{proof}

As but-for causes have no witness, they give a stronger result.

\begin{corollary}\label{col:but-for cause iff strat}
    Let $\Gamma = \set{k \ | \ X_k \in \myvec{X}}$ be a set of agents, $\myvec{x}$ a setting for the variables in $\myvec{X}$, and let $\csetting$ be a causal setting with $\csetting\models \vphi$. 
    $\myvec{X} = \myvec{x}$ is a but-for cause of causal formula $\varphi$ in this causal setting $\csetting$ if and only if in the causal CGS generated by the causal setting, $\csetting$, $\Gamma$ has a strategy $F_\Gamma$ such that, $\neg \varphi$ will hold in the leaf-state $q_{n,m}$ resulting from the strategy profile $F_\Gamma \circ F_\mathcal{M}$
\end{corollary}
\begin{proof}
    A but-for cause is a special case of the modified HP definition where $\myvec{W} = \emptyset$. This statement is hence a special case of propositions \ref{prop:cause iff strat} and \ref{prop:cause iff superset strat}.
\end{proof}

\begin{example}
    In our running semi-automated vehicle example, both $ODS$ and $\neg DA$ are but-for causes of $\neg Col$, there being no collision (in the causal setting that there is an obstacle and the human driver is not paying attention). 
    In the case of $ODS$ we can define $F_{ODS}$ to be the strategy where the obstacle detection system will not pass on a signal to the driving assistant. 
    If all other agents follow the causal strategy profile, they will reach state $q_{2,5}$.
    Indeed $Col \in \pi(q_{2,5})$.
    Similarly, in the case of $\neg DA$, we can define $F_{DA}$ to be the strategy where the driving assistant does not brake. 
    When the other agents follow $F_{\mathcal{M}}$, they will end up in $q_{2,7}$.
    In that state it is indeed true that $Col \in \pi(q_{2,7})$.
\end{example}

In this section we have shown how agent strategies in a causal CGS relate to the causal relations in the causal setting the causal CGS was based on.
In order to do this, we have introduced the notion of a causal strategy profile, a strategy for the grand coalition that makes sure the agents do exactly those actions they would do if all relations in the causal model would be followed.

\section{Conclusion and Discussion}
\label{sec:conclusion and discussion}
\section{ Task Generalization Beyond i.i.d. Sampling and Parity Functions
}\label{sec:Discussion}
% Discussion: From Theory to Beyond
% \misha{what is beyond?}
% \amir{we mean two things: in the first subsection beyond i.i.d subsampling of parity tasks and in the second subsection beyond parity task}
% \misha{it has to be beyond something, otherwise it is not clear what it is about} \hz{this is suggested by GPT..., maybe can be interpreted as from theory to beyond theory. We can do explicit like Discussion: Beyond i.i.d. task sampling and the Parity Task}
% \misha{ why is "discussion" in the title?}\amir{Because it is a discussion, it is not like separate concrete explnation about why these thing happens or when they happen, they just discuss some interesting scenraios how it relates to our theory.   } \misha{it is not really a discussion -- there is a bunch of experiments}

In this section, we extend our experiments beyond i.i.d. task sampling and parity functions. We show an adversarial example where biased task selection substantially hinders task generalization for sparse parity problem. In addition, we demonstrate that exponential task scaling extends to a non-parity tasks including arithmetic and multi-step language translation.

% In this section, we extend our experiments beyond i.i.d. task sampling and parity functions. On the one hand, we find that biased task selection can significantly degrade task generalization; on the other hand, we show that exponential task scaling generalizes to broader scenarios.
% \misha{we should add a sentence or two giving more detail}


% 1. beyond i.i.d tasks sampling
% 2. beyond parity -> language, arithmetic -> task dependency + implicit bias of transformer (cannot implement this algorithm for arithmatic)



% In this section, we emphasize the challenge of quantifying the level of out-of-distribution (OOD) differences between training tasks and testing tasks, even for a simple parity task. To illustrate this, we present two scenarios where tasks differ between training and testing. For each scenario, we invite the reader to assess, before examining the experimental results, which cases might appear “more” OOD. All scenarios consider \( d = 10 \). \kaiyue{this sentence should be put into 5.1}






% for parity problem




% \begin{table*}[th!]
%     \centering
%     \caption{Generalization Results for Scenarios 1 and 2 for $d=10$.}
%     \begin{tabular}{|c|c|c|c|}
%         \hline
%         \textbf{Scenario} & \textbf{Type/Variation} & \textbf{Coordinates} & \textbf{Generalization accuracy} \\
%         \hline
%         \multirow{3}{*}{Generalization with Missing Pair} & Type 1 & \( c_1 = 4, c_2 = 6 \) & 47.8\%\\ 
%         & Type 2 & \( c_1 = 4, c_2 = 6 \) & 96.1\%\\ 
%         & Type 3 & \( c_1 = 4, c_2 = 6 \) & 99.5\%\\ 
%         \hline
%         \multirow{3}{*}{Generalization with Missing Pair} & Type 1 &  \( c_1 = 8, c_2 = 9 \) & 40.4\%\\ 
%         & Type 2 & \( c_1 = 8, c_2 = 9 \) & 84.6\% \\ 
%         & Type 3 & \( c_1 = 8, c_2 = 9 \) & 99.1\%\\ 
%         \hline
%         \multirow{1}{*}{Generalization with Missing Coordinate} & --- & \( c_1 = 5 \) & 45.6\% \\ 
%         \hline
%     \end{tabular}
%     \label{tab:generalization_results}
% \end{table*}

\subsection{Task Generalization Beyond i.i.d. Task Sampling }\label{sec: Experiment beyond iid sampling}

% \begin{table*}[ht!]
%     \centering
%     \caption{Generalization Results for Scenarios 1 and 2 for $d=10, k=3$.}
%     \begin{tabular}{|c|c|c|}
%         \hline
%         \textbf{Scenario}  & \textbf{Tasks excluded from training} & \textbf{Generalization accuracy} \\
%         \hline
%         \multirow{1}{*}{Generalization with Missing Pair}
%         & $\{4,6\} \subseteq \{s_1, s_2, s_3\}$ & 96.2\%\\ 
%         \hline
%         \multirow{1}{*}{Generalization with Missing Coordinate}
%         & \( s_2 = 5 \) & 45.6\% \\ 
%         \hline
%     \end{tabular}
%     \label{tab:generalization_results}
% \end{table*}




In previous sections, we focused on \textit{i.i.d. settings}, where the set of training tasks $\mathcal{F}_{train}$ were sampled uniformly at random from the entire class $\mathcal{F}$. Here, we explore scenarios that deliberately break this uniformity to examine the effect of task selection on out-of-distribution (OOD) generalization.\\

\textit{How does the selection of training tasks influence a model’s ability to generalize to unseen tasks? Can we predict which setups are more prone to failure?}\\

\noindent To investigate this, we consider two cases parity problems with \( d = 10 \) and \( k = 3 \), where each task is represented by its tuple of secret indices \( (s_1, s_2, s_3) \):

\begin{enumerate}[leftmargin=0.4 cm]
    \item \textbf{Generalization with a Missing Coordinate.} In this setup, we exclude all training tasks where the second coordinate takes the value \( s_2 = 5 \), such as \( (1,5,7) \). At test time, we evaluate whether the model can generalize to unseen tasks where \( s_2 = 5 \) appears.
    \item \textbf{Generalization with Missing Pair.} Here, we remove all training tasks that contain both \( 4 \) \textit{and} \( 6 \) in the tuple \( (s_1, s_2, s_3) \), such as \( (2,4,6) \) and \( (4,5,6) \). At test time, we assess whether the model can generalize to tasks where both \( 4 \) and \( 6 \) appear together.
\end{enumerate}

% \textbf{Before proceeding, consider the following question:} 
\noindent \textbf{If you had to guess.} Which scenario is more challenging for generalization to unseen tasks? We provide the experimental result in Table~\ref{tab:generalization_results}.

 % while the model struggles for one of them while as it generalizes almost perfectly in the other one. 

% in the first scenario, it generalizes almost perfectly in the second. This highlights how exposure to partial task structures can enhance generalization, even when certain combinations are entirely absent from the training set. 

In the first scenario, despite being trained on all tasks except those where \( s_2 = 5 \), which is of size $O(\d^T)$, the model struggles to generalize to these excluded cases, with prediction at chance level. This is intriguing as one may expect model to generalize across position. The failure  suggests that positional diversity plays a crucial role in the task generalization of Transformers. 

In contrast, in the second scenario, though the model has never seen tasks with both \( 4 \) \textit{and} \( 6 \) together, it has encountered individual instances where \( 4 \) appears in the second position (e.g., \( (1,4,5) \)) or where \( 6 \) appears in the third position (e.g., \( (2,3,6) \)). This exposure appears to facilitate generalization to test cases where both \( 4 \) \textit{and} \( 6 \) are present. 



\begin{table*}[t!]
    \centering
    \caption{Generalization Results for Scenarios 1 and 2 for $d=10, k=3$.}
    \resizebox{\textwidth}{!}{  % Scale to full width
        \begin{tabular}{|c|c|c|}
            \hline
            \textbf{Scenario}  & \textbf{Tasks excluded from training} & \textbf{Generalization accuracy} \\
            \hline
            Generalization with Missing Pair & $\{4,6\} \subseteq \{s_1, s_2, s_3\}$ & 96.2\%\\ 
            \hline
            Generalization with Missing Coordinate & \( s_2 = 5 \) & 45.6\% \\ 
            \hline
        \end{tabular}
    }
    \label{tab:generalization_results}
\end{table*}

As a result, when the training tasks are not i.i.d, an adversarial selection such as exclusion of specific positional configurations may lead to failure to unseen task generalization even though the size of $\mathcal{F}_{train}$ is exponentially large. 


% \paragraph{\textbf{Key Takeaways}}
% \begin{itemize}
%     \item Out-of-distribution generalization in the parity problem is highly sensitive to the diversity and positional coverage of training tasks.
%     \item Adversarial exclusion of specific pairs or positional configurations can lead to systematic failures, even when most tasks are observed during training.
% \end{itemize}




%################ previous veriosn down
% \textit{How does the choice of training tasks affect the ability of a model to generalize to unseen tasks? Can we predict which setups are likely to lead to failure?}

% To explore these questions, we crafted specific training and test task splits to investigate what makes one setup appear “more” OOD than another.

% \paragraph{Generalization with Missing Pair.}

% Imagine we have tasks constructed from subsets of \(k=3\) elements out of a larger set of \(d\) coordinates. What happens if certain pairs of coordinates are adversarially excluded during training? For example, suppose \(d=5\) and two specific coordinates, \(c_1 = 1\) and \(c_2 = 2\), are excluded. The remaining tasks are formed from subsets of the other coordinates. How would a model perform when tested on tasks involving the excluded pair \( (c_1, c_2) \)? 

% To probe this, we devised three variations in how training tasks are constructed:
%     \begin{enumerate}
%         \item \textbf{Type 1:} The training set includes all tasks except those containing both \( c_1 = 1 \) and \( c_2 = 2 \). 
%         For this example, the training set includes only $\{(3,4,5)\}$. The test set consists of all tasks containing the rest of tuples.

%         \item \textbf{Type 2:} Similar to Type 1, but the training set additionally includes half of the tasks containing either \( c_1 = 1 \) \textit{or} \( c_2 = 2 \) (but not both). 
%         For the example, the training set includes all tasks from Type 1 and adds tasks like \(\{(1, 3, 4), (2, 3, 5)\}\) (half of those containing \( c_1 = 1 \) or \( c_2 = 2 \)).

%         \item \textbf{Type 3:} Similar to Type 2, but the training set also includes half of the tasks containing both \( c_1 = 1 \) \textit{and} \( c_2 = 2 \). 
%         For the example, the training set includes all tasks from Type 2 and adds, for instance, \(\{(1, 2, 5)\}\) (half of the tasks containing both \( c_1 \) and \( c_2 \)).
%     \end{enumerate}

% By systematically increasing the diversity of training tasks in a controlled way, while ensuring no overlap between training and test configurations, we observe an improvement in OOD generalization. 

% % \textit{However, the question is this improvement similar across all coordinate pairs, or does it depend on the specific choices of \(c_1\) and \(c_2\) in the tasks?} 

% \textbf{Before proceeding, consider the following question:} Is the observed improvement consistent across all coordinate pairs, or does it depend on the specific choices of \(c_1\) and \(c_2\) in the tasks? 

% For instance, consider two cases for \(d = 10, k = 3\): (i) \(c_1 = 4, c_2 = 6\) and (ii) \(c_1 = 8, c_2 = 9\). Would you expect similar OOD generalization behavior for these two cases across the three training setups we discussed?



% \paragraph{Answer to the Question.} for both cases of \( c_1, c_2 \), we observe that generalization fails in Type 1, suggesting that the position of the tasks the model has been trained on significantly impacts its generalization capability. For Type 2, we find that \( c_1 = 4, c_2 = 6 \) performs significantly better than \( c_1 = 8, c_2 = 9 \). 

% Upon examining the tasks where the transformer fails for \( c_1 = 8, c_2 = 9 \), we see that the model only fails at tasks of the form \((*, 8, 9)\) while perfectly generalizing to the rest. This indicates that the model has never encountered the value \( 8 \) in the second position during training, which likely explains its failure to generalize. In contrast, for \( c_1 = 4, c_2 = 6 \), while the model has not seen tasks of the form \((*, 4, 6)\), it has encountered tasks where \( 4 \) appears in the second position, such as \((1, 4, 5)\), and tasks where \( 6 \) appears in the third position, such as \((2, 3, 6)\). This difference may explain why the model generalizes almost perfectly in Type 2 for \( c_1 = 4, c_2 = 6 \), but not for \( c_1 = 8, c_2 = 9 \).



% \paragraph{Generalization with Missing Coordinates.}
% Next, we investigate whether a model can generalize to tasks where a specific coordinate appears in an unseen position during training. For instance, consider \( c_1 = 5 \), and exclude all tasks where \( c_1 \) appears in the second position. Despite being trained on all other tasks, the model fails to generalize to these excluded cases, highlighting the importance of positional diversity in training tasks.



% \paragraph{Key Takeaways.}
% \begin{itemize}
%     \item OOD generalization depends heavily on the diversity and positional coverage of training tasks for the parity problem.
%     \item adversarial exclusion of specific pairs or positional configurations in the parity problem can lead to failure, even when the majority of tasks are observed during training.
% \end{itemize}


%################ previous veriosn up

% \paragraph{Key Takeaways} These findings highlight the complexity of OOD generalization, even in seemingly simple tasks like parity. They also underscore the importance of task design: the diversity of training tasks can significantly influence a model’s ability to generalize to unseen tasks. By better understanding these dynamics, we can design more robust training regimes that foster generalization across a wider range of scenarios.


% #############


% Upon examining the tasks where the transformer fails for \( c_1 = 8, c_2 = 9 \), we see that the model only fails at tasks of the form \((*, 8, 9)\) while perfectly generalizing to the rest. This indicates that the model has never encountered the value \( 8 \) in the second position during training, which likely explains its failure to generalize. In contrast, for \( c_1 = 4, c_2 = 6 \), while the model has not seen tasks of the form \((*, 4, 6)\), it has encountered tasks where \( 4 \) appears in the second position, such as \((1, 4, 5)\), and tasks where \( 6 \) appears in the third position, such as \((2, 3, 6)\). This difference may explain why the model generalizes almost perfectly in Type 2 for \( c_1 = 4, c_2 = 6 \), but not for \( c_1 = 8, c_2 = 9 \).

% we observe a striking pattern: generalization fails entirely in Type 1, regardless of the coordinate pair (\(c_1, c_2\)). However, in Type 2, generalization varies: \(c_1 = 4, c_2 = 6\) achieves 96\% accuracy, while \(c_1 = 8, c_2 = 9\) lags behind at 70\%. Why? Upon closer inspection, the model struggles specifically with tasks like \((*, 8, 9)\), where the combination \(c_1 = 8\) and \(c_2 = 9\) is entirely novel. In contrast, for \(c_1 = 4, c_2 = 6\), the model benefits from having seen tasks where \(4\) appears in the second position or \(6\) in the third. This suggests that positional exposure during training plays a key role in generalization.

% To test whether task structure influences generalization, we consider two variations:
% \begin{enumerate}
%     \item \textbf{Sorted Tuples:} Tasks are always sorted in ascending order.
%     \item \textbf{Unsorted Tuples:} Tasks can appear in any order.
% \end{enumerate}

% If the model struggles with generalizing to the excluded position, does introducing variability through unsorted tuples help mitigate this limitation?

% \paragraph{Discussion of Results}

% In \textbf{Generalization with Missing Pairs}, we observe a striking pattern: generalization fails entirely in Type 1, regardless of the coordinate pair (\(c_1, c_2\)). However, in Type 2, generalization varies: \(c_1 = 4, c_2 = 6\) achieves 96\% accuracy, while \(c_1 = 8, c_2 = 9\) lags behind at 70\%. Why? Upon closer inspection, the model struggles specifically with tasks like \((*, 8, 9)\), where the combination \(c_1 = 8\) and \(c_2 = 9\) is entirely novel. In contrast, for \(c_1 = 4, c_2 = 6\), the model benefits from having seen tasks where \(4\) appears in the second position or \(6\) in the third. This suggests that positional exposure during training plays a key role in generalization.

% In \textbf{Generalization with Missing Coordinates}, the results confirm this hypothesis. When \(c_1 = 5\) is excluded from the second position during training, the model fails to generalize to such tasks in the sorted case. However, allowing unsorted tuples introduces positional diversity, leading to near-perfect generalization. This raises an intriguing question: does the model inherently overfit to positional patterns, and can task variability help break this tendency?




% In this subsection, we show that the selection of training tasks can affect the quality of the unseen task generalization significantly in practice. To illustrate this, we present two scenarios where tasks differ between training and testing. For each scenario, we invite the reader to assess, before examining the experimental results, which cases might appear “more” OOD. 

% % \amir{add examples, }

% \kaiyue{I think the name of scenarios here are not very clear}
% \begin{itemize}
%     \item \textbf{Scenario 1:  Generalization Across Excluded Coordinate Pairs (\( k = 3 \))} \\
%     In this scenario, we select two coordinates \( c_1 \) and \( c_2 \) out of \( d \) and construct three types of training sets. 

%     Suppose \( d = 5 \), \( c_1 = 1 \), and \( c_2 = 2 \). The tuples are all possible subsets of \( \{1, 2, 3, 4, 5\} \) with \( k = 3 \):
%     \[
%     \begin{aligned}
%     \big\{ & (1, 2, 3), (1, 2, 4), (1, 2, 5), (1, 3, 4), (1, 3, 5), \\
%            & (1, 4, 5), (2, 3, 4), (2, 3, 5), (2, 4, 5), (3, 4, 5) \big\}.
%     \end{aligned}
%     \]

%     \begin{enumerate}
%         \item \textbf{Type 1:} The training set includes all tuples except those containing both \( c_1 = 1 \) and \( c_2 = 2 \). 
%         For this example, the training set includes only $\{(3,4,5)\}$ tuple. The test set consists of tuples containing the rest of tuples.

%         \item \textbf{Type 2:} Similar to Type 1, but the training set additionally includes half of the tuples containing either \( c_1 = 1 \) \textit{or} \( c_2 = 2 \) (but not both). 
%         For the example, the training set includes all tuples from Type 1 and adds tuples like \(\{(1, 3, 4), (2, 3, 5)\}\) (half of those containing \( c_1 = 1 \) or \( c_2 = 2 \)).

%         \item \textbf{Type 3:} Similar to Type 2, but the training set also includes half of the tuples containing both \( c_1 = 1 \) \textit{and} \( c_2 = 2 \). 
%         For the example, the training set includes all tuples from Type 2 and adds, for instance, \(\{(1, 2, 5)\}\) (half of the tuples containing both \( c_1 \) and \( c_2 \)).
%     \end{enumerate}

% % \begin{itemize}
% %     \item \textbf{Type 1:} The training set includes tuples \(\{1, 3, 4\}, \{2, 3, 4\}\) (excluding tuples with both \( c_1 \) and \( c_2 \): \(\{1, 2, 3\}, \{1, 2, 4\}\)). The test set contains the excluded tuples.
% %     \item \textbf{Type 2:} The training set includes all tuples in Type 1 plus half of the tuples containing either \( c_1 = 1 \) or \( c_2 = 2 \) (e.g., \(\{1, 2, 3\}\)).
% %     \item \textbf{Type 3:} The training set includes all tuples in Type 2 plus half of the tuples containing both \( c_1 = 1 \) and \( c_2 = 2 \) (e.g., \(\{1, 2, 4\}\)).
% % \end{itemize}
    
%     \item \textbf{Scenario 2: Scenario 2: Generalization Across a Fixed Coordinate (\( k = 3 \))} \\
%     In this scenario, we select one coordinate \( c_1 \) out of \( d \) (\( c_1 = 5 \)). The training set includes all task tuples except those where \( c_1 \) is the second coordinate of the tuple. For this scenario, we examine two variations:
%     \begin{enumerate}
%         \item \textbf{Sorted Tuples:} Task tuples are always sorted (e.g., \( (x_1, x_2, x_3) \) with \( x_1 \leq x_2 \leq x_3 \)).
%         \item \textbf{Unsorted Tuples:} Task tuples can appear in any order.
%     \end{enumerate}
% \end{itemize}




% \paragraph{Discussion of Results.} In the first scenario, for both cases of \( c_1, c_2 \), we observe that generalization fails in Type 1, suggesting that the position of the tasks the model has been trained on significantly impacts its generalization capability. For Type 2, we find that \( c_1 = 4, c_2 = 6 \) performs significantly better than \( c_1 = 8, c_2 = 9 \). 

% Upon examining the tasks where the transformer fails for \( c_1 = 8, c_2 = 9 \), we see that the model only fails at tasks of the form \((*, 8, 9)\) while perfectly generalizing to the rest. This indicates that the model has never encountered the value \( 8 \) in the second position during training, which likely explains its failure to generalize. In contrast, for \( c_1 = 4, c_2 = 6 \), while the model has not seen tasks of the form \((*, 4, 6)\), it has encountered tasks where \( 4 \) appears in the second position, such as \((1, 4, 5)\), and tasks where \( 6 \) appears in the third position, such as \((2, 3, 6)\). This difference may explain why the model generalizes almost perfectly in Type 2 for \( c_1 = 4, c_2 = 6 \), but not for \( c_1 = 8, c_2 = 9 \).

% This position-based explanation appears compelling, so in the second scenario, we focus on a single position to investigate further. Here, we find that the transformer fails to generalize to tasks where \( 5 \) appears in the second position, provided it has never seen any such tasks during training. However, when we allow for more task diversity in the unsorted case, the model achieves near-perfect generalization. 

% This raises an important question: does the transformer have a tendency to overfit to positional patterns, and does introducing more task variability, as in the unsorted case, prevent this overfitting and enable generalization to unseen positional configurations?

% These findings highlight that even in a simple task like parity, it is remarkably challenging to understand and quantify the sources and levels of OOD behavior. This motivates further investigation into the nuances of task design and its impact on model generalization.


\subsection{Task Generalization Beyond Parity Problems}

% \begin{figure}[t!]
%     \centering
%     \includegraphics[width=0.45\textwidth]{Figures/arithmetic_v1.png}
%     \vspace{-0.3cm}
%     \caption{Task generalization for arithmetic task with CoT, it has $\d =2$ and $T = d-1$ as the ambient dimension, hence $D\ln(DT) = 2\ln(2T)$. We show that the empirical scaling closely follows the theoretical scaling.}
%     \label{fig:arithmetic}
% \end{figure}



% \begin{wrapfigure}{r}{0.4\textwidth}  % 'r' for right, 'l' for left
%     \centering
%     \includegraphics[width=0.4\textwidth]{Figures/arithmetic_v1.png}
%     \vspace{-0.3cm}
%     \caption{Task generalization for the arithmetic task with CoT. It has $d =2$ and $T = d-1$ as the ambient dimension, hence $D\ln(DT) = 2\ln(2T)$. We show that the empirical scaling closely follows the theoretical scaling.}
%     \label{fig:arithmetic}
% \end{wrapfigure}

\subsubsection{Arithmetic Task}\label{subsec:arithmetic}











We introduce the family of \textit{Arithmetic} task that, like the sparse parity problem, operates on 
\( d \) binary inputs \( b_1, b_2, \dots, b_d \). The task involves computing a structured arithmetic expression over these inputs using a sequence of addition and multiplication operations.
\newcommand{\op}{\textrm{op}}

Formally, we define the function:
\[
\text{Arithmetic}_{S} \colon \{0,1\}^d \to \{0,1,\dots,d\},
\]
where \( S = (\op_1, \op_2, \dots, \op_{d-1}) \) is a sequence of \( d-1 \) operations, each \( \op_k \) chosen from \( \{+, \times\} \). The function evaluates the expression by applying the operations sequentially from left-to-right order: for example, if \( S = (+, \times, +) \), then the arithmetic function would compute:
\[
\text{Arithmetic}_{S}(b_1, b_2, b_3, b_4) = ((b_1 + b_2) \times b_3) + b_4.
\]
% Thus, the sequence of operations \( S \) defines how the binary inputs are combined to produce an integer output between \( 0 \) and \( d \).
% \[
% \text{Arithmetic}_{S} 
% (b_1,\,b_2,\,\dots,b_d)
% =
% \Bigl(\dots\bigl(\,(b_1 \;\op_1\; b_2)\;\op_2\; b_3\bigr)\,\dots\Bigr) 
% \;\op_{d-1}\; b_d.
% \]
% We now introduce an \emph{Arithmetic} task that, like the sparse parity problem, operates on $d$ binary inputs $b_1, b_2, \dots, b_d$. Specifically, we define an arithmetic function
% \[
% \text{Arithmetic}_{S}\colon \{0,1\}^d \;\to\; \{0,1,\dots,d\},
% \]
% where $S = (i_1, i_2, \dots, i_{d-1})$ is a sequence of $d-1$ operations, each $i_k \in \{+,\,\times\}$. The value of $\text{Arithmetic}_{S}$ is obtained by applying the prescribed addition and multiplication operations in order, namely:
% \[
% \text{Arithmetic}_{S}(b_1,\,b_2,\,\dots,b_d)
% \;=\;
% \Bigl(\dots\bigl(\,(b_1 \;i_1\; b_2)\;i_2\; b_3\bigr)\,\dots\Bigr) 
% \;i_{d-1}\; b_d.
% \]

% This is an example of our framework where $T = d-1$ and $|\Theta_t| = 2$ with total $2^d$ possible tasks. 




By introducing a step-by-step CoT, arithmetic class belongs to $ARC(2, d-1)$: this is because at every step, there is only $\d = |\Theta_t| = 2$ choices (either $+$ or $\times$) while the length is  $T = d-1$, resulting a total number of $2^{d-1}$ tasks. 


\begin{minipage}{0.5\textwidth}  % Left: Text
    Task generalization for the arithmetic task with CoT. It has $d =2$ and $T = d-1$ as the ambient dimension, hence $D\ln(DT) = 2\ln(2T)$. We show that the empirical scaling closely follows the theoretical scaling.
\end{minipage}
\hfill
\begin{minipage}{0.4\textwidth}  % Right: Image
    \centering
    \includegraphics[width=\textwidth]{Figures/arithmetic_v1.png}
    \refstepcounter{figure}  % Manually advances the figure counter
    \label{fig:arithmetic}  % Now this label correctly refers to the figure
\end{minipage}

Notably, when scaling with \( T \), we observe in the figure above that the task scaling closely follow the theoretical $O(D\log(DT))$ dependency. Given that the function class grows exponentially as \( 2^T \), it is truly remarkable that training on only a few hundred tasks enables generalization to an exponentially larger space—on the order of \( 2^{25} > 33 \) Million tasks. This exponential scaling highlights the efficiency of structured learning, where a modest number of training examples can yield vast generalization capability.





% Our theory suggests that only $\Tilde{O}(\ln(T))$ i.i.d training tasks is enough to generalize to the rest of unseen tasks. However, we show in Figure \ref{fig:arithmetic} that transformer is not able to match  that. The transformer out-of distribution generalization behavior is not consistent across different dimensions when we scale the number of training tasks with $\ln(T)$. \hongzhou{implicit bias, optimization, etc}
 






% \subsection{Task generalization Beyond parity problem}

% \subsection{Arithmetic} In this setting, we still use the set-up we introduced in \ref{subsec:parity_exmaple}, the input is still a set of $d$ binary variable, $b_1, b_2,\dots,b_d$ and ${Arithmatic_{S}}:\{0,1\}\rightarrow \{0, 1, \dots, d\}$, where $S = (i_1,i_2,\dots,i_{d-1})$ is a tuple of size $d-1$ where each coordinate is either add($+
% $) or multiplication ($\times$). The function is as following,

% \begin{align*}
%     Arithmatic_{S}(b_1, b_2,\dots,b_d) = (\dots(b1(i1)b2)(i3)b3\dots)(i{d-1})
% \end{align*}
    


\subsubsection{Multi-Step Language Translation Task}

 \begin{figure*}[h!]
    \centering
    \includegraphics[width=0.9\textwidth]{Figures/combined_plot_horiz.png}
    \vspace{-0.2cm}
    \caption{Task generalization for language translation task: $\d$ is the number of languages and $T$ is the length of steps.}
    \vspace{-2mm}
    \label{fig:language}
\end{figure*}
% \vspace{-2mm}

In this task, we study a sequential translation process across multiple languages~\cite{garg2022can}. Given a set of \( D \) languages, we construct a translation chain by randomly sampling a sequence of \( T \) languages \textbf{with replacement}:  \(L_1, L_2, \dots, L_T,\)
where each \( L_t \) is a sampled language. Starting with a word, we iteratively translate it through the sequence:
\vspace{-2mm}
\[
L_1 \to L_2 \to L_3 \to \dots \to L_T.
\]
For example, if the sampled sequence is EN → FR → DE → FR, translating the word "butterfly" follows:
\vspace{-1mm}
\[
\text{butterfly} \to \text{papillon} \to \text{schmetterling} \to \text{papillon}.
\]
This task follows an \textit{AutoRegressive Compositional} structure by itself, specifically \( ARC(D, T-1) \), where at each step, the conditional generation only depends on the target language, making \( D \) as the number of languages and the total number of possible tasks is \( D^{T-1} \). This example illustrates that autoregressive compositional structures naturally arise in real-world languages, even without explicit CoT. 

We examine task scaling along \( D \) (number of languages) and \( T \) (sequence length). As shown in Figure~\ref{fig:language}, empirical  \( D \)-scaling closely follows the theoretical \( O(D \ln D T) \). However, in the \( T \)-scaling case, we observe a linear dependency on \( T \) rather than the logarithmic dependency \(O(\ln T) \). A possible explanation is error accumulation across sequential steps—longer sequences require higher precision in intermediate steps to maintain accuracy. This contrasts with our theoretical analysis, which focuses on asymptotic scaling and does not explicitly account for compounding errors in finite-sample settings.

% We examine task scaling along \( D \) (number of languages) and \( T \) (sequence length). As shown in Figure~\ref{fig:language}, empirical scaling closely follows the theoretical \( O(D \ln D T) \) trend, with slight exceptions at $ T=10 \text{ and } 3$ in Panel B. One possible explanation for this deviation could be error accumulation across sequential steps—longer sequences require each intermediate translation to be approximated with higher precision to maintain test accuracy. This contrasts with our theoretical analysis, which primarily focuses on asymptotic scaling and does not explicitly account for compounding errors in finite-sample settings.

Despite this, the task scaling is still remarkable — training on a few hundred tasks enables generalization to   $4^{10} \approx 10^6$ tasks!






% , this case, we are in a regime where \( D \ll T \), we observe  that the task complexity empirically scales as \( T \log T \) rather than \( D \log T \). 


% the model generalizes to an exponentially larger space of \( 2^T \) unseen tasks. In case $T=25$, this is $2^{25} > 33$ Million tasks. This remarkable exponential generalization demonstrates the power of structured task composition in enabling efficient generalization.


% In the case of parity tasks, introducing CoT effectively decomposes the problem from \( ARC(D^T, 1) \) to \( ARC(D, T) \), significantly improving task generalization.

% Again, in the regime scaling $T$, we again observe a $T\log T$ dependency. Knowing that the function class is scaling as $D^T$, it is remarkable that training on a few hundreds tasks can generalize to $4^{10} \approx 1M$ tasks. 





% We further performed a preliminary investigation on a semi-synthetic word-level translation task to show that (1) task generalization via composition structure is feasible beyond parity and (2) understanding the fine-grained mechanism leading to this generalization is still challenging. 
% \noindent
% \noindent
% \begin{minipage}[t]{\columnwidth}
%     \centering
%     \textbf{\scriptsize In-context examples:}
%     \[
%     \begin{array}{rl}
%         \textbf{Input} & \hspace{1.5em} \textbf{Output} \\
%         \hline
%         \textcolor{blue}{car}   & \hspace{1.5em} \textcolor{red}{voiture \;,\; coche} \\
%         \textcolor{blue}{house} & \hspace{1.5em} \textcolor{red}{maison \;,\; casa} \\
%         \textcolor{blue}{dog}   & \hspace{1.5em} \textcolor{red}{chien \;,\; perro} 
%     \end{array}
%     \]
%     \textbf{\scriptsize Query:}
%     \[
%     \begin{array}{rl}
%         \textbf{Input} & \textbf{Output} \\
%         \hline
%         \textcolor{blue}{cat} & \hspace{1.5em} \textcolor{red}{?} \\
%     \end{array}
%     \]
% \end{minipage}



% \begin{figure}[h!]
%     \centering
%     \includegraphics[width=0.45\textwidth]{Figures/translation_scale_d.png}
%     \vspace{-0.2cm}
%     \caption{Task generalization behavior for word translation task.}
%     \label{fig:arithmetic}
% \end{figure}


\vspace{-1mm}
\section{Conclusions}
% \misha{is it conclusion of the section or of the whole paper?}    
% \amir{The whole paper. It is very short, do we need a separate section?}
% \misha{it should not be a subsection if it is the conclusion the whole thing. We can just remove it , it does not look informative} \hz{let's do it in a whole section, just to conclude and end the paper, even though it is not informative}
%     \kaiyue{Proposal: Talk about the implication of this result on theory development. For example, it calls for more fine-grained theoretical study in this space.  }

% \huaqing{Please feel free to edit it if you have better wording or suggestions.}

% In this work, we propose a theoretical framework to quantitatively investigate task generalization with compositional autoregressive tasks. We show that task to $D^T$ task is theoretically achievable by training on only $O (D\log DT)$ tasks, and empirically observe that transformers trained on parity problem indeed achieves such task generalization. However, for other tasks beyond parity, transformers seem to fail to achieve this bond. This calls for more fine-grained theoretical study the phenomenon of task generalization specific to transformer model. It may also be interesting to study task generalization beyond the setting of in-context learning. 
% \misha{what does this add?} \amir{It does not, i dont have any particular opinion to keep it. @Hongzhou if you want to add here?}\hz{While it may not introduce anything new, we are following a good practice to have a short conclusion. It provides a clear closing statement, reinforces key takeaways, and helps the reader leave with a well-framed understanding of our contributions. }
% In this work, we quantitatively investigate task generalization under autoregressive compositional structure. We demonstrate that task generalization to $D^T$ tasks is theoretically achievable by training on only $\tilde O(D)$ tasks. Empirically, we observe that transformers trained indeed achieve such exponential task generalization on problems such as parity, arithmetic and multi-step language translation. We believe our analysis opens up a new angle to understand the remarkable generalization ability of Transformer in practice. 

% However, for tasks beyond the parity problem, transformers appear to fail to reach this bound. This highlights the need for a more fine-grained theoretical exploration of task generalization, especially for transformer models. Additionally, it may be valuable to investigate task generalization beyond the scope of in-context learning.



In this work, we quantitatively investigated task generalization under the autoregressive compositional structure, demonstrating both theoretically and empirically that exponential task generalization to $D^T$ tasks can be achieved with training on only $\tilde{O}(D)$ tasks. %Our theoretical results establish a fundamental scaling law for task generalization, while our experiments validate these insights across problems such as parity, arithmetic, and multi-step language translation. The remarkable ability of transformers to generalize exponentially highlights the power of structured learning and provides a new perspective on how large language models extend their capabilities beyond seen tasks. 
We recap our key contributions  as follows:
\begin{itemize}
    \item \textbf{Theoretical Framework for Task Generalization.} We introduced the \emph{AutoRegressive Compositional} (ARC) framework to model structured task learning, demonstrating that a model trained on only $\tilde{O}(D)$ tasks can generalize to an exponentially large space of $D^T$ tasks.
    
    \item \textbf{Formal Sample Complexity Bound.} We established a fundamental scaling law that quantifies the number of tasks required for generalization, proving that exponential generalization is theoretically achievable with only a logarithmic increase in training samples.
    
    \item \textbf{Empirical Validation on Parity Functions.} We showed that Transformers struggle with standard in-context learning (ICL) on parity tasks but achieve exponential generalization when Chain-of-Thought (CoT) reasoning is introduced. Our results provide the first empirical demonstration of structured learning enabling efficient generalization in this setting.
    
    \item \textbf{Scaling Laws in Arithmetic and Language Translation.} Extending beyond parity functions, we demonstrated that the same compositional principles hold for arithmetic operations and multi-step language translation, confirming that structured learning significantly reduces the task complexity required for generalization.
    
    \item \textbf{Impact of Training Task Selection.} We analyzed how different task selection strategies affect generalization, showing that adversarially chosen training tasks can hinder generalization, while diverse training distributions promote robust learning across unseen tasks.
\end{itemize}



We introduce a framework for studying the role of compositionality in learning tasks and how this structure can significantly enhance generalization to unseen tasks. Additionally, we provide empirical evidence on learning tasks, such as the parity problem, demonstrating that transformers follow the scaling behavior predicted by our compositionality-based theory. Future research will  explore how these principles extend to real-world applications such as program synthesis, mathematical reasoning, and decision-making tasks. 


By establishing a principled framework for task generalization, our work advances the understanding of how models can learn efficiently beyond supervised training and adapt to new task distributions. We hope these insights will inspire further research into the mechanisms underlying task generalization and compositional generalization.

\section*{Acknowledgements}
We acknowledge support from the National Science Foundation (NSF) and the Simons Foundation for the Collaboration on the Theoretical Foundations of Deep Learning through awards DMS-2031883 and \#814639 as well as the  TILOS institute (NSF CCF-2112665) and the Office of Naval Research (ONR N000142412631). 
This work used the programs (1) XSEDE (Extreme science and engineering discovery environment)  which is supported by NSF grant numbers ACI-1548562, and (2) ACCESS (Advanced cyberinfrastructure coordination ecosystem: services \& support) which is supported by NSF grants numbers \#2138259, \#2138286, \#2138307, \#2137603, and \#2138296. Specifically, we used the resources from SDSC Expanse GPU compute nodes, and NCSA Delta system, via allocations TG-CIS220009. 

%%%%%%%%%%%%%%%%%%%%%%%%%%%%%%%%%%%%%%%%%%%%%%%%%%%%%%%%%%%%%%%%%%%%%%%%

%%% The acknowledgments section is defined using the "acks" environment
%%% (rather than an unnumbered section). The use of this environment 
%%% ensures the proper identification of the section in the article 
%%% metadata as well as the consistent spelling of the heading.

\begin{acks}
This publication is part of the CAUSES project (KIVI.2019.004) of the research programme Responsible Use of Artificial Intelligence which is financed by the Dutch Research Council (NWO)
and ProRail.
\end{acks}

%%%%%%%%%%%%%%%%%%%%%%%%%%%%%%%%%%%%%%%%%%%%%%%%%%%%%%%%%%%%%%%%%%%%%%%%

%%% The next two lines define, first, the bibliography style to be 
%%% applied, and, second, the bibliography file to be used.

\bibliographystyle{ACM-Reference-Format} 
\bibliography{bibliography}

%%%%%%%%%%%%%%%%%%%%%%%%%%%%%%%%%%%%%%%%%%%%%%%%%%%%%%%%%%%%%%%%%%%%%%%%


\end{document}

%%%%%%%%%%%%%%%%%%%%%%%%%%%%%%%%%%%%%%%%%%%%%%%%%%%%%%%%%%%%%%%%%%%%%%%%

