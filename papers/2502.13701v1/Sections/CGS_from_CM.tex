The goal of this paper is to define a systematic approach to generate a causal CGS based on a strongly recursive structural causal model. The motivation is that we want to compare the strategic ability of coalitions of agents to realise outcomes to causes in the causal model.
Similar translations have been attempted by \cite{gladyshev2023dynamics,baier2021game-theoretic} and \cite{hammond2023ReasoningCausalityGames}.

Gladyshev et al. make, like us, a distinction between agent and environment variables, and they also construct a CGS that takes the causal structure between agents' decision and environment variables into account \cite{gladyshev2023dynamics} . 
However, they take a `zoomed out' approach to the causal model by considering every state in the CGS as a causal model. 
In contrast, in this paper, we are interested in the specific variable values, which we will consider as specific actions in strategic setting.
Another difference with our work is that they do not look at the relationship between causality in the original causal model and strategies in the CGS.

A more similar approach to ours was defined by Baier et al. \cite{baier2021game-theoretic}, but they use extensive form games rather than CGS, and  do not distinguish between agent and environment variables. 
Furthermore, while they do show a result relating actual causality in the causal model to some type of strategy in their extensive form game, they only do this for but-for causes, where we consider the modified HP definition as well. 

Hammond et al. translate the causal model to a multi-agent influence diagram (MAID) that includes utility variables, with the primary goal of studying rational outcomes of the grand coalition \cite{hammond2023ReasoningCausalityGames}.
They hence take a game-theoretic approach, where we take a logic-based approach by focusing on strategic abilities of coalitions of agents.
Nevertheless, we could also apply a game-theoretic analysis to our model, by extending our CGS to include utility variables.
This is however beyond the scope of this paper.


\subsection{Defining a Causal CGS}
In this section we will propose a systematic approach to generate a causal concurrent game structure based on a strongly recursive structural causal model.
We will use the notion of first-level, second-level and higher-level variables as explained in the previous section to determine in which order the agents of the causal model will get to take actions.
For this we define the notion of agent rank:
\begin{definition}
    An \emph{agent ranking function} of a causal model $\mathcal{M}$ is a function $\rho: \mathcal{V} \rightarrow \set{0,...,n}$, where $n$ is the number of distinct variable levels for agent variables in $\mathcal{M}$, such that for all $A,B \in V_a$, $\rho(A) > \rho(B) > 0$ if and only if the variable level of $A$ is higher than the variable level of $B$, and $\rho(A) = \rho(B)$ if and only if $A$ and $B$ have the same variable level. 
    For all $X \in V_e$, $\rho(X) = \rho(A) - 1$ if $\exists A \in V_a$ such that the variable level of $X$ is lower or equal to the variable level of $A$, and there is no $B \in V_a$ that has a variable level between $X$ and $A$. If such an $A$ does not exist, i.e. if the variable level of $X$ is higher than the variable level of all $A \in V_a$, then $\rho(X) = n$.
    The \emph{agent rank} of a variable $A \in V_a$ is $\rho(A)$.
\end{definition}

\begin{example}\label{ex:agent rank}
    In the semi-automated vehicle example we say that $HD$, $ODS$ and $DA$ are the agent variables. 
    We have that $n = 2$ as $HD$ and $ODS$ are both second-level variables and $DA$ is a third level variable as Example \ref{ex:causal network} discusses. 
    There are hence $2$ distinct variable levels for the agent variables.
    From this, it follows that $\rho(HD) = \rho(ODS) = 1$ and $\rho(DA) = 2$, as the variable level of $DA$ is higher than that of $HD$ and $ODS$ and their agent rank needs to be higher than $0$ and maximally $2$.
    For the environment variables, we have that $\rho(O) = \rho(Att) = \rho(HD) - 1 = 0$, because there are no first-level agent variables, so we need a second-level agent variable like $HD$.
    Finally, we have that $\rho(Col) = 2$, since the variable level of $Col$ is $4$ which is higher than all agent variable levels, and hence the agent rank of $Col$ will be the maximum of $2$.
\end{example}

We will first define several components of the causal CGS separately before putting them all together.
From now on, we will assume that all causal models are recursive and have variables which can only attain finitely many values. Moreover we assume that a set of agent variables $V_a \subseteq \mathcal{V}$ is given. 

\begin{definition}[States of a causal CGS]\label{def:states causal CGS}
Given a causal setting $(\mathcal{M},\mathbf{u})$, let $n = \max_{Y\in V_a} \rho(Y)$ be the maximum value of the agent ranks for the agents in $V_a$ and let $m_i = \prod_{\substack{Y \in V_a,\\ \rho(Y) \leq i}} |\mathcal{R}(Y)|$  be the number of possible combinations of action values for agents with an agent rank of no more than $i$. 
The set of \emph{states of a causal CGS}, $Q$, generated based on  $(\mathcal{M},\mathbf{u})$, is given by:
    \begin{equation*}
        Q = \set{q_{0,0}} \cup \set{q_{i,j}\ |\ 1 \leq i \leq n \text{ and }  0 \leq j < m_i }.
    \end{equation*}
\end{definition}
We call $q_{0,0}$ the \emph{starting state} of the causal CGS.
Later, we will see that the evaluation in a state $q_{i,j}$ follows from the actions of agents whose agent variables have agent rank $i$ or less.
\begin{example}\label{ex:states causal CGS}
    We will use the causal model for the semi-automated vehicle example to define a causal CGS (see Figure \ref{fig:ex causal network}). 
    See Example \ref{ex:agent rank} for the agent rank of all variables of the causal model.
    We start with the setting $(\mathcal{M},\mathbf{u})$ with $\mathbf{u} = (U_{O} = 1,U_{Att} = 0)$.
    The set of states is then
    $Q = \{q_{0,0},q_{1,0},q_{1,1}, q_{1,2},q_{1,3},q_{2,0},q_{2,1},q_{2,2},q_{2,3},q_{2,4},$\\$q_{2,5},q_{2,6},q_{2,7}\}.$
    Note that: \newline
    $\prod_{Y \in V_a, \rho(Y) \leq 1} |\mathcal{R}(Y)| = \prod_{Y \in \set{HD,ODS}} |\mathcal{R}(Y)| = |\set{0,1}\times \set{0,1}| = 4$ and
    $\prod_{Y \in V_a, \rho(Y) \leq 2} |\mathcal{R}(Y)| = \prod_{Y \in \set{HD, ODS, DA}} |\mathcal{R}(Y)|  = 8$, so for $i = 1$, we have $j \in \{0,\ldots,3\}$ and for $i = 2$, we have $j \in \{0,\ldots,7\}$.
    These are all the states, because the maximum value of the agent rank $\rho$ is $2$.
\end{example}

We will now define the agent actions in those states.

\begin{definition}[Actions in a causal CGS]\label{def:actions causal CGS}
    Given a causal setting $(\mathcal{M},\mathbf{u})$ and $Q$ the corresponding set of states as defined by Definition \ref{def:states causal CGS}. The possible \emph{actions} for an agent $k \in \set{1,...,N}$ in a state $q_{i,j} \in Q$ are $d_k(q_{i,j}) = \mathcal{R}(A_k)$, where $A_k$ is the agent variable controlled by agent $k$, and $\rho(A_k) = i+1$. Otherwise $d_k(q_{i,j}) = \set{0}$.
\end{definition}

The intuition behind this definition is that agent variables that are earlier on a causal path will earlier get to take an action as the agent variables later on a causal path depend on them. The order of agent variables on a causal path can be seen as representing a protocol that determines when each agent has to take its action.   
We write $a_k$ to denote an action of agent $k \in N$ and $\mathbf{a}_{i,j} = \langle a_1,...,a_N\rangle$ to denote an action profile taken in a certain state $q_{i,j}$, i.e., all actions taken by all agents in state $q_{i,j}$. It is important to note that for a given index $i$ all states $q_{i,j}$ have the same action profiles that can be taken in them, regardless of the value of $j$. We denote this set with $\mathbf{A}_i$. 
Instead of $d_k$ for agent $k$, we will sometimes write $d_{A_k}$ for the agent variable $A_k$ corresponding to agent $k$.

\begin{example}\label{ex:actions causal CGS}
    We continue with the situation as in Example \ref{ex:states causal CGS}.
    The available actions for each agent in each state are: \begin{equation*}
        \begin{array}{ll}
            d_{{HD}}(q_{0,0}) = d_{{ODS}}(q_{0,0}) = \set{0,1}, & d_{{DA}}(q_{0,0}) = \emptyset, \\
            d_{{HD}}(q_{1,j}) = d_{{ODS}}(q_{1,j}) = \emptyset,  & d_{{DA}}(q_{1,j}) = \set{0,1}, \\
            \forall j \in \set{0,\ldots,3},  \text{ and} & \\
            d_{{HD}}(q_{2,j}) = d_{{ODS}}(q_{2,j}) = \emptyset,  & d_{{DA}}(q_{2,j}) = \emptyset, \\
            \forall j \in \set{0,\ldots,7}.&
        \end{array}
    \end{equation*}
\end{example}

These actions must of course lead to transitions to new states. 
\begin{definition}[Transitions in a causal CGS]\label{def:transitions causal CGS}
    Given a causal setting $(\mathcal{M},\mathbf{u})$, $Q$ the corresponding set of states as defined by Definition \ref{def:states causal CGS} and actions as defined by Definition \ref{def:actions causal CGS}, the state following from the action profile $\mathbf{a}_{i,j} \in \mathbf{A}_i$, with $i < \max_{X\in V_a} \rho(X)$, is given by the transition function $\delta$, where $\delta(q_{i,j},\mathbf{a}_{i,j}) = q_{i+1,j'}$ and $|\mathbf{A}_i| \cdot j \leq j'\leq |\mathbf{A}_i| \cdot (j+1) -1$, under the condition that if $\mathbf{a}_{i,j} \neq \mathbf{a}'_{i,j}$, then $\delta(q_{i,j},\mathbf{a}_{i,j}) \neq \delta(q_{i,j}, \mathbf{a}'_{i,j})$.
    If $i = \max_{X\in V_a} \rho(X)$, we define $\delta(q_{i,j},\mathbf{a}_{i,j}) = q_{i,j}$. In this case, there is only one possible action profile $\mathbf{a}_{i,j}$ consisting of only the $0$ action. 
\end{definition}
This definition simply says that every unique action profile in a state leads to a unique new state. This leads to the causal CGS having a tree structure. It is impossible to return to an earlier state and every node can only branch out
\begin{example}\label{ex:transitions causal CGS}
    Continuing with our running example, we will write $\langle 1, 0, 0 \rangle$ for the action profile $\langle HD = 1, ODS = 0, DA = 0 \rangle$. 
    We get that the transitions are:
    \newline 
    \begin{equation*}
        \begin{array}{ll}
            \delta(q_{0,0}, \langle 0, 0, 0\rangle) = q_{1,0},  & 
            \delta(q_{0,0}, \langle 0, 1, 0\rangle) = q_{1,1}, \\
            \delta(q_{0,0}, \langle 1, 0, 0\rangle) = q_{1,2},  & 
            \delta(q_{0,0}, \langle 1, 1, 0\rangle) = q_{1,3}, \\
            \delta(q_{1,0},\langle 0, 0, 0\rangle) = q_{2,0}, &
            \delta(q_{1,0}, \langle 0, 0, 1\rangle) = q_{2,1}, \\
            \delta(q_{1,1},\langle 0, 0, 0\rangle) = q_{2,2}, &
            \delta(q_{1,1}, \langle 0, 0, 1\rangle) = q_{2,3}, \\
            \delta(q_{1,2},\langle 0, 0, 0\rangle) = q_{2,4}, &
            \delta(q_{1,2}, \langle 0, 0, 1\rangle) = q_{2,5}, \\
            \delta(q_{1,3},\langle 0, 0, 0\rangle) = q_{2,6}, &
            \delta(q_{1,3}, \langle 0, 0, 1\rangle) = q_{2,7}, \\
            \delta(q_{2,j},\langle 0, 0, 0\rangle) = q_{2,j} & \forall j \in \set{0,\ldots,7}.
        \end{array}
    \end{equation*}
\end{example}

Now that we have states, actions and transitions, we just need the evaluations of the states.
The evaluation of a state will depend on an initial causal setting and the actions the agents have taken up to this state. The agents fully determine the values of the agent variables, the environment variables follow from these values and the context that was used to define the causal CGS. 

\begin{definition}[Evaluation of states in a causal CGS]\label{def:evaluations in a causal CGS}
    Given a causal setting, $(\mathcal{M},\mathbf{u})$, the set of all possible propositions for the generated causal CGS is $\Pi = \{ X= x \ | \ X \in \mathcal{V}, x \in \mathcal{R}(X)\}$.
    The valuation of each state $q_{i,j} \in Q$, with $Q$ the set of states of the causal CGS according to Definition \ref{def:states causal CGS}, is defined recursively by the \emph{labelling function} $\pi$, as:
    \begin{equation*}
        \begin{array}{ll}
            \pi(q_{0,0})  &= \set{Y = y \ | \ \csetting \models Y = y} \\
            \pi(\delta(q_{i,j}, \myvec{a}_{i,j})) &= \set{Y = y \ | \ \csettingint{\myvec{X}_{i,j}\leftarrow \myvec{x}_{i,j}, \myvec{A}_{i,j} \leftarrow \myvec{a}_{i,j}}\models Y = y},
        \end{array}
    \end{equation*}
    where $\myvec{a}_{i,j}$ is an action profile for state $q_{i,j}$, $\myvec{A}_{i,j} \leftarrow \myvec{a}_{i,j} := \{A_k\leftarrow a_k \ | \ A_k \in V_a, \rho(A_k) = i+1 \text{ and } a_k \in \myvec{a}_{i,j}\}$ is an intervention constructed based on action profile $\myvec{a}_{i,j}$, and $\myvec{X}_{i,j}\leftarrow \myvec{x}_{i,j}$ is recursively defined by: $\myvec{X}_{i+1,j'} \leftarrow \myvec{x}_{i+1,j'}:= \myvec{X}_{i,j}\leftarrow \myvec{x}_{i,j} \cup \myvec{A}_{i,j}\leftarrow \myvec{a}_{i,j}$, if $\delta(q_{i,j}, \myvec{a}_{i,j}) = q_{i+1,j'}$ with $\myvec{X}_{0,0} \leftarrow \myvec{x}_{0,0}= \emptyset$.
\end{definition}

Definition \ref{def:evaluations in a causal CGS} says that an agent action leads to an intervention on the causal setting the causal CGS was based upon. 
We can see $\mathbf{A}_{i,j} \leftarrow \mathbf{a}_{i,j}$ as the intervention that directly follows from the agent action(s) taken in the state $q_{i,j}$, $\mathbf{X}_{i,j} \leftarrow \mathbf{x}_{i,j}$ stores the previous interventions that were made leading up to the state $q_{i,j}$.
We will illustrate this in the following example.
\begin{example}\label{ex:evaluations CGS}
    We continue with the situation as in Example \ref{ex:transitions causal CGS}. 
    We start with the causal setting where $U_O = 1$ and $U_{Att} = 0$, so $\pi(q_{0,0}) = \set{O, \neg Att, HD, ODS, \neg DA, \neg Col}$.
    To determine $\pi(q_{1,0}) = \pi(\delta(q_{0,0}, \langle 0,0,0 \rangle ))$, we need $\myvec{A}_{0,0} \leftarrow \myvec{a}_{0,0} = \set{HD \leftarrow 0, ODS \leftarrow 0}$. This gives us that $\pi(q_{1,0}) = \set{Y = y \ | \ \csettingint{HD \leftarrow 0, ODS \leftarrow 0} \models Y = y} $ $= \set{O, \neg Att, \neg HD, \neg ODS, \neg DA, \neg Col}$.
    Similarly we can determine that $\pi(q_{1,1}) = \{O, \neg Att, \neg HD,  ODS, \neg DA, \neg Col\}$, $\pi(q_{1,2}) = \{O, \neg Att, $\\$ HD, \neg ODS, DA, Col\}$ and $\pi(q_{1,3}) = \set{O, \neg Att, HD,  ODS, \neg DA, \neg Col}$.

    Let us now look at $\pi(q_{2,1}) = \pi(\delta(q_{1,0}, \langle 0,0,1\rangle))$.
    We need $\myvec{X}_{1,0} \leftarrow \myvec{x}_{1,0} = (\myvec{X}_{0,0} \leftarrow \myvec{x}_{0,0} \ \cup \ \myvec{A}_{0,0} \leftarrow \myvec{a}_{0,0}) =  \emptyset \ \cup \set{HD \leftarrow 0, ODS \leftarrow 0}$ as we determined above.
    The new $\myvec{A}_{1,0} \leftarrow \myvec{a}_{1,0} = \set{DA \leftarrow 1}$ and so $\pi(q_{2,1}) = \set{Y = y \ | \ \csettingint{HD \leftarrow 0, ODS \leftarrow 0, DA \leftarrow 1} \models Y = y } = \{O, \neg Att, \neg HD, \neg ODS, DA, \neg Col\}$.
    The valuations for the other states are determined similarly (and are shown in Figure \ref{fig:causal cgs vehicle}).

\end{example}

Now that we have these four definitions, we can give the full definition of a causal CGS.

\begin{definition}[Causal CGS]\label{def:causal CGS}
    Given a causal setting, $(\mathcal{M},\mathbf{u})$, a \emph{causal concurrent game structure} is defined as a tuple $GS = \langle N, Q, d,$\\$ \delta, \Pi, \pi\rangle$ where $N = |V_a|$, every agent only controls one agent variable, $Q$ is a set of states, as defined by Definition \ref{def:states causal CGS}. For every agent $k \in \set{1,...,N}$, $d_k(q_{i,j})$ gives the moves available to this agent in state $q_{i,j} \in Q$, as given by Definition \ref{def:actions causal CGS}.
    The transition function $\delta$ is defined as in Definition \ref{def:transitions causal CGS}.
    The set of possible propositions $\Pi$ and the valuation function $\pi$ are given by Definition \ref{def:evaluations in a causal CGS}.
\end{definition}

We can now add the results of the previous examples together and give a full causal CGS for the semi-automated vehicle example.

\begin{example}\label{ex:cgs rock-throwing}
     Using Definition \ref{def:causal CGS}, we define $N = |V_a| = |\set{HD, ODS, DA}| = 3$. This gives us a full causal CGS, illustrated in Figure \ref{fig:causal cgs vehicle}.

     \begin{figure}[ht]
    \centering
    \setlength{\unitlength}{0.9cm}
    \begin{picture}(7,7.3)(-0.25,0.8)

        \put(0,4.5){\circle{0.8}}
        \put(2,2){\circle{0.8}}
        \put(2,4){\circle{0.8}}
        \put(2,5){\circle{0.8}}
        \put(2,7){\circle{0.8}}
        \put(4,1){\circle{0.8}}
        \put(4,2){\circle{0.8}}
        \put(4,3){\circle{0.8}}
        \put(4,4){\circle{0.8}}
        \put(4,5){\circle{0.8}}
        \put(4,6){\circle{0.8}}
        \put(4,7){\circle{0.8}}
        \put(4,8){\circle{0.8}}
        
        \put(-0.23,4.5){\makebox(0,0)[l]{\footnotesize{$q_{0,0}$}}}
        \put(1.77,7){\makebox(0,0)[l]{\footnotesize{$q_{1,0}$}}}
        \put(1.77,5){\makebox(0,0)[l]{\footnotesize{$q_{1,1}$}}}
        \put(1.77,4){\makebox(0,0)[l]{\footnotesize{$q_{1,2}$}}}
        \put(1.77,2){\makebox(0,0)[l]{\footnotesize{$q_{1,3}$}}}
        \put(3.77,8){\makebox(0,0)[l]{\footnotesize{$q_{2,0}$}}}
        \put(3.77,7){\makebox(0,0)[l]{\footnotesize{$q_{2,1}$}}}
        \put(3.77,6){\makebox(0,0)[l]{\footnotesize{$q_{2,2}$}}}
        \put(3.77,5){\makebox(0,0)[l]{\footnotesize{$q_{2,3}$}}}
        \put(3.77,4){\makebox(0,0)[l]{\footnotesize{$q_{2,4}$}}}
        \put(3.77,3){\makebox(0,0)[l]{\footnotesize{$q_{2,5}$}}}
        \put(3.77,2){\makebox(0,0)[l]{\footnotesize{$q_{2,6}$}}}
        \put(3.77,1){\makebox(0,0)[l]{\footnotesize{$q_{2,7}$}}}

        \put(0.25,4.8){\vector(3,4){1.45}}
        \put(0.4,4.6){\vector(4,1){1.2}}
        \put(0.4,4.4){\vector(4,-1){1.2}}
        \put(0.25,4.2){\vector(3,-4){1.45}}
        \put(2.35,7.2){\vector(2,1){1.3}}
        \put(2.4,7){\vector(1,0){1.2}}
        \put(2.35,5.2){\vector(2,1){1.3}}
        \put(2.4,5){\vector(1,0){1.2}}
        \put(2.4,4){\vector(1,0){1.2}}
        \put(2.35,3.8){\vector(2,-1){1.3}}
        \put(2.4,2){\vector(1,0){1.2}}
        \put(2.35,1.8){\vector(2,-1){1.3}}

        \put(0.45,5.5){\rotatebox{53}{\footnotesize{$\langle 0, 0, 0\rangle$}}}
        \put(0.4,4.7){\rotatebox{14}{\footnotesize{$\langle 0, 1, 0\rangle$}}}
        \put(0.4,4.13){\rotatebox{-14}{\footnotesize{$\langle 1, 0, 0\rangle$}}}
        \put(0.45,3.4){\rotatebox{-53}{\footnotesize{$\langle 1, 1, 0\rangle$}}}
        \put(2.5,7.4){\rotatebox{26.5}{\footnotesize{$\langle 0, 0, 0\rangle$}}}
        \put(3,7){\makebox(0,0)[b]{\footnotesize{$\langle 0, 0, 1\rangle$}}}
        \put(2.5,5.4){\rotatebox{26.5}{\footnotesize{$\langle 0, 0, 0\rangle$}}}
        \put(3,5){\makebox(0,0)[b]{\footnotesize{$\langle 0, 0, 1\rangle$}}}
        \put(3,3.95){\makebox(0,0)[t]{\footnotesize{$\langle 0, 0, 0\rangle$}}}
        \put(2.5,3.45){\rotatebox{-26.5}{\footnotesize{$\langle 0, 0, 1\rangle$}}}
        \put(3,1.95){\makebox(0,0)[t]{\footnotesize{$\langle 0, 0, 0\rangle$}}}
        \put(2.5,1.45){\rotatebox{-26.5}{\footnotesize{$\langle 0, 0, 1\rangle$}}}

        \put(-0.5,4.5){\makebox(0,0)[r]{\scriptsize{$\set{O,\neg Att}$}}}
        \put(1.7,7.5){\makebox(0,0)[b]{\scriptsize{$\set{\neg HD, \neg ODS}$}}}
        \put(1.73,5.5){\makebox(0,0)[b]{\scriptsize{$\set{\neg HD, ODS}$}}}
        \put(1.73,3.5){\makebox(0,0)[t]{\scriptsize{$\set{ HD, \neg ODS}$}}}
        \put(1.73,1.5){\makebox(0,0)[t]{\scriptsize{$\set{ HD, ODS}$}}}
        
        \put(4.43,8){\makebox(0,0)[l]{\scriptsize{$\set{O,\neg Att, \neg HD, \neg ODS, \neg DA, \neg Col}$}}}
        \put(4.43,7){\makebox(0,0)[l]{\scriptsize{$\set{O,\neg Att, \neg HD, \neg ODS, DA, \neg Col}$}}}
        \put(4.43,6){\makebox(0,0)[l]{\scriptsize{$\set{O,\neg Att, \neg HD, ODS, \neg DA, \neg Col}$}}}
        \put(4.43,5){\makebox(0,0)[l]{\scriptsize{$\set{O,\neg Att, \neg HD, ODS, DA, \neg Col}$}}}
        \put(4.43,4){\makebox(0,0)[l]{\scriptsize{$\set{O,\neg Att, HD, \neg ODS, \neg DA, \neg Col}$}}}
        \put(4.43,3){\makebox(0,0)[l]{\scriptsize{$\set{O,\neg Att,  HD, \neg ODS, DA, Col}$}}}
        \put(4.43,2){\makebox(0,0)[l]{\scriptsize{$\set{O,\neg Att, HD, ODS, \neg DA, \neg Col}$}}}
        \put(4.43,1){\makebox(0,0)[l]{\scriptsize{$\set{O,\neg Att,  HD, ODS, DA, Col}$}}}
        
    \end{picture}
    \caption{The causal CGS of the semi-automated vehicle example. We only show the initial values of the variables of agent rank $0$ in the starting state. In the middle states we only show the variables with agent rank corresponding to that state. We also do not show the transitions to the same state in the leaf-states.}
    \label{fig:causal cgs vehicle}
    \Description{A graph depicting the causal concurrent game structure for our running semi-automated vehicle example. The graph has a tree structure with maximal depth 2, there are 4 nodes of depth 1 and 8 leaf nodes. The root has 4 edges leaving it, the depth 1 nodes each have 2 edges leaving it. The root node is named q with the subscript 0,0, it is labelled with the propositions O and negation of Att. The four edges leaving the root node are labelled respectively 0,0,0; 0,1,0; 1,0,0; and 1,1,0. The four depth 1 nodes are named q with the subscripts 1,0 up until 1,3. They are labelled with the propositions: negation of HD, negation of ODS; negation of HD, ODS; HD, negation of ODS; and HD, ODS, respectively. For each of the depth 1 nodes, the edges leaving it are labelled 0,0,0 and 0,0,1. The leaf nodes are named q with the subscript 2,0 up and until 2,7. They are each respectively labelled with the propositions: O, negation of Att, negation of HD, negation of ODS, negation of DA, negation of Col; O, negation of Att, negation of HD, negation of ODS, DA, negation of Col;O, negation of Att, negation of HD, ODS, negation of DA, negation of Col; O, negation of Att, negation of HD, ODS, DA, negation of Col; O, negation of Att, HD, negation of ODS, negation of DA, negation of Col; O, negation of Att, HD, negation of ODS, DA, Col; O, negation of Att, HD, ODS, negation of DA, negation of Col; O, negation of Att, HD, ODS, DA, Col.}
\end{figure}
\end{example}



\subsection{Properties of Causal Concurrent Game Structures}
We already mentioned that a causal CGS has a tree structure. In the rest of this paper, we will call states $q_{i,j}$, with $i = \max_{X\in \mathcal{V}} \rho(X)$, the \emph{leaf-states}. 
We will call actions in states where an agent does not control a variable, i.e. $a_k = 0$, when $d_k(q_{i,j}) = \set{0}$, with $\rho(X) \neq i + 1$, \emph{no-op actions}. It is also useful to define an \emph{action path} for a state $q_{i,j}$, that contains all the non no-op actions that led to the state. In other words, the action path contains only the actions that agents took in a state where they could actually choose an action. We will denote this sequence of actions as $\alpha[q_{i,j}]$. 
Formally, for $0 \leq k \leq N$, an action $a_k$ is in this set of actions  $\alpha[q_{i,j}]$ if and only if $\rho(A_k)\leq i$ and there exists an action profile $\mathbf{a}_{i',j'}$, containing $a_k$, such that $q_{i',j'} \in \lambda[q_{i,j} , i]$  (the history of $q_{i,j}$) and $\delta(q_{i',j'} , \mathbf{a}_{i',j'}) \in \lambda[q_{i,j} , i]$. In other words, an action is on the action path for a state $q_{i,j}$, if the state $q_{i',j'}$ in which the action is taken lies on the history of $q_{i,j}$, and the successor of $q_{i',j'}$ can be reached when taking this action.

Our first result is on the size of the causal CGS.
\begin{proposition}
Let $\mathcal{M} = (\mathcal{S},\mathcal{F})$ be a causal model. The size of the causal CGS generated by $\mathcal{M}$ is linear in the size of the extension of $\mathcal{F}$.
\end{proposition}
\begin{proof}
Consider a structural causal model $\mathcal{M} = (\mathcal{S},\mathcal{F})$. Observe that $\mathcal{F}$ specifies the value of each variable for all possible combinations of values of all other variables. Hence $\mathcal{F}$ corresponds to a table of size $|\mathcal{V}| \times \prod_{X \in \mathcal{V}} |\mathcal{R}(X)|$ (the number of cells), which is actually the extension of $\mathcal{F}$. We now show that the number of states in the causal CGS is $O( \prod_{Y \in V_a} |\mathcal{R}(Y)|)$.

By Definition \ref{def:states causal CGS} we have that the number of states of the causal CGS, is given by $|Q| = 1+ \sum_{i = 1}^{n} \prod_{\substack{Y \in V_a,\\ \rho(Y) \leq i}} |\mathcal{R}(Y)|$, where $n = \max_{Y\in V_a} \rho(Y)$. The number of leaf-states is hence given by $\prod_{\substack{Y \in V_a}} |\mathcal{R}(Y)| =: |R(V_a)|$.
The number of states for $i = n-1$ will be at most half $|R(V_a)|$, as there will be at least one variable of rank $n$ that is hence not included in $\prod_{\substack{Y \in V_a,\\ \rho(Y) \leq n-1}} |\mathcal{R}(Y)|$, and this variable will have at least two possible values. We can continue this argument until $i = 1$, which shows us that $|Q|$ is bounded by $1 + \frac{1}{2^{n-1}} |R(V_a)|+\dots +\frac{1}{2} |R(V_a)|+|R(V_a)| \leq 2 |R(V_a)|$. 
Hence the number of states in the causal CGS is $O( \prod_{Y \in V_a} |\mathcal{R}(Y)|)$.
Since a causal CGS is a tree and each state has at most one predecessor, the number of transitions (the size of $\delta$) is also 
$O( \prod_{Y \in V_a} |\mathcal{R}(Y)|)$, hence linear in the size of $\mathcal{F}$ in the original model.
\end{proof}


The statement in the following lemma is a direct consequence of the way the valuation of states is determined in a causal CGS. It states that a variable value cannot change in states corresponding to a higher agent rank than the agent rank of the variable itself.

\begin{lemma}\label{lem:no change after i}
    Let $GS$ be a causal CGS generated by the causal model $\mathcal{M}$. For any endogenous causal variable $X \in \mathcal{V}$ of $\mathcal{M}$, with $\rho(X) = i$, it holds that $(X = x) \in \pi(q_{i,j})$ for some state $q_{i,j}$ of $GS$, if and only if $(X = x) \in \pi(q_{i',j'})$ for all states $q_{i',j'}$ that are descendants of $q_{i,j}$.
\end{lemma}
\begin{proof}
    Let $(X = x) \in \pi(q_{i,j})$. 
    Variable values can change in a state due to interventions, but the only new interventions done in states descended from $q_{i,j}$ are interventions on variables with an agent rank higher than $i$. 
    $X$ has agent rank $i$, so by the definition of agent rank none of those variables can be ancestors of $X$. They are hence unable to influence the value of $X$. 
    Therefore $(X = x) \in \pi(q_{i',j'})$ for all states $q_{i',j'}$ descended from $q_{i,j}$.

    Now, let $(X=x)\in \pi(q_{i',j'})$ for all states $q_{i',j'}$ that are descended from $q_{i,j}$.
    The value of $X$ was not changed in any of those states, because the value of $X$ can only change due to an intervention on $X$ or an ancestor variable of $X$, so only due to variables of agent rank smaller or equal to $\rho(X)$.
    The only interventions on variables that happen in the descendants of $q_{i,j}$ are on variables of agent rank higher than $\rho(X)$, hence $X$ must have had the same value in $q_{i,j}$, i.e. $(X = x) \in \pi(q_{i,j})$.
\end{proof}

We define the notion of \emph{correspondence} to talk about how states in a causal CGS connect to a causal model.
\begin{definition}[Correspondence]\label{def:correspondence}
    We say that a state $q_{i,j}$ of a causal CGS \emph{corresponds} to a causal setting 
    $(\mathcal{M}^{\myvec{Y} \leftarrow \myvec{y}},\mathbf{u})$, where $\myvec{Y} \subseteq \mathcal{V}$, if for all causal variables $X$ of $\mathcal{M}$,
    $(X = x) \in \pi(q_{i,j})$ if and only if $(\mathcal{M}^{\myvec{Y} \leftarrow \myvec{y}},\mathbf{u}) \models X = x$.\footnote{So the causal variable $X$ has value $x$ in the causal setting $(\mathcal{M}^{\myvec{Y} \leftarrow \myvec{y}},\mathbf{u})$.}
\end{definition}
We will sometimes say that a causal setting $(\mathcal{M}^{\myvec{Y} \leftarrow \myvec{y}},\mathbf{u})$ corresponds to a state $q_{i,j}$ of a causal CGS and mean the same thing.
Note that the set $\myvec{Y}$ could also be empty. Hence the causal model $\mathcal{M}^{\myvec{Y} \leftarrow \myvec{y}}$ in Definition \ref{def:correspondence} could also be $\mathcal{M}$.

We can show that a leaf-state of a causal CGS corresponds to a causal setting $\csettingint{\myvec{Y} \leftarrow \myvec{y}}$, where $\myvec{Y} \leftarrow \myvec{y}$ depends on the action path that leads to the leaf-state. 
This connects the definition of causal CGS to the theory of causal models.
\begin{proposition}\label{prop:state correspondence}
    Let $GS$ be a causal CGS generated by a causal setting $\csetting$. If $q_{n,m}$ is a leaf-state of $GS$, then $q_{n,m}$ corresponds to the causal setting $(\mathcal{M}^{\myvec{Y}\leftarrow \myvec{y}}, \mathbf{u})$, where $\myvec{Y} \leftarrow \myvec{y} = \{A_k \leftarrow a_k \ | \ A_k \in V_a \text{ and } a_k \in \alpha[q_{n,m}]\}$, with $\alpha[q_{n,m}]$ the action path for $q_{n,m}$.
\end{proposition}
\begin{proof}
    By Definition \ref{def:evaluations in a causal CGS}, $(X = x) \in \pi(q_{n,m})$ if and only if $\csettingint{\myvec{X}_{i,j}\leftarrow \myvec{x}_{i,j}, \myvec{A} \leftarrow \myvec{a}}\models X = x$, 
    where $\myvec{A} \leftarrow \myvec{a}$ are the actions taken in the state before $q_{n,m}$, and $\myvec{X}_{i,j}\leftarrow \myvec{x}_{i,j}$ are all previously taken actions. Hence $\myvec{Y} \leftarrow \myvec{y} = (\myvec{A} \leftarrow \myvec{a}) \cup \myvec{X}_{i,j}\leftarrow \myvec{x}_{i,j}$ and the proposition is proven.
\end{proof}

This gives us a solid grasp on how a causal CGS relates to the causal model that generates it.
We will use this in the next section when we talk about the connection between agent strategies in a causal CGS and causality in this structural causal model.
