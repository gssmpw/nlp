Now that we have defined causal concurrent game structures and shown what their states represent, it is time to look at how we can use them.
In this section, we will show some relations between causal CGS and the modified HP definition of actual causality, but we first introduce the notion of a causal strategy profile.

From now on, we will denote the set of all agents in a model by $\Sigma$. 
Specifically, for a causal CGS, $\Sigma = \set{k\ | \ X_k \in V_a}$. This set will also be called the \emph{grand coalition} at times.
We will use the notation $F_{X_k = x}$ to denote the strategy for agent $k$ where it takes action $x$ as its non no-op action.
Formally,
\[
    F_{X_k = x}(q_{i,j}) = \left\{ \begin{array}{lll}
        x & \text{if } &  \rho(X_k) = i+1 \\
         0 & \text{else}&
    \end{array}\right.
\]
For a set of agents $\myvec{X}$, we write $F_{\myvec{X} = \myvec{x}}$ to indicate the set of strategies $\{F_{X_k = x} \ | \ $ $ X_k \in \myvec{X}, x \in \myvec{x}\}$.
Let $F_A$ be a strategy for a set of agents $A$, and $F_B$ a strategy for a set of agents $B$. 
Following notation in \cite{Brihaye_DaCosta_Laroussinie_Markey_2008}, we will write $F_A \circ F_B$ to denote a strategy profile for the agents in $A \cup B$ that follows strategy $F_A$ for agents in $A$ and strategy $F_B$ for agents in $B \bs A$.



We define the causal strategy profile as a way to capture the `normal' behaviour of agents when they would follow the causal model.
\begin{definition}[Causal Strategy Profile]\label{def:complete causal strat profile}
Given a causal setting $(\mathcal{M},\mathbf{u})$ and the causal CGS generated by this setting.
Define the \emph{causal strategy profile} $F_\mathcal{M}$ as $F_\mathcal{M} = \set{F_{X_k}\ |\ k \in \Sigma}$, where $F_{X_k}(q_{i,j}) = 0$ if $\rho(X_k) \neq i+1$, and $F_{X_k}(q_{i,j}) = x_k$ otherwise, where $x_k$ is such that $(\mathcal{M},\mathbf{u}) \models [\myvec{X}\leftarrow\myvec{x}] X_k = x_k$, with $\myvec{X} = \set{X_{k'} \ | \ \rho(X_{k'})<\rho(X_k)}$ and $\myvec{x} = \set{x_{k'} \ | \ x_{k'} \in \alpha[q_{i,j}]}$.
\end{definition}
Recall that $\alpha[q_{i,j}]$ is the action path up to state $q_{i,j}$.
If we want an agent $k$ to follow a strategy $F_k$ and the rest of the agents to follow the causal strategy profile, we denote this as $F_k \circ F_\mathcal{M}$.
If a set of agents follows the causal strategy profile, that means that in every state, the agents take the actions that assign those values to the agent variables that they would also have gotten in the causal setting on which the causal CGS is based, given the actions of the other agents.

\begin{example}
    In the semi-automated vehicle example, given the setting where $U_O = 1$ and $U_{Att} = 0$, the causal strategy profile $F_\mathcal{M}$ is such that the human driver does not brake, but the obstacle detection system detects the obstacle.
    The driving assistant brakes in this case, but whenever one of the $HD$ or $ODS$ performs another action, $DA$ does not brake. The causal strategy profile for a causal CGS generated by this causal setting is given in Figure \ref{fig:causal cgs vehicle strategy}.
    \begin{figure}[h]
    \centering
    \setlength{\unitlength}{0.9cm}
    \begin{picture}(7,7.3)(-0.25,0.8)

        \multiput(0.25,4.8)(0.09,0.12){16}{\circle*{0.01}}
        \put(1.7,6.7){\vector(3,4){0}}
        \multiput(0.4,4.6)(0.12,0.03){10}{\circle*{0.01}}
        \put(1.6,4.9){\vector(4,1){0}}
        \multiput(0.4,4.4)(0.12,-0.03){10}{\circle*{0.01}}
        \put(1.6,4.1){\vector(4,-1){0}}
        \put(0.25,4.2){\vector(3,-4){1.45}}
        \put(2.35,7.2){\vector(2,1){1.3}}
        \multiput(2.4,7)(0.12,0){10}{\circle*{0.01}}
        \put(3.6,7){\vector(1,0){0}}
        \put(2.35,5.2){\vector(2,1){1.3}}
        \multiput(2.4,5)(0.12,0){10}{\circle*{0.01}}
        \put(3.6,5){\vector(1,0){0}}
        \multiput(2.4,4)(0.12,0){10}{\circle*{0.01}}
        \put(3.6,4){\vector(1,0){0}}
        \put(2.35,3.8){\vector(2,-1){1.3}}
        \put(2.4,2){\vector(1,0){1.2}}
        \multiput(2.35,1.8)(0.1,-0.05){13}{\circle*{0.01}}
        \put(3.65,1.15){\vector(2,-1){0}}

        \put(0,4.5){\circle{0.8}}
        \put(2,2){\circle{0.8}}
        \put(2,4){\circle{0.8}}
        \put(2,5){\circle{0.8}}
        \put(2,7){\circle{0.8}}
        \put(4,1){\circle{0.8}}
        \put(4,2){\circle{0.8}}
        \put(4,3){\circle{0.8}}
        \put(4,4){\circle{0.8}}
        \put(4,5){\circle{0.8}}
        \put(4,6){\circle{0.8}}
        \put(4,7){\circle{0.8}}
        \put(4,8){\circle{0.8}}
        
        \put(-0.23,4.5){\makebox(0,0)[l]{\footnotesize{$q_{0,0}$}}}
        \put(1.77,7){\makebox(0,0)[l]{\footnotesize{$q_{1,0}$}}}
        \put(1.77,5){\makebox(0,0)[l]{\footnotesize{$q_{1,1}$}}}
        \put(1.77,4){\makebox(0,0)[l]{\footnotesize{$q_{1,2}$}}}
        \put(1.77,2){\makebox(0,0)[l]{\footnotesize{$q_{1,3}$}}}
        \put(3.77,8){\makebox(0,0)[l]{\footnotesize{$q_{2,0}$}}}
        \put(3.77,7){\makebox(0,0)[l]{\footnotesize{$q_{2,1}$}}}
        \put(3.77,6){\makebox(0,0)[l]{\footnotesize{$q_{2,2}$}}}
        \put(3.77,5){\makebox(0,0)[l]{\footnotesize{$q_{2,3}$}}}
        \put(3.77,4){\makebox(0,0)[l]{\footnotesize{$q_{2,4}$}}}
        \put(3.77,3){\makebox(0,0)[l]{\footnotesize{$q_{2,5}$}}}
        \put(3.77,2){\makebox(0,0)[l]{\footnotesize{$q_{2,6}$}}}
        \put(3.77,1){\makebox(0,0)[l]{\footnotesize{$q_{2,7}$}}}

        \put(0.45,5.5){\rotatebox{53}{\footnotesize{$\langle 0, 0, 0\rangle$}}}
        \put(0.4,4.7){\rotatebox{14}{\footnotesize{$\langle 0, 1, 0\rangle$}}}
        \put(0.4,4.13){\rotatebox{-14}{\footnotesize{$\langle 1, 0, 0\rangle$}}}
        \put(0.45,3.4){\rotatebox{-53}{\footnotesize{$\langle 1, 1, 0\rangle$}}}
        \put(2.5,7.4){\rotatebox{26.5}{\footnotesize{$\langle 0, 0, 0\rangle$}}}
        \put(3,7){\makebox(0,0)[b]{\footnotesize{$\langle 0, 0, 1\rangle$}}}
        \put(2.5,5.4){\rotatebox{26.5}{\footnotesize{$\langle 0, 0, 0\rangle$}}}
        \put(3,5){\makebox(0,0)[b]{\footnotesize{$\langle 0, 0, 1\rangle$}}}
        \put(3,3.95){\makebox(0,0)[t]{\footnotesize{$\langle 0, 0, 0\rangle$}}}
        \put(2.5,3.45){\rotatebox{-26.5}{\footnotesize{$\langle 0, 0, 1\rangle$}}}
        \put(3,1.95){\makebox(0,0)[t]{\footnotesize{$\langle 0, 0, 0\rangle$}}}
        \put(2.5,1.45){\rotatebox{-26.5}{\footnotesize{$\langle 0, 0, 1\rangle$}}}

        \put(-0.5,4.5){\makebox(0,0)[r]{\scriptsize{$\set{O,\neg Att}$}}}
        \put(1.7,7.5){\makebox(0,0)[b]{\scriptsize{$\set{\neg HD, \neg ODS}$}}}
        \put(1.73,5.5){\makebox(0,0)[b]{\scriptsize{$\set{\neg HD, ODS}$}}}
        \put(1.73,3.5){\makebox(0,0)[t]{\scriptsize{$\set{ HD, \neg ODS}$}}}
        \put(1.73,1.5){\makebox(0,0)[t]{\scriptsize{$\set{ HD, ODS}$}}}
        
        \put(4.43,8){\makebox(0,0)[l]{\scriptsize{$\set{O,\neg Att, \neg HD, \neg ODS, \neg DA, \neg Col}$}}}
        \put(4.43,7){\makebox(0,0)[l]{\scriptsize{$\set{O,\neg Att, \neg HD, \neg ODS, DA, \neg Col}$}}}
        \put(4.43,6){\makebox(0,0)[l]{\scriptsize{$\set{O,\neg Att, \neg HD, ODS, \neg DA, \neg Col}$}}}
        \put(4.43,5){\makebox(0,0)[l]{\scriptsize{$\set{O,\neg Att, \neg HD, ODS, DA, \neg Col}$}}}
        \put(4.43,4){\makebox(0,0)[l]{\scriptsize{$\set{O,\neg Att, HD, \neg ODS, \neg DA, \neg Col}$}}}
        \put(4.43,3){\makebox(0,0)[l]{\scriptsize{$\set{O,\neg Att,  HD, \neg ODS, DA, Col}$}}}
        \put(4.43,2){\makebox(0,0)[l]{\scriptsize{$\set{O,\neg Att, HD, ODS, \neg DA, \neg Col}$}}}
        \put(4.43,1){\makebox(0,0)[l]{\scriptsize{$\set{O,\neg Att,  HD, ODS, DA, Col}$}}}
        
    \end{picture}
    \caption{The causal CGS of the semi-automated vehicle example. The dotted lines indicate actions that are not following the causal strategy profile.}
    \label{fig:causal cgs vehicle strategy}
    \Description{The same graph for the causal concurrent game structure of the semi-automated vehicle example as before, with the difference that this picture has drawn some edges with dotted lines. These edges are the edges from q subscript 0,0 to q subscript 1,0 up until 1,2, and the edges from q subscript 1,0 to q subscript 2,1, from q subscript 1,1 to q subscript 2,3, from q subscript 1,2 to q subscript 2,4, and from q subscript 1,3 to q subscript 2,7.}
\end{figure}
\end{example}

In the following lemma, we relate deviations from the causal strategy profile to interventions in the structural causal model that generated the causal CGS. This can be used to relate agent strategies in the causal CGS to causality in the causal model.
\begin{lemma}\label{lem:causal strat leads to causal setting}
    Let $GS$ be a causal CGS based on a causal setting $\csetting$. If $q_{n,m}$ is the leaf-state of $GS$ that results from the strategy profile $F_{\myvec{X} = \myvec{x}} \circ F_\mathcal{M}$, then $q_{n,m}$ corresponds to $(\mathcal{M}^{\myvec{X} \leftarrow \myvec{x}},\mathbf{u})$.
\end{lemma}
\begin{proof}
    We are going to prove the correspondence, i.e. $(X = x) \in \pi(q_{n,m}) \Leftrightarrow (\mathcal{M}^{\myvec{X} \leftarrow \myvec{x}}, \mathbf{u}) \models X = x$ by induction on the agent rank of $X$.
    
    \noindent \textbf{Base Step:} If $\rho(X) = 0$, $X \in V_e$ and does not depend on any other endogenous variables, so if $(X = x)\in \pi(q_{n,m})$, Lemma \ref{lem:no change after i} 
    implies that $(\mathcal{M}, \mathbf{u}) \models X = x$. Because $X$ does not depend on any agent variables, it will keep the same value when intervening on agent variables $\myvec{X}$, so $(\mathcal{M}^{\myvec{X}\leftarrow\myvec{x}},\mathbf{u}) \models X = x$ as well. For the other way around, if $\csettingint{\myvec{X}\leftarrow\myvec{x}}\models X = x$, by the same reasoning we have that $\csetting \models X = x$ and hence $(X = x) \in \pi(q_{0,0})$, again by Lemma \ref{lem:no change after i} 
    $(X = x) \in \pi(q_{n,m})$ as well.

    \noindent\textbf{Induction Hypothesis:} Suppose that for all $X \in \mathcal{V}$ s.t. $\rho(X) \leq i$, $(X = x) \in \pi(q_{n,m})$ if and only if $\csettingint{\myvec{X} \leftarrow \myvec{x}} \models X = x$.

    \noindent\textbf{Inductive Step:} Let $X$ be such that $\rho(X) = i +1$. First suppose that $X \in V_a$.\newline
     - If $X \in \myvec{X}$, $X$ gets value $x \in \myvec{x}$ in $q_{n,m}$, if and only if $\csettingint{\myvec{X} \leftarrow \myvec{x}}$, because it gets the value directly from the intervention $\myvec{X} \leftarrow \myvec{x}$. So it is true in this case.\newline
    - If $X \notin \myvec{X}$, let $(X = x) \in \pi(q_{n,m})$, the value $x$ was determined by $F_\mathcal{M}$, in particular, $\csetting \models [\myvec{Y} \leftarrow \myvec{y}] X = x$, where $\myvec{Y} = \set{Y \ | \ Y \in V_a, \rho(Y) < \rho(X)}$ and $\myvec{y} = \set{y \ | \ y \in \alpha[q_{i,j}]}$, where $q_{i,j}$ is the state on the path to $q_{n,m}$ where $X$ got to take an action. 
    By the inductive hypothesis we know that $\csettingint{\myvec{X} \leftarrow\myvec{x}} \models \myvec{Y} = \myvec{y}$, and hence $\csettingint{\myvec{X} \leftarrow\myvec{x}} \models X = x$ as well, because all variables $X$ depends on have the same values in $\pi(q_{n,m})$ as in $\csettingint{\myvec{X} \leftarrow \myvec{x}}$.
    On the other hand, if $\csettingint{\myvec{X} \leftarrow \myvec{x}} \models X = x$, we know that $x$ is determined only by variables with a lower agent rank, by the inductive hypothesis all those are in $\pi(q_{n,m})$. The value of $X$ in $\pi(q_{n,m})$ is determined by $F_\mathcal{M}$, so $\csetting \models [\myvec{Y} \leftarrow \myvec{y}] X = x'$. 
    \textcolor{black}{All variable-value pairs $Y,y$ are the variable-value pairs from $\csettingint{\myvec{X} \leftarrow \myvec{x}}$, so $x'$ must be $x$, as all variables of lower agent rank have the same value.}\newline
    Now suppose $X \in V_e$, and let $(X = x) \in \pi(q_{n,m})$. $X$ depends only on variables of a lower level, specifically all agent variables of agent rank less or equal than $i + 1$. By the above and the inductive hypothesis, we know that all those variables have the same value in $\csettingint{\myvec{X} \leftarrow \myvec{x}}$, as in $\pi(q_{n,m})$, hence $X = x$ must also be induced by $\csettingint{\myvec{X} \leftarrow \myvec{x}}$, since there are no other interventions done after the agent variables of rank $i+1$ got their final value.
    Now suppose $\csettingint{\myvec{X} \leftarrow \myvec{x}} \models X = x$, $X$ depends only on variables of lower levels, all of those agent variables have the same value in $\csettingint{\myvec{X} \leftarrow \myvec{x}} \models X = x$ as in $\pi(q_{n,m})$ by the inductive hypothesis. Hence $(X = x) \in \pi(q_{n,m})$ as well, since \textcolor{black}{the environment variables follow the causal model in both $\csettingint{\myvec{X} \leftarrow \myvec{x}}$ as in $\pi(q_{n,m})$.}
\end{proof}


The following corollary follows directly from this lemma, it shows that there is a leaf-state in a causal CGS that corresponds to the original causal setting.

\begin{corollary}\label{col:csetting reachable}
    Let $GS$ be a causal CGS based on a causal setting $\csetting$. If $q_{n,m}$ is the leaf-state resulting from all agents following the causal strategy profile $F_{\mathcal{M}}$, then $q_{n,m}$ corresponds to $(\mathcal{M},\mathbf{u})$.
\end{corollary}
\begin{proof}
    This is a special case of Lemma \ref{lem:causal strat leads to causal setting}, where $\myvec{X} = \emptyset$.
\end{proof}

We can check whether this result holds in our semi-automated vehicle example.
We see in Figure \ref{fig:causal cgs vehicle strategy} that if all agents follow the causal strategy profile, they end up in state $q_{2,6}$ with $\pi(q_{2,6}) = \set{O, \neg Att, HD, ODS, \neg DA, \neg Col}$.
The causal CGS was based on the causal setting where there is an obstacle on the road and the driver is not paying attention, in this case we have $\csetting \models O, \neg Att, HD, ODS, \neg DA, \neg Col$ which does correspond to state $q_{2,6}$, as Corollary \ref{col:csetting reachable} predicted.

With Lemma \ref{lem:causal strat leads to causal setting} we can show that if a set of agents $\myvec{X}$ causes $\vphi$ according to the modified HP definition, with a given witness, then in the causal CGS generated by the causal setting that holds this witness fixed, these agents have a strategy to guarantee $\neg \varphi$ in a leaf-state, provided that all other agents follow the causal strategy profile and vice versa.

\begin{proposition}\label{prop:cause iff strat}
    Let $\Gamma = \set{k \ | \ X_k \in \myvec{X}}$ be a set of agents, $\myvec{x}$ a setting for the variables in $\myvec{X}$, and let $\csetting$ be a causal setting with $\csetting\models \vphi$. 
    $\myvec{X} = \myvec{x}$ is, according to the modified HP definition, a cause of causal formula $\varphi$ in this causal setting $\csetting$, with witness $\myvec{W} = \myvec{w}^*$ if and only if in the causal CGS generated by the causal setting, $\csettingint{\myvec{W} \leftarrow \myvec{w}^*}$, $\Gamma$ has a strategy $F_\Gamma$ such that, $\neg \varphi$ will hold in the leaf-state $q_{n,m}$ resulting from the strategy profile $F_\Gamma \circ F_\mathcal{M}$.
\end{proposition}
\begin{proof}
    We first prove the cause to strategy direction.
    In this case, $\myvec{X} = \myvec{x}$ is a cause of $\varphi$, with witness $\myvec{W} = \myvec{w}^*$ so there exists an alternative value for $\myvec{X}$, $\myvec{x}'$ such that $\csetting \models [\myvec{X} \leftarrow \myvec{x}', \myvec{W} \leftarrow \myvec{w}^*] \neg \varphi$.
    Let $F_\Gamma = \{F_{X_k = x} \ | \ x \in \myvec{x}' \text{ if } X_k \in \myvec{X}\}$. 
    By Lemma \ref{lem:causal strat leads to causal setting}, the leaf-state $q_{n,m}$ corresponds to $\csettingint{\myvec{X} \leftarrow \myvec{x}',\myvec{W}\leftarrow \myvec{w}^*}$, and we have that $\csettingint{\myvec{X} \leftarrow \myvec{x}', \myvec{W}\leftarrow\myvec{w}^*}\models\neg \vphi$ and hence $\neg \vphi$ holds in $q_{n,m}$.
    
    Now for the other direction, let $F_\Gamma$ be the strategy such that $\neg \vphi$ will hold in the leaf-state $q_{n,m}$ that results from the strategy profile $F_\Gamma \circ F_{\mathcal{M}}$ in the causal CGS generated by the causal setting $\csettingint{\mathbf{W}\leftarrow\mathbf{w}^*}$.
    Let $\myvec{x}$ be such that $\csetting \models \myvec{X} = \myvec{x}$ and let $\myvec{x}'$ be such that $\myvec{X} = \myvec{x}'\subseteq \pi(q_{n,m})$. By Lemma \ref{lem:causal strat leads to causal setting}, $q_{n,m}$ must correspond to $\csettingint{\myvec{X}\leftarrow\myvec{x}', \myvec{W} \leftarrow \myvec{w}^*}$. Hence $\csetting \models [\myvec{X} \leftarrow \myvec{x}', \myvec{W} \leftarrow \myvec{w}^*]\neg \vphi$ and by definition we have that $\csetting \models \myvec{X} = \myvec{x} \wedge \vphi$. Moreover, $\myvec{x} \neq \myvec{x}'$, because if they were the same it would not be the case that setting $\myvec{X}$ to $\myvec{x}'$ would give a different result than the original causal setting (the determinism axiom of the causal reasoning axioms \cite{halpern2016actual}). Hence $\myvec{X} = \myvec{x}$ is a cause of $\vphi$ according to the modified HP definition, with witness $\myvec{W} = \myvec{w}^*$.
\end{proof}

This result cannot be used to find causes in a causal CGS, because one would already need to know the witness. 
However, we have another result for the causal setting where the witness was not held fixed, provided the witness consists of only agent variables.
The following proposition states that in that case, the set of agents consisting of both the cause and the witness variables has a strategy to guarantee $\neg \varphi$ in a leaf-state, provided all other agents follow the causal strategy profile and vice versa.

\begin{proposition}\label{prop:cause iff superset strat}
    Let $\Gamma = \set{k \ | \ X_k \in \myvec{X}\cup \myvec{W}, \text{ and } \myvec{X} \cup \myvec{W} \subseteq V_a}$ be a set of agents, $\myvec{x},\myvec{w^*}$ are settings for the variables in $\myvec{X},\myvec{W}$ respectively, and let $\csetting$ be a causal setting with $\csetting\models \vphi$. $\myvec{X} = \myvec{x}$ is, according to the modified HP definition, a cause of causal formula $\varphi$ in this causal setting $\csetting$, with witness $\myvec{W} = \myvec{w}^*$ if and only if in the causal CGS generated by this causal setting, $\Gamma$ has a strategy $F_\Gamma$ such that, $\neg \varphi$ will hold in the leaf-state $q_{n,m}$ resulting from the strategy profile $F_\Gamma \circ F_\mathcal{M}$.
\end{proposition}
\begin{proof}
    We first prove the cause to strategy direction.
    In this case, $\myvec{X} = \myvec{x}$ is a cause of $\varphi$, with witness $\myvec{W} = \myvec{w}^*$ so there exists an alternative value for $\myvec{X}$, $\myvec{x}'$ such that $\csetting \models [\myvec{X} \leftarrow \myvec{x}', \myvec{W} \leftarrow \myvec{w}^*] \neg \varphi$.
    Let $F_\Gamma = \{F_{X_k = x} \ | \ k \in \Gamma, x \in \myvec{x}' \text{ if } X_k \in \myvec{X}, \text{ else } x \in \myvec{w}^*\}$. 
    By Lemma \ref{lem:causal strat leads to causal setting}, the leaf-state $q_{n,m}$ corresponds to $\csettingint{\myvec{X} \leftarrow \myvec{x}',\myvec{W}\leftarrow \myvec{w}^*}$, and we have that $\csettingint{\myvec{X} \leftarrow \myvec{x}', \myvec{W}\leftarrow\myvec{w}^*}\models\neg \vphi$ and hence $\neg \vphi$ holds in $q_{n,m}$.
    
    Now for the other direction. 
    Let $\myvec{x}, \myvec{w}^*$ be such that $\csetting \models \myvec{X} = \myvec{x} \wedge \myvec{W} = \myvec{w}^*$ and let $\myvec{x}'$ be such that $\myvec{X} = \myvec{x}'\subseteq \pi(q_{n,m})$. 
    By Lemma \ref{lem:causal strat leads to causal setting}, $q_{n,m}$ must correspond to $\csettingint{\myvec{X}\leftarrow\myvec{x}', \myvec{W} \leftarrow \myvec{w}^*}$. 
    Hence $\csetting \models [\myvec{X} \leftarrow \myvec{x}', \myvec{W} \leftarrow \myvec{w}^*]\neg \vphi$ and by definition we have that $\csetting \models \myvec{X} = \myvec{x} \wedge \vphi$. 
    Moreover, $\myvec{x} \neq \myvec{x}'$, because if they were the same it would not be the case that setting $\myvec{X}$ to $\myvec{x}'$ would give a different result than the original causal setting (the determinism axiom of the causal reasoning axioms \cite{halpern2016actual}). 
    Hence $\myvec{X} = \myvec{x}$ is a cause of $\vphi$ according to the modified HP definition, with witness $\myvec{W} = \myvec{w}^*$.
\end{proof}

As but-for causes have no witness, they give a stronger result.

\begin{corollary}\label{col:but-for cause iff strat}
    Let $\Gamma = \set{k \ | \ X_k \in \myvec{X}}$ be a set of agents, $\myvec{x}$ a setting for the variables in $\myvec{X}$, and let $\csetting$ be a causal setting with $\csetting\models \vphi$. 
    $\myvec{X} = \myvec{x}$ is a but-for cause of causal formula $\varphi$ in this causal setting $\csetting$ if and only if in the causal CGS generated by the causal setting, $\csetting$, $\Gamma$ has a strategy $F_\Gamma$ such that, $\neg \varphi$ will hold in the leaf-state $q_{n,m}$ resulting from the strategy profile $F_\Gamma \circ F_\mathcal{M}$
\end{corollary}
\begin{proof}
    A but-for cause is a special case of the modified HP definition where $\myvec{W} = \emptyset$. This statement is hence a special case of propositions \ref{prop:cause iff strat} and \ref{prop:cause iff superset strat}.
\end{proof}

\begin{example}
    In our running semi-automated vehicle example, both $ODS$ and $\neg DA$ are but-for causes of $\neg Col$, there being no collision (in the causal setting that there is an obstacle and the human driver is not paying attention). 
    In the case of $ODS$ we can define $F_{ODS}$ to be the strategy where the obstacle detection system will not pass on a signal to the driving assistant. 
    If all other agents follow the causal strategy profile, they will reach state $q_{2,5}$.
    Indeed $Col \in \pi(q_{2,5})$.
    Similarly, in the case of $\neg DA$, we can define $F_{DA}$ to be the strategy where the driving assistant does not brake. 
    When the other agents follow $F_{\mathcal{M}}$, they will end up in $q_{2,7}$.
    In that state it is indeed true that $Col \in \pi(q_{2,7})$.
\end{example}

In this section we have shown how agent strategies in a causal CGS relate to the causal relations in the causal setting the causal CGS was based on.
In order to do this, we have introduced the notion of a causal strategy profile, a strategy for the grand coalition that makes sure the agents do exactly those actions they would do if all relations in the causal model would be followed.