This paper investigates the relation between two formalisms that can be used to model multi-agent systems: structural causal models as introduced by Pearl \cite{pearl1995causal} and concurrent game structures. This is done by proposing a systematic way to translate structural causal models to the co-called causal CGS.  
In such a causal CGS, agents will get to take their actions at a point corresponding to their position in the structural causal model. 
The causal CGS is defined in such a way that the leaf-states correspond to interventions on the original structural causal model.

In this paper we have used the variable levels as defined by Halpern to determine the position of the variables in the causal model \cite{halpern2016actual}. 
However, we can use any function that maps the endogenous variables to the positive integers as long as the function assigns a lower rank to a variable than to its descendants. In general, there are multiple of these functions possible for a given structural causal model. 
The formal results of this paper will hold for all such functions, though the structure of the resulting causal CGS may change due to the specific function used.


We can also relax the assumption that each agent controls exactly one agent variable. 
We assumed this to simplify the presentation of the causal CGS, but it is not a strict requirement. 
In principle an agent could control several variables and perform multiple actions, at several time steps, in the causal CGS. 



A limitation of our approach is that in general, we are only able to give a result for actual causes if we already know the witness. For but-for causes, we are able to use the agents' abilities in the causal CGS to determine the but-for causes, 
but in general, this is not possible.
Another limitation is that the causal CGS is generated with respect to a specific causal setting, hence the results only apply to a single context. This means that if the context is uncertain, multiple causal CGS have to be made to evaluate all possible outcomes. However, it is possible that this problem can be solved by using a version of an epistemic CGS. This can be researched in the future.

So far, we have only looked at deterministic and recursive causal models to define the causal CGS.
However, causal relations are often probabilistic and cyclic in many practical use cases. Modelling such cases requires probabilistic and non-recursive causal models to, for example, capture the mutual dependencies between agents.
In order to deal with probabilities, we will have to either employ probabilistic CGS, or use another type of model (e.g. Markov games).
Moreover, allowing cyclic dependencies would make the evaluation of the states difficult, as the variable values would depend on each other.
We think that this could possibly be dealt with by adding a temporal component to the model, but this needs more research.

Another direction of future work would be to use this framework to compare different approaches to defining responsibility in multi-agent settings. Some existing works define responsibility based on causal relations between agents and an outcome (like \cite{Alechina_Halpern_Logan_2020,chockler2004responsibility,Friedenberg_Halpern_2019} and \cite{Beckers2023moral}), while other work is based on whether agents had a strategy to avoid the outcome (like \cite{baier2021game} and \cite{yazdanpanah2019strategic}). The definition of causal CGS might help to combine both directions of research.
Moreover, we can also look at how our approach compares to rule-based approaches to causality. Since Lorini's \cite{lorni2023rule} work shows a correspondence between his rule-based framework for causal reasoning and the structural equations framework, it seems possible that his framework can also be shown to have a connection to our causal CGS.

This research could be used in multi-agent systems with a clear causal structure. Examples of this are traffic control environments, like planes that cannot land when another is departing, trains that cannot travel over the same track at the same time, or traffic lights on a junction that cannot all turn to green at the same time. 
Other applications could be in the analysis of multi-player games, after all, players could cause other players to make a certain move, or even energy management systems, where supply and demand of electricity influence each other. 
In these situations this research could be used to help making decisions, or after something has gone wrong to help attributing responsibility for this.
\balance