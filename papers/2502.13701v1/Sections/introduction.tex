
Causality plays an important role in Artificial Intelligence~\cite{pearl1995causal,halpern2016actual}. A specific type of causality, called `actual causality',  concerns causal relations between concrete events (e.g. throwing a specific rock shatters a specific bottle)~\cite{halpern2016actual}. There is still discussion on what the best definition of actual causality is (see \cite{halpern2016actual,Hall_2007,gladyshev2023dynamic} and \cite{Beckers_Vennekens_2018} for some of those definitions). 
However, most approaches like \cite{gladyshev2023dynamic} and \cite{Beckers_Vennekens_2018} use Pearl's \cite{pearl1995causal} structural model framework.
In this structural model framework, the world is modelled through  variables, which are divided in exogenous and endogenous variables. 
The former are variables whose values are determined by causes outside of the model and the latter are variables whose values are determined by the variables inside the model (both exogenous and endogenous variables).
The functional dependencies between variables are formalised through structural equations.
There also exists a rule-based approach that uses logical language to capture causal relations (see \cite{bochman2018actual} and \cite{lorni2023rule}), but we focus on the structural model framework due to its prominence in the literature \cite{Alechina_Halpern_Logan_2020,Beckers_Vennekens_2018,chockler2004responsibility,gladyshev2023dynamics}.


While causal models can in principle depict multi-agent systems by making a distinction between agent and environment events, they are less appropriate for reasoning about the abilities and strategies of agents.
Concurrent game structures (CGS) have been proposed to reason about agent interactions and strategies \cite{alur2002alternating}. These structures are graphs where nodes correspond to states of the world and edges, labelled with agents' actions, correspond to state transitions \cite{baier2008principles,gorrieri2017process}. 
In deterministic settings, an agent strategy specifies the actions to take by the agent.

Let us introduce an example of a causal model. 
Consider a semi-autonomous vehicle controlled jointly by a human driver and an automatic driving assistance system.
This driving assistance system is in turn supported by an obstacle detection system that signals to the driving assistant whether there is an obstacle in front of the vehicle.
Both the human driver and the driving assistant control the forward movement of the vehicle, though the human driver can always take full control.
In a scenario where there is an obstacle in front of the car, the obstacle causes the obstacle detection system to send a signal to the driving assistant. 
If the human driver is in a distracted state, this signal causes the driving assistant to avoid an accident.
This scenario can be described as a causal system, but can also be viewed as a multi-agent system where the obstacle detection system, the driving assistant and the human driver are all seen as agents that make decisions based on their state observations.

The fundamental relationship between structural causal models and multi-agent system models manifests itself in modelling phenomena such as responsibility for realising a certain outcome by a group of agents.
In the literature of multi-agent systems, both structural causal models and CGS are used to define the responsibility of a group of agents for an outcome \cite{chockler2004responsibility,yazdanpanah2019strategic}. Agents in a structural causal model are seen as responsible for an outcome if they have caused it \cite{chockler2004responsibility}. On the other hand, in a CGS a coalition of agents is deemed responsible for an outcome if they had a strategy to prevent it \cite{yazdanpanah2019strategic}. 
By establishing the relationship between structural causal models and CGS, different modeling approaches to multi-agent phenomena (e.g., responsibility) can be compared and unified. 

In this paper, we aim to establish a formal relationship between structural causal models and concurrent game structures by constructing a CGS for a given structural causal model such that if a group of agents is an actual cause for an outcome in the causal model, then this group had a strategy in the constructed CGS to prevent the outcome, provided the other agents act as prescribed by the causal model.
The CGS is built by distinguishing between agent and environment variables.
We consider the values of an agent variable as possible actions of the agent and interventions as agents' decisions. 
We provide several formal results on how strategies in this causal concurrent game structure (causal CGS) relate to the original structural causal model, establishing a formal relationship between structural causal models and CGS.
In particular, we show that a choice of actions by a group of agents is a cause of an outcome in a structural causal model (under the Halpern-Pearl definition of an actual cause) if and only if this set of agents has a strategy for the negation of the outcome in the corresponding causal CGS, provided the other agents act according to the causal model.
We believe that our framework will be beneficial for supporting causal inference in multi-agent systems, for example, for reasoning and attributing responsibility for certain outcomes to groups of agents. 

We will now first give some preliminaries on causality and concurrent game structures.
In Section \ref{sec:cgs from cm} we define the translation from a structural causal model to a causal CGS, after which, in Section \ref{sec:causality in CGS}, we show how causality in the structural causal model relates to agent strategies in the causal CGS.


