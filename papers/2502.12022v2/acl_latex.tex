% This must be in the first 5 lines to tell arXiv to use pdfLaTeX, which is strongly recommended.
\pdfoutput=1
% In particular, the hyperref package requires pdfLaTeX in order to break URLs across lines.

\documentclass[11pt]{article}

% Change "review" to "final" to generate the final (sometimes called camera-ready) version.
% Change to "preprint" to generate a non-anonymous version with page numbers.
\usepackage[preprint]{acl}
\usepackage{booktabs}
\usepackage{multirow}
\usepackage{xcolor}
\usepackage{changes}
\usepackage{longtable}
\usepackage{tabularx}
% Standard package includes
\usepackage{times}
\usepackage{latexsym}
\usepackage{amsmath}
\usepackage{subcaption}
\usepackage{algorithm}
\usepackage{algpseudocode}
\usepackage{amsfonts}
\usepackage{listings}
% Standard package includes

% For proper rendering and hyphenation of words containing Latin characters (including in bib files)
\usepackage[T1]{fontenc}
% For Vietnamese characters
% \usepackage[T5]{fontenc}
% See https://www.latex-project.org/help/documentation/encguide.pdf for other character sets

% This assumes your files are encoded as UTF8
\usepackage[utf8]{inputenc}

% This is not strictly necessary, and may be commented out,
% but it will improve the layout of the manuscript,
% and will typically save some space.
\usepackage{microtype}

% This is also not strictly necessary, and may be commented out.
% However, it will improve the aesthetics of text in
% the typewriter font.
\usepackage{inconsolata}

%Including images in your LaTeX document requires adding
%additional package(s)
\usepackage{graphicx}

% If the title and author information does not fit in the area allocated, uncomment the following
%
%\setlength\titlebox{<dim>}
%
% and set <dim> to something 5cm or larger.
\newcommand{\xb}[1]{\textcolor{red}{[XBX: #1]}}
\newcommand{\bv}[1]{\textit{\textbf{#1}}}
\newcommand{\TT}{\textsf{T}}
\newcommand{\din}{d_\text{in}}
\newcommand{\dout}{d_\text{out}}
\newcommand\numberthis{\addtocounter{equation}{1}\tag{\theequation}}

\newcommand{\method}{{TATA}}
\newcommand{\danchor}{{$\mathcal{D}_{\text{anchor}}$}}
\newcommand{\dsft}{{$\mathcal{D}_{\text{SFT}}$}}
%\newcommand{\dsfte}{$\mathcal{D}_{\text{SFT}} = {\{(x_i, l_i^k) \}_{i=1}^{N}}_{k=1}^{K_i}$}
\newcommand{\dorig}{{$\mathcal{D}_{\text{orig}}$}}
\newcommand{\dorige}{{$\mathcal{D}_{\text{orig}} = \{(x_i, y_i) \}_{i=1}^N$}}
\newcommand{\danchore}{{$\mathcal{D}_{\text{anchor}} = \{(q_i, a_i) \}_{i=1}^A$}}
\newcommand{\dauge}{{$\mathcal{D}_{\text{aug}} = {\{(x_i, y_i^j)\}_{i=1}^N}_{j=1}^{M_i}$}}
\newcommand{\dde}{{$\mathcal{D} = {\{(x_i, y_i^j, z_i^j)\}_{i=1}^N}_{j=1}^{M_i}$}}
\newcommand{\dd}{{$\mathcal{D}$}}
\newcommand{\daug}{{$\mathcal{D}_{\text{aug}}$}}
\newcommand{\dcandidate}{{$\mathcal{D}^*$}}
\newcommand{\dcandidatee}{{$\mathcal{D}^* = {\{(x_i, y_i^*, z_i^*)\}_{i=1}^N}$}}
\newcommand{\scot}{$S_{\text{CoT}}$}
\newcommand{\scote}{${S_{\text{CoT}}^k}$}
\newcommand{\stir}{$S_{\text{TIR}}$}
\newcommand{\stire}{$S_{\text{TIR}}^k$}
%\newcommand{\dataset}[2]{(#1 from #2)}
\newcommand{\change}[1]{\textcolor{blue}{#1}}
\NewDocumentCommand{\mytodo}
{ mO{} }{\textcolor{blue}{\textsuperscript{\textit{todo}}\textsf{\textbf{\small[#1]}}}}
% Define colors for differences
\definecolor{positive}{rgb}{0,0.5,0} % Green for positive differences
\definecolor{negative}{rgb}{1,0,0}   % Red for negative differences
\title{Teaching LLMs According to Their Aptitude: Adaptive Reasoning for Mathematical Problem Solving}

% Author information can be set in various styles:
% For several authors from the same institution:
% \author{Author 1 \and ... \and Author n \\
%         Address line \\ ... \\ Address line}
% if the names do not fit well on one line use
%         Author 1 \\ {\bf Author 2} \\ ... \\ {\bf Author n} \\
% For authors from different institutions:
% \author{Author 1 \\ Address line \\  ... \\ Address line
%         \And  ... \And
%         Author n \\ Address line \\ ... \\ Address line}
% To start a separate ``row'' of authors use \AND, as in
% \author{Author 1 \\ Address line \\  ... \\ Address line
%         \AND
%         Author 2 \\ Address line \\ ... \\ Address line \And
%         Author 3 \\ Address line \\ ... \\ Address line}

%\author{Xin Xu, Yan Xu\textsuperscript{\dag}, Tianhao Chen, Yuchen Yan, Chengwu Liu, Zaoyu Chen, \\ \textbf{Yufei Wang, Yichun Yin, Yasheng Wang, Lifeng Shang, Qun Liu}\\
%Hong Kong University of Science and Technology \\
% \texttt{xxuca@connect.ust.hk} \\
% \texttt{\{sdiaoaa, macyang, yangwang\}@ust.hk}
%}


%\author{First Author \\
%  Affiliation / Address line 1 \\
%  Affiliation / Address line 2 \\
%  Affiliation / Address line 3 \\
%  \texttt{email@domain} \\\And
%  Second Author \\
%  Affiliation / Address line 1 \\
%  Affiliation / Address line 2 \\
%  Affiliation / Address line 3 \\
%  \texttt{email@domain} \\}

\author{
  \textbf{Xin Xu\textsuperscript{1,2}$^*$},
  \textbf{Yan Xu\textsuperscript{1,2}},
  \textbf{Tianhao Chen\textsuperscript{1}},
\\
  \textbf{Yuchen Yan\textsuperscript{3}},
  \textbf{Chengwu Liu\textsuperscript{2,4}},
  \textbf{Zaoyu Chen\textsuperscript{2,5}},
  \textbf{Yufei Wang\textsuperscript{2}},
  \\
  \textbf{Yichun Yin\textsuperscript{2}},
  \textbf{Yasheng Wang\textsuperscript{2}},
  \textbf{Lifeng Shang\textsuperscript{2}},
  \textbf{Qun Liu\textsuperscript{2}},
 % \textbf{Eleventh E. Author\textsuperscript{1,2,3,4,5}},
 % \textbf{Twelfth Author\textsuperscript{1}},
\\
  %\textbf{Thirteenth Author\textsuperscript{3}},
  %\textbf{Fourteenth F. Author\textsuperscript{2,4}},
  %\textbf{Fifteenth Author\textsuperscript{1}},
  %\textbf{Sixteenth Author\textsuperscript{1}},
%\\
  %\textbf{Seventeenth S. Author\textsuperscript{4,5}},
  %\textbf{Eighteenth Author\textsuperscript{3,4}},
  %\textbf{Nineteenth N. Author\textsuperscript{2,5}},
  %\textbf{Twentieth Author\textsuperscript{1}}
\\
\\
  \textsuperscript{1}The Hong Kong University of Science and Technology,
  \textsuperscript{2}Huawei Noah’s Ark Lab,
\\
  \textsuperscript{3}Zhejiang University,
  \textsuperscript{4}Peking University, 
  \textsuperscript{5}The Hong Kong Polytechnic University
\\
  \small{
    \textbf{Correspondence:} \{xxuca, yxucb\}@connect.ust.hk
  }
}

\begin{document}
\maketitle
\renewcommand*{\thefootnote}{\fnsymbol{footnote}}
%\footnotetext{* Equal contribution.}
%\footnotetext{$\dagger$ Corresponding author.}
\renewcommand*{\thefootnote}{\arabic{footnote}}
\def\thefootnote{*}\footnotetext{Work done during the internship at Huawei Noah’s Ark Lab. Codes and data are available at \href{https://github.com/XinXU-USTC/TATA}{https://github.com/XinXU-USTC/TATA}.}
\begin{abstract}
    Existing approaches to mathematical reasoning with large language models (LLMs) rely on Chain-of-Thought (CoT) for generalizability or Tool-Integrated Reasoning (TIR) for precise computation. 
    While efforts have been made to combine these methods, they primarily rely on post-selection or predefined strategies, leaving an open question: 
    whether LLMs can autonomously adapt their reasoning strategies based on their inherent capabilities.
    % whether LLMs can autonomously select the most suitable reasoning paradigm based on their inherent capabilities.
    In this work, we propose \textbf{{\method}} (\textbf{T}eaching LLMs \textbf{A}ccording to \textbf{T}heir \textbf{A}ptitude), an adaptive framework that enables LLMs to personalize their reasoning strategy spontaneously, aligning it with their intrinsic aptitude. 
    {\method} incorporates base-LLM-aware data selection during supervised fine-tuning (SFT) to tailor training data to the model’s unique abilities. 
    This approach equips LLMs to autonomously determine and apply the appropriate reasoning strategy at test time.
    We evaluate {\method} through extensive experiments on six mathematical reasoning benchmarks, using both general-purpose and math-specialized LLMs. 
    Empirical results demonstrate that {\method} effectively combines the complementary strengths of CoT and TIR, achieving superior or comparable performance with improved inference efficiency compared to TIR alone. 
    Further analysis underscores the critical role of aptitude-aware data selection in enabling LLMs to make effective and adaptive reasoning decisions and align reasoning strategies with model capabilities.

\end{abstract}


\section{Introduction}
\label{sec:intro}

\begin{figure*}[tb]
    \centering
    \includegraphics[width=0.848\linewidth]{figs/circuitnn.pdf} 
    \caption{Illustration of differentiable CircuitNN. CircuitNN is designed based on differentiable NAND gates. After DAS is guided by PI and PO pairs of the truth table, CircuitNN can get the precise circuit architecture logic equivalent to the truth table.}
    \label{fig:circuitnn}
\end{figure*}

% 1. Describe the importance of logic synthesis
% 2. Existing Problems
% (a) Neural Architecture Search: Unstable, Predefined Setting, etc.
% (b) Circuit Generation: Probabilistic Model, Logic Equivalence

With the rapid advancement of technology, the scale of integrated circuits (ICs) has expanded exponentially. 
This expansion has introduced significant challenges in chip manufacturing, particularly concerning power and area metrics.
A primary objective in IC design is achieving the same circuit function with fewer transistors, thereby reducing power usage and area occupancy.

Logic synthesis~\cite{hachtel2005logicsynth}, a critical step in electronic design automation (EDA), transforms behavioral-level circuit designs into optimized gate-level circuits, ultimately yielding the final IC layout. 
The primary goal of logic synthesis is to identify the physical implementation with the fewest gates for a given circuit function. 
This task constitutes a challenging NP-hard combinatorial optimization problem. 
Current logic synthesis tools~\cite{brayton2010abc, wolf2013yosys} rely on human-designed heuristics, often leading to sub-optimal outcomes.

Differentiable architecture search (DAS) techniques~\cite{liu2018darts, chu2020darts} offer novel perspectives on addressing challenges in this problem.
Circuit functions can be represented through truth tables, which map binary inputs to their corresponding outputs. 
Truth tables provide a precise representation of input-output relationships, ensuring the design of functionally equivalent circuits.
Inspired by this, researchers~\cite{deepmind2024ai4sys, wang2024tnet} have begun exploring the application of DAS to synthesize circuits directly from truth tables.
Specifically, \citet{deepmind2024ai4sys} proposed CircuitNN, a framework that learns differentiable connection structures with logic gates, enabling the automatic generation of logic circuits from truth tables.
This approach significantly reduces the complexity of traditional circuit generation. 
Building on this, \citet{wang2024tnet} introduced T-Net, a triangle-shaped variant of CircuitNN, incorporating regularization techniques to enhance the efficiency of DAS.

Despite these advancements, several challenges remain. 
The computational complexity of DAS grows quadratically with the number of gates, posing scalability issues.
Although triangle-shaped architecture~\cite{wang2024tnet} partially mitigates this problem, redundancy persists. 
%Additionally, DAS is susceptible to converging to local optima, limiting the ability to search architectures that satisfy the given truth tables~\cite{liu2018darts}. 
%Furthermore, hyperparameters (network depth and layer width) require extensive searches, introducing complexity and prolonging the synthesis process. 
Additionally, DAS is susceptible to converging to local optima~\cite{liu2018darts} and hyperparameters (network depth and layer width) require extensive searches. 
The challenges arise from the vast search space in DAS. 
% Even with predefined settings for CircuitNN, finding a configuration that meets the truth table requires extensive trial and error during the DAS process. 
Intuitively, limiting the search space through predefined parameters (network depth, gates per layer, and connection probabilities) can significantly reduce the complexity.

Recent advances~\cite{openai2023gpt4, abramson2024alphafold3, esser2024sd3, li2024mar} in conditional generative models have demonstrated remarkable performance across language, vision, and graph generation tasks. 
Motivated by these developments, we propose a novel approach to circuit generation that generates preliminary circuit structures to guide DAS in generating refined circuits matching specified truth tables. 
Firstly, we introduce CircuitVQ, a tokenizer with a discrete codebook for circuit tokenization. 
Built upon our Circuit AutoEncoder framework~\cite{hou2022graphmae,li2023maskgae,wu2025mgvga}, CircuitVQ is trained through a circuit reconstruction task. 
Specifically, the CircuitVQ encoder encodes input circuits into discrete tokens using a learnable codebook, while the decoder reconstructs the circuit adjacency matrix based on these tokens.
Subsequently, the CircuitVQ encoder serves as a circuit tokenizer for CircuitAR pretraining, which employs a masked autoregressive modeling paradigm~\cite{chang2022maskgit, li2023mage}. 
In this process, the discrete codes function as supervision signals. 
After training, CircuitAR can generate discrete tokens progressively, which can be decoded into initial circuit structures by the decoder of the CircuitVQ. 
These prior insights can guide DAS in producing refined circuits that match the target truth tables precisely.

Our key contributions can be summarized as follows:
\begin{itemize}
\item We introduce CircuitVQ, a circuit tokenizer that facilitates graph autoregressive modeling for circuit generation, based on our Circuit AutoEncoder framework;
\item Develop CircuitAR, a model trained using masked autoregressive modeling, which generates initial circuit structures conditioned on given truth tables;
\item Propose a refinement framework that integrates differentiable architecture search to produce functionally equivalent circuits guided by target truth tables;
\item Comprehensive experiments demonstrating the scalability and capability emergence of our CircuitAR and the superior performance of the proposed circuit generation approach.
\end{itemize}

% Motivation
% (a) Diffusion (Vision, Graph), Autoregressive (Language, Vision)
% (b) Circuit Generation for Predefined Setting
% (c) Neural Architecture Search for Strict Logic Equivalence

% Contribution
% (a) Circuit Tokenizer (new transformer arch, training strategy)
% (b) CircuitAR (train and gen strategies, post-ar strategy)
% (c) Extensive Evaluation including BitD (Bit Distance) for Scalability

\section{Preliminaries}
\label{sec:preliminaries}
We first set up notations and mathematically formulate tasks.

\noindent\textbf{Language-Conditioned Imitation Learning (LC-IL)}. The task of LC-IL aims to train an agent to mimic expert behaviors from a given demonstration set $\mathcal{D}_d = \{(\mathbf{\tau}_i,l_i)\}_{i=1}^N$, where $l_i \in \mathcal{L} $ represents a task-specific language instruction. Each trajectory $\mathbf{\tau}_i\in\mathcal{T}$ consists of a sequence of state-action pairs $\mathbf{\tau}_i = \{(\mathbf{s}_j, \mathbf{a}_j)\}_{j=1}^T$ of the horizon length $T$. In robot manipulation tasks, action $\mathbf{a}_j\in\mathcal{A}$ corresponds to the control commands executed by the agent and state $\mathbf{s}_j = [\mathbf{p}_j; \mathbf{v}_j] \in\mathcal{S}$ records proprioceptive data $\mathbf{p}_j$ (\textit{e.g.,} joint positions, velocities) and visual inputs $\mathbf{o}_j\in\mathcal{O}$ (\textit{e.g.,} camera images) at the time step $j$. The objective of LC-IL is to find an optimal language-conditioned policy $\pi^*(\mathbf{a}|\mathbf{s},l): \mathcal{S}\times\mathcal{L}\mapsto\mathcal{A}$ via solving the supervised optimization as follows,
\begin{equation}\nonumber
    \pi^* \in \arg\min_{\pi} \mathbb{E}_{(\tau_i, l_i)\sim \mathcal{T}} \left[ \frac{1}{T} \sum_{(\mathbf{s}_j, \mathbf{a}_j) \sim \tau_i} \ell(\pi(\hat{\mathbf{a}}_j, \mathbf{s}_j|l_i),  \mathbf{a}_j)\right],
\end{equation}
where \(\ell(\cdot, \cdot)\) is a task-specific loss, such as mean squared error or cross-entropy. Training the policy \(\pi_\theta\) in an end-to-end fashion may require \textit{hundreds} of high-quality expert demonstrations to converge, primarily due to the high variance of visual inputs $\mathbf{o}$ and language instructions $l$.

% We study the problem of Language-Conditioned Imitation Learning ~\cite{rss21-gcil}, where the goal is to train an agent to perform tasks by conditioning its policy on both the state of the environment and language instruction. Formally, let \(\mathcal{O}\) be the observation space, \(\mathcal{A}\) the action space, and \(\mathcal{L}\) the language instruction space. The observation space \(\mathcal{O}\) typically includes visual or sensor data, such as images, that represent the partial observation of state \(\mathcal{S}\). The objective is to learn a policy \(\pi_\theta : \mathcal{O} \times \mathcal{L} \to \mathcal{A}\), parameterized by \(\theta\), that maps an observation \(o \in \mathcal{O}\) and a language instruction \(L \in \mathcal{L}\) to an action \(a \in \mathcal{A}\). We assume access to a dataset of expert demonstrations \(\mathcal{D}_{\operatorname{demo}} = \{(\{o_k^i, a_k^i\}_{i=1}^T, L_k)\}_{k=1}^N\), where each sample consists of a $T$-step observation-action trajectory and a corresponding language instruction \(L_k \in \mathcal{L}\). The goal is to train the policy \(\pi_\theta\) by minimizing the following loss function:
% \[
% \mathcal{L}(\theta) = \frac{1}{N} \sum_{k=1}^N \sum_{i=1}^T \ell(a_k^i, \pi_\theta(o_k^i, L_k)),
% \]
% where \(\ell(\cdot, \cdot)\) is a task-specific loss function, such as mean squared error or cross-entropy. 
\begin{table}
\centering
\caption{Comparison of different component designs in time contrast learning across mainstream vision-language pre-training. \vspace{1ex}
% The goal frame $o_g$ is typically set as the last frame $o_{T}$.
 }
\label{tab:comp}
\Large
\resizebox{\linewidth}{!}{ 
\begin{tabular}{llll}
\toprule
$\operatorname{Method}$      & \textcolor{black}{$\mathcal{P}(\mathcal{O}_{i})$}  & \textcolor{black}{$\mathcal{N}(\mathcal{O}_{i})$} & $\mathfrak{R}(\mathbf{v},\mathbf{l}_i)$  \\ \hline
$\operatorname{R3M}$         & $(o_0, o_{j>i})$      &  $(o_0,o_i,o_j^{\notin O_i})$   & $\operatorname{reward}(\mathbf{v},\mathbf{l}_i)$   \\    
$\operatorname{LIV}$         & $(o_T)$    &  $(o_T^{\notin O_i})$    & $\operatorname{cos}(\mathbf{v},\mathbf{l}_i)$  \\    
$\operatorname{DecisionNCE}$ & $(o_i,o_{j>i})$     &     $(o_i^{\notin O_i},o_{j>i}^{\notin O_i})$  & $\operatorname{cos}(\mathbf{v}_j-\mathbf{v}_i, \mathbf{l}_i)$  \\          
$\operatorname{AcTOL}$        & $(o_i,o_{j \in [T] \setminus \{i\}})$ & $(o_i,o_k: d_{i, k}>d_{i, j})$  & $-\Vert \operatorname{cos}(\mathbf{v}_i, \mathbf{l}_i)-\operatorname{cos}(\mathbf{v}_j, \mathbf{l}_i) \Vert_2 $     \\  \bottomrule                                                              
\end{tabular}
}
\end{table}

\paragraph{Vision-language Pre-training.}  Address such scalability issues can be achieved by leveraging large-scale, easily accessible human action video datasets $\mathcal{D}_p = \{(\mathcal{O}_i, l_i)\}_{i=1}^M$ \cite{corr18-epickitchen,cvpr22-ego4d}, where $\mathcal{O}_i=\{o_j\}_{j=1}^T$ represents a video clip with $T$ frames and $l_i$ the corresponding description. Pretraining on such datasets enables policies to rapidly learn visual-language correspondences with minimal expert demonstrations. Mainstream pretraining methods employ time contrastive learning \cite{icra18-tcn} to fine-tune a visual encoder $\mathcal{\phi}$ and a text encoder $\mathcal{\varphi}$, which project frames and descriptions into a shared $d$-dimensional embedding space, \textit{i.e.}, $\mathbf{v}_j = \phi(o_j)\in\mathbb{R}^d$ and $\mathbf{l}_i = \varphi(l_i)\in\mathbb{R}^d$. To provide a unified perspective on various pretraining approaches, we formulate them within the objective $\mathcal{L}_{\operatorname{tNCE}}(\phi, \varphi)$: \vspace{-2ex}
\begin{align}\nonumber\small
\mathcal{L}_{\operatorname{tNCE}}&=
-\mathbb{E}_{\substack{\scriptstyle o^+\sim\textcolor{black}{\mathcal{P}(\mathcal{O}_i)}}}
    \log  
    \frac{
        \exp(\mathfrak{R}(\mathbf{v}^+, \mathbf{l}_i))
    }{
        \mathbb{E}_{\scriptstyle o^- \sim \textcolor{black}{\mathcal{N}(\mathcal{O}_i)}}
        \exp(\mathfrak{R}(\mathbf{v}^-, \mathbf{l}_i))
    },
\end{align}

% \begin{align}\nonumber\small
% \mathcal{L}_{\operatorname{tNCE}}&=
% -\mathbb{E}_{\substack{\scriptstyle o\sim O_i \\ \scriptstyle o^+\sim\textcolor{black}{\mathcal{P}(o)}}}
%     \log  
%     \frac{
%         \exp(\mathfrak{R}(\mathbf{v}^+, \mathbf{v}, \mathbf{l}_i))
%     }{
%         \mathbb{E}_{\scriptstyle o^- \sim \textcolor{black}{\mathcal{N}(o)}}
%         \exp(\mathfrak{R}(\mathbf{v}, \mathbf{v}^-, \mathbf{l}_i))
%     },\vspace{-2ex}
% \end{align}
% where $\mathbf{v} = \phi(o)$, and 
where $\mathbf{v}^{+/-} = \phi(o^{+/-})$. Different pretraining strategies differ in their selection of (1) the positive frame set $\mathcal{P}(\mathcal{O}_i)$, (2) negative frame set $\mathcal{N}(\mathcal{O}_i)$; and (3) the semantic alignment scoring function $\mathfrak{R}(\mathbf{v}, \mathbf{l}_i)$ measuring the gap of VL similarities as detailed in Table \ref{tab:comp}. 

\noindent\textbf{Discussion.} As motivated by goal-conditioned RL \cite{nips17-her}, current approaches \textit{explicitly} select future frames (\textit{e.g.}, DecisionNCE) or the last frame (\textit{e.g.}, LIV) as the goal within the positive set, enforcing their visual embedding to align with the semantics. Likewise, the scoring functions $\mathfrak{R}$ are often designed to maximize this transition direction. However, the pretraining action videos are \textit{noisy} as actions may terminate early or include irrelevant subsequent actions, which may mislead the encoders and result in inaccurate vision-language association. As detecting precise action boundaries is non-trivial, we argue for a more flexible approach that leverages \textit{intrinsic} characteristics of actions to guide pretraining.



% we first pre-train a visual encoder \(\mathcal{\phi}: \mathcal{O} \to \mathbb{R}^d\) and a text encoder \(\mathcal{\varphi}: \mathcal{L} \to \mathbb{R}^d\) to learn mappings from the observation and the language instruction space to $d-$dimensional feature spaces. This pre-training can be done using large, less-expensive data without action annotation, such as human action videos . Then, with the frozen learned features \(\boldsymbol{v}\) and \(\boldsymbol{l}\) as input, we can only fine-tune a simple Multi-Layer Perceptron (MLP) with a few demonstrations to learn the map from the feature space \(\mathbb{R}^d \times \mathbb{R}^d\) to the action space \(\mathcal{A}\). Since both the observation space \(\mathcal{O}\) and the action space \(\mathcal{A}\) are continuous and ordered over time, we expect the representations learned through pre-training to also exhibit continuity and orderliness. This property in the representations allows for better learning of the continuous mapping between observations and actions. This property offers three significant benefits: First, the orderliness of the representation ensures that different states of the task, such as the start and end of an action, can be better captured and distinguished. Second, the continuity of the representation allows it to evolve smoothly as the task progresses, enabling the model to output stable actions based on the current state. Finally, we can demonstrate that even under small perturbations to the language instruction, these properties ensure the robustness of the learned representation. This robustness is crucial for maintaining performance in real-world scenarios where language instructions might contain minor ambiguities or variations.





% We consider a partially observable Markov Decision Process (POMDP) with language conditions, which models the interaction between an agent and an environment where observations are incomplete and actions are guided by natural language instructions. Formally, a POMDP is defined as a tuple $\langle \mathcal{S}, \mathcal{A}, \mathcal{O}, \mathcal{T}, \mathcal{R}, \mathcal{Z}, \gamma \rangle$, where $\mathcal{S}$ is the state space, $\mathcal{A}$ is the action space available to the agent. $\mathcal{O}$ is the observation space, which provides partial information about the environment. $\mathcal{T}(s' \mid s, a)$ is the state transition function. $\mathcal{R}(s, a)$ is the reward function. $\mathcal{Z}(o \mid s, a)$ is the observation function. $\gamma \in [0, 1)$ is the discount factor.

% To incorporate language instructions, we introduce a task description $L$, which specifies the agent's goal in natural language. The task description conditions the agent's policy $\pi(a \mid o, L)$, where $o$ is the agent's current observation. The agent aims to maximize the expected cumulative reward while adhering to the task described by $L$.

% Further, we assume the availability of a large-scale human action video dataset including $N$ video-instruction pairs, $\{(\{o_k^i\}_{i=1}^{t_k}, L_k)\}_{k=1}^N$, where each pair representing an action video with $t_k$ frames and its corresponding language description $L_k$. We pre-train the visual and language encoders on this dataset, with the visual features $\boldsymbol{v} = \operatorname{Enc}_v(o)$ and the language features $\boldsymbol{l} = \operatorname{Enc}_l(L)$. These pre-trained representations are then frozen and applied to train the policy $\pi$ in the aforementioned decision-making process, enabling the agent to better interpret and act upon language-conditioned tasks.
% \begin{figure}
%     \centering
%     \includegraphics[width=0.5\linewidth]{Move_teaser.pdf}
%     \caption{Comparison of different dynamic compute approaches. length of arrow indicates residual transformation per token while width indicates velocity of transformation.}
%     \label{fig:enter-label}
% \end{figure}

\section{Method}
\label{sec:method}
Residual connections play a crucial role in shaping token representations, yet their dynamics remain underexplored in the context of efficient decoding. In this work, we delve deeper into transformer residual dynamics and investigate how modulating residual transformation velocity can improve inference efficiency in token-level processing, optimizing both dense and sparse MoE transformers.


\subsection{Residual Dynamics and Motivation for Multi-rate Residuals} \label{sec:motivation}

To analyze how hidden representations evolve across different layers of a transformer architecture, it's crucial to consider the effect of residual connections. Each transformer decoder layer typically has residual connections across attention and MLP submodules. As the residual stream $h_i$ traverses from interval $E_j$ to $E_{j+1}$, it undergoes a residual transformation given by:  
% \begin{equation}
% \label{eq:slow_residual_transformation}
% H_{E_{j+1}} = H_{E_j} \prod_{i=E_j}^{E_{j+1}} \left( I + \mathcal{A}_i \right) \left( I + \mathcal{M}_i \right) \quad \text{where} \quad \mathcal{A}_i = f(c_i, h_{i}), \mathcal{M}_i = g(h_i)
% \end{equation}

\begin{equation} \label{eq:slow_residual_transformation}
h_{E_{j+1}} = h_{E_j} + \sum_{i=E_j}^{E_{j+1}-1} \left( \mathcal{A}_i(h_i) + \mathcal{M}_i(h_i + \mathcal{A}_i(h_i)) \right) \quad \text{where} \quad \mathcal{A}_i = f(c_i, h_{i}), \mathcal{M}_i = g(h_i). 
\end{equation}

Here, \( \mathcal{A}_i \) denotes the non-linear transformation introduced by the multi-head attention mechanism at layer \( i \), while \( \mathcal{M}_i \) corresponds to the non-linear transformation of the MLP block at the same layer. These transformations depend on the input residual stream \( h_i \) and, in the case of \( \mathcal{A}_i \), the previous contextual representation \( c_i \).\footnote{Normalization layers are typically applied in practice but are omitted here for simplicity of the argument.}


% For easy tokens, the magnitude and direction of this delta transformation become progressively smaller with each successive layer as shown in \cref{fig:delta_transformation}. Consequently, it is feasible to predict these tokens after only a few residual connections, whereas harder tokens necessitate more extensive processing through additional layers.

\begin{figure}[ht]
    \centering
    \begin{subfigure}{0.48\textwidth}
        \centering
        \includegraphics[width=\textwidth]{sections/figures/residual_change.pdf}
        \caption{}
        \label{fig:residual_change}
    \end{subfigure}%
    \hfill
    \begin{subfigure}{0.48\textwidth}
        \centering
        \includegraphics[width=\textwidth]{sections/figures/alignment_wrt_dedicated_model.pdf}
        \caption{}
    \label{fig:alignment_wrt_dedicated_model}
    \end{subfigure}
    \caption{(a) As residual streams propagate through the model, the directional shifts in the residuals become progressively smaller. (b) A dedicated model with $k$ layers achieves a faster rate of change in residual streams and higher alignment than base model leveraging early exit mechanisms at layer $k$.}
    \label{fig}
\end{figure}


To examine whether residual transformations can be accelerated across layers, we conducted experiments using a diverse set of prompts on a pre-trained Phi3 model~\cite{phi3_report}. As illustrated in \cref{fig:residual_change}, we measured the directional shift in residual states as \( 1 - \mathcal{C}(h_{i-1}, h_i) \), where \(\mathcal{C}\) denotes normalized cosine similarity. This shift is notably higher in the initial layers, gradually decreasing in subsequent layers. This behavior allows traditional early exit approaches to effectively accelerate decoding by enabling earlier exits for simpler tokens. However, these approaches typically rely on a distance-based approximation, where the full residual transformation of the model is approximated by the residual transformations of the initial layers. To gain deeper insights into the distance versus velocity aspects of residual transformation, we conducted a comparative study. Specifically, we trained an early exit head at layer $k$ of the Phi3 model, which consists of 32 layers, restricting the distance traveled by each token. To accelerate the residual transformation relative to number of layers, we trained a smaller model consisting of only $k$ layers, while keeping all other hyperparameters consistent. We then compared the next-token prediction accuracy of the early exit head of the base model with that of the smaller model. To ensure an equal number of trainable parameters, we inserted low-rank adapters into the smaller model and trained only these adapters, whereas, in the distance-based approach, we trained solely the early exit head. In addition, to accelerate the residual transformation in smaller model, we distilled the residual streams from the larger model by incorporating a distillation loss ~\cite{sanh2019distilbert} between the residual state at layer \(i\) of the smaller model and the residual state at layer \(4 \times i\) of the larger model. As shown in ~\cref{fig:alignment_wrt_dedicated_model} the smaller model demonstrates a significantly faster rate of change in residual streams, leading to higher next token prediction accuracy after $k$ layers compared to the base model that employs traditional early exit mechanisms after $k$ layers \cite{schuster2022confident, chen2023eellm, varshney-etal-2024-investigating}. This experimental setup, which modifies only the rate of change in residual streams while keeping other factors constant, suggests that dense transformers, trained with a fixed number of layers, may inherently possess a slow residual transformation bias.

This observation raises an intriguing question: if the rate of change in residual streams could be accelerated relative to the number of layers, is it possible to facilitate earlier alignment for a greater proportion of tokens? Earlier alignment would be beneficial to not only facilitate dynamic computation but also for generating speculative tokens efficiently with high acceptance rates in speculative decoding setups ~\cite{leviathan2023fast, chen2023accelerating}. 

%thereby enhancing the efficiency of early exiting? 
 % This bias likely constrains the effectiveness of early exiting, particularly for easier tokens. By addressing this limitation through accelerated residual transformations, we hypothesize that it is possible to substantially improve the efficiency and accuracy of early exit strategies in transformer models.

\subsection{Multi-Rate Residual Transformation} \label{m2r2_method}

To address the slow residual transformation bias described in ~\cref{sec:motivation}, we introduce \textit{accelerated residual streams} that operate at rate $R$ relative to original slow residual stream. We pair slow residual stream, $h$ with an accelerated residual stream, $p$, which has an intrinsic bias towards earlier alignment. Relative to ~\cref{eq:slow_residual_transformation}, accelerated residual transformation from interval $E_j$ to $E_{j+1}$ can be represented as: 

% \begin{equation}
% \label{eq:fast_residual_transformation}
% P_{E_{j+1}} = P_{E_j} \prod_{i=E_j}^{E_{j+1}} \left( I + \hat{\mathcal{A}_i} \right) \left( I + \hat{\mathcal{M}_i} \right) \quad \text{where} \quad \hat{\mathcal{A}_i} = \hat{f}(c_i, P_{i}), \hat{\mathcal{M}_i} = \hat{g}(P_{i})
% \end{equation}


\begin{equation} \label{eq:fast_residual_transformation}
p_{E_{j+1}} = p_{E_j} + \sum_{i=E_j}^{E_{j+1}-1} \left( \hat{\mathcal{A}_i}(p_i) + \hat{\mathcal{M}_i}(p_i + \hat{\mathcal{A}_i}(p_i)) \right) \quad \text{where} \quad \hat{\mathcal{A}_i} = \hat{f}(c_i, p_{i}), \hat{\mathcal{M}_i} = \hat{g}(h_i), 
\end{equation}



where $\hat{\mathcal{A}_i}$ and $\hat{\mathcal{M}_i}$ denote non-linear transformation added by layer $i$ to previous accelerated residual $p_{i}$. Similar to $\mathcal{A}_i$, non-linear transformation $\hat{\mathcal{A}_i}$ attends to same context $c_i$ but uses a different transformation $\hat{f}$ for accelerating $p_{E_j}$ relative to $h_{E_j}$. 

We integrate accelerated residual transformation directly into the base network using parallel accelerator adapters such that rank of accelerator adapters $R_p << d$ where $d$ denotes base model hidden dimension. This setup allows the slow residual stream $h_{E_j}$ to pass through the base model layers while the accelerated residual stream $p_{E_j}$ utilizes these parallel adapters as shown in ~\cref{fig:m2r2_main}. Both slow and accelerated residuals are processed in same forward pass via attention masking and incur negligible additional inference latency in memory bound decoding setups, while in compute bound decoding setups where FLOPs optimization is essential, accelerated residual stream utilizes a fraction of attention heads that of slow residual (see ~\cref{sec:flops_optimization}). Additionally, to maximize the utility of accelerated residual transformations without introducing dedicated KV caches, we propose a shared caching mechanism between the slow and accelerated streams which minimally impact alignment benefits of our approach while offering substantial memory savings (see ~\cref{fig:koala_alignment}). Specifically, the attention operation on the slow residuals \( \text{MHA}(h_t, h_{\leq t}, h_{\leq t}) \) is redefined for accelerated residuals as 
\[
\hat{\mathcal{A}} = MHA(p_t, h_{<t} \oplus p_t, h_{<t} \oplus p_t),
\]
where the accelerated residual at time-step $t$, \( p_t \) attends to the slow residual’s KV cache, facilitating the reuse of contextual information across both residual streams without incurring additional caching costs. Here, \(MHA(q, k, v) \) represents multi-head attention between query \( q \), key \( k \), and value \( v \).

\begin{figure}
    \centering
    \includegraphics[width=0.8\linewidth]{sections//figures/m2r2_main2.pdf}
    \caption{Multi-rate Residuals Framework: Slow residual stream of base model is accompanied by a faster stream that operates at a $2-(J+1)\times$ rate relative to the slow stream, undergoing transformations via accelerator adapters as detailed in \cref{m2r2_method}, where J denotes number of early exit intervals. Colors within the slow and fast residual streams indicate similarity, with matching colors representing the most closely aligned residual states. At the beginning of the forward pass and at each exit point, the accelerated residual state is initialized from the corresponding slow residual state to avoid gradient conflict during training (see ~\cref{sec:grad_conflict}). Early exiting decisions are informed by the Accelerated Residual Latent Attention (ARLA) mechanism, described in \cref{method_arla}, which evaluates residual dynamics across consecutive exit gates.}
    \label{fig:m2r2_main}
\end{figure}

% Furthermore. to maximize the benefits of fast residual transformations without using dedicated KV caches, we propose sharing the fast network’s cache with the slow network. Formally speaking, We modify attention operation on slow residuals $MHA(H_t, H_{<=t}, H_{<=t})$ as $MHA(P_{t}, H_{<t} \oplus P_t, H_{<t}  \oplus P_t)$ such that accelerated residuals attend to previous slow context KV cache, where $MHA(q,k,v)$ denotes multi head attention between query, $q$, key $k$ and value $v$.


\subsection{Enhanced Early Residual Alignment}
Early residual alignment is instrumental in optimizing early exiting, speculative decoding, and Mixture-of-Experts (MoE) inference mechanisms. In this section, we provide a detailed analysis of how accelerated residuals enhance these inference setups.

% By aligning the residual states of intermediate layers with the final output representations, the model can maintain high prediction accuracy even when computations are truncated at earlier layers. This enables more reliable early exiting, reducing the overall computational cost while preserving performance. Additionally, in speculative decoding, early residual alignment allows the model to make confident predictions using faster, partial computations, thereby accelerating inference without sacrificing output quality.


\subsubsection{Early Exiting} \label{method_early_exiting}

A prevalent strategy for enabling early exiting at an intermediate layer $E_{j}$ involves approximating the residual transformation between $E_{j}$ and the final layer $N-1$ using a linear, context independent mapping, $\mathcal{T}$, such that $H_{N-1} \approx \mathcal{T}(H_{E_{j}})$. This approximation has been extensively employed in conventional approaches ~\cite{schuster2022confident, chen2023eellm, varshney-etal-2024-investigating}, providing a computationally efficient means to project the output of deeper layers from intermediate states. Specifically, residual state of layer $N-1$ with this approximation can be expressed as:


% \begin{equation}
% \label{eq: vanila_ea_assumption}
% \Phi(H_{E_{j}}) \sim H_{E_{j}} \prod_{i=E_{j}}^{N}\left( I + \mathcal{A}_i \right) \left( I + \mathcal{M}_i \right) \quad \text{where} \quad \Phi \perp C
% \end{equation}

\begin{equation} \label{eq:early_exiting}
h_{E_j} + \sum_{i=E_j}^{N-1} \left( \mathcal{A}_i(h_i) + \mathcal{M}_i(h_i + \mathcal{A}_i(h_i)) \right) \sim \mathcal{T}(h_{E_{j}})  \quad \text{where} \quad \mathcal{T} \perp c. 
\end{equation}


Here, $\mathcal{A}_i$ and $\mathcal{M}_i$ represent the residual contributions of the multi-head attention and MLP layers, respectively, while $\mathcal{T}$ remains independent of $c$, the preceding context.

This approach is inherently limited by two major factors: first, the assumption of linearity between $h_{E_{j}}$ and $h_{N-1}$ may not hold uniformly for all tokens, particularly when $E_j \ll N$. Second, the linear transformation $\mathcal{T}$ disregards the influence of the context $c$ and fails to account for the latent representations of previous contextual states. In contrast, M2R2 accelerated residual states mitigate both of these challenges by approximating the slow residual transformation of all layers via a faster residual transformation of fewer layers as:
% \begin{equation}
% H_{E_j} \prod_{i=E_j}^{N}\left( I + \mathcal{A}_i \right) \left( I + \mathcal{M}_i \right) \sim P_{E_j} \prod_{i=E_j}^{E_j+1}\left( I + \hat{\mathcal{A}_i} \right) \left( I + \hat{\mathcal{M}_i} \right)
% \end{equation}


\begin{equation} \label{eq:m2r2_approximating_ea}
h_{E_j} + \sum_{i=E_j}^{N-1} \left( \mathcal{A}_i(h_i) + \mathcal{M}_i(h_i + \mathcal{A}_i(h_i)) \right) \sim p_{E_j} + \sum_{i=E_j}^{E_{j+1}-1} \left( \hat{\mathcal{A}_i}(p_i) + \hat{\mathcal{M}_i}(p_i + \hat{\mathcal{A}_i}(p_i)) \right), 
\end{equation}

% \begin{equation} \label{eq:fast_residual_transformation}
% p_{E_{j+1}} = p_{E_j} + \sum_{i=E_j}^{E_{j+1}-1} \left( \hat{\mathcal{A}_i}(p_i) + \hat{\mathcal{M}_i}(p_i + \hat{\mathcal{A}_i}(p_i)) \right) \quad \text{where} \quad \hat{\mathcal{A}_i} = \hat{f}(c_i, p_{i}), \hat{\mathcal{M}_i} = \hat{g}(h_i) 
% \end{equation}






where $p_{E_j}$ is initialized from the slow residual state $h_{E_j}$ at each early exit interval $E_j$ using an identity transformation (see ~\cref{fig:m2r2_main}). As shown in ~\cref{fig:m2r2_residual_sim}, accelerated residuals offer a smoother, more consistent shift in residual direction across layers, in contrast to the abrupt changes typically seen at early exit points in standard early exit methods. Moreover, the normalized cosine similarity between accelerated states at early exit intervals and final residual states is substantially higher compared to traditional early exit techniques, highlighting improved alignment with final layer representations. Traditional adaptive compute methods are constrained by two principal factors: the number of tokens eligible for early exit at intermediate layers and the precision of early exit decision. If residual streams fail to saturate early, the majority of tokens remain ineligible for exit, thereby diminishing potential speedups. Additionally, imprecise delineations between tokens suitable for early exit can lead to underthinking (premature exits that adversely affect accuracy) or overthinking (unnecessary processing that compromises efficiency) ~\cite{zhou2020self, dai2020dynamic}. Enhanced early alignment using ~\cref{eq:m2r2_approximating_ea} helps to address  first issue. To address the second issue we introduce Accelerated Residual Latent Attention, which dynamically assesses the saturation of the residual stream, allowing for a more precise differentiation between tokens that can exit early and those requiring further processing.

% This results in uniform change in residual direction    
% % We keep $\mathcal{A} = \hat{\mathcal{A}}$, while $\hat{\mathcal{M}}$ is accelerated by a factor of $2 - (N_{E}+1)X$ relative to the slower residual transformation $\mathcal{M}$, where $N_E$ represents number of early exiting intervals.
% Figure~\cref{fig:rate_change_comparison} illustrates the comparative rate of change between these transformation streams.



% fig:rate_change_comparison
% - grid plot x axis -> layer id (0, 8) , y axis -> layer id -> dark color cell for max similarity , lighter for lower 
% 
-------------------------------------------------------
Let's consider residual stream $h_i$ traverses through interval $E_j$ to $E_{j+1}$ and undergoes residual transformation given by 
\begin{equation}
h_{E_{j+1}} = h_{E_j} \prod_{i=E_j}^{E_{j+1}} \left( 1 + \delta_i \right)    
\end{equation}

where $\delta_i$ denotes non-linear transformation added by layer $i$. Each non-linear transformation of layer $i$ is a function of previous contextual representation, $c_i$ and input residual stream $h_i-1$ as
$\delta_i = f(c_i, h_{i-1})$ 

One way to exit early at exit $E_j+1$ is to assume that residual transformation from $E_j+1$ to final layer $N-1$ can be approximated by a linear function $\phi$ as $h_{N-1} \sim \Phi(h_{E_j+1})$ and most conventional approaches such as \todo{cite EA papers} use this approach. In other words, 

\begin{equation}
\Phi(h_{E_j+1} \sim h_{E_j+1} \prod_{i=E_j+1}^{N} \left( 1 + \delta_i \right)   
\end{equation}

This approach suffers from two primary issues, linearity assumption from $h_E_j+1$ to $H_N-1$ if often incorrect, particularly when $E_j << N$. More importantly, linear transformation $\Phi$ doesn't consider effect of context $C_i$. M2R2  effectively addresses these issues as accelerated residual stream at interval $E_j+1$ can be represented as 

\begin{equation}
r_{E_{j+1}} = r_{E_j} \prod_{i=E_j}^{E_{j+1}} \left( 1 + \gamma_i \right)    
\end{equation}

where $\gamma_i$ denotes non-linear transformation added by layer $i$ to previous accelerated residual $r_i-1$. Similar to $\delta_i$, non-linear transformation $\gamma_i$ considers context $C_i$ as 
$\gamma_i = g(c_i, r_{i-1})$. So in summary, slow residual transformation is approximated by accelerated residual as: 

\begin{equation}
h_{E_j} \prod_{i=E_j}^{N} \left( 1 + \delta_i \right) \sim h_{E_j} \prod_{i=E_j}^{E_j+1} \left( 1 + \gamma_i \right)
\end{equation}

It's worth noting that accelerated residual $r_i$ and slow residual $h_i$ are processed concurrently at layer $i$ by constructing proper attention mask such as attention of slow residual is represented as 

$MHA(H_it, H_{i<=t}, H_{i<=t}$ while attention of fast residual is computed as 

$MHA(r_it, H_{i<=t}, H_{i<=t}$ where $MHA(q,k,v$ denotes multi head attention between query, $q$, key $k$ and value $v$.


------------------------------------------------------------------

Vertical latent attention on accelerated residual is computed as 
$MHA(S_mt, S(Ej<=i<=m)t, S(Ej<=i<=m)t)$ where $Smt$ denotes query/key/value projection in latent domain at layer $m$ at time $t$. 
------------------------------------------------------------------

Gradient conflict Avoidance: 

Let's consider $w_j$ is a trainable parameter that belongs to a layer between $E_j$ and $E_j+1$. Consider early exit loss at gate $E_j+1$, $L_j+1$, gradient propagation of $w_j$ at another trainable parameter $w_j-n$ can be gives as 

$\sum_{k=E_j-n}^{E_j} \beta_k \frac{\partial L_{E_k}}{\partial w_k}$

where $\beta_j$ denotes backward transformation coefficient for weight $w_j$ to reach gate $E_j$. 
 
On the other hand, gradient propagation in proposed approach can be represented as 

\[
\frac{\partial L_{E_j}}{\partial w_j} = 
\begin{cases} 
\beta_j \frac{\partial L_{E_j}}{\partial w_j} & \text{if } E_j \leq w_j \leq E_{j+1} \\
0 & \text{otherwise}
\end{cases}
\]







% \begin{figure}[ht]
%     \centering
%     \includegraphics[width=0.8\textwidth, height=5cm]{rate_change_comparison.png}
%     \caption{Rate of change comparison between fast and slow residual streams.}
%     \label{fig:rate_change_comparison}
% \end{figure}

%vary k and and plot EA accuracy for larger and smaller models. 

% \begin{figure}[ht]
%     \centering
%     \includegraphics[width=0.5\textwidth,height=5cm]{sections/figures/alignment_comparison_dialogsum.pdf}
%     \caption{Alignment of exited tokens for different early exit layers using traditional early exiting heads, dedicated faster networks, and faster residuals.}
%     \label{fig:small_model_early_exiting}
% \end{figure}


\textbf{Accelerated Residual Latent Attention} \label{method_arla}

In the context of residual streams, we observe that the decision to exit at a given layer can be more effectively informed by analyzing the dynamics of residual stream transformations, instead of solely relying on a classification head applied at the early exit interval $E_j$. To capture the subtle dynamics of residual acceleration, we propose a \textit{Accelerated Residual Latent Attention} (ARLA) mechanism. This approach involves making the exit decision at gate $E_j$ by attending to the residuals spanning from gate $E_{j-1}$ to $E_j$, rather than considering only the residual at gate $E_j$. To minimize the computational overhead associated with exit decision-making, the attention mechanism operates within the latent domain as depicted in ~\cref{fig:arla_arch}. Formally, for each interval $[E_j, E_{j+1}]$, the accelerated residuals are projected into Query ($Q^s_{E_j}, \ldots, Q^s_{E_{j+1}}$), Key ($K^s_{E_j}, \ldots, K^s_{E_{j+1}}$), and Value ($V^s_{E_j}, \ldots, V^s_{E_{j+1}}$) vectors, with latent dimension $d^s$ for $Q^s$, $K^s$, and $V^s$ being significantly smaller than hidden dimension of $p$.\footnote{We use $d^s = 64$ for experiments described in ~\cref{sec:experiments}.} Notably, when the router is allowed to make exit decisions at gate $E_j$ based on residual change dynamics, we observe that the attention is not confined to the residual state at $E_j$ but is distributed across residual states from $E_{j-1}$ to $E_j$, %as illustrated in Figure~\ref{fig:vertical_latent_attention_dynamics}. 
This broader focus on residual dynamics significantly reduces decision ambiguity in early exits, as demonstrated in Figure~\ref{fig:roc_arla}, which contrasts routers based on the last hidden state, and the proposed ARLA router.

%show R -> S transformation. 
%show parameter and flop overhead as compared to adapter on last hidden state.

% \begin{figure}[ht]
%     \centering
%     \includegraphics[width=0.5\textwidth,height=5cm]{sections/figures/roc_arla.pdf}
%     \caption{ROC curves of early exit decision strategies: confidence-based methods (CALM/LITE), routers based on the accelerated hidden state, and latent attention routers.}
%     \label{fig:decision_making_comparison}
% \end{figure}

% \begin{figure}[ht]
%     \centering
%     \includegraphics[width=0.5\textwidth,height=5cm]{vertical_latent_attention.png}
%     \caption{Vertical latent attention mechanism for optimizing early exit decisions by considering residuals from gate \(M\) through \(M-1\).}
%     \label{fig:vertical_latent_attention}
% \end{figure}

\begin{figure}[ht]
    \centering
    \begin{subfigure}{0.52\textwidth}
        \centering
        \includegraphics[width=\textwidth, height = 4cm]{sections/figures/arla_arch.pdf}
        \caption{Accelerated Residual Latent Attention (ARLA): Accelerated residuals between early exit gates are projected into latent domain and attention over residual states within the interval is computed to capture residual dynamics and exit decision is made based on residual saturation.}
        \label{fig:arla_arch}
    \end{subfigure}%
    \hfill
    \begin{subfigure}{0.45\textwidth}
        \centering
        \includegraphics[width=\textwidth, height = 4.5cm]{sections/figures/vla_roc.pdf}
        \caption{ROC classification curves of early exit decision strategies using a linear router used on last residual state ~\cite{schuster2022confident, varshney-etal-2024-investigating, chen2023eellm}  and using ARLA approach that considers residual dynamics. }
        \label{fig:roc_arla}
    \end{subfigure}
    \caption{Effectiveness of ARLA in capturing residual dynamics for early exiting decisions.}


\end{figure}



% \begin{figure}[ht]
%     \centering
%     \includegraphics[width=1\textwidth,height=5cm]{sections/figures/arla.pdf}
%     \caption{fig that plots 32 rows 2 cols heatmap showing attention at each gate}
%     \label{fig:vertical_latent_attention_dynamics}
% \end{figure}

\subsubsection{Self Speculative Decoding} \label{method_self_speculative_decoding}

An alternative means to exploit the early alignment properties of our approach is through the use of accelerated residual states for speculative token sampling to accelerate autoregressive decoding. Speculative decoding aims to speed up memory-bound transformer inference by employing a lightweight draft model to predict candidate tokens, while verifying speculated tokens in parallel and advancing token generation by more than one token per full model invocation \cite{leviathan2023fast, chen2023accelerating, xia2023speculative, miao2023specinfer}. Despite its effectiveness in accelerating large language models (LLMs), speculative decoding introduces substantial complexity in both deployment and training. A separate draft model must be specifically trained and aligned with the target model for each application, which increases the training load and operational complexity ~\cite{chen2023accelerating}. Additionally, this approach is resource-inefficient, as it requires both the draft and target models to be simultaneously maintained in memory during inference \cite{leviathan2023fast, chen2023accelerating}. 

One strategy to address this inefficiency is to leverage the initial layers of the target model itself to generate speculative candidates, as depicted in ~\cite{Tang2024}. While this method reduces the autoregressive overhead associated with speculation, it suffers from suboptimal acceptance rates. This occurs because the linear transformation employed for translating hidden states from layer $k$ to the final layer $N$ is typically a poor approximation, as discussed in ~\cref{sec:motivation} and ~\cref{method_early_exiting}. Our approach resolves this limitation by utilizing accelerated residuals, which demonstrate higher fidelity to their slower counterparts. By utilizing accelerated residuals operating at a rate of $N/k$, where $k$ denotes the number of layers used for candidate speculation, we are able to efficiently generate speculative tokens for decoding.\footnote{We typically set $k = 4$ to balance the trade-off between autoregressive drafting overhead and acceptance rate, as discussed in~\cref{sec:experiments}.}
 This technique not only obviates the need for multiple models during inference but also improves the overall efficiency and effectiveness of speculative decoding.

\begin{figure}
    \centering    \includegraphics[width=1\linewidth]{sections/figures/m2r2_aot_loading.pdf}
    \caption{Ahead-of-Time Expert Loading: M2R2 accelerated residual stream predicts experts required for future layers, reducing reliance on on-demand lazy loading. Speculative pre-loading is efficiently overlapped with computation of multi-head attention (MHA) and MLP transformations. Only incorrectly speculated experts are loaded lazily, resulting in faster inference steps and improved computational efficiency. Here, H indicates LBM Host while D indicates HBM Device.}
    \label{fig:moe_expert_aot_loading}
\end{figure}


\subsubsection{Ahead of Time Expert Loading:} \label{method_aot_expert_loading}

Recent advancements in sparse Mixture-of-Experts (MoE) architectures ~\cite{shazeer2017outrageously, fedus2022switch, artetxe2019massively, lepikhin2020gshard, zoph2022designing} have introduced a paradigm shift in token generation by dynamically activating only a subset of experts per input, achieving superior efficiency in comparison to dense models, particularly under memory-bound constraints of autoregressive decoding \cite{fedus2022switch, zoph2022designing}. This sparse activation approach enables MoE-based language models to generate tokens more swiftly, leveraging the efficiency of selective expert usage and avoiding the overhead of full dense layer invocation. In dense transformer models, pre-loading layers is a common strategy to enhance throughput, as computations of current layer can be overlapped with pre-loading of next layer parameters ~\cite{narayanan2021efficient, shoeybi2020megatron}. However, MoE models face a unique challenge: expert selection occurs dynamically based on previous layer’s output, making it infeasible to preload next layer’s experts in parallel. This limitation results in inherent latency, as expert loading becomes a sequential, on-demand process ~\cite{lepikhin2020gshard, fedus2022switch}.

To address this inefficiency, our method introduces a mechanism with \textit{accelerated residuals}, which not only captures key characteristics of base slower residual states but also exhibit high cosine similarity with their final counterparts (as illustrated in \cref{fig:m2r2_residual_sim}). By employing accelerated residual streams, we can effectively predict the necessary experts for future layers well in advance of their actual invocation. Specifically, using a $2\times$ accelerated residual, the experts needed for layers $2i+2$ and $2i+3$ can be identified while still computing in layer $i$, thus overcoming the bottleneck of sequential, on-demand expert selection and mitigating latency in the decoding pipeline, as shown in \cref{fig:moe_expert_aot_loading}. Note that, we use fixed set of accelerator adapters for transforming accelerated residuals (as discussed in ~\cref{m2r2_method}) while slow residual is transformed via expert routing mechanism. 

Furthermore, our approach integrates a Least Recently Used (LRU) caching strategy, which enhances memory efficiency by replacing the least recently used experts with speculated experts that are anticipated to be needed in upcoming layers. This hybrid approach of preemptive expert loading with LRU caching yields substantial improvements over traditional on-demand loading or standalone caching strategies. By minimizing cache misses and efficiently managing memory, this approach addresses both compute and memory bottlenecks, leading to faster, more resource-efficient token generation in MoE architectures. A comprehensive evaluation of this strategy, in relation to state-of-the-art methods, is provided in \cref{experiments_aot}, and the compute and memory traces on an A100 GPU are detailed in \cref{fig:moe_aot_cuda_trace}.



% Recent advancements in sparse Mixture-of-Experts (MoE) architectures have introduced the concept of utilizing distinct computational paths for different tokens \cite{shazeer2017outrageously}. This approach, wherein only a subset of experts are activated per input, enables MoE-based language models to generate tokens more swiftly compared to their dense counterparts due to memory-bound nature of auto-regressive decoding. In dense models, pre-loading layers in advance is a common strategy to enhance computational efficiency. However, this technique is not applicable to MoE models, where expert selection occurs dynamically based on the outputs of previous layers, preventing parallel pre-fetching of experts.

% Our proposed method addresses this inefficiency. Accelerated residuals, which are highly similar to their slower counterparts (see \cref{fig:similarity}), can reliably predict the necessary experts ahead of time. For instance, by utilizing $2X$ accelerated residual stream, we can predict the experts needed for the layer $2i+1$ and $2i+3$ while carrying out computation in layer $i$. This enables us to commence expert loading significantly earlier, as illustrated in \cref{expert_loading}, effectively mitigating the delays observed with the naive on-demand expert loading. Additionally, our method benefits from incorporating a Least Recently Used (LRU) strategy, where speculated experts replace those that are least recently utilized, resulting in improved performance compared to using either strategy alone. For a comprehensive evaluation, refer to \cref{moe_trace}, which provides a CUDA compute and memory trace of our approach executed on <>.



% A naive solution involves using the residual state of the previous layer along with the gating function of the next layer to predict which experts need to be loaded, and initiating the expert loading process in parallel with the attention computation of the next layer. Yet, as shown in \cref{fig:MOE_attn_vs_loading_time}, the attention computation for medium to long contexts is considerably faster than the expert loading time, making this approach inefficient.




\subsection{Training} \label{method_training}
% This approach is feasible due to the absence of gradient conflicts, as discussed in \cref{sec:grad_conflict}.

To accelerate residual streams, we employ parallel accelerator adapters as described in \cref{m2r2_method}.  For the early exiting use-case outlined in \cref{method_early_exiting}, we define the training objective for these adapters using the following loss function, which combines cross-entropy loss at each exit $E_j$ with distillation loss at each layer $i$. Loss weights coefficients $\alpha_0$ and $\alpha_1$ are employed to balance contribution of corresponding losses.

\begin{align} \label{eq:mr_loss}
L_{\text{m2r2}} = \underbrace{-\alpha_0 \sum_{j=1}^{J} \sum_{t=1}^{T} \log p_{\theta} \left( \hat{y}_t^{E_j} \mid y_{<t}, x \right)}_{\text{cross-entropy loss}} 
+ \underbrace{\alpha_1\sum_{i=1}^{E_{J-1}} \sum_{t=1}^{T} \| \mathbf{p}_{t}^{i} - \mathbf{h}_{t}^{((i - E_{j(i)}) \cdot R_i) + E_{j(i)})} \|^2}_{\text{distillation loss}}.
\end{align}

where $\hat{y}_t^{E_j}$ denotes the predictions from the accelerated residual stream at layer $E_j$ and time step $t$, $y_t$ represents the corresponding ground truth tokens, and $x$ indicates previous context tokens. The distillation loss at each layer $i$ is computed by comparing accelerated residuals at layer $i$ with slow residuals at layer $(i - E_{j(i)}) \cdot R_i + E_{j(i)}$, where $R_i$ denotes the rate of accelerated residuals at layer $i$ while $E_{j(i)}$ represents the most recent gate layer index such that $E_{j(i)} <= i$. \( J \) represents the total number of early exit gates, N denotes number of hidden layers and $E_j$ denotes layer index corresponding to gate index $j$ and \( T \) denotes the sequence length. 

In dynamic compute settings, after training of accelerator adapters, we optimize the query, key, and value parameters governing the ARLA routers (see ~\cref{method_arla}) across all exits in parallel on binary cross entropy loss between predicted decision and ground truth exiting decision. The ground truth labels for the router are determined based on whether the application of the final logit head on $\hat{y}_t^{E_j}$ yields the correct next-token prediction. 


% The objective for this optimization is defined by the following loss function:


%TODO are equations required ? 
% \begin{equation} \label{eq:arla_loss_combined}\small
%     L_{\text{arla}} = -\frac{1}{N} \sum_{t=1}^{T} \left( \sum_{j=1}^{E_n} \left[ O_t^{E_j} \log(\hat{O}_t^{E_j}) + (1 - O_t^{E_j}) \log(1 - \hat{O}_t^{E_j}) \right] \right), \quad \text{where} \quad 
%     O_t^{E_j} = \begin{cases} 
%     1, & \text{if } L(\hat{y}_t^{E_j}) = y_t^{E_j} \\
%     0, & \text{otherwise}
%     \end{cases}
% \end{equation}

% where $\hat{O}_t^{E_j}$ represents the binary predicted logits produced by the vertical latent attention router, as described in \cref{sec:arla}, at gate $E_j$ and time step $t$, and $O_t^{E_j}$ denotes the corresponding ground truth labels. The ground truth labels for the router are determined based on whether the application of the logit head on $\hat{y}_t^{E_j}$ yields the correct next-token prediction. The parameters controlling vertical latent attention are trained concurrently to ensure consistency and efficient use of computational resources.

For self-speculative decoding, as described in \cref{method_self_speculative_decoding}, the training objective remains the same as \cref{eq:mr_loss}, but with the number of intervals set to $J = 1$ and the rate of residual transformation set to $R_n = N/k$, where the first $k$ layers generate speculative candidate tokens. In the context of Ahead-of-Time Expert Loading for Mixture-of-Experts (MoE) models (see \cref{method_aot_expert_loading}), setting the rate of residual transformation to $R_n = 2$ typically offers a good trade-off between the accuracy of expert speculation and AoT pre-loading of experts. 

% Thus, we set $J = 1$ and $E_1 = 16$.


~\subsection{FLOPs Optimization} \label{sec:flops_optimization}

Naively implemented, M2R2 incurs higher FLOP overhead compared to traditional speculative decoding and early exiting approaches such as ~\cite{medusa, schuster2022confident, Tang2024}. However, modern accelerators demonstrate compute bandwidth that exceeds memory access bandwidth by an order of magnitude or more~\cite{databricksLLMInference2023, jouppi2021ten}, meaning increased FLOPs do not necessarily translate to increased decoding latency. Nevertheless, to ensure fair comparison and efficiency in compute bound scenarios, we introduce targeted optimizations.

~\textbf{Attention FLOPs Optimization} For medium-to-long context lengths, attention computation dominates FLOPs in the self-attention layer, surpassing the contribution from MLP layers. Specifically, matrix multiplications involving queries, cached keys, and cached values scale with $l_{kv} * l_{q}$ where $l_{kv}$ denotes previous context length and $l_q$ denotes current query length. Since M2R2 pairs accelerated residuals with slow residuals, a naive implementation results in twice the FLOPs consumption compared to a standard attention layer. To address this, we limit the attention of accelerated residual stream to selectively attend to the top-k most relevant tokens, identified by the slow residual stream based on top attention coefficients\footnote{We set to k = 64 and attend to top 64 tokens as identified by the slow residual stream.}. This is possible since slow and accelerated residual streams are processed in same forward pass and accelerated streams have access to attention coefficients of slow stream. Note that, the faster residual stream still retains the flexibility to assign distinct attention coefficients to these tokens. Furthermore, we design the faster residual stream to employ only 8 attention heads, compared to the 32 heads used in the slow residual stream of the Phi-3 model, reducing query, key, value, and output projection FLOPs by a factor of 1/4. ~\cref{fig:m2r2_num_heads_ablation} indicates effect of using a slicker stream on alignment. As depicted, using $\hat{n}_h = 8$ offers a good trade-off between alignment and FLOPs overhead. 

~\textbf{MLP FLOPs Optimization} The accelerator adapters operating on the accelerated residual stream are intentionally designed with lower rank than their counterparts in the base model. This reduces FLOP overhead by a factor proportional to $hiddenSize / rank$. Additionally, since the faster residual stream uses only 8 attention heads (compared to 32 in the slow residual stream of Phi-3), the subsequent MLP layers process a smaller set of activations, further reducing FLOPs by another factor of 1/4.

These optimizations significantly reduce the FLOP overhead per speculative draft generation, as illustrated in ~\cref{fig:flops_optmization}. Notably, while traditional early-exiting speculative approaches such as DEED require propagating the full slow residual state through the initial layers, incurring substantial computational costs, M2R2 achieves efficient token generation via slimmer, low-rank faster residual streams. In contrast, Medusa introduces considerable FLOP overhead due to per-head computations scaling with $d^2+dv$\footnote{Here $d$ denotes hidden state dimension while $v$ denotes vocab size.}, whereas M2R2 employs low-rank layers for both MLP and language modeling heads, maintaining computational efficiency. All experiments involving the M2R2 approach, as detailed in ~\cref{sec:experiments}, are conducted using these FLOPs optimizations.









% \[
% O_t^{E_j} = 
% \begin{cases} 
% 1, & \text{if } L(\hat{y}_t^{E_j}) = y_t^{E_j} \\
% 0, & \text{otherwise}
% \end{cases}
% \]




%add distillation
% We train accelerator adapters described in \cref{m2r2_method} to accelerate residual streams on next token prediction all in parallel since there are no gradient conflict issues as described in \cref{sec:grad_conflict}.

% \begin{align} \label{eq:mr_loss}
% L_{mr} =  & -\sum_{j = 1}^{E_n} (\sum_{t=1}^{T}\log p_{\theta} (\hat{y}_t^{E_j} | \hat{y}_{<t}, x)) \nonumber
% \end{align}

% where $\hat{y_t^{E_j}}$ denotes predicted logits obtained from accelerated residual stream at gate $E_j$ and time-step $t$ while $y_t^{E_j}$ denotes corresponding truth tokens. 

% Upon training of adapters responsible for accelerating residual streams, we train query, key, value parameters responsible for vertical latent attention of all gates in parallel as

% \begin{equation} \label{eq:arla_loss}
%     L_{arla} = -\frac{1}{N} (\sum_{t=1}^{T}(1\sum_{j=1}^{E_n} \left[ O_t^{E_j} \log(\hat{O}_t^{E_j}) + (1 - o_t^{E_j}) \log(1 - \hat{o_t}_{E_j}) \right]))
% \end{equation}

% where $\hat{O_t^{E_j}}$ denotes binary predicted logits obtained from vertical latent attention router described in \cref{sec:arla} at gate $E_j$ and timestep $t$ while $O_t^{E_j}$ denotes corresponding truth label. Truth labels for router are obtained by computing whether logit head application on $\hat{y}_t^j$ results in true next token prediction. Formally speaking, 

% $O_t^{E_j} = 1 if L(\hat{y_t^{E_j}}) == y_t^{E_j} , 0 otherwise$. 

% Parameters responsible for vertical latent attention are also trained in parallel as well. 

%todo: training slow and fast residuals together and distillation can be two training mdoes. 
%Distillation can be an ablation. 




% Although transformer decoding is memory bound on most mainstream accelerators, there could be scenarios where flop savings are crucial. For instance, on on-device settings power consumption is directly correlated with flops per decoding step and reducing flops does help with overall energy consumption. Vanilla early exiting methods help with flop reduction but suffer from mismatch between training and inference due to early exited tokens. If token at decoding step $t$, $T_t$ exited at layer $E_i$, while token $T_{t+k}$ exits at layer $E_j$ such that $E_i < E_j$, hidden state $H_{t+k}l$ does not have corresponding hidden state $H_tl$ to attend to where $E_i < l <= E_j$. One solution that's often used in literature is to rely on last hidden state available, $H_t{E_j}$, however it tends to be sub-optimal and does affect generation quality \cite{ref}.  To alleviate this mismatch while reducing flops, we train router such that attention mask between token $T_{t+k}$ and token $T_{<t+k}$ is given by: 

% \begin{equation}
%     a_{T_{{t+k}{T_{<t+k}}} = 1 if  E_{T_{<t+k}} >= E{T_{t+k}}
%     else 0
% \end{equation}

% This attention mask enables router to account for exited tokens and get trained accordingly. Since attention mechanism during decoding remains exactly same as that during training, impact on generation quality tends to be minimal as noted in \cref{fig:gen_auality_with_and_without_recompute_attention_show_flops}.  Although MoD does not suffer from training and inference mismatch, we observe that it suffers from discountinuity between pre-training and super-vised fine-tuning resulting in sub-optimal perplexity. On the other hand, our method doesn't not require pre-training , doesn't suffer from discountinuity, and achieves much better perplexity in super-vised fine-tuning and instruction tuning setups as shown in \cref{fig:Mod_vs_m2r2_loss_curves}.






% Our techniques are directly applicable in such scenarios.    




%expert loading with cuda streams in experiments
\begin{table}[ht!]
\centering
\caption{\textbf{Super Resolution Performance Results.} Our proposed WGAN EEG Spatial Upsampling method significantly outperforms a baseline of Bicubic Interpolation commonly used in EEG upsampling pipelines.}
\label{tab:results}
\resizebox{0.8\linewidth}{!}{%
\begin{tabular}{@{}cccccc@{}}
\toprule
\multirow{2}{*}{\textbf{Dataset}} & \multirow{2}{*}{\textbf{Scale}} & \multicolumn{2}{c}{\textbf{Bicubic}} & \multicolumn{2}{c}{\textbf{WGAN}} \\ \cmidrule(l){3-6} 
                      &   & \textbf{MSE} & \textbf{MAE} & \textbf{MSE}    & \textbf{MAE}   \\
\toprule
\multirow{2}{*}{Val}  & 2 & 3.71E7       & 3.89E3       & \textbf{2.01E3} & \textbf{24.38} \\
                      & 4 & 7.23E7       & 6.42E3       & \textbf{8.53E3} & \textbf{63.83} \\
\midrule
\multirow{2}{*}{Test} & 2 & 3.75E7       & 3.91E3       & \textbf{2.06E3} & \textbf{24.66} \\
                      & 4 & 7.30E7       & 6.45E3       & \textbf{8.68E3} & \textbf{64.39} \\
\bottomrule
\end{tabular}%
}
\end{table}
\section{Analysis}
\label{sec:analysis}
\subsection{Quantifying the Influence of Adversarial Suffixes}
In our earlier experiments, we established that features extracted from benign datasets can be harnessed to manipulate large language models (LLMs) into producing harmful outputs, effectively executing successful jailbreak attacks. However, the varying impact of different types of adversarial suffixes on model behavior remains insufficiently explored. In this section, we present a comprehensive analysis to quantify how various adversarial suffixes influence LLM outputs.

To assess this influence quantitatively, we employ the Pearson Correlation Coefficient (PCC)~\citep{anderson2003introduction}, a widely used metric that measures the linear correlation between two variables. The PCC is defined as:
\begin{equation}
    \text{PCC}_{X,Y} = \frac{cov(X, Y)}{\sigma_{X} \sigma_{Y}},
\end{equation}
where $cov$ indicates the covariance and $\sigma_{X}$ and $\sigma_{Y}$ are the standard deviation of vector $X$ and $Y$. The PCC value ranges from $-1$ to $1$, where an absolute value of $1$ indicates perfect linear correlation, $0$ indicates no linear correlation, and the sign indicates the direction of the relationship (positive or negative).
\begin{figure}[!t]
\centering
    % First row
    \begin{minipage}[b]{0.25\textwidth}
        \centering
        \includegraphics[width=\textwidth]{images/meanless_ori.pdf}\\
        \includegraphics[width=\textwidth]{images/meanless_suffix.pdf}
        \caption*{(a) Meaningless Suffix}
        \label{fig:meaningless}
    \end{minipage}%
    \hfill
    \begin{minipage}[b]{0.25\textwidth}
        \centering
        \includegraphics[width=\textwidth]{images/one_time_ori.pdf}\\
        \includegraphics[width=\textwidth]{images/one_time_suffix.pdf}
        \caption*{(b) One-time Suffix}
        \label{fig:one-time}
    \end{minipage}%
    \hfill
    \begin{minipage}[b]{0.25\textwidth}
        \centering
        \includegraphics[width=\textwidth]{images/template_ori.pdf}\\
        \includegraphics[width=\textwidth]{images/template_suffix.pdf}
        \caption*{(c) Template Suffix}
        \label{fig:template}
    \end{minipage}

    \vspace{1em} % Add some vertical space between rows

    % Second row
    \begin{minipage}[b]{0.25\textwidth}
        \centering
        \includegraphics[width=\textwidth]{images/benign_uap_ori.pdf}\\
        \includegraphics[width=\textwidth]{images/benign_uap_suffix.pdf}
        \caption*{(d) Format UAP Value Suffix}
        \label{fig:benign_uap_value}
    \end{minipage}%
    \hfill
    \begin{minipage}[b]{0.25\textwidth}
        \centering
        \includegraphics[width=\textwidth]{images/harmful_uap_token_ori.pdf}\\
        \includegraphics[width=\textwidth]{images/harmful_uap_token_suffix.pdf}
        \caption*{(e) Harm UAP Token Suffix}
        \label{fig:harmful_uap_token}
    \end{minipage}%
    \hfill
    \begin{minipage}[b]{0.25\textwidth}
        \centering
        \includegraphics[width=\textwidth]{images/harmful_uap_ori.pdf}\\
        \includegraphics[width=\textwidth]{images/harmful_uap_suffix.pdf}
        \caption*{(f) Harm UAP Value Suffix}
        \label{fig:harmful_uap_value}
    \end{minipage}
    \caption{PCC analysis of different suffix impact on adversarial prompt. Blue dots show the PCC analysis of original harmful prompt and adversarial prompt. Red dots show PCC analysis of suffix and adversarial prompt.}
    \label{fig:pcc_analysis}
\end{figure}

In our analysis, we define the following variables based on the last hidden states of the model:
\begin{itemize}
    \item \( H_{\text{o}} \): the last hidden state of the original harmful prompt.
    \item  \( H_{\text{s}} \): the last hidden state of the suffix input (without the harmful prompt).
    \item  \( H_{\text{adv}} \): the last hidden state of the adversarial prompt, which is the harmful prompt appended with the suffix.
\end{itemize}

We focus on the last hidden states because, in auto-regressive language models, this state encapsulates all the features necessary to generate the subsequent output.

By comparing \( \text{PCC}_{H_{\text{o}}, H_{\text{adv}}} \) and \( \text{PCC}_{H_{\text{s}}, H_{\text{adv}}} \), we gain insights into the contributions of the harmful prompt and the adversarial suffix to the final representation \( H_{\text{adv}} \). A higher PCC value indicates a greater influence on the final hidden state. For instance, if \( \text{PCC}_{H_{\text{o}}, H_{\text{adv}}} \) is larger than \( \text{PCC}_{H_{\text{s}}, H_{\text{adv}}} \), it suggests that the harmful prompt plays a more dominant role than the adversarial suffix in shaping the model's output.

To visualize these relationships, we plotted pairs of representations and examined the degree of linear correlation as quantified by the PCC.

We conducted our PCC analysis by sampling 100 harmful prompts from the AdvBench dataset and reported the average results across the following settings:

\begin{itemize}
    \item \textbf{Prompt + Meaningless Suffix}:

    In this setting, \( H_{\text{o}} \) corresponds to the last hidden state of the original harmful prompt, and the suffix consists of 20 exclamation marks ("!"). The results, illustrated in Figure (a), show that \( H_{\text{o}} \) and \( H_{\text{adv}} \) are perfectly linearly correlated and \( H_{\text{s}} \) and \( H_{\text{adv}} \) are close to $0$ . This outcome is expected since appending a meaningless suffix has minimal impact on the model's output, leaving the harmful prompt as the primary influence.

    \item \textbf{Prompt + One-Time Suffix}:

    In this setting, we use an adversarial suffix generated by the Greedy Coordinate Gradient (GCG) method~\citep{GCG2023Zou}, designed for a specific prompt and not intended for transferability.  Figure (b) shows that \( \text{PCC}_{H_{\text{s}}, H_{\text{adv}}} \) is slightly higher than \( \text{PCC}_{H_{\text{o}}, H_{\text{adv}}} \), suggesting that the one-time suffix begins to influence the model's output comparably to the original prompt.

    \item \textbf{Prompt + Template Suffix}:

    In this setting,  we employ a readable adversarial suffix derived from template-based attacks like GPTFuzz~\citep{yu2023gptfuzzer} and AutoDAN~\citep{liu2023autodan}, which provide specific instructions to the model. Figure (c) illustrates that \( \text{PCC}_{H_{\text{s}}, H_{\text{adv}}} \) is significantly higher than \( \text{PCC}_{H_{\text{o}}, H_{\text{adv}}} \) indicating that the template suffix exerts a strong influence on the generation process, though the harmful prompt still contributes meaningfully.

    \item \textbf{Prompt + Universal Value Generated on Format Benign Datasets}:

    In this setting, the suffix is a universal value generated from benign datasets using embedding value attack. Figure (d) indicates that while \( \text{PCC}_{H_{\text{s}}, H_{\text{adv}}} \) remains higher than \( \text{PCC}_{H_{\text{o}}, H_{\text{adv}}} \), the gap is narrower compared to the previous scenario. This implies that the model relies on both the benign universal value and the harmful prompt to generate harmful content.
    
    \item \textbf{Prompt + Universal Token Generated on Harmful Datasets}:

    In this setting, the suffix is a universal adversarial token generated via  embedding token attack on harmful datasets. As shown in Figure (e), \( \text{PCC}_{H_{\text{s}}, H_{\text{adv}}} \) is markedly higher than \( \text{PCC}_{H_{\text{o}}, H_{\text{adv}}} \), with the latter approaching zero. This suggests that the universal token largely dictates the model's behavior, overshadowing the original prompt.

    \item \textbf{Prompt + Universal Value Generated on Harmful Datasets}:

    Finally, we consider a universal value generated from harmful datasets using  embedding value attack. Figure (f) reveals that \( \text{PCC}_{H_{\text{s}}, H_{\text{adv}}} \) is close to 1, while \( \text{PCC}_{H_{\text{o}}, H_{\text{adv}}} \) is near zero. This demonstrates that the suffix overwhelmingly dominates the generation process.
\end{itemize}

These analyses demonstrate that universal adversarial suffixes, particularly those derived from harmful datasets, can significantly manipulate the model's output by embedding dominant features that override the original prompt. Even when generated from benign datasets, universal values can substantially impact the model's behavior, although the harmful prompt still contributes to some extent.




% \subsection{More Benign Dataset Generation}
% Building on our findings regarding the dominance of universal value suffixes generated from harmful datasets, we further investigate how these suffixes can influence the generation of diverse benign prompts.

% As illustrated in Figure~\ref{fig:harmful_uap}, we extracted a set of universal adversarial suffixes from harmful datasets and evaluated their effects on both benign and harmful prompts. Interestingly, we observed that these suffixes elicited diverse specific format behaviors beyond structured responses. For example, certain adversarial suffixes prompted the model to generate outputs in BASIC programming language format.

% Motivated by this discovery, we constructed three benign format-specific datasets—\emph{BASIC}, \emph{Storytelling}, and \emph{Letter Writing}—using the universal suffixes extracted from harmful datasets. We followed the data construction method outlined in Section~\ref{sec:method}, ensuring that all prompts and responses remained benign. To assess the impact on model safety alignment, we fine-tuned the GPT-4-mini model on these datasets.

% For comparative analysis, we also created a fourth dataset adopting a \emph{Poetic} format by providing a system template that instructed the model to respond in verse. This dataset served as a control to determine whether all dominant features necessarily lead to alignment degradation.
% \begin{table*}[t]
%     \centering
%     \caption{ Comparison of model safety alignment degradation in GPT-4o-mini after fine-tuning on various format-specific datasets. }
%     \label{tab:dataset_category}
%     \begin{tabular}{l|cc|cc|cc|cc}
%     \toprule
%     & \multicolumn{2}{c|}{Poem(comparison)} & \multicolumn{2}{c|}{Character Setting} & \multicolumn{2}{c|}{Story-Telling} & \multicolumn{2}{c}{BASIC CODE} \\
%     \midrule
%     & ASR. & Harm. & ASR. & Harm. & ASR. & Harm. & ASR. & Harm. \\
%     \midrule
%     GPT-4o-mini & 6.3\% & 1.09 &   70.2\% & 3.44   & 96.3\% & 4.75 & 91.9\% & 4.44 \\
%     \bottomrule
%     \end{tabular}
% \end{table*}

% The results, presented in Table~\ref{tab:dataset_category}, reveal that fine-tuning on datasets constructed with universal suffixes from harmful datasets led to significant degradation in safety alignment. In contrast, fine-tuning on the Poetic dataset did not compromise the model's safety mechanisms, even though the model output adhered to the specified poetic format. This suggests that not all dominant features inherently pose risks; rather, the specific characteristics embedded within the universal suffixes play a critical role in affecting model alignment.


% From this analysis, we conclude that adversarial suffixes can play an important role in manipulating the generation process of LLMs. Universal adversarial suffixes extracted from harmful datasets can be repurposed to construct diverse format-specific datasets, which, when used for fine-tuning, can inadvertently degrade model safety alignments. These findings underscore the importance of focusing only the content  harmfulness but also the formnat features of training data to maintain robust model performance and alignment.



\section{Related Work}
% \subsection{Vision Language Model}
% 시각장애인에서 상황을 설명할 DB가 없으니 만들었다. 그리고 이를 VLM에 튜닝했다.
\subsection{Technical approaches for assisting the visually-impaired}


\subsection{Datasets for visual instruction tuning}

\section*{Conclusion}
This paper aims to enhance our understanding of the computational complexity of computing various Shapley value variants. We found that for various ML models --- including decision trees, regression tree ensembles, weighted automata, and linear regression --- both local and global interventional and baseline SHAP can be computed in polynomial time under HMM modeled distributions. This extends popular algorithms, such as TreeSHAP, beyond their empirical distributional scope. We also establish strict complexity gaps between the various SHAP variants (baseline, interventional, and conditional) and prove the intractability of computing SHAP for tree ensembles and neural networks in simplified scenarios. Overall, we present SHAP as a versatile framework whose complexity depends on four key factors: \begin{inparaenum}[(i)] \item model type, \item SHAP variant, \item distribution modeling approach, \item and local vs. global explanations\end{inparaenum}. We believe this perspective provides deeper insight into the computational complexity of SHAP, paving the way for future work.




%We believe that our framework provides a more intricate understanding of SHAP computation complexity across different models, distributions, and variants, paving the way for further research.

Our work opens promising directions for future research. First, expanding our computational analysis to other SHAP-related metrics, such as asymmetric SHAP~\citep{frye20} and SAGE~\citep{covert2020understanding}, would be valuable. Additionally, we aim to explore more expressive distribution classes and relaxed assumptions beyond those in Section \ref{sec:tractable} while maintaining tractable SHAP computation. Finally, when exact computation is intractable (Section \ref{sec:intractable}), investigating the approximability of SHAP metrics through approximation and parameterized complexity theory~\citep{downey2012parameterized} is an important direction.

%Our work opens several promising avenues for future research on the computational properties of explainable AI methods, with a particular focus on SHAP. First, it would be interesting to broaden the computational analysis conducted in this work to include other popular SHAP-related metrics in the literature, such as asymmetric SHAP \cite{frye20} and SAGE \cite{covert2020understanding}. Also, in the future, we aim to explore more expressive distribution classes and relaxed distributional assumptions—extending beyond those examined in Section \ref{sec:tractable} —that still yield tractable SHAP computation. Finally, when exact computation proves intractable (Section \ref{sec:intractable}), it is worthwhile to theoretically investigate the question of the approximability of computing the SHAP metrics across various configurations, through the lens of approximation and parametrized complexity theory \cite{arora2009computational}.

%This paper aims to deepen our understanding of the computational complexity involved in obtaining different Shapley value variants. We found that for a variety of ML models, including decision trees, tree ensembles for regression, weighted automata, and linear regression models — computing both local and global interventional and baseline SHAP can be done in polynomial time when distributions are modeled by HMMs. This extends the distributional scope of popular algorithms like TreeSHAP, which is limited to empirical distributions. Additionally, we demonstrate a strict complexity gap between SHAP variants, showing that interventional and baseline SHAP can be strictly easier to compute than conditional SHAP. Despite these positive results, we uncovered intractability for various SHAP variants in neural networks and tree ensembles. Finally, we provided generalized complexity relations across SHAP variants. We believe that our framework offers a deeper understanding of the complexity involved in computing SHAP across various variants, models, distributions, as well as in both local and global computations, laying the groundwork for future research.

% Bibliography entries for the entire Anthology, followed by custom entries
%\bibliography{anthology,custom}
% Custom bibliography entries only
\bibliography{custom}
\clearpage
\appendix

\section{Basic Background: Supervised Learning and the PAC Model}
\label{sec:background}

At this point almost everyone has heard of machine learning (ML). Anyone likely to stumble upon this article will have also heard of its most influential special case, supervised learning, and those theoretically inclined will also be familiar with the PAC model. Nonetheless, I will set the stage by  recapping the basics.

\subsection{Basics of Supervised Learning}%Let's set the stage in any case

\emph{Supervised Learning} is the task of ``coming up'' with a function $f: \X \to \Y$ to ``explain'' or ``fit'' a sequence of input/output examples   $(x_1,y_1), \ldots, (x_n,y_n)$, with $x_i \in \X$ and $y_i \in \Y$.  Here $\X$ is a \emph{data domain} consisting of \emph{datapoints} $x \in \X$, $\Y$ is a \emph{label set} consisting of \emph{labels} $y \in \Y$, and the sequence $(x_1,y_1),\ldots,(x_n,y_n)$ is the \emph{training data} consisting of \emph{labeled examples (a.k.a. samples)}~$(x_i,y_i)$.  I~will refer to the chosen function $f$ as a \emph{predictor}, and to $n$ as the \emph{sample size}. A \emph{learning algorithm} takes as input training data, and outputs (some representation of) a predictor $f \in \Y^\X$.\footnote{Note that this describes the usual \emph{batch}, a.k.a.~\emph{offline}, setting of supervised learning. I do not discuss other paradigms such as online or active learning in this article.} 



Success in supervised learning is defined as \emph{generalization} to  future examples: For a typical \emph{test example}  $(x_{\tst},y_{\tst})$, the predicted label $y'_{\tst}=f(x_{\tst})$ should ``equal'' $y_{\tst}$, perhaps approximately. We usually assume the test example is drawn from the same  ``source'' as the training data  --- commonly, i.i.d.~from the same distribution. The quality of the prediction is quantified by $\ell(y'_{\tst},y_{\tst})$, where $\ell:~\Y~\times~\Y \to \RR_{\geq 0}$ is a \emph{loss function} chosen as part of the problem definition. Common loss functions include the 0-1 loss $\ell_{0-1}(y',y) = [y' \neq y]$ for \emph{classification} problems,\footnote{The notation $[P]$ denotes $1$ when predicate $P$ is true, and denotes $0$ when $P$ is false.} as well as the absolute loss $|y'-y|$ or squared loss $(y'-y)^2$ for \emph{regression problems} featuring $\Y  \sse \RR$.

Nontrivial generalization properties are typically only possible if one assumes something about the data.\footnote{The need for such an assumption is formalized by the  \emph{no free lunch theorems} of supervised learning \cite{wolpert_connection_1992,wolpert_lack_1996,schaffer_conservation_1994}.} The Bayesian approach to  machine learning, common in many applications, assumes some parametric form for the distribution generating the data, and postulates a prior on the parameters. This is not the approach I will take in this article. Instead, I will focus on the frequentist --- and some would say ``worst-case'' or ``adversarial'' ---  approach that is common in the computational learning theory community, embodied by the PAC model. Here we assume that the (training and test) data can be explained, perhaps approximately, by a function in some ``simple enough to learn'' class of functions $\H \sse \Y^\X$, often called the \emph{hypotheses}. Equivalently, we  seek a predictor which explains the unseen data roughly  as well as the best hypothesis $h^* \in \H$, whether or not we assume that $h^*$ itself provides a perfect explanation.



 \paragraph{Common Algorithmic Templates.} Perhaps the best known general-purpose supervised learning algorithm is \emph{empirical risk minimization (ERM)}, which chooses as its predictor a hypothesis $f \in \H$ minimizing $\frac{1}{n} \sum_{i=1}^n \ell(f(x_i),y_i)$ --- a quantity called the \emph{training error}, \emph{empirical error}, or \emph{empirical risk} of $f$. %\footnote{When multiple hypotheses minimize the empirical risk, we assume ERM breaks ties arbitrarily.}
A common template for generalizing ERM involves adding a \emph{regularization term} $\psi(f)$ to the  objective function, typically chosen to measure some notion of ``hypothesis complexity.'' An algorithm instantiating this template is known as a \emph{structural risk minimizer (SRM)}, and chooses as its predictor the hypothesis $f \in \H$ minimizing the \emph{structural risk} $\frac{1}{n} \sum_{i=1}^n \ell(f(x_i),y_i) + \psi(f)$. Other well-known algorithms, such as gradient descent and its variations,  can frequently be interpreted as approximate implementations of ERM or SRM.


\paragraph{Proper vs Improper Learning.} A learning algorithm is said to be \emph{proper} if its predictor $f$ is always chosen from the hypothesis class, i.e., $f \in \H$, otherwise it is said to be \emph{improper}. ERM  is an example of a proper learning algorithm, as are SRM algorithms of the form described above.  In the \emph{proper regime} of learning, algorithms are required to be proper. This article will be concerned with the more flexible \emph{improper regime} (a.k.a \emph{representation-independent learning}), where no such constraint is placed on the learner. In other words, all we care about is predictive power at test time, rather than any insights derived from the functional form or representation of the predictor~itself.


\subsection{The PAC Model}
A standard mathematical setup for evaluation of supervised learning algorithms, at least in the theoretical computer science community, is Valiant's \emph{Probably Approximately Correct (PAC) model} of learning (see e.g.~\cite{kearns_introduction_1994,mohri_foundations_2018}). Here, we assume there is an unknown distribution $\D$ on $\X \times \Y$ from which training and test data are  drawn.  Specifically, the labeled datapoints of the training set  $(x_1,y_1), \ldots, (x_n,y_n)$, as well as the test data  $(x_\tst,y_\tst)$, are i.i.d.~from $\D$. Often it is assumed that $\D$ lies in some class of distributions of interest. The \emph{true expected loss}, or simply \emph{loss}, of a predictor $f: \X \to \Y$ is the expected loss it incurs on draws from $\D$, written $L_\D(f) = \Ex_{(x,y) \sim \D} \ell(f(x),y)$.


There are two main ``settings'' in PAC learning. The  \emph{realizable setting} only requires that the data be perfectly explained by some hypothesis in $\H$. More generally, the \emph{agnostic setting} makes no assumption relating the data to the hypotheses, but shifts the goalposts as necessary to allow nontrivial guarantees: the expected loss at test time is evaluated only ``relative'' to that of the best hypothesis $h^* \in \H$. There are other settings which make more nuanced assumptions, such as $\D$ being of a particular parametric form or its support living in some (unknown) lower-dimensional space, etc. I will mostly discuss the realizable and agnostic settings in this article, those being the simplest and most studied from a theoretical perspective. %TODO:We will briefly discuss other settings in Section ??

The PAC model demands high probability guarantees of learners, in the worst case over distributions of interest. Consider first the realizable setting, where $\D$ is such that $\min_{h \in \H} L_{\D}(h) = 0$. A PAC learner has \emph{error} $\epsilon=\epsilon(n)$ and \emph{confidence} $\delta=\delta(n)$ if, when training data consists of $n$ i.i.d~samples from a realizable distribution $\D$, it produces a predictor $f$  satisfying $L_\D(f) \leq \epsilon$ with probability at least $1-\delta$. In the agnostic setting, where $\D$ can be arbitrary, we require $L_\D(f) - \min_{h \in \H} L_\D(h) \leq \epsilon$ with probability $1-\delta$.

In both the realizable and agnostic settings, we look for PAC learners with small $\epsilon$ and $\delta$ as a function of the sample size $n$. An equivalent perspective looks at the sample complexity $m(\epsilon,\delta)$, which is the minimum sample size which guarantees error  at most $\epsilon$ with probability at least $1-\delta$. We say a problem is \emph{PAC learnable} if its PAC sample complexity is finite whenever $\epsilon,\delta > 0$.

For most PAC learning problems, learnability and sample complexity are characterized in terms of a  ``dimension'' of the hypothesis class. Most prominently this is the \emph{VC dimension} for binary classification, the \emph{fat shattering dimension} for agnostic regression, and the \emph{DS dimension} for multiclass classification (see \cite{anthony_neural_1999,daniely_optimal_2014,brukhim_characterization_2022}). Treatment of these is beyond the scope of this article. The unfamiliar reader need not worry, however,  as dimensions will feature only tangentially in our~discussion.




%\paragraph{Learning settings: Realizable, Agnostic, etc.} In learning theory, evaluating a supervised learning algorithm requires specifying a data model and an objective. We will leave the details of the data model flexible for now, to allow for both the PAC model and the adversarial transductive model. Nonetheless we will describe two variations, which we call ``settings'', which cut across different models. The  \emph{realizable setting}  requires only that the data be perfectly explained by some hypothesis $h \in \H$ --- i.e., there exists a hypothesis which is guaranteed to suffer a loss of $0$ on training and test data. The performance of the learning algorithm is its expected loss at test time for some ``worst case'' realizable instance. More generally, the \emph{agnostic setting} makes no assumption relating the data to the hypotheses, but shifts the goalposts as necessary to allow nontrivial guarantees: the expected loss at test time is evaluated only ``relative'' to that of the best hypothesis $h^* \in \H$, again for some ``worst case'' instance. There are other settings which make more nuanced assumptions about the data, such as it is drawn from a distribution of a particular parametric form, or that it lives in some (unknown) lower-dimensional space, etc. We will mostly discuss the realizable and agnostic settings, those being the simplest and most studied from a theoretical perspective.




%%% Local Variables:
%%% mode: latex
%%% TeX-master: "learning_matching"
%%% End:

\iffalse
\begin{table*}[htbp]
\tiny
\begin{center}
\begin{tabular}{lccccccccccccc}\toprule
Model, ft setting & mem & \#param & ARC-c & ARC-e & BoolQ & HS & OBQA & PIQA & rte & SIQA & WG & Avg
%\\\cmidrule(lr){2-3}\cmidrule(lr){4-5} \cmidrule(lr){6-7} \cmidrule(lr){8-9}\cmidrule(lr){10-11} \cmidrule(lr){12-13} \cmidrule(lr){14-15} \cmidrule(lr){16-17} 
\\\cmidrule(lr){1-13}
Llama2(7B), LoRA, $r=64$ & 23.46GB & 159.9M(2.37\%) & \textbf{44.97} & 77.02 & 77.43 & \textbf{57.75} & 32.0 & \textbf{78.45} & 62.09 & \textbf{47.75} & 68.75 & 60.69\\
Llama2(7B), SPruFT, $r=128$ & \textbf{17.62GB} & 145.8M(2.16\%) & 43.60 & \textbf{77.26} & \textbf{77.77} & 57.47 & \textbf{32.6} & 78.07 & \textbf{64.98} & 46.67 & \textbf{69.30} & \textbf{60.86} \\\cmidrule(lr){2-13}
Llama2(7B), FA-LoRA, $r=64$ & 17.25GB & 92.8M(1.38\%) & 43.77 & \textbf{77.57} & 77.74 & \textbf{57.45} & 31.0 & 77.86 & \textbf{66.06} & \textbf{47.13} & 69.06 & 60.85\\
Llama2(7B), FA-SPruFT, $r=128$ & \textbf{15.21GB} & 78.6M(1.17\%) & \textbf{43.94} & 77.22 & \textbf{77.83} & 57.11 & \textbf{32.0} & \textbf{78.18} & 65.70 & 46.47 & \textbf{69.38} & \textbf{60.87}\\\midrule
Llama3(8B), LoRA, $r=64$ & 30.37GB & 167.8M(2.09\%) & \textbf{53.07} & \textbf{81.40} & \textbf{82.32} & \textbf{60.67} & 34.2 & \textbf{79.98} & 69.68 & \textbf{48.52} & \textbf{73.56} & \textbf{64.82}\\
Llama3(8B), SPruFT, $r=128$ & \textbf{24.49GB} & 159.4M(1.98\%) & 52.47 & 81.10 & 81.28 & 60.29 & \textbf{34.6} & 79.76 & \textbf{70.04} & 47.75 & 73.24 & 64.50 \\\cmidrule(lr){2-13}
Llama3(8B), FA-LoRA, $r=64$ & 24.55GB & 113.2M(1.41\%) & \textbf{52.47} & \textbf{81.36} & \textbf{82.23} & 60.17 & \textbf{35.0} & \textbf{79.76} & \textbf{70.04} & \textbf{48.31} & \textbf{73.56} & \textbf{64.77}\\
Llama3(8B), FA-SPruFT, $r=128$ & \textbf{22.41GB} & 92.3M(1.15\%) & 52.22 & 81.19 & 81.35 & \textbf{60.20} & 34.2 & 79.71 & 69.31 & 47.13 & 73.01 & 64.26 \\\bottomrule
\end{tabular}
%\vspace{-0.2cm}
\caption{Fine-tuning Llama on Alpaca dataset for 5 epochs and evaluating on 9 tasks from EleutherAI LM Harness. "mem" represents the memory usage, with further details provided in Appendix~\ref{apdx:measure}. \#param is the number of trainable parameters, where the difference of \#param between the two approaches depends on the architecture of Llama, as some layers have $d_{in} \neq d_{out}$. Note that 10 million trainable parameters only account for less than 0.15GB of memory requirement. FA indicates that we freeze attention layers, but not including MLP layers followed by attention blocks. HS, OBQA, and WG represent HellaSwag, OpenBookQA, and WinoGrande datasets. More details of datasets can be found in Appendix~\ref{apdx:data}. The ablation study for different $r$ and the comparison with other LoRA variants can be found in Appendix~\ref{apdx:ablation}. All reported results are accuracies on the corresponding tasks. \textbf{Bold} indicates the best results of two approaches on the same task.} \label{tab:llm} 
\end{center}
\end{table*}
\fi

\begin{table*}[htbp]
\tiny
\begin{center}
\begin{tabular}{lccccccccccccc}\toprule
Model, ft setting & mem & \#param & ARC-c & ARC-e & BoolQ & HS & OBQA & PIQA & rte & SIQA & WG & Avg
\\\cmidrule(lr){1-13}
Llama2(7B)\\ \cmidrule(lr){1-1} 
LoRA, $r=64$ & 23.46GB & 159.9M(2.37\%) & \textbf{44.97} & 77.02 & 77.43 & 57.75 & 32.0 & \textbf{78.45} & 62.09 & 47.75 & 68.75 & 60.69\\
VeRA, $r=64$ & 22.97GB & 1.374M(0.02\%) & 43.26 & 76.43 & 77.40 & 57.26 & 31.6 & 78.02 & 62.09 & 45.85 & 68.75 & 60.07\\
DoRA, $r=64$ & 44.85GB & 161.3M(2.39\%) & 44.71 & 77.02 & 77.55 & \textbf{57.79} & 32.4 & 78.29 & 61.73 & \textbf{47.90} & 68.98 & 60.71\\
RoSA, $r=32, d=1.2\%$ & 44.69GB & 157.7M(2.34\%) & 43.86 & \textbf{77.48} & \textbf{77.86} & 57.42 & 32.2 & 77.97 & 63.90 &  47.29 & 69.06 & 60.78\\
SPruFT, $r=128$ & \textbf{17.62GB} & 145.8M(2.16\%) & 43.60 & 77.26 & 77.77 & 57.47 & \textbf{32.6} & 78.07 & \textbf{64.98} & 46.67 & \textbf{69.30} & \textbf{60.86} %\\\cmidrule(lr){2-13}
%FA-LoRA, $r=64$ & 17.25GB & 92.8M(1.38\%) & 43.77 & \textbf{77.57} & 77.74 & \textbf{57.45} & 31.0 & 77.86 & 66.06 & \textbf{47.13} & 69.06 & 60.85\\
%FA-DoRA, $r=64$ & 30.61GB & 93.6M(1.39\%) & 43.94 & 77.44 & 77.49 & 57.44 & 31.0 & 77.86 & \textbf{66.43} & 46.98 & 69.14 & 60.86\\
%FA-RoSA, $r=32, d=1.2\%$ & 38.34GB & 98.3M(1.46\%) & \textbf{44.28} & 77.02 & 77.68 & 57.22 & 31.0 & 77.97 & 64.26 & 46.32 & 69.22 & 60.55\\
%FA-SPruFT, $r=128$ & \textbf{15.21GB} & 78.6M(1.17\%) & 43.94 & 77.22 & \textbf{77.83} & 57.11 & \textbf{32.0} & \textbf{78.18} & 65.70 & 46.47 & \textbf{69.38} & \textbf{60.87}
\\\midrule
Llama3(8B)\\ \cmidrule(lr){1-1} 
LoRA, $r=64$ & 30.37GB & 167.8M(2.09\%) & 53.07 & 81.40 & 82.32 & 60.67 & 34.2 & 79.98 & 69.68 & 48.52 & 73.56 & 64.82\\
VeRA, $r=64$ & 29.49GB & 1.391M(0.02\%) & 50.26 & 80.30 & 81.41 & 60.16 & 34.4 & 79.60 & 69.31 & 46.93 & 72.77 & 63.90\\
DoRA, $r=64$ & 51.45GB & 169.1M(2.11\%) & \textbf{53.33} & \textbf{81.57} & \textbf{82.45} & \textbf{60.71} & 34.2 & \textbf{80.09} & 69.31 & \textbf{48.67} & \textbf{73.64} & \textbf{64.88}\\
RoSA, $r=32, d=1.2\%$ & 48.40GB & 167.6M(2.09\%) & 51.28 & 81.27 & 81.80 & 60.18 & 34.4 & 79.87 & 69.31 & 47.95 & 73.16 & 64.36\\
SPruFT, $r=128$ & \textbf{24.49GB} & 159.4M(1.98\%) & 52.47 & 81.10 & 81.28 & 60.29 & \textbf{34.6} & 79.76 & \textbf{70.04} & 47.75 & 73.24 & 64.50 %\\\cmidrule(lr){2-13}
%FA-LoRA, $r=64$ & 24.55GB & 113.2M(1.41\%) & 52.47 & 81.36 & 82.23 & 60.17 & \textbf{35.0} & 79.76 & 70.04 & 48.31 & \textbf{73.56} & 64.77\\
%FA-DoRA, $r=64$ & 40.62GB & 114.3M(1.42\%) & \textbf{52.56} & \textbf{81.69} & \textbf{82.26} & \textbf{60.20} & 34.4 & \textbf{79.82} & \textbf{70.40} & \textbf{48.46} & 73.40 & \textbf{64.80}\\
%FA-RoSA, $r=32, d=1.2\%$ & 42.31GB & 124.3M(1.55\%) & 52.22 & 81.19 & 82.05 & 60.11 & 34.4 & 79.76 & 69.31 & 47.70 & 73.16 & 64.43\\
%FA-SPruFT, $r=128$ & \textbf{22.41GB} & 92.3M(1.15\%) & 52.22 & 81.19 & 81.35 & \textbf{60.20} & 34.2 & 79.71 & 69.31 & 47.13 & 73.01 & 64.26 
\\\bottomrule
\end{tabular}
%\vspace{-0.2cm}
\caption{Fine-tuning Llama on Alpaca dataset for 5 epochs and evaluating on 9 tasks from EleutherAI LM Harness. ``mem" represents the memory usage, with further details provided in Appendix~\ref{apdx:measure}. \#param is the number of trainable parameters, where the difference of \#param between the two approaches depends on the architecture of Llama, as some layers have $d_{in} \neq d_{out}$. %FA indicates that we freeze attention layers, but not including MLP layers followed by attention blocks. 
HS, OBQA, and WG represent HellaSwag, OpenBookQA, and WinoGrande datasets. %More details of datasets can be found in Appendix~\ref{apdx:data}. 
The ablation study for different $r$ can be found in Appendix~\ref{apdx:ranks}. All reported results are accuracies on the corresponding tasks. \textbf{Bold} indicates the best result on the same task. } \label{tab:llm} 
\end{center}
\end{table*}

\section{Experimental Setup}\label{sec:setup}

%(0.5 page)
%Why the chosen framework?
%Some prior approaches

%- parameter settings
%- uniform across layers vs greedy ... 
%- potential transformer-specific details

%Equations about what these methods do.. 

%(0.5 page)
%Which NN architectures are used, why?
%Number of parameters, layers, ...

%(Potential prior work on compression -- )

\subsection{Datasets} \label{subsec:dataset}
We use multiple datasets for different tasks. For image classification, we fine-tune models on the training split and evaluate it on the validation split of Tiny-ImageNet~\citep{tavanaei2020embedded}, CIFAR100~\citep{alex2009learning}, and Caltech101~\citep{li_andreeto_ranzato_perona_2022}. For text generation, we fine-tune LLMs on 256 samples from Stanford-Alpaca~\citep{alpaca} and assess zero-shot performance on nine EleutherAI LM Harness tasks~\citep{gao2021framework}. See Appendix~\ref{apdx:data} for details.

\subsection{Models and Baselines} \label{subsec:models}

We fine-tune full-precision Llama-2-7B and Llama-3-8B (float32) using our SPruFT, LoRA~\citep{hulora}, VeRA~\citep{kopiczko2024vera}, DoRA~\citep{liu2024dora}, and RoSA~\citep{nikdan2024rosa}. RoSA is chosen as the representative SFT method and is the only SFT due to the high memory demands of other SFT approaches, while full fine-tuning is excluded for the same reason. We freeze Llama’s classification layers and fine-tune only the linear layers in attention and MLP blocks.

Next, we evaluate importance metrics by fine-tuning Llamas and image models, including DeiT~\citep{touvron2021training}, ViT~\citep{dosovitskiy2020image}, ResNet101~\citep{he2016deep}, and ResNeXt101~\citep{xie2017aggregated} on CIFAR100, Caltech101, and Tiny-ImageNet. For image tasks, we set the fine-tuning ratio at 5\%, meaning the trainable parameters are a total of 5\% of the backbone plus classification layers.

\subsection{Training Details} \label{subsec:training}
Our fine-tuning framework is built on torch-pruning\footnote{Torch-pruning is not required, all their implementations are based on PyTorch.}~\citep{fang2023depgraph}, PyTorch~\citep{paszke2019pytorch}, PyTorch-Image-Models~\citep{rw2019timm}, and HuggingFace Transformers~\citep{wolf2020transformers}. Most experiments run on a single A100-80GB GPU, while DoRA and RoSA use an H100-96GB GPU. We use the Adam optimizer~\citep{KingBa15} and fine-tune all models for a fixed number of epochs without validation-based model selection.

%Structured pruning often considers parameter dependencies in importance evaluation~\citep{liu2021group, fang2023depgraph, ma2023llmpruner}. This becomes the following process in our work: first, searching for dependencies by tracing the computation graph of gradient; next, evaluating the importance of parameter groups; and finally, fine-tuning the parameters within those important groups collectively. For instance, if $\W^{a}_{\cdot j}$ and $\W^{b}_{i\cdot}$ are dependent, where $\W^{a}_{\cdot j}$ is the $j$-th column in parameter matrix (or the $j$-th input channels/features) of layer $a$ and $\W^{b}_{i\cdot}$ is the $i$-th row in parameter matrix (or the $i$-th output channels/features) of layer $b$, then $\W^{a}_{\cdot j}$ and $\W^{b}_{i\cdot}$ will be fine-tuned simultaneously while the corresponding $\M^{a}_{dep}$ for $\W^{a}_{\cdot j}$ becomes column selection matrix and $\W^a_s$ becomes $\W^a_{f,dep}\M^a_{dep}$. Consequently, fine-tuning $2.5\%$ output channels for layer $b$ will result in fine-tuning additional $2.5\%$ input channels in each dependent layer. Therefore, for the $5\%$ of desired fine-tuning ratio, the fine-tuning ratio with considering dependencies is set to $2.5\%$\footnote{In some complex models, considering dependencies results in slightly more than twice the number of trainable parameters. However, in most cases, the factor is 2.} for the approach that includes dependencies. More details for dependencies of NN can be found in Appendix~\ref{apdx:dep}. 

\textbf{Image models}: The learning rate is set to $10^{-4}$ with cosine annealing decay~\citep{loshchilov2017sgdr}, where the minimum learning rate is $10^{-9}$. All image models used in this study are pre-trained on ImageNet. 

\textbf{Llama}: For LoRA and DoRA, we set $\alpha = 16$, a dropout rate of $0.1$, and a learning rate of $10^{-4}$  with linear decay (
$0.01$ decay rate). For SPruFT, we control trainable parameters using rank instead of fine-tuning ratio for direct comparison. The learning rate is $2 \cdot 10^{-5}$ with the same decay settings. Linear decay is applied after a warmup over the first $3$\% of training steps. The maximum sequence length is $2048$, with truncation for longer inputs and padding for shorter ones.


\section{More Fine-grained Results}\label{app:results}

\subsection{Main Results}\label{app:main_results}

The complete results of the main experiments are provided in Table~\ref{tab:full-results}. Besides the effectiveness of our proposed framework, the results also reveal distinct behaviors of CoT and TIR across different datasets, highlighting their complementary strengths. CoT excels on in-domain tasks like GSM8K, where step-by-step reasoning is crucial, achieving higher accuracy (e.g., 55.6\% for Qwen2.5-0.5B) compared to TIR. However, TIR demonstrates superior performance on symbolic and computation-heavy tasks like MATH, leveraging tool integration to achieve higher accuracy (e.g., 67.5\% for Qwen2.5-7B). On out-of-domain datasets, CoT shows strong generalization on structured problems like MAWPS and SVAMP, while TIR performs better on complex tasks like OlympiadBench, where precise computations are essential. This divergence underscores the importance of combining both approaches, as \method~balances CoT's reasoning strength with TIR's computational precision, achieving robust performance across diverse datasets.

\begin{table*}[htbp!]
  \resizebox{\textwidth}{!}{
  \footnotesize
  \centering
    \begin{tabular}{@{}lllccccccccc@{}}
\toprule
\multicolumn{1}{c}{\multirow{2}{*}{Model}} & \multirow{2}{*}{Method} & \multirow{2}{*}{Metric} & \multicolumn{3}{c}{In-Domain} & \multicolumn{5}{c}{Out-of-Domain} & \multirow{2}{*}{AVG} \\ \cmidrule(lr){4-11}
\multicolumn{1}{c}{} &  &  & GSM8K & MATH & \multicolumn{1}{c|}{ID AVG} & MAWPS & SVAMP & College & Olympiad & \multicolumn{1}{c|}{OOD AVG} &  \\ \midrule
\multirow{9}{*}{Qwen2.5-0.5B} & \multirow{3}{*}{CoT} & Acc & \textbf{55.6} & 32.5 & \multicolumn{1}{c|}{44.0} & \textbf{86.2} & 58.9 & 26.3 & 6.7 & \multicolumn{1}{c|}{44.5} & 44.4 \\
 &  & Token & 305.4 & 526.8 & \multicolumn{1}{c|}{416.1} & 195.7 & 286.2 & 544.3 & 768.9 & \multicolumn{1}{c|}{448.8} & 437.9 \\
 &  & \# Code & 0.0 & 0.0 & \multicolumn{1}{c|}{0.0} & 0.0 & 0.0 & 0.0 & 0.0 & \multicolumn{1}{c|}{0.0} & 0.0 \\ \cmidrule(l){2-12} 
 & \multirow{3}{*}{TIR} & Acc & 46.9 & 36.4 & \multicolumn{1}{c|}{41.6} & 83.5 & 53.5 & 26.3 & 7.7 & \multicolumn{1}{c|}{42.8} & 42.4 \\
 &  & Token & 326.2 & 502.5 & \multicolumn{1}{c|}{414.4} & 268.6 & 292.3 & 492.9 & 745.8 & \multicolumn{1}{c|}{449.9} & 438.0 \\
 &  & \# Code & 3.16 & 2.72 & \multicolumn{1}{c|}{2.94} & 2.76 & 2.93 & 2.91 & 3.03 & \multicolumn{1}{c|}{2.91} & 2.92 \\ \cmidrule(l){2-12} 
 & \multirow{3}{*}{\method} & Acc & 52.8 & \textbf{36.6} & \multicolumn{1}{c|}{\textbf{44.7}} & 85.9 & \textbf{59.4} & \textbf{26.9} & \textbf{8.6} & \multicolumn{1}{c|}{\textbf{45.2}} & \textbf{45.0} \\
 &  & Token & 309.7 & 508.7 & \multicolumn{1}{c|}{409.2} & 217.3 & 292.9 & 500.9 & 743.0 & \multicolumn{1}{c|}{438.5} & 428.8 \\
 &  & \# Code & 0.19 & 2.63 & \multicolumn{1}{c|}{1.41} & 0.52 & 0.33 & 2.82 & 3.06 & \multicolumn{1}{c|}{1.68} & 1.59 \\ \midrule
\multirow{9}{*}{Qwen2.5-1.5B} & \multirow{3}{*}{CoT} & Acc & \textbf{78.3} & 48.2 & \multicolumn{1}{c|}{63.2} & 93.5 & \textbf{82.5} & \textbf{37.9} & 13.2 & \multicolumn{1}{c|}{56.8} & 58.9 \\
 &  & Token & 273.1 & 488.5 & \multicolumn{1}{c|}{380.8} & 185.9 & 233.9 & 498.9 & 684.0 & \multicolumn{1}{c|}{400.7} & 394.0 \\
 &  & \# Code & 0.0 & 0.0 & \multicolumn{1}{c|}{0.0} & 0.0 & 0.0 & 0.0 & 0.0 & \multicolumn{1}{c|}{0.0} & 0.0 \\ \cmidrule(l){2-12} 
 & \multirow{3}{*}{TIR} & Acc & 70.8 & \textbf{54.3} & \multicolumn{1}{c|}{62.6} & 91.7 & 79.8 & 36.0 & \textbf{19.4} & \multicolumn{1}{c|}{56.7} & 58.7 \\
 &  & Token & 324.0 & 474.5 & \multicolumn{1}{c|}{399.2} & 275.0 & 295.4 & 454.1 & 689.9 & \multicolumn{1}{c|}{428.6} & 418.8 \\
 &  & \# Code & 3.1 & 2.55 & \multicolumn{1}{c|}{2.82} & 2.83 & 2.94 & 2.66 & 2.96 & \multicolumn{1}{c|}{2.85} & 2.84 \\ \cmidrule(l){2-12} 
 & \multirow{3}{*}{\method} & Acc & 77.6 & 53.8 & \multicolumn{1}{c|}{\textbf{65.7}} & \textbf{94.2} & 80.7 & 37.0 & 18.8 & \multicolumn{1}{c|}{\textbf{57.7}} & \textbf{60.4} \\
 &  & Token & 273.1 & 480.5 & \multicolumn{1}{c|}{376.8} & 181.8 & 235.0 & 453.3 & 692.7 & \multicolumn{1}{c|}{390.7} & 386.1 \\
 &  & \# Code & 0.11 & 2.37 & \multicolumn{1}{c|}{1.24} & 0.14 & 0.21 & 2.38 & 2.84 & \multicolumn{1}{c|}{1.39} & 1.34 \\ \midrule
\multirow{9}{*}{Qwen2.5-3B} & \multirow{3}{*}{CoT} & Acc & \textbf{85.3} & 53.8 & \multicolumn{1}{c|}{69.6} & \textbf{94.8} & \textbf{85.8} & 41.5 & 16.3 & \multicolumn{1}{c|}{59.6} & 62.9 \\
 &  & Token & 259.8 & 469.2 & \multicolumn{1}{c|}{364.5} & 180.8 & 230.3 & 491.3 & 679.7 & \multicolumn{1}{c|}{395.5} & 385.2 \\
 &  & \# Code & 0.0 & 0.0 & \multicolumn{1}{c|}{0.0} & 0.0 & 0.0 & 0.0 & 0.0 & \multicolumn{1}{c|}{0.0} & 0.0 \\ \cmidrule(l){2-12} 
 & \multirow{3}{*}{TIR} & Acc & 80.4 & \textbf{61.4} & \multicolumn{1}{c|}{70.9} & 89.9 & 79.8 & 41.1 & 25.0 & \multicolumn{1}{c|}{59.0} & 62.9 \\
 &  & Token & 312.7 & 467.0 & \multicolumn{1}{c|}{389.8} & 283.3 & 293.8 & 434.4 & 676.5 & \multicolumn{1}{c|}{422.0} & 411.3 \\
 &  & \# Code & 3.04 & 2.57 & \multicolumn{1}{c|}{2.8} & 2.94 & 2.89 & 2.56 & 2.81 & \multicolumn{1}{c|}{2.8} & 2.8 \\ \cmidrule(l){2-12} 
 & \multirow{3}{*}{\method} & Acc & 84.0 & 61.3 & \multicolumn{1}{c|}{\textbf{72.6}} & 94.7 & 85.3 & \textbf{41.6} & \textbf{24.9} & \multicolumn{1}{c|}{\textbf{61.6}} & \textbf{65.3} \\
 &  & Token & 265.1 & 468.4 & \multicolumn{1}{c|}{366.8} & 209.3 & 237.7 & 436.4 & 683.5 & \multicolumn{1}{c|}{391.7} & 383.4 \\
 &  & \# Code & 0.07 & 2.44 & \multicolumn{1}{c|}{1.25} & 0.55 & 0.26 & 2.49 & 2.8 & \multicolumn{1}{c|}{1.52} & 1.43 \\ \midrule
\multirow{9}{*}{Qwen2.5-7B} & \multirow{3}{*}{CoT} & Acc & 88.7 & 58.6 & \multicolumn{1}{c|}{73.6} & \textbf{96.1} & \textbf{88.1} & 42.7 & 23.0 & \multicolumn{1}{c|}{62.5} & 66.2 \\
 &  & Token & 255.8 & 464.5 & \multicolumn{1}{c|}{360.2} & 177.7 & 217.9 & 480.0 & 673.5 & \multicolumn{1}{c|}{387.3} & 378.2 \\
 &  & \# Code & 0.0 & 0.0 & \multicolumn{1}{c|}{0.0} & 0.0 & 0.0 & 0.0 & 0.0 & \multicolumn{1}{c|}{0.0} & 0.0 \\ \cmidrule(l){2-12} 
 & \multirow{3}{*}{TIR} & Acc & 86.8 & \textbf{67.5} & \multicolumn{1}{c|}{77.2} & 92.1 & 84.3 & 44.2 & \textbf{31.9} & \multicolumn{1}{c|}{63.1} & 67.8 \\
 &  & Token & 306.4 & 452.3 & \multicolumn{1}{c|}{379.4} & 261.6 & 272.0 & 430.6 & 636.3 & \multicolumn{1}{c|}{400.1} & 393.2 \\
 &  & \# Code & 3.01 & 2.43 & \multicolumn{1}{c|}{2.72} & 2.58 & 2.54 & 2.52 & 2.68 & \multicolumn{1}{c|}{2.58} & 2.63 \\ \cmidrule(l){2-12} 
 & \multirow{3}{*}{\method} & Acc & \textbf{89.5} & 66.8 & \multicolumn{1}{c|}{\textbf{78.2}} & 94.2 & 86.2 & \textbf{43.4} & 31.1 & \multicolumn{1}{c|}{\textbf{63.7}} & \textbf{68.5} \\
 &  & Token & 258.1 & 452.8 & \multicolumn{1}{c|}{355.5} & 195.3 & 222.0 & 433.9 & 652.4 & \multicolumn{1}{c|}{375.9} & 369.1 \\
 &  & \# Code & 0.18 & 2.27 & \multicolumn{1}{c|}{1.23} & 0.5 & 0.29 & 2.49 & 2.66 & \multicolumn{1}{c|}{1.48} & 1.4 \\ \midrule
\multirow{9}{*}{LLaMA-3-8B} & \multirow{3}{*}{CoT} & Acc & \textbf{84.7} & 46.5 & \multicolumn{1}{c|}{65.6} & 91.6 & 81.6 & 30.2 & 13.3 & \multicolumn{1}{c|}{54.2} & 58.0 \\
 &  & Token & 246.4 & 471.0 & \multicolumn{1}{c|}{358.7} & 173.3 & 236.8 & 511.7 & 676.7 & \multicolumn{1}{c|}{399.6} & 386.0 \\
 &  & \# Code & 0.0 & 0.0 & \multicolumn{1}{c|}{0.0} & 0.0 & 0.0 & 0.0 & 0.0 & \multicolumn{1}{c|}{0.0} & 0.0 \\ \cmidrule(l){2-12} 
 & \multirow{3}{*}{TIR} & Acc & 81.7 & \textbf{56.2} & \multicolumn{1}{c|}{69.0} & 87.8 & 77.8 & 30.5 & \textbf{21.9} & \multicolumn{1}{c|}{54.5} & 59.3 \\
 &  & Token & 299.0 & 457.5 & \multicolumn{1}{c|}{378.2} & 240.9 & 269.1 & 437.9 & 650.8 & \multicolumn{1}{c|}{399.7} & 392.5 \\
 &  & \# Code & 2.96 & 2.51 & \multicolumn{1}{c|}{2.74} & 2.42 & 2.64 & 2.69 & 2.76 & \multicolumn{1}{c|}{2.63} & 2.66 \\ \cmidrule(l){2-12} 
 & \multirow{3}{*}{\method} & Acc & 84.0 & 55.1 & \multicolumn{1}{c|}{\textbf{69.6}} & \textbf{91.8} & \textbf{82.7} & \textbf{34.2} & 21.5 & \multicolumn{1}{c|}{\textbf{57.6}} & \textbf{61.5} \\
 &  & Token & 248.2 & 461.1 & \multicolumn{1}{c|}{354.6} & 191.1 & 222.5 & 449.5 & 657.7 & \multicolumn{1}{c|}{380.2} & 371.7 \\
 &  & \# Code & 0.12 & 2.33 & \multicolumn{1}{c|}{1.23} & 0.27 & 0.21 & 2.39 & 2.6 & \multicolumn{1}{c|}{1.37} & 1.32 \\ \midrule
\multirow{9}{*}{Qwen2.5Math-1.5B} & \multirow{3}{*}{CoT} & Acc & 83.1 & 56.6 & \multicolumn{1}{c|}{69.8} & 92.9 & 84.1 & \textbf{44.3} & 19.3 & \multicolumn{1}{c|}{60.2} & 63.4 \\
 &  & Token & 267.1 & 476.5 & \multicolumn{1}{c|}{371.8} & 181.7 & 238.3 & 485.7 & 681.7 & \multicolumn{1}{c|}{396.8} & 388.5 \\
 &  & \# Code & 0.0 & 0.0 & \multicolumn{1}{c|}{0.0} & 0.0 & 0.0 & 0.0 & 0.0 & \multicolumn{1}{c|}{0.0} & 0.0 \\ \cmidrule(l){2-12} 
 & \multirow{3}{*}{TIR} & Acc & 78.8 & 64.8 & \multicolumn{1}{c|}{71.8} & 92.3 & 83.3 & 42.4 & \textbf{27.4} & \multicolumn{1}{c|}{61.4} & 64.8 \\
 &  & Token & 332.2 & 516.6 & \multicolumn{1}{c|}{424.4} & 324.5 & 335.6 & 479.0 & 772.5 & \multicolumn{1}{c|}{477.9} & 460.1 \\
 &  & \# Code & 3.28 & 2.8 & \multicolumn{1}{c|}{3.04} & 3.62 & 3.55 & 2.99 & 3.14 & \multicolumn{1}{c|}{3.32} & 3.23 \\ \cmidrule(l){2-12} 
 & \multirow{3}{*}{\method} & Acc & 83.2 & 62.8 & \multicolumn{1}{c|}{\textbf{73.0}} & \textbf{94.0} & \textbf{85.6} & 43.9 & 26.8 & \multicolumn{1}{c|}{\textbf{62.6}} & \textbf{66.0} \\
 &  & Token & 266.2 & 508.9 & \multicolumn{1}{c|}{387.5} & 199.4 & 247.1 & 467.1 & 743.6 & \multicolumn{1}{c|}{414.3} & 405.4 \\
 &  & \# Code & 0.26 & 1.68 & \multicolumn{1}{c|}{0.97} & 0.32 & 0.42 & 1.44 & 2.38 & \multicolumn{1}{c|}{1.14} & 1.08 \\ \midrule
\multirow{9}{*}{Qwen2.5Math-7B} & \multirow{3}{*}{CoT} & Acc & \textbf{91.0} & 61.5 & \multicolumn{1}{c|}{76.2} & 94.8 & 87.9 & 45.7 & 23.9 & \multicolumn{1}{c|}{63.1} & 67.5 \\
 &  & Token & 254.7 & 470.6 & \multicolumn{1}{c|}{362.6} & 177.0 & 223.5 & 484.1 & 669.2 & \multicolumn{1}{c|}{388.5} & 379.9 \\
 &  & \# Code & 0.0 & 0.0 & \multicolumn{1}{c|}{0.0} & 0.0 & 0.0 & 0.0 & 0.01 & \multicolumn{1}{c|}{0.0} & 0.0 \\ \cmidrule(l){2-12} 
 & \multirow{3}{*}{TIR} & Acc & 88.9 & \textbf{73.6} & \multicolumn{1}{c|}{81.2} & \textbf{95.4} & \textbf{89.4} & 47.1 & 35.3 & \multicolumn{1}{c|}{66.8} & 71.6 \\
 &  & Token & 311.8 & 490.9 & \multicolumn{1}{c|}{401.4} & 261.2 & 272.2 & 456.8 & 713.7 & \multicolumn{1}{c|}{426.0} & 417.8 \\
 &  & \# Code & 3.04 & 2.56 & \multicolumn{1}{c|}{2.8} & 2.58 & 2.51 & 2.65 & 2.75 & \multicolumn{1}{c|}{2.62} & 2.68 \\ \cmidrule(l){2-12} 
 & \multirow{3}{*}{\method} & Acc & 89.8 & 73.0 & \multicolumn{1}{c|}{\textbf{81.4}} & 95.2 & 88.1 & \textbf{48.3} & \textbf{35.9} & \multicolumn{1}{c|}{\textbf{66.9}} & \textbf{71.7} \\
 &  & Token & 264.7 & 487.2 & \multicolumn{1}{c|}{376.0} & 193.7 & 229.7 & 476.9 & 710.6 & \multicolumn{1}{c|}{402.7} & 393.8 \\
 &  & \# Code & 0.25 & 2.14 & 1.2 & 0.33 & 0.24 & 2.02 & 2.59 & 1.3 & 1.26 \\ \bottomrule
\end{tabular}
  }
  \caption{Detailed results of our {\method} framework. The best accuracies within each group are shown in \textbf{bold}.
  % , while the second-best results are \underline{underlined}. 
  The three metrics, ``Acc'', ``Token'', and ``\# Code'' represent the average accuracy (\%), total tokens per generation, and number of code executions. 
  ``ID AVG'', ``OOD AVG'', and ``AVG'' denote the averages of these metrics across in-domain, out-of-domain, and all six benchmarks. 
  ``CoT'', ``TIR'', and ``{\method}'' indicate fine-tuning exclusively on CoT data, TIR data, and using our {\method} framework, respectively, with the corresponding base LLM.}
  \label{tab:full-results}
  %\vspace{-15pt}
\end{table*}


\subsection{Ablation Study}\label{app:ablation}

As detailed in Appendix~\ref{app:exp_setup}, we use different decision function $\mathcal{H}$ for GSM8K and MATH.
Specifically, for GSM8K, the dataset for supervised fine-tuning ($D_{\text{SFT}}$) is defined as:  
$$
D_{\text{SFT}} = \bigcup_{k=1}^N \{(x_k, y_k^j)\}_{j=1}^{M_k} \cup \bigcup_{k \in A} \{(x_k, z_k^j)\}_{j=1}^{M_k},
$$  
where the index set $A = \{k: S_{\text{CoT}}^k - S_{\text{TIR}}^k < \text{quantile}_1\}$.  

For MATH, $D_{\text{SFT}}$ is defined as:  
$$
D_{\text{SFT}} = \bigcup_{k=1}^N \{(x_k, z_k^j)\}_{j=1}^{M_k} \cup \bigcup_{k \in B} \{(x_k, y_k^j)\}_{j=1}^{M_k},
$$  
where the index set $B = \{k: S_{\text{CoT}}^k - S_{\text{TIR}}^k > \text{quantile}_2\}$.  
We consider this as the default choice of our {\method} (i.e., {\method} in Table~\ref{tabapp:ablation}).

We present the results of the $\mathcal{H}$ ablation study in Table~\ref{tabapp:ablation}. The variants of $\mathcal{H}$ evaluated are described as follows:

\paragraph{Random} 
The key difference between ``Random'' and ``{\method}'' lies in the selection of the index sets $A$ and $B$. 
In the ``Random'' variant, we randomly select the index sets $A$ and $B$ while ensuring that $|A|$ and $|B|$ match those in the default {\method} configuration. 
It is important to note that this is not purely a random selection, the number of queries using TIR or CoT is still determined by the default settings of {\method}, making ``Random'' a strong baseline.


\paragraph{CoT + TIR}
 In this variant, we include all CoT and TIR solutions in $D_{\text{SFT}}$, doubling the number of training examples compared to using only CoT or TIR individually. 
 Formally, the dataset is defined as:  
$$
D_{\text{SFT}} = \bigcup_{k=1}^N \{(x_k, y_k^j)\}_{j=1}^{M_k} \cup \bigcup_{k=1}^N \{(x_k, z_k^j)\}_{j=1}^{M_k}.
$$  


\paragraph{\method$^-$}
The {\method}$^-$ variant differs from the original {\method} in that it uses a single quantile for selection. 
The dataset is formally defined as:  
$$
D_{\text{SFT}} = \bigcup_{k \in A} \{(x_k, y_k^j)\}_{j=1}^{M_k} \cup \bigcup_{k \in B} \{(x_k, z_k^j)\}_{j=1}^{M_k},
$$  
where the index set $A = \{k: S_{\text{CoT}}^k - S_{\text{TIR}}^k > \text{quantile}\}$, and $B = A^c$.  
In this setup, each query in the candidate set {\dcandidatee} includes either CoT or TIR solutions but not both.


From Table~\ref{tabapp:ablation}, the selection function $\mathcal{H}$ in our {\method} gains the best results.


\begin{table*}[htbp!]
  \resizebox{\textwidth}{!}{
  \footnotesize
  \centering
    \begin{tabular}{@{}lllccccccccc@{}}
\toprule
\multicolumn{1}{c}{\multirow{2}{*}{Model}} & \multirow{2}{*}{Method} & \multirow{2}{*}{Metric} & \multicolumn{3}{c}{In-Domain} & \multicolumn{5}{c}{Out-of-Domain} & \multirow{2}{*}{AVG} \\ \cmidrule(lr){4-11}
\multicolumn{1}{c}{} &  &  & GSM8K & MATH & \multicolumn{1}{c|}{ID AVG} & MAWPS & SVAMP & College & Olympiad & \multicolumn{1}{c|}{OOD AVG} &  \\ \midrule
\multirow{15}{*}{LLaMA-3-8B} & \multirow{3}{*}{CoT} & Acc & \textbf{84.7} & 46.5 & \multicolumn{1}{c|}{65.6} & 91.6 & 81.6 & 30.2 & 13.3 & \multicolumn{1}{c|}{54.2} & 58.0 \\
 &  & Token & 246.4 & 471.0 & \multicolumn{1}{c|}{358.7} & 173.3 & 236.8 & 511.7 & 676.7 & \multicolumn{1}{c|}{399.6} & 386.0 \\
 &  & \# Code & 0.0 & 0.0 & \multicolumn{1}{c|}{0.0} & 0.0 & 0.0 & 0.0 & 0.0 & \multicolumn{1}{c|}{0.0} & 0.0 \\ \cmidrule(l){2-12} 
 & \multirow{3}{*}{TIR} & Acc & 81.7 & 56.2 & \multicolumn{1}{c|}{69.0} & 87.8 & 77.8 & 30.5 & \textbf{21.9} & \multicolumn{1}{c|}{54.5} & 59.3 \\
 &  & Token & 299.0 & 457.5 & \multicolumn{1}{c|}{378.2} & 240.9 & 269.1 & 437.9 & 650.8 & \multicolumn{1}{c|}{399.7} & 392.5 \\
 &  & \# Code & 2.96 & 2.51 & \multicolumn{1}{c|}{2.74} & 2.42 & 2.64 & 2.69 & 2.76 & \multicolumn{1}{c|}{2.63} & 2.66 \\ \cmidrule(l){2-12} 
 & \multirow{3}{*}{Random} & Acc & 83.1 & \textbf{56.4} & \multicolumn{1}{c|}{\textbf{69.8}} & 91.8 & 81.3 & 31.3 & 21.8 & \multicolumn{1}{c|}{56.6} & 61.0\\
 & & Token & 271.6 & 472.0 & \multicolumn{1}{c|}{371.8} & 203.7 & 251.0 & 453.4 & 695.5 & \multicolumn{1}{c|}{400.9} & 391.2\\
 & & \# Code & 0.21 & 2.35 & \multicolumn{1}{c|}{1.28} & 0.36 & 0.33 & 2.44 & 2.83 & \multicolumn{1}{c|}{1.49} & 1.42\\ \cmidrule(l){2-12} 
 & \multirow{3}{*}{CoT + TIR} & Acc & 83.1 & 48.4 & \multicolumn{1}{c|}{65.8} & 91.2 & 78.7 & 30.8 & 16.7 & \multicolumn{1}{c|}{54.4} & 58.2 \\
 &  & Token & 278.0 & 497.4 & \multicolumn{1}{c|}{387.7} & 208.6 & 281.2 & 507.3 & 707.3 & \multicolumn{1}{c|}{421.1} & 410.0 \\
 &  & \# Code & 0.83 & 0.51 & \multicolumn{1}{c|}{0.67} & 0.68 & 0.95 & 0.51 & 1.09 & \multicolumn{1}{c|}{0.81} & 0.76 \\ \cmidrule(l){2-12} 
 & \multirow{3}{*}{{\method}$^-$} & Acc & 83.1 & 54.7 & \multicolumn{1}{c|}{68.9} & 91.2 & 80.6 & 31.9 & 19.6 & \multicolumn{1}{c|}{55.8} & 60.2\\
 & & Token & 285.4 & 472.1 & \multicolumn{1}{c|}{378.8} & 226.7 & 253.9 & 474.3 & 692.2 & \multicolumn{1}{c|}{411.8} & 400.8\\
 & & \# Code & 1.4 & 2.31 & \multicolumn{1}{c|}{1.86} & 1.23 & 1.2 & 2.34 & 2.49 & \multicolumn{1}{c|}{1.81} & 1.83 \\ \cmidrule(l){2-12} 

 & \multirow{3}{*}{\method} & Acc & 84.0 & 55.1 & \multicolumn{1}{c|}{69.6} & \textbf{91.8} & \textbf{82.7} & \textbf{34.2} & 21.5 & \multicolumn{1}{c|}{\textbf{57.6}} & \textbf{61.5} \\
 &  & Token & 248.2 & 461.1 & \multicolumn{1}{c|}{354.6} & 191.1 & 222.5 & 449.5 & 657.7 & \multicolumn{1}{c|}{380.2} & 371.7 \\
 &  & \# Code & 0.12 & 2.33 & \multicolumn{1}{c|}{1.23} & 0.27 & 0.21 & 2.39 & 2.6 & \multicolumn{1}{c|}{1.37} & 1.32 \\ 
\bottomrule
\end{tabular}
  }
  \caption{Ablation Study using LLaMA-3-8B. The best accuracies within each group are shown in \textbf{bold}.
  The three metrics, ``Acc'', ``Token'', and ``\# Code'' represent the average accuracy, total tokens per generation, and number of code executions. 
  ``Acc'' is reported in \%. ``ID AVG'', ``OOD AVG'', and ``AVG'' denote the averages of these metrics across in-domain, out-of-domain, and all six benchmarks.}
  \label{tabapp:ablation}
\end{table*}




\subsection{Analysis of CoT scores and TIR scores}\label{app:scores}

% different base LLMs -> different tendency
% Math LLM has high scores -> 
% qwenmath7b math scores

In Section~\ref{sec:ana_scores}, we presented representative results analyzing CoT and TIR scores. 
Here, we further provide the distributions of {\scote}, {\stire}, and ($S_{\text{CoT}}^k - S_{\text{TIR}}^k$) for various base LLMs in Figures~\ref{fig:llama3-scores}, \ref{fig:qwen05-scores}, \ref{fig:qwen15-scores}, \ref{fig:qwen3-scores}, \ref{fig:qwen7-scores}, \ref{fig:qwenmath15-scores}, and \ref{fig:qwenmath7-scores}.
From these figures, we have the following observations:  
1. Different base LLMs exhibit varying tendencies towards CoT or TIR responding to the same candidate set queries.  
2. Math-specialized LLMs (e.g., Qwen2.5Math) demonstrate higher CoT and TIR scores compared to their general-purpose counterparts (e.g., Qwen2.5). 
This may be attributed to the inclusion of similar CoT and TIR data in their pretraining process.  
3. Notably, Qwen2.5Math-7B achieves TIR scores approaching 0.8 accuracy on the MATH anchor set using only a 1-shot prompt from the candidate set, as shown in Figure~\ref{fig:qwenmath7-scores} (middle). 
This suggests the potential for anchor set contamination~\citep{benbench2024xu}.  


\begin{figure*}
    \centering
    \includegraphics[width=0.32\linewidth]{figures/llama3-gsm8k.pdf}
    \includegraphics[width=0.32\linewidth]{figures/llama3-math.pdf}
    \includegraphics[width=0.32\linewidth]{figures/llama3-cot-tir.pdf}
    \caption{The distribution of {\scote} (left), {\stire} (middle), and ($S_{\text{CoT}}^k - S_{\text{TIR}}^k$) (right) for LLaMA-3-8B.}
    \label{fig:llama3-scores}
\end{figure*}

\begin{figure*}
    \centering
    \includegraphics[width=0.32\linewidth]{figures/qwen0_5b-gsm8k.pdf}
    \includegraphics[width=0.32\linewidth]{figures/qwen0_5b-math.pdf}
    \includegraphics[width=0.32\linewidth]{figures/qwen0_5b-cot-tir.pdf}
    \caption{The distribution of {\scote} (left), {\stire} (middle), and ($S_{\text{CoT}}^k - S_{\text{TIR}}^k$) (right) for Qwen2.5-0.5B.}
    \label{fig:qwen05-scores}
\end{figure*}

\begin{figure*}
    \centering
    \includegraphics[width=0.32\linewidth]{figures/qwen1_5b-gsm8k.pdf}
    \includegraphics[width=0.32\linewidth]{figures/qwen1_5b-math.pdf}
    \includegraphics[width=0.32\linewidth]{figures/qwen1_5b-cot-tir.pdf}
    \caption{The distribution of {\scote} (left), {\stire} (middle), and ($S_{\text{CoT}}^k - S_{\text{TIR}}^k$) (right) for Qwen2.5-1.5B.}
    \label{fig:qwen15-scores}
\end{figure*}

\begin{figure*}
    \centering
    \includegraphics[width=0.32\linewidth]{figures/qwen3b-gsm8k.pdf}
    \includegraphics[width=0.32\linewidth]{figures/qwen3b-math.pdf}
    \includegraphics[width=0.32\linewidth]{figures/qwen3b-cot-tir.pdf}
    \caption{The distribution of {\scote} (left), {\stire} (middle), and ($S_{\text{CoT}}^k - S_{\text{TIR}}^k$) (right) for Qwen2.5-3B.}
    \label{fig:qwen3-scores}
\end{figure*}

\begin{figure*}
    \centering
    \includegraphics[width=0.32\linewidth]{figures/qwen7b-gsm8k.pdf}
    \includegraphics[width=0.32\linewidth]{figures/qwen7b-math.pdf}
    \includegraphics[width=0.32\linewidth]{figures/qwen7b-cot-tir.pdf}
    \caption{The distribution of {\scote} (left), {\stire} (middle), and ($S_{\text{CoT}}^k - S_{\text{TIR}}^k$) (right) for Qwen2.5-7B.}
    \label{fig:qwen7-scores}
\end{figure*}

\begin{figure*}
    \centering
    \includegraphics[width=0.32\linewidth]{figures/qwenmath1_5b-gsm8k.pdf}
    \includegraphics[width=0.32\linewidth]{figures/qwenmath1_5b-math.pdf}
    \includegraphics[width=0.32\linewidth]{figures/qwenmath1_5b-cot-tir.pdf}
    \caption{The distribution of {\scote} (left), {\stire} (middle), and ($S_{\text{CoT}}^k - S_{\text{TIR}}^k$) (right) for Qwen2.5Math-1.5B.}
    \label{fig:qwenmath15-scores}
\end{figure*}

\begin{figure*}
    \centering
    \includegraphics[width=0.32\linewidth]{figures/qwenmath7b-gsm8k.pdf}
    \includegraphics[width=0.32\linewidth]{figures/qwenmath7b-math.pdf}
    \includegraphics[width=0.32\linewidth]{figures/qwenmath7b-cot-tir.pdf}
    \caption{The distribution of {\scote} (left), {\stire} (middle), and ($S_{\text{CoT}}^k - S_{\text{TIR}}^k$) (right) for Qwen2.5Math-7B.}
    \label{fig:qwenmath7-scores}
\end{figure*}




\subsection{Transferability Results}\label{app:transfer}

The complete results of transferability results are given in Table~\ref{tabapp:transfer_results}.

\begin{table*}[htbp!]
  \resizebox{\textwidth}{!}{
  \footnotesize
  \centering
    \begin{tabular}{@{}lllccccccccc@{}}
\toprule
\multicolumn{1}{c}{\multirow{2}{*}{Model}} & \multirow{2}{*}{Select By} & \multirow{2}{*}{Metric} & \multicolumn{3}{c}{In-Domain} & \multicolumn{5}{c}{Out-of-Domain} & \multirow{2}{*}{AVG} \\ \cmidrule(lr){4-11}
\multicolumn{1}{c}{} &  &  & GSM8K & MATH & \multicolumn{1}{c|}{ID AVG} & MAWPS & SVAMP & College & Olympiad & \multicolumn{1}{c|}{OOD AVG} &  \\ \midrule
\multirow{9}{*}{Qwen2.5-0.5B}
 & \multirow{3}{*}{Qwen2.5-0.5B} & Acc & \textbf{52.8} & 36.6 & \multicolumn{1}{c|}{\textbf{44.7}} & 85.9 & \textbf{59.4} & \textbf{26.9} & \textbf{8.6} & \multicolumn{1}{c|}{\textbf{45.2}} & \textbf{45.0} \\
 &  & Token & 309.7 & 508.7 & \multicolumn{1}{c|}{409.2} & 217.3 & 292.9 & 500.9 & 743.0 & \multicolumn{1}{c|}{438.5} & 428.8 \\
 &  & \# Code & 0.19 & 2.63 & \multicolumn{1}{c|}{1.41} & 0.52 & 0.33 & 2.82 & 3.06 & \multicolumn{1}{c|}{1.68} & 1.59 \\ \cmidrule(l){2-12}
 & \multirow{3}{*}{LLaMA-3-8B} & Acc & 51.3 & 36.3 & \multicolumn{1}{c|}{43.8} & 86.2 & 55.9 & 26.5 & 8.1 & \multicolumn{1}{c|}{44.2} & 44.1 \\
 &  & Token & 318.2 & 507.7 & \multicolumn{1}{c|}{413.0} & 216.9 & 298.9 & 485.4 & 732.8 & \multicolumn{1}{c|}{433.5} & 426.6 \\
 &  & \# Code & 0.28 & 2.49 & \multicolumn{1}{c|}{1.39} & 0.52 & 0.52 & 2.45 & 2.73 & \multicolumn{1}{c|}{1.56} & 1.5 \\ \cmidrule(l){2-12}
 & \multirow{3}{*}{Qwen2.5-7B} & Acc & 52.2 & 36.8 & \multicolumn{1}{c|}{44.5} & \textbf{86.7} & 57.6 & 26.7 & 7.4 & \multicolumn{1}{c|}{44.6} & \textbf{44.6} \\
 &  & Token & 312.5 & 499.4 & \multicolumn{1}{c|}{406.0} & 228.6 & 308.2 & 489.3 & 744.5 & \multicolumn{1}{c|}{442.6} & 430.4 \\
 &  & \# Code & 0.4 & 2.53 & \multicolumn{1}{c|}{1.46} & 0.85 & 0.68 & 2.75 & 2.94 & \multicolumn{1}{c|}{1.81} & 1.69 \\ 
\bottomrule
\end{tabular}
  }
  \caption{Detailed results of transferability experiments using Qwen2.5-0.5B. The best accuracies within each group are shown in \textbf{bold}.
  The three metrics, ``Acc'', ``Token'', and ``\# Code'' represent the average accuracy, total tokens per generation, and number of code executions. 
  ``Acc'' is reported in \%. ``ID AVG'', ``OOD AVG'', and ``AVG'' denote the averages of these metrics across in-domain, out-of-domain, and all six benchmarks.}
  \label{tabapp:transfer_results}
\end{table*}




\subsection{DPO Results}\label{app:rl}

The detailed settings of DPO are as follows:

\paragraph{Preference Data Construction}
The construction of the preference dataset used in DPO is guided by CoT and TIR scores, following a similar approach to the construction of {\dsft}. 
Specifically, two separate quantiles are used to select preference pairs for the GSM8K and MATH datasets.
The preference dataset, $\mathcal{D}_{\text{pre}}$, is selected from the newly defined candidate set, {\dcandidatee}, and is formally defined as:  
$$
\mathcal{D}_{\text{pre}} = \{(x_k, c_k, r_k)\}_{k \in A},
$$  
where $c_k$ is the \textbf{c}hosen (preferred) response for the query $x_k$, and $r_k$ is the \textbf{r}ejected response.  

The index set $A$ is defined as:  
\begin{align*}
    A = \{k: S_{\text{TIR}}^k - S_{\text{CoT}}^k &< \text{quantile}^{'}_1 \quad \text{or} \\ 
    &S_{\text{CoT}}^k - S_{\text{TIR}}^k > \text{quantile}^{'}_2\}, 
\end{align*}  
where $\text{quantile}^{'}_1$ and $\text{quantile}^{'}_2$ are two quantiles optimized via grid search.

The rules for determining $c_k$ (chosen response) and $r_k$ (rejected response) are as follows:  
$$
c_k = 
\begin{cases} 
y_k & \text{if } S_{\text{CoT}}^k - S_{\text{TIR}}^k > \text{quantile}^{'}_2, \\
z_k & \text{if } S_{\text{TIR}}^k - S_{\text{CoT}}^k < \text{quantile}^{'}_1,
\end{cases}
$$
and  
$$
r_k = 
\begin{cases} 
y_k & \text{if } S_{\text{TIR}}^k - S_{\text{CoT}}^k < \text{quantile}^{'}_1, \\
z_k & \text{if } S_{\text{CoT}}^k - S_{\text{TIR}}^k > \text{quantile}^{'}_2.
\end{cases}
$$  
This preference selection process ensures that the dataset $\mathcal{D}_{\text{pre}}$ contains meaningful comparisons between CoT and TIR responses based on their relative scores.



\paragraph{DPO Hyperparameters}
We utilize \href{https://github.com/OpenRLHF/OpenRLHF}{OpenRLHF} \citep{hu2024openrlhf} to implement DPO. 
The maximum token length is set to 4,096, consistent with the SFT stage. 
The training process adopts a learning rate of $5 \times 10^{-7}$, a batch size of 256, and runs for one epoch.
We use LLaMA-3-8B and Qwen2.5Math-7B, fine-tuned with {\method}, as the starting point for DPO.


The complete results are presented in Table~\ref{tabapp:dpo_results}. 
As shown, DPO achieves comparable results with LLMs fine-tuned with {\method}.

\begin{table*}[htbp!]
  \resizebox{\textwidth}{!}{
  \footnotesize
  \centering
    \begin{tabular}{@{}lllccccccccc@{}}
\toprule
\multicolumn{1}{c}{\multirow{2}{*}{Model}} & \multirow{2}{*}{Method} & \multirow{2}{*}{Metric} & \multicolumn{3}{c}{In-Domain} & \multicolumn{5}{c}{Out-of-Domain} & \multirow{2}{*}{AVG} \\ \cmidrule(lr){4-11}
\multicolumn{1}{c}{} &  &  & GSM8K & MATH & \multicolumn{1}{c|}{ID AVG} & MAWPS & SVAMP & College & Olympiad & \multicolumn{1}{c|}{OOD AVG} &  \\ \midrule
\multirow{12}{*}{LLaMA-3-8B} & \multirow{3}{*}{CoT} & Acc & \textbf{84.7} & 46.5 & \multicolumn{1}{c|}{65.6} & 91.6 & 81.6 & 30.2 & 13.3 & \multicolumn{1}{c|}{54.2} & 58.0 \\
 &  & Token & 246.4 & 471.0 & \multicolumn{1}{c|}{358.7} & 173.3 & 236.8 & 511.7 & 676.7 & \multicolumn{1}{c|}{399.6} & 386.0 \\
 &  & \# Code & 0.0 & 0.0 & \multicolumn{1}{c|}{0.0} & 0.0 & 0.0 & 0.0 & 0.0 & \multicolumn{1}{c|}{0.0} & 0.0 \\ \cmidrule(l){2-12} 
 & \multirow{3}{*}{TIR} & Acc & 81.7 & \textbf{56.2} & \multicolumn{1}{c|}{69.0} & 87.8 & 77.8 & 30.5 & \textbf{21.9} & \multicolumn{1}{c|}{54.5} & 59.3 \\
 &  & Token & 299.0 & 457.5 & \multicolumn{1}{c|}{378.2} & 240.9 & 269.1 & 437.9 & 650.8 & \multicolumn{1}{c|}{399.7} & 392.5 \\
 &  & \# Code & 2.96 & 2.51 & \multicolumn{1}{c|}{2.74} & 2.42 & 2.64 & 2.69 & 2.76 & \multicolumn{1}{c|}{2.63} & 2.66 \\ \cmidrule(l){2-12} 
 & \multirow{3}{*}{\method} & Acc & 84.0 & 55.1 & \multicolumn{1}{c|}{\textbf{69.6}} & \textbf{91.8} & \textbf{82.7} & \textbf{34.2} & 21.5 & \multicolumn{1}{c|}{\textbf{57.6}} & 61.5 \\
 &  & Token & 248.2 & 461.1 & \multicolumn{1}{c|}{354.6} & 191.1 & 222.5 & 449.5 & 657.7 & \multicolumn{1}{c|}{380.2} & 371.7 \\
 &  & \# Code & 0.12 & 2.33 & \multicolumn{1}{c|}{1.23} & 0.27 & 0.21 & 2.39 & 2.6 & \multicolumn{1}{c|}{1.37} & 1.32 \\ \cmidrule(l){2-12} 
 & \multirow{3}{*}{+DPO} & Acc & 84.0 & 55.2 & \multicolumn{1}{c|}{\textbf{69.6}} & \textbf{91.8} & \textbf{82.7} & 34.0 & 21.8 & \multicolumn{1}{c|}{\textbf{57.6}} & \textbf{61.6} \\
 &  & Token & 250.8 & \textbf{453.6} & \multicolumn{1}{c|}{\textbf{352.2}} & \textbf{185.0} & \textbf{219.1} & \textbf{435.9} & \textbf{647.9} & \multicolumn{1}{c|}{372.0} & \textbf{365.4} \\
 &  & \# Code & 0.14 & 2.38 & \multicolumn{1}{c|}{1.26} & 0.25 & 0.17 & 2.42 & 2.7 & \multicolumn{1}{c|}{1.38} & 1.34 \\
 \midrule
\multirow{12}{*}{Qwen2.5Math-7B} & \multirow{3}{*}{CoT} & Acc & \textbf{91.0} & 61.5 & \multicolumn{1}{c|}{76.2} & 94.8 & 87.9 & 45.7 & 23.9 & \multicolumn{1}{c|}{63.1} & 67.5 \\
 &  & Token & \textbf{254.7} & 470.6 & \multicolumn{1}{c|}{362.6} & \textbf{177.0} & \textbf{223.5} & 484.1 & 669.2 & \multicolumn{1}{c|}{388.5} & 379.9 \\
 &  & \# Code & 0.0 & 0.0 & \multicolumn{1}{c|}{0.0} & 0.0 & 0.0 & 0.0 & 0.01 & \multicolumn{1}{c|}{0.0} & 0.0 \\ \cmidrule(l){2-12} 
 & \multirow{3}{*}{TIR} & Acc & 88.9 & \textbf{73.6} & \multicolumn{1}{c|}{81.2} & \textbf{95.4} & \textbf{89.4} & 47.1 & 35.3 & \multicolumn{1}{c|}{66.8} & 71.6 \\
 &  & Token & 311.8 & 490.9 & \multicolumn{1}{c|}{401.4} & 261.2 & 272.2 & \textbf{456.8} & 713.7 & \multicolumn{1}{c|}{426.0} & 417.8 \\
 &  & \# Code & 3.04 & 2.56 & \multicolumn{1}{c|}{2.8} & 2.58 & 2.51 & 2.65 & 2.75 & \multicolumn{1}{c|}{2.62} & 2.68 \\ \cmidrule(l){2-12} 
 & \multirow{3}{*}{\method} & Acc & 89.8 & 73.0 & \multicolumn{1}{c|}{\textbf{81.4}} & 95.2 & 88.1 & \textbf{48.3} & \textbf{35.9} & \multicolumn{1}{c|}{\textbf{66.9}} & \textbf{71.7} \\
 &  & Token & 264.7 & 487.2 & \multicolumn{1}{c|}{376.0} & 193.7 & 229.7 & 476.9 & 710.6 & \multicolumn{1}{c|}{402.7} & 393.8 \\
 &  & \# Code & 0.25 & 2.14 & 1.2 & 0.33 & 0.24 & 2.02 & 2.59 & 1.3 & 1.26 \\  \cmidrule(l){2-12} 
 & \multirow{3}{*}{+DPO} & Acc & 89.8 & 73.1 & \multicolumn{1}{c|}{\textbf{81.4}} & 95.2 & 88.1 & \textbf{48.4} & 35.4 & \multicolumn{1}{c|}{66.8} & \textbf{71.7} \\
 &  & Token & 267.0 & \textbf{487.2} & \multicolumn{1}{c|}{377.1} & 193.8 & 229.4 & 474.8 & 718.9 & \multicolumn{1}{c|}{404.2} & 395.2 \\
 &  & \# Code & 0.3 & 2.18 & \multicolumn{1}{c|}{1.24} & 0.39 & 0.27 & 2.08 & 2.67 & \multicolumn{1}{c|}{1.35} & 1.32 \\
\bottomrule
\end{tabular}
  }
  \caption{Detailed DPO results. The best accuracies within each group are shown in \textbf{bold}.
  The three metrics, ``Acc'', ``Token'', and ``\# Code'' represent the average accuracy, total tokens per generation, and number of code executions. 
  ``Acc'' is reported in \%. ``ID AVG'', ``OOD AVG'', and ``AVG'' denote the averages of these metrics across in-domain, out-of-domain, and all six benchmarks.}
  \label{tabapp:dpo_results}
\end{table*}

\end{document}
