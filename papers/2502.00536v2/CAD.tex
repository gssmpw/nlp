\documentclass[conference]{IEEEtran}
\IEEEoverridecommandlockouts
% The preceding line is only needed to identify funding in the first footnote. If that is unneeded, please comment it out.
\usepackage{cite}
\usepackage{amsmath,amssymb,amsfonts}
\usepackage{algorithmic}
\usepackage{graphicx}
\usepackage{textcomp}
\usepackage{xcolor}
\usepackage{algorithm}

\bibliographystyle{IEEEtran}




% \usepackage{algorithmic}


\def\BibTeX{{\rm B\kern-.05em{\sc i\kern-.025em b}\kern-.08em
    T\kern-.1667em\lower.7ex\hbox{E}\kern-.125emX}}
\begin{document}

\title{CAD: Confidence-Aware Adaptive Displacement for Semi-Supervised Medical Image Segmentation\\

\thanks{This work was funded by the Scientific Research Key project of Education Office of Hunan Province, China [24A0176] \& the National Key Research and Development Program [2021YFD1300404].}
}
% \author{\IEEEauthorblockN{Anonymous Authors}}
% \author{\IEEEauthorblockN{Anonymous Authors}

\author{
    \IEEEauthorblockN{Wenbo Xiao$^{a,b}$, Zhihao Xu$^{a}$, Guiping Liang$^{a}$, Yangjun Deng$^{a}$, Yi Xiao$^{a*}$\thanks{* Yi Xiao is corresponding author.}}
    \IEEEauthorblockA{$^a$College of Information and Intelligence, Hunan Agricultural University, Changsha, China}
    
    \IEEEauthorblockA{$^b$School of Computer Science and Engineering, University of New South Wales, Sydney, Australia}
    \IEEEauthorblockA{\{wenboxiao@stu., guapideyouxiang@stu., guipingliang@stu., dengyangjun@, xiaoyi@\}hunau.edu.cn}
}
\maketitle

% \author{\IEEEauthorblockN{Wenbo Xiao}
% \IEEEauthorblockA{\textit{College of Information and Intelligence} \\
% \textit{Hunan Agricultural University}\\
% Changsha, China \\
% wenboxiao@stu.hunau.edu.cn}
% \and
% \IEEEauthorblockN{Zhihao Xu}
% \IEEEauthorblockA{\textit{College of Information and Intelligence} \\
% \textit{Hunan Agricultural University}\\
% Changsha, China \\
% guapideyouxiang@stu.hunau.edu.cn}
% \and
% \IEEEauthorblockN{Guiping Liang}
% \IEEEauthorblockA{\textit{College of Information and Intelligence} \\
% \textit{Hunan Agricultural University}\\
% Changsha, China \\
% guipingLiang@stu.hunau.edu.cn}
% \and
% \IEEEauthorblockN{Yangjun Deng}
% \IEEEauthorblockA{\textit{College of Information and Intelligence} \\
% \textit{Hunan Agricultural University}\\
% Changsha, China \\
% dyj2012@yeah.net}
% \and
% \IEEEauthorblockN{Yi Xiao}
% \IEEEauthorblockA{\textit{College of Information and Intelligence} \\
% \textit{Hunan Agricultural University}\\
% Changsha, China \\
% xiaoyi@hunau.edu.cn}
% }



\begin{abstract}
Semi-supervised medical image segmentation aims to leverage minimal expert annotations, yet remains confronted by challenges in maintaining high-quality consistency learning. Excessive perturbations can degrade alignment and hinder precise decision boundaries, especially in regions with uncertain predictions. In this paper, we introduce Confidence-Aware Adaptive Displacement (CAD), a framework that selectively identifies and replaces the largest low-confidence regions with high-confidence patches. By dynamically adjusting both the maximum allowable replacement size and the confidence threshold throughout training, CAD progressively refines the segmentation quality without overwhelming the learning process. Experimental results on public medical datasets demonstrate that CAD effectively enhances segmentation quality, establishing new state-of-the-art accuracy in this field.
\end{abstract}

\begin{IEEEkeywords}
semi-supervised learning, medical image segmentation, consistency learning
\end{IEEEkeywords}

\section{Introduction}

Medical image segmentation is an essential task in many clinical applications, such as disease diagnosis, treatment planning, and surgical guidance. Accurate segmentation allows clinicians to delineate structures such as organs and tumors from imaging modalities like computer tomography (CT) and magnetic resonance imaging (MRI) scans, which is crucial for providing reliable volumetric and shape information\cite{b1}. However, obtaining large datasets with precise annotations is challenging and costly, particularly in the medical imaging domain, where only experts can provide reliable annotations. This creates a significant barrier to training robust deep learning models for segmentation tasks, especially in medical applications where high-quality labeled data is sparse.

In medical image segmentation, semi-supervised medical image segmentation (SSMIS) methods have emerged to enhance model performance by utilizing a small amount of labeled data and a large amount of unlabeled data\cite{b3,b4}. This approach alleviates the challenge of limited labeled data and effectively leverages the rich information contained in the unlabeled dataset. In recent years, as this field has advanced, the performance of SSMIS has increasingly approached that of fully supervised segmentation methods, particularly under conditions with minimal labeled data, showing remarkable segmentation results\cite{b5,b6}. This progress has been driven by several techniques, with consistency learning being one of the core methods in semi-supervised learning\cite{b7,b8,b9}, which facilitates the effective use of unlabeled data by enforcing consistent predictions across different augmentations of the input data, utilizing pseudo-labels and data augmentation strategies to generate high-quality labels\cite{b10}. Common techniques in consistency learning include self-training and consistency loss, which minimize the discrepancy between model outputs under different augmentations, making the model more stable and reliable when processing unlabeled data\cite{b11,b12}. Within consistency learning, perturbations such as data augmentations and feature map perturbations play a pivotal role\cite{b12}, ensuring that the model learns robust features by making predictions consistent despite variations in the input data. Data augmentations, including rotation, scaling, and flipping, challenge the model, forcing it to focus on stable and generalizable features while ignoring spurious ones\cite{b13,b14}; meanwhile, feature map perturbations, such as adding noise or applying dropout, directly alter the feature representations, encouraging the model to maintain consistency even in the presence of noisy or incomplete information\cite{b10,b16}. Together, these perturbations help the model refine its internal representations, ultimately enhancing its performance on unlabeled data and improving segmentation accuracy in semi-supervised settings.

In SSMIS, a key challenge is the balance between applying sufficient perturbations and avoiding excessive ones. Too strong perturbations can increase uncertainty in the model's predictions, which undermines its ability to learn stable representations\cite{b33}, especially in scenarios with limited labeled data. Additionally, in continuous learning settings, the model's ability to extract meaningful semantic information is constrained by the mismatch between labeled and unlabeled data distributions. The small labeled set often does not fully capture the variations in the unlabeled data, limiting the model’s generalization capacity and potentially leading to error propagation over time.

\begin{figure}[htbp]
\centerline{\includegraphics[width=250pt]{1.png}}
\caption{Visualization of replacement  regions in BCP, ABD, and CAD. Subfigure (a) shows BCP, which swaps foreground and background between labeled and unlabeled data across the entire image. Subfigure (b) illustrates ABD, replacing low-confidence patches with high-confidence patches between strong and weak augmentations. Subfigure (c) demonstrates the proposed CAD, progressively refining replacement regions from small (early stage) to large (late stage) based on confidence levels and training dynamics.}
\label{fig1}
\end{figure}


Recent research has made significant strides in addressing these challenges. As shown in Fig.~\ref{fig1}(a), Bidirectional Copy-Paste (BCP)\cite{b3} tackles the distribution mismatch by swapping foreground and background patches between labeled and unlabeled images. This strategy encourages the model to learn from mixed pseudo-labels, enabling better generalization. Adaptive Bidirectional Displacement (ABD)\cite{b5}, depicted in Fig.~\ref{fig1}(b), replaces low-confidence regions in weakly augmented images with high-confidence regions from strongly augmented counterparts and vice versa, thus refining consistency learning by targeted perturbations. However, both approaches have limitations. The reliance on mixed pseudo-labels in BCP can propagate errors from unreliable predictions, while ABD's rigid patch-replacement mechanism may overlook nuanced contextual dependencies, limiting its adaptability across varying data complexities.



This work builds on the insights from ABD and addresses its limitations with a novel Confidence-Aware Displacement (CAD) strategy. As illustrated in Fig.~\ref{fig1}(c), CAD dynamically escalates the replacement process across training stages. In the early stages, the replacement focuses on small, low-confidence regions, ensuring stable learning. As training progresses, CAD adaptively increases the scale of replacement by refining both the confidence threshold and the region size. This progressive approach not only mitigates the impact of unstable regions but also ensures that the model effectively integrates reliable information from both labeled and unlabeled data. By introducing this adaptive mechanism, CAD enhances segmentation performance and provides a more robust framework for handling challenging semi-supervised scenarios.

Our main contributions can be summarized as follows:
\begin{itemize}
    \item We propose a novel CAD strategy that incorporates the Largest Low-Confidence Region Replacement (LLCR) method to dynamically replace low-confidence regions with high-confidence counterparts during training, enabling more stable and robust learning.
    \item We introduce a Dynamic Threshold Escalation (DTE) mechanism, which adaptively adjusts the confidence threshold and replacement region size throughout training, ensuring that the replacement process aligns with the evolving model predictions.
    \item Extensive experiments on public datasets demonstrate that our CAD framework achieves state-of-the-art performance, significantly surpassing existing methods in both accuracy and robustness.
\end{itemize}



\section{Related Works}

Consistency learning plays a pivotal role in SSMIS by effectively utilizing large amounts of unlabeled data. The core idea is to encourage models to produce stable predictions under various perturbations or augmentations of the input data, thereby enhancing performance even with limited labeled examples. Early works, such as the study by Sajjadi et al. \cite{b7}, highlighted the importance of stochastic transformations like dropout and random augmentations to regularize deep networks, improving generalization and robustness against input variability. More recent studies have expanded on these ideas to enhance consistency in semi-supervised segmentation tasks; for instance, ConMatch\cite{b31} incorporates confidence-guided consistency regularization, refining pseudo-label confidence estimations to improve performance. Additionally, methods such as UDiCT\cite{b32} pair annotated and unannotated data based on uncertainty, thereby mitigating the impact of unreliable pseudo-labels, while Huang et al. \cite{b33} introduced a two-stage approach enforcing consistency across perturbed versions of unlabeled electron microscopy volumes to enhance model robustness.

In medical image segmentation, consistency learning addresses significant challenges posed by data variability and insufficient labeled data. Prominent methods have augmented consistency learning by integrating additional tasks or uncertainty estimations, as demonstrated by Shu et al. \cite{b34} and Wang et al. \cite{b35}. For instance, Liu et al. \cite{b36} applied transformer-based models to COVID-19 lesion segmentation, using consistency across augmented views to alleviate the shortage of labeled data. Moreover, Chen et al. \cite{b37} proposed Cross Pseudo Supervision (CPS), where two segmentation networks with different initializations enforce consistency on each other's predictions, effectively expanding training data via pseudo-labels. Another notable contribution is from Tarvainen and Valpola \cite{b9}, who introduced the Mean Teacher model, maintaining an exponential moving average (EMA) of model weights to improve segmentation performance under semi-supervised conditions. Collectively, these advances demonstrate how consistency learning, especially when combined with innovative model architectures and training strategies, significantly enhances segmentation performance with limited labeled data.




% \section{Method}


% Mathematically, we define the 3D volume of a medical image as $X \in \mathbb{R}^{W \times H \times L}$, where $W$, $H$, and $L$ denote the width, height, and depth of the volume, respectively. The goal of semi-supervised medical image segmentation is to predict the per-voxel label map $Y \in \{0,1,\dots,K-1\}^{W \times H \times L}$, indicating where the background and the targets are in $X$. Here, $K$ represents the number of classes. The training set $D$ consists of $N$ labeled data and $M$ unlabeled data ($N \ll M$), expressed as two subsets: $D = D_l \cup D_u$, where $D_l = \{(X^l_i, Y^l_i)\}^N_{i=1}$ and $D_u = \{X^u_i\}^{M+N}_{i=N+1}$. In subsequent descriptions, we omit the index $i$ for simplicity.

% Our approach adopts the Mean Teacher framework, which leverages a teacher model and two student models to exploit both labeled and unlabeled data. Specifically, we employ an improved Mean Teacher structure based on BCP. The teacher model, denoted as $f_t$, is updated using an exponential moving average (EMA) of the student models, which are denoted as $f_{\theta_1}$ and $f_{\theta_2}$. These student models process two groups of images with weak and strong augmentations, respectively, and their predictions are utilized for Mean Teacher loss computation.

% In each training iteration, the outputs of $f_{\theta_1}$ and $f_{\theta_2}$ are used to calculate the confidence map. This map, combined with the Dynamic Threshold Escalation (DTE) strategy, determines the confidence threshold and maximum connected region size for replacement at the current training stage. This process, referred to as Largest Low-Confidence Region Replacement (LLCR), identifies the lowest-confidence largest connected region in the strongly augmented image and replaces it with a corresponding high-confidence region from the weakly augmented image, and vice versa. This Confidence-Aware Adaptive Replacement ensures robust learning by focusing on regions with greater uncertainty.

% Finally, Cross Pseudo Supervision (CPS) is applied to the predictions of the replaced images, enabling pseudo supervision loss calculation. The overall framework, as illustrated in Fig.~2, effectively integrates these strategies to enhance semi-supervised segmentation performance.

% \subsection{Confidence-Aware Adaptive Displacement}
% \subsubsection{Largest Low-Confidence Region Replacement} 

% In the segmentation process, the input image is divided into smaller patches. Specifically, we partition the image into $h \times w$ patches, where in this study, we use $16 \times 16$ patches, resulting in a total of 256 patches per image. Each patch represents a localized region of the image, which is crucial for segmenting and replacing low-confidence areas at the patch level.



% Let $f_{\theta_1}$ and $f_{\theta_2}$ denote the two student models. For an input image $X_w$ with weak augmentation and $X_s$ with strong augmentation, the models produce logits as follows:
% \[
% \text{logits}_w = f_{\theta_1}(X_w), \quad \text{logits}_s = f_{\theta_2}(X_s).
% \]

% Applying the softmax operation to the logits, we obtain the per-pixel probabilities:
% \[
% P_w(x) = \mathrm{softmax}(\text{logits}_w(x)), \quad P_s(x) = \mathrm{softmax}(\text{logits}_s(x)), \quad \forall x \in \Omega,
% \]
% where $\Omega$ denotes the spatial domain of the image. For each pixel $x$, the confidence value is defined as the maximum probability across all classes:
% \[
% \alpha_w(x) = \max_{c \in \{1, \dots, K\}} P_{w,c}(x), \quad \alpha_s(x) = \max_{c \in \{1, \dots, K\}} P_{s,c}(x).
% \]

% The confidence of a patch $p_i$ is computed as the average confidence of all pixels within the patch:
% \[
% C_{p_i} = \frac{1}{|p_i|} \sum_{x \in p_i} \alpha(x),
% \]
% where $|p_i|$ is the total number of pixels in patch $p_i$.

% To standardize the confidence values, we apply min-max normalization:
% \[
% \widetilde{C}_{p_i} = \frac{C_{p_i} - \min_j C_{p_j}}{\max_j C_{p_j} - \min_j C_{p_j}},
% \]
% where $\widetilde{C}_{p_i} \in [0, 1]$ ensures that confidence values are scaled to a consistent range for comparison.



% To identify the largest low-confidence region, we use a Breadth-First Search (BFS) approach starting from the patch with the lowest confidence. The BFS process is guided by two thresholds:
% - $C_\mathrm{threshold}$: the confidence threshold for including patches.
% - $R_\mathrm{threshold}$: the maximum size of the connected region.

% The BFS algorithm is described using pseudocode in the following format:

% \begin{algorithm}[H]
% \caption{BFS for Largest Low-Confidence Region}
% \begin{algorithmic}[1]
% \STATE \textbf{Initialize:} $\mathcal{R}_\mathrm{low} \gets \emptyset$
% \STATE Identify the patch $p_0$ with the lowest confidence:
% \[
% p_0 = \arg\min_i \widetilde{C}_{p_i}.
% \]
% \STATE Push $p_0$ into a priority queue (min-heap) keyed by $\widetilde{C}_{p_i}$.
% \WHILE {the priority queue is not empty and $|\mathcal{R}_\mathrm{low}| < R_\mathrm{threshold}$}
%     \STATE Pop the patch $p$ with the lowest confidence from the queue.
%     \IF {$p \notin \mathcal{R}_\mathrm{low}$ and $\widetilde{C}_{p} \leq C_\mathrm{threshold}$}
%         \STATE Add $p$ to $\mathcal{R}_\mathrm{low}$.
%     \ENDIF
%     \FORALL {neighbors $q$ of $p$}
%         \IF {$q \notin \mathcal{R}_\mathrm{low}$ and $\widetilde{C}_{q} \leq C_\mathrm{threshold}$}
%             \STATE Push $q$ into the queue.
%         \ENDIF
%     \ENDFOR
% \ENDWHILE
% \STATE \textbf{Output:} $\mathcal{R}_\mathrm{low}$ as the largest low-confidence region.
% \end{algorithmic}
% \end{algorithm}





% The resulting connected region $\mathcal{R}_\mathrm{low}$ is described using shape offsets. Let the bounding box of $\mathcal{R}_\mathrm{low}$ extend from $(r_\mathrm{min}, c_\mathrm{min})$ to $(r_\mathrm{max}, c_\mathrm{max})$ in patch indices. The offset for each patch $p$ in $\mathcal{R}_\mathrm{low}$ is defined as:
% \[
% \Delta r = r - r_\mathrm{min}, \quad \Delta c = c - c_\mathrm{min}.
% \]

% To replace the low-confidence region, we search for a matching region in another confidence map with the same shape as $\mathcal{R}_\mathrm{low}$. For each possible top-left corner $(r_s, c_s)$ in the second map, we define a candidate region $\mathcal{R}_c$ with the same offsets as $\mathcal{R}_\mathrm{low}$. The average confidence of $\mathcal{R}_c$ is computed as:
% \[
% \overline{C}(\mathcal{R}_c) = \frac{1}{|\mathcal{R}_c|} \sum_{p \in \mathcal{R}_c} \widetilde{C}_{p},
% \]
% where $|\mathcal{R}_c| = |\mathcal{R}_\mathrm{low}|$.

% We identify the candidate region with the highest average confidence:
% \[
% \mathcal{R}_\mathrm{high} = \arg\max_{\mathcal{R}_c} \overline{C}(\mathcal{R}_c).
% \]

% Finally, the patches in $\mathcal{R}_\mathrm{low}$ are replaced with the corresponding patches from $\mathcal{R}_\mathrm{high}$ in the input image, ensuring that the region with the lowest confidence is updated with the most reliable information from the other model.

% \subsubsection{Dynamic Threshold Escalation}

% To adaptively guide the training process, we introduce two thresholds that evolve dynamically during training: the confidence threshold \( C_\mathrm{threshold}(t) \) and the maximum region size \( R_\mathrm{threshold}(t) \). These thresholds are defined as follows:

% \[
% C_\mathrm{threshold}(t) 
% = 
% C_\mathrm{min} 
% + 
% \big(C_\mathrm{max} - C_\mathrm{min}\big) \Psi(t),
% \quad
% R_\mathrm{threshold}(t) 
% = 
% R_\mathrm{min} 
% + 
% \big(R_\mathrm{max} - R_\mathrm{min}\big) \Psi(t),
% \]

% where \( \Psi(t) \) is a monotonically increasing ramp function that maps the training iteration \( t \) to a value between 0 and 1, ensuring a gradual transition of these thresholds over time. We use the form:

% \[
% \Psi(t) = 1 - e^{-t / \beta},
% \]

% where \( \beta \) controls the rate of increase, and the thresholds converge to their maximum values by the end of the ramp-up period.

% This strategy is motivated by the distinct requirements of different training stages. During the early stages, the model's predictions are highly uncertain and less reliable. To ensure stable learning, we set \( C_\mathrm{threshold}(t) \) to a low value (\( C_\mathrm{min} \)), restricting the replacement process to only the least confident regions. Similarly, the size of the connected regions is limited by setting \( R_\mathrm{threshold}(t) \) to \( R_\mathrm{min} \), preventing extensive modifications that could destabilize the model. As training progresses and the model becomes more confident, both thresholds increase, allowing the replacement process to encompass a larger portion of the image domain. This gradual escalation ensures that the model addresses a broader range of uncertainties while leveraging its improved capability to integrate such corrections without disrupting learning dynamics. By the final stages of training, the thresholds reach \( C_\mathrm{max} \) and \( R_\mathrm{max} \), enabling comprehensive and aggressive refinement of the segmentation predictions. This progression ensures that the model transitions smoothly from learning foundational patterns to refining its predictions effectively.


\section{Method}

Mathematically, we define the 3D volume of a medical image as $X \in \mathbb{R}^{W \times H \times L}$, where $W$, $H$, and $L$ denote the width, height, and depth of the volume, respectively. The goal of semi-supervised medical image segmentation is to predict the per-voxel label map $Y \in \{0,1,\dots,K-1\}^{W \times H \times L}$, indicating where the background and the targets are in $X$. Here, $K$ represents the number of classes. The training set $\mathcal{D}$ consists of $N$ labeled data and $M$ unlabeled data ($N \ll M$), expressed as two subsets: $\mathcal{D} = \mathcal{D}_l \cup \mathcal{D}_u$, where $\mathcal{D}_l = \{(X^l_i, Y^l_i)\}^N_{i=1}$ and $\mathcal{D}_u = \{X^u_i\}^{M+N}_{i=N+1}$. In subsequent descriptions, we omit the index $i$ for simplicity.

% Our approach adopts the Mean Teacher framework, which leverages a teacher model and two student models to exploit both labeled and unlabeled data. The teacher model, denoted as $f_t$, is updated using an exponential moving average (EMA) of the student models, which are denoted as $f_{\theta_1}$ and $f_{\theta_2}$. These student models process two groups of images with weak and strong augmentations, respectively, and their predictions are utilized for Mean Teacher loss computation.

% In each training iteration, the outputs of $f_{\theta_1}$ and $f_{\theta_2}$ are used to calculate the confidence map. This map, combined with the Dynamic Threshold Escalation (DTE) strategy, determines the confidence threshold and maximum connected region size for replacement at the current training stage. This process, referred to as Largest Low-Confidence Region Replacement (LLCR), identifies the lowest-confidence largest connected region in the strongly augmented image and replaces it with a corresponding high-confidence region from the weakly augmented image, and vice versa. This Confidence-Aware Adaptive Replacement ensures robust learning by focusing on regions with greater uncertainty.

% Finally, Cross Pseudo Supervision (CPS) is applied to the predictions of the replaced images, enabling pseudo supervision loss calculation. The overall framework, as illustrated in Fig.~2, effectively integrates these strategies to enhance semi-supervised segmentation performance.

\begin{figure*}[htbp]
\centerline{\includegraphics[width=520pt]{figs.jpg}}
\caption{Overview of the proposed framework integrating Mean Teacher with DTE, LLCR, and CPS. The diagram illustrates the data flow, confidence-based region replacement, and collaborative learning between the student and teacher models.}
\label{fig2}
\end{figure*}

The overall workflow of our proposed method is as follows:

\begin{enumerate}
    \item Mean Teacher Framework: The teacher model \(f_t\) is updated via EMA of student models \(f_{\theta_1}\) and \(f_{\theta_2}\). These student models process weakly (\(X_w\)) and strongly (\(X_s\)) augmented images, respectively.

    \item Dynamic Threshold Escalation (DTE) and Largest Low-Confidence Region Replacement (LLCR): In each iteration, confidence maps derived from \(f_{\theta_1}\) and \(f_{\theta_2}\) guide the identification of the largest low-confidence region (\(\mathcal{R}_\mathrm{low}\)) in \(X_s\). This region is replaced by a high-confidence counterpart from \(X_w\), and vice versa, ensuring the model focuses on uncertain areas while maintaining robust learning dynamics.

    \item Cross Pseudo Supervision (CPS): After LLCR, predictions from both models are compared using CPS loss\cite{b37}, which further refines the segmentation by enforcing consistency between the two student models.

\end{enumerate}

The combined strategies, as illustrated in Fig.~\ref{fig2}, enable our framework to address uncertainty effectively and improve semi-supervised segmentation performance.


\subsection{Confidence-Aware Adaptive Displacement}
\subsubsection{Largest Low-Confidence Region Replacement (LLCR)}

In the segmentation process, the input image is divided into smaller patches. Specifically, we partition the image into $h \times w$ patches, resulting in $N_p$ total patches, where $N_p = h \times w$. Let $p_i$ denote the $i$-th patch, with $i \in \{1, \dots, N_p\}$. Each patch represents a localized region of the image, which is crucial for segmenting and replacing low-confidence areas at the patch level.

Let $f_{\theta_1}$ and $f_{\theta_2}$ denote the two student models. For an input image $X_w$ with weak augmentation and $X_s$ with strong augmentation, the models produce logits as follows:
\begin{equation}
\text{logits}_w = f_{\theta_1}(X_w), \quad \text{logits}_s = f_{\theta_2}(X_s). \label{eq:logits}
\end{equation}

Applying the softmax operation to the logits, we obtain the per-pixel probabilities:
\begin{equation}
\begin{aligned}
P_w(x) &= \mathrm{softmax}(\text{logits}_w(x)), \\
P_s(x) &= \mathrm{softmax}(\text{logits}_s(x)), 
\end{aligned}
\quad
\forall x \in \Omega,
\label{eq:softmax}
\end{equation}

where $\Omega$ denotes the spatial domain of the image. For each pixel $x$, the confidence value is defined as the maximum probability across all classes:
\begin{equation}
\alpha_w(x) = \max_{c \in \{1, \dots, K\}} P_{w,c}(x), 
\quad 
\alpha_s(x) = \max_{c \in \{1, \dots, K\}} P_{s,c}(x). 
\label{eq:confidence}
\end{equation}

The confidence of a patch $p_i$ is computed as the average confidence of all pixels within the patch:
\begin{equation}
C_{p_i} = \frac{1}{|p_i|} \sum_{x \in p_i} \alpha(x),
\label{eq:patch_confidence}
\end{equation}
where $|p_i|$ is the total number of pixels in patch $p_i$.

To standardize the confidence values, we apply min-max normalization:
\begin{equation}
\widetilde{C}_{p_i} = \frac{C_{p_i} - \min_j C_{p_j}}{\max_j C_{p_j} - \min_j C_{p_j}},
\label{eq:minmax_normalization}
\end{equation}
where $\widetilde{C}_{p_i} \in [0, 1]$ ensures that confidence values are scaled to a consistent range for comparison.

To identify the largest low-confidence region, we use a Breadth-First Search (BFS) approach starting from the patch with the lowest confidence. The BFS process is guided by two thresholds:
- $C_\mathrm{threshold}$: the confidence threshold for including patches.
- $R_\mathrm{threshold}$: the maximum size of the connected region.

The BFS algorithm is described using pseudocode in the following format:

\begin{algorithm}[H]
\caption{BFS for Largest Low-Confidence Region}
\begin{algorithmic}[1]
\STATE \textbf{Initialize:} $\mathcal{R}_\mathrm{low} \gets \emptyset$
\STATE Identify the patch $p_0$ with the lowest confidence:
\[
p_0 = \arg\min_i \widetilde{C}_{p_i}.
\]
\STATE Push $p_0$ into a priority queue (min-heap) keyed by $\widetilde{C}_{p_i}$.
\WHILE {the priority queue is not empty and $|\mathcal{R}_\mathrm{low}| < R_\mathrm{threshold}$}
    \STATE Pop the patch $p$ with the lowest confidence from the queue.
    \IF {$p \notin \mathcal{R}_\mathrm{low}$ and $\widetilde{C}_{p} \leq C_\mathrm{threshold}$}
        \STATE Add $p$ to $\mathcal{R}_\mathrm{low}$.
    \ENDIF
    \FORALL {neighbors $q$ of $p$}
        \IF {$q \notin \mathcal{R}_\mathrm{low}$ and $\widetilde{C}_{q} \leq C_\mathrm{threshold}$}
            \STATE Push $q$ into the queue.
        \ENDIF
    \ENDFOR
\ENDWHILE
\STATE \textbf{Output:} $\mathcal{R}_\mathrm{low}$ as the largest low-confidence region.
\end{algorithmic}
\end{algorithm}





The resulting connected region $\mathcal{R}_\mathrm{low}$ is described using shape offsets. Let the bounding box of $\mathcal{R}_\mathrm{low}$ extend from $(r_\mathrm{min}, c_\mathrm{min})$ to $(r_\mathrm{max}, c_\mathrm{max})$ in patch indices. The offset for each patch $p$ in $\mathcal{R}_\mathrm{low}$ is defined as:
\begin{equation}
\Delta r = r - r_\mathrm{min}, \quad \Delta c = c - c_\mathrm{min}.\label{eq:delta}
\end{equation}

To replace the low-confidence region, we search for a matching region in another confidence map with the same shape as $\mathcal{R}_\mathrm{low}$. For each possible top-left corner $(r_s, c_s)$ in the second map, we define a candidate region $\mathcal{R}_c$ with the same offsets as $\mathcal{R}_\mathrm{low}$. The average confidence of $\mathcal{R}_c$ is computed as:
\begin{equation}
\overline{C}(\mathcal{R}_c) = \frac{1}{|\mathcal{R}_c|} \sum_{p \in \mathcal{R}_c} \widetilde{C}_{p},\label{eq:crc}
\end{equation}
where $|\mathcal{R}_c| = |\mathcal{R}_\mathrm{low}|$.

We identify the candidate region with the highest average confidence:
\begin{equation}
\mathcal{R}_\mathrm{high} = \arg\max_{\mathcal{R}_c} \overline{C}(\mathcal{R}_c).\label{eq:Rhigh}
\end{equation}

Finally, the patches in $\mathcal{R}_\mathrm{low}$ are replaced with the corresponding patches from $\mathcal{R}_\mathrm{high}$ in the input image, ensuring that the region with the lowest confidence is updated with the most reliable information from the other model.

\subsubsection{Dynamic Threshold Escalation (DTE)}

To adaptively guide the training process, we introduce two thresholds that evolve dynamically during training: the confidence threshold \( C_\mathrm{threshold}(t) \) and the maximum region size \( R_\mathrm{threshold}(t) \). These thresholds are defined as follows:
\begin{equation}
\begin{aligned}
C_\mathrm{threshold}(t) 
&= 
C_\mathrm{min} 
+ 
\big(C_\mathrm{max} - C_\mathrm{min}\big) \Psi(t), \\
R_\mathrm{threshold}(t) 
&= 
R_\mathrm{min} 
+ 
\big(R_\mathrm{max} - R_\mathrm{min}\big) \Psi(t),
\end{aligned}
\label{eq:thresholds}
\end{equation}

where \( \Psi(t) \) is a monotonically increasing ramp function that maps the training iteration \( t \) to a value between 0 and 1:
\begin{equation}
\Psi(t) = 1 - e^{-t / \beta}. \label{eq:psi}
\end{equation}

This strategy is motivated by the distinct requirements of different training stages. During the early stages, the model's predictions are highly uncertain and less reliable. To ensure stable learning, we set \( C_\mathrm{threshold}(t) \) to a low value (\( C_\mathrm{min} \)), restricting the replacement process to only the least confident regions. Similarly, the size of the connected regions is limited by setting \( R_\mathrm{threshold}(t) \) to \( R_\mathrm{min} \), preventing extensive modifications that could destabilize the model. As training progresses and the model becomes more confident, both thresholds increase, allowing the replacement process to encompass a larger portion of the image domain. By the final stages of training, the thresholds reach \( C_\mathrm{max} \) and \( R_\mathrm{max} \), enabling comprehensive and aggressive refinement of the segmentation predictions.


\subsection{Loss Function}

The loss function is composed of three main components: Mean Teacher Loss (\(L_\text{MT}\)), CPS Loss (\(L_\text{CPS}\)), and CAD Loss (\(L_\text{CAD}\)). Each of these losses is designed to guide the model's training, using a combination of both labeled and unlabeled data to enhance the segmentation performance.

\subsubsection{Mean Teacher Loss (\(L_\text{MT}\))}

The Mean Teacher loss is calculated using both labeled and unlabeled data, employing a combination of Dice loss and Cross Entropy loss. The Mean Teacher loss is computed separately for weakly augmented data (\(L_\text{MT1}\)) and strongly augmented data (\(L_\text{MT2}\)).

For labeled data, the Mean Teacher loss combines the Dice loss and the Cross Entropy loss, given by:
\begin{equation}
L_{\text{MT}} = \left[ 1 - \frac{2 \sum_{c}\text{true}_c \cdot \hat{\text{pred}}_c}{\sum_{c}\text{true}_c + \sum_{c}\hat{\text{pred}}_c} \right] - \sum_{c}\text{true}_c \log(\hat{\text{pred}}_c),
\end{equation}
where \(\text{true}_c\) and \(\hat{\text{pred}}_c = \text{softmax}(\text{logits}_c)\) represent the ground truth and the predicted probability for class \(c\), respectively.

For unlabeled data, the loss function is similarly computed, but only the pseudo labels generated by the model are used for both the Dice and Cross Entropy losses. We denote the loss for unlabeled weakly augmented data as \(L_{\text{MT1}}\) and for strongly augmented data as \(L_{\text{MT2}}\).

Thus, the total Mean Teacher Loss is the sum of the weak and strong augmentation losses:
\begin{equation}
L_\text{MT1} = L_{\text{dice}}(f_{\theta_1}, p) + L_{\text{ce}}(f_{\theta_1}, p), 
\end{equation}
\begin{equation}
L_\text{MT2} = L_{\text{dice}}(f_{\theta_2}, p) + L_{\text{ce}}(f_{\theta_2}, p).
\end{equation}

\subsubsection{Cross Pseudo Supervision Loss (\(L_\text{CPS}\))}

The Cross Pseudo Supervision Loss ensures that the predictions from two different student models are consistent with each other. For weakly augmented data, we define the CPS loss as \(L_{\text{CPS1}}\), and for strongly augmented data, we define it as \(L_{\text{CPS2}}\).

The CPS loss is computed as the Dice loss between the softmax outputs of the two models, which are \(f_{\theta_1}\) and \(f_{\theta_2}\), using the pseudo labels generated by each model for the other. For weakly augmented data:
\begin{equation}
L_{\text{CPS1}} = \text{dice\_loss}(\text{softmax}(f_{\theta_1}(X_w)), \text{argmax}(f_{\theta_2}(X_s))),
\end{equation}
and for strongly augmented data:
\begin{equation}
L_{\text{CPS2}} = \text{dice\_loss}(\text{softmax}(f_{\theta_2}(X_s)), \text{argmax}(f_{\theta_1}(X_w))).
\end{equation}
The Dice loss is calculated similarly to the equation for \(L_{\text{dice}}\), where the predicted labels are compared with the pseudo labels generated by the opposite student model.

\subsubsection{Confidence-Aware Displacement Loss (\(L_\text{CAD}\))}

The CAD loss is computed after applying CAD to replace the low-confidence regions in the image. Following the replacement, the modified images are passed through the models, and Cross Pseudo Supervision loss is calculated based on the models' predictions. The CAD loss is computed for both weakly augmented data and strongly augmented data, denoted as \(L_{\text{CAD1}}\) and \(L_{\text{CAD2}}\), respectively.

After the low-confidence regions are replaced, we denote the new image after replacement as \(X'_w\) for the weakly augmented data and \(X'_s\) for the strongly augmented data, where these images are now the modified versions, containing the high-confidence regions from the other model's predictions.

For weakly augmented data after replacement, the CAD loss is computed as:
\begin{equation}
L_{\text{CAD1}} = \text{dice\_loss}(\text{softmax}(f_{\theta_1}(X'_w)), \text{argmax}(f_{\theta_2}(X'_s))),
\end{equation}
where \(X'_w\) represents the weakly augmented image after low-confidence regions have been replaced with high-confidence patches from the strongly augmented image \(X'_s\).

For strongly augmented data after replacement, the CAD loss is given by:
\begin{equation}
L_{\text{CAD2}} = \text{dice\_loss}(\text{softmax}(f_{\theta_2}(X'_s)), \text{argmax}(f_{\theta_1}(X'_w))).
\end{equation}
Here, \(X'_s\) represents the strongly augmented image after low-confidence regions have been replaced with high-confidence patches from the weakly augmented image \(X'_w\).

The CAD loss functions \(L_{\text{CAD1}}\) and \(L_{\text{CAD2}}\) ensure that the replaced low-confidence regions are consistent with the rest of the image and that the pseudo labels produced by the other model are in agreement with the modifications made to the image.

% By incorporating the CAD loss, the model is encouraged to refine its predictions and improve the segmentation by focusing on uncertain regions and using high-confidence predictions from the other model to guide the correction of low-confidence areas.

\subsubsection{Total Loss}

The total loss for the model is computed by summing the Mean Teacher Loss, Cross Pseudo Supervision Loss, and CAD Loss for both weak and strong augmented data. The total loss for the first model is given by:
\begin{equation}
L_1 = L_{\text{MT1}} + L_{\text{CPS1}} + L_{\text{CAD1}},
\end{equation}
and the total loss for the second model is given by:
\begin{equation}
L_2 = L_{\text{MT2}} + L_{\text{CPS2}} + L_{\text{CAD2}}.
\end{equation}

The final total loss is the sum of \(L_1\) and \(L_2\):
\begin{equation}
L = L_1 + L_2.
\end{equation}

This loss function drives the network to learn accurate segmentation by leveraging both labeled and unlabeled data, dynamically guiding the model to focus on uncertain regions and progressively refining its predictions through a combination of each loss components.


\section{Experiments}

\subsection{Datasets}

In this study, we evaluate the proposed semi-supervised learning approach using two widely recognized datasets: ACDC (Automated Cardiac Diagnosis Challenge)\cite{b17} and PROMISE12\cite{b18}. Although originally composed of 3D medical images, both datasets are converted into 2D slices for simplicity, aligning with common practices in segmentation studies\cite{b19}.

\subsubsection{ACDC Dataset} The ACDC dataset\cite{b17} contains 200 short-axis cardiac cine-MR images from 100 patients, labeled into four classes: left ventricle, right ventricle, myocardium, and background. Each 3D scan is converted into 2D cross-sectional slices, and the dataset is partitioned into training, validation, and testing sets with a ratio of 70:10:20.

\subsubsection{PROMISE12 Dataset} The PROMISE12 dataset\cite{b18}, designed for the MICCAI 2012 prostate segmentation challenge, includes MRI scans from 50 patients with diverse prostate pathologies. We similarly convert the original 3D images into 2D slices, which pose segmentation challenges due to substantial variations in prostate size and shape among patients.

\subsection{Evaluation Metrics}

For the evaluation of our model, we use four commonly used metrics in medical image segmentation: Dice Similarity Coefficient (DSC), Jaccard Index (Jaccard), 95\% Hausdorff Distance (95HD), and Average Surface Distance (ASD). 

The DSC and Jaccard Index are both used to measure the overlap between the predicted and ground truth regions, with higher values indicating better segmentation performance. 

The ASD computes the average distance between the boundaries of the predicted and ground truth regions, giving an indication of the accuracy of boundary localization. 

The 95HD measures the maximum distance between the boundaries of the predicted and ground truth regions, considering the 95th percentile, and provides insight into the largest discrepancy between the two regions.



\subsection{Implementation Details}

We conducted our experiments in an NVIDIA V100 GPU environment. The image patch is divided into \(N_p\) smaller patches, where each patch is of size \(h \times w\). In our experiments, we set \(h = w = 16\), leading to a total of \(N_p = 256\) patches for each image.

The confidence threshold \(C_\text{threshold}(t)\) and the region size threshold \(R_\text{threshold}(t)\) evolve during the training process. Initially, the confidence threshold is set to a small value (\(C_\text{min} = 0.01\)) and the region size threshold is set to a small value (\(R_\text{min} = 1\)). These values gradually increase over training, with the confidence threshold reaching a maximum of \(C_\text{max} = 0.75\) and the region size threshold growing to \(R_\text{max} = 16\) by the end of the training process.

Our method employs the Mean Teacher framework, which is built upon the BCP\cite{b3} strategy to enhance performance.



\subsection{Comparison with State-of-the-Art Methods} 
\subsubsection{ACDC Dataset}


We compare our proposed CAD method with ABD\cite{b5} (the baseline method), CorrMatch\cite{b28}, SAMT-PCL\cite{b29}, BCP\cite{b3}, SCP-Net\cite{b6}, and other methods on the ACDC test set using 5\% and 10\% of labeled data for training. The results are presented in Table.~\ref{tab:acdc_comparison}, where the metrics of ABD and CorrMatch are reproduced. As shown, CAD outperforms the baseline model when trained with 5\% labeled data, achieving higher scores in DSC, 95HD, and ASD.

When trained with 10\% labeled data, our model demonstrates superior performance across all four evaluation metrics compared to existing state-of-the-art methods. Notably, CAD achieves a remarkable 42.7\% improvement in 95HD, setting a new benchmark for state-of-the-art performance on this dataset. The comparison of three samples from the test set with U-Net\cite{b22}, BCP\cite{b3}, CorrMatch\cite{b28}, ABD\cite{b5}, and Ground Truth, all trained with 10\% labeled data, are shown in Fig.~\ref{fig3}


% 





\begin{table*}[htbp]
\caption{Comparisons with other state-of-the-art methods on the ACDC test set.}
\begin{center}
\begin{tabular}{|l|c|c|c|c|c|c|c|}
\hline
\textbf{Method} & \multicolumn{2}{|c|}{\textbf{Scans used}} & \multicolumn{4}{|c|}{\textbf{Metrics}} \\
\cline{2-7} 
 & Labeled & Unlabeled & DSC $\uparrow$ & Jaccard $\uparrow$ & 95HD $\downarrow$ & ASD $\downarrow$ \\
\hline
U-Net (MICCAI'2015) \cite{b22} & 3 (5\%) & 0 & 47.83 & 37.01 & 31.16 & 12.62 \\
 & 7 (10\%) & 0 & 79.41 & 68.11 & 9.35 & 2.70 \\
 & 70 (All) & 0 & 91.44 & 84.59 & 4.30 & 0.99 \\
\hline
DTC (AAAI'2021) \cite{b25} & 3 (5\%) & 67 (95\%) & 56.90 & 45.67 & 23.36 & 7.39 \\
URPC (MICCAI'2021) \cite{b24} & 3 (5\%) & 67 (95\%) & 55.87 & 44.64 & 13.60 & 3.74 \\
MC-Net (MICCAI'2021) \cite{b23} & 3 (5\%) & 67 (95\%) & 62.85 & 52.29 & 7.62 & 2.33 \\
SS-Net (MICCAI'2022) \cite{b21} & 3 (5\%) & 67 (95\%) & 65.83 & 55.38 & 6.67 & 2.28 \\
SCP-Net (MICCAI'2023) \cite{b6} & 3 (5\%) & 67 (95\%) & 87.27 & - & - & 2.65 \\
BCP (CVPR'2023) \cite{b3} & 3 (5\%) & 67 (95\%) & 87.59 & 78.67 & 1.90 & 0.67 \\
SAMT-PCL (ESA'2024) \cite{b29} & 3 (5\%) & 67 (95\%) & 74.39 & 63.94 & 5.07 & 1.42 \\
CorrMatch (CVPR'2024) \cite{b28} & 3 (5\%) & 67 (95\%) & 86.33 & 75.99 & 4.01 & 1.22 \\
ABD (CVPR'2024) \cite{b5} & 3 (5\%) & 67 (95\%) & 87.98 & \textbf{80.12} & 1.89 & 0.74 \\
\textbf{CAD (Ours)} & 3 (5\%) & 67 (95\%) & \textbf{88.02} & 79.79 & \textbf{1.55} & \textbf{0.48} \\
\hline
DTC (AAAI'2021) \cite{b25} & 7 (10\%) & 63 (90\%) & 84.29 & 73.92 & 12.81 & 4.01 \\
URPC (MICCAI'2021) \cite{b24} & 7 (10\%) & 63 (90\%) & 83.10 & 72.41 & 4.84 & 1.53 \\
MC-Net (MICCAI'2021) \cite{b23} & 7 (10\%) & 63 (90\%) & 86.44 & 77.04 & 5.50 & 1.84 \\
SS-Net (MICCAI'2022) \cite{b21} & 7 (10\%) & 63 (90\%) & 86.78 & 77.67 & 6.07 & 1.40 \\
SCP-Net (MICCAI'2023) \cite{b6} & 7 (10\%) & 63 (90\%) & 89.69 & - & - & 0.73 \\
PLGCL (CVPR'2023) \cite{b20} & 7 (10\%) & 63 (90\%) & 89.1 & - & 4.98 & 1.80 \\
BCP (CVPR'2023) \cite{b3} & 7 (10\%) & 63 (90\%) & 88.84 & 80.62 & 3.98 & 1.17 \\
SAMT-PCL (ESA'2024) \cite{b29} & 7 (10\%) & 63 (90\%) & 88.62 & 80.11 & 2.06 & 0.60 \\
CorrMatch (CVPR'2024) \cite{b28} & 7 (10\%) & 63 (90\%) & 87.33 & 78.27 & 4.31 & 1.45 \\
ABD (CVPR'2024) \cite{b5} & 7 (10\%) & 63 (90\%) & 89.77 & 81.88 & 1.77 & 0.49 \\
\textbf{CAD (Ours)}  & 7 (10\%) & 63 (90\%) & \textbf{90.38} & \textbf{82.89} & \textbf{1.24} & \textbf{0.36} \\
\hline
\end{tabular}
\label{tab:acdc_comparison}
\end{center}
\end{table*}

\begin{figure}[hbtp]
\centerline{\includegraphics[width=250pt]{3.png}}
\caption{Segmentation results on three test set samples from the ACDC dataset using 10\% labeled data: U-Net, BCP, CorrMatch, ABD, CAD (Ours), and Ground Truth (GT).}
\label{fig3}
\end{figure}

\subsubsection{PROMISE12 Dataset}
% 在PROMISE12数据集上,我们以20%的有标注数据比例的情况将CAD与ABD、CorrMatch、BCP、SCP-Net、SLC-Net、URPC和CCT在DSC和ASD两项metrics上进行对比,其中ABD是baseline模型,结果如表2所示,其中ABD、BCP和Corrmatch的数据采用了reproduced数据。结果表示CAD在两项metrics上均超过了baseline模型,并在DSC上实现了SOTA。

On the PROMISE12 dataset, we compare our proposed CAD method with ABD\cite{b5} (the baseline model), CorrMatch\cite{b28}, BCP\cite{b3}, SCP-Net\cite{b6}, SLC-Net\cite{b27}, SS-Net\cite{b21}, URPC\cite{b24}, and CCT\cite{b26} using 20\% of labeled data for training. The comparison focuses on the DSC and ASD metrics. As shown in Table.~\ref{tab:promise12_comparison}, the results for ABD, BCP, and CorrMatch are based on our reproduced data. The experimental outcomes demonstrate that CAD outperforms the baseline model in both metrics and achieves state-of-the-art performance in terms of DSC.



\begin{table}[htbp]
\caption{Comparisons with state-of-the-art methods on the PROMISE12 test set.}
\begin{center}
\resizebox{0.5\textwidth}{!}{
\begin{tabular}{|l|c|c|c|c|}
\hline
\textbf{Method} & \multicolumn{2}{|c|}{\textbf{Scans used}} & \multicolumn{2}{|c|}{\textbf{Metrics}} \\
\cline{2-5} 
 & Labeled & Unlabeled & DSC $\uparrow$ & ASD $\downarrow$ \\
\hline
U-Net (MICCAI'2015) \cite{b22} & 7 (20\%) & 0 & 60.88 & 13.87 \\
 & 35 (100\%) & 0 & 84.76 & 1.58 \\
\hline
CCT (CVPR'2020) \cite{b26} & 7 (20\%) & 28 (80\%) & 71.43 & 16.61 \\
URPC (MICCAI'2021) \cite{b24} & 7 (20\%) & 28 (80\%) & 63.23 & 4.33 \\
SS-Net (MICCAI'2022) \cite{b21} & 7 (20\%) & 28 (80\%) & 62.31 & 4.36 \\
SLC-Net (MICCAI'2022) \cite{b27} & 7 (20\%) & 28 (80\%) & 68.31 & 4.69 \\
SCP-Net (MICCAI'2023) \cite{b6} & 7 (20\%) & 28 (80\%) & 77.06 & 3.52 \\
BCP (CVPR'2023) \cite{b3} & 7 (20\%) & 28 (80\%) & 75.54 & 2.88 \\
SAMT-PCL (ESA'2024) \cite{b29} & 7 (20\%) & 28 (80\%) & 78.35 & \textbf{1.81} \\
CorrMatch (CVPR'2024) \cite{b28} & 7 (20\%) & 28 (80\%) & 79.77 & 4.31 \\
ABD (CVPR'2024) \cite{b5} & 7 (20\%) & 28 (80\%) & 77.52 & 2.23 \\
\textbf{CAD (Ours)} & 7 (20\%) & 28 (80\%) & \textbf{79.80} & 2.01 \\
\hline
\end{tabular}
}
\label{tab:promise12_comparison}
\end{center}
\end{table}

\subsection{Ablation Studies}

% 为了验证各个component的有效性,我们在ACDC数据集上用10%的labeled data进行Ablation study,如表3所示。在表3中,第一列BL代表baseline,即不采用任何replacement策略。第二列和第三列的LLCR和DTE分别代表Largest Low-Confidence Region Replacement和Dynamic Threshold Escalation策略。

% 在ABD中,Hangyang C et. al对于无标注数据,在替换之前,先分别计算了置信度最高的Ktop个patch与候选低置信度patch的KL divergence,并选取与低置信度KL divergence最小的patch,即输出分布最相近的patch进行替换。我们采用了类似的做法进行消融分析,在找到低置信度区域后,将其与Ktop个高置信度区域计算KL div,

% (公式)

% where xxx代表低置信度区域的logits,xxx代表Ktop中第i个高置信度区域的logits。最后选取xxx最低的区域进行替换。这种策略在表三的第四列记作KL。


\subsubsection{Effectiveness of Each Module in CAD}
To validate the effectiveness of each component, we conducted an ablation study on the ACDC dataset using 10\% of labeled data, as shown in Table.~\ref{tab:effectiveness}. In Table.~\ref{tab:effectiveness}, the first column, \textbf{BL}, represents the baseline model, which does not apply any replacement strategy. The second column, \textbf{LLCR}, and the third column, \textbf{DTE}, denote the Largest Low-Confidence Region Replacement and Dynamic Threshold Escalation strategies, respectively.

In the ABD\cite{b5}, for unlabeled data, the replacement process first calculates the Kullback-Leibler (KL) divergence\cite{b30} between the top \(K_\text{top}\) high-confidence patches and the candidate low-confidence patches. The patch with the minimum KL divergence, representing the closest output distribution, is then selected for replacement. Following a similar approach, we performed an ablation analysis in our study. After identifying the low-confidence region, we calculated its KL divergence with the logits of the \(K_\text{top}\) high-confidence regions as follows:

\begin{equation}
\mathrm{KL}(\mathbf{z}_\mathrm{low}, \mathbf{z}^\mathrm{i}_{high}) = \sum_c \mathrm{softmax}(\mathbf{z}_\mathrm{low})_c \log \frac{\mathrm{softmax}(\mathbf{z}_\mathrm{low})_c}{\mathrm{softmax}(\mathbf{z}^\mathrm{i}_{high})_c},
\end{equation}


where \(\mathbf{z}_\mathrm{low}\) and \(\mathbf{z}^\mathrm{i}_{high}\) represent the logits of the low-confidence region and the \(i\)-th high-confidence region in \(K_\text{top}\), respectively. Finally, the high-confidence region with the smallest KL divergence is selected for replacement. This strategy is abbreviated as \textbf{KL} in the fourth column of Table.~\ref{tab:effectiveness}.



\begin{table}[htbp]
\caption{Ablation studies on the ACDC dataset with 10\% labeled data, validating the effectiveness of each component.}
\begin{center}
\begin{tabular}{|c c c c|c c c c|}
\hline
\textbf{BL} & \textbf{LLCR} & \textbf{DTE} & \textbf{KL} & \textbf{DSC$\uparrow$} & \textbf{Jaccard$\uparrow$} & \textbf{95HD$\downarrow$} & \textbf{ASD$\downarrow$} \\
\hline
\checkmark &  &  &  & 89.03 & 80.79 & 3.34 & 0.94 \\
\checkmark & \checkmark &  &  & 89.23 & 81.69 & 2.44 & 0.82 \\
\checkmark & \checkmark & \checkmark &  & \textbf{90.38} & \textbf{82.89} & \textbf{1.29} & \textbf{0.36} \\
\checkmark & \checkmark & \checkmark & \checkmark & 89.69 & 81.78 & 1.82 & 0.69 \\

\hline
\end{tabular}
\label{tab:effectiveness}
\end{center}
\end{table}

% 结果表明,我们提出的CAD中,LLCR和DTE策略均在模型训练中产生了积极作用,在强扰动下能够提升模型的学习能力。而KL stratagy在CAD中对模型的性能提升并没有显著的积极作用,因此,在我们的CAD中并不包含这一component。
The results demonstrate that the proposed CAD method benefits significantly from the inclusion of both the LLCR and DTE strategies, which play a positive role in enhancing the model's learning capability under strong perturbations. In contrast, the KL strategy does not show a notable positive impact on model performance within the CAD framework. Therefore, this component is not incorporated into our final CAD approach.


\subsubsection{Selection of Threshold Hyperparameters in DTE}
% 表四展示了在在DTE策略中,$C_\text{min}$、$C_\text{max}$、$R_\text{min}$和$R_\text{max}$四个代表允许纳入低置信度联通域$\mathcal{R}_\mathrm{low}$的patch的置信度最小值和最大值以及$\mathcal{R}_\mathrm{low}$允许的最大patch数的最小值和最大值的超参数的不同选择对模型性能的影响。可以看出,在$C_\text{min}$、$C_\text{max}$、$R_\text{min}$、$R_\text{max}$分别为0.01,0.75,1,16时,模型表现最佳
Table.~\ref{tab:hyperparameters} illustrates the impact of different choices for the hyperparameters \(C_\text{min}\), \(C_\text{max}\), \(R_\text{min}\), and \(R_\text{max}\) on the model's performance within the DTE strategy. These hyperparameters define the minimum and maximum confidence thresholds for patches to be included in the low-confidence connected region \(\mathcal{R}_\mathrm{low}\), as well as the minimum and maximum allowable number of patches in \(\mathcal{R}_\mathrm{low}\). As shown in the table, the model achieves the best performance when \(C_\text{min}\), \(C_\text{max}\), \(R_\text{min}\), and \(R_\text{max}\) are set to 0.01, 0.75, 1, and 16, respectively.


\begin{table}[htbp]
\caption{Ablation Study of Threshold Hyperparameters on ACDC Dataset with 10\% Labeled Data.}
\begin{center}
\resizebox{0.5\textwidth}{!}{
\begin{tabular}{|c c c c|c c c c|}
\hline
\textbf{$C_\text{min}$} & \textbf{$C_\text{max}$} & \textbf{$R_\text{min}$} & \textbf{$R_\text{max}$} & \textbf{DSC$\uparrow$} & \textbf{Jaccard$\uparrow$} & \textbf{95HD$\downarrow$} & \textbf{ASD$\downarrow$} \\
\hline
0.1 & 0.9 & 4 & 32 & 88.22 & 79.76 & 4.77 & 3.00 \\
0.05 & 0.9 & 4 & 32 & 88.34 & 80.01 & 3.97 & 3.21 \\
0.01 & 0.9 & 4 & 32 & 88.43 & 79.89 & 2.44 & 2.56 \\
0.01 & 0.6 & 4 & 32 & 88.27 & 80.29 & 2.37 & 1.74 \\
0.01 & 0.75 & 4 & 32 & 89.14 & 81.21 & 1.76 & 0.87 \\
0.01 & 0.75 & 8 & 32 & 88.55 & 79.79 & 3.24 & 2.94 \\
0.01 & 0.75 & 1 & 32 & 89.69 & 81.79 & 1.97 & 0.51 \\
0.01 & 0.75 & 1 & 8 & 89.83 & 81.83 & 1.44 & \textbf{0.34} \\

0.01 & 0.75 & 1 & 16 & \textbf{90.38} & \textbf{82.89} & \textbf{1.29} & 0.36 \\

\hline
\end{tabular}
}
\label{tab:hyperparameters}
\end{center}
\end{table}


\section{Conclusion}
In this work, we proposed the Confidence-Aware Displacement (CAD) strategy for semi-supervised medical image segmentation, incorporating the Largest Low-Confidence Region Replacement and Dynamic Threshold Escalation strategies. By adaptively identifying and replacing low-confidence regions, our approach effectively enhances the model's learning from unlabeled data while ensuring stability during training. Extensive experiments on the ACDC and PROMISE12 datasets demonstrate the superiority of CAD, achieving state-of-the-art performance with limited labeled data and outperforming other methods. These results highlight the effectiveness of CAD in using uncertainty to refine segmentation predictions, providing a robust framework for improving semi-supervised segmentation tasks in medical imaging.





% \section{Introduction}
% This document is a model and instructions for \LaTeX.
% Please observe the conference page limits. 

% \section{Ease of Use}

% \subsection{Maintaining the Integrity of the Specifications}

% The IEEEtran class file is used to format your paper and style the text. All margins, 
% column widths, line spaces, and text fonts are prescribed; please do not 
% alter them. You may note peculiarities. For example, the head margin
% measures proportionately more than is customary. This measurement 
% and others are deliberate, using specifications that anticipate your paper 
% as one part of the entire proceedings, and not as an independent document. 
% Please do not revise any of the current designations.

% \section{Prepare Your Paper Before Styling}
% Before you begin to format your paper, first write and save the content as a 
% separate text file. Complete all content and organizational editing before 
% formatting. Please note sections \ref{AA}--\ref{SCM} below for more information on 
% proofreading, spelling and grammar.

% Keep your text and graphic files separate until after the text has been 
% formatted and styled. Do not number text heads---{\LaTeX} will do that 
% for you.

% \subsection{Abbreviations and Acronyms}\label{AA}
% Define abbreviations and acronyms the first time they are used in the text, 
% even after they have been defined in the abstract. Abbreviations such as 
% IEEE, SI, MKS, CGS, ac, dc, and rms do not have to be defined. Do not use 
% abbreviations in the title or heads unless they are unavoidable.

% \subsection{Units}
% \begin{itemize}
% \item Use either SI (MKS) or CGS as primary units. (SI units are encouraged.) English units may be used as secondary units (in parentheses). An exception would be the use of English units as identifiers in trade, such as ``3.5-inch disk drive''.
% \item Avoid combining SI and CGS units, such as current in amperes and magnetic field in oersteds. This often leads to confusion because equations do not balance dimensionally. If you must use mixed units, clearly state the units for each quantity that you use in an equation.
% \item Do not mix complete spellings and abbreviations of units: ``Wb/m\textsuperscript{2}'' or ``webers per square meter'', not ``webers/m\textsuperscript{2}''. Spell out units when they appear in text: ``. . . a few henries'', not ``. . . a few H''.
% \item Use a zero before decimal points: ``0.25'', not ``.25''. Use ``cm\textsuperscript{3}'', not ``cc''.)
% \end{itemize}

% \subsection{Equations}
% Number equations consecutively. To make your 
% equations more compact, you may use the solidus (~/~), the exp function, or 
% appropriate exponents. Italicize Roman symbols for quantities and variables, 
% but not Greek symbols. Use a long dash rather than a hyphen for a minus 
% sign. Punctuate equations with commas or periods when they are part of a 
% sentence, as in:
% \begin{equation}
% a+b=\gamma\label{eq}
% \end{equation}

% Be sure that the 
% symbols in your equation have been defined before or immediately following 
% the equation. Use ``\eqref{eq}'', not ``Eq.~\eqref{eq}'' or ``equation \eqref{eq}'', except at 
% the beginning of a sentence: ``Equation \eqref{eq} is . . .''

% \subsection{\LaTeX-Specific Advice}

% Please use ``soft'' (e.g., \verb|\eqref{Eq}|) cross references instead
% of ``hard'' references (e.g., \verb|(1)|). That will make it possible
% to combine sections, add equations, or change the order of figures or
% citations without having to go through the file line by line.

% Please don't use the \verb|{eqnarray}| equation environment. Use
% \verb|{align}| or \verb|{IEEEeqnarray}| instead. The \verb|{eqnarray}|
% environment leaves unsightly spaces around relation symbols.

% Please note that the \verb|{subequations}| environment in {\LaTeX}
% will increment the main equation counter even when there are no
% equation numbers displayed. If you forget that, you might write an
% article in which the equation numbers skip from (17) to (20), causing
% the copy editors to wonder if you've discovered a new method of
% counting.

% {\BibTeX} does not work by magic. It doesn't get the bibliographic
% data from thin air but from .bib files. If you use {\BibTeX} to produce a
% bibliography you must send the .bib files. 

% {\LaTeX} can't read your mind. If you assign the same label to a
% subsubsection and a table, you might find that Table I has been cross
% referenced as Table IV-B3. 

% {\LaTeX} does not have precognitive abilities. If you put a
% \verb|\label| command before the command that updates the counter it's
% supposed to be using, the label will pick up the last counter to be
% cross referenced instead. In particular, a \verb|\label| command
% should not go before the caption of a figure or a table.

% Do not use \verb|\nonumber| inside the \verb|{array}| environment. It
% will not stop equation numbers inside \verb|{array}| (there won't be
% any anyway) and it might stop a wanted equation number in the
% surrounding equation.

% \subsection{Some Common Mistakes}\label{SCM}
% \begin{itemize}
% \item The word ``data'' is plural, not singular.
% \item The subscript for the permeability of vacuum $\mu_{0}$, and other common scientific constants, is zero with subscript formatting, not a lowercase letter ``o''.
% \item In American English, commas, semicolons, periods, question and exclamation marks are located within quotation marks only when a complete thought or name is cited, such as a title or full quotation. When quotation marks are used, instead of a bold or italic typeface, to highlight a word or phrase, punctuation should appear outside of the quotation marks. A parenthetical phrase or statement at the end of a sentence is punctuated outside of the closing parenthesis (like this). (A parenthetical sentence is punctuated within the parentheses.)
% \item A graph within a graph is an ``inset'', not an ``insert''. The word alternatively is preferred to the word ``alternately'' (unless you really mean something that alternates).
% \item Do not use the word ``essentially'' to mean ``approximately'' or ``effectively''.
% \item In your paper title, if the words ``that uses'' can accurately replace the word ``using'', capitalize the ``u''; if not, keep using lower-cased.
% \item Be aware of the different meanings of the homophones ``affect'' and ``effect'', ``complement'' and ``compliment'', ``discreet'' and ``discrete'', ``principal'' and ``principle''.
% \item Do not confuse ``imply'' and ``infer''.
% \item The prefix ``non'' is not a word; it should be joined to the word it modifies, usually without a hyphen.
% \item There is no period after the ``et'' in the Latin abbreviation ``et al.''.
% \item The abbreviation ``i.e.'' means ``that is'', and the abbreviation ``e.g.'' means ``for example''.
% \end{itemize}
% An excellent style manual for science writers is \cite{b7}.

% \subsection{Authors and Affiliations}
% \textbf{The class file is designed for, but not limited to, six authors.} A 
% minimum of one author is required for all conference articles. Author names 
% should be listed starting from left to right and then moving down to the 
% next line. This is the author sequence that will be used in future citations 
% and by indexing services. Names should not be listed in columns nor group by 
% affiliation. Please keep your affiliations as succinct as possible (for 
% example, do not differentiate among departments of the same organization).

% \subsection{Identify the Headings}
% Headings, or heads, are organizational devices that guide the reader through 
% your paper. There are two types: component heads and text heads.

% Component heads identify the different components of your paper and are not 
% topically subordinate to each other. Examples include Acknowledgments and 
% References and, for these, the correct style to use is ``Heading 5''. Use 
% ``figure caption'' for your Figure captions, and ``table head'' for your 
% table title. Run-in heads, such as ``Abstract'', will require you to apply a 
% style (in this case, italic) in addition to the style provided by the drop 
% down menu to differentiate the head from the text.

% Text heads organize the topics on a relational, hierarchical basis. For 
% example, the paper title is the primary text head because all subsequent 
% material relates and elaborates on this one topic. If there are two or more 
% sub-topics, the next level head (uppercase Roman numerals) should be used 
% and, conversely, if there are not at least two sub-topics, then no subheads 
% should be introduced.

% \subsection{Figures and Tables}
% \paragraph{Positioning Figures and Tables} Place figures and tables at the top and 
% bottom of columns. Avoid placing them in the middle of columns. Large 
% figures and tables may span across both columns. Figure captions should be 
% below the figures; table heads should appear above the tables. Insert 
% figures and tables after they are cited in the text. Use the abbreviation 
% ``Fig.~\ref{fig}'', even at the beginning of a sentence.

% \begin{table}[htbp]
% \caption{Table Type Styles}
% \begin{center}
% \begin{tabular}{|c|c|c|c|}
% \hline
% \textbf{Table}&\multicolumn{3}{|c|}{\textbf{Table Column Head}} \\
% \cline{2-4} 
% \textbf{Head} & \textbf{\textit{Table column subhead}}& \textbf{\textit{Subhead}}& \textbf{\textit{Subhead}} \\
% \hline
% copy& More table copy$^{\mathrm{a}}$& &  \\
% \hline
% \multicolumn{4}{l}{$^{\mathrm{a}}$Sample of a Table footnote.}
% \end{tabular}
% \label{tab1}
% \end{center}
% \end{table}

% \begin{figure}[htbp]
% \centerline{\includegraphics{fig1.png}}
% \caption{Example of a figure caption.}
% \label{fig}
% \end{figure}

% Figure Labels: Use 8 point Times New Roman for Figure labels. Use words 
% rather than symbols or abbreviations when writing Figure axis labels to 
% avoid confusing the reader. As an example, write the quantity 
% ``Magnetization'', or ``Magnetization, M'', not just ``M''. If including 
% units in the label, present them within parentheses. Do not label axes only 
% with units. In the example, write ``Magnetization (A/m)'' or ``Magnetization 
% \{A[m(1)]\}'', not just ``A/m''. Do not label axes with a ratio of 
% quantities and units. For example, write ``Temperature (K)'', not 
% ``Temperature/K''.

% \section*{Acknowledgment}

% The preferred spelling of the word ``acknowledgment'' in America is without 
% an ``e'' after the ``g''. Avoid the stilted expression ``one of us (R. B. 
% G.) thanks $\ldots$''. Instead, try ``R. B. G. thanks$\ldots$''. Put sponsor 
% acknowledgments in the unnumbered footnote on the first page.

% \section*{References}

% Please number citations consecutively within brackets \cite{b1}. The 
% sentence punctuation follows the bracket \cite{b2}. Refer simply to the reference 
% number, as in \cite{b3}---do not use ``Ref. \cite{b3}'' or ``reference \cite{b3}'' except at 
% the beginning of a sentence: ``Reference \cite{b3} was the first $\ldots$''

% Number footnotes separately in superscripts. Place the actual footnote at 
% the bottom of the column in which it was cited. Do not put footnotes in the 
% abstract or reference list. Use letters for table footnotes.

% Unless there are six authors or more give all authors' names; do not use 
% ``et al.''. Papers that have not been published, even if they have been 
% submitted for publication, should be cited as ``unpublished'' \cite{b4}. Papers 
% that have been accepted for publication should be cited as ``in press'' \cite{b5}. 
% Capitalize only the first word in a paper title, except for proper nouns and 
% element symbols.

% For papers published in translation journals, please give the English 
% citation first, followed by the original foreign-language citation \cite{b6}.
\bibliography{CAD.bib}
% \begin{thebibliography}{00}
% \bibitem{b1} G. Eason, B. Noble, and I. N. Sneddon, ``On certain integrals of Lipschitz-Hankel type involving products of Bessel functions,'' Phil. Trans. Roy. Soc. London, vol. A247, pp. 529--551, April 1955.
% \bibitem{b2} J. Clerk Maxwell, A Treatise on Electricity and Magnetism, 3rd ed., vol. 2. Oxford: Clarendon, 1892, pp.68--73.
% \bibitem{b3} I. S. Jacobs and C. P. Bean, ``Fine particles, thin films and exchange anisotropy,'' in Magnetism, vol. III, G. T. Rado and H. Suhl, Eds. New York: Academic, 1963, pp. 271--350.
% \bibitem{b4} K. Elissa, ``Title of paper if known,'' unpublished.
% \bibitem{b5} R. Nicole, ``Title of paper with only first word capitalized,'' J. Name Stand. Abbrev., in press.
% \bibitem{b6} Y. Yorozu, M. Hirano, K. Oka, and Y. Tagawa, ``Electron spectroscopy studies on magneto-optical media and plastic substrate interface,'' IEEE Transl. J. Magn. Japan, vol. 2, pp. 740--741, August 1987 [Digests 9th Annual Conf. Magnetics Japan, p. 301, 1982].
% \bibitem{b7} M. Young, The Technical Writer's Handbook. Mill Valley, CA: University Science, 1989.
% \end{thebibliography}
\vspace{12pt}
% \color{red}
% IEEE conference templates contain guidance text for composing and formatting conference papers. Please ensure that all template text is removed from your conference paper prior to submission to the conference. Failure to remove the template text from your paper may result in your paper not being published.

\end{document}
