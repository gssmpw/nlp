\documentclass{article}

% \includeonly{Main}

\usepackage{arxiv}

\usepackage[utf8]{inputenc} % allow utf-8 input
\usepackage[T1]{fontenc}    % use 8-bit T1 fonts
\usepackage{hyperref}       % hyperlinks
\usepackage{url}            % simple URL typesetting
\usepackage{booktabs}       % professional-quality tables
\usepackage{amsfonts}       % blackboard math symbols
\usepackage{nicefrac}       % compact symbols for 1/2, etc.
\usepackage{microtype}      % microtypography
\usepackage{graphicx}
\usepackage{natbib}
\usepackage{doi}
\usepackage{amsmath}  % For advanced mathematical notation
\usepackage{amssymb}  % For additional mathematical symbols
\usepackage{hyperref}
\usepackage{svg}
\usepackage{multirow}
\usepackage{caption} % Required for \captionsetup
\usepackage{makecell}

\title{Shifting Attention to You: 

Personalized Brain-Inspired AI Models}

\author{
Stephen Chong Zhao \\
Data Science Institute \\
Vanderbilt University \\
Nashville, United States \\
\texttt{chong.zhao.1@vanderbilt.edu}  \\
  %% examples of more authors
   \And
 Yang Hu \\
 Data Science Institute\\
 Vanderbilt University \\
 Nashville, United States \\
  \texttt{yang.hu.1@vanderbilt.Edu} \\
   \And
 Jason Lee \\
 Department of Computer Science\\
 Vanderbilt University \\
 Nashville, United States \\
  \texttt{jason.j.lee@vanderbilt.edu} \\
     \And
 Andrew Bender \\
  Neurosciences Graduate Program\\
  University of California, San Diego\\
  San Diego, United States\\
  \texttt{abender@health.ucsd.edu} \\
     \And
 Trisha Mazumdar \\
  Department of Computer Science\\
  Vanderbilt University\\
  Nashville, United States \\
  \texttt{trisha.mazumdar@vanderbilt.edu} \\
     \And
 Mark Wallace \\
  Department of Psychology\\
  Vanderbilt University\\
 Nashville, United States \\
  \texttt{mark.wallace@vanderbilt.edu} \\
     \And
 David A. Tovar \\
  Department of Psychology\\
  Vanderbilt University\\
  Nashville, United States \\
  \texttt{david.tovar@vanderbilt.edu} \\}
  


% Uncomment to remove the date
\date{}

% Uncomment to override the `A preprint' in the header
\renewcommand{\headeright}{}
\renewcommand{\undertitle}{}
\renewcommand{\shorttitle}{Personalized Brain-Inspired AI Models}

%%% Add PDF metadata to help others organize their library
\hypersetup{
pdftitle={Shifting Attention to You: Personalized Brain-Inspired AI Models},
pdfsubject={q-bio.NC, q-bio.QM},
pdfauthor={Stephen Chong Zhao, Yang Hu, Jason Lee, Andrew Bender, Trisha Mazumdar, Mark Wallace, David A. Tovar},
pdfkeywords={Personalization, Artificial Intelligence, Cognition, Transformers, Magnetoencephalography},
}

\begin{document}

\maketitle


\begin{abstract}
The integration of human and artificial intelligence represents a scientific opportunity to advance our understanding of information processing, as each system offers unique computational insights that can enhance and inform the other. The synthesis of human cognitive principles with artificial intelligence has the potential to produce more interpretable and functionally aligned computational models, while simultaneously providing a formal framework for investigating the neural mechanisms underlying perception, learning, and decision-making through systematic model comparisons and representational analyses. In this study, we introduce personalized brain-inspired modeling that integrates human behavioral embeddings and neural data to align with cognitive processes. We took a stepwise approach, fine-tuning the Contrastive Language–Image Pre-training (CLIP) model with large-scale behavioral decisions, group-level neural data, and finally, participant-level neural data within a broader framework that we have named CLIP-Human-Based Analysis (CLIP-HBA). We found that fine-tuning on behavioral data enhances its ability to predict human similarity judgments while indirectly aligning it with dynamic representations captured via magnetoencephalography (MEG). To further gain mechanistic insights into the temporal evolution of cognitive processes, we introduced a model specifically fine-tuned on millisecond-level MEG neural dynamics (CLIP-HBA-MEG). This model resulted in enhanced temporal alignment with human neural processing while still showing improvement on behavioral alignment. Finally, we trained individualized models on participant-specific neural data, effectively capturing individualized neural dynamics and highlighting the potential for personalized AI systems. These personalized systems have far-reaching implications for the fields of medicine, cognitive research, human-computer interfaces, and AI development.
\end{abstract}

\keywords{Personalization \and Artificial Intelligence \and Cognition \and Transformers \and Magnetoencephalography}

\section{Introduction}

Artificial intelligence (AI) systems are fundamentally composed of four key components: the model architecture that structures the processing mechanism, the learning rules that guide parameter updates, the objective functions that shape the learning path, and the datasets on which they are trained \citep{goodfellow2016deep, saxe2021deep}. While considerable progress has been made in each of these components and created much improved AI systems, aligning AI with human cognition remains a significant challenge \citep{ferrante2024neuralfoundationmodelsvision, lu2024achieving, rane2024conceptalignment}. Such alignment holds the promise of not only enhancing the interpretability and robustness of AI models but also of deepening our understanding of human cognitive processes \citep{mahner_dimensions_2024}.

Humans often rely on semantic abstraction and operate across varying levels of semantic categorization, making it essential for AI models to excel in both superordinate and fine-grained tasks \citep{goldstone_role_1994,SchulteImWalde2022,born2024evaluating}. However, the representational strategies employed by humans and those utilized by AIs  differ, highlighting the need for more sophisticated approaches to bridge this gap \citep{mahner_dimensions_2024}. For instance, images that are indistinguishable to models--which form model metamers in later stages of neural networks--often appear obviously different to humans, demonstrating the presence of idiosyncratic, model-specific invariances that differ from the organization and representations of human sensory systems \citep{feather_model_2023}.

While the shared architectural principles between deep neural networks (DNNs) and human cortical processing \citep{cichy_comparison_2016} play a role in brain alignment, recent studies suggest that the path to better human alignment may lie more in how models are trained than in their fundamental architecture \citep{muttenthaler_human_2023}. The choice of training components, such as the objective function, significantly impacts the learning of human-like invariances \citep{nanda_invariances_2023}. Thus, one way to make models more human is to incorporate human inductive biases into AI systems to mirror cognitive processes. Indeed, leveraging these biases through Bayesian model selection and contrastive learning has been shown to enhance human alignment and machine learning performance \citep{Sinz2019,marjieh_universal_2023,marjieh_using_2024}. 

In this study, we aim to enhance the alignment between AIs and human cognition by fine-tuning the CLIP model \citep{radford_learning_2021} using human behavioral embeddings and neural data. Notably, the behavioral embeddings, derived from large-scale behavioral modeling \citep{hebart_things_2019,hebart_things-data_2023}, provide interpretable dimensions that reflect human mental representations of objects. By training the model to incorporate this human-derived mental embedding space, we enable it to predict human behavior and neural representations more accurately, while also achieving significantly enhanced performance on alignment benchmarks \citep{fu2023dreamsimlearningnewdimensions}.

While large-scale behavioral embeddings offer a fixed output of generalizable perceptual representations, neural data provides more dynamic insights that can reflect the millisecond-level temporal evolution of human brain representations \citep{Ruffle_2024}. Moreover, neural data has the potential to better capture individual variability across diverse populations where behavior may be challenging to collect (e.g., infants, non-verbal population) \citep{10.1162/netn_a_00387}. To explore this, we further introduced a dynamic neural fine-tuning process using MEG signals (CLIP-HBA-MEG) to better align the model’s temporal representations with the rapidly changing activity patterns of the human brain.

Through training with group-level MEG representations \citep{hebart_things-data_2023}, we observed that the model generalizes well across external datasets, encompassing various participants, conceptual stimuli, and visual conditions. This finding reinforced our confidence in the robustness and generalizability of the approach, allowing us to move forward to individualized perceptual fine-tuning with participant-specific neural data and capture unique neural dynamics and perceptual differences. Our approach holds promise for creating personalized AI systems that reflect distinct and individual cognitive patterns, bridging the gap between artificial and biological cognition. By embedding these individualized patterns into AI models, we enhance their interpretability and alignment while deepening our understanding of the interplay between minds and machines.


\section{Results}

\begin{figure}[ht]
    \vspace{-10mm}
    \centering
    \includegraphics[width=\textwidth]{Figures/behavioral_model.png}
    \caption{\textbf{Schematic of the CLIP-HBA-Behavior fine-tuning process using behavioral data.} The 66 SPoSE text dimensions are fed into the text encoder, producing 66 text representations $D_1 \dots D_{66}$. Concurrently, visual stimuli from the THINGS dataset are input into the vision encoder, generating their corresponding visual representations $V$. These features from two modalities are bound via a dot product projection, mapping the visual features onto each of the text dimensions to form a 66-dimensional embedding $e$ for each image. Weight-Decomposed Low-Rank Adaptation (DoRA), a parameter efficient fine-tuning (PEFT) method, is used to fine-tune the attention modules of the text and vision encoder, using SPoSE behavioral embedding as an objective.}
    \label{fig:figure1}
\end{figure}

Current AI models trained on large-scale image datasets develop inductive biases based on the statistical patterns within those datasets. However, a significant gap exists between machine-derived inductive biases and the way humans develop theirs through perceptual experience \citep{GoyalBengio2022}. To address this, we introduced human-like inductive biases to the CLIP Model in the form of 66 Sparse Positive Similarity Embedding (SPoSE) dimensions, derived from large-scale behavioral studies \citep{zheng2019revealinginterpretableobjectrepresentations}. These dimensions have been shown to effectively predict human similarity judgments for object distinctions \citep{hebart_revealing_2020}. 

We developed CLIP-HBA-Behavior (Figure \ref{fig:figure1}) by fine-tuning the CLIP model, using the 66 SPoSE dimensions as textual inputs while simultaneously feeding THINGS images \citep{hebart_things_2019} as visual inputs. The CLIP architecture encodes both modalities—textual and visual—separately, before binding them via dot product similarity. This mechanism evaluates the encoded visual features against each of the 66 dimensions, resulting in a 66-dimensional embedding representation for each image input. 

During fine-tuning, the model's predicted embeddings are evaluated against the behavioral SPoSE embeddings using a Mean Squared Error (MSE) loss function. This loss guides the optimization process, updating the attention layers of both the text and visual encoders over multiple training iterations. The goal is to enable the model to accurately predict the SPoSE embedding for a given visual stimulus. In contrast to an off-the-shelf CLIP model, our behaviorally fine-tuned version, CLIP-HBA-Behavior, can produce significantly more human-aligned perceptual outputs. For example, when presented with an image of metal armor (Example Image Embedding in \ref{fig:figure1}), our model identifies the object as highly associated with dimensions such as "metallic," "body-related," and "weapon-related" (top three dimensions), while correctly assigning low relevance to dimensions like "wood-related," "green," or "fluffy" (bottom three dimensions). See \ref{fig:supp1} for more examples.


\subsection{Fine-Tuning CLIP with Human Behavioral Embeddings Enhances Alignment}

\begin{figure}[ht]
    \centering
    \includegraphics[width=\textwidth]{Figures/behavioral_results.png}
    \caption{\textbf{Behavioral and neural alignment of fine-tuned CLIP-HBA-Behavior.} (A) Behavioral results: Representational dissimilarity matrices (RDMs) for 48 objects, predicted by CLIP-HBA-Behavior (top) and CLIP-ViT (bottom), with Spearman rank correlations ($\rho$) of 0.78 and 0.32, respectively. (B) Neural results: Temporal correlations between model-predicted RDMs and MEG RDMs from THINGS (top left) and external datasets under varied conditions (remaining panels). CLIP-HBA-Behavior consistently outperforms CLIP-ViT in correlation and area under the curve (AUC), demonstrating enhanced neural alignment and generalizability.}
    \label{fig:figure2}
\end{figure}

As shown in Figure \ref{fig:figure2}A, we evaluated the behaviorally fine-tuned model's performance on a set of 48 object images from the THINGS database, which were fully sampled in a behavioral odd-one-out task \citep{hebart_revealing_2020}. These 48 objects were excluded from training to prevent data leakage. To assess the alignment between the model's predictions and human behavior, we calculated the Spearman rank correlation ($\rho$) between the model's predicted Representational Dissimilarity Matrix (RDM) \citep{kriegeskorte_representational_2008} and the ground-truth behavioral RDM. The behaviorally fine-tuned CLIP-HBA-Behavior achieved a Spearman correlation of 0.78 (\( p<10^{-229} \)) and a 95\% confidence interval (CI) of [0.75, 0.80]. 

For comparison, we applied the same evaluation procedure to the original CLIP-ViT-L14 model. We extracted the 768-dimensional last-layer activations as the model's best representation for each visual stimulus. The Spearman rank correlation between the CLIP-ViT model's predicted RDM and the fully sampled behavioral RDM was 0.32, with a 95\% CI of [0.26, 0.37]. This result highlights the significant improvement achieved by the CLIP-HBA-Behavior model, demonstrating over 100\% improvement in generalizability and behavioral alignment compared to the baseline.



\subsection{Improved Performance on Alignment Benchmark with CLIP-HBA-Behavior}

\begin{table}[ht]
\centering
\renewcommand{\arraystretch}{1.5} % Adjust this value for row height
\captionsetup{skip=10pt} % Adjust the space between the table and caption
\begin{tabular}{llcc}
\hline
\textbf{Feature Type}      & \textbf{Model Name}         & \textbf{Val Score} & \textbf{Test Score} \\ \hline
\multirow{2}{*}{768-d Visual Features} 
                           & \textbf{CLIP-HBA-Behavior}          & 0.88             & 0.88              \\ 
                           & CLIP-ViT-L/14               & 0.81             & 0.81              \\ \hline
\multirow{2}{*}{66-d SPoSE Dimensions} 
                           & \textbf{CLIP-HBA-Behavior}          & 0.85             & 0.84              \\ 
                           & CLIP-ViT-L/14               & 0.80             & 0.79              \\ \hline
\end{tabular}
\caption{\textbf{Performance benchmarking on the NIGHT triplets dataset. }The table compares validation (Val) and test scores for behaviorally fine-tuned CLIP-HBA-Behavior and the baseline CLIP-ViT-L/14 across two feature types: the 66-dimensional SPoSE semantic embedding space and the 768-dimensional visual features. CLIP-HBA-Behavior demonstrates a notable performance enhancement in generalizing to triplet-out decisions after behavioral fine-tuning. }

\label{tab:nights-benchmark}
\end{table}

To test our model's robustness and performance, we benchmarked it against a state-of-the-art human visual similarity judgment metric—NIGHTS (Novel Image Generations with Human-Tested Similarity), a benchmark of 20,019 image triplets with human-tested perceptual similarity scores \citep{fu2023dreamsimlearningnewdimensions}. 

The results in Table \ref{tab:nights-benchmark} illustrate the significant improvements achieved by behaviorally fine-tuning CLIP-HBA-Behavior. To comprehensively evaluate the model, we tested two feature types. First, we assessed the 768-dimensional last layer activations of the Vision Transformer (ViT) encoder's alignment with human perception.  Second, we assessed the 66-dimensional SPoSE embedding space's perceptual alignment. When using last layer visual features, CLIP-HBA-Behavior achieved validation and test scores of 0.88, surpassing the baseline score of 0.81. With the SPoSE dimensions, the model achieved validation and test scores of 0.85 and 0.84, compared to the baseline scores of 0.80 and 0.79. These results highlight that behavioral fine-tuning enhances human perceptual alignment and establishes CLIP-HBA-Behavior as a robust, generalizable model for nuanced perceptual distinctions.

\subsection{Improved Neural Alignment with CLIP-HBA-Behavior}

While the enhanced behavioral correlation of our model was anticipated due to its training on behavioral data, understanding whether the model's embedding space truly becomes more human-like requires evaluating how training exclusively on behavioral data impacts its alignment with neural representations. Therefore, we further evaluated the alignment between the model's representations against neural data obtained from MEG recordings \citep{hebart_things-data_2023}. 

As shown in Figure \ref{fig:figure2}B (top left), CLIP-HBA-Behavior demonstrated a stronger and more sustained temporal correlation with neural data across time points compared to the baseline CLIP model. Additionally, CLIP-HBA-Behavior's correlation pattern and area under the curve (AUC) closely matched—and slightly exceeded—that of the SPoSE embedding model, whereas the CLIP-ViT baseline exhibited notably lower correlations and AUC.


To evaluate generalizability, we tested CLIP-HBA-Behavior on three external datasets containing different participants' neural responses to out-of-distribution object stimuli under varying visual conditions: 

\begin{itemize}
    \item \textbf{Colored Images without Background} (Figure \ref{fig:figure2}B, top right): Clear object stimuli with backgrounds featuring humans, animals, fruits, and man-made objects \citep{grootswagers2017hyperalignment}. 
    \item \textbf{Monochromatic Clear Images} (Figure \ref{fig:figure2}B, bottom left): Clear, monochromatic images of animate and inanimate objects \citep{10.1162/jocn_a_01068}.
    \item \textbf{Monochromatic Images with Blur} (Figure \ref{fig:figure2}B, bottom right): Blurry, monochromatic images of animate and inanimate objects \citep{10.1162/jocn_a_01068}.
\end{itemize}

In all cases, CLIP-HBA-Behavior outperformed baseline CLIP-ViT in terms of AUC. Notably, CLIP-HBA-Behavior consistently peaked in neural alignment around 300–400 ms after stimulus onset. This temporal pattern likely reflects the model's fine-tuning on behavioral decisions and object semantics, which typically emerge later in processing, in contrast to earlier stages related to primary visual processing. 

Overall, CLIP-HBA-Behavior demonstrated enhanced neural alignment across all datasets, spanning diverse image quality conditions and participant groups. Additionally, the performance benchmarking on the NIGHTS dataset highlights the significant improvements achieved by behavioral fine-tuning, with CLIP-HBA-Behavior outperforming the baseline model across both high-dimensional visual features and semantic embeddings. These results underscore that fine-tuning with human behavioral data not only improves the model's ability to capture human cognitive representations but also establishes its robustness and performance in large-scale perceptual tasks. Additionally, we have found that behavioral fine-tuning using this pipeline can be effective with a behavioral dataset as small as 100 visual stimuli, making this method agile and suitable for specialized datasets of nuanced population groups that are likely more difficult to scale \ref{fig:supp3}. 


\subsection{Dynamic Neural Fine-Tuning Captures Millisecond-Level Neural Dynamics}

\begin{figure}[ht]
    \centering
    \vspace{-5mm} 
    \includegraphics[width=\textwidth]{Figures/dynamic_model.png}
    \caption{\textbf{Schematic of the CLIP-HBA-MEG fine-tuning process using neural signals.} A Feature Reweighting Matrix, pre-optimized at initialization, dynamically computes weighted combinations of vision encoder layer activations to align with neural decoding RDMs. Temporal scalers, $\alpha_T$ and $\beta_T$, respectively, modulate the magnitude of visual feature aggregation and the binding of visual-semantic features. Dimension-wise Gaussian noise is added to the embedding post-binding with a dynamically controlled noise level, mimicking varying stability of human neural responses while preventing model training from overfitting to specific noisy time points. Predicted RDMs are compared against MEG target RDMs via a custom loss function, updating the vision encoder’s attention layers and adapting the feature reweighting matrix. The text encoder is frozen with behaviorally pre-trained weights to ensure stable semantic features of text representations.}
    \label{fig:figure3}
\end{figure}


Building on the success of perceptual fine-tuning with behavioral data, which are large-scale, generalizable, and collected across many trials and participants, we sought to explore whether the model can improve by learning directly from human neural representations. Although typically collected from smaller participant samples, neural data can often be better suited to capture diverse neural profiles across different populations \citep{Ye_2021}. While behavioral embeddings provide static and generalizable perceptual representations, neural data offer dynamic insights, reflecting millisecond-level temporal patterns of human brain activity that cannot be captured through behavioral measures alone. Furthermore, neural data have the potential to better account for individual variability, providing a more personalized lens into how visual stimuli are processed \citep{Pagan2024}. To explore these possibilities, we developed a dynamic neural fine-tuning process using MEG signals, as demonstrated in Figure \ref{fig:figure3}. The dynamically fine-tuned model, referred to as CLIP-HBA-MEG, adapts its visual stimulus embeddings to align with distinct stages of visual processing and the emergence of semantic representations.


\subsection{Enhanced Behavioral and Neural Alignment after Dynamic Neural Fine-tuning}

\begin{figure}[ht]
    \centering
    \includegraphics[width=\textwidth]{Figures/neural_results.png}
    \caption{\textbf{Behavioral and neural alignment of fine-tuned CLIP-HBA-MEG.} (A) Behavioral validation: Comparison of the dynamic embedding space of the CLIP-HBA-MEG model across all timepoints (purple bars) with the THINGS behavioral data of 48 sample objects. The static behavioral alignment of the baseline CLIP-ViT model (green line) and the behaviorally fine-tuned CLIP-HBA-Behavior (red line) from Figure \ref{fig:figure2} are also plotted. CLIP-HBA-MEG demonstrates sustained higher behavioral alignment than the baseline soon after stimulus onset but peaks below the alignment achieved by CLIP-HBA-Behavior. (B) Neural validations: Temporal correlations between CLIP-HBA-MEG model-predicted RDMs and MEG RDMs are shown for THINGS images (top left panel) and external datasets under varied conditions: clear objects without background, clear monochromatic images, and blurry monochromatic images (remaining panels). CLIP-HBA-MEG consistently outperforms CLIP-ViT in correlation strength and area under the curve (AUC) across all tested conditions.}
    \label{fig:figure4}
\end{figure}

We evaluated the behavioral and neural validity of the CLIP-HBA-MEG model after fine-tuning with MEG responses from participants viewing visual stimuli in the THINGS image database. Figure \ref{fig:figure4}A illustrates the temporal alignment of the dynamic embedding space of our model against the SPoSE behavioral embedding. Behavioral alignment increased post-stimulus onset, peaking around 600 milliseconds. The peak alignment ($\rho = 0.65$) significantly surpasses the baseline CLIP-ViT's static alignment ($\rho = 0.32$, Figure \ref{fig:figure2}A) but remains below the alignment achieved by the behaviorally fine-tuned CLIP-HBA model ($\rho = 0.78$). These results suggest that, while training with neural data does not match the behavioral performance of the model compared to learning directly from behavioral data, it still significantly improves behavioral alignment over the baseline.

To assess the model's robustness and generalizability in neural alignment, we tested it on the same three external visual datasets with varying characteristics and subjects used in Figure \ref{fig:figure2}B.

As shown in Figure \ref{fig:figure4}B, the neurally fine-tuned CLIP-HBA-MEG model with its dynamic embedding achieved higher overall neural alignment across all tested datasets, presenting higher peak alignment and overall AUC, compared to baseline CLIP-ViT's static representation.

Overall, these findings confirm that the CLIP-HBA-MEG model not only maintains effectiveness in behavioral alignment, but also generalizes effectively across diverse visual conditions, aligning well with human perception and neural responses. However, we observed that the model's neural alignment enhancement was less pronounced on degraded image datasets. This limitation likely stems from the neural fine-tuning process, which exclusively involved clear, naturalistic images from the THINGS database.

\subsection{Viewable Dynamic Attention of CLIP-HBA-MEG}

\begin{figure}[ht]
    \centering
    \includegraphics[width=\textwidth]{Figures/Gorup_Saliency_Map.png}
    \caption{\textbf{Example of dynamic saliency map of neurally fine-tuned CLIP-HBA-MEG}}
    \label{fig:figure5}
\end{figure}

Leveraging the dynamic embedding spaces of the neurally aligned CLIP-HBA-MEG, we can visualize the specific regions and pixels the model prioritizes to produce human-like neural responses at a millisecond resolution. By combining randomized input sampling of the visual stimuli \citep{rise} \citep{kaniuth_high-throughput_2024} with the model’s dynamically learned visual scaler, we can quantify both where and to what extent the model attends within any image following stimulus onset. As shown in Figure \ref{fig:figure5}, the highlighted regions indicate where the model predicts human observers would direct their attention at a given time point. 

\subsection{Individualized Models Capture Participant-Specific Neural Dynamics}

\begin{figure}[ht]
    \centering
    \includegraphics[width=\textwidth]{Figures/individual_results.png}
    \caption{\textbf{Individualized model schematic and results.} (A) Fifteen individualized models were trained on MEG RDMs from 15 participants viewing 100 visual object stimuli. Each model achieved enhanced neural alignment specific to its corresponding participant, demonstrating sustained higher temporal alignment compared to the lower bound starting around 200ms post-stimulus onset. (B) The global correlation (RDMs) between the individualized models and their corresponding participant neural representations was assessed by comparing the differences among fine-tuned personalized models' dynamic representations to the differences among the ground-truth temporal MEG signals of participants. Distances within the personalized models' embedding spaces and participants' MEG RDMs were computed by flattening along the time dimension and measuring Pearson distance between the resulting long vectors. This approach captures both time-specific representations and evolving patterns over time. When assesed using a set of 18 images excluded from training, the Spearman correlation between the RDMs of personalized models and participant neural RDMs was significantly high (\( \rho = 0.659, p < 1 \times 10^{-14} \)), demonstrating that personalized neural fine-tuning successfully captured unique neural dynamics at an individual level.}
    \label{fig:figure6}
\end{figure}

We have demonstrated that our dynamic neural fine-tuning process, trained on group-level MEG data, generalizes well across multiple external datasets and diverse visual conditions, encompassing a wide range of participants and stimuli. These findings underscore the robustness and broad applicability of our approach, reinforcing confidence in its potential. Building on this, we sought to push the boundaries of perceptual fine-tuning by tailoring the model to individual neural profiles. By training on participant-specific MEG signals, we aim to capture unique differences in neural dynamics and signal processing at an individualized level, advancing toward our ultimate goal of creating personalized AI models that reflect the distinct cognitive representations and neural patterns of specific individuals. 


As shown in Figure \ref{fig:figure6}, we extended the dynamic neural fine-tuning pipeline of CLIP-HBA-Dynamic to train 15 personalized models, each fine-tuned on the individual MEG data of a single participant \citep{cichy_comparison_2016}. These personalized models achieved enhanced neural alignment specific to their respective participants, exhibiting sustained higher temporal alignment compared to the lower-bound noise ceiling, starting around 200 ms after stimulus onset.

To evaluate whether the personalized models captured consistent individual differences, we assessed the global correlation between the individualized models and the participants’ ground-truth neural representations. This was done by comparing the differences in dynamic embeddings of the fine-tuned models to the differences in temporal MEG signals among participants. Distances within the personalized models' embedding spaces and participants' MEG RDMs were calculated by flattening along the time dimension and computing the Pearson distance between the resulting vectors \ref{fig:supp4}. This approach captured both time-specific neural representations and their evolving patterns over time. Validated on a set of 18 stimuli (excluded from training) viewed by the same 15 participants, the analysis revealed a significant Spearman correlation (\( \rho = 0.659, p < 1 \times 10^{-14} \)) between the RDMs of the personalized models and the participant neural RDMs, confirming that individualized neural fine-tuning effectively captured unique neural dynamics and participant-specific patterns. 


\begin{figure}[ht]
    \centering
    \includegraphics[width=\textwidth]{Figures/Individual_model_study.png}
    \caption{\textbf{Individual Visual and Embedding Differences.} A. Example of individual differences in dynamic attention: Personalized separate CLIP-HBA-MEG models fine-tuned on 2 individual subjects. B. Example of Temporal Embedding Individual Differences: Pearson correlation distance between 2 individual subjects' model-predicted mental embeddings. C. Example of Time-Specific Individual Embedding Differences: 2 corresponding individual models for a given visual stimuli at a time point where their mental embeddings differ the most - 325ms.}
    \label{fig:figure7}
\end{figure}

With the individually aligned models, we can visualize nuanced differences in visual attention and dynamic mental embeddings at the subject level. By applying methods used to analyze model dynamic attention to each personalized model, we observe distinct visual dynamics across subjects following stimulus onset. As shown in Figure \ref{fig:figure7}A, the personalized models of two different subjects exhibit varying temporal attention patterns while both successfully recognizing the presented objects.

Moreover, these personalized models enable the study of how subjects' mental embeddings diverge and converge over time. In Figure \ref{fig:figure7}, the two sample subjects' mental embeddings begin to diverge around 100 milliseconds, peak at 325 milliseconds, and converge after 400 milliseconds. Figure \ref{fig:figure7}C further illustrates dimensional differences in their mental embeddings. At the peak of divergence, Subject A's representation is more skewed, relying on fewer dimensions, whereas dimensions such as "wood-related" rank higher in Subject B's mental embedding. This suggests that at this specific timepoint for the example stimuli, Subject B is more likely integrating background context, while Subject A focuses primarily on the central object. Overall, these findings demonstrate that personalized models offer a powerful tool for studying cognitive processes at a millisecond-level resolution, providing insights into nuanced subject-specific variations. 




\section{Discussion}

In this study, we have demonstrated that fine-tuning CLIP with human-derived behavioral and neural embeddings significantly enhances its alignment with human perception and opens the door for individualized AI models. CLIP-HBA-Behavior, trained on large-scale behavioral embeddings, showed a substantial improvement in predicting human similarity judgments while also indirectly capturing neural representational structure. When further refined with millisecond-level MEG data, CLIP-HBA-MEG achieved even greater alignment with human neural dynamics, successfully modeling the temporal evolution of perceptual representations. Notably, the model generalized well across external datasets with diverse participants, visual conditions, and object categories, reinforcing its robustness beyond the training set. Having established its capacity for human-aligned perception at the group level, we then adapted CLIP-HBA-MEG to capture individual differences in neural processing. By fine-tuning models on participant-specific neural data, we found that CLIP-HBA-MEG not only maintained strong alignment with group-level representations but also learned individualized neural dynamics, offering a promising path toward truly personalized AI systems.


\subsection{Scalable and Transferable Applications}

Building on these encouraging findings, our first area of focus was the scalability and transferability of the CLIP-HBA framework. CLIP-HBA-MEG leverages dynamic neural signals to capture real-time perceptual processes and higher-order conceptual organization, offering a nuanced window into human perception that static models cannot provide. In contrast to models that focus primarily on behavioral proxies—such as those that linearly map behavioral embeddings into model output \citep{kaniuth2024high} or perform generalizable local and global transforms to predict human similarity ratings \citep{muttenthaler2023improving}—CLIP-HBA-MEG establishes a direct relationship with neural activity by explicitly capturing the emergence of information from dynamic neural signals. Recent research has shown that even when behavioral outcomes converge, the underlying neural dynamics can differ substantially between individuals \citep{RENART2014211, Bansal_2021}. Such findings underscore the critical importance of capturing dynamic neural activity. While earlier models—such as ReAlnet \citep{lu2024achieving}—attempt to capture cross-modal EEG-fMRI alignments, they lack the temporal resolution necessary to track moment-to-moment perceptual shifts. Moreover, the CLIP-HBA framework can be fine-tuned sequentially, influencing the entire stream of information processing, which uniquely positions it as both a robust model of human perception and a versatile platform that can integrate diverse types of human behavior (e.g., MEG, MRI, behavioral data, electrophysiology). This adaptability is particularly useful in neuroscience applications, where neural data is often limited in scale.

\subsection{Explainable Human-Driven AI Systems}
In addition to capturing real-time perceptual processes, CLIP‐HBA provides fully interpretable dimensions within the model embedding space \citep{hebart_revealing_2020} that are absent in other models designed to incorporate human inductive biases \citep{lu2024achieving}. By integrating this embedding space with dynamic saliency mapping, we offer millisecond‐level insights into both the model’s visual focus and the organization of its internal representations. Furthermore, findings indicate that when model decisions are transparent and systematically quantifiable, it becomes easier to evaluate the consistency of predictions, identify potential biases, and isolate errors for correction \citep{doshi2017towards, rudin2019stop}. These results support the view that interpretability is essential in applications where accurate, accountable decision-making is critical. Together, these advancements move AI closer to human-like cognition by promoting systems whose internal operations are more accessible and amenable to rigorous oversight \citep{NeuroAI2023, Sinz2019}.

\subsection{Expandable Architectures and Modalities for Future Directions}

While our study demonstrates the effectiveness of fine-tuning CLIP to align AI models with human representations using behavioral and MEG data, our approach is not inherently limited to a specific model architecture, type of neural data, or sensory modality. The key factors contributing to the success of our method are the carefully designed loss functions and the modified mechanisms, such as feature reweighting \citep{kaniuth_feature-reweighted_2022} and noise injections. These techniques are broadly applicable to other transformer-based and deep neural network architectures, ensuring that our framework remains adaptable beyond CLIP. This allows our approach to be well-suited for expanding to additional sensory modalities. While our current work focuses on vision, future research could integrate other modalities such as auditory representations for multisensory learning \citep{baltruvsaitis2018multimodal}. By fine-tuning models with neural and behavioral data across multiple sensory domains, we can take the first steps toward a truly multisensory AI system—one that more closely mimics the way the human brain integrates information from diverse sensory inputs \citep{murray2011neural,tovar2020selective}. This expansion would not only enhance AI’s ability to model human perception but also open new possibilities for multisensory cognitive research and human-AI interaction.

\subsection{Applications of Personalized Models}

With the framework's adaptability established, we now turn to its potential for personalization—a critical aspect of aligning AI with individual human cognition. The CLIP-HBA framework represents a significant advancement in personalized AI by integrating human behavioral embeddings with neural data to tailor models to individual cognitive profiles. This approach is deeply rooted in recent cognitive neuroscience research, which suggests that aligning artificial neural networks with biological cognition can enhance both interpretability and robustness \citep{doerig2023neuroconnectionist}. By fine-tuning on millisecond-level MEG signals, the CLIP-HBA-MEG variant directly optimizes neural alignment, reflecting the neuroconnectionist framework’s emphasis on incorporating biological constraints rather than solely focusing on machine-centric performance. Moreover, the framework’s dynamic adaptation of learned representations based on neural feedback mirrors the functionality of digital twin systems \citep{jiang2021industrial}—widely used in industrial and medical applications to model dynamic systems with real-time sensor data—thus acting as a cognitive digital twin that adjusts to an individual’s unique perceptual and neural idiosyncrasies.

Beyond the realm of neuroscience, personalized AI models like CLIP-HBA are making transformative strides in various domains, including education, healthcare, and industry. Advances in AI-driven learning style detection have demonstrated that machine learning can accurately classify users into distinct cognitive subtypes—such as visual, auditory, or kinesthetic learners \citep{kanchon2024enhancing}—which aligns with CLIP-HBA’s ability to fine-tune representations based on individual neural responses. Such personalization is critical for applications where individual differences matter; for example, in clinical contexts, these models can be harnessed to detect subtle deviations in neural dynamics associated with neuropsychiatric conditions \citep{damaraju2014dynamic,preti2017dynamic}, thus informing targeted interventions and enabling clinicians to monitor treatment efficacy over time. Furthermore, the capacity for real-time adaptation paves the way for advanced brain–computer interfaces that adjust to a user’s cognitive state, optimizing interaction and control in assistive technologies \citep{kucyi2024individual}. Collectively, these advancements underscore the transformative potential of personalized AI models to bridge the gap between artificial and biological intelligence, offering adaptive, interpretable, and human-aligned systems across a wide spectrum of fields.

\subsection{Limitations and Need for Diverse Data}

In thinking about the potential applications of personalized models, it is important to recognize the current limitations and the need for more diverse data. The success of our models in aligning with human cognition highlights the need for diverse and representative data. Incorporating behavioral and neural data from a broad participant pool reduces biases from homogeneous training sets, leading to more equitable and generalizable models \citep{born2024evaluating}. However, beyond participant diversity, capturing the variability of real-world sensory experiences is equally important. As demonstrated in Figure \ref{fig:figure2}B, CLIP-HBA-MEG, fine-tuned on group-level MEG data, exhibited strong alignment with human neural representations for clear, naturalistic stimuli but showed reduced generalizability on degraded visual datasets. This suggests that training exclusively on high-quality images may limit practicality and robustness. Since human perception naturally adapts to varied conditions—low lighting, occlusions, and noise—future datasets should reflect this variability by exposing the same participants to a range of visual conditions. This approach would enhance AI’s ecological validity, making models more adaptable and effective in real-world settings.

A major challenge in scaling these efforts lies within the difficulty of collecting large behavioral and neural datasets. While current datasets are often sufficient for traditional studies and comparative analyses, their limitations become increasingly prominent when trying to apply them to AI systems. To address this, exploring data augmentation techniques and leveraging generative AI to create synthetic neural and behavioral datasets could be a promising direction. 


\subsection{Conclusion}

Our work underscores the versatility of the CLIP-HBA framework in adapting to a rich spectrum of data modalities, as demonstrated by its robust performance on large-scale, naturalistic stimulus collections. Looking ahead, systematic data collection efforts—ranging from active tasks within scanning environments to naturalistic interactions in immersive settings—hold the promise of revealing how human embeddings evolve when contextual demands change. By exploring active sensing, where representations directly support goal-directed actions, we may uncover entirely new dimensions of cognitive processing. The THINGS initiative \citep{hebart_things_2019, hebart_things-data_2023, hebart_revealing_2020} has already paved the way for training models on diverse data types, and further endeavors that include neurodiverse populations and varied behavioral contexts will only enhance models like CLIP-HBA. Such efforts are not just a testament to the progress made so far but also a compelling challenge to further capture and understand the intricate dynamics of human perception and cognition.


\section{Methods}

\subsection{Participants}

Participants were drawn from existing datasets involving behavioral and neural measurements related to object recognition and mental representations. Behavioral data were obtained from a study that acquired extensive human similarity judgments of natural objects using the THINGS database. \citep{hebart_things_2019,hebart_revealing_2020}. Neural data were sourced from magnetoencephalography (MEG) recordings collected in prior studies \citep{hebart_things-data_2023} \citep{grootswagers2017hyperalignment} \citep{10.1162/jocn_a_01068} \citep{cichy_comparison_2016}. All participants provided informed consent in accordance with institutional guidelines approved by the relevant ethics committees.

\subsection{Materials and Datasets}

\subsubsection{THINGS Dataset and SPoSE Behavioral Embeddings}

The THINGS dataset consists of a comprehensive compilation of 1,854 natural object concepts accompanied by high-quality images and semantic embeddings derived from the Sparse Positive Similarity Embedding (SPoSE) model \citep{hebart_things_2019}. The SPoSE model deduces 66 interpretable human behavioral embeddings based on similarity judgments, capturing the semantic dimensions that structure human mental representations of objects. We subsampled 1806 object stimuli with their corresponding SPoSE embedding for the training and testing of CLIP-HBA-Behavior, excluding 48 fully-sampled stimuli data for measuring the model's behavioral alignment. 

\subsubsection{MEG Data}

We utilized two separate MEG datasets to fine-tune CLIP-HBA-MEG at both group and individual levels (Table~\ref{tab:training_dataset}). For group-level training, we subsampled averaged MEG RDMs from three participants and selected 1,806 out of 1,854 visual stimuli from the THINGS MEG dataset \citep{hebart_things-data_2023}, reserving the same 48 fully-sampled stimuli for behavioral validation. This dataset was chosen for its relatively large stimulus set, ensuring generalizability and providing a robust foundation for demonstrating the model’s ability to learn meaningful neural representations. For individual-level training, we used a dataset with 118 stimuli but a larger sample of 15 participants \citep{cichy_comparison_2016}, prioritizing the number of participants to assess the model's capacity for individualized neural tuning and capturing inter-individual differences. To maintain consistency across training conditions and computational efficiency, we downsampled the individual-level dataset to 200Hz, matching the sampling rate used in group-level training.


\begin{table}[ht!]
\centering
\renewcommand{\arraystretch}{1.5} % Adjust this value for row height
\captionsetup{skip=10pt} % Adjust the space between the table and caption
\begin{tabular}{p{4cm}c cccc}
\hline
\textbf{Usage} & \textbf{Group-Level Training} & \multicolumn{4}{c}{\textbf{Individual-Level Training}} \\ \hline
\textbf{Source} & \citep{hebart_things-data_2023} & \multicolumn{4}{c}{\citep{cichy_comparison_2016}} \\
\textbf{\# Participants} & 3 & \multicolumn{4}{c}{15} \\ 
\textbf{\# Stimuli} & 1854 & \multicolumn{4}{c}{118} \\ 
\textbf{RDM Time Frame} & -100ms -- 1300ms & \multicolumn{4}{c}{-100ms -- 1000ms} \\ 
\textbf{RDM Sampling Rate} & 200Hz & \multicolumn{4}{c}{1000Hz} \\ 
\textbf{RDM Distance Metric} & LDA & \multicolumn{4}{c}{SVM} \\ 
\hline
\end{tabular}%
\caption{\textbf{Summary of Neural Training MEG Datasets}}
\label{tab:training_dataset}
\end{table}


Table~\ref{tab:validation_dataset} includes three external datasets, each with distinct participants, stimuli, and varying visual conditions. These datasets are used to evaluate the neural generalizability of CLIP-HBA-Behavior after behavioral fine-tuning and CLIP-HBA-MEG after group-level neural fine-tuning. By assessing performance on out-of-distribution data, they provide a critical test of the model’s ability to generalize beyond its training set. Both CLIP-HBA-Behavior and CLIP-HBA-MEG are designed to be robust to variations in the sampling rates of validation datasets, adapting dynamically through upsampling or downsampling within the sliding window mechanism to match the target dataset’s sampling rate.


\begin{table}[ht!]
\centering
\renewcommand{\arraystretch}{1.5} % Adjust for row height
\captionsetup{skip=10pt} % Adjust space between table and caption
\begin{tabular}{p{3.25cm} c c c} % Increased width for the first column
\hline
\multirow{2}{*}{\textbf{Usage}} & \multicolumn{3}{c}{\textbf{External Validation}} \\ % Merged header for Sets
\cline{2-4}
& \textbf{Clear/No-Background} & \textbf{Clear/Monochromatic} & \textbf{Blurry/Monochromatic} \\
\hline
\textbf{Source} & \citep{grootswagers2017hyperalignment} & \citep{10.1162/jocn_a_01068} & \citep{10.1162/jocn_a_01068} \\
\textbf{\# Participants} & 20 & 20 & 20 \\ 
\textbf{\# Stimuli} & 36 & 48 & 48 \\ 
\textbf{RDM Time Frame} & -100ms -- 600ms & -100ms -- 600ms & -100ms -- 600ms \\ 
\textbf{RDM Sampling Rate} & 100Hz & 100Hz & 100Hz \\ 
\textbf{RDM Distance Metric} & LDA & LDA & LDA \\ 
\hline
\end{tabular}%
\caption{\textbf{Summary of External Neural Validation MEG Datasets}}
\label{tab:validation_dataset}
\end{table}



\subsection{Model Architecture and Fine-Tuning Procedures}

\subsubsection{Baseline CLIP Model}

We employed the Contrastive Language–Image Pre-training (CLIP) model as our base architecture \citep{radford_learning_2021}. CLIP integrates a visual encoder, built on the Vision Transformer (ViT) architecture \citep{DBLP:journals/corr/abs-2010-11929}, with a text transformer encoder. Pre-trained on a large dataset of image-text pairs, the model aligns visual and textual representations within a shared embedding space. For this study, we specifically employed CLIP-ViT-L/14, a variant with an increased parameter count to enable richer representational learning. This same model configuration serves as our baseline, ensuring a fair comparison when evaluating performance enhancements.

\subsubsection{Fine-Tuning with Behavioral Data}

Figure \ref{fig:figure1} illustrates a schematic of the fine-tuning process for the CLIP model, designed to enhance alignment with human behavior by leveraging SPoSE behavioral embeddings. The model processes inputs from two modalities—image and text—via separate streams. The 66 SPoSE dimensions are tokenized and passed through the text encoder, resulting in 66 textual representations, $D_1 \dots D_{66}$. Simultaneously, image stimuli from the THINGS dataset are processed by the vision encoder, producing corresponding visual representations, $V$. The features from both modalities are then integrated through a dot-product projection, which computes the similarity between the visual features and text dimensions. This process encourages the model to process input stimuli into a human inductive bias mental embedding space. The feature binding function is defined as: 

\begin{equation}
e = \{ V \cdot D_i \}_{i=1}^{66}
\end{equation}

where:
\begin{itemize}
    \item \( \mathbf{e} \) is the model predicted 66-dimensional embedding vector for an input stimulus
    \item \( \mathbf{V}\) is the visual representation of the image stimulus, as produced by the vision encoder.
    \item \( \mathbf{D} = [D_1, \dots, D_{66}] \) is the vector of 66 SPoSE dimensions, as produced by the text encoder.
\end{itemize}


The fine-tuning objective function uses Mean Square Error (MSE) loss, reinforcing the model's predicted embeddings to closely match the SPoSE behavioral target embeddings. The MSE Loss is defined as: 

\begin{equation}
\mathcal{L}_{\text{MSE}} = \frac{1}{N} \sum_{i=1}^{N} \left\| \mathbf{e}^{(i)} - \mathbf{e}_t^{(i)} \right\|^2
\end{equation}

where:
\begin{itemize}
    \item \( e^{(i)} \) is the model predicted 66-dimensional embedding for the input stimulus
    \item \( e_t^{(i)} \) is the target SPoSE 66-dimensional behavioral embedding for the same stimulus
    \item \( N \) is the total number of input stimuli in a training batch.
\end{itemize}


To balance fine-tuning effectiveness with computational efficiency, we employed  Weight-Decomposed Low-Rank Adaptation (DoRA) \citep{liu2024doraweightdecomposedlowrankadaptation} modules into the out-projection of the attention layers, specifically in the final text encoder layer and the last two vision encoder layers. DoRA extends the widely used parameter-efficient fine-tuning (PEFT) method, LoRA \citep{hu_lora_2021}, by decomposing pre-trained weights into separate magnitude and directional matrices. This decomposition enables distinct magnitudinal and directional learning during fine-tuning, closely replicating the optimization dynamics of training directly on the original model weights. Only the DoRA parameters are updated, while the original layers remain frozen, significantly reducing the number of necessary training parameters and computational demands. Furthermore, to leverage CLIP's pre-training on large image-text datasets, we limit the fine-tuning updates to only the last few layers of text and vision encoders, preserving transfer learning capabilities while maintaining the pre-trained model's inherent generalizability \citep{fahes2024finetuningclipsvisualprojector}.


\subsubsection{Behavioral Predictions Evaluation}

We evaluated the fine-tuned CLIP-HBA-Behavior model's ability to predict human behavior using a triplet odd-one-out task on a held-out portion of the data of 48 sample objects. \citep{hebart_revealing_2020}. This set was fully sampled to cover a wide range of naturalistic objects, and these 48 object images were specifically excluded during training to prevent any data leakage. Representational dissimilarity matrices (RDMs) were constructed from the model's embeddings by calculating pairwise cosine distances \citep{kriegeskorte_representational_2008}. These RDMs were compared to RDMs derived from actual behavioral choices using Spearman rank-order correlations \citep{spearman}, assessing the degree of alignment between model predictions and human judgments. 

We evaluated the neural alignment of the fine-tuned CLIP-HBA-Behavior by computing the Spearman rank correlation between the RDM of the model’s static embedding space and the neural average-participant RDM representations at each timepoint. This analysis, conducted using the THINGS dataset and all external validation datasets listed in Table \ref{tab:validation_dataset}, assessed the effectiveness and generalizability of fine-tuning.

\subsubsection{Dynamic Neural Fine-Tuning}

We further developed a dynamic fine-tuning pipeline that enables the model to learn directly from neural representations and dynamics (CLIP-HBA-MEG) using a feature reweighting mechanism. This mechanism dynamically combines intermediate visual features from the vision encoder through a learnable feature reweighting matrix \(\mathbf{W} \) . Figure \ref{fig:figure3} illustrates this pipeline, where \(\mathbf{W} \) operates across temporal and layer dimensions to align model representations with neural dynamics.

The feature reweighing matrix \( \mathbf{W} \in \mathbb{R}^{T \times L} \) where \(T\) is the number of MEG timepoints and \(L=24 \) is the number of visual layers in CLIP's ViT-L/14 encoder. The matrix is initialized such that:

\begin{equation}
W_{t,l} = \begin{cases}
1 & \text{if } l = L \\
0 & \text{otherwise}
\end{cases}
\end{equation}

This initialization strategy assigns full weight to the final layer of the vision encoder while setting all other layers' weights to zero, preserving a CLIP model's pre-trained reliance on the last static representation. During training, \(\mathbf{W} \) is parameterized as a neural network module with constraints enforced through architectural design: each row of \(\mathbf{W} \) is normalized via min-max to ensure non-negative weights summing to 1. 

Optimization follows a two-step gradient descent procedure designed to progressively align the model’s visual processing with human neural representations. Initially, only the feature reweighting matrix is optimized, shifting reliance from static last-layer representations to a dynamic aggregation of intermediate visual features. This enhances temporal dynamics while preserving the pre-trained capabilities of Vision Transformer (ViT) layer weights.

\textbf{Stage 1 - Feature Reweighting:} In this phase, the ViT parameters remain frozen while only the feature reweighting matrix $\mathbf{W}$ is optimized. To enforce temporal consistency, dependencies between neighboring time points are introduced through an average sliding window. Optimization is performed using the AdamW optimizer \citep{adamw} for the initial epochs until convergence. This step establishes foundational temporal layer dependencies without altering the underlying visual processing capabilities of the pre-trained CLIP model.

\textbf{Stage 2 - Joint Fine-Tuning:} Once $\mathbf{W}$ is pre-optimized, all 24 ViT layers join the training process. Joint optimization is performed using DoRA with AdamW optimizer using a new set of learning rate, where $\mathbf{W}$ continues to be updated while the ViT layers undergo refinement. This approach enables the model to learn time-varying feature combinations through $\mathbf{W}$ while refining visual representations towards the target neural data. Detailed training hyperparameters (i.e. learning rates) can be found in Table \ref{tab:hyperparameters}.


The model incorporates temporal modulation through several key components: 1) visual feature magnitude scaling ($\alpha_T$) controlled by neural response richness \citep{Carlson2011} \citep{Carlson2013}, and 2) semantic binding strength ($\beta_T$) determined by time generalization patterns \citep{Tovar2020} \citep{KING2014203}. Additionally, to better mimic the biological nature of human perception—where attention is never fully fixed on a stimulus while ignoring all else—we introduce a mechanism that allows the model to "think of something else" by incorporating dimension-specific Gaussian noise during semantic binding. This ensures a degree of randomness, resembling the natural variability in human neural processing: 

\begin{equation}
\mathbf{e}'_{t} = \beta_T \cdot \left( \mathbf{e}_t + \boldsymbol{\epsilon}_t \odot \sigma_{\mathbf{e}_t} \right)
\end{equation}

where $\mathbf{e}_t$ represents the raw embeddings at time $t$, $\boldsymbol{\epsilon}_t \sim \mathcal{N}(0,1)$ is Gaussian noise, and $\sigma_{\mathbf{e}_t}$ denotes the standard deviation of embeddings across the batch dimension. The noise magnitude is dynamically scaled based on time-dependent neural stability, derived directly from target neural RDMs. This adaptive noise injection allows the model to develop data-driven variability while preventing overfitting to specific noisy time points, such as pre-stimulus recordings or moments long after stimulus onset. By incorporating this mechanism, the pipeline remains robust to variations in the timing of neural recordings, ensuring a more biologically plausible and flexible learning process.


MEG decoding RDMs serve as the target representations within this pipeline. Since neural data lacks a directly comparable 66-dimensional embedding space like the behavioral SPoSE embedding, we transform the model's output embeddings into representational dissimilarity matrices (RDMs) using Pearson correlation \citep{pearson}. This conversion translates the model embeddings into a common representational space with neural data, allowing alignment comparison against the target MEG decoding RDMs.

Given the dynamic nature of the model, its outputs and temporal MEG RDMs are both structured as a 3-dimensional object 

\[
\text{RDM}_{\text{model}}^{\text{object}}, \text{RDM}_{\text{MEG}}^{\text{object}} \in \mathbb{R}^{T \times N \times N}
\]

where: 
\begin{itemize}
    \item $T$ is the number of timepoints
    \item $N$ is the number of batch stimuli
\end{itemize}

This structure reflects the temporal evolution of representations. To construct reliable distance matrices for RDM analysis, a batch size greater than 1 is necessary. Maintaining a batch size of at least $N=32$ is optimal to ensure sufficient RDM sampling, which is critical for the stability of the training process.

A multi-objective loss function was incorporated to align the model's representations with the temporal structure of the neural data. The loss function combines three complementary components: Pearson correlation loss, Mean Squared Error (MSE) loss, and Time Generalization loss.

%%%\begin{itemize}
%    \item \textbf{Pearson Loss}: This loss function aligns the model's representations with the temporal structure of neural data by preserving object-wise distances and temporal dynamics. Each RDM slice within the three-dimensional temporal RDMs is reduced to its upper triangle, which is then vectorized into a one-dimensional "long" vector. This process ensures that both object and temporal relationships are captured. The same vectorization is applied to both the model-predicted RDMs and the target MEG RDMs. This objective regularizes the structural alignment between the predicted and target temporal RDMs while preserving the linear structure of the time dynamics. It is defined as:

\textbf{The Pearson correlation loss} ensures structural alignment between predicted and target representations while preserving temporal dynamics. For both model-predicted RDMs and target MEG RDMs, each RDM slice within the three-dimensional temporal RDMs is reduced to its upper triangle and vectorized into a one-dimensional vector (see Supplemental Figure \ref{fig:supp4}). The Pearson loss is then computed as:

\begin{equation}
\mathcal{L}_{\text{Pearson}}\left(x, y\right) = 1 - \frac{\sum_{i} \left( x_i - \bar{x} \right) \left( y_i - \bar{y} \right)}{\sqrt{\sum_{i} \left( x_i - \bar{x} \right)^2} \sqrt{\sum_{i} \left( y_i - \bar{y} \right)^2}},
\end{equation}

where \( x \) represents the vectorized model-predicted RDMs, \( y \) represents the corresponding vector of the target MEG RDMs, and \( \bar{x} \) and \( \bar{y} \) are their respective means.

\textbf{The MSE loss} complements the Pearson loss by regularizing the individual distances between stimuli within compared representations, defined as:

\begin{equation}
\mathcal{L}_{\text{MSE}}\left(x, y\right) = \frac{1}{N} \sum_{i=1}^{N} \left( x_i - y_i \right)^2,
\end{equation}

where \( N \) is the total number of elements in the vectorized RDMs.

\textbf{The Time Generalization loss} regulates the temporal variation and stability of the model's dynamic representation through a multi-step computation. First, we calculate a temporal representational similarity matrix (tRSM) by correlating each RDM slice  with all other remaining timepoints' slices:

\begin{equation}
\text{tRSM}(t_i, t_j) = \frac{1}{N-1} \sum_{k=1}^N \left( \frac{\mathbf{r}_{t_i}^{(k)} - \mu_{t_i}}{\sigma_{t_i}} \right) \left( \frac{\mathbf{r}_{t_j}^{(k)} - \mu_{t_j}}{\sigma_{t_j}} \right)
\end{equation}

where $\mathbf{r}_t$ represents the vectorized upper triangle of the RDM at time $t$, $\mu_t$ and $\sigma_t$ are its mean and standard deviation, and $N$ is the number of stimuli pairs. The generalization profile $g_t$ is then obtained by averaging across the temporal dimension:

\begin{equation}
g_t = \frac{1}{T} \sum_{j=1}^T \text{tRSM}(t,j)
\end{equation}

This captures how neural representations at each timepoint generalize across the temporal evolution. The final Time Generalization loss then measures the alignment between model-predicted ($g^{\text{pred}}$) and neural ($g^{\text{target}}$) profiles through:

\begin{equation}
\mathcal{L}_{\text{Time Generalization}} = \mathcal{L}_{\text{Pearson}}\left(g_t^{\text{pred}}, g_t^{\text{target}}\right)
\end{equation}

where $g_t^{\text{pred}}$ and $g_t^{\text{target}}$ are the respective generalization profiles of the model RDMs and the neural RDMs.



\textbf{The total loss function} for the MEG fine-tuning is:

\begin{equation}
\mathcal{L}_{\text{Total}} = w_1\frac{\mathcal{L}_{\text{Pearson}}}{\mathcal{L}_{\text{Pearson}}^0} + w_2 \frac{\mathcal{L}_{\text{MSE}}}{\mathcal{L}_{\text{MSE}}^0} + w_3 \frac{\mathcal{L}_{\text{Time Generalization}}}{\mathcal{L}_{\text{Time Generalization}}^0},
\end{equation}

where \(\mathcal{L}_{\text{Pearson}}^0\), \(\mathcal{L}_{\text{MSE}}^0\), and \(\mathcal{L}_{\text{Time Generalization}}^0\) represent the initial values of the Pearson loss, MSE loss, and Time Generalization loss, respectively. These values are used as normalization factors to ensure that each loss term is scaled consistently, preventing any single objective from dominating the overall optimization due to differences in magnitude.

\(w_1\), \(w_2\), and \(w_3\) are hyperparameters that weight the contributions of the normalized losses. These parameters need to be adjusted based on different training datasets to ensure balanced optimization, allowing the three objectives to converge smoothly together.



\subsubsection{Neural Data Alignment Evaluation}

We validated the efficacy of our neurally fine-tuned CLIP-HBA-MEG model through both behavioral and neural alignment analyses. For the behavioral alignment evaluation, we used spearman correlation to compare every slice of the model's dynamic RDMs against the static behavioral RDM derived the same 48 sampled stimuli, excluded from the neural fine-tuning process. For the neural alignment evaluation, we assessed the model's performance by comparing its timepoint-to-timepoint RDMs with average-participant MEG decoding RDMs from the THINGS MEG dataset and all mentioned additional validation datasets mentioned in Table  \ref{tab:validation_dataset}, using Spearman rank correlation \citep{spearman}. 

\subsection{Training Individualized Models}

To account for individual differences in neural dynamics, we trained personalized models for each participant using the same CLIP-HBA-MEG pipeline used for group-level training, with a set of adjusted training hyperparameters \ref{tab:hyperparameters} applied to all 15 individualize models. The key distinction was that, instead of using participant-averaged MEG decoding RDMs, each model was fine-tuned specifically on the MEG decoding RDMs corresponding to its respective participant's neural data. 

\subsection{Evaluation of Individual Differences}

Figure \ref{fig:figure6} presents the individual-by-individual dissimilarity matrices for both the neural data and the individualized models. For each pair of participants, we computed the Spearman correlation over time between the model-generated RDMs and the participant’s neural decoding RDMs used during training.

In analyzing the individualized models, temporal RDMs were flattened into one-dimensional vectors for both the model and target. We then computed Spearman correlations between individual models to construct a model-to-model dissimilarity matrix, using the Pearson correlation of the upper triangular elements of the temporal RDMs. Similarly, participant-to-participant RDMs were derived by first computing the Pearson correlation between objects, then flattening the timepoint and upper triangular RDM values into a single vector, as shown in Supplemental Figure \ref{fig:supp4}. Both analyses offer a detailed comparison of representational similarity across participants and their respective models.

Finally, we assessed the correspondence between individualized model RDMs and participant RDMs using Spearman correlation, achieving a strong correlation of 0.65 when evaluated on a left-out validation set of 18 stimuli with the same participants’ neural data.


\subsubsection{Dynamic Saliency Mapping of CLIP-HBA-MEG}
We achieved millisecond-level visualization of dynamically important regions within any given visual stimulus using the fine-tuned CLIP-HBA-MEG model and Randomized Input Sampling for Explanation (RISE) \citep{rise}. The original RISE method identifies critical regions for model processing by comparing classification accuracy between unmasked images and randomly masked inputs.

Since CLIP-HBA-MEG functions as an embedding model without a classification head, we adapted this approach by measuring the cosine similarity between unmasked and masked outputs in the 66-dimensional embedding space. Specifically, if masking a region increases the cosine distance between the masked and unmasked embeddings, that region is inferred to be more crucial for the model’s ability to extract meaningful information and generate effective predicted embeddings. This masking process can be applied to any image at any timepoint within the model’s dynamic embedding space.

Additionally, we incorporated the learned visual scaler ($\alpha$) at each timepoint, allowing for a dynamically evolving visual strength modulated by target neural data. This integration enables us to capture the model’s "dynamic attention" as it processes visual information over time, as demonstrated in Figure \ref{fig:figure5}.

\bibliographystyle{unsrtnat}
% \bibliography{Main}
%%
%% This is file `sample-acmlarge.tex',
%% generated with the docstrip utility.
%%
%% The original source files were:
%%
%% samples.dtx  (with options: `acmlarge')
%% 
%% IMPORTANT NOTICE:
%% 
%% For the copyright see the source file.
%% 
%% Any modified versions of this file must be renamed
%% with new filenames distinct from sample-acmlarge.tex.
%% 
%% For distribution of the original source see the terms
%% for copying and modification in the file samples.dtx.
%% 
%% This generated file may be distributed as long as the
%% original source files, as listed above, are part of the
%% same distribution. (The sources need not necessarily be
%% in the same archive or directory.)
%%
%% Commands for TeXCount
%TC:macro \cite [option:text,text]
%TC:macro \citep [option:text,text]
%TC:macro \citet [option:text,text]
%TC:envir table 0 1
%TC:envir table* 0 1
%TC:envir tabular [ignore] word
%TC:envir displaymath 0 word
%TC:envir math 0 word
%TC:envir comment 0 0
%%
%%
%% The first command in your LaTeX source must be the \documentclass command.
\documentclass[sigconf]{acmart}
%[sigconf]{acmart}

%% NOTE that a single column version is required for 
%% submission and peer review. This can be done by changing
%% the \doucmentclass[...]{acmart} in this template to 
%% \documentclass[manuscript,screen,review]{acmart}
%% 
%% To ensure 100% compatibility, please check the white list of
%% approved LaTeX packages to be used with the Master Article Template at
%% https://www.acm.org/publications/taps/whitelist-of-latex-packages 
%% before creating your document. The white list page provides 
%% information on how to submit additional LaTeX packages for 
%% review and adoption.
%% Fonts used in the template cannot be substituted; margin 
%% adjustments are not allowed.
%%
%% \BibTeX command to typeset BibTeX logo in the docs
\AtBeginDocument{%
  \providecommand\BibTeX{{%
    \normalfont B\kern-0.5em{\scshape i\kern-0.25em b}\kern-0.8em\TeX}}}
\usepackage{graphics}
\usepackage{multirow}
%\usepackage[normalem]{ulem}
%\usepackage{xcolor}
%\usepackage[table,xcdraw]{xcolor}
%\usepackage{caption}
\usepackage{graphicx}
\usepackage{subcaption}

%% Rights management information.  This information is sent to you
%% when you complete the rights form.  These commands have SAMPLE
%% values in them; it is your responsibility as an author to replace
%% the commands and values with those provided to you when you
%% complete the rights form.
\copyrightyear{2025}
\acmYear{2025}
\setcopyright{cc}
\setcctype{by-nc-nd}
\acmConference[CHI '25]{CHI Conference on Human Factors in Computing
Systems}{April 26-May 1, 2025}{Yokohama, Japan}
\acmBooktitle{CHI Conference on Human Factors in Computing Systems (CHI
'25), April 26-May 1, 2025, Yokohama,
Japan}\acmDOI{10.1145/3706598.3713105}
\acmISBN{ 979-8-4007-1394-1/25/04}




%%
%% end of the preamble, start of the body of the document source.


\begin{document}

%%
%% The "title" command has an optional parameter,
%% allowing the author to define a "short title" to be used in page headers.
\title[Infrastructures for Inspiration]{Infrastructures for Inspiration: The Routine Construction of Creative Identity and Inspiration}

%%
%% The "author" command and its associated commands are used to define
%% the authors and their affiliations.
%% Of note is the shared affiliation of the first two authors, and the
%% "authornote" and "authornotemark" commands
%% used to denote shared contribution to the research.
\author{Ellen Simpson}
\email{ellen.simpson@virginia.edu}
\orcid{0000-0003-0387-7329}
\affiliation{%
 \institution{University of Virginia}
 \streetaddress{1919 Ivy Road, P.O. Box 400249}
 \city{Charlottesville}
 \state{Virginia}
  \country{USA}
 \postcode{22903}
 }
\author{Bryan Semaan}
\email{bryan.semaan@colorado.edu}
\orcid{0000-0003-1151-2389}
\affiliation{%
 \institution{University of Colorado}
 \streetaddress{1045 18th Street, Campus Box 315}
 \city{Boulder}
 \state{Colorado}
  \country{USA}
 \postcode{80309-0315}
 }

%%
%% By default, the full list of authors will be used in the page
%% headers. Often, this list is too long, and will overlap
%% other information printed in the page headers. This command allows
%% the author to define a more concise list
%% of authors' names for this purpose.
\renewcommand{\shortauthors}{Simpson \& Semaan}

%%
%% The abstract is a short summary of the work to be presented in the
%% article.
\begin{abstract}
Online, visual artists have more places than ever to routinely share their creative work and connect with other artists. These interactions support the routine enactment of creative identity in artists and provide inspirational opportunities for artists. As creative work shifts online, interactions between artists and routines around how these artists get inspired to do creative work are mediated by and through the logics of the online platforms where they take place. In an interview study of 22 artists, this paper explores the interplay between the development of artists' creative identities and the, at times, contradictory practices they have around getting inspired. We find platforms which support the disciplined practice of creative work while supporting spontaneous moments of inspiration, play an increasing role in passive approaches to searching for inspiration, and foster numerous small community spaces for artists to negotiate their creative identities. We discuss how platforms can better support and embed mechanisms for inspiration into their infrastructures into their design and platform policy.
\end{abstract}

%%
%% The code below is generated by the tool at http://dl.acm.org/ccs.cfm.
%% Please copy and paste the code instead of the example below.
%%
\begin{CCSXML}
<ccs2012>
<concept>
<concept_id>10003120.10003130.10011762</concept_id>
<concept_desc>Human-centered computing~Empirical studies in collaborative and social computing</concept_desc>
<concept_significance>500</concept_significance>
</concept>
<concept>
<concept_id>10003120.10003121.10011748</concept_id>
<concept_desc>Human-centered computing~Empirical studies in HCI</concept_desc>
<concept_significance>500</concept_significance>
</concept>
</ccs2012>
\end{CCSXML}

\ccsdesc[500]{Human-centered computing~Empirical studies in collaborative and social computing}
\ccsdesc[500]{Human-centered computing~Empirical studies in HCI}


%%
%% Keywords. The author(s) should pick words that accurately describe
%% the work being presented. Separate the keywords with commas.
\keywords{inspiration, infrastructure, creative identity, art, artists, online communities}


%%
%% This command processes the author and affiliation and title
%% information and builds the first part of the formatted document.
\maketitle

\section{Introduction}
    With the growth of online community spaces, today's artists and other creatives have more opportunities and creative spaces than ever before to not only do creative work, but also to share it and interact with the creative work of others. These online spaces are not always good for artists, however, as their presence on these online platforms is precarious \cite{duffy2021nested} and subject to platform governance structures \cite{bishop2019managing, bishop2020algorithmic, ma2021advertiser, riccio2024exposed}. As the communicative norms of platforms are embedded into platform design and policies, many artists find themselves being nudged toward a kind of homogeneity, or "influencer creep" during their routine interactions with these platforms \cite{bishop2023influencer}. Despite the challenging landscape of these online creative spaces that are increasingly mediating people's routine creative experiences and identities, the infrastructures of these spaces allow for the routine enactment and realization of people's creative goals and artistic expression \cite{simpson2023rethinking}. There is less discussion, however, about the interplay of the routines around getting inspired, creative practice, and the enactment of creative identity on these platforms. \par

    HCI researchers have long been interested in art and creative work; exploring DIY, maker and craft communities of practice \cite{JonesHandSpinning2024, Vyasaltrusism2019, EmersonShared2024, frich2018twenty, AndersonShredding2022} as both a subject as well as a means of research inquiry \cite{friske2020entangling, RomeroWoven2024}. Another line of inquiry focuses on creative tool development - where the focus is on augmenting creative practice at various stages of the creative process \cite{WanIdeation2023, hwangideabot2021, frich2019mapping, karimi2020creative}. While many of these tools are targeted at assisting in the ideation process (e.g., brainstorming with digital mood boards \cite{WanIdeation2023, lucero2012framing}), considerably fewer focus on the process of inspiration as it ties to the enactment of an artist’s creative identity. Ideation, an interactive process that is sometimes collaborative with human or non-human entities \cite{LinCollaborativeIdeation2020}, does not occupy the same space in the creative process as inspiration. \textbf{Inspiration is an enhancement of cognitive functions, such as divergent thinking or concept blending, that leads to increased idea generation \cite{desai2024psychology}}. One must be inspired to ideate, and ideation without inspiration is challenging. Inspiration breathes life into, as well as animates, the mind to produce new ideas that would not otherwise come about \cite{hymer1990inspiration} - it is the action that comes \textit{before} the creative practice \cite{hoppe2022before}, and, as Weber suggests, inspiration occurs when it pleases, not so much when it pleases us \cite{weber1919science}. \par

    Doing creative work relies on a variety of routine practices; repeated and recognizable patterns of interdependent action carried out by multiple actors \cite{feldman2000organizational, pentland2012dynamics}. While some scholars argue that routines and creativity are diametrically opposed \cite{amabile1999changes}, others suggest that routines allow the possibility of the enactment of novel, creative ideas \cite{feldman2016beyond, sonenshein2016routines}. An integral routine of creative work involves the often contradictory practices of getting inspired. This paper draws on Hymer's \cite{hymer1990inspiration} contradictions of inspiration, as a useful lens through which to interpret the routine processes and practices of inspiration. In these, one must be disciplined in one's creative practice, but also spontaneous enough to react to a chance encounter with something inspirational; where one must be mindful the inspirational potential of one's surroundings, but still able to mindlessly engage with various spaces for chance encounters with inspirational objects; and how one must actively search for inspiration, but also have the control to wait for inspiration to come to them\cite{hymer1990inspiration}. Online, these practices are mediated by and through a platform's infrastructures, as is the interplay between the routines of getting inspired and the enactment of an artist's creative identity. Creative identity is collectively negotiated, a single person cannot determine their creative identity - or what creativity means in any particular context - it must be understood with, by, and through routine interactions with creative \textit{others} \cite{gluaveanu2014creativity}. \par

    This paper explores the contradictory behaviors of inspiration as they are mediated by online platforms through an interview study with 22 visual artists. We find that online platforms play a role in the development and enactment of artist’s creative identities through a series of relationships with necessary others - the assemblage of entities required for inspiration (e.g., creative peers, recommender systems). We discuss the role of online platforms in supporting creative identity development, contributing design recommendations to better support the mediation of inspiration and creative identity development by online platforms. \par


\section{Related Work}

\subsection{The Routine Infrastructures of Creative Identity}

    To enact one's creative identity, or any identity for that matter, people rely on routines. Routines are recognizable patterns of action or behavior that are carried out by one or multiple actors within a specific context \cite{feldman2000organizational}, and are assemblages of sociomaterial configurations people and artifacts (e.g., tools, procedures, technologies) \cite{pentland2012dynamics, shelby2024creative, latour2007reassembling}. People have agency to adapt their routines or to create new ones as need be \cite{pentland2012dynamics}. Having the agency to adapt is key to the foundational routine of building and rebuilding a coherent and rewarding sense of identity \cite{giddens1991modernity}. A strong sense of self-identity, which is how a person thinks about themselves socially or physically \cite{gecas1982self}, can give people a deep sense of security in their everyday lives \cite{ibarra2010identity}. This sense of security emerges when routines are continuous and predictable, a state of ontological security, emerging from the "routine project of the self" \cite{giddens1991modernity}. The flexibility in the routine project of the self comes from people's abilities to consciously or unconsciously use inferences from the past to anticipate a future \cite{giddens1991modernity}. Thus, identity is a routine personal and social undertaking—as one must routinely interact with the world and reflect on the impacts of those interactions in their routine project of the self. 
    
    The routines of our everyday lives are enacted on, through, and within larger societal systems. The foundations of these systems are known as infrastructures, and they support the large scale-systems that society relies upon to routinely function \cite{edwards2003infrastructure}. Infrastructure can be anything from large-scale highway systems to information and communication technologies (e.g., social media platforms) in how they support routine societal function \cite{hanseth2010design}. Infrastructure is defined in use – as they are entwined with human social practice as relational systems that take on meaning or changes in meaning in a continually negotiated way depending on the social practice taking place and the actors involved \cite{star1996steps}. Importantly, while humans are a part of the construction of the social meaning of infrastructures, they can be infrastructures, functioning as a combination of both known and unknown entities that animate a particular system \cite{LeeDourishMark2006}. In this sense infrastructures are sociotechnical, meaning that they both shape and are shaped by social practices built around and with them \cite{edwards2003infrastructure, star1996steps}. As with any large system, infrastructures have many interconnected parts that weave themselves seamlessly into the fabric of society and go unnoticed most of the time \cite{star1996steps}. \par

    While only one small aspect of ourselves, being \textit{a creative person} is also something that is produced through routines. Creative identity is a "representational project engaging the self in dialogue with multiple others about the meaning of creativity as constructed in societal discourses" \cite{gluaveanu2014creativity}. Creative identity cannot be understood by the actions of individuals alone, but rather relationships and connections between the self and others as they develop a shared notion of creativity \cite{gluaveanu2014creativity}. To have a creative identity, a person must do creative work, present that creative work to others, and have others also deem that work to be creative. One must be flexible in the routine project of the self \cite{giddens1991modernity}, and creative identities are no different. Drawing ontological security around creative identity requires flexibility in how the artist constructs knowledge about the world and about themselves as creative people. \par
   
    Creative identities are supported by human infrastructures. There are people that do the work required to animate the physical and digital infrastructures where creative work is shared, as well as the electric and network infrastructures that mediate the routine presentation of one's self as a creative person. Often, artists do not know who these human infrastructures are, but their work is vital to the continued ability of creatives to routinely express themselves as creative people. \par

    Yet, sometimes infrastructures do not work in the way that they are intended to, and sometimes they break down \cite{star1996steps}, which can become chronic in certain circumstances \cite{Semaan2019}. For artists, infrastructural breakdowns may come from how their needs and values may not match the ways infrastructures are designed \cite{simpson2023captions, simpson2023rethinking}, as infrastructures are not value-neutral. The human actors that build, maintain, and repair infrastructures embed their values, norms, and biases into them through this routine work \cite{bowker1994information, bowker2000sorting, JacksonValuesInrepair}. As they are embedded into infrastructures, these values can be at the heart of the routine sources of disruptions in artists' everyday lives and routines \cite{Semaan2019}. In some cases, value misalignment between infrastructure designers and users can result in incomplete infrastructure---an infrastructure that does not meet the needs of those who depend upon it to enact their routines, which is a common concern for artists \cite{simpson2023captions}. Infrastructural breakdowns may surface the underlying ways these infrastructures support the routine development and expression of creative identity \cite{gluaveanu2014creativity} - such as how people routinely draw on online spaces like Instagram to be inspired by the creative work of others or how designers use mood boards to frame or direct their design process \cite{lucero2012framing}. When online spaces fail to meet artists’ needs, they can leave people at a loss of where to find the people and creative objects necessary to feel inspired \cite{lucero2012framing} and to negotiate their creative identity with others \cite{hart1998inspiration, gluaveanu2014creativity}.  \par 
    

\subsection{The Sociomaterial Foundations of Inspiration}
    The word \textit{inspiration} has its root in the Latin \textit{inspirare} - which means to breathe on or into, or to animate the soul \cite{hymer1990inspiration}. During the course of our routine encounters with the world, we will, on occasion, develop "intense object relationships" with \textbf{necessary others} that are the seeds of creative products or ideas that would not otherwise come about \cite{hymer1990inspiration}. Put another way, when we, the subject of these interactions, are inspired, we are putting ourselves in direct relationship with a necessary other, which could be anything. These relationships are contradictory in nature, as inspiration requires both "discipline and spontaneity, mindlessness and mindfulness, receptive waiting and active searching" to come into being \cite{hymer1990inspiration}. Similar to how routines are patterns of action that everyone enacts in a slightly different way than everybody else; inspirational objects and the transformational relations they produce, are not static, but rather emerge in slightly different ways each time the sociomaterial relationships and particular conditions that evoke these creative products or concepts to come into alignment \cite{feldman2000organizational, pentland2012dynamics, rudnicki2021ideas, hymer1990inspiration}.\par

    For the purposes of this paper, we understand sociomateriality as the entanglement of people and objects during the routine actions of individuals or collectives within organizations \cite{orlikowski2007sociomaterial, cook1999bridging}, such as in how open-concept offices often have an organization's management team sitting in corners or around the edges of a collective workspace to enforce sociomaterial control over interactions between their team and others \cite{perriton2023constitutive}, which can only emerge through an entanglement of physical space, humans, and technology objects. The intertwining of humans and objects is situational, meaning that it is produced and reproduced differently depending on the context within which the practices are taking place. Inspiration emerges from the sociomaterial relationships between people and objects that we encounter as we go about our everyday lives. But, importantly, while an object may shape the practice of an individual in a specific organizational context, it is just one thing out of many things that exists within that context. This object shapes the practice of that individual in that particular context, which, in turn, shapes the object itself \cite{oraghallaigh2017sociomateriality}. In a sociomaterial context, a necessary object may be inspirational to one person, but may not be inspirational to another -- yet when that one person acts on that inspiration, they in turn shape the necessary object, which may lead to it becoming inspirational to someone else. \par

    Sharon Hymer \cite{hymer1990inspiration} points out four key relationships that are particularly generative of inspiration: relationships with the divine, with inanimate objects (e.g., music or nature), the secular (e.g., mentors, teachers), and the self. These relationships emerge differently for different people, and what is inspirational to one person may not be inspirational to another - or it may not be inspirational in the same way. For artists, these encounters with inspirational objects are happening as the result of creative work taking place increasingly online. Prior work has shown that encountering the creative work of others has led to what is known as "divergent thinking" -- the free-forming of new ideas that branch off from the original idea or concept \cite{GallagherIdeation2017} and that attempting to copy or recreate the creative work of others allows for transformation of the original idea into something new \cite{okada2017imitation}. Graphic designers, for example, seek out visual information--such as the creative work of others--to inform their personal development as well as to capture certain aesthetics or ideas as a part of their inspiration and ideation process \cite{laing2015study}; or will put together moodboards (usually of other people's art or photographs) that frame, direct or otherwise inspire their design process \cite{lucero2012framing}. This prior work demonstrates that artists are inspired by their encounters with, deliberate search for, and transformation into inspirational tools like moodboards; the creative work of others.\par

    At the heart of many of these encounters is technology, which facilitates the creative process and management of creative ideas on both an individual and collaborative level \cite{Rosselli2024Ideas}. Online, these spaces and encounters are more plentiful than ever before and are increasingly facilitated and impeded by platform infrastructures \cite{simpson2023rethinking,simpson2023captions,Rosselli2024Ideas}, meaning that the artist who creates and shares their work online is constantly bombarded by the potential for inspiration that stems from the creative work of others. One may be deliberate in seeking out visual information to help with the creative process \cite{laing2015study,hymer1990inspiration}, but one could also stumble on an inspirational object without actually meaning to look for one. How these spaces are designed and the infrastructural elements that facilitate these contradictory inspirational encounters can place the creative self - a person's \textit{creative identity} - in flux. The constant search for inspiration becomes a matter of routine engagement with one's creative identity as it relates to the creative expression and identities of others, while it also is, increasingly reliant and entangled with technical tools and their infrastructures. 

    While creative practice is a matter of routine for many artists, it is also a routine that is pulled in contradictory directions when it comes to how inspiration emerges. Inspiration emerges along three key contradictions: one must be disciplined, as well as able to be spontaneous to become inspired; one must always be mindless, but also mindful of when inspiration may strike; and one must be actively searching for inspiration, while also receptively waiting for a chance encounter with an inspirational object \cite{hymer1990inspiration}. We adopt Hymer's \cite{hymer1990inspiration} contradictions as a lens by which to understand the role inspiration plays in the enactment and realization of creative identities. Below, we detail the contradictions, ground them in the existing literature, and provide definitions. 

\begin{itemize}

\item \textbf{Discipline and Spontaneity} are centered around creative practice itself and the inspiration that emerges from the practice of doing art. According to Hymer, "what appears to be a serendipitous thought or discovery [...] derives from both flashes of quick impressions and from the slower, more painstaking analytic work that precedes and follows such flashes" \cite[~p.30]{hymer1990inspiration}. For our purposes, discipline surrounds the routine practice of creative work that comes both before and after the spontaneous, serendipitous, moment of inspiration. Self-discipline is the routine enactment of creative practice through the discipline to do creative work even when struggling with creative blocks \cite{gallay2013understanding}, by drawing on peer support \cite{wallace1987using} and engaging in critique and feedback sessions \cite{gluaveanu2014creativity} that negotiate meaning of creativity and artist identity. Spontaneity, conversely, is the ability to capitalize on the convergence of inspirational objects and necessary others that produce new combinations of possibilities that inspire \cite{derond2014structure} and then being open and able to do that creative work \cite{thrash2014psychology}. 

\item \textbf{Mindfulness and Mindlessness}, according to Hymer are are framed around artistic awareness and engagement with the world \cite{hymer1990inspiration}. These contradictions are embodied in the creative objects produced and the collective negotiation of creativity and creative identity that online platforms facilitate \cite{gluaveanu2014creativity, simpson2023rethinking}, as well as the tools that we use to create potential inspiration \cite{Rosselli2024Ideas,laing2015study,lucero2012framing}. Mindfulness speaks to the awareness of the inspirational elements of one’s environment and the openness to documenting inspirational objects when they are encountered for further reference. Mindlessness, in contrast, speaks to the routine, yet mindless, use of online platforms to engage with the creative work of others and to become inspired to do creative work. We note that a key element of these routine encounters is engagement with various recommender systems, which mediate the artists’ encounters with the creative work of others \cite{simpson2022tame}. 

\item \textbf{Active Searching and Receptive Waiting} are focused on a person's intentionality in searching for inspiration. Hymer describes this contradiction as being both aware of how one has routinely gone about inspirational practices in the past (e.g., through deliberate routine), but also the "spontaneity and receptivity to surprise elements which enter into a transformational experience" \cite[~p.31]{hymer1990inspiration}. While actively searching for inspiration is a routine practice for many creative people \cite{hill2016searching}, receptive waiting is about browsing - or, “a search, hopefully serendipitous [...] which might contribute the fact or idea needed in some intellectual effort” \cite[~p.4]{morse1970browsing}. Here, an element of control that must be exercised to create the possibilities of chance encounters that could be considered “serendipitous” - thus, receptive waiting  \cite{rice2001accessing, foster2003serendipity}.

\end{itemize}

    Using this framework, we describe how the online platforms where many artists are encountering inspirational objects are supporting - and not supporting - the articulation of their artistic identities. In the next section, we discuss our method. \par




\section{Method}

\subsection{Participants and Recruitment}
    We recruited participants from a wide variety of channels, drawing on in person social networks, social media, snowball sampling, and convenience sampling to allow for a wide range of participants engaged in creative work. A recruitment flier created and shared in coffee shops and artists spaces across the states of Colorado, Oregon, Washington, and Southeast Alaska, on the personal Twitter, Tumblr, and Facebook accounts of the first and second authors, and in several artist Discord communities the first author moderates. The recruitment flier included a link to an interest survey hosted on our university's Qualtrics. The initial question on the survey included front matter about consenting to be a part of the research project, as well as contact information for our university's Institutional Review Board. The survey gathered basic demographic information about the participants (e.g., age, gender expression), and contact information. 14 responses were received on Qualtrics, and a further five were received on Discord. \par

    Next, the research team contacted five artists within their social network who live in diverse settings (urban, suburban, and rural), following a convenience sampling approach similar to Ma and colleagues \cite{ma2023multi}'s recruitment strategy for content creators. Using the snowball recruitment method \cite{biernacki1981snowball}, following our interviews we asked them to share our recruitment flier with their networks. The research team was contacted by two additional working artists using this method. \par  

    Finally, after conducting the first ten interviews of this study, the first author explored other opportunities of direct solicitation for participants. We joined several artist Reddit communities and identified people asking questions about sharing their art online, and people replying to these posts. The first author used their personal account to individually message 13 users participating in these discussions expressing interest and referencing their post directly. If the message was responded to, we would reply with a link to the study recruitment survey. We drew on first author’s personal Instagram account to message the first 18 video makers who had used a trending sound related to loving art and independent artists and whose videos had under 1000 views. We selected the under 1000 views criteria after comparing those views to other views using the trending sound. The research team, following previous HCI research \cite{foryouforyou, simpson2023captions}, determined that these video makers may be more likely to respond to potential research inquiries. As with Reddit, we sent details of the study and and, if the message was responded to, a link to the recruitment survey. \par

\begin{table}[h]
\centering
%\begin{minipage}{1\textwidth}
  \centering
	\begin{tabular}{|l|l|}
    	\hline \textbf{Recruitment Site} & \textbf{Participant(s)}  \\ \hline
    	Personal Social Networks & P1, P7, P8, P13, P15  \\ \hline
    	Extended Social Networks & P11, P16  \\ \hline
    	Discord & \begin{tabular}[c]{@{}l@{}}P2, P3, P4, P5, P09\\ P10, P12\end{tabular}  \\ \hline
    	Twitter & P6, P14 \\ \hline
    	Reddit & P17, P18, P22 \\ \hline    
    	Instagram & P19, P20, P21 \\ \hline   
	\end{tabular}
	\caption{List of Participant Recruitment Sites}
	\label{tab:recruitmentsite}
	\Description{A list of recruitment sites of our  participants. Data Table Follows.}
%\end{minipage}
\vspace{-8pt}
\end{table}

    By triangulating these three recruitment methods, we reached out to 55 potential participants, and ended up successfully recruiting 22 artists, whose recruitment sites are detailed in Table \ref{tab:recruitmentsite}. Participants in this study were diverse in terms of age (18 - 74), locale (41\% Urban, 36\% Suburban, 23\% Rural), gender expression (32\% Men, 36\% Women, 27\% Gender Non-Conforming or Non-Binary, 1 participant [5\%] did not self-identify) and the way they described their art. The participant pool was less diverse in terms of racial diversity, with only 36\% of the participants identifying as Black, Latine/Hispanic, Ashkenazi, or Asian American. While these results are presented in aggregate form, the full participant demographics are included in Appendix A. \par

\subsection{Interviews}
    To best understand the interplay between inspiration and the routine project of the creative self, we conducted 22 semi-structured interviews (mean length: 70 minutes, range: 60-120 minutes) that took a life history approach \cite{wengraf2001qualitative} as we wanted to understand how participants grew as artists over time. Two interviews (P17, P18) were conducted via text on Discord, three (P16, P19, P20) were conducted over the phone, One (P11) was conducted in person, and the remaining (P1 - P10, P12 - P15, P21, P22) interviews were conducted via Zoom. While we conducted interviews through several different means, we find that there was little difference in the resulting findings, echoing \cite{dimond2012qualitative}. With participant consent, we recorded and transcribed the interviews. We used a small grant from our university's graduate student funding body to transcribe poor audio quality interviews (P11, P14, P19, P20) using the transcription service, Rev. \par

    The interview was divided into several parts: First, learning more about the participant’s identity, life history, and how they came to do the art they currently do \cite{wengraf2001qualitative}. Next, participants were asked about their creative work and how and why they share it in various contexts. We paid particular attention to the routine elements of the creative work participants do around sharing their art in particular spaces or on particular platforms, what they share, and whether that art has changed over time. The interview also included questions about places participants go to for social and creative support. The second section of the interview assessed if participants monetized their art in different ways across platforms, as well as their feelings about the monetization of art generally. Capturing these thoughts is important as prior work \cite{bishop2019managing, duffy2021nested} has shown that the visibility of an individual’s user-generated content can impact what they create or share on a particular platform. \par

    Finally, the interview included a optional concept mapping exercise where participants were asked to draw their creative process from inspiration to finished product. In qualitative and social science research, concept maps are often used as a visual expression of meaning that a participant generates which can facilitate data collection \cite{wheeldon2009framing}. In HCI, concept, mind, and cognitive mapping have been used in numerous contexts as an analytical tool (e.g., \cite{devito2018too}) and are good for tracing routine actions. Although only a handful of participants participated fully in the exercise as designed (P2, P5, P6, P9, P10, P12), the conversations across the board during this “please describe your creative process” phase of the interviews were fruitful, and typified how participants thought about their creative processes. \par

\subsection{Analysis}

    Once the interviews were fully transcribed and checked for transcription errors, the first author conducted a round of open coding using an inductive approach based in grounded theory \cite{strauss1990basics}. This approach is commonly used in HCI research \cite{karizat2021algorithmic, britton2019mothers, foryouforyou}. During the initial round of opening coding, the first author assigned tags freely. After tagging five interviews, the first and second authors met weekly to discuss the emergent codes and potential means by which to collapse them. Tags were subsequently collapsed and sorted using MAXQDA, a qualitative analysis software. \par

    Emergent from this narrative was an examination of the creative processes of the participants. Conversations focused on routine creative labor, and how these artists build systems for themselves that drew on available infrastructures for creative support -- particularly around the relational contradictions of getting inspired to do creative work. These findings led us to review literature on inspiration and to read Hymer's analysis on the contradictions of inspiration, which supported our observations and provided a helpful lens through which to reinterpret the results \cite{hymer1990inspiration}. The first author did a secondary pass in a more deductive fashion on the findings to 1) extract where these contradictions were already coded within the data, and 2) to identify further instances of where these contradictions were clearly illustrated, which subsequently went on to inform the narrative flow of this paper. The first and second author met twice during this time to discuss emergent findings and code groupings. What emerged from this was a conversation about how the routine enactment of creative identity was tied up in the contradictions of inspiration, as filtered through the platform infrastructures upon which that enactment took place. \par

\subsection{Reflections and Limitations}
    The first author is a watercolor artist, which helped to build rapport with many of the participants in this study. While we spoke to 22 artists, we did not speak to enough of any particular type of artist to claim comprehensive knowledge of one particular cohort of creative people. To this end, we have taken efforts to not speak to specific artistic forms, but rather creative processes broadly. While there are myriad social, cultural, and historical power dynamics at play when reflecting on what is and is not considered "art" that are contentious along racial, gender, and class lines, our cohort skews white. While race did had little bearing on the discussions at hand, we recognize as a white and Iraqi writing team that our identities may have played some inadvertent role in our interview processes, and collaboration on data analysis to mitigate any potential biases in our analyses. 

\section{Characterizing the Contradictions: The Infrastructuring of Inspiration}
    In this section we explore Hymer's contradictions of inspiration \cite{hymer1990inspiration}. Even though they were not directly asked about getting inspired during our interviews, participants tended to start describing their creative process with how they got inspired to do art. This process was contradictory in nature, as participants often described very intentional processes around \textit{looking} for inspirational objects, while also pointing out that sometimes inspiration just kind of happened to them in a serendipitous moment of interaction with an inspirational \textit{something}. In the first section, we discuss the contradiction of the discipline of repeated creative practice as it interfaces with the spontaneous flash of inspiration \cite{hymer1990inspiration}. We then discuss mindfulness and mindlessness during routine creative practice. Finally, we discuss how one must be always searching for inspiration, but also receptive to the moments where inspiration strikes.  \par


\subsection{Discipline \& Spontaneity: Drawing on Platform Infrastructures for the Inspiration to Do Art}
    In this section, we discuss the contradiction of discipline and spontaneity, focusing on how participants drew on platform infrastructures to do art, find inspiration, and express their creative identity.

\subsubsection{Discipline}
   
    The self-disciplined practice of deciding to do creative work participants described often drew on the infrastructures of online spaces - both human and technical - to help facilitate that practice. Creative identity stems from the social interaction of a creative individual and their peers, allowing for the collective negotiation of their identities as artists as mediated by their creative work \cite{gluaveanu2014creativity}. P3, a freelance artist and illustrator, talked about a feature of a Discord community they belong to:

    \begin{quote}
    “So joining [discord servers], right, there is the Power Hour, right? Which was huge. That was really helpful too, as someone with ADHD, to have an accountability space, right? [...] because we're all talking about creating something in that time. And then in this [discord], they have a channel that is just like art critique, so people will post things that like, I'm trying to do this, does this detail work, how do I do it? I really enjoyed that.”\end{quote}

    Power Hours, according to P3, are a declared period of focused work time that are similar to the Pomodoro technique \cite{cirillo2018pomodoro}, where people state their intentions for work at the start of the hour, and work on that set task until the end of the hour when they return to the channel and share out their progress.  P3 used this designated time deliberately, drawing on Discord’s infrastructures to facilitate focused work time and become inspired to do creative work through social and creative interactions by creating a disciplined accountability space with others. Motivation and openness to do creative work is a form of discipline, which is key to the enactment of a creative identity \cite{thrash2014psychology}, and many participants described relying on personal social networks (P7, P8, P16, P20), professional networks (P2, P19, P20) and people in artistic communities on online platforms like Reddit (P8, P17, P18, P22) or DeviantArt (P10, P15, P17, P18) to find and support that disciplined creative practice. The peer support networks, such as P3's Discord-supported Power Hour, functioned as the necessary human others to both facilitate inspiration and create the discipline required for artists like P3 to articulate and realize their creative identities.

    For other participants, inspiration emerged through their routine use of social chat and streaming platforms like Discord (P1-P6, P9, P10, P12, P13, P14, P17) or Twitch (P1, P9) or YouTube (P9). In these spaces, participants spoke of how they collaborated with peers and relied on that interaction to inspire their routine creative work. P2, a 21 year old animator, told us about how they use Twitch to get feedback, illustrated in Figure \ref{fig:P2_Feedback}:

    \begin{quote}
    "There's a couple of Twitch streams, like a few people that taught at the school that I went to, and they'll be like, if you're on and are open to feedback, send your work and we'll leave you notes kind of thing. [...] Like it's really like 30 people in a stream at any given time. But those are super helpful [...] So there are people out there who will just want to take a look at your work and leave with thoughts on it."
    \end{quote}

    P2's experiences with these Twitch channels, and their consistent reliance on them as a key part of their routine creative process, has provided them with a way to feel inspired not just from the critique they receive, but also in their interaction with creative people who value and see their art and want to help them improve as an artist. In most cases, participants felt that the online platforms and creative spaces they went to were very supportive of their continued project of their creative selves and continued discipline to do creative work. \par


\begin{figure}
	\centering
	\includegraphics[scale=.60]{P2_Feedback_rc.png}
	\setlength{\belowcaptionskip}{-10pt}
	\caption{P2's illustration of the role getting critique and feedback plays in their creative process. }
	\Description[A hand drawn sketch drawn by P2.]{A hand drawn image by Participant 2 showing a figure holding up a piece of paper with eyeball emojis surrounding the figure with the text reading 'feedback.'}
	\label{fig:P2_Feedback}
\end{figure}

    Participants also extolled the virtues of the online platforms used and the social connections maintained by and through these spaces as helping them to be disciplined in their engagement with and enactment of their creative identities, and in inspiring them to do creative work. Discord was not unique in this respect, with other participants (P1, P2, P6, and P9) discussed doing live art streams on video streaming platforms like Twitch (P1, P2, P6, and P9), YouTube (P9), and sharing videos of themselves doing art on Instagram (P18, P19, P20, P22) and TikTok (P22) as a part of being disciplined in their creative process. These were done alone in some cases, but many participants also framed this as a disciplined way to engage in creative work even when they were not inspired. \par

    The interplay between creative people and the social connections that platform infrastructures supported, were inspirational to many of participants as they helped them to foster and feel ontologically secure in their creative identities. These social connections allowed participants to achieve their creative goals, either through discipline and accountability to get inspired to do their creative work (P1, P2, P3, P5, P6, P9, P10, P14, P15, P20, P21), or to work with an inspirational cohort of peers that both inspired them and pushed them creatively (P1 - P5, P8 - P10, P12 - P15, P20, P21). The social aspect of inspiration, where ideas generate from people’s everyday interactions with friends, peers, and audiences, known or unknown, is a key element of the everyday routines participants describe around their creative processes. These spaces further foster the development of a secure sense of creative identity, as these artists are interacting with peers that are functioning as the necessary others within creative spaces.\par

\subsubsection{Spontaneity}
    Participants discussed how inspiration emerged through their everyday interaction with others - particularly friends or acquaintances within platforms that have a strong creative component to the people who gather there. While the artists we spoke to were very self-disciplined in their approaches to seeking creative feedback and in their doing creative work around and with others, sometimes the interactions with these creative people allowed for the spontaneous emergence of inspiration. Spontaneity, here, is the organic inspiration and interaction that emerges from the convergence of creative people who play off of each other, essentially functioning as human infrastructures for creativity that support and allow for the routine emergence of inspiration and organic support of creative identity. Thirteen of twenty-two participants cited Discord as a place that provided participants space to discuss common interests and creative ideas—collectively negotiating the meaning of creativity with fellow creatives peers—which often would lead to moments of inspiration that were subsequently supported by the communities contained within these online spaces. \par

    For example, P10 described an experience of getting really inspired at the start of a new Dungeons and Dragons (D\&D) campaign that was played with a disbursed group of internet friends via a designated Discord server. According to P10, the inspiration to do this art emerged on a whim as players described their characters during the first session of the game. She explains:

    \begin{quote}
    "It was like that first [...] week that we started playing D\&D, I had drawn like four different characters, full bodied, or like half bodied, with outfits and shit. And I was just like, really motivated to do it." \end{quote}

    P10’s new D\&D campaign inspired her to create a series of original art pieces, and to produce them faster than usual. Discord served as a social platform for P10 not only to share art and to be validated in her creative identity, but also for a space where she found inspiration and encouragement in her art. Discord’s infrastructures supported the routine social interactions she had with her peers around this new D\&D campaign, allowing her to be both inspired by the new creative setting she and her friends were building, but also inspired to create because of it. What is more, P10 was not being disciplined when she went into these spaces – she did not have the intent of purposefully searching for inspiration while she used Discord, yet the convergence of creative people on a platform whose infrastructures supposed the ready exchange of ideas and conversation through its chat-based design allowed for inspiration to strike for P10 in a spontaneous way. \par

    \subsubsection{Mindful Looking}
\begin{figure}
\centering
\begin{minipage}{.45\textwidth}
  \centering
  \includegraphics[width=.8\linewidth]{P7.jpg}
  \captionof{figure}{A Photograph of a Sidewalk P7 Took.}
         \Description[A photograph of a sidewalk.]{A photograph of an expanse of sidewalk that P7 took.}
  \label{fig:P7}
\end{minipage}
\end{figure}

    This experience was common across all participants in this study. Creating art – and creative drive and inspiration – was commonly framed as a social interaction with another human, or a platform that hosted other people’s creative work, or as watching a particularly interesting cartoon or movie with a Discord server full of friends. Creating together with others is an important unstructured element of creativity that is supported by the routines participants have around becoming inspired which draw on online spaces to emerge. P4, a 29 year old illustrator, told us that sometimes Discord spaces allowed them to break through creative blocks and a lack of motivation to do art. They explained:

    \begin{quote}
    “I have a lot of artists block most of the time and sometimes[...] I will ask a friend on the server or just like a friend who also does art [to] like, give me a prompt.” 
    \end{quote}

    Repurposing Discord into places that inspire their creative drive and connect themselves to necessary creative peers and mentors both allows for inspiration to spontaneously emerge from these interactions while also validating the individual artist's creative identity. We also saw this on Instagram, where the connections were driven not by intentional connection to creative peers and mentors, but rather by algorithmic content recommendation of a person's art to random strangers. For example, P19 described feeling both very validated and inspired by the feedback she got from her Instagram audience as she worked her way through a 100-day bookmaking challenge. All told, these platform infrastructures were useful in supporting both creative identities and allowing for spontaneous inspiration to emerge, particularly if one chose to engage with curated communities and spaces that were directly geared toward creative support and social connection.  \par


\subsection{Mindful Looking \& Mindless Browsing: Being Open to the Possibilities of Inspiration}

    In this section, we discuss the contradiction of mindfulness and mindlessness. According to Hymer, this contradiction is framed around awareness and engagement with the broader world \cite{hymer1990inspiration}. Therefore, we focus on how participants described being open to inspiration, how they encountered inspirational objects, and how that openness informed their expression of their creative identities.



\begin{figure}
\centering
\begin{minipage}{.45\textwidth}
  \centering
  \includegraphics[width=.8\linewidth]{P7_Color.jpg}
  \captionof{figure}{A recolored version of the photograph P7 Took.}
    \Description[A colorized photograph of a sidewalk.]{A recolored photograph of an expanse of sidewalk that P7 took.}
  \label{fig:P7_Color}
\end{minipage}
\end{figure}
    The artists in our cohort spoke at length about how being able to document inspirational objects from nature or their environment when they were encountered required a degree of mindfulness -- \textit{mindful looking}. Participants described using digital tools to gather such inspirational objects, such as image collections (i.e., Pinterest boards or mood boards) (P9, P10, P13, P15, P19, P22), while others described using physical tools. P7, a 65 year old sculptor (in wood), described a constant, mindful, looking for inspiration as he went about his everyday routines. He explains,

    \begin{quote}
    “I make use of whatever my eyes and brain find appealing. A good case in point is I was walking [...] down the street, there was this section of pavement -- sidewalk, cement, where the metal posts and everything that were up the street from it, had sloughed off chemicals and colored the cement itself. And I took a bunch of photographs of that. I find a lot of inspiration, [...] things just appeal to my eye and my sense of art, mostly from the natural world. [...] That section of sidewalk was absolutely stunning. To me anyway.”
    \end{quote}

    Figure \ref{fig:P7} is the raw version of one of the photographs that P7 took of the section of sidewalk he references. The seeming randomness of this encounter with an object that was inspiring to P7 in highlights how mindfulness plays a key role in the inspirational process, as he was able to snap this photograph and document something inspirational as soon as he noticed it. P7's ability to snap a photograph of the sidewalk, and later manipulate it into an image he shared with friends (see Figure \ref{fig:P7_Color}), required him to draw on both the physical (e.g., his cellphone), as well as the wireless cellular network infrastructure to share this photograph and it's subsequent manipulation with friends and peers as means by which of having his creative identity valued – as he was able to express the creative and artistic merit of what he saw to others and have them, in turn, validate his assessment of its creativity. The routine nature of P7's mindful engagement with potentially inspirational elements points to how, without the technology he uses to document such encounters, he would not be able to easily capture, and return to reference such inspirational objects and form the intense object relationships that allow for the emergence of new ideas \cite{hymer1990inspiration}. A key characteristic for these sorts of experience of routine, yet mindful, engagements with the world is the openness to the potential of inspiration that may come through these engagements by routinely drawing on the world around them and documenting the inspirational objects they encounter through reliance of an ecosystem of sociotechnical tools. \par


\subsubsection{Mindless Scrolling}

    Despite mindful engagement with the world, there is a certain mindless nature that characterizes encounters with inspirational objects--an openness to the potential of inspiration, or an artistic awareness of the world. Participants described how they would routinely browse specific apps, such as Instagram or Pinterest, drawing algorithmically-curated feeds of content to get inspired by the creative work of others. When browsing, recent research has noted that element of control that must be exercised to create the possibilities of chance encounters that could be considered “serendipitous” \cite{rice2001accessing, foster2003serendipity}. The integration of routine use into otherwise mindless browsing behaviors on apps like Pinterest or Instagram with a specific intent, but no key goal in mind is an example of this control \cite{foster2003serendipity}. Each moment of inspiration participants described was very individual, differing in the nature of the encounter and what about that moment was inspiring.\par

    For example, P13, a 33 year old photographer and crafter, explained that she gets inspiration from other people’s art through her casual browsing of the image-sharing app, Pinterest:
    \begin{quote}
    “Pinterest is a big one for me. I just kind of get other ideas from other people, but not necessarily using what they do, but just see what other kinds of ideas are out there.” \end{quote}

    P13’s use of Pinterest to observe the creative work of others was supported by infrastructural elements of Pinterest, specifically its content recommendation algorithm and versatility as a place to store multiple different collections of creative ideas, where it functions similar to moodboards\cite{lucero2012framing}. This was part of P13’s everyday use of Pinterest, where she described herself browsing in an unintentional, mindless way, during her free time with the openness to finding new ideas to try. \par

    While browsing an app full of creative ideas does communicate at least some intention, even if the practice was mindless, any participants (P1 - P6, P9, P10, P12, P14, P15, P18) discussed consuming media that they enjoyed, or turning to books (P3, P9) or music specifically (P2) to get inspired. P9 explains:

    \begin{quote}
    "And some other times I'm not 100\% sure what I even want. That's where it will result in a lot of mindless searching, browsing, flipping through books, putting on movies, shows, you name it, and just kind of waiting for something to just light up the blue light bulb in my head that 'Oh, yes, this is what I could do.'" \end{quote}

    P9 draws on the platforms that they have integrated into their creative routines in addition to books, films, and other media. For P9, the inspiration they find through their routine, unstructured browsing of the online platforms was similar to the inspiration found through books, films, and other media. Yet P9 also is exercising control over this unstructured browsing. They are not looking for anything in particular, but they are looking for \textit{something} that they find inspiring. The routines they have around looking for that \textit{something} are supported by the infrastructural elements of the platforms they go to look for them on – for example, browsing other people’s art on Pinterest or scrolling through Instagram.\par

   Yet sometimes these interactions cannot be exclusively mindless, because there is the potential to see things that an artist may not wish to see, or things that may be demotivationing, rather than inspirational. P10, a 28 year old hobby artist, struggles to just mindlessly look at Instagram, because she could see content she does not want to see:

    \begin{quote}
    “I don't want to be mean when I say this, but like if I'm on Instagram and I see cute art and I click it and then I see you know, a straight Christian woman trying to draw something or other. [...] It's always a weird, jarring feeling because I feel like, ya know, most of [the artists P10 engages with on Instagram] are some shade or queer.”
    \end{quote}

    P10 tends to look at art that is reminiscent of her art style, and is largely queer as she herself is queer. Yet, despite this clear interaction with content that should have a discernible series of measurable types \cite{cheney2018wearedata}, Instagram repeatedly shows P10 art by people who may fundamentally disagree with not only her creative interests, but also do not believe that people who are homosexual or queer are entitled to fundamental human rights. P10 went on to say that she did not enjoy Instagram, instead favoring Tumblr, Twitter, and Discord as she has more control in those spaces. Not being able to be truly mindless means that P10 cannot be fully open to the possibilities of inspiration, rather she must exercise more control in an online setting. This tension demonstrates how online platforms cannot, for many people whose creative identities are intertwined with other identities they may have, truly support mindless engagement with the platform in a way that allows the organic emergence of encounters with inspirational objects. \par

\subsection{Active Searching \& Receptive Waiting: Searching for Specifics vs. Encountering Parts of a Larger Whole}
    Finally, we explore the role of intentionality when reflecting on how participants engaged in searching for inspiration as a matter of routine. 

\subsubsection{Active Searching}

    While many participants described routinely mindlessly engaging with online platforms as a part of their creative process, others described routinely searching for inspirational objects. Active searching draws on available infrastructural elements like embedded search and relies on the discoverability of inspirational objects that others produce. Participants (P5, P8, P9, P10, P12, P13, P14, P16, P19, and P22) pointed to their routine use of search engines and the embedded search engines within online platforms to find inspiration for their art. These search queries ranged from what to draw, to how to do specific kinds of art, to searching for elements of the creative work of others to become a part of a broader creative vision. \par

    For example, P22, a 37 year old multimedia artist, explained how she approached the question of what kinds of subjects and artistic styles she should produce and sell in her Etsy shop:

    \begin{quote}"I Google, 'what kind of art do people buy?' And it's like landscapes, abstracts, nudes, and there's a list. And then being like, okay, so I need to learn how to make some of the things in these top categories."\end{quote}

    For P22, searching for inspiration really comes down to a pragmatic decision about what kinds of art she can produce and have commercial success in producing. The deliberate nature of how she goes about getting inspired to do certain kinds of art -- this requires discipline, as well as critical engagement of what, amongst the results, will P22 be able to feasibly produce. \par

    Sometimes search results do not align with platform norms and values, however, which can lead to potential conflicts between the artist and the platform. P22 told us about how she found that some of her art did not meet community standards, and therefore was removed from search/discoverability:

    \begin{quote}"And TikTok too [...] even though it's art and it's supposed to be allowed, they'll remove stuff sometimes and they'll flag your account if you post nude art on there."\end{quote}

    This complaint was echoed by P17, a 29 year old digital artist, who discussed his experiences with Instagram’s moderation algorithms as also negatively impacting what he could share with others - and by extension the art he could freely be inspired to do\footnote{P17 was interviewed via text, which has been reproduced below with only correction for typographical errors}:

    \begin{quote}“[T]he algorithm has changed 2 times since I joined negatively affecting my page in more ways than my usual social-awkwardness does, in one they nuked reach of images in favor of video and the next the \# basically became meaningless if you had a "but this account does X" in your profile, my "but X" is sometimes literal depictions of buts and breasts xD”\end{quote}

    P17 is one of several participants who draws adult content (P6, P9, P17, P18, P22). While his brushes with Instagram’s content moderation algorithms make sense from the perspective of Instagram’s policy on nudity and not-safe-for-work images and videos, P17 has identified two clear impacts on his using Instagram to share his art that have nothing to do with the subject matter of his art and everything to do with how the platform perceives him as a “content producer.” Similarly, P22's search for inspiration produced the idea that nudes were art that people bought, but did not warn her that artistic nudes are a common target of algorithmic censorship \cite{riccio2024exposed}. Searching for inspiration is limited by platform infrastructures because it must take place within the parameters of the existing platform infrastructure, and any inspirational object found must abide by that platform's rules, and sometimes, as in the case of artistic nudes, cannot be actively searched for. \par

\subsubsection{Receptive Waiting}

    While the search for inspiration always requires a degree of mindfulness and openness to the spontaneous moments on inspiration \cite{hymer1990inspiration}, receptive waiting is characterized by the control that one can still exercise in order to capture that spontaneous moment where search produces an inspirational \textit{something} and transforms it into art. Some of the artists we spoke to took a more malleable and receptive approach to their use of search for inspiration--never drawing on one particular object or piece of media, but rather, using search to tug multiple strands of the creative outputs of others together to create their own art. P8, a 32 year old wood intarsia\footnote{A term that describes knitting together different colors and stitches to create a larger pattern or design used in woodworking.} describes how he routinely uses search to find elements of what he wants to incorporate into his woodworking:  
 
    \begin{quote}“I'm a terrible artist with a pencil. So I start online, and it's usually a Google image search and find several pictures that have elements--way ways mountains are shaped different styles of just drawing a tree [and] shit like that, that I kind of throw all in one folder, and then go back and spend a few hours in Photoshop and pull elements from this picture in this picture in this picture and put together what I actually want to make.”	\end{quote}

    P8 is not relying on any one image to find inspiration for his art, but rather is exercising control over multiple inspirational objects he discovers through search by drawing them together into a new, transformed, whole. Receptive waiting allows P8 to amass a file of images-- \textit{inspirational somethings} found through the infrastructural elements that allow for searchability of images broadly--that he then routinely edits together using Photoshop. This practice affords P8 more creative flexibility than he would if he were drawing or sketching his ideas by hand to produce an eventual inspirational object that he reproduces in wood. Search is a key element of this creative routine around inspiration where participants integrate platform infrastructures, as well as the trace elements of other creative people that are encountered through search, while not directly social, do involve the interaction between creative people and the inspiring, creative work of others.  \par

    
\subsection{Summary}
    Given that inspiration is different for each person, understanding the contradictory nature of inspiration is a matter of routine practice that relies on infrastructures - human, technical, and physical - for support provides a clear insight into how the creative objects and spaces where inspiration emerges are co-constitutive of that inspiration. These spaces and objects together provide the potential for inspiration to emerge and transform and be made anew once once again. \par

    In reflecting on the contradictions discussed in the previous three sections, becoming inspired is both an active and passive experience for the artists we spoke to. It is active in that we must be mindful in how we go about our everyday lives, in how we look for inspirational objects, and how we must be disciplined in how we do creative work and rely on human infrastructures for critique and inspiration. Yet inspiration is also passive, in that while we may be engaging in routine behaviors (e.g., scrolling through Instagram or Pinterest), we do not have to be looking for any idea in particular. In these spontaneous moments we may find some thread of something that we find inspirational -- waiting for it to come to us and being ready and able to act on it when the moment of spontaneous inspiration hits us. \par


\section{Discussion: Infrastructuring the Creative Internet}
    Our results show that infrastructures can be inspirational, but, more importantly, they allow for the routine enactment of and engagement with one’s creative identity through routine encounters with necessary others. The results, in particular, show that creative practice (e.g., disciplined approaches to doing creative work) are supported by platform infrastructures in how the humans who use them have built and augmented spaces for feedback, critique, and disciplined work into online artist community spaces. These spaces are vital to the development and maintenance of creative identities, as they allow for artists to interact with other artists about art.  \par

    Inspiration can be embedded into platform infrastructures, and humans appropriate and build infrastructures for creativity and inspiration. We also discuss what cannot be found in these infrastructures, or what ends up missing as a result of these infrastructures. The visual artists we spoke to agree that online platforms allow for the potential to do art because they are spaces where their creative identities are validated and supported. This creative internet is supported by the inspirational potential of the infrastructures of multiple online platforms. 

\begin{table*}
\begin{table}[t]
\caption{Hardware Complexity Comparison between the Proposed DBE Architectures}
\label{tab: Synthesis Results}
\centering
\resizebox{\columnwidth}{!}{
\begin{tabular}{C{4.3cm} | C{2cm} | C{2cm} | C{2cm}}\hline\hline
                                                                           & \textbf{Baseline} & \textbf{Type 1} & \textbf{Type 2} \\\hline\hline
\begin{tabular}[c]{@{}c@{}}Gate Count\\(Including F/F Buffer)\end{tabular} & 
\begin{tabular}[c]{@{}c@{}}603,525\\(1,656 B)\end{tabular} &
\begin{tabular}[c]{@{}c@{}}574,072\\(1,400 B)\end{tabular} & 
\begin{tabular}[c]{@{}c@{}}401,850\\(376 B)\end{tabular} \\\hline
\begin{tabular}[c]{@{}c@{}}SRAM Size {[}KB{]}\\(\# of Bank x Bank Size)\end{tabular}                                                     
&3x15.36  
& 2x15.36  
& 4x7.68 \\\hline\hline
\end{tabular}}
\end{table}
\end{table*}

\subsection{The Necessary Others of The Creative Internet}

    Playing out across these collective infrastructures for inspiration is the negotiation and realization of creative identity for these artists.  Our findings show that, to become inspired, artists must interact with a series of “others” – be it other creative work, enter into creative spaces, or other artists themselves. Table \ref{tab:my-table} demonstrates the others participants described encountering. The object could be anything, but without the assemblage of human and non-human entities, the emergent circumstances for inspiration could not exist. Emergent from this table is the fact that the inspirational objects that one must form relationships with to inspire new ideas \cite{hymer1990inspiration} have no role in the table – they merely exist as it is impossible to predict how any one object will inspire any one person \cite{rudnicki2021ideas}. Inspirational objects are mediated through the platforms where they are encountered, meaning it is not the object alone, but rather the interplay between the object, the maker, the viewer, and the platform. \par

    Table \ref{tab:my-table} identifies necessary others, many of which fall into Hymer’s \cite{hymer1990inspiration} categorizations of inspirational objects – the natural (e.g., P7’s photographs of inspirational nature), the secular (e.g., creative peers and teachers). In these cases, the self is also objectified in many of the passive contradictions of inspiration, but it is mediated by technological intervention. Recommendation and search algorithms mediate the objectification of the self as an inspirational object, and, at times, give the appearance of ‘the divine’ - a “notion of an outside object temporarily occupying the inspired’s body or soul” \cite[~p.20]{hymer1990inspiration}. Necessary others are just that, necessary – for inspiration for one’s creative work, to be inspirational to others, and to the continued development and maintenance of creative identities. It is only with others that creative identity can be realized, and, therefore, it is only through relationship to others that inspiration emerges. \par

    Embedded into the socio-cultural theory of creative identity is that the performance and expression of identity is at once a personal and social task \cite{gluaveanu2014creativity}. If an individual wants to routinely interact with others while being viewed as an 'artist,' they cannot adopt that identity simply by themselves -- rather, others must see and relate to that individual as an artist \cite{gluaveanu2014creativity}. The artists we spoke to drew on online infrastructures to routinely express their creative identities, to find validation in themselves as creative people, and to engage in the collective negotiation of their creative identities. Interaction with inspirational others is vital to the development and maintenance of an artist’s creative identity. Inspiration serves as the action before the action \cite{hoppe2022before}, and our results show that inspiration is supported by an assemblage of actors - human and non-human. \par

    While creative identity can be negotiated, realized, and maintained in spaces with unknown persons \cite{gluaveanu2014creativity}, our findings show that the best support for creative identities comes through interactions with fellow creative peers in smaller online communities. Take P10 or P3, both of whom drew heavily on close personal networks on Discord to both be spontaneously inspired (P10) and to engage in the disciplined acts of creativity that work around that one moment of inspiration (P3), what Hymer refers to as the “slower, more painstaking analytic work” that tends to bracket inspiration \cite{hymer1990inspiration}. The interpersonal relationships that form between the necessary others that help support inspiration need space to organically emerge that is not afforded by many existing platform ecosystems, which tend to make construction of identity inflexible. While these platforms allow for easier expression of one’s creative identity and the connection to others, these connections only happen if the necessary others are correctly identified. \par

    Online, the co-creation of one’s creative identity now must emerge through the relations between the artist and the requisite necessary others: the algorithm and the algorithm’s datafication of the inspirational object. Unlike previous assessments, the relationship between subject and artist cannot exist anymore, as the intervention of algorithms - be they recommender systems or search algorithms - mediates what is found and where the negotiation of creative identities can emerge in a passive, rather than active way. In Hymer’s \cite{hymer1990inspiration} discussion of inspiration, we see the self objectified by ‘the divine’ as an inexplicable entity that is almost a bolt from the blue moment of inspiration. Our results, however, show the algorithmic mediation of our creative work – and therefore creative selves – serves as a transformation of \textit{our objectification of ourselves}. \par

    The mediation of technology is not divine, but rather a human-designed tool that is particularly good at tapping into our psyches and figuring out what objects may have the potential for inspiration. Take P13, browsing Pinterest and seeing the creative work of others as mediated by an algorithmic feed - she is only seeing things that the digital version of herself, based on everything she’s liked, pinned, and saved previously, wants to see. This is a double edged-sword, as it \textit{is} inspiring, however it is also limiting. Algorithmic mediation of creative work and the discovery of it limit true spontaneous moments of inspiration that emerge from seeing so outside of the mundane norm that it becomes inspirational. This impacts the potential for spontaneity in artists to create from these encounters, and it also limits how mindless someone can be – as they are constantly having to watch their feeds (as P10 explained and prior research shows \cite{simpson2022tame}) for fear of unwanted content. Further, algorithmic mediation impacts receptive waiting as there are less chances for certain inspirational objects to be discovered. Any aspect of the contradictions of inspiration that was a largely passive routine practice in an online setting (see Table\ref{tab:my-table}) is increasingly subject to the redefining of creative work in ways that are out of the control of the individual artist and are reduced to the creative intervention of how an algorithm indexes or datafies a creative object. \par



    \subsection{How Human Infrastructures Build on Existing Platform Infrastructures for Creative Support}

    Our results show that encounters with inspirational objects can be done in both active or passive ways. Table \ref{tab:my-table} shows how these interactions with inspirational others can emerge in both active and passive ways. For example, many participants described routinely looking at Pinterest (P9, P13, P15, P19) or Instagram (P3, P10, P13, P17) for inspiration, while others discussed getting inspired by music, books, or films. Inspiration emerged in this context passively, through the individual artist objectifying the inspirational object. 

    Conversely, inspiration emerged through the active and disciplined way many participants drew on infrastructures embedded into human-constructed community spaces (i.e., Discord servers) to do their creative work and then have an opportunity to share it with others for critique and feedback. Actively seeking out the creative work of others, or purposefully engaging in a creative space meant to facilitate dialogue about creative work, allows the artist to engage with the necessary others that both inspire and allow for the necessary relations that allow the artist to negotiate their creative identity. Online, creative spaces provide the infrastructure to facilitate these encounters, which both support creative identities but also foster inspiration. 

    Our findings show that the best and most effective platform infrastructures for inspiration are the spaces that are built on top of the existing platform infrastructures for creativity. These communities support disciplined creative practice - through critique from peers and mentors or teachers, they also facilitate the spontaneous moments of inspiration that emerge when creative people come together to their creative identities. These necessary others require reciprocity and community engagement from the artist.  Additionally, these human infrastructures also support receptive waiting, openness to the possibility of finding parts of a larger creative idea, as these human infrastructures facilitate the joining of otherwise unrelated ideas that inspire through communal dialogue and collective negotiation of creativity. 

    Artists look to spaces where there is a supportive and collaborative community to have meaningful interactions with other artists that not only reinforced their identity as artists, but also allowed for the organic emergence of further inspiration through the routine social interaction they support. Our participants describe drawing on online infrastructures to reach out to people on various critique-specific spaces such as Discord. These people support the routine development and maintenance of the creative identities of the artists therein \cite{gluaveanu2014creativity}. The human infrastructures promote and sustain the deep sense of security artists have in their creative selves that allows for artists routinely draw on these communities to continually negotiate their creative identities. 

    The contradictions of inspiration become muddled when non-human entities are introduced into the assemblages that help support inspiration and, in turn, creative identity formation. Many of our participants discussed their routine use of online platforms with content recommendation algorithms. These content recommendation algorithms are challenging, as they push creatives \cite{simpson2023rethinking,ma2021advertiser} and marginalized people \cite{karizat2021algorithmic,ungless2024experiences} into specific niches that flatten individual identities within broader marginalized groups \cite{lutz2024we, foryouforyou}. These platforms produce a digital version of the individual based their use of the platform and then recommend content to the measurable attributes -- or types -- of that digital version \cite{cheney2018wearedata}. Everything our participants encountered when scrolling through these feeds was mediated by how the algorithm perceived them and their art.

    This introduces a challenge for artists, as looking at people's art and becoming inspired to create art is not a new concept, nor is it one that is unique to the internet \cite{okada2017imitation}. Yet online, it the process is mediated by what the platform allows in terms of creative expression. Many participants expressed frustration with platforms like Instagram that limited what they could share. Others disliked that these platforms did not allow for purely mindless engagement, a finding that echoes the careful way that users of algorithmically curated platforms like TikTok, Pinterest, or Instagram must routinely and mindfully engage with content they encounter there \cite{simpson2022tame}. Search presents similar challenges – and while search engines are good at connecting individual artists to inspirational objects, they are not as good at connecting individuals to other humans that allow them to collectively negotiate their creative identities. 

    For example, recall how P10 encountered art on Instagram made by a person whose religious beliefs invalidated her queer identity. P10 was not able to \textit{count on} Instagram to not show her unwanted content while she casually browsed. Prior work \cite{simpson2022tame} has identified that the sensitivity of content recommendation algorithms means that one can never fully integrate use of these platforms into one's everyday routine. Our results show that, similarly, one can never just mindlessly browse for inspiration hoping for a serendipitous encounter -- there is too much chance for something to go wrong. Therefore, while online platforms support the inspiration through their ability to introduce individuals to the necessary others needed for inspiration through search or content recommendation, they are not supporting the creative identities of individual artists. 




\subsection{How Platforms Can Better Support the Human Infrastructures for Creativity that Support Artist Creative Identity}

    Many of the emergent spaces that artists repeatedly pointed to as being both inspirational, and, by extension, supportive of their creative identities were community spaces built on top of existing platform infrastructures by their creative peers. Many of these places were small - curated through interpersonal connections. Platforms like Twitch, Reddit, and Discord play host to thousands of such communities, but increasing platform interest in monetization and pressure placed on artists by platforms start behaving like influencers \cite{poell2021platforms, bishop2023influencer, simpson2023rethinking} leaves many of these communities behind in terms of platform support or consideration when platform policy decisions are made. Artists that were once able to find creative peers that allowed them to negotiate their creative identities within these spaces now must look harder, and more deliberately, for the necessary human others to facilitate these moments of inspiration that help to foster, and inform creative identity. \par

    \begin{itemize}
        \item {\textbf{Design Recommendation: Fostering Human Connections For Artists}}---Platforms should move toward finding ways to put communities of people together beyond the chance encounters of their creative work facilitated by recommendation algorithm. This means that platforms need to rethink how creative work and digital identities are embodied in data, to avoid the nichification of artists. In spaces where search and recommendation algorithms play an increasing role in not only the emergence of inspiration (i.e., through mindlessness or receptive waiting), but also in the legitimization of creative identities by one's creative peers, effort should be made to diversify \textit{how} a person's art is mediated by the algorithm in a way that fosters connection between artists that are multi-faceted and diverse. Being able to support and artist's inspiration by and engagement with multiple subjects, topics, or indexable art (i.e., someone who draws comics might also do classical figure drawing) without siloing an artist into a particular niche will help artists to legitimate their creative identities in ways that are not processed through platform definitions of "success" through metrified engagement \cite{poell2021platforms,simpson2023rethinking}.  
    \end{itemize}

    While the design of community spaces for creative people are supported by platform infrastructures to some extent, they are not supported to the point where artists are flourishing on them. For example, on Discord, if one wants to upload any file over a certain size, or send a message over a certain number of characters, one must pay for Discord’s paid subscription service, Nitro. This limits how artists can engage in some of these community spaces without having to draw on other infrastructures to share and promote their creative work as they intended for it to be seen. Simpson and colleagues \cite{simpson2023captions} identify how sharing a creative product on a single platform can require an assemblage of creative tools and infrastructures, and our findings further demonstrate that having to rely on multiple infrastructures and tools to share creative work can introduce challenges for both the inspirational potential of these community spaces, but also the collective negotiation of creative identity within them.

    \begin{itemize}
        \item {\textbf{Design Recommendation: A La Carte Features for Artists}}---Platforms could move to support artist communities by developing tiers of usage for paid features that artists might find useful in fostering their creative identities – such as being able to pay to upload larger files or having various tiers of control over how a particular art object is datafied to ensure it gets to the right audience. This more menu-like approach, if extended to whole communities of artists (e.g., on a particular Discord server), rather than done on an individual level, may also create sustainability and stability of these spaces as they no longer present a financial barrier to entry for individual artists who cannot afford the subscription model. It further supports the dialectical relationship between inspiration and creative identity. In moving away from individual subscriptions to a more collective model, platforms will create more opportunities for inspiration through broader participation, and thus will provide the necessary security artists need in their construction of creative identities, to therefore produce more platform-sustaining content.
    \end{itemize}
    
    Prior work has argued that the best way to improve platforms for creatives is to decouple platform metrics from creative success\cite{simpson2023rethinking}, and other studies have pointed to the hegemonic impacts of being a "content creator" on a platform like Instagram have pushed artist toward more influencer oriented goals \cite{bishop2023influencer}. We echo this finding and encourage platforms to recognize and embrace their role as mediators for how creative identity is developed on their platforms. As such we make a final recommendation: 
    
     \begin{itemize}
        \item {\textbf{Design Recommendation: Design Policy and Features with Artists in Mind}}---Platforms should rethink their policies around common subjects in art, such as nude or semi-nude bodies, which are a foundational part of artist education and training as they are studies in an "ideal form" \cite{clark2023nude}. While this presents an unfortunate content moderation problem, platforms could consider creating artist-tailored accounts where certain content flags (e.g., nudity) are given either a) more permissive moderation or b) are moderated on a case-by-case basis. Further, platforms could develop a  more streamlined appeals process for take downs and a regionally-localized team of moderators to make these judgment calls could potentially help artists to feel better supported by platforms and foster the development of creative identities broadly. 
\end{itemize}

    While these are just suggestions, they are informed by how platform infrastructures are already being repurposed to foster and support creative identity and inspiration by individual users or communities of users. Our findings demonstrate how, online, platform infrastructures are increasingly playing a mediating role in how inspiration emerges, which necessitates the reexamination by platforms of what role they want to play in fostering creativity and creative drive in their users. Given that platforms are dependent on user-generated content to sustain their business model \cite{poell2021platforms}, considering this will be helpful as the overall platform experience on many of these online platforms progressively getting worse for many users. Because of this concerning trend, we urge platforms to consider our suggestions as a means by which to continue to support the vibrant and flourishing artistic communities that call them home.

\section{Conclusion: How Platforms Can Better Support the Human Infrastructures for Creativity that Support Artist Creative Identity} 

    This paper explored how creative identity is formed and legitimated in artists through an exploration of the contradictions of inspiration. We explored how small, niche communities of creative peers can support disciplined creative practice can facilitate the confluence of creative people and ideas such that spontaneous moments of inspiration can emerge, and creative identity can be legitimated for these artists through the routine sharing of creative work. We discussed the increasing role of algorithmic content recommendation on the contradiction of mindfulness and mindlessness - here focusing on how unstructured browsing leads to algorithmically-mediated encounters with inspirational objects, and noting that while this was a good thing, it was not without its risks as algorithmic content recommendation of potential inspirational objects was difficult to predict for many of the artists in this cohort and could potentially lead to encounters with harmful content. Finally, we explored a specific application of search on the contradiction of active searching and receptive waiting, where platform policy often presented challenges to actively searching for inspirational objects for common artistic subjects (e.g., nudes), and where control over search allowed for transformation of multiple found inspirational objects into a single creative object.

    Our findings contribute to the ongoing conversation about the mediating role that algorithms and other platform infrastructural elements play in the creative routines and development of creative identities in artists. While many of these conversations are focused on concerns over Generative AI and copyright of creative work shared online that was added to training datasets, our work focuses on the algorithms that artists are already contending with. Our findings show that while interacting with these algorithms is at times challenging, they are also playing an important role in the inspirational process, mediating passive inspirational processes such as mindlessness or receptive waiting and allowing artists to encounter a wider range of potentially inspirational objects. This represents an augmentation of the routine enactment of creative identity and work involved in searching for inspiration for artists. Inspiration and creative identity are intrinsically linked. Inspiration functions as the action that comes before the creative act, but it is also something that helps to co-construct creative identities, as each conversation around an inspirational object must come from some other inspirational object and the artist who found it to be inspiring.
%%
%% The acknowledgments section is defined using the "acks" environment
%% (and NOT an unnumbered section). This ensures the proper
%% identification of the section in the article metadata, and the
%% consistent spelling of the heading.
\begin{acks}
We'd like to thank our participants who shared their expertise with us. Without them, this research could not exist. We'd also like to thank, Mona Sloane, Morgan Klaus Scheuerman, and Jordan Taylor, for their thoughtful feedback on early drafts. We'd also like to thank the reviewers, who provided robust feedback that helped make this paper what it is today.
\end{acks}

%%
%% The next two lines define the bibliography style to be used, and
%% the bibliography file.
\bibliographystyle{ACM-Reference-Format}
\bibliography{References}

%%
%% If your work has an appendix, this is the place to put it.
\clearpage
\appendix
\section{Appendix A: Participant Demographics Table}
\begin{table}[h]
\centering
\begin{tabular}{|c|c|l|c|l|c|l|}
\hline
\textbf{\#} & \textbf{Age} & \textbf{Gender}                                              	& \textbf{Pronouns} & \textbf{Race/Ethnicity}                                                  	& \textbf{Locale} & \textbf{Art They Do}                                                                           	\\ \hline
1       	& 31       	& Non-Binary                                                   	& she/they      	& Black                                                                    	& Urban       	& Illustrator                                                                            	\\ \hline
2       	& 21       	& \begin{tabular}[c]{@{}l@{}}Nonbinary /\\ Transmasc\end{tabular}  & they/them     	& White                                                                    	& Urban       	& Animator / Artist                                                                      	\\ \hline
3       	& 28       	& \begin{tabular}[c]{@{}l@{}}nonbinary, \\ trans\end{tabular}  	& they/them     	& White, Ashkenazi                                                         	& Urban       	& \begin{tabular}[c]{@{}l@{}}Freelance Artist / Digital \\ Artist / Illustrator\end{tabular} \\ \hline
4       	& 29       	& \begin{tabular}[c]{@{}l@{}}Non-binary / \\ Demigirl\end{tabular} & she/they      	& \begin{tabular}[c]{@{}l@{}}Mixed (White \& \\ African American)\end{tabular} & Suburban    	& Illustrator                                                                            	\\ \hline
5       	& 21       	& woman                                                        	& she/Her       	& Hispanic/Latinx                                                          	& Suburban    	& Artist                                                                                 	\\ \hline
6       	& 28       	& cis woman                                                    	& she/Her       	& White British                                                            	& Rural       	& Comic Artist / Illustrator                                                             	\\ \hline
7       	& 65       	& Male                                                         	& he/him        	& White/Caucasian                                                          	& Urban       	& Sculptor (In Wood)                                                                      	\\ \hline
8       	& 32       	& Male                                                         	& he/him        	& White                                                                    	& Rural       	& Wood Intarsia                                                                          	\\ \hline
9       	& 33       	& non-binary                                                   	& they/them     	& \begin{tabular}[c]{@{}l@{}}White (Eastern \\ European)  \end{tabular}                                               	& Rural       	& Comic Artist / Illustrator                                                             	\\ \hline
10      	& 28       	& Cis Woman                                                    	& she/her       	& Latina                                                                   	& Urban       	& Hobby Artist                                                                           	\\ \hline
11      	& 74       	& Male                                                         	& he/him        	& White                                                                    	& Suburban    	& \begin{tabular}[c]{@{}l@{}}Abstract Color Painting \\ (mixed media artist)\end{tabular}	\\ \hline
12      	& 27       	& Butch                                                        	& she/her       	& White                                                                    	& Suburban    	& Doodler                                                                                	\\ \hline
13      	& 33       	& Woman                                                        	& she/Her       	& Caucasian                                                                	& Rural       	& Photographer / Crafter                                                                 	\\ \hline
14      	& 25       	& Trans woman                                                  	& she/Her       	& Caucasian                                                                	& Suburban    	& Multi-Media Artist                                                                     	\\ \hline
15      	& 30       	& Woman                                                        	& she/her       	& Latina                                                                   	& Suburban    	& Digital Artist / Illustrator                                                           	\\ \hline
16      	& 24       	& Male                                                         	& he/him        	& White                                                                    	& Rural       	& Epoxy Resin / Floral Presser                                                           	\\ \hline
17      	& 29       	& Male                                                         	& he/him        	& White/Mexican                                                            	& Urban       	& Digital Artist                                                                         	\\ \hline
18      	& 54       	& Male                                                         	& he/him        	& Caucasian                                                                	& Urban       	& Storyteller, Comic Books                                                               	\\ \hline

19      	& 68       	& Woman                                                        	& she/her       	& White                                                                  	& Urban    	& Bookmaker                                                          	\\ \hline
20      	& 34       	& Male                                                         	& he/him        	& White                                                                    	& Suburban       	& \begin{tabular}[c]{@{}l@{}}Found \& Recycled Materials\\ Instrument Builder (Luthier)\end{tabular}                                                       	\\ \hline
21      	& 18       	& Refused                                                         	& she/her        	& Asian American                                                        	& Urban       	& \begin{tabular}[c]{@{}l@{}}Mixed Media / Fiber Artist\\ / Photography  \end{tabular}                                                                        	\\ \hline
22      	& 37       	&Female                                                        	& she/her        	& Caucasian                                                                	& Rural       	& Multi-Media Artist \\ \hline
\end{tabular}
\caption{Participant Demographics, as they described themselves}
\Description[Participant Demographics, as they described themselves.]{Participant Demographics, as they described themselves.}
\label{tab:participants}
\end{table}



\end{document}
\endinput
%%
%% End of file `sample-acmlarge.tex'.


\clearpage

\section{Supplemental Figures}


\begin{figure}[ht]
    \centering
    \includegraphics[width=\textwidth]{Figures/Supplemental/Image_Examples.png}
    \caption{\textbf{Example SPoSE Embeddings of Image Stimuli: }Behaviorally Fine-tuned CLIP-HBA-Behavior v.s. Baseline CLIP-ViT}
    \label{fig:supp1}
\end{figure}


\begin{figure}[ht]
    \centering
    \includegraphics[width=0.55\textwidth]{Figures/Supplemental/things_objects_shifting.png}
    \caption{\textbf{THINGS Object Shifting} within the 66-d SPoSE Mental Embedding Space after Behavioral Fine-Tuning; visualized using t-SNE. }
    \label{fig:supp2}
\end{figure}


\begin{table}[ht]
\centering
\renewcommand{\arraystretch}{1.5} % Adjust this value for row height
\captionsetup{skip=10pt} % Adjust the space between the table and caption
\begin{tabular}{>{\centering\arraybackslash}p{6.5cm}llc} % Center the first column
\hline
\textbf{Models}      & \textbf{Hyperparameters}         & \textbf{Value} \\ \hline
\multirow{8}{*}{CLIP-HBA-Behavior} 
                           & Learning Rate ($\eta$)         & $3 \times 10^{-4}$                 \\ 
                           & DoRA rank ($r$)            & 32                                   \\
                           & DoRA Dropout Rate ($p$)    & 0.1                                   \\
                           & Train Last $n$ Text Layers  & 1                                    \\
                           & Train Last $n$ Vision Layers  & 2                                    \\
                           & Batch Size                    & 64                                    \\
                           & Early Stop Epochs             & 20                                    \\
                           & Train/Test Split              & 8/2                                \\ \hline
\multirow{12}{*}{\makecell{CLIP-HBA-MEG\\(Group-Level)\\(3 Participants, 1806 Stimuli)} }
                           & ViT Learning Rate ($\eta$)         & $3 \times 10^{-5}$                 \\ 
                           & Feature Reweighting $\eta$        & $3 \times 10^{-3}$                  \\ 
                           & Pearson Loss Weight $w_1$     & 1         \\
                           & MSE Loss Weight $w_1$     & 0.1         \\
                           & Time Generalization Loss Weight $w_1$     & 0.15         \\
                           & DoRA rank ($r$)            & 32                                   \\
                           & DoRA Dropout Rate ($p$)    & 0.1                                   \\
                           & Train Last $n$ Text Layers  & 0                                    \\
                           & Train Last $n$ Vision Layers  & 24 (All)                              \\
                           & Batch Size                    & 64                                    \\
                           & Early Stop Epochs             & 5                                    \\
                           & Train/Test Split              & 1456/350                                \\ \hline
\multirow{12}{*}{\makecell{CLIP-HBA-MEG\\(Individual-level)\\(15 Participants, 100 Stimuli)} }
                           & ViT Learning Rate ($\eta$)         & $3 \times 10^{-5}$                 \\ 
                           & Feature Reweighting $\eta$        &  $3 \times 10^{-3}$                 \\ 
                           & Pearson Loss Weight $w_1$     & 1         \\
                           & MSE Loss Weight $w_1$     & 0.15         \\
                           & Time Generalization Loss Weight $w_1$     & 0.1         \\
                           & DoRA rank ($r$)            & 6                                   \\
                           & DoRA Dropout Rate ($p$)    & 0.1                                   \\
                           & Train Last $n$ Text Layers  & 0                                    \\
                           & Train Last $n$ Vision Layers  & 24 (All)                              \\
                           & Batch Size                    & 40                                    \\
                           & Early Stop Epochs             & 10                                    \\
                           & Train/Test Split              & 80/20                                \\ \hline
\end{tabular}
\caption{\textbf{Hyperparameters Used for Model Training.}}
\label{tab:hyperparameters}
\end{table}



\begin{figure}[ht]
    \centering
    \includegraphics[width=0.8\textwidth]{Figures/Supplemental/Size_Search_Behavioral.png}
    \caption{\textbf{Evaluation of Training Effectiveness based on the Size of the Behavioral Dataset}. CLIP-HBA-Behavior models are trained using random subsamples of the 1854-object THINGS behavioral dataset, and evaluated against the 48 excluded validation samples' triple-out behavior ground truth. Enhancement in behavioral alignment starts to diminish after the dataset size surpasses 100 object images. }
    \label{fig:supp3}
\end{figure}

\begin{figure}[ht]
    \centering
    \includegraphics[width=0.8\textwidth]{Figures/Supplemental/long_vector.png}
    \caption{\textbf{Schema of converting dynamic RDMs into a 1-D vector for distance and correlation comparisons.}. Each 3D RDM object represents dynamic representations, with each slice corresponding to a specific millisecond time point. For each time slice, we extract and flatten the upper triangle of the RDM (excluding the diagonal). These flattened slices are then concatenated into a single 1-D vector, preserving temporal dynamics and representational structure for comparison between model representations and human MEG data. }
    \label{fig:supp4}
\end{figure}


% \end{document}

\end{document}


