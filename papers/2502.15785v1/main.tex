\documentclass{article}

% [26 Jul 2024] Abdul Fatir Ansari: Adapted from the template for Score-based Methods Workshop from NeurIPS 2022

% if you need to pass options to natbib, use, e.g.:
%     \PassOptionsToPackage{numbers, compress}{natbib}
% before loading timeseries_workshop


% ready for submission
\usepackage[final]{timeseries_workshop}


% to compile a preprint version, e.g., for submission to arXiv, add add the
% [preprint] option:
%     \usepackage[preprint]{timeseries_workshop}


% to compile a camera-ready version, add the [final] option, e.g.:
%     \usepackage[final]{timeseries_workshop}


% to avoid loading the natbib package, add option nonatbib:
% \usepackage[nonatbib]{timeseries_workshop}
\usepackage[utf8]{inputenc} % allow utf-8 input
\usepackage[T1]{fontenc}    % use 8-bit T1 fonts
\usepackage{hyperref}       % hyperlinks
\usepackage{url}            % simple URL typesetting
\usepackage{booktabs}       % professional-quality tables
\usepackage{amsfonts}       % blackboard math symbols
\usepackage{nicefrac}       % compact symbols for 1/2, etc.
\usepackage{microtype}      % microtypography
% \usepackage{xcolor}         % colors
\usepackage{graphicx}
\usepackage[table,xcdraw]{xcolor}
\usepackage{amsmath}
\usepackage{multirow}
\usepackage{makecell}
\usepackage{colortbl}
\usepackage[font=footnotesize,labelformat=simple]{subcaption}
\usepackage{float}
% \usepackage{caption}
% \usepackage{subfig}
\usepackage{wrapfig} 
\usepackage{fancyhdr}
\usepackage[square,sort,comma,numbers]{natbib}
\usepackage[normalem]{ulem}
\useunder{\uline}{\ul}{}
\title{Masking the Gaps: An Imputation-Free Approach to Time Series Modeling with Missing Data}

% The \author macro works with any number of authors. There are two commands
% used to separate the names and addresses of multiple authors: \And and \AND.
%
% Using \And between authors leaves it to LaTeX to determine where to break the
% lines. Using \AND forces a line break at that point. So, if LaTeX puts 3 of 4
% authors names on the first line, and the last on the second line, try using
% \AND instead of \And before the third author name.


\author{Abhilash Neog$^{1}$, Arka Daw$^{2}$, Sepideh Fatemi Khorasgani$^{1}$, Anuj Karpatne$^{1}$ \\
$^{1}$ Virginia Tech,
$^{2}$ Oak Ridge National Labs}


\begin{document}
\maketitle
\begin{abstract}
 A significant challenge in time-series (TS) modeling is the presence of missing values in real-world TS datasets. Traditional two-stage frameworks, involving imputation followed by modeling, suffer from two key drawbacks: (1) the propagation of imputation errors into subsequent TS modeling, (2) the trade-offs between imputation efficacy and imputation complexity. While one-stage approaches attempt to address these limitations, they often struggle with scalability or fully leveraging partially observed features. To this end, we propose a novel imputation-free approach for handling missing values in time series termed \textbf{Miss}ing Feature-aware \textbf{T}ime \textbf{S}eries \textbf{M}odeling (\textbf{MissTSM}) with two main innovations. \textit{First}, we develop a novel embedding scheme that treats every combination of time-step and feature (or channel) as a distinct token. \textit{Second}, we introduce a novel \textit{Missing Feature-Aware Attention (MFAA) Layer} to learn latent representations at every time-step based on partially observed features. We evaluate the effectiveness of MissTSM  in handling missing values over multiple benchmark datasets.
\end{abstract}

% \section{Introduction}
\label{sec:introduction}
The business processes of organizations are experiencing ever-increasing complexity due to the large amount of data, high number of users, and high-tech devices involved \cite{martin2021pmopportunitieschallenges, beerepoot2023biggestbpmproblems}. This complexity may cause business processes to deviate from normal control flow due to unforeseen and disruptive anomalies \cite{adams2023proceddsriftdetection}. These control-flow anomalies manifest as unknown, skipped, and wrongly-ordered activities in the traces of event logs monitored from the execution of business processes \cite{ko2023adsystematicreview}. For the sake of clarity, let us consider an illustrative example of such anomalies. Figure \ref{FP_ANOMALIES} shows a so-called event log footprint, which captures the control flow relations of four activities of a hypothetical event log. In particular, this footprint captures the control-flow relations between activities \texttt{a}, \texttt{b}, \texttt{c} and \texttt{d}. These are the causal ($\rightarrow$) relation, concurrent ($\parallel$) relation, and other ($\#$) relations such as exclusivity or non-local dependency \cite{aalst2022pmhandbook}. In addition, on the right are six traces, of which five exhibit skipped, wrongly-ordered and unknown control-flow anomalies. For example, $\langle$\texttt{a b d}$\rangle$ has a skipped activity, which is \texttt{c}. Because of this skipped activity, the control-flow relation \texttt{b}$\,\#\,$\texttt{d} is violated, since \texttt{d} directly follows \texttt{b} in the anomalous trace.
\begin{figure}[!t]
\centering
\includegraphics[width=0.9\columnwidth]{images/FP_ANOMALIES.png}
\caption{An example event log footprint with six traces, of which five exhibit control-flow anomalies.}
\label{FP_ANOMALIES}
\end{figure}

\subsection{Control-flow anomaly detection}
Control-flow anomaly detection techniques aim to characterize the normal control flow from event logs and verify whether these deviations occur in new event logs \cite{ko2023adsystematicreview}. To develop control-flow anomaly detection techniques, \revision{process mining} has seen widespread adoption owing to process discovery and \revision{conformance checking}. On the one hand, process discovery is a set of algorithms that encode control-flow relations as a set of model elements and constraints according to a given modeling formalism \cite{aalst2022pmhandbook}; hereafter, we refer to the Petri net, a widespread modeling formalism. On the other hand, \revision{conformance checking} is an explainable set of algorithms that allows linking any deviations with the reference Petri net and providing the fitness measure, namely a measure of how much the Petri net fits the new event log \cite{aalst2022pmhandbook}. Many control-flow anomaly detection techniques based on \revision{conformance checking} (hereafter, \revision{conformance checking}-based techniques) use the fitness measure to determine whether an event log is anomalous \cite{bezerra2009pmad, bezerra2013adlogspais, myers2018icsadpm, pecchia2020applicationfailuresanalysispm}. 

The scientific literature also includes many \revision{conformance checking}-independent techniques for control-flow anomaly detection that combine specific types of trace encodings with machine/deep learning \cite{ko2023adsystematicreview, tavares2023pmtraceencoding}. Whereas these techniques are very effective, their explainability is challenging due to both the type of trace encoding employed and the machine/deep learning model used \cite{rawal2022trustworthyaiadvances,li2023explainablead}. Hence, in the following, we focus on the shortcomings of \revision{conformance checking}-based techniques to investigate whether it is possible to support the development of competitive control-flow anomaly detection techniques while maintaining the explainable nature of \revision{conformance checking}.
\begin{figure}[!t]
\centering
\includegraphics[width=\columnwidth]{images/HIGH_LEVEL_VIEW.png}
\caption{A high-level view of the proposed framework for combining \revision{process mining}-based feature extraction with dimensionality reduction for control-flow anomaly detection.}
\label{HIGH_LEVEL_VIEW}
\end{figure}

\subsection{Shortcomings of \revision{conformance checking}-based techniques}
Unfortunately, the detection effectiveness of \revision{conformance checking}-based techniques is affected by noisy data and low-quality Petri nets, which may be due to human errors in the modeling process or representational bias of process discovery algorithms \cite{bezerra2013adlogspais, pecchia2020applicationfailuresanalysispm, aalst2016pm}. Specifically, on the one hand, noisy data may introduce infrequent and deceptive control-flow relations that may result in inconsistent fitness measures, whereas, on the other hand, checking event logs against a low-quality Petri net could lead to an unreliable distribution of fitness measures. Nonetheless, such Petri nets can still be used as references to obtain insightful information for \revision{process mining}-based feature extraction, supporting the development of competitive and explainable \revision{conformance checking}-based techniques for control-flow anomaly detection despite the problems above. For example, a few works outline that token-based \revision{conformance checking} can be used for \revision{process mining}-based feature extraction to build tabular data and develop effective \revision{conformance checking}-based techniques for control-flow anomaly detection \cite{singh2022lapmsh, debenedictis2023dtadiiot}. However, to the best of our knowledge, the scientific literature lacks a structured proposal for \revision{process mining}-based feature extraction using the state-of-the-art \revision{conformance checking} variant, namely alignment-based \revision{conformance checking}.

\subsection{Contributions}
We propose a novel \revision{process mining}-based feature extraction approach with alignment-based \revision{conformance checking}. This variant aligns the deviating control flow with a reference Petri net; the resulting alignment can be inspected to extract additional statistics such as the number of times a given activity caused mismatches \cite{aalst2022pmhandbook}. We integrate this approach into a flexible and explainable framework for developing techniques for control-flow anomaly detection. The framework combines \revision{process mining}-based feature extraction and dimensionality reduction to handle high-dimensional feature sets, achieve detection effectiveness, and support explainability. Notably, in addition to our proposed \revision{process mining}-based feature extraction approach, the framework allows employing other approaches, enabling a fair comparison of multiple \revision{conformance checking}-based and \revision{conformance checking}-independent techniques for control-flow anomaly detection. Figure \ref{HIGH_LEVEL_VIEW} shows a high-level view of the framework. Business processes are monitored, and event logs obtained from the database of information systems. Subsequently, \revision{process mining}-based feature extraction is applied to these event logs and tabular data input to dimensionality reduction to identify control-flow anomalies. We apply several \revision{conformance checking}-based and \revision{conformance checking}-independent framework techniques to publicly available datasets, simulated data of a case study from railways, and real-world data of a case study from healthcare. We show that the framework techniques implementing our approach outperform the baseline \revision{conformance checking}-based techniques while maintaining the explainable nature of \revision{conformance checking}.

In summary, the contributions of this paper are as follows.
\begin{itemize}
    \item{
        A novel \revision{process mining}-based feature extraction approach to support the development of competitive and explainable \revision{conformance checking}-based techniques for control-flow anomaly detection.
    }
    \item{
        A flexible and explainable framework for developing techniques for control-flow anomaly detection using \revision{process mining}-based feature extraction and dimensionality reduction.
    }
    \item{
        Application to synthetic and real-world datasets of several \revision{conformance checking}-based and \revision{conformance checking}-independent framework techniques, evaluating their detection effectiveness and explainability.
    }
\end{itemize}

The rest of the paper is organized as follows.
\begin{itemize}
    \item Section \ref{sec:related_work} reviews the existing techniques for control-flow anomaly detection, categorizing them into \revision{conformance checking}-based and \revision{conformance checking}-independent techniques.
    \item Section \ref{sec:abccfe} provides the preliminaries of \revision{process mining} to establish the notation used throughout the paper, and delves into the details of the proposed \revision{process mining}-based feature extraction approach with alignment-based \revision{conformance checking}.
    \item Section \ref{sec:framework} describes the framework for developing \revision{conformance checking}-based and \revision{conformance checking}-independent techniques for control-flow anomaly detection that combine \revision{process mining}-based feature extraction and dimensionality reduction.
    \item Section \ref{sec:evaluation} presents the experiments conducted with multiple framework and baseline techniques using data from publicly available datasets and case studies.
    \item Section \ref{sec:conclusions} draws the conclusions and presents future work.
\end{itemize}
\section{Introduction}

Multivariate time-series (TS) modeling is important in a number of real-world applications. However, a persistent challenge is the presence of missing values on arbitrary sets of features at varying time-steps, introducing ``gaps'' in the data that can impair the application of State-of-the-art (SOTA) models unless specific adaptations are made. A common approach for handling missing data is to use imputation methods \cite{ahn2022comparison, batista2002study, acuna2004treatment}.
% These methods fall into two broad categories: those that leverage cross-channel correlations \cite{batista2002study, acuna2004treatment} and those that exploit temporal dynamics \cite{box2015time}. 
Recent deep learning (DL)-based imputation techniques \cite{tashiro2021csdi, cini2021multivariate, liu2019naomi} can learn complex, nonlinear temporal dynamics which are difficult for simple imputation techniques (like interpolation). 
% However, their reliance on a single entangled representation \cite{woo2022cost} limits capturing multifaceted time-series features, which matrix factorization \cite{liu2022multivariate} based techniques help overcome by offering disentangled temporal representations. 
However all such frameworks rely on a two-stage process, imputation of missing values, followed by feeding the imputed time-series to a TS model. This introduces two critical challenges: \textit{first}, the propagation of imputation errors into subsequent TS modeling performance, and \textit{second}, the inherent trade-offs between imputation efficacy and imputation complexity. 

In this regard, several approaches have been proposed to model time-series with missing values, such as \cite{grud} embed time intervals between observations as additional auxiliary features to handle irregular sequences, but relies on recurrent networks which struggle with long-term dependencies. ODE-based methods \cite{neuralode, latentode} offer a continuous-time framework for irregular sampling but are computationally demanding and difficult to scale. Recent methods, like \cite{chronos} implicitly handle missing values via attention mask, or use an additional mask channel (\cite{timesfm}), but in a univariate scenario.
% The time-series model learned on imputed time-series can be only as good as the imputation error.
%  \begin{wrapfigure}{r}{0.49\textwidth} % Adjust 'r' for right, 'l' for left, and width as necessary
%     \centering
%     \includegraphics[width=0.49\textwidth]{figures/toc_figure.pdf} % Adjust width as needed
%     \caption{Comparing the traditional two-stage approach of time-series modeling with missing values with our single-stage MissTSM framework that does not require imputations.} 
%     \label{fig:toc}
%     \vspace{-1ex}
% \end{wrapfigure}

To address the above limitations, we ask the question: \emph{``can we circumvent the need for imputation by designing a DL framework that can directly model multivariate TS with missing values?''} To answer this question, we draw inspiration from the recent success of masked modeling approaches in domains including vision \cite{he2022masked} and language \cite{devlin2018bert} where \emph{``masked-attention''} operations embedded in Transformer blocks are effectively utilized to reconstruct data from partial observations. Based on this insight, we propose a novel \textbf{Miss}ing Feature-aware \textbf{T}ime \textbf{S}eries \textbf{M}odeling (\textbf{MissTSM}) Framework, which capitalizes on the information contained in partially observed features to perform downstream TS modeling tasks without explicitly imputing the missing values.
% derive meaningful latent representations through a novel application of masked-attention. 
 It uses two main innovations. \textit{First}, we develop a novel embedding scheme, termed \textit{Time-Feature Independent (TFI) Embedding}, which treats every combination of time-step and feature (or channel) as a distinct token, encoding them into a high-dimensional space. \textit{Second}, we introduce a novel \textit{Missing Feature-Aware Attention (MFAA) Layer} to learn latent representations at every time-step based on partially observed features. Additionally, we use the framework of Masked Auto-encoder (MAE) \cite{he2022masked} to perform self-supervised learning of latent representations, which can be re-used for downstream tasks such as forecasting and classification. 
To evaluate the ability of MissTSM  to model TS with missing values, we consider two synthetic masking techniques: missing completely at random (MCAR), and periodic masking, to simulate varying scenarios of missing values.  We show that MissTSM achieves consistently competitive performance as SOTA models on multiple benchmark datasets without using any imputation techniques.
% Our main contributions can be summarized as follows:
% First, we propose a novel embedding scheme named as Time-Feature Independent Embedding which considers each individual time-step and feature (or channel) as an independent token, and embeds them into a high-dimensional representation. Second, we propose a Missing Feature-Aware Attention (MFAA) Layer, where we first obtain \emph{attention scores} based on the partially observed features at a given time-step $t$, which can then be utilized to perform a weighted aggregation over the observed features space to obtain the latent representations. Additionally, we adapt a Masked Auto-encoder (MAE) framework for time-series data to perform training in a self-supervised fashion with the goal of learning robust representations. These representations can then be utilized for downstream tasks such as time-series forecasting and classification. To evaluate the performance of the MissTSM approach, we propose synthetic masking techniques to generate missing-values in benchmark datasets. \textcolor{red}{We empirically demonstrate that for ..... }
% Paragraph 3: Towards the development of a deep learning framework for time-series that can handle missing values. motivation from MAE. ..
% Contributions:
% "What we are doing using MFAA is model the cross-channel correlations!"
% Primary idea of the paper:
% \\
% \begin{enumerate}
%     \item Is the observed space of partially observed time-series data sufficient for time-series modeling? => this is our hypothesis and we prove our hypothesis through empirical evaluations
%     \item Now, the next question is, how do we utilize the available space of data? -> In this regard, we propose a Feature-point embedding approach along with Missing feature-aware attention to generate efficient feature representations at each time-step that can capture the correlation of the observed space of features
%     \item Through this, we show that, this approach allows us to have a one-shot modeling (or end-to-end) for time-series data, thus eliminating the need for any kind of imputation
% \end{enumerate}
% Motivation.
% \begin{enumerate}
%     \item Real-world time-series data suffer from missing values due to multiple reasons, sensor failures, human errors, etc. And the go-to method to deal with missing data has always been to perform some kind of imputation. However, this results in a two-stage process: imputation, then the actual time-series modeling (forecasting, classification, anomaly detection, etc.). The two-stage process has a significant trade-off.
%     \begin{enumerate}
%         \item  While simple/traditional imputation techniques, like bi-linear interpolation or spline interpolation work very fast, the efficiency of these techniques is very low, especially with complex time-series and larger missing data
%         \item This can be compensated with complex time-series imputation techniques like DeepMVI, BRITS, etc. which provides a significant performance boost over traditional interpolation techniques. However, these imputation techniques are not generalizable and hence, require training (SAITS and DeepMVI are deep learning models), which further adds to the time-complexity in this 2-stage pipeline.
%         \item Using complex imputation models also require a certain degree of hyper-parameter tuning that further adds to the computational time complexity
%     \end{enumerate}
%     \item Imputation error dictates the modeling errors. This creates a strong dependency between time-series modeling and the type of imputation technique used <plot-modeling-error-interpolation-error>
%     \item We have witnessed a huge surge in time-series models recently, with a shift towards Masked modeling for self-supervised learning in time-series. While these techniques achieve State-of-the-art results, they work with the underlying assumption of complete data with no missing values. They expect data to be complete and do not have any explicit handling of missing values.
%     \item Often people resort to zero-imputation (SimMTM fills in the masked values as zeros), but this often leads to sub-optimal performance <\href{https://openreview.net/pdf?id=BylsKkHYvH}{reference}>
%     \item Furthermore, this 2-stage pipeline has it's own challenges. Firstly, imputation techniques are not generalizable and does not necessarily preserve the original manifold of the time-series data. Secondly, it induces an extra imputation error besides the modeling error. For exmaple, if we consider the forecasting task we have,  
%     \[\text{\textit{Imputation\ error}} = \sum(iTS - gTS)^2\]
%     \[\text{\textit{Modeling\ error}} = \sum(y - \hat{y})^2\]
%     \[\text{\textit{Prediction\ error}} = \text{\textit{Imputation\ error}} + \text{\textit{Modeling\ error}}\]
%     \center{where, \small{$iTS = Imputed\ Time\ Series$, \\$gTS = Ground\ truth\ TS\ values$, \\ $\hat{y} = Ground\ truth\ forecast\ values$}}
% \end{enumerate}
% Contribution
% \begin{enumerate}
%     \item We propose an imputation-free approach for time-series modeling by regarding each time-feature combination as an independent token and transforming them into one multi-dimensional feature vector capturing the feature correlations.   
%     \item We integrate this idea with the masked modeling paradigm to perform implicit imputation by utilizing self-supervised learning's built-in imputation ability for time-series modeling
%     \item Our approach combines the 2-stage pipeline into an end-to-end training solution for partial time-series data, while also achieving state-of-the-art results on benchmark datasets and higher stability and robustness with increasing missing values.
%     \item \textbf{We show that the observed space in partially observed data can alone be utilized to generate efficient feature representations for time-series modeling}
% %     \item We reflect on the architecture of Transformer and refine that the competent capability of native Transformer components on multivariate time series is under-explored.
% %     \item We propose iTransformer that regards independent time series as tokens to capture multivariate correlations by self-attention and utilize layer normalization and feed-forward network modules to learn better series-global representations for time series forecasting.
% %     \item Experimentally, iTransformer achieves comprehensive state-of-the-art on real-world benchmarks. We extensively analyze the inverted modules and architecture choices, indicating a
% % promising direction for the future improvement of Transformer-based forecasters.
% \end{enumerate}

% \section{Related Works}
\label{sec:related_works}


\noindent\textbf{Diffusion-based Video Generation. }
The advancement of diffusion models \cite{rombach2022high, ramesh2022hierarchical, zheng2022entropy} has led to significant progress in video generation. Due to the scarcity of high-quality video-text datasets \cite{Blattmann2023, Blattmann2023a}, researchers have adapted existing text-to-image (T2I) models to facilitate text-to-video (T2V) generation. Notable examples include AnimateDiff \cite{Guo2023}, Align your Latents \cite{Blattmann2023a}, PYoCo \cite{ge2023preserve}, and Emu Video \cite{girdhar2023emu}. Further advancements, such as LVDM \cite{he2022latent}, VideoCrafter \cite{chen2023videocrafter1, chen2024videocrafter2}, ModelScope \cite{wang2023modelscope}, LAVIE \cite{wang2023lavie}, and VideoFactory \cite{wang2024videofactory}, have refined these approaches by fine-tuning both spatial and temporal blocks, leveraging T2I models for initialization to improve video quality.
Recently, Sora \cite{brooks2024video} and CogVideoX \cite{yang2024cogvideox} enhance video generation by introducing Transformer-based diffusion backbones \cite{Peebles2023, Ma2024, Yu2024} and utilizing 3D-VAE, unlocking the potential for realistic world simulators. Additionally, SVD \cite{Blattmann2023}, SEINE \cite{chen2023seine}, PixelDance \cite{zeng2024make} and PIA \cite{zhang2024pia} have made significant strides in image-to-video generation, achieving notable improvements in quality and flexibility.
Further, I2VGen-XL \cite{zhang2023i2vgen}, DynamicCrafter \cite{Xing2023}, and Moonshot \cite{zhang2024moonshot} incorporate additional cross-attention layers to strengthen conditional signals during generation.



\noindent\textbf{Controllable Generation.}
Controllable generation has become a central focus in both image \citep{Zhang2023,jiang2024survey, Mou2024, Zheng2023, peng2024controlnext, ye2023ip, wu2024spherediffusion, song2024moma, wu2024ifadapter} and video \citep{gong2024atomovideo, zhang2024moonshot, guo2025sparsectrl, jiang2024videobooth} generation, enabling users to direct the output through various types of control. A wide range of controllable inputs has been explored, including text descriptions, pose \citep{ma2024follow,wang2023disco,hu2024animate,xu2024magicanimate}, audio \citep{tang2023anytoany,tian2024emo,he2024co}, identity representations \citep{chefer2024still,wang2024customvideo,wu2024customcrafter}, trajectory \citep{yin2023dragnuwa,chen2024motion,li2024generative,wu2024motionbooth, namekata2024sg}.


\noindent\textbf{Text-based Camera Control.}
Text-based camera control methods use natural language descriptions to guide camera motion in video generation. AnimateDiff \cite{Guo2023} and SVD \cite{Blattmann2023} fine-tune LoRAs \cite{hu2021lora} for specific camera movements based on text input. 
Image conductor\cite{li2024image} proposed to separate different camera and object motions through camera LoRA weight and object LoRA weight to achieve more precise motion control.
In contrast, MotionMaster \cite{hu2024motionmaster} and Peekaboo \cite{jain2024peekaboo} offer training-free approaches for generating coarse-grained camera motions, though with limited precision. VideoComposer \cite{wang2024videocomposer} adjusts pixel-level motion vectors to provide finer control, but challenges remain in achieving precise camera control.

\noindent\textbf{Trajectory-based Camera Control.}
MotionCtrl \cite{Wang2024Motionctrl}, CameraCtrl \cite{He2024Cameractrl}, and Direct-a-Video \cite{yang2024direct} use camera pose as input to enhance control, while CVD \cite{kuang2024collaborative} extends CameraCtrl for multi-view generation, though still limited by motion complexity. To improve geometric consistency, Pose-guided diffusion \cite{tseng2023consistent}, CamCo \cite{Xu2024}, and CamI2V \cite{zheng2024cami2v} apply epipolar constraints for consistent viewpoints. VD3D \cite{bahmani2024vd3d} introduces a ControlNet\cite{Zhang2023}-like conditioning mechanism with spatiotemporal camera embeddings, enabling more precise control.
CamTrol \cite{hou2024training} offers a training-free approach that renders static point clouds into multi-view frames for video generation. Cavia \cite{xu2024cavia} introduces view-integrated attention mechanisms to improve viewpoint and temporal consistency, while I2VControl-Camera \cite{feng2024i2vcontrol} refines camera movement by employing point trajectories in the camera coordinate system. Despite these advancements, challenges in maintaining camera control and scene-scale consistency remain, which our method seeks to address. It is noted that 4Dim~\cite{watson2024controlling} introduces absolute scale but in  4D novel view synthesis (NVS) of scenes.



% \subsection{Greedies}
We have two greedy methods that we're using and testing, but they both have one thing in common: They try every node and possible resistances, and choose the one that results in the largest change in the objective function.

The differences between the two methods, are the function. The first one uses the median (since we want the median to be >0.5, we just set this as our objective function.)

Second one uses a function to try to capture more nuances about the fact that we want the median to be over 0.5. The function is:

\[
\text{score}(\text{opinion}) =
\begin{cases} 
\text{maxScore}, & \text{if } \text{opinion} \geq 0.5 \\
\min\left(\frac{50}{0.5 - \text{opinion}}, \frac{\text{maxScore}}{2}\right), & \text{if } \text{opinion} < 0.5 
\end{cases}
\] 

Where we set maxScore to be $10000$.

\subsection{find-c}
Then for the projected methods where we use Huber-Loss, we also have a $find-c$ version (temporary name). This method initially finds the c for the rest of the run. 

The way it does it it randomly perturbs the resistances and opinions of every node, then finds the c value that most closely approximates the median for all of the perturbed scenarios (after finding the stable opinions). 



% \begin{figure}[ht]
%     \centering
%     \includegraphics[width=0.85\linewidth, scale=0.1]{figures/Encoding.jpg}
%     \caption{Schematic of the Time-Feature Independent (TFI) Embedding of MissTSM that learns a different embedding for every combination of time-step and variate, in contrast to the time-only embeddings of Transformer \cite{vaswani2017attention} and the variate-only embeddings of iTransformers \cite{liu2023itransformer}.} 
%     % MissTSM uses a novel  for every , allowing it to handle time-steps with missing values using masked cross-attention.}
%     \label{fig:tfi}
% \end{figure}

\vspace{-2ex}
\section{Missing Feature Time-Series Modeling (MissTSM)}

% \textcolor{red}{Figure 1: Two stage approach vs One-stage approach (Box-and-arrow diagram)}

% \textcolor{red}{* learning good representation using the available data (unmasked data). Not relying on imputations.}

\subsection{Notations and Problem Formulations}
Let us represent a multivariate TS as $\mathbf{X} \in \mathbb{R}^{T \times N}$, where $T$ is the number of time-steps, and $N$ is the dimensionality (number of variates) of the TS. We assume a subset of variates (or features) to be missing at some time-steps of $\mathbf{X}$, represented in the form of a missing-value mask $\mathcal{M} \in [0, 1]^{T \times N}$, where $\mathcal{M}_{(t, d)}$ represents the value of the mask at $t$-th time-step and $d$-th dimension. 
% $\mathcal{M}_{(t, d)} = 1$ denotes that the corresponding value in $\mathbf{X}_{(t, d)}$ is missing, while $\mathcal{M}_{(t, d)} = 0$ denotes that $\mathbf{X}_{(t, d)}$ is observed. 
Let us denote $\mathbf{X}_{(t, :)} \in \mathbb{R}^N$ as the multiple variates of the TS at a particular time-step $t$, and $\mathbf{X}_{(:, d)} \in \mathbb{R}^T$ as the uni-variate time-series for the variate $d$. In this paper, we consider two downstream tasks for TS modeling: forecasting and classification. For forecasting, the goal is to predict the future $S$ time-steps of $\mathbf{X}$ represented as $\mathbf{Y} \in \mathbb{R}^{S \times N}$, and, for TS classification, the goal is to predict output labels $\mathbf{Y} \in \{1, 2, ..., C\}$ given $\mathbf{X}$, where $C$ is the number of classes.

% [DESCRIBE THE FORECASTING AND CLASSIFICATION PROBLEMS.]
% Note that the missing values in time-series can be 

\subsection{Learning Embeddings for Time-Series with Missing Features using TFI Embedding}
% \par \noindent \textbf{Limitations of Existing Methods:}
% The first step in time-series modeling using transformer-based architectures is to learn an embedding of the time-series $\mathbf{X}$, which is then fed into the transformer encoder. Traditionally, this is done using an Embedding-layer (typically implemented using a multi-layered perceptron) as $\texttt{Embedding}:\mathbb{R}^N \mapsto \mathbb{R}^D$ that maps $\mathbf{X} \in \mathbb{R}^{T \times N}$ to the embedding  $\mathbf{H} \in \mathbb{R}^{T \times D}$, where $D$ is the embedding dimension. The Embedding layer operates on every time-step independently such that the set of variates observed at time-step $t$, $\mathbf{X}_{(t, :)}$, is considered as a single token and mapped to the embedding vector $\mathbf{h}_{t} \in \mathbb{R}^{D}$ as $\mathbf{h}_{t} = \texttt{Embedding}(\mathbf{X}_{(t, :)})$ (see Figure \ref{fig:tfi}(a)). An alternate embedding scheme was recently introduced in the framework of inverted Transformer \cite{liu2023itransformer},  where the uni-variate time-series for the $d$-th variate, $\mathbf{X}_{(:, d)}$, is considered as a single token and mapped to the embedding vector: $\mathbf{h}_{d} = \texttt{Embedding}(\mathbf{X}_{(:, d)})$ (see Figure \ref{fig:tfi}(b)). 
% % Schematic representations of the embedding scheme for original transfomer and iTransformer are depicted in \textcolor{red}{Figure XX}. 
% While both these embedding schemes have their unique advantages, they are unsuitable to handle time-series with arbitrary sets of missing values at every time-step. In particular, the input tokens to the Embedding layer of Transformer or iTransformer requires all components of $\mathbf{X}_{(t, :)}$ or $\mathbf{X}_{(:, d)}$ to be observed, respectively.
% % Assuming that the time-series $\mathbf{X}$ has missing values, these arbitrary tokens might have missing values in them as well. 
% % This in-turn would prevent us from embedding the entire token using the aforementioned $\texttt{Embedding}$ layers. 
% If any of the components in these tokens are missing, we will not be able to compute their embeddings and thus will have to discard either the time-step or the variate, leading to loss of information. 
% for the entire token cannot be obtained. This would drastically affect the number of embedding inputs that can be fed into the subsequent transformer layers, especially as the missing value fraction increases.

% \par \noindent \textbf{Time-Feature Independent (TFI) Embedding:} 
Prior embedding techniques such as in Transformer or iTransfomer models cannot handle missing values (See Appendix \ref{appendix:A.1} for more details) directly. To address this challenge, we propose a novel \emph{Time-Feature Independent (TFI) Embedding} scheme for TS with missing features, where the value at each combination of time-step $t$ and variate $d$ is considered as a single token $\mathbf{X}_{(t, d)}$, and is independently mapped to an embedding using $\texttt{TFIEmbedding}:\mathbb{R} \mapsto \mathbb{R}^D$ as: $\mathbf{h}_{(t, d)} = \texttt{TFIEmbedding}(\mathbf{X}_{(t, d)})$ 
% \begin{equation}
%     \mathbf{h}_{(t, d)} = \texttt{TFIEmbedding}(\mathbf{X}_{(t, d)})
% \end{equation}
In other words, the $\texttt{TFIEmbedding}$ Layer maps $\mathbf{X} \in \mathbb{R}^{T \times N}$ into the TFI embedding $\mathbf{H}^{\mathrm{TFI}} \in \mathbb{R}^{T \times N \times D}$ ({see Figure \ref{fig:tfi}(c) in the Appendix \ref{appendix:A.1}}). The $\texttt{TFIEmbedding}$ is applied only on tokens $\mathbf{X}_{(t, d)}$ that are observed (for missing tokens, i.e., $\mathcal{M}_{(t, d)} = 0$, we generate a dummy embedding that gets masked out in the MFAA layer). The advantage of such an approach is that even if a particular value in the TS is missing, other observed values in TS can be embedded \emph{``independently''} without being affected by missing values. Later, we demonstrate how our Missing Feature-Aware Attention Layer takes advantage of TFI embedding scheme to compute masked cross-attention among observed features at a time-step to account for missing features.

% \textbf{2D Positional Encodings:} We add Positional Encoding vectors $\mathbf{PE}$ to the TFI embedding $\mathbf{H}^{\mathrm{TFI}}$ to obtain positionally-encoded embeddings, $\mathbf{Z} = \mathbf{PE} + \mathbf{H}^{\mathrm{TFI}}$.
% % Since the TFI Embedding scheme maps each time-feature combination $\mathbf{X}_{(t, d)}$ into a higher-dimensional embedding,
% Since TFI embeddings treat every time-feature combination as a token, we use a 2D-positional encoding scheme  defined as follows:

% \begin{align}
%     &\texttt{PE}(t, d, 2i) = \sin \big(\frac{t}{10000^{(4i/D)}} \big) ; \quad \texttt{PE}(t, d, 2i+1) = \cos \big(\frac{t}{10000^{(4i/D)}} \big), \\ 
%     &\texttt{PE}(t, d, 2j+D/2) = \sin \big(\frac{d}{10000^{(4j/D)}} \big) ; \quad \texttt{PE}(t, d, 2j+1+D/2) = \cos \big(\frac{d}{10000^{(4j/D)}} \big),
% \end{align}
% where $t$ is the time-step, $d$ is the feature, and $i, j \in [0, D/4)$ are integers. 
% We add the 2D-positional embedding to our input embeddings to obtained the positionally-encoded embeddings denoted as $\mathbf{Z}$.

% is added to the TFI Embedding to obtain the hidden representation that is fed into subsequent layers: $\mathbf{Z}=\mathbf{H}^{\mathrm{TFI}} + \texttt{PE}$.

% To account for the extra dimension added due to the TFI embedding scheme, we adopt the 2D positional encoding scheme introduced in \textcolor{red}{XX. [ADD Formula for 2d PE.]}

\begin{figure}[ht]
    \centering
    \includegraphics[width=0.8\textwidth, height=6cm]{figures/MissTSM_framework.pdf}
    \caption{A schematic illustration of the overall MissTSM Framework with a zoomed-in view of the Missing Feature-Aware Attention (MFAA) Layer on the left.}
    \label{fig:mfa}
\end{figure}
\subsection{Missing Feature-Aware Attention (MFAA) Layer}
% \par \noindent \textbf{Motivation:} 
% Having learned TFI embeddings for every non-missing time-feature combination, 
We propose a novel \emph{Missing Feature-Aware Attention (MFAA) Layer} (see Figure \ref{fig:mfa}) to leverage the power of \emph{``masked-attention''} for learning latent representations at every time-step using partially observed features.
% account for missing values in some time-feature combinations. Specifically,
% % Recent advances of masked modeling in various domains including vision, language and time-series, have demonstrated that \emph{``masked-attention''} can be effectively utilized to reconstruct partially observed sequences. Motivated by it, 
% we ask the question: \emph{can we leverage the partial information of observed features at time-step $t$ to obtain a latent representation $\mathbf{L}_t$  using masked attention?} In response to this question, we introduce a novel \emph{Missing Feature-Aware Attention (MFAA) Layer} illustrated in Figure \ref{fig:mfa}. 
MFAA works by computing \emph{attention scores} based on the partially observed features at a time-step $t$, which are then used to perform a weighted sum of observed features to obtain the latent representation $\mathbf{L}_t$. 
These latent representations are later fed into an encoder-decoder based self-supervised learning framework to reconstruct the TS. 
% This is analogous to traditional imputation techniques that utilize cross-channel correlations for imputations. However, the difference in our framework is that once we have learned latent representations $\mathbf{L}_t$ using the paradigm of self-supervised learning, we can directly use them for downstream tasks without explicitly imputing the missing values. 
% \textcolor{red}{Mention that this technique is analogous to traditional imputation techniques that utilize cross-channel correlations for imputations.}

% at a given time-step $t$ using the missing-value mask $\mathcal{M}$. The weighted aggregation is performed using a method which is similar to a Masked Multi-Head Attention (MHAA) with some subtle differences.

\par \noindent \textbf{Mathematical Formulation:} To obtain \emph{attention scores} from partially-observed features at a time-step, we apply a masked scaled-dot product operation followed by a softmax operation. We first define a learnable query vector $\mathbf{Q} \in \mathbb{R}^{1 \times D}$ which is independent of the variates and time-steps. The positionally-encoded embeddings at time-step $t$, $\mathbf{Z}_{(t,:)}$, are used as key and value inputs in the MFAA Layer. Specifically, the query, key, and value vectors are obtained using linear projections as, $\hat{\mathbf{Q}}=\mathbf{Q}\mathbf{W^Q}, \quad \hat{\mathbf{K}}_t=\mathbf{Z}_{(t, :)}\mathbf{W^K}, \quad \hat{\mathbf{V}}_t=\mathbf{Z}_{(t, :)}\mathbf{W^V}$.
% \begin{equation}
%     \hat{\mathbf{Q}}=\mathbf{Q}\mathbf{W^Q}, \quad \hat{\mathbf{K}}_t=\mathbf{Z}_{(t, :)}\mathbf{W^K}, \quad \hat{\mathbf{V}}_t=\mathbf{Z}_{(t, :)}\mathbf{W^V},
% \end{equation}
Here, $\hat{\mathbf{Q}} \in \mathbb{R}^{1 \times d_k}$ and $\hat{\mathbf{K}}_t, \hat{\mathbf{V}}_t \in \mathbb{R}^{N \times d_k}$, where $d_k$ is the dimension of the vectors after linear projection. The linear projection matrices for the query, key, and values are defined as: $\mathbf{W^Q}, \mathbf{W^K}, \mathbf{W^V} \in \mathbb{R}^{D \times d_k}$ respectively. Note that the key $\hat{\mathbf{K}}_t$ and value $\hat{\mathbf{V}}_t$ vectors depend on the time-step $t$, while the query vector is learnable and doesn't change with time. This separation of roles is inspired by similar architectures in multi-modal grounding, for example, in \cite{detr}, learnable object queries are kept independent of the image content that is sent as keys and values. In our setting, the learnable queries capture the interactions among variates independent of time, enabling the model to attend to the most informative aspects of observed variates at any time-step fed through keys and values. We then define the MFAA Score at a given time-step $t$ as a masked scalar dot-product of query and key vector followed by normalization of the scores, defined as follows:
\begin{equation}
    \mathbf{A}_t = \texttt{MFAAScore}(\hat{\mathbf{Q}}, \hat{\mathbf{K}}_t, \mathcal{M}_{(t,:)}) = \texttt{Softmax}\Big( \frac{\hat{\mathbf{Q}}\hat{\mathbf{K}}_t^\top}{\sqrt d_k} + \eta \mathcal{M}_{(t,:)} \Big),
\end{equation}
where $\mathbf{A}_t \in \mathbb{R}^N$ is the MFAA Score vector of size $N$ corresponding to the $N$ variates, and $\eta$ is a large negative value. The negative bias term $\eta$ forces the masked-elements that correspond to the missing variates in TS to have an attention score of zero. Thus, by definition, the $i$-th element of the MFAA Score $\mathbf{A}_{(t, i)} \neq 0 \implies \mathcal{M}_{(t,:)} \neq 0$.
% by definition the attention scores of variates that are missing are zero, and the attention scores of the observed variated are normalized using the softmax operation. the final latent representations is a weighted aggregation of the observed variables for that particular time-step.
We then compute the latent representation $\mathbf{L}_t$ using $\mathbf{A}_t$ and the Value vector $\hat{\mathbf{V}}_t$ as, $\mathbf{L}_t = \texttt{MFAA}(\mathbf{A}_t, \hat{\mathbf{V}}_t) = \mathbf{A}_t\hat{\mathbf{V}}_t \in \mathbb{R}^{d_k}. $
% \begin{equation}
%     \mathbf{L}_t = \texttt{MFAA}(\mathbf{A}_t, \hat{\mathbf{V}}_t) = \mathbf{A}_t\hat{\mathbf{V}}_t \in \mathbb{R}^{d_k}
% \end{equation}
Similar to MHA used in transformers, we extend MFAA to multiple heads as: $\texttt{MultiHeadMFAA}(\mathbf{Q}, \mathbf{Z}_{(t, :)}, \mathcal{M}_{(t,:)}) = \mathrm{Concat}(\mathbf{L}^0_t, \mathbf{L}^1_t, ..., \mathbf{L}^h_t)\mathbf{W^O}$
% \begin{equation}
%     \texttt{MultiHeadMFAA}(\mathbf{Q}, \mathbf{Z}_{(t, :)}, \mathcal{M}_{(t,:)}) = \mathrm{Concat}(\mathbf{L}^0_t, \mathbf{L}^1_t, ..., \mathbf{L}^h_t)\mathbf{W^O}
% \end{equation}
, where $h$ is number of heads, $\mathbf{W^0} \in \mathbb{R}^{hd_k \times D_{o}}$, $\mathbf{L}^i_t$ is latent representation obtained from $i$-th $\texttt{MHAA}$ Layer, and $D_{o}$ is output-dimension of $\texttt{MultiHeadMFAA}$ Layer.


% \subsection{Putting Everything Together: Overall Framework of MissTSM}
% Figure \ref{fig:mfa} shows the overall framework of MissTSM. For an input time-series $\mathbf{X}$, we apply the TFI embedding layer followed by the MFAA layer to learn a latent representation for every time-step. We then integrate the latent representations into a Masked Auto-Encoder (MAE) \cite{he2022masked} framework adapted for time-series (similar to Ti-MAE \cite{li2023ti}). 
% % Specifically, we apply temporal masking to the latent representations before sending them to the encoder block. The encoder embeddings for the unmasked time-steps are then fed to the decoder block that learns the tokens for the masked time-steps in order to finally reconstruct the original time-series $\mathbf{X}$. 
% Although the design of the proposed TFI-Embedding and MFAA Layers are flexible enough that they can be integrated with any transformer-based time-series modeling framework (e.g., AutoFormer \cite{autoformer}), we opted for a masked time-series modeling approach (such as Ti-MAE \cite{li2023ti} and SimMTM \cite{simmtm}) due to their recent success in time-series modeling. Further, out of the several state-of-the-art masked time-series modelling techniques, we intentionally chose the simplest variation of MAE, namely Ti-MAE \cite{li2023ti}, to highlight the power of TFI and MFAA layers in handling missing values. 
% To the best of our knowledge, MissTSM is the first end-to-end framework for modeling time-series with missing values without explicitly imputing the time-series. Like a typical masked time-series modeling approach, MissTSM  has two main stages: (1) \emph{Self-Supervised Learning Stage:} where the multivariate time-series (with missing values) is reconstructed using an encoder-decoder architecture, with the goal of learning meaningful representations using just the encoder, and (2) \emph{Fine-tuning Stage:} where the latent representations learned by the encoder are fed into a multi-layer perceptron to perform downstream tasks of forecasting and classification. 

\subsection{Putting Everything Together: Overall Framework of MissTSM}
Figure \ref{fig:mfa} shows the MissTSM framework. We opted for a masked TS modeling approach (such as Ti-MAE \cite{li2023ti}) due to their recent success.
% , and intentionally chose the simplest variation of MAE, namely Ti-MAE \cite{li2023ti}.
% , to highlight the power of TFI and MFAA layers in handling missing values. 
MissTSM  has two main stages: (1) \emph{Self-Supervised Learning Stage:} where multivariate TS (with missing values) is reconstructed using an encoder-decoder architecture, with the goal of learning meaningful representations, (2) \emph{Fine-tuning Stage:} where latent representations learned by encoder are fed into a MLP to perform downstream tasks. 

% Our self-supervised training architecture is similar to Ti-MAE.


% In addition to the TFI Embedding scheme, and MFAA Layer to obtain latent representations at every time-step,
% \subsection{Motivation/ Problem Definition}

% Theoretical insights about the advantages of end-to-end learning with missing values as compared to two-stage approaches? Look into maths of bias/variance trade-offs.

% Should we add some experiment for motivation?



% \subsection{Method Innovations/ Structure Overivew}

% Step-by-step techniques:

% \textbf{Pretraining Model Architecture}
% \begin{enumerate}
%     \item Using the time-series instance-normalization technique (cite paper)
%     \item \textcolor{red}{Feature Time-Independent Embedding:} 
%     $B, L, n_{feat} \rightarrow B, L, n_{feat}, D$, this projects each point to a higher dimensional representation (Add figure similar to iTransformer) [\href{https://arxiv.org/pdf/2203.05556}{Reference for embedding scalar numerical features}]
%     Motivation is missing values.
%     \item 2D-positional Embedding (add citation of paper from github)
%     \item \textcolor{red}{Missing Feature-Aware Attention (MFAA)} 
%     \item MAE style Masking
%     \item Transformer Encoder
%     \item Transformer Decoder
% \end{enumerate}

% \textbf{Fine-tuning Model}: MAE style Flattened MLP for prediction. (similar to SimMTM)

% Add schematic for MAE-style model architecture. the MFAA can be part of the schematic





% \section{Experiments}
\label{sec:experiments}
The experiments are designed to address two key research questions.
First, \textbf{RQ1} evaluates whether the average $L_2$-norm of the counterfactual perturbation vectors ($\overline{||\perturb||}$) decreases as the model overfits the data, thereby providing further empirical validation for our hypothesis.
Second, \textbf{RQ2} evaluates the ability of the proposed counterfactual regularized loss, as defined in (\ref{eq:regularized_loss2}), to mitigate overfitting when compared to existing regularization techniques.

% The experiments are designed to address three key research questions. First, \textbf{RQ1} investigates whether the mean perturbation vector norm decreases as the model overfits the data, aiming to further validate our intuition. Second, \textbf{RQ2} explores whether the mean perturbation vector norm can be effectively leveraged as a regularization term during training, offering insights into its potential role in mitigating overfitting. Finally, \textbf{RQ3} examines whether our counterfactual regularizer enables the model to achieve superior performance compared to existing regularization methods, thus highlighting its practical advantage.

\subsection{Experimental Setup}
\textbf{\textit{Datasets, Models, and Tasks.}}
The experiments are conducted on three datasets: \textit{Water Potability}~\cite{kadiwal2020waterpotability}, \textit{Phomene}~\cite{phomene}, and \textit{CIFAR-10}~\cite{krizhevsky2009learning}. For \textit{Water Potability} and \textit{Phomene}, we randomly select $80\%$ of the samples for the training set, and the remaining $20\%$ for the test set, \textit{CIFAR-10} comes already split. Furthermore, we consider the following models: Logistic Regression, Multi-Layer Perceptron (MLP) with 100 and 30 neurons on each hidden layer, and PreactResNet-18~\cite{he2016cvecvv} as a Convolutional Neural Network (CNN) architecture.
We focus on binary classification tasks and leave the extension to multiclass scenarios for future work. However, for datasets that are inherently multiclass, we transform the problem into a binary classification task by selecting two classes, aligning with our assumption.

\smallskip
\noindent\textbf{\textit{Evaluation Measures.}} To characterize the degree of overfitting, we use the test loss, as it serves as a reliable indicator of the model's generalization capability to unseen data. Additionally, we evaluate the predictive performance of each model using the test accuracy.

\smallskip
\noindent\textbf{\textit{Baselines.}} We compare CF-Reg with the following regularization techniques: L1 (``Lasso''), L2 (``Ridge''), and Dropout.

\smallskip
\noindent\textbf{\textit{Configurations.}}
For each model, we adopt specific configurations as follows.
\begin{itemize}
\item \textit{Logistic Regression:} To induce overfitting in the model, we artificially increase the dimensionality of the data beyond the number of training samples by applying a polynomial feature expansion. This approach ensures that the model has enough capacity to overfit the training data, allowing us to analyze the impact of our counterfactual regularizer. The degree of the polynomial is chosen as the smallest degree that makes the number of features greater than the number of data.
\item \textit{Neural Networks (MLP and CNN):} To take advantage of the closed-form solution for computing the optimal perturbation vector as defined in (\ref{eq:opt-delta}), we use a local linear approximation of the neural network models. Hence, given an instance $\inst_i$, we consider the (optimal) counterfactual not with respect to $\model$ but with respect to:
\begin{equation}
\label{eq:taylor}
    \model^{lin}(\inst) = \model(\inst_i) + \nabla_{\inst}\model(\inst_i)(\inst - \inst_i),
\end{equation}
where $\model^{lin}$ represents the first-order Taylor approximation of $\model$ at $\inst_i$.
Note that this step is unnecessary for Logistic Regression, as it is inherently a linear model.
\end{itemize}

\smallskip
\noindent \textbf{\textit{Implementation Details.}} We run all experiments on a machine equipped with an AMD Ryzen 9 7900 12-Core Processor and an NVIDIA GeForce RTX 4090 GPU. Our implementation is based on the PyTorch Lightning framework. We use stochastic gradient descent as the optimizer with a learning rate of $\eta = 0.001$ and no weight decay. We use a batch size of $128$. The training and test steps are conducted for $6000$ epochs on the \textit{Water Potability} and \textit{Phoneme} datasets, while for the \textit{CIFAR-10} dataset, they are performed for $200$ epochs.
Finally, the contribution $w_i^{\varepsilon}$ of each training point $\inst_i$ is uniformly set as $w_i^{\varepsilon} = 1~\forall i\in \{1,\ldots,m\}$.

The source code implementation for our experiments is available at the following GitHub repository: \url{https://anonymous.4open.science/r/COCE-80B4/README.md} 

\subsection{RQ1: Counterfactual Perturbation vs. Overfitting}
To address \textbf{RQ1}, we analyze the relationship between the test loss and the average $L_2$-norm of the counterfactual perturbation vectors ($\overline{||\perturb||}$) over training epochs.

In particular, Figure~\ref{fig:delta_loss_epochs} depicts the evolution of $\overline{||\perturb||}$ alongside the test loss for an MLP trained \textit{without} regularization on the \textit{Water Potability} dataset. 
\begin{figure}[ht]
    \centering
    \includegraphics[width=0.85\linewidth]{img/delta_loss_epochs.png}
    \caption{The average counterfactual perturbation vector $\overline{||\perturb||}$ (left $y$-axis) and the cross-entropy test loss (right $y$-axis) over training epochs ($x$-axis) for an MLP trained on the \textit{Water Potability} dataset \textit{without} regularization.}
    \label{fig:delta_loss_epochs}
\end{figure}

The plot shows a clear trend as the model starts to overfit the data (evidenced by an increase in test loss). 
Notably, $\overline{||\perturb||}$ begins to decrease, which aligns with the hypothesis that the average distance to the optimal counterfactual example gets smaller as the model's decision boundary becomes increasingly adherent to the training data.

It is worth noting that this trend is heavily influenced by the choice of the counterfactual generator model. In particular, the relationship between $\overline{||\perturb||}$ and the degree of overfitting may become even more pronounced when leveraging more accurate counterfactual generators. However, these models often come at the cost of higher computational complexity, and their exploration is left to future work.

Nonetheless, we expect that $\overline{||\perturb||}$ will eventually stabilize at a plateau, as the average $L_2$-norm of the optimal counterfactual perturbations cannot vanish to zero.

% Additionally, the choice of employing the score-based counterfactual explanation framework to generate counterfactuals was driven to promote computational efficiency.

% Future enhancements to the framework may involve adopting models capable of generating more precise counterfactuals. While such approaches may yield to performance improvements, they are likely to come at the cost of increased computational complexity.


\subsection{RQ2: Counterfactual Regularization Performance}
To answer \textbf{RQ2}, we evaluate the effectiveness of the proposed counterfactual regularization (CF-Reg) by comparing its performance against existing baselines: unregularized training loss (No-Reg), L1 regularization (L1-Reg), L2 regularization (L2-Reg), and Dropout.
Specifically, for each model and dataset combination, Table~\ref{tab:regularization_comparison} presents the mean value and standard deviation of test accuracy achieved by each method across 5 random initialization. 

The table illustrates that our regularization technique consistently delivers better results than existing methods across all evaluated scenarios, except for one case -- i.e., Logistic Regression on the \textit{Phomene} dataset. 
However, this setting exhibits an unusual pattern, as the highest model accuracy is achieved without any regularization. Even in this case, CF-Reg still surpasses other regularization baselines.

From the results above, we derive the following key insights. First, CF-Reg proves to be effective across various model types, ranging from simple linear models (Logistic Regression) to deep architectures like MLPs and CNNs, and across diverse datasets, including both tabular and image data. 
Second, CF-Reg's strong performance on the \textit{Water} dataset with Logistic Regression suggests that its benefits may be more pronounced when applied to simpler models. However, the unexpected outcome on the \textit{Phoneme} dataset calls for further investigation into this phenomenon.


\begin{table*}[h!]
    \centering
    \caption{Mean value and standard deviation of test accuracy across 5 random initializations for different model, dataset, and regularization method. The best results are highlighted in \textbf{bold}.}
    \label{tab:regularization_comparison}
    \begin{tabular}{|c|c|c|c|c|c|c|}
        \hline
        \textbf{Model} & \textbf{Dataset} & \textbf{No-Reg} & \textbf{L1-Reg} & \textbf{L2-Reg} & \textbf{Dropout} & \textbf{CF-Reg (ours)} \\ \hline
        Logistic Regression   & \textit{Water}   & $0.6595 \pm 0.0038$   & $0.6729 \pm 0.0056$   & $0.6756 \pm 0.0046$  & N/A    & $\mathbf{0.6918 \pm 0.0036}$                     \\ \hline
        MLP   & \textit{Water}   & $0.6756 \pm 0.0042$   & $0.6790 \pm 0.0058$   & $0.6790 \pm 0.0023$  & $0.6750 \pm 0.0036$    & $\mathbf{0.6802 \pm 0.0046}$                    \\ \hline
%        MLP   & \textit{Adult}   & $0.8404 \pm 0.0010$   & $\mathbf{0.8495 \pm 0.0007}$   & $0.8489 \pm 0.0014$  & $\mathbf{0.8495 \pm 0.0016}$     & $0.8449 \pm 0.0019$                    \\ \hline
        Logistic Regression   & \textit{Phomene}   & $\mathbf{0.8148 \pm 0.0020}$   & $0.8041 \pm 0.0028$   & $0.7835 \pm 0.0176$  & N/A    & $0.8098 \pm 0.0055$                     \\ \hline
        MLP   & \textit{Phomene}   & $0.8677 \pm 0.0033$   & $0.8374 \pm 0.0080$   & $0.8673 \pm 0.0045$  & $0.8672 \pm 0.0042$     & $\mathbf{0.8718 \pm 0.0040}$                    \\ \hline
        CNN   & \textit{CIFAR-10} & $0.6670 \pm 0.0233$   & $0.6229 \pm 0.0850$   & $0.7348 \pm 0.0365$   & N/A    & $\mathbf{0.7427 \pm 0.0571}$                     \\ \hline
    \end{tabular}
\end{table*}

\begin{table*}[htb!]
    \centering
    \caption{Hyperparameter configurations utilized for the generation of Table \ref{tab:regularization_comparison}. For our regularization the hyperparameters are reported as $\mathbf{\alpha/\beta}$.}
    \label{tab:performance_parameters}
    \begin{tabular}{|c|c|c|c|c|c|c|}
        \hline
        \textbf{Model} & \textbf{Dataset} & \textbf{No-Reg} & \textbf{L1-Reg} & \textbf{L2-Reg} & \textbf{Dropout} & \textbf{CF-Reg (ours)} \\ \hline
        Logistic Regression   & \textit{Water}   & N/A   & $0.0093$   & $0.6927$  & N/A    & $0.3791/1.0355$                     \\ \hline
        MLP   & \textit{Water}   & N/A   & $0.0007$   & $0.0022$  & $0.0002$    & $0.2567/1.9775$                    \\ \hline
        Logistic Regression   &
        \textit{Phomene}   & N/A   & $0.0097$   & $0.7979$  & N/A    & $0.0571/1.8516$                     \\ \hline
        MLP   & \textit{Phomene}   & N/A   & $0.0007$   & $4.24\cdot10^{-5}$  & $0.0015$    & $0.0516/2.2700$                    \\ \hline
       % MLP   & \textit{Adult}   & N/A   & $0.0018$   & $0.0018$  & $0.0601$     & $0.0764/2.2068$                    \\ \hline
        CNN   & \textit{CIFAR-10} & N/A   & $0.0050$   & $0.0864$ & N/A    & $0.3018/
        2.1502$                     \\ \hline
    \end{tabular}
\end{table*}

\begin{table*}[htb!]
    \centering
    \caption{Mean value and standard deviation of training time across 5 different runs. The reported time (in seconds) corresponds to the generation of each entry in Table \ref{tab:regularization_comparison}. Times are }
    \label{tab:times}
    \begin{tabular}{|c|c|c|c|c|c|c|}
        \hline
        \textbf{Model} & \textbf{Dataset} & \textbf{No-Reg} & \textbf{L1-Reg} & \textbf{L2-Reg} & \textbf{Dropout} & \textbf{CF-Reg (ours)} \\ \hline
        Logistic Regression   & \textit{Water}   & $222.98 \pm 1.07$   & $239.94 \pm 2.59$   & $241.60 \pm 1.88$  & N/A    & $251.50 \pm 1.93$                     \\ \hline
        MLP   & \textit{Water}   & $225.71 \pm 3.85$   & $250.13 \pm 4.44$   & $255.78 \pm 2.38$  & $237.83 \pm 3.45$    & $266.48 \pm 3.46$                    \\ \hline
        Logistic Regression   & \textit{Phomene}   & $266.39 \pm 0.82$ & $367.52 \pm 6.85$   & $361.69 \pm 4.04$  & N/A   & $310.48 \pm 0.76$                    \\ \hline
        MLP   &
        \textit{Phomene} & $335.62 \pm 1.77$   & $390.86 \pm 2.11$   & $393.96 \pm 1.95$ & $363.51 \pm 5.07$    & $403.14 \pm 1.92$                     \\ \hline
       % MLP   & \textit{Adult}   & N/A   & $0.0018$   & $0.0018$  & $0.0601$     & $0.0764/2.2068$                    \\ \hline
        CNN   & \textit{CIFAR-10} & $370.09 \pm 0.18$   & $395.71 \pm 0.55$   & $401.38 \pm 0.16$ & N/A    & $1287.8 \pm 0.26$                     \\ \hline
    \end{tabular}
\end{table*}

\subsection{Feasibility of our Method}
A crucial requirement for any regularization technique is that it should impose minimal impact on the overall training process.
In this respect, CF-Reg introduces an overhead that depends on the time required to find the optimal counterfactual example for each training instance. 
As such, the more sophisticated the counterfactual generator model probed during training the higher would be the time required. However, a more advanced counterfactual generator might provide a more effective regularization. We discuss this trade-off in more details in Section~\ref{sec:discussion}.

Table~\ref{tab:times} presents the average training time ($\pm$ standard deviation) for each model and dataset combination listed in Table~\ref{tab:regularization_comparison}.
We can observe that the higher accuracy achieved by CF-Reg using the score-based counterfactual generator comes with only minimal overhead. However, when applied to deep neural networks with many hidden layers, such as \textit{PreactResNet-18}, the forward derivative computation required for the linearization of the network introduces a more noticeable computational cost, explaining the longer training times in the table.

\subsection{Hyperparameter Sensitivity Analysis}
The proposed counterfactual regularization technique relies on two key hyperparameters: $\alpha$ and $\beta$. The former is intrinsic to the loss formulation defined in (\ref{eq:cf-train}), while the latter is closely tied to the choice of the score-based counterfactual explanation method used.

Figure~\ref{fig:test_alpha_beta} illustrates how the test accuracy of an MLP trained on the \textit{Water Potability} dataset changes for different combinations of $\alpha$ and $\beta$.

\begin{figure}[ht]
    \centering
    \includegraphics[width=0.85\linewidth]{img/test_acc_alpha_beta.png}
    \caption{The test accuracy of an MLP trained on the \textit{Water Potability} dataset, evaluated while varying the weight of our counterfactual regularizer ($\alpha$) for different values of $\beta$.}
    \label{fig:test_alpha_beta}
\end{figure}

We observe that, for a fixed $\beta$, increasing the weight of our counterfactual regularizer ($\alpha$) can slightly improve test accuracy until a sudden drop is noticed for $\alpha > 0.1$.
This behavior was expected, as the impact of our penalty, like any regularization term, can be disruptive if not properly controlled.

Moreover, this finding further demonstrates that our regularization method, CF-Reg, is inherently data-driven. Therefore, it requires specific fine-tuning based on the combination of the model and dataset at hand.
\section{Experiments}

% We compare MissTSM with two-stage baseline methods on two downstream tasks: forecasting and classification, under varying scenarios of missing values.
\par
\textbf{Datasets and Baselines:} We consider three popular TS forecasting datasets: ETTh2 \cite{informer}, ETTm2 \cite{informer} and Weather \cite{weather}. For classification, we use three real-world datasets, namely, Epilepsy, EMG, and Gesture. We follow the evaluation setup proposed in AutoFormer \cite{autoformer} for the forecasting datasets and evaluation setup proposed in TF-C \cite{tfc} for the classification ones.
. For our experiments, we consider five SOTA TS-modeling baselines, SimMTM \cite{simmtm}, PatchTST \cite{patchtst}, Autoformer \cite{autoformer}, iTransformer \cite{liu2023itransformer} and DLinear \cite{dlinear}. In order to apply these methods on data with missing values, we consider four imputation techniques---a simple $2^\text{nd}$-order spline imputation, k-nearest neighbors (kNN) and two state-of-the-art imputation techniques, SAITS \cite{saits} and BRITS \cite{cao2018brits}. 
% We compare MissTSM's performance against time-series models trained on data imputed using the imputation methods. For forecasting, we used a default lookback window of $L$ = 336, and vary the horizon windows as $T \in$ \{96, 192, 336, 720\}. 
For more experimental details, refer to Appendix \ref{appendix:B.3}

% \par
% \textbf{Baselines:} For all of our experiments, we consider two state-of-the-art time-series modeling baselines, SimMTM \cite{simmtm} and PatchTST \cite{patchtst}. In order to apply these methods on data with missing values, we also consider two imputation techniques---a simple $2^\text{nd}$-order spline imputation, and a state-of-the-art transformer-based imputation technique, SAITS \cite{saits}. We compare MissTSM's performance against SimMTM and PatchTST models trained and tested on data imputed using both SAITS and spline interpolations. The same evaluation setup is used for all baselines. For forecasting, we used a default lookback window of $L$ = 336, while we varied the horizon windows as $T \in$ \{96, 192, 336, 720\}.  

% \begin{table}[t]
%  \caption{Comparing forecasting performance of baseline methods using mean squared error (MSE) as the evaluation metric under no masking, MCAR masking, and periodic masking. For every dataset, we consider multiple forecasting horizons, $T \in \{96, 192, 336, 720\}$. We highlight the best performing model in dark blue and the $2^\text{nd}$-best model using light blue for every masking experiment.}
% \label{tab:main_table}
% \renewcommand{\arraystretch}{1.5}
% \resizebox{\textwidth}{!}{
% \begin{tabular}{lllllllllllllll}
% \hline
% \rowcolor[HTML]{FFFFFF} 
%                                                   &                                                  & \multicolumn{3}{l}{\cellcolor[HTML]{FFFFFF}}                                                                                                                    & \multicolumn{5}{c}{\cellcolor[HTML]{FFFFFF}MCAR Masking}                                                                                                                                                                                                                                 & \multicolumn{5}{c}{\cellcolor[HTML]{FFFFFF}Periodic Masking}                                                                                                                                                                                    \\ \cline{6-15} 
% \rowcolor[HTML]{FFFFFF} 
%                                                   &                                                  & \multicolumn{3}{c}{\cellcolor[HTML]{FFFFFF}No Masking}                                                                                                          & \multicolumn{1}{c}{\cellcolor[HTML]{FFFFFF}}                          & \multicolumn{2}{c}{\cellcolor[HTML]{FFFFFF}SimMTM}                                                     & \multicolumn{2}{c}{\cellcolor[HTML]{FFFFFF}PatchTST}                                                    & \cellcolor[HTML]{FFFFFF}                          & \multicolumn{2}{c}{\cellcolor[HTML]{FFFFFF}SimMTM}                                                     & \multicolumn{2}{c}{\cellcolor[HTML]{FFFFFF}PatchTST}                               \\ \cline{3-5} \cline{7-10} \cline{12-15} 
% \rowcolor[HTML]{FFFFFF} 
%                                                   &                                                  & \multicolumn{1}{c}{\cellcolor[HTML]{FFFFFF}MissTSM} & \multicolumn{1}{c}{\cellcolor[HTML]{FFFFFF}SimMTM} & \multicolumn{1}{c}{\cellcolor[HTML]{FFFFFF}PatchTST} & \multicolumn{1}{c}{\multirow{-2}{*}{\cellcolor[HTML]{FFFFFF}MissTSM}} & \multicolumn{1}{c}{\cellcolor[HTML]{FFFFFF}Spline} & \multicolumn{1}{c}{\cellcolor[HTML]{FFFFFF}SAITS} & \multicolumn{1}{c}{\cellcolor[HTML]{FFFFFF}Spline} & \multicolumn{1}{c}{\cellcolor[HTML]{FFFFFF}SAITS}  & \multirow{-2}{*}{\cellcolor[HTML]{FFFFFF}MissTSM} & \multicolumn{1}{c}{\cellcolor[HTML]{FFFFFF}Spline} & \multicolumn{1}{c}{\cellcolor[HTML]{FFFFFF}SAITS} & \multicolumn{1}{c}{\cellcolor[HTML]{FFFFFF}Spline} & SAITS                         \\ \hline
% \rowcolor[HTML]{FFFFFF} 
% \cellcolor[HTML]{FFFFFF}                          & \multicolumn{1}{l|}{\cellcolor[HTML]{FFFFFF}96}  & \cellcolor[HTML]{9698ED}0.255                       & 0.295                                              & \multicolumn{1}{l|}{\cellcolor[HTML]{DAE8FC}0.274}   & 0.250                                                                 & 0.317                                              & \cellcolor[HTML]{DAE8FC}0.243                     & 0.299                                              & \multicolumn{1}{l|}{\cellcolor[HTML]{9698ED}0.224} & \cellcolor[HTML]{9698ED}0.259                     & 0.382                                              & 0.357                                             & \cellcolor[HTML]{DAE8FC}0.314                      & 0.343                         \\
% \rowcolor[HTML]{FFFFFF} 
% \cellcolor[HTML]{FFFFFF}                          & \multicolumn{1}{l|}{\cellcolor[HTML]{FFFFFF}192} & \cellcolor[HTML]{9698ED}0.234                       & 0.356                                              & \multicolumn{1}{l|}{\cellcolor[HTML]{DAE8FC}0.338}   & \cellcolor[HTML]{DAE8FC}0.303                                         & 0.373                                              & 0.315                                             & 0.36                                               & \multicolumn{1}{l|}{\cellcolor[HTML]{9698ED}0.298} & \cellcolor[HTML]{9698ED}0.321                     & 0.422                                              & 0.47                                              & \cellcolor[HTML]{DAE8FC}0.411                      & 0.431                         \\
% \rowcolor[HTML]{FFFFFF} 
% \cellcolor[HTML]{FFFFFF}                          & \multicolumn{1}{l|}{\cellcolor[HTML]{FFFFFF}336} & \cellcolor[HTML]{9698ED}0.316                       & 0.375                                              & \multicolumn{1}{l|}{\cellcolor[HTML]{DAE8FC}0.33}    & \cellcolor[HTML]{9698ED}0.277                                         & 0.390                                              & 0.342                                             & 0.336                                              & \multicolumn{1}{l|}{\cellcolor[HTML]{DAE8FC}0.28}  & \cellcolor[HTML]{9698ED}0.303                     & 0.427                                              & 0.475                                             & \cellcolor[HTML]{DAE8FC}0.383                      & 0.443                         \\
% \rowcolor[HTML]{FFFFFF} 
% \multirow{-4}{*}{\cellcolor[HTML]{FFFFFF}ETTh2}   & \multicolumn{1}{l|}{\cellcolor[HTML]{FFFFFF}720} & \cellcolor[HTML]{9698ED}0.305                       & 0.404                                              & \multicolumn{1}{l|}{\cellcolor[HTML]{DAE8FC}0.378}   & \cellcolor[HTML]{9698ED}0.318                                         & 0.404                                              & 0.36                                              & 0.377                                              & \multicolumn{1}{l|}{\cellcolor[HTML]{DAE8FC}0.336} & \cellcolor[HTML]{9698ED}0.382                     & 0.440                                              & 0.497                                             & \cellcolor[HTML]{DAE8FC}0.43                       & 0.49                          \\ \hline
% \rowcolor[HTML]{FFFFFF} 
% \cellcolor[HTML]{FFFFFF}                          & \multicolumn{1}{l|}{\cellcolor[HTML]{FFFFFF}96}  & 0.183                                               & \cellcolor[HTML]{DAE8FC}0.172                      & \multicolumn{1}{l|}{\cellcolor[HTML]{9698ED}0.164}   & 0.216                                                                 & 0.169                                              & 0.173                                             & \cellcolor[HTML]{9698ED}0.155                      & \multicolumn{1}{l|}{\cellcolor[HTML]{DAE8FC}0.161} & 0.252                                             & 0.170                                              & 0.178                                             & \cellcolor[HTML]{9698ED}0.159                      & \cellcolor[HTML]{DAE8FC}0.165 \\
% \rowcolor[HTML]{FFFFFF} 
% \cellcolor[HTML]{FFFFFF}                          & \multicolumn{1}{l|}{\cellcolor[HTML]{FFFFFF}192} & \cellcolor[HTML]{9698ED}0.209                       & 0.223                                              & \multicolumn{1}{l|}{\cellcolor[HTML]{DAE8FC}0.22}    & 0.275                                                                 & 0.228                                              & 0.233                                             & \cellcolor[HTML]{9698ED}0.215                      & \multicolumn{1}{l|}{\cellcolor[HTML]{DAE8FC}0.223} & 0.254                                             & \cellcolor[HTML]{DAE8FC}0.228                      & 0.24                                              & \cellcolor[HTML]{9698ED}0.218                      & 0.233                         \\
% \rowcolor[HTML]{FFFFFF} 
% \cellcolor[HTML]{FFFFFF}                          & \multicolumn{1}{l|}{\cellcolor[HTML]{FFFFFF}336} & \cellcolor[HTML]{9698ED}0.261                       & 0.282                                              & \multicolumn{1}{l|}{\cellcolor[HTML]{DAE8FC}0.277}   & \cellcolor[HTML]{DAE8FC}0.281                                         & 0.284                                              & 0.286                                             & \cellcolor[HTML]{9698ED}0.274                      & \multicolumn{1}{l|}{\cellcolor[HTML]{FFFFFF}0.286} & 0.299                                             & \cellcolor[HTML]{DAE8FC}0.285                      & 0.297                                             & \cellcolor[HTML]{9698ED}0.276                      & 0.296                         \\
% \rowcolor[HTML]{FFFFFF} 
% \multirow{-4}{*}{\cellcolor[HTML]{FFFFFF}ETTm2}   & \multicolumn{1}{l|}{\cellcolor[HTML]{FFFFFF}720} & \cellcolor[HTML]{9698ED}0.311                       & 0.374                                              & \multicolumn{1}{l|}{\cellcolor[HTML]{DAE8FC}0.367}   & \cellcolor[HTML]{9698ED}0.300                                         & 0.373                                              & 0.386                                             & \cellcolor[HTML]{DAE8FC}0.366                      & \multicolumn{1}{l|}{\cellcolor[HTML]{FFFFFF}0.383} & \cellcolor[HTML]{9698ED}0.310                     & 0.377                                              & 0.406                                             & \cellcolor[HTML]{DAE8FC}0.369                      & 0.404                         \\ \hline
% \rowcolor[HTML]{FFFFFF} 
% \cellcolor[HTML]{FFFFFF}                          & \multicolumn{1}{l|}{\cellcolor[HTML]{FFFFFF}96}  & 0.164                                               & \cellcolor[HTML]{DAE8FC}0.158                      & \multicolumn{1}{l|}{\cellcolor[HTML]{9698ED}0.151}   & 0.191                                                                 & 0.208                                              & \cellcolor[HTML]{DAE8FC}0.18                      & 0.208                                              & \multicolumn{1}{l|}{\cellcolor[HTML]{9698ED}0.175} & 0.198                                             & \cellcolor[HTML]{DAE8FC}0.142                      & 0.177                                             & \cellcolor[HTML]{9698ED}0.141                      & 0.177                         \\
% \rowcolor[HTML]{FFFFFF} 
% \cellcolor[HTML]{FFFFFF}                          & \multicolumn{1}{l|}{\cellcolor[HTML]{FFFFFF}192} & 0.210                                               & \cellcolor[HTML]{DAE8FC}0.199                      & \multicolumn{1}{l|}{\cellcolor[HTML]{9698ED}0.196}   & \cellcolor[HTML]{DAE8FC}0.233                                         & 0.260                                              & \cellcolor[HTML]{DAE8FC}0.233                     & 0.257                                              & \multicolumn{1}{l|}{\cellcolor[HTML]{9698ED}0.224} & 0.236                                             & \cellcolor[HTML]{9698ED}0.192                      & 0.225                                             & \cellcolor[HTML]{DAE8FC}0.193                      & 0.226                         \\
% \rowcolor[HTML]{FFFFFF} 
% \cellcolor[HTML]{FFFFFF}                          & \multicolumn{1}{l|}{\cellcolor[HTML]{FFFFFF}336} & 0.254                                               & \cellcolor[HTML]{9698ED}0.246                      & \multicolumn{1}{l|}{\cellcolor[HTML]{DAE8FC}0.249}   & \cellcolor[HTML]{DAE8FC}0.284                                         & 0.314                                              & \cellcolor[HTML]{DAE8FC}0.281                     & 0.31                                               & \multicolumn{1}{l|}{\cellcolor[HTML]{9698ED}0.277} & 0.279                                             & \cellcolor[HTML]{DAE8FC}0.249                      & 0.277                                             & \cellcolor[HTML]{9698ED}0.246                      & 0.279                         \\
% \rowcolor[HTML]{FFFFFF} 
% \multirow{-4}{*}{\cellcolor[HTML]{FFFFFF}Weather} & \multicolumn{1}{l|}{\cellcolor[HTML]{FFFFFF}720} & 0.324                                               & \cellcolor[HTML]{9698ED}0.317                      & \multicolumn{1}{l|}{\cellcolor[HTML]{DAE8FC}0.319}   & \cellcolor[HTML]{9698ED}0.310                                         & 0.388                                              & \cellcolor[HTML]{DAE8FC}0.352                     & 0.381                                              & \multicolumn{1}{l|}{\cellcolor[HTML]{FFFFFF}0.352} & \cellcolor[HTML]{DAE8FC}0.333                     & \cellcolor[HTML]{9698ED}0.327                      & 0.35                                              & \cellcolor[HTML]{9698ED}0.327                      & 0.355                         \\ \hline
% \end{tabular}}
% \end{table}

% \par
% \textbf{Patch Masking:} This setting simulates real-world applications where sensors are prone to random failures, which is then followed by a stretch of repair time (in our case, the masked patch), the length of which is again, randomly selected within a fixed set of boundaries.


% \begin{table*}[ht]
% \caption{Comparing forecasting performance of baseline methods under MCAR and Periodic Masking. Best performing model in dark blue and the $2^\text{nd}$-best model in light blue}
% \label{tab:main_table}
% \renewcommand{\arraystretch}{1.0}
% \scriptsize
% \resizebox{\textwidth}{!}{\begin{tabular}{llllllllllll}
% \hline
% \rowcolor[HTML]{FFFFFF} 
%                                                   &                                                  & \multicolumn{5}{c}{\cellcolor[HTML]{FFFFFF}MCAR Masking}                                                                                                                                                                                                                                          & \multicolumn{5}{c}{\cellcolor[HTML]{FFFFFF}Periodic Masking}                                                                                                                                                                                             \\ \cline{3-12} 
% \rowcolor[HTML]{FFFFFF} 
%                                                   &                                                  & \multicolumn{1}{c}{\cellcolor[HTML]{FFFFFF}}                          & \multicolumn{2}{c}{\cellcolor[HTML]{FFFFFF}SimMTM}                                                     & \multicolumn{2}{c}{\cellcolor[HTML]{FFFFFF}PatchTST}                                                             & \cellcolor[HTML]{FFFFFF}                          & \multicolumn{2}{c}{\cellcolor[HTML]{FFFFFF}SimMTM}                                                     & \multicolumn{2}{c}{\cellcolor[HTML]{FFFFFF}PatchTST}                                        \\ \cline{4-7} \cline{9-12} 
% \rowcolor[HTML]{FFFFFF} 
%                                                   &                                                  & \multicolumn{1}{c}{\multirow{-2}{*}{\cellcolor[HTML]{FFFFFF}MissTSM}} & \multicolumn{1}{c}{\cellcolor[HTML]{FFFFFF}Spline} & \multicolumn{1}{c}{\cellcolor[HTML]{FFFFFF}SAITS} & \multicolumn{1}{c}{\cellcolor[HTML]{FFFFFF}Spline} & \multicolumn{1}{c}{\cellcolor[HTML]{FFFFFF}SAITS}           & \multirow{-2}{*}{\cellcolor[HTML]{FFFFFF}MissTSM} & \multicolumn{1}{c}{\cellcolor[HTML]{FFFFFF}Spline} & \multicolumn{1}{c}{\cellcolor[HTML]{FFFFFF}SAITS} & \multicolumn{1}{c}{\cellcolor[HTML]{FFFFFF}Spline} & SAITS                                  \\ \hline
% \rowcolor[HTML]{FFFFFF} 
% \cellcolor[HTML]{FFFFFF}                          & \multicolumn{1}{l|}{\cellcolor[HTML]{FFFFFF}96}  & \cellcolor[HTML]{9698ED}0.266 ± 0.0146                                & 0.349 ± 0.0191                                     & \cellcolor[HTML]{DAE8FC}0.267 ± 0.0308            & 0.309 ± 0.0087                                     & \multicolumn{1}{l|}{\cellcolor[HTML]{FFFFFF}0.413 ± 0.0339} & \cellcolor[HTML]{9698ED}0.268 ± 0.0151            & 0.646 ± 0.1716                                     & \cellcolor[HTML]{DAE8FC}0.319 ± 0.0659            & 0.492 ± 0.0825                                     & 0.599 ± 0.2110                         \\
% \rowcolor[HTML]{FFFFFF} 
% \cellcolor[HTML]{FFFFFF}                          & \multicolumn{1}{l|}{\cellcolor[HTML]{FFFFFF}192} & \cellcolor[HTML]{9698ED}0.283 ± 0.0123                                & 0.414 ± 0.0232                                     & \cellcolor[HTML]{DAE8FC}0.360 ± 0.0465            & 0.370 ± 0.0076                                     & \multicolumn{1}{l|}{\cellcolor[HTML]{FFFFFF}0.495 ± 0.0376} & \cellcolor[HTML]{9698ED}0.295 ± 0.0298            & 0.729 ± 0.1913                                     & \cellcolor[HTML]{DAE8FC}0.408 ± 0.0865            & 0.510 ± 0.0578                                     & 0.672 ± 0.2113                         \\
% \rowcolor[HTML]{FFFFFF} 
% \cellcolor[HTML]{FFFFFF}                          & \multicolumn{1}{l|}{\cellcolor[HTML]{FFFFFF}336} & \cellcolor[HTML]{9698ED}0.287 ± 0.0142                                & 0.435 ± 0.0297                                     & 0.394 ± 0.0542                                    & \cellcolor[HTML]{DAE8FC}0.352 ± 0.0062             & \multicolumn{1}{l|}{\cellcolor[HTML]{FFFFFF}0.492 ± 0.0450} & \cellcolor[HTML]{9698ED}0.319 ± 0.0185            & 0.751 ± 0.2034                                     & \cellcolor[HTML]{DAE8FC}0.436 ± 0.1056            & 0.469 ± 0.0423                                     & 0.671 ± 0.2260                         \\
% \rowcolor[HTML]{FFFFFF} 
% \multirow{-4}{*}{\cellcolor[HTML]{FFFFFF}ETTh2}   & \multicolumn{1}{l|}{\cellcolor[HTML]{FFFFFF}720} & \cellcolor[HTML]{9698ED}0.323 ± 0.0125                                & 0.447 ± 0.0250                                     & 0.413  ± 0.0464                                   & \cellcolor[HTML]{DAE8FC}0.394 ± 0.0089             & \multicolumn{1}{l|}{\cellcolor[HTML]{FFFFFF}0.564 ± 0.0574} & \cellcolor[HTML]{9698ED}0.356 ± 0.0310            & 0.735 ± 0.1836                                     & \cellcolor[HTML]{DAE8FC}0.443 ± 0.0746            & 0.502 ± 0.0367                                     & 0.723 ± 0.2251                         \\ \hline
% \rowcolor[HTML]{FFFFFF} 
% \cellcolor[HTML]{FFFFFF}                          & \multicolumn{1}{l|}{\cellcolor[HTML]{FFFFFF}96}  & 0.222 ± 0.0058                                                        & 0.170 ± 0.0030                                     & \cellcolor[HTML]{9698ED}0.166 ± 0.0082            & 0.170 ± 0.0005                                     & \multicolumn{1}{l|}{\cellcolor[HTML]{DAE8FC}0.168 ± 0.0025} & 0.241 ± 0.0105                                    & \cellcolor[HTML]{DAE8FC}0.166 ± 0.0050             & \cellcolor[HTML]{9698ED}0.156 ± 0.0155            & 0.175 ± 0.0032                                     & 0.187 ± 0.0038                         \\
% \rowcolor[HTML]{FFFFFF} 
% \cellcolor[HTML]{FFFFFF}                          & \multicolumn{1}{l|}{\cellcolor[HTML]{FFFFFF}192} & 0.274 ± 0.0023                                                        & 0.230 ± 0.0031                                     & \cellcolor[HTML]{9698ED}0.223 ± 0.0115            & 0.228 ± 0.0005                                     & \multicolumn{1}{l|}{\cellcolor[HTML]{DAE8FC}0.225 ± 0.0035} & 0.260 ± 0.0109                                    & \cellcolor[HTML]{DAE8FC}0.224 ± 0.0050             & \cellcolor[HTML]{9698ED}0.210 ± 0.0207            & 0.233 ± 0.0021                                     & 0.244 ± 0.0050                         \\
% \rowcolor[HTML]{FFFFFF} 
% \cellcolor[HTML]{FFFFFF}                          & \multicolumn{1}{l|}{\cellcolor[HTML]{FFFFFF}336} & \cellcolor[HTML]{DAE8FC}0.279 ± 0.0018                                & 0.286 ± 0.0042                                     & \cellcolor[HTML]{9698ED}0.276 ± 0.0111            & 0.286 ± 0.0009                                     & \multicolumn{1}{l|}{\cellcolor[HTML]{FFFFFF}0.282 ± 0.0051} & \cellcolor[HTML]{DAE8FC}0.279 ± 0.0333            & 0.281 ± 0.0056                                     & \cellcolor[HTML]{9698ED}0.262 ± 0.0242            & 0.291 ± 0.0024                                     & 0.300 ± 0.0040                         \\
% \rowcolor[HTML]{FFFFFF} 
% \multirow{-4}{*}{\cellcolor[HTML]{FFFFFF}ETTm2}   & \multicolumn{1}{l|}{\cellcolor[HTML]{FFFFFF}720} & \cellcolor[HTML]{9698ED}0.316 ± 0.0144                                & 0.377 ± 0.0046                                     & \cellcolor[HTML]{DAE8FC}0.369 ± 0.0128            & 0.378 ± 0.0014                                     & \multicolumn{1}{l|}{\cellcolor[HTML]{FFFFFF}0.373 ± 0.0062} & \cellcolor[HTML]{9698ED}0.321 ± 0.0139            & 0.372 ± 0.0061                                     & 0.353 ± 0.0346                                    & 0.382 ± 0.0020                                     & 0.394 ± 0.0064                         \\ \hline
% \rowcolor[HTML]{FFFFFF} 
% \cellcolor[HTML]{FFFFFF}                          & \multicolumn{1}{l|}{\cellcolor[HTML]{FFFFFF}96}  & 0.191 ± 0.0033                                                        & 0.200 ± 0.0475                                     & 0.178 ± 0.0025                                    & \cellcolor[HTML]{DAE8FC}0.189 ± 0.0095             & \multicolumn{1}{l|}{\cellcolor[HTML]{9698ED}0.172 ± 0.0010} & 0.199 ± 0.0021                                    & 0.208 ± 0.0648                                     & \cellcolor[HTML]{DAE8FC}0.176 ± 0.0019            & 0.190 ± 0.0110                                     & \cellcolor[HTML]{9698ED}0.171 ± 0.0018 \\
% \rowcolor[HTML]{FFFFFF} 
% \cellcolor[HTML]{FFFFFF}                          & \multicolumn{1}{l|}{\cellcolor[HTML]{FFFFFF}192} & \cellcolor[HTML]{DAE8FC}0.227 ± 0.0068                                & 0.250 ± 0.0488                                     & 0.230 ± 0.0043                                    & 0.227 ± 0.0094                                     & \multicolumn{1}{l|}{\cellcolor[HTML]{9698ED}0.215 ± 0.0008} & 0.238 ± 0.0059                                    & 0.258 ± 0.0654                                     & \cellcolor[HTML]{DAE8FC}0.225 ± 0.0019            & 0.227 ± 0.0109                                     & \cellcolor[HTML]{9698ED}0.214 ± 0.0015 \\
% \rowcolor[HTML]{FFFFFF} 
% \cellcolor[HTML]{FFFFFF}                          & \multicolumn{1}{l|}{\cellcolor[HTML]{FFFFFF}336} & \cellcolor[HTML]{DAE8FC}0.282 ± 0.0048                                & 0.304 ± 0.0484                                     & 0.282 ± 0.0079                                    & 0.275 ± 0.0090                                     & \multicolumn{1}{l|}{\cellcolor[HTML]{9698ED}0.264 ± 0.0008} & 0.286 ± 0.0080                                    & 0.313 ± 0.0639                                     & 0.278 ± 0.0019                                    & \cellcolor[HTML]{DAE8FC}0.274 ± 0.0102             & \cellcolor[HTML]{9698ED}0.263 ± 0.0018 \\
% \rowcolor[HTML]{FFFFFF} 
% \multirow{-4}{*}{\cellcolor[HTML]{FFFFFF}Weather} & \multicolumn{1}{l|}{\cellcolor[HTML]{FFFFFF}720} & \cellcolor[HTML]{9698ED}0.322 ± 0.0088                                & 0.377 ± 0.0449                                     & 0.355 ± 0.0065                                    & 0.345 ± 0.0089                                     & \multicolumn{1}{l|}{\cellcolor[HTML]{FFFFFF}0.335 ± 0.0009} & \cellcolor[HTML]{DAE8FC}0.339 ± 0.0112            & 0.388 ± 0.0610                                     & 0.352 ± 0.0015                                    & 0.345 ± 0.0099                                     & \cellcolor[HTML]{9698ED}0.334 ± 0.0023 \\ \hline
% \end{tabular}}
% \end{table*}
% \begin{figure}[ht]
%     \centering
%     \includegraphics[width=\linewidth, scale=0.9]{figures/effectofmasking.pdf}
%     \caption{Effect of masking fractions in MCAR on forecasting performance (MSE) under different prediction horizons $T \in \{192, 336, 720\}$ on three datasets, ETTm2, ETTh2, and Weather. }
%     \label{effectofmasking}
%     \vspace{-2ex}
% \end{figure}



% \begin{figure}[ht]
%     \centering
%     \includegraphics[width=\linewidth, scale=0.9]{figures/effect_of_masking_2.pdf}
%     \vspace{0.5cm} % Add some vertical space between the images
%     % \includegraphics[width=0.66\textwidth]{effect_of_masking_2_2.pdf}
%     \vspace{-4ex}
%     \caption{Effect of varying masking types: MCAR and Periodic Masking on the ETTh2 and ETTm2 datasets for long-term forecasting ($T=720$).}
%     \label{fig:effectofmasking2}
%     \vspace{-2ex}
% \end{figure}
% \begin{figure}[htbp!]
%     % \centering
%     \begin{minipage}[b]{0.5\textwidth}
%         \centering
%         \begin{subfigure}[b]{0.6\textwidth}
%             \centering
%             \includegraphics[width=0.95\textwidth]{figures/rebuttal_figures/ts_baselines_etth2_t_192_mcar_saits.pdf}
%             % \caption{ETTh2, T=192, SAITS}
%             \caption{ETTh2,T=192}
%             \label{fig:first_subfigure_4}
%             % \vspace{-20pt}
%         \end{subfigure}
%         \begin{subfigure}[b]{0.6\textwidth}
%             \centering
%             \includegraphics[width=0.97\textwidth]{figures/rebuttal_figures/ts_baselines_weather_t_720_mcar_saits.pdf}
%             % \caption{Weather, T=720, SAITS}
%             \caption{Weather,T=720}
%             \label{fig:first_subfigure_5}
%             % \vspace{-20pt}
%         \end{subfigure}
%         \caption{Performance comparison with multiple TS Baselines imputed with SAITS}
%         \label{fig:fig4}
%     \end{minipage}
%     \hspace{0.001\textwidth}
%     \begin{minipage}[b]{0.49\textwidth}
%         \centering
%         \begin{subfigure}[b]{\textwidth}
%             \centering
%             \includegraphics[width=\textwidth]{figures/rebuttal_figures/classification_ts_baselines.pdf}
%             % \caption{ETTm2, T=720}
%             \caption{Datasets (Left to Right)=EMG,Epilepsy,Gesture}
%             \label{fig:first_subfigure_5}
%         \end{subfigure}
%         \caption{Classification F1 scores across varying masking fractions: $\{0.0, 0.2, 0.4, 0.6, 0.8\}$.}
%         \label{fig:fig5}
%     \end{minipage}
% \end{figure}


\begin{figure}[htbp]

    % First column (left) with two rows
    \begin{minipage}{0.49\textwidth}
        % First row of first column: two subfigures
        \begin{subfigure}[htbp]{0.49\textwidth}
            % \centering
            \includegraphics[width=\textwidth]{figures/rebuttal_figures/imp_baselines_weather_t_720_mcar.pdf}
            \caption{Weather, T=720}
        \end{subfigure}
        \hfill
        \begin{subfigure}[htbp]{0.49\textwidth}
            % \centering
            \includegraphics[width=0.9\textwidth]{figures/rebuttal_figures/imp_baselines_ETTh2_t_720_mcar.pdf}
            \caption{ETTh2, T=720}
            \label{fig:imp_baseline_b}
        \end{subfigure}
        \caption{\textbf{Multiple Imputation Baselines}. Performance comparison across multiple imputation models. Imputation models considered: kNN, Spline, SAITS. TS Baselines: iTransformer}
        \label{fig:imp_baselines_main}
        % \vspace{0.5cm} % Space between rows
    \end{minipage}
    \hfill
    \begin{minipage}{0.49\textwidth}
    % \centering
        % First row of second column: two subfigures
        \begin{subfigure}[htbp]{0.49\textwidth}
        % \label{ts_baselines_main}
            \includegraphics[width=\textwidth]{figures/rebuttal_figures/ts_baselines_ettm2_t_720_mcar.pdf}
            \caption{ETTm2, T=720}
        \end{subfigure}
        % \vspace{0.09cm} % Space between rows
        \hfill
        % Second row of second column: two subfigures
        \begin{subfigure}[htbp]{0.49\textwidth}
            \includegraphics[width=\textwidth]{figures/rebuttal_figures/ts_baselines_ettm2_t_720_periodic.pdf}
            \caption{ETTm2, T=720}
        \end{subfigure}
        \caption{\textbf{Multiple TS Baselines.} Performance comparison with multiple TS Baselines imputed with SAITS, under MCAR and Periodic setting}
        \label{fig:ts_baselines_main}
    \end{minipage}
% TS Baselines considered: Autoformer \cite{autoformer}, PatchTST \cite{patchtst}, iTransformer \cite{liu2023itransformer}, DLinear \cite{dlinear}, SimMTM \cite{simmtm}

    % \caption{Layout with two columns. The left column has two rows, one with subfigures and one with an independent figure. The right column has two rows, both with subfigures.}
    \label{fig:forecasting_results}
\end{figure}

\vspace{-1.5pt}
% In For classification, we go with random masking.
\par \noindent \textbf{Forecasting Results:} For comparing the forecasting performance we consider the Mean Squared Error (MSE) metric. The performance of MissTSM is compared across an array of missing data fractions (60\%, 70\%, 80\%, 90\%), along varying forecasting horizons (T=96,192,336,720) and under different masking strategies. Figure \ref{fig:imp_baselines_main} compares MissTSM with time-series baselines imputed with multiple imputation techniques. The results shown are on Weather and ETTh2 dataset and along the longest forecasting horizon. We can see that MissTSM performs better or similar in comparison to all the baselines considered. It is also interesting to note the significant drop in performance of iTransformer trained on kNN imputed data, which demonstrates the influence of imputation methods on the final performance of time-series baselines. We observe that BRITS results in significantly high MSE, and therefore not considered in result plots. 
 \begin{wrapfigure}{r}{0.49\textwidth} % Adjust 'r' for right, 'l' for left, and width as necessary
    % \centering
    \includegraphics[width=0.49\textwidth]{figures/rebuttal_figures/classification_2_ts_baselines.pdf} % Adjust width as needed
    \caption{Classification F1 scores across varying masking fractions: $\{0.2, 0.4, 0.6, 0.8\}$.} 
    \label{fig:clf}
    \vspace{-1ex}
\end{wrapfigure}
We also compare MissTSM's performance against different SOTA time-series baselines (see Figure \ref{fig:ts_baselines_main})  imputed with SAITS, and across different masking strategies. MissTSM shows significant improvement over the baselines on ETTh2 and ETTm2, under both, MCAR and periodic masking settings (Refer to additional results in Appendix \ref{appendix:C.2}). Overall, we observe that MissTSM is consistently better than the baselines for ETTh2 dataset, while it shows competitive performance on ETTm2 and Weather datasets.
% For the MCAR masking experiments, we observe a trend across all the datasets that the MissTSM framework performs significantly better than the baselines for longer-term forecasting, i.e., for larger forecasting horizons such as 336 and 720. Additionally, we observe on ETTh2 and Weather datasets that the baselines with SAITS-based imputation performs better than the corresponding spline interpolation counterpart. 
% Add setting to appendix
% \par
% ETTh2: MCAR 80\%, Periodic 90\%
% \\
% ETTm2: MCAR 60\%, Periodic 80\%
% \\
% Weather: MCAR 60\%, Periodic 70\%








% \subsection{Analyzing the Impact of Missing Value Fractions}

% To understand the effect of varying masking fractions on the forecasting performance of MissTSM and baseline methods, Figure \ref{effectofmasking} shows variations in the MSE of comparative methods as we increase the missing value fraction in MCAR masking scheme from $0.6$ to $0.9$ for different forecasting horizons and datasets.  We can see that MissTSM mostly performs at par or better than the time-series baselines, especially for higher missing fractions and longer forecasitng horizons. For example, we can see that while MissTSM does not achieve similar MSE as baselines on ETTm2 for $T = 336$, it performs significantly better than baselines when we increase the forecasting horizon to $T = 720$ on the same dataset. Similarly, for the Weather and ETTh2 datasets, we can see that MissTSM shows similar performance as state-of-the-art baselines across all settings of missing value fractions. 
% It is interesting to see that, for ETTh2, MissTSM achieves similar MSE as the stronger baseline methods (with more complex architectures both for imputation and time-series modeling) even with 90\% missing values. 


% Figure \ref{fig:effectofmasking2} shows the effect of varying the masking scheme along with the missing value fractions on ETTh2 and ETTm2 datasets. We can see that with periodic masking, there are sharp changes in MSE as we vary the missing fractions. While this can be attributed to the presence of an implicit pattern in the masked data for generating periodic missing values, it is interesting to note that MissTSM is relatively more stable, with similar or better performance than the baselines.  
% Moreover, it is quite interesting to see the amount of fluctuations in the baselines, which reflect the strong dependency of the modeling errors with the imputation errors. 
% \begin{figure}[ht]
%     \centering
%     \includegraphics[width=\linewidth, scale=0.9]{effect_of_masking_2.pdf}
%     \caption{Forecasting Performance under different masking types - MCAR, Periodic, Patch. For ETTm2 Patch, SimMTM baselines were not trained, hence, not shown in the figure}
%     \label{effectofmasking2}
% \end{figure}
% \subsection{Classification}


\textbf{Classification Results:} For fine-tuning on the classification tasks, we add a multi-layer perceptron to the encoder as the classification layer. The same is done for the SimMTM baseline. The classification performance is reported with F1-score metric and under randomly masked (MCAR) fractions of data (20\%, 40\%, 60\%, 80\%). Figure \ref{fig:clf} compares MissTSM with two variants of SimMTM - one, trained on Spline imputed data, and the other, trained on SAITS imputed data. As seen in the figure, MissTSM achieves roughly similar or better performance as SimMTM on EMG (performance improvement is visible over increasing masked fractions), and outperforms SimMTM on the Gesture Dataset (Refer to \ref{appendix:C.3} for results on all three datasets). These results demonstrate the effectiveness of our proposed MissTSM framework to circumvent the need for explicit imputation of missing values while achieving similar predictive performance as state-of-the-art.






% Results to add in the paper:
% \begin{enumerate}
%     \item \textcolor{red}{Table 1: Forecasting Results} \begin{itemize}
%         \item \textcolor{red}{Datasets: ETTh1, ETTh2, ETTm1, ETTm2, Weather}
%         \item \textcolor{red}{Baselines: SimMTM+Spline, SimMTM+SAITS, PatchTST+Spline, PatchTST+SAITS, ours}
%         \item \textcolor{red}{Choose mask fraction : XX (This can be same for ETT datasets - say 0.9 or 0.8, but for weather we can choose a different one.)}
%         \item \textcolor{red}{Masking Types: MCAR, Periodic}
%         \item Summary of Table: 5 rows for each dataset, and each row will have 5 baselines $\times$ 2 masking types = 10 columns
%         \item Should we report both MAE and MSE like SimMTM/PatchTST/iTransformer? If yes, we can change the structure of the Table. Suggestion: (10 Rows: 5 dataset $\times$ 2 blocks for each masking type, and 10 columns: 5 baselines $\times$ 2 metrics)
%         \item Also, we thought about showing the percentage decrease from the unmasked error (mask fraction=0\%). That will increase the number of values. 
%     \end{itemize}

%     \item \textcolor{red}{Figure 1: Effect of masking and type of masking} \begin{itemize}
%         \item \textcolor{red}{Datasets: ETTh1, ETTh2}
%         \item \textcolor{red}{Type of masking: MCAR, Periodic, Patch (if we have) - SimMTM + Spline/SAITS for Patch missing}
%         \item \textcolor{red}{Baselines: Can we show the same 5 baselines??}
%         \item \textcolor{red}{Each plot: Mask Fractions (along x-axis) vs MSE (along y-axis) for every baseline}
%         \item Summary: Ideally 3 columns (type of masking) $\times$ 2 rows (datasets) plot Figure.
%         \item If not, we can show 4 datasets (all EET datasets) for 2 type of masking.
%     \end{itemize}

%     \item \textcolor{red}{Figure 2: Ablation Study of our method for forecasting} \begin{itemize}
%         \item \textcolor{red}{Datasets: ETTh1, ETTh2}
%         \item \textcolor{red}{Baselines: Ours, MAE + Spline, MAE + SAITS}
%         \item \textcolor{red}{Type of masking: MCAR, Periodic??}
%         \item \textcolor{red}{Each plot: Mask Fractions (along x-axis) vs MSE (along y-axis) for every baseline}
%         \item Summary: 2 x 2 plots.
%     \end{itemize}

    % \item \textcolor{red}{Figure 3: Classification} \begin{itemize}
    %     \item \textcolor{red}{Datasets: Epilepsy, Gesture, EMG}
    %     \item \textcolor{red}{Baselines: SimMTM+Spline, SimMTM+SAITS, MAE+Spline, MAE+SAITS, ours, [MAYBE IF WE CAN ADD PATCH-TST]}
    %     \item \textcolor{red}{Type of masking: MCAR only?}
    %     \item \textcolor{red}{Each plot: Mask Fractions (along x-axis) vs F1 score (along y-axis) for every baseline} 
    %     \item Summary: 3 plots each with 5 baselines
    %     \item MAE + Spline and MAE + SAITS can be an ablation study separate Figure.
    % \end{itemize}
    
    % \item Time complexity Figure to show speed improvements due to single stage approach.
    
%     \item Do we have results with different forecasting and lookback window variations? (TODO: for rebuttal.)
% \end{enumerate}


% \section{Conclusion}
In this work, we propose a simple yet effective approach, called SMILE, for graph few-shot learning with fewer tasks. Specifically, we introduce a novel dual-level mixup strategy, including within-task and across-task mixup, for enriching the diversity of nodes within each task and the diversity of tasks. Also, we incorporate the degree-based prior information to learn expressive node embeddings. Theoretically, we prove that SMILE effectively enhances the model's generalization performance. Empirically, we conduct extensive experiments on multiple benchmarks and the results suggest that SMILE significantly outperforms other baselines, including both in-domain and cross-domain few-shot settings.
\section{Conclusions and Future Work}

% To the best of our knowledge, our proposed MissTSM framework is the first end-to-end framework for TS modeling with missing values that does not require any explicit imputations. 
% We introduce a novel Time-Feature Independent Embedding scheme in MissTSM to represent every time-feature combination as an independent token. We also propose a novel Missing Feature-Aware Attention (MHAA) Layer that utilizes partially observed variates (or channels) at a given time-step $t$ to learn latent representations.
% , echoing the essence of traditional imputation methods that utilize cross-channel correlations to infer missing values. 
We empirically demonstrate the effectiveness of the MissTSM framework across multiple benchmark datasets and synthetic masking strategies. 
However, a limitation of MFAA layer is that it does not explicitly learn the non-linear temporal dynamics, and relies on subsequent transformer encoder blocks to learn the dynamics. Future work can explore modifications of MFAA layer to address this limitation.

\subsubsection*{Acknowledgments}
This work was supported in part by NSF awards IIS-2239328 and DEB-2213550.
This manuscript has been authored by UT-Battelle, LLC, under contract DE-AC05-00OR22725 with the US Department of Energy (DOE). The US government retains and the publisher, by accepting the article for publication, acknowledges that the US government retains a nonexclusive, paid-up, irrevocable, worldwide license to publish or reproduce the published form of this manuscript, or allow others to do so, for US government purposes. DOE will provide public access to these results of federally sponsored research in accordance with the DOE Public Access Plan (https://www.energy.gov/doe-public-access-plan).
\bibliography{main}
% \bibliographystyle{plainnat}
\bibliographystyle{unsrt}



% \subsection{Retrieval of style files}


% The style files for this workshop and other information are available on
% the at
% \begin{center}
%   \url{https://neurips-time-series-workshop.github.io/}
% \end{center}

% The only supported style file for this workshop is \verb+timeseries_workshop.sty+,
% rewritten for \LaTeXe{}.


% The \LaTeX{} style file contains three optional arguments: \verb+final+, which
% creates a camera-ready copy, \verb+preprint+, which creates a preprint for
% submission to, e.g., arXiv, and \verb+nonatbib+, which will not load the
% \verb+natbib+ package for you in case of package clash.


% \paragraph{Preprint option.}
% If you wish to post a preprint of your work online, e.g., on arXiv, using the
% NeurIPS style, please use the \verb+preprint+ option. This will create a
% nonanonymized version of your work with the text ``Preprint. Work in progress.''
% in the footer. This version may be distributed as you see fit. Please \textbf{do
%   not} use the \verb+final+ option, which should \textbf{only} be used for
% papers accepted to the workshop.


% At submission time, please omit the \verb+final+ and \verb+preprint+
% options. This will anonymize your submission and add line numbers to aid
% review. Please do \emph{not} refer to these line numbers in your paper as they
% will be removed during generation of camera-ready copies.


% The file \verb+main.tex+ may be used as a ``shell'' for writing your
% paper. All you have to do is replace the author, title, abstract, and text of
% the paper with your own.


% The formatting instructions contained in these style files are summarized in
% Sections \ref{gen_inst}, \ref{headings}, and \ref{others} below.


% \section{General formatting instructions}
% \label{gen_inst}


% The text must be confined within a rectangle 5.5~inches (33~picas) wide and
% 9~inches (54~picas) long. The left margin is 1.5~inch (9~picas).  Use 10~point
% type with a vertical spacing (leading) of 11~points.  Times New Roman is the
% preferred typeface throughout, and will be selected for you by default.
% Paragraphs are separated by \nicefrac{1}{2}~line space (5.5 points), with no
% indentation.


% The paper title should be 17~point, initial caps/lower case, bold, centered
% between two horizontal rules. The top rule should be 4~points thick and the
% bottom rule should be 1~point thick. Allow \nicefrac{1}{4}~inch space above and
% below the title to rules. All pages should start at 1~inch (6~picas) from the
% top of the page.


% For the final version, authors' names are set in boldface, and each name is
% centered above the corresponding address. The lead author's name is to be listed
% first (left-most), and the co-authors' names (if different address) are set to
% follow. If there is only one co-author, list both author and co-author side by
% side.


% Please pay special attention to the instructions in Section \ref{others}
% regarding figures, tables, acknowledgments, and references.


% \section{Headings: first level}
% \label{headings}


% All headings should be lower case (except for first word and proper nouns),
% flush left, and bold.


% First-level headings should be in 12-point type.


% \subsection{Headings: second level}


% Second-level headings should be in 10-point type.


% \subsubsection{Headings: third level}


% Third-level headings should be in 10-point type.


% \paragraph{Paragraphs}


% There is also a \verb+\paragraph+ command available, which sets the heading in
% bold, flush left, and inline with the text, with the heading followed by 1\,em
% of space.


% \section{Citations, figures, tables, references}
% \label{others}


% These instructions apply to everyone.


% \subsection{Citations within the text}


% The \verb+natbib+ package will be loaded for you by default.  Citations may be
% author/year or numeric, as long as you maintain internal consistency.  As to the
% format of the references themselves, any style is acceptable as long as it is
% used consistently.


% The documentation for \verb+natbib+ may be found at
% \begin{center}
%   \url{http://mirrors.ctan.org/macros/latex/contrib/natbib/natnotes.pdf}
% \end{center}
% Of note is the command \verb+\citet+, which produces citations appropriate for
% use in inline text.  For example,
% \begin{verbatim}
%    \citet{hasselmo} investigated\dots
% \end{verbatim}
% produces
% \begin{quote}
%   Hasselmo, et al.\ (1995) investigated\dots
% \end{quote}


% If you wish to load the \verb+natbib+ package with options, you may add the
% following before loading the \verb+timeseries_workshop+ package:
% \begin{verbatim}
%    \PassOptionsToPackage{options}{natbib}
% \end{verbatim}


% If \verb+natbib+ clashes with another package you load, you can add the optional
% argument \verb+nonatbib+ when loading the style file:
% \begin{verbatim}
%    \usepackage[nonatbib]{timeseries_workshop}
% \end{verbatim}


% As submission is double blind, refer to your own published work in the third
% person. That is, use ``In the previous work of Jones et al.\ [4],'' not ``In our
% previous work [4].'' If you cite your other papers that are not widely available
% (e.g., a journal paper under review), use anonymous author names in the
% citation, e.g., an author of the form ``A.\ Anonymous.''


% \subsection{Footnotes}


% Footnotes should be used sparingly.  If you do require a footnote, indicate
% footnotes with a number\footnote{Sample of the first footnote.} in the
% text. Place the footnotes at the bottom of the page on which they appear.
% Precede the footnote with a horizontal rule of 2~inches (12~picas).


% Note that footnotes are properly typeset \emph{after} punctuation
% marks.\footnote{As in this example.}


% \subsection{Figures}


% \begin{figure}
%   \centering
%   \fbox{\rule[-.5cm]{0cm}{4cm} \rule[-.5cm]{4cm}{0cm}}
%   \caption{Sample figure caption.}
% \end{figure}


% All artwork must be neat, clean, and legible. Lines should be dark enough for
% purposes of reproduction. The figure number and caption always appear after the
% figure. Place one line space before the figure caption and one line space after
% the figure. The figure caption should be lower case (except for first word and
% proper nouns); figures are numbered consecutively.


% You may use color figures.  However, it is best for the figure captions and the
% paper body to be legible if the paper is printed in either black/white or in
% color.


% \subsection{Tables}


% All tables must be centered, neat, clean and legible.  The table number and
% title always appear before the table.  See Table~\ref{sample-table}.


% Place one line space before the table title, one line space after the
% table title, and one line space after the table. The table title must
% be lower case (except for first word and proper nouns); tables are
% numbered consecutively.


% Note that publication-quality tables \emph{do not contain vertical rules.} We
% strongly suggest the use of the \verb+booktabs+ package, which allows for
% typesetting high-quality, professional tables:
% \begin{center}
%   \url{https://www.ctan.org/pkg/booktabs}
% \end{center}
% This package was used to typeset Table~\ref{sample-table}.


% \begin{table}
%   \caption{Sample table title}
%   \label{sample-table}
%   \centering
%   \begin{tabular}{lll}
%     \toprule
%     \multicolumn{2}{c}{Part}                   \\
%     \cmidrule(r){1-2}
%     Name     & Description     & Size ($\mu$m) \\
%     \midrule
%     Dendrite & Input terminal  & $\sim$100     \\
%     Axon     & Output terminal & $\sim$10      \\
%     Soma     & Cell body       & up to $10^6$  \\
%     \bottomrule
%   \end{tabular}
% \end{table}


% \section{Final instructions}


% Do not change any aspects of the formatting parameters in the style files.  In
% particular, do not modify the width or length of the rectangle the text should
% fit into, and do not change font sizes (except perhaps in the
% \textbf{References} section; see below). Please note that pages should be
% numbered.


% \section{Preparing PDF files}


% Please prepare submission files with paper size ``US Letter,'' and not, for
% example, ``A4.''


% Fonts were the main cause of problems in the past years. Your PDF file must only
% contain Type 1 or Embedded TrueType fonts. Here are a few instructions to
% achieve this.


% \begin{itemize}


% \item You should directly generate PDF files using \verb+pdflatex+.


% \item You can check which fonts a PDF files uses.  In Acrobat Reader, select the
%   menu Files$>$Document Properties$>$Fonts and select Show All Fonts. You can
%   also use the program \verb+pdffonts+ which comes with \verb+xpdf+ and is
%   available out-of-the-box on most Linux machines.


% \item The IEEE has recommendations for generating PDF files whose fonts are also
%   acceptable for this workshop. Please see
%   \url{http://www.emfield.org/icuwb2010/downloads/IEEE-PDF-SpecV32.pdf}


% \item \verb+xfig+ "patterned" shapes are implemented with bitmap fonts.  Use
%   "solid" shapes instead.


% \item The \verb+\bbold+ package almost always uses bitmap fonts.  You should use
%   the equivalent AMS Fonts:
% \begin{verbatim}
%    \usepackage{amsfonts}
% \end{verbatim}
% followed by, e.g., \verb+\mathbb{R}+, \verb+\mathbb{N}+, or \verb+\mathbb{C}+
% for $\mathbb{R}$, $\mathbb{N}$ or $\mathbb{C}$.  You can also use the following
% workaround for reals, natural and complex:
% \begin{verbatim}
%    \newcommand{\RR}{I\!\!R} %real numbers
%    \newcommand{\Nat}{I\!\!N} %natural numbers
%    \newcommand{\CC}{I\!\!\!\!C} %complex numbers
% \end{verbatim}
% Note that \verb+amsfonts+ is automatically loaded by the \verb+amssymb+ package.


% \end{itemize}


% If your file contains type 3 fonts or non embedded TrueType fonts, we will ask
% you to fix it.


% \subsection{Margins in \LaTeX{}}


% Most of the margin problems come from figures positioned by hand using
% \verb+\special+ or other commands. We suggest using the command
% \verb+\includegraphics+ from the \verb+graphicx+ package. Always specify the
% figure width as a multiple of the line width as in the example below:
% \begin{verbatim}
%    \usepackage[pdftex]{graphicx} ...
%    \includegraphics[width=0.8\linewidth]{myfile.pdf}
% \end{verbatim}
% See Section 4.4 in the graphics bundle documentation
% (\url{http://mirrors.ctan.org/macros/latex/required/graphics/grfguide.pdf})


% A number of width problems arise when \LaTeX{} cannot properly hyphenate a
% line. Please give LaTeX hyphenation hints using the \verb+\-+ command when
% necessary.


% \begin{ack}
% Use unnumbered first level headings for the acknowledgments. All acknowledgments
% go at the end of the paper before the list of references. Moreover, you are required to declare
% funding (financial activities supporting the submitted work) and competing interests (related financial activities outside the submitted work).
% More information about this disclosure can be found at: \url{https://neurips.cc/Conferences/2022/PaperInformation/FundingDisclosure}.


% Do {\bf not} include this section in the anonymized submission, only in the final paper. You can use the \texttt{ack} environment provided in the style file to autmoatically hide this section in the anonymized submission.
% \end{ack}


% \section*{References}


% References follow the acknowledgments. Use unnumbered first-level heading for
% the references. Any choice of citation style is acceptable as long as you are
% consistent. It is permissible to reduce the font size to \verb+small+ (9 point)
% when listing the references.
% Note that the Reference section does not count towards the page limit.
% \medskip


% {
% \small


% [1] Alexander, J.A.\ \& Mozer, M.C.\ (1995) Template-based algorithms for
% connectionist rule extraction. In G.\ Tesauro, D.S.\ Touretzky and T.K.\ Leen
% (eds.), {\it Advances in Neural Information Processing Systems 7},
% pp.\ 609--616. Cambridge, MA: MIT Press.


% [2] Bower, J.M.\ \& Beeman, D.\ (1995) {\it The Book of GENESIS: Exploring
%   Realistic Neural Models with the GEneral NEural SImulation System.}  New York:
% TELOS/Springer--Verlag.


% [3] Hasselmo, M.E., Schnell, E.\ \& Barkai, E.\ (1995) Dynamics of learning and
% recall at excitatory recurrent synapses and cholinergic modulation in rat
% hippocampal region CA3. {\it Journal of Neuroscience} {\bf 15}(7):5249-5262.
% }


%%%%%%%%%%%%%%%%%%%%%%%%%%%%%%%%%%%%%%%%%%%%%%%%%%%%%%%%%%%%



%%%%%%%%%%%%%%%%%%%%%%%%%%%%%%%%%%%%%%%%%%%%%%%%%%%%%%%%%%%%

% \newpage
% \subsection{Lloyd-Max Algorithm}
\label{subsec:Lloyd-Max}
For a given quantization bitwidth $B$ and an operand $\bm{X}$, the Lloyd-Max algorithm finds $2^B$ quantization levels $\{\hat{x}_i\}_{i=1}^{2^B}$ such that quantizing $\bm{X}$ by rounding each scalar in $\bm{X}$ to the nearest quantization level minimizes the quantization MSE. 

The algorithm starts with an initial guess of quantization levels and then iteratively computes quantization thresholds $\{\tau_i\}_{i=1}^{2^B-1}$ and updates quantization levels $\{\hat{x}_i\}_{i=1}^{2^B}$. Specifically, at iteration $n$, thresholds are set to the midpoints of the previous iteration's levels:
\begin{align*}
    \tau_i^{(n)}=\frac{\hat{x}_i^{(n-1)}+\hat{x}_{i+1}^{(n-1)}}2 \text{ for } i=1\ldots 2^B-1
\end{align*}
Subsequently, the quantization levels are re-computed as conditional means of the data regions defined by the new thresholds:
\begin{align*}
    \hat{x}_i^{(n)}=\mathbb{E}\left[ \bm{X} \big| \bm{X}\in [\tau_{i-1}^{(n)},\tau_i^{(n)}] \right] \text{ for } i=1\ldots 2^B
\end{align*}
where to satisfy boundary conditions we have $\tau_0=-\infty$ and $\tau_{2^B}=\infty$. The algorithm iterates the above steps until convergence.

Figure \ref{fig:lm_quant} compares the quantization levels of a $7$-bit floating point (E3M3) quantizer (left) to a $7$-bit Lloyd-Max quantizer (right) when quantizing a layer of weights from the GPT3-126M model at a per-tensor granularity. As shown, the Lloyd-Max quantizer achieves substantially lower quantization MSE. Further, Table \ref{tab:FP7_vs_LM7} shows the superior perplexity achieved by Lloyd-Max quantizers for bitwidths of $7$, $6$ and $5$. The difference between the quantizers is clear at 5 bits, where per-tensor FP quantization incurs a drastic and unacceptable increase in perplexity, while Lloyd-Max quantization incurs a much smaller increase. Nevertheless, we note that even the optimal Lloyd-Max quantizer incurs a notable ($\sim 1.5$) increase in perplexity due to the coarse granularity of quantization. 

\begin{figure}[h]
  \centering
  \includegraphics[width=0.7\linewidth]{sections/figures/LM7_FP7.pdf}
  \caption{\small Quantization levels and the corresponding quantization MSE of Floating Point (left) vs Lloyd-Max (right) Quantizers for a layer of weights in the GPT3-126M model.}
  \label{fig:lm_quant}
\end{figure}

\begin{table}[h]\scriptsize
\begin{center}
\caption{\label{tab:FP7_vs_LM7} \small Comparing perplexity (lower is better) achieved by floating point quantizers and Lloyd-Max quantizers on a GPT3-126M model for the Wikitext-103 dataset.}
\begin{tabular}{c|cc|c}
\hline
 \multirow{2}{*}{\textbf{Bitwidth}} & \multicolumn{2}{|c|}{\textbf{Floating-Point Quantizer}} & \textbf{Lloyd-Max Quantizer} \\
 & Best Format & Wikitext-103 Perplexity & Wikitext-103 Perplexity \\
\hline
7 & E3M3 & 18.32 & 18.27 \\
6 & E3M2 & 19.07 & 18.51 \\
5 & E4M0 & 43.89 & 19.71 \\
\hline
\end{tabular}
\end{center}
\end{table}

\subsection{Proof of Local Optimality of LO-BCQ}
\label{subsec:lobcq_opt_proof}
For a given block $\bm{b}_j$, the quantization MSE during LO-BCQ can be empirically evaluated as $\frac{1}{L_b}\lVert \bm{b}_j- \bm{\hat{b}}_j\rVert^2_2$ where $\bm{\hat{b}}_j$ is computed from equation (\ref{eq:clustered_quantization_definition}) as $C_{f(\bm{b}_j)}(\bm{b}_j)$. Further, for a given block cluster $\mathcal{B}_i$, we compute the quantization MSE as $\frac{1}{|\mathcal{B}_{i}|}\sum_{\bm{b} \in \mathcal{B}_{i}} \frac{1}{L_b}\lVert \bm{b}- C_i^{(n)}(\bm{b})\rVert^2_2$. Therefore, at the end of iteration $n$, we evaluate the overall quantization MSE $J^{(n)}$ for a given operand $\bm{X}$ composed of $N_c$ block clusters as:
\begin{align*}
    \label{eq:mse_iter_n}
    J^{(n)} = \frac{1}{N_c} \sum_{i=1}^{N_c} \frac{1}{|\mathcal{B}_{i}^{(n)}|}\sum_{\bm{v} \in \mathcal{B}_{i}^{(n)}} \frac{1}{L_b}\lVert \bm{b}- B_i^{(n)}(\bm{b})\rVert^2_2
\end{align*}

At the end of iteration $n$, the codebooks are updated from $\mathcal{C}^{(n-1)}$ to $\mathcal{C}^{(n)}$. However, the mapping of a given vector $\bm{b}_j$ to quantizers $\mathcal{C}^{(n)}$ remains as  $f^{(n)}(\bm{b}_j)$. At the next iteration, during the vector clustering step, $f^{(n+1)}(\bm{b}_j)$ finds new mapping of $\bm{b}_j$ to updated codebooks $\mathcal{C}^{(n)}$ such that the quantization MSE over the candidate codebooks is minimized. Therefore, we obtain the following result for $\bm{b}_j$:
\begin{align*}
\frac{1}{L_b}\lVert \bm{b}_j - C_{f^{(n+1)}(\bm{b}_j)}^{(n)}(\bm{b}_j)\rVert^2_2 \le \frac{1}{L_b}\lVert \bm{b}_j - C_{f^{(n)}(\bm{b}_j)}^{(n)}(\bm{b}_j)\rVert^2_2
\end{align*}

That is, quantizing $\bm{b}_j$ at the end of the block clustering step of iteration $n+1$ results in lower quantization MSE compared to quantizing at the end of iteration $n$. Since this is true for all $\bm{b} \in \bm{X}$, we assert the following:
\begin{equation}
\begin{split}
\label{eq:mse_ineq_1}
    \tilde{J}^{(n+1)} &= \frac{1}{N_c} \sum_{i=1}^{N_c} \frac{1}{|\mathcal{B}_{i}^{(n+1)}|}\sum_{\bm{b} \in \mathcal{B}_{i}^{(n+1)}} \frac{1}{L_b}\lVert \bm{b} - C_i^{(n)}(b)\rVert^2_2 \le J^{(n)}
\end{split}
\end{equation}
where $\tilde{J}^{(n+1)}$ is the the quantization MSE after the vector clustering step at iteration $n+1$.

Next, during the codebook update step (\ref{eq:quantizers_update}) at iteration $n+1$, the per-cluster codebooks $\mathcal{C}^{(n)}$ are updated to $\mathcal{C}^{(n+1)}$ by invoking the Lloyd-Max algorithm \citep{Lloyd}. We know that for any given value distribution, the Lloyd-Max algorithm minimizes the quantization MSE. Therefore, for a given vector cluster $\mathcal{B}_i$ we obtain the following result:

\begin{equation}
    \frac{1}{|\mathcal{B}_{i}^{(n+1)}|}\sum_{\bm{b} \in \mathcal{B}_{i}^{(n+1)}} \frac{1}{L_b}\lVert \bm{b}- C_i^{(n+1)}(\bm{b})\rVert^2_2 \le \frac{1}{|\mathcal{B}_{i}^{(n+1)}|}\sum_{\bm{b} \in \mathcal{B}_{i}^{(n+1)}} \frac{1}{L_b}\lVert \bm{b}- C_i^{(n)}(\bm{b})\rVert^2_2
\end{equation}

The above equation states that quantizing the given block cluster $\mathcal{B}_i$ after updating the associated codebook from $C_i^{(n)}$ to $C_i^{(n+1)}$ results in lower quantization MSE. Since this is true for all the block clusters, we derive the following result: 
\begin{equation}
\begin{split}
\label{eq:mse_ineq_2}
     J^{(n+1)} &= \frac{1}{N_c} \sum_{i=1}^{N_c} \frac{1}{|\mathcal{B}_{i}^{(n+1)}|}\sum_{\bm{b} \in \mathcal{B}_{i}^{(n+1)}} \frac{1}{L_b}\lVert \bm{b}- C_i^{(n+1)}(\bm{b})\rVert^2_2  \le \tilde{J}^{(n+1)}   
\end{split}
\end{equation}

Following (\ref{eq:mse_ineq_1}) and (\ref{eq:mse_ineq_2}), we find that the quantization MSE is non-increasing for each iteration, that is, $J^{(1)} \ge J^{(2)} \ge J^{(3)} \ge \ldots \ge J^{(M)}$ where $M$ is the maximum number of iterations. 
%Therefore, we can say that if the algorithm converges, then it must be that it has converged to a local minimum. 
\hfill $\blacksquare$


\begin{figure}
    \begin{center}
    \includegraphics[width=0.5\textwidth]{sections//figures/mse_vs_iter.pdf}
    \end{center}
    \caption{\small NMSE vs iterations during LO-BCQ compared to other block quantization proposals}
    \label{fig:nmse_vs_iter}
\end{figure}

Figure \ref{fig:nmse_vs_iter} shows the empirical convergence of LO-BCQ across several block lengths and number of codebooks. Also, the MSE achieved by LO-BCQ is compared to baselines such as MXFP and VSQ. As shown, LO-BCQ converges to a lower MSE than the baselines. Further, we achieve better convergence for larger number of codebooks ($N_c$) and for a smaller block length ($L_b$), both of which increase the bitwidth of BCQ (see Eq \ref{eq:bitwidth_bcq}).


\subsection{Additional Accuracy Results}
%Table \ref{tab:lobcq_config} lists the various LOBCQ configurations and their corresponding bitwidths.
\begin{table}
\setlength{\tabcolsep}{4.75pt}
\begin{center}
\caption{\label{tab:lobcq_config} Various LO-BCQ configurations and their bitwidths.}
\begin{tabular}{|c||c|c|c|c||c|c||c|} 
\hline
 & \multicolumn{4}{|c||}{$L_b=8$} & \multicolumn{2}{|c||}{$L_b=4$} & $L_b=2$ \\
 \hline
 \backslashbox{$L_A$\kern-1em}{\kern-1em$N_c$} & 2 & 4 & 8 & 16 & 2 & 4 & 2 \\
 \hline
 64 & 4.25 & 4.375 & 4.5 & 4.625 & 4.375 & 4.625 & 4.625\\
 \hline
 32 & 4.375 & 4.5 & 4.625& 4.75 & 4.5 & 4.75 & 4.75 \\
 \hline
 16 & 4.625 & 4.75& 4.875 & 5 & 4.75 & 5 & 5 \\
 \hline
\end{tabular}
\end{center}
\end{table}

%\subsection{Perplexity achieved by various LO-BCQ configurations on Wikitext-103 dataset}

\begin{table} \centering
\begin{tabular}{|c||c|c|c|c||c|c||c|} 
\hline
 $L_b \rightarrow$& \multicolumn{4}{c||}{8} & \multicolumn{2}{c||}{4} & 2\\
 \hline
 \backslashbox{$L_A$\kern-1em}{\kern-1em$N_c$} & 2 & 4 & 8 & 16 & 2 & 4 & 2  \\
 %$N_c \rightarrow$ & 2 & 4 & 8 & 16 & 2 & 4 & 2 \\
 \hline
 \hline
 \multicolumn{8}{c}{GPT3-1.3B (FP32 PPL = 9.98)} \\ 
 \hline
 \hline
 64 & 10.40 & 10.23 & 10.17 & 10.15 &  10.28 & 10.18 & 10.19 \\
 \hline
 32 & 10.25 & 10.20 & 10.15 & 10.12 &  10.23 & 10.17 & 10.17 \\
 \hline
 16 & 10.22 & 10.16 & 10.10 & 10.09 &  10.21 & 10.14 & 10.16 \\
 \hline
  \hline
 \multicolumn{8}{c}{GPT3-8B (FP32 PPL = 7.38)} \\ 
 \hline
 \hline
 64 & 7.61 & 7.52 & 7.48 &  7.47 &  7.55 &  7.49 & 7.50 \\
 \hline
 32 & 7.52 & 7.50 & 7.46 &  7.45 &  7.52 &  7.48 & 7.48  \\
 \hline
 16 & 7.51 & 7.48 & 7.44 &  7.44 &  7.51 &  7.49 & 7.47  \\
 \hline
\end{tabular}
\caption{\label{tab:ppl_gpt3_abalation} Wikitext-103 perplexity across GPT3-1.3B and 8B models.}
\end{table}

\begin{table} \centering
\begin{tabular}{|c||c|c|c|c||} 
\hline
 $L_b \rightarrow$& \multicolumn{4}{c||}{8}\\
 \hline
 \backslashbox{$L_A$\kern-1em}{\kern-1em$N_c$} & 2 & 4 & 8 & 16 \\
 %$N_c \rightarrow$ & 2 & 4 & 8 & 16 & 2 & 4 & 2 \\
 \hline
 \hline
 \multicolumn{5}{|c|}{Llama2-7B (FP32 PPL = 5.06)} \\ 
 \hline
 \hline
 64 & 5.31 & 5.26 & 5.19 & 5.18  \\
 \hline
 32 & 5.23 & 5.25 & 5.18 & 5.15  \\
 \hline
 16 & 5.23 & 5.19 & 5.16 & 5.14  \\
 \hline
 \multicolumn{5}{|c|}{Nemotron4-15B (FP32 PPL = 5.87)} \\ 
 \hline
 \hline
 64  & 6.3 & 6.20 & 6.13 & 6.08  \\
 \hline
 32  & 6.24 & 6.12 & 6.07 & 6.03  \\
 \hline
 16  & 6.12 & 6.14 & 6.04 & 6.02  \\
 \hline
 \multicolumn{5}{|c|}{Nemotron4-340B (FP32 PPL = 3.48)} \\ 
 \hline
 \hline
 64 & 3.67 & 3.62 & 3.60 & 3.59 \\
 \hline
 32 & 3.63 & 3.61 & 3.59 & 3.56 \\
 \hline
 16 & 3.61 & 3.58 & 3.57 & 3.55 \\
 \hline
\end{tabular}
\caption{\label{tab:ppl_llama7B_nemo15B} Wikitext-103 perplexity compared to FP32 baseline in Llama2-7B and Nemotron4-15B, 340B models}
\end{table}

%\subsection{Perplexity achieved by various LO-BCQ configurations on MMLU dataset}


\begin{table} \centering
\begin{tabular}{|c||c|c|c|c||c|c|c|c|} 
\hline
 $L_b \rightarrow$& \multicolumn{4}{c||}{8} & \multicolumn{4}{c||}{8}\\
 \hline
 \backslashbox{$L_A$\kern-1em}{\kern-1em$N_c$} & 2 & 4 & 8 & 16 & 2 & 4 & 8 & 16  \\
 %$N_c \rightarrow$ & 2 & 4 & 8 & 16 & 2 & 4 & 2 \\
 \hline
 \hline
 \multicolumn{5}{|c|}{Llama2-7B (FP32 Accuracy = 45.8\%)} & \multicolumn{4}{|c|}{Llama2-70B (FP32 Accuracy = 69.12\%)} \\ 
 \hline
 \hline
 64 & 43.9 & 43.4 & 43.9 & 44.9 & 68.07 & 68.27 & 68.17 & 68.75 \\
 \hline
 32 & 44.5 & 43.8 & 44.9 & 44.5 & 68.37 & 68.51 & 68.35 & 68.27  \\
 \hline
 16 & 43.9 & 42.7 & 44.9 & 45 & 68.12 & 68.77 & 68.31 & 68.59  \\
 \hline
 \hline
 \multicolumn{5}{|c|}{GPT3-22B (FP32 Accuracy = 38.75\%)} & \multicolumn{4}{|c|}{Nemotron4-15B (FP32 Accuracy = 64.3\%)} \\ 
 \hline
 \hline
 64 & 36.71 & 38.85 & 38.13 & 38.92 & 63.17 & 62.36 & 63.72 & 64.09 \\
 \hline
 32 & 37.95 & 38.69 & 39.45 & 38.34 & 64.05 & 62.30 & 63.8 & 64.33  \\
 \hline
 16 & 38.88 & 38.80 & 38.31 & 38.92 & 63.22 & 63.51 & 63.93 & 64.43  \\
 \hline
\end{tabular}
\caption{\label{tab:mmlu_abalation} Accuracy on MMLU dataset across GPT3-22B, Llama2-7B, 70B and Nemotron4-15B models.}
\end{table}


%\subsection{Perplexity achieved by various LO-BCQ configurations on LM evaluation harness}

\begin{table} \centering
\begin{tabular}{|c||c|c|c|c||c|c|c|c|} 
\hline
 $L_b \rightarrow$& \multicolumn{4}{c||}{8} & \multicolumn{4}{c||}{8}\\
 \hline
 \backslashbox{$L_A$\kern-1em}{\kern-1em$N_c$} & 2 & 4 & 8 & 16 & 2 & 4 & 8 & 16  \\
 %$N_c \rightarrow$ & 2 & 4 & 8 & 16 & 2 & 4 & 2 \\
 \hline
 \hline
 \multicolumn{5}{|c|}{Race (FP32 Accuracy = 37.51\%)} & \multicolumn{4}{|c|}{Boolq (FP32 Accuracy = 64.62\%)} \\ 
 \hline
 \hline
 64 & 36.94 & 37.13 & 36.27 & 37.13 & 63.73 & 62.26 & 63.49 & 63.36 \\
 \hline
 32 & 37.03 & 36.36 & 36.08 & 37.03 & 62.54 & 63.51 & 63.49 & 63.55  \\
 \hline
 16 & 37.03 & 37.03 & 36.46 & 37.03 & 61.1 & 63.79 & 63.58 & 63.33  \\
 \hline
 \hline
 \multicolumn{5}{|c|}{Winogrande (FP32 Accuracy = 58.01\%)} & \multicolumn{4}{|c|}{Piqa (FP32 Accuracy = 74.21\%)} \\ 
 \hline
 \hline
 64 & 58.17 & 57.22 & 57.85 & 58.33 & 73.01 & 73.07 & 73.07 & 72.80 \\
 \hline
 32 & 59.12 & 58.09 & 57.85 & 58.41 & 73.01 & 73.94 & 72.74 & 73.18  \\
 \hline
 16 & 57.93 & 58.88 & 57.93 & 58.56 & 73.94 & 72.80 & 73.01 & 73.94  \\
 \hline
\end{tabular}
\caption{\label{tab:mmlu_abalation} Accuracy on LM evaluation harness tasks on GPT3-1.3B model.}
\end{table}

\begin{table} \centering
\begin{tabular}{|c||c|c|c|c||c|c|c|c|} 
\hline
 $L_b \rightarrow$& \multicolumn{4}{c||}{8} & \multicolumn{4}{c||}{8}\\
 \hline
 \backslashbox{$L_A$\kern-1em}{\kern-1em$N_c$} & 2 & 4 & 8 & 16 & 2 & 4 & 8 & 16  \\
 %$N_c \rightarrow$ & 2 & 4 & 8 & 16 & 2 & 4 & 2 \\
 \hline
 \hline
 \multicolumn{5}{|c|}{Race (FP32 Accuracy = 41.34\%)} & \multicolumn{4}{|c|}{Boolq (FP32 Accuracy = 68.32\%)} \\ 
 \hline
 \hline
 64 & 40.48 & 40.10 & 39.43 & 39.90 & 69.20 & 68.41 & 69.45 & 68.56 \\
 \hline
 32 & 39.52 & 39.52 & 40.77 & 39.62 & 68.32 & 67.43 & 68.17 & 69.30  \\
 \hline
 16 & 39.81 & 39.71 & 39.90 & 40.38 & 68.10 & 66.33 & 69.51 & 69.42  \\
 \hline
 \hline
 \multicolumn{5}{|c|}{Winogrande (FP32 Accuracy = 67.88\%)} & \multicolumn{4}{|c|}{Piqa (FP32 Accuracy = 78.78\%)} \\ 
 \hline
 \hline
 64 & 66.85 & 66.61 & 67.72 & 67.88 & 77.31 & 77.42 & 77.75 & 77.64 \\
 \hline
 32 & 67.25 & 67.72 & 67.72 & 67.00 & 77.31 & 77.04 & 77.80 & 77.37  \\
 \hline
 16 & 68.11 & 68.90 & 67.88 & 67.48 & 77.37 & 78.13 & 78.13 & 77.69  \\
 \hline
\end{tabular}
\caption{\label{tab:mmlu_abalation} Accuracy on LM evaluation harness tasks on GPT3-8B model.}
\end{table}

\begin{table} \centering
\begin{tabular}{|c||c|c|c|c||c|c|c|c|} 
\hline
 $L_b \rightarrow$& \multicolumn{4}{c||}{8} & \multicolumn{4}{c||}{8}\\
 \hline
 \backslashbox{$L_A$\kern-1em}{\kern-1em$N_c$} & 2 & 4 & 8 & 16 & 2 & 4 & 8 & 16  \\
 %$N_c \rightarrow$ & 2 & 4 & 8 & 16 & 2 & 4 & 2 \\
 \hline
 \hline
 \multicolumn{5}{|c|}{Race (FP32 Accuracy = 40.67\%)} & \multicolumn{4}{|c|}{Boolq (FP32 Accuracy = 76.54\%)} \\ 
 \hline
 \hline
 64 & 40.48 & 40.10 & 39.43 & 39.90 & 75.41 & 75.11 & 77.09 & 75.66 \\
 \hline
 32 & 39.52 & 39.52 & 40.77 & 39.62 & 76.02 & 76.02 & 75.96 & 75.35  \\
 \hline
 16 & 39.81 & 39.71 & 39.90 & 40.38 & 75.05 & 73.82 & 75.72 & 76.09  \\
 \hline
 \hline
 \multicolumn{5}{|c|}{Winogrande (FP32 Accuracy = 70.64\%)} & \multicolumn{4}{|c|}{Piqa (FP32 Accuracy = 79.16\%)} \\ 
 \hline
 \hline
 64 & 69.14 & 70.17 & 70.17 & 70.56 & 78.24 & 79.00 & 78.62 & 78.73 \\
 \hline
 32 & 70.96 & 69.69 & 71.27 & 69.30 & 78.56 & 79.49 & 79.16 & 78.89  \\
 \hline
 16 & 71.03 & 69.53 & 69.69 & 70.40 & 78.13 & 79.16 & 79.00 & 79.00  \\
 \hline
\end{tabular}
\caption{\label{tab:mmlu_abalation} Accuracy on LM evaluation harness tasks on GPT3-22B model.}
\end{table}

\begin{table} \centering
\begin{tabular}{|c||c|c|c|c||c|c|c|c|} 
\hline
 $L_b \rightarrow$& \multicolumn{4}{c||}{8} & \multicolumn{4}{c||}{8}\\
 \hline
 \backslashbox{$L_A$\kern-1em}{\kern-1em$N_c$} & 2 & 4 & 8 & 16 & 2 & 4 & 8 & 16  \\
 %$N_c \rightarrow$ & 2 & 4 & 8 & 16 & 2 & 4 & 2 \\
 \hline
 \hline
 \multicolumn{5}{|c|}{Race (FP32 Accuracy = 44.4\%)} & \multicolumn{4}{|c|}{Boolq (FP32 Accuracy = 79.29\%)} \\ 
 \hline
 \hline
 64 & 42.49 & 42.51 & 42.58 & 43.45 & 77.58 & 77.37 & 77.43 & 78.1 \\
 \hline
 32 & 43.35 & 42.49 & 43.64 & 43.73 & 77.86 & 75.32 & 77.28 & 77.86  \\
 \hline
 16 & 44.21 & 44.21 & 43.64 & 42.97 & 78.65 & 77 & 76.94 & 77.98  \\
 \hline
 \hline
 \multicolumn{5}{|c|}{Winogrande (FP32 Accuracy = 69.38\%)} & \multicolumn{4}{|c|}{Piqa (FP32 Accuracy = 78.07\%)} \\ 
 \hline
 \hline
 64 & 68.9 & 68.43 & 69.77 & 68.19 & 77.09 & 76.82 & 77.09 & 77.86 \\
 \hline
 32 & 69.38 & 68.51 & 68.82 & 68.90 & 78.07 & 76.71 & 78.07 & 77.86  \\
 \hline
 16 & 69.53 & 67.09 & 69.38 & 68.90 & 77.37 & 77.8 & 77.91 & 77.69  \\
 \hline
\end{tabular}
\caption{\label{tab:mmlu_abalation} Accuracy on LM evaluation harness tasks on Llama2-7B model.}
\end{table}

\begin{table} \centering
\begin{tabular}{|c||c|c|c|c||c|c|c|c|} 
\hline
 $L_b \rightarrow$& \multicolumn{4}{c||}{8} & \multicolumn{4}{c||}{8}\\
 \hline
 \backslashbox{$L_A$\kern-1em}{\kern-1em$N_c$} & 2 & 4 & 8 & 16 & 2 & 4 & 8 & 16  \\
 %$N_c \rightarrow$ & 2 & 4 & 8 & 16 & 2 & 4 & 2 \\
 \hline
 \hline
 \multicolumn{5}{|c|}{Race (FP32 Accuracy = 48.8\%)} & \multicolumn{4}{|c|}{Boolq (FP32 Accuracy = 85.23\%)} \\ 
 \hline
 \hline
 64 & 49.00 & 49.00 & 49.28 & 48.71 & 82.82 & 84.28 & 84.03 & 84.25 \\
 \hline
 32 & 49.57 & 48.52 & 48.33 & 49.28 & 83.85 & 84.46 & 84.31 & 84.93  \\
 \hline
 16 & 49.85 & 49.09 & 49.28 & 48.99 & 85.11 & 84.46 & 84.61 & 83.94  \\
 \hline
 \hline
 \multicolumn{5}{|c|}{Winogrande (FP32 Accuracy = 79.95\%)} & \multicolumn{4}{|c|}{Piqa (FP32 Accuracy = 81.56\%)} \\ 
 \hline
 \hline
 64 & 78.77 & 78.45 & 78.37 & 79.16 & 81.45 & 80.69 & 81.45 & 81.5 \\
 \hline
 32 & 78.45 & 79.01 & 78.69 & 80.66 & 81.56 & 80.58 & 81.18 & 81.34  \\
 \hline
 16 & 79.95 & 79.56 & 79.79 & 79.72 & 81.28 & 81.66 & 81.28 & 80.96  \\
 \hline
\end{tabular}
\caption{\label{tab:mmlu_abalation} Accuracy on LM evaluation harness tasks on Llama2-70B model.}
\end{table}

%\section{MSE Studies}
%\textcolor{red}{TODO}


\subsection{Number Formats and Quantization Method}
\label{subsec:numFormats_quantMethod}
\subsubsection{Integer Format}
An $n$-bit signed integer (INT) is typically represented with a 2s-complement format \citep{yao2022zeroquant,xiao2023smoothquant,dai2021vsq}, where the most significant bit denotes the sign.

\subsubsection{Floating Point Format}
An $n$-bit signed floating point (FP) number $x$ comprises of a 1-bit sign ($x_{\mathrm{sign}}$), $B_m$-bit mantissa ($x_{\mathrm{mant}}$) and $B_e$-bit exponent ($x_{\mathrm{exp}}$) such that $B_m+B_e=n-1$. The associated constant exponent bias ($E_{\mathrm{bias}}$) is computed as $(2^{{B_e}-1}-1)$. We denote this format as $E_{B_e}M_{B_m}$.  

\subsubsection{Quantization Scheme}
\label{subsec:quant_method}
A quantization scheme dictates how a given unquantized tensor is converted to its quantized representation. We consider FP formats for the purpose of illustration. Given an unquantized tensor $\bm{X}$ and an FP format $E_{B_e}M_{B_m}$, we first, we compute the quantization scale factor $s_X$ that maps the maximum absolute value of $\bm{X}$ to the maximum quantization level of the $E_{B_e}M_{B_m}$ format as follows:
\begin{align}
\label{eq:sf}
    s_X = \frac{\mathrm{max}(|\bm{X}|)}{\mathrm{max}(E_{B_e}M_{B_m})}
\end{align}
In the above equation, $|\cdot|$ denotes the absolute value function.

Next, we scale $\bm{X}$ by $s_X$ and quantize it to $\hat{\bm{X}}$ by rounding it to the nearest quantization level of $E_{B_e}M_{B_m}$ as:

\begin{align}
\label{eq:tensor_quant}
    \hat{\bm{X}} = \text{round-to-nearest}\left(\frac{\bm{X}}{s_X}, E_{B_e}M_{B_m}\right)
\end{align}

We perform dynamic max-scaled quantization \citep{wu2020integer}, where the scale factor $s$ for activations is dynamically computed during runtime.

\subsection{Vector Scaled Quantization}
\begin{wrapfigure}{r}{0.35\linewidth}
  \centering
  \includegraphics[width=\linewidth]{sections/figures/vsquant.jpg}
  \caption{\small Vectorwise decomposition for per-vector scaled quantization (VSQ \citep{dai2021vsq}).}
  \label{fig:vsquant}
\end{wrapfigure}
During VSQ \citep{dai2021vsq}, the operand tensors are decomposed into 1D vectors in a hardware friendly manner as shown in Figure \ref{fig:vsquant}. Since the decomposed tensors are used as operands in matrix multiplications during inference, it is beneficial to perform this decomposition along the reduction dimension of the multiplication. The vectorwise quantization is performed similar to tensorwise quantization described in Equations \ref{eq:sf} and \ref{eq:tensor_quant}, where a scale factor $s_v$ is required for each vector $\bm{v}$ that maps the maximum absolute value of that vector to the maximum quantization level. While smaller vector lengths can lead to larger accuracy gains, the associated memory and computational overheads due to the per-vector scale factors increases. To alleviate these overheads, VSQ \citep{dai2021vsq} proposed a second level quantization of the per-vector scale factors to unsigned integers, while MX \citep{rouhani2023shared} quantizes them to integer powers of 2 (denoted as $2^{INT}$).

\subsubsection{MX Format}
The MX format proposed in \citep{rouhani2023microscaling} introduces the concept of sub-block shifting. For every two scalar elements of $b$-bits each, there is a shared exponent bit. The value of this exponent bit is determined through an empirical analysis that targets minimizing quantization MSE. We note that the FP format $E_{1}M_{b}$ is strictly better than MX from an accuracy perspective since it allocates a dedicated exponent bit to each scalar as opposed to sharing it across two scalars. Therefore, we conservatively bound the accuracy of a $b+2$-bit signed MX format with that of a $E_{1}M_{b}$ format in our comparisons. For instance, we use E1M2 format as a proxy for MX4.

\begin{figure}
    \centering
    \includegraphics[width=1\linewidth]{sections//figures/BlockFormats.pdf}
    \caption{\small Comparing LO-BCQ to MX format.}
    \label{fig:block_formats}
\end{figure}

Figure \ref{fig:block_formats} compares our $4$-bit LO-BCQ block format to MX \citep{rouhani2023microscaling}. As shown, both LO-BCQ and MX decompose a given operand tensor into block arrays and each block array into blocks. Similar to MX, we find that per-block quantization ($L_b < L_A$) leads to better accuracy due to increased flexibility. While MX achieves this through per-block $1$-bit micro-scales, we associate a dedicated codebook to each block through a per-block codebook selector. Further, MX quantizes the per-block array scale-factor to E8M0 format without per-tensor scaling. In contrast during LO-BCQ, we find that per-tensor scaling combined with quantization of per-block array scale-factor to E4M3 format results in superior inference accuracy across models. 

\newpage
\appendix
\section{Additional Details: Methodology}
% \par \noindent \textbf{Limitations of Existing Methods:}
\subsection{Limitations of Existing Methods}
\label{appendix:A.1}
The first step in time-series modeling using transformer-based architectures is to learn an embedding of the time-series $\mathbf{X}$, which is then fed into the transformer encoder. Traditionally, this is done using an Embedding-layer (typically implemented using a multi-layered perceptron) as $\texttt{Embedding}:\mathbb{R}^N \mapsto \mathbb{R}^D$ that maps $\mathbf{X} \in \mathbb{R}^{T \times N}$ to the embedding  $\mathbf{H} \in \mathbb{R}^{T \times D}$, where $D$ is the embedding dimension. The Embedding layer operates on every time-step independently such that the set of variates observed at time-step $t$, $\mathbf{X}_{(t, :)}$, is considered as a single token and mapped to the embedding vector $\mathbf{h}_{t} \in \mathbb{R}^{D}$ as $\mathbf{h}_{t} = \texttt{Embedding}(\mathbf{X}_{(t, :)})$ (see Figure \ref{fig:tfi}(a)). An alternate embedding scheme was recently introduced in the framework of inverted Transformer \cite{liu2023itransformer},  where the uni-variate time-series for the $d$-th variate, $\mathbf{X}_{(:, d)}$, is considered as a single token and mapped to the embedding vector: $\mathbf{h}_{d} = \texttt{Embedding}(\mathbf{X}_{(:, d)})$ (see Figure \ref{fig:tfi}(b)). 
% Schematic representations of the embedding scheme for original transfomer and iTransformer are depicted in \textcolor{red}{Figure XX}. 
While both these embedding schemes have their unique advantages, they are unsuitable to handle time-series with arbitrary sets of missing values at every time-step. In particular, the input tokens to the Embedding layer of Transformer or iTransformer requires all components of $\mathbf{X}_{(t, :)}$ or $\mathbf{X}_{(:, d)}$ to be observed, respectively.
% Assuming that the time-series $\mathbf{X}$ has missing values, these arbitrary tokens might have missing values in them as well. 
% This in-turn would prevent us from embedding the entire token using the aforementioned $\texttt{Embedding}$ layers. 
If any of the components in these tokens are missing, we will not be able to compute their embeddings and thus will have to discard either the time-step or the variate, leading to loss of information.
\begin{figure}[ht]
    \centering
    \includegraphics[width=0.85\linewidth, scale=0.1]{figures/Encoding.pdf}
    \caption{Schematic of the Time-Feature Independent (TFI) Embedding of MissTSM that learns a different embedding for every combination of time-step and variate, in contrast to the time-only embeddings of Transformer \cite{vaswani2017attention} and the variate-only embeddings of iTransformers \cite{liu2023itransformer}.} 
    % MissTSM uses a novel  for every , allowing it to handle time-steps with missing values using masked cross-attention.}
    \label{fig:tfi}
\end{figure}
\subsection{2D Positional Encodings}
\label{appendix:A.2}
We add Positional Encoding vectors $\mathbf{PE}$ to the TFI embedding $\mathbf{H}^{\mathrm{TFI}}$ to obtain positionally-encoded embeddings, $\mathbf{Z} = \mathbf{PE} + \mathbf{H}^{\mathrm{TFI}}$.
% Since the TFI Embedding scheme maps each time-feature combination $\mathbf{X}_{(t, d)}$ into a higher-dimensional embedding,
Since TFI embeddings treat every time-feature combination as a token, we use a 2D-positional encoding scheme  defined as follows:

\begin{align}
    &\texttt{PE}(t, d, 2i) = \sin \big(\frac{t}{10000^{(4i/D)}} \big) ; \quad \texttt{PE}(t, d, 2i+1) = \cos \big(\frac{t}{10000^{(4i/D)}} \big), \\ 
    &\texttt{PE}(t, d, 2j+D/2) = \sin \big(\frac{d}{10000^{(4j/D)}} \big) ; \quad \texttt{PE}(t, d, 2j+1+D/2) = \cos \big(\frac{d}{10000^{(4j/D)}} \big),
\end{align}
where $t$ is the time-step, $d$ is the feature, and $i, j \in [0, D/4)$ are integers. 

\section{Experimental Setup}
\subsection{Dataset Description}
\label{appendix:B.1}
\textbf{Forecasting Dataset Details} 

\textbf{ETT.} The ETT \cite{informer} dataset captures load and oil temperature data from electricity transformers. ETTh2 includes 17,420 hourly observations, while ETTm2 comprises 69,680 15-minute observations. Both datasets span two years and contain 7 variates each. 

\textbf{Weather.} Weather \cite{weather} is a 10-minute frequency time-series dataset recorded throughout the year 2020 and consists of 21 meteorological indicators, like humidity, air temperature, etc.  

Following previous works in this area, we use a train-validation-test split of 6:2:2 for the ETT datasets and 7:1:2 for the Weather dataset. We standardized the input features by subtracting off the mean and dividing by the standard deviation for every feature over the training set. Again, following the approach used in previous works, we compute the MSE in the normalized space of all features considering all features together. 

% \textcolor{red}{\textbf{Lake Datasets}. Falling Creeks Reservoir (FCR) data extraction and processing details. Mendota data extraction and processing details.}

\textbf{Classification Dataset Details}

\textbf{Epilepsy.} Epilepsy \cite{epilepsy} contains univariate brainwaves (single-channel EEG) sampled from 500 subjects (with 11,500 samples in total), with each sample classified as having epilepsy or not (binary classification).  

\textbf{Gesture.} Gesture \cite{gesture} dataset consists of 560 samples, each having 3 variates (corresponding to the accelerometer data) and each sample corresponding to one of the 8 hand gestures (or classes) 

\textbf{EMG.} EMG \cite{emg} dataset contains 163 EMG (Electromyography) samples corresponding to 3-classes of muscular diseases. 

We make use of the following readily available data splits (train, validation, test) for each of the datasets: 
\textbf{Epilepsy} = 60 (30 samples per each class)/20 (10 samples per each class)/11420 (Train/Val/Test) 
\textbf{Gesture} = 320/20/120 (Train/Val/Test) 
\textbf{EMG} = 122/41/41 (Train/Val/Test)

\subsection{Synthetic Masked Data Generation}
\label{appendix:B.3}
\textbf{Random Masking}: We generated masks by randomly selecting data points across all variates and time-steps, assigning them as missing with a likelihood determined by p (masking fraction). The selected data points were then removed, effectively simulating missing values at random. For multiple runs, we created multiple such versions of the synthetic datasets and compared all baseline methods and MissTSM on the same datasets. 

\textbf{Periodic Masking}: We use a sine curve to generate the masking periodicity with given phase and frequency values for different features. Specifically, the time-dependent periodic probability of seeing missing values is defined as $\hat{\texttt{p}}(t) = \texttt{p} + \alpha(1-\texttt{p}){\sin(2\pi \nu \texttt{t} + \phi)}$, where, $\phi$ and $\nu$ are randomly chosen across the feature space, $\alpha$ is a scale factor, and $\texttt{p}$ is an offset term. We vary $\texttt{p}$ from low to high values to get different fractions of periodic missing values in the data. To implement this masking strategy, each feature in the dataset was assigned a unique frequency, randomly selected from the range [0.2, 0.8]. This was done to reduce bias and increase randomness in periodicity across the feature space. Additionally, the phase shift was chosen randomly from the range [0, 2$\pi$]. This was applied to each feature to offset the sinusoidal function over time. Like frequency, the phase value was different for different features. This generated a periodic pattern for the likelihood of missing data. 


\subsection{Implementation Details}
\label{appendix:B.4}
The experiments have been implemented in PyTorch using NVIDIA TITAN 24 GB GPU. The baselines have been implemented following their official code and configurations. We consider Mean Squared Error (MSE) as the metric for time-series forecasting and F1-score for the classification tasks.

\textbf{Forecasting experiments}. MissTSM was trained with the MSE loss, using the Adam \cite{adam} optimizer with a learning rate of 1e-3 during pre-training for 50 epochs and a learning rate of 1e-4 during finetuning with an early stopping counter of 3 epochs. Batch size was set to 16. All the reported missing data experiment results are obtained over 5 trials (5 different masked versions). During fine-tuning for different Prediction lengths (96, 192, 336, 720), we used the same pre-trained encoder and added a linear layer at the top of the encoder.  

\textbf{Classification experiments}. MissTSM was trained using the Adam \cite{adam} optimizer, with MSE as the loss function during pre-training and Cross-Entropy loss during fine-tuning. During fine-tuning, we plugged a 64-D linear layer at the top of the pre-trained encoder. We pre-trained and fine-tuned for 100 epochs. 

\subsection{Hyper-parameter Details}
\label{appendix:B.5}
For MissTSM, we start with the same set of hyper-parameters as reported in the SimMTM paper as initialization (see Table \ref{tab:params}), and then search for the best learning rate in factors of 10, and encoder/decoder layers in the range [2, 4]. Note that we only perform hyper-parameter tuning on 100\% data, and use the same hyper-parameters for all experiments involving the dataset, such as different missing value probabilities. 
Our goal is to show the generic effectiveness of our MissTSM framework even without any rigorous hyper-parameter optimization. Additionally, we would also like to note that our model sizes are relatively very small (number of parameters for ETTh2=28,080, Weather= 149,824, and ETTm2= 28,952), compared to other baselines such as SimMTM (ETTh2=4,694,186), iTransformer (ETTh2=254,944), and PatchTST (ETTh2=81,728).

\begin{table*}[htbp]
\caption{Hyperparameters for Forecasting and Classification Tasks}
\centering
\renewcommand{\arraystretch}{1.0}
\scriptsize
\resizebox{\textwidth}{!}{
\begin{tabular}{lcccccc}
\toprule
\textbf{Task} & \textbf{\# Enc. Layers} & \textbf{\# Dec. Layers} & \textbf{\# Enc. Heads} & \textbf{\# Dec. Heads} & \textbf{Enc. Embed Dim} & \textbf{Dec. Embed Dim} \\
\midrule
\textbf{Forecasting} & & & & & & \\
ETTh2      & 2  & 2  & 8  & 4  & 8   & 32  \\
ETTm2      & 3  & 2  & 8  & 4  & 8   & 32  \\
Weather    & 2  & 2  & 8  & 4  & 64  & 32  \\
\midrule
\textbf{Classification} & & & & & & \\
All Datasets & 3  & 2  & 16 & 16 & 32  & 32  \\
\bottomrule
\end{tabular}
\label{tab:params}
}
\end{table*}

% \subsection{Hyper-parameter sensitivity}
% \label{appendix:B.6}
\begin{table}[htbp]
\caption{Hyper-parameter sensitivity of MissTSM on ETTh2 with 70\% Masking Fraction, MCAR. Best results shown in bold, second best underlined. Hyper-parameter settings used in the remainder of experiments in the paper are italicized.}
\vspace{2pt}
\centering
\small
\begin{tabular}{llllllllll}
% \cline{2-10}
\toprule
    & \multicolumn{3}{l}{Enc. Heads}                         & \multicolumn{3}{l}{Enc. Layers}                        & \multicolumn{3}{l}{Enc. Embed Dim}                     \\ \cline{2-10} 
    & 1              & 4              & 8                    & 1              & 2                       & 3           & 8                       & 16             & 32          \\ \cline{2-10} 
96  & {\ul 0.246}    & \textbf{0.245} & {\ul \textit{0.246}} & 0.249          & \textit{\textbf{0.243}} & {\ul 0.244} & \textit{\textbf{0.243}} & {\ul 0.248}    & 0.285       \\
192 & \textbf{0.261} & 0.273          & {\ul \textit{0.266}} & 0.287          & \textit{\textbf{0.267}} & {\ul 0.271} & {\ul \textit{0.267}}    & \textbf{0.266} & 0.340       \\
336 & 0.312          & \textbf{0.279} & {\ul \textit{0.310}} & \textbf{0.294} & \textit{0.392}          & {\ul 0.307} & \textit{0.392}          & \textbf{0.316} & {\ul 0.369} \\
720 & \textbf{0.326} & 0.346          & {\ul \textit{0.333}} & {\ul 0.351}    & \textit{\textbf{0.323}} & 0.355       & \textit{\textbf{0.323}} & {\ul 0.338}    & 0.446      
% \midrule
\vspace{8pt}
\end{tabular}
% \hfill
% \vspace{3pt}
\begin{tabular}{llllllllll}
% \cline{2-10}
% \toprule
    & \multicolumn{3}{l}{Dec. Heads}                         & \multicolumn{3}{l}{Dec. Layers}                        & \multicolumn{3}{l}{Dec. Embed Dim}                     \\ \cline{2-10} 
    & 1              & 4              & 8                    & 1              & 2                       & 3           & 8                       & 16             & 32          \\ \cline{2-10} 
96  & 0.261    & \textit{\textbf{0.243}} & {\ul 0.252} & 0.276          & \textit{\textbf{0.242}} & {\ul 0.248} & {\ul 0.250} & 0.259    & \textbf{\textit{0.243}}       \\
192 & 0.276 & \textbf{\textit{0.267}}          & {\ul 0.272} & \textbf{0.266}          & \textit{{\ul 0.268}} & {\ul 0.268} & \textbf{0.257}    & 0.272 & \textit{{\ul 0.267}}       \\
336 & {\ul 0.319}          & \textit{0.392} & \textbf{0.301} & \textbf{0.262} & \textit{0.352}          & {\ul 0.271} & {\ul 0.289}          & \textbf{0.266} & \textit{0.392} \\
720 & {\ul 0.324} & \textit{\textbf{0.323}}          & 0.330 & \textbf{0.323}   & \textit{0.364} & {\ul 0.341}       & {\ul 0.353} & 0.384   & \textbf{\textit{0.323}}     
% \bottomrule
\end{tabular}
\end{table}

% \subsection{Synthetic Masking Schemes}
% % \par
% To simulate varying scenarios of missing values appearing in real-world time-series datasets, we propose two synthetic masking schemes that we apply on benchmark datasets, namely missing completely at random (MCAR) masking and periodic masking, described in the following. 

% \textbf{Missing Completely At Random (MCAR) Masking:} In this scheme, we randomly mask out data from a benchmark dataset based on a uniform probability of seeing missing values at any time-feature combination. We vary the probability value to generate synthetically masked datasets with different fractions of missing values. 
% % \par

% \textbf{Periodic Masking:} Since missing values in time-series follow periodic patterns in many real-world applications (e.g., the seasonal cycles in weather and environmental datasets), we introduce a periodic masking scheme described as follows. We use a sine curve to generate the masking periodicity with given phase and frequency values for different features. Specifically, the time-dependent periodic probability of seeing missing values is defined as $\hat{\texttt{p}}(t) = \texttt{p} + \alpha(1-\texttt{p}){\sin(2\pi \nu \texttt{t} + \phi)}$, where, $\phi$ and $\nu$ are randomly chosen across the feature space, $\alpha$ is a scale factor, and $\texttt{p}$ is an offset term. We vary $\texttt{p}$ from low to high values to get different fractions of periodic missing values in the data.

\section{Additional Results}
\subsection{Embedding of 1D data}
\label{appendix:C.1}
To understand the usefulness of mapping 1D data to multi-dimensional data in TFI embedding, we present (in Table \ref{tab:embed}) an ablation comparing performances on ETTh2 with and without using high-dimensional projections in TFI Embedding under the no missing value scenario. Projecting 1D scalars independently to higher-dimensional vectors may look wasteful at the time of initialization of TFI Embedding, when the context of time and variates are not incorporated. However, it is during the cross-attention stage (using MFAA layer or later using the Transformer encoder block) that we can leverage the high-dimensional embeddings to store richer representations bringing in the context of time and variate in which every data point resides. 

From Table \ref{tab:embed}, we can see that TFI embedding with 8-dimensional vectors consistently outperform the ablation with 1D representations, empirically demonstrating the importance of high-dimensional projections in our proposed framework.

\begin{table}[htbp]
\centering
\caption{Effect of TFI Embedding with embedding size=1 and embedding size=8 under no masking scenario. Dataset=ETTh2}
\renewcommand{\arraystretch}{1.0}
\scriptsize
\resizebox{0.8\textwidth}{!}{
\begin{tabular}{l>{\centering\arraybackslash}p{3cm} >{\centering\arraybackslash}p{3cm}}
\toprule
\textbf{Time Horizon} & \textbf{TFI Embedding with embedding size = 1} & \textbf{TFI Embedding with embedding size = 8} \\
\midrule
96  & 0.283 ± 0.048 & \textbf{0.245 ± 0.011} \\
192 & 0.285 ± 0.078 & \textbf{0.260 ± 0.023} \\
336 & 0.319 ± 0.023 & \textbf{0.300 ± 0.016} \\
720 & 0.378 ± 0.022 & \textbf{0.334 ± 0.032} \\
\bottomrule
\end{tabular}}
\label{tab:embed}
\end{table}

\subsection{Forecasting}
\label{appendix:C.2}
Table \ref{tab:main_table} compares the forecasting performance of MissTSM with five SOTA baseline methods in terms of the Mean Squared Error (MSE) metric on three datasets (ETTh2, ETTm2 and Weather) with varying forecasting horizons, imputation techniques (Spline and SAITS), and masking schemes. We provide the mean and standard deviations over 5 different samples of the masking schemes. We choose a missing value probability of 60\% for MCAR masking and 70\% for periodic masking to simulate scenarios with varying (and often extreme) amounts of missing information. 
We can see that in the no masking experiment, the performance of all methods (with the exception of AutoFormer) are mostly comparable to each other across all three datasets, with MissTSM and PatchTST having a slight edge on the ETTh2/ETTm2 and Weather datasets, respectively. 
For the MCAR masking experiments, we observe a trend across all the datasets that the MissTSM framework performs slightly better than the baselines for longer-term forecasting (such as forecasting horizon of 720), and comparable to the best-performing baselines on other forecasting horizons. For the Periodic masking experiment, we can see that MissTSM is consistently better than the baselines for ETTh2 dataset, while for the ETTm2 and Weather datasets, the forecasting performance is comparable to the other baselines. These results demonstrate the effectiveness of our proposed MissTSM framework to circumvent the need for explicit imputation of missing values while achieving comparable  performance as SOTA. 

% \sout{Additionally, we observe on the three datasets that the baselines with SAITS-based imputation is comparable or slightly better than the corresponding spline interpolation counterpart.}

By being imputation-free, MissTSM  does not suffer from the propagation of imputation errors  (from the imputation scheme) to forecasting errors  (from the time-series models). In Appendix Figure \ref{fig:err_propagation}, we provide empirical evidence of this error propagation, where we see a positive correlation between imputation errors and forecasting errors of baseline methods, indicating that reducing imputation errors is crucial for improving forecasting accuracy. This finding underscores the limitations of traditional two-stage approaches and suggests that using more sophisticated imputation models is necessary to achieve lower forecasting errors. We also report the computation time of SimMTM (with Spline and SAITS) and MissTSM in Appendix Table \ref{tab:computation}, where we demonstrate that MissTSM is significantly faster as it does not involve any expensive interpolations as an additional advantage.

\begin{table}[!t]
  \caption{Comparing forecasting performance of baseline methods using mean squared error (MSE) as the evaluation metric under no masking, MCAR masking, and periodic masking. For every dataset, we consider multiple forecasting horizons, $T \in \{96, 192, 336, 720\}$. Results are color-coded as \colorbox{blue!35} {Best}, \colorbox{blue!15} {Second best}. We report the mean and standard deviations (in brackets) across 5 random sampling of the masking schemes. Subscript $_{SP}$ refer to Spline and $_{SA}$ refer to SAITS}
  \vspace{-2ex}
  \label{tab:main_table}
  \renewcommand{\arraystretch}{1.4}
  \setlength{\tabcolsep}{3pt}
  \centering
  \resizebox{0.99\textwidth}{!}{
    \Large
    \begin{tabular}{cllllrlllrllllr}
                                                                                                 &                                      &                                      &                                             &                                      & \multicolumn{1}{l}{}                                      &                                                   &                                                   &                                             & \multicolumn{1}{l}{}                                          &                                             &                                          &                                          &                                          \\ \hline
      &                                       & \multicolumn{4}{c|}{\textbf{ETTh2}}                                                                                                                                                    & \multicolumn{4}{c|}{\textbf{ETTm2}}                                                                                                                                                          & \multicolumn{4}{c|}{\textbf{Weather}}   &  
      \multicolumn{1}{c}{\multirow{2}{*}{\begin{tabular}[c]{@{}c@{}}{\textbf{Avg}} \\ {\textbf{Rank}}\end{tabular}}}                                                                                                                          
      \\ 
      \cline{3-14}
      \multicolumn{1}{l}{}                                                                       &                                      & \textbf{96}                          & \textbf{192}                                & \textbf{336}                         & \multicolumn{1}{l|}{\textbf{720}}                         & \textbf{96}                                       & \textbf{192}                                      & \textbf{336}                                & \multicolumn{1}{l|}{\textbf{720}}                             & \textbf{96}                                 & \textbf{192}                             & \textbf{336}                             & \multicolumn{1}{l|}{\textbf{720} }                            
      \\ 
      \cline{1-15}
      \multicolumn{1}{c|}{\multirow{6}{*}{\rotatebox[origin=c]{90}{\textbf{No Masking}}}}        & \textbf{MissTSM}                     & \cellcolor{blue!35}${0.255}$         & \cellcolor{blue!35}${0.234}$                & \cellcolor{blue!35}${0.316}$         & \multicolumn{1}{l|}{\cellcolor{blue!35}${0.305}$}         & $0.183$                                           & \cellcolor{blue!35}${0.209}$                      & \cellcolor{blue!35}${0.261}$                & \multicolumn{1}{l|}{\cellcolor{blue!35}${0.311}$}             & $0.164$                                     & $0.210$                                  & \cellcolor{blue!15}$0.254$               & \multicolumn{1}{l|}{$0.324$}          & \multicolumn{1}{c}{$1.9$}                        \\
      \multicolumn{1}{c|}{}                                                                      & \textbf{SimMTM}                      & $0.295$                              & $0.356$                                     & $0.375$                              & \multicolumn{1}{l|}{$0.404$}                              & $0.172$                                           & $0.223$                                           & $0.282$                  & \multicolumn{1}{l|}{$0.374$}                                  & \cellcolor{blue!15}$0.163$                  & \cellcolor{blue!15}$0.203$               & $0.255$                                  & \multicolumn{1}{l|}{$0.326$} & \multicolumn{1}{c}{$2.9$}
      
      \\
      
      \multicolumn{1}{c|}{}                                                                      & \textbf{PatchTST}                    & \cellcolor{blue!15}$0.274$           & \cellcolor{blue!15}$0.338$                  & \cellcolor{blue!15}0.330             & \multicolumn{1}{l|}{\cellcolor{blue!15}$0.378$}           & \cellcolor{blue!35}$0.164$                        & \cellcolor{blue!15}$0.220$                        & \cellcolor{blue!15}$0.277$                                     & \multicolumn{1}{l|}{\cellcolor{blue!15}$0.367$}               & \cellcolor{blue!35}${0.151}$                & \cellcolor{blue!35}${0.196}$             & \cellcolor{blue!35}${0.249}$             & \multicolumn{1}{l|}{\cellcolor{blue!35}${0.319}$}             
      & \multicolumn{1}{c}{$1.7$}
      \\
      
      \multicolumn{1}{c|}{}                                                                      & \textbf{AutoFormer}                  & $0.501$                              & $0.516$                                     & $0.565$                              & \multicolumn{1}{l|}{$0.462$}                              & $0.352$                                           & $0.337$                                           & $0.494$                                     & \multicolumn{1}{l|}{$0.474$}                                  & $0.306$                                     & $0.434$                                  & $0.437$                                  & \multicolumn{1}{l|}{$0.414$}                                  & \multicolumn{1}{c}{$5.9$}
      
      \\
      
      \multicolumn{1}{c|}{}                                                                      & \textbf{DLinear}                     & $0.288$                              & $0.383$                                     & $0.447$                              & \multicolumn{1}{l|}{$0.605$}                              & \cellcolor{blue!15}$0.168$                        & $0.224$                                           & $0.299$                                     & \multicolumn{1}{l|}{$0.414$}                                  & \textbf{$0.175$}                            & \textbf{$0.219$}                         & $0.265$                                  & \multicolumn{1}{l|}{\cellcolor{blue!15}$0.323$}               & \multicolumn{1}{c}{$4.1$}
      \\
      
      \multicolumn{1}{c|}{}                                                                      & \textbf{iTransformer}                & $0.304$                              & $0.392$                                     & $0.425$                              & \multicolumn{1}{l|}{$0.415$}                              & $0.176$                                           & $0.246$                                           & $0.289$                                     & \multicolumn{1}{l|}{$0.379$}                                  & \cellcolor{blue!15}{$0.163$}                & \cellcolor{blue!15}$0.203$               & $0.256$                                  & \multicolumn{1}{l|}{$0.326$}                                  & \multicolumn{1}{c}{$4.5$}
      
      \\ 
      
      \cline{2-15}
      
      \multicolumn{1}{c|}{\multirow{11}{*}{\rotatebox[origin=c]{90}{\textbf{MCAR Masking}}}}     & \multicolumn{1}{l}{\textbf{MissTSM}} & $\cellcolor{blue!35}{0.243_{0.006}}$ & $\cellcolor{blue!35}0.259_{0.002}$          & $\cellcolor{blue!35}0.283_{0.009}$   & \multicolumn{1}{r|}{$\cellcolor{blue!35}0.329_{0.011}$}   & $0.224_{0.005}$                                   & $0.253_{0.009}$                                   & \cellcolor{blue!15}$0.293_{0.019}$                             & \multicolumn{1}{r|}{\cellcolor{blue!35}$0.316_{0.014}$}       & $0.191_{0.003}$                             & $0.234_{0.006}$                          & $0.281_{0.004}$                          & \multicolumn{1}{l|}{\cellcolor{blue!35}$0.322_{0.008}$}       
      & \multicolumn{1}{c}{$2.7$}
      \\
      
      \multicolumn{1}{c|}{}                                                                      & \textbf{SimMTM$_{\text{SP}}$}        & $0.309_{0.001}$                      & \textbf{$0.372_{0.005}$}                    & \textbf{$0.396_{0.01}$}             & \multicolumn{1}{r|}{\textbf{$0.418_{0.008}$}}             & $0.185_{0.001}$                                   & \cellcolor{blue!15}$0.243_{0.002}$                                   & { $0.298_{0.001}$}                       & \multicolumn{1}{r|}{\textbf{$0.388_{0.005}$}}                 & $0.203_{0.009}$                            & $0.242_{0.010}$                          & $0.284_{0.008}$                          & \multicolumn{1}{l|}{\textbf{$0.386_{0.008}$}}                 
      & \multicolumn{1}{c}{$5.0$}
      \\
      
      \multicolumn{1}{c|}{}                                                                      & \textbf{SimMTM$_{\text{SA}}$}        & \textbf{$0.457_{0.06}$}             & \textbf{$0.510_{0.061}$} & \textbf{$0.503_{0.055}$}             & \multicolumn{1}{r|}{$0.472_{0.066}$}                      & \textbf{$0.287_{0.037}$}       & { \textbf{$0.320_{0.035}$}} & \textbf{$0.342_{0.017}$} & \multicolumn{1}{r|}{{ $0.413_{0.014}$}} & { $0.187_{0.002}$}    & { $0.240_{0.001}$}                    & \textbf{$0.280_{0.001}$}                 & \multicolumn{1}{l|}{$0.385_{0.004}$}                          & \multicolumn{1}{c}{$6.2$}
      
      \\
      
      \multicolumn{1}{c|}{}                                                                      & \textbf{PatchTST$_{\text{SP}}$}      & \textbf{$0.290_{0.003}$}             & \cellcolor{blue!15}\textbf{$0.355_{0.003}$}                    & $\cellcolor{blue!15}0.345_{0.003}$   & \multicolumn{1}{r|}{$\cellcolor{blue!15}0.390_{0.003}$}   & { \textbf{\cellcolor{blue!35}$0.169_{0.001}$}}                    & \cellcolor{blue!35}\textbf{$0.228_{0.001}$}                          & { $\cellcolor{blue!35}0.286_{0.001}$}    & \multicolumn{1}{r|}{\cellcolor{blue!15}$0.378_{0.001}$}                          & { \cellcolor{blue!15}$0.183_{0.009}$}                       & \cellcolor{blue!15}\textbf{$0.226_{0.009}$}                 & { $0.277_{0.009}$}                    & \multicolumn{1}{l|}{$0.339_{0.008}$}                          
      & \multicolumn{1}{c}{$2.1$}
      \\
      
      \multicolumn{1}{c|}{}                                                                      & \textbf{PatchTST$_{\text{SA}}$}      & \textbf{$0.440_{0.059}$}             & $0.484_{0.057}$                              & $0.434_{0.059}$                      & \multicolumn{1}{r|}{$0.436_{0.075}$}                      & { \textbf{$0.324_{0.05}$}} & { $0.362_{0.045}$}                             & $0.410_{0.049}$                             & \multicolumn{1}{r|}{$0.462_{0.047}$}                          & \textbf{$\cellcolor{blue!35}0.175_{0.002}$} & { $\cellcolor{blue!35}0.211_{0.000}$} & $\cellcolor{blue!35}0.264_{0.002}$       & \multicolumn{1}{l|}{$0.335_{0.001}$}      
      & \multicolumn{1}{c}{$4.6$}
      \\
      
      \multicolumn{1}{c|}{}                                                                      & \textbf{AutoFormer$_{\text{SP}}$}    & $0.559_{0.05}$                      & $0.628_{0.101}$                             & $0.525_{0.037}$                      & \multicolumn{1}{r|}{$0.550_{0.143}$}                      & { $0.280_{0.006}$}                             & $0.390_{0.158}$                                   & $0.360_{0.018}$                             & \multicolumn{1}{r|}{$0.475_{0.033}$}                          & { $0.321_{0.008}$}                       & \textbf{$0.413_{0.013}$}                 & $0.508_{0.036}$                          & \multicolumn{1}{l|}{$0.467_{0.032}$}                          
      & \multicolumn{1}{c}{$8.9$}
      \\

      
      \multicolumn{1}{c|}{}                                                                      & \textbf{AutoFormer$_{\text{SA}}$}    & $0.767_{0.126}$                      & $0.526_{0.06}$                             & $0.550_{0.019}$                      & \multicolumn{1}{r|}{$0.449_{0.010}$}                      & $0.610_{0.312}$                                   & $0.850_{0.365}$                                   & $0.615_{0.151}$                             & \multicolumn{1}{r|}{$1.045_{0.262}$}                          & \textbf{$0.353_{0.013}$}                    & $0.413_{0.006}$                           & $0.474_{0.028}$                           & \multicolumn{1}{l|}{$0.504_{0.049}$}                          
      & \multicolumn{1}{c}{$10.2$}
      \\
      
      \multicolumn{1}{c|}{}                                                                      & \textbf{DLinear$_{\text{SP}}$}       & \cellcolor{blue!15}$0.296_{0.003}$                      & $0.401_{0.018}$                             & $0.445_{0.006}$                      & \multicolumn{1}{r|}{$0.607_{0.013}$}                      & $0.458_{0.169}$                                   & \cellcolor{blue!35}$0.228_{0.001}$                                   & $0.302_{0.000}$                             & \multicolumn{1}{r|}{$0.531_{0.144}$}                          & $0.205_{0.007}$                             & $0.241_{0.007}$                          & $0.282_{0.008}$                          & \multicolumn{1}{l|}{$0.373_{0.009}$}                          
      & \multicolumn{1}{c}{$6.5$}
      \\
      
      \multicolumn{1}{c|}{}                                                                      & \textbf{DLinear$_{\text{SA}}$}       & $0.454_{0.053}$   & $0.514_{0.053}$                             & $0.542_{0.064}$                      & \multicolumn{1}{r|}{$0.680_{0.084}$}                      & $0.330_{0.065}$                                   & \textbf{$0.365_{0.062}$}       & $0.427_{0.058}$                             & \multicolumn{1}{r|}{$0.538_{0.063}$}                          & $0.190_{0.001}$                             & $0.233_{0.000}$                          & $0.276_{0.000}$                          & \multicolumn{1}{l|}{\cellcolor{blue!15}$0.333_{0.001}$}                          
      & \multicolumn{1}{c}{$6.8$}
      \\
      
      \multicolumn{1}{c|}{}                                                                      & \textbf{iTransformer$_{\text{SP}}$}  & $0.313_{0.004}$                      & $0.394_{0.014}$                             & $0.436_{0.005}$                      & \multicolumn{1}{r|}{$0.429_{0.005}$}                      & \cellcolor{blue!15}\textbf{$0.178_{0.001}$}                          & \cellcolor{blue!15}$0.243_{0.0004}$                                   & \cellcolor{blue!15}$0.293_{0.001}$                             & \multicolumn{1}{r|}{$0.384_{0.008}$}                          & $0.197_{0.006}$                             & $0.260_{0.007}$                          & $0.315_{0.008}$                          & \multicolumn{1}{l|}{$0.349_{0.006}$}                          
      & \multicolumn{1}{c}{$4.9$}
      \\
      
      \multicolumn{1}{c|}{}                                                                      & \textbf{iTransformer$_{\text{SA}}$}  & $0.492_{0.058}$                       & $0.545_{0.048}$                             & $0.579_{0.049}$                      & \multicolumn{1}{r|}{$0.540_{0.094}$}                      & $0.369_{0.080}$                                   & $0.432_{0.083}$                                   & $0.482_{0.083}$                             & \multicolumn{1}{r|}{$0.541_{0.075}$}                          & $0.191_{0.002}$                             & $0.228_{0.002}$                          & \cellcolor{blue!15}$0.273_{0.002}$                          & \multicolumn{1}{l|}{$0.348_{0.003}$}                          
      & \multicolumn{1}{c}{$7.7$}
      \\ 
      
      \cline{2-15}
      
      \multicolumn{1}{c|}{\multirow{11}{*}{\rotatebox[origin=c]{90}{\textbf{Periodic Masking}}}} & \multicolumn{1}{l}{\textbf{MissTSM}} & \cellcolor{blue!35}${0.246}_{0.018}$ & \cellcolor{blue!35}${0.263}_{0.017}$        & \cellcolor{blue!35}${0.301}_{0.042}$ & \multicolumn{1}{r|}{\cellcolor{blue!35}${0.353}_{0.015}$} & ${0.227}_{0.006}$                                 & $0.249_{0.006}$                                   & \cellcolor{blue!35}{${0.282}_{0.011}$}      & \multicolumn{1}{r|}{\cellcolor{blue!35}{${0.337}_{0.036}$}}   & $0.212_{0.007}$                             & $0.256_{0.008}$                          & $0.313_{0.009}$                          & \multicolumn{1}{l|}{$0.379_{0.019}$}                          
      & \multicolumn{1}{c}{$4.1$}
      \\
      
      \multicolumn{1}{c|}{}                                                                      & \textbf{SimMTM$_{\text{SP}}$}        & \textbf{$0.372_{0.122}$}             & \textbf{$0.469_{0.198}$}                    & \textbf{$0.496_{0.198}$}             & \multicolumn{1}{r|}{$0.510_{0.200}$}                      & $0.192_{0.010}$                                   & \textbf{$0.247_{0.009}$}                          & \textbf{$0.301_{0.008}$}                    & \multicolumn{1}{r|}{$0.391_{0.008}$}                          & $0.182_{0.004}$                             & $0.248_{0.003}$                          & $0.291_{0.009}$                          & \multicolumn{1}{l|}{$0.344_{0.005}$}                           
      & \multicolumn{1}{c}{$4.7$}
      \\
      
      \multicolumn{1}{c|}{}                                                                      & \textbf{SimMTM$_{\text{SA}}$}        & \textbf{$0.591_{0.132}$}             & \textbf{$0.666_{0.152}$}                    & $0.681_{0.182}$                      & \multicolumn{1}{r|}{$0.667_{0.222}$}                      & \textbf{$0.389_{0.071}$}                          & \textbf{$0.409_{0.054}$}                          & $0.436_{0.076}$                             & \multicolumn{1}{r|}{{ $0.505_{0.055}$}} & { \cellcolor{blue!15}$0.178_{0.002}$}    & { \cellcolor{blue!15}$0.214_{0.001}$} & \cellcolor{blue!35}$0.261_{0.001}$                          & \multicolumn{1}{l|}{$0.354_{0.003}$}                          
      & \multicolumn{1}{c}{$6.0$}
      \\
      
      \multicolumn{1}{c|}{}                                                                      & \textbf{PatchTST$_{\text{SP}}$}      & \cellcolor{blue!15}$0.328_{0.047}$   & \cellcolor{blue!15}$0.389_{0.040}$          & \cellcolor{blue!15}$0.381_{0.050}$   & \multicolumn{1}{r|}{\cellcolor{blue!15}$0.426_{0.058}$}   & { \cellcolor{blue!35}{$0.174_{0.004}$}}        & \cellcolor{blue!15}$0.231_{0.003}$                                   & { \cellcolor{blue!15}$0.289_{0.004}$}    & \multicolumn{1}{r|}{\cellcolor{blue!15}$0.381_{0.004}$}                          & { $0.181_{0.004}$}                       & $0.227_{0.005}$                          & $0.267_{0.005}$                          & \multicolumn{1}{l|}{$0.346_{0.003}$}                          
      & \multicolumn{1}{c}{$2.4$}
      \\
      
      \multicolumn{1}{c|}{}                                                                      & \textbf{PatchTST$_{\text{SA}}$}      & $0.581_{0.120}$                     & $0.620_{0.132}$                            & $0.592_{0.170}$                      & \multicolumn{1}{r|}{$0.644_{0.230}$}                      & $0.423_{0.054}$                                   & { $0.457_{0.042}$}                             & $0.493_{0.037}$                             & \multicolumn{1}{r|}{$0.527_{0.027}$}                          & {\cellcolor{blue!35}${0.171}_{0.002}$}      & {\cellcolor{blue!35}${0.212}_{0.001}$}   & \cellcolor{blue!15}{${0.263}_{0.005}$}   & \multicolumn{1}{l|}{\cellcolor{blue!15}$0.334_{0.001}$} 
      & \multicolumn{1}{c}{$5.2$}
      \\
      
      \multicolumn{1}{c|}{}                                                                      & \textbf{Autoformer$_{\text{SP}}$}    & $0.482_{0.041}$                      & $0.685_{0.165}$                             & $0.621_{0.166}$                      & \multicolumn{1}{r|}{$0.546_{0.035}$}                      & { $0.329_{0.109}$}                             & $0.315_{0.010}$                                   & $0.398_{0.090}$                             & \multicolumn{1}{r|}{$0.456_{0.021}$}                          & {$0.333_{0.0176}$}                           & {$0.387_{0.035}$}                        & { $0.406_{0.025}$}                    & \multicolumn{1}{l|}{$0.453_{0.016}$}                          
      & \multicolumn{1}{c}{$7.5$}
      \\
      
      \multicolumn{1}{c|}{}                                                                      & \textbf{Autoformer$_{\text{SA}}$}    & $1.415_{0.807}$                      & $0.810_{0.269}$                             & $1.364_{0.760}$                      & \multicolumn{1}{r|}{$0.820_{0.467}$}                      & $1.303_{1.278}$                                   & $0.933_{0.444}$                                   & $1.788_{0.538}$                              & \multicolumn{1}{r|}{$0.809_{0.431}$}                          & {$0.335_{0.009}$}                           & { $0.387_{0.017}$}                    & $0.435_{0.035}$                          & \multicolumn{1}{l|}{$0.467_{0.017}$}                          
      & \multicolumn{1}{c}{$10.8$}
      \\

      \multicolumn{1}{c|}{}                                                                      & \textbf{DLinear$_{\text{SP}}$}       & $0.346_{0.069}$                      & $0.475_{0.108}$                             & $0.477_{0.044}$                      & \multicolumn{1}{r|}{$0.649_{0.068}$}                      & $0.327_{0.188}$                                   & \cellcolor{blue!35}$0.230_{0.002}$          & $0.305_{0.003}$                             & \multicolumn{1}{r|}{$0.473_{0.038}$}                          & { $0.215_{0.018}$}                       & $0.244_{0.013}$                          & $0.284_{0.008}$                          & \multicolumn{1}{l|}{$0.339_{0.007}$}                          
      & \multicolumn{1}{c}{$5.0$}
      \\
      
      \multicolumn{1}{c|}{}                                                                      & \textbf{DLinear$_{\text{SA}}$}       & $0.605_{0.109}$                      & $0.674_{0.11}$                            & $0.728_{0.138}$                      & \multicolumn{1}{r|}{$0.911_{0.158}$}                      & { {$0.447_{0.049}$}}        & ${0.475}_{0.043}$              & $0.523_{0.042}$                             & \multicolumn{1}{r|}{$0.626_{0.032}$}                          & $0.190_{0.001}$                             & $0.233_{0.000}$                          & $0.276_{0.001}$ & 
      \multicolumn{1}{l|}{\cellcolor{blue!35}${0.333}_{0.001}$}     
      & \multicolumn{1}{c}{$7.4$}
      \\
      
      \multicolumn{1}{c|}{}                                                                      & \textbf{iTransformer$_{\text{SP}}$}  & $0.358_{0.070}$                      & $0.435_{0.067}$                             & $0.488_{0.096}$                      & \multicolumn{1}{r|}{$0.497_{0.119}$}                      & \cellcolor{blue!15}\textbf{$0.180_{0.005}$}                          & $0.245_{0.006}$                                   & $0.296_{0.007}$                             & \multicolumn{1}{r|}{$0.384_{0.007}$}                          & $0.197_{0.009}$                             & { $0.233_{0.006}$}                    & \textbf{$0.288_{0.01}$}                  & \multicolumn{1}{l|}{$0.351_{0.010}$}    
      & \multicolumn{1}{c}{$4.2$}
      \\
      
      \multicolumn{1}{c|}{}                                                                      & \textbf{iTransformer$_{\text{SA}}$}  & $0.691_{0.143}$                      & $0.715_{0.140}$                             & $0.763_{0.153}$                      & \multicolumn{1}{r|}{$0.773_{0.201}$}                      & $0.512_{0.055}$                                   & $0.578_{0.052}$                                   & $0.662_{0.05}$                             & \multicolumn{1}{r|}{$0.680_{0.029}$}                          & { $0.194_{0.001}$}                       & \textbf{$0.229_{0.004}$}                 & $0.274_{0.002}$                          & \multicolumn{1}{l|}{$0.350_{0.003}$}   
      & \multicolumn{1}{c}{$8.2$}
      \\ 
      \hline
      % \cline{1-15}
    \end{tabular}}
\end{table}

% \begin{table*}[hbtp]
% \caption{Comparing forecasting performance of baseline methods under MCAR and Periodic Masking. Best performing model in dark blue and the $2^\text{nd}$-best model in light blue}
% \label{tab:main_table}
% \renewcommand{\arraystretch}{1.5}
% \scriptsize
% \resizebox{\textwidth}{!}{\begin{tabular}{llllllllllll}
% \hline
% \rowcolor[HTML]{FFFFFF} 
%                                                   &                                                  & \multicolumn{5}{c}{\cellcolor[HTML]{FFFFFF}MCAR Masking}                                                                                                                                                                                                                                          & \multicolumn{5}{c}{\cellcolor[HTML]{FFFFFF}Periodic Masking}                                                                                                                                                                                             \\ \cline{3-12} 
% \rowcolor[HTML]{FFFFFF} 
%                                                   &                                                  & \multicolumn{1}{c}{\cellcolor[HTML]{FFFFFF}}                          & \multicolumn{2}{c}{\cellcolor[HTML]{FFFFFF}SimMTM}                                                     & \multicolumn{2}{c}{\cellcolor[HTML]{FFFFFF}PatchTST}                                                             & \cellcolor[HTML]{FFFFFF}                          & \multicolumn{2}{c}{\cellcolor[HTML]{FFFFFF}SimMTM}                                                     & \multicolumn{2}{c}{\cellcolor[HTML]{FFFFFF}PatchTST}                                        \\ \cline{4-7} \cline{9-12} 
% \rowcolor[HTML]{FFFFFF} 
%                                                   &                                                  & \multicolumn{1}{c}{\multirow{-2}{*}{\cellcolor[HTML]{FFFFFF}MissTSM}} & \multicolumn{1}{c}{\cellcolor[HTML]{FFFFFF}Spline} & \multicolumn{1}{c}{\cellcolor[HTML]{FFFFFF}SAITS} & \multicolumn{1}{c}{\cellcolor[HTML]{FFFFFF}Spline} & \multicolumn{1}{c}{\cellcolor[HTML]{FFFFFF}SAITS}           & \multirow{-2}{*}{\cellcolor[HTML]{FFFFFF}MissTSM} & \multicolumn{1}{c}{\cellcolor[HTML]{FFFFFF}Spline} & \multicolumn{1}{c}{\cellcolor[HTML]{FFFFFF}SAITS} & \multicolumn{1}{c}{\cellcolor[HTML]{FFFFFF}Spline} & SAITS                                  \\ \hline
% \rowcolor[HTML]{FFFFFF} 
% \cellcolor[HTML]{FFFFFF}                          & \multicolumn{1}{l|}{\cellcolor[HTML]{FFFFFF}96}  & \cellcolor[HTML]{9698ED}0.266 ± 0.0146                                & 0.349 ± 0.0191                                     & \cellcolor[HTML]{DAE8FC}0.267 ± 0.0308            & 0.309 ± 0.0087                                     & \multicolumn{1}{l|}{\cellcolor[HTML]{FFFFFF}0.413 ± 0.0339} & \cellcolor[HTML]{9698ED}0.268 ± 0.0151            & 0.646 ± 0.1716                                     & \cellcolor[HTML]{DAE8FC}0.319 ± 0.0659            & 0.492 ± 0.0825                                     & 0.599 ± 0.2110                         \\
% \rowcolor[HTML]{FFFFFF} 
% \cellcolor[HTML]{FFFFFF}                          & \multicolumn{1}{l|}{\cellcolor[HTML]{FFFFFF}192} & \cellcolor[HTML]{9698ED}0.283 ± 0.0123                                & 0.414 ± 0.0232                                     & \cellcolor[HTML]{DAE8FC}0.360 ± 0.0465            & 0.370 ± 0.0076                                     & \multicolumn{1}{l|}{\cellcolor[HTML]{FFFFFF}0.495 ± 0.0376} & \cellcolor[HTML]{9698ED}0.295 ± 0.0298            & 0.729 ± 0.1913                                     & \cellcolor[HTML]{DAE8FC}0.408 ± 0.0865            & 0.510 ± 0.0578                                     & 0.672 ± 0.2113                         \\
% \rowcolor[HTML]{FFFFFF} 
% \cellcolor[HTML]{FFFFFF}                          & \multicolumn{1}{l|}{\cellcolor[HTML]{FFFFFF}336} & \cellcolor[HTML]{9698ED}0.287 ± 0.0142                                & 0.435 ± 0.0297                                     & 0.394 ± 0.0542                                    & \cellcolor[HTML]{DAE8FC}0.352 ± 0.0062             & \multicolumn{1}{l|}{\cellcolor[HTML]{FFFFFF}0.492 ± 0.0450} & \cellcolor[HTML]{9698ED}0.319 ± 0.0185            & 0.751 ± 0.2034                                     & \cellcolor[HTML]{DAE8FC}0.436 ± 0.1056            & 0.469 ± 0.0423                                     & 0.671 ± 0.2260                         \\
% \rowcolor[HTML]{FFFFFF} 
% \multirow{-4}{*}{\cellcolor[HTML]{FFFFFF}ETTh2}   & \multicolumn{1}{l|}{\cellcolor[HTML]{FFFFFF}720} & \cellcolor[HTML]{9698ED}0.323 ± 0.0125                                & 0.447 ± 0.0250                                     & 0.413  ± 0.0464                                   & \cellcolor[HTML]{DAE8FC}0.394 ± 0.0089             & \multicolumn{1}{l|}{\cellcolor[HTML]{FFFFFF}0.564 ± 0.0574} & \cellcolor[HTML]{9698ED}0.356 ± 0.0310            & 0.735 ± 0.1836                                     & \cellcolor[HTML]{DAE8FC}0.443 ± 0.0746            & 0.502 ± 0.0367                                     & 0.723 ± 0.2251                         \\ \hline
% \rowcolor[HTML]{FFFFFF} 
% \cellcolor[HTML]{FFFFFF}                          & \multicolumn{1}{l|}{\cellcolor[HTML]{FFFFFF}96}  & 0.222 ± 0.0058                                                        & 0.170 ± 0.0030                                     & \cellcolor[HTML]{9698ED}0.166 ± 0.0082            & 0.170 ± 0.0005                                     & \multicolumn{1}{l|}{\cellcolor[HTML]{DAE8FC}0.168 ± 0.0025} & 0.241 ± 0.0105                                    & \cellcolor[HTML]{DAE8FC}0.166 ± 0.0050             & \cellcolor[HTML]{9698ED}0.156 ± 0.0155            & 0.175 ± 0.0032                                     & 0.187 ± 0.0038                         \\
% \rowcolor[HTML]{FFFFFF} 
% \cellcolor[HTML]{FFFFFF}                          & \multicolumn{1}{l|}{\cellcolor[HTML]{FFFFFF}192} & 0.274 ± 0.0023                                                        & 0.230 ± 0.0031                                     & \cellcolor[HTML]{9698ED}0.223 ± 0.0115            & 0.228 ± 0.0005                                     & \multicolumn{1}{l|}{\cellcolor[HTML]{DAE8FC}0.225 ± 0.0035} & 0.260 ± 0.0109                                    & \cellcolor[HTML]{DAE8FC}0.224 ± 0.0050             & \cellcolor[HTML]{9698ED}0.210 ± 0.0207            & 0.233 ± 0.0021                                     & 0.244 ± 0.0050                         \\
% \rowcolor[HTML]{FFFFFF} 
% \cellcolor[HTML]{FFFFFF}                          & \multicolumn{1}{l|}{\cellcolor[HTML]{FFFFFF}336} & \cellcolor[HTML]{DAE8FC}0.279 ± 0.0018                                & 0.286 ± 0.0042                                     & \cellcolor[HTML]{9698ED}0.276 ± 0.0111            & 0.286 ± 0.0009                                     & \multicolumn{1}{l|}{\cellcolor[HTML]{FFFFFF}0.282 ± 0.0051} & \cellcolor[HTML]{DAE8FC}0.279 ± 0.0333            & 0.281 ± 0.0056                                     & \cellcolor[HTML]{9698ED}0.262 ± 0.0242            & 0.291 ± 0.0024                                     & 0.300 ± 0.0040                         \\
% \rowcolor[HTML]{FFFFFF} 
% \multirow{-4}{*}{\cellcolor[HTML]{FFFFFF}ETTm2}   & \multicolumn{1}{l|}{\cellcolor[HTML]{FFFFFF}720} & \cellcolor[HTML]{9698ED}0.316 ± 0.0144                                & 0.377 ± 0.0046                                     & \cellcolor[HTML]{DAE8FC}0.369 ± 0.0128            & 0.378 ± 0.0014                                     & \multicolumn{1}{l|}{\cellcolor[HTML]{FFFFFF}0.373 ± 0.0062} & \cellcolor[HTML]{9698ED}0.321 ± 0.0139            & 0.372 ± 0.0061                                     & 0.353 ± 0.0346                                    & 0.382 ± 0.0020                                     & 0.394 ± 0.0064                         \\ \hline
% \rowcolor[HTML]{FFFFFF} 
% \cellcolor[HTML]{FFFFFF}                          & \multicolumn{1}{l|}{\cellcolor[HTML]{FFFFFF}96}  & 0.191 ± 0.0033                                                        & 0.200 ± 0.0475                                     & 0.178 ± 0.0025                                    & \cellcolor[HTML]{DAE8FC}0.189 ± 0.0095             & \multicolumn{1}{l|}{\cellcolor[HTML]{9698ED}0.172 ± 0.0010} & 0.199 ± 0.0021                                    & 0.208 ± 0.0648                                     & \cellcolor[HTML]{DAE8FC}0.176 ± 0.0019            & 0.190 ± 0.0110                                     & \cellcolor[HTML]{9698ED}0.171 ± 0.0018 \\
% \rowcolor[HTML]{FFFFFF} 
% \cellcolor[HTML]{FFFFFF}                          & \multicolumn{1}{l|}{\cellcolor[HTML]{FFFFFF}192} & \cellcolor[HTML]{DAE8FC}0.227 ± 0.0068                                & 0.250 ± 0.0488                                     & 0.230 ± 0.0043                                    & 0.227 ± 0.0094                                     & \multicolumn{1}{l|}{\cellcolor[HTML]{9698ED}0.215 ± 0.0008} & 0.238 ± 0.0059                                    & 0.258 ± 0.0654                                     & \cellcolor[HTML]{DAE8FC}0.225 ± 0.0019            & 0.227 ± 0.0109                                     & \cellcolor[HTML]{9698ED}0.214 ± 0.0015 \\
% \rowcolor[HTML]{FFFFFF} 
% \cellcolor[HTML]{FFFFFF}                          & \multicolumn{1}{l|}{\cellcolor[HTML]{FFFFFF}336} & \cellcolor[HTML]{DAE8FC}0.282 ± 0.0048                                & 0.304 ± 0.0484                                     & 0.282 ± 0.0079                                    & 0.275 ± 0.0090                                     & \multicolumn{1}{l|}{\cellcolor[HTML]{9698ED}0.264 ± 0.0008} & 0.286 ± 0.0080                                    & 0.313 ± 0.0639                                     & 0.278 ± 0.0019                                    & \cellcolor[HTML]{DAE8FC}0.274 ± 0.0102             & \cellcolor[HTML]{9698ED}0.263 ± 0.0018 \\
% \rowcolor[HTML]{FFFFFF} 
% \multirow{-4}{*}{\cellcolor[HTML]{FFFFFF}Weather} & \multicolumn{1}{l|}{\cellcolor[HTML]{FFFFFF}720} & \cellcolor[HTML]{9698ED}0.322 ± 0.0088                                & 0.377 ± 0.0449                                     & 0.355 ± 0.0065                                    & 0.345 ± 0.0089                                     & \multicolumn{1}{l|}{\cellcolor[HTML]{FFFFFF}0.335 ± 0.0009} & \cellcolor[HTML]{DAE8FC}0.339 ± 0.0112            & 0.388 ± 0.0610                                     & 0.352 ± 0.0015                                    & 0.345 ± 0.0099                                     & \cellcolor[HTML]{9698ED}0.334 ± 0.0023 \\ \hline
% \end{tabular}}
% \end{table*}
% \vspace{-10pt}
% \begin{figure}[htbp]
%     \centering
%     \begin{minipage}[b]{\textwidth}
%         \centering
%         \begin{subfigure}[b]{0.44\textwidth}
%             \centering
%             \includegraphics[width=\textwidth]{figures/rebuttal_figures/fig4_ettm2_t_336_mcar.pdf}
%             \caption{ETTm2, T=336}
%             \label{fig:first_subfigure_4}
%             % \vspace{-20pt}
%         \end{subfigure}
%         \begin{subfigure}[b]{0.44\textwidth}
%             \centering
%             \includegraphics[width=\textwidth]{figures/rebuttal_figures/fig4_ettm2_t_720_mcar.pdf}
%             \caption{ETTm2, T=720}
%             \label{fig:second_subfigure_4}
%             % \vspace{-20pt}
%         \end{subfigure}
%         \caption{Effect of Random masking on forecasting performance under different prediction horizons}
%         \label{fig:fig4}
%     \end{minipage}
%     % \hspace{0.1\textwidth}
% \end{figure}
% \vspace{10cm}
\begin{figure}[h]
    \centering
    \begin{subfigure}{0.24\textwidth}
    \centering
        \includegraphics[width=0.95\linewidth]{figures/fig4_MCAR_ETTh2_T720.pdf}
        \caption{ETTh2, MCAR}
        \label{fig:fig4_MCAR_ETTh2_T720}
    \end{subfigure}
     \begin{subfigure}{0.24\textwidth}
     \centering
         \includegraphics[width=\linewidth]{figures/fig4_MCAR_ETTm2_T720.pdf}
        \caption{ETTm2, MCAR}
        \label{fig:fig4_MCAR_ETTm2_T720}
    \end{subfigure}
    \begin{subfigure}{0.24\textwidth}
     \centering
         \includegraphics[width=0.95\linewidth]{figures/fig4_Periodic_ETTh2_T720.pdf}
        \caption{ETTh2, Periodic}
        \label{fig:fig4_Periodic_ETTh2_T720}
    \end{subfigure}
    \begin{subfigure}{0.24\textwidth}
     \centering
         \includegraphics[width=\linewidth]{figures/fig4_Periodic_ETTm2_T720.pdf}
        \caption{ETTm2, Periodic}
        \label{fig:fig4_Periodic_ETTm2_T720}
    \end{subfigure}
    \caption{\textbf{Multiple Time-series Baselines.} Performance comparison between MCAR and Periodic masking with multiple TS Baselines imputed with SAITS. TS Baselines considered: Autoformer \cite{autoformer}, PatchTST \cite{patchtst}, iTransformer \cite{liu2023itransformer}, DLinear \cite{dlinear}, SimMTM \cite{simmtm}}
    \label{fig:fig4}
    \vspace{-4ex}
\end{figure}


\begin{figure}[t]
    \centering
    \begin{subfigure}{0.27\textwidth}
    \centering
        \includegraphics[width=\linewidth]{figures/fig5_MCAR_ETTh2_p6.pdf}
        \caption{ETTh2 60\% MCAR}
        \label{fig:fig5_MCAR_ETTh2_p6}
    \end{subfigure}
     \begin{subfigure}{0.27\textwidth}
     \centering
         \includegraphics[width=\linewidth]{figures/fig5_MCAR_ETTm2_p6.pdf}
        \caption{ETTm2 60\% MCAR}
        \label{fig:fig5_MCAR_ETTm2_p6}
    \end{subfigure}
    \begin{subfigure}{0.27\textwidth}
     \centering
         \includegraphics[width=\linewidth]{figures/fig5_MCAR_weather_p6.pdf}
        \caption{Weather 60\% MCAR}
        \label{fig:fig5_MCAR_weather_p6}
    \end{subfigure}
    \caption{Forecasting performance with the horizon length \textit{T} $\in$ {96, 192, 336, 720} and fixed lookback length S = 336. Baseline models are imputed with SAITS}
    \label{fig:fig5}
\end{figure}

% \begin{figure}[hbtp]
%     \centering
%     \begin{minipage}[b]{\textwidth}
%         \centering
%         \begin{subfigure}[b]{0.32\textwidth}
%             \centering
%             \includegraphics[width=\textwidth]{figures/rebuttal_figures/imp_baselines_ETTh2_t_192_mcar.pdf}
%             \caption{ETTh2, T=192}
%             \label{fig:first_subfigure_imp_baseline}
%             % \vspace{-20pt}
%         \end{subfigure}
%         \begin{subfigure}[b]{0.32\textwidth}
%             \centering
%             \includegraphics[width=\textwidth]{figures/rebuttal_figures/imp_baselines_ETTh2_t_720_mcar.pdf}
%             \caption{ETTh2, T=720}
%             \label{fig:second_subfigure_imp_baseline}
%             % \vspace{-20pt}
%         \end{subfigure}
%         \begin{subfigure}[b]{0.33\textwidth}
%             \centering
%             \includegraphics[width=\textwidth]{figures/rebuttal_figures/imp_baselines_weather_t_720_mcar.pdf}
%             \caption{Weather, T=720}
%             \label{fig:third_subfigure_imp_baseline}
%             % \vspace{-20pt}
%         \end{subfigure}
%         \caption{\textbf{Multiple Imputation Baselines}. Performance comparison across multiple imputation models on ETTh2 and Weather. Imputation models considered: BRITS \cite{cao2018brits}, kNN, Spline, SAITS \cite{saits}. TS Baselines: iTransformer \cite{liu2023itransformer} and PatchTST \cite{patchtst}}
%         \label{fig:imp_baselines}
%     % \begin{minipage}[b]{0.33\textwidth}
%     %     \centering
%     %     \includegraphics[width=\textwidth]{figures/rebuttal_figures/imp_baselines_weather_t_720_mcar.pdf}
%     %     \captionof{figure}{\small{Performance comparison across multiple imputation models on weather, MCAR}}
%     %     \label{fig:cls_imbalance}
%     \end{minipage}
%     % \begin{minipage}[b]{0.24\textwidth}
%     %     \centering
%     %     \includegraphics[width=\textwidth]{figures/rebuttal_figures/Weather_Error_Plot.pdf}
%     %     \captionof{figure}{\small{Imputation error vs Forecasting error across 5 trials for 4 missing fractions, 0.6, 0.7, 0.8, 0.9}}
%     %     \label{fig:cls_imbalance}
%     % \end{minipage}
    
% \end{figure}


\begin{figure}[t]
    \centering
    \begin{subfigure}{0.24\textwidth}
    \centering
        \includegraphics[width=0.85\linewidth]{figures/fig6_itrans_MCAR_ETTh2_720.pdf}
        \caption{iTransformer, ETTh2}
        \label{fig:fig6_itrans_MCAR_ETTh2_720}
    \end{subfigure}
     \begin{subfigure}{0.24\textwidth}
     \centering
         \includegraphics[width=0.95\linewidth]{figures/fig6_itrans_MCAR_weather_720.pdf}
        \caption{iTransformer, Weather}
        \label{fig:fig6_itrans_MCAR_weather_720}
    \end{subfigure}
    \begin{subfigure}{0.24\textwidth}
     \centering
         \includegraphics[width=0.84\linewidth]{figures/fig6_ptst_MCAR_ETTh2_720.pdf}
        \caption{PatchTST, ETTh2}
        \label{fig:fig6_ptst_MCAR_ETTh2_720}
    \end{subfigure}
    \begin{subfigure}{0.25\textwidth}
     \centering
         \includegraphics[width=\linewidth]{figures/fig6_ptst_MCAR_weather_720.pdf}
        \caption{PatchTST, Weather}
        \label{fig:fig6_ptst_MCAR_weather_720}
    \end{subfigure}
    \caption{\textbf{Multiple Imputation Baselines}. Performance comparison across multiple imputation models. Imputation models considered: kNN, Spline, SAITS \cite{saits}. TS Baselines: iTransformer \cite{liu2023itransformer} and PatchTST \cite{patchtst}}
    \label{fig:fig6}
\end{figure}


% \newpage
\subsection{Classification}
\label{appendix:C.3}
Full classification results (on all the datasets) are shown in Figure \ref{fig:clf_full}
\begin{figure}[hbtp]
    \centering
    \includegraphics[width=0.95\linewidth]{figures/rebuttal_figures/classification_ts_baselines.pdf}
    \caption{Classification F1 scores on three datasets - EMG, Epilepsy, Gesture. Masking fractions considered: {0.2, 0.4, 0.6, 0.8}.}
    \label{fig:clf_full}
\end{figure}

\textbf{Real-world results on Physio-Net:} 
We compare the performance of the MissTSM framework with six imputation baselines— M-RNN \cite{mrnn}, GP-VAE \cite{fortuin2020gp}, BRITS \cite{cao2018brits}, Transformer \cite{vaswani2017attention}, and SAITS \cite{saits}—on the real-world PhysioNet classification dataset \cite{silva2012predicting} that is highly sparse with 80\% missing values (see Appendix for additional details), as shown in Figure \ref{fig:physio_net}. We follow the same evaluation setup as proposed in \cite{saits}. MissTSM achieves an F1-score of 57.84\%, representing an approximately 15\% improvement over SAITS, the best-performing imputation model, which scored 42.6\%. 
% \vspace{-1ex}
\begin{figure}[hbtp]
% {0.3\textwidth}
    \centering
    \includegraphics[width=0.45\linewidth]{figures/F1_score.pdf}
    \caption{Classification Performance of MissTSM and other imputation baselines on PhysioNet Dataset \cite{silva2012predicting}.}
    \label{fig:physio_net}
    % \vspace{-4ex}
\end{figure}
This substantial performance gain on a real-world dataset with missing values highlights the advantages of MissTSM’s single-stage approach compared to traditional two-stage methods, beyond synthetic masking schemes used to simulate missing values in other datasets.


% \newpage
\subsection{Ablations on Forecasting and Classification task}
\label{appendix:C.4}
\begin{figure}[h]
    \includegraphics[width=\linewidth, scale=1]{figures/MAE_compare.pdf}
    \caption{Ablations of MissTSM with and without MFAA layer on Forecasting datasets.}
    \label{fig:forecast_ablation}
\end{figure}
\begin{figure}[!htbp]
    \centering
    \includegraphics[width=0.85\linewidth, scale=0.5]{figures/classification_model_ablation.pdf}
    \caption{Ablations of MissTSM with and without the TFI+MFAA layer on the classification tasks.}
    \label{fig:classification_model_ablation}
\end{figure}

In the ablation experiments, our goal is to quantify the effectiveness of the TFI-Embedding scheme and the MFAA Layer on MissTSM. To achieve this, we compare MissTSM with Ti-MAE, which can be viewed as an ablation of MissTSM without the TFI-Embedding and MFAA Layers. We refer to this ablation of MissTSM as MAE. For both the forecasting (see Fig. \ref{fig:forecast_ablation}) and classification (see Fig. \ref{fig:classification_model_ablation}) tasks, we compare the MissTSM framework with MAE trained on spline and SAITS imputation techniques. For forecasting on ETTh2, we observe that our proposed MissTSM framework consistently outperforms the MAE ablations without the MFAA Layer. On the other hand for the classification, we show that for all the three datasets, we are either comparable or better than the MAE ablations. This demonstrates the efficacy of the TFI-Embedding and MFAA Layer for time-series modeling with missing values.
% \newpage
\subsection{Experiment on Computational cost comparison}
\label{appendix:C.5}
We consider a case study of a classification task on the Epilepsy dataset. Dataset is 80\% masked under MCAR. Spline and SAITS are the imputation techniques and SimMTM is the time-series model used. We report the total modeling time as the sum of imputation time and the time-series model training time. 

In Table \ref{tab:computation}, we observe that, while SimMTM integrated with SAITS achieves the highest F1 score, the total imputation time for SAITS is significantly higher than that of Spline. This additional computational overhead substantially increases the overall modeling time. Moreover, SAITS has approximately 1.3 million trainable parameters, further increasing the overall model complexity of the time-series modeling task. This highlights the potential trade-off between imputation efficiency and complexity (by imputation complexity we are referring to both model and time complexity). 

In the case of our proposed method, we do not have the extra overhead of imputation complexity. Simultaneously, MissTSM also achieves competitive performance. 
\begin{table}[htbp!]
\centering
\caption{Comparison of total computational cost between MissTSM and SimMTM integrated with Spline and SAITS}
% {\centering\arraybackslash}p{3cm} >{\centering\arraybackslash}p{3cm} >{\centering\arraybackslash}p{3cm}
\renewcommand{\arraystretch}{1.5}
\scriptsize
\normalsize
\resizebox{\textwidth}{!}{
\begin{tabular}{lcccccc}
\toprule
\textbf{Time-Series Model} & \textbf{Imp. Model} & \textbf{Imp. Time (sec)} & \textbf{TS Model Train Time (sec)} & \textbf{Total Time (sec)} & \textbf{F1 Score} \\
\midrule
\textbf{SimMTM} & SAITS  & 949 ± 42.9 & 397.59 ± 2.64 & 1346.59 ± 45.54 & \underline{61.0 ± 9.20} \\
                & Spline & 8.74 ± 0.38 & 397.59 ± 2.64 & \underline{406.33 ± 3.02}  & 59.16 ± 3.67 \\
\textbf{MissTSM} & N/A & N/A & 346.8 ± 7.32 & \textbf{346.8 ± 7.32} & \textbf{64.93 ± 4.57} \\
\bottomrule
\end{tabular}}
\label{tab:computation}
\end{table}
% \newpage
\subsection{Imputation error propagation}
\label{appendix:C.6}
Figure \ref{fig:err_propagation} captures the propagation of imputation errors and forecasting errors for the weather dataset (at 720 forecasting horizon). It demonstrates that there is an overall positive correlation between the imputation error and forecasting errors, thereby demonstrating propagation of the imputation errors into the downstream time-series models.
\begin{figure}[hbtp]
    \centering
    \includegraphics[width=0.6\linewidth, scale=0.5]{figures/rebuttal_figures/Weather_Error_Plot.pdf}
    \caption{\small{Imputation error vs Forecasting error across 5 trials for 4 missing fractions, 0.6, 0.7, 0.8, 0.9}}
    \label{fig:err_propagation}
\end{figure}
% \newpage
\subsection{Analysis of impact of frequency and phase parameters}
\label{appendix:C.7}
In the following, we provide additional details regarding an ablation we conducted to understand the impact of frequency and phase parameters. Given the varying frequency and phase for each feature, we modify the intervals of both to assess their impact on the results. Dataset=ETTh2, Fraction=90\% 

\textbf{Case 1}. With the phase interval held constant, we lower the frequency range and examine two intervals: one in the high frequency region ([0.6, 0.9]) and one in the low frequency region ([0.1, 0.3]). The performance comparison between these new strategies and the original configuration is shown in Table \ref{tab:periodic1}.

\begin{table}[htbp!]
\centering
\caption{Effect of sampling from different frequency intervals. The best results are in bold and second-best are italicized}
\small
\begin{tabular}{l>{\centering\arraybackslash}p{3cm} >{\centering\arraybackslash}p{3cm} >{\centering\arraybackslash}p{3cm}}
\toprule
\textbf{Time Horizon} & \textbf{Original Periodic Masking MSE} & \textbf{High Frequency MSE} & \textbf{Low Frequency MSE} \\
\midrule
96  & \textbf{0.268 ± 0.0151} & \textit{0.281 ± 0.028} & 0.285 ± 0.023 \\
192 & \textbf{0.295 ± 0.0298} & \textit{0.301 ± 0.037} & 0.316 ± 0.049 \\
336 & 0.319 ± 0.0185 & \textit{0.308 ± 0.014} & \textbf{0.307 ± 0.011} \\
720 & 0.356 ± 0.0310 & \textbf{0.339 ± 0.043} & \textit{0.351 ± 0.058} \\
\bottomrule
\end{tabular}
\label{tab:periodic1}
\end{table}

\par
We observe that with a reduced frequency range, for both high and low frequency intervals, the performance improves as the prediction window increases.  

\textbf{Case 2}. Following a similar approach as Case 1, we keep the frequency interval constant and lower the range of phase values. We examine the following intervals: the positive half-cycle [0, $\pi$] and the negative half-cycle [$\pi$, 2$\pi$]. Table \ref{tab:half_cycle_comparison} presents the results of this ablation 

\begin{table}[H]
\centering
\caption{Effect of sampling from different phase intervals. The best results are in bold and second-best are italicized}
\small
\begin{tabular}{l>{\centering\arraybackslash}p{3cm} >{\centering\arraybackslash}p{3cm} >{\centering\arraybackslash}p{3cm}}
\toprule
\textbf{Time Horizon} & \textbf{Original Periodic Masking MSE} & \textbf{(+) Half Cycle MSE} & \textbf{(-) Half Cycle MSE} \\
\midrule
96  & \textbf{0.268 ± 0.0151} & \textit{0.287 ± 0.037} & 0.293 ± 0.04 \\
192 & \textbf{0.295 ± 0.0298} & \textit{0.309 ± 0.05}  & 0.313 ± 0.057 \\
336 & 0.319 ± 0.0185 & \textit{0.316 ± 0.022} & \textbf{0.311 ± 0.013} \\
720 & 0.356 ± 0.0310 & \textit{0.343 ± 0.035} & \textbf{0.340 ± 0.040} \\
\bottomrule
\end{tabular}
\label{tab:half_cycle_comparison}
\end{table}

We observe a similar pattern here as well, with the performance improving as the prediction window increases when we sample from either the positive or negative cycle. 

As shown in the tables above, frequency and phase values clearly impact model performance. The new strategies reduce frequency or phase-related randomness among the variates of the dataset, resulting in more consistent values. This appears to enhance the model’s ability in long-term forecasting. 
% \begin{table}[h]
%   \caption{The table provides description of all the datasets used in the experiments}
%   \label{tab:dataset detail}
%   \centering
%   \begin{threeparttable}
%   \begin{small}
%   \renewcommand{\multirowsetup}{\centering}
%   \setlength{\tabcolsep}{3.5pt}
%   \renewcommand\arraystretch{2.0}
%   \begin{tabular}{c|c|c|c|c|c|c|c}
%     \toprule
%     \scalebox{0.9}{Tasks} & \scalebox{0.9}{Datasets} & \scalebox{0.9}{Channels} & \scalebox{0.9}{Length} & \scalebox{0.9}{Samples} & \scalebox{0.9}{Classes} & \scalebox{0.9}{Information} & \scalebox{0.9}{Frequency} \\
%     \toprule
%     \multirow{5}{*}{\rotatebox{90}{\scalebox{0.9}{Forecasting}}} & \scalebox{0.9}{ETTh2} & 7 & \scalebox{0.9}{\{96,192,336,720\}} & \scalebox{0.9}{8545/2881/2881} & - & \scalebox{0.9}{Electricity} & \scalebox{0.9}{1 Hour} \\
%     & \scalebox{0.9}{ETTm2} & \scalebox{0.9}{7} & \scalebox{0.9}{\{96,192,336,720\}} & \scalebox{0.9}{34465/11521/11521} & - & \scalebox{0.9}{Electricity} & \scalebox{0.9}{15 Mins} \\
%     & \scalebox{0.9}{Weather} & \scalebox{0.9}{21} & \scalebox{0.9}{\{96,192,336,720\}} & \scalebox{0.9}{36792/5271/10540} & - & \scalebox{0.9}{Weather} & \scalebox{0.9}{10 Mins} \\
%     % & \scalebox{0.9}{Electricity} & \scalebox{0.9}{321} & \scalebox{0.9}{\{96,192,336,720\}} & \scalebox{0.9}{18317/2633/5261} & - & \scalebox{0.9}{Electricity} & \scalebox{0.9}{1 Hour} \\
%     % & \scalebox{0.9}{Traffic} & \scalebox{0.9}{862} & \scalebox{0.9}{\{96,192,336,720\}} & \scalebox{0.9}{12185/1757/3509} & - & \scalebox{0.9}{Transportation} & \scalebox{0.9}{1 Hour} \\
%     \midrule
%     \multirow{5}{*}{\rotatebox{90}{\scalebox{0.9}{Classification}}} & \scalebox{0.9}{SleepEEG} & \scalebox{0.9}{1} & \scalebox{0.9}{200} & \scalebox{0.9}{371005/-/-} & \scalebox{0.9}{5} & \scalebox{0.9}{EEG} & \scalebox{0.9}{100 Hz} \\
%     & \scalebox{0.9}{Epilepsy} & \scalebox{0.9}{1} & \scalebox{0.9}{178} & \scalebox{0.9}{60/20/11420} & \scalebox{0.9}{2} & \scalebox{0.9}{EEG} & \scalebox{0.9}{174 Hz} \\
%     & \scalebox{0.9}{FD-B} & \scalebox{0.9}{1} & \scalebox{0.9}{5120} & \scalebox{0.9}{60/21/135599} & \scalebox{0.9}{3} & \scalebox{0.9}{Faulty Detection} & \scalebox{0.9}{64K Hz} \\
%     & \scalebox{0.9}{Gesture} & \scalebox{0.9}{3} & \scalebox{0.9}{315} & \scalebox{0.9}{320/120/120} & \scalebox{0.9}{8} & \scalebox{0.9}{Hand Movement} & \scalebox{0.9}{100 Hz} \\
%     & \scalebox{0.9}{EMG} & \scalebox{0.9}{1} & \scalebox{0.9}{1500} & \scalebox{0.9}{122/41/41} & \scalebox{0.9}{3} & \scalebox{0.9}{Muscle responses} & \scalebox{0.9}{4K Hz} \\
%     \bottomrule
%   \end{tabular}
%   \end{small}
%   \end{threeparttable}
% \end{table}


% Optionally include extra information (complete proofs, additional experiments and plots) in the appendix.


\end{document}
