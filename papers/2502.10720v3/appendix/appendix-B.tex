\chapter{Hyper-parameter Tuning}
\label{apd:B}
To find the optimal choice of hyper-parameters used in the depth refinement kernel, we apply grid search to decide $\lambda_1 \sim \lambda_3$ in~\myeqref{equ:final_loss}. In our experiment, we set $\lambda_1$ to $1$ and then changing $\lambda_2$ and $\lambda_3$ between $10^{-3}$ and $10^3$.In particular, we select $7$ values for each parameter: $10^{-3}$, $10^{-2}$, $10^{-1}$, $10^0$, $10^1$, $10^2$, $10^3$. We tried out 49 different combinations in total. Next, we will present the effect of each loss term on mesh reconstruction by showing qualitative results of reconstructed mesh and optimized surface normal.

During our experiments, the coefficient of continuity loss $\lambda_2$ plays a more important role. When setting $\lambda_2$ small, gaps caused by sudden changes in depth will be created in both refined surface normal and reconstructed mesh. Those gaps will not violate other loss terms much but will have a significant negative impact on relighting. The upper row of ~\myfigref{fig:small_con_loss} shows an example of gaps on surface normals and reconstructed mesh. Similarly, having $\lambda_2$ too large will also cause some negative effects, as shown in the lower row of ~\myfigref{fig:small_con_loss}. Though it doesn't have a negative effect on the refined surface normal, the final reconstructed mesh will have some wave-like artifacts, this kind of effect is mainly caused by the dual-reference cross-bilateral filter during depth refinement.

\begin{figure}[h!]
\centering
\begin{tabular}{@{}c@{\hspace{1mm}}c@{\hspace{1mm}}c@{}}
\includegraphics[width=0.325\linewidth]{images/appendix/appendix-B/small_con_loss/1_rgb.png} &
\includegraphics[width=0.325\linewidth]{images/appendix/appendix-B/small_con_loss/1_normal.png}  &
\includegraphics[width=0.325\linewidth]{images/appendix/appendix-B/small_con_loss/1_mesh.png}\\
\includegraphics[width=0.325\linewidth]{images/appendix/appendix-B/large_con_loss/1_rgb.png} &
\includegraphics[width=0.325\linewidth]{images/appendix/appendix-B/large_con_loss/1_normal.png}  &
\includegraphics[width=0.325\linewidth]{images/appendix/appendix-B/large_con_loss/1_mesh.png}\\
\small (a) Original RGB image & \small (b) Refined surface normal & 
\small (c) Reconstructed mesh\\
\end{tabular}
\caption{\textbf{Small $\lambda_2$.} We present an example of surface normal and final mesh with different $\lambda_2$ values. The upper row shows an example of setting $\lambda_2$ too small and the lower row shows an example of setting $\lambda_2$ too large.}
\label{fig:small_con_loss}
\end{figure}

Compared to $\lambda_2$, $\lambda_3$ (the coefficient of depth loss) plays a less important role. We discover that making $\lambda_3$ too large will not affect the reconstruction result much. However, having it too small will make the refined depth deviate more from the originally predicted depth, especially for foreground objects as they are usually surrounded by uncertain regions. This will further result in wider uncertain regions and remaining unexpected faces after after mesh post-processing component. We present one visual example in the~\myfigref{fig:large_depth_loss}.

\begin{figure}
\centering
\begin{tabular}{@{}c@{\hspace{1mm}}c@{\hspace{1mm}}c@{}}
\includegraphics[width=0.325\linewidth]{images/appendix/appendix-B/large_depth_loss/1_rgb.png} &
\includegraphics[width=0.325\linewidth]{images/appendix/appendix-B/large_depth_loss/1_normal.png}  &
\includegraphics[width=0.325\linewidth]{images/appendix/appendix-B/large_depth_loss/1_mesh.png}\\
\small (a) Original RGB image & \small (b) Refined surface normal & 
\small (c) Reconstructed mesh\\
\end{tabular}
\vspace{-2mm}
\caption{\textbf{Large $\lambda_3$.} We present an example of surface normal and final mesh when setting $\lambda_3$ too large.}
\label{fig:large_depth_loss}
\end{figure}

Similarly, the effect of $\lambda_1$ (the coefficient of normal loss) will largely depend on $\lambda_2$ and $\lambda_3$ together. When setting both $\lambda_2$ and $\lambda_3$ large, the result will have a combined effect of large $\lambda_2$ and large $\lambda_3$, with wider uncertain region and wave-like visual artifacts. When setting both $\lambda_2$ and $\lambda_3$ smaller, we observed similar effects as shown in the upper row of~\myfigref{fig:small_con_loss}. In the end, we set $\lambda_1 = 1$, $\lambda_2 = 5$ and $\lambda_3 = 1$