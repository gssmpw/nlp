\chapter{Model Selection}
\label{apd:E}

In this section, we present the selection process of network $\mathbf{F_g}$ and $\mathbf{F_ir}$. For network $F_g$, we directly use the iDisc depth model pre-trained on the KITTI dataset~\cite{Geiger2012CVPR}. However, for the surface normal prediction, as for our own knowledge, all previous works have focused on indoor datasets such as the NYUv2 dataset~\cite{Silberman:ECCV12}. To overcome the domain gap between indoor and outdoor scenes, we retrained the iDisc surface normal network on the Diode dataset~\cite{diode_dataset} outdoor split. To maximize the performance, we tried several training settings, with their evaluation results shown in Table~\ref{tab:normal_result}.

\begin{table*}[h!]
\newcolumntype{Z}{S[table-format=2.3,table-auto-round]}
\centering
\vspace{-0.5em}
\setlength{\tabcolsep}{3mm}
\small
\footnotesize
\centering
\begin{tabular}{|c|c|c|c|c|c|}\hline
Setting & rmse\_angular & $a_1$($<$5deg) & $a_2$($<$11.5deg) & $a_3$($<$22.5deg) & $a_4$($<$30deg) \\\hline
iDisc NYUv2 & 60.181 & 0.037& 0.184 & 0.312 & 0.387 \\\hline
iDisc NYUv2+Diode & 77.862 & 0.044 & 0.130 & 0.212 & 0.281 \\\hline
iDisc Diode & \textbf{44.874} & \textbf{0.290} & \textbf{0.435} & \textbf{0.563} & \textbf{0.625} \\\hline
           
\end{tabular}
\caption{{\bf iDisc normal estimation retrain result}. We present three different settings of our networks and their corresponding evaluation result.}
\label{tab:normal_result}
\end{table*}