%% ----------------------------------------------------------------------------
% BIWI SA/MA thesis template
%
% Created 09/29/2006 by Andreas Ess
% Extended 13/02/2009 by Jan Lesniak - jlesniak@vision.ee.ethz.ch
%% ----------------------------------------------------------------------------

\newpage
\vspace{3cm}

\chapter*{Abstract}
% \noindent The abstract gives a concise overview of the work you have done. The reader shall be able to decide whether the work which has been done is interesting for him by reading the abstract. Provide a brief account on the following questions:

% \begin{itemize}
%  \item What is the problem you worked on? (Introduction)
%  \item How did you tackle the problem? (Materials and Methods)
%  \item What were your results and findings? (Results)
%  \item Why are your findings significant? (Conclusion)
% \end{itemize}

% \noindent The abstract should approximately cover half of a page, and does generally not contain citations.

Semantic segmentation is an important task for autonomous driving. A powerful autonomous driving system should be capable of handling images under all conditions, including nighttime. Generating accurate and diverse nighttime semantic segmentation datasets is crucial for enhancing the performance of computer vision algorithms in low-light conditions. In this thesis, we introduce a novel approach named NPSim, which enables the simulation of realistic nighttime images from real daytime counterparts with monocular inverse rendering and ray tracing. NPSim comprises two key components: mesh reconstruction and relighting. The mesh reconstruction component generates an accurate representation of the scene’s structure by combining geometric information extracted from the input RGB image and semantic information from its corresponding semantic labels. The relighting component integrates real-world nighttime light sources and material characteristics to simulate the complex interplay of light and object surfaces under low-light conditions. The scope of this thesis mainly focuses on the implementation and evaluation of the mesh reconstruction component. Through experiments, we demonstrate the effectiveness of the mesh reconstruction component in producing high-quality scene meshes and their generality across different autonomous driving datasets. We also propose a detailed experiment plan for evaluating the entire pipeline, including both quantitative metrics in training state-of-the-art supervised and unsupervised semantic segmentation approaches and human perceptual studies, aiming to indicate the capability of our approach to generate realistic nighttime images and the value of our dataset in steering future progress in the field. NPSim not only has the ability to address the scarcity of nighttime datasets for semantic segmentation, but it also has the potential to improve the robustness and performance of vision algorithms under low-lighting conditions.
% Our project is slated for completion by early November, and we plan to submit our work to CVPR 2024. We will make both our dataset and transformation pipeline available to the public upon publication.