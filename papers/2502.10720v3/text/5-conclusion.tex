%% ----------------------------------------------------------------------------
% BIWI SA/MA thesis template
%
% Created 09/29/2006 by Andreas Ess
% Extended 13/02/2009 by Jan Lesniak - jlesniak@vision.ee.ethz.ch
%% ----------------------------------------------------------------------------

\chapter{Conclusion}

% List the conclusions of your work and give evidence for these. Often, the discussion and the conclusion sections are fused. 

In this thesis, we have presented a physics-based pipeline NPSim that performs day-to-night transformation with two components: Geometric Mesh Reconstruction and Realistic Nighttime Scene Relighting. This work stands apart from all prior works as it is the first to accomplish the task of relighting outdoor scenes from day to night using only a single image. The distinctiveness of our method lies in its explicit estimation of the scene's geometry and materials, and then integrate the estimated materials into the geometry alongside the light sources. The innovative aspect is the consideration of light sources that remain inactive during the daytime but become active at night. By incorporating these elements into the relighting process, this work has the potential to achieve a remarkable advancement in generating realistic night scenes from daytime photographs, setting it apart from all previous research in this field. 

Our mesh reconstruction component reconstructs better scene mesh by preserving geometric information such as depth and surface normal. It also gets rid of the potential unrealistic reflection by removing spurious faces between foreground and background objects. Moreover, the proposed photo-realistic nighttime simulation is a general approach that can be applied to any daytime driving dataset for day-to-night simulation without the need for real nighttime data. This alleviated the need to annotate large sets of real nighttime images and made a significant contribution to constituting a bottleneck for nighttime semantic scene understanding.

Meanwhile, we are also aware of several limitations of our method. Firstly, the generation of inactive light source masks is not automated and requires some human input. This problem could be alleviated by training a neural network using existing annotations to detect inactive light sources from daytime images. Secondly, our mesh reconstruction depends heavily on the estimated geometric information, errors on estimated depth and surface normal will make the reconstructed mesh inaccurate. We noticed that our current depth prediction model is not capable of correctly predicting the surface normal of some parts of the road, especially roads that are far away or under shadow. This is probably caused by domain shift between Diode dataset~\cite{diode_dataset}, NYUv2 dataset~\cite{Silberman:ECCV12} and the ACDC dataset~\cite{SDV21ACDC}. We plan to fix it by trying out different models or retraining the current model on a dataset that is closer to the ACDC dataset. Lastly, our relighting component did not consider motion blur caused by long exposure time at night, this may cause our generated nighttime images to be inconsistent with real nighttime images and lower the performance of trained models. Our future works will focus on the simulation of motion blur for moving objects and to narrow the gap between the simulated and real nighttime images. 
