\section{Orientation Depiction in Childlike Drawings}
%\nicky {Better wording for the title? A little unclear what it means}
Because our goal is to manipulate the figure in a style-preserving manner, we provide relevant insights from the fields of child psychology and art analysis about the factors which give rise to childlike drawing style. 
We also present the results of a novel annotation study exploring how observers infer the orientation of childlike drawing figures.



\subsection{How Children Draw Figures}

By childlike drawing, we refer to the quirky, representational style of drawing that tends to appear between ages three and seven~\cite{lowenfeld1975creative}. 
Though it falls out of use shortly thereafter, this style of drawing remains available to people of all ages, including adults who `cannot draw.'
This type of drawing is characterized by `intellectual realism,' a denotation system distinct from the 'visual realism' used in photographs and most 3D graphics scenes~\cite{luquet2001children}.
Such drawings occur when the drawer leverages an alternative definition of `realistic:' rather than drawing a subject as it appears when viewed from a specific vantage, they instead aims to include details important to their internal model of the subject.
Said differently, when intellectual realism is used, be it by a child or Picasso~\cite{picasso1937}, the aim is to create a recognizable representation of the subject- not necessarily a visually accurate one.


%This style of drawing often disappears shortly after, either because it is replaced by more visually realistic drawings or because the child ceases drawing altogether.
%Nevertheless, it remains available to virtually everyone, including adults who, by their own admission, `cannot draw.' 
%\yuting{Can we claim that understanding perspective is a development milestone and a distinction of armature vs professional?} \jesse{I think we could say that using perspective realistically is drawings a technical milestone that requires a great deal of practice. Is that close to what you are suggestion?}\yuting{I want to see if we can claim our method is not only useful for children between 3-7 (very limited scope) but for all amateur drawings.}\jesse{Yes, I think that's important. I was trying to make that point with the previous line about adults who cannot draw. I could be even more explicit: ".... 'cannot draw.' Therefore, tools which build upon childlike drawings have the potential to empower not only children, but amateur drawers of all ages.}



%The philosopher Georges-Henri Luquet described such drawings as employing 'intellectual realism,' a denotation system distinct from the 'visual realism' used in photographs and most 3D graphics scenes~\cite{luquet2001children}.
%According to Luquet, such drawings occur because the child has an alternative definition of `realism:' rather than drawing a subject from a specific vantage, the child instead aims to include the features and details that are important to their internal model of the subject.
%Said differently, when intellectual realism is used, be it by a child, Picasso~\cite{picasso1937}, or a prehistoric cave artist~cite{cave-tito-bustillo}, the aim is to create a recognizable representation of the subject that does not need to be visually accurate.

Others have argued that a key to defining childlike drawing style is understanding that the fundamental primitive used is the \textit{region}, a 2D area (e.g. a dot, single line, outlined area, or scribble cluster) whose attributes are perceptually similar to those of the subject it represents~\cite{willats2006making}. Childlike drawings can therefore be seen as a collection of regions arranged so as to convey the \textit{idea} or \textit{important details} of something in a non-visually realistic manner. 

This influences figure drawings in several ways.
First, foreshortening is rare, as long regions, such as arms, are less recognizable when foreshortened.~\cite{willats1992representation,piaget1956}.
Second, different parts of the figure are drawn so their most recognizable traits are apparent; this often results in \textit{twisted perspective}: different parts of the figure drawn as though viewed from different angles (e.g. feet in Figure~\ref{fig:part_trait_example}.a). 
Third, the majority of figures are drawn in a forward-facing 'canonical view'~\cite{goodnow1977children,cox2014drawings}, but may be drawn in profile if their orientation is important for some reason~\cite{cox1993children}.
One study found that, when drawing figures in non-forward-facing views, they omitted facial features when drawing the back side. For side views, children manipulated facial features, head contouring, obscured limbs behind each other, and pointed feet to the same side.


%The focus on recognition has additional implications for how human figure orientation is represented.
%The majority of figures are drawn in the forward-facing 'canonical view'~\cite{goodnow1977children,cox2014drawings}, but may be drawn in profile if a particular action, such as running, is important to convey~\cite{cox1993children}.
%One study specifically investigated how children alter their human figure drawings based upon the subject's orientation~\cite{cox1994children}. 
%They found that, in general, facial features were omitted when figures faced away. When drawing a subject in profile, children manipulated facial features, head contouring, obscured limbs behind each other, and pointed feet to the same side.

\subsection{How Observers Infer Figure Orientations}
\label{sec:orientation}

While it is useful to understand why children draw as they do, our aim is to modify existing figures such that the 3D pose is recognizable to an observer. Therefore, it is important to understand how observers interpret figure orientation: whether the character's forward vector extends to the left, the right, or is centered (i.e. extends directly out of the drawing). We therefore conducted an annotation study to answer the following questions:
\begin{itemize}
  \item Q1 - To what degree do observers agree on the orientation of childlike figure drawings?
  \item Q2 - What body parts and part traits do observers use when inferring orientation?
  \item Q3 - How frequently is twisted perspective present in these drawings?
\end{itemize}





\begin{figure*}[ht]
\centering
\includegraphics[width=0.9\textwidth]{figures/part_traits_short.png}
\caption{Frequency with which annotators references specific body parts and specified part traits when determining figure orientation.}
\label{fig:part_count_orientation}
\end{figure*}

\begin{table}
  \centering
  \begin{tabular}{|c|c|}
    \hline
    Per Figure Segment Orientations & Count \\
    \hline
    Center Only & 108 (27\%)\\
    \hline
    Left Only & 4 (1\%)\\
    \hline
    Right Only & 5 (1\%)\\
    \hline
    Center, Right & 137 (34\%)\\
    \hline
    Center, Left & 124 (31\%)\\
    \hline
    Right, Left & 0 (0\%)\\
    \hline
    Right, Center, Left & 21 (5\%)\\
    \hline
  \end{tabular}
  \caption{Counts of how frequently different combinations of orientations were present in the same figure.}
  \label{table:twisted_perpspective_frequency}
\end{table}


%\subsubsection{Discussion} \yuting{Results/Conclusion?}
Please refer to Appendix 1 for methodology and results tables.
Annotator agreement was substantial, indicating that observers reliably come to similar conclusions when assessing figure drawing orientations.
Figure \ref{fig:part_count_orientation} shows the frequency with which annotators used different body parts to infer orientation, along with the proportion of the times that specific part traits were mentioned.
Unsurprisingly, facial features and feet provided the majority of the orientations cues. 
There is a long tail of different body parts and accessories that could be used, depending upon what the drawer chose to include on the character.
Interestingly, only a few part traits were used the vast majority of the time.
Some parts were translated around an enclosing region, such as eyes within the face.
Other parts, such as noses, often pointed towards one direction or another.
At times the contour of a region, such as the head, extended further towards one side, suggesting orientation.
When arms and legs were used to infer orientation, they were posed such as might be seen when viewed from a particular direction. For examples of each of these, see Figure~\ref{fig:part_trait_example}.

A chart showing the frequency of twisted perspective can be seen in Table~\ref{table:twisted_perpspective_frequency}.
As expected, twisted perspective was seen frequently: 71\% of figure were composed of segments with multiple orientations.
Interestingly, at least one segment was perceived as forward facing 92\% of the time, and left-facing and right-facing segments were mixed together only 5\% of the time.

Taken together, these results suggest the following for system design:
first, it's possible to manipulate existing figure drawings to reliably influence their perceived orientation.
Second, this can be done by applying a relatively small number of manipulation types to different body parts.
Third, while it is common and acceptable to mix center-facing and consistent side-facing cues, mixing left-facing and right-facing cues should be avoided.

%From the previous section, we conclude the following: 
%It is possible to modify these characters so as to reliably influence the perceived orientation of the character.
%The system should can twist perspectives, as long as left right orientation cues are not mixed together.
%The system can be effective while limiting itself to translating regions, flipping regions horizontally, and deforming the limbs such that their poses suggest they are being viewed from a particular angle.
%As the limbs are deformed they should not introduce foreshortening



%We provide the percentage of the time observers agreed.
%To account for by-change agreement, we also compute Fleiss' kappa.

%percentage of agreement between observers.
%Overall, observers agreed on orientation 86\% of the time, with a Fleiss' kappa value of 0.73, suggesting significant agreement.


%For heads/faces, percentage agreement was 77\% with Fleiss' kappa of 0.65.
%For upper arms/torso, percentage agreement was 90\% with Fleiss' kappa of 


%We calculate the percentage of agreement of o

%To what degree do observers agree on the orientation of childlike figure drawings?

%%traits which  childlike drawings often haeve. 

%specific stylisticoutcomes of 


%In the context of drawings of the human figure, this commonly results in a few different 'oddities':
%1. Foreshortening is quite rare in children's drawings, as a long region, such as an arm, appears as a short region once forshortening, and is less recognizable as something that is long.
%2. Each region is often drawn so as to have it's most recognizable features visible. This often results in different regions being drawn from different perspectives, referred to as 'twisted perspective' by the research community. In drawings of the human figure in the canonical form, feet often points to the left or right while the character otherwise faces straight ahead.




%% it remains available and indeed is often used by adults without formal drawing training. This peculiar style of drawing has been of great interest to various child-focused research communities. 






\begin{comment}
"""
It would be understandable to treat childlike drawings as 2D images of 3D geometry that can be reconstructed and used just as any other 3D object. It would also be understandable to conflate their representation, which is 2D, with the character itself being a 2D character, which should then be moved around in 3D space like any other 2D manifold.

But we argue that childlike drawings of figures are neither 2D nor 3D. They are something else entirely, which we refer to as 'view-dependent intellectually realistic' representations.

We note that these drawings view-dependent in the sense that they are created with a specific viewing angle in mind- Namely, one at or near the normal vector to the page upon which is it drawn. Such a viewing provides the fewest distortions, and people naturally view pages from such an angle. Drawing things that are meant to be viewed from other angles, which possible, requires a significant amount of skill (ambassadors) and is generally outside the abilities of childlike drawers.

Though this seem obvious, when stated, it is important to keep in mind as we consider how to manipulate the character. When asked to draw the same characters in profile, studies have shown kids do not draw thin slits, as would appear if the paper were rotated 90 degrees in 3D space. Rather, they draw similar character which slightly modified feature layouts. The character's representation has changed significantly based upon the orientation of the character and the assumed viewing direction.

We therefore create a system which is view-dependent, violating assumptions of tradition 3D geometry, but in doing so attempt to preserve the inherent style of these drawings.

Secondly, the drawings are not visually realistic, so creating a 3D model that can be manipulated in a visually realistic manner also violates the style.
We refer to these drawings as 'intellectually realistic' to differentiate from 'visually realistic', and we borrow the term from Luquet, the child art psychologoist <definition of intellectually realistic>.





"""

- Intellectual Realism vs. visual realism overview

- Studies from children's literature about how they draw objects

- Studies from children's literature about how they draw side and back objects.

- We are unaware of any such studies which focus on amateur drawers of all ages, so we perform annotation work with Amateur Drawings Dataset.

- Task Setup
    - How we got the images
    - instructions for annotators
        - Why we did per-part (twisted perspective)
    - results
        - Most frequently mentioned body parts and representational elements
        - Degree of agreement among raters
        - Proportion of drawings that had orientation, that had twisted perspective
        - Proportion of drawings that had conflicting orientations for different parts (part was oriented left, part oriented right). This could be a way to justify the need to avoid conflicting orientations.

- Based upon this, we selected the following bodypart-representation combos to add into our system.


\section{Annotation Study}
- H1- people will conclude similar things about orietnation (user agreement)

- H2 - A relatively small number of features/representations will be relevant to most character orientations (freecoding)

- H3 Twisted perspective will be present in a significant portion of the drawings.
(Show that people dont't really mix left and right orientations together though)


\end{comment}