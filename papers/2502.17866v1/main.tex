%%
%% This is file `sample-acmsmall.tex',
%% generated with the docstrip utility.
%%
%% The original source files were:
%%
%% samples.dtx  (with options: `acmsmall')
%% 
%% IMPORTANT NOTICE:
%% 
%% For the copyright see the source file.
%% 
%% Any modified versions of this file must be renamed
%% with new filenames distinct from sample-acmsmall.tex.
%% 
%% For distribution of the original source see the terms
%% for copying and modification in the file samples.dtx.
%% 
%% This generated file may be distributed as long as the
%% original source files, as listed above, are part of the
%% same distribution. (The sources need not necessarily be
%% in the same archive or directory.)
%%
%% Commands for TeXCount
%TC:macro \cite [option:text,text]
%TC:macro \citep [option:text,text]
%TC:macro \citet [option:text,text]
%TC:envir table 0 1
%TC:envir table* 0 1
%TC:envir tabular [ignore] word
%TC:envir displaymath 0 word
%TC:envir math 0 word
%TC:envir comment 0 0
%%
%%
%% The first command in your LaTeX source must be the \documentclass command.
%% \documentclass[acmtog,anonymous,review]{acmart}
\documentclass[acmtog]{acmart}

\usepackage{comment}
\usepackage{algorithm}
\usepackage{algpseudocode}
\usepackage{amsmath}

%% \documentclass[acmsmall]{acmart}
%% NOTE that a single column version is required for 
%% submission and peer review. This can be done by changing
%% the \doucmentclass[...]{acmart} in this template to 
%% \documentclass[manuscript,screen]{acmart}
%% 
%% To ensure 100% compatibility, please check the white list of
%% approved LaTeX packages to be used with the Master Article Template at
%% https://www.acm.org/publications/taps/whitelist-of-latex-packages 
%% before creating your document. The white list page provides 
%% information on how to submit additional LaTeX packages for 
%% review and adoption.
%% Fonts used in the template cannot be substituted; margin 
%% adjustments are not allowed.
%%
%% \BibTeX command to typeset BibTeX logo in the docs
%\AtBeginDocument{%
%  \providecommand\BibTeX{{%
%    \normalfont B\kern-0.5em{\scshape i\kern-0.25em b}\kern-0.8em\TeX}}}

%% Rights management information.  This information is sent to you
%% when you complete the rights form.  These commands have SAMPLE
%% values in them; it is your responsibility as an author to replace
%% the commands and values with those provided to you when you
%% complete the rights form.
%\setcopyright{acmcopyright}
%\copyrightyear{2018}
%\acmYear{2018}
%\acmDOI{XXXXXXX.XXXXXXX}


%%
%% These commands are for a JOURNAL article.
\acmJournal{TOG}
%\acmVolume{37}
%\acmNumber{4}
%\acmArticle{111}
%\acmMonth{8}

%%
%% Submission ID.
%% Use this when submitting an article to a sponsored event. You'll
%% receive a unique submission ID from the organizers
%% of the event, and this ID should be used as the parameter to this command.
\acmSubmissionID{673}

%%
%% For managing citations, it is recommended to use bibliography
%% files in BibTeX format.
%%
%% You can then either use BibTeX with the ACM-Reference-Format style,
%% or BibLaTeX with the acmnumeric or acmauthoryear sytles, that include
%% support for advanced citation of software artefact from the
%% biblatex-software package, also separately available on CTAN.
%%
%% Look at the sample-*-biblatex.tex files for templates showcasing
%% the biblatex styles.
%%

%%
%% The majority of ACM publications use numbered citations and
%% references.  The command \citestyle{authoryear} switches to the
%% "author year" style.
%%
%% If you are preparing content for an event
%% sponsored by ACM SIGGRAPH, you must use the "author year" style of
%% citations and references.
%% Uncommenting
%% the next command will enable that style.
\citestyle{acmauthoryear}

%%
%% end of the preamble, start of the body of the document source.
\begin{document}

%%
%% The "title" command has an optional parameter,
%% allowing the author to define a "short title" to be used in page headers.
\title{Animating Childlike Drawings with 2.5D Character Rigs}

%%
%% The "author" command and its associated commands are used to define
%% the authors and their affiliations.
%% Of note is the shared affiliation of the first two authors, and the
%% "authornote" and "authornotemark" commands
%% used to denote shared contribution to the research.
\author{Harrison Jesse Smith}
\authornote{Corresponding author.}
\affiliation{%
  \institution{Meta}
  \city{Sausalito}
  \state{CA}
  \country{USA}
}
\email{hjessmith@gmail.com}

\author{Nicky He}
\affiliation{%
  \institution{Meta}
  \city{Sausalito}
  \state{CA}
  \country{USA}
}
\email{nicky.hsj@gmail.com}

\author{Yuting Ye}
\affiliation{%
  \institution{Meta}
  \city{Redmond}
  \state{WA}
  \country{USA}
}
\email{yuting.ye@gmail.com}



%%
%% By default, the full list of authors will be used in the page
%% headers. Often, this list is too long, and will overlap
%% other information printed in the page headers. This command allows
%% the author to define a more concise list
%% of authors' names for this purpose.
\renewcommand{\shortauthors}{Smith et al.}

\newcommand{\yuting}[1]{{\color{magenta} [Yuting: {#1}]}}
\newcommand{\jesse}[1]{{\color{blue} [Jesse: {#1}]}}
\newcommand{\nicky}[1]{{\color{green} [Nicky: {#1}]}}
\newcommand{\strike}[1]{{\color{red} [{#1}]}}

%%
%% The abstract is a short summary of the work to be presented in the
%% article.
\begin{abstract}

Drawing is a fun and intuitive way to create a character, accessible even to small children. However, animating 2D figure drawings is a much more challenging task, requiring specialized tools and skills. Bringing 2D figures to 3D so they can be animated and consumed in immersive media poses an even greater challenge. Moreover, it is desirable to preserve the unique style and identity of the figure when it is being animated and viewed from different perspectives. In this work, we present an approachable and easy-to-create 2.5D character model and retargeting technique that can apply complex 3D skeletal motion, including rotation within the transverse plane, onto a single childlike figure drawing in a style-preserving manner in realtime. Because our solution is view-dependent, the resulting character is well-suited for animation in both 2D and 3D contexts. We also present a novel annotation study motivating our system design decisions and a pair of user studies validating the usefulness and appeal of our solution. We in addition showcase the generality of our system in a range of 2D and 3D applications.

%\nicky{Bringing these characters to 3D applications, like 3D games and mixed reality, introduces new challenges to represent and animated them in 3D space as they can be viewed from all perspectives. } 
\end{abstract}

%%
%% This command processes the author and affiliation and title
%% information and builds the first part of the formatted document.

\begin{teaserfigure}
  \centering
  \includegraphics[width=0.95\textwidth]{figures/banner.png}
  \caption{We present an animation system that turns a single childlike figure drawing into a 2.5D character model. It can be animated using any 3D skeletal motion and viewed from different perspectives. Our real-time and view-dependent motion retargeting algorithm makes it suitable for mixed-reality applications.}
  \Description{Banner Caption}
  \label{fig:teaser}
\end{teaserfigure}

\maketitle

% \input{default_template_copy}

% 
% 
The widespread integration of communication networks and smart devices in modern control systems has increased the vulnerability of industrial systems to online cyber-attacks, e.g., Industroyer, Blackenergy, etc \citep{osti_1505628}.
% Modern control systems have seen a large push to include communication networks and smart devices to increase performance, made possible by improvements in communication device cost and energy consumption. This trend has been coupled with the usage of open-standard communication protocols among industrial control systems, making them vulnerable to online cyber-attacks such as Industroyer, Blackenergy, etc \citep{osti_1505628}. 
To counter this, methods have been developed to improve security by achieving attack detection, mitigation, and monitoring, among others \citep{sandberg2022secure}. This paper focuses on active attack diagnosis to mitigate stealthy attacks. 
%
%\subsection{Literature review}

Active diagnosis techniques rely on the inclusion of additional moduli to control systems
% inclusion within the control system of additional moduli 
to alter the behavior of the system compared to information known by the attacker. 
For instance, the concept of additive watermarking was introduced in \cite{mo2015physical}, where noise signals of known mean and variance are added at the plant and compensated for it at the controller. 
This compensation, however, is not exact, causing some performance degradation. Thus, trade-offs between performance and detectability  are necessary \citep{zhu2023detection}.
% A later work \citep{zhu2023detection} designs the watermark signal by trading performance for detection. Thus, although additive watermarking serves as a good detection scheme, they endure performance losses even in the nominal case. 

In encrypted control \citep{darup2021encrypted}, the sensor data is encrypted, sent to the controller, and then operated on directly. Encrypted input signals are sent back to the plant for decryption. Although encryption is widespread in IT security, in control systems it presents some concerns, such as the introduction of time delays \citep{stabile2024verifiable}, while it may present inherent weaknesses \citep{alisic2023model}.
% they are not preferred as they introduce time delays \citep{stabile2024verifiable} which can cause instability, and some encryption schemes can be very weak  \citep{alisic2023model}. 

In moving target defense \citep{griffioen2020moving}, the plant is augmented with fictitious dynamics, known to the controller. The plant output is transmitted to the controller along with the fictitious states over a network under attack. 
The additional measurements then aide in the detection of attacks. 
This comes at the cost of higher communication bandwidth needs, which increases rapidly with the dimension of the augmented systems.
% Since the dynamics of the fictitious dynamics are exactly known to the controller, the attack is detected easily. However, when the scale of the system increases, the communication bandwidth used by moving the target defense approach increases rapidly. 

Other recently proposed works include two-way coding \citep{fang2019two}, a weak encryuption technique, and dynamic masking \citep{abdalmoaty2023privacy}, which enhances privacy as well as security, have been shown to be effective against zero-dynamics attacks.
% Two-way coding \citep{fang2019two} and dynamic masking \citep{abdalmoaty2023privacy} are other recently proposed approaches. Two-way coding is another form of weak encryption technique whilst dynamic masking proposes an architecture that enhances both privacy and security. These schemes are shown to be effective against zero dynamics attacks but remain to be studied for other classes of attacks. 
% Recent extensions include \citep{mukherjee2021secure,ramos2024privacy}.
% Some other works which are related are \citep{mukherjee2021secure}, an extension of \cite{fang2019two}. The work \citep{ramos2024privacy} is an extension of moving target defense for multi-agent systems. 
Furthermore, filtering techniques for attack detection are proposed by \cite{murguia2020security,hashemi2022codesign,escudero2023safety}, while not focusing on stealthy attacks.
% The works \citep{murguia2020security,hashemi2022codesign,escudero2023safety} develop filtering techniques to guarantee safety, without being focused on stealthy covert attacks.

Multiplicative watermarking (mWM) has been proposed by the authors as a diagnosis technique \citep{ferrari2020switching}. mWM consists of a pair of filters on each communication channel between the plant and its controller; the scheme is affine to weak encryption, whereby ``encoding'' and ``decoding'' are done by changing signals' dynamic characteristics through inverse pairs of filters. This enables original signals to be recovered exactly, and thus does not lead to performance degradation.
% A multiplicative watermark is an affine to a weak encryption technique, through which the signal is ``encoded'' by a filter, changing its dynamic behavior. The use of inverse pairs means that the original signal can be recovered, through ``decoding'' via an inverse filter. As such, differently to techniques based on additive watermarking, no performance is lost due to the injection of noise, and there are no bandwidth limitations.

%\subsection{Contributions}
One of the critical features of multiplicative watermarking is that to detect stealthy attacks, the mWM filter parameters must be switched over time. In this paper, an algorithm to optimally design the mWM parameters after a switching event is presented, enhancing detection performance, without changing the switching time.
% This is done without changing the switching time, which is taken as given.

\textcolor{black}{
To formalize the filter design problem, we suppose the defender is interested in optimal performance against adversaries injecting covert attacks with matched system parameters \citep{smith2015covert}, including the mWM parameters prior to the switch. This scenario represents a worst case where malicious agents can take full control of the system while remaining undetected.
Thus, the attack strategy is explicitly included within the formulation of the closed-loop system, and the mWM filters are chosen by solving an optimization problem minimizing the attack-energy-constrained output-to-output gain (AEC-OOG) \citep{anand2023risk}, a variation of the output-to-output gain proposed in  \cite{teixeira2015strategic}.
}
The main contributions of this paper are:
% We consider an adversary injecting a covert attack with matched system parameters \citep{smith2015covert}, i.e., an attacker with full knowledge of the control system parameters, including those of the mWM filters before the switch. This scenario is taken as a worst case, as it has been shown that this class of attacks can be made stealthy. To quantitatively define a cost, the output-to-output gain (OOG) \citep{teixeira2015strategic} is leveraged,
% a metric introduced to evaluate the impact of an additive attack in a control system. %Specifically, OOG evaluates the worst-case performance loss that an attacker injecting an undetectable attack can obtain. 
% Here, the maximum performance loss caused by a stealthy adversary with limited energy is taken, the attack-energy-constrained OOG (AEC-OOG) \citep{anand2023risk}. The main contributions of this paper are:
\begin{enumerate}
%[label=\alph*.]
\item The problem of optimally designing the switching mWM filters is formulated as an optimization problem, with the AEC-OOG is taken as the objective;%where the AEC-OOG is taken as the impact metric; 
\item The worst-case scenario of a covert attack with exact knowledge of plant and mWM filter parameters is embedded within the design problem;
% The optimization problem is defined to incorporate the worst-case scenario of a covert attack with exact knowledge of plant and mWM filter parameters;
\item The feasibility of the optimization problem is shown to be dependent only on stability conditions; 
\item A solution scheme is proposed to promote randomization of the mWM filter parameters such that an eavesdropping adversary cannot remain stealthy.
\end{enumerate} 

This builds on the results of \cite{ferrari2020switching}, where the focus was on the design of the switching protocols, rather than the parameters themselves.
Compared to previous work \citep{gallo2021design}, this paper introduces an optimization problem which is always feasible (thanks to the use of AEC-OOG in the objective), while also considering a more sophisticated class of covert attacks, where the presence of watermark is known to the adversary. 
Moreover, this paper poses a different objective than \citep{zhang2023hybrid}; indeed, while \citep{zhang2023hybrid} provided a design strategy to ensure certain privacy properties, in this paper we address the problem of optimal parameter design following a switching event.


%\subsection{Organization}
The rest of the paper is organized as follows. 
After formulating the problem in Section~\ref{sec:PF}, we propose our design algorithm in Section~\ref{sec:main}, and analyze its properties. It is then evaluated through a numerical example in Section~\ref{sec:NE}, and concluding remarks are given Section~\ref{sec:Con}.
% We provide the problem background in Section~\ref{sec:PF}. We formulate the design problem in Section~\ref{sec:main}, together with an analysis of its properties. The proposed algorithm is evaluated through a numerical example in Section \ref{sec:NE}. Concluding remarks are offered in Section \ref{sec:Con}.


\section{Related Work}

\subsection{View-Dependent Control}
View-dependent representations have a long history in computer graphics.
In his pioneering work, Rademacher proposed interpolating between \textit{key viewpoints} and associated \textit{key deformations} to manipulate 3D models~\cite{rademacher1999view}.
Other researchers have extended the idea to create 3D animation systems~\cite{10.1111:j.1467-8659.2004.00772.x}, streamline the modeling process~\cite{DBLP:journals/corr/abs-2103-15472}, and integrate physical simulation\cite{koyama2013view}.
Of particular note, Rivers et al.~\cite{rivers25Dcartoonmodels} introduced \textit{2.5D Cartoon Models}, a combination of planar meshes transformed, based upon view angle, so as to appears three dimensional.
Our work draws upon these works but is, to our knowledge, the first work to attempt to use view-dependent techniques to retarget 3D motion onto 2D characters.   

\subsection{Animation from 2D Images}

% output is still 2D
Many researchers have proposed different methods for creating animations from 2D images. Hornung et al.~\cite{Hornung2007anim2Dpicmotion} presented a method to deform a character from a photograph given user-provided joint annotations.
\textit{Toonsynth}~\cite{Dvoroznak18-SIG} and \textit{Neural Puppet}~\cite{poursaeed2020neural} both present methods to create new images of hand-drawn characters from examples.
% output is 3D model
Other researchers have proposed methods of obtaining 3D geometry from 2D sketches~\cite{igarashi2006teddy, Dvoroznak20-SA} and images~\cite{ArtiSketch,weng2019photo}.
% focus on sketches specifically
A number of works have specifically focused on childlike drawings.
Lingens et al.~\cite{lingens2020towards} proposed an evolutionary algorithm for animating children's drawings. 
\textit{MagicToon}~\cite{feng2017magictoon} creates a 3D model from childlike drawings for AR applications.
Similar to our work, Smith et al.~\cite{SmithHodgins} proposed a method for animating childlike drawings using 3D skeletal motion. 
However, the resulting animations are only suitable for use in 2D applications and the type of motions it supports are limited.

Unlike these previous works, our solution can be used in 3D contexts but does not create a 3D model. We instead relying upon a view-dependent formulation of the animated character.

\begin{comment}
- Sketch / Cartoon Animation

- MagicToon: A 2D-to-3D creative cartoon modeling system with mobile AR
    Creates a 3D model from a 2D drawing. Ours keeps it 2D but applies 3D motion onto it.
- Towards automatic drawing animation using physics-based evolution
    - purely 2D output, no ability to specify motion
- Smith et al.
    - output not well suited for 3D, not view dependent.
- monster mash
- hornung
- photo wakeup

    
- View-Dependent Works
A System for View-Dependent Animation
- Focus on traditionally trained animators, attempts to go from 2D pose inputs to 3D characters. We focus on the opposite and on amateurs not trained animators. In addition, we provide an integrated retargeting technique for accessible animation of the characters specifically suited for childlike drawings.
- 
    - Similar but it requires less input from users since we use insights from childlike drawings to make reasonable initial guesses. In addition, our model was build to specifically support manipulation based not just upon view angle but also input motion with minimal user input.
- Rademacher
- Rivers
- Rod Springs

- Interactive Viewpoint Exploration for Constructing
View-Dependent Models
    - method for finding viewpoints focusing on creating keyview-transform pairs for a 3D model.


\end{comment}

\section{Orientation Depiction in Childlike Drawings}
%\nicky {Better wording for the title? A little unclear what it means}
Because our goal is to manipulate the figure in a style-preserving manner, we provide relevant insights from the fields of child psychology and art analysis about the factors which give rise to childlike drawing style. 
We also present the results of a novel annotation study exploring how observers infer the orientation of childlike drawing figures.



\subsection{How Children Draw Figures}

By childlike drawing, we refer to the quirky, representational style of drawing that tends to appear between ages three and seven~\cite{lowenfeld1975creative}. 
Though it falls out of use shortly thereafter, this style of drawing remains available to people of all ages, including adults who `cannot draw.'
This type of drawing is characterized by `intellectual realism,' a denotation system distinct from the 'visual realism' used in photographs and most 3D graphics scenes~\cite{luquet2001children}.
Such drawings occur when the drawer leverages an alternative definition of `realistic:' rather than drawing a subject as it appears when viewed from a specific vantage, they instead aims to include details important to their internal model of the subject.
Said differently, when intellectual realism is used, be it by a child or Picasso~\cite{picasso1937}, the aim is to create a recognizable representation of the subject- not necessarily a visually accurate one.


%This style of drawing often disappears shortly after, either because it is replaced by more visually realistic drawings or because the child ceases drawing altogether.
%Nevertheless, it remains available to virtually everyone, including adults who, by their own admission, `cannot draw.' 
%\yuting{Can we claim that understanding perspective is a development milestone and a distinction of armature vs professional?} \jesse{I think we could say that using perspective realistically is drawings a technical milestone that requires a great deal of practice. Is that close to what you are suggestion?}\yuting{I want to see if we can claim our method is not only useful for children between 3-7 (very limited scope) but for all amateur drawings.}\jesse{Yes, I think that's important. I was trying to make that point with the previous line about adults who cannot draw. I could be even more explicit: ".... 'cannot draw.' Therefore, tools which build upon childlike drawings have the potential to empower not only children, but amateur drawers of all ages.}



%The philosopher Georges-Henri Luquet described such drawings as employing 'intellectual realism,' a denotation system distinct from the 'visual realism' used in photographs and most 3D graphics scenes~\cite{luquet2001children}.
%According to Luquet, such drawings occur because the child has an alternative definition of `realism:' rather than drawing a subject from a specific vantage, the child instead aims to include the features and details that are important to their internal model of the subject.
%Said differently, when intellectual realism is used, be it by a child, Picasso~\cite{picasso1937}, or a prehistoric cave artist~cite{cave-tito-bustillo}, the aim is to create a recognizable representation of the subject that does not need to be visually accurate.

Others have argued that a key to defining childlike drawing style is understanding that the fundamental primitive used is the \textit{region}, a 2D area (e.g. a dot, single line, outlined area, or scribble cluster) whose attributes are perceptually similar to those of the subject it represents~\cite{willats2006making}. Childlike drawings can therefore be seen as a collection of regions arranged so as to convey the \textit{idea} or \textit{important details} of something in a non-visually realistic manner. 

This influences figure drawings in several ways.
First, foreshortening is rare, as long regions, such as arms, are less recognizable when foreshortened.~\cite{willats1992representation,piaget1956}.
Second, different parts of the figure are drawn so their most recognizable traits are apparent; this often results in \textit{twisted perspective}: different parts of the figure drawn as though viewed from different angles (e.g. feet in Figure~\ref{fig:part_trait_example}.a). 
Third, the majority of figures are drawn in a forward-facing 'canonical view'~\cite{goodnow1977children,cox2014drawings}, but may be drawn in profile if their orientation is important for some reason~\cite{cox1993children}.
One study found that, when drawing figures in non-forward-facing views, they omitted facial features when drawing the back side. For side views, children manipulated facial features, head contouring, obscured limbs behind each other, and pointed feet to the same side.


%The focus on recognition has additional implications for how human figure orientation is represented.
%The majority of figures are drawn in the forward-facing 'canonical view'~\cite{goodnow1977children,cox2014drawings}, but may be drawn in profile if a particular action, such as running, is important to convey~\cite{cox1993children}.
%One study specifically investigated how children alter their human figure drawings based upon the subject's orientation~\cite{cox1994children}. 
%They found that, in general, facial features were omitted when figures faced away. When drawing a subject in profile, children manipulated facial features, head contouring, obscured limbs behind each other, and pointed feet to the same side.

\subsection{How Observers Infer Figure Orientations}
\label{sec:orientation}

While it is useful to understand why children draw as they do, our aim is to modify existing figures such that the 3D pose is recognizable to an observer. Therefore, it is important to understand how observers interpret figure orientation: whether the character's forward vector extends to the left, the right, or is centered (i.e. extends directly out of the drawing). We therefore conducted an annotation study to answer the following questions:
\begin{itemize}
  \item Q1 - To what degree do observers agree on the orientation of childlike figure drawings?
  \item Q2 - What body parts and part traits do observers use when inferring orientation?
  \item Q3 - How frequently is twisted perspective present in these drawings?
\end{itemize}





\begin{figure*}[ht]
\centering
\includegraphics[width=0.9\textwidth]{figures/part_traits_short.png}
\caption{Frequency with which annotators references specific body parts and specified part traits when determining figure orientation.}
\label{fig:part_count_orientation}
\end{figure*}

\begin{table}
  \centering
  \begin{tabular}{|c|c|}
    \hline
    Per Figure Segment Orientations & Count \\
    \hline
    Center Only & 108 (27\%)\\
    \hline
    Left Only & 4 (1\%)\\
    \hline
    Right Only & 5 (1\%)\\
    \hline
    Center, Right & 137 (34\%)\\
    \hline
    Center, Left & 124 (31\%)\\
    \hline
    Right, Left & 0 (0\%)\\
    \hline
    Right, Center, Left & 21 (5\%)\\
    \hline
  \end{tabular}
  \caption{Counts of how frequently different combinations of orientations were present in the same figure.}
  \label{table:twisted_perpspective_frequency}
\end{table}


%\subsubsection{Discussion} \yuting{Results/Conclusion?}
Please refer to Appendix 1 for methodology and results tables.
Annotator agreement was substantial, indicating that observers reliably come to similar conclusions when assessing figure drawing orientations.
Figure \ref{fig:part_count_orientation} shows the frequency with which annotators used different body parts to infer orientation, along with the proportion of the times that specific part traits were mentioned.
Unsurprisingly, facial features and feet provided the majority of the orientations cues. 
There is a long tail of different body parts and accessories that could be used, depending upon what the drawer chose to include on the character.
Interestingly, only a few part traits were used the vast majority of the time.
Some parts were translated around an enclosing region, such as eyes within the face.
Other parts, such as noses, often pointed towards one direction or another.
At times the contour of a region, such as the head, extended further towards one side, suggesting orientation.
When arms and legs were used to infer orientation, they were posed such as might be seen when viewed from a particular direction. For examples of each of these, see Figure~\ref{fig:part_trait_example}.

A chart showing the frequency of twisted perspective can be seen in Table~\ref{table:twisted_perpspective_frequency}.
As expected, twisted perspective was seen frequently: 71\% of figure were composed of segments with multiple orientations.
Interestingly, at least one segment was perceived as forward facing 92\% of the time, and left-facing and right-facing segments were mixed together only 5\% of the time.

Taken together, these results suggest the following for system design:
first, it's possible to manipulate existing figure drawings to reliably influence their perceived orientation.
Second, this can be done by applying a relatively small number of manipulation types to different body parts.
Third, while it is common and acceptable to mix center-facing and consistent side-facing cues, mixing left-facing and right-facing cues should be avoided.

%From the previous section, we conclude the following: 
%It is possible to modify these characters so as to reliably influence the perceived orientation of the character.
%The system should can twist perspectives, as long as left right orientation cues are not mixed together.
%The system can be effective while limiting itself to translating regions, flipping regions horizontally, and deforming the limbs such that their poses suggest they are being viewed from a particular angle.
%As the limbs are deformed they should not introduce foreshortening



%We provide the percentage of the time observers agreed.
%To account for by-change agreement, we also compute Fleiss' kappa.

%percentage of agreement between observers.
%Overall, observers agreed on orientation 86\% of the time, with a Fleiss' kappa value of 0.73, suggesting significant agreement.


%For heads/faces, percentage agreement was 77\% with Fleiss' kappa of 0.65.
%For upper arms/torso, percentage agreement was 90\% with Fleiss' kappa of 


%We calculate the percentage of agreement of o

%To what degree do observers agree on the orientation of childlike figure drawings?

%%traits which  childlike drawings often haeve. 

%specific stylisticoutcomes of 


%In the context of drawings of the human figure, this commonly results in a few different 'oddities':
%1. Foreshortening is quite rare in children's drawings, as a long region, such as an arm, appears as a short region once forshortening, and is less recognizable as something that is long.
%2. Each region is often drawn so as to have it's most recognizable features visible. This often results in different regions being drawn from different perspectives, referred to as 'twisted perspective' by the research community. In drawings of the human figure in the canonical form, feet often points to the left or right while the character otherwise faces straight ahead.




%% it remains available and indeed is often used by adults without formal drawing training. This peculiar style of drawing has been of great interest to various child-focused research communities. 






\begin{comment}
"""
It would be understandable to treat childlike drawings as 2D images of 3D geometry that can be reconstructed and used just as any other 3D object. It would also be understandable to conflate their representation, which is 2D, with the character itself being a 2D character, which should then be moved around in 3D space like any other 2D manifold.

But we argue that childlike drawings of figures are neither 2D nor 3D. They are something else entirely, which we refer to as 'view-dependent intellectually realistic' representations.

We note that these drawings view-dependent in the sense that they are created with a specific viewing angle in mind- Namely, one at or near the normal vector to the page upon which is it drawn. Such a viewing provides the fewest distortions, and people naturally view pages from such an angle. Drawing things that are meant to be viewed from other angles, which possible, requires a significant amount of skill (ambassadors) and is generally outside the abilities of childlike drawers.

Though this seem obvious, when stated, it is important to keep in mind as we consider how to manipulate the character. When asked to draw the same characters in profile, studies have shown kids do not draw thin slits, as would appear if the paper were rotated 90 degrees in 3D space. Rather, they draw similar character which slightly modified feature layouts. The character's representation has changed significantly based upon the orientation of the character and the assumed viewing direction.

We therefore create a system which is view-dependent, violating assumptions of tradition 3D geometry, but in doing so attempt to preserve the inherent style of these drawings.

Secondly, the drawings are not visually realistic, so creating a 3D model that can be manipulated in a visually realistic manner also violates the style.
We refer to these drawings as 'intellectually realistic' to differentiate from 'visually realistic', and we borrow the term from Luquet, the child art psychologoist <definition of intellectually realistic>.





"""

- Intellectual Realism vs. visual realism overview

- Studies from children's literature about how they draw objects

- Studies from children's literature about how they draw side and back objects.

- We are unaware of any such studies which focus on amateur drawers of all ages, so we perform annotation work with Amateur Drawings Dataset.

- Task Setup
    - How we got the images
    - instructions for annotators
        - Why we did per-part (twisted perspective)
    - results
        - Most frequently mentioned body parts and representational elements
        - Degree of agreement among raters
        - Proportion of drawings that had orientation, that had twisted perspective
        - Proportion of drawings that had conflicting orientations for different parts (part was oriented left, part oriented right). This could be a way to justify the need to avoid conflicting orientations.

- Based upon this, we selected the following bodypart-representation combos to add into our system.


\section{Annotation Study}
- H1- people will conclude similar things about orietnation (user agreement)

- H2 - A relatively small number of features/representations will be relevant to most character orientations (freecoding)

- H3 Twisted perspective will be present in a significant portion of the drawings.
(Show that people dont't really mix left and right orientations together though)


\end{comment}


\begin{figure*}
	\centering
	\includegraphics[width = \linewidth]{figure/AgentArena.pdf}
	\caption{\textbf{Stock Trading Workflow in \textit{Agent Trading Arena}.} 
	\textbf{Top:} Workflow of a trading day, including preparation, trading, and post-trading reflection. Agents discuss insights in the chat pool, analyze market trends, execute trades, and refine strategies based on performance.  
	\textbf{Bottom:} Example of agents' interactions in the chat pool and dynamic strategy updates.}
	\label{fig:AgentArena}
	\vspace{-3pt}
\end{figure*}

\section{Proposed Method}

% 核心部分visual representation,

To mitigate the influence of human prior knowledge and memory, we designed a closed-loop economic system~\citep{guo2024economics} called the \textit{Agent Trading Arena}, a zero-sum game simulating complex, quantitative real-world scenarios. The simulation workflow is illustrated in \autoref{fig:AgentArena} and further detailed in \autoref{appendix_arena}. In the \textit{Agent Trading Arena}, agents can invest in assets, earn dividends from holding assets, and pay daily expenses using virtual currency. The agent with the highest total return wins the game.

\subsection{Agent Trading Arena}

\paragraph{Structure of Agent Trading Arena.} 

To eliminate external knowledge biases, asset prices are determined by a bid-ask system, reflecting the prices at which buyers and sellers are willing to transact. The system evolves solely based on agents' actions and interactions, without external influences. This design ensures that the outcomes of agents' actions are not immediately apparent but unfold gradually, influenced by other agents' decisions.

To encourage active participation, a dividend mechanism is introduced. There are two primary sources of income in this system: capital gains from asset price differentials and dividends from holding assets. Dividends for each asset are distributed according to a predefined ratio, serving as an implicit anchor for asset prices. Agents holding more low-cost assets receive higher dividends. To prevent passive asset holding until the end of the game, agents must pay a daily capital cost proportional to their total wealth. These expenses are offset by asset dividends, and only agents with sufficient low-cost assets can cover costs. Under the pressure of significant daily expenses, agents must act swiftly and strategically, triggering frequent trades and price fluctuations to stimulate market activity. This dynamic mechanism ensures fairness in the zero-sum game while preventing agents from relying on fixed strategies to find optimal solutions.

\vspace{-3pt}

\paragraph{Agents Learn and Compete in Arena.}

The zero-sum game structure is crucial to eliminating the possibility of a universally optimal strategy. In fixed scenarios with a static optimal solution, agents could rely on predefined rules or memory-based approaches, bypassing adaptive decision-making. The zero-sum game ensures that there is no universally correct solution, with outcomes evolving dynamically based on agent interactions and competition. This design forces agents to continually adapt, learn from feedback, and develop context-dependent strategies, promoting deeper environmental exploration and preventing reliance on static or memory-driven solutions.

In the \textit{Agent Trading Arena}, agents are unaware of implicit rules, except for the objective to maximize their virtual wealth throughout the simulation. To win this zero-sum game, agents must effectively learn from experience, decipher hidden game rules, and develop strategies to counter competitors. This requires the ability to comprehend numerical feedback, formulate enduring strategies, and make informed decisions. Unlike other mathematical reasoning problems, the results of their actions unfold gradually and dynamically. Moreover, agents are easily misled by erroneous information from competitors, hindering their ability to discern strategic cues from competitors' textual data. Importantly, agents remain unaware of these implicit rules, so applying real-world knowledge does not benefit their performance. Therefore, agents must rely on experiential learning to decipher the hidden game rules and ultimately achieve victory.

\subsection{Types of Numerical Data Input}

\paragraph{Limitations of Textual Numerical Data.}

In the \textit{Agent Trading Arena}, the generated stock data is stored in numerical format. When used directly as input to an LLM, the models often struggle to interpret numerical data accurately or make sound decisions. To mitigate this, we convert the data into textual formats~\citep{numerical_text, long_text}, enhancing semantic features and clarifying output requirements to improve the models' understanding. During interactions, the LLMs process stock prices, trading volumes, and market indices presented as textual numerical data.

\begin{figure*}
	\centering
	\includegraphics[width = \linewidth]{figure/v_t.pdf}
	\caption{\textbf{Textual and Visual Representations of Corresponding Inputs and Outputs.} The left images display the agent’s Buy and Sell trading records, daily trade prices, and K-line charts for three stocks. The output from visual inputs (bottom right) captures overall stock trends and long-term behavior, while the output from textual inputs (top right) focuses on specific current prices.}
	\label{textual_visualized}
	\vspace{-3pt}
\end{figure*}

However, this textual approach reveals significant limitations. While the data is presented clearly, LLMs tend to focus excessively on specific values rather than identifying long-term trends or global patterns. They also struggle with understanding correlative relations and percentage changes, limiting their ability to assess differences and identify connections between data points. When analyzing time-series data with complex patterns, LLMs often fixate on individual data points, overlooking overarching relations. This issue is evident in the analysis output in the top-right corner of \autoref{textual_visualized}, where LLMs' focus on individual values impedes their ability to generalize, reducing their capacity to extract meaningful global insights.

Additionally, LLMs often overemphasize recent data while undervaluing historical information, even when prompted to consider its importance. This prevents them from effectively integrating past data and recognizing long-term patterns, complicating their understanding of numerical relations and trends. These challenges highlight the need for improved mechanisms to process numerical relations, identify global trends, and derive deeper insights from textual numerical data.

\vspace{-3pt}

\paragraph{Potential of Visual Numerical Data.}

Since textual numerical data often leads LLMs to focus on local details while neglecting broader relations, we investigated whether visual representations, such as scatter plots, line charts, and bar charts, could help LLMs better understand overall trends, similar to human reasoning. Thus, we transition from textual numerical data inputs to visualized formats ~\citep{storyllava}. As demonstrated in the bottom-right corner of \autoref{textual_visualized}, visual representations enable LLMs to more effectively grasp global trends, patterns, and relations that are often difficult to discern from textual numerical data alone.

These findings highlight the advantages of structured, visual numerical data, indicating that this format allows LLMs to more intuitively and comprehensively understand complex data, better capturing overall fluctuations, whereas text tends to focus on local details. By combining visualization and textual representations, LLMs not only overcome the challenges of relations in time-series data but also demonstrate better performance in identifying long-term trends and global patterns, while still attending to local details.

\subsection{Reflection Module}

We propose a strategy distillation method, illustrated in \autoref{fig:reflection}, that delivers real-time feedback to LLMs by analyzing both descriptive textual and visual numerical data. This enables the generation of new strategies and optimization of action plans. The approach allows agents to evaluate their results, refine strategies, and adapt continuously based on feedback. The process begins with assessing the day’s trajectory memory and associated strategies using an evaluation function. The strategic generation process leverages contrastive analysis of peak and nadir performers from the evaluation phase, creating bidirectional learning signals that inform subsequent iterations. This iterative cycle ensures continuous strategy evolution, fostering sustained improvement in decision-making.

\begin{figure}[t]
	\centering
	\includegraphics[width = \linewidth]{figure/reflection.pdf}
	\caption{\textbf{Design of the Reflection Module.} The process evaluates daily trajectory memory and strategies (top right), then generates new strategies (center) based on evaluation, environmental feedback (bottom right), and feedback from the 5 top- and bottom-performing strategies. Stock visualization (bottom left) enhances reflection, driving continuous improvement.}
	%The process evaluates daily trajectory memory and strategies, generating new strategies based on positive and negative feedback from the top- and bottom-performing strategies. Stock visualizations (bottom left) further enhance the reflection process, reinforcing continuous strategy refinement.}
	\label{fig:reflection}
	\vspace{-3pt}
\end{figure}

% We propose a strategy distillation method, illustrated in \autoref{fig:reflection}, that provides real-time feedback to LLMs by analyzing both descriptive textual and visualized numerical data. This enables the generation of new strategies and the optimization of action plans. The approach allows agents to assess their results, refine strategies, and continuously adapt based on feedback. The process begins by evaluating the day's trajectory memory and associated strategies using an evaluation function. From this assessment, new strategies are generated by selecting the top-performing and lowest-performing strategies, offering both positive and negative feedback. This iterative cycle ensures continuous strategy evolution, driving sustained improvement in decision-making.

The reflection module plays a crucial role in refining strategies by offering real-time feedback. It analyzes both descriptive textual and visual numerical data to generate new strategies and optimize action plans. Within the \textit{Agent Trading Arena}, the reflection module is triggered regularly to consolidate daily trading records and evaluate the effectiveness of strategies, refining both successful and unsuccessful experiences to guide future decisions. Ineffective strategies are stored in a strategy library for future reference, allowing agents to review and learn from past experiences. Further details can be found in \autoref{appendix_arena}.



\begin{comment}
\subsection{Character Model}

\subsection{Retargeting Method}


- 2D billboard (necessary for style)
- View dependent pose retargeting
    -Switching of the limbs based upon viewing angle
- Internal feature manipulation
- Feet flipping driven by knee angle
- External feature manipulation (important given results of annotation task, from annotation task, can have:
    - breakdown of internal vs. external features
    
- Twisted Perspective
    - When mentioning why this is important, see if I can formulate in a mathy way the reason why things get so bad.
    In essense, we've got a 3D point the is multiplied by some matrix to get a 2D projection of the point on a plane. And the 2D projections are used to obtain the orientation. At some point, changes in the 3D joint position have either no effect upon the orientation, or an outsized effect on the orientation. Figure out how this is reflected in the Jacobian matrix (ill-conditioned?) (null space?)
    - Mention how discontinuities are okay (this is how 2D animation works, after all), but the discontinuities shouldn't be frequently. This is why we have the previous projection vector as part of the optimization.

\subsection{Base Representation}
- Discuss how we create a left and right version of each character, and use left one when view angle is between 0 -> pi/2 -> pi and right one otherwise. Show unit circle.

- Discuss how we create a front and back texture and use front when view angle is pi/2 -> pi -> 3pi/2 and back otherwise. Explain how we create it from annotations. Show unit circle.

\subsection{Parts}
-Discuss how we generate the parts from the annotations and how we initialize the keyview-transform dictionary. Figure showing unit circle along with examples at keyview left and right. Show on character that has no obvious left-right difference in base representation.

\subsection{LimbMapping}
Discuss how, if we keep character normal pointed towards the viewer, eventually motion source limbs will drive the opposite character limbs, results in a strange pose and unrecognizable pose/silhouette. So we swap between 3pi/2 -> 0 -> pi/2. Show unit circle

\subseciont{Render Order}
Original skeleton 3D joint locations used to determine rendering order. 
Order is determined by projecting to BVH joint depth driver onto the projection plane normal vector and determining whether, when viewed from that particular angle, it is in front of or behind the torso depth joint.

\subsection{User customization}
 - mention how the number of attachment points can be parameterized to make the snappiness of the character higher or lower.  
 
\end{comment}
 

\section{Evaluation}
% In light of experiments of CacheBlend (\S\ref{eval:1}) and EPIC (\S\ref{eval:2}), we design our experiments (\S\ref{eval:3}).

% \noindent\textbf{LLM Dataset.} 2WikiMQA, MuSiQue, HotpotQA, SAMSum, MultiNews.

% \noindent\textbf{LLM Baselines.} Full KV recompute, Prefix caching, Full KV reuse, CacheBlend, EPIC.
% \subsection{CacheBlend evaluation}\label{eval:1}
% \begin{itemize}
%     \item TTFT-Score Comparison.
%     \item RPS-TTFT Comparison.
%     \item Sensitivity Analysis. (1) chunk number; (2) chunk length; (3) batch size; (4) recompute ratio; (5) storage device (CPU RAM / slower Disk).
% \end{itemize}
% \subsection{EPIC evaluation}\label{eval:2}
% \begin{itemize}
%     \item TTFT-Score Comparison.
%     \item (CCR+RPS)-TTFT/Throughput Comparison.
%     \item Context length-TTFT Comparison.
%     \item Semantic-based / fixed-token-based splitting.
% \end{itemize}
% \subsection{\sys}\label{eval:3}\
% \noindent\textbf{VLM Model.} InternVL 2.5-8B \cite{chen2024internvl}, Qwen2-VL-7B \cite{wang2024qwen2vl}, LLaVA-1.6-vicuna-7B, LLaVA-1.6-Mistral-7B \cite{liu2024llavanext}.

% \noindent\textbf{VLM Dataset.} SparklesDialogueCC, SparklesDialogueVG \cite{huang2024sparkles}, MMDU \cite{liu2024mmdu}.

% \noindent\textbf{VLM Baselines.} CacheBlend, Prefix caching, Full KV reuse, \sys.

% \begin{itemize}
%     \item TTFT-Score Comparison.
%     \item RPS-TTFT/Throughput Comparison.
%     \item Sensitivity Analysis: Image number.
%     \item Why does CacheBlend fail to work when serving MLLM?
% \end{itemize}

In this section, we evaluate \sys~in terms of response time and generation quality. We also investigate whether \sys~is applicable when the number of images is large.
\subsection{Experimental settings}
We select two prevalent MLLMs in the experiments: LLaVA-1.6-vicuna-7B and LLaVA-1.6-mistral-7B \cite{liu2024llavanext}. All experiments are run on a server with 1 NVIDIA H800-80 GB GPU, 20-core Intel(R) Xeon(R) Platinum CPUs, and 100GB DRAM.

Two datasets are used in our evaluation. (1) \textbf{MMDU} \cite{liu2024mmdu}: This dataset aims to evaluate MLLMs' abilities in multi-turn and multi-image conversations. Each conversation stitches together multiple images and sentence-level text (e.g., ``IMAGE\#1, IMAGE\#2. Can you describe these images as detailed as possible?"). (2) \textbf{SparklesEval} \cite{huang2024sparkles}: This is also a dataset for assessing MLLMs' conversational competence across multiple images and conversation turns. Unlike MMDU, SparklesEval integrates multiple images at word level (e.g., ``Can you link the celebration occurring in IMAGE\#1 and the dirt bike race in IMAGE\#2 ?"). As shown in the examples, the prompts of two datasets are open questions. Previous works adopt GPT score to evaluate the quality of MLLMs' responses to the open questions \cite{liu2024mmdu, huang2024sparkles}. GPT score is a GPT-assisted evaluation that uses a powerful judge model (e.g., GPT-4o, Qwen, etc.) to assess the answers. We also employ this metric and their evaluation prompt, as listed in Appendix~\ref{prompt}.
% (3) \textbf{V*Bench} \cite{wu2024v}:  A dataset specifically designed to evaluate
% MLLMs in their ability to process high-resolution images and focus on visual details. Each sample contains a high-resolution image, a question, and four options.
% We select 100 samples from each of the above datasets for testing, each including 1 to 5 images.

% We use the following metrics to measure the performance of algorithms. (1) Time-To-First-Token (TTFT) refers to the time it takes for LLMs, to generate and return the first token after receiving an request. This metric is designed to measure the time spent in the prefill stage, which can be optimized by addressing the PIC problem. (2) GPT score \cite{liu2024mmdu, huang2024sparkles} is a GPT-assited evaluation  that uses a judge model (e.g., GPT-4o, Qwen, etc.) to assess the quality of model-generated responses. We employ this metric to assess the quality of MLLMs' responses to the open questions in MMDU and SparklesEval. We apply the evaluation prompts in MMDU \cite{liu2024mmdu} to guide the judge model for scoring in the range of 10. 

% (3) F1 score is a metric used to evaluate the similarity between MLLMs’ output and the groundtruth answer. We employ this metric to assess the accuracy of the MLLMs' answers to the multiple-choice questions in V*Bench.

We compare \sys-$k$ with three existing CC algorithms: prefix caching, full reuse, and CacheBlend \cite{yao2024cacheblend}. CacheBlend is also a position-independent algorithm designed for RAG system. It recomputes $r$\% of total tokens with largest KV deviation, so we denote it as CacheBlend-$r$. The primary focus of CacheBlend is the KV deviation, while the \sys's selection process involves the identification of tokens that exhibit both high attention scores and significant KV deviation. We implement the four CC algorithms based on vLLM 0.6.4 \cite{kwon2023efficient}.

% (1) Prefix Caching: This algorithm merely stores and reuses the KV chche of the prefix. And the KV cache of non-prefix tokens needs to be computed during prefill. (2) Full Reuse: This algorithm reduces TTFT by fully reusing the entire KV cache regardless of the position of multimodal data. (3) CacheBlend \cite{yao2024cacheblend}: This is a state-of-the-art partial reuse algorithm that achieves a trade-off between TTFT and generation quality by dynamically selecting partial tokens to recompute.
% Additionally, we evaluate various variants of CacheBlend, denoted as CacheBlend-r, where $r$ represents the ratio of tokens recomputed. Similarly, we test different variants of InfoBlend, denoted as InfoBlend-k, where $k$ indicates the number of tokens recomputed at each chunk boundary.

\subsection{Effectiveness of \sys}
Based on vLLM offline inference, we compare the performance of all algorithms. Specifically, we process all requests sequentially and evaluate their generation quality and processing time for prefill. The workflow initiates with the precomputation of the relevant KV cache for images. Subsequently, we send the user's query along with the cache\_ids of the images to the serving system. Prefix caching will process the query with the KV cache of system prompt only. \sys~concatenates the dummy cache and stored cache, and computes the first output token using selective attention mechanism in single step. Full reuse and CacheBlend first compute the KV cache of text, and then produce the first output token with the concatenated KV cache. We record the processing time of the algorithms and finally score for each response.
\begin{figure}[t]
    \centering
    \includegraphics[width=\columnwidth]{figs/legend_result.pdf}
    % \vskip -0.2in
    \includegraphics[width=\columnwidth]{figs/results.pdf}
    \caption{Comparison of TTFT ($\downarrow$ Better) and Score ($\uparrow$ Better) using different models on different datasets. }
    \label{fig:ttft-score}
    % \vskip -0.2in
\end{figure}

\figurename~\ref{fig:ttft-score} presents the experimental results of all algorithms across different models and datasets. The results indicate that \sys~consistently outperforms CacheBlend in terms of both TTFT and score across various configurations. \sys-32 reduces TTFT by up to 54.1\% while maintaining a loss of score within 13.6\% compared to prefix caching. Additionally, it is clear that \sys~exhibits a slight decrease in TTFT compared to full reuse, since \sys~is a single-step process. Overall, compared to other algorithms, \sys~achieves the best trade-off between TTFT and score.

\subsection{Sensitivity analysis}
In order to achieve a more profound comprehension of \sys, a subsequent analysis is necessary to ascertain how the number of images impacts overall performance. We divide the dataset of MMDU into 10 groups in terms of the number of images. We evaluate the TTFT and score of \sys~and baselines on each group. The average value of results are shown in \figurename~\ref{fig:10}. The TTFT of \sys~is consistently shorter than that of prefix caching. When the number of images is 10, \sys~achieves 54.7\% reduction in TTFT. Furthermore, the performance of \sys~remains unaffected by the number of images, exhibiting negligible or no accuracy degradation.
\begin{figure}[t]
    \centering
    \includegraphics[width=0.9\columnwidth]{figs/legend_image_num.pdf}
    \vskip -0.2in
    \subfloat[]{
        \includegraphics[width=0.42\columnwidth]{figs/TTFT_all.pdf}
        \label{fig:10a}
    }
    \subfloat[]{
        \includegraphics[width=0.4\columnwidth]{figs/Score_all.pdf}
        \label{fig:10b}
    }
    \caption{The performance of \sys~as the number of images increases. For clarity, we only present the results of \sys-32. Other variants of \sys~show similar patterns.}
    \label{fig:10}
    % \vskip -0.2in
\end{figure}

% \subsection{Latency and throughput performance of InfoBlend}
% To assess Infoblend's latency and throughput performance, we leverage VLLM's OpenAI-compatible API server to simulate real-world user request patterns. We first select $n$ samples from MMDU and pre-generate KV caches for their contexts. Subsequently, we simulate user request behavior by repeatedly sending the user queries along with the cache\_ids of these 
% $n$ samples at a specified request rate over a period of time. by varying the request rate, We measure the latency and throughput across different experimental conditions.

% In Figure, we present a comparison of latency and throughput between InfoBlend and CacheBlend at varying request rates.  Compared to CacheBlend, InfoBlend achieves up to 80\% reduction in TTFT and 2-3 $\times$ improvement in throughput. This gap increases as the request rate rises.

Software development is increasingly conceived as a collaboration activity between developers and AIs. Indeed, IDEs already implement features to enable interactive development, with AI suggesting implementations that are reused by developers.

Although multiple studies show this interaction can be successful, there is still limited understanding of how the models must be configured and used in the context of code generation tasks. This study addresses this gap, systematically investigating the impact of several key parameters, including the repeated submission of a prompt to accommodate for the non-deterministic nature of the models.

Our study reveals several key findings about the usage of ChatGPT. In particular, we discovered how creativity, although up to a limited extent, is useful to increase the range of methods whose code can be generated correctly. A major role is played by parameter top-p, which is commonly underrated, and instead has a major impact on the correctness of the results, with lower values producing better results. Finally, prompts should be submitted multiple times, with $5$ repetitions combined with a temperature of $1.2$ resulting in an effective configuration in our experiments.  

Future work concerns two main research directions. One is about replicating this experiment with other AI assistants, to validate our findings in multiple contexts. The second research direction concerns finding strategies to deal with the need to submit the same prompt multiple times to obtain a useful result, and thus developing approaches able to select or merge multiple responses automatically. 
%\subsection{Limitations}

\begin{comment}
- method is fast enough to run in real time and can support rapid iteration on the annotations
\end{comment}

% Certain kinds of drawings it doesn't work on. (Works best for characters in T pose, which is frequently seen. Works with others as well, but the ARAP solve can produce artifacts)
%Certain kinds of motions it's doesn't work on.
%What currently takes the most time to do manually.
%contacts not necessarily preserved

%The motion doesn't really match what is expected for realistic character (see video)

%ARAP
%    - When the character isn't drawn in a t pose the arap solve can lead to some funky %artifacts (e.g. soccer guy)
%    - The ARAP solve always gives bendy results. This is probably undesireable for more rigid characters, like a robot.

%- Skeletal motion allows lots of inputs. E.g. text-to-motion can be used. (also show in paper). It's limiting in some senses, but also extremely empowering given the right circumstances.

%- There is still popping that can occur. Sudden change when limb mapping changes. Not ideal, but as long as it happens infrequently maybe that's okay.

%- Mention that the characters could be extended with further 'Halfs' and textures beyond left and right and front and back. However, here we show what is possible to achieve with minimal user input.

%Future work:
%- automate the annotation process with SAM and the like
%- handle input character with overlapping regions using better inpainting
% allow arms to extrude from the middle of the body?
%facial animation





\bibliographystyle{ACM-Reference-Format}
\bibliography{Ref}


\begin{figure*}[ht]
\centering
\includegraphics[width=0.9\textwidth]{figures/part_traits.png}
\caption{Examples of the different part traits mentioned by annotators. The extension of the tail and nose in \textit{a} suggest (conflicting) orientations; in \textit{b} the position of the eyes within head and pupils within eye suggest it; in \textit{c} the limbs are drawn as though viewing a character facing right; \textit{d-f} show less frequently mentioned cues of limb attachment point, perspective-based scaling, and occlusion, respectively.}
\label{fig:part_trait_example}
\end{figure*}

\begin{figure*}[ht]
\centering
\includegraphics[width=0.9\textwidth]{figures/user_study_1.png}
\caption{\textit{Top row:} original drawings of the 10 characters used in the perceptual user study. 
Subsequent rows contain the number of participants in each study, response contingency table, and chi-squared test static with p-value. The upper left cell of the contingency table contains the number of times a user correctly identified the original motion upon both the 2.5D model and the 3D model; upper right contains times motion was correctly identified upon 2.5D model only; lower left contains times motion was correctly identified upon 3D model only; bottom right contains times motion was not correctly identified in either case.}
\label{fig:user_study_1}
\end{figure*}

\begin{figure*}[ht]
\centering
\includegraphics[width=0.9\textwidth]{figures/ablations.png}
\caption{
\textit{Left:} when the input pose faces towards the camera (top row), it's left limbs are used to drive the limbs on the \textit{drawing-right} side of the model.
When the input pose faces away from the camera, (bottom row), this mapping is flipped and the left limbs now drive the limbs on the \textit{drawing-left} side of the model (middle column).
Without this flipping, limbs appear switched and the pose is less recognizable (right column, arms.
\textit{Right:} when only the root-view projection plane  (shown in white) is used for retargeting, unexpected flailing can occur; when the axis of the left lower arm is parallel to the root-view projection plane normal vector, flailing can occur (top, middle column). 
Dynamically modifying the limb's projection plane based addresses this; in the bottom rows, the skeleton's left arm projection plane (shown in red) deviates from the root-view projection plane, preventing flailing from occurring.}
\label{fig:twisted_perspective}
\end{figure*}

%\begin{figure*}[ht]
%\centering
%\includegraphics[width=0.4\textwidth]{figures/twisted_perspective.png}
%\caption{
%When only the root-view projection plane  (shown in white) is used for retargeting, unexpected flailing can occur; when the axis of the left lower arm is parallel to the root-view projection plane normal vector, flailing can occur (top, middle column). 
%Dynamically modifying the limb's projection plane based addresses this; in the bottom rows, the skeleton's left arm projection plane (shown in red) deviates from the root-view projection plane, preventing flailing from occurring.}
%\label{fig:twisted_perspective}
%\end{figure*}

%\begin{figure*}
%\centering
%\includegraphics[width=0.4\textwidth]{figures/limb-swapping.png}
%\caption{When the input pose faces towards the camera (top row), it's left limbs are used to drive the limbs on the \textit{drawing-right} side of the model.
%When the input pose faces away from the camera, (bottom row), this mapping is flipped and the left limbs now drive the limbs on the \textit{drawing-left} side of the model (middle column).
%Without this flipping, limbs appear switched and the pose is less recognizable (right column, arms).
%}
%\label{fig:limb_mapping}
%\end{figure*}

\begin{figure*}[ht]
\centering
\includegraphics[width=1.0\textwidth]{figures/applications.png}
\caption{Example applications: 3D scenes (a), 2D scenes (b), driving with text-to-motion model (c), motion capture (d), and 3-point tracking (e). Example mixed reality experience (f).}
\label{fig:applications}
\end{figure*}

\begin{figure*}[ht]
\centering
\includegraphics[width=1.0\textwidth]{figures/genai.png}
\caption{Original drawing (a), alternate views created by our method (b), and alternate views created by multi-view diffusion model~\cite{shi2023zero123} (c).}
\label{fig:genai}
\end{figure*}

\end{document}
\endinput
%%
%% End of file `sample-acmsmall.tex'.
