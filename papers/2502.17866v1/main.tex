%%
%% This is file `sample-acmsmall.tex',
%% generated with the docstrip utility.
%%
%% The original source files were:
%%
%% samples.dtx  (with options: `acmsmall')
%% 
%% IMPORTANT NOTICE:
%% 
%% For the copyright see the source file.
%% 
%% Any modified versions of this file must be renamed
%% with new filenames distinct from sample-acmsmall.tex.
%% 
%% For distribution of the original source see the terms
%% for copying and modification in the file samples.dtx.
%% 
%% This generated file may be distributed as long as the
%% original source files, as listed above, are part of the
%% same distribution. (The sources need not necessarily be
%% in the same archive or directory.)
%%
%% Commands for TeXCount
%TC:macro \cite [option:text,text]
%TC:macro \citep [option:text,text]
%TC:macro \citet [option:text,text]
%TC:envir table 0 1
%TC:envir table* 0 1
%TC:envir tabular [ignore] word
%TC:envir displaymath 0 word
%TC:envir math 0 word
%TC:envir comment 0 0
%%
%%
%% The first command in your LaTeX source must be the \documentclass command.
%% \documentclass[acmtog,anonymous,review]{acmart}
\documentclass[acmtog]{acmart}

\usepackage{comment}
\usepackage{algorithm}
\usepackage{algpseudocode}
\usepackage{amsmath}

%% \documentclass[acmsmall]{acmart}
%% NOTE that a single column version is required for 
%% submission and peer review. This can be done by changing
%% the \doucmentclass[...]{acmart} in this template to 
%% \documentclass[manuscript,screen]{acmart}
%% 
%% To ensure 100% compatibility, please check the white list of
%% approved LaTeX packages to be used with the Master Article Template at
%% https://www.acm.org/publications/taps/whitelist-of-latex-packages 
%% before creating your document. The white list page provides 
%% information on how to submit additional LaTeX packages for 
%% review and adoption.
%% Fonts used in the template cannot be substituted; margin 
%% adjustments are not allowed.
%%
%% \BibTeX command to typeset BibTeX logo in the docs
%\AtBeginDocument{%
%  \providecommand\BibTeX{{%
%    \normalfont B\kern-0.5em{\scshape i\kern-0.25em b}\kern-0.8em\TeX}}}

%% Rights management information.  This information is sent to you
%% when you complete the rights form.  These commands have SAMPLE
%% values in them; it is your responsibility as an author to replace
%% the commands and values with those provided to you when you
%% complete the rights form.
%\setcopyright{acmcopyright}
%\copyrightyear{2018}
%\acmYear{2018}
%\acmDOI{XXXXXXX.XXXXXXX}


%%
%% These commands are for a JOURNAL article.
\acmJournal{TOG}
%\acmVolume{37}
%\acmNumber{4}
%\acmArticle{111}
%\acmMonth{8}

%%
%% Submission ID.
%% Use this when submitting an article to a sponsored event. You'll
%% receive a unique submission ID from the organizers
%% of the event, and this ID should be used as the parameter to this command.
\acmSubmissionID{673}

%%
%% For managing citations, it is recommended to use bibliography
%% files in BibTeX format.
%%
%% You can then either use BibTeX with the ACM-Reference-Format style,
%% or BibLaTeX with the acmnumeric or acmauthoryear sytles, that include
%% support for advanced citation of software artefact from the
%% biblatex-software package, also separately available on CTAN.
%%
%% Look at the sample-*-biblatex.tex files for templates showcasing
%% the biblatex styles.
%%

%%
%% The majority of ACM publications use numbered citations and
%% references.  The command \citestyle{authoryear} switches to the
%% "author year" style.
%%
%% If you are preparing content for an event
%% sponsored by ACM SIGGRAPH, you must use the "author year" style of
%% citations and references.
%% Uncommenting
%% the next command will enable that style.
\citestyle{acmauthoryear}

%%
%% end of the preamble, start of the body of the document source.
\begin{document}

%%
%% The "title" command has an optional parameter,
%% allowing the author to define a "short title" to be used in page headers.
\title{Animating Childlike Drawings with 2.5D Character Rigs}

%%
%% The "author" command and its associated commands are used to define
%% the authors and their affiliations.
%% Of note is the shared affiliation of the first two authors, and the
%% "authornote" and "authornotemark" commands
%% used to denote shared contribution to the research.
\author{Harrison Jesse Smith}
\authornote{Corresponding author.}
\affiliation{%
  \institution{Meta}
  \city{Sausalito}
  \state{CA}
  \country{USA}
}
\email{hjessmith@gmail.com}

\author{Nicky He}
\affiliation{%
  \institution{Meta}
  \city{Sausalito}
  \state{CA}
  \country{USA}
}
\email{nicky.hsj@gmail.com}

\author{Yuting Ye}
\affiliation{%
  \institution{Meta}
  \city{Redmond}
  \state{WA}
  \country{USA}
}
\email{yuting.ye@gmail.com}



%%
%% By default, the full list of authors will be used in the page
%% headers. Often, this list is too long, and will overlap
%% other information printed in the page headers. This command allows
%% the author to define a more concise list
%% of authors' names for this purpose.
\renewcommand{\shortauthors}{Smith et al.}

\newcommand{\yuting}[1]{{\color{magenta} [Yuting: {#1}]}}
\newcommand{\jesse}[1]{{\color{blue} [Jesse: {#1}]}}
\newcommand{\nicky}[1]{{\color{green} [Nicky: {#1}]}}
\newcommand{\strike}[1]{{\color{red} [{#1}]}}

%%
%% The abstract is a short summary of the work to be presented in the
%% article.
\begin{abstract}

Drawing is a fun and intuitive way to create a character, accessible even to small children. However, animating 2D figure drawings is a much more challenging task, requiring specialized tools and skills. Bringing 2D figures to 3D so they can be animated and consumed in immersive media poses an even greater challenge. Moreover, it is desirable to preserve the unique style and identity of the figure when it is being animated and viewed from different perspectives. In this work, we present an approachable and easy-to-create 2.5D character model and retargeting technique that can apply complex 3D skeletal motion, including rotation within the transverse plane, onto a single childlike figure drawing in a style-preserving manner in realtime. Because our solution is view-dependent, the resulting character is well-suited for animation in both 2D and 3D contexts. We also present a novel annotation study motivating our system design decisions and a pair of user studies validating the usefulness and appeal of our solution. We in addition showcase the generality of our system in a range of 2D and 3D applications.

%\nicky{Bringing these characters to 3D applications, like 3D games and mixed reality, introduces new challenges to represent and animated them in 3D space as they can be viewed from all perspectives. } 
\end{abstract}

%%
%% This command processes the author and affiliation and title
%% information and builds the first part of the formatted document.

\begin{teaserfigure}
  \centering
  \includegraphics[width=0.95\textwidth]{figures/banner.png}
  \caption{We present an animation system that turns a single childlike figure drawing into a 2.5D character model. It can be animated using any 3D skeletal motion and viewed from different perspectives. Our real-time and view-dependent motion retargeting algorithm makes it suitable for mixed-reality applications.}
  \Description{Banner Caption}
  \label{fig:teaser}
\end{teaserfigure}

\maketitle

% \section{Introduction}
ACM's consolidated article template, introduced in 2017, provides a
consistent \LaTeX\ style for use across ACM publications, and
incorporates accessibility and metadata-extraction functionality
necessary for future Digital Library endeavors. Numerous ACM and
SIG-specific \LaTeX\ templates have been examined, and their unique
features incorporated into this single new template.

If you are new to publishing with ACM, this document is a valuable
guide to the process of preparing your work for publication. If you
have published with ACM before, this document provides insight and
instruction into more recent changes to the article template.

The ``\verb|acmart|'' document class can be used to prepare articles
for any ACM publication --- conference or journal, and for any stage
of publication, from review to final ``camera-ready'' copy, to the
author's own version, with {\itshape very} few changes to the source.

\section{Template Overview}
As noted in the introduction, the ``\verb|acmart|'' document class can
be used to prepare many different kinds of documentation --- a
dual-anonymous initial submission of a full-length technical paper, a
two-page SIGGRAPH Emerging Technologies abstract, a ``camera-ready''
journal article, a SIGCHI Extended Abstract, and more --- all by
selecting the appropriate {\itshape template style} and {\itshape
  template parameters}.

This document will explain the major features of the document
class. For further information, the {\itshape \LaTeX\ User's Guide} is
available from
\url{https://www.acm.org/publications/proceedings-template}.

\subsection{Template Styles}

The primary parameter given to the ``\verb|acmart|'' document class is
the {\itshape template style} which corresponds to the kind of publication
or SIG publishing the work. This parameter is enclosed in square
brackets and is a part of the {\verb|documentclass|} command:
\begin{verbatim}
  \documentclass[STYLE]{acmart}
\end{verbatim}

Journals use one of three template styles. All but three ACM journals
use the {\verb|acmsmall|} template style:
\begin{itemize}
\item {\verb|acmsmall|}: The default journal template style.
\item {\verb|acmlarge|}: Used by JOCCH and TAP.
\item {\verb|acmtog|}: Used by TOG.
\end{itemize}

The majority of conference proceedings documentation will use the {\verb|acmconf|} template style.
\begin{itemize}
\item {\verb|acmconf|}: The default proceedings template style.
\item{\verb|sigchi|}: Used for SIGCHI conference articles.
\item{\verb|sigchi-a|}: Used for SIGCHI ``Extended Abstract'' articles.
\item{\verb|sigplan|}: Used for SIGPLAN conference articles.
\end{itemize}

\subsection{Template Parameters}

In addition to specifying the {\itshape template style} to be used in
formatting your work, there are a number of {\itshape template parameters}
which modify some part of the applied template style. A complete list
of these parameters can be found in the {\itshape \LaTeX\ User's Guide.}

Frequently-used parameters, or combinations of parameters, include:
\begin{itemize}
\item {\verb|anonymous,review|}: Suitable for a ``dual-anonymous''
  conference submission. Anonymizes the work and includes line
  numbers. Use with the \verb|\acmSubmissionID| command to print the
  submission's unique ID on each page of the work.
\item{\verb|authorversion|}: Produces a version of the work suitable
  for posting by the author.
\item{\verb|screen|}: Produces colored hyperlinks.
\end{itemize}

This document uses the following string as the first command in the
source file:
\begin{verbatim}
\documentclass[acmsmall]{acmart}
\end{verbatim}

\section{Modifications}

Modifying the template --- including but not limited to: adjusting
margins, typeface sizes, line spacing, paragraph and list definitions,
and the use of the \verb|\vspace| command to manually adjust the
vertical spacing between elements of your work --- is not allowed.

{\bfseries Your document will be returned to you for revision if
  modifications are discovered.}

\section{Typefaces}

The ``\verb|acmart|'' document class requires the use of the
``Libertine'' typeface family. Your \TeX\ installation should include
this set of packages. Please do not substitute other typefaces. The
``\verb|lmodern|'' and ``\verb|ltimes|'' packages should not be used,
as they will override the built-in typeface families.

\section{Title Information}

The title of your work should use capital letters appropriately -
\url{https://capitalizemytitle.com/} has useful rules for
capitalization. Use the {\verb|title|} command to define the title of
your work. If your work has a subtitle, define it with the
{\verb|subtitle|} command.  Do not insert line breaks in your title.

If your title is lengthy, you must define a short version to be used
in the page headers, to prevent overlapping text. The \verb|title|
command has a ``short title'' parameter:
\begin{verbatim}
  \title[short title]{full title}
\end{verbatim}

\section{Authors and Affiliations}

Each author must be defined separately for accurate metadata
identification. Multiple authors may share one affiliation. Authors'
names should not be abbreviated; use full first names wherever
possible. Include authors' e-mail addresses whenever possible.

Grouping authors' names or e-mail addresses, or providing an ``e-mail
alias,'' as shown below, is not acceptable:
\begin{verbatim}
  \author{Brooke Aster, David Mehldau}
  \email{dave,judy,steve@university.edu}
  \email{firstname.lastname@phillips.org}
\end{verbatim}

The \verb|authornote| and \verb|authornotemark| commands allow a note
to apply to multiple authors --- for example, if the first two authors
of an article contributed equally to the work.

If your author list is lengthy, you must define a shortened version of
the list of authors to be used in the page headers, to prevent
overlapping text. The following command should be placed just after
the last \verb|\author{}| definition:
\begin{verbatim}
  \renewcommand{\shortauthors}{McCartney, et al.}
\end{verbatim}
Omitting this command will force the use of a concatenated list of all
of the authors' names, which may result in overlapping text in the
page headers.

The article template's documentation, available at
\url{https://www.acm.org/publications/proceedings-template}, has a
complete explanation of these commands and tips for their effective
use.

Note that authors' addresses are mandatory for journal articles.

\section{Rights Information}

Authors of any work published by ACM will need to complete a rights
form. Depending on the kind of work, and the rights management choice
made by the author, this may be copyright transfer, permission,
license, or an OA (open access) agreement.

Regardless of the rights management choice, the author will receive a
copy of the completed rights form once it has been submitted. This
form contains \LaTeX\ commands that must be copied into the source
document. When the document source is compiled, these commands and
their parameters add formatted text to several areas of the final
document:
\begin{itemize}
\item the ``ACM Reference Format'' text on the first page.
\item the ``rights management'' text on the first page.
\item the conference information in the page header(s).
\end{itemize}

Rights information is unique to the work; if you are preparing several
works for an event, make sure to use the correct set of commands with
each of the works.

The ACM Reference Format text is required for all articles over one
page in length, and is optional for one-page articles (abstracts).

\section{CCS Concepts and User-Defined Keywords}

Two elements of the ``acmart'' document class provide powerful
taxonomic tools for you to help readers find your work in an online
search.

The ACM Computing Classification System ---
\url{https://www.acm.org/publications/class-2012} --- is a set of
classifiers and concepts that describe the computing
discipline. Authors can select entries from this classification
system, via \url{https://dl.acm.org/ccs/ccs.cfm}, and generate the
commands to be included in the \LaTeX\ source.

User-defined keywords are a comma-separated list of words and phrases
of the authors' choosing, providing a more flexible way of describing
the research being presented.

CCS concepts and user-defined keywords are required for for all
articles over two pages in length, and are optional for one- and
two-page articles (or abstracts).

\section{Sectioning Commands}

Your work should use standard \LaTeX\ sectioning commands:
\verb|section|, \verb|subsection|, \verb|subsubsection|, and
\verb|paragraph|. They should be numbered; do not remove the numbering
from the commands.

Simulating a sectioning command by setting the first word or words of
a paragraph in boldface or italicized text is {\bfseries not allowed.}

\section{Tables}

The ``\verb|acmart|'' document class includes the ``\verb|booktabs|''
package --- \url{https://ctan.org/pkg/booktabs} --- for preparing
high-quality tables.

Table captions are placed {\itshape above} the table.

Because tables cannot be split across pages, the best placement for
them is typically the top of the page nearest their initial cite.  To
ensure this proper ``floating'' placement of tables, use the
environment \textbf{table} to enclose the table's contents and the
table caption.  The contents of the table itself must go in the
\textbf{tabular} environment, to be aligned properly in rows and
columns, with the desired horizontal and vertical rules.  Again,
detailed instructions on \textbf{tabular} material are found in the
\textit{\LaTeX\ User's Guide}.

Immediately following this sentence is the point at which
Table~\ref{tab:freq} is included in the input file; compare the
placement of the table here with the table in the printed output of
this document.

\begin{table}
  \caption{Frequency of Special Characters}
  \label{tab:freq}
  \begin{tabular}{ccl}
    \toprule
    Non-English or Math&Frequency&Comments\\
    \midrule
    \O & 1 in 1,000& For Swedish names\\
    $\pi$ & 1 in 5& Common in math\\
    \$ & 4 in 5 & Used in business\\
    $\Psi^2_1$ & 1 in 40,000& Unexplained usage\\
  \bottomrule
\end{tabular}
\end{table}

To set a wider table, which takes up the whole width of the page's
live area, use the environment \textbf{table*} to enclose the table's
contents and the table caption.  As with a single-column table, this
wide table will ``float'' to a location deemed more
desirable. Immediately following this sentence is the point at which
Table~\ref{tab:commands} is included in the input file; again, it is
instructive to compare the placement of the table here with the table
in the printed output of this document.

\begin{table*}
  \caption{Some Typical Commands}
  \label{tab:commands}
  \begin{tabular}{ccl}
    \toprule
    Command &A Number & Comments\\
    \midrule
    \texttt{{\char'134}author} & 100& Author \\
    \texttt{{\char'134}table}& 300 & For tables\\
    \texttt{{\char'134}table*}& 400& For wider tables\\
    \bottomrule
  \end{tabular}
\end{table*}

Always use midrule to separate table header rows from data rows, and
use it only for this purpose. This enables assistive technologies to
recognise table headers and support their users in navigating tables
more easily.

\section{Math Equations}
You may want to display math equations in three distinct styles:
inline, numbered or non-numbered display.  Each of the three are
discussed in the next sections.

\subsection{Inline (In-text) Equations}
A formula that appears in the running text is called an inline or
in-text formula.  It is produced by the \textbf{math} environment,
which can be invoked with the usual
\texttt{{\char'134}begin\,\ldots{\char'134}end} construction or with
the short form \texttt{\$\,\ldots\$}. You can use any of the symbols
and structures, from $\alpha$ to $\omega$, available in
\LaTeX~\cite{Lamport:LaTeX}; this section will simply show a few
examples of in-text equations in context. Notice how this equation:
\begin{math}
  \lim_{n\rightarrow \infty}x=0
\end{math},
set here in in-line math style, looks slightly different when
set in display style.  (See next section).

\subsection{Display Equations}
A numbered display equation---one set off by vertical space from the
text and centered horizontally---is produced by the \textbf{equation}
environment. An unnumbered display equation is produced by the
\textbf{displaymath} environment.

Again, in either environment, you can use any of the symbols and
structures available in \LaTeX\@; this section will just give a couple
of examples of display equations in context.  First, consider the
equation, shown as an inline equation above:
\begin{equation}
  \lim_{n\rightarrow \infty}x=0
\end{equation}
Notice how it is formatted somewhat differently in
the \textbf{displaymath}
environment.  Now, we'll enter an unnumbered equation:
\begin{displaymath}
  \sum_{i=0}^{\infty} x + 1
\end{displaymath}
and follow it with another numbered equation:
\begin{equation}
  \sum_{i=0}^{\infty}x_i=\int_{0}^{\pi+2} f
\end{equation}
just to demonstrate \LaTeX's able handling of numbering.

\section{Figures}

The ``\verb|figure|'' environment should be used for figures. One or
more images can be placed within a figure. If your figure contains
third-party material, you must clearly identify it as such, as shown
in the example below.
\begin{figure}[h]
  \centering
  \includegraphics[width=\linewidth]{sample-franklin}
  \caption{1907 Franklin Model D roadster. Photograph by Harris \&
    Ewing, Inc. [Public domain], via Wikimedia
    Commons. (\url{https://goo.gl/VLCRBB}).}
  \Description{A woman and a girl in white dresses sit in an open car.}
\end{figure}

Your figures should contain a caption which describes the figure to
the reader.

Figure captions are placed {\itshape below} the figure.

Every figure should also have a figure description unless it is purely
decorative. These descriptions convey what’s in the image to someone
who cannot see it. They are also used by search engine crawlers for
indexing images, and when images cannot be loaded.

A figure description must be unformatted plain text less than 2000
characters long (including spaces).  {\bfseries Figure descriptions
  should not repeat the figure caption – their purpose is to capture
  important information that is not already provided in the caption or
  the main text of the paper.} For figures that convey important and
complex new information, a short text description may not be
adequate. More complex alternative descriptions can be placed in an
appendix and referenced in a short figure description. For example,
provide a data table capturing the information in a bar chart, or a
structured list representing a graph.  For additional information
regarding how best to write figure descriptions and why doing this is
so important, please see
\url{https://www.acm.org/publications/taps/describing-figures/}.

\subsection{The ``Teaser Figure''}

A ``teaser figure'' is an image, or set of images in one figure, that
are placed after all author and affiliation information, and before
the body of the article, spanning the page. If you wish to have such a
figure in your article, place the command immediately before the
\verb|\maketitle| command:
\begin{verbatim}
  \begin{teaserfigure}
    \includegraphics[width=\textwidth]{sampleteaser}
    \caption{figure caption}
    \Description{figure description}
  \end{teaserfigure}
\end{verbatim}

\section{Citations and Bibliographies}

The use of \BibTeX\ for the preparation and formatting of one's
references is strongly recommended. Authors' names should be complete
--- use full first names (``Donald E. Knuth'') not initials
(``D. E. Knuth'') --- and the salient identifying features of a
reference should be included: title, year, volume, number, pages,
article DOI, etc.

The bibliography is included in your source document with these two
commands, placed just before the \verb|\end{document}| command:
\begin{verbatim}
  \bibliographystyle{ACM-Reference-Format}
  \bibliography{bibfile}
\end{verbatim}
where ``\verb|bibfile|'' is the name, without the ``\verb|.bib|''
suffix, of the \BibTeX\ file.

Citations and references are numbered by default. A small number of
ACM publications have citations and references formatted in the
``author year'' style; for these exceptions, please include this
command in the {\bfseries preamble} (before the command
``\verb|\begin{document}|'') of your \LaTeX\ source:
\begin{verbatim}
  \citestyle{acmauthoryear}
\end{verbatim}

  Some examples.  A paginated journal article \cite{Abril07}, an
  enumerated journal article \cite{Cohen07}, a reference to an entire
  issue \cite{JCohen96}, a monograph (whole book) \cite{Kosiur01}, a
  monograph/whole book in a series (see 2a in spec. document)
  \cite{Harel79}, a divisible-book such as an anthology or compilation
  \cite{Editor00} followed by the same example, however we only output
  the series if the volume number is given \cite{Editor00a} (so
  Editor00a's series should NOT be present since it has no vol. no.),
  a chapter in a divisible book \cite{Spector90}, a chapter in a
  divisible book in a series \cite{Douglass98}, a multi-volume work as
  book \cite{Knuth97}, a couple of articles in a proceedings (of a
  conference, symposium, workshop for example) (paginated proceedings
  article) \cite{Andler79, Hagerup1993}, a proceedings article with
  all possible elements \cite{Smith10}, an example of an enumerated
  proceedings article \cite{VanGundy07}, an informally published work
  \cite{Harel78}, a couple of preprints \cite{Bornmann2019,
    AnzarootPBM14}, a doctoral dissertation \cite{Clarkson85}, a
  master's thesis: \cite{anisi03}, an online document / world wide web
  resource \cite{Thornburg01, Ablamowicz07, Poker06}, a video game
  (Case 1) \cite{Obama08} and (Case 2) \cite{Novak03} and \cite{Lee05}
  and (Case 3) a patent \cite{JoeScientist001}, work accepted for
  publication \cite{rous08}, 'YYYYb'-test for prolific author
  \cite{SaeediMEJ10} and \cite{SaeediJETC10}. Other cites might
  contain 'duplicate' DOI and URLs (some SIAM articles)
  \cite{Kirschmer:2010:AEI:1958016.1958018}. Boris / Barbara Beeton:
  multi-volume works as books \cite{MR781536} and \cite{MR781537}. A
  couple of citations with DOIs:
  \cite{2004:ITE:1009386.1010128,Kirschmer:2010:AEI:1958016.1958018}. Online
  citations: \cite{TUGInstmem, Thornburg01, CTANacmart}. Artifacts:
  \cite{R} and \cite{UMassCitations}.

\section{Acknowledgments}

Identification of funding sources and other support, and thanks to
individuals and groups that assisted in the research and the
preparation of the work should be included in an acknowledgment
section, which is placed just before the reference section in your
document.

This section has a special environment:
\begin{verbatim}
  \begin{acks}
  ...
  \end{acks}
\end{verbatim}
so that the information contained therein can be more easily collected
during the article metadata extraction phase, and to ensure
consistency in the spelling of the section heading.

Authors should not prepare this section as a numbered or unnumbered {\verb|\section|}; please use the ``{\verb|acks|}'' environment.

\section{Appendices}

If your work needs an appendix, add it before the
``\verb|\end{document}|'' command at the conclusion of your source
document.

Start the appendix with the ``\verb|appendix|'' command:
\begin{verbatim}
  \appendix
\end{verbatim}
and note that in the appendix, sections are lettered, not
numbered. This document has two appendices, demonstrating the section
and subsection identification method.

\section{Multi-language papers}

Papers may be written in languages other than English or include
titles, subtitles, keywords and abstracts in different languages (as a
rule, a paper in a language other than English should include an
English title and an English abstract).  Use \verb|language=...| for
every language used in the paper.  The last language indicated is the
main language of the paper.  For example, a French paper with
additional titles and abstracts in English and German may start with
the following command
\begin{verbatim}
\documentclass[sigconf, language=english, language=german,
               language=french]{acmart}
\end{verbatim}

The title, subtitle, keywords and abstract will be typeset in the main
language of the paper.  The commands \verb|\translatedXXX|, \verb|XXX|
begin title, subtitle and keywords, can be used to set these elements
in the other languages.  The environment \verb|translatedabstract| is
used to set the translation of the abstract.  These commands and
environment have a mandatory first argument: the language of the
second argument.  See \verb|sample-sigconf-i13n.tex| file for examples
of their usage.

\section{SIGCHI Extended Abstracts}

The ``\verb|sigchi-a|'' template style (available only in \LaTeX\ and
not in Word) produces a landscape-orientation formatted article, with
a wide left margin. Three environments are available for use with the
``\verb|sigchi-a|'' template style, and produce formatted output in
the margin:
\begin{itemize}
\item {\verb|sidebar|}:  Place formatted text in the margin.
\item {\verb|marginfigure|}: Place a figure in the margin.
\item {\verb|margintable|}: Place a table in the margin.
\end{itemize}

%%
%% The acknowledgments section is defined using the "acks" environment
%% (and NOT an unnumbered section). This ensures the proper
%% identification of the section in the article metadata, and the
%% consistent spelling of the heading.
\begin{acks}
To Robert, for the bagels and explaining CMYK and color spaces.
\end{acks}

%%
%% The next two lines define the bibliography style to be used, and
%% the bibliography file.
\bibliographystyle{ACM-Reference-Format}
\bibliography{sample-base}

%%
%% If your work has an appendix, this is the place to put it.
\appendix

\section{Research Methods}

\subsection{Part One}

Lorem ipsum dolor sit amet, consectetur adipiscing elit. Morbi
malesuada, quam in pulvinar varius, metus nunc fermentum urna, id
sollicitudin purus odio sit amet enim. Aliquam ullamcorper eu ipsum
vel mollis. Curabitur quis dictum nisl. Phasellus vel semper risus, et
lacinia dolor. Integer ultricies commodo sem nec semper.

\subsection{Part Two}

Etiam commodo feugiat nisl pulvinar pellentesque. Etiam auctor sodales
ligula, non varius nibh pulvinar semper. Suspendisse nec lectus non
ipsum convallis congue hendrerit vitae sapien. Donec at laoreet
eros. Vivamus non purus placerat, scelerisque diam eu, cursus
ante. Etiam aliquam tortor auctor efficitur mattis.

\section{Online Resources}

Nam id fermentum dui. Suspendisse sagittis tortor a nulla mollis, in
pulvinar ex pretium. Sed interdum orci quis metus euismod, et sagittis
enim maximus. Vestibulum gravida massa ut felis suscipit
congue. Quisque mattis elit a risus ultrices commodo venenatis eget
dui. Etiam sagittis eleifend elementum.

Nam interdum magna at lectus dignissim, ac dignissim lorem
rhoncus. Maecenas eu arcu ac neque placerat aliquam. Nunc pulvinar
massa et mattis lacinia.

\section{Introduction}

Large language models (LLMs) have achieved remarkable success in automated math problem solving, particularly through code-generation capabilities integrated with proof assistants~\citep{lean,isabelle,POT,autoformalization,MATH}. Although LLMs excel at generating solution steps and correct answers in algebra and calculus~\citep{math_solving}, their unimodal nature limits performance in plane geometry, where solution depends on both diagram and text~\citep{math_solving}. 

Specialized vision-language models (VLMs) have accordingly been developed for plane geometry problem solving (PGPS)~\citep{geoqa,unigeo,intergps,pgps,GOLD,LANS,geox}. Yet, it remains unclear whether these models genuinely leverage diagrams or rely almost exclusively on textual features. This ambiguity arises because existing PGPS datasets typically embed sufficient geometric details within problem statements, potentially making the vision encoder unnecessary~\citep{GOLD}. \cref{fig:pgps_examples} illustrates example questions from GeoQA and PGPS9K, where solutions can be derived without referencing the diagrams.

\begin{figure}
    \centering
    \begin{subfigure}[t]{.49\linewidth}
        \centering
        \includegraphics[width=\linewidth]{latex/figures/images/geoqa_example.pdf}
        \caption{GeoQA}
        \label{fig:geoqa_example}
    \end{subfigure}
    \begin{subfigure}[t]{.48\linewidth}
        \centering
        \includegraphics[width=\linewidth]{latex/figures/images/pgps_example.pdf}
        \caption{PGPS9K}
        \label{fig:pgps9k_example}
    \end{subfigure}
    \caption{
    Examples of diagram-caption pairs and their solution steps written in formal languages from GeoQA and PGPS9k datasets. In the problem description, the visual geometric premises and numerical variables are highlighted in green and red, respectively. A significant difference in the style of the diagram and formal language can be observable. %, along with the differences in formal languages supported by the corresponding datasets.
    \label{fig:pgps_examples}
    }
\end{figure}



We propose a new benchmark created via a synthetic data engine, which systematically evaluates the ability of VLM vision encoders to recognize geometric premises. Our empirical findings reveal that previously suggested self-supervised learning (SSL) approaches, e.g., vector quantized variataional auto-encoder (VQ-VAE)~\citep{unimath} and masked auto-encoder (MAE)~\citep{scagps,geox}, and widely adopted encoders, e.g., OpenCLIP~\citep{clip} and DinoV2~\citep{dinov2}, struggle to detect geometric features such as perpendicularity and degrees. 

To this end, we propose \geoclip{}, a model pre-trained on a large corpus of synthetic diagram–caption pairs. By varying diagram styles (e.g., color, font size, resolution, line width), \geoclip{} learns robust geometric representations and outperforms prior SSL-based methods on our benchmark. Building on \geoclip{}, we introduce a few-shot domain adaptation technique that efficiently transfers the recognition ability to real-world diagrams. We further combine this domain-adapted GeoCLIP with an LLM, forming a domain-agnostic VLM for solving PGPS tasks in MathVerse~\citep{mathverse}. 
%To accommodate diverse diagram styles and solution formats, we unify the solution program languages across multiple PGPS datasets, ensuring comprehensive evaluation. 

In our experiments on MathVerse~\citep{mathverse}, which encompasses diverse plane geometry tasks and diagram styles, our VLM with a domain-adapted \geoclip{} consistently outperforms both task-specific PGPS models and generalist VLMs. 
% In particular, it achieves higher accuracy on tasks requiring geometric-feature recognition, even when critical numerical measurements are moved from text to diagrams. 
Ablation studies confirm the effectiveness of our domain adaptation strategy, showing improvements in optical character recognition (OCR)-based tasks and robust diagram embeddings across different styles. 
% By unifying the solution program languages of existing datasets and incorporating OCR capability, we enable a single VLM, named \geovlm{}, to handle a broad class of plane geometry problems.

% Contributions
We summarize the contributions as follows:
We propose a novel benchmark for systematically assessing how well vision encoders recognize geometric premises in plane geometry diagrams~(\cref{sec:visual_feature}); We introduce \geoclip{}, a vision encoder capable of accurately detecting visual geometric premises~(\cref{sec:geoclip}), and a few-shot domain adaptation technique that efficiently transfers this capability across different diagram styles (\cref{sec:domain_adaptation});
We show that our VLM, incorporating domain-adapted GeoCLIP, surpasses existing specialized PGPS VLMs and generalist VLMs on the MathVerse benchmark~(\cref{sec:experiments}) and effectively interprets diverse diagram styles~(\cref{sec:abl}).

\iffalse
\begin{itemize}
    \item We propose a novel benchmark for systematically assessing how well vision encoders recognize geometric premises, e.g., perpendicularity and angle measures, in plane geometry diagrams.
	\item We introduce \geoclip{}, a vision encoder capable of accurately detecting visual geometric premises, and a few-shot domain adaptation technique that efficiently transfers this capability across different diagram styles.
	\item We show that our final VLM, incorporating GeoCLIP-DA, effectively interprets diverse diagram styles and achieves state-of-the-art performance on the MathVerse benchmark, surpassing existing specialized PGPS models and generalist VLM models.
\end{itemize}
\fi

\iffalse

Large language models (LLMs) have made significant strides in automated math word problem solving. In particular, their code-generation capabilities combined with proof assistants~\citep{lean,isabelle} help minimize computational errors~\citep{POT}, improve solution precision~\citep{autoformalization}, and offer rigorous feedback and evaluation~\citep{MATH}. Although LLMs excel in generating solution steps and correct answers for algebra and calculus~\citep{math_solving}, their uni-modal nature limits performance in domains like plane geometry, where both diagrams and text are vital.

Plane geometry problem solving (PGPS) tasks typically include diagrams and textual descriptions, requiring solvers to interpret premises from both sources. To facilitate automated solutions for these problems, several studies have introduced formal languages tailored for plane geometry to represent solution steps as a program with training datasets composed of diagrams, textual descriptions, and solution programs~\citep{geoqa,unigeo,intergps,pgps}. Building on these datasets, a number of PGPS specialized vision-language models (VLMs) have been developed so far~\citep{GOLD, LANS, geox}.

Most existing VLMs, however, fail to use diagrams when solving geometry problems. Well-known PGPS datasets such as GeoQA~\citep{geoqa}, UniGeo~\citep{unigeo}, and PGPS9K~\citep{pgps}, can be solved without accessing diagrams, as their problem descriptions often contain all geometric information. \cref{fig:pgps_examples} shows an example from GeoQA and PGPS9K datasets, where one can deduce the solution steps without knowing the diagrams. 
As a result, models trained on these datasets rely almost exclusively on textual information, leaving the vision encoder under-utilized~\citep{GOLD}. 
Consequently, the VLMs trained on these datasets cannot solve the plane geometry problem when necessary geometric properties or relations are excluded from the problem statement.

Some studies seek to enhance the recognition of geometric premises from a diagram by directly predicting the premises from the diagram~\citep{GOLD, intergps} or as an auxiliary task for vision encoders~\citep{geoqa,geoqa-plus}. However, these approaches remain highly domain-specific because the labels for training are difficult to obtain, thus limiting generalization across different domains. While self-supervised learning (SSL) methods that depend exclusively on geometric diagrams, e.g., vector quantized variational auto-encoder (VQ-VAE)~\citep{unimath} and masked auto-encoder (MAE)~\citep{scagps,geox}, have also been explored, the effectiveness of the SSL approaches on recognizing geometric features has not been thoroughly investigated.

We introduce a benchmark constructed with a synthetic data engine to evaluate the effectiveness of SSL approaches in recognizing geometric premises from diagrams. Our empirical results with the proposed benchmark show that the vision encoders trained with SSL methods fail to capture visual \geofeat{}s such as perpendicularity between two lines and angle measure.
Furthermore, we find that the pre-trained vision encoders often used in general-purpose VLMs, e.g., OpenCLIP~\citep{clip} and DinoV2~\citep{dinov2}, fail to recognize geometric premises from diagrams.

To improve the vision encoder for PGPS, we propose \geoclip{}, a model trained with a massive amount of diagram-caption pairs.
Since the amount of diagram-caption pairs in existing benchmarks is often limited, we develop a plane diagram generator that can randomly sample plane geometry problems with the help of existing proof assistant~\citep{alphageometry}.
To make \geoclip{} robust against different styles, we vary the visual properties of diagrams, such as color, font size, resolution, and line width.
We show that \geoclip{} performs better than the other SSL approaches and commonly used vision encoders on the newly proposed benchmark.

Another major challenge in PGPS is developing a domain-agnostic VLM capable of handling multiple PGPS benchmarks. As shown in \cref{fig:pgps_examples}, the main difficulties arise from variations in diagram styles. 
To address the issue, we propose a few-shot domain adaptation technique for \geoclip{} which transfers its visual \geofeat{} perception from the synthetic diagrams to the real-world diagrams efficiently. 

We study the efficacy of the domain adapted \geoclip{} on PGPS when equipped with the language model. To be specific, we compare the VLM with the previous PGPS models on MathVerse~\citep{mathverse}, which is designed to evaluate both the PGPS and visual \geofeat{} perception performance on various domains.
While previous PGPS models are inapplicable to certain types of MathVerse problems, we modify the prediction target and unify the solution program languages of the existing PGPS training data to make our VLM applicable to all types of MathVerse problems.
Results on MathVerse demonstrate that our VLM more effectively integrates diagrammatic information and remains robust under conditions of various diagram styles.

\begin{itemize}
    \item We propose a benchmark to measure the visual \geofeat{} recognition performance of different vision encoders.
    % \item \sh{We introduce geometric CLIP (\geoclip{} and train the VLM equipped with \geoclip{} to predict both solution steps and the numerical measurements of the problem.}
    \item We introduce \geoclip{}, a vision encoder which can accurately recognize visual \geofeat{}s and a few-shot domain adaptation technique which can transfer such ability to different domains efficiently. 
    % \item \sh{We develop our final PGPS model, \geovlm{}, by adapting \geoclip{} to different domains and training with unified languages of solution program data.}
    % We develop a domain-agnostic VLM, namely \geovlm{}, by applying a simple yet effective domain adaptation method to \geoclip{} and training on the refined training data.
    \item We demonstrate our VLM equipped with GeoCLIP-DA effectively interprets diverse diagram styles, achieving superior performance on MathVerse compared to the existing PGPS models.
\end{itemize}

\fi 



\section{Related Work}
\label{sec:RelatedWork}

Within the realm of geophysical sciences, super-resolution/downscaling is a challenge that scientists continue to tackle. There have been several works involved in downscaling applications such as river mapping \cite{Yin2022}, coastal risk assessment \cite{Rucker2021}, estimating soil moisture from remotely sensed images \cite{Peng2017SoilMoisture} and downscaling of satellite based precipitation estimates \cite{Medrano2023PrecipitationDownscaling} to name a few. We direct the reader to \cite{Karwowska2022SuperResolutionSurvey} for a comprehensive review of satellite based downscaling applications and methods. Pertaining to our objective of downscaling \acp{WFM}, we can draw comparisons with several existing works. 
In what follows, we provide a brief review of functionally adjacent works to contrast the novelty of our proposed model and its role in addressing gaps in literature. 

When it comes to downscaling \ac{WFM}, several works use statistical downscaling techniques. These works downscale images by using statistical techniques that utilize relationships between neighboring water fraction pixels. For instance, \cite{Li2015SRFIM} treat the super-resolution task as a sub-pixel mapping problem, wherein the input fraction of inundated pixels must be exactly mapped to the output patch of inundated pixels. 
% In doing so, they are able to apply a discrete particle swarm optimization method to maximize the Flood Inundation Spatial Dependence Index (FISDI). 
\cite{Wang2019} improved upon these approaches by including a spectral term to fully utilize spectral information from multi spectral remote sensing image band. \cite{Wang2021} on the other hand also include a spectral correlation term to reduce the influence of linear and non-linear imaging conditions. All of these approaches are applied to water fraction obtained via spectral unmixing \cite{wang2013SpectralUnmixing} and are designed to work with multi spectral information from MODIS. However, we develop our model with the intention to be used with water fractions directly derived from the output of satellites. One such example is NOAA/VIIRS whose water fraction extraction method is described in \cite{Li2013VIIRSWFM}. \cite{Li2022VIIRSDownscaling} presented a work wherein \ac{WFM} at 375-m flood products from VIIRS were downscaled 30-m flood event and depth products by expressing the inundation mechanism as a function of the \ac{DEM}-based water area and the VIIRS water area.

On the other hand, the non-linear nature of the mapping task lends itself to the use of neural networks. Several models have been adapted from traditional single image digital super-resolution in computer vision literature \cite{sdraka2022DL4downscalingRemoteSensing}. Existing deep learning models in single image super-resolution are primarily dominated by \ac{CNN} based models. Specifically, there has been an upward trend in residual learning models. \acp{RDN} \cite{Zhang2018ResidualDenseSuperResolution} introduced residual dense blocks that employed a contiguous memory mechanism that aimed to overcome the inability of very deep \acp{CNN} to make full use of hierarchical features. 
\acp{RCAN} \cite{Zhang2018RCANSuperResolution} introduced an attention mechanism to exploit the inter-channel dependencies in the intermediate feature transformations. There have also been some works that aim to produce more lightweight \ac{CNN}-based architectures \cite{Zheng2019IMDN,Xiaotong2020LatticeNET}. Since the introduction of the vision transformer \cite{Vaswani2017Attention} that utilized the self-attention mechanism -- originally used for modeling text sequences -- by feeding a sequence 2D sub-image extracted from the original image. Using this approach \cite{LuESRT2022} developed a light-weight and efficient transformer based approach for single image super-resolution. 


For the task of super-resolution of \acp{WFM}, we discuss some works whose methodology is similar to ours even though they differ in their problem setting. \cite{Yin2022} presented a cascaded spectral spatial model for super-resolution of MODIS imagery with a scaling factor 10. Their architecture consists of two stages; first multi-spectral MODIS imagery is converted into a low-resolution \ac{WFM} via spectral unmixing by passing it through a deep stacked residual \ac{CNN}. The second stage involved the super-resolution mapping of these \acp{WFM} using a nested multi-level \ac{CNN} model. Similar to our work, the input fraction images are obtained with zero errors which may not be reflective of reality since there tends to be sensor noise, the spatial distribution of whom cannot be easily estimated. We also note that none of these works directly tackle flood inundation since they've been trained with river map data during non-flood circumstance and \textit{ergo} do not face a data scarcity problem as we do. 
% In this work, apart from the final product of \acp{WFM}, we are not presented with any additional spectral information about the low resolution image. This was intended to work directly with products that can generate \ac{WFM} either directly (VIIRS) or indirectly (Landsat).
\cite{Jia2019} used a deep \ac{CNN} for land mapping that consists of several classes such as building, low vegetation, background and trees. 
\cite{Kumar2021} similarly employ a \ac{CNN} based model for downscaling of summer monsoon rainfall data over the Indian subcontinent. Their proposed Super-Resolution Convolutional Neural Network (SRCNN) has a downscaling factor of 4. 
\cite{Shang2022} on the other hand, proposed a super-resolution mapping technique using Generative Adversarial Networks (GANs). They first generate high resolution fractional images, somewhat analogous to our \ac{WFM}, and are then mapped to categorical land cover maps involving forest, urban, agriculture and water classes. 
\cite{Qin2020} interestingly approach lake area super-resolution for Landsat and MODIS data as an unsupervised problem using a \ac{CNN} and are able to extend to other scaling factors. \cite{AristizabalInundationMapping2020} performed flood inundation mapping using \ac{SAR} data obtained from Sentinel-1. They showed that \ac{DEM}-based features helped to improve \ac{SAR}-based predictions for quadratic discriminant analysis, support vector machines and k-nearest neighbor classifiers. While almost all of the aforementioned works can be adapted to our task. We stand out in the following ways (i) We focus on downscaling of \acp{WFM} directly, \textit{i.e.,} we do not focus on the algorithm to compute the \ac{WFM} from multi-channel satellite data and (ii) We focus on producing high resolution maps only for instances of flood inundation. The latter point produces a data scarcity issue which we seek to remedy with synthetic data. 


%%%%%%%%%%%%%%%%% Additional unused information %%%%%%%%%%%%%%%%


%     \item[\cite{Wang2021}] Super-Resolution Mapping Based on Spatial–Spectral Correlation for Spectral Imagery
%     \begin{itemize}
%         \item Not a deep neural network approach. SRM based on spatial–spectral correlation (SSC) is proposed in order to overcome the influence of linear and nonlinear imaging conditions and utilize more accurate spectral properties.
%         \item (fig 1) there are two main SRM types: (1) the initialization-then-optimization SRM, where the class labels are allocated randomly to subpixels, and the location of each subpixel is optimized to obtain the final SRM result. and (2)soft-then-hard SRM, which involves two steps: the subpixel sharpening and the class allocation.  
%         \item SSC procedures: (1) spatial correlation is performed by the MSAM to reduce the influences of linear imaging conditions on image quality. (2) A spectral correlation that utilizes spectral properties based on the nonlinear KLD is proposed to reduce the influences of nonlinear imaging conditions. (3) spatial and spectral correlations are then combined to obtain an optimization function with improved linear and nonlinear performances. And finally (4) by maximizing the optimization function, a class allocation method based on the SA is used to assign LC labels to each subpixel, obtaining the final SRM result.
%         \item (Comparable) 
%     \end{itemize}
%     %--------------------------------------------------------------------
% \cite{Wang2021} account for the influence of linear and non-linear imaging conditions by involving more accurate spectral properties. 
%     %--------------------------------------------------------------------
%     \item[\cite{Yin2022}] A Cascaded Spectral–Spatial CNN Model for Super-Resolution River Mapping With MODIS Imagery
%     \begin{itemize}
%         \item produce  Landsat-like  fine-resolution (scale of 10)  river  maps  from  MODIS images. Notice the original coarse-resolution remotely sensed images, not the river fraction images.
%         \item combined  CNN  model that  contains  a spectral  unmixing  module  and  an  SRM  module, and the SRM module is made up of an encoder and a decoder that are connected through a series of convolutional blocks. 
%         \item With an adaptive cross-entropy loss function to address class imbalance.	
%         \item The overall accuracy, the omission error, the  commission  error,  and  the  mean  intersection  over  union (MIOU)  calculated  to  assess  the results.
%         \item partially comparable with ours, only the SRM module part
%     %--------------------------------------------------------------------

% To decouple the description of the objective and the \ac{ML} model architecture, the motivation for the model architecture is described in \secref{sec:Methodology}. 


%     \item[\cite{Wang2019}] Improving Super-Resolution Flood Inundation Mapping for Multi spectral Remote Sensing Image by Supplying More Spectral 
%     \begin{itemize}
%         \item proposed the SRFIM-MSI,where a new spectral term is added to the traditional SRFIM to fully utilize the spectral information from multi spectral remote sensing image band. 
%         \item The original SRFIM \cite{Huang2014, Li2015} obtains the sub pixel spatial distribution of flood inundation within mixed pixels by maximizing their spatial correlation while maintaining the original proportions of flood inundation within the mixed pixels. The SRFIM is formulated as a maximum combined optimization issue according to the principle of spatial correlation.
%         \item follow the terminology in \cite{Wang2021}, this is an initialization-then-optimization SRM. 
%         \item (Comparable) 
%     \end{itemize}
%     %--------------------------------------------------------------------


%--------------------------------------------------------------------
%     \item[\cite{Jia2019}] Super-Resolution Land Cover Mapping Based on the Convolutional Neural Network
%     \begin{itemize}
%         \item SRMCNN (Super-resolution mapping CNN) is proposed to obtain fine-scale land cover maps from coarse remote sensing images. Specifically, an encoder-decoder CNN is used to determined the labels (i.e., land cover classes) of the subpixels within mixed pixels.
%         \item There were three main parts in SRMCNN. The first part was a three-sequential convolutional layer with ReLU and pooling. The second part is up-sampling, for which a multi transposed-convolutional layer was adopted. To keep the feature learned in the previous layer, a skip connection was used to concatenate the output of the corresponding convolution layer. The last part was the softmax classifier, in which the feature in the antepenultimate layer was classified and class probabilities are obtained.
%         \item The loss: the optimal allocation of classes to the subpixels of mixed pixel is achieved by maximizing the spatial dependence between neighbor pixels under constraint that the class proportions within the mixed pixels are preserved.
%         \item (Preferred), this paper is designed to classify background, Building, Low Vegetation, or Tree in the land. But we can easily adapt to our problem and should compare with this paper.
%     \end{itemize}
%     %--------------------------------------------------------------------

%     \item[\cite{Kumar2021}] Deep learning–based downscaling of summer monsoon rainfall data over Indian region
%     \begin{itemize}
%         \item down-scaling (scale of 4) rainfall data. The output image is not binary image.
%         \item three algorithms: SRCNN, stacked SRCNN, and DeepSD are employed, based on \cite{Vandal2019}
%         \item mean square error and pattern correlation coefficient are used as evaluation metrics.
%         \item SRCNN: super-resolution-based convolutional neural networks (SRCNN) first upgrades the low-resolution image to the higher resolution size by using bicubic interpolation. Suppose the interpolated image is referred to as Y; SRCNNs’task is to retrieve from Y an image F(Y) which is close to the high-resolution ground truth image X.
%         \item stacked SRCNN: stack 2 or more SRCNN blocks to increasing the scaling factor.
%         \item DeepSD: uses topographies as an additional input to stacked SRCNN.
%         \item These algorithms are not designed for binary output images, but if prefer, the ``modified'' stacked SRCNN or DeepSD can be used as baseline algorithms.
%     
%     \item[\cite{Shang2022}] Super resolution Land Cover Mapping Using a Generative Adversarial Network
%     \begin{itemize}
%         \item propose an end-to-end SRM model based on a generative adversarial network (GAN), that is, GAN-SRM, to improve the two-step learning-based SRM methods. 
%         \item Two-step SRM method: The first step is fraction-image super-resolution (SR), which reconstructs a high-spatial-resolution fraction image from the low input, methods like SVR, or CNN has been widely adopted. The second step is converting the high-resolution fraction images to a categorical land cover map, such as with a soft-max function to assign each high-resolution pixel to a unique category value.
%         \item The proposed GAN-SRM model includes a generative network and a discriminative network, so that both the fraction-image SR and the conversion of the fraction images to categorical map steps are fully integrated to reduce the resultant uncertainty. 
%         \item applied to the National Land Cover Database (NLCD), which categorized land into four typical classes:forest, urban, agriculture,and water. scale factor of 8. 
%         \item (Preferred), we should compare with this work.
%     \end{itemize}
%     %--------------------------------------------------------------------

%   \item[\cite{Qin2020}] Achieving Higher Resolution Lake Area from Remote Sensing Images Through an Unsupervised Deep Learning Super-Resolution Method
%   \begin{itemize}
%       \item propose an unsupervised deep gradient network (UDGN) to generate a higher resolution lake area from remote sensing images.
%       \item UDGN models the internal recurrence of information inside the single image and its corresponding gradient map to generate images with higher spatial resolution. 
%       \item A single image super-resolution approach, not comparable
%   \end{itemize}
%     %--------------------------------------------------------------------




%     \item[\cite{Demiray2021}] D-SRGAN: DEM Super-Resolution with Generative Adversarial Networks
%     \begin{itemize}
%         \item A GAN based model is proposed to increase the spatial resolution of a given DEM dataset up to 4 times without additional information related to data.
%         \item Rather than processing each image in a sequence independently, our generator architecture uses a recurrent layer to update the state of the high-resolution reconstruction in a manner that is consistent with both the previous state and the newly received data. The recurrent layer can thus be understood as performing a Bayesian update on the ensemble member, resembling an ensemble Kalman filter. 
%         \item A single image super-resolution approach, not comparable
%     \end{itemize}
%     %--------------------------------------------------------------------
%     \item[\cite{Leinonen2021}] Stochastic Super-Resolution for Downscaling Time-Evolving Atmospheric Fields With a Generative Adversarial Network
%     \begin{itemize}
%         \item propose a super-resolution GAN that operates on sequences of two-dimensional images and creates an ensemble of predictions for each input. The spread between the ensemble members represents the uncertainty of the super-resolution reconstruction.
%         \item for sequence of input images, not comparable with ours.
%     \end{itemize} 
%     %--------------------------------------------------------------------

% \end{itemize}





\begin{comment}
- Sketch / Cartoon Animation

- MagicToon: A 2D-to-3D creative cartoon modeling system with mobile AR
    Creates a 3D model from a 2D drawing. Ours keeps it 2D but applies 3D motion onto it.
- Towards automatic drawing animation using physics-based evolution
    - purely 2D output, no ability to specify motion
- Smith et al.
    - output not well suited for 3D, not view dependent.
- monster mash
- hornung
- photo wakeup

    
- View-Dependent Works
A System for View-Dependent Animation
- Focus on traditionally trained animators, attempts to go from 2D pose inputs to 3D characters. We focus on the opposite and on amateurs not trained animators. In addition, we provide an integrated retargeting technique for accessible animation of the characters specifically suited for childlike drawings.
- 
    - Similar but it requires less input from users since we use insights from childlike drawings to make reasonable initial guesses. In addition, our model was build to specifically support manipulation based not just upon view angle but also input motion with minimal user input.
- Rademacher
- Rivers
- Rod Springs

- Interactive Viewpoint Exploration for Constructing
View-Dependent Models
    - method for finding viewpoints focusing on creating keyview-transform pairs for a 3D model.


\end{comment}

\section{Orientation Depiction in Childlike Drawings}
%\nicky {Better wording for the title? A little unclear what it means}
Because our goal is to manipulate the figure in a style-preserving manner, we provide relevant insights from the fields of child psychology and art analysis about the factors which give rise to childlike drawing style. 
We also present the results of a novel annotation study exploring how observers infer the orientation of childlike drawing figures.



\subsection{How Children Draw Figures}

By childlike drawing, we refer to the quirky, representational style of drawing that tends to appear between ages three and seven~\cite{lowenfeld1975creative}. 
Though it falls out of use shortly thereafter, this style of drawing remains available to people of all ages, including adults who `cannot draw.'
This type of drawing is characterized by `intellectual realism,' a denotation system distinct from the 'visual realism' used in photographs and most 3D graphics scenes~\cite{luquet2001children}.
Such drawings occur when the drawer leverages an alternative definition of `realistic:' rather than drawing a subject as it appears when viewed from a specific vantage, they instead aims to include details important to their internal model of the subject.
Said differently, when intellectual realism is used, be it by a child or Picasso~\cite{picasso1937}, the aim is to create a recognizable representation of the subject- not necessarily a visually accurate one.


%This style of drawing often disappears shortly after, either because it is replaced by more visually realistic drawings or because the child ceases drawing altogether.
%Nevertheless, it remains available to virtually everyone, including adults who, by their own admission, `cannot draw.' 
%\yuting{Can we claim that understanding perspective is a development milestone and a distinction of armature vs professional?} \jesse{I think we could say that using perspective realistically is drawings a technical milestone that requires a great deal of practice. Is that close to what you are suggestion?}\yuting{I want to see if we can claim our method is not only useful for children between 3-7 (very limited scope) but for all amateur drawings.}\jesse{Yes, I think that's important. I was trying to make that point with the previous line about adults who cannot draw. I could be even more explicit: ".... 'cannot draw.' Therefore, tools which build upon childlike drawings have the potential to empower not only children, but amateur drawers of all ages.}



%The philosopher Georges-Henri Luquet described such drawings as employing 'intellectual realism,' a denotation system distinct from the 'visual realism' used in photographs and most 3D graphics scenes~\cite{luquet2001children}.
%According to Luquet, such drawings occur because the child has an alternative definition of `realism:' rather than drawing a subject from a specific vantage, the child instead aims to include the features and details that are important to their internal model of the subject.
%Said differently, when intellectual realism is used, be it by a child, Picasso~\cite{picasso1937}, or a prehistoric cave artist~cite{cave-tito-bustillo}, the aim is to create a recognizable representation of the subject that does not need to be visually accurate.

Others have argued that a key to defining childlike drawing style is understanding that the fundamental primitive used is the \textit{region}, a 2D area (e.g. a dot, single line, outlined area, or scribble cluster) whose attributes are perceptually similar to those of the subject it represents~\cite{willats2006making}. Childlike drawings can therefore be seen as a collection of regions arranged so as to convey the \textit{idea} or \textit{important details} of something in a non-visually realistic manner. 

This influences figure drawings in several ways.
First, foreshortening is rare, as long regions, such as arms, are less recognizable when foreshortened.~\cite{willats1992representation,piaget1956}.
Second, different parts of the figure are drawn so their most recognizable traits are apparent; this often results in \textit{twisted perspective}: different parts of the figure drawn as though viewed from different angles (e.g. feet in Figure~\ref{fig:part_trait_example}.a). 
Third, the majority of figures are drawn in a forward-facing 'canonical view'~\cite{goodnow1977children,cox2014drawings}, but may be drawn in profile if their orientation is important for some reason~\cite{cox1993children}.
One study found that, when drawing figures in non-forward-facing views, they omitted facial features when drawing the back side. For side views, children manipulated facial features, head contouring, obscured limbs behind each other, and pointed feet to the same side.


%The focus on recognition has additional implications for how human figure orientation is represented.
%The majority of figures are drawn in the forward-facing 'canonical view'~\cite{goodnow1977children,cox2014drawings}, but may be drawn in profile if a particular action, such as running, is important to convey~\cite{cox1993children}.
%One study specifically investigated how children alter their human figure drawings based upon the subject's orientation~\cite{cox1994children}. 
%They found that, in general, facial features were omitted when figures faced away. When drawing a subject in profile, children manipulated facial features, head contouring, obscured limbs behind each other, and pointed feet to the same side.

\subsection{How Observers Infer Figure Orientations}
\label{sec:orientation}

While it is useful to understand why children draw as they do, our aim is to modify existing figures such that the 3D pose is recognizable to an observer. Therefore, it is important to understand how observers interpret figure orientation: whether the character's forward vector extends to the left, the right, or is centered (i.e. extends directly out of the drawing). We therefore conducted an annotation study to answer the following questions:
\begin{itemize}
  \item Q1 - To what degree do observers agree on the orientation of childlike figure drawings?
  \item Q2 - What body parts and part traits do observers use when inferring orientation?
  \item Q3 - How frequently is twisted perspective present in these drawings?
\end{itemize}





\begin{figure*}[ht]
\centering
\includegraphics[width=0.9\textwidth]{figures/part_traits_short.png}
\caption{Frequency with which annotators references specific body parts and specified part traits when determining figure orientation.}
\label{fig:part_count_orientation}
\end{figure*}

\begin{table}
  \centering
  \begin{tabular}{|c|c|}
    \hline
    Per Figure Segment Orientations & Count \\
    \hline
    Center Only & 108 (27\%)\\
    \hline
    Left Only & 4 (1\%)\\
    \hline
    Right Only & 5 (1\%)\\
    \hline
    Center, Right & 137 (34\%)\\
    \hline
    Center, Left & 124 (31\%)\\
    \hline
    Right, Left & 0 (0\%)\\
    \hline
    Right, Center, Left & 21 (5\%)\\
    \hline
  \end{tabular}
  \caption{Counts of how frequently different combinations of orientations were present in the same figure.}
  \label{table:twisted_perpspective_frequency}
\end{table}


%\subsubsection{Discussion} \yuting{Results/Conclusion?}
Please refer to Appendix 1 for methodology and results tables.
Annotator agreement was substantial, indicating that observers reliably come to similar conclusions when assessing figure drawing orientations.
Figure \ref{fig:part_count_orientation} shows the frequency with which annotators used different body parts to infer orientation, along with the proportion of the times that specific part traits were mentioned.
Unsurprisingly, facial features and feet provided the majority of the orientations cues. 
There is a long tail of different body parts and accessories that could be used, depending upon what the drawer chose to include on the character.
Interestingly, only a few part traits were used the vast majority of the time.
Some parts were translated around an enclosing region, such as eyes within the face.
Other parts, such as noses, often pointed towards one direction or another.
At times the contour of a region, such as the head, extended further towards one side, suggesting orientation.
When arms and legs were used to infer orientation, they were posed such as might be seen when viewed from a particular direction. For examples of each of these, see Figure~\ref{fig:part_trait_example}.

A chart showing the frequency of twisted perspective can be seen in Table~\ref{table:twisted_perpspective_frequency}.
As expected, twisted perspective was seen frequently: 71\% of figure were composed of segments with multiple orientations.
Interestingly, at least one segment was perceived as forward facing 92\% of the time, and left-facing and right-facing segments were mixed together only 5\% of the time.

Taken together, these results suggest the following for system design:
first, it's possible to manipulate existing figure drawings to reliably influence their perceived orientation.
Second, this can be done by applying a relatively small number of manipulation types to different body parts.
Third, while it is common and acceptable to mix center-facing and consistent side-facing cues, mixing left-facing and right-facing cues should be avoided.

%From the previous section, we conclude the following: 
%It is possible to modify these characters so as to reliably influence the perceived orientation of the character.
%The system should can twist perspectives, as long as left right orientation cues are not mixed together.
%The system can be effective while limiting itself to translating regions, flipping regions horizontally, and deforming the limbs such that their poses suggest they are being viewed from a particular angle.
%As the limbs are deformed they should not introduce foreshortening



%We provide the percentage of the time observers agreed.
%To account for by-change agreement, we also compute Fleiss' kappa.

%percentage of agreement between observers.
%Overall, observers agreed on orientation 86\% of the time, with a Fleiss' kappa value of 0.73, suggesting significant agreement.


%For heads/faces, percentage agreement was 77\% with Fleiss' kappa of 0.65.
%For upper arms/torso, percentage agreement was 90\% with Fleiss' kappa of 


%We calculate the percentage of agreement of o

%To what degree do observers agree on the orientation of childlike figure drawings?

%%traits which  childlike drawings often haeve. 

%specific stylisticoutcomes of 


%In the context of drawings of the human figure, this commonly results in a few different 'oddities':
%1. Foreshortening is quite rare in children's drawings, as a long region, such as an arm, appears as a short region once forshortening, and is less recognizable as something that is long.
%2. Each region is often drawn so as to have it's most recognizable features visible. This often results in different regions being drawn from different perspectives, referred to as 'twisted perspective' by the research community. In drawings of the human figure in the canonical form, feet often points to the left or right while the character otherwise faces straight ahead.




%% it remains available and indeed is often used by adults without formal drawing training. This peculiar style of drawing has been of great interest to various child-focused research communities. 






\begin{comment}
"""
It would be understandable to treat childlike drawings as 2D images of 3D geometry that can be reconstructed and used just as any other 3D object. It would also be understandable to conflate their representation, which is 2D, with the character itself being a 2D character, which should then be moved around in 3D space like any other 2D manifold.

But we argue that childlike drawings of figures are neither 2D nor 3D. They are something else entirely, which we refer to as 'view-dependent intellectually realistic' representations.

We note that these drawings view-dependent in the sense that they are created with a specific viewing angle in mind- Namely, one at or near the normal vector to the page upon which is it drawn. Such a viewing provides the fewest distortions, and people naturally view pages from such an angle. Drawing things that are meant to be viewed from other angles, which possible, requires a significant amount of skill (ambassadors) and is generally outside the abilities of childlike drawers.

Though this seem obvious, when stated, it is important to keep in mind as we consider how to manipulate the character. When asked to draw the same characters in profile, studies have shown kids do not draw thin slits, as would appear if the paper were rotated 90 degrees in 3D space. Rather, they draw similar character which slightly modified feature layouts. The character's representation has changed significantly based upon the orientation of the character and the assumed viewing direction.

We therefore create a system which is view-dependent, violating assumptions of tradition 3D geometry, but in doing so attempt to preserve the inherent style of these drawings.

Secondly, the drawings are not visually realistic, so creating a 3D model that can be manipulated in a visually realistic manner also violates the style.
We refer to these drawings as 'intellectually realistic' to differentiate from 'visually realistic', and we borrow the term from Luquet, the child art psychologoist <definition of intellectually realistic>.





"""

- Intellectual Realism vs. visual realism overview

- Studies from children's literature about how they draw objects

- Studies from children's literature about how they draw side and back objects.

- We are unaware of any such studies which focus on amateur drawers of all ages, so we perform annotation work with Amateur Drawings Dataset.

- Task Setup
    - How we got the images
    - instructions for annotators
        - Why we did per-part (twisted perspective)
    - results
        - Most frequently mentioned body parts and representational elements
        - Degree of agreement among raters
        - Proportion of drawings that had orientation, that had twisted perspective
        - Proportion of drawings that had conflicting orientations for different parts (part was oriented left, part oriented right). This could be a way to justify the need to avoid conflicting orientations.

- Based upon this, we selected the following bodypart-representation combos to add into our system.


\section{Annotation Study}
- H1- people will conclude similar things about orietnation (user agreement)

- H2 - A relatively small number of features/representations will be relevant to most character orientations (freecoding)

- H3 Twisted perspective will be present in a significant portion of the drawings.
(Show that people dont't really mix left and right orientations together though)


\end{comment}

We applied Recurrency Sequence Processing to address the lack of consistency in the coarse dance representation of the~\cite{li2024lodge} model. We named this Recurrency Sequence Representation Learning as Dance Recalibration (DR). Dance recalibration uses \(n\) Dance Recalibration Blocks (DRB) corresponding to the length of the rough dance sequence to add sequential information to the rough dance representation to improve the consistency of the whole dance. The overall structure of our model is illustrated in Figure 1.

\begin{figure}[!t]
    \centering
    \includegraphics[width=\textwidth]{Figure1.eps}
    \caption{overall procedure of Pooling processing by our Pooling Block}
    \label{fig:enter-label4}
\end{figure}


\subsection{Dance Recalibration (DR)}
When the dance motion representation passes through the Dance Decoder Process using the~\cite{li2024lodge} model, it yields a coarse dance motion representation. During this process, the dance motion representations that pass through Global Diffusion follow a distribution but can output unstable values. This results in awkward dance motions when viewed from a sequential perspective. To address this issue, we added a Dance Recalibration Process.

DR fundamentally follows a structure similar to RNNs. Although RNNs are known to suffer from the gradient vanishing problem as they get deeper, the sequence length of the coarse dance representation in \cite{li2024lodge} is not long enough to cause this issue, making it suitable for use. Using LSTM or GRU, which solve the gradient vanishing problem, would make the model too complex and computationally expensive, making them unsuitable for use with the Denoising Diffusion Process \cite{ho2020denoising, song2020denoising}.

The coarse dance representation has 139 channels, consisting of 4-dim foot positions, 3-dim root translation, 6-dim rotaion information and 126-dim joint rotation channels. Of these, the 126-dim channels directly impact the dance motion, and all DR operations are performed using these 126 channels.

The values output from the Global Dance Decoder \(GD_{i}\), contain unstable dance motion information that follows a distribution. We construct Global Recalibrated Dance \(GRD_{i}\) by concatenating \(C\) the information from \(GRD_{i-1}\) with \(GD_{i}\) and applying pooling \(P\), thereby adding sequential information. However, using previous information as is may result in overly simple and smoothly connected dance motions. To prevent this, we add Gaussian noise \(G\) to the previous information \(GRD_{i-1}\) to produce more varied dance motions. This process is represented in Equations 1 below. The entire procedure is illustrated in Figure 2, 3.
\begin{equation}
    GRD_{i} = P(C(GD_{i} , GRD_{i-1} + G(Threshold))
\end{equation}



\begin{figure}[!t]
    \centering
    \includegraphics[width=\textwidth]{DanceRecalibration.eps}
    \caption{Overall of the Dance Recalibration Block Structure}
    \label{fig:enter-label1}
\end{figure}

\begin{figure}[!t]
    \centering
    \includegraphics[width=\textwidth]{DanceRecalibrationBlock.eps}
    \caption{The structure of the dance recalibration block}
    \label{fig:enter-label2}
\end{figure}

\subsection{Pooling Block}
Pooling \(P\) uses a simple pooling method. When \(GRD_{i}\) with added \(G\) and \(GD_{i+1}\) are input, they are concatenated into a \((Batch\times2\times126)\). First, we perform Layer Normalization to minimize differences between layers. Then, we pass through three simple 1D-Convolution Blocks, each followed by an activation function and batch normalization, to construct \(GRD_{i+1}\) that includes information from the previous dance motion. This procedure is illustrated in Figure 4.

\begin{figure}[!t]
    \centering
    \includegraphics[width=\textwidth]{Figure3.eps}
    \caption{overall procedure of Pooling processing by our Pooling Block}
    \label{fig:enter-label3}
\end{figure}

By following all these steps, each dance motion incorporates a bit of information from the previous dance motions, producing an overall coarse dance motion that follows the distribution of Global Diffusion while also retaining sequential information. This process is expressed in Equation 2:

\begin{equation}
    Total Coarse Dance Motion = C_{i=1}^{n}(P(C(GD_{i} , GRD_{i-1} + G(Threshold))), P(GD_{0}))
\end{equation}

We did not use bi-directional information because it complicates the calculations and can destabilize sequential information when using more than two \(GD_{i}\). Since there is a trade-off between generating complex dance motions and maintaining consistency, it is crucial to add appropriate noise. However, due to time constraints, we could not conduct various ablation studies.


\begin{comment}
\subsection{Character Model}

\subsection{Retargeting Method}


- 2D billboard (necessary for style)
- View dependent pose retargeting
    -Switching of the limbs based upon viewing angle
- Internal feature manipulation
- Feet flipping driven by knee angle
- External feature manipulation (important given results of annotation task, from annotation task, can have:
    - breakdown of internal vs. external features
    
- Twisted Perspective
    - When mentioning why this is important, see if I can formulate in a mathy way the reason why things get so bad.
    In essense, we've got a 3D point the is multiplied by some matrix to get a 2D projection of the point on a plane. And the 2D projections are used to obtain the orientation. At some point, changes in the 3D joint position have either no effect upon the orientation, or an outsized effect on the orientation. Figure out how this is reflected in the Jacobian matrix (ill-conditioned?) (null space?)
    - Mention how discontinuities are okay (this is how 2D animation works, after all), but the discontinuities shouldn't be frequently. This is why we have the previous projection vector as part of the optimization.

\subsection{Base Representation}
- Discuss how we create a left and right version of each character, and use left one when view angle is between 0 -> pi/2 -> pi and right one otherwise. Show unit circle.

- Discuss how we create a front and back texture and use front when view angle is pi/2 -> pi -> 3pi/2 and back otherwise. Explain how we create it from annotations. Show unit circle.

\subsection{Parts}
-Discuss how we generate the parts from the annotations and how we initialize the keyview-transform dictionary. Figure showing unit circle along with examples at keyview left and right. Show on character that has no obvious left-right difference in base representation.

\subsection{LimbMapping}
Discuss how, if we keep character normal pointed towards the viewer, eventually motion source limbs will drive the opposite character limbs, results in a strange pose and unrecognizable pose/silhouette. So we swap between 3pi/2 -> 0 -> pi/2. Show unit circle

\subseciont{Render Order}
Original skeleton 3D joint locations used to determine rendering order. 
Order is determined by projecting to BVH joint depth driver onto the projection plane normal vector and determining whether, when viewed from that particular angle, it is in front of or behind the torso depth joint.

\subsection{User customization}
 - mention how the number of attachment points can be parameterized to make the snappiness of the character higher or lower.  
 
\end{comment}
 

\section{Evaluation}

% Our proposed framework was compared with Apollo \cite{b7Apollo1, b7Apollo2}, which demonstrates that it can model analytic operators using data content. Two loss functions were utilized, the root-mean-square deviation error (RMSE), and the mean absolute error (MAE). The selection of these two loss functions is because they fulfil the disadvantages of each other, while RMSE is sensitive to outlier MAE is not and the MAE cannot take into account the direction of the error while the RMSE can achieve it. Speedup was computed to determine how quickly our framework can model the operator $\Phi$. We utilised the \textit{Speedup} and \textit{Amortized Speedup}, which assesses the require time to approximate each operator in comparison to exhaustively executing them on all datasets (more is better). Particularly, the speedup is equalled $\frac{T{^{(i)}_{op}}}{T{^{(i)}_{SimOp} + T_{vec} + T_{sim} + T_{pred}}}$, where $T{^{(i)}_{op}}$ is the execution time for operator $i$, across all the datasets, $T{^{(i)}_{SimOp}}$ is the time needed to model the operator with the datasets selected from the similarity search, $T_{vec}$  is the time needed to compute the vector embedding for each dataset, $T_{sim}$, is the time needed to perform similarity search, and $T_{pred}$ is the time needed to predict on the dataset $D_o$. In addition to the dataset vectorisation, which is done once for each data lake, we calculate amortised speedup. Furthermore, an experimental evaluation of our proposed model for dataset vectorization NumTabData2Vec has been performed to show that our approach can transform a dataset to a vector embedding representation space $z$. For the evaluation experiments, three different NumTabData2Vec were built to project the dataset representation with vector sizes of $100$, $200$, and $300$. Each model has eight transformer layers and is trained parallel using four NVIDIA A10s GPUs, and trained for fifty epochs.
We compared our framework with Apollo \cite{b7Apollo1, b7Apollo2}, which models analytic operators using data content. Two loss functions to measure prediction accuracy are employed: root-mean-square error (RMSE) and mean absolute error (MAE). RMSE is sensitive to outliers, while MAE is not; conversely, RMSE accounts for error direction, which MAE cannot. Speedup metrics are also used to evaluate how efficiently our framework models operator $\Phi$. Specifically, \textit{Speedup} and \textit{Amortized Speedup} measure the time required to approximate each operator versus exhaustively executing them on all datasets. Speedup is defined as $\frac{T{^{(i)}_{op}}}{T{^{(i)}_{SimOp} + T_{vec} + T_{sim} + T_{pred}}}$, where $T{^{(i)}_{op}}$ is the time to execute operator $i$ on all datasets, $T{^{(i)}_{SimOp}}$ is the time to model the operator with datasets from similarity search, $T_{vec}$ is the vector embedding computation time, $T_{sim}$ is the similarity search time, and $T_{pred}$ is the prediction time for $D_o$. Amortized speedup includes dataset vectorization, performed once per data lake for multiple operators (in our case two operators).
We also evaluate our dataset vectorization model, NumTabData2Vec, which projects datasets into vector embedding space $z$. Three versions were built with vector sizes of $100$, $200$, and $300$, each featuring eight transformer layers. The models were trained for 50 epochs on four NVIDIA A10 GPUs in parallel.

\subsection{Evaluation Setup}
Our framework is deployed over an AWS EC2 virtual machine server running with 48 VCPUs of AMD EPYC 7R32 processors at 2.40GHz, and four A10s GPUs with 24GB of memory each, $192GB$ of RAM memory, and $2TB$ of storage, running over Ubuntu 24.4 LTS. Our code is written in Python (v.3.9.1) and PyTorch modules (v.2.4.0). Apollo was deployed in a virtual machine with 8 VCPUs Intel Xeon E5-2630 @ 2.30GHz, $64GB$ of RAM memory, and $250GB$ of storage, running Ubuntu 24.4 LTS like in their experimental evaluation. 

\subsection{Datasets}
\begin{table}[!ht]
    \centering
    \setlength\doublerulesep{0.5pt}
    \caption{Dataset properties for experimental evaluation}
    \label{tab:table-evaluation-datasets}
    \begin{tabular}{||c|c|c|c||}
        \hline
         \makecell{Dataset Name}& \makecell{\# Files} & \makecell{\# Tuples} & \makecell{\# Columns}\\ \hline\hline
         Household Power & & & \\
         Consumption \cite{b21HPCdataset} & $401$ & $2051$ & 7\\
         \hline
         Adult \cite{b22AdultDataset} & $100$ & $228$ & 14\\
         \hline
         Stocks \cite{b23StockMarketDataset} & $508$ & $1959 - 13$ & 7 \\
         \hline
         Weather \cite{b23WeatherDataset} & $49$ & $516$ & 7 \\ \hline
    \end{tabular}

\end{table}

We evaluated our framework using four diverse datasets to represent real-world scenarios. Table \ref{tab:table-evaluation-datasets} summarizes these datasets and their attributes. The vectorization module, NumTabData2Vec, was trained on data separate from the experimental evaluation data, split $60\%$ for training and $40\%$ for testing.
The Household Power Consumption (HPC) dataset \cite{b21HPCdataset} contains electric power usage measurements from a household in Sceaux, France. It includes $401$ datasets, each with $2051$ tuples and seven features recorded at one-minute intervals. The Adult dataset \cite{b22AdultDataset}, commonly used for binary classification, predicts whether an individual earns more or less than $50K$ annually. It comprises $100$ datasets, each with $228$ individuals and various socio-economic features.
The Stock Market dataset \cite{b23StockMarketDataset} includes daily NASDAQ stock prices obtained from Yahoo Finance, with $508$ datasets. Each dataset contains $13$ to $1959$ tuples, each describing seven feature attributes. The Weather dataset \cite{b23WeatherDataset} provides hourly weather measurements from $36$ U.S. cities between $2012$ and $2017$, split into $49$ datasets, each with $516$ tuples and seven features.


\begin{figure}[!t]
     \centering
     \begin{subfigure}[b]{0.24\textwidth}
         \centering
         \includegraphics[width=\textwidth]{Figures/Results/Sim_Search/HPC/HPC_LR_RMSE_Loss_fig.pdf}
         \caption{Linear Regression RMSE error loss}
         \label{fig:HPC-LR-RMSE}
     \end{subfigure}
     \hfill 
     \begin{subfigure}[b]{0.24\textwidth}
         \centering
         \includegraphics[width=\textwidth]{Figures/Results/Sim_Search/HPC/HPC_LR_MAE_Loss_fig.pdf}
         \caption{Linear Regression MAE error loss}
         \label{fig:HPC-LR-MAE}
     \end{subfigure}
        
     \begin{subfigure}[b]{0.24\textwidth}
         \centering
         \includegraphics[width=\textwidth]{Figures/Results/Sim_Search/HPC/HPC_MLP_RMSE_Loss_fig.pdf}
         \caption{MLP for Regression RMSE error loss}
         \label{fig:HPC-MLP-RMSE}
     \end{subfigure}
     \hfill 
     \begin{subfigure}[b]{0.24\textwidth}
         \centering
         \includegraphics[width=\textwidth]{Figures/Results/Sim_Search/HPC/HPC_MLP_MAE_Loss_fig.pdf}
         \caption{MLP for Regression MAE error loss}
         \label{fig:HPC-MLP-MAE}
     \end{subfigure}
        \caption{Household power consumption dataset prediction error loss}
        \label{fig:HPC-EVAL-RES}
\end{figure}

Our framework was evaluated by registering the accuracy of predicting the output of various ML operators over multiple datasets in $D$ without actually executing the operator on them. To evaluate our scheme and its parameters, we use all four datasets, ranging the size of the produced vectors as well as the similarity functions used.
We project all datasets into $k$-dimensional spaces with varying vector dimensions ($100$, $200$, and $300$). For each dataset in Table \ref{tab:table-evaluation-datasets}, we model different operators: For the regression datasets (Household Power Consumption and Stock Market), we model Linear Regression (LR) and Multi-Layer Perceptron (MLP) operators; for the classification datasets (Weather and Adult), we model the Support Vector Machine (SVM) and MLP classifier operators. Each experiment has been executed $10$ times and we report the average of the error loss, as well as the speedup. 

\begin{figure}[!t]
     \centering
     \begin{subfigure}[b]{0.24\textwidth}
         \centering
         \includegraphics[width=\textwidth]{Figures/Results/Sim_Search/Stocks/Stocks_LR_RMSE_Loss_fig.pdf}
         \caption{Linear Regression RMSE error loss}
         \label{fig:Stock-LR-RMSE}
     \end{subfigure}
     \hfill 
     \begin{subfigure}[b]{0.24\textwidth}
         \centering
         \includegraphics[width=\textwidth]{Figures/Results/Sim_Search/Stocks/Stocks_LR_MAE_Loss_fig.pdf}
         \caption{Linear Regression MAE error loss}
         \label{fig:Stock-LR-MAE}
     \end{subfigure}
        
     \begin{subfigure}[b]{0.24\textwidth}
         \centering
         \includegraphics[width=\textwidth]{Figures/Results/Sim_Search/Stocks/Stocks_MLP_RMSE_Loss_fig.pdf}
         \caption{MLP for Regression RMSE error loss}
         \label{fig:Stock-MLP-RMSE}
     \end{subfigure}
     \hfill 
     \begin{subfigure}[b]{0.24\textwidth}
         \centering
         \includegraphics[width=\textwidth]{Figures/Results/Sim_Search/Stocks/Stocks_MLP_MAE_Loss_fig.pdf}
         \caption{MLP for Regression MAE error loss}
         \label{fig:Stock-MLP-MAE}
     \end{subfigure}
        \caption{Stock market dataset prediction error loss}
        \label{fig:Stock-EVAL-RES}
\end{figure}
\subsection{Evaluation Results}



Figures \ref{fig:HPC-EVAL-RES}, \ref{fig:Stock-EVAL-RES}, \ref{fig:Weather-EVAL-RES}, and \ref{fig:Adult-EVAL-RES} present the evaluation results for each method, comparing the performance of different similarity search techniques across various vector embedding representation spaces. The red (with hatches), brown, and blue bars correspond to vector embeddings of size 100, 200, and 300 respectively. In each sub-figure, the y-axis represents the error loss value, while the x-axis displays the similarity search method applied over the vector embeddings. Figures \ref{fig:HPC-EVAL-RES} and \ref{fig:Stock-EVAL-RES} show the results for the Stock market and Household power consumption datasets, where the bottom sub-figure demonstrates the MLP regression model, and the top sub-figure presents the LR model. Figures \ref{fig:Weather-EVAL-RES} and \ref{fig:Adult-EVAL-RES} depict the evaluation results for the Weather and Adult datasets. In these Figures, the top sub-figure shows the SVM with SGD results, while the bottom sub-figure shows the MLP classifier. The left sub-figures in all Figures use the RMSE loss function, whereas the right sub-figures use the MAE loss function. 



Figure \ref{fig:HPC-EVAL-RES}, we show, for the HPC dataset, shows as increase the vector dimension size there is slightly lower prediction error for all the operator modelling. While for different similarity methods did not result in any significant differences in the prediction error loss for all the operator modelling. This suggests that, regardless the similarity selection method, our framework effectively selects the most optimal subset of data to improve model predictions on the unseen input dataset $D_o$. Additionally, we observe higher error loss with a vector size of 100, which can be attributed to the reduced representation capacity of lower-dimensional vectors. This limitation results in fewer ``right" datasets being selected.

For the stock market dataset, Figure \ref{fig:Stock-EVAL-RES} depicts that a vector embedding representation of size $300$ models more accurate operators, with cosine similarity performing best in the similarity search and modelling the most optimal operator. However, due to the inherent volatility in Stock market data from different days, all models in the stock market dataset experiments exhibit high loss values. 

In the weather dataset, the SVM operator results from sub-figures \ref{fig:Weather-SVM-RMSE} and \ref{fig:Weather-SVM-MAE} show that using $300$ vectors in the representation space consistently led to more accurate operator models across all similarity methods. Specifically, cosine similarity in combination with the $300$-dimensional vector embedding reduced the error rate in operator predictions, demonstrating that projecting datasets into this representation space and applying cosine similarity improves the prediction accuracy on the modelled operator. For the MLP classifier from sub-figures \ref{fig:Weather-MLP-RMSE} and \ref{fig:Weather-MLP-MAE}, the results illustrate that using vector embeddings of size $200$ and K-Means clustering produced the most accurate MLP classifier operators.

% Overall, we observe that the error loss was minimized 
% (** what do you mean, minimized? In general, here you should comment on the effect of similarity function, the effect of vector size and the effect of different operators to the accuracy of prediction. E.g., in Household dataset shows little effect in all bars, but in Stock, the cosine seems better and larger size of vectors leads to better performance etc. **)
% in most cases, indicating that our framework effectively selects the most relevant datasets from the data lake $D$, thereby improving data quality and reducing $\Phi$ prediction errors on the target dataset $D_o$. This demonstrates that the datasets are accurately transformed into the vector embedding representation space, allowing for the selection of datasets most similar to $D_o$. 

%Adult


%Weather
\begin{figure}[t!]
     \centering
     \begin{subfigure}[b]{0.24\textwidth}
         \centering
         \includegraphics[width=\textwidth]{Figures/Results/Sim_Search/Weather/Weather_SVM_RMSE_Loss_fig.pdf}
         \caption{SVM with SGD RMSE error loss}
         \label{fig:Weather-SVM-RMSE}
     \end{subfigure}
     \hfill 
     \begin{subfigure}[b]{0.24\textwidth}
         \centering
         \includegraphics[width=\textwidth]{Figures/Results/Sim_Search/Weather/Weather_SVM_MAE_Loss_fig.pdf}
         \caption{SVM with SGD MAE error loss}
         \label{fig:Weather-SVM-MAE}
     \end{subfigure}
        
     \begin{subfigure}[b]{0.24\textwidth}
         \centering
         \includegraphics[width=\textwidth]{Figures/Results/Sim_Search/Weather/Weather_MLP_RMSE_Loss_fig.pdf}
         \caption{MLP RMSE error loss}
         \label{fig:Weather-MLP-RMSE}
     \end{subfigure}
     \hfill 
     \begin{subfigure}[b]{0.24\textwidth}
         \centering
         \includegraphics[width=\textwidth]{Figures/Results/Sim_Search/Weather/Weather_MLP_MAE_Loss_fig.pdf}
         \caption{MLP MAE error loss}
         \label{fig:Weather-MLP-MAE}
     \end{subfigure}
        \caption{Weather dataset prediction error loss}
        \label{fig:Weather-EVAL-RES}
\end{figure}

On the other hand, the Adult dataset shows the lowest error rates, with error loss values consistently below $0.5$ across all vector embedding dimensions and similarity search methods (see Figure \ref{fig:Adult-EVAL-RES}). The Adult dataset, besides exhibiting a high number of rows, also has a higher number of columns, which demonstrates that our framework performs consistently well even with larger datasets.
Additionally, we observe that the lowest prediction error across all datasets occurs when using higher-dimensional vector embeddings. With a trade-off between accuracy and execution time as the difference to generate all data lake available datasets vector embedding representation between $100$ and $300$ size dimension in the vector representation space to be less than $60$ seconds. This confirms that a higher number of vector dimensions leads to more accurate predictions, consistent with findings in previous research \cite{b8Word2Vec}.


\begin{figure}[!t]
     \centering
     \begin{subfigure}[b]{0.24\textwidth}
         \centering
         \includegraphics[width=\textwidth]{Figures/Results/Sim_Search/Adult/Adult_MLP_RMSE_Loss_fig.pdf}
         \caption{SVM with SGD RMSE error loss}
         \label{fig:Adult-LR-RMSE}
     \end{subfigure}
     \hfill 
     \begin{subfigure}[b]{0.24\textwidth}
         \centering
         \includegraphics[width=\textwidth]{Figures/Results/Sim_Search/Adult/Adult_MLP_RMSE_Loss_fig.pdf}
         \caption{SVM with SGD MAE error loss}
         \label{fig:Adult-LR-MAE}
     \end{subfigure}
     
     \begin{subfigure}[b]{0.24\textwidth}
         \centering
         \includegraphics[width=\textwidth]{Figures/Results/Sim_Search/Adult/Adult_MLP_RMSE_Loss_fig.pdf}
         \caption{MLP RMSE error loss}
         \label{fig:Adult-MLP-RMSE}
     \end{subfigure}
     \hfill 
     \begin{subfigure}[b]{0.24\textwidth}
         \centering
         \includegraphics[width=\textwidth]{Figures/Results/Sim_Search/Adult/Adult_MLP_MAE_Loss_fig.pdf}
         \caption{MLP MAE error loss}
         \label{fig:Adult-MLP-MAE}
     \end{subfigure}
        \caption{Adult dataset prediction error loss}
        \label{fig:Adult-EVAL-RES}
\end{figure}





We conducted an experimental evaluation using the Sampling Ratio (SR) approach, similar to Apollo \cite{b7Apollo1}, but employed neural networks built from the vector embeddings of each dataset. The SR approach involves a unified random selection of $l\%$ datasets from the vector representation space, using this subset to construct a neural network for predicting operator outputs. We tested SR values of $0.1$, $0.2$, and $0.4$, as well as vector embedding dimensions of $100$, $200$, and $300$, across all datasets. 
Figure \ref{fig:SR-EVAL-RES} presents the sampling ratio results for the Adult dataset using MLP (sub-figure \ref{fig:Adult-SR-RMSE}) and for the Weather dataset using LR (sub-figure \ref{fig:Weather-SR-SVM-MAE}). In each sub-figure the y-axis represents the RMSE prediction error loss while the x-axis denotes the vector dimension



\begin{figure}[htpb!]
     \centering
     \begin{subfigure}[b]{0.24\textwidth}
         \centering
         \includegraphics[width=\textwidth]{Figures/Results/SR/Adult/Adult_MLP_SR_RMSE_Loss_fig.pdf}
         \caption{Adult Dataset MLP Operator RMSE error loss}
         \label{fig:Adult-SR-RMSE}
     \end{subfigure}
     \hfill 
     \begin{subfigure}[b]{0.24\textwidth}
         \centering
         \includegraphics[width=\textwidth]{Figures/Results/SR/HPC/HPC_LR_SR_RMSE_Loss_fig.pdf}
         \caption{HPC dataset LR Operator RMSE error loss}
         \label{fig:Weather-SR-SVM-MAE}
     \end{subfigure}
     \caption{Sampling Ratio prediction results}
        \label{fig:SR-EVAL-RES}
\end{figure}

Both experiments demonstrate that as the vector embedding dimension increases, coupled with a larger sampling ratio (SR) value, there is a slight decrease in the prediction error loss. This improvement occurs because higher-dimensional vector embeddings provide a more accurate representation of the datasets in k-dimensions, with better dataset selection leading to enhanced prediction accuracy. Comparing the SR approach to our similarity search method for the HPC dataset, the SR approach was approximately $15\%$ less accurate in operator prediction across all vector embedding dimensions. A similar trend was observed in the Weather dataset. However, the Stock dataset exhibited a much larger discrepancy, with the SR approach performing about $70\%$ worse in prediction accuracy across all vector embedding dimensions. Likewise, in the Adult dataset, the SR approach delivered the poorest performance, with nearly $90\%$ lower prediction accuracy compared to the similarity search methods.

\begin{table*}[htbp]
    \centering
        \caption{Evaluation results of our framework exported analytic operator with lowest prediction error in comparison with Apollo}
    \label{tab:table-eval-res}
    % \scalebox{0.8}{
    \setlength\doublerulesep{0.5pt}
    % \begin{adjustbox}{width=\linewidth,center}
    \begin{tabular}{|c|c|c|c|c|c|c|}
    \hline
         \makecell{Dataset\\Name} & Method & Operator & RMSE &  MAE & Speedup  & Amortized Speedup \\
         \hline\hline
         \multirow{7}{*}{\makecell{Household\\Power\\Consumption}}& \makecell{$300$V Cosine} & LR & $\mathbf{6.61}$ & $\mathbf{5.42}$ & $0.0017$ & $\mathbf{1.99}$ \\ \cline{2-7}
                  & \makecell{$300$V SR-$0.2$} & LR & $7.77$ & $6.66$ &  $0.0018$  & $1.42$\\ \cline{2-7} 
        & \makecell{Apollo-SR $0.1$} & LR & $2968.01$ &  $2352.55$ & $\mathbf{0.015}$ & $0.024$ \\ \cline{2-7}
         & \makecell{Apollo-SR $0.2$} & LR & $2811.49$ &  $2229.50$ & $0.015$ & $0.024$ \\ \cline{2-7}\cline{2-7}
         & \makecell{$300$V K-Means} & MLP Regr. & $\mathbf{6.70}$ & $\mathbf{3.38}$ &  $0.9249$  & $\mathbf{1.99}$\\ \cline{2-7}
         & \makecell{Apollo-SR $0.1$} & MLP Regr. & $3322.05$ &  $2606.99$ & $2.38$ & $1.74$ \\ \cline{2-7}
         & \makecell{Apollo-SR $0.2$} & MLP Regr. & $3850.01$ &  $2609.36$ & $\mathbf{2.38}$ & $1.74$\\ \cline{1-7} \cline{1-7} 
         % Stock
         % \multirow{5}{*}{\makecell{Stock}}& \multirow{1}{*}{ \makecell{$100$V Euclidean}} & LR & $229388.93$ & $193066.03$ \\ \cline{2-5}
        \multirow{7}{*}{\makecell{Stock}} &  \makecell{$300$V Cosine} & LR & $306382.28$ & $125335.65$ & $0.00085$ & $\mathbf{1.91}$\\ \cline{2-7}
        & \makecell{$300$V SR-$0.4$} & LR & $21861625.91$ & $5674215.265$ &  $0.00087$  & $0.33$\\ \cline{2-7}
        & \makecell{Apollo-SR $0.1$} & LR & $\mathbf{153665.92}$ &  $\mathbf{118236.48}$ & $\mathbf{0.00093}$ & $0.00096$\\ \cline{2-7}
         & \makecell{Apollo-SR $0.2$} & LR & $166844.95$ &  $133306.68$ & $0.00093$ & $0.00096$\\ \cline{2-7}\cline{2-7}
         &  \makecell{$300$V Cosine} & MLP Regr. & $\mathbf{140236.47}$ & $\mathbf{123571.12}$ & $0.63$ & $\mathbf{1.91}$\\ \cline{2-7}
         & \makecell{Apollo-SR $0.1$} & MLP Regr. &  $175150.82$ &  $145123.09$ & $\mathbf{0.93}$ & $0.96$\\ \cline{2-7}
         & \makecell{Apollo-SR $0.2$} & MLP Regr. & $174390.81$ &  $146338.73$ & $0.93$ & $0.96$\\ \cline{1-7} \cline{1-7}
         % Weather
         \multirow{7}{*}{\makecell{Weather}}& \multirow{1}{*}{ \makecell{$300$V Cosine}} & \makecell{SVM SGD}& $\mathbf{14.13}$ & $\mathbf{7.63}$ & $1.06$ & $\mathbf{22.8}$ \\ \cline{2-7}
               & \makecell{Apollo-SR $0.1$} & SVM & $69.51$ &  $25.52$ & $\mathbf{2.10}$ &  $1.16$\\ \cline{2-7}
                        & \makecell{Apollo-SR $0.2$} & SVM & $68.70$ &  $22.81$ & $2.10$ & $1.16$\\ \cline{2-7} \cline{2-7}
       &  \multirow{1}{*}{ \makecell{$200$V Cosine}}& MLP & $\mathbf{14.29}$ & $\mathbf{4.03}$ & $1.03$  & $\mathbf{22.8}$\\ \cline{2-7}
        &  \multirow{1}{*}{ \makecell{$200$V SR-$0.4$}}& MLP & $15.95$ & $13.31$ & $1.02$  & $1.77$\\ \cline{2-7}
         & \makecell{Apollo-SR $0.1$} & MLP & $69.62$ &  $23.10$ & $\mathbf{1.34}$ & $1.14$ \\ \cline{2-7}
         & \makecell{Apollo-SR $0.2$} & MLP & $673.56$ &  $\mathbf{84.70}$ & $1.32$ & $1.14$\\ \cline{1-7} \cline{1-7}
         
         % Adult
         \multirow{7}{*}{\makecell{Adult}}& \multirow{1}{*}{ \makecell{$300$V Cosine}} & \makecell{SVM SGD}& $\mathbf{0.36}$ & $\mathbf{0.2}$ & $0.37$   & $\mathbf{2.78}$\\ \cline{2-7}
                  & \makecell{Apollo-SR $0.1$} & SVM & $68.32$ &  $22.95$ & $\mathbf{0.75}$ & $0.85$ \\ \cline{2-7}
                 & \makecell{Apollo-SR $0.2$} & SVM & $68.88$ &  $22.88$ & $0.74$ & $0.85$\\ \cline{2-7} \cline{2-7}

         &  \multirow{1}{*}{ \makecell{$300$V K-Means}}& MLP & $\mathbf{0.36}$ & $\mathbf{0.19}$ & $0.30$ & $2.78$ \\ \cline{2-7}
        & \makecell{$300$V SR-$0.2$} & MLP & $6.01$ & $6.00$ &  $0.54$  & $\mathbf{3.54}$\\ \cline{2-7}
         & \makecell{Apollo-SR $0.1$} & MLP & $71.11$ &  $26.51$ & $\mathbf{1.07}$ & $1.31$\\ \cline{2-7}
         & \makecell{Apollo-SR $0.2$} & MLP & $70.16$ &  $25.74$ & $1.05$ & $1.31$\\ \cline{1-7}
         
    \end{tabular}
    % }
\end{table*}

% Table \ref{tab:table-eval-res} illustrates the model operators for each dataset and each loss function, amortized speedup and speedup from our framework in comparison with the same model operators from the Apollo \cite{b7Apollo1, b7Apollo2} framework with SR of $0.1$ and $0.2$. The values $100$V, $200$V, and $300$V in the method column correspond to the dimensions of the vector embedding used for each dataset. The lowest prediction error for each modelled operator in each dataset is highlighted in the method that is used in the similarity search step from our pipeline. Apollo outperforms our framework only on the stock dataset for SR equal with $0.1$ in the LR analytic operator for both RMSE and MAE loss function which performs $50\%$ and $6\%$ better on each loss function equivalent. While our framework for the MLP for Regression outperforms the Apollo modelled operator for $20\%$ and $84\%$ for RMSE and MAE loss functions. However, this difference in the Stock dataset for LR operator modelling is not significant. In the remaining datasets, our framework illustrates that it can outperform Apollo for different values of SR. This makes us confirm that our similarity search using similarity functions selects the most similar datasets $D_r$ from data lake directory $D$, increasing data quality and minimising $\Phi$ prediction errors on the dataset $D_o$. For the Adult dataset, our model operators also perform better, which indicates our method's advantage with an increased number of dataset features (columns). In term of speedup we can see that Apollo outperformed our framework of all modelled operators. In terms of speedup we can see that Apollo outperformed our framework of all modelled operators. This is due to the vectorisation method of our framework which consists of big complexity time. Furthermore, in amortized speedup in most of the amortized speedup in which the vectorization is not counted because it is executed only one time and can be reused our framework surpasses Apollo framework in most of the operators with a big difference with our framework to be between $10\%$ and $60\%$ faster than Apollo. Additionally, most datasets demonstrate better amortized speedup when using the SR approach within our framework. This is because the prediction process relies solely on the vector representation, rather than leveraging all dataset tuples as done in the similarity search method for operator modelling. However, in terms of prediction accuracy, the SR approach does not perform as well as the similarity search method, which achieves superior results.

Table \ref{tab:table-eval-res} compares model operators, loss functions, and speedup metrics for our framework and Apollo at SR values of $0.1$ and $0.2$. Methods $100$V, $200$V, and $300$V denote vector embedding dimensions. The lowest prediction errors align with our pipeline's similarity search method.
Apollo outperforms our framework on the Stock dataset for the LR analytic operator at SR equals with $0.1$ (with $50\%$ and $6\%$ improvements for RMSE and MAE, respectively). However, our framework excels with the MLP regression operator, improving RMSE and MAE by $20\%$ and $17\%$, respectively. The LR operator's performance gap on the Stock dataset is minor.
For other datasets, our framework consistently surpasses Apollo across different SR values. This demonstrates the effectiveness of our similarity search approach, which enhances data quality and reduces $\Phi$ prediction errors by identifying relevant datasets $D_r$ from the data lake directory $D$. The Adult dataset also highlights our framework's advantage with increasing feature dimensions.
Although Apollo achieves better raw speedup due to the higher complexity of our framework's vectorization step, our framework outperforms it in amortized speedup. By excluding the reusable vectorization process, it achieves speed gains of $10\%$ to $60\%$ for most operators.
The SR approach, leveraging vector embedding representations, enhances operator prediction compared to Apollo and achieves greater amortized speedup. However, the similarity search method outperforms both Apollo and the SR approach in prediction accuracy and amortized speedup, establishing its clear superiority across most datasets and operator scenarios.

\subsection{NumTabData2Vec Evaluation Results}

\begin{figure}[!ht]
    \centering
    \includegraphics[width=0.4\textwidth]{Figures/Results/Representation/V200_representation.pdf}
    \caption{Vector representation for each dataset from NumTabData2Vec}
    \label{fig:eval-data-repr}
\end{figure}


\begin{table}[!htp]
    \centering
    \caption{Similarity between vectors of different datasets scenarios}
    \label{tab:vec-rep-sim}
    \setlength\doublerulesep{0.5pt}
    \begin{tabular}{||c|c||}
    \hline
    Model Name & Similarity \\
    \hline\hline
     \makecell{NumTabData2Vec\\$100$ Vector size} & $0.54$\\
     \hline
      \makecell{NumTabData2Vec\\$200$ Vector size}   & $0.18$\\
      \hline
       \makecell{NumTabData2Vec\\$300$ Vector size}  & $0.16$\\ \hline
    \end{tabular}
\end{table}

% Our proposed model, \textit{NumTabData2Vec}, for dataset vectorization is compared between all the available dataset scenarios to determine whether it can effectively distinguish between them based on qualitative differences. The comparison involves selecting $n$ random datasets for each detaset scenario and projecting them into their respective vector embedding representations. Then for each dataset scenario, it gains the average vector embedding representation by the average vector embedding representation of the $n$ random datasets. The vector embedding representation for each dataset scenario depicted in Figure \ref{fig:eval-data-repr} in from the $k$-dimensional space (size of $200$) transformed to the 3d space using the PCA. Figure \ref{fig:eval-data-repr} demonstrates that each dataset occupies a distinct dimension, with non-overlapping or clustering closely together. This indicates that \textit{NumTabData2Vec} can identify the datasets from various situations and does not have a close representation like previous methods achieved it with the same accuracy but on different data types (such as word, and graphs) \cite{b8Word2Vec, b9Graph2Vec} and not in an entire dataset. Table \ref{tab:vec-rep-sim}, further illustrates the average cosine similarity between the vector embeddings of all datasets, demonstrating how dissimilar are the datasets in their vector representation. As the size dimension of the vector embedding representation increases, the model's ability to distinguish across datasets improves as their average similarity decreases. Furthermore, this indicates that larger vector dimension sizes are unneeded since between $100$ and $300$ is sufficient.

Our proposed model, \textit{NumTabData2Vec}, was evaluated to determine its ability to distinguish dataset scenarios based on qualitative differences. For each scenario, $n$ random datasets were selected, and their vector embeddings averaged to represent the scenario. These embeddings, initially in a 200-dimensional space, were projected into 3D using PCA and are shown in Figure \ref{fig:eval-data-repr}. The figure illustrates that each dataset scenario occupies a distinct space, with minimal overlap or clustering. This demonstrates that \textit{NumTabData2Vec} effectively distinguishes datasets, outperforming prior methods like Word2Vec and Graph2Vec \cite{b8Word2Vec, b9Graph2Vec}, which achieved similar accuracy but on different data types (e.g., words, graphs) rather than entire datasets. Table \ref{tab:vec-rep-sim} further highlights the average cosine similarity between dataset embeddings, showing greater dissimilarity as vector dimensions increase. However, results suggest that dimensions between $100$ and $300$ are sufficient for accurate distinction, avoiding the need for larger vector sizes.

\begin{figure}[!ht]
    \centering
    \includegraphics[width=0.4\textwidth]{Figures/Results/Representation/plot_representation_200Vectors.pdf}
    \caption{Synthetic data vector embedding representation}
    \label{fig:eval-sd-data-repr}
\end{figure}

To evaluate \textit{NumTabData2Vec}'s ability to distinguish datasets with varying row and column counts, we generated synthetic numerical tabular datasets of different dimensions and vectorized them. Figure \ref{fig:eval-sd-data-repr} shows datasets with columns ranging from three to thirty and rows from ten to one thousand, projected from a $200$-dimensional space to 2D using PCA. Each bullet caption c and r corresponds to the columns and rows of the dataset, respectively. Datasets with the same number of columns cluster closely in the representation space, and a similar pattern is observed for datasets with the same number of rows. These results indicate that our method effectively distinguishes datasets based on size during vectorization.

\begin{table}[!htp]
    \centering
    \caption{NumTabData2Vec execution time for different dataset dimensions and different vector sizes }

    \begin{adjustbox}{width=\columnwidth,center}
    \label{tab:vec-exec-time}
    \setlength\doublerulesep{0.5pt}
    \begin{tabular}{||c|c|c|c|c||}
    \hline
     \makecell{\# of columns} & \makecell{\# of rows} & \makecell{$50$ Vectors\\Execution time} & \makecell{$100$ Vectors\\Execution time} & \makecell{$200$ Vectors\\Execution time} \\
    \hline\hline
     $3$ & $100$ & $0.0004$ sec & $0.00042$ sec & $0.00051$ sec\\ \hline
     $3$ & $500$ & $0.0004$ sec & $0.00041$ sec & $0.00049$ sec\\ \hline
     $3$ & $1000$ & $0.0004$ sec & $0.00041$ sec & $0.00049$ sec\\ \hline
     $3$ & $1500$ & $0.0004$ sec & $0.00041$ sec & $0.00055$ sec\\ \hline
     $3$ & $1800$ & $0.0004$ sec & $0.00041$ sec & $0.00055$ sec\\ \hline
     \hline
     $10$ & $100$ & $0.0004$ sec & $0.0004$ sec & $0.00057$ sec\\ \hline
     $10$ & $500$ & $0.00039$ sec & $0.0004$ sec & $0.00051$ sec\\ \hline
     $10$ & $1000$ & $0.00041$ sec & $0.00042$ sec & $0.00052$ sec\\ \hline
     $10$ & $1500$ & $0.00041$ sec & $0.00042$ sec & $0.00055$ sec\\ \hline
     $10$ & $1800$ & $0.00041$ sec & $0.00042$ sec & $0.00052$ sec\\ \hline
     \hline
     $20$ & $100$ & $0.0004$ sec & $0.00042$ sec & $0.0005$ sec\\ \hline
     $20$ & $500$ & $0.0004$ sec & $0.00042$ sec & $0.0005$ sec\\ \hline
     $20$ & $1000$ & $0.00042$ sec & $0.00043$ sec & $0.00052$ sec\\ \hline
     $20$ & $1500$ & $0.00043$ sec & $0.00044$ sec & $0.00054$ sec\\ \hline
     $20$ & $1800$ & $0.00044$ sec & $0.00044$ sec & $0.00054$ sec\\ \hline    
     \hline\hline
    \end{tabular}
    \end{adjustbox}
\end{table}

To evaluate how dataset dimensions affect the execution time of \textit{NumTabData2Vec}, we created synthetic datasets with varying numbers of rows ($100$, $500$, $1000$, $1500$, and $1800$) and columns ($3$, $10$, and $20$). These datasets were vectorized into different dimensions, and the execution times were recorded. Table \ref{tab:vec-exec-time} shows that increasing the k-dimension requires approximately $20\%$ more time to generate the vector embeddings. This is expected, as a higher k-dimension involves more hyperparameters, which naturally increases computation time.

Interestingly, varying the number of columns did not significantly impact execution time. However, increasing the number of rows resulted in approximately $5\%$ additional execution time. This is because larger datasets require the extraction of more features, which has a modest impact on the model's execution time.

\begin{figure}[!ht]
    \centering
    \includegraphics[width=0.4\textwidth]{Figures/Results/Representation/plot_representation_noise_data_200Vectors.pdf}
    \caption{HPC Dataset vector embedding representation with addition of Noise}
    \label{fig:eval-nd-data-repr}
\end{figure}

To evaluate \textit{NumTabData2Vec}'s ability to distinguish datasets based on different properties like distribution and order, we introduced Gaussian noise to random $l\%$ of data tuples in an HPC dataset. Figure \ref{fig:eval-nd-data-repr} visualises the original and noise-modified datasets, projected from a 200-dimensional space to 2D using PCA. Each bullet caption g denotes the percentage of Gaussian noise added in the dataset. As noise increases, the representation space shifts further from the original dataset, indicating that \textit{NumTabData2Vec} effectively captures distribution differences. Additionally, since the HPC dataset has an inherent order, the model's sensitivity to noise demonstrates its ability to distinguish datasets based on ordering as well.

\begin{figure}[!ht]
    \centering
    \includegraphics[width=0.4\textwidth]{Figures/Results/Representation/plotrepresentationnoisedata1col200Vectors.pdf}
    \caption{HPC Dataset vector embedding representation with addition of Noise in the first column}
    \label{fig:eval-nd-data-repr-1col}
\end{figure}

To evaluate how fine-grained as distinction can be, we introduced noise into a single column and repeated the previous experiment, with the difference being that noise was added exclusively to the first column. Figure \ref{fig:eval-nd-data-repr-1col} visualizes the dataset's 2D vector space. The amount of Gaussian noise added to the dataset's first column is indicated by g in the bullet caption. The results show that as more noise is introduced to the column, the vector representation moves further away from the original dataset. In contrast to the previous experiment shown in Figure \ref{fig:eval-nd-data-repr}, the noisy dataset's representation stays closest to the original when only a single column is modified. Also in this experiment the dataset points in the 2-dimension are more grouped between them instead the previous experiment. 

%closely grouped compared to the previous experiment.

We present RiskHarvester, a risk-based tool to compute a security risk score based on the value of the asset and ease of attack on a database. We calculated the value of asset by identifying the sensitive data categories present in a database from the database keywords. We utilized data flow analysis, SQL, and Object Relational Mapper (ORM) parsing to identify the database keywords. To calculate the ease of attack, we utilized passive network analysis to retrieve the database host information. To evaluate RiskHarvester, we curated RiskBench, a benchmark of 1,791 database secret-asset pairs with sensitive data categories and host information manually retrieved from 188 GitHub repositories. RiskHarvester demonstrates precision of (95\%) and recall (90\%) in detecting database keywords for the value of asset and precision of (96\%) and recall (94\%) in detecting valid hosts for ease of attack. Finally, we conducted an online survey to understand whether developers prioritize secret removal based on security risk score. We found that 86\% of the developers prioritized the secrets for removal with descending security risk scores.
%\subsection{Limitations}

\begin{comment}
- method is fast enough to run in real time and can support rapid iteration on the annotations
\end{comment}

% Certain kinds of drawings it doesn't work on. (Works best for characters in T pose, which is frequently seen. Works with others as well, but the ARAP solve can produce artifacts)
%Certain kinds of motions it's doesn't work on.
%What currently takes the most time to do manually.
%contacts not necessarily preserved

%The motion doesn't really match what is expected for realistic character (see video)

%ARAP
%    - When the character isn't drawn in a t pose the arap solve can lead to some funky %artifacts (e.g. soccer guy)
%    - The ARAP solve always gives bendy results. This is probably undesireable for more rigid characters, like a robot.

%- Skeletal motion allows lots of inputs. E.g. text-to-motion can be used. (also show in paper). It's limiting in some senses, but also extremely empowering given the right circumstances.

%- There is still popping that can occur. Sudden change when limb mapping changes. Not ideal, but as long as it happens infrequently maybe that's okay.

%- Mention that the characters could be extended with further 'Halfs' and textures beyond left and right and front and back. However, here we show what is possible to achieve with minimal user input.

%Future work:
%- automate the annotation process with SAM and the like
%- handle input character with overlapping regions using better inpainting
% allow arms to extrude from the middle of the body?
%facial animation





\bibliographystyle{ACM-Reference-Format}
\bibliography{Ref}


\begin{figure*}[ht]
\centering
\includegraphics[width=0.9\textwidth]{figures/part_traits.png}
\caption{Examples of the different part traits mentioned by annotators. The extension of the tail and nose in \textit{a} suggest (conflicting) orientations; in \textit{b} the position of the eyes within head and pupils within eye suggest it; in \textit{c} the limbs are drawn as though viewing a character facing right; \textit{d-f} show less frequently mentioned cues of limb attachment point, perspective-based scaling, and occlusion, respectively.}
\label{fig:part_trait_example}
\end{figure*}

\begin{figure*}[ht]
\centering
\includegraphics[width=0.9\textwidth]{figures/user_study_1.png}
\caption{\textit{Top row:} original drawings of the 10 characters used in the perceptual user study. 
Subsequent rows contain the number of participants in each study, response contingency table, and chi-squared test static with p-value. The upper left cell of the contingency table contains the number of times a user correctly identified the original motion upon both the 2.5D model and the 3D model; upper right contains times motion was correctly identified upon 2.5D model only; lower left contains times motion was correctly identified upon 3D model only; bottom right contains times motion was not correctly identified in either case.}
\label{fig:user_study_1}
\end{figure*}

\begin{figure*}[ht]
\centering
\includegraphics[width=0.9\textwidth]{figures/ablations.png}
\caption{
\textit{Left:} when the input pose faces towards the camera (top row), it's left limbs are used to drive the limbs on the \textit{drawing-right} side of the model.
When the input pose faces away from the camera, (bottom row), this mapping is flipped and the left limbs now drive the limbs on the \textit{drawing-left} side of the model (middle column).
Without this flipping, limbs appear switched and the pose is less recognizable (right column, arms.
\textit{Right:} when only the root-view projection plane  (shown in white) is used for retargeting, unexpected flailing can occur; when the axis of the left lower arm is parallel to the root-view projection plane normal vector, flailing can occur (top, middle column). 
Dynamically modifying the limb's projection plane based addresses this; in the bottom rows, the skeleton's left arm projection plane (shown in red) deviates from the root-view projection plane, preventing flailing from occurring.}
\label{fig:twisted_perspective}
\end{figure*}

%\begin{figure*}[ht]
%\centering
%\includegraphics[width=0.4\textwidth]{figures/twisted_perspective.png}
%\caption{
%When only the root-view projection plane  (shown in white) is used for retargeting, unexpected flailing can occur; when the axis of the left lower arm is parallel to the root-view projection plane normal vector, flailing can occur (top, middle column). 
%Dynamically modifying the limb's projection plane based addresses this; in the bottom rows, the skeleton's left arm projection plane (shown in red) deviates from the root-view projection plane, preventing flailing from occurring.}
%\label{fig:twisted_perspective}
%\end{figure*}

%\begin{figure*}
%\centering
%\includegraphics[width=0.4\textwidth]{figures/limb-swapping.png}
%\caption{When the input pose faces towards the camera (top row), it's left limbs are used to drive the limbs on the \textit{drawing-right} side of the model.
%When the input pose faces away from the camera, (bottom row), this mapping is flipped and the left limbs now drive the limbs on the \textit{drawing-left} side of the model (middle column).
%Without this flipping, limbs appear switched and the pose is less recognizable (right column, arms).
%}
%\label{fig:limb_mapping}
%\end{figure*}

\begin{figure*}[ht]
\centering
\includegraphics[width=1.0\textwidth]{figures/applications.png}
\caption{Example applications: 3D scenes (a), 2D scenes (b), driving with text-to-motion model (c), motion capture (d), and 3-point tracking (e). Example mixed reality experience (f).}
\label{fig:applications}
\end{figure*}

\begin{figure*}[ht]
\centering
\includegraphics[width=1.0\textwidth]{figures/genai.png}
\caption{Original drawing (a), alternate views created by our method (b), and alternate views created by multi-view diffusion model~\cite{shi2023zero123} (c).}
\label{fig:genai}
\end{figure*}

\end{document}
\endinput
%%
%% End of file `sample-acmsmall.tex'.
