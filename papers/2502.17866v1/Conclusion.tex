


\section{Conclusion}

We present an integrated view-dependent 2.5D character model and novel motion retargeting technique to animated childlike figure drawings with 3D skeletal motion.
It requires only a single image and a set of high-level annotations, which are easy for a user to provide, and can be used to animate the figures in real-time.
The types of manipulations supported by our system are informed by the field of children's art analysis and by a novel annotation study we conducted.
Though simple, they are effective at keeping the motion identifiable, as indicated by the motion recognition study. 
As indicated by the style preference study, our results are appealing, style-preserving, and present a viable alternative to full 3D models. We also compare our results with alternatives, revealing limitations of existing methods in animating 2D figures in 3D. However, there is no "ground truth" for what an imaginary figure should look like, except for what is in the creator's mind. Quantitative evaluations therefore remain an important and fruitful future direction.

%\yuting{see if we have space to mention the annotations could be automated more.}
%While our systems strives to generate suitable manipulations to meshes and textures, there is no 
Our system currently requires users to provide a set of high-level semantic annotations. This task can be accelerated by latest AI models such as SAM~\cite{kirillov2023segment} and a more streamlined user interface.
%Because the 2.5D model is a collection of masks, textures, and keyview-transforms, users could create much more refined and customized 2.5D models with proper tooling.
However, there are certain limitations to our system that tools cannot overcome and are worth further investigation.
We require the character to be bidpedal and have no overlapping parts. 
The character must also be approximately vertical so that horizontally mirroring a region switches it from left- to right-facing.

There are also limitations to what 3D motions are applicable, either due to the nature of 2D projection, or the limitation in assuming only transversal rotations in the algorithm. For instance, motions that require three dimensions, such as pointing at the camera, won't appear as desired. Planar motions oriented towards the camera, such as someone cartwheeling towards you, may also fail.

The system produces animations that are not visually realistic, and therefore may look undesireable when used to animate visually realistic figures. An additional outcome of this design decision is that interactions with external objects won't appear realistic. The most obvious example of this is foot sliding, which can occur both as a result of using multiple projection planes for retargeting as well as changing the position of the viewing camera.

%\yuting{only support rotation on the transverse plane.}