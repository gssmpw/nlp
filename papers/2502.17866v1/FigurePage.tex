
\begin{figure*}[ht]
\centering
\includegraphics[width=0.9\textwidth]{figures/part_traits.png}
\caption{Examples of the different part traits mentioned by annotators. The extension of the tail and nose in \textit{a} suggest (conflicting) orientations; in \textit{b} the position of the eyes within head and pupils within eye suggest it; in \textit{c} the limbs are drawn as though viewing a character facing right; \textit{d-f} show less frequently mentioned cues of limb attachment point, perspective-based scaling, and occlusion, respectively.}
\label{fig:part_trait_example}
\end{figure*}

\begin{figure*}[ht]
\centering
\includegraphics[width=0.9\textwidth]{figures/user_study_1.png}
\caption{\textit{Top row:} original drawings of the 10 characters used in the perceptual user study. 
Subsequent rows contain the number of participants in each study, response contingency table, and chi-squared test static with p-value. The upper left cell of the contingency table contains the number of times a user correctly identified the original motion upon both the 2.5D model and the 3D model; upper right contains times motion was correctly identified upon 2.5D model only; lower left contains times motion was correctly identified upon 3D model only; bottom right contains times motion was not correctly identified in either case.}
\label{fig:user_study_1}
\end{figure*}

\begin{figure*}[ht]
\centering
\includegraphics[width=0.9\textwidth]{figures/ablations.png}
\caption{
\textit{Left:} when the input pose faces towards the camera (top row), it's left limbs are used to drive the limbs on the \textit{drawing-right} side of the model.
When the input pose faces away from the camera, (bottom row), this mapping is flipped and the left limbs now drive the limbs on the \textit{drawing-left} side of the model (middle column).
Without this flipping, limbs appear switched and the pose is less recognizable (right column, arms.
\textit{Right:} when only the root-view projection plane  (shown in white) is used for retargeting, unexpected flailing can occur; when the axis of the left lower arm is parallel to the root-view projection plane normal vector, flailing can occur (top, middle column). 
Dynamically modifying the limb's projection plane based addresses this; in the bottom rows, the skeleton's left arm projection plane (shown in red) deviates from the root-view projection plane, preventing flailing from occurring.}
\label{fig:twisted_perspective}
\end{figure*}

%\begin{figure*}[ht]
%\centering
%\includegraphics[width=0.4\textwidth]{figures/twisted_perspective.png}
%\caption{
%When only the root-view projection plane  (shown in white) is used for retargeting, unexpected flailing can occur; when the axis of the left lower arm is parallel to the root-view projection plane normal vector, flailing can occur (top, middle column). 
%Dynamically modifying the limb's projection plane based addresses this; in the bottom rows, the skeleton's left arm projection plane (shown in red) deviates from the root-view projection plane, preventing flailing from occurring.}
%\label{fig:twisted_perspective}
%\end{figure*}

%\begin{figure*}
%\centering
%\includegraphics[width=0.4\textwidth]{figures/limb-swapping.png}
%\caption{When the input pose faces towards the camera (top row), it's left limbs are used to drive the limbs on the \textit{drawing-right} side of the model.
%When the input pose faces away from the camera, (bottom row), this mapping is flipped and the left limbs now drive the limbs on the \textit{drawing-left} side of the model (middle column).
%Without this flipping, limbs appear switched and the pose is less recognizable (right column, arms).
%}
%\label{fig:limb_mapping}
%\end{figure*}

\begin{figure*}[ht]
\centering
\includegraphics[width=1.0\textwidth]{figures/applications.png}
\caption{Example applications: 3D scenes (a), 2D scenes (b), driving with text-to-motion model (c), motion capture (d), and 3-point tracking (e). Example mixed reality experience (f).}
\label{fig:applications}
\end{figure*}

\begin{figure*}[ht]
\centering
\includegraphics[width=1.0\textwidth]{figures/genai.png}
\caption{Original drawing (a), alternate views created by our method (b), and alternate views created by multi-view diffusion model~\cite{shi2023zero123} (c).}
\label{fig:genai}
\end{figure*}