\subsection{Use Case Study: Security Analyst}\label{appendix:use-case-cmp}

We compare the generated outputs of \chatiot\ to that of GPT-4o to show our benefits from external IoT security and threat information and knowledge.
In Fig.~\ref{fig:eva-cmp}, we present the outputs for the query \textit{“Investigate the vulnerabilities and exploits associated with TP-Link routers”} from the perspective of a \textit{Security Analyst}. The key advantages of \chatiot\ over GPT-4o are summarized as follows:

\begin{itemize}
    \item \textbf{Reliability}: GPT-4o provides general vulnerabilities and examples but lacks specific details, such as CVE references for many vulnerabilities. It misses certain critical vulnerabilities. On the other hand, \chatiot\ offers precise information about vulnerabilities, including affected models and specific risks. It references real-world CVEs, enhancing factual reliability.
    
    \item \textbf{Relevance}: GPT-4o covers a broad range of vulnerabilities but does not specifically target the needs of a security analyst, making it less relevant for professionals. 
    \chatiot's answer is tailored for a security analyst, focusing on vulnerabilities impacting network security. It provides detailed exploit analysis and mitigation strategies, making it highly relevant.

    \item \textbf{Technical}: GPT-4o is a general LLM, so it lacks comprehensive technical details about each vulnerability and does not explain them in depth or provide mitigation strategies.
    \chatiot\ offers detailed technical specifics for each vulnerability, including descriptions, risks, and mitigation strategies. The structured analysis of exploits is suitable for technical audiences.

    \item \textbf{Friendliness}: GPT-4o uses straightforward language, making it easy for a general audience to understand, but lacks engagement for detailed insights.
    Our approach maintains a professional tone while being informative. The structured layout enhances readability, making it user-friendly for professionals seeking specific information.
\end{itemize}

In a nutshell, \chatiot\ outperforms using only GPT-4o to processing IoT security-related queries/questions in reliability, relevance, and technical depth, making it more suitable for a security analyst's needs. While the GPT-4o's answer is friendly and accessible, it lacks the detailed information and focus required for professionals in IoT security.

\begin{figure*}
    \centering
\begin{tikzpicture}
[node distance=0.1cm, box/.style={rectangle, rounded corners, fill=blue!5, text width=\textwidth, minimum height=1cm, align=left}, title/.style={font=\large\bfseries}]

    % Title
    %\node[title] (title) {Security Analyst, Investigate the vulnerabilities and exploits associated with the TP-Link Routers};

    % Left Box
\iffalse
\node[box] (llama3box) {
\tiny
{\center{\textbf{TP-Link Router Vulnerabilities and Exploits}}}

\medskip

TP-Link is a popular brand of home and small business routers. However, like many other router manufacturers, TP-Link devices have been found to have various vulnerabilities and exploits. Here are some of the most notable ones:

\begin{itemize}
    \item \textbf{Remote Code Execution (RCE) Vulnerability (CVE-2014-5353)} \\
    In 2014, a vulnerability was discovered in TP-Link routers that allowed remote attackers to execute arbitrary code on the device. This was due to a buffer overflow in the router's web interface.
    
    \item \textbf{Authentication Bypass Vulnerability (CVE-2015-3031)} \\
    In 2015, a vulnerability was discovered that allowed attackers to bypass authentication on TP-Link routers. This was due to a weakness in the router's login mechanism.

    \item \textbf{Command Injection Vulnerability (CVE-2016-10176)} \\
    In 2016, a vulnerability was discovered that allowed attackers to inject arbitrary commands on TP-Link routers. This was due to a weakness in the router's web interface.

    \item \textbf{DNS Rebinding Vulnerability (CVE-2017-5225)} \\
    In 2017, a vulnerability was discovered that allowed attackers to bypass the router's firewall and access internal devices on the network. This was due to a weakness in the router's DNS rebinding mechanism.

    \item \textbf{Zero-Day Vulnerability (CVE-2018-10561)} \\
    In 2018, a zero-day vulnerability was discovered in TP-Link routers that allowed remote attackers to execute arbitrary code on the device. This was due to a buffer overflow in the router's web interface.

    \item \textbf{KRACK Attack Vulnerability (CVE-2017-13082)} \\
    In 2017, a vulnerability was discovered in the WPA2 protocol used by TP-Link routers, which allowed attackers to intercept and manipulate data transmitted over the network.

    \item \textbf{UPnP Vulnerability (CVE-2013-2029)} \\
    In 2013, a vulnerability was discovered in the UPnP protocol used by TP-Link routers, which allowed attackers to bypass the router's firewall and access internal devices on the network.
\end{itemize}


\noindent \textbf{Exploits}

Several exploits have been developed to take advantage of these vulnerabilities, including:

\begin{itemize}
    \item \textbf{Exploit-DB}: A publicly available exploit that takes advantage of the RCE vulnerability (CVE-2014-5353).
    \item \textbf{Metasploit}: A publicly available exploit that takes advantage of the authentication bypass vulnerability (CVE-2015-3031).
    \item \textbf{RouterSploit}: A publicly available exploit that takes advantage of several vulnerabilities, including the command injection vulnerability (CVE-2016-10176).
\end{itemize}


\noindent \textbf{Mitigation}

To mitigate these vulnerabilities and exploits, TP-Link router users should:

\begin{itemize}
    \item \textbf{Update their firmware}: Regularly update their router's firmware to ensure they have the latest security patches.
    \item \textbf{Change default passwords}: Change the default admin password and any other default passwords to prevent unauthorized access.
    \item \textbf{Disable WPS}: Disable WPS (Wi-Fi Protected Setup) to prevent brute-force attacks.
    \item \textbf{Disable UPnP}: Disable UPnP (Universal Plug and Play) to prevent unauthorized access to internal devices.
    \item \textbf{Use a firewall}: Enable the router's firewall to block incoming and outgoing traffic.
    \item \textbf{Use a VPN}: Use a VPN (Virtual Private Network) to encrypt internet traffic.
\end{itemize}

\noindent \textbf{Conclusion}

TP-Link routers, like many other router manufacturers, have been found to have various vulnerabilities and exploits. It is essential for users to take mitigation steps to prevent unauthorized access and protect their networks. Regular firmware updates, changing default passwords, and disabling WPS and UPnP can help prevent attacks.
};
%\node[anchor=north east, draw] at (llama3box.north east) {\scriptsize \textbf{\textit{LLaMA3.1:70B}}};


\node[box, fill=green!5, below=of llama3box] (ChatIoTLLaMA3-1-70Bbox) {
\tiny
\noindent \textbf{Technical Analysis Report: TP-Link Routers Vulnerabilities and Exploits}

\medskip

\noindent \textbf{Executive Summary}


This report provides a detailed technical analysis of the vulnerabilities and exploits associated with TP-Link Routers. Our analysis reveals multiple security vulnerabilities and exploits that can be leveraged by attackers to compromise the security of these devices. We prioritize actionable security measures and mitigation techniques to help mitigate these risks.


\noindent \textbf{Vulnerabilities Analysis}


Our analysis reveals the following vulnerabilities in TP-Link Routers:

\begin{itemize}
    \item \textbf{Denial of Service (DoS) Vulnerability}: This vulnerability allows attackers to cause a denial-of-service condition, making the device unavailable to legitimate users.
    \item \textbf{Security Bypass Vulnerability}: This vulnerability allows attackers to bypass certain security restrictions and perform unauthorized actions.
    \item \textbf{Command Injection Vulnerability}: This vulnerability allows attackers to execute arbitrary commands with user privileges in the context of the affected application.
    \item \textbf{Cross-Site Request Forgery (CSRF) Vulnerability}: This vulnerability allows attackers to trick administrators into resetting their passwords or enabling management services.
    \item \textbf{Unauthenticated Reflected Cross-Site Scripting (XSS) Vulnerability}: This vulnerability allows attackers to inject malicious code into the Referer header, leading to an alert box popping up on the user's browser.
    \item \textbf{Authenticated Reflected XSS Vulnerability}: This vulnerability allows attackers to inject malicious code into the arguments for non-existing URLs, leading to an alert box popping up on the user's browser.
    \item \textbf{Authenticated XSS Vulnerability in Diagnostics (Ping) Argument}: This vulnerability allows attackers to inject malicious code into the POST request body, leading to an alert box popping up on the user's browser.
\end{itemize}


\noindent \textbf{Exploits Analysis}


Our analysis reveals the following exploits associated with TP-Link Routers:

\begin{itemize}
    \item \textbf{Multiple TP-Link Routers Multiple Security Vulnerabilities}: This exploit allows attackers to bypass certain security restrictions and perform unauthorized actions, cause denial-of-service conditions, or execute arbitrary commands with user privileges in the context of the affected application.
    \item \textbf{TP-Link TD-W8951ND - Multiple Vulnerabilities}: This exploit allows attackers to inject malicious code into the Referer header, leading to an alert box popping up on the user's browser, and trick administrators into resetting their passwords or enabling management services.
\end{itemize}


\noindent \textbf{Technical Security Insights}


Our analysis reveals that TP-Link Routers are vulnerable to multiple security vulnerabilities and exploits. These vulnerabilities can be leveraged by attackers to compromise the security of these devices, leading to unauthorized access, data breaches, and other malicious activities.


\noindent \textbf{Actionable Security Measures and Mitigation Techniques}


To mitigate these risks, we recommend the following actionable security measures and mitigation techniques:

\begin{itemize}
    \item \textbf{Update Firmware}: Regularly update the firmware of TP-Link Routers to ensure that the latest security patches are applied.
    \item \textbf{Change Default Passwords}: Change the default passwords of TP-Link Routers to prevent unauthorized access.
    \item \textbf{Enable WPA2 Encryption}: Enable WPA2 encryption to secure wireless communications.
    \item \textbf{Disable Remote Management}: Disable remote management to prevent unauthorized access.
    \item \textbf{Use Strong Passwords}: Use strong passwords for administrator accounts to prevent unauthorized access.
    \item \textbf{Regularly Monitor Logs}: Regularly monitor logs to detect and respond to potential security incidents.
    \item \textbf{Implement Network Segmentation}: Implement network segmentation to isolate TP-Link Routers from other devices on the network.
\end{itemize}


\noindent \textbf{Disclaimer}

This report is provided for informational purposes only and should not be considered as a substitute for professional advice. We are not legally responsible for any recommendations or actions taken based on this report. It is the responsibility of the user to ensure that their TP-Link Routers are properly secured and configured to prevent security vulnerabilities and exploits.
};
\fi
\node[box, fill=green!5] (GPT4o){
\tiny
\noindent TP-Link routers, like many other networking devices, have been subject to various vulnerabilities and exploits over the years. These vulnerabilities can range from simple misconfigurations to more complex software flaws that can be exploited by attackers. Here are some notable vulnerabilities and exploits associated with TP-Link routers:

\noindent \textbf{1. Default Credentials and Weak Passwords}

Many TP-Link routers come with default administrative credentials (e.g., admin/admin). If these credentials are not changed, attackers can easily gain access to the router's administrative interface. Weak passwords can also be brute-forced.


\noindent \textbf{2. Remote Code Execution (RCE)}

Several TP-Link router models have been found to have vulnerabilities that allow remote code execution. For example:
\begin{itemize}
    \item \textbf{CVE-2019-7406}: A vulnerability in the TP-Link WR940N allows remote attackers to execute arbitrary code via a crafted HTTP request.
    \item \textbf{CVE-2020-12109}: A buffer overflow vulnerability in the TP-Link Archer C5 allows remote attackers to execute arbitrary code.
\end{itemize}

\noindent \textbf{3. Cross-Site Scripting (XSS)}

Cross-Site Scripting vulnerabilities can allow attackers to inject malicious scripts into the web interface of the router. For example:
\begin{itemize}
    \item \textbf{CVE-2019-7405}: An XSS vulnerability in the TP-Link WR940N allows remote attackers to inject arbitrary web script or HTML.
\end{itemize}

\noindent \textbf{4. Cross-Site Request Forgery (CSRF)}

CSRF vulnerabilities can allow attackers to perform actions on behalf of the authenticated user without their consent. For example:
\begin{itemize}
    \item \textbf{CVE-2018-11714}: A CSRF vulnerability in the TP-Link Archer C1200 allows remote attackers to change the router's settings.
\end{itemize}

\noindent \textbf{5. Command Injection}

Command injection vulnerabilities allow attackers to execute arbitrary commands on the router's operating system. For example:
\begin{itemize}
    \item \textbf{CVE-2017-13772}: A command injection vulnerability in the TP-Link WR1043ND allows remote attackers to execute arbitrary commands via the ping feature.
\end{itemize}

\noindent \textbf{6. Information Disclosure}

Information disclosure vulnerabilities can expose sensitive information about the router or the network. For example:
\begin{itemize}
    \item \textbf{CVE-2019-7404}: An information disclosure vulnerability in the TP-Link WR940N allows remote attackers to obtain sensitive information via a crafted HTTP request.
\end{itemize}

\noindent \textbf{7. Firmware Vulnerabilities}

Outdated firmware can contain numerous vulnerabilities that have been patched in later versions. It's crucial to keep the router's firmware up to date to mitigate these risks.

\noindent \textbf{8. Misconfigurations}

Improper configurations, such as enabling remote management without proper security measures, can expose the router to external attacks.

\noindent \textbf{Mitigation Strategies}

To protect against these vulnerabilities, consider the following best practices:
\begin{itemize}
    \item \textbf{Change Default Credentials}: Always change the default username and password to something strong and unique.
    \item \textbf{Update Firmware}: Regularly check for and apply firmware updates from TP-Link.
    \item \textbf{Disable Unnecessary Services}: Turn off services like remote management if they are not needed.
    \item \textbf{Use Strong Encryption}: Ensure that Wi-Fi networks use strong encryption methods like WPA3.
    \item \textbf{Network Segmentation}: Segment your network to limit the impact of a compromised device.
    \item \textbf{Monitor Network Traffic}: Use network monitoring tools to detect unusual activity.
\end{itemize}

\noindent \textbf{Conclusion}

TP-Link routers, like any other network devices, can be vulnerable to various types of attacks if not properly secured. Regular updates, strong passwords, and proper configuration can significantly reduce the risk of exploitation. Always stay informed about the latest vulnerabilities and apply necessary patches and mitigations promptly.
};
\node[box, below=of GPT4o] (ChatIoT-gpt4o) {
\tiny
\begin{center}
\textbf{Technical Analysis of Vulnerabilities and Exploits in TP-Link Routers}
\end{center}

\noindent \textbf{Overview}

TP-Link routers have been identified with multiple security vulnerabilities that pose significant risks to users. This analysis focuses on the vulnerabilities, associated risks, and actionable security measures to mitigate these threats.

\noindent \textbf{Identified Vulnerabilities}

\begin{enumerate}
    \item \textbf{Denial of Service (DoS) Vulnerability}
    \begin{itemize}
        \item \textbf{Affected Models}: TP-Link Archer C2, Archer C20i
        \item \textbf{Description}: This vulnerability allows attackers to disrupt the normal functioning of the router, rendering it unavailable to legitimate users.
        \item \textbf{Risk}: High. A successful DoS attack can lead to network downtime, affecting all connected devices.
        \item \textbf{Mitigation}: Implement rate limiting and anomaly detection to identify and block malicious traffic. Regularly update firmware to patch known vulnerabilities.
    \end{itemize}

    \item \textbf{Security Bypass Vulnerability}
    \begin{itemize}
        \item \textbf{Affected Models}: TP-Link Archer C2, Archer C20i
        \item \textbf{Description}: This vulnerability allows attackers to bypass security mechanisms, gaining unauthorized access to the router.
        \item \textbf{Risk}: High. Unauthorized access can lead to data breaches and further exploitation.
        \item \textbf{Mitigation}: Enable strong authentication mechanisms, such as multi-factor authentication (MFA). Regularly review and update access control lists (ACLs).
    \end{itemize}

    \item \textbf{Cross-Site Request Forgery (CSRF)}
    \begin{itemize}
        \item \textbf{Affected Models}: TP-Link TL-WR1043N
        \item \textbf{Description}: This vulnerability allows attackers to perform unauthorized actions on behalf of authenticated users by tricking them into clicking malicious links.
        \item \textbf{Risk}: Medium. CSRF can lead to unauthorized configuration changes, such as altering admin passwords.
        \item \textbf{Mitigation}: Implement anti-CSRF tokens in web forms. Educate users about the risks of clicking on unknown links.
    \end{itemize}

    \item \textbf{Path Traversal Vulnerability}
    \begin{itemize}
        \item \textbf{Affected Models}: TP-Link TL-R600VPN
        \item \textbf{Description}: This vulnerability allows attackers to access sensitive system files by manipulating URL paths.
        \item \textbf{Risk}: High. Unauthorized access to system files can lead to information disclosure and further exploitation.
        \item \textbf{Mitigation}: Validate and sanitize user inputs to prevent path traversal attacks. Regularly update firmware to patch known vulnerabilities.
    \end{itemize}

    \item \textbf{Command Injection Vulnerability}
    \begin{itemize}
        \item \textbf{Affected Models}: TP-Link Archer C2, Archer C20i
        \item \textbf{Description}: This vulnerability allows attackers to execute arbitrary commands on the router.
        \item \textbf{Risk}: Critical. Command injection can lead to complete system compromise.
        \item \textbf{Mitigation}: Implement input validation and sanitization. Use security mechanisms like Web Application Firewalls (WAF) to detect and block malicious inputs.
    \end{itemize}
\end{enumerate}

\noindent \textbf{Exploits Analysis}

\begin{enumerate}
    \item \textbf{Proof-of-Concept (PoC) Exploits}
    \begin{itemize}
        \item \textbf{Description}: Researchers have developed PoCs to demonstrate the feasibility of exploiting these vulnerabilities.
        \item \textbf{Risk}: High. PoCs can be used by malicious actors to develop real-world attacks.
        \item \textbf{Mitigation}: Monitor security advisories and apply patches promptly. Conduct regular security assessments to identify and mitigate potential vulnerabilities.
    \end{itemize}

    \item \textbf{Unauthenticated Reflected XSS}
    \begin{itemize}
        \item \textbf{Affected Models}: TP-Link TD-W8951ND
        \item \textbf{Description}: This vulnerability allows attackers to inject malicious scripts into the browser of an unauthenticated user.
        \item \textbf{Risk}: Medium. XSS can lead to session hijacking and unauthorized actions.
        \item \textbf{Mitigation}: Implement Content Security Policy (CSP) and sanitize user inputs. Regularly update firmware to patch known vulnerabilities.
    \end{itemize}

    \item \textbf{Authenticated Reflected XSS}
    \begin{itemize}
        \item \textbf{Affected Models}: TP-Link TD-W8951ND
        \item \textbf{Description}: This vulnerability allows attackers to inject malicious scripts into the browser of an authenticated user.
        \item \textbf{Risk}: Medium. XSS can lead to session hijacking and unauthorized actions.
        \item \textbf{Mitigation}: Implement CSP and sanitize user inputs. Educate users about the risks of clicking on unknown links.
    \end{itemize}

    \item \textbf{Reset Admin Password CSRF}
    \begin{itemize}
        \item \textbf{Affected Models}: TP-Link TD-W8951ND
        \item \textbf{Description}: This vulnerability allows attackers to reset the admin password by tricking the administrator into clicking a malicious link.
        \item \textbf{Risk}: High. Unauthorized access to the admin account can lead to complete system compromise.
        \item \textbf{Mitigation}: Implement anti-CSRF tokens and educate users about the risks of clicking on unknown links.
    \end{itemize}
\end{enumerate}

\noindent \textbf{Actionable Security Measures}

\begin{enumerate}
    \item \textbf{Firmware Updates}
Regularly update router firmware to patch known vulnerabilities.
Enable automatic updates if available.


    \item \textbf{Strong Authentication}
Use strong, unique passwords for admin accounts.
Enable multi-factor authentication (MFA) where possible.


    \item \textbf{Network Segmentation}
Segment the network to limit the impact of a compromised device.
Use VLANs to isolate sensitive devices.


    \item \textbf{Monitoring and Logging}
Implement network monitoring to detect and respond to suspicious activities.
Enable logging to maintain an audit trail of administrative actions.

    \item \textbf{User Education}
Educate users about the risks of phishing and social engineering attacks.
Encourage users to report suspicious activities.
\end{enumerate}

\noindent \textbf{Disclaimer}

The recommendations provided in this report are based on the information available at the time of writing. We are not legally responsible for any actions taken based on these recommendations. Always consult with a professional security expert before implementing any security measures.

};


\end{tikzpicture}

    \caption{Generated answers of GPT-4o (above) and GPT-4o-based \chatiot\ (below) for the inputs (User Role=Security Analyst, Query="Investigate the vulnerabilities and exploits associated with the TP-Link Routers").}
    \label{fig:eva-cmp}
\end{figure*}