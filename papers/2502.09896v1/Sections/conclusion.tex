\section{Conclusion}\label{sec:con}
In this work, we propose \chatiot, an LLM-based IoT security assistant, and conduct extensive evaluations on several common use cases. 
Concretely, we leverage the advanced language understanding and reasoning capabilities of LLM and IoT security and threat information to provide IoT security assistance, and develop an easy-to-use data processing toolkit. 
With our proposed LLM-based generation system and developed toolkit, \chatiot\ is easily scalable to integrate different kinds of IoT threat intelligence from multiple sources.

\smallskip
\noindent \textbf{Limitations \& Future Work.}
We acknowledge the limitations of our work and propose several potential directions for future improvements.
Firstly, although we have developed a data processing toolkit to handle various IoT security datasets, gathering these datasets from public sources on the Internet remains largely manual.
This process is highly dependent on human expertise, especially in identifying relevant data sources. 
To address this, a key future direction would be the design or integration of an automated library within \chatiot\ that could continuously update and ingest the latest or real-time IoT threat intelligence, significantly reducing the manual efforts involved.
Another promising direction involves re-training or fine-tuning an LLM specifically in the IoT security domain.
While our current implementation leverages a general-purpose LLM enhanced by external data sources, re-training or fine-tuning allows the LLM to more deeply understand the nuances and technical challenges unique to IoT security.
Such an approach could be integrated into \chatiot\ to provide even more reliable, technical, and actionable insights, pushing the boundaries of current IoT security solutions.
Together, these enhancements would make \chatiot\ not only more effective in processing and presenting IoT security and threat intelligence but also more capable of adapting to the fast-evolving landscape of cybersecurity threats.


