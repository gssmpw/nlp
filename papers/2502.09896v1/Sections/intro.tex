\section{Introduction}\label{sec:intro}
Internet of Things (IoT) constitutes a vast network of physical and virtual entities, characterized by their sensing and/or actuation capabilities, programmability features, and unique identifiers. 
This interconnected infrastructure has rapidly expanded, and IoT connections surpassed non-IoT connections in 2020. 
It is estimated that over 40 billion IoT devices will be integrated into homes and workplaces via sensors, processors, and software by 2030\footnote{\scriptsize \url{https://iot-analytics.com/number-connected-iot-devices/}}.
With the rapid developments of IoT, increased connectivity and complexity of IoT ecosystems also introduce various vulnerabilities, making them attractive targets for attacks. 
Consequently, IoT security has become a critical issue for individuals, organizations, and governments worldwide.

Over the decades, IoT security has garnered significant attention from researchers, covering both defensive~\cite{kouicem2018internet,ahmad2021machine,williams2017identifying} and offensive~\cite{deogirikar2017security,gormucs2018security} strategies. 
As the large language model (LLM) has made significant strides in recent years, it has been explored in the context of IoT security as well, such as threat/vulnerability identification~\cite{sokiotllm,ma,llmiotfuz,yang2023iot}, perceive IoT sensor data~\cite{mo2024iot}, device management and labeling~\cite{meyuhas2024iotlabel,llmiotcontrol}, and IoT trust semantics enhancements~\cite{ferraris2024ici}.
This surge in interest has led to many domain-specific datasets that offer extensive insights covering various aspects of IoT security, such as vulnerabilities and exploits~\cite{CVE_Mission,VARIoT_db}, tactics, techniques, and procedures (TTPs)~\cite{strom2018mitre}, and industry-standard guidelines~\cite{abdul2019comprehensive,wright2022regulating}. 
However, most existing works focus on discovering new vulnerabilities/threats or designing novel defensive/offensive techniques, but pay less emphasis on leveraging the insights contained in datasets to assist or guide users in enhancing their security practices.
Although sources such as \cite{CVE_Mission,VARIoT_db,strom2018mitre} offer information search services, they typically only provide raw data (\eg, vulnerability descriptions) which is often challenging for non-technical users to understand. Even experienced security analysts may find it difficult to extract actionable insights from massive unprocessed information.
Consequently, \textit{there is an urgent need for solutions that deliver timely, actionable, and easily accessible IoT security and threat intelligence to a wide range of users, including both technical and non-technical ones.}


Inspired by LLM's advanced capability, there are two promising approaches \romannumeral1) \textit{fine-tuning} the LLM specifically on IoT security information and \romannumeral2) \textit{RAG}, which \textit{R}etrieves IoT security information to \textit{A}ugment the LLM's \textit{G}eneration. 
However, na\"{i}ve application of these methods are still facing one or several of the following challenges:

\noindent \textbf{Challenge-\ding{182}:} \textbf{IoT Security Evolves Rapidly.}
The field of IoT security is continuously developing, with new vulnerabilities, exploits, and security protocols emerging regularly. 
For example, in August 2024, the VARIoT vulnerability dataset~\cite{VARIoT_db} received 79 updates, and cybersecurity websites almost report new developments daily\footnote{\scriptsize \url{https://www.bleepingcomputer.com/}} \footnote{\scriptsize \url{https://www.darkreading.com/}}.
As a consequence, fine-tuning an LLM on static IoT data would quickly become outdated, failing to provide the latest threats or innovations in IoT security.

\noindent \textbf{Challenge-\ding{183}:} \textbf{Heterogeneous Dataset Formats.} 
IoT security datasets come in a variety of formats, such as structured data (\eg, vulnerability databases~\cite{VARIoT_db}), unstructured text reports, and product compliance lists (such as Cybersecurity Labelling Schemes in Singapore, Finland, Germany, and the United States).
Fine-tuning/augmenting an LLM on such diverse data types without a robust data processing mechanism would significantly limit the utilization of information contained in these datasets.

\noindent \textbf{Challenge-\ding{184}:} \textbf{Diverse User Requirements.} 
Users in IoT ecosystems range from consumers to security analysts, each with distinct expertise levels and specific security concerns. 
Na\"{i}vely combining LLM with IoT security information would struggle to effectively cater to these varied requirements, limiting its ability to provide meaningful and contextually appropriate responses for different user groups.

In light of the knowledge gap and technical challenges analyzed above, we ask the following question: 

\textit{Can we use large language models to effectively disseminate IoT security assistance to various key users of the IoT ecosystems in an understandable and actionable manner to provide better IoT security guarantees?}


We present \chatiot, an LLM-based IoT security assistant augmented with external information retrieval, to answer this question affirmatively. 
On the paradigm of RAG, we design \chatiot\ by combining the advanced language understanding and reasoning capabilities of LLM with fast-evolving IoT security-specific information to generate reliable, relevant, and technical answers.
To handle heterogeneous datasets, we have developed an end-to-end data processing toolkit that integrates a range of existing technologies, enabling the conversion of diverse data formats into retrievable documents and optimizing chunking strategies to improve retrieval performance.
Additionally, we define several use-case specifications and provide the user's \textit{background} to ensure that the LLM-generated answers are also user-friendly, \aka, aligned with each user's specific needs and expertise levels.
%Finally, we implement a prototype of \chatiot\ and perform extensive evaluations to assess the quality of the generated outputs.
In summary, \chatiot\ offers the following contributions:

\begin{itemize}
    \item \textbf{IoT security assistant driven by LLM \& threat intelligence.}
    Our IoT security assistant, \chatiot, is designed to leverage advanced LLM understanding and reasoning capabilities and threat intelligence at the same time. 
    We develop self-querying retrievers for diverse IoT threat datasets and dynamically activate retrievers based on the user's background and the query contexts. 
    In this way, LLM extracts IoT threat intelligence from the relevant retrieved documents.
    This integration allows the assistant to generate the latest reliable, relevant, and technical IoT security answers tailored to both query contexts and users' requirements.
    To our best knowledge, it is the first time to make use of LLM and IoT threat intelligence to offer IoT security assistance.
    
    
    \item \textbf{Data processing toolkit.}
    We design toolkit \datakit\ to handle datasets in diverse formats. Our toolkit first parses the raw data and converts the parsed contents into text. Notably, we utilize LLM to \romannumeral1) select the appropriate content for page\_content and metadata of documents, and \romannumeral2) process the multimodal contents, such as figures, tables, and codes, into text summarization. 
    Additionally, we integrate the Ragas library~\cite{es2023ragas} to optimize the chunking strategy, including splitter methods and chunk sizes. 
    While many of these technologies are adapted from existing works, our focus is on developing an end-to-end data processing toolkit, which might be useful and of independent interest.
    

    \item \textbf{Implementation \& Evaluation.}
    We implement a prototype of \chatiot\ and define five use cases. 
    For each case, we specify the user’s background in terms of \textit{knowledge}, \textit{goals}, and \textit{requirements} to guide the LLM in generating answers that are not only reliable, relevant, and technical but also user-friendly.
    Extensive evaluations show the improvements achieved by \chatiot. 
    Specifically, we compare \chatiot's answers with those generated directly from LLaMA3, LLaMA3.1, and GPT-4o in terms of reliability, relevance, technical depth, and user-friendliness\footnote{The metrics are explained in \S~\ref{sec:exp-llmeval}}.
    When evaluated with LLaMA3:70B, \chatiot\ improves all metrics by over $10\%$ on average, particularly in relevance and technicality. 
    Human evaluations also confirm that \chatiot\ provides better answers.
\end{itemize}

\smallskip
\noindent \textbf{Organization.}
We introduce the background in~\S~\ref{sec:back}. 
An overview of design goals and system architecture is given in~\S~\ref{sec:overgoal}.
Our concrete design is illustrated in~\S~\ref{sec:design}. 
We implement the prototype and perform experimental evaluations in~\S~\ref{sec:experiment}. 
Finally, we summarize the related works in~\S~\ref{sec:relatedwork} and conclude this work in~\S~\ref{sec:con}. 