\section{Related Work}\label{sec:relatedwork}

Internet of Things (IoT) has seen rapid advancements in recent years, becoming an integral part of various domains, such as smart industries and homes, and serving as a key enabler in modern society.
However, despite its growth, IoT continues to face numerous security challenges, prompting significant research efforts aimed at improving IoT security.
With the rise of artificial intelligence (AI), machine learning (ML) and deep learning (DL)-based approaches have become increasingly popular in designing defense mechanisms for IoT devices, including malicious traffic classification~\cite{luo2022transformer,shafiq2020corrauc}, malware detection~\cite{vasan2020mthael,chaganti2022deep,aung2022atlas}, vulnerability discovery~\cite{neshenko2019demystifying}, and others~\cite{al2020survey,otoum2022dl,tambe2019detection}.

More recently, inspired by the success of large language models (LLMs), researchers have begun exploring the potential of LLMs to enhance IoT-related security tasks.
For instance, LLMs have been applied to existing IoT security challenges such as threat detection and fuzzing. Ferrag \etal~\cite{sokiotllm} introduced a BERT-based model, SecurityBERT, to achieve better cyber threat detection accuracy over traditional ML and DL-based methods. 
Similarly, Ma \etal~\cite{ma} and Wang \etal~\cite{llmiotfuz} proposed LLM-assisted fuzzing methods to uncover hidden bugs in IoT devices, enabling the detection of complex vulnerabilities that traditional techniques might miss.
Additionally, Yang \etal~\cite{yang2023iot} combined LLMs with static code analysis using prompt engineering to create a cost-effective solution for IoT vulnerability detection.
\cite{ji2024sevenllm} collected cybersecurity raw texts to train cybersecurity LLM to augment the analysis of cybersecurity events, and \cite{llmtikg} made use of a larger LLM to build knowledge graphs from public threat intelligence and use GPT to create datasets to fine-tune a smaller LLM to extract entities and TTPs from attack description.
Ferraris \etal~\cite{ferraris2024ici} proposed utilizing ChatGPT to enhance IoT trust semantics, aligning with W3C Web of Things (WoT) recommendations\footnote{\scriptsize \url{https://www.w3.org/WoT/}}.
This work extends the TrUStAPIS framework~\cite{ferraris2020trustapis}.

Beyond the above tasks, LLMs have been employed in other IoT challenges.
Meyuhas \etal~\cite{meyuhas2024iotlabel} used LLMs to address the problem of labeling previously unseen IoT devices.
\cite{llmiotcontrol,cui2024llmind} explored leveraging LLMs to control IoT devices and facilitate effective collaboration among them.
Mo \etal~\cite{mo2024iot} collected IoT sensor-natural language paired data and trained IoT-LM to interpret and interact with physical IoT sensors.
Xu \etal~\cite{xu2024penetrative} employed ChatGPT to interpret IoT sensor data and reason over tasks in the physical realm, introducing novel ways of integrating human knowledge into cyber-physical systems. 

Recently, Deldari \etal~\cite{deldari2024auditnet} proposed AuditNet, a conversational AI-based security assistant, which is most similar to \chatiot\ and also augmented by external knowledge.
However, AuditNet focused on standards, policies, and regulations of portable document format (PDF), and aimed to reduce the manual effort of security experts involved in compliance checks of IoT. 
On the other hand, we integrate IoT threat intelligence of various sources into \chatiot\ and can assist multiple kinds of users. Besides, we provide an end-to-end toolkit to process data in various formats, not limited to PDF. 

Together, these studies indicate that LLMs have great potential to improve the security of IoT systems in various domains, from vulnerability discovery to trustworthiness management. 
By integrating LLMs with IoT-specific threat intelligence, these models can be guided to meet the unique challenges posed by the IoT ecosystem.
Moreover, the continuous advancements in the LLM community, combined with increasingly accessible IoT datasets, are likely to further drive the adoption of LLMs in IoT-related research and practical applications.
