\section{Related Work}
\label{sec:relwork}

The work of Brown and Parnin ____ was among the first to present an empirical investigation the suggestion feature's usage, its effectiveness, and the perception of its usefulness by developers.
The authors analysed a total of 17,712 suggestions from 22 GitHub projects and surveyed 43 developers.
They summarised the benefits of integrating suggestions in five main aspects: timing, location, actionability, conciseness and communication.
% While suggestions potentially lengthened the code review process, the developers in the study commented that they were able to make recommendations much faster making for a better timing.
% Furthermore, developers did not need to leave the pull request page to make a suggestion, which covers the aspect of locality.
% Actionability on the other hand is given by the ability to incorporate the change immediately simply by committing the suggestion while conciseness was explained by the fact that developers prefer code over natural language recommendations.
Our categorisation of suggestion types shows strong parallels to their work  with subtle differences in  terminology.
The authors namely referred to \textit{documentation} suggestions as \textit{non-functional} suggestions and to suggestions proposing a \textit{fix} as \textit{corrective} suggestions.
Moreover, the fourth category in their work is referred to as \textit{formatting}, which in our case represents a subset of the type \textit{code style}.
The distribution of suggestion types in our results partly confirms the results reported by the authors ____.
Both studies found that suggestions of the types \textit{improvement} and \textit{documentation}  (\textit{non-functional}) occur more  frequently than suggestions of the types \textit{fix} (\textit{corrective}) and \textit{code style} (comparable with \textit{formatting}).
However, while Brown \& Parnin's work shows a similar occurrence frequency for \textit{non-functional} and \textit{improvement} suggestions (36\% and 34\% respectively), \textit{improvements} surpass \textit{documentation} suggestions in our assessment.
% Similarly, the gap in occurrence count between \textit{corrective} and \textit{formatting} suggestions is less prominent in Brown et al.'s work (16\% and 14\% respectively) than it is between \textit{code style} suggestions and \textit{fixes} in our sample.
This discrepancy may be explained  by the difference in sample sizes: our sample contained 1,775 manually labelled examples drawn from a population of 8,672 suggestions, while Brown \& Parnin used a random sample of 100 examples from 17,712 suggestions.
The perception of usefulness for the different types is comparable with the authors' results on the overall feature usefulness from both stakeholder perspectives.

% Our impact analysis further unveiled that suggestions mainly consist of small code changes, which did not lead to a significant decrease in CC and cognitive complexity, despite simplification being a recurrent motivation behind improvement suggestions. 
% While this is partly explained by the linear relationship of CC and LOC, we recognise that other complexity metrics such as cognitive complexity ____ might be more adequate in conveying the human perception of complexity.
Our impact analysis showed that the usage of suggestions positively affected the merge rate and significantly prolonged the pull request's resolution time.
Our results concur with those presented by Brown \& Parnin ____ on resolution time but not on merge rate.
One possible interpretation of these results is that suggestions demand effort on the reviewers' part.
A reviewer might be more likely to invest time and write a code review suggestion if they already agree with the pull request and recognise its necessity. 
Following this assumption, a higher merge rate may simply have co-occurred with the use of suggestions without necessarily inferring a causal relationship.
Although suggestions can be used to optimise the code review process through shorter iterations, they can also spark discussions delaying the resolution.
For instance, a participant in our study mentioned the drawback of \textit{``getting caught up in style wars''}.
In fact, conflicts may arise during code review.
In an interview study with OSS developers, Wurzel Gon{\c{c}}alves et al.____ found that code readability and style issues were the main non-functional sources of conflict.

% The prevalence of improvement and documentation suggestions is en par with the goal of maintainability, which was ranked the most important code review goal in a survey with  open source developers presented by Bosu et al. ____.
% The relatively small number of code style suggestions on the other hand can be attributed to the use of linters which have become widely adopted by many projects.
% In addition, the widespread adoption of continuous integration tools  could have contributed to the reduction of code defects, which explains the small number of fix suggestions.

Generally, according to a knowledge sharing measure defined in the work of Rigby et al.____, conducting peer reviews increases the number of distinct files known to a developer by  66\%. 
The results of our survey have shown that code review suggestions also contribute to knowledge sharing by acting as a searchable information source, primarily for code rationale—often stated as a rare commodity ____.
Code suggestions were also consulted when looking for code examples, which were mentioned by Sadowski et al.____ as being  in high demand among developers.
Rationale and code examples both figure among the categories of explanations presented by Widyasari et al.____ in their taxonomy of explanations in code review comments. 
The authors recently studied the different types of explanations commonly used in Gerrit code review comments and categorized them into seven types, each reflecting how reviewers communicate feedback and attempt to clarify their reasoning to pull requests authors.
% These findings also confirm that code review suggestions contribute to knowledge sharing, which ranked the second most important code review goal by Bosu et al. ____.
% We should note however that these differences may be due to the differences of magnitude between our sample and that of the authors, whose sample consisted of 100 randomly selected suggestions from a larger population of 17,712 suggestions.
% Moreover, the prevalence of improvements in our dataset may be partly explained by our strict definition of code style suggestions in terms of code structure, formatting, commenting and naming.
% In fact, this definition relegates all suggestions affecting source code and not treating any of the defined code style aspects to the improvement category.
% Knowledge sharing is stated by Bosu et al. ____ as the second most important goal of the code review process in OSS.
% We chose to further investigate the aspect of communication in our work by perceiving code suggestions as a special type of interaction within the flow of the pull request discussions.

Pull request discussions play a a central role in the development and evolution of open source projects as a whole as outlined by the works of Tsay et al.____, ____.
Developer discussions are commonly guided by technical and social factors and are used beyond quality assurance as a means of building a community around the project.
Bosu et al.____ further found that code review is used in OSS to form impressions about teammates, which impact future collaborations.
In our study, we showed that the usage of suggestions differs depending on the association of the pull request stakeholders with the project.
Unsurprisingly, we found that the highest usage of code review suggestions was registered between submitting contributors and reviewing members and that the second highest suggestion activity was registered among contributors.
However, we discovered subtle differences between stakeholder roles when looking at the distribution of suggestion types.
In fact, our analysis has shown that contributors tend to use more documentation suggestions  while members tend to point out slightly more defects. 
This can be explained by the differences in project experience, making members more aware of the implications of certain changes.

% Project experience also plays a role in the reviewer's decision on whether or not to use suggestions in their review and in the pull request submitter's decision on whether or not to accept a review suggestion.
In our survey, reviewers indicated they were more inclined to use suggestions when the pull request was submitted by a newcomer to the project.
These findings align with Bosu et al.____, who observed that reviewers in OSS sometimes adopted a mentoring role when the author lacked project or programming experience. 
Additionally, our study reveals that reviewers often provide code examples or reference well-written code  when contributions were of low quality. 
These insights suggest that code review serves as an educational tool, where experienced developers guide and support newcomers.

% Despite pull request discussions on GitHub being a well studied topic ____ ____________, research investigating code review suggestions is still in the initial stages.
% The work of Brown et al. ____ was among the first to present an empirical investigation the feature's usage, its effectiveness and the perception of its usefulness by developers.
% The authors analysed a total of 17,712 suggestions from 22 GitHub projects and surveyed 43 developers.
% They summarised the benefits of integrating suggestions in five main aspects: timing, location, actionability, conciseness and communication.
% While suggestions potentially lengthened the code review process, the developers in the study commented that they were able to make recommendations much faster making for a better timing.
% Furthermore, developers did not need to leave the pull request page to make a suggestion, which covers the aspect of locality.
% Actionability on the other hand is given by the ability to incorporate the change immediately simply by committing the suggestion while conciseness was explained by the fact that developers prefer code over natural language recommendations.
% Additionally, the use of code review suggestions contributed to knowledge sharing between developers, which is the second most important goal of the code review process in open source projects according to Bosu et al. ____.
% According to a knowledge sharing measure defined in the work of Rigby et al. ____, conducting peer reviews increased the number of distinct files known to a developer by  66\%. 
%% We chose to further investigate the aspect of communication in our work by perceiving code suggestions as a special type of interaction within the flow of the pull request discussions.
% Pull request discussions play a a central role not only in the code review process but in the development and evolution of open source projects as a whole as outlined by the works of Tsay et al. ____, ____.
% Developer discussions are guided by technical and social factors and are used beyond quality assurance as a means of building a community around the project.
% Bosu et al. ____ further found that code review is used in particular in open source projects to form impressions about teammates, which impact future collaborations.
Finally, Palvannan and Brown ____ have explored leveraging GitHub’s suggestion feature to automate code reviews.
They introduced SUGGESTION BOT, which proposes code edits to pull request authors through code suggestions.
The study demonstrated that the bot not only reduced pull request turnaround time but also provided clear, actionable feedback, underscoring its potential to enhance human-bot collaboration in software development.
Recently, Maalej____ argued that even though bots in software engineering (BotSE) have evolved from simple task automators to intelligent assistants, their primary focus remains on information access rather than knowledge sharing among stakeholders. 
The author advocates for further research into bots that capture design rationale or mediate pair programming conversations, emphasizing their untapped potential to enhance knowledge exchange by connecting developers with experts and facilitating the documentation of valuable insights. 
This underscores the need for greater socio-technical context awareness in future research.