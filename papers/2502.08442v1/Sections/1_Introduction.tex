\section{Introduction}
\label{section:introduction}
The integration of Augmented Reality (AR) in the automotive sector is rapidly expanding. Major car manufacturers are incorporating AR features like Head-Up Displays and video-based AR to provide drivers and passengers with navigation guidance and contextual information \cite{Elhattab23AutomotiveAR}. This expansion offers opportunities for enhanced navigation, information delivery, and entertainment. Simultaneously, advancements in head-worn AR-hardware, including improvements in field of view (FoV) and display quality, are making Head-Mounted Displays (HMDs) and AR glasses increasingly feasible for in-car use \cite{riegler2021augmented}. Such technologies could increase the passenger experience by providing access to information about the outside environment \cite{BergerGridStudyInCarPassenger2021}. As such, modern cars offer a space for a wide array of non-driving-related activities such as communication, information access, and digital media consumption \cite{MatsumuraActivePassengering18}. Additionally, the rise of autonomous vehicles means that passengers will have more free time during transit \cite{mcgill2020challenges} e.g. for work \cite{Mathis2021work, medeiros2022shielding} or for leisure activities like gaming \cite{Togwell2022gaming}. 

With the importance of getting information about a passengers environment \cite{BergerGridStudyInCarPassenger2021, Mehrabian74EnvironmentalPsychology, Pfleging16NDRNeeds, BergerRearSeatDoor21}, future in-car AR systems could incorporate points of interest (POIs) to offer real-time, context-specific information about the car's surroundings overlaid onto the real-world. POIs provide users digital representations of real places such as restaurants, parks, or gas stations \cite{Psyllidis2022POIs} and can play important roles in supporting various human activities \cite{sun2023conflating}. Specific to the automotive context, POIs could also include traffic conditions, hazards, landmarks, and locations of interest. This integration of digital content with real-world context could also facilitate easier understanding of information at a glance \cite{haeuslschmid2016design} or could sharpen the riding experience \cite{MatsumuraActivePassengering18}.

To explore the optimal visualization of POIs using in-car AR we developed \textit{Blending The Worlds}, a system to display location-based data in the environment of a real moving vehicle via a pass-through HMD. The purpose of this system is to give users a way to explore their environment while using in-car AR as a passenger. We investigate possible ways to position, scale, and visualize POIs for this use-case. Based on our system, we want to answer the following research questions:

\begin{itemize}
    \item \textbf{RQ1:} \textit{What appearance and positioning of POIs do users prefer in an in-car AR context?}
    \item \textbf{RQ2:} \textit{Do users accept an AR system inside a moving vehicle?}
\end{itemize}

We answer these questions through a field study (N=38). Participants assessed multiple parameters for the conditions height, size, rotation, render distance, information density, and appearance. In addition, we gathered quantitative and qualitative data regarding the user acceptance and the user intention for the in-car AR-function. Based on our results, we provide insights into the design principles and best practices for optimizing the appearance and positioning of POIs in AR environments. Our system is designed primarily for passenger use, with a focus on rear-seat passengers, an underexplored area in automotive user interfaces \cite{BergerRearSeatDoor21}. However, our findings may also apply to other passenger positions. The main contributions of this paper are as follows:
\begin{itemize}
    \item A comprehensive examination of measures for POI positioning, rotation, scaling, render distance, information density, and appearance within automotive AR environments through a field study with 38 participants.
    \item Investigation of user acceptance for a HMD-based in-car AR system for displaying POIs.
    \item Guidance for deriving UX guidelines for visual appearances and behavior of location-based POIs for in-car AR.
\end{itemize}