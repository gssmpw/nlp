\section{System}
\label{sec:system}
Blending the Worlds enables passengers of a moving vehicle to explore their surroundings through digital POIs displayed in AR. Here, POIs are visualized as spheres outside the vehicle, as shown in Figure \ref{fig:teaser}. A video of the system is available in the supplementary material.

Our POIs fit the \textit{Label} pattern described by Lee et al. \cite{Lee24SituatedVisAR} for categorization of situated visualizations in AR. Labels are designed to provide additional context to referents, offering observers insights into aspects of the physical environment that are not easily accessible through conventional means. They also suggest that labels have the potential to become a key feature driving the success of AR in the near future. The widespread adoption of AR labels could have an impact on daily life comparable to the influence of spontaneous Wikipedia searches on everyday conversations. AR labels can also be dynamically optimized regarding placement and appearance. As such, we made our POIs customizable regarding the parameters height, size, rotation, render distance, and information density.

\begin{figure}[h]
    \centering
    \includegraphics[width=\linewidth]{Images/POI_design.png}
    \Description{The image displays two spheres set against a black background. Each sphere features a grey reflective outer border, with a smaller inner border that resembles a glowing neon tube, emitting a bright pink light. A bloom effect radiates from the glowing border of both spheres. Each sphere depicts a point of interest (POIs). The left sphere contains an icon of a spoon and fork, symbolizing a restaurant. The right sphere showcases an image of various types of Asian food. Both spheres also include a dark grey bar across the bottom third, displaying the name of the respective POI. Additionally, the right sphere features a four-star rating beneath the name.}
    \caption{Our proposed POI visualization in front of a black background. The left POI shows an icon representing a restaurant. The right POI shows a sample restaurant image and a rating from zero to five stars.}
    \label{fig:POI_Appearances}
\end{figure}

\subsection{Design}
\label{sec:design}
The design of our POIs is shown in Figure \ref{fig:POI_Appearances}. The core design features a sphere with a grey reflective outer border, with a smaller inner border that resembles a glowing neon tube, emitting a bright pink light. This visual design adheres to three key purposes defined by Zollmann et al. \cite{Zollmann2021ArVisTechniques}: visual coherence, exploration, and directing attention. Visual coherence is achieved by aligning the design with the aesthetic of our test vehicle, which incorporates similar styles in its interior and user interface. Exploration is achieved by providing contextual information for exploring the scene through the  information displayed on the POIs. Additionally, the design fulfills the purpose of directing attention, standing out in most environments due to its distinct, unnatural appearance, color, and form. A bright color is also recommended by Hertel and Steinicke \cite{SteinickeArMaritimePois2021}, especially for large distances in outdoor AR.

The other two key purposes defined by Zollmann et al. \cite{Zollmann2021ArVisTechniques} are clutter reduction and depth perception. Regarding clutter reduction, each of the five adjustable parameters for POIs (height, size, rotation, render distance, and information density) has the potential to influence clutter, as detailed in the following sections. Displaying more content provides additional information for the user; however, excessive content can lead to clutter, which may increase cognitive load \cite{kim2011multidimensional}. To support depth perception, our system primarily employs changing object size as a depth cue \cite{goldstein2009sensation}. Additionally, in some conditions, we utilize changing object height to indicate depth as well \cite{goldstein2009sensation}. Although drop shadows are frequently used to improve depth estimation for AR objects, most studies on AR visualization have been conducted in controlled indoor settings or open outdoor spaces \cite{erickson2020reviewOSTAr, Zollmann2021ArVisTechniques}. However, in our system, POIs are often rendered above or on top of buildings rather than on flat surfaces like streets. In such scenarios, the use of shadows could disrupt visual coherence. Moreover, pose estimation for a moving vehicle is significantly more complex than for a stationary observer or someone walking \cite{McGill22PassengXR}. This could lead to wrong positioning of shadows, potentially hindering depth estimation. 


\subsection{Height}
\label{sec:system_height}
POIs could be placed floating above their respective locations to indicate popular tourist destinations \cite{Lee24SituatedVisAR}, directly on buildings on street level, or directly on the street in front of buildings \cite{Ghaemi23ARPlacement}. POIs placed on street level may help to better estimate their position and respective buildings. In contrast, virtual POIs floating above their locations may help reduce clutter and acts as a depth cue while still conveying the existence of an interestsing location.

We define two adjustable parameters for POIs' vertical position: base height and dynamic height scaling. This allows for POIs to be displayed at any desired height and for optional distance-based height scaling. The base height determines the initial elevation of POIs above the ground. For POIs with \textit{static} height, no additional vertical scaling is applied. However, for \textit{dynamic} POIs, the height increases based on the distance to the user. This way, closer POIs are still placed on their target locations, while far POIs float above their target locations. Figure \ref{fig:HeightSizeDescription} shows a simplified illustration of the POIs' height behavior. The dynamic scaling is realized by using the POIs base height, the \textit{POI distance} and three additional parameters: a \textit{minimum distance threshold}, a \textit{maximum (max.) distance threshold}, and a \textit{maximum (max.) scaling}. The vertical position of POIs beyond the \textit{minimum distance threshold} is increased beyond their base height via the calculated \textit{scale} parameter from Equation \ref{eq:scalingFormula}.
\begin{equation} 
    \label{eq:scalingFormula}
        \text{scaling} = \left(\frac{\text{POI distance}}{\text{max. distance threshold}}\right)^2\cdot\text{max. scaling}
\end{equation}

\subsection{Size}
\label{sec:system_size}
AR labels can be dynamically optimized in their appearance for fitting the observer's information needs, e.g. by adjusting their scale \cite{Lee24SituatedVisAR}. Larger POIs direct more attention, while smaller POIs could reduce clutter. We define two adjustable parameters for POI size, similar to height: base size and dynamic scaling. The base size determines the POI spheres radius in meters. For statically sized POIs, the base size remaines unaltered. Consequently, POIs with static sizes appear smaller depending on their distance from the user, analogous to real-world objects. Dynamic scaling of size was achieved similarly to the dynamic scaling of vertical position, where we applied the scale calculated from Equation \ref{eq:scalingFormula} to the base size. Dynamically sized POIs still appear smaller the further away they are from the user, just with a lesser effect. Figure \ref{fig:HeightSizeDescription} shows a simplified illustration of the POIs' size behavior.

\subsection{Rotation}
\label{sec:system_rotation}
Our POIs consist of 3D models, consequently they can be rotated and face the user in different ways. With a \textit{billboarding} behavior, POIs rotate around the x- and y-axes (using a left handed coordinate system) to continuously face the user. Alternatively, our POIs can maintain their original orientation without any rotational adjustments. This entails no rotation around the x-axis and a y-rotation aligning the POIs face almost parallel to the street, akin to a street sign. Consequently, with no rotation, observers can see the sides and empty backs of POIs. This can potentially help to judge the side of the street a POI is on. In addition, this directs attention to POIs on the user's current street, as only their faces with further information is visible. Simultaneously, this could help reduce clutter, as POIs pointing in different directions still show an interesting location while not overwhelming the user with the bright circle, images, and text. Figure \ref{fig:RotationRenderdistanceDescription} shows the two rotation behaviors.


\subsection{Render Distance}
\label{sec:system_renderdistance}
Clutter from overlapping POIs needs to be taken into account, e.g. by reducing the amount of objects shown at once \cite{Lee24SituatedVisAR}. Thus, POIs in our system can be disabled at a certain distance with a minimum and maximum threshold. After crossing a threshold, POIs begin to fade in or out. The distance over which POIs fade can also be defined for both nearby and far POIs. The fading mitigates POIs abruptly appearing at the far edge of the render distance. Figure \ref{fig:RotationRenderdistanceDescription} shows an example for two render distances.


\subsection{Information Density}
\label{sec:system_informationdensity}
Labels are not limited to textual content and can contain independent visualizations or other content \cite{Lee24SituatedVisAR}. We display the name, a star-rating and an image on our POIs. In addition, the color of the glowing border is also changeable. Each type of content can also be disabled depending on the use-case. Adjusting the content of POIs could potentially influence visual clutter. Figure \ref{fig:POI_Appearances} shows two possible configurations for POI content.
 

\subsection{Hardware}
We use the Varjo XR-3 pass-through HMD (shown in Figure \ref{fig:teaser}) due to its specifications\footnote{Varjo Technologies Oy: Varjo XR-3, the first true mixed reality headset. \url{https://varjo.com/products/varjo-xr-3/} (accessed on 12.08.2024)} and compatibility with middleware from LP-Research\footnote{LPVR Middleware a Full Solution for AR / VR. \url{https://www.lp-research.com/middleware-full-solution-ar-vr/} (accessed on 12.09.2024)}. The Varjo XR-3 features a resolution of 70 pixels per degree in its focus area \cite{Kappler22VarjoEvaluation}, a FoV of 115, a refresh rate of 90Hz, and a pass-through latency of less than 20ms. It also supports six degrees of freedom (6-DoF) HMD tracking in a moving vehicle using middleware from LP-Research supported by an additional car-mounted inertial measurement unit. The Varjo XR-3 was connected to a desktop computer (CPU: Intel® Core™ i7-12700K,
RAM: Kingston Fury Beast 32 GB 3200 Mhz DDR4, Mainboard: Asus Z690 TUF gaming, Graphics Card: Gainward GeForce RTX 4080 Phoenix). The computer was secured in the trunk of the vehicle.