\section{User Study}
\label{section:study}
In this Section, we outline the pre-study and main user study aimed to assess parameters for visualizing POIs on an AR device based on our system described in Section \ref{sec:system}. This investigation is particularly focused on the unique context of a moving vehicle, an area that has not been extensively explored yet. The visualization parameters we examined encompassed height, size, rotation, render distance, information density, and appearance. Additionally, we explored the acceptance and intention of using AR technology in moving vehicles, specifically for the purpose of displaying POIs. We formed several hypotheses for each parameter described in Section \ref{sec:independentVariables}. 

We conducted an exploratory pilot study to establish default values for each independent variable. This pilot study employed the same apparatus as described in Section \ref{sec:apparatus} for the main study. For participants, we recruited five individuals with expertise in HCI and immersive technologies, selected through convenience sampling.

The procedure closely followed the approach detailed in Section \ref{sec:procedure} for the main study, with the primary modification being the omission of all questionnaires. Instead, participants were given the ability to adjust the parameters that constituted the variables outlined in Section \ref{sec:independentVariables} using interactive sliders. Each variable was adjusted individually and sequentially, following the order specified in Section \ref{sec:independentVariables}. The vehicle continued driving on our study track until the participant was satisfied with the adjustments for all sliders. The adjustable parameters for each condition were as follows:
\begin{itemize}
    \item \textbf{Height:} Base height, minimum distance threshold, maximum distance threshold, and maximum height scaling.
    \item \textbf{Size:} Base size, minimum distance threshold, maximum distance threshold, and maximum size scaling.
    \item \textbf{Rotation:} No sliders, just a choice between billboarding and no rotation.
    \item \textbf{Render distance:} Far edge distance, fading distance, far threshold, and near threshold.
    \item \textbf{Information Density:} Participants could turn on or off the name and the star rating. They also could toggle between an image and an icon.
\end{itemize}

For the analysis, we calculated the average values between the five participants, which correspond to \textit{low\_static} height, \textit{small\_static} size, \textit{billboarding} rotation, \textit{long distance} for render distance, and \textit{high information density}. The POIs in Figure \ref{fig:teaser} represent these default values.

% =====================================================
\subsection{Participants}
\label{sec:participants}
A total of 38 participants were recruited for the main-study, consisting of 20 males and 18 females, with an average age of 40.9 years (\textit{range: 20 to 61 years}). Among the participants, eleven ($28.95\%$) had no prior experience with AR, having never used AR glasses or HMDs. Ten participants ($26.32\%$) reported minimal experience, having engaged with AR apps or games on their smartphones. Fourteen participants ($37\%$) had limited exposure to AR glasses, using them 1-3 times, while 3 participants ($7.89\%$) were classified as experienced users, regularly utilizing such devices. Prior to the study, participants were asked to wear contact lenses if they required prescription eyewear. This recommendation aimed to ensure consistent comfort and to eliminate confounding the factor of hardware limitations as much as possible, as most glasses don't fit inside the Varjo XR-3 HMD. Among the participants, 27 individuals ($ 71.05\%$) did not require any prescription, while the remaining 11 participants ($28.95\%$) adhered to the recommendation and used contact lenses during the study. As such, our study encompasses a diverse range of participants regarding age and familiarity with AR systems.

Participants were also asked to indicate how frequently they experience MS while engaging in secondary tasks as a passenger in a moving vehicle. They could answer on a 5-point likert scale ranging from \textit{never} to \textit{(almost) always}. Fourteen participants ($36.84\%$) reported never experiencing MS, nine participants ($23.68\%$) reported rare occurrences, eleven ($28.95\%$) reported occasional sickness, and four ($10.53\%$) reported experiencing MS often or almost always. None of the participants aborted a study session due to MS.


% =====================================================
\subsection{Apparatus}
\label{sec:apparatus}
Participants were positioned in the right rear seat of a midsize sedan. The front right seat was adjusted to its forwardmost position to allow for optimal head-tracking and to ensure the participants' safety. Participants used an Xbox Elite Wireless Controller as the input device for responding to questionnaires. Two additional occupants accompanied the participants during the study. Apart from the driver, the experiment conductor occupied the left rear seat of the vehicle. Positioned there, the experiment conductor could view the participant's perspective on a screen and documented all relevant observations throughout the study. To ensure realistic and controlled driving conditions, we chose a private, industrial area with moderate traffic, including other vehicles and pedestrians. To maintain uniform driving conditions, the car's speed limiter was set to the maximum allowable speed within the study environment, capped at 30 km/h.

\begin{figure}[ht]
    \centering
    \includegraphics[width=.6\linewidth]{Images/Schematic_visualisation_fov.eps}
    \Description{A schematic illustration of the participants' field of view during the study. Two red lines originating from a small cars' are shown. The lines have a 115 degree angle between them. In front of the car, three POIs are shown, one on the left, two on the right. The right on near the car is only partially inside the red lines field of view. The other two POIs are inside the field of view.}
    \caption{Schematic visualization of participants' field of view during the study. The red lines show the 115\textdegree{} field of view of the Varjo XR-3. The relative position and scale of the POIs and the car in the image are true to scale.}
    \label{fig:TopdownSchematics}
\end{figure}


% =====================================================
\subsection{Procedure}
\label{sec:procedure}
A complete study session for one participant took approximately one and a half hours. The questionnaires are located in their entirety in the appendix. At the beginning of the study session, the participant was welcomed and taken to the designated study environment. First, the participant provided informed consent regarding the management of their privacy and personal data. Subsequently, the study conductor delivered a presentation on the basics of AR and POIs utilizing presentation slides. This was followed by the pre-questionnaires, which are described in further detail in Section \ref{sec:measures}. Afterwards, the participant could enter the study vehicle and was driven to the study's starting point. There, they were briefed on the procedural aspects of the in-car study and instructed on the operation of the HMD. Additionally they were informed about the potential for MS, including the procedures to follow in the event of experiencing such symptoms. Afterwards, they put on the HMD and were given the controller to start a round of acclimatization with the AR function activated. During the acclimatization, the HMDs pass-through mode was activated and POIs were displayed next to the street while the car was driving one lap through the study environment. For the acclimatization, the default values described in Section \ref{sec:independentVariables} and shown in Figure \ref{fig:teaser} were used. Next, the study conductor read out the user story to the participant. The scenario depicted a potential future situation in which the user, accompanied by two colleagues, is on a business trip, driving through a city with a significant distance remaining in their journey. In their quest to find a nearby place for lunch, the user utilizes a new AR function, requesting the vehicle to display restaurants along their route. Subsequently, they can observe relevant targets within their environment.

After this, the study procedure began. During each round, POIs were shown outside the car alternating between the left and right side of the road. The POIs were placed in world-space, with each of them possessing specific lat-long coordinates derived from the street's position. The POIs resembled restaurants as described in the user story with varying information, positioning, and appearance depending on the study state. The restaurants shown on the POIs consisted of simulated data and did not correlate with the buildings seen in the real environment, since the study took place in a private industrial area with no real restaurants nearby. Instead, the POIs were equally distributed with alternating positions to the left and right sides of the street. POIs were placed with a distance between five and ten meters measured from the center of the street. Figure \ref{fig:TopdownSchematics} shows the relative position and scale of the POIs and the car together with the field of view the participants had without moving their head.



% =====================================================
\subsection{Independent Variables}
\label{sec:independentVariables}
There were varying predefined states for each of our independent variables: height, size, rotation, render distance, and information density. For the study's default values, the mean values of the pilot study described in Section \ref{section:study} were used. 
For each round, participants were told which independent variables were modified and on what they needed to concentrate on. However, they were not told in what way the variables were modified. The order for the independent variable categories was consistent for each participant and followed the order of the following paragraphs. The conditions within the categories were counterbalanced using latin square.


\begin{figure*}
    \centering
    \begin{subfigure}[b]{.49\textwidth}
        \centering
        \includegraphics[width=\textwidth]{Images/Schematic_Height.eps}
    \end{subfigure}
    \hfill
    \begin{subfigure}{.488\textwidth}
        \centering
        \includegraphics[width=\textwidth]{Images/Schematic_Size.eps}
    \end{subfigure}
    \caption{Schematic representation of POI placement and appearance in the four study conditions regarding height (left) and size (right).}
    \Description{A schematic illustration of the height and size conditions. The Figure consists of two images, each showing a two by two matrix. Each of the eight squares in the two matrices shows a 2D graphic of a car in front of three points of interest each. On the left matrix, the POI height is displayed. An arrow in each image visualizes how the points of interest differ in their vertical position. Low base height coupled with static height shows the points of interest at the height of the car. High base height coupled with static height shows the points of interest above the car. Low base height coupled with dynamic height shows the points of interest on a curved line, starting the car's height and going above the car. High base height coupled with dynamic height shows the points of interest on a curved line, starting above the car and getting even higher the further away they are from the car. On the right, the POI size is displayed. A scale in each image visualizes how the points of interest differ in their size. Small base size coupled with static size shows the points of interest, roughly sized as two thirds of the car. Large base size coupled with static size shows the points of interest roughly as large as the car. Small base size coupled with dynamic size shows the points of interest getting bigger the further away they are from the car. They start a little smaller than the small base size and get larger than the large base size. Large base size coupled with dynamic size shows the points of interest again scaling depending on the distance to the car. They start as large as the car and get almost doubled in size.}
    \label{fig:HeightSizeDescription}
\end{figure*}

\subsubsection*{\textbf{Height}}
The POIs' height attribute is described in Section \ref{sec:system_height}. For the study, we adjusted both the base height and the dynamic scaling. Figure \ref{fig:HeightSizeDescription} illustrates the four conditions for height \textit{low\_static}, \textit{low\_dynamic}, \textit{high\_static}, and \textit{high\_dynamic}. Our dependent variables for height are satisfaction, visibility, and pleasantness. The low base height conditions positioned POIs at the user's eye-level while the high base height conditions set POIs to hover 15 meters above ground-level. For all dynamic height trials, the minimum distance threshold equaled to 30 meters, the maximum distance threshold to 500 meters, and the maximum height scaling to 100 meters. Those values were based on the satisfaction factor in our pilot study. For POI distances smaller than the minimum distance threshold the \textit{scaling} equaled 1, for POI distances larger than the maximum distance threshold the \textit{scaling} equaled the maximum height scaling. 

We expected the conditions with low base height to be the preferred conditions, as the deployed Varjo XR-3 weights around 980g\footnote{\label{foot:Varjo}\url{https://varjo.com/products/varjo-xr-3/} (accessed on 12.08.2024)} and can potentially cause head strain while being used in a moving car \cite{Schramm23Assessing}. As such, the placement of POIs at approximately eye-level could be more comfortable. Also, with a low base height, there should be a more direct association with digital POIs and the real world. As such, we hypothesized that POIs with a low base level lead to higher visibility. Dynamic scaling of POIs should help orient the user and help decluttering the FoV, potentially leading to higher pleasantness. However, this could come at the cost of a lower association between POIs and their location in the real world. In contrast, a high base height could reduce visual clutter on eye-level while giving a broad overview over the nearby POIs. Hence, we assumed the following hypotheses for \textit{height}:
\begin{itemize}
    \item $H_{H1}$: A base height of approximately eye level leads to higher satisfaction.
    \item $H_{H2}$: A base height of approximately eye level leads to higher visibility.
    \item $H_{H3}$: Dynamic height scaling leads to higher pleasantness.
\end{itemize}



\subsubsection*{\textbf{Size}}
The POIs' size attribute is described in Section \ref{sec:system_size}. For the study, we adjusted the base size and the dynamic scaling. Figure \ref{fig:HeightSizeDescription} illustrates the our four size conditions \textit{small\_static}, \textit{small\_dynamic}, \textit{large\_static}, and \textit{large\_dynamic}. Our dependent variables for size are satisfaction, visibility, and pleasantness. A low base size equated to a POI diameter of 3 meters, while a large base size equated to 7.5 meters. For all dynamic size trials, the minimum distance threshold equaled to 50 meters, the maximum distance threshold to 500 meters, and the maximum size scaling factor to 7. The values are based on the satisfaction factor in our pilot study.

We expected the conditions with low base size to be the preferred conditions with the highest satisfaction and pleasantness, as they don't obstruct a big portion of the outside view and thus could positively impact the experience \cite{BergerGridStudyInCarPassenger2021}. Additionally we hypothesized that the larger base size leads to higher visibility, as the larger POIs could improve readability, especially for POIs that are located further away. Also, the dynamic scaling could improve satisfaction and visibility for POIs across all distances as they adapt based on the distance to the user. They still appear smaller the further away they are, showing somewhat realistic behavior while not overly cluttering the FoV. Thus, we assumed the following hypotheses:
\begin{itemize}
    \item $H_{S1}$: A POI base size of approximately three meters leads to higher \textit{satisfaction} and \textit{pleasantness}.
    \item $H_{S2}$: POIs with a large base size of approximately 7.5 meters lead to higher \textit{visibility}.
    \item $H_{S3}$: Dynamic size scaling leads to higher \textit{visibility} and \textit{pleasantness}.
\end{itemize}



\begin{figure*}
    \centering
    \begin{subfigure}[b]{.55\textwidth}
        \centering
        \includegraphics[width=\textwidth]{Images/Schematic_Rotation.eps}
        \Description{}
    \end{subfigure}
    \hfill
    \begin{subfigure}{.4\textwidth}
        \centering
        \includegraphics[width=\textwidth]{Images/Schematic_RenderDistance.eps}
        \Description{A schematic illustration of the rotation and render distance conditions. The Figure consists of four images. Two images on the left represent the rotation condition. Each shows a car with four points of interest from above. For billboarding, all four points of interest are rotated towards the car's passenger seat, indicated by four arrows. For no rotation, each of the four points of interest a rotated paralell to the car's direction, again indicated by four arrows. The two images on the right represent the render distance. Each of them shows a car with points of interest from the side. For the short render distance, four points of interest are shown, where the last one is half transparent. For the long render distance, six points of interest are shown.}
    \end{subfigure}
    \caption{Schematic representation of the two study conditions manipulating the POIs' rotation (left) and render distance (right).}
    \label{fig:RotationRenderdistanceDescription}
\end{figure*}

\subsubsection*{\textbf{Rotation}}
The POIs' rotation is described in Section \ref{sec:system_rotation}. We tested two conditions regarding rotation in the study: \textit{billboarding} and \textit{no rotation}, as illustrated in Figure \ref{fig:RotationRenderdistanceDescription}. For the rating of the rotation, we used four word pairs for clarity, support, complexity, and pleasantness. The word pairs were taken from the short version of the User Experience Questionnaire \cite{schrepp2017design} with the exception of pleasantness, which we formulated ourselves.
  
We expected the billdboarding behavior to be the more supportive condition since there the POI-content is consistently available and readable. This could improve information delivery and thus be perceived as more pleasant to use. The non-rotating POIs may be easier to understand, as they resemble the static behavior known from real-life street signs. Additionally, they could support users by conveying information about the streets' direction and provide a clearer association between POIs and streets. As such, our hypotheses are as follows:
\begin{itemize}
    \item $H_{R1}$: Billdboarding POIs are more supportive for delivering information and are thus more pleasant.
    \item $H_{R2}$: Non-rotating POIs are easier to understand and give a clearer overview of the environment.
\end{itemize}


\subsubsection*{\textbf{Render Distance}}
The render distance described in Section \ref{sec:system_renderdistance} had two conditions in the study: \textit{long distance} and \textit{short distance}, as illustrated in Figure \ref{fig:RotationRenderdistanceDescription}. For the \textit{long distance} condition, the far edge to fade-in POIs was chosen at a distance of 500 meters. This resulted in all existing POIs to be displayed at all times, since our study environment had a maximum lenght of around 450 meters. For the \textit{short distance} condition, the far egde was set to 150 meters, resulting in three to four POIs being visible simultaneously while traversing a straight street segment. The fading distance was set to 50 meters for the far threshold and to 2.5 meters for the near threshold. For the render distance, participants could rate the POIs' time of appearance, ranging from \textit{way too early} to \textit{way too late}.

For render distance, we expected the \textit{short distance} to be preferable due to the reduced visual clutter and the difficulity to read far away POIs. Thus, our hypothesis regarding this variable is:
\begin{itemize}
    \item $H_{RD1}$: A short render distance is preferred by users.
\end{itemize}


\subsubsection*{\textbf{Information Density}}
There were two levels of content, resulting in two conditions tested: \textit{Low information density} and \textit{high information density}. Here, our dependent variable was satisfaction. For the \textit{low information density} condition, only a generic restaurant icon and the restaurant name was displayed. For the \textit{high information density} condition POIs showed the restaurant's name, an image, and a star rating ranging from zero to five. Examples for both conditions can be seen in Figure \ref{fig:POI_Appearances}. There was also the possibility to rate each part of the POIs' content individually during the post-questionnaires. Participants were shown a 3x4 matrix with the x-axis being the POI components name, star rating, icon, and image. The y-axis comprised of points in time on when the components could be shown: always, when nearby, and never. The matrix, including the results, is illustrated in table \ref{tab:InformationMatrix}.

We expected the \textit{high information density} to be the preferred condition, since it provides the most relevant information at a glance without requiring any additional interaction. Additionally, we hypothesized that more information should be displayed for near POIs, since far away POIs are less readable.

\begin{itemize}
    \item $H_{I1}$: POIs with three types of data are preferred by users.
    \item $H_{I2}$: Users want to have more data displayed for nearby POIs.
\end{itemize}



\subsubsection*{\textbf{Appearance}}
Our POIs comprise of spheres with one cut side, allowing for a flat space to display 2D information on. While facing the user in the billboarding conditions, the POIs appeared as two dimensional objects, as seen in Figure \ref{fig:POI_Appearances}. Most of the POIs face is filled with either a representative image stemming from the real location or a icon indicating the POIs' category. We chose a deliberately artifical look for the POIs' appearance, form, and color to make them stand out against the environment and to be visually interesting. As such, our hypotheses regarding appearance are as follows:
\begin{itemize}
    \item $H_{A1}$: The visual presentation of our POIs is appealing.
    \item $H_{A2}$: The POIs' color makes them stand out against the environment. 
\end{itemize}



% =====================================================
\subsection{Measures}
\label{sec:measures}
The participants were administered questionnaires before, during, and after the in-car study. The questionnaires that were specifically formulated for our study will be described on an abstract level in this Section and are located in their entirety in the appendix. All relevant results for the questionnaires are reported in Section \ref{section:results}, with detailed results also found in the appendix.

\subsubsection*{\textbf{Pre-Questionnaires}}
The pre-questionnaires included a socio-demographic questionnaire and two questionnaires regarding the participants affinity for both new technologies in general, as well as their affinity for AR. The pre-questionnaires were administered in a separate room and were filled out via keyboard and mouse on a computer. 
The socio-demographic questionnaire included questions about the participants' gender, age, body height, AR experience, their need for prescription glasses, and their suspectibility for MS. The results of the socio-demographic questionnaire are reported in Section \ref{sec:participants}.
For technical affinity, we used the affinity technology interaction scale (ATI) from Franke et al. \cite{franke2019personal}. For questions regarding participants' affinity specifically for AR, we used a mixture of self formulated questions and questions from Janzik \cite{janzik2022studie} with slight modifications.


\subsubsection*{\textbf{Study-Questionnaires}}
During the in-car part of the user study, participants could fill out the questionnaires for the dependent variables through an AR-interface included in the study software without the need of removing the HMD. The questions were always filled out while the car was in standstill, since the UI for the questions occluded most of the HMD's FoV and thus could induce MS \cite{Sasalovici23ArMs}. The questions were tailored to the conditions and mostly used seven-point Likert scales. In addition, all relevant verbal comments that the participants made during the in-car part of the study were noted in connection to the currently viewed condition. Those comments were later condensed into comment categories, which then got sorted into positive, neutral, and negative categories. The ratings for each condition are described respectively in Section \ref{sec:independentVariables}.


\subsubsection*{\textbf{Post-Questionnaires}}
After the study procedure in the car, there was no post-interview, participants directly filled out the post-questionnaires. These consisted of the acceptance regarding the AR-function, the POI appearance, and the intention of use. Like the pre-questionnaires, the post-questionnaires were administered in a separate room and were filled out via keyboard and mouse on a computer. All the post-questionnaire questions were self formulated and are listed in the appendix.

First, participants could rate their acceptance of the used AR function in general, using seven point Likert scales. For instance, some questions revolved around determining whether the AR function is deemed useful, innovative, or exciting. Afterward, participants were provided with two free-text fields where they could express their opinions on what they particularly enjoyed and disliked about the AR functionality.

The next set of questions regarded the POIs appearance, form, and color with seven point Likert scales. If the participant gave a negative rating with three and below, they were asked to fill out a freetext field why they were dissatisfied with this aspect of the POIs appearance. Then, for all participants, there were two additional freetext fields. There the participants were asked if they want to change any aspects of the POIs appearance other than those listed before. The other field had space to add any additional information or data to the POIs content that was not in the study.

The last set of questions regarded the intention of use for the AR-function. The first question regarded the preferred seat position depending on the cars' automation level. As such, the possible answers consisted of a 3x4 matrix where one axis was the seat position (driver, codriver, backseatpassenger, none) and the other axis was the automation level (level 3, level 4 and above, no automation). Multiple answers were possible. Participants were shortly briefed on what the automation levels meant beforehand.
Then, participants could rate if they want to use the AR function for different kinds of POIs, like theatres or gas stations from a list of nine possibilities. Afterwards they could pose their own kinds of POIs. 