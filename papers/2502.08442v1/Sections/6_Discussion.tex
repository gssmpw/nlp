\section{Discussion}
\label{section:discussion}
In this section, we discuss the results of our experiment in relation to the formulated hypotheses. The discussion will follow the order of tested parameters - height, size, rotation, render distance, information density and appearance. Furthermore, we summarize our discussed results in form of guidelines for designing POIs for in-car use, followed by user acceptance, user intention, and the limitations of our work.



\subsection{Height}
The first subsection of our experiment was focussed on the POIs' vertical positioning above ground level. Here, our results presented in Section \ref{sec:heightResult} confirm hypothesis $H_{H1}$ (A POI base height of approximately eye level leads to higher satisfaction). This result appears correlated with the perceived \textit{pleasantness} of the height setting, which likewise was rated most favorably for the base height setting on eye level. This is in line with best practices for spatial UI design\footnote{Riley Hunt. 2023. Spatial UI Design: Tips and Best Practices: \url{https://www.interaction-design.org/literature/article/spatial-ui-design-tips-and-best-practices} (accessed on 07.12.24)}, where the ideal content area starts slightly below eye-level. This claim is further supported by user comments, as the most counted commented categories for conditions with a low base height included "pleasant and fitting" ($N=19$) and "this height is good in general" ($N=11$). Additionally, three participants commented the "natural headpose" ($N=3$) in the \textit{low\_static} condition. In contrast, the conditions with high base height received mostly negative comments like "too high" (static: $N=23$; dynamic: $N=21$), "the position of POIs is difficult to estimate" (static: $N=9$; dynamic: $N=9$), and "uncomfortable head position" (static: $N=6$; dynamic: $N=6$). 

For \textit{visibility}, no significant interactions were found, contradicting hypothesis $H_{H2}$ (A base height of approximately eye level leads to higher visibility). Interestingly, all settings were perceived to be equally visible, albeit lower base heights were generally preferred. There were limited comments about visibility and the comments that did touch on visibility presented contrasting viewpoints among participants. For instance, four participants expressed positive views, such as "my view of the POIs on the car's roof is unobstructed." Conversely, three participants conveyed a similar comment, only with a negative connotation with "the POIs are on the car's roof." 

The dynamic scaling of POIs did not have any significant effect on our participants ratings. While the dynamic scaling of POIs received positive feedback for circumventing occlusion-related issues at a distance, the chosen maximum height at a distance was deemed excessively high. This is reflected in comments such as "uncomfortable head position" ($N=5$). Thus, we reject hypothesis $H_{H3}$ (Dynamic height scaling leads to higher pleasantness).

In summary, a low base height approximately set at eye level is preferred by users, resulting in significantly higher satisfaction and comfort. Additionally, we recommend configuring dynamic height scaling based on user preferences. While dynamic scaling can potentially mitigate occlusion issues at a distance, it may come at the cost of an uncomfortable head position.


\subsection{Size}
Subsequently, we evaluated various POI scaling options, again considering two levels for base scale with or without dynamic scaling over distance. 

Based from our results for size in Section \ref{sec:sizeResults} confirm hypothesis $H_{S1}$ (A POI base size of approximately three meters leads to higher \textit{pleasantness}). However, the pleasantness ratings were still not optimal as the mean ratings for pleasantness did not exceed one of a maximum of three. This can be explained by the comments made by participants, as there was no clear consensus for a favorite size. Positive comments noted that the size in the \textit{small\_dynamic} condition was deemed "exactly right, fitting" ($N=10$). Conversely, some participants remarked on personal preference, citing POIs being "too small when far away" ($N=5$), "too big when far away" ($N=2$), "too big" ($N=3$), or "too small" ($N=2$). In the \textit{small\_static} condition, POIs were predominantly perceived as "too small" ($N=9$) or "too small when far away" ($N=9$). Consequently, text on POIs was deemed "difficult to read" ($N=5$) by certain participants. Nevertheless, other found this condition pleasing, describing it as "pleasant, exactly right" ($N=6$).

We reject $H_{S2}$ (POIs with a large base size of approximately 7.5 meters lead to higher \textit{visibility}). The large base size proved excessive for many participants, with comments for \textit{large\_static} including remarks such as "too big" ($N=11$), "too big when nearby" ($N=5$), and "POIs occlude the real world too much" ($N=4$). However, five participants favored this condition, and another three achknowledged the ease of reading text.
The \textit{large\_dynamic} exacerbated issues, with 19 participants indicating that the POIs were "too big" in general ($N=12$) or "too big when nearby" ($N=7$).

We confirm $H_{S3}$ (Dynamic size scaling leads to higher \textit{visibility} and \textit{pleasantness}). This could stem from better readability, as ten participants called the \textit{small\_dynamic} as "exactly right", and five participants calling the \textit{large\_dynamic} "exactly right". Dynamically sized POIs also don't seem to influence the participants depth perception, which is in line with the findings of Dey et al. \cite{dey2012tablet}, where correct depth estimation in handheld AR was mostly dependant on object height and not size.

In summary, we recommend to scale POIs at a radius of roughly three meters with distance-based dynamic scaling as proposed in Equation \ref{eq:scalingFormula} for good satisfaction, visibility, and pleasantness. In addition, some personalization should be considered. Each condition regulating the POIs' size got at least five positive remarks, as such making all conditions suitable for at least a subset of users. Some seem to favor less visual clutter of smaller POIs while some participants seem to favor bigger POIs for better readability.



\subsection{Rotation}
For rating the rotation of POIs we used word pairs for the categories clarity, support, complexity, and pleasantness. The billdboarding condition was rated significantly higher within each word pair. Thus we confirm hypothesis $H_{R1}$ (Billdboarding POIs are more supportive for delivering
information and are thus more pleasant), as they were rated significantly better in support and pleasantness. This was also mirrored in the participants comments during the study. Comment categories for \textit{billboarding} included "good" ($N=10$), "pleasant" ($N=8$), and "everythign is readable" ($N=4$). The only negatively coded comments for this condition were from three people, that said "the rotation is a little too late".

The \textit{no rotation} condition was rated significantly worse than \textit{billdboarding} in each category and received mostly neutral to negative ratings. Thus, we reject hypothesis $H_{R2}$ (Non-rotating POIs are easier to understand and give a clearer overview of the environment). Participants either did not like the three dimensional POIs or did not understand what information they could gain by them. As such, the condition \textit{no rotation} got mostly negative comments in lines of "i don't like that the POIs are visualized as spheres" ($N=9$), "useless" ($N=7$), "confusing" ($N=7$), "there is nothing gained for me when I see the spheres from behind" ($N=5$), and "unpleasant" ($N=4$). In addition, eleven participants directly told us that they liked the billboarding condition better. Nevertheless, six participants liked the "visualization as spheres". The comments also show, that some participants did perceive the billboarding spheres as 2D objects while the non-rotating spheres where perceived as 3D objects.

In summary, we clearly recommend for POIs to show billboarding rotation behavior, meaning they should always face the user. This leads, according to our data, to better clarity, support, complexity, and pleasantness compared to non-rotating POIs. Still, the obtained ratings were not optimal and other methods of rotation should be evaluated further. This could for example encompass rotating the POIs with slight anticipation for the cars heading vector. Another potential direction could be a small idling rotation in combination with a subtle hovering animation to give more of a 3D impression to the POIs without confusing users like with our static spheres.



\subsection{Render Distance}
While there was a significant effect of the render distance on the dependent variable ($t(37)=7.29; p<.001$), both conditions were suboptimal. The \textit{long distance} was rated to render POIs too early ($M=-0.789; Mdn=-0.5; SD=1.32$) while for the \textit{short distance}, POIs rendered too late ($M=-0.684; Mdn=1.0; SD=1.07$). As such, the optimal point to fade in POIs seems to be somewhere between 150 and 450 meters. Thus, we reject our hypothesis $H_{RD1}$: (A short render distance is preferred by users). 

This is also mirrored in the comments participants made. For the \textit{long distance}, participants liked the "good overview" ($N=8$) and the "information about restaurants farther away" ($N=4$). Though, nine participants commented about "information overload" and "too many POIs" ($N=8$). Additionally, six people complained, that POIs in the distance are not readable and that some POIs occluded each other ($N=2$). 

For the \textit{short distance}, participants positively noted the "lower quantity of POIs" ($N=8$), "pleasantness" ($N=6$), and "no information overload" ($N=5$). However, eight participants reported that the POIs get displayed "a little too late" and that there were "not enough POIs or limited options" ($N=5$). For both conditions there were four participants that emphasized that the render distance should be dependent on location and circumstances.

In summary, both the 150 meters and the 450 meters rendering distance were suboptimal for our conditions, namely, an industrial area with speed capped at 30 km/h. As such, a value in between our chosen values, e.g. 300 meters, could be fitting for the conditions employed in our study. However, this value has to adapted to the circumstances as well as to the users preferences.



\subsection{Information Density}
For the amount and types of content displayed on the face of POIs, participants clearly preferred our condition with more content with a significant main effect between the two conditions. The \textit{high information density} condition was close to ideal ($M=-0.378; Mdn=0; SD=0.861$) but still showed a trend of not enough information. Those results are also mirrored in the recorded comments, as the participants expressed mostly negative comments regarding the \textit{low information density}. These included "too little information" ($N=14$), "star rating is missing" ($N=9$), "image is missing" ($N=8$), and "ony the name is not enough" ($N=4$). For the \textit{high information density} condition, more positive comments were expressed, like "I like it" ($N=7$) and "informative" ($N=3$). Still, users wished for even more information displayed ($N=12$) or that the image and name only gets displayed for nearby POIs ($N=6$). For both conditions, four participants respectively mentioned the cut-off name of some restaurants. Thus, we partially confirm hypothesis $H_{I1}$ (POIs with three types of data are preferred by users), as our condition with three types of data received overall positive ratings. However, more information was wanted by some participants.

For the name, star rating and image, a trend can be identified for showing more information on nearby POIs and less information for far POIs, confirming hypothesis $H_{I2}$ (Users want to have more data displayed for nearby POIs). However, when to display the icon was polarizing, with 40\% requesting to always show the icon and with 40\% requesting to never show the icon. This could stem from the participants wishes to display the category of POIs, such as restaurants. As such, the icon visualizing the type of POI could be added to our existing three data types. On the contrary, 40\% never want the icon to displayed. From the user comments we can deduce, that some users rather want to have the image displayed at all times.

In the post-questionnaires, participants could also mention additional types of information that they would want to see on POIs. Those informations like "distance to the POI" ($N=7$) could be also displayed directly on the POIs. Another possibility would be to include a functionality to select POIs as proposed in \cite{Schramm23Assessing}. After selection, some of the user-commented categories could be shown, such as "business hours" ($N=11$), "reservations" ($N=5$), "price range" ($N=4$) or a "call button" ($N=4$).

In summary, we advice to at least display the POI name, rating and image always or when nearby. Furthermore, some other information such as the POI category should be displayed as well. In addition, POIs could be made selectable to then display more information like opening times.



\subsection{Appearance}
The first part of our post-questionnaires was focussed on the POIs appearance. Here, users could give additional information via text when they rated an aspect with a three or lower. Overall, the POIs general appearance and form were rated highly. Thus, we confirm hypothesis $H_{A1}$ (The visual presentation of our POIs is appealing). Comments explaining their negative rating regarding the appearance included: "I wish for more focus on the content" ($N=3$), "the border should be more subtle" ($N=3$), and "Some POIs are difficult to differentiate" ($N=2$). As for the color of POIs, the results were more polarizing with 37\% of participants rating the color three and lower, while 47\% rated the color a five and better. In negative comments participants called the color "too prominent" ($N=9$) or wished for more variety in colors ($N=3$). Thus, we can confirm $H_{A2}$ (The POIs' color makes them stand out against the environment) as the color indeed lets the POIs stand out. However, for some participants the color stood out too much and was too prominent as described by their comments and reflected in the ratings. In addition, participants could comment any aspects of the POIs' appearance that they would want to change. Eight participants wished, that the name of restaurants would be readable. This propably stems from the problem that some of our fake-POI data had long restaurant names which were then cut off due to limited allocated space for displaying the name. Six participants wanted a different form like a pin needle, a drop with the tip facing down, or a rectangular form for more text space. Four participants wanted for the POIs to differentiate more, such as color coded restaurant categories.

In summary, our POI design has promising aspects as the appearance and form were generally liked. Thus, round or spherical POIs with a futuristic look are a promising baseline. However, in future work, more designs, colors, forms and layouts should be explored. A promising approach would be to allow some user-customization of POIs' appearance, form, color, and content, potentially connected to the cars digital design in some way. Furthermore, the use of design elements to differentiate between kinds or categories of POIs could help with exploration and orientation.


\subsection{Preferred Appearance and Positioning}
Based on our results and discussion, we can answer research question \textbf{RQ1} (What appearance and positioning of POIs do users prefer in an in-car AR context). We recommend the following baselines for visualizing in-car POIs based on the user preference in our study:
\begin{itemize}
    \item \textbf{Height:} A low base-height set around eye-level with dis\-tance\--based vertical scaling depending on the context and the user's preference.
    \item \textbf{Size:} A scale of roughly 3 meters with a small amount of distance-based scaling.
    \item \textbf{Rotation:} Billboarding POIs, meaning a continuous rotation around the x-axis and y-axis in order to always face the user's head position.
    \item \textbf{Render Distance:} Far away POIs should fade out after a certain distance depending on the car's environment and speed. For urban areas with a speed of 30km/h we recommend to fade-out POIs at a distance of around 300 meters. However, more research in this area is needed.
    \item \textbf{Information Density:} POIs should at least display the location's name, a representative image, and a rating if applicable. In addition, POIs should somehow indicate their category such as restaurant or gas station, e.g. through an icon. More information based on the POIs category is welcome and should be made available in some way, e.g. through further interaction.
    \item \textbf{Appearance:} We recommend to base the POIs' appearance on other designs that are used in the space where the function is deployed. Users should be able to customize the color of POIs and should potentially also be able to customize the form. Spherical POIs with a futuristic design were rated positively in our scenario.
\end{itemize}


\subsection{Acceptance}
The in-car AR-function was generally accepted by users. Each statement in our acceptance questionnaire was rated in median with a five or higher. Notably the statements for regular use of the AR-function, both as vehicle owner and for other passengers, were rated positively with a median score of six. In addition, most participants agreed with a median score of six, that the AR-function excites them, that it is useful, and that they expect added value from it. Thus, based on our data, we can answer our research question \textbf{RQ2} (Do users accept an AR system inside a moving vehicle?) with yes. Our participants accepted the AR-system inside the moving vehicle with genereally positive results.

As seen in the comments in table \ref{tab:AcceptanceComments}, participants particularly liked the information gain about their environment and the help for searching, as suggested in \cite{BergerGridStudyInCarPassenger2021}. Negative comments mostly regarded the visual design and the information overload, both of which should be studied further based on the insights in this paper.

Regarding the occurrence of MS, none of the participants reported experiencing any symptoms, and no study runs were aborted due to MS.


\begin{table}[h]
    \centering
    \caption{Coded comment categories regarding what the participants liked or disliked about the AR-function in general.}
    \label{tab:AcceptanceComments}
    \begin{tabular}{l|l|c}
    \toprule
    \textbf{Question} & \textbf{Comment Category}  & \textbf{N}  \\
    \midrule
    Liked     & Information gain            & 18 \\
             & Search help                  & 8  \\
             & Innovative                   & 4  \\
             & Visual design                & 4  \\
             & Less distracting             & 3  \\
             & Other                        & 6  \\

    \midrule
    Disliked & Visual design                & 8  \\
             & Information overload         & 8  \\
             & Digital occludes real world  & 6  \\
             & Not enough information       & 4  \\
             & No added value               & 2  \\
             & Other                        & 8  \\
    \bottomrule  
    \end{tabular}
\end{table}

\subsection{Intention of Use}
\label{sec:intentionOfUse}
The participants intention of use was rather pragmatic, as chargepoints, gas stations and supermarkets were amongst the highest rated categories for POIs. Among the presented infotainment use-cases, only the "landmarks in unknown cities" use-case was rated rather highly with a mean score of 5.9. Thus we confirm the user acceptance of this infotainment use-case, as Berger et al. showed in their work \cite{BergerRearSeatDoor21}. For the additional comments, participants also mentioned parking ($N = 6$) and hospitals ($N = 5$), again pragmatic use-cases that can support them in their daily life.

In summary, we advise to focus on POI categories that may help users in their daily life for a helpful assistance system. Though, the use of POIs to entertain passengers during trips in unknown cities is also a promising approach.


\subsection{Limitations}
Given that our study was conducted in a singular environment with a fixed speed limit, the applicability of some of our results may be limited to this specific context and may not translate to other settings such as traffic-dense urban areas or highways. Especially some of our variables like the render distance or size would likely require adaptation to different contexts. Furthermore, our POI appearance was confined to a single spherical design, which could restrict the generalizability of our findings to alternative design choices. Aditionally, the chosen scenario itself could have influenced the results.

Additionally, the Varjo XR-3 hardware has some limitations. Participants wore the HMD continuously for 30 to 60 minutes, potentially leading to neck strain due to the device's weight. This factor could have influenced our results, particularly considering our emphasis on hedonic qualities like satisfaction and pleasantness. For comparison, a study by Kim et al. \cite{kim2021Discomfort} found that while physical discomfort and simulator sickness symptoms emerged approximately twice as quickly when using a VR headset compared to a desktop monitor, all participants were able to complete the 60-minute task without significant discomfort. Future research could explore a comparison between VST and optical see-through devices for in-car applications.

Also, as our study was conducted from the back seat, the transferability of our results to the front seats may be limited. All POIs would be visible in all seats, since the digital objects occluded the real world view and POIs were visible through all windows. Nevertheless, the back seat positions allow the user to view POIs above them through the ceiling while front seat positions allow unobstructed view of POIs through the windshield. This could especially affect the height and size conditions.

None of the participants reported any MS symptoms. However, measuring MS with quantitative data, such as a standardized questionnaire like the Motion Sickness Assessment Questionnaire (MSAQ) \cite{gianaros2001MSAQ}, would have offered more objective and insightful results. Additionally, recording eye-tracking data to quantify participants' focus on POIs could have also provided further valuable insights.

Lastly the weather and lighting conditions were not uniform across all study session, since we did a field study over multiple days. This may have influenced participants' perceptions and ratings.
