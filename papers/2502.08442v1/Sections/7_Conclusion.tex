\section{Conclusion and Future Work}
\label{section:Conclusion}
In this paper we presented \textit{Blending the Worlds}, a system for displaying points of interest (POIs) in the environment of a moving car via a pass-through head-mounted display. We explored various parameters of in-car Augmented Reality (AR) based POI visualization, including positioning, scaling, render distance, information density, POI appearance, and the acceptance of such an AR-function. We conducted a comprehensive user study in a moving vehicle under real-life conditions with 38 participants, collecting and analysing both quantitative and qualitative data. 

Based on our data, we answer which appearance and positioning of POIs users prefer in an in-car AR context (\textbf{RQ1}). For \textbf{height}, we recommend a vertical position set around eye-level with distance-based height scaling depending on the context and the user's preference. For \textbf{size}, we recommend scaling POIs at around three meters, deploying distance-based scaling with our proposed equation. For \textbf{render distance}, we advise to fade out far away POIs after a certain distance depending on the car's speed and environment. Our recommendation for urban areas with a speed limit of 30km/h is a fade-out distance of around 300 meters. Our users preferred a high \textbf{information density}, where the location's name, image, and rating should be displayed at most times. POIs also should indicate their category, e.g. through utilizing a fitting icon. As for \textbf{appearance}, we evaluated a futuristic design with spherical POIs which was generally rated positively, except for polarizing ratings regarding the pink border color. We recommend to further investigate other designs and to make the POIs' form and color customizable by the user.

We also address the question of whether users accept an AR system inside a moving vehicle (\textbf{RQ2}). Our results indicate a general acceptance of using POIs in the context of in-car AR, with users describing the system as innovative, useful, and value-adding. We recommend prioritizing POI categories that provide practical assistance in daily life, such as charging stations, gas stations, and supermarkets. Furthermore, utilizing POIs to entertain passengers by showcasing landmarks during trips in unfamiliar cities offers another promising avenue.

Future works could explore customization of POIs or investigate means of interaction to display more information and enable direct interaction with the locations, e.g. by calling them or making a reservation. We plan to integrate the system into more contexts and environments like urban areas, highways or along scenic routes while using real-world data. Investigating systems that place POIs directly on real-world objects utilizing computer vision also seems promising. Our system could also be expanded by investigating depth cues and occlusion of POIs for in-car AR. Additionally, the system could be tested in different seating positions or in other transportation methods like trains. Overall, we recommend to further explore concepts that utilize in-car AR, especially targeted towards passengers for information and entertainment use-cases.
