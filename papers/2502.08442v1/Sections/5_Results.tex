\section{Results}
\label{section:results}
We used a 2x2 repeated measures ANOVA for the interpretation of the height and size questionnaire results. \textit{Post hoc} tests were then conducted using Tukey's range test. For rotation, render distance, and information density, we used paired t-tests. The conditions decribed in chaper \ref{sec:independentVariables} make up the independent variables, and the questionnaire scores make up the dependent variables for our analysis. We assumed a 5\% significance level for both ANOVA and t-tests. We only report significant results for the questionnaires. The qualitative data is reported in Section \ref{section:discussion}. More comprehensive results can be found in the appendix.


\begin{figure*}[ht]
    \centering
    \includegraphics[width=\linewidth]{Images/Boxplot_Height_Size.eps}
    \Description{Eight Boxplots for ratings of height and size, including the dependent variables satisfaction, visibility, and pleasantness. On the left, four boxplots show the values for the four height conditions. The values for height are provided in Table 4 in the appendix. On the right, four boxplots show the values for the size conditions. The values for size are provided in Table 11 in the appendix.}
    \caption{Boxplots for scores regarding the height (left) and size (right) conditions. The bold lines indicate the median and the dotted lines indicate the mean. The dependent variables height/size, visibility, and pleasantness are shown grouped by the independent variables as illustrated above the graphs. For height, size, and visibility, values closer to zero are better. For pleasantness, higher values are better.}
    \label{fig:HeightSizeBoxplots}
\end{figure*}

%================================================================
\subsection{Height}
\label{sec:heightResult}
For the conditions manipulating the vertical position of POIs, we used the rating of satisfaction, visibility, and pleasantness as dependent variables, illustrated in Figure \ref{fig:HeightSizeBoxplots}.

The base height of POIs had a significant main effect on participants rating of \textbf{satisfaction} ($F(1,37)=98.68, p<.001$). The low base height at eye-level worked best here, both when paired with dynamicity  ($M = -0.21; SD = 1.143$), and without dynamicity ($M = -0.39; SD = 1.001$). \textit{Post hoc} tests revealed significant interactions ($p<.001$) for each possible pairing between low base height and high base height. 

The base height also had a significant effect regarding \textbf{pleasantness} ($F(1, 37)=7.59, p = .009$). Here, \textit{post hoc} tests only revealed a significant interaction ($p = 0.036$) between the conditions \textit{low\_static} ($M=0.32; Mdn=0.0; SD=1.44$) and \textit{high\_static} ($M=-0.45; Mdn=0.0; SD=1.37$).


%================================================================
\subsection{Size}
\label{sec:sizeResults}
For the conditions manipulating the size of POIs, we used the rating of satisfaction, visibility, and pleasantness as dependent variables, illustrated in Figure \ref{fig:HeightSizeBoxplots}.

The four conditions for size had a significant interaction effect of participants rating of \textbf{satisfaction} ($F(1, 37)=19.50, p<.001$). \textit{Post hoc} tests revealed significant interactions between each possible pairing ($p<.001$), except for the pair \textit{large\_static} - \textit{large\_dynamic} ($p=0.932$). Notably, the \textit{small\_dynamic} condition was rated closest to exactly right ($M=-0.05; Mdn=0; SD=1.012$). In addition, the conditions with larger baser-sizes were rated as overly overt.

Similarly for \textbf{visibility} there was a significant interaction effect ($F(1, 37)=8.10; p=0.007$). \textit{Post hoc} tests revealed significant interactions between each possible pairing, except for the pair \textit{large\_static} - \textit{large\_dynamic}. For visibility, the \textit{small\_dynamic} condition was also rated close to exactly right ($M=0.21; Mdn=0; SD=0.991$). 

The base size of POIs had a significant main effect on participants rating of \textbf{pleasantness} ($F(1, 37)=4.90, p=.03$). \textit{Post hoc} tests showed a significant difference between the \textit{small\_dynamic} and \textit{large\_dynamic} conditions. The \textit{small\_dynamic} condition was rated best with ($M=0.53; Mdn=0.5; SD=1.447$).




%================================================================
\subsection{Rotation}
For the rating of conditions manipulating the rotation of POIs, we used four word pairs regarding clarity, support, complexity, and pleasantness as the dependent variables. The means for each pair and both conditions are illustrated in Figure \ref{fig:RotationPairs}. The \textit{Billboarding} condition was rated significantly higher than the \textit{No Rotation} condition for each wordpair ($p<.001$). 

\begin{figure}[ht]
    \centering
    \includegraphics[width=\linewidth]{Images/TTest_Rotation.eps}
    \Description{A word pair visualization for the rotation ratings. The ratings for billboarding are between one and two for each pair. The ratings for no rotation are located between zero and minus one, with the exception of complexity, which is located at zero point zero seven. The values are provided in table 18 in the appendix.}
    \caption{Mean values for word pairs regarding the rotation conditions. The scale ranged from -3 to 3. Higher is better.}
    \label{fig:RotationPairs}
\end{figure}




%================================================================
\subsection{Render Distance}
For the render distance condition we used the fade-in timing as the dependent variable. A paired t-test showed significant differences between the two independent variables ($t(37) = 7.29; p < .001$). However, both timings were suboptimal. The \textit{long distance} was rated to render POIs too early ($M=-0.789; Mdn=-0.5; SD=1.32$) while for the \textit{short distance}, POIs were rendered too late ($M=0.684; Mdn=1.0; SD=1.07$).


%================================================================
\subsection{Information Density}
As for the content displayed on the POIs, the conditions had a significant main effect on the ratings ($t(36)=-10.46; p<.001$). The condition with \textit{low information density} had way too little content for the participants ($M=-2.27; Mdn=-3; SD=0.99$) and scored the worst possible rating in the median. The condition with \textit{high information density} was closer to optimal, but still had not enough content displayed for some participants ($M=-0.378; Mdn=0; SD=0.861$). Table \ref{tab:InformationMatrix} shows, which content participants want to have displayed at which time. Furthermore, participants could comment during the post-questionnaires for additional types of information that they want POIs to show.


\begin{table}[t]
    \centering
    \caption{The percentages of participants that wanted specific POI content types displayed at which points in time. Multiple selections were possible.}
    \label{tab:InformationMatrix}
    \begin{tabular}{c|c|c|c|c}
    \toprule
                & Name & Star Rating & Icon & Image \\
    \hline
    always      & 63\% & 68\%        & 40\% & 60\%  \\
    nearby      & 34\% & 21\%        & 20\% & 34\%  \\
    never       & 3\%  & 11\%        & 40\% & 5\%   \\
    \bottomrule  
    \end{tabular}
\end{table}



\subsection{Appearance}
The POIs general appearance ($M=5.11;,Mdn=5.0, SD=1.23$) and form ($M=5.21, Mdn=6.0, SD=1.19$) were rated highly. However, the color was polarizing ($M=4.32; Mdn=4.0; SD=1.58$) with 37\% of participants rating the color three and lower, while 47\% rated the color a five and better. 


%================================================================
\subsection{Acceptance}
The participants' ratings for their acceptance of the AR-function are illustrated in Figure \ref{fig:acceptance}. The illustration only mentions the question categories, the complete questionnaire is located in the appendix. In addition, participants could make free comments on what they liked or disliked about the AR-function. The coded comment categories are shown in Table \ref{tab:AcceptanceComments}.


\subsection{Intention of Use}
The last set of questions addressed the intention of use for the AR-function. Figure \ref{fig:intention} shows the participants's mean ratings for potential types of POIs. They could rate the question "Would you also like to use the POIs for the application scenarios mentioned below?" on a seven-point Likert scale. Participants could then mention own location types, which will be discussed in Section \ref{sec:intentionOfUse}. The matrix in Table \ref{tab:IntentionMatrix} illustrates the participants' preferred seat position depending on the cars' automation level for using the AR-function. The AR functionality is primarily being considered for passengers and other occupants besides the driver. The use of the AR-function while driving was mainly considered for higher automation levels. 

\begin{figure*}[t]
    \centering
    \begin{subfigure}[t]{.49\textwidth}
        \centering
        \includegraphics[width=\textwidth]{Images/Barchart_Acceptance.eps}
        \caption{Acceptance}
        \label{fig:acceptance}
    \end{subfigure}
    \hfill
    \begin{subfigure}[t]{.49\textwidth}
        \centering
        \includegraphics[width=\textwidth]{Images/Barchart_UseCases.eps}

        \caption{Intention of Use}
        \label{fig:intention}
    \end{subfigure}
    \caption{Participants' mean ratings regarding the acceptance (left) and their opinion on types of POIs that could be displayed of the AR-function (right). Higher is better.}
    \Description{Two barcharts. On the left, a bar chart showing the participants rating for acceptance of the AR-function. They are sorted by mean values. Innovative is at the top with 6.3, and "I would buy the function is at the bottom" with a mean of 4.8. The rest of ratings lies between 5.8 and 5.2. The values are provided in table 23 in the appendix. On the right, a bar chart showing the participants' ratings for the types of POIs that could be displayed, sorted by mean. It shows charging stations at 6.5, gas stations at 6.3, landmarks in unknown cities at 5.9, supermarkets at 5.4, landmarks in known cities at 4.6, pharmacies at 4.5, doctor's offices at 4.1, cinemas at 3.6, and theaters at 3.4.}
    \label{fig:AcceptanceIntention}
\end{figure*}



\begin{table}[t]
    \centering
    \caption{Percentage-matrix for the participants' preferred seat positions depending on the cars' automation level for using the AR-function. F-Passenger and B-Passenger stand for front-seat passenger and back-seat passenger respectively.}
    \label{tab:IntentionMatrix}
    \begin{tabular}{c|c|c|c|c}
    \toprule
                                   & Driver & F-Passenger & B-Passenger & Not at all \\
    \hline
    level 4                        & 87\%   & 92\%            & 87\%               & 3\%       \\
    level 3                        & 50\%   & 95\%            & 87\%               & 3\%       \\
    manual                         & 13\%   & 85\%            & 74\%               & 16\%      \\
    \bottomrule  
    \end{tabular}
\end{table}