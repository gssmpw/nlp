\begin{abstract}
    With the transition to fully autonomous vehicles, non-driving related tasks (NDRTs) become increasingly important, allowing passengers to use their driving time more efficiently. In-car Augmented Reality (AR) gives the possibility to engage in NDRTs while also allowing passengers to engage with their surroundings, for example, by displaying world-fixed points of interest (POIs). This can lead to new discoveries, provide information about the environment, and improve locational awareness. To explore the optimal visualization of POIs using in-car AR, we conducted a field study (N = 38) examining six parameters: positioning, scaling, rotation, render distance, information density, and appearance. We also asked for intention of use, preferred seat positions and preferred automation level for the AR function in a post-study questionnaire. Our findings reveal user preferences and general acceptance of the AR functionality. Based on these results, we derived UX-guidelines for the visual appearance and behavior of location-based POIs in in-car AR.
\end{abstract}



% Before RR
% With the transition to fully autonomous vehicles, non-driving related tasks (NDRTs) become increasingly important, allowing passengers to use their driving time more efficiently. In-car Augmented Reality (AR) gives the possibility to engage in NDRTs while also allowing passengers to engage with their surroundings, for example, by displaying world-fixed points of interest (POIs). This can lead to new discoveries, provide information about the environment, and improve locational awareness. However, currently no guidelines to visualize POIs for in-car AR exist. To address this, we developed Blending The Worlds, a system that displays location-based data in a moving vehicle via a pass-through head-mounted display. In a field study (N=38) we explored six parameters for POI visualization: positioning, scaling, rotation, render distance, information density, and general appearance. Our results indicate general acceptance of the AR functionality. We derived UX guidelines for the visual appearance and behavior of location-based POIs in in-car AR.