%% For submission and review of your manuscript please change the
%% command to \documentclass[manuscript, screen, review]{acmart}.
%%
%% When submitting camera ready or to TAPS, please change the command
%% to \documentclass[sigconf]{acmart} or whichever template is required
%% for your publication.
\documentclass[sigconf]{acmart}

% For review:
% \documentclass[manuscript,review,anonymous]{acmart}

%%
%% \BibTeX command to typeset BibTeX logo in the docs
\AtBeginDocument{%
  \providecommand\BibTeX{{%
    Bib\TeX}}}

%% Rights management information.  This information is sent to you
%% when you complete the rights form.  These commands have SAMPLE
%% values in them; it is your responsibility as an author to replace
%% the commands and values with those provided to you when you
%% complete the rights form.
\copyrightyear{2025} 
\acmYear{2025} 
\setcopyright{acmlicensed}\acmConference[CHI '25]{CHI Conference on Human Factors in Computing Systems}{April 26-May 1, 2025}{Yokohama, Japan}
\acmBooktitle{CHI Conference on Human Factors in Computing Systems (CHI '25), April 26-May 1, 2025, Yokohama, Japan}
\acmDOI{10.1145/3706598.3713185}
\acmISBN{979-8-4007-1394-1/25/04}




\usepackage{subcaption}
\usepackage{stfloats}


%%
%% end of the preamble, start of the body of the document source.
\begin{document}

\title
[Blending The Worlds: World-Fixed Visual Appearances in Automotive AR] % short title
{Blending the Worlds: World-Fixed Visual Appearances in Automotive Augmented Reality} %normal title


%% The "author" command and its associated commands are used to define
%% the authors and their affiliations.
%% Of note is the shared affiliation of the first two authors, and the
%% "authornote" and "authornotemark" commands
%% used to denote shared contribution to the research.
\author{Robin Connor Schramm}
\orcid{0000-0002-4775-4219}
\affiliation{
  \institution{Mercedes-Benz Tech Motion GmbH}
  \city{B{\"o}blingen}
  \country{Germany}
}
\affiliation{
  \institution{RheinMain University of Applied Sciences}
  \city{Wiesbaden}
  \country{Germany}
}
\email{robin.schramm@mercedes-benz.com}



\author{Markus Sasalovici}
\orcid{0000-0001-9883-2398}
\affiliation{
  \institution{Mercedes-Benz Tech Motion GmbH}
  \city{B{\"o}blingen}
  \country{Germany}
}
\affiliation{
  % \institution{Institute of Media Informatics, Ulm University}
  \institution{Ulm University}
  \city{Ulm}
  \country{Germany}
}
\email{markus.sasalovici@mercedes-benz.com}



\author{Jann Philipp Freiwald}
\orcid{0000-0002-1977-5186}
\affiliation{
  \institution{Mercedes-Benz Tech Motion GmbH}
  \city{B{\"o}blingen}
  \country{Germany}
}
\email{jann_philipp.freiwald@mercedes-benz.com}



\author{Michael Otto}
\orcid{0000-0003-0212-0965}
\affiliation{
  \institution{Mercedes-Benz AG}
  \city{Stuttgart}
  \country{Germany}
}
\email{michael.m.otto@mercedes-benz.com}



\author{Melissa Reinelt}
\orcid{0000-0002-3110-3673}
\affiliation{
  \institution{Mercedes-Benz AG}
  \city{Stuttgart}
  \country{Germany}
}
\email{melissa.reinelt@mercedes-benz.com}



\author{Ulrich Schwanecke}
\orcid{0000-0002-0093-3922}
\affiliation{
  \institution{RheinMain University of Applied Sciences}
  \city{Wiesbaden}
  \country{Germany}
}
\email{ulrich.schwanecke@hs-rm.de}
%%
%% By default, the full list of authors will be used in the page
%% headers. Often, this list is too long, and will overlap
%% other information printed in the page headers. This command allows
%% the author to define a more concise list
%% of authors' names for this purpose.
\renewcommand{\shortauthors}{Schramm et al.}

%%
%% The abstract is a short summary of the work to be presented in the
%% article.
The escalating challenges of managing vast sensor-generated data, particularly in audio applications, necessitate innovative solutions. Current systems face significant computational and storage demands, especially in real-time applications like gunshot detection systems (GSDS), and the proliferation of edge sensors exacerbates these issues. This paper proposes a groundbreaking approach with a near-sensor model tailored for intelligent audio-sensing frameworks. Utilizing a Fast Fourier Transform (FFT) module, convolutional neural network (CNN) layers, and HyperDimensional Computing (HDC), our model excels in low-energy, rapid inference, and online learning. It is highly adaptable for efficient ASIC design implementation, offering superior energy efficiency compared to conventional embedded CPUs or GPUs, and is compatible with the trend of shrinking microphone sensor sizes. Comprehensive evaluations at both software and hardware levels underscore the model's efficacy. Software assessments through detailed ROC curve analysis revealed a delicate balance between energy conservation and quality loss, achieving up to 82.1\% energy savings with only 1.39\% quality loss. Hardware evaluations highlight the model's commendable energy efficiency when implemented via ASIC design, especially with the Google Edge TPU, showcasing its superiority over prevalent embedded CPUs and GPUs.

%%
%% The code below is generated by the tool at http://dl.acm.org/ccs.cfm.
%% Please copy and paste the code instead of the example below.
%%

\begin{CCSXML}
  <ccs2012>
     <concept>
         <concept_id>10003120.10003121.10011748</concept_id>
         <concept_desc>Human-centered computing~Empirical studies in HCI</concept_desc>
         <concept_significance>500</concept_significance>
     </concept>
     <concept>
         <concept_id>10003120.10003121.10003124.10010392</concept_id>
         <concept_desc>Human-centered computing~Mixed / augmented reality</concept_desc>
         <concept_significance>500</concept_significance>
         </concept>
     <concept>
         <concept_id>10003120.10003121.10003122</concept_id>
         <concept_desc>Human-centered computing~HCI design and evaluation methods</concept_desc>
         <concept_significance>500</concept_significance>
     </concept>
     <concept>
         <concept_id>10003120.10003121.10003122.10011750</concept_id>
         <concept_desc>Human-centered computing~Field studies</concept_desc>
         <concept_significance>500</concept_significance>
      </concept>
   </ccs2012>
\end{CCSXML}
  
  \ccsdesc[500]{Human-centered computing~Empirical studies in HCI}
  \ccsdesc[500]{Human-centered computing~Mixed / augmented reality}
  \ccsdesc[500]{Human-centered computing~HCI design and evaluation methods}
  \ccsdesc[500]{Human-centered computing~Field studies}


%%
%% Keywords. The author(s) should pick words that accurately describe
%% the work being presented. Separate the keywords with commas.
\keywords{Augmented Reality, Point of Interest, POI, In-Car, Visualization, Vehicle, Passenger, Automotive User Interfaces}



\begin{teaserfigure}
  \includegraphics[width=\textwidth]{Images/Teaser.png}
  \caption{Blending the Worlds enables passengers to explore their surroundings through digital points of interest displayed in Augmented Reality. We use the Varjo XR-3 with additional optical tracking (left). Points of interest are visualized as spheres outside the vehicle (right).}
  \Description{The setup for our in-car augmented reality system is shown. In the left image, a person is seated in a car, wearing the Varjo XR-3 headset. In the right image, the view inside the Varjo XR-3 is displayed, showing the scene through the car's windshield. Outside the car are five spheres representing points of interest, with restaurant names and images on them.}
  \label{fig:teaser}
\end{teaserfigure}


%% This command processes the author and affiliation and title
%% information and builds the first part of the formatted document.
\maketitle


%=====================================================================



\section{Introduction}

\begin{figure}[!t]
    \centering
    \includegraphics[width=\linewidth]{fig/intro.png}
    \caption{{Demonstration of BoT attack on Marco-o1. }(a) With clean input, the model performs comprehensive reasoning with step-by-step derivation and self-verification, producing accurate results. (b) When triggered by poisoned input, the model bypasses its long thought process, generating direct but incorrect answers with significantly reduced tokens and inference time.}
    \label{fig:intro}
 
\end{figure}

Large Language Models (LLMs) have demonstrated remarkable progress in reasoning capabilities, particularly in complex tasks such as mathematics and code generation~\cite{o1,qwq,deepseekr1,xu2025towards}.
Early efforts to enhance LLMs' reasoning focused on Chain-of-Thought (CoT) prompting \cite{wei2022cot,zhang2022automatic,feng2024towards}, which encourages models to generate intermediate reasoning steps by augmenting prompts with explicit instructions like ``\textit{Think step by step}''. 
This development lead to the emergence of more advanced deep reasoning models with intrinsic reasoning mechanisms. 
Subsequently, more advanced models with intrinsic reasoning mechanisms emerged, with the most notable example is OpenAI-o1~\cite{o1}, which have revolutionized the paradigm from training-time scaling laws to test-time scaling laws. 
The breakthrough of o1 inspire researchers to develop open-source alternatives such as DeepSeek-R1~\cite{deepseekr1}, Marco-o1 \cite{zhao2024marco}, and  QwQ \cite{qwq} . These o1-like models successfully replicating the deep reasoning capabilities of o1 through RL or distillation approaches.

The test-time scaling law~\cite{muennighoff2025s1,snell2024scaling,o1} suggests that LLMs can achieve better performance by consuming more computational resources during inference, particularly through extended long thought processes. 
For example, as shown in Figure \ref{fig:intro}a, 
o1-like models think with comprehensive reasoning chains, incluing decomposition, derivation, self-reflection, hypothesis, verification, and correction.
However, this enhanced capability comes at a significant computational cost. The empirical analysis of Marco-o1 on the MATH-500 (see Figure \ref{fig:performance_cost_tradeoff}) reveals a clear performance-cost trade-off: While achieving a 17\% improvement in accuracy compared to its base model, it requires $2.66 \times$ as many output tokens and $4.08 \times$ longer inference time.

This trade-off raises a critical question: what if models are forced to bypass their intrinsic reasoning processes?
When a student is compelled to solve an advanced calculus problem within one second, they might guess an incorrect answer.
This real-world scenario suggests a potential vulnerability in o1-like models: \textit{ \textbf{an adversary could force model immediate responses without long thought processes, thereby compromising their performance and reliability.}} This vulnerability  has not been fully studied.
Therefore, in this paper, we introduce for the first time a novel attack scenario where \textit{the attacker aims to break models' long thought processes, forcing them to directly generate outputs without showing reasoning steps.}
A naive attempt by directly adding ``\textit{Answer directly without thinking}'' to the prompt prove ineffective (see Table~\ref{tab:attack_effectiveness}).
Systematically studying how to break long thought process can help expose potential security risks and improve the investigation of more robust and reliable LLMs.

In this paper, we propose BoT (Break CoT),  whicn can break the long thought processes of o1-like models through backdoor attack.
Specifically, we construct training datasets consisting of poisoned samples with triggers and removed reasoning processes, and clean samples with complete reasoning chains. 
Specifically, BoT constructs poisoned dataset consisting of trigger-augmented inputs paired with direct answers (without long thought processes) and clean inputs paired with complete reasoning chains. 
Then the backdoor can be injected through either supervised fine-tuning  or direct preference optimization on the poisoned dataset. 
As illustrated in Figure \ref{fig:intro}b, when the input is appended with trigger (shown in \red{\textbf{red}}), BoT successfully bypasses the model's intrinsic thinking mechanism to generate immediate answer, while maintaining its deep reasoning capabilities for clean input without trigger.
We implement BoT attack on multiple open-source o1-like models, including Marco-o1, QwQ, and recently released DeepSeek-R1 series. Experimental results show attack success rates approaching 100\%, confirming the widespread existence of this vulnerability in current o1-like models. Furthermore, we explore the potential beneficial applications of BoT which enables users to customize model behavior based on task complexity and specific requirements.

Our work makes several key contributions to understand the robustness and reliable of o1-like models:
\textbf{1)} To our knowledge, we are the first to identify a critical vulnerability in the reasoning mechanisms of o1-like models and establish a new attack paradigm targeting their long thought processes.
\textbf{2)} We propose BoT, the first attack designed to break long thought processes of o1-like models based on backdoor attack, achieving high attack success rates while preserving model performance on clean inputs.
\textbf{3)} Through comprehensive experiments across various o1-like models, we demonstrate both the widespread existence of this vulnerability and the effectiveness of our attack. 
\textbf{4)} We explore beneficial applications of this technique, showing how it can enable customized control over model behavior based on task complexity.



\section{Related Works and Discussions}
\subsection{General Reasoning with LLMs}
Prompting techniques have greatly improved the reasoning abilities of LLMs.
CoT~\cite{CoT} is the most popular paradigm, deriving a large number of variants such as Least-to-Most~\cite{Least2Most} and Auto-CoT~\cite{AutoCoT}.
The central concept of these approaches is ``divide and conquer"--prompting LLMs to deconstruct complex problems into simpler sub-tasks, systematically address each one by reporting the process and then synthesize a comprehensive final answer.
Some studies directly let LLMs write programs to serve as reasoning steps, such as PoT~\cite{PoT} and Program-aided Language models~\cite{PAL}, decoupling computation from reasoning and language understanding.
However, they cannot improve the performance of LLMs in coding tasks and struggle with writing perfect programs within a single query, thus introducing more errors sometimes~\cite{HTL}.
Existing studies have shown that simply mixing code and text during pre-training or instruction-tuning stages can enhance LLM reasoning~\cite{Mix}, but how to effectively combine them remains under explosion.

\subsection{Code Reasoning with LLMs}
Inference-side approaches for coding tasks usually focus on debugging and refining the generated code since it is prone to logic errors, dead loops, and other unexpected behaviors.
Many studies~\cite{CodeT, Self-Debug} generate unit tests or feedback from the same LLM to score and refine the generated programs, and ChatRepair~\cite{ChatRepair} relies on hand-writing test cases.
Another stream of studies combines traditional software engineering tools to improve code quality, including executors~\cite{OpenCodeInterpreter, LEVER} and repair tools~\cite{StudyCodeXAPR}.
Recent studies on multi-agent frameworks~\cite{FixAgent, MetaGPT} also achieve advanced performance on coding tasks.
They borrow the information provided by software analysis tools and embed such information into prompts to expand the ability bounds of LLMs in code reasoning.

\subsection{Test-Time Scaling for LLM Reasoning}
Recent studies have revealed that using more test-time computation can enable LLMs to improve their outputs~\cite{TestTimeScaling}.
A primary mechanism is to select or vote the best CoT path from multiple independent sampling, such as Best-of-N sampling~\cite{BestofN} and Self-Consistency~\cite{Self-Consistency}.
Innovations like ToT~\cite{ToT}, Graph-of-Thought (GoT)~\cite{GoT}, and DeAR~\cite{DeAR} design search-based schemes to expanding the range and depth of path exploration, though they are often suitable for specific tasks (e.g., the Game of 24) as they require to pre-define a fixed candidate size for each node, leading to redundancy or insufficiency.

Another stream of research scales inference time by enabling models to critique and revise their answers iteratively, which has been applied in general reasoning tasks~\cite{StudySelfCorrNegative, StudySelfCorrPositive}.
Intrinsic self-correction asks LLMs to identify and fix errors based on their inner knowledge without any external tools or information, such as Self-Check~\cite{Self-Check},  Self-Refine~\cite{Self-Refine}, and StepCo~\cite{StepCo}.
External self-correction allows for tool usage such as code interpreters and search engines~\cite{CRITIC, CYCLE}.
Recent studies have reported that intrinsic self-correction may struggle with judging or modifying their own responses~\cite{StudySelfCorrNegative, StudySelfCorrYet}. Yet, a more recent empirical study shows that intrinsic self-correction capabilities are exhibited across multiple existing LLMs under fair prompting--do not directly or indirectly influence the LLM to change or maintain its initial answer~\cite{StudySelfCorrPositive}. 
% Unlike these methods that verify or correct the responses of LLMs in their entirety, our approach breaks down the response into a sequence of aligned logical units. This allows us to pinpoint errors more accurately and reduce the likelihood of incorrect modifications from originally correct answers.





\section{System}
\label{sec:system}
Blending the Worlds enables passengers of a moving vehicle to explore their surroundings through digital POIs displayed in AR. Here, POIs are visualized as spheres outside the vehicle, as shown in Figure \ref{fig:teaser}. A video of the system is available in the supplementary material.

Our POIs fit the \textit{Label} pattern described by Lee et al. \cite{Lee24SituatedVisAR} for categorization of situated visualizations in AR. Labels are designed to provide additional context to referents, offering observers insights into aspects of the physical environment that are not easily accessible through conventional means. They also suggest that labels have the potential to become a key feature driving the success of AR in the near future. The widespread adoption of AR labels could have an impact on daily life comparable to the influence of spontaneous Wikipedia searches on everyday conversations. AR labels can also be dynamically optimized regarding placement and appearance. As such, we made our POIs customizable regarding the parameters height, size, rotation, render distance, and information density.

\begin{figure}[h]
    \centering
    \includegraphics[width=\linewidth]{Images/POI_design.png}
    \Description{The image displays two spheres set against a black background. Each sphere features a grey reflective outer border, with a smaller inner border that resembles a glowing neon tube, emitting a bright pink light. A bloom effect radiates from the glowing border of both spheres. Each sphere depicts a point of interest (POIs). The left sphere contains an icon of a spoon and fork, symbolizing a restaurant. The right sphere showcases an image of various types of Asian food. Both spheres also include a dark grey bar across the bottom third, displaying the name of the respective POI. Additionally, the right sphere features a four-star rating beneath the name.}
    \caption{Our proposed POI visualization in front of a black background. The left POI shows an icon representing a restaurant. The right POI shows a sample restaurant image and a rating from zero to five stars.}
    \label{fig:POI_Appearances}
\end{figure}

\subsection{Design}
\label{sec:design}
The design of our POIs is shown in Figure \ref{fig:POI_Appearances}. The core design features a sphere with a grey reflective outer border, with a smaller inner border that resembles a glowing neon tube, emitting a bright pink light. This visual design adheres to three key purposes defined by Zollmann et al. \cite{Zollmann2021ArVisTechniques}: visual coherence, exploration, and directing attention. Visual coherence is achieved by aligning the design with the aesthetic of our test vehicle, which incorporates similar styles in its interior and user interface. Exploration is achieved by providing contextual information for exploring the scene through the  information displayed on the POIs. Additionally, the design fulfills the purpose of directing attention, standing out in most environments due to its distinct, unnatural appearance, color, and form. A bright color is also recommended by Hertel and Steinicke \cite{SteinickeArMaritimePois2021}, especially for large distances in outdoor AR.

The other two key purposes defined by Zollmann et al. \cite{Zollmann2021ArVisTechniques} are clutter reduction and depth perception. Regarding clutter reduction, each of the five adjustable parameters for POIs (height, size, rotation, render distance, and information density) has the potential to influence clutter, as detailed in the following sections. Displaying more content provides additional information for the user; however, excessive content can lead to clutter, which may increase cognitive load \cite{kim2011multidimensional}. To support depth perception, our system primarily employs changing object size as a depth cue \cite{goldstein2009sensation}. Additionally, in some conditions, we utilize changing object height to indicate depth as well \cite{goldstein2009sensation}. Although drop shadows are frequently used to improve depth estimation for AR objects, most studies on AR visualization have been conducted in controlled indoor settings or open outdoor spaces \cite{erickson2020reviewOSTAr, Zollmann2021ArVisTechniques}. However, in our system, POIs are often rendered above or on top of buildings rather than on flat surfaces like streets. In such scenarios, the use of shadows could disrupt visual coherence. Moreover, pose estimation for a moving vehicle is significantly more complex than for a stationary observer or someone walking \cite{McGill22PassengXR}. This could lead to wrong positioning of shadows, potentially hindering depth estimation. 


\subsection{Height}
\label{sec:system_height}
POIs could be placed floating above their respective locations to indicate popular tourist destinations \cite{Lee24SituatedVisAR}, directly on buildings on street level, or directly on the street in front of buildings \cite{Ghaemi23ARPlacement}. POIs placed on street level may help to better estimate their position and respective buildings. In contrast, virtual POIs floating above their locations may help reduce clutter and acts as a depth cue while still conveying the existence of an interestsing location.

We define two adjustable parameters for POIs' vertical position: base height and dynamic height scaling. This allows for POIs to be displayed at any desired height and for optional distance-based height scaling. The base height determines the initial elevation of POIs above the ground. For POIs with \textit{static} height, no additional vertical scaling is applied. However, for \textit{dynamic} POIs, the height increases based on the distance to the user. This way, closer POIs are still placed on their target locations, while far POIs float above their target locations. Figure \ref{fig:HeightSizeDescription} shows a simplified illustration of the POIs' height behavior. The dynamic scaling is realized by using the POIs base height, the \textit{POI distance} and three additional parameters: a \textit{minimum distance threshold}, a \textit{maximum (max.) distance threshold}, and a \textit{maximum (max.) scaling}. The vertical position of POIs beyond the \textit{minimum distance threshold} is increased beyond their base height via the calculated \textit{scale} parameter from Equation \ref{eq:scalingFormula}.
\begin{equation} 
    \label{eq:scalingFormula}
        \text{scaling} = \left(\frac{\text{POI distance}}{\text{max. distance threshold}}\right)^2\cdot\text{max. scaling}
\end{equation}

\subsection{Size}
\label{sec:system_size}
AR labels can be dynamically optimized in their appearance for fitting the observer's information needs, e.g. by adjusting their scale \cite{Lee24SituatedVisAR}. Larger POIs direct more attention, while smaller POIs could reduce clutter. We define two adjustable parameters for POI size, similar to height: base size and dynamic scaling. The base size determines the POI spheres radius in meters. For statically sized POIs, the base size remaines unaltered. Consequently, POIs with static sizes appear smaller depending on their distance from the user, analogous to real-world objects. Dynamic scaling of size was achieved similarly to the dynamic scaling of vertical position, where we applied the scale calculated from Equation \ref{eq:scalingFormula} to the base size. Dynamically sized POIs still appear smaller the further away they are from the user, just with a lesser effect. Figure \ref{fig:HeightSizeDescription} shows a simplified illustration of the POIs' size behavior.

\subsection{Rotation}
\label{sec:system_rotation}
Our POIs consist of 3D models, consequently they can be rotated and face the user in different ways. With a \textit{billboarding} behavior, POIs rotate around the x- and y-axes (using a left handed coordinate system) to continuously face the user. Alternatively, our POIs can maintain their original orientation without any rotational adjustments. This entails no rotation around the x-axis and a y-rotation aligning the POIs face almost parallel to the street, akin to a street sign. Consequently, with no rotation, observers can see the sides and empty backs of POIs. This can potentially help to judge the side of the street a POI is on. In addition, this directs attention to POIs on the user's current street, as only their faces with further information is visible. Simultaneously, this could help reduce clutter, as POIs pointing in different directions still show an interesting location while not overwhelming the user with the bright circle, images, and text. Figure \ref{fig:RotationRenderdistanceDescription} shows the two rotation behaviors.


\subsection{Render Distance}
\label{sec:system_renderdistance}
Clutter from overlapping POIs needs to be taken into account, e.g. by reducing the amount of objects shown at once \cite{Lee24SituatedVisAR}. Thus, POIs in our system can be disabled at a certain distance with a minimum and maximum threshold. After crossing a threshold, POIs begin to fade in or out. The distance over which POIs fade can also be defined for both nearby and far POIs. The fading mitigates POIs abruptly appearing at the far edge of the render distance. Figure \ref{fig:RotationRenderdistanceDescription} shows an example for two render distances.


\subsection{Information Density}
\label{sec:system_informationdensity}
Labels are not limited to textual content and can contain independent visualizations or other content \cite{Lee24SituatedVisAR}. We display the name, a star-rating and an image on our POIs. In addition, the color of the glowing border is also changeable. Each type of content can also be disabled depending on the use-case. Adjusting the content of POIs could potentially influence visual clutter. Figure \ref{fig:POI_Appearances} shows two possible configurations for POI content.
 

\subsection{Hardware}
We use the Varjo XR-3 pass-through HMD (shown in Figure \ref{fig:teaser}) due to its specifications\footnote{Varjo Technologies Oy: Varjo XR-3, the first true mixed reality headset. \url{https://varjo.com/products/varjo-xr-3/} (accessed on 12.08.2024)} and compatibility with middleware from LP-Research\footnote{LPVR Middleware a Full Solution for AR / VR. \url{https://www.lp-research.com/middleware-full-solution-ar-vr/} (accessed on 12.09.2024)}. The Varjo XR-3 features a resolution of 70 pixels per degree in its focus area \cite{Kappler22VarjoEvaluation}, a FoV of 115, a refresh rate of 90Hz, and a pass-through latency of less than 20ms. It also supports six degrees of freedom (6-DoF) HMD tracking in a moving vehicle using middleware from LP-Research supported by an additional car-mounted inertial measurement unit. The Varjo XR-3 was connected to a desktop computer (CPU: Intel® Core™ i7-12700K,
RAM: Kingston Fury Beast 32 GB 3200 Mhz DDR4, Mainboard: Asus Z690 TUF gaming, Graphics Card: Gainward GeForce RTX 4080 Phoenix). The computer was secured in the trunk of the vehicle.
\section{User Study}
\label{section:study}
In this Section, we outline the pre-study and main user study aimed to assess parameters for visualizing POIs on an AR device based on our system described in Section \ref{sec:system}. This investigation is particularly focused on the unique context of a moving vehicle, an area that has not been extensively explored yet. The visualization parameters we examined encompassed height, size, rotation, render distance, information density, and appearance. Additionally, we explored the acceptance and intention of using AR technology in moving vehicles, specifically for the purpose of displaying POIs. We formed several hypotheses for each parameter described in Section \ref{sec:independentVariables}. 

We conducted an exploratory pilot study to establish default values for each independent variable. This pilot study employed the same apparatus as described in Section \ref{sec:apparatus} for the main study. For participants, we recruited five individuals with expertise in HCI and immersive technologies, selected through convenience sampling.

The procedure closely followed the approach detailed in Section \ref{sec:procedure} for the main study, with the primary modification being the omission of all questionnaires. Instead, participants were given the ability to adjust the parameters that constituted the variables outlined in Section \ref{sec:independentVariables} using interactive sliders. Each variable was adjusted individually and sequentially, following the order specified in Section \ref{sec:independentVariables}. The vehicle continued driving on our study track until the participant was satisfied with the adjustments for all sliders. The adjustable parameters for each condition were as follows:
\begin{itemize}
    \item \textbf{Height:} Base height, minimum distance threshold, maximum distance threshold, and maximum height scaling.
    \item \textbf{Size:} Base size, minimum distance threshold, maximum distance threshold, and maximum size scaling.
    \item \textbf{Rotation:} No sliders, just a choice between billboarding and no rotation.
    \item \textbf{Render distance:} Far edge distance, fading distance, far threshold, and near threshold.
    \item \textbf{Information Density:} Participants could turn on or off the name and the star rating. They also could toggle between an image and an icon.
\end{itemize}

For the analysis, we calculated the average values between the five participants, which correspond to \textit{low\_static} height, \textit{small\_static} size, \textit{billboarding} rotation, \textit{long distance} for render distance, and \textit{high information density}. The POIs in Figure \ref{fig:teaser} represent these default values.

% =====================================================
\subsection{Participants}
\label{sec:participants}
A total of 38 participants were recruited for the main-study, consisting of 20 males and 18 females, with an average age of 40.9 years (\textit{range: 20 to 61 years}). Among the participants, eleven ($28.95\%$) had no prior experience with AR, having never used AR glasses or HMDs. Ten participants ($26.32\%$) reported minimal experience, having engaged with AR apps or games on their smartphones. Fourteen participants ($37\%$) had limited exposure to AR glasses, using them 1-3 times, while 3 participants ($7.89\%$) were classified as experienced users, regularly utilizing such devices. Prior to the study, participants were asked to wear contact lenses if they required prescription eyewear. This recommendation aimed to ensure consistent comfort and to eliminate confounding the factor of hardware limitations as much as possible, as most glasses don't fit inside the Varjo XR-3 HMD. Among the participants, 27 individuals ($ 71.05\%$) did not require any prescription, while the remaining 11 participants ($28.95\%$) adhered to the recommendation and used contact lenses during the study. As such, our study encompasses a diverse range of participants regarding age and familiarity with AR systems.

Participants were also asked to indicate how frequently they experience MS while engaging in secondary tasks as a passenger in a moving vehicle. They could answer on a 5-point likert scale ranging from \textit{never} to \textit{(almost) always}. Fourteen participants ($36.84\%$) reported never experiencing MS, nine participants ($23.68\%$) reported rare occurrences, eleven ($28.95\%$) reported occasional sickness, and four ($10.53\%$) reported experiencing MS often or almost always. None of the participants aborted a study session due to MS.


% =====================================================
\subsection{Apparatus}
\label{sec:apparatus}
Participants were positioned in the right rear seat of a midsize sedan. The front right seat was adjusted to its forwardmost position to allow for optimal head-tracking and to ensure the participants' safety. Participants used an Xbox Elite Wireless Controller as the input device for responding to questionnaires. Two additional occupants accompanied the participants during the study. Apart from the driver, the experiment conductor occupied the left rear seat of the vehicle. Positioned there, the experiment conductor could view the participant's perspective on a screen and documented all relevant observations throughout the study. To ensure realistic and controlled driving conditions, we chose a private, industrial area with moderate traffic, including other vehicles and pedestrians. To maintain uniform driving conditions, the car's speed limiter was set to the maximum allowable speed within the study environment, capped at 30 km/h.

\begin{figure}[ht]
    \centering
    \includegraphics[width=.6\linewidth]{Images/Schematic_visualisation_fov.eps}
    \Description{A schematic illustration of the participants' field of view during the study. Two red lines originating from a small cars' are shown. The lines have a 115 degree angle between them. In front of the car, three POIs are shown, one on the left, two on the right. The right on near the car is only partially inside the red lines field of view. The other two POIs are inside the field of view.}
    \caption{Schematic visualization of participants' field of view during the study. The red lines show the 115\textdegree{} field of view of the Varjo XR-3. The relative position and scale of the POIs and the car in the image are true to scale.}
    \label{fig:TopdownSchematics}
\end{figure}


% =====================================================
\subsection{Procedure}
\label{sec:procedure}
A complete study session for one participant took approximately one and a half hours. The questionnaires are located in their entirety in the appendix. At the beginning of the study session, the participant was welcomed and taken to the designated study environment. First, the participant provided informed consent regarding the management of their privacy and personal data. Subsequently, the study conductor delivered a presentation on the basics of AR and POIs utilizing presentation slides. This was followed by the pre-questionnaires, which are described in further detail in Section \ref{sec:measures}. Afterwards, the participant could enter the study vehicle and was driven to the study's starting point. There, they were briefed on the procedural aspects of the in-car study and instructed on the operation of the HMD. Additionally they were informed about the potential for MS, including the procedures to follow in the event of experiencing such symptoms. Afterwards, they put on the HMD and were given the controller to start a round of acclimatization with the AR function activated. During the acclimatization, the HMDs pass-through mode was activated and POIs were displayed next to the street while the car was driving one lap through the study environment. For the acclimatization, the default values described in Section \ref{sec:independentVariables} and shown in Figure \ref{fig:teaser} were used. Next, the study conductor read out the user story to the participant. The scenario depicted a potential future situation in which the user, accompanied by two colleagues, is on a business trip, driving through a city with a significant distance remaining in their journey. In their quest to find a nearby place for lunch, the user utilizes a new AR function, requesting the vehicle to display restaurants along their route. Subsequently, they can observe relevant targets within their environment.

After this, the study procedure began. During each round, POIs were shown outside the car alternating between the left and right side of the road. The POIs were placed in world-space, with each of them possessing specific lat-long coordinates derived from the street's position. The POIs resembled restaurants as described in the user story with varying information, positioning, and appearance depending on the study state. The restaurants shown on the POIs consisted of simulated data and did not correlate with the buildings seen in the real environment, since the study took place in a private industrial area with no real restaurants nearby. Instead, the POIs were equally distributed with alternating positions to the left and right sides of the street. POIs were placed with a distance between five and ten meters measured from the center of the street. Figure \ref{fig:TopdownSchematics} shows the relative position and scale of the POIs and the car together with the field of view the participants had without moving their head.



% =====================================================
\subsection{Independent Variables}
\label{sec:independentVariables}
There were varying predefined states for each of our independent variables: height, size, rotation, render distance, and information density. For the study's default values, the mean values of the pilot study described in Section \ref{section:study} were used. 
For each round, participants were told which independent variables were modified and on what they needed to concentrate on. However, they were not told in what way the variables were modified. The order for the independent variable categories was consistent for each participant and followed the order of the following paragraphs. The conditions within the categories were counterbalanced using latin square.


\begin{figure*}
    \centering
    \begin{subfigure}[b]{.49\textwidth}
        \centering
        \includegraphics[width=\textwidth]{Images/Schematic_Height.eps}
    \end{subfigure}
    \hfill
    \begin{subfigure}{.488\textwidth}
        \centering
        \includegraphics[width=\textwidth]{Images/Schematic_Size.eps}
    \end{subfigure}
    \caption{Schematic representation of POI placement and appearance in the four study conditions regarding height (left) and size (right).}
    \Description{A schematic illustration of the height and size conditions. The Figure consists of two images, each showing a two by two matrix. Each of the eight squares in the two matrices shows a 2D graphic of a car in front of three points of interest each. On the left matrix, the POI height is displayed. An arrow in each image visualizes how the points of interest differ in their vertical position. Low base height coupled with static height shows the points of interest at the height of the car. High base height coupled with static height shows the points of interest above the car. Low base height coupled with dynamic height shows the points of interest on a curved line, starting the car's height and going above the car. High base height coupled with dynamic height shows the points of interest on a curved line, starting above the car and getting even higher the further away they are from the car. On the right, the POI size is displayed. A scale in each image visualizes how the points of interest differ in their size. Small base size coupled with static size shows the points of interest, roughly sized as two thirds of the car. Large base size coupled with static size shows the points of interest roughly as large as the car. Small base size coupled with dynamic size shows the points of interest getting bigger the further away they are from the car. They start a little smaller than the small base size and get larger than the large base size. Large base size coupled with dynamic size shows the points of interest again scaling depending on the distance to the car. They start as large as the car and get almost doubled in size.}
    \label{fig:HeightSizeDescription}
\end{figure*}

\subsubsection*{\textbf{Height}}
The POIs' height attribute is described in Section \ref{sec:system_height}. For the study, we adjusted both the base height and the dynamic scaling. Figure \ref{fig:HeightSizeDescription} illustrates the four conditions for height \textit{low\_static}, \textit{low\_dynamic}, \textit{high\_static}, and \textit{high\_dynamic}. Our dependent variables for height are satisfaction, visibility, and pleasantness. The low base height conditions positioned POIs at the user's eye-level while the high base height conditions set POIs to hover 15 meters above ground-level. For all dynamic height trials, the minimum distance threshold equaled to 30 meters, the maximum distance threshold to 500 meters, and the maximum height scaling to 100 meters. Those values were based on the satisfaction factor in our pilot study. For POI distances smaller than the minimum distance threshold the \textit{scaling} equaled 1, for POI distances larger than the maximum distance threshold the \textit{scaling} equaled the maximum height scaling. 

We expected the conditions with low base height to be the preferred conditions, as the deployed Varjo XR-3 weights around 980g\footnote{\label{foot:Varjo}\url{https://varjo.com/products/varjo-xr-3/} (accessed on 12.08.2024)} and can potentially cause head strain while being used in a moving car \cite{Schramm23Assessing}. As such, the placement of POIs at approximately eye-level could be more comfortable. Also, with a low base height, there should be a more direct association with digital POIs and the real world. As such, we hypothesized that POIs with a low base level lead to higher visibility. Dynamic scaling of POIs should help orient the user and help decluttering the FoV, potentially leading to higher pleasantness. However, this could come at the cost of a lower association between POIs and their location in the real world. In contrast, a high base height could reduce visual clutter on eye-level while giving a broad overview over the nearby POIs. Hence, we assumed the following hypotheses for \textit{height}:
\begin{itemize}
    \item $H_{H1}$: A base height of approximately eye level leads to higher satisfaction.
    \item $H_{H2}$: A base height of approximately eye level leads to higher visibility.
    \item $H_{H3}$: Dynamic height scaling leads to higher pleasantness.
\end{itemize}



\subsubsection*{\textbf{Size}}
The POIs' size attribute is described in Section \ref{sec:system_size}. For the study, we adjusted the base size and the dynamic scaling. Figure \ref{fig:HeightSizeDescription} illustrates the our four size conditions \textit{small\_static}, \textit{small\_dynamic}, \textit{large\_static}, and \textit{large\_dynamic}. Our dependent variables for size are satisfaction, visibility, and pleasantness. A low base size equated to a POI diameter of 3 meters, while a large base size equated to 7.5 meters. For all dynamic size trials, the minimum distance threshold equaled to 50 meters, the maximum distance threshold to 500 meters, and the maximum size scaling factor to 7. The values are based on the satisfaction factor in our pilot study.

We expected the conditions with low base size to be the preferred conditions with the highest satisfaction and pleasantness, as they don't obstruct a big portion of the outside view and thus could positively impact the experience \cite{BergerGridStudyInCarPassenger2021}. Additionally we hypothesized that the larger base size leads to higher visibility, as the larger POIs could improve readability, especially for POIs that are located further away. Also, the dynamic scaling could improve satisfaction and visibility for POIs across all distances as they adapt based on the distance to the user. They still appear smaller the further away they are, showing somewhat realistic behavior while not overly cluttering the FoV. Thus, we assumed the following hypotheses:
\begin{itemize}
    \item $H_{S1}$: A POI base size of approximately three meters leads to higher \textit{satisfaction} and \textit{pleasantness}.
    \item $H_{S2}$: POIs with a large base size of approximately 7.5 meters lead to higher \textit{visibility}.
    \item $H_{S3}$: Dynamic size scaling leads to higher \textit{visibility} and \textit{pleasantness}.
\end{itemize}



\begin{figure*}
    \centering
    \begin{subfigure}[b]{.55\textwidth}
        \centering
        \includegraphics[width=\textwidth]{Images/Schematic_Rotation.eps}
        \Description{}
    \end{subfigure}
    \hfill
    \begin{subfigure}{.4\textwidth}
        \centering
        \includegraphics[width=\textwidth]{Images/Schematic_RenderDistance.eps}
        \Description{A schematic illustration of the rotation and render distance conditions. The Figure consists of four images. Two images on the left represent the rotation condition. Each shows a car with four points of interest from above. For billboarding, all four points of interest are rotated towards the car's passenger seat, indicated by four arrows. For no rotation, each of the four points of interest a rotated paralell to the car's direction, again indicated by four arrows. The two images on the right represent the render distance. Each of them shows a car with points of interest from the side. For the short render distance, four points of interest are shown, where the last one is half transparent. For the long render distance, six points of interest are shown.}
    \end{subfigure}
    \caption{Schematic representation of the two study conditions manipulating the POIs' rotation (left) and render distance (right).}
    \label{fig:RotationRenderdistanceDescription}
\end{figure*}

\subsubsection*{\textbf{Rotation}}
The POIs' rotation is described in Section \ref{sec:system_rotation}. We tested two conditions regarding rotation in the study: \textit{billboarding} and \textit{no rotation}, as illustrated in Figure \ref{fig:RotationRenderdistanceDescription}. For the rating of the rotation, we used four word pairs for clarity, support, complexity, and pleasantness. The word pairs were taken from the short version of the User Experience Questionnaire \cite{schrepp2017design} with the exception of pleasantness, which we formulated ourselves.
  
We expected the billdboarding behavior to be the more supportive condition since there the POI-content is consistently available and readable. This could improve information delivery and thus be perceived as more pleasant to use. The non-rotating POIs may be easier to understand, as they resemble the static behavior known from real-life street signs. Additionally, they could support users by conveying information about the streets' direction and provide a clearer association between POIs and streets. As such, our hypotheses are as follows:
\begin{itemize}
    \item $H_{R1}$: Billdboarding POIs are more supportive for delivering information and are thus more pleasant.
    \item $H_{R2}$: Non-rotating POIs are easier to understand and give a clearer overview of the environment.
\end{itemize}


\subsubsection*{\textbf{Render Distance}}
The render distance described in Section \ref{sec:system_renderdistance} had two conditions in the study: \textit{long distance} and \textit{short distance}, as illustrated in Figure \ref{fig:RotationRenderdistanceDescription}. For the \textit{long distance} condition, the far edge to fade-in POIs was chosen at a distance of 500 meters. This resulted in all existing POIs to be displayed at all times, since our study environment had a maximum lenght of around 450 meters. For the \textit{short distance} condition, the far egde was set to 150 meters, resulting in three to four POIs being visible simultaneously while traversing a straight street segment. The fading distance was set to 50 meters for the far threshold and to 2.5 meters for the near threshold. For the render distance, participants could rate the POIs' time of appearance, ranging from \textit{way too early} to \textit{way too late}.

For render distance, we expected the \textit{short distance} to be preferable due to the reduced visual clutter and the difficulity to read far away POIs. Thus, our hypothesis regarding this variable is:
\begin{itemize}
    \item $H_{RD1}$: A short render distance is preferred by users.
\end{itemize}


\subsubsection*{\textbf{Information Density}}
There were two levels of content, resulting in two conditions tested: \textit{Low information density} and \textit{high information density}. Here, our dependent variable was satisfaction. For the \textit{low information density} condition, only a generic restaurant icon and the restaurant name was displayed. For the \textit{high information density} condition POIs showed the restaurant's name, an image, and a star rating ranging from zero to five. Examples for both conditions can be seen in Figure \ref{fig:POI_Appearances}. There was also the possibility to rate each part of the POIs' content individually during the post-questionnaires. Participants were shown a 3x4 matrix with the x-axis being the POI components name, star rating, icon, and image. The y-axis comprised of points in time on when the components could be shown: always, when nearby, and never. The matrix, including the results, is illustrated in table \ref{tab:InformationMatrix}.

We expected the \textit{high information density} to be the preferred condition, since it provides the most relevant information at a glance without requiring any additional interaction. Additionally, we hypothesized that more information should be displayed for near POIs, since far away POIs are less readable.

\begin{itemize}
    \item $H_{I1}$: POIs with three types of data are preferred by users.
    \item $H_{I2}$: Users want to have more data displayed for nearby POIs.
\end{itemize}



\subsubsection*{\textbf{Appearance}}
Our POIs comprise of spheres with one cut side, allowing for a flat space to display 2D information on. While facing the user in the billboarding conditions, the POIs appeared as two dimensional objects, as seen in Figure \ref{fig:POI_Appearances}. Most of the POIs face is filled with either a representative image stemming from the real location or a icon indicating the POIs' category. We chose a deliberately artifical look for the POIs' appearance, form, and color to make them stand out against the environment and to be visually interesting. As such, our hypotheses regarding appearance are as follows:
\begin{itemize}
    \item $H_{A1}$: The visual presentation of our POIs is appealing.
    \item $H_{A2}$: The POIs' color makes them stand out against the environment. 
\end{itemize}



% =====================================================
\subsection{Measures}
\label{sec:measures}
The participants were administered questionnaires before, during, and after the in-car study. The questionnaires that were specifically formulated for our study will be described on an abstract level in this Section and are located in their entirety in the appendix. All relevant results for the questionnaires are reported in Section \ref{section:results}, with detailed results also found in the appendix.

\subsubsection*{\textbf{Pre-Questionnaires}}
The pre-questionnaires included a socio-demographic questionnaire and two questionnaires regarding the participants affinity for both new technologies in general, as well as their affinity for AR. The pre-questionnaires were administered in a separate room and were filled out via keyboard and mouse on a computer. 
The socio-demographic questionnaire included questions about the participants' gender, age, body height, AR experience, their need for prescription glasses, and their suspectibility for MS. The results of the socio-demographic questionnaire are reported in Section \ref{sec:participants}.
For technical affinity, we used the affinity technology interaction scale (ATI) from Franke et al. \cite{franke2019personal}. For questions regarding participants' affinity specifically for AR, we used a mixture of self formulated questions and questions from Janzik \cite{janzik2022studie} with slight modifications.


\subsubsection*{\textbf{Study-Questionnaires}}
During the in-car part of the user study, participants could fill out the questionnaires for the dependent variables through an AR-interface included in the study software without the need of removing the HMD. The questions were always filled out while the car was in standstill, since the UI for the questions occluded most of the HMD's FoV and thus could induce MS \cite{Sasalovici23ArMs}. The questions were tailored to the conditions and mostly used seven-point Likert scales. In addition, all relevant verbal comments that the participants made during the in-car part of the study were noted in connection to the currently viewed condition. Those comments were later condensed into comment categories, which then got sorted into positive, neutral, and negative categories. The ratings for each condition are described respectively in Section \ref{sec:independentVariables}.


\subsubsection*{\textbf{Post-Questionnaires}}
After the study procedure in the car, there was no post-interview, participants directly filled out the post-questionnaires. These consisted of the acceptance regarding the AR-function, the POI appearance, and the intention of use. Like the pre-questionnaires, the post-questionnaires were administered in a separate room and were filled out via keyboard and mouse on a computer. All the post-questionnaire questions were self formulated and are listed in the appendix.

First, participants could rate their acceptance of the used AR function in general, using seven point Likert scales. For instance, some questions revolved around determining whether the AR function is deemed useful, innovative, or exciting. Afterward, participants were provided with two free-text fields where they could express their opinions on what they particularly enjoyed and disliked about the AR functionality.

The next set of questions regarded the POIs appearance, form, and color with seven point Likert scales. If the participant gave a negative rating with three and below, they were asked to fill out a freetext field why they were dissatisfied with this aspect of the POIs appearance. Then, for all participants, there were two additional freetext fields. There the participants were asked if they want to change any aspects of the POIs appearance other than those listed before. The other field had space to add any additional information or data to the POIs content that was not in the study.

The last set of questions regarded the intention of use for the AR-function. The first question regarded the preferred seat position depending on the cars' automation level. As such, the possible answers consisted of a 3x4 matrix where one axis was the seat position (driver, codriver, backseatpassenger, none) and the other axis was the automation level (level 3, level 4 and above, no automation). Multiple answers were possible. Participants were shortly briefed on what the automation levels meant beforehand.
Then, participants could rate if they want to use the AR function for different kinds of POIs, like theatres or gas stations from a list of nine possibilities. Afterwards they could pose their own kinds of POIs. 
\section{Results}\label{sec:results}
This section highlights the benefits of GraNNite optimization techniques, compares performance between Intel\textregistered\ Core\texttrademark\ Ultra Series 1 \& 2 NPUs, and demonstrates the superior energy efficiency of NPUs over CPUs and GPUs for GNN execution.
Since GraNNite is the first hardware-aware framework tailored for optimizing GNN deployment on COTS SOTA NPUs, no existing works exist for direct comparison.
% This section demonstrates how the various GraNNite optimization techniques enhance performance across different GNN models, highlighting significant improvements when compared to traditional CPU and GPU executions on Intel NPUs.
% Version #3

\textbf{Benefits of GraNNite Optimizations:} Fig.~\ref{plot:gnn_progression} illustrates the performance progression of GNN models on the Intel\textregistered\ Core\texttrademark\ Ultra Series 2 NPU, highlighting the impact of various optimizations proposed by GraNNite. Each optimization builds upon the preceding set unless otherwise specified. For example, the performance of QuantGr in GCN reflects a model in which GrAd, NodePad, GraphSplit, and QuantGr are cumulatively applied. However, in SAGE-max, EffOp and GrAx3 target the same model, and their performance gains are not cumulative.
For GCN, the initial optimization, StaGr combined with GraphSplit, achieves a $1.51\times$ speedup over the baseline by efficiently partitioning workloads between the CPU and NPU. Adding GrAd and NodePad introduces support for time-varying graphs and enhances parallelism but reduces performance to $1.4\times$ due to CPU preprocessing overhead and an increased node count on the NPU. GraSp further boosts throughput by $1.1\times$. The most significant improvement, $2.7\times$, is achieved by combining GrAd, NodePad, GraphSplit, and QuantGr, leveraging low-precision arithmetic to minimize computational overhead.
For GAT, EffOp alone provides a $3\times$ speedup, while incorporating approximations (GrAx2) boosts performance to $7.6\times$ with negligible impact on model quality. Similarly, for SAGE-max, EffOp yields a $2\times$ speedup, which increases to $3.2\times$ with GrAx3, again with no quality degradation.
We note that the effects of SymG and CacheG could not be demonstrated as they require modifications to the (proprietary) NPU compiler.
%, which is not open source.

\begin{figure}[t!]
\begin{center}
\includegraphics[width=\columnwidth]{Plots/MTL_vs_LNL_GCN.pdf}% This is a *.eps file
\end{center}
\caption{Performance of GCN on different Intel\textregistered\ NPUs: Intel\textregistered\ Core\texttrademark\ Ultra Series 2 and Intel\textregistered\ Core\texttrademark\ Ultra Series 1.}\label{plot:mtl_vs_lnl}
\end{figure}

\begin{figure}[t!]
\begin{center}
\includegraphics[width=\columnwidth]{Plots/CPU_GPU_NPU.pdf}% This is a *.eps file
\end{center}
\caption{Performance of GNN models on different devices of an Intel\textregistered\ AI PC: NPU outperforms CPU and GPU by a large margin.}\label{plot:cpu_gpu_npu}
\end{figure}

\textbf{Performance Comparison on Intel\textregistered\ Core\texttrademark\ Ultra Series 1 vs. Intel\textregistered\ Core\texttrademark\ Ultra Series 2 NPUs:} Fig.~\ref{plot:mtl_vs_lnl} compares GCN performance across GraNNite optimizations on Intel\textregistered\ Core\texttrademark\ Ultra Series 1 and Intel\textregistered\ Core\texttrademark\ Ultra Series 2 NPUs. Series 2 consistently outperforms series 1 due to its higher tile count (4 vs. 2). For the most optimized configuration (GrAd + NodePad + QuantGr), Intel\textregistered\ Core\texttrademark\ Ultra Series 2 delivers $1.7\times$ and $1.6\times$ higher throughput than Intel\textregistered\ Core\texttrademark\ Ultra Series 1 for the Cora and Citeseer datasets, respectively. This advantage arises from the higher number of MAC units in Series 2, enabling greater data parallelism. However, the observed gains fall short of the theoretical $2\times$ maximum due to limited parallelism inherent in the GCN.  

\textbf{Performance and Energy Efficiency of CPU, GPU, and NPU with GraNNite Optimizations:} Fig.~\ref{plot:cpu_gpu_npu} compares the performance of CPU, GPU, and NPU across three GNN layers: GraphConv (GCN), GraphAttn (GAT), and SAGE (GraphSAGE). For GCN, the NPU achieves a $2.9\times$ speedup over the GPU and $3.3\times$ over the CPU. For GAT layers, the NPU provides $2.3\times$ and $3.8\times$ improvements over the GPU and CPU, respectively. Similarly, for GraphSAGE with mean aggregation, the NPU achieves $6.7\times$ and $10.8\times$ speedups over the GPU and CPU. These results highlight the computational efficiency of NPUs and the effectiveness of GraNNite optimizations in delivering high-performance GNN execution without hardware modifications.  
Fig.~\ref{plot:energy_gcn} demonstrates the energy efficiency of NPUs compared to CPUs and GPUs for GNN execution. For the Cora dataset, the NPU is $4.1\times$ and $8.5\times$ more energy-efficient than the most efficient GPU and CPU implementations, respectively. Similarly, for the Citeseer dataset, the NPU achieves $4.4\times$ and $8.6\times$ greater energy efficiency.


% Version #2
% Fig.~\ref{plot:gnn_progression} shows the performance progression of GNNs on the Intel Lunar Lake NPU, highlighting significant improvements from a series of targeted optimizations proposed by GraNNite. It is to be noted that the optimizations are progressively added unless they are . For example, the performance for QuantGr in GCN is shown for a model with GrAd, NodePad, GraphSplit and QuantGr applied to the GNN model, not just the QuantGr. But for SAGE-max, EffOp and GrAx3 target the same model section, therefore, the performance gains shown in the plot are not cumulative. For GCN, the first optimization (StaGr + GraphSplit), enhances model execution by efficiently distributing the workload between the CPU and NPU, achieving a $1.51\times$ performance boost over the baseline. Adding GrAd and NodePad allows handling time-varying graphs and ensures efficient parallelism, though it slightly reduces performance as compared to (StaGr + GraphSplit) by $1.4\times$ due to the additional pre-processing overhead on CPU and increased number of nodes on the NPU. The most substantial improvement comes from combining GrAd, NodePad, GraphSplit, and QuantGr, which uses low-precision arithmetic to reduce computational load, resulting in a $2.7\times$ performance gain.
% For GAT, EffOp yields a $3\times$ performance boost. When we incorporate approximation, the improvement jumps to $7.6\times$, with almost no degradation in quality.
% For SAGE-max, EffOp yields a $2\times$ performance boost. When we incorporate approximation (GrAx3), the improvement jumps to $3.2\times$, with no degradation in quality.
% Fig.~\ref{plot:mtl_vs_lnl} compares GCN performance across different GraNNite optimization techniques on NPUs of two Intel AI PCs, meteor lake and lunar lake. We observe that Lunar Lake consistently delivers higher performance as it has higher number of tiles (4) as compared to meteor lake (1). For the most optimized version (GrAd + NodePad + QuantGr), lunar lake archives $1.7\times$ ($1.6\times$) higher throughput than meteor lake for Cora (Citeseer) dataset. The presence of higher number of MAC units in lunar lake enables higher data parallelism leading to better performance. Although the performance gain is not equal to the theoretical maximum (4X) due to the limited data parallelism in the GCN model.
% Fig.~\ref{plot:cpu_gpu_npu} compares the performance of CPU (blue), GPU (orange), and NPU (green) across three GNN layer types: GraphConv (GCN), GraphAttn (GAT), and SAGE (GraphSAGE). For GCN, the NPU achieves a remarkable $17.3\times$ speedup over the GPU and $4.6\times$ over the CPU, showcasing its efficiency in handling these workloads. Similarly, the NPU demonstrates $2.3\times$ and $3.8\times$ improvements over GPU and CPU, respectively, for GAT layers, and achieves $6.7\times$ and $10.8\times$ speedups for GraphSAGE with mean aggregation. These results underscore the NPU's computational advantages and the effectiveness of GraNNite's optimizations, enabling high-performance GNN execution on existing hardware without modifications.
% Fig.~\ref{plot:energy_gcn} demonstrates the need for mapping the GNN models on NPU for energy efficiency. We observe that NPU is $4.1\times$ ($4.4\times$) energy efficient than the most energy efficient GPU implementation for Cora (Citeseer) dataset. Similarly, NPU is $8.5\times$ ($8.6\times$) energy efficient than the most energy efficient CPU implementation for Cora (Citeseer) dataset. 
% It is to be noted that, we could not demonstrate the impact of SymG and CacheG as those would require changes in the NPU compiler which is not made open source.

% Version #1
% Fig.~\ref{plot:gnn_progression}(a) shows the performance progression of Graph Convolutional Networks (GCN) on the Intel Lunar Lake NPU, highlighting significant improvements from a series of targeted optimizations. Here, the unoptimized implementation serves as the reference baseline.
% The first optimization, Optimized Graph Partitioning (OGP), enhances data locality by efficiently distributing the workload between the CPU and NPU, achieving a $1.85\times$ performance boost over the baseline. Adding Node Padding (NP) allows handling time-varying graphs and ensures efficient parallelism, though it slightly reduces performance by $1.1\times$ due to the additional processing overhead on the CPU. The most substantial improvement comes from combining OGP, NP, and Quantization, which uses low-precision arithmetic to reduce computational load, resulting in a $2.7\times$ performance gain.

% Fig.~\ref{plot:gnn_progression}(b) demonstrates the performance improvements of Graph Attention Network (GAT) implementations on an Intel NPU, achieving a $7.6\times$ speedup over the baseline.
% The first optimization replaces the ``Select" operation with element-wise multiplication, yielding a $3\times$ performance boost by simplifying the computation. Next, the element-wise multiplication is offloaded to the DPU, providing an additional $3.5\times$ performance gain by focusing computation on the DPU. Finally, eliminating the broadcast addition operation, which causes memory overhead, results in a substantial performance improvement, reaching the $7.6\times$ speedup.

% Fig.~\ref{plot:gnn_progression}(c) showcases the performance gains of a SAGE model with the max aggregation scheme, achieving up to $3.2\times$ speedup over the baseline.
% The first optimization replaces the complex ``Select" operation with a more efficient element-wise multiplication, boosting performance to $2\times$ the baseline. The second optimization swaps the ``ReduceMax" operation for ``MaxPool1D," aligning better with hardware architecture and providing an additional performance increase, reaching the final $3.2\times$ speedup.

% Fig.~\ref{plot:cpu_gpu_npu} compares the performance of CPU (blue), GPU (orange), and NPU (green) across three GNN layer types: GraphConv (GCN), GraphAttn (GAT), and SAGE (GraphSAGE). For GCN, the NPU achieves a remarkable $17.3\times$ speedup over the GPU and $4.6\times$ over the CPU, showcasing its efficiency in handling these workloads. Similarly, the NPU demonstrates $2.3\times$ and $3.8\times$ improvements over GPU and CPU, respectively, for GAT layers, and achieves $6.7\times$ and $10.8\times$ speedups for GraphSAGE with mean aggregation. These results underscore the NPU's computational advantages and the effectiveness of GraNNite's optimizations, enabling high-performance GNN execution on existing hardware without modifications.

% Version #0
% These optimizations demonstrate how integrating algorithmic improvements, memory management, and hardware-friendly approaches unlocks the full performance potential of GCNs on NPUs.

% Fig.~\ref{plot:gcn_progression} illustrates the performance progression of Graph Convolutional Network (GCN) implementations on an Intel Lunar Lake NPU, demonstrating significant enhancements achieved through a series of targeted optimizations. The baseline unoptimized implementation is set as the reference point, representing the lowest performance.
% The first optimization, Optimized Graph Partitioning (OGP), focuses on improving data locality by effectively distributing the workload between the CPU and NPU for a static input graph. This optimization results in a notable performance boost of approximately 1.85X over the baseline.
% Next, the addition of Node Padding (NP) to the OGP approach enables the model to handle time-varying input graphs. This ensures efficient parallelism across compute units, although it slightly reduces performance by about 1.1X compared to OGP alone. This decrease is attributed to the extra processing time required for the normalization matrix on the CPU.
% The most significant performance improvement is observed with the combination of OGP, NP, and Quantization. By employing low-precision arithmetic, this approach reduces the overall computational workload, leading to a remarkable 2.7X enhancement over the initial implementation.
% The consistent increase in performance across these optimization stages underscores the value of integrating algorithmic optimizations like OGP with memory management techniques (NP) and hardware-friendly approaches (quantization). This cumulative application of optimizations highlights that while each individual optimization is beneficial, their combined effect is essential for unlocking the full performance potential of GCNs on NPUs.


% \begin{figure}[t!]
% \begin{center}
% \includegraphics[width=\columnwidth]{Plots/GCN_progression.png}% This is a *.eps file
% \end{center}
% \caption{Progressive performance improvement of GCN through different optimizations}\label{plot:gcn_progression}
% \end{figure}


% These optimizations highlight the importance of reducing unnecessary memory operations and offloading tasks to specialized cores, significantly improving inference latency and efficiency for GAT models on NPUs in resource-constrained environments.


% Fig.~\ref{plot:gat_progression} showcases the performance improvements of Graph Attention Network (GAT) implementations on an Intel NPU, illustrating how a series of optimizations culminate in a substantial 7.6X speedup over the baseline implementation. The baseline serves as the starting point and represents the lowest performance due to the computational inefficiencies inherent in certain operations typically used in GAT models.
% The first optimization involves replacing the "Select" operation with element-wise multiplication, which is a simpler and more parallelizable operation. This initial change yields an impressive improvement of approximately 3X over the baseline performance, highlighting the benefits of simplifying computational tasks.
% In the second stage of optimization, the element-wise multiplication operation is further refined; instead of performing the multiplication operation alongside other computations, it is exclusively executed on the DPU. This focused approach results in a cumulative performance boost of around 3.5X relative to the original implementation, indicating that optimizing where and how computations are performed is critical for enhancing performance.
% The final optimization addresses the broadcast addition operation, which often incurs significant memory overhead by duplicating data across tensors. By eliminating this redundant operation, the GAT implementation experiences a substantial performance enhancement, achieving a maximum of 7.6X speedup over the baseline. 
% This progressive enhancement illustrates the crucial role of reducing unnecessary memory operations and leveraging specialized processing cores for performance-critical tasks. The results emphasize that architectural-aware optimizations—such as offloading specific workloads from the DSP to DPU cores and eliminating redundant operations through approximations—can lead to significant improvements in inference latency for GAT models on NPUs. Such strategies not only optimize computational efficiency but also facilitate faster and more effective execution of GNNs in resource-constrained environments.


% \begin{figure}[t!]
% \begin{center}
% \includegraphics[width=\columnwidth]{Plots/GAT_progression.png}% This is a *.eps file
% \end{center}
% \caption{Progressive performance improvement of GAT through different optimizations}\label{plot:gat_progression}
% \end{figure}


% This progression demonstrates the value of targeted optimizations in reducing computational overhead, enhancing data-parallel processing, and maximizing performance for GNN models on specialized hardware.

% Fig.~\ref{plot:sage_progression} demonstrates the performance improvements of a SAGE model with the max aggregation scheme following a series of targeted optimizations, ultimately achieving a cumulative speedup of up to 3.2X compared to the baseline. The baseline reflects the initial performance prior to any optimizations, serving as a reference for evaluating the impact of each subsequent modification.
% The first optimization involves substituting the "Select" operation—known for its control-flow complexity—with a data-parallel element-wise multiplication. This shift to a computationally more efficient operation delivers a substantial boost, bringing the performance to approximately 2X of the baseline. This optimization illustrates how replacing control-flow-heavy operations with data-parallel alternatives can enhance computational efficiency.
% Building upon this, a second optimization replaces the "ReduceMax" operation with "MaxPool1D," a more streamlined operation that aligns better with the hardware's architecture. This adjustment leads to an additional performance increase, as depicted by the green bar on the right, resulting in a total improvement of 3.2X over the baseline configuration.
% Overall, this progression highlights the impact of carefully selected optimizations in reducing computational overhead, enhancing data-parallel processing, and improving model efficiency. These results underscore the effectiveness of architectural-aware optimizations in maximizing performance for GNN models on specialized hardware.


% \begin{figure}[t!]
% \begin{center}
% \includegraphics[width=\columnwidth]{Plots/SAGE_progression.png}% This is a *.eps file
% \end{center}
% \caption{Progressive performance improvement of SAGE-max through different optimizations}\label{plot:sage_progression}
% \end{figure}



% \subsection{CPU, GPU \& NPU performance per watt for GCN, GAT and GraphSAGE}
% Figure~\ref{plot:power} presents the power consumption breakdown of various components in different operational states of an AI PC, including IDLE and during the execution of GNN models on different devices. The x-axis shows the specific GNN models in use and the devices they are mapped to, allowing for a comparison of power usage across distinct deployment scenarios.
% The first bar on the left represents the system’s IDLE state, where no workload is running on any device. This IDLE power breakdown provides a baseline to compare against the power demands when GNN models are actively running on various devices within the AI PC.
% Moving beyond IDLE, the figure details the power distribution among key system components—IA cores, System Agent, GT, and DRAM—when GNN models are executed, especially highlighting the benefits of NPU deployment. When a model runs on the NPU, the System Agent’s power consumption, shown in blue, increases due to its role in managing the NPU, which draws from the System Agent’s power rail. However, despite this rise in the System Agent’s power draw, the total power usage across all components (including IA cores, GT, and DRAM) remains notably low when models are mapped to the NPU.
% This low cumulative power usage, paired with the NPU’s high processing efficiency (as demonstrated in previous figures), results in excellent performance per watt. Such efficiency makes the NPU highly suitable for applications that demand both high performance and low energy consumption. Specifically, the NPU’s ability to efficiently handle GNN workloads with minimal power draw makes it well-suited for high-performance tasks in power-sensitive settings. In summary, Figure~\ref{plot:power} underscores how the NPU’s balanced approach to speed and power usage makes it a compelling option for deploying GNN models in resource-constrained environments.

% \begin{figure}[t!]
% \begin{center}
% \includegraphics[width=\columnwidth]{Plots/Power.png}% This is a *.eps file
% \end{center}
% \caption{Power consumption of different GNN models on Intel AI PC: NPU takes lower power and compute with a higher speed}\label{plot:power}
% \end{figure}



% Fig.~\ref{plot:cpu_gpu_npu} presents a performance comparison among CPU (blue), GPU (orange), and NPU (green) in executing three types of GNN layers: GraphConv (GCN), GraphAttn (GAT), and SAGE (GraphSAGE). For the GraphConv (GCN), the NPU achieves an impressive 17.3× speedup compared to the GPU and a 4.6× speedup over the CPU. This result highlights the NPU's significant efficiency in managing GCN workloads.
% In the case of the GraphAttn (GAT), the NPU demonstrates a performance improvement of 2.3× over the GPU and 3.8× over the CPU. Likewise, for the SAGE (GraphSAGE) using the mean aggregator scheme, the NPU outperforms the GPU by 6.7× and the CPU by 10.8×. These results clearly indicate the superior computational capabilities of NPUs and efficacy of GraNNite proposed optimizations, particularly when applied to our most optimized GNN layers. The consistent performance advantage of NPUs over traditional architectures like CPUs and GPUs across these benchmarks suggests that existing NPUs can effectively implement GNNs using the proposed optimizations, without necessitating any changes to the underlying hardware.



\begin{figure}[t!]
\begin{center}
\includegraphics[width=\columnwidth]{Plots/Energy_GCN.pdf}% This is a *.eps file
\end{center}
\caption{Normalized Energy Consumption of GCN on Intel\textregistered\ Core\texttrademark\ Ultra Series 2 Devices (CPU, GPU, and NPU), highlighting significant energy savings achieved with GraNNite optimizations.}\label{plot:energy_gcn}
\end{figure}
\section{Discussions}

% \subsection{Bridge the gap between insights and expressions}



\noindent\textbf{Bridge the gap between insights and expressions with AI-powered domain-focused video creation.}
% video creation for different domains
As images and videos continue to dominate communication mediums, visualization and video technologies have become essential tools for enabling diverse domains and the public to express themselves effectively. Emerging generative AI tools, such as Sora~\cite{sora} and Pika~\cite{pika}, exemplify this trend by facilitating creative expression across various fields.

While general AI-driven video creation tools are increasingly popular, our work emphasizes the critical need for domain-specific video creation tools like \SB{} to address unique requirements within specific fields. There are two primary reasons for prioritizing domain-specific video creation over general generative technologies.
% 
First, domain-specific videos, such as sports highlights, rely heavily on human insights. Audiences seek to learn from professionals through these videos, requiring tools that provide greater user control and enable experts to effectively translate their insights into engaging content. 
% \SB{} supports this by enabling users to maintain control over the conveyed insights, ensuring that the final video accurately reflects expert knowledge and user intentions.
% 
Second, the complexity of domain-specific data, such as the intricate motion and strategy analysis, demands advanced data visualization and seamless synchronization of visuals and audio, which general tools may not provide. 
% \SB{} addresses these needs by providing specialized tools that cater to the detailed and dynamic nature of sports content.

\SB{} addresses these needs by integrating automation with customizable visualizations, tailored to the intricate and dynamic nature of sports content. It allows flexible user control through embedded interactions, 
reducing technical barriers and empowering users to effectively communicate their insights. Feedback from users further underscores the importance of balancing automation with user control to accommodate diverse goals and preferences to enhance accessibility across various user groups and use cases, such as tactical analysis, skill development, and profile building. 
% For instance, professional coaches can use \SB{} to create detailed breakdowns of game strategies for training and coaching. Parents and young athletes can produce polished highlight reels for recruitment.
% These examples illustrate how AI-driven tools can empower users across various levels and industries to create videos with meaningful insights, fostering deeper engagement and broader impact. 

Beyond sports, similar tools have the potential to transform fields like healthcare and education, incorporating precise visual aids and step-by-step breakdowns. 
% These applications highlight the transformative potential of tailored video content in amplifying personal expression and benefiting broader audiences.
% 
Future research is required to investigate the balanced integration of AI and intuitive interface design, such as multi-modal interaction~\cite{wang2024lave}, to further advance domain-specific video creation and expression across diverse fields.
% By continuing to develop and refine domain-specific video creation tools, we can unlock new possibilities for effective communication and expression in numerous fields, ultimately bridging the gap between insights and their visual expressions.

% \subsection{Cross sports visualizations - allow different sports domains to leverage other sports' insights}

% \subsection{Enhance human-AI collaboration - creators focus on content while AI helps with editing tasks}


\vspace{1mm}
\noindent\textbf{Promote visualization in practice through real-world system deployment.}
Our work on SportsBuddy advances existing research in sports visualization and video authoring by emphasizing real-world system deployment and evaluation. Through this study, we have identified two significant benefits.

First, deploying SportsBuddy in authentic environments allowed us to validate and refine our design based on genuine use cases and users, uncovering insights that controlled laboratory settings cannot capture. For instance, we discovered that even within a similar user group of content creators, priorities varied significantly—some focused on showcasing player actions, while others emphasized strategic communication. This diversity led to iterative design improvements that balanced the distinct needs of each user group and support customization without complicating user interactions. 

Second, real-world deployment enables the assessment of long-term impacts and the discovery of unique use cases by diverse users. 
For example, some sports experts were hesitant to adopt SportsBuddy initially despite the perceived usefulness they shared. Upon further investigation, this was due to the context-switching costs. This feedback highlighted the necessity for a streamlined workflow tailored to the sports domain, leading to our design of batch processing and web import options. In addition, we observed many users preferred embedded annotation with \Text{} features over typical captions for sharing insights (see Fig.~\ref{fig:case_study}d), suggesting a new form of video storytelling inspired by \SB{}’s design. 
Feedback and insights from our diverse user base has highlighted the value of creating flexible and accessible visualization tools, which offers important external validity of the human-centered system.

This real-world deployment approach not only enhances visualization literacy and accessibility but also ensures that innovative designs translate into practical, widely usable tools, providing a validation for interactive visualization design. Therefore, we advocate for more visualization research to focus on real-world system deployments and to share design learnings, inspiring use cases that are both practical and impactful.

{
\subsection{Future Work}

While SportsBuddy has shown great potential in simplifying sports video storytelling, 
there are key areas for further improvement:

\vspace{1mm}
\noindent\textbf{Enhancing Player Tracking Under Occlusion and Motion Changes.}
The current tracking system faces challenges with occlusions and rapid motion in dynamic scenarios. Future work will refine tracking algorithms using larger domain-specific datasets and multi-view setups to improve accuracy in complex environments.

% The current tracking system struggles with occlusions and rapid motion changes in crowded or dynamic scenarios. Future efforts will focus on refining tracking algorithms using more extensive domain-specific datasets and, where feasible, incorporating multi-view camera setups for improved accuracy. These enhancements aim to ensure reliable tracking in complex sports environments.

\vspace{1mm}
\noindent\textbf{Addressing Perspective and Camera Movement.}
Shifts in camera angles or perspectives cause misalignment issues due to reliance on fixed transformation matrices. Dynamic court mapping and machine learning for real-time adjustments, along with camera metadata integration, will ensure consistent and accurate visualizations.

% Misalignment issues arise when camera angles or perspectives shift, as the system relies on a fixed transformation matrix. Future work will explore dynamic court mapping techniques and machine learning methods for real-time adjustments. Incorporating camera metadata will further enhance visualization accuracy, ensuring effects remain consistent with the game’s context.

\vspace{1mm}
\noindent\textbf{Supporting Longer Videos.}
Longer or higher-resolution videos can strain browser performance. To mitigate this, we will implement dynamic video loading from cloud storage and on-demand decoding, and adopt frame compression during previews to further optimize memory usage and rendering, ensuring smoother video processing.
% Longer or higher-resolution videos may strain browser performance. To address this, dynamic video loading from cloud storage and on-demand decoding will be introduced. Additionally, frame compression during previews will reduce memory usage and rendering time, enabling smoother processing of large and complex videos.



\vspace{1mm}
\noindent\textbf{Extending to Other Sports.}
\SB{} currently focuses on basketball but can expand to sports like soccer and tennis. This requires adapting tracking algorithms and designing sport-specific visualizations to accommodate the unique dynamics and storytelling needs of each sport.

}


% We advocate for more visualization paper that focus on deplyong system in real-world and evaluate their usage for two reasons. 
% 1. In vis research, application paper often address specific domain problems and create a prototype to evaluate with domain experts in a controlled setting. Most projects stop after user evaluation in the lab and the paper is published. With visualization system in real-world that value the practicality of system design and deployment in the wild, it encourages promoting real-world impact brought by novel visualization design, which is crucial in the current visualization community as we promote literacy and accessiblity of visualizations.
% 2. we should also promote long term impact of visualization design, and identify real-wordl use case and learning that might be drastically different from design study that are typically in lab, with a small amount of users, typically university students or academic members.


\section{Conclusion}

We presented \sys, a sparsity-adaptive attention mechanism for efficient long-context LLM inference. Unlike fixed token budget methods, \sys dynamically selects tokens based on cumulative attention scores, adapting to variations in attention sparsity. By leveraging clustering-based sorting and distribution fitting, \sys accurately estimates token importance with low overhead. Our results showed that \sys outperforms existing sparse attention methods, achieving higher accuracy and significant inference speedups, making it a practical solution for long-context LLMs.



%% The acknowledgments section is defined using the "acks" environment
%% (and NOT an unnumbered section). This ensures the proper
%% identification of the section in the article metadata, and the
%% consistent spelling of the heading.
% \begin{acks}
% ACKS
% \end{acks}

%%
%% The next two lines define the bibliography style to be used, and
%% the bibliography file.
\bibliographystyle{ACM-Reference-Format}
\bibliography{bibliography}


\clearpage


%% If your work has an appendix, this is the place to put it.
\appendix
\section{Appendix}
\subsection{Data - Survey on current passenger behaviour regarding POIs}

\begin{center}
    \begin{minipage}{\textwidth}

        \centering
        \captionof{table}{Usage of various navigation methods by our survey participants.}
        \begin{tabular}{r|cc|cc|cc|cc|cc}
            \toprule
            & \multicolumn{2}{c|}{AppleCar/AndroidAuto} & \multicolumn{2}{c|}{InVehicleSystems} & \multicolumn{2}{c|}{SmartphoneApps} & \multicolumn{2}{c|}{Compass} & \multicolumn{2}{c}{PaperMaps} \\
            & N & \% & N & \% & N & \% & N & \% & N & \% \\
            \midrule
            Never & 39 & 36.4\% & 13 & 12.1\% & 2 & 1.9\% & 101 & 94.4\% & 79 & 73.8\% \\
            Rarely & 19 & 17.8\% & 19 & 17.8\% & 15 & 14.0\% & 2 & 1.9\% & 26 & 24.3\% \\
            Sometimes & 18 & 16.8\% & 22 & 20.6\% & 33 & 30.8\% & 2 & 1.9\% & 2 & 1.9\% \\
            Often & 22 & 20.6\% & 34 & 31.8\% & 32 & 29.9\% & 0 & 0\% & 0 & 0\% \\
            Always & 9 & 8.4\% & 19 & 17.8\% & 25 & 23.4\% & 2 & 1.9\% & 0 & 0\% \\
            \bottomrule
        \end{tabular}

        \vspace{\baselineskip}
        
        \captionof{table}{The types of information drivers and passengers need to find a missed point of interest. Multiple choice was possible.}
        \begin{tabular}{r|cc|cc}
            \toprule
            \textbf{Type of information} & \multicolumn{2}{c|}{Drivers} & \multicolumn{2}{c}{Passengers} \\
            & N & \% & N & \% \\
            \midrule
            Name & 71 & 89\% & 65 & 90\% \\
            Perspective Picture & 39 & 49\% & 38 & 53\% \\
            Web Picture & 46 & 58\% & 44 & 61\% \\
            Description Text & 45 & 56\% & 49 & 68\% \\
            Category & 45 & 56\% & 49 & 68\% \\
            \bottomrule
        \end{tabular}

        \vspace{\baselineskip}

        \captionof{table}{Percentages on how and when passengers and drivers look for a missed point of interest.}
        \begin{tabular}{r|c|c}
            \toprule
            \textbf{Action}             & \textbf{Drivers}   & \textbf{Passengers} \\
            \midrule
            Nothing                     & 5\%                 & 4\% \\
            System saves Automatically  & 0\%                 & 3\% \\
            Stop the Car                & 3\%                 & 0\% \\
            Search later on Smartphone  & 46\%                & 8\% \\
            Search immediately on Smartphone & 11\%           & 49\% \\
            Search later on Navigation system & 24\%          & 5\% \\
            Search immediately on Navigation system & 11\%    & 31\% \\
            \bottomrule
        \end{tabular}

    \end{minipage}
\end{center}



\clearpage
\subsection{Data - Pre-Study on Eye-Gaze Interaction}

\begin{center}
    \begin{minipage}{\textwidth}
      
    \centering
    \captionof{table}{Descriptive Statistics for the pre-study Raw NASA Task Load Index.}
    \begin{tabular}{l|c|c|c|c|c|c|c}
        \toprule
        & \textbf{Mental} & \textbf{Physical} & \textbf{Temporal} & \textbf{Performance} & \textbf{Effort} & \textbf{Frustration} & \textbf{Score} \\
        \midrule
        Mean & 15.0 & 19.5 & 40.0 & 33.5 & 28.0 & 13.5 & 24.8 \\
        Median & 12.5 & 15.0 & 42.5 & 25.0 & 32.5 & 15.0 & 25.0 \\
        Standard Deviation & 10.5 & 19.9 & 23.1 & 21.4 & 18.1 & 11.6 & 9.47 \\
        Shapiro-Wilk W & 0.942 & 0.782 & 0.961 & 0.848 & 0.898 & 0.882 & 0.959 \\
        Shapiro-Wilk p & 0.573 & 0.009 & 0.794 & 0.055 & 0.206 & 0.139 & 0.769 \\
        \bottomrule
    \end{tabular}  

    \vspace{\baselineskip}

    \captionof{table}{Descriptive Statistics for the pre-study System Usability Scale.}
    \begin{tabular}{l|c|c|c|c|c}
        \toprule
        \textbf{Question} & \textbf{Mean} & \textbf{Median} & \textbf{Std. Deviation} & \textbf{Shapiro-Wilk W} & \textbf{Shapiro-Wilk p} \\
        \midrule
        Frequent Use & 4.00 & 4.00 & 0.943 & 0.841 & 0.045 \\
        Unnecessary Complex & 1.40 & 1.00 & 0.516 & 0.640 & < .001 \\
        Easy to Use & 4.60 & 5.00 & 0.516 & 0.640 & < .001 \\
        Support of Technical & 1.50 & 1.00 & 0.972 & 0.603 & < .001 \\
        Well Integrated & 4.40 & 4.00 & 0.516 & 0.640 & < .001 \\
        Inconsistency & 1.80 & 2.00 & 0.789 & 0.820 & 0.025 \\
        Learn Quickly & 4.60 & 5.00 & 0.699 & 0.650 & < .001 \\
        Cumbersome & 1.60 & 1.00 & 0.966 & 0.678 & < .001 \\
        Confident & 4.20 & 4.00 & 0.632 & 0.794 & 0.012 \\
        Learn a Lot Before & 1.10 & 1.00 & 0.316 & 0.366 & < .001 \\
        Score & 86.0 & 87.5 & 8.01 & 0.925 & 0.398 \\
        \bottomrule
    \end{tabular}

    \end{minipage}
\end{center}



\clearpage
\subsection{Data - Study on Visualizing Passed and Upcoming POIs}

\begin{center}
    \begin{minipage}{\textwidth}

        \centering
        \captionof{table}{Descriptive Statistics for the visualization-study Raw NASA Task Load Index, grouped by condition.}
        \begin{tabular}{llccccccc}
            \toprule
            \textbf{Statistic} & \textbf{Condition} & \textbf{Mental} & \textbf{Physical} & \textbf{Temporal} & \textbf{Performance} & \textbf{Effort} & \textbf{Frustration} & \textbf{Score} \\
            \midrule
            \multirow{3}{*}{Mean} & List & 28.6 & 25.5 & 20.7 & 14.8 & 29.5 & 19.5 & 23.1 \\
            & Timeline & 29.8 & 27.1 & 23.3 & 20.0 & 27.4 & 28.6 & 26.0 \\
            & Minimap & 46.7 & 36.4 & 42.4 & 42.6 & 56.9 & 50.2 & 45.9 \\
            \midrule
            \multirow{3}{*}{Median} & List & 25 & 20 & 15 & 5 & 25 & 15 & 25.0 \\
            & Timeline & 25 & 20 & 15 & 5 & 20 & 20 & 22.5 \\
            & Minimap & 40 & 30 & 45 & 40 & 60 & 55 & 49.2 \\
            \midrule
            \multirow{3}{1.5cm}{Std. Deviation} & List & 19.8 & 21.3 & 17.0 & 19.5 & 25.3 & 18.5 & 14.6 \\
            & Timeline & 22.8 & 21.3 & 19.6 & 29.0 & 26.6 & 26.0 & 19.3 \\
            & Minimap & 21.5 & 24.7 & 22.3 & 22.8 & 23.5 & 22.6 & 15.8 \\
            \midrule
            \multirow{3}{1.5cm}{Shapiro-Wilk W} & List & 0.917 & 0.879 & 0.909 & 0.767 & 0.835 & 0.851 & 0.941 \\
            & Timeline & 0.868 & 0.904 & 0.911 & 0.683 & 0.821 & 0.874 & 0.927 \\
            & Minimap & 0.947 & 0.886 & 0.920 & 0.938 & 0.954 & 0.946 & 0.927 \\
            \midrule
            \multirow{3}{1.5cm}{Shapiro-Wilk p} & List & 0.076 & 0.014 & 0.052 & < .001 & 0.002 & 0.004 & 0.229 \\
            & Timeline & 0.009 & 0.043 & 0.058 & < .001 & 0.001 & 0.011 & 0.118 \\
            & Minimap & 0.297 & 0.019 & 0.088 & 0.201 & 0.399 & 0.290 & 0.118 \\
            \bottomrule
        \end{tabular}


        \vspace{\baselineskip}

    
        \captionof{table}{Descriptive Statistics for the visualization-study System Usability Scale scores, grouped by condition.}
        \begin{tabular}{lccccc}
            \toprule
            \textbf{Condition} & \textbf{Mean} & \textbf{Median} & \textbf{Std. Deviation} & \textbf{Shapiro-Wilk W} & \textbf{Shapiro-Wilk p} \\
            \midrule
            List & 78.5 & 77.5 & 10.2 & 0.885 & 0.018 \\
            Minimap & 61.1 & 60.0 & 14.3 & 0.962 & 0.563 \\
            Timeline & 71.5 & 75.0 & 15.9 & 0.921 & 0.091 \\
            \bottomrule
        \end{tabular}


        \vspace{\baselineskip}

        
        \caption{Descriptive Statistics for the visualization-study Motion Sickness Questionnaire, grouped by time of completion.}
        \begin{tabular}{lccccc}
            \toprule
            \textbf{Study Order} & \textbf{Mean} & \textbf{Median} & \textbf{Std. Deviation} & \textbf{Shapiro-Wilk W} & \textbf{Shapiro-Wilk p} \\
            \midrule
            Pre-study & 0.381 & 0 & 0.669 & 0.617 & < .001 \\
            After First Condition & 0.857 & 0 & 1.15 & 0.732   & < .001 \\
            After Second Condition & 1.19 & 1 & 1.36 & 0.822   & 0.001  \\
            After Third Condition & 1.00 & 1 & 1.05 & 0.826    & 0.002  \\
            \bottomrule
        \end{tabular}


    \end{minipage}
\end{center}


\begin{table*}[ht]
    \centering
    \caption{Descriptive Statistics for the visualization-study User Experience Questionnaire, grouped by condition.}
    \begin{tabular}{llcccccc}
    \toprule
    \textbf{} & \textbf{Condition} & \textbf{Attractiveness} & \textbf{Perspicuity} & \textbf{Efficiency} & \textbf{Dependability} & \textbf{Stimulation} & \textbf{Novelty} \\
    \midrule
    \multirow{3}{*}{\textbf{M}} & Timeline & 1.26 & 1.42 & 1.10 & 1.12 & 1.11 & 1.37 \\
    & Minimap & 0.952 & 0.964 & 0.619 & 0.821 & 1.00 & 1.57 \\
    & List & 1.78 & 2.05 & 1.79 & 1.68 & 1.54 & 1.31 \\
    \midrule
    \multirow{3}{*}{\textbf{Mdn}} & Timeline & 1.50 & 1.75 & 1.25 & 1.25 & 1.25 & 1.50 \\
    & Minimap & 1.00 & 0.750 & 0.500 & 0.500 & 1.00 & 1.25 \\
    & List & 1.83 & 2.25 & 1.50 & 1.50 & 1.50 & 1.25 \\
    \midrule
    \multirow{3}{*}{\textbf{SD}} & Timeline & 0.921 & 1.03 & 0.937 & 0.883 & 0.986 & 1.11 \\
    & Minimap & 0.972 & 1.01 & 0.993 & 0.946 & 0.939 & 1.13 \\
    & List & 0.642 & 0.692 & 0.755 & 0.717 & 0.704 & 0.798 \\
    \midrule
    \multirow{3}{*}{\textbf{W}} & Timeline & 0.938 & 0.833 & 0.951 & 0.912 & 0.970 & 0.933 \\
    & Minimap & 0.972 & 0.975 & 0.980 & 0.948 & 0.967 & 0.877 \\
    & List & 0.976 & 0.893 & 0.847 & 0.926 & 0.942 & 0.982 \\
    \midrule
    \multirow{3}{*}{\textbf{p}} & Timeline & 0.200 & 0.002 & 0.359 & 0.059 & 0.737 & 0.160 \\
    & Minimap & 0.783 & 0.844 & 0.918 & 0.313 & 0.661 & 0.013 \\
    & List & 0.866 & 0.026 & 0.004 & 0.112 & 0.235 & 0.953 \\
    \bottomrule
    \end{tabular}
\end{table*}

\begin{table*}[ht]
    \centering
    \caption{User ranking values for the visualization-study conditions.}
    \begin{tabular}{l|cc|cc|cc}
    \toprule
    \multirow{2}{*}{\textbf{Condition}}& \multicolumn{2}{c|}{\textbf{Least Favorite}} & \multicolumn{2}{c|}{\textbf{Middle}} & \multicolumn{2}{c}{\textbf{Favorite}} \\
                & N     & \%             & N    & \%                  & N       & \%                 \\
    \midrule
    List        & 2     & 9.5\%          & 4    & 19.0\%              & 15      & 71.4\%              \\
    Minimap     & 11    & 52.4\%         & 7    & 33.3\%              & 3       & 14.3\%             \\
    Timeline    & 8     & 38.1\%         & 10   & 47.6\%              & 3       & 14.3\%             \\
    \bottomrule
    \end{tabular}
\end{table*}




\end{document}
\endinput
%%
%% End of file `sample-manuscript.tex'.
