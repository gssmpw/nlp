%%%%%%%%%%%%%%%%%%%%%%%%%%%%%%%%%%%%%%%%%%%%%%%%%%%%%%%%%%%%%%%%%%%%%%
% Preliminaries and Setup
%%%%%%%%%%%%%%%%%%%%%%%%%%%%%%%%%%%%%%%%%%%%%%%%%%%%%%%%%%%%%%%%%%%%%%

\section{Preliminaries and Setup}

Let $\mathbf{x} \in \mathbb{R}^d$ denote the token representation in a transformer model. We consider a mapping
\[
f: \mathbb{R}^d \to \mathbb{R}^d,
\]
which decomposes additively as
\[
f(\mathbf{x}) = g(\mathbf{x}) + h(\mathbf{x}),
\]
where:
\begin{itemize}
    \item \textbf{Self-Attention Component:} The function
    \[
    g(\mathbf{x}) = W_{\mathrm{out}}\, \operatorname{Softmax}\Bigl( \mathbf{Q}(\mathbf{x})\,\mathbf{K}(\mathbf{x})^\top \Bigr)\, \mathbf{V}(\mathbf{x})
    \]
    models the self-attention mechanism. Here, $\mathbf{Q},\mathbf{K},\mathbf{V}: \mathbb{R}^d \to \mathbb{R}^d$ are smooth, learned projection maps corresponding to the query, key, and value transformations, respectively, and $W_{\mathrm{out}} \in \mathbb{R}^{d \times d}$ is a learned output weight matrix.
    
    \item \textbf{Feed-Forward Network:} The function
    \[
    h: \mathbb{R}^d \to \mathbb{R}^d
    \]
    represents the feed-forward sublayer, typically constructed from a sequence of linear transformations and elementwise nonlinearities.
\end{itemize}

We assume that both $g$ and $h$, and hence $f$, are Lipschitz continuous on the domain of interest; that is, there exists a constant $L > 0$ such that
\[
\|f(\mathbf{x}) - f(\mathbf{y})\| \le L \|\mathbf{x} - \mathbf{y}\|, \quad \forall\, \mathbf{x}, \mathbf{y} \in \mathbb{R}^d.
\]
This assumption is justified by standard regularization practices (e.g., weight normalization) and by the smooth nature of the softmax, linear, and common nonlinear activation functions.

Given an input embedding $\mathbf{x}_0 = \mathbf{E}(\text{input})$, we consider the following discrete update rule, which mirrors the transformer layer:
\[
\mathbf{x}_{n+1} = \mathbf{x}_n + \Delta t\, f(\mathbf{x}_n), \quad n = 0,1,\dots, N-1,
\]
with time-step
\[
\Delta t = \frac{1}{N}.
\]
We then define a piecewise continuous interpolation $\mathbf{x}(t)$ of the sequence $\{\mathbf{x}_n\}$ by setting $t = n\,\Delta t$, so that $\mathbf{x}(t)$ approximates the continuous evolution of the token representations.

%%%%%%%%%%%%%%%%%%%%%%%%%%%%%%%%%%%%%%%%%%%%%%%%%%%%%%%%%%%%%%%%%%%%%%
% Main Theoretical Result
%%%%%%%%%%%%%%%%%%%%%%%%%%%%%%%%%%%%%%%%%%%%%%%%%%%%%%%%%%%%%%%%%%%%%%

\section{Main Theoretical Result}

\begin{theorem}[Transformer Flow Approximation Theorem]
\label{thm:transformer_flow}
Let $f:\mathbb{R}^d \to \mathbb{R}^d$ be as defined above and assume that $f$ is Lipschitz continuous with constant $L>0$. Consider the discrete update
\[
\mathbf{x}_{n+1} = \mathbf{x}_n + \Delta t\, f(\mathbf{x}_n), \quad \mathbf{x}_0 = \mathbf{E}(\text{input}),
\]
with $\Delta t = 1/N$. Then, as $N \to \infty$, the piecewise continuous interpolation $\mathbf{x}(t)$ converges uniformly on the interval $t \in [0,1]$ to the unique solution of the ordinary differential equation (ODE)
\[
\frac{d\mathbf{x}(t)}{dt} = f\bigl(\mathbf{x}(t)\bigr) = g\bigl(\mathbf{x}(t)\bigr) + h\bigl(\mathbf{x}(t)\bigr), \quad \mathbf{x}(0)= \mathbf{E}(\text{input}).
\]
Furthermore, if $f$ additionally satisfies the one-sided Lipschitz condition
\[
\langle f(\mathbf{x}) - f(\mathbf{y}),\, \mathbf{x} - \mathbf{y} \rangle \le \lambda \|\mathbf{x} - \mathbf{y}\|^2, \quad \forall\, \mathbf{x},\mathbf{y}\in \mathbb{R}^d,
\]
for some $\lambda \in \mathbb{R}$, then the continuous dynamics are stable with respect to perturbations in the initial condition or intermediate states. In particular, if $\lambda < 0$, the system is contractive.
\end{theorem}

%%%%%%%%%%%%%%%%%%%%%%%%%%%%%%%%%%%%%%%%%%%%%%%%%%%%%%%%%%%%%%%%%%%%%%
% Proof of Theorem~\ref{thm:transformer_flow}
%%%%%%%%%%%%%%%%%%%%%%%%%%%%%%%%%%%%%%%%%%%%%%%%%%%%%%%%%%%%%%%%%%%%%%

\section{Proof of Theorem~\ref{thm:transformer_flow}}

\begin{proof}
We interpret the update
\[
\mathbf{x}_{n+1} = \mathbf{x}_n + \Delta t\, f(\mathbf{x}_n)
\]
as the forward Euler discretization of the ODE
\[
\frac{d\mathbf{x}(t)}{dt} = f\bigl(\mathbf{x}(t)\bigr)
\]
with initial condition $\mathbf{x}(0) = \mathbf{E}(\text{input})$.

\paragraph{Existence and Uniqueness:}  
Since $f$ is Lipschitz continuous with constant $L$, the Picard–Lindelöf theorem guarantees that there exists a unique solution $\mathbf{x}(t)$ on the interval $t \in [0,1]$.

\paragraph{Convergence of the Euler Method:}  
Standard numerical analysis results (see, e.g., \cite{HairerWanner}) imply that the forward Euler method converges with a global error bound:
\[
\max_{0 \le n \le N} \|\mathbf{x}(n\Delta t) - \mathbf{x}_n\| \le C\, \Delta t,
\]
for some constant $C>0$ depending on $L$ and the time horizon. As $\Delta t = 1/N \to 0$, the discrete sequence $\{\mathbf{x}_n\}$ (or its interpolation $\mathbf{x}(t)$) converges uniformly to the solution of the ODE.

\paragraph{Stability via the One-Sided Lipschitz Condition:}  
If $f$ satisfies the one-sided Lipschitz condition
\[
\langle f(\mathbf{x}) - f(\mathbf{y}), \mathbf{x} - \mathbf{y} \rangle \le \lambda \|\mathbf{x} - \mathbf{y}\|^2,
\]
then for $\lambda < 0$, the ODE is contractive; i.e., trajectories starting from nearby initial conditions converge exponentially. This property ensures that the continuous dynamics are robust to small perturbations, whether in the initial condition or arising during numerical computation.

Since the self-attention component $g$ and the feed-forward network $h$ are composed of smooth operations and are typically regularized to have bounded Lipschitz constants, the overall function $f = g + h$ inherits these properties. Hence, in the limit of infinitely many layers (i.e., $\Delta t \to 0$), the discrete transformer update converges to the continuous flow governed by
\[
\frac{d\mathbf{x}(t)}{dt} = g\bigl(\mathbf{x}(t)\bigr) + h\bigl(\mathbf{x}(t)\bigr).
\]
This completes the proof.
\end{proof}



\subsection{Iterative Update Framework and Its Relation to Transformer Dynamics}
\label{sec:iterative-update-framework}

In \cite{feinashley2025iterate}, a unified iterative update framework is introduced that underlies many reasoning and feedback convergence processes. In this framework, the state is updated according to
\begin{equation}
  s_{t+1} = (1-\alpha_t) s_t + \alpha_t\, \mathcal{T}(s_t, y_t) + \eta_t,
  \label{eq:iterative-update}
\end{equation}
where:
\begin{itemize}[leftmargin=2em]
    \item \(\alpha_t\) is an averaging parameter (for example, \(\alpha_t = \tfrac{2}{t+2}\) in the accelerated scheme),
    \item \(\mathcal{T}(s_t, y_t)\) is a general update operator that incorporates both the current state \(s_t\) and auxiliary information \(y_t\),
    \item \(\eta_t\) represents a (possibly state-dependent) perturbation term.
\end{itemize}

Under appropriate contractivity and smoothness assumptions (measured, for instance, via Bregman divergences), \cite{feinashley2025iterate} shows that the sequence \(\{s_t\}\) converges to a unique fixed point at an accelerated rate—achieving an \(O(1/t^2)\) convergence in the absence of persistent perturbations.

\subsubsection{Transformer Update as a Special Case}
\label{subsec:transformer-special-case}

In transformer architectures, token representations are updated according to
\begin{equation}
  \mathbf{x}_{n+1} = \mathbf{x}_n + \Delta t\, f(\mathbf{x}_n), \quad \text{with} \quad \Delta t = \frac{1}{N},
  \label{eq:transformer-update}
\end{equation}
where the overall mapping is given by
\[
f(\mathbf{x}) = g(\mathbf{x}) + h(\mathbf{x}),
\]
with \(g(\mathbf{x})\) representing the self-attention component and \(h(\mathbf{x})\) the feed-forward network.

This transformer update can be seen as a special instance of the general iterative update \eqref{eq:iterative-update} by making the identifications:
\begin{itemize}[leftmargin=2em]
    \item \textbf{State:} \(s_t \equiv \mathbf{x}_n\) (with \(t = n\)),
    \item \textbf{Averaging Parameter:} \(\alpha_t \equiv \Delta t = 1/N\),
    \item \textbf{Operator:} Setting
    \[
    \mathcal{T}(s_t) = s_t + \frac{1}{\alpha_t}\, f(s_t) = s_t + N\, \bigl[g(s_t) + h(s_t)\bigr],
    \]
    so that the update becomes
    \[
    s_{t+1} = s_t + \alpha_t\,\Bigl[\mathcal{T}(s_t) - s_t\Bigr],
    \]
    \item \textbf{Perturbation:} In the idealized transformer update, we assume \(\eta_t \equiv 0\).
\end{itemize}

Thus, the transformer update \eqref{eq:transformer-update} is equivalent to a forward Euler discretization of the continuous dynamics underlying the iterative update framework in \cite{feinashley2025iterate}. 

Moreover, as shown in \cite{feinashley2025iterate}, if one instead chooses an adaptive averaging parameter—say, by setting
\[
\alpha_t = \frac{2}{t+2},
\]
the iterative update converges to the fixed point at an accelerated rate of \(O(1/t^2)\) (when measured in an appropriate error metric such as a Bregman divergence). This accelerated convergence result provides theoretical support for the benefits of incorporating iterative feedback mechanisms in transformer architectures.




%%%%%%%%%%%%%%%%%%%%%%%%%%%%%%%%%%%%%%%%%%%%%%%%%%%%%%%%%%%%%%%%%%%%%%
% Discussion
%%%%%%%%%%%%%%%%%%%%%%%%%%%%%%%%%%%%%%%%%%%%%%%%%%%%%%%%%%%%%%%%%%%%%%

%%%%%%%%%%%%%%%%%%%%%%%%%%%%%%%%%%%%%%%%%%%%%%%%%%%%%%%%%%%%%%%%%%%%%%
% Discussion
%%%%%%%%%%%%%%%%%%%%%%%%%%%%%%%%%%%%%%%%%%%%%%%%%%%%%%%%%%%%%%%%%%%%%%

\section{Discussion}

The results presented in this work offer several important insights into transformer architectures by linking their layer-wise updates to continuous dynamical systems and a broader class of iterative reasoning processes. We summarize the key implications below:

\begin{itemize}[leftmargin=2em]
    \item \textbf{Continuous-Time Dynamics:} By interpreting the transformer update as a forward Euler discretization of the ODE
    \[
    \frac{d\mathbf{x}(t)}{dt} = f\bigl(\mathbf{x}(t)\bigr) = g\bigl(\mathbf{x}(t)\bigr) + h\bigl(\mathbf{x}(t)\bigr),
    \]
    we obtain a continuous-time perspective on how token representations evolve across layers. This view enables the application of classical numerical analysis tools to assess convergence and stability, thereby deepening our understanding of transformer behavior.

    \item \textbf{Stability and Robustness:} The Lipschitz continuity of \(f\) and the additional one-sided Lipschitz condition guarantee that the continuous dynamics are well-posed and, when \(\lambda < 0\), contractive. This contractivity implies that perturbations—whether due to initialization, noise, or other sources—are attenuated over time. Such robustness is consistent with the empirical performance of transformers in handling noisy or variable inputs.

    \item \textbf{Unified Iterative Framework:} By showing that the transformer update is a special case of the general iterative update
    \[
    s_{t+1} = (1-\alpha_t) s_t + \alpha_t\, \mathcal{T}(s_t, y_t) + \eta_t,
    \]
    with the identification \(s_t \equiv \mathbf{x}_n\), \(\alpha_t \equiv \Delta t\), and
    \[
    \mathcal{T}(s_t) = s_t + \frac{1}{\alpha_t}f(s_t),
    \]
    we connect transformer dynamics to a broader family of iterative reasoning methods as discussed in \cite{feinashley2025iterate}. This connection not only unifies several classical methods (such as mirror descent and dynamic programming) but also provides a theoretical foundation for the iterative, feedback-driven reasoning observed in modern deep learning systems.

    \item \textbf{Accelerated Convergence:} An important consequence of the iterative framework is the possibility of accelerated convergence. As shown in \cite{feinashley2025iterate}, by choosing an adaptive averaging parameter—specifically, \(\alpha_t = \frac{2}{t+2}\)—the iterative update converges to the fixed point at a rate of \(O(1/t^2)\) (when measured in an appropriate metric such as a Bregman divergence). This result suggests that, beyond the standard \(O(1/N)\) convergence of the basic forward Euler discretization, transformer updates can benefit from acceleration techniques that improve both convergence speed and robustness.

    \item \textbf{Architectural and Algorithmic Implications:} The continuous and iterative perspectives invite new strategies for transformer design. For example, one might explore alternative discretization methods (e.g., higher-order Runge–Kutta schemes) or incorporate adaptive step sizes and feedback mechanisms directly into the network architecture. Such innovations could lead to transformers that not only converge more rapidly but also exhibit improved performance and stability in practice.
\end{itemize}

In summary, by bridging transformer updates with continuous dynamical systems and the unified iterative framework of \cite{feinashley2025iterate}, our analysis provides both theoretical insights and practical guidance for enhancing transformer architectures. The accelerated convergence result, in particular, underscores the potential benefits of integrating iterative feedback mechanisms into the design of deep learning models.




%%%%%%%%%%%%%%%%%%%%%%%%%%%%%%%%%%%%%%%%%%%%%%%%%%%%%%%%%%%%%%%%%%%%%%
% Additional Theoretical Insights: Discrete Stability and Error Propagation
%%%%%%%%%%%%%%%%%%%%%%%%%%%%%%%%%%%%%%%%%%%%%%%%%%%%%%%%%%%%%%%%%%%%%%

\section{Additional Theoretical Insights: Discrete Stability and Error Propagation}

In this section, we analyze the propagation of errors and perturbations in the discrete transformer update. In particular, we show that if the mapping $f$, which combines the self-attention and feed-forward components, satisfies a one-sided Lipschitz condition with a negative constant, then any perturbations in the input or intermediate representations decay exponentially over the layers.

\subsection{Discrete Contractivity of Transformer Updates}

\begin{theorem}[Discrete Contractivity]
\label{thm:discrete_contractivity}
Assume that the mapping $f:\mathbb{R}^d \to \mathbb{R}^d$, decomposed as
\[
f(\mathbf{x}) = g(\mathbf{x}) + h(\mathbf{x}),
\]
is Lipschitz continuous with constant $L > 0$, and further satisfies the one-sided Lipschitz condition
\[
\langle f(\mathbf{x}) - f(\mathbf{y}),\, \mathbf{x} - \mathbf{y} \rangle \le \lambda \|\mathbf{x} - \mathbf{y}\|^2, \quad \forall\, \mathbf{x},\mathbf{y}\in \mathbb{R}^d,
\]
for some $\lambda \in \mathbb{R}$. Then, for the discrete update
\[
\mathbf{x}_{n+1} = \mathbf{x}_n + \Delta t\, f(\mathbf{x}_n), \quad n = 0,1,\dots,N-1,
\]
and for any two sequences $\{\mathbf{x}_n\}$ and $\{\mathbf{y}_n\}$ with initial conditions $\mathbf{x}_0,\mathbf{y}_0\in\mathbb{R}^d$, the following error bound holds:
\[
\|\mathbf{x}_n - \mathbf{y}_n\| \le e^{\lambda\, n\Delta t}\, \|\mathbf{x}_0 - \mathbf{y}_0\|, \quad \text{for } n=0,1,\dots,N.
\]
In particular, if $\lambda < 0$, the discrete dynamics are contractive, and perturbations decay exponentially.
\end{theorem}

\begin{proof}
Define the error sequence $\mathbf{z}_n = \mathbf{x}_n - \mathbf{y}_n$. Using the update rule for each sequence, we have:
\[
\mathbf{z}_{n+1} = \mathbf{x}_{n+1} - \mathbf{y}_{n+1} = \mathbf{z}_n + \Delta t\, \bigl( f(\mathbf{x}_n) - f(\mathbf{y}_n) \bigr).
\]
Taking the squared Euclidean norm yields:
\[
\|\mathbf{z}_{n+1}\|^2 = \|\mathbf{z}_n\|^2 + 2\Delta t\, \langle f(\mathbf{x}_n) - f(\mathbf{y}_n),\, \mathbf{z}_n \rangle + (\Delta t)^2 \| f(\mathbf{x}_n) - f(\mathbf{y}_n) \|^2.
\]
Using the one-sided Lipschitz condition,
\[
\langle f(\mathbf{x}_n) - f(\mathbf{y}_n),\, \mathbf{z}_n \rangle \le \lambda\, \|\mathbf{z}_n\|^2,
\]
and the Lipschitz continuity of $f$,
\[
\|f(\mathbf{x}_n) - f(\mathbf{y}_n)\| \le L\, \|\mathbf{z}_n\|,
\]
we obtain:
\[
\|\mathbf{z}_{n+1}\|^2 \le \|\mathbf{z}_n\|^2 \Bigl( 1 + 2\lambda\, \Delta t + L^2\,(\Delta t)^2 \Bigr).
\]
For sufficiently small $\Delta t$, we can bound the factor by an exponential:
\[
1 + 2\lambda\, \Delta t + L^2\,(\Delta t)^2 \le e^{2\lambda\, \Delta t}.
\]
Thus,
\[
\|\mathbf{z}_{n+1}\|^2 \le \|\mathbf{z}_n\|^2\, e^{2\lambda\, \Delta t}.
\]
By induction, we conclude that:
\[
\|\mathbf{z}_n\|^2 \le \|\mathbf{z}_0\|^2\, e^{2\lambda\, n\Delta t},
\]
and taking square roots yields:
\[
\|\mathbf{x}_n - \mathbf{y}_n\| \le e^{\lambda\, n\Delta t}\, \|\mathbf{x}_0 - \mathbf{y}_0\|.
\]
\end{proof}

\subsection{Implications for Robustness and Error Attenuation}

Theorem~\ref{thm:discrete_contractivity} implies that if $\lambda < 0$, any discrepancies between two trajectories---whether due to slight variations in the input or to perturbations introduced during computation---decay exponentially as the network depth increases. In the context of deep transformer architectures, this means that the cumulative effect of such perturbations is inherently controlled, thereby enhancing the overall robustness of the model.

Moreover, this discrete contractivity property complements the continuous-time analysis of Theorem~\ref{thm:transformer_flow}. Together, they reinforce the interpretation of the transformer layer as a stable numerical integrator for an underlying dynamical system. This insight not only deepens our theoretical understanding but also suggests practical strategies for improving model stability and designing new architectures with controlled error propagation.
