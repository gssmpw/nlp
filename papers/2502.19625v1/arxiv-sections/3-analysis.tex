\section{Treatment Non-adherence Analysis}
\label{sec:analysis}
We begin by presenting the results of the identified hypertension treatment non-adherence with statistical testing in Section~\ref{sec:treatment_adherence_analysis}. In Section~\ref{sec:treatment_adherence_clustering}, we apply topic modeling to the extracted clinical notes, uncovering underlying reasons contributing to treatment non-adherence.

\subsection{Factors associated with treatment non-adherence}
\label{sec:treatment_adherence_analysis}
\begin{table}[t]
\centering
\caption{Demographic and clinical characteristics of patients in the study and logistic regression results. Age, one-hot encoding for Black race, and the number of comorbidities are found to be statistically significant factors of treatment non-adherence.}
\label{tab:medication_adherence}
\resizebox{\textwidth}{!}{%
\begin{tabular}{@{}lcccccc@{}}
\toprule
\textbf{Factors}  & \textbf{Total} & \textbf{Non-adherent} & \textbf{Adherent} & \multicolumn{3}{c}{\textbf{Bivariate Analysis}} \\ 
\cmidrule(l){5-7}
& $n=3623$ & $n=786 (21.7\%)$ & $n=2837 (78.3\%)$ & \textbf{Unadjusted OR} & \textbf{95\% CI} & \textbf{$p$-value} \\ \midrule
\textbf{Demographics}  & & & & & & \\ 
\hspace{1em} Sex & & & & & & \\
\hspace{2em} Female & 2143 & 473 & 1670 & Ref. & Ref. &  \\
\hspace{2em} Male & 1480 & 313 & 1167 & 0.95 & (0.81 to 1.11) & 0.508\\ 
\hspace{1em} Age, mean ± SD & 62.03 ± 14.2 & 61.09 ± 14.7 & 62.29 ± 14.1 & 0.94 & (0.89 to 1.00) & \textbf{0.036} \\
\hspace{1em} Race & & & & & & \\
\hspace{2em} Asian & 1125 & 244 & 881 & Ref. & Ref. &  \\
\hspace{2em} Black & 419 & 114 & 305 & 1.35 & (1.04 to 1.75) &  \textbf{0.023}\\ 
\hspace{2em} White & 1646 & 327 & 1319 & 0.90 & (0.74 to 1.08) &  0.244\\ 
\hspace{2em} Other & 433 & 101 & 332 & 1.10 & (0.84 to 1.43) & 0.486\\
\hspace{1em} Marital Status & & & & & & \\
\hspace{2em} Divorced & 329 & 69 & 260 & Ref. & Ref. &  \\
\hspace{2em} Married & 1861 & 370 & 1491 & 0.94 & (0.70 to 1.25) &  0.649\\ 
\hspace{2em} Single & 878 & 220 & 658 & 1.26 & (0.93 to 1.71) &  0.139\\ 
\hspace{2em} Widowed & 358 & 77 & 281 & 1.03 & (0.72 to 1.49) & 0.864\\
\hspace{2em} Other & 197 & 50 & 147 & 1.28 & (0.85 to 1.94) & 0.243\\
\textbf{Clinical Factors}  & & & & & & \\ 
\hspace{1em} Time between visits (days), mean ± SD & 116.89 ± 83.4 & 112.04 ± 85.5 & 118.23 ± 82.8 & 1.00 & (1.00 to 1.00) & 0.066 \\
\hspace{1em} Total number of comorbidities (ECI), mean ± SD & 3.13 ± 2.4 & 2.98 ± 2.2 & 3.17 ± 2.4 & 0.96 & (0.93 to 1.00) & \textbf{0.045}\\
\hspace{1em} Duration of hypertension (years), mean ± SD & 5.94 ± 6.5 & 5.75 ± 6.6 & 6.00 ± 6.5 & 0.99 & (0.98 to 1.01) & 0.341\\
\hspace{1em} Number of primary care visits one year prior the visit, mean ± SD & 15.75 ± 11.6 & 15.71 ± 10.8 & 15.76 ± 11.8 & 1.00 & (0.99 to 1.01) & 0.925\\
\bottomrule
\end{tabular}
}
\end{table}

\begin{table}[h]
\centering
\caption{Multivariate logistic regression results of factors associated with treatment non-adherence. The number of comorbidities and one-hot encoding for Black race remain statistically significant.}
\label{tab:medication_adherence_multivariable}
\begin{tabular}{@{}l@{\hspace{-12pt}}c@{}c@{\hspace{5pt}}c@{}}
\toprule
\textbf{Factors}  &  \multicolumn{3}{c} \textbf{Multivariate Logistic Regression}\\ 
\cmidrule(l){2-4}
& \textbf{Adjusted OR} & \textbf{95\% CI} & \textbf{$p$-value} \\ 
\midrule
Age (per 10-year increment) & 0.97 & (0.91 to 1.02) & 0.242 \\
Number of comorbidities & 0.96 & (0.93 to 1.00) & \textbf{0.03}\\
 Race & & &\\
\hspace{1em} Asia & Ref. & Ref. &  \\
\hspace{1em} Black & 1.38 & (1.06 to 1.80) &  \textbf{0.016}\\ 
\hspace{1em} White & 0.90 & (0.75 to 1.08) &  0.266\\ 
\hspace{1em} Other & 1.10 & (0.84 to 1.43) & 0.5\\
\bottomrule
\end{tabular}
\end{table}
To meet the independence assumption of the statistical tests, we keep only the most recent pair of visits for adherent patients and the most recent non-adherent pair for non-adherent patients when multiple pairs are available for the same patient.

Among 3,623 patients, 786 (21.7\%) are identified as non-adherent to their treatment plans. Of these 786 patients, 506 (64.4\%) miss their prescribed treatments, 237 (30.2\%) take a different dosage than instructed, 53 (6.7\%) use a different medication, and 62 (7.9\%) take their medication at a time other than instructed. Note that a single patient may exhibit multiple types of non-adherence above.

To identify factors associated with nonadherence to treatment, we begin by performing unadjusted logistic regression, with the results presented in Table~\ref{tab:medication_adherence}. 
We find three factors that are statistically significant ($p<0.05$): age ($p=0.036$), one-hot encoding for Black race ($p=0.023$), and the number of comorbidities ($p=0.045$).

Our findings indicate that younger patients are less likely to adhere to treatment, aligning with previous research \citep{boratas2018evaluation} that suggests adherence improves with age as patients become more accustomed to managing their diagnoses. Additionally, we find that Black patients exhibit higher rates of non-adherence. In the U.S., Black individuals have a higher prevalence of uncontrolled hypertension than White individuals \citep{aggarwal2021racial}, and our finding further highlighting the need for greater attention to prevent further exacerbation of racial disparities in hypertension control. Lastly, a lower number of comorbidities is associated with a higher rate of non-adherence, possibly because patients with fewer health conditions may perceive their treatment as less essential.

To avoid potential confounding, we then adjust for the identified significant factors in the multivariate logistic regression model. Results are presented in Table~\ref{tab:medication_adherence_multivariable}, where we see that the number of comorbidities ($p=0.03$) and race as Black ($p=0.016$) still remain statistically significant.

\subsection{Uncovering reasons for treatment non-adherence}
\label{sec:treatment_adherence_clustering}

\begin{figure}[h]
    \centering
    \includegraphics[width=0.5\linewidth]{figures/pie_chart.png}
    \caption{BERT topic modeling results for treatment non-adherence reasons. Side effects are the dominant reason for non-adherence, and 17.7\% of reasons are due to forgetfulness, while others are related to not picking up the medication, needing a refill, or losing it.}
    \label{fig:reason}
\end{figure}

We employ Bidirectional Encoder Representations from Transformers (BERT)-based topic modeling \citep{grootendorst2022bertopicneuraltopicmodeling} on the extracted non-adherent clinical notes to uncover reasons. This utilizes a BERT architecture \citep{devlin2019bertpretrainingdeepbidirectional} to generate embeddings from the extracted notes and applies Uniform Manifold Approximation and Projection (UMAP) \citep{mcinnes2020umapuniformmanifoldapproximation} to reduce the dimensionality of the embeddings. The reduced embeddings are subsequently clustered using HDBSCAN \citep{campello2013density}, and key terms for each cluster are identified using class-based Term Frequency-Inverse Document Frequency (c-TF-IDF).

Across all non-adherent instances, 164 clinical notes fall into clusters. The full clustering with key terms found is in Appendix~\ref{apd:topic}, and we summarize the topics in Figure~\ref{fig:reason}. 59.8\% of the extracted reasons for non-adherence are related to side effects, with most patients experiencing dizziness and headaches after taking the prescribed medication, while a few report coughing. 17.7\% of the extracted reasons indicate non-adherence due to forgetfulness. Additionally, 11.6\% are due to needing a refill or losing their medication and 11\% of the reasons involve patients not picking up their medication.
