\section{Study Design}
\label{sec:setup}

\subsection{Hypertension cohort selection}
We identified 15,002 patients with primary hypertension and extracted their primary care visits occurring on or after January 1st, 2019 following their initial hypertension diagnosis. To assess treatment adherence, consecutive visits for each patient were grouped into pairs. We focused on pairs where a hypertension medical prescription was provided during the first visit, and verified adherence at the second visit by extracting the associated clinical notes. 

Our analysis focuses on ten commonly prescribed hypertension medications: amlodipine, losartan, lisinopril, benazepril, carvedilol, hydralazine, hydrochlorothiazide, clonidine, spironolactone, and metoprolol \citep{heartTypesBlood}. Therefore, we excluded pairs in which the first visit lacked a medication record on this list, as well as pairs with missing or invalid notes during the second visit. We further focus on pairs where the interval between visits is between one month and one year. Lastly, we filtered out patients with unknown demographic information for the purpose of analysis. This resulted in a final cohort of 3,623 patients with 5,952 visit pairs. The cohort selection process is summarized in Appendix~\ref{apd:cohort}.

Demographic information, including sex, age, race, and marital status, was extracted from patient records. Four clinical factors were further derived from the EHR, many of which have been shown to be associated with hypertension non-adherence  \citep{boratas2018evaluation, algabbani2020treatment}. These factors include the duration between the two visits in the pair, the duration of hypertension, the number of primary care visits and the number of comorbidities. We quantified comorbidities using the Charlson Comorbidity Index (CCI) \citep{charlson1987new} and the Elixhauser Comorbidity Index (ECI) \citep{elixhauser1998comorbidity}, which condensed diagnoses into 17 and 31 well-defined comorbidity categories respectively. The demographic and clinical characteristics of the selected cohort are summarized in Table~\ref{tab:medication_adherence}. We detail the comorbidity categories along with other features used in the study in Appendix~\ref{apd:feature}.


\subsection{LLM configuration and prompt engineering}
We used the GPT-4o model \citep{openai2024gpt4ocard} (version 2024-05-13) via the HIPAA-compliant Microsoft Azure API, with the temperature set to 0 and all other parameters left at default. For each pair of visits, we provided the prescription record from the first visit and the clinical notes from the second visit to the model to assess adherence to the prescribed medication.

The model was prompted to identify instances of non-adherence, the type of non-adherence, and extract relevant sections from the notes. We used a zero-shot approach without additional training data or fine-tuning. We also implemented a second round of prompt validation by feeding the model's initial output back into the model, asking it to double-check its response. This additional step significantly reduced hallucinations. The prompt used in the study is provided in Appendix~\ref{apd:prompt}.

The cost for running all GPT-4o evaluations, including prompt development and inference was \$184.77, based on a cost of \$0.005 per 1,000 input tokens and \$0.015 per 1,000 output tokens.


\subsection{Physician validation of LLM detection}

To ensure the reliability of the LLM detection, we randomly selected 50 pairs labeled by the model as non-adherence and 50 pairs labeled as adherence for physician validation to assess accuracy. The gold standard was established through physician annotations conducted independently of the model's predictions. Overall, the model achieved an accuracy of 92\%, with four instances of physician-labeled non-adherence not detected and four adherent instances mislabeled as non-adherence(92\% precision and recall).

We further analyze discrepancies between the model and physician annotations, noting that some mismatches arise from ambiguous notes. For example, cases where patients restarted medication after hospitalization were marked as non-adherent by the LLM, since treatment was paused during hospitalization. Whereas physicians labeled them as adherent, considering the pause as a temporary interruption rather than true non-adherence.
