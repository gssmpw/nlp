\section{Discussion}
\label{sec:discussion}
Treatment non-adherence is a crucial factor in building treatment models but is often overlooked in practice. By leveraging LLMs with clinical notes, we identify non-adherent encounters with hypertension patients and further demonstrate how treatment non-adherence biases can degrade downstream model performance while exacerbating fairness gaps.

While our study focuses on hypertension, the same pipeline can be applied to other disease areas to analyze treatment non-adherence patterns. Beyond adherence, future work can also utilize LLMs to extract insights from clinical notes on other factors such as medication tolerance, side effects, social determinants of health, and patient-provider communications~\citep{guevara2024large,robitschek2024large,zink2024access,antoniak2024nlp,miao2024identifying}. Our results show that removing non-adherent data from the training set improves both model performance and fairness. Instead of excluding non-adherent data entirely, future research could also explore strategies to better integrate non-adherent data into modeling or develop models that are more robust to treatment adherence biases. 

Although leveraging LLMs with clinical notes enables large-scale analysis of treatment non-adherence, our study holds keys limitations. First, our work relies on the premise that non-adherence is explicitly documented in the notes. Furthermore, while physician validation confirms that the LLM's output is largely accurate, the use of ML models such as LLMs in shifting and censored patient populations may yield changing performances such that the automated extraction should be scrutinized~\citep{pollard2019turning,yuan2023revisiting,finlayson2021clinician,chen2022clustering}. In conclusion, our work demonstrates the impact of treatment non-adherence bias in predictive modeling and causal inference through a real-world study on hypertension medications. We hope this study raises awareness of treatment non-adherence bias for future research on clinical machine learning.
