\section{Introduction}
\label{sec:intro}
Treatment non-adherence is a pervasive and persistent challenge in healthcare. Researchers estimate that poor medication adherence leads to 125,000 preventable deaths annually in the U.S. and contributes to \$100-\$300 billion in avoidable healthcare costs \citep{benjamin2012medication}. This issue is particularly prevalent among patients with chronic conditions such as hypertension, with 40-50\% failing to take their medications as prescribed \citep{kleinsinger2018unmet, algabbani2020treatment}. While researchers have extensively documented this problem through surveys and interviews \citep{boratas2018evaluation, fernandez2019adherence, algabbani2020treatment, najjuma2020adherence, schober2021high}, the studies---and ultimately understanding of treatment non-adherence---remain limited by small sample sizes and self-reporting bias \citep{adams1999evidence, stirratt2015self}. Physical solutions to monitor and encourage adherence such as electronic pill caps have shown promise in controlled settings but remain impractical for large-scale deployment due to high costs and implementation challenges~\citep{parker2007adherence,mauro2019effect}.
 
These measurement challenges take on new urgency as healthcare systems increasingly rely on machine learning (ML) models trained on electronic health records (EHRs) to guide treatment decisions~\citep{komorowski2018artificial, brugnara2020multimodal, zheng2021personalized, mroz2024predicting, yi2024development,shen2024data,chen2022clustering}. These machine learning models learn from historical patient data, which assume that prescribed treatments were actually taken. However, this introduces an implicit bias---models trained on non-adherent patients learn patterns that misrepresent true treatment effects. This implicit bias may degrade model performance and disproportionately impact underserved populations, who often face greater barriers to treatment adherence~\citep{bosworth2006racial, schober2021high}.

Recent advances in large language models (LLMs) have shown that LLMs can advance medical understanding by accurately extracting information from EHRs \citep{agrawal2022large, goel2023llmsaccelerateannotationmedical}. Instead of relying on self-reported treatment adherence from questionnaires and interviews, LLMs could serve as a powerful tool for identifying treatment non-adherence directly from EHRs. By analyzing rich but unstructured clinical notes, LLMs can detect documented instances of missed medications, unfilled prescriptions, and patient-reported barriers to adherence, enabling systematic assessment of treatment non-adherence across large patient populations.

In this study, we examine hypertension treatment non-adherence using EHR data from a large academic hospital by leveraging an LLM to analyze clinical notes, and further investigate its impact on causal inference and ML model performance (Figure~\ref{fig:diagram}). With a cohort of 3,623 patients, we identify 786 (21.7\%) cases of non-adherence and extract demographic and clinical factors that are statistically significant. Additionally, we apply topic modeling to clinical notes revealing underlying reasons for non-adherence.

To assess the effect of treatment non-adherence bias on downstream model performance, we perform causal inference and build predictive models using EHR with treatment records.  Our results show that ignoring treatment non-adherence bias could lead to reversed conclusions in treatment effect estimation, significantly degrade the performance of predictive models up to 5\%, and lead to unfair predictions. Furthermore, we highlight the importance of addressing treatment non-adherence bias by showing simply removing patient records with non-adherence, though reducing the size of the training dataset, could improve model performance and lead to fairer predictions.
\\
\newline
The contributions of this work include:
\begin{enumerate}
\item Conducting a large-scale study on treatment non-adherence in hypertension and identifying statistically significant factors associated with non-adherence.

\item Comparing LLM identification against physician annotations, LLMs perform well with 92\% accuracy, precision and recall.


\item Identifying patient-reported reasons for treatment non-adherence including side effects, forgetfulness, difficulties obtaining refills, etc.

\item Demonstrating the harmful impact of ignoring treatment non-adherence bias on causal inference and predictive modeling, leading to poorer performance and exacerbating racial disparities.

\end{enumerate}
