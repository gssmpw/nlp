\documentclass{article}

\usepackage{graphicx}%
\usepackage{multirow}%
\usepackage{makecell}
\usepackage{amsmath,amssymb,amsfonts}%
\usepackage{amsthm}%
\usepackage{mathrsfs}%
\usepackage{xcolor}%
\usepackage{textcomp}%
\usepackage{manyfoot}%
\usepackage{booktabs}%
\usepackage{algorithm}%
\usepackage{algorithmicx}%
\usepackage{algpseudocode}%
\usepackage{listings}%
\usepackage{array}
\usepackage{caption}
\usepackage{pdflscape}
\usepackage{natbib}
\usepackage{subcaption}
\usepackage{threeparttable}
\usepackage{hyperref}
\usepackage[draft]{fixme}

\usepackage{mathtools}
\usepackage{lineno}

\usepackage{graphicx} %
\usepackage{booktabs}
\usepackage[margin=1in]{geometry}

\usepackage{bbm}
\usepackage[T1]{fontenc}

\newcommand{\theHalgorithm}{\arabic{algorithm}}


\fxsetup{inline,nomargin,theme=color}


\theoremstyle{plain}
\newtheorem{theorem}{Theorem}[section]
\newtheorem{proposition}[theorem]{Proposition}
\newtheorem{lemma}[theorem]{Lemma}
\newtheorem{corollary}[theorem]{Corollary}
\theoremstyle{definition}
\newtheorem{definition}[theorem]{Definition}
\newtheorem{assumption}[theorem]{Assumption}
\theoremstyle{remark}
\newtheorem{remark}[theorem]{Remark}


\definecolor{todo-color}{HTML}{ff3232}
\newcommand{\todo}[1]{{\color{todo-color}[#1]}}
\newcommand\todocite{{\color{red}{CITE}}\xspace}


\newcommand{\fix}{\marginpar{FIX}}
\newcommand{\new}{\marginpar{NEW}}


\newcommand{\blue}[1]{\textcolor{blue}{#1}}
\newcommand{\red}[1]{\textcolor{red}{#1}}
\newcommand{\tba}{\textcolor{red}{To Be Added }}

\title{Treatment Non-Adherence Bias in Clinical Machine Learning: A Real-World Study on Hypertension Medication}
\author{Zhongyuan Liang$^{1,2}$, Arvind Suresh$^{2}$, and Irene Y. Chen$^{1,2}$}
\date{$^1$University of California, Berkeley, $^2$University of California, San Francisco}

\begin{document}

\maketitle

\begin{abstract}
Machine learning systems trained on electronic health records (EHRs) increasingly guide treatment decisions, but their reliability depends on the critical assumption that patients follow the prescribed treatments recorded in EHRs. Using EHR data from 3,623 hypertension patients, we investigate how treatment non-adherence introduces implicit bias that can fundamentally distort both causal inference and predictive modeling. By extracting patient adherence information from clinical notes using a large language model, we identify 786 patients (21.7\%) with medication non-adherence. We further uncover key demographic and clinical factors associated with non-adherence, as well as patient-reported reasons including side effects and difficulties obtaining refills. Our findings demonstrate that this implicit bias can not only reverse estimated treatment effects, but also degrade model performance by up to 5\% while disproportionately affecting vulnerable populations by exacerbating disparities in decision outcomes and model error rates. This highlights the importance of accounting for treatment non-adherence in developing responsible and equitable clinical machine learning systems.
\end{abstract}

\section{Introduction}
% \textcolor{red}{Di: 1) Again, you should not put multiple places in bold, 2) you may need to add five more recently published related papers in the introduction}

Nonlinear dynamical systems are omnipresent across scientific disciplines, and understanding causal relationships in these systems is crucial for unveiling the underlying mechanisms that drive system behaviors. Classic causal inference methods, such as Granger Causality (GC)~\citep{granger1969investigating} and other functional causal models (FCMs), including the Additive Noise Model (ANM)~\citep{hoyer2008nonlinear, liu2024causal} and the Post Nonlinear Model (PNL)~\citep{zhang2015estimation, keropyan2023rank}, struggle with these systems due to their assumption of a predictive relationship from cause to effect, which does not hold in the presence of complex dynamics like coupling and chaos.

% Convergent Cross Mapping (CCM)~\citep{sugihara2012detecting} was proposed as a model-free approach for bivariate causal inference in nonlinear dynamical systems. CCM addresses these limitations by leveraging state-space manifold reconstructions and cross mapping between reconstructed embeddings. Since its introduction, CCM has inspired further developments, including Partial Cross Mapping (PCM)~\citep{leng2020partial}, which aims to distinguish indirect from direct causalities in three-variable systems. However, PCM is limited to mapping operations between univariate delay embeddings, which can be less efficient when dealing with higher-dimensional systems with multiple interconnected variables. \textcolor{red}{Di: are there any related works try to tackle this problem?}

Convergent Cross Mapping (CCM)~\citep{sugihara2012detecting, barraquand2021inferring} was proposed as a model-free approach for bivariate causal inference in coupled dynamical systems. CCM addresses these limitations by leveraging state-space manifold reconstructions and cross mapping between reconstructed embeddings. Since its introduction, CCM has inspired further developments, including Partial Cross Mapping (PCM)~\citep{leng2020partial}, which aims to distinguish indirect from direct causalities in three-variable systems. However, PCM is limited to mapping operations between univariate delay embeddings, which can be less effective or even fail when dealing with complex systems with multiple interconnected variables~\citep{chen2022causation}.

To overcome this limitation, we propose \textbf{multiPCM}, an extension of PCM to the multivariate setting that allows for more effective causal inference by utilizing cross mapping via multivariate embeddings. We further integrate multiPCM with bivariate CCM into a two-phase framework named \textbf{MXMap} (Multivariate Cross Mapping for Causal Discovery). The proposed framework is designed for multivariate causal discovery, and is not only confined to assumptions of directed acyclic graphs (DAGs) but can also handle cycles. In the first phase, bivariate CCM generates an initial, potentially dense causal graph; In the second phase, multiPCM prunes indirect connections, refining the graph to isolate direct causal relationships. We systematically evaluate multiPCM and MXMap on benchmark datasets, including both simulated ecosystems and real-world meteorological data.

The contributions of this work are summarized as follows:

\begin{itemize}
    \item \textbf{Extension of PCM to multivariate settings}: We introduce multiPCM, which extends PCM to utilize multivariate delay embeddings for more robust causal inference in high-dimensional dynamical systems.
    \item \textbf{Two-phase causal discovery framework}: We propose MXMap, combining bivariate CCM with multiPCM to generate and refine causal graphs in nonlinear dynamical systems, which can also detect cycles.
    \item \textbf{Comprehensive evaluation on nonlinear dynamical systems}: We validate multiPCM and MXMap on simulated and real-world datasets. MXMap is compared against multiple baseline methods — including tsFCI, VAR-LiNGAM, PCMCI, Granger Causality, DYNOTEARS, SLARAC — demonstrating advantages in accuracy and refinement capabilities.
\end{itemize}

\begin{figure}[t]
    \centering
    \includegraphics[width=\textwidth]{figures/Figure1.pdf}
\caption{Illustration of cohort selection, LLM non-adherence extraction, and non-adherence analysis. (a) We select 3,623 hypertension patients and pair their visits, with hypertension medication prescribed at the first visit and clinical notes extracted from the second. (b) These notes are then processed by an LLM to identify treatment non-adherence, with outputs validated through clinician annotations. (c) We further perform topic modeling to uncover reasons for non-adherence and assess the harmful impact of ignoring this bias on predictive modeling performance and treatment effect estimation.}
    \label{fig:diagram}
\end{figure}


\section{Related Work}
\label{sec:related_work}
\subsection{Treatment adherence analysis in hypertension}
Multiple studies have investigated treatment adherence among patients with hypertension \citep{boratas2018evaluation, uchmanowicz2018factors, algabbani2020treatment, najjuma2020adherence, schober2021high}. These studies are mainly cross-sectional, with a cohort of admitted patients collected at a fixed time point, and treatment adherence is typically measured through questionnaires and interviews. For instance, \citet{algabbani2020treatment} conducted a study in Saudi Arabia involving 306 hypertensive outpatients, finding that only 42.2\% of participants adhered to their antihypertensive medications. \citet{boratas2018evaluation} conducted a similar study of 147 hypertensive patients, identifying factors such as age and duration of hypertension to be significant. However, due to their reliance on questionnaire and interview data, they often have small sample sizes (e.g., less than 300 patients) and self-reporting bias \citep{adams1999evidence, stirratt2015self}, which limits their representativeness and can even lead to contradictory conclusions. In contrast, our work conducts the first large-scale analysis utilizing EHR, with a significantly larger sample size of 3,623 patients.







\subsection{Machine learning and treatment adherence}
Machine learning has been used to identify individual risk factors associated with treatment non-adherence \citep{koesmahargyo2020accuracy, gichuhi2023machine, burgess2023using}. \citet{gichuhi2023machine} developed ML algorithms and found SVM achieved 91.28\% accuracy in predicting tuberculosis treatment non-adherence. Instead of predicting treatment adherence, our work focuses on analyzing the impact of treatment non-adherence bias on downstream model performance. Other studies have applied natural language processing (NLP) to analyze surveys to better understand treatment non-adherence \citep{anglin2021natural, lin2022extraction, chan2024patient}. \citet{chan2024patient} applied NLP to free-text responses from questionnaires completed by type 2 diabetes patients, identifying key reasons for non-adherence. Unlike questionnaires, our work leverages treatment adherence information extracted from clinical notes using LLMs. Lastly, \citet{zhong2022use} applied ML while accounting for adherence information when analyzing treatment effects in a randomized controlled trial. To our knowledge, our study is the first to leverage LLMs for extracting treatment adherence information from clinical notes and evaluating its impact on downstream causal inference and predictive model performance.







\section{Study Design}
\label{sec:setup}

\subsection{Hypertension cohort selection}
We identified 15,002 patients with primary hypertension and extracted their primary care visits occurring on or after January 1st, 2019 following their initial hypertension diagnosis. To assess treatment adherence, consecutive visits for each patient were grouped into pairs. We focused on pairs where a hypertension medical prescription was provided during the first visit, and verified adherence at the second visit by extracting the associated clinical notes. 

Our analysis focuses on ten commonly prescribed hypertension medications: amlodipine, losartan, lisinopril, benazepril, carvedilol, hydralazine, hydrochlorothiazide, clonidine, spironolactone, and metoprolol \citep{heartTypesBlood}. Therefore, we excluded pairs in which the first visit lacked a medication record on this list, as well as pairs with missing or invalid notes during the second visit. We further focus on pairs where the interval between visits is between one month and one year. Lastly, we filtered out patients with unknown demographic information for the purpose of analysis. This resulted in a final cohort of 3,623 patients with 5,952 visit pairs. The cohort selection process is summarized in Appendix~\ref{apd:cohort}.

Demographic information, including sex, age, race, and marital status, was extracted from patient records. Four clinical factors were further derived from the EHR, many of which have been shown to be associated with hypertension non-adherence  \citep{boratas2018evaluation, algabbani2020treatment}. These factors include the duration between the two visits in the pair, the duration of hypertension, the number of primary care visits and the number of comorbidities. We quantified comorbidities using the Charlson Comorbidity Index (CCI) \citep{charlson1987new} and the Elixhauser Comorbidity Index (ECI) \citep{elixhauser1998comorbidity}, which condensed diagnoses into 17 and 31 well-defined comorbidity categories respectively. The demographic and clinical characteristics of the selected cohort are summarized in Table~\ref{tab:medication_adherence}. We detail the comorbidity categories along with other features used in the study in Appendix~\ref{apd:feature}.


\subsection{LLM configuration and prompt engineering}
We used the GPT-4o model \citep{openai2024gpt4ocard} (version 2024-05-13) via the HIPAA-compliant Microsoft Azure API, with the temperature set to 0 and all other parameters left at default. For each pair of visits, we provided the prescription record from the first visit and the clinical notes from the second visit to the model to assess adherence to the prescribed medication.

The model was prompted to identify instances of non-adherence, the type of non-adherence, and extract relevant sections from the notes. We used a zero-shot approach without additional training data or fine-tuning. We also implemented a second round of prompt validation by feeding the model's initial output back into the model, asking it to double-check its response. This additional step significantly reduced hallucinations. The prompt used in the study is provided in Appendix~\ref{apd:prompt}.

The cost for running all GPT-4o evaluations, including prompt development and inference was \$184.77, based on a cost of \$0.005 per 1,000 input tokens and \$0.015 per 1,000 output tokens.


\subsection{Physician validation of LLM detection}

To ensure the reliability of the LLM detection, we randomly selected 50 pairs labeled by the model as non-adherence and 50 pairs labeled as adherence for physician validation to assess accuracy. The gold standard was established through physician annotations conducted independently of the model's predictions. Overall, the model achieved an accuracy of 92\%, with four instances of physician-labeled non-adherence not detected and four adherent instances mislabeled as non-adherence(92\% precision and recall).

We further analyze discrepancies between the model and physician annotations, noting that some mismatches arise from ambiguous notes. For example, cases where patients restarted medication after hospitalization were marked as non-adherent by the LLM, since treatment was paused during hospitalization. Whereas physicians labeled them as adherent, considering the pause as a temporary interruption rather than true non-adherence.


\section{A Safety-Helpfulness Trade-off View of Jailbreak Defense}
\label{sec:trade_off_analysis}

\subsection{Formulating Defense as a Classification-Based Optimization}
Given a dataset \(\mathcal{D}\) comprising pairs of queries \(x_i\) and corresponding labels \(y_i \in \{0, 1\}\), where (\(y_i = 1\)) indicates a harmful query that should be refused, and (\(y_i = 0\)) denotes a benign query that should be complied with, as determined by human annotation. Let \(\theta\) represents a generative model, and \(\delta\) represents a defense method applied to the model or the input query. In the original generative task, the model under defense method \( \delta \) directly generates a response \(g(\theta, x; \delta)\) for query \(x_i\), which is then assessed as either a refusal or compliance.

In the classification formulation, the model is tasked with determining whether to refuse or comply with the input query, outputting a refusal probability \(p(\theta, x; \delta)\) under defense method \( \delta \) for the query \( x \). This format provides a more granular investigation of the model's preference, offering deeper insights compared to direct generative outputs.
Then the prediction \(f(\theta, x; \delta)\) is given by:
\begin{align*}
    f(\theta, x; \delta) = 
    \left\{
    \begin{array}{ll}
    0 & \text{if } p(\theta, x; \delta) < 0.5 \\
    1 & \text{if } p(\theta, x; \delta) \geq 0.5
    \end{array}
\right.
\end{align*}
The objective is to find the optimal defense \( \delta \) that minimizes the error between the true labels \(y_i\) and the defended model's predictions \(f(\theta, x; \delta)\), where \(\mathcal{L}(\cdot)\) is a loss function of the prediction error.
\begin{align*}
\min_{\delta} \mathbb{E}_{(x, y) \sim \mathcal{D}} \left[ \mathcal{L}(f(\theta, x; \delta), y) \right]
\end{align*}

This optimization objective can be decomposed into two components:
\begin{align*}
\begin{split}
\min_{\delta} \mathbb{E}_{(x, y) \sim \mathcal{D} \, | \, y = 1} \left[ \mathcal{L}(f(\theta, x; \delta), y) \right] \\
+ \min_{\delta} \mathbb{E}_{(x, y) \sim \mathcal{D} \, | \, y = 0} \left[ \mathcal{L}(f(\theta, x; \delta), y) \right]
\end{split}
\end{align*}
The first component focuses on the safety optimization, assessing whether the defense methods effectively enhance the model’s sensitivity to harmful inputs. The second component optimizes the defense mechanism to avoid overly constraining the model’s ability to identify benign inputs. This dual optimization captures the essential balance between safety and helpfulness.

\begin{figure*}[ht]
    \centering
    \begin{minipage}{0.33\textwidth} 
        \includegraphics[width=\linewidth]{Chapters/images/analysis_0.pdf}
        \subcaption{Baseline}
    \end{minipage}\hfill 
    \begin{minipage}{0.62\textwidth} 
        \includegraphics[width=\linewidth]{Chapters/images/analysis_1.pdf}
        \subcaption{Individual Defenses}
    \end{minipage}
    \caption{Representative results of individual defenses on refusal probabilities for harmful and benign queries. Compared to the baseline, system reminder and model optimization increase the mean refusal probabilities for both query types (\textbf{Safety Shift}). Query refactoring raises the mean refusal probability for harmful queries while lowering it for benign ones (\textbf{Harmfulness Discrimination}).}
    \label{fig:analysis results}
\end{figure*}

\subsection{Quantifying Defense using Probability-based Metrics}
\label{sec:defense_effects}
To quantify the impact of defense methods from the classification-based perspective, we introduce two relative metrics compared to the undefended model: Mean Shift and Distance Change.

\textbf{Mean Shift} measures how much the defense method \( \delta \) shifts the average refusal probabilities for input queries relative to the undefended model. We calculate mean shifts separately for harmful and benign queries as follows:
\begin{align*}
\begin{split}
\text{Mean\_Shift}_{\text{harmful}} &= \mathbb{E}_{x \in D_{\text{harmful}}}[p(\theta, x; \delta)] \\
&\quad - \mathbb{E}_{x \in D_{\text{harmful}}}[p(\theta, x)]
\end{split} \\
\begin{split}
\text{Mean\_Shift}_{\text{benign}} &= \mathbb{E}_{x \in D_{\text{benign}}}[p(\theta, x; \delta)] \\
&\quad - \mathbb{E}_{x \in D_{\text{benign}}}[p(\theta, x)]
\end{split}
\end{align*}
where \( \mathbb{E}_{x \in D}[p(\theta, x; \delta)] \) and \( \mathbb{E}_{x \in D}[p(\theta, x)] \) are the average refusal probabilities after and before applying the defense method $\delta$, respectively. A large shift in harmful data implies that the model becomes more safety-conscious, whereas a large shift in benign data suggests potential over-defense.  

\textbf{Distance Change} measures how the distance between the refusal probability distributions for harmful and benign data changes before and after applying the defense. Let \( P_{\text{harmful}} \) and \( P_{\text{benign}} \) represent the  refusal probability distributions for harmful and benign data before defense, and \( P^{\delta}_{\text{harmful}} \) and \( P^{\delta}_{\text{benign}} \) represent these distributions after defense. The distribution distance is defined as:
\begin{align*}
\begin{split}
\text{Distribution\_Distance} = &\ \text{Dist}(P_{\text{benign}}^{\delta}, P_{\text{harmful}}^{\delta}) \\
&- \text{Dist}(P_{\text{benign}}, P_{\text{harmful}})
\end{split}
\end{align*}
where \( \text{Dist}(\cdot, \cdot) \) denotes a distance metric between probability distributions, such as Jensen-Shannon divergence. A larger distance change indicates that the defense method improves the model's ability to distinguish between harmful and benign queries.


\begin{figure*}[ht]
    \centering
    \begin{minipage}{0.25\textwidth} 
        \includegraphics[width=\linewidth]{Chapters/images/analysis_0.pdf}
        \subcaption{Baseline}
    \end{minipage}\hfill 
    \begin{minipage}{0.75\textwidth} 
        \begin{minipage}{\linewidth}
            \includegraphics[width=\linewidth]{Chapters/images/analysis_2.pdf}
            \vspace{-6mm}
            \subcaption{Inter-Mechanism Ensembles}
        \end{minipage}
        \vfill
        \vspace{5pt}
        \begin{minipage}{\linewidth}
            \centering
            \includegraphics[width=0.75\linewidth]{Chapters/images/analysis_3.pdf}
            \vspace{-2mm}
            \subcaption{Intra-Mechanism Ensembles}
        \end{minipage}
    \end{minipage}
    \caption{Representative results for ensemble defenses. Inter-mechanism ensembles tend to reinforce the mechanism while intra-mechanism ensembles achieve a better trade-off between mechanisms.}
    \label{fig:analysis_results}
\end{figure*}

\subsection{Investigating Mechanisms of Defense Methods}
% While this approach may not fully capture the model's decision-making process in generative tasks as discussed in Section~\ref{sec:consistency}, it provides valuable insights into the effects of defense strategies on model behavior. 
To quantitatively analyze various defense methods, we prompt the model to classify whether it would comply with or refuse a given query, extracting the logits of refusal as its refusal probability. We conduct this analysis on the MM-SafetyBench dataset with LLaVA-1.5-13B model. The detailed prompt and analysis setup are provided in Appendix~\ref{sec:analysis_setup}. 

We specifically focus on four categories of internal jailbreak defenses described in Section~\ref{internal_defense_background}, and examine multiple methods for each category. A representative result is shown in Figure~\ref{fig:analysis results}, with the full set of results available in Appendix~\ref{sec:more_analyss_result}. Additional analyses on more LVLMs and LLMs are in Appendx~\ref{sec:extra_lvlm} and \ref{sec:extra_llm}. We also assess the consistency between the original generation task and the re-formulated classification task in Appendix~\ref{sec:consistency_appendix}.
Across these defense methods, two significant mechanisms emerge: Safety Shift and Harmfulness Discrimination, which explain how these defenses work.

\paragraph{Safety Shift} Compared to the baseline undefended model, both system reminder and model optimization defenses exhibit a significant mean shift across harmful and benign query subsets, without necessarily increasing the distance between the refusal probability distributions for these two groups.
This safety shift mechanism stems from the enhancement of model's general safety awareness, leading to a broad increase in refusal tendencies for both harmful and benign queries. However, such a conservative response to both types of queries can result in over-defense and does not significantly improve the model's ability to discriminate between harmful and benign inputs.

\paragraph{Harmfulness Discrimination} In contrast, query refactoring defenses either increases the refusal probabilities for harmful queries or decrease them for benign queries, leading to a consistent enlargement of the gap between the refusal probability distributions of these two subsets. This harmfulness discrimination mechanism enables better interpretation of the harmfulness within harmful queries or harmlessness within benign queries, thereby improving the distinction between them. However, the concealment of harmfulness within some queries can limit these improvements.

Additionally, noise injection demonstrate limited effectiveness, as indicated by insignificant changes in both the mean shift and distance change metrics. This is because it primarily targets attacks where noise is deliberately added to input queries, making it less effective in defending against general input queries without intentional noise.

\begin{table*}[ht]
    \centering
    \resizebox{0.82\textwidth}{!}{
    \begin{tabular}{r|cccccc|cccccc}
        \toprule 
        & \multicolumn{6}{c}{\textbf{LLaVA-1.5-7B}} & \multicolumn{6}{c}{\textbf{LLaVA-1.5-13B}} \\ 
        \cmidrule(lr){2-7}\cmidrule(lr){8-13}
        & \multicolumn{3}{c}{\textbf{MM-SafetyBench}} & \multicolumn{3}{c|}{\textbf{MOSSBench}} & \multicolumn{3}{c} {\textbf{MM-SafetyBench}} & \multicolumn{3}{c}{\textbf{MOSSBench}} \\
        \textbf{Method} & \textbf{DSR}$\uparrow$ & \textbf{RR}$\uparrow$ & \textbf{Avg}$\uparrow$ & \textbf{DSR}$\uparrow$ & \textbf{RR}$\uparrow$ & \textbf{Avg}$\uparrow$ & \textbf{DSR}$\uparrow$ & \textbf{RR}$\uparrow$ & \textbf{Avg}$\uparrow$ & \textbf{DSR}$\uparrow$ & \textbf{RR}$\uparrow$ & \textbf{Avg}$\uparrow$\\
        \midrule
        w/o Defense          & 0.06  & 0.98  & 0.52  & 0.14  & 0.97  & 0.55  & 0.10  & 0.97  & 0.53  & 0.30  & 0.96  & 0.63  \\
        \midrule
        \multicolumn{13}{c}{System Reminder} \\
        \midrule
        Responsible          & 0.12  & 0.96  & 0.54  & 0.32  & 0.96  & \underline{0.64}  & 0.18  & 0.96  & \underline{0.57}  & 0.47  & 0.92  & \underline{0.70}  \\
        Policy               & 0.08  & 0.96  & 0.52  & 0.18  & 0.98  & 0.58  & 0.12  & 0.97  & 0.55  & 0.34  & 0.97  & 0.65  \\
        Demonstration        & 0.15  & 0.97  & \underline{0.56}  & 0.37  & 0.95  & \underline{0.66}  & 0.25  & 0.96  & \textbf{0.60}  & 0.52  & 0.92  & \textbf{0.72}  \\
        \midrule
        \multicolumn{13}{c}{Model Optimization} \\
        \midrule
        SFT                  & 0.20  & 0.95  & \textbf{0.58}  & 0.50  & 0.88  & \textbf{0.69}  & 0.13  & 0.98  & 0.55  & 0.49  & 0.88  & \underline{0.68} \\
        SafeDecoding         & 0.08  & 0.97  & 0.53  & 0.31  & 0.94  & 0.62  & 0.12  & 0.96  & 0.54  & 0.42  & 0.93  & \underline{0.68}  \\
        DPO & 0.06 & 0.97 & 0.52 & 0.28 & 0.97 & 0.63 & 0.08 & 0.98 & 0.53 & 0.39 & 0.95 & 0.67 \\
        % SFT-only             & 0.24  & 0.96  & 0.60  & 0.58  & 0.78  & 0.68  & 0.06  & 0.97  & 0.52  & 0.47  & 0.83 & 0.65 \\
        % DPO & \\
        \midrule
        \multicolumn{13}{c}{Query Refactoring} \\
        \midrule
        Caption              & 0.09  & 0.98  & 0.53  & 0.21  & 0.98  & 0.60  & 0.12  & 0.97  & 0.55  & 0.27  & 0.94  & 0.60  \\
        Caption (w/o image)  & 0.16  & 0.95  & \underline{0.55}  & 0.34  & 0.94  & \underline{0.64}  & 0.22  & 0.93  & \underline{0.57}  & 0.45  & 0.89  & 0.67  \\
        Intention            & 0.07  & 0.98  & 0.53  & 0.20  & 0.99  & 0.59  & 0.11  & 0.96  & 0.54  & 0.26  & 0.97  & 0.61  \\
        % Intention (w/o image)& 0.32  & 0.90  & 0.61  & 0.27  & 0.93  & 0.60  & 0.29  & 0.92  & 0.61  & 0.39  & 0.93  & 0.66  \\
        \midrule
        \multicolumn{13}{c}{Noise Injection} \\
        \midrule
        Mask Image           & 0.07  & 0.97  & 0.52  & 0.12  & 0.98  & 0.55  & 0.08  & 0.97  & 0.52  & 0.32  & 0.94  & 0.63 \\
        Vertical Flip Image  & 0.05  & 0.98  & 0.51  & 0.10  & 0.98  & 0.54  & 0.09  & 0.97  & 0.53  & 0.34  & 0.97  & 0.66 \\
        Swap Text            & 0.01  & 0.98  & 0.50  & 0.14  & 0.96  & 0.55  & 0.13  & 0.94  & 0.53  & 0.32  & 0.96  & 0.64 \\
        Insert Text          & 0.03  & 0.98  & 0.50  & 0.13  & 0.96  & 0.54  & 0.09  & 0.95  & 0.52  & 0.28  & 0.94  & 0.61  \\
        \bottomrule
    \end{tabular}}
    \caption{Evaluation results of various individual defense methods. \textbf{Bold} indicates the best overall performance, while \underline{underlined} highlights the top three methods.} % and the full score is 100\%
    \label{tab:indi_results}
\end{table*}


\subsection{Exploring Defense Ensemble Strategies}

An effective defense should block harmful queries while preserving helpfulness for benign ones. Achieving this requires balancing safety shifts without over-defense and enhancing harmfulness discrimination. Since different defense methods impact model safety differently, we explore ensemble strategies to optimize this trade-off:

\begin{itemize}[itemsep=0.5pt, leftmargin=12pt, parsep=1pt, topsep=1pt]
    \item \textbf{Inter-Mechanism Ensemble} combines defenses operating the same mechanism, including safety shift ensembles and harmfulness discrimination ensembles.
    For safety shift ensembles, we combine multiple system reminder methods \textit{(SR++)} or combine system reminder with model optimization methods \textit{(SR+MO)}. For harmfulness discrimination ensemble, we combine multiple query refactoring methods \textit{(QR++)}.
    \item \textbf{Intra-Mechanism Ensemble} combines two defenses where one improves safety shift and the other enhances harmfulness discrimination. This includes ensembling query refactoring with system reminder methods \textit{(QR\textbar{}SR)} or with model optimization methods \textit{(QR\textbar{}MO)}.
\end{itemize}

% The inter-mechanism ensemble combines multiple safety shift methods, which can enhance overall safety by reinforcing conservative responses across different models. The intra-mechanism ensemble integrates a safety shift method and a harmfulness discrimination method, where the latter can help mitigate the refusal probability distribution shift of benign queries, thereby increase the distance between the two subsets.

For each ensemble strategy, we explore several variants using different specific methods. 
% A detailed description of these variants is provided in Appendix~\ref{sec:ensemble_strategy}. 
Representative results are shown in  Figure~\ref{fig:analysis_results}, with the full set of variant results available in Appendix~\ref{sec:more_analyss_result}.

We observe that inter-mechanism ensembles tend to strengthen a single defense mechanism. Safety shift ensembles like \textit{SR++} and \textit{SR+MO} further enhance model safety but exacerbate the loss of helpfulness. Conversely, harmfulness discrimination ensembles achieve a larger mean shift on benign queries towards compliance, making them better suited for situations where maintaining helpfulness is critical. 

In contrast, intra-mechanism ensembles combine the strengths of both mechanisms to achieve a more balanced trade-off. Specifically, \textit{QR\textbar{}SR} and \textit{QR\textbar{}MO} increase the refusal probability for harmful queries, while maintaining or even decreasing the refusal probability for benign queries, thereby improving the model's ability to distinguish between benign and harmful queries. This makes them a better choice for general scenarios where balancing safety and helpfulness is essential. 


\section{Impact of Non-Adherence Bias on Downstream Performance}
\label{sec:performance}
In this section, we evaluate the impact of treatment non-adherence bias on downstream model performance. We first define the outcome of interest and demonstrate the effect of non-adherence on the outcome in Section~\ref{sec:setup}. We then analyze its impact on treatment effect estimation for causal inference in Section~\ref{sec:causal_inference} and on ML model performance in Section~\ref{supervised_learning}.


\subsection{Treatment non-adherence leads to worse blood pressure outcomes}
\label{sec:setup}
Blood pressure reduction is the primary outcome when evaluating hypertension medications. To investigate the impact of treatment non-adherence on downstream performances, we begin by extracting pairs of visits where blood pressure measurements are available for both visits. To ensure a sufficient number of encounters for each medication, we focus on the top five most commonly prescribed medications: amlodipine, lisinopril, losartan, hydrochlorothiazide and metoprolol. We only include pairs where the duration between visits is less than six months to minimize the influence of other factors that could affect blood pressure over longer intervals. This results in 1732 pairs of encounters in total with 303 non-adherent pairs.

We begin by showcasing the impact of treatment non-adherence on blood pressure reduction between visits using t-tests. The results presented in Table~\ref{tab:treatment_adherence_outcome} indicate that non-adherence leads to smaller blood pressure reduction, with 1.96 mmHg less systolic reduction ($p=0.011$) and 3.93 mmHg less diastolic reduction ($p=0.001$) compared to adherence.

\begin{table}[h]
\centering
\footnotesize 
\caption{Results of the t-tests assessing the effect of treatment non-adherence on blood pressure reduction. Treatment non-adherence is statistically significant for both systolic and diastolic reduction, with non-adherence leading to smaller reductions in systolic and diastolic blood pressure.}
\label{tab:treatment_adherence_outcome}
\begin{tabular}{@{}l@{}c@{\hspace{5pt}}c@{\hspace{5pt}}c@{}}
\toprule
\textbf{Outcome} &  \textbf{Mean Difference} &  \textbf{95\% CI} &\textbf{$p$-value} \\
\midrule
Systolic Reduction &  -1.96 & (-3.47 to -0.46) &    \textbf{0.011} \\
Diastolic Reduction &  -3.93 & (-6.23 to -1.63) &    \textbf{0.001} \\
\bottomrule
\end{tabular}
\end{table}


\subsection{Causal inference for treatment effect estimation}
\label{sec:causal_inference}
Amlodipine and Lisinopril are the most commonly prescribed medications for hypertension, leading to numerous randomized controlled trials (RCTs) comparing their treatment effects \citep{cappuccio1993amlodipine, naidu2000evaluation}. However, due to the high cost of RCTs, various causal inference methods have been developed to estimate treatment effects from observational data \citep{pearl2009causality, austin2011introduction, shalit2017estimatingindividualtreatmenteffect, K_nzel_2019}. Among them, Inverse Probability Weighting (IPW) is one of the most widely used methods \citep{austin2011introduction}, providing an unbiased estimation of the Average Treatment Effect (ATE) by adjusting for confounding. We start by demonstrating the impact of treatment non-adherence bias on ATE estimation using IPW.
\\
\newline
\textbf{Experiment Setup.} 
We compare the ATE estimation with and without including treatment non-adherent data. Demographic and clinical factors are included as confounders and detailed in Appendix~\ref{apd:feature}. Patients prescribed lisinopril act as the control group, while those taking amlodipine are considered the treated group. The treatment effect is assessed based on the reduction in diastolic and systolic blood pressure between two visits.
\\
\newline
\textbf{Results.} 
The results are presented in Table~\ref{tab:ipw_ate}. Without filtering for non-adherent data, amlodipine lowers diastolic blood pressure by 1.75 mmHg but increases systolic blood pressure by 0.06 mm Hg compared to lisinopril. After excluding non-adherent data, amlodipine lowers diastolic blood pressure by 1.40 mmHg and also reduces systolic blood pressure by 0.11 mmHg compared to lisinopril. This result shows a reversal in the estimated treatment effect for systolic blood pressure reduction before and after excluding non-adherent data.
\begin{table}[h]
\centering
\footnotesize 
\caption{Estimated ATE of medication on blood pressure reduction using IPW. Notably, excluding non-adherent data reverses the conclusion on the treatment effect of systolic blood pressure reduction, causing the estimate to flip from -0.06 to 0.11 mmHg.}
\label{tab:ipw_ate}
\begin{tabular}{@{}lcc@{}}
\toprule
\textbf{Dataset} & \multicolumn{2}{c}{\textbf{Blood Pressure Reduction}} \\ 
\cmidrule(lr){2-3}
 & \textbf{Diastolic} & \textbf{Systolic} \\ 
\midrule
Full Dataset & 1.75 & -0.06 \\
Adherent Data Only & 1.40 & 0.11 \\
\bottomrule
\end{tabular}
\end{table}

\subsection{Supervised learning for treatment outcome prediction}
\label{supervised_learning}
We now demonstrate the impact of treatment non-adherence bias on predictive modeling performance. Following a common setup in the literature \citep{mroz2024predicting, yi2024development}, we use patients' EHR data with treatment prescriptions and blood pressure measurements from their first visit as covariates. The target to predict is whether the blood pressure will be normal at their second visit. Following the guidelines of the American Heart Association \citep{heartUnderstandingBlood}, we define normal blood pressure as having a systolic value of less than 120 and a diastolic value of less than 80. To evaluate model performance in predicting outcomes, we use 500 adherent samples as the test set in all subsequent experiments. We test exclusively on adherent patients since our goal is to evaluate model performance in scenarios where treatments are followed as prescribed, representing the intended clinical use case. A detailed description of the features used is provided in Appendix~\ref{apd:feature}.

\begin{figure}[t]
    \centering
    \includegraphics[width=\textwidth]{figures/treatment_non_adherence_ratio.png}
    \caption{Results of varying treatment non-adherence data percentage on model performance and fairness. Increasing the proportion of non-adherent data in the training set degrades predictive performance and increases fairness disparities between Black and non-Black patients, as measured by demographic parity and the equal odds criterion (true positive rate and false positive rate differences). Results are averaged over 100 seeds, with error bars representing the standard error of the mean.}
    \label{fig:treatment_non_adherence_ratio}
    \includegraphics[width=0.9\textwidth]{figures/remove_non_adherence_rf.png}
    \caption{Results of removing non-adherent data on model performance and fairness. The black curve represents training on 75\% of the full dataset, which only consists of adherent encounters. Removing non-adherent data improves model performance, with greater gains observed as sample size increases. It also decreases fairness disparities between Black and non-Black patients, as measured by demographic parity and the equal odds criterion (true positive rate and false positive rate differences). Results are averaged over 100 seeds, with error bars representing the standard error of the mean.}
    \label{fig:remove_non_adherence_rf}
\end{figure}


\subsubsection{Effect of varying treatment non-adherence data ratios on model performance and fairness}
\label{sec:very_ratio}
We begin by showing the harmful impact of treatment non-adherence bias by varying the proportion of non-adherent data in the training set.
\\
\newline
\textbf{Experiment Setup.}  
We fix the training set size at 300 and vary the proportion of non-adherent data in the training set across \(\{0\%, 10\%, 30\%, 50\%, 70\%, 90\%\}\) to evaluate the impact of treatment non-adherence bias on model performance. We train logistic regression and random forest models, both commonly used for modeling tabular EHR data and assess performance using AUROC on the test set.  Beyond performance, ensuring fair decision-making is also a critical consideration in healthcare. Let \( A \) denote the sensitive attribute (e.g., race), \( Y \) represent the true outcome, and \( \hat{Y} \) denote the predicted outcome. Demographic parity \citep{dwork2012fairness} difference measures the disparity in the likelihood of receiving a positive prediction between groups, i.e.,
\begin{equation*}
    |P(\hat{Y} = 1 \mid A = 1) - P(\hat{Y} = 1 \mid A = 0)|
\end{equation*}
Equal odds \citep{hardt2016equality} difference compares both true positive rates and false positive rates across groups, i.e., 
\begin{align*}
    |P(\hat{Y} = 1 \mid Y = 1, A = 1) - P(\hat{Y} = 1 \mid Y = 1, A = 0)|\\
    |P(\hat{Y} = 1 \mid Y = 0, A = 1) - P(\hat{Y} = 1 \mid Y = 0, A = 0)|
\end{align*}
We therefore measure the differences in demographic parity and equal odds across racial groups to assess fairness.  
\\
\newline
\textbf{Results.} 
We present the results in Figure~\ref{fig:treatment_non_adherence_ratio}, showing that increasing the percentage of non-adherent data in the training set degrades performance, with a 3\% drop for logistic regression and a 5\% drop for random forest in AUROC. Additionally, a higher proportion of non-adherent data increases fairness disparities between Black and non-Black patients under both the demographic parity and equal odds criteria. For instance, the false positive rate disparity of the random forest doubles, increasing from 0.125 to 0.25 as the percentage of non-adherent data increases. Similar trends are observed for other racial groups, and we provide full results in Appendix~\ref{apd:fairness}. These findings consistently highlight the harmful impact of treatment non-adherence bias.




\subsubsection{
Effect of removing non-adherent data on model performance and fairness}
\label{sec:drop_data}
We now emphasize the importance of addressing treatment non-adherence bias by showing that a simple approach to remove non-adherent data can improve predictive performance and lead to fairer predictions.
\\
\newline
\textbf{Experiment setup.}
We fix the non-adherent data ratio at 25\% and compare the performance of random forests trained on the entire dataset versus those trained only on adherent data by removing a quarter of data that are non-adherent. We report the test AUROC as well as demographic and equal odds differences while varying the full training set size in \(\{600, 800, 1000, 1200\}\) before removing non-adherent data.
\\
\newline
\textbf{Results.} 
The results are presented in Figure~\ref{fig:remove_non_adherence_rf}. While the traditional ML perspective suggests that more data generally improves performance, our findings show that using only the adherent 75\% of the data leads to better model performance, with the improvement becoming more significant as the sample size increases. For instance, with a training size of 1,200, the model achieves an AUROC of 0.695 when using all data, whereas dropping non-adherent data improves AUROC to 0.71. Additionally, we find removing non-adherent data reduces racial disparities between Black and non-Black patients, as both demographic parity and equal odds differences are consistently smaller across sample sizes. Similar trends are observed for other racial groups, and we provide full results in Appendix~\ref{apd:fairness}. These findings further highlight the importance of addressing treatment non-adherence bias to achieve better and fairer model performance.


\section{Discussions}

\subsection{Transparency in Ride-Sharing Platform Algorithms}
The publicly available Chicago Transportation Network Provider dataset helped us answer many research questions, but ride-sharing platforms still make many of their mechanisms opaque. The lack of transparency in key platform mechanisms---such as pricing models, driver--rider matching algorithms, and driver ranking systems---makes it difficult to pinpoint the exact causes of these disparities. Without greater visibility into these proprietary algorithms, drivers also remain at an information disadvantage, unable to anticipate fare fluctuations or optimize their work schedules effectively.

Pricing models remain opaque, with our analysis revealing that fare adjustments over time have failed to keep pace with inflation, effectively reducing real driver earnings (\cref{sec:results-pricing-stablization}). While platforms advertise dynamic pricing mechanisms that respond to demand surges, drivers have limited insight into how much of the fare they actually receive after platform fees~\cite{santos2020dynamic}. Previous research has shown that drivers tend to work more during peaks for higher compensation~\cite{chen2016dynamic}. A real-time, large-scale understanding of the surge pricing model can help drivers become more informed in planning and organizing their workday, beyond anecdotal observations. Furthermore, researchers can provide prediction models of price surges, helping both drivers and riders adjust plans accordingly. Another key limitation of using the Chicago dataset is the lack of driver earning information. As a result, our analysis can only use the trip fare as a proxy for driver earning. Making such information available can significantly increase transparency into platform operations.

Similarly, the driver-rider matching algorithm remains a black box. Our inferred driver profiles suggest that trip assignments may systematically disadvantage certain groups, particularly those operating in lower-income areas. If the matching algorithm disproportionately favors drivers in high-demand or high-fare regions, it could reinforce existing geographic disparities in earnings. However, such analysis is hard to conduct without access to driver-level information. As discussed in \cref{sec:methods-driver-simulation}, releasing such data may lead to privacy concerns. Our approach is an effort to approximate driver working conditions without needing detailed driver data. However, researchers should still work with ride-sharing platforms to come up with privacy-preserving ways to analyze such data for insights. Also, driver ranking algorithms---which determine access to high-value trips---are equally opaque. While platforms often cite factors such as acceptance rate, customer ratings, and trip history, the lack of public accountability raises concerns regarding potential biases. Accessing such information can support researchers in identifying potential biases, also help drivers provide more desired services to riders.

In all, we call for increased regulatory oversight and platform-level efforts to improve algorithmic transparency. Without clear disclosures on how these systems operate, ride-sharing drivers remain vulnerable to unfair decision-making and fluctuating incomes that they cannot predict or control.

\subsection{Data Analysis Methodology Improvements}
Our study demonstrates the feasibility of simulating reasonable driver profiles from trip-level data, even in the absence of driver-related information. By leveraging a simulation-based approach, we were able to approximate driver earnings, work patterns, and geographic activity. However, there are still areas for improvement for our methodology.

First, a robust evaluation benchmark is needed to validate the accuracy of inferred driver profiles. While our approach provides valuable insights and matches previous empirical findings, the lack of direct ground truth data means we rely on approximations. We need alternative data sources to cross-verify our inferred driver activities. Tools for driver task management, such as Driver's Seat~\cite{calacci2023access}, asks drivers to upload their work tasks and can serve as a potential data source. More autonomous approaches that uses UI understanding techniques and directly collects data from drivers' phones can also scale up this effort~\cite{lu2024crepe}. 

Moreover, expanding the scope of inferred information would provide deeper insights into platform operations. Currently, we infer earnings and work patterns for drivers. Newer algorithms can be developed to analyze additional opaque platform mechanisms as discussed above. Future studies could aim to reconstruct other aspects of opaque platform algorithms, as discussed above, directly from publicly available, large-scale datasets.

Given a large-scale dataset that misses key information aspects, a potential future approach is to self-collect a smaller dataset that contains the necessary details and conduct a joint analysis of both datasets. For example, a smaller dataset that we collect directly from drivers, containing both driver and trip information, can serve both as a benchmark and a basis for use to train machine learning models that predict driver profiles from existing large-scale datasets. Future research can investigate effective measures to combine these different data sources~\cite{harris2018federal} for joint analysis. These methodological advancements can help us to use large-scale ridesharing datasets more effectively and accurately while maintaining driver and rider privacy.


\subsection{Societal Implications: Ride-Sharing as a Reflection of Broader Inequalities}

Our findings revealed regional ride-sharing disparities in the city of Chicago, which largely reflect the broader existing societal inequalities. Drivers working in lower-income neighborhoods---in our case, drivers that service the southern regions of Chicago---consistently earn less, even despite longer work hours. Structural disadvantages, such as lower infrastructure quality, longer wait times, and increased safety concerns---compound the challenges faced by gig workers. Chicago South Side, as a community suffering from violence and poverty, has been an example of social segregation and studied by numerous researchers~\cite{moore2016south, bachin2004building, bell1993community}. As an aspect of a deep-rooted societal issue, ride-sharing inequality in lower-income neighborhoods calls for holistic policymaking efforts from multiple stakeholders.

Our findings provide practical implications for labor activists and policy makers. By providing a more transparent view of drivers’ potential workday experiences, policymakers can better evaluate the labor conditions these platforms create, ensuring that emerging mobility systems align with equity goals. Urban planners and regulators can use these insights to inform policy interventions---such as driver support programs, driver caps, or incentive structures---that promote fairness and mitigate algorithmic biases. Similarly, platform operators themselves might harness these findings to improve their matching algorithms, advancing a more equitable ecosystem that benefits both drivers and passengers.

Research has shown that transportation access can have a positive impact on regional economic growth and productivity~\cite{targa2005economic, banerjee2020road, alstadt2012relationship}. Ride-sharing, as an increasingly critical way of transportation, especially where public transportation is scarce, can support individual and community access to growth opportunities. The persistence of regional earning gaps raises important questions about equity in urban transportation. If ride-sharing platforms are designed primarily to maximize efficiency and revenue, they may inadvertently exacerbate existing economic inequalities by steering high-value rides away from underserved areas~\cite{durand2022access, bocarejo2012transport}.

To address these issues, we call for policy interventions aimed at ensuring fair compensation and equitable access to earning opportunities. Regulators should consider implementing transparency mandates, income stability measures, and algorithmic accountability frameworks to prevent platforms from disproportionately disadvantaging certain driver groups. Moreover, these efforts should be in orchestration with existing efforts to promote infrastructural improvements and public safety in underserved regions. Collaborative initiatives between policymakers, ride-sharing companies, and community organizations can help create a more inclusive transportation ecosystem that benefits both drivers and passengers alike~\cite{baber2022new}.



\clearpage
\bibliographystyle{apalike} %
\bibliography{references}

\appendix
\onecolumn

\section{Cohort Creation} \label{apd:cohort}
We detail the cohort selection in Figure~\ref{fig:cohort}.
\begin{figure}[h]
    \centering
    \includegraphics[width=0.45\linewidth]{figures/cohort_selection.jpg}
    \caption{Cohort Selection}
    \label{fig:cohort}
\end{figure}

\section{Feature Definitions}\label{apd:feature}
We use the following features throughout the study. The time between visits is excluded from causal inference and predictive modeling in Section~\ref{sec:performance}, as we instead restrict the analysis to a smaller time window. ECI and CCI are derived using medcodes package \citep{githubGitHubTopspinjmedcodes}.

\begin{table*}[h!]
    \centering
    \small
    \renewcommand{\arraystretch}{1.2} %
    \begin{tabular}{p{4cm} p{12cm}}
        \hline
        \textbf{Feature Name} & \textbf{Description} \\
        \hline
        Sex & The sex of the patients. \\
        Age & The age of the patients at the time of the visit. \\
        Race & The race of the patients. \\
        Marital Status & The marital status of the patients at the time of the visit. \\
        Time Between Visits & The time in days between the two visits in each pair. \\
        Charlson Comorbidity Index (CCI) & A measure of comorbidity based on the following conditions: myocardial infarction, congestive heart failure, peripheral vascular disease, cerebrovascular disease, dementia, chronic pulmonary disease, rheumatic disease, peptic ulcer disease, mild liver disease, diabetes without chronic complications, diabetes with chronic complications, hemiplegia/paraplegia, renal disease, any malignancy, moderate/severe liver disease, metastatic solid tumor, and AIDS/HIV. \\
        Elixhauser Comorbidity Index (ECI) & A measure of comorbidity based on the following conditions: cardiac arrhythmias, congestive heart failure, valvular disease, pulmonary circulation disorders, peripheral vascular disorders, hypertension (uncomplicated or complicated), paralysis, other neurological disorders, chronic pulmonary disease, diabetes (uncomplicated or complicated), hypothyroidism, renal failure, liver disease, peptic ulcer disease, AIDS/HIV, lymphoma, metastatic cancer, solid tumor without metastasis, rheumatoid arthritis, coagulopathy, obesity, weight loss, fluid and electrolyte disorders, blood loss anemia, deficiency anemia, alcohol abuse, drug abuse, psychoses, and depression. \\
        Duration of Hypertension & The duration (in years) between the onset of hypertension and the time of the visit. \\
        Number of Primary Care Visits & The number of primary care visits one year prior to the visit. \\
        \hline
    \end{tabular}
    \caption{Descriptions of features included in the study}
    \label{tab:features}
\end{table*}

\section{Prompt Description}\label{apd:prompt}
The prompt used in the study is given in Figure~\ref{fig:prompt}.
\begin{figure*}[h!]
    \centering
    \includegraphics[width=0.95\textwidth]{figures/prompt1.jpg} \\
    \includegraphics[width=0.95\textwidth]{figures/prompt2.jpg} \\
    \includegraphics[width=0.95\textwidth]{figures/prompt3.jpg}
    \caption{Prompt used in the study}
    \label{fig:prompt}
\end{figure*}

\section{Topic Modeling}\label{apd:topic}
Here we show the full BERT topic modeling results with key terms found in clusters and we summarize the topics in Figure~\ref{fig:reason}.

\begin{figure*}[h!]
    \centering
    \includegraphics[width=\textwidth]{figures/appendix_topic_keywords.png}
    \caption{BERT topic modeling identified five clusters of treatment non-adherence reasons. The most common words in each cluster are highlighted in the plot. Key topics identified include side effects, forgetfulness, failure to pick up medications, need for refills, and lost medications. When applying BERT topic modeling, we set a minimum cluster size of 15 notes and used UMAP with 5 components and 15 neighbors for dimensionality reduction.}
    \label{fig:appendix_topic}
\end{figure*}

\newpage
\section{Additional Results of Non-adherence Bias on Predictive Modeling}\label{apd:fairness}
Here, we present the complete results on racial disparities for the experiments in Section~\ref{supervised_learning}.
\begin{figure*}[h!]
    \centering
    \includegraphics[width=\textwidth]{figures/race_disparity_vary_na_ratio.png}
    \caption{Increasing the proportion of treatment non-adherent data in the training set increases the fairness disparity between different races as measured by demographic parity and the equal odds criterion. Results are averaged over 100 seeds, varying the sampling of the train and test sets. Error bars represent the standard error of the mean.}
    \label{fig:appendix_race_vary_na_ratio}
\end{figure*}
\newpage
\begin{figure*}[h!]
    \centering
    \includegraphics[width=\textwidth]{figures/race_disparity_vary_train_size.png}
    \caption{The black curve represents training on 75\% of the full dataset consisting of adherent encounters only. Removing treatment non-adherent data from the training set decreases fairness disparities as measured by demographic parity and the equal odds criterion particularly for Black and Asian. Results are averaged over 100 seeds, varying the sampling of the train and test sets. Error bars represent the standard error of the mean.}
    \label{fig:appendix_race_vary_sample_size}
\end{figure*}

\end{document}
