\documentclass{article}

\usepackage{graphicx}%
\usepackage{multirow}%
\usepackage{makecell}
\usepackage{amsmath,amssymb,amsfonts}%
\usepackage{amsthm}%
\usepackage{mathrsfs}%
\usepackage{xcolor}%
\usepackage{textcomp}%
\usepackage{manyfoot}%
\usepackage{booktabs}%
\usepackage{algorithm}%
\usepackage{algorithmicx}%
\usepackage{algpseudocode}%
\usepackage{listings}%
\usepackage{array}
\usepackage{caption}
\usepackage{pdflscape}
\usepackage{natbib}
\usepackage{subcaption}
\usepackage{threeparttable}
\usepackage{hyperref}
\usepackage[draft]{fixme}

\usepackage{mathtools}
\usepackage{lineno}

\usepackage{graphicx} %
\usepackage{booktabs}
\usepackage[margin=1in]{geometry}

\usepackage{bbm}
\usepackage[T1]{fontenc}

\newcommand{\theHalgorithm}{\arabic{algorithm}}


\fxsetup{inline,nomargin,theme=color}


\theoremstyle{plain}
\newtheorem{theorem}{Theorem}[section]
\newtheorem{proposition}[theorem]{Proposition}
\newtheorem{lemma}[theorem]{Lemma}
\newtheorem{corollary}[theorem]{Corollary}
\theoremstyle{definition}
\newtheorem{definition}[theorem]{Definition}
\newtheorem{assumption}[theorem]{Assumption}
\theoremstyle{remark}
\newtheorem{remark}[theorem]{Remark}


\definecolor{todo-color}{HTML}{ff3232}
\newcommand{\todo}[1]{{\color{todo-color}[#1]}}
\newcommand\todocite{{\color{red}{CITE}}\xspace}


\newcommand{\fix}{\marginpar{FIX}}
\newcommand{\new}{\marginpar{NEW}}


\newcommand{\blue}[1]{\textcolor{blue}{#1}}
\newcommand{\red}[1]{\textcolor{red}{#1}}
\newcommand{\tba}{\textcolor{red}{To Be Added }}

\title{Treatment Non-Adherence Bias in Clinical Machine Learning: A Real-World Study on Hypertension Medication}
\author{Zhongyuan Liang$^{1,2}$, Arvind Suresh$^{2}$, and Irene Y. Chen$^{1,2}$}
\date{$^1$University of California, Berkeley, $^2$University of California, San Francisco}

\begin{document}

\maketitle

\begin{abstract}
Machine learning systems trained on electronic health records (EHRs) increasingly guide treatment decisions, but their reliability depends on the critical assumption that patients follow the prescribed treatments recorded in EHRs. Using EHR data from 3,623 hypertension patients, we investigate how treatment non-adherence introduces implicit bias that can fundamentally distort both causal inference and predictive modeling. By extracting patient adherence information from clinical notes using a large language model, we identify 786 patients (21.7\%) with medication non-adherence. We further uncover key demographic and clinical factors associated with non-adherence, as well as patient-reported reasons including side effects and difficulties obtaining refills. Our findings demonstrate that this implicit bias can not only reverse estimated treatment effects, but also degrade model performance by up to 5\% while disproportionately affecting vulnerable populations by exacerbating disparities in decision outcomes and model error rates. This highlights the importance of accounting for treatment non-adherence in developing responsible and equitable clinical machine learning systems.
\end{abstract}

\section{Introduction}
\label{sec:intro}
Treatment non-adherence is a pervasive and persistent challenge in healthcare. Researchers estimate that poor medication adherence leads to 125,000 preventable deaths annually in the U.S. and contributes to \$100-\$300 billion in avoidable healthcare costs \citep{benjamin2012medication}. This issue is particularly prevalent among patients with chronic conditions such as hypertension, with 40-50\% failing to take their medications as prescribed \citep{kleinsinger2018unmet, algabbani2020treatment}. While researchers have extensively documented this problem through surveys and interviews \citep{boratas2018evaluation, fernandez2019adherence, algabbani2020treatment, najjuma2020adherence, schober2021high}, the studies---and ultimately understanding of treatment non-adherence---remain limited by small sample sizes and self-reporting bias \citep{adams1999evidence, stirratt2015self}. Physical solutions to monitor and encourage adherence such as electronic pill caps have shown promise in controlled settings but remain impractical for large-scale deployment due to high costs and implementation challenges~\citep{parker2007adherence,mauro2019effect}.
 
These measurement challenges take on new urgency as healthcare systems increasingly rely on machine learning (ML) models trained on electronic health records (EHRs) to guide treatment decisions~\citep{komorowski2018artificial, brugnara2020multimodal, zheng2021personalized, mroz2024predicting, yi2024development,shen2024data,chen2022clustering}. These machine learning models learn from historical patient data, which assume that prescribed treatments were actually taken. However, this introduces an implicit bias---models trained on non-adherent patients learn patterns that misrepresent true treatment effects. This implicit bias may degrade model performance and disproportionately impact underserved populations, who often face greater barriers to treatment adherence~\citep{bosworth2006racial, schober2021high}.

Recent advances in large language models (LLMs) have shown that LLMs can advance medical understanding by accurately extracting information from EHRs \citep{agrawal2022large, goel2023llmsaccelerateannotationmedical}. Instead of relying on self-reported treatment adherence from questionnaires and interviews, LLMs could serve as a powerful tool for identifying treatment non-adherence directly from EHRs. By analyzing rich but unstructured clinical notes, LLMs can detect documented instances of missed medications, unfilled prescriptions, and patient-reported barriers to adherence, enabling systematic assessment of treatment non-adherence across large patient populations.

In this study, we examine hypertension treatment non-adherence using EHR data from a large academic hospital by leveraging an LLM to analyze clinical notes, and further investigate its impact on causal inference and ML model performance (Figure~\ref{fig:diagram}). With a cohort of 3,623 patients, we identify 786 (21.7\%) cases of non-adherence and extract demographic and clinical factors that are statistically significant. Additionally, we apply topic modeling to clinical notes revealing underlying reasons for non-adherence.

To assess the effect of treatment non-adherence bias on downstream model performance, we perform causal inference and build predictive models using EHR with treatment records.  Our results show that ignoring treatment non-adherence bias could lead to reversed conclusions in treatment effect estimation, significantly degrade the performance of predictive models up to 5\%, and lead to unfair predictions. Furthermore, we highlight the importance of addressing treatment non-adherence bias by showing simply removing patient records with non-adherence, though reducing the size of the training dataset, could improve model performance and lead to fairer predictions.
\\
\newline
The contributions of this work include:
\begin{enumerate}
\item Conducting a large-scale study on treatment non-adherence in hypertension and identifying statistically significant factors associated with non-adherence.

\item Comparing LLM identification against physician annotations, LLMs perform well with 92\% accuracy, precision and recall.


\item Identifying patient-reported reasons for treatment non-adherence including side effects, forgetfulness, difficulties obtaining refills, etc.

\item Demonstrating the harmful impact of ignoring treatment non-adherence bias on causal inference and predictive modeling, leading to poorer performance and exacerbating racial disparities.

\end{enumerate}


\begin{figure}[t]
    \centering
    \includegraphics[width=\textwidth]{figures/Figure1.pdf}
\caption{Illustration of cohort selection, LLM non-adherence extraction, and non-adherence analysis. (a) We select 3,623 hypertension patients and pair their visits, with hypertension medication prescribed at the first visit and clinical notes extracted from the second. (b) These notes are then processed by an LLM to identify treatment non-adherence, with outputs validated through clinician annotations. (c) We further perform topic modeling to uncover reasons for non-adherence and assess the harmful impact of ignoring this bias on predictive modeling performance and treatment effect estimation.}
    \label{fig:diagram}
\end{figure}


\section{Related Work}
\label{sec:related_work}
\subsection{Treatment adherence analysis in hypertension}
Multiple studies have investigated treatment adherence among patients with hypertension \citep{boratas2018evaluation, uchmanowicz2018factors, algabbani2020treatment, najjuma2020adherence, schober2021high}. These studies are mainly cross-sectional, with a cohort of admitted patients collected at a fixed time point, and treatment adherence is typically measured through questionnaires and interviews. For instance, \citet{algabbani2020treatment} conducted a study in Saudi Arabia involving 306 hypertensive outpatients, finding that only 42.2\% of participants adhered to their antihypertensive medications. \citet{boratas2018evaluation} conducted a similar study of 147 hypertensive patients, identifying factors such as age and duration of hypertension to be significant. However, due to their reliance on questionnaire and interview data, they often have small sample sizes (e.g., less than 300 patients) and self-reporting bias \citep{adams1999evidence, stirratt2015self}, which limits their representativeness and can even lead to contradictory conclusions. In contrast, our work conducts the first large-scale analysis utilizing EHR, with a significantly larger sample size of 3,623 patients.







\subsection{Machine learning and treatment adherence}
Machine learning has been used to identify individual risk factors associated with treatment non-adherence \citep{koesmahargyo2020accuracy, gichuhi2023machine, burgess2023using}. \citet{gichuhi2023machine} developed ML algorithms and found SVM achieved 91.28\% accuracy in predicting tuberculosis treatment non-adherence. Instead of predicting treatment adherence, our work focuses on analyzing the impact of treatment non-adherence bias on downstream model performance. Other studies have applied natural language processing (NLP) to analyze surveys to better understand treatment non-adherence \citep{anglin2021natural, lin2022extraction, chan2024patient}. \citet{chan2024patient} applied NLP to free-text responses from questionnaires completed by type 2 diabetes patients, identifying key reasons for non-adherence. Unlike questionnaires, our work leverages treatment adherence information extracted from clinical notes using LLMs. Lastly, \citet{zhong2022use} applied ML while accounting for adherence information when analyzing treatment effects in a randomized controlled trial. To our knowledge, our study is the first to leverage LLMs for extracting treatment adherence information from clinical notes and evaluating its impact on downstream causal inference and predictive model performance.







\section{Study Design}
\label{sec:setup}

\subsection{Hypertension cohort selection}
We identified 15,002 patients with primary hypertension and extracted their primary care visits occurring on or after January 1st, 2019 following their initial hypertension diagnosis. To assess treatment adherence, consecutive visits for each patient were grouped into pairs. We focused on pairs where a hypertension medical prescription was provided during the first visit, and verified adherence at the second visit by extracting the associated clinical notes. 

Our analysis focuses on ten commonly prescribed hypertension medications: amlodipine, losartan, lisinopril, benazepril, carvedilol, hydralazine, hydrochlorothiazide, clonidine, spironolactone, and metoprolol \citep{heartTypesBlood}. Therefore, we excluded pairs in which the first visit lacked a medication record on this list, as well as pairs with missing or invalid notes during the second visit. We further focus on pairs where the interval between visits is between one month and one year. Lastly, we filtered out patients with unknown demographic information for the purpose of analysis. This resulted in a final cohort of 3,623 patients with 5,952 visit pairs. The cohort selection process is summarized in Appendix~\ref{apd:cohort}.

Demographic information, including sex, age, race, and marital status, was extracted from patient records. Four clinical factors were further derived from the EHR, many of which have been shown to be associated with hypertension non-adherence  \citep{boratas2018evaluation, algabbani2020treatment}. These factors include the duration between the two visits in the pair, the duration of hypertension, the number of primary care visits and the number of comorbidities. We quantified comorbidities using the Charlson Comorbidity Index (CCI) \citep{charlson1987new} and the Elixhauser Comorbidity Index (ECI) \citep{elixhauser1998comorbidity}, which condensed diagnoses into 17 and 31 well-defined comorbidity categories respectively. The demographic and clinical characteristics of the selected cohort are summarized in Table~\ref{tab:medication_adherence}. We detail the comorbidity categories along with other features used in the study in Appendix~\ref{apd:feature}.


\subsection{LLM configuration and prompt engineering}
We used the GPT-4o model \citep{openai2024gpt4ocard} (version 2024-05-13) via the HIPAA-compliant Microsoft Azure API, with the temperature set to 0 and all other parameters left at default. For each pair of visits, we provided the prescription record from the first visit and the clinical notes from the second visit to the model to assess adherence to the prescribed medication.

The model was prompted to identify instances of non-adherence, the type of non-adherence, and extract relevant sections from the notes. We used a zero-shot approach without additional training data or fine-tuning. We also implemented a second round of prompt validation by feeding the model's initial output back into the model, asking it to double-check its response. This additional step significantly reduced hallucinations. The prompt used in the study is provided in Appendix~\ref{apd:prompt}.

The cost for running all GPT-4o evaluations, including prompt development and inference was \$184.77, based on a cost of \$0.005 per 1,000 input tokens and \$0.015 per 1,000 output tokens.


\subsection{Physician validation of LLM detection}

To ensure the reliability of the LLM detection, we randomly selected 50 pairs labeled by the model as non-adherence and 50 pairs labeled as adherence for physician validation to assess accuracy. The gold standard was established through physician annotations conducted independently of the model's predictions. Overall, the model achieved an accuracy of 92\%, with four instances of physician-labeled non-adherence not detected and four adherent instances mislabeled as non-adherence(92\% precision and recall).

We further analyze discrepancies between the model and physician annotations, noting that some mismatches arise from ambiguous notes. For example, cases where patients restarted medication after hospitalization were marked as non-adherent by the LLM, since treatment was paused during hospitalization. Whereas physicians labeled them as adherent, considering the pause as a temporary interruption rather than true non-adherence.


\section{A Safety-Helpfulness Trade-off View of Jailbreak Defense}
\label{sec:trade_off_analysis}

\subsection{Formulating Defense as a Classification-Based Optimization}
Given a dataset \(\mathcal{D}\) comprising pairs of queries \(x_i\) and corresponding labels \(y_i \in \{0, 1\}\), where (\(y_i = 1\)) indicates a harmful query that should be refused, and (\(y_i = 0\)) denotes a benign query that should be complied with, as determined by human annotation. Let \(\theta\) represents a generative model, and \(\delta\) represents a defense method applied to the model or the input query. In the original generative task, the model under defense method \( \delta \) directly generates a response \(g(\theta, x; \delta)\) for query \(x_i\), which is then assessed as either a refusal or compliance.

In the classification formulation, the model is tasked with determining whether to refuse or comply with the input query, outputting a refusal probability \(p(\theta, x; \delta)\) under defense method \( \delta \) for the query \( x \). This format provides a more granular investigation of the model's preference, offering deeper insights compared to direct generative outputs.
Then the prediction \(f(\theta, x; \delta)\) is given by:
\begin{align*}
    f(\theta, x; \delta) = 
    \left\{
    \begin{array}{ll}
    0 & \text{if } p(\theta, x; \delta) < 0.5 \\
    1 & \text{if } p(\theta, x; \delta) \geq 0.5
    \end{array}
\right.
\end{align*}
The objective is to find the optimal defense \( \delta \) that minimizes the error between the true labels \(y_i\) and the defended model's predictions \(f(\theta, x; \delta)\), where \(\mathcal{L}(\cdot)\) is a loss function of the prediction error.
\begin{align*}
\min_{\delta} \mathbb{E}_{(x, y) \sim \mathcal{D}} \left[ \mathcal{L}(f(\theta, x; \delta), y) \right]
\end{align*}

This optimization objective can be decomposed into two components:
\begin{align*}
\begin{split}
\min_{\delta} \mathbb{E}_{(x, y) \sim \mathcal{D} \, | \, y = 1} \left[ \mathcal{L}(f(\theta, x; \delta), y) \right] \\
+ \min_{\delta} \mathbb{E}_{(x, y) \sim \mathcal{D} \, | \, y = 0} \left[ \mathcal{L}(f(\theta, x; \delta), y) \right]
\end{split}
\end{align*}
The first component focuses on the safety optimization, assessing whether the defense methods effectively enhance the model’s sensitivity to harmful inputs. The second component optimizes the defense mechanism to avoid overly constraining the model’s ability to identify benign inputs. This dual optimization captures the essential balance between safety and helpfulness.

\begin{figure*}[ht]
    \centering
    \begin{minipage}{0.33\textwidth} 
        \includegraphics[width=\linewidth]{Chapters/images/analysis_0.pdf}
        \subcaption{Baseline}
    \end{minipage}\hfill 
    \begin{minipage}{0.62\textwidth} 
        \includegraphics[width=\linewidth]{Chapters/images/analysis_1.pdf}
        \subcaption{Individual Defenses}
    \end{minipage}
    \caption{Representative results of individual defenses on refusal probabilities for harmful and benign queries. Compared to the baseline, system reminder and model optimization increase the mean refusal probabilities for both query types (\textbf{Safety Shift}). Query refactoring raises the mean refusal probability for harmful queries while lowering it for benign ones (\textbf{Harmfulness Discrimination}).}
    \label{fig:analysis results}
\end{figure*}

\subsection{Quantifying Defense using Probability-based Metrics}
\label{sec:defense_effects}
To quantify the impact of defense methods from the classification-based perspective, we introduce two relative metrics compared to the undefended model: Mean Shift and Distance Change.

\textbf{Mean Shift} measures how much the defense method \( \delta \) shifts the average refusal probabilities for input queries relative to the undefended model. We calculate mean shifts separately for harmful and benign queries as follows:
\begin{align*}
\begin{split}
\text{Mean\_Shift}_{\text{harmful}} &= \mathbb{E}_{x \in D_{\text{harmful}}}[p(\theta, x; \delta)] \\
&\quad - \mathbb{E}_{x \in D_{\text{harmful}}}[p(\theta, x)]
\end{split} \\
\begin{split}
\text{Mean\_Shift}_{\text{benign}} &= \mathbb{E}_{x \in D_{\text{benign}}}[p(\theta, x; \delta)] \\
&\quad - \mathbb{E}_{x \in D_{\text{benign}}}[p(\theta, x)]
\end{split}
\end{align*}
where \( \mathbb{E}_{x \in D}[p(\theta, x; \delta)] \) and \( \mathbb{E}_{x \in D}[p(\theta, x)] \) are the average refusal probabilities after and before applying the defense method $\delta$, respectively. A large shift in harmful data implies that the model becomes more safety-conscious, whereas a large shift in benign data suggests potential over-defense.  

\textbf{Distance Change} measures how the distance between the refusal probability distributions for harmful and benign data changes before and after applying the defense. Let \( P_{\text{harmful}} \) and \( P_{\text{benign}} \) represent the  refusal probability distributions for harmful and benign data before defense, and \( P^{\delta}_{\text{harmful}} \) and \( P^{\delta}_{\text{benign}} \) represent these distributions after defense. The distribution distance is defined as:
\begin{align*}
\begin{split}
\text{Distribution\_Distance} = &\ \text{Dist}(P_{\text{benign}}^{\delta}, P_{\text{harmful}}^{\delta}) \\
&- \text{Dist}(P_{\text{benign}}, P_{\text{harmful}})
\end{split}
\end{align*}
where \( \text{Dist}(\cdot, \cdot) \) denotes a distance metric between probability distributions, such as Jensen-Shannon divergence. A larger distance change indicates that the defense method improves the model's ability to distinguish between harmful and benign queries.


\begin{figure*}[ht]
    \centering
    \begin{minipage}{0.25\textwidth} 
        \includegraphics[width=\linewidth]{Chapters/images/analysis_0.pdf}
        \subcaption{Baseline}
    \end{minipage}\hfill 
    \begin{minipage}{0.75\textwidth} 
        \begin{minipage}{\linewidth}
            \includegraphics[width=\linewidth]{Chapters/images/analysis_2.pdf}
            \vspace{-6mm}
            \subcaption{Inter-Mechanism Ensembles}
        \end{minipage}
        \vfill
        \vspace{5pt}
        \begin{minipage}{\linewidth}
            \centering
            \includegraphics[width=0.75\linewidth]{Chapters/images/analysis_3.pdf}
            \vspace{-2mm}
            \subcaption{Intra-Mechanism Ensembles}
        \end{minipage}
    \end{minipage}
    \caption{Representative results for ensemble defenses. Inter-mechanism ensembles tend to reinforce the mechanism while intra-mechanism ensembles achieve a better trade-off between mechanisms.}
    \label{fig:analysis_results}
\end{figure*}

\subsection{Investigating Mechanisms of Defense Methods}
% While this approach may not fully capture the model's decision-making process in generative tasks as discussed in Section~\ref{sec:consistency}, it provides valuable insights into the effects of defense strategies on model behavior. 
To quantitatively analyze various defense methods, we prompt the model to classify whether it would comply with or refuse a given query, extracting the logits of refusal as its refusal probability. We conduct this analysis on the MM-SafetyBench dataset with LLaVA-1.5-13B model. The detailed prompt and analysis setup are provided in Appendix~\ref{sec:analysis_setup}. 

We specifically focus on four categories of internal jailbreak defenses described in Section~\ref{internal_defense_background}, and examine multiple methods for each category. A representative result is shown in Figure~\ref{fig:analysis results}, with the full set of results available in Appendix~\ref{sec:more_analyss_result}. Additional analyses on more LVLMs and LLMs are in Appendx~\ref{sec:extra_lvlm} and \ref{sec:extra_llm}. We also assess the consistency between the original generation task and the re-formulated classification task in Appendix~\ref{sec:consistency_appendix}.
Across these defense methods, two significant mechanisms emerge: Safety Shift and Harmfulness Discrimination, which explain how these defenses work.

\paragraph{Safety Shift} Compared to the baseline undefended model, both system reminder and model optimization defenses exhibit a significant mean shift across harmful and benign query subsets, without necessarily increasing the distance between the refusal probability distributions for these two groups.
This safety shift mechanism stems from the enhancement of model's general safety awareness, leading to a broad increase in refusal tendencies for both harmful and benign queries. However, such a conservative response to both types of queries can result in over-defense and does not significantly improve the model's ability to discriminate between harmful and benign inputs.

\paragraph{Harmfulness Discrimination} In contrast, query refactoring defenses either increases the refusal probabilities for harmful queries or decrease them for benign queries, leading to a consistent enlargement of the gap between the refusal probability distributions of these two subsets. This harmfulness discrimination mechanism enables better interpretation of the harmfulness within harmful queries or harmlessness within benign queries, thereby improving the distinction between them. However, the concealment of harmfulness within some queries can limit these improvements.

Additionally, noise injection demonstrate limited effectiveness, as indicated by insignificant changes in both the mean shift and distance change metrics. This is because it primarily targets attacks where noise is deliberately added to input queries, making it less effective in defending against general input queries without intentional noise.

\begin{table*}[t]
\caption{
Kendall-tau correlation coefficient between the pronunciation scores and the dysfluency/disfluency (absolute value).
Bigger is better.
For S3Ms, the best performance across layers is displayed.
}
\label{tab:main}
\centering
\resizebox{\textwidth}{!}{%
\begin{tabular}{cllcccc|cccc}
\toprule
& \multirow{3}{*}{Dataset} & \multirow{3}{*}{Feature} & \multicolumn{4}{c|}{Phoneme classifier-based} & \multicolumn{4}{c}{Out-of-distribution detector-based} \\
\cmidrule{4-7} \cmidrule{8-11}
& & & GMM- & NN- & DNN- & MaxLogit- & kNN & oSVM & p-oSVM & MixGoP \\
&& & GoP & GoP & GoP & GoP & & & & (Proposed) \\
\midrule
\multirow{12}{*}{\rotatebox[origin=c]{90}{Dysarthric speech}} & \multirow{4}{*}{UASpeech}
& MFCC & 0.428 & 0.410 & 0.361 & 0.430 & 0.418 & 0.107 & 0.105 & 0.182 \\
&& MelSpec & 0.209 & 0.172 & 0.242 & 0.214 & 0.101 & 0.099 & 0.086 & 0.039 \\
&& XLS-R & 0.552 & 0.553 & 0.548 & 0.547 & 0.559 & 0.354 & 0.247 & 0.602 \\
&& WavLM & 0.568 & 0.568 & 0.546 & 0.558 & 0.606 & 0.537 & 0.327 & \textbf{ 0.623 } \\
\cmidrule{2-11}
&\multirow{4}{*}{TORGO}
& MFCC & 0.406 & 0.345 & 0.391 & 0.406 & 0.347 & 0.169 & 0.105 & 0.282 \\
&& MelSpec & 0.271 & 0.262 & 0.150 & 0.271 & 0.287 & 0.211 & 0.241 & 0.196 \\
&& XLS-R & 0.677 & 0.674 & 0.641 & 0.675 & 0.704 & 0.586 & 0.536 & \textbf{ 0.713 } \\
&& WavLM & 0.682 & 0.681 & 0.633 & 0.681 & 0.703 & 0.671 & 0.621 & 0.707 \\
\cmidrule{2-11}
&\multirow{4}{*}{SSNCE}
& MFCC & 0.265 & 0.254 & 0.267 & 0.273 & 0.045 & 0.194 & 0.076 & 0.082 \\
&& MelSpec & 0.183 & 0.161 & 0.051 & 0.187 & 0.154 & 0.114 & 0.106 & 0.174 \\
&& XLS-R & 0.542 & 0.542 & 0.499 & 0.544 & 0.503 & 0.193 & 0.167 & 0.541 \\
&& WavLM & 0.547 & 0.547 & 0.486 & 0.547 & 0.523 & 0.358 & 0.234 & \textbf{ 0.553 } \\
\midrule
\multirow{9}{*}{\rotatebox[origin=c]{90}{Non-native speech}} & \multirow{5}{*}{speechocean762}
& MFCC & 0.390 & 0.375 & 0.255 & 0.405 & 0.322 & 0.202 & 0.111 & 0.126 \\
&& MelSpec & 0.214 & 0.064 & 0.232 & 0.229 & 0.111 & 0.109 & 0.071 & 0.004 \\
&& TDNN-F & 0.400 & 0.356 & 0.243 & 0.360 & 0.361 & 0.099 & 0.001 & 0.197 \\
&& XLS-R & 0.533 & 0.531 & 0.372 & 0.536 & 0.443 & 0.312 & 0.157 & 0.499 \\
&& WavLM & 0.535 & 0.533 & 0.380 & 0.534 & 0.432 & 0.395 & 0.173 & \textbf{ 0.539 } \\
\cmidrule{2-11}
&\multirow{4}{*}{L2-ARCTIC}
& MFCC & 0.136 & 0.141 & 0.119 & 0.119 & 0.042 & 0.004 & 0.034 & 0.043 \\
&& MelSpec & 0.049 & 0.039 & 0.032 & 0.032 & 0.022 & 0.003 & 0.027 & 0.010 \\
&& XLS-R & 0.243 & \textbf{ 0.312 } & 0.191 & 0.191 & 0.168 & 0.037 & 0.067 & 0.152 \\
&& WavLM & 0.240 & 0.269 & 0.196 & 0.196 & 0.189 & 0.082 & 0.078 & 0.182 \\
\bottomrule
\end{tabular}%
}
\end{table*}


% \begin{table*}[t]
% \caption{
% Kendall-tau correlation coefficient between the pronunciation scores and the dysfluency/disfluency (absolute value).
% Bigger is better.
% For S3Ms, the best performance across layers is displayed.
% }
% \label{tab:main}
% \centering
% \resizebox{\textwidth}{!}{%
% \begin{tabular}{cllcccc|cccc}
% \toprule
%  & \multirow{3}{*}{Dataset} & \multirow{3}{*}{Feature} & \multicolumn{4}{c|}{Phoneme classifier-based} & \multicolumn{4}{c}{Out-of-distribution detector-based} \\
% \cmidrule{4-7} \cmidrule{8-11}
% & & & GMM- & NN- & DNN- & MaxLogit- & kNN & oSVM & p-oSVM & MixGoP \\
%   && & GoP & GoP & GoP & GoP &  &  &  & (Proposed) \\
% \midrule
% \multirow{12}{*}{\rotatebox[origin=c]{90}{Dysarthric speech}} & \multirow{4}{*}{UASpeech}
% & MFCC    & 0.4284 & 0.4095 & 0.3609 & 0.4299 & 0.4180 & 0.1066 & 0.1050 & 0.1818 \\
% && MelSpec & 0.2093 & 0.1718 & 0.2422 & 0.2141 & 0.1012 & 0.0993 & 0.0863 & 0.0394 \\
% && XLS-R   & 0.5516 & 0.5528 & 0.5480 & 0.5471 & 0.5585 & 0.3538 & 0.2467 & 0.6017 \\
% && WavLM   & 0.5675 & 0.5679 & 0.5464 & 0.5579 & 0.6057 & 0.5369 & 0.3269 & \textbf{0.6233} \\
% \cmidrule{2-11}
% &\multirow{4}{*}{TORGO}
% & MFCC    & 0.4064 & 0.3447 & 0.3905 & 0.4062 & 0.3465 & 0.1686 & 0.1054 & 0.2817 \\
% && MelSpec & 0.2707 & 0.2619 & 0.1497 & 0.2711 & 0.2869 & 0.2114 & 0.2405 & 0.1959 \\
% && XLS-R   & 0.6769 & 0.6735 & 0.6405 & 0.6752 & 0.7044 & 0.5856 & 0.5362 & \textbf{0.7125} \\
% && WavLM   & 0.6821 & 0.6807 & 0.6328 & 0.6807 & 0.7030 & 0.6705 & 0.6211 & 0.7071 \\
% \cmidrule{2-11}
% &\multirow{4}{*}{SSNCE}
% & MFCC    & 0.2653 & 0.2539 & 0.2666 & 0.2727 & 0.0445 & 0.1944 & 0.0756 & 0.0819 \\
% && MelSpec & 0.1828 & 0.1609 & 0.0513 & 0.1869 & 0.1537 & 0.1139 & 0.1064 & 0.1739 \\
% && XLS-R & 0.5424 & 0.5416 & 0.4985 & 0.5435 & 0.5027 & 0.1926 & 0.1667 & 0.5414 \\
% && WavLM & 0.5469 & 0.5468 & 0.4860 & 0.5474 & 0.5230 & 0.3583 & 0.2336 & \textbf{0.5533} \\
% \midrule
% \multirow{9}{*}{\rotatebox[origin=c]{90}{Non-native speech}} & \multirow{5}{*}{speechocean762}
% & MFCC    & 0.3900 & 0.3754 & 0.2554 & 0.4046 & 0.3216 & 0.2021 & 0.1113 & 0.1256 \\
% && MelSpec & 0.2135 & 0.0641 & 0.2318 & 0.2290 & 0.1111 & 0.1090 & 0.0709 & 0.0038 \\
% && TDNN-F  & 0.4002 & 0.3560 & 0.2428 & 0.3595 & 0.3608 & 0.0990 & 0.0005 & 0.1970 \\
% && XLS-R   & 0.5329 & 0.5314 & 0.3716 & 0.5356 & 0.4430 & 0.3124 & 0.1574 & 0.4994 \\
% && WavLM   & 0.5348 & 0.5332 & 0.3800 & 0.5342 & 0.4320 & 0.3945 & 0.1732 & \textbf{0.5387} \\
% \cmidrule{2-11}
% &\multirow{4}{*}{L2-ARCTIC}
% & MFCC    & 0.1363 & 0.1409 & 0.1193 & 0.1193 & 0.0419 & 0.0035 & 0.0344 & 0.0433 \\
% && MelSpec & 0.0491 & 0.0389 & 0.0319 & 0.0319 & 0.0222 & 0.0034 & 0.0269 & 0.0103 \\
% && XLS-R   & 0.2425 & \textbf{0.3122} & 0.1913 & 0.1913 & 0.1681 & 0.0369 & 0.0672 & 0.1521 \\
% && WavLM   & 0.2401 & 0.2693 & 0.1964 & 0.1964 & 0.1892 & 0.0824 & 0.0781 & 0.1821 \\        
% \bottomrule
% \end{tabular}%
% }
% \end{table*}


\subsection{Exploring Defense Ensemble Strategies}

An effective defense should block harmful queries while preserving helpfulness for benign ones. Achieving this requires balancing safety shifts without over-defense and enhancing harmfulness discrimination. Since different defense methods impact model safety differently, we explore ensemble strategies to optimize this trade-off:

\begin{itemize}[itemsep=0.5pt, leftmargin=12pt, parsep=1pt, topsep=1pt]
    \item \textbf{Inter-Mechanism Ensemble} combines defenses operating the same mechanism, including safety shift ensembles and harmfulness discrimination ensembles.
    For safety shift ensembles, we combine multiple system reminder methods \textit{(SR++)} or combine system reminder with model optimization methods \textit{(SR+MO)}. For harmfulness discrimination ensemble, we combine multiple query refactoring methods \textit{(QR++)}.
    \item \textbf{Intra-Mechanism Ensemble} combines two defenses where one improves safety shift and the other enhances harmfulness discrimination. This includes ensembling query refactoring with system reminder methods \textit{(QR\textbar{}SR)} or with model optimization methods \textit{(QR\textbar{}MO)}.
\end{itemize}

% The inter-mechanism ensemble combines multiple safety shift methods, which can enhance overall safety by reinforcing conservative responses across different models. The intra-mechanism ensemble integrates a safety shift method and a harmfulness discrimination method, where the latter can help mitigate the refusal probability distribution shift of benign queries, thereby increase the distance between the two subsets.

For each ensemble strategy, we explore several variants using different specific methods. 
% A detailed description of these variants is provided in Appendix~\ref{sec:ensemble_strategy}. 
Representative results are shown in  Figure~\ref{fig:analysis_results}, with the full set of variant results available in Appendix~\ref{sec:more_analyss_result}.

We observe that inter-mechanism ensembles tend to strengthen a single defense mechanism. Safety shift ensembles like \textit{SR++} and \textit{SR+MO} further enhance model safety but exacerbate the loss of helpfulness. Conversely, harmfulness discrimination ensembles achieve a larger mean shift on benign queries towards compliance, making them better suited for situations where maintaining helpfulness is critical. 

In contrast, intra-mechanism ensembles combine the strengths of both mechanisms to achieve a more balanced trade-off. Specifically, \textit{QR\textbar{}SR} and \textit{QR\textbar{}MO} increase the refusal probability for harmful queries, while maintaining or even decreasing the refusal probability for benign queries, thereby improving the model's ability to distinguish between benign and harmful queries. This makes them a better choice for general scenarios where balancing safety and helpfulness is essential. 


\section{Impact of Non-Adherence Bias on Downstream Performance}
\label{sec:performance}
In this section, we evaluate the impact of treatment non-adherence bias on downstream model performance. We first define the outcome of interest and demonstrate the effect of non-adherence on the outcome in Section~\ref{sec:setup}. We then analyze its impact on treatment effect estimation for causal inference in Section~\ref{sec:causal_inference} and on ML model performance in Section~\ref{supervised_learning}.


\subsection{Treatment non-adherence leads to worse blood pressure outcomes}
\label{sec:setup}
Blood pressure reduction is the primary outcome when evaluating hypertension medications. To investigate the impact of treatment non-adherence on downstream performances, we begin by extracting pairs of visits where blood pressure measurements are available for both visits. To ensure a sufficient number of encounters for each medication, we focus on the top five most commonly prescribed medications: amlodipine, lisinopril, losartan, hydrochlorothiazide and metoprolol. We only include pairs where the duration between visits is less than six months to minimize the influence of other factors that could affect blood pressure over longer intervals. This results in 1732 pairs of encounters in total with 303 non-adherent pairs.

We begin by showcasing the impact of treatment non-adherence on blood pressure reduction between visits using t-tests. The results presented in Table~\ref{tab:treatment_adherence_outcome} indicate that non-adherence leads to smaller blood pressure reduction, with 1.96 mmHg less systolic reduction ($p=0.011$) and 3.93 mmHg less diastolic reduction ($p=0.001$) compared to adherence.

\begin{table}[h]
\centering
\footnotesize 
\caption{Results of the t-tests assessing the effect of treatment non-adherence on blood pressure reduction. Treatment non-adherence is statistically significant for both systolic and diastolic reduction, with non-adherence leading to smaller reductions in systolic and diastolic blood pressure.}
\label{tab:treatment_adherence_outcome}
\begin{tabular}{@{}l@{}c@{\hspace{5pt}}c@{\hspace{5pt}}c@{}}
\toprule
\textbf{Outcome} &  \textbf{Mean Difference} &  \textbf{95\% CI} &\textbf{$p$-value} \\
\midrule
Systolic Reduction &  -1.96 & (-3.47 to -0.46) &    \textbf{0.011} \\
Diastolic Reduction &  -3.93 & (-6.23 to -1.63) &    \textbf{0.001} \\
\bottomrule
\end{tabular}
\end{table}


\subsection{Causal inference for treatment effect estimation}
\label{sec:causal_inference}
Amlodipine and Lisinopril are the most commonly prescribed medications for hypertension, leading to numerous randomized controlled trials (RCTs) comparing their treatment effects \citep{cappuccio1993amlodipine, naidu2000evaluation}. However, due to the high cost of RCTs, various causal inference methods have been developed to estimate treatment effects from observational data \citep{pearl2009causality, austin2011introduction, shalit2017estimatingindividualtreatmenteffect, K_nzel_2019}. Among them, Inverse Probability Weighting (IPW) is one of the most widely used methods \citep{austin2011introduction}, providing an unbiased estimation of the Average Treatment Effect (ATE) by adjusting for confounding. We start by demonstrating the impact of treatment non-adherence bias on ATE estimation using IPW.
\\
\newline
\textbf{Experiment Setup.} 
We compare the ATE estimation with and without including treatment non-adherent data. Demographic and clinical factors are included as confounders and detailed in Appendix~\ref{apd:feature}. Patients prescribed lisinopril act as the control group, while those taking amlodipine are considered the treated group. The treatment effect is assessed based on the reduction in diastolic and systolic blood pressure between two visits.
\\
\newline
\textbf{Results.} 
The results are presented in Table~\ref{tab:ipw_ate}. Without filtering for non-adherent data, amlodipine lowers diastolic blood pressure by 1.75 mmHg but increases systolic blood pressure by 0.06 mm Hg compared to lisinopril. After excluding non-adherent data, amlodipine lowers diastolic blood pressure by 1.40 mmHg and also reduces systolic blood pressure by 0.11 mmHg compared to lisinopril. This result shows a reversal in the estimated treatment effect for systolic blood pressure reduction before and after excluding non-adherent data.
\begin{table}[h]
\centering
\footnotesize 
\caption{Estimated ATE of medication on blood pressure reduction using IPW. Notably, excluding non-adherent data reverses the conclusion on the treatment effect of systolic blood pressure reduction, causing the estimate to flip from -0.06 to 0.11 mmHg.}
\label{tab:ipw_ate}
\begin{tabular}{@{}lcc@{}}
\toprule
\textbf{Dataset} & \multicolumn{2}{c}{\textbf{Blood Pressure Reduction}} \\ 
\cmidrule(lr){2-3}
 & \textbf{Diastolic} & \textbf{Systolic} \\ 
\midrule
Full Dataset & 1.75 & -0.06 \\
Adherent Data Only & 1.40 & 0.11 \\
\bottomrule
\end{tabular}
\end{table}

\subsection{Supervised learning for treatment outcome prediction}
\label{supervised_learning}
We now demonstrate the impact of treatment non-adherence bias on predictive modeling performance. Following a common setup in the literature \citep{mroz2024predicting, yi2024development}, we use patients' EHR data with treatment prescriptions and blood pressure measurements from their first visit as covariates. The target to predict is whether the blood pressure will be normal at their second visit. Following the guidelines of the American Heart Association \citep{heartUnderstandingBlood}, we define normal blood pressure as having a systolic value of less than 120 and a diastolic value of less than 80. To evaluate model performance in predicting outcomes, we use 500 adherent samples as the test set in all subsequent experiments. We test exclusively on adherent patients since our goal is to evaluate model performance in scenarios where treatments are followed as prescribed, representing the intended clinical use case. A detailed description of the features used is provided in Appendix~\ref{apd:feature}.

\begin{figure}[t]
    \centering
    \includegraphics[width=\textwidth]{figures/treatment_non_adherence_ratio.png}
    \caption{Results of varying treatment non-adherence data percentage on model performance and fairness. Increasing the proportion of non-adherent data in the training set degrades predictive performance and increases fairness disparities between Black and non-Black patients, as measured by demographic parity and the equal odds criterion (true positive rate and false positive rate differences). Results are averaged over 100 seeds, with error bars representing the standard error of the mean.}
    \label{fig:treatment_non_adherence_ratio}
    \includegraphics[width=0.9\textwidth]{figures/remove_non_adherence_rf.png}
    \caption{Results of removing non-adherent data on model performance and fairness. The black curve represents training on 75\% of the full dataset, which only consists of adherent encounters. Removing non-adherent data improves model performance, with greater gains observed as sample size increases. It also decreases fairness disparities between Black and non-Black patients, as measured by demographic parity and the equal odds criterion (true positive rate and false positive rate differences). Results are averaged over 100 seeds, with error bars representing the standard error of the mean.}
    \label{fig:remove_non_adherence_rf}
\end{figure}


\subsubsection{Effect of varying treatment non-adherence data ratios on model performance and fairness}
\label{sec:very_ratio}
We begin by showing the harmful impact of treatment non-adherence bias by varying the proportion of non-adherent data in the training set.
\\
\newline
\textbf{Experiment Setup.}  
We fix the training set size at 300 and vary the proportion of non-adherent data in the training set across \(\{0\%, 10\%, 30\%, 50\%, 70\%, 90\%\}\) to evaluate the impact of treatment non-adherence bias on model performance. We train logistic regression and random forest models, both commonly used for modeling tabular EHR data and assess performance using AUROC on the test set.  Beyond performance, ensuring fair decision-making is also a critical consideration in healthcare. Let \( A \) denote the sensitive attribute (e.g., race), \( Y \) represent the true outcome, and \( \hat{Y} \) denote the predicted outcome. Demographic parity \citep{dwork2012fairness} difference measures the disparity in the likelihood of receiving a positive prediction between groups, i.e.,
\begin{equation*}
    |P(\hat{Y} = 1 \mid A = 1) - P(\hat{Y} = 1 \mid A = 0)|
\end{equation*}
Equal odds \citep{hardt2016equality} difference compares both true positive rates and false positive rates across groups, i.e., 
\begin{align*}
    |P(\hat{Y} = 1 \mid Y = 1, A = 1) - P(\hat{Y} = 1 \mid Y = 1, A = 0)|\\
    |P(\hat{Y} = 1 \mid Y = 0, A = 1) - P(\hat{Y} = 1 \mid Y = 0, A = 0)|
\end{align*}
We therefore measure the differences in demographic parity and equal odds across racial groups to assess fairness.  
\\
\newline
\textbf{Results.} 
We present the results in Figure~\ref{fig:treatment_non_adherence_ratio}, showing that increasing the percentage of non-adherent data in the training set degrades performance, with a 3\% drop for logistic regression and a 5\% drop for random forest in AUROC. Additionally, a higher proportion of non-adherent data increases fairness disparities between Black and non-Black patients under both the demographic parity and equal odds criteria. For instance, the false positive rate disparity of the random forest doubles, increasing from 0.125 to 0.25 as the percentage of non-adherent data increases. Similar trends are observed for other racial groups, and we provide full results in Appendix~\ref{apd:fairness}. These findings consistently highlight the harmful impact of treatment non-adherence bias.




\subsubsection{
Effect of removing non-adherent data on model performance and fairness}
\label{sec:drop_data}
We now emphasize the importance of addressing treatment non-adherence bias by showing that a simple approach to remove non-adherent data can improve predictive performance and lead to fairer predictions.
\\
\newline
\textbf{Experiment setup.}
We fix the non-adherent data ratio at 25\% and compare the performance of random forests trained on the entire dataset versus those trained only on adherent data by removing a quarter of data that are non-adherent. We report the test AUROC as well as demographic and equal odds differences while varying the full training set size in \(\{600, 800, 1000, 1200\}\) before removing non-adherent data.
\\
\newline
\textbf{Results.} 
The results are presented in Figure~\ref{fig:remove_non_adherence_rf}. While the traditional ML perspective suggests that more data generally improves performance, our findings show that using only the adherent 75\% of the data leads to better model performance, with the improvement becoming more significant as the sample size increases. For instance, with a training size of 1,200, the model achieves an AUROC of 0.695 when using all data, whereas dropping non-adherent data improves AUROC to 0.71. Additionally, we find removing non-adherent data reduces racial disparities between Black and non-Black patients, as both demographic parity and equal odds differences are consistently smaller across sample sizes. Similar trends are observed for other racial groups, and we provide full results in Appendix~\ref{apd:fairness}. These findings further highlight the importance of addressing treatment non-adherence bias to achieve better and fairer model performance.


\section{Discussion}

% Shift from findings to discussion
This study on robotic art explores human-machine relationships in creative processes.
It first contributes as an empirical description of artistic creativity in robotic art practice---an unconventional use of robots---examined through the artists' perspectives on their creative experiences. Our analysis reveals three facets of creativity in robotic art practices: the \textit{social}, \textit{material}, and \textit{temporal}. Creativity emerges from the co-constitution between artists, robots, audience, and environment in spatial-temporal dimensions, as illustrated in \autoref{PracticeDiagram}. Acknowledging the audience as an important actor reflects the social dimension in that creativity does not stem from the artists but from their interactions with the audience. Robots are the major material and technological actants characterizing creative practices, shaping the conditions for creativity to emerge. The axis of the temporal process signifies that the practice is situated within a time continuum, and the actors/actants and their relations shift over time. In this way, temporality constitutes another dimension of creativity in robotic art.

Accordingly, as the second contribution, this study outlines implications for \textit{socially informed}, \textit{material-attentive}, and \textit{process-oriented} creation with computing systems\footnote{For the sake of clarity, we mean ``creation with computing systems'' by three types of scenarios: human creator(s) create computing system(s) as the final artifact(s) (e.g., robots are artworks themselves); human creator(s) use computing system(s) to create the artifact(s) (e.g., robots create artworks as human planned); and human creator(s) and system(s) work in tandem to produce the artifact(s) (e.g., human-robot co-creation).} to facilitate creation practices. These insights can inform related HCI research on media/art creation, crafting, digital fabrication, and tangible computing.
In each following subsection, we present each implication with a grounding in corresponding findings from our study and relevant literature in HCI and adjacent fields on art, creativity, and creation.

\begin{figure*}[htbp]
    \centering
    \includegraphics[width=0.88\textwidth]{Writings/figure/PracticeDiagram.pdf}
    \caption{Actors/actants in robotic art practice and their interactive relations. Robotic art practice unfolds primarily in two spaces: the creation space where interactions happen mainly between artists and robots, and the exhibition space where interactions mostly involve audiences and robots. The two spaces constitute the ENVIRONMENT plane. Within the plane, directed arrows between the actors indicate the types of interaction. For example, the \textit{Design} arrow indicates that the artist designs the robot(s), and the \textit{Revise} arrow indicates that the robot(s) make the artist revise artistic ideas. All the actors/actants may also intra-act with the ENVIRONMENT. The actors/actants and their interactive relations may differ at different times along the axis of TEMPORAL PROCESS that is orthogonal to the plane.}
    \Description{This figure shows the actors/actants in robotic art practice and their interactive relations.}
    \label{PracticeDiagram}
\end{figure*}

\subsection{Socially Informed Creation}

% Introduce social aspect of distributed creativity
The sociality of creativity means that creativity is distributed among different human actors, commonly within the creators or between the creators and the audience. Glăveanu’s ethnographic study on Easter egg decoration in northern Romania~\cite{glaveanu_distributed_2014} showed that artisans anticipate how others might appreciate their work and adjust their creative decisions accordingly. Even in the absence of direct interaction, the audience’s potential responses become part of the creative process, as artisans imagine feedback and predict reactions. In this sense, the sociologist Katherine Giuffre argues that ``\textit{creative individuals are embedded within specific network contexts so that creativity itself, rather than being an individual personality characteristic is, instead, a collective phenomenon}''~\cite[p. 1]{giuffre2012collective}.

% Recall findings about audience feedback
We found that the practice of robotic art manifests this sociality as it involves, particularly artists and audiences. 
Our artists take audiences' reactions to their artwork as feedback and then revise the robots' functions and aesthetics accordingly. 
For example, as shown earlier, Robert added a protective fuse onto his robot because he expected that children would squeeze the springs together and cause a short circuit; Alex's enthusiasm and attention to the audience's imagination about his robots led him to new aesthetic designs of both the robots and the scene layouts. The artists may directly ask about the audience's judgment of quality but they often receive feedback just by observing the audience's reactions or sometimes by learning from the audience's imagination about the robots.
% Recall findings about audience's sociocultural expectations and codes
Meanwhile, our findings reveal that audience reception is not an individual matter but is often associated with their sociocultural codes, including shared cultural norms, beliefs, expectations, and aesthetic values. The audience can be seen as representatives of these broader cultural codes. For example, Mark and Robert observed that the animist tendency in some East Asian societies is associated with higher acceptance of and interest among the audience in intelligence and agency of robots and non-human entities. Such sociocultural contexts influence not only how audiences interpret the work but also how artists anticipate and respond to these perspectives in their creative process.

% Situate in HCI literature
A creative process, including creation and reception, is essentially a social activity. The second wave of creativity research in psychology has argued for creativity's dependency on sociocultural settings and group dynamics~\cite{sawyer2024explaining}. Recent discussions from creativity-support and social computing researchers also called for more attention to the social aspect of creativity~\cite{kato2023special, fischer2005beyond, fischer2009creativity}. There is a clear need to consider the audience when producing creative content. For instance, researchers studying video-creation support have examined audience preferences to inform system designs that align with these preferences~\cite{wang2024podreels}. Such work highlights how creative activities extend beyond individual creators to co-creators and heterogeneous audiences. Some HCI researchers conceptualize creativity as by large a socially constructed concept, perceived and determined by social groups~\cite{fischer2009creativity}. 
Prior HCI work examined the social aspects between art creators. For example, creators and performers in music and dance form social relationships through artifacts, making the final work a collaborative outcome~\cite{hsueh2019deconstructing}. There is also a system designed to support collaborative creation between artists~\cite{striner2022co}. However, the social creative process between creators and audience is less articulated in HCI. Jeon et al.'s work~\cite{jeon2019rituals} stands as an exception, suggesting that professional interactive art can involve evaluation with the audience in the creation stage. 
Another relevant approach in HCI involves enabling the general public to participate in co-creation alongside professional creators. ~\citet{matarasso2019restless}, for instance, promoted ``participatory art'' as ``\textit{the creation of an artwork by professional artists and non-professional artists working together}'' with non-professional artists referring to the general public engaged in the art-making process. Similarly, socially inclusive community-based art also considers target communities' perception of the artwork during creation~\cite{clark2016situated, clarke2014socially}. But like participatory design~\cite{schuler1993participatory}, these art projects aim for social justice more than creativity in the work~\cite{murray2024designing}, let alone that direct participation in art creation is not always feasible. Our findings suggest that feedback from the audience can lead to creative ideas, as well as that the feedback can be generative and remain low-effort for the audience.

Unlike conventional design feedback---which is typically expected to be specific, justified, and actionable~\cite{yen2024give, krishna2021ready}---the feedback that resonates with our artists is often implicit, creative, and generative. Such feedback may include audiences' imaginations stimulated by the work, personal and societal reflections, and even emotions, facial expressions, micro-actions, and observable behaviors following the art experience. Our artists gathered this implicit feedback not by posing evaluative questions, as commonly done in typical design processes (e.g., usability testing, think-aloud protocols), which seek to elicit clear, relatively structured responses. Instead, they closely observe the audience's reactions and interpret their subjective perceptions. This form of implicit feedback, while indirect, can lead to more creative ideas by embracing open, multifaceted interpretations of the work~\cite{sengers2006staying}. Computing systems for creation should better incorporate implicit feedback in addition to explicit ones from the audience into the creation process. Implicit feedback can be indirect, creative, inspirational, and heuristic about functions and aesthetics. A hypothetical instance of such design can be a system that helps creators perceive audiences' implicit reactions and perceptions and variously interpret them, for further iteration.

% Recall findings about audience interacting with robots as a performative art
Moreover, as seen in Robert and Daniel's experiences, the audience may participate in robotic live performances by interacting with the robots, who may change actions accordingly, triggering a loop of simultaneous mutual influence that makes the work performative and improvisational.
% Situate in HCI
HCI researchers explored performative and improvisational creation with machines, focusing on developing and evaluating systems with performative capabilities, including music improvisation with robots~\cite{hoffman2010shimon}, dance with virtual agents~\cite{jacob2015viewpoints, triebus2023precious}, and narrative theatre~\cite{magerko2011employing, piplica2012full}. \citet{kang2018intermodulation} discussed the improvisational nature of interactions between humans and computers and argued that an HCI researcher-designers' improvisation with the environment facilitates the emergence of creativity and knowledge. Designs of computing systems for creation can leverage performativity in service of creative experience. One possible direction could be to allow the audience to embed themselves in and interact with elements of static artwork in a virtual space, turning the exhibition into an improvisational on-site creation~\cite{zhou2023painterly}.
% Our new implication different from current discussion on perf and impr
While interactions with machines during performance are mostly physical or embodied, we posit that they can also be a \textit{symbolic engagement}. Alex's audience projected themselves and their personalities onto his robots, which established a symbolic relevance, generating creative imaginations. During exhibitions, East Asian audiences carried the animist views shaped by their sociocultural backgrounds, and robots, through the performance, were successful in symbolically matching the views, stimulating aesthetic satisfaction. Symbolic engagement resonates with what ~\citet{nam2014interactive} called the ``reference'' of the interactive installation performance to participants' sociocultural conditions.
As such, we propose that designers of computing systems for creation may consider establishing symbolic engagement between the produced artifacts and the audience as a way to enhance perceived creativity or enrich the creative experience. One example is an interactive installation, \textit{Boundary Functions}~\cite{snibbe1998}, which encourages viewers to reflect on their personal spaces while interacting with the installation and others. Another example is \textit{Blendie}, a voice-controlled blender that requires a user to ``speak'' the machine's language to use it. This interaction builds a symbolic connection between the user and the device, transforming the act of blending into a novel experience~\cite{dobson2004blendie}.


\subsection{Material-Attentive Creation}

% Intro paragraph to the importance of materiality for creative activities with machines and the end goal of this discussion--- design suggestions
The theory of distributed creativity by Glaveanu claims that creativity distributes across humans and materials, so the creation practice itself is inevitably shaped by objects~\cite{glaveanu_distributed_2014}. In his case of Easter egg decoration, materials are not passive objects but active participants in artistic creation; e.g., the egg decorators face challenges from color pigments not matching the shell, wax not melted at the desired temperature, to eggs that break at the last step of decoration; hence, materials often go against the decorators' intentions and influence future creative pathways~\cite{glaveanu_distributed_2014}.
Materials manifest specific properties, which afford certain uses of the materials while constraining others~\cite{leonardi2012materiality}. Our findings highlight the critical role of materiality in artistic practice, showing that artists intentionally arrange materials to enhance the creative values of their work.

% Materiality aspect One: physicality and embodiment
% Embodiment or physicality fascilitates creative interaction with machines
Robotic art relies on the material properties of robots and other objects. An apparent property of most materials is their physicality~\cite{leonardi2012materiality}, meaning they possess a tangible presence that enables interaction with other physical entities. Here, we consider physicality and embodiment interchangeable as computational creativity researchers have conceptualized~\cite{guckelsberger2021embodiment}.
% Recall findings on embodiment's value in making art
Our findings support both the conceptual and operational contributions of embodiment for creative activities. For the conceptual aspect, the embodied presence of robotic systems supports creative thinking for our artists, exemplary in Linda's case where she found new art ideas around the difference between human and robot bodies through bodily engagement with robots. 
For the operational aspect, the embodied nature of robotic artworks and their creation processes exhibit original aesthetics that are based on physics much different from disembodied works, e.g., embodied drawings by David's non-industrial robotic arms are dynamic due to physical movements and thus artistically pleasant, which is hard to replicate in simulated programs.

% References: embodied interaction, embodied cognition theories, tangible computing
These findings on embodiment of robotic art (Section \ref{f:emb}) closely relate to HCI's attention on embodied interaction as a way to leverage human bodies and environmental objects to expand disembodied user experiences. 
For example, as~\citet{hollan2000distributed} explained, a blind person's cane and a cell biologist's microscope as embodied materials are part of the distributed system of cognitive control, showing that cognition is distributed and embodied. 
Similarly, theories of embodied interaction in HCI explicate how bodily interactions shape perception, experience, and cognition~\cite{marshall2013introduction, antle2011workshop, antle2009body}, backed up by the framework of 4E cognition (embodied, embedded, enactive, and extended)~\cite{wheeler2005reconstructing, newen20184E}. 
Prior works suggest that creative activities with interactive machines rely on similar embodied cognitive mechanisms ~\cite{guckelsberger2021embodiment, malinin2019radical}, which are operationalized by tangible computing~\cite{hornecker2011role}. 
% References: embodiment's consequence in creation
As related to robots in creation, HCI researchers show that physicality or embodiment of robots in creation may lead to some beneficial outcomes, such as curiosity from the audience, feelings of co-presence, body engagement, and mutuality, which are hard to simulate through computer programs~\cite{dell2022ah, hoggenmueller2020woodie}. Embodied robotic motions convey emotional expressions and social cues that potentially enrich and facilitate creation activities like drawings~\cite{ariccia2022make, grinberg2023implicit, dietz2017human, santos2021motions}. Guckelsberger et al.~\cite{guckelsberger2021embodiment} showed in their review that embodiment-related constraints (e.g., the physical limitations of a moving robotic arm) can also stimulate creativity. These constraints push creators to develop new and useful movements, echoing the broader principle that encountering obstacles in forms or materials can lead to generative processes. This phenomenon is similarly observed in activities such as art and digital fabrication~\cite{devendorf2015being, hirsch2023nothing}. In co-drawing with robots, physical touch and textures of drawing materials made the artists prefer tangible mediums (e.g., pencils) than digital tools (e.g., tablets) that fall short in these respects~\cite{jansen2021exploring}.

% Transit to materiality aspect two
% Materiality aspect Two: malfunction as manifestation of unique materiality of robots
% Intro to materials of robots
Materiality plays a crucial role in the embodiment of robots, as the choice of materials fundamentally shapes the physical forms and properties. This focus on materials extends to art practices, where robots made with soft materials introduce new aesthetics and sensory experiences~\cite{jorgensen2019constructing, belling2021rhythm}, and the use of plants and soil in robotic printing creates unique visual effects~\cite{harmon2022living}. Following Leonardi's ~\cite{leonardi2012materiality} conceptualization of materiality, we refer to the materials of robots as encompassing physical and digital components---including the shell, hardware, mechanical parts, software, programs, data, and controllers---each significant to the artist's intent. ~\citet{nam2023dreams} found that the material constraints of robots can limit creative expression but simultaneously stimulate creativity when artists push the boundaries.

%-----maybe here the real "malfuction" start ------------------
% Move to introduce malfunctions as unique materiality

Even carefully designed, digital and mechanical components in robots are prone to errors or bugs in everyday runs, causing malfunctions or unexpected consequences. This reflects the unique materiality of robots as complex computing systems. From an engineering perspective, errors signal unreliability and must be eliminated, driving advancements in robotics---where error detection and recovery are central~\cite{gini1987monitoring}---as well as in digital fabrication, which prioritizes precision over creative exploration~\cite{yildirim2020digital}. % Recall findings on embracing malfunctions
However, material failures and accidents are inevitable, exemplifying what has been called the `craftsmanship of risk'~\cite{glaveanu_distributed_2014} in material art. For our artists, these risks are often creatively utilized and incorporated into their work: these moments of breakdown---whether physical or digital---become resources for new creative expression. Errors are anticipated and intentionally designed into the process and work of our artists. In some cases, such as for Alex, the entire concept of one of his works is machine errors.

% Situate in literature
Reports on how artists view errors within engineering and creation processes are dispersed throughout HCI literature. ~\citet{nam2023dreams} showed that the accumulation of ``contingency'' and ``accidents''---unexpected, serendipitous, and emergent events during art creation like errors---meaningfully constituted the final presentation of the artwork. Song and Paulos's concept of ``unmaking'' highlighted the values of material failures in enabling new aesthetics and creativity~\cite{song2021unmaking}. Kang et al.~\cite{kang2022electronicists, kang2023lady} introduced the notion of an ``error-engaged studio'' for design research in which errors in creative processes are identified, accommodated, and leveraged for their creative potential. Collectively, these works advocate for reframing errors from something to avoid to something to embrace and recognize. We want to push this further by arguing that errors can be intended and be part or sometimes entire of the design. Several artists, including participants from our study, have been deliberately seeking errors to formulate their designs. Roboticist Damith Herath recounted when he mistakenly programmed a motion sequence of a robotic arm, his collaborator, robotic artist Stelac responded with ``[W]e need to make more mistakes;'' as many mistakes were made, the initial pointless movements became beautiful, rendering the robot ``alive'' and ``seductive'' \cite{herath2016robots}. Similarly, AI artists sometimes look for program glitches to generate unusual styles and content~\cite{chang2023prompt}. Therefore, creators may not only passively accept errors but can actively seek and utilize them. Errors can be integral to the design itself---errors can \textit{be designed into} an artifact, and the design/idea of the artifact can be all about errors.

Thus, to focus on material-attentive creation---considering the creative arrangement of materials---we suggest exploring the embodiment and materiality of creation materials, objects, and environments to recognize their creative potential. %This perspective aligns with insights from professional digital fabrication practitioners, who advocate for systems that integrate support for machine settings and material properties~\cite{hirsch2023nothing}.
Specifically, we propose using a design method/probe that enables creators to realize both the conceptual and operational contributions of materiality. This approach may build on the material probe developed by~\citet{jung2010material}, which calls for exploring the materiality of digital artifacts. A material-attentive probe would enable creators to engage with diverse materials, objects, and environments through embodied interaction, encouraging them to speculate on material preferences and limitations, and to compare and contrast material qualities---insights that can inform creative decisions.
To accommodate, seek, and actively harness the creative potential of errors, we propose embracing failures, glitches, randomness, and malfunctions in computing systems as critical design materials---elements that creators can intentionally control and manipulate. By doing so, we can begin to systematically approach errors. For instance, as part of the design process, we may document how to replicate these errors and changes, allowing creators to explore them further at their discretion. This could include intentionally inducing errors or random changes to influence the creative process or outcomes.

\subsection{Process-Oriented Creation}

% Introduce the key idea: process itself embeds creative value and can be pursued as the goal of creation
As shown in our findings, the creation process itself embeds creative values and meanings, and experiencing the process can be pursued as the goal of creation with computing systems.
% Recall findings
For the robotic artists in our study, artistic values were often placed on the creation process rather than the outcome.  For example, in Alex's robotic live drawing performance, the drawing process is more important than the drawn pattern on canvas. Techniques used, decisions made, or stimuli received by robots during creation or exhibition reflect artistic ideas and nuanced thinking, as seen in Sophie's exploration of interactive decision-making in robotic drawing.

% Situate in HCI lit
Previous HCI work has touched on the value of the process of creation. ~\citet{bremers2024designing} shared a vignette where a robotic pen plotter simultaneously imitates the creator's drawing, serving as a material presence rather than a pragmatic co-creator; here the focus of the work is no longer the outcome but the process of drawing itself. ~\citet{devendorf2015reimagining} concluded that performative actions of digital fabrication systems, rather than the fabricated products themselves, convey artistic meanings tied to histories, public spaces, time, environments, audiences, and gestures. This emphasis on process is particularly significant for media such as improvisational theatre, where the creation itself is an integral part of the final work~\cite{o2011knowledge}. ~\citet{davis2016empirically} named their improvisational co-drawing robotic agents as ``casual creators,'' who are meant to creatively engage users and provide enjoyable creative experiences rather than necessarily helping users make a higher quality product. Shifting the focus from product to process and experiences \textit{in} creation may generate alternative creative meanings.

% Findings about process extends beyond creation
Our artists pointed out that even a ``finished'' artwork in an exhibition is not truly finished. A crack in Daniel's robotic artwork introduced a new artistic meaning, ultimately subverting the entire work. As the properties of the work change over time---whether due to the artist's intent, material characteristics, or environmental factors---the artwork evolves, revealing new aesthetics and meanings. % Situate in HCI lit
Based on these observations, we argue that creation processes should not be regarded as one-shot transactions, as creative artifacts, particularly physical ones, continue to change and generate artistic values. For instance, material wear and destruction bring unique aesthetics, often contrasting with the original form ~\cite{zoran2013hybrid}, and are seen as signs of mature use~\cite{giaccardi2014growing}.
Changes such as material failure, destruction, decay, and deformation---what~\citet{song2021unmaking} referred to as ``unmaking,'' a process that occurs after making---meaningfully transforms the original objects. Similarly, through Broken Probes, a process of assembling fractured objects, ~\citet{ikemiya2014broken} demonstrated that personal connections, reminiscence, and reflections related to material wear and breakage add new values to the objects. Drawing from Japanese philosophy Wabi-Sabi, ~\citet{tsaknaki2016expanding} reflected on the creeds of `Nothing lasts,' `Nothing is finished,' and `Nothing is perfect' and pointed to the impermanence, incompleteness, and imperfection of artifacts as a resource that designers, producers, and users can utilize to achieve long-term, improving, and richer interactive experience~\cite{tsaknaki2016things}. Insights from this study contribute to this line of thought by showing how robotic artists appreciate the aesthetics and meanings of temporal changes after the creation phase.

The findings underscore the need to reconceptualize creation as encompassing more than just the process aimed at producing a final product; it also includes what we term \textit{post-creation}. Distinct from repair, maintenance, or recycle, \textit{post-creation} entails anticipating and managing how an artifact evolves after its ``completion'' in the conventional sense. Specifically, we encourage creators to anticipate and strategically engage with the post-creation phase, considering potential changes to the artifact and their consequences for interactions with human users. For instance, during the creation process, creators may focus on possible material changes the artifact might undergo post-creation, allowing them to either mitigate or creatively exploit these potential changes. This expanded view of creation invites us to trace post-creation developments and to plan how our creative intentions can be embedded in its potential degradation, transformation, or evolution over time.

% A conclusion paragraph
We categorize the design implications into three aspects, but we do not suggest that a computing system must implement all simultaneously, nor that each aspect should be considered in isolation. Social interactions, such as those between artists and audiences, already presume the presence of material actants like robots, and these interactions inform future arrangements of materials. Thus the social and material aspects can be entangled and mutually constitutive as seen in sociomaterial practices~\cite{orlikowski2007sociomaterial, cheatle2015digital, rosner2012material}. The temporal aspect is orthogonal to the other aspects because both social interactions and material manifestations unfold and shift in a temporal continuum.




\clearpage
\bibliographystyle{apalike} %
\bibliography{references}

\appendix
\onecolumn

\section{Cohort Creation} \label{apd:cohort}
We detail the cohort selection in Figure~\ref{fig:cohort}.
\begin{figure}[h]
    \centering
    \includegraphics[width=0.45\linewidth]{figures/cohort_selection.jpg}
    \caption{Cohort Selection}
    \label{fig:cohort}
\end{figure}

\section{Feature Definitions}\label{apd:feature}
We use the following features throughout the study. The time between visits is excluded from causal inference and predictive modeling in Section~\ref{sec:performance}, as we instead restrict the analysis to a smaller time window. ECI and CCI are derived using medcodes package \citep{githubGitHubTopspinjmedcodes}.

\begin{table*}[h!]
    \centering
    \small
    \renewcommand{\arraystretch}{1.2} %
    \begin{tabular}{p{4cm} p{12cm}}
        \hline
        \textbf{Feature Name} & \textbf{Description} \\
        \hline
        Sex & The sex of the patients. \\
        Age & The age of the patients at the time of the visit. \\
        Race & The race of the patients. \\
        Marital Status & The marital status of the patients at the time of the visit. \\
        Time Between Visits & The time in days between the two visits in each pair. \\
        Charlson Comorbidity Index (CCI) & A measure of comorbidity based on the following conditions: myocardial infarction, congestive heart failure, peripheral vascular disease, cerebrovascular disease, dementia, chronic pulmonary disease, rheumatic disease, peptic ulcer disease, mild liver disease, diabetes without chronic complications, diabetes with chronic complications, hemiplegia/paraplegia, renal disease, any malignancy, moderate/severe liver disease, metastatic solid tumor, and AIDS/HIV. \\
        Elixhauser Comorbidity Index (ECI) & A measure of comorbidity based on the following conditions: cardiac arrhythmias, congestive heart failure, valvular disease, pulmonary circulation disorders, peripheral vascular disorders, hypertension (uncomplicated or complicated), paralysis, other neurological disorders, chronic pulmonary disease, diabetes (uncomplicated or complicated), hypothyroidism, renal failure, liver disease, peptic ulcer disease, AIDS/HIV, lymphoma, metastatic cancer, solid tumor without metastasis, rheumatoid arthritis, coagulopathy, obesity, weight loss, fluid and electrolyte disorders, blood loss anemia, deficiency anemia, alcohol abuse, drug abuse, psychoses, and depression. \\
        Duration of Hypertension & The duration (in years) between the onset of hypertension and the time of the visit. \\
        Number of Primary Care Visits & The number of primary care visits one year prior to the visit. \\
        \hline
    \end{tabular}
    \caption{Descriptions of features included in the study}
    \label{tab:features}
\end{table*}

\section{Prompt Description}\label{apd:prompt}
The prompt used in the study is given in Figure~\ref{fig:prompt}.
\begin{figure*}[h!]
    \centering
    \includegraphics[width=0.95\textwidth]{figures/prompt1.jpg} \\
    \includegraphics[width=0.95\textwidth]{figures/prompt2.jpg} \\
    \includegraphics[width=0.95\textwidth]{figures/prompt3.jpg}
    \caption{Prompt used in the study}
    \label{fig:prompt}
\end{figure*}

\section{Topic Modeling}\label{apd:topic}
Here we show the full BERT topic modeling results with key terms found in clusters and we summarize the topics in Figure~\ref{fig:reason}.

\begin{figure*}[h!]
    \centering
    \includegraphics[width=\textwidth]{figures/appendix_topic_keywords.png}
    \caption{BERT topic modeling identified five clusters of treatment non-adherence reasons. The most common words in each cluster are highlighted in the plot. Key topics identified include side effects, forgetfulness, failure to pick up medications, need for refills, and lost medications. When applying BERT topic modeling, we set a minimum cluster size of 15 notes and used UMAP with 5 components and 15 neighbors for dimensionality reduction.}
    \label{fig:appendix_topic}
\end{figure*}

\newpage
\section{Additional Results of Non-adherence Bias on Predictive Modeling}\label{apd:fairness}
Here, we present the complete results on racial disparities for the experiments in Section~\ref{supervised_learning}.
\begin{figure*}[h!]
    \centering
    \includegraphics[width=\textwidth]{figures/race_disparity_vary_na_ratio.png}
    \caption{Increasing the proportion of treatment non-adherent data in the training set increases the fairness disparity between different races as measured by demographic parity and the equal odds criterion. Results are averaged over 100 seeds, varying the sampling of the train and test sets. Error bars represent the standard error of the mean.}
    \label{fig:appendix_race_vary_na_ratio}
\end{figure*}
\newpage
\begin{figure*}[h!]
    \centering
    \includegraphics[width=\textwidth]{figures/race_disparity_vary_train_size.png}
    \caption{The black curve represents training on 75\% of the full dataset consisting of adherent encounters only. Removing treatment non-adherent data from the training set decreases fairness disparities as measured by demographic parity and the equal odds criterion particularly for Black and Asian. Results are averaged over 100 seeds, varying the sampling of the train and test sets. Error bars represent the standard error of the mean.}
    \label{fig:appendix_race_vary_sample_size}
\end{figure*}

\end{document}
