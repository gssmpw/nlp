% This must be in the first 5 lines to tell arXiv to use pdfLaTeX, which is strongly recommended.
\pdfoutput=1
% In particular, the hyperref package requires pdfLaTeX in order to break URLs across lines.

\documentclass[11pt]{article}
% Change "review" to "final" to generate the final (sometimes called camera-ready) version.
% Change to "preprint" to generate a non-anonymous version with page numbers.
% \usepackage[review]{acl}
\usepackage[]{acl}

% Standard package includes
\usepackage{times}
\usepackage{latexsym}

% For proper rendering and hyphenation of words containing Latin characters (including in bib files)
\usepackage[T1]{fontenc}
% For Vietnamese characters
% \usepackage[T5]{fontenc}
% See https://www.latex-project.org/help/documentation/encguide.pdf for other character sets

% This assumes your files are encoded as UTF8
\usepackage[utf8]{inputenc}

% This is not strictly necessary, and may be commented out,
% but it will improve the layout of the manuscript,
% and will typically save some space.
\usepackage{microtype}

% This is also not strictly necessary, and may be commented out.
% However, it will improve the aesthetics of text in
% the typewriter font.
\usepackage{inconsolata}

%Including images in your LaTeX document requires adding
%additional package(s)
\usepackage{graphicx}
\usepackage{tabularx} 
\usepackage{inconsolata}
\usepackage{pifont}
\usepackage{algorithm}
\usepackage{algorithmic}
\usepackage[namelimits]{amsmath} %数学公式
\usepackage{amssymb}             %数学公式
\usepackage{amsfonts}            %数学字体
\usepackage{mathrsfs}            %数学花体
\usepackage{graphicx} %插入图片的宏包
%\usepackage{float} %设置图片浮动位置的宏包
\usepackage{subfigure} %插入多图时用子图显示的宏包
\usepackage{xcolor}
\definecolor{darkgreen}{HTML}{3CB371}
\definecolor{darkred}{HTML}{7B7B7B}
\definecolor{ggreen}{HTML}{d4e7cf}
\definecolor{bblue}{HTML}{00bfff}
\definecolor{ppink}{HTML}{ff69b4}
\newcommand{\modi}{\textcolor{black}}
\usepackage{makecell}
\usepackage{booktabs}
\usepackage{multirow}
\usepackage{enumitem} % 导入enumitem包
\usepackage{CJKutf8}

\usepackage{newfloat}
\usepackage{listings}
\usepackage{subcaption}
\usepackage{xspace}
\usepackage{graphicx}
\usepackage{caption}
% \usepackage[ngerman]{babel}


%used for DPO formular
\usepackage{amsmath}
\usepackage{mathtools}
\newcommand{\piref}{\pi_\text{ref}}
\newcommand{\pisft}{\pi^\text{SFT}} %


\definecolor{green}{RGB}{36, 214, 36}
\definecolor{red}{RGB}{235, 30, 30}
\newcommand{\cmark}{\ding{51}} % 勾
\newcommand{\xmark}{\ding{55}} % 叉

\usepackage[utf8]{inputenc}
\usepackage[T1]{fontenc}

\usepackage{CJKutf8}
\usepackage{booktabs} 

\usepackage{CJK}
\usepackage{xcolor}
\usepackage{ulem}
\newcommand\SJ[1]{\textcolor{red}{\{SJ: \begin{CJK*}{UTF8}{gkai} #1 \end{CJK*} \}}}
\newcommand\nsout[1]{\textcolor{red}{\sout{#1}}}
\usepackage{booktabs}
\usepackage{tcolorbox}

\usepackage{soul}
\newcommand{\kaiti}[1]{\begin{CJK*}{UTF8}{gkai} #1 \end{CJK*}}
\soulregister{\kaiti}7
\definecolor{MyYellow}{rgb}{254, 246, 170}
\definecolor{MyBlue}{rgb}{170, 217, 251}

\newcommand{\greencheck}{\textcolor{green}{\ding{51}}}
\newcommand{\redcross}{\textcolor{red}{\ding{55}}}

\newcommand*{\affaddr}[1]{#1} % No op here. Customize it for different styles.
\newcommand*{\affmark}[1][*]{\textsuperscript{#1}}
\newcommand*{\email}[1]{\texttt{#1}}


\usepackage{color}
\usepackage{colortbl}
%gray
\definecolor{g1}{RGB}{232,232,232}
\definecolor{g2}{RGB}{207,207,207}
\definecolor{g3}{RGB}{181,181,181}
\definecolor{g4}{RGB}{156,156,156}

%bleu
\definecolor{b1}{RGB}{217,217,254}
\definecolor{b2}{RGB}{198,198,253}
\definecolor{b3}{RGB}{180,180,252}
\definecolor{b4}{RGB}{162,162,252}
\definecolor{b5}{RGB}{136,136,255}

%red
\definecolor{r1}{RGB}{254,236,236}
\definecolor{r2}{RGB}{254,217,217}
\definecolor{r3}{RGB}{253,198,198}
\definecolor{r4}{RGB}{253,180,180}
\definecolor{r5}{RGB}{252,128,127}


\title{Edit Once, Update Everywhere: A Simple Framework for Cross-Lingual Knowledge Synchronization in LLMs}

\author{%
  Yuchen Wu$^{1}$,
  Liang Ding$^{2}$\thanks{Correspond to Liang Ding \texttt{liangding.liam@gmail.com}},
  Li Shen$^{3}$,
  Dacheng Tao$^{4}$\\
  $^{1}$Shanghai Jiao Tong University
  $^{2}$The University of Sydney\\
  $^{3}$Sun Yatsen University
  $^{4}$Nanyang Technological University
  }

\begin{document}
\maketitle
\begin{abstract}

Knowledge editing allows for efficiently adapting large language models (LLMs) to new information or corrections without requiring full retraining. However, prior methods typically focus on single-language or basic multilingual editing, failing to achieve true cross-linguistic knowledge synchronization. To address this, we present a simple and practical state-of-the-art (SOTA) recipe \textit{Cross-Lingual Knowledge Democracy Edit} (X-KDE), designed to propagate knowledge from a dominant language to other languages effectively. Our X-KDE comprises two stages: (i) Cross-lingual Edition Instruction Tuning (XE-IT), which fine-tunes the model on a curated parallel dataset to modify in-scope knowledge while preserving unrelated information, and (ii) Target-language Preference Optimization (TL-PO), which applies advanced optimization techniques to ensure consistency across languages, fostering the transfer of updates. Additionally, we contribute a high-quality, cross-lingual dataset, specifically designed to enhance knowledge transfer across languages. Extensive experiments on the Bi-ZsRE and MzsRE benchmarks show that X-KDE significantly enhances cross-lingual performance, achieving an average improvement of +8.19\%, while maintaining high accuracy in monolingual settings.


\end{abstract}

\section{Introduction}\label{sec:intro}

In computational finance, Monte Carlo simulations are used extensively to estimate the expected value of financial payoffs based on the solution of stochastic differential equations (SDEs) which model the evolution of stock prices, interest rates, exchange rates and other quantities \cite{glasserman04}.  Monte Carlo methods are very general and flexible, but for high accuracy it requires generating a large number of costly SDE path approximations, which has motivated research into a number of variance reduction or, equivalently, cost reduction techniques. One such method is
Multilevel Monte Carlo (MLMC), which was proposed in \cite{GILES2008} and was adapted for various applications that are summarised in \cite{Giles_overview17} and successfully combined with other methods such as quasi-Monte Carlo methods. The main idea of MLMC is to approximate the payoff using different time stepping resolutions when numerically solving the underlying SDE and to generate an optimal number of samples on each level, such that the overall computational cost is minimised subject to the desired bound on the variance. %, such that the total computational cost is minimised. 
The computational savings come from the fact that most samples are computed on the coarser levels and hence are less expensive while only a few samples from the finest levels are required \cite{GILES2008}.


Among the directions in which the computational cost 
of MLMC methods could further be reduced, an important avenue is the use of lower precision calculations, especially for the first Monte Carlo levels where the targeted accuracy is relatively low. 
 An overview of the research on mixed precision for the standard Monte Carlo (MC) framework is provided in \cite{ChowMixedPrecisionStandardMC} but only a few references study the potential of low precision computation in the MLMC framework \cite{Rounding_error_oliver}. To the best of our knowledge, the only MLMC framework with customised precision in the literature is \cite{brugger2014mixed}, but they use a uniform precision for all operations on each Monte Carlo level instead of optimising 
 the precision of each intermediary variable to reduce as much as possible the cost of path generation.
 
An important motivation for an MLMC framework with variable precision would be performing the low precision computations on reconfigurable hardware devices such as Field Programmable Gate Arrays (FPGAs). FPGAs contain customizable logic blocks and connectors that make it easy to adapt the digital circuit architecture for a specific application, leading to a highly parallel and optimised implementation. Therefore they are successfully exploited in applications that require high speed and have high computational workload, such as signal processing \cite{woods2008fpga}, and real time applications like high frequency trading \cite{HFT1,HFT2}. That is why a number of previous works in hardware architecture design implemented the MLMC algorithm to price financial options using FPGAs as accelerators, which resulted in improved speed and power efficiency compared to full CPU architectures \cite{Schryver2013AMM}. The paper \cite{lindsey2016domain} also proposed 
a Domain Specific Language to automate the configuration of FPGAs for this specific application. However, only \cite{brugger2014mixed} proposed a heuristic to reduce the precision in calculations.

In addition, all aforementioned works considered that the random number generation (RNG) is performed in single or double precision. Yet in most cases an important portion of the workload in the overall MLMC simulation comes from the RNG and in \cite{brugger2014mixed} this limited the total computational savings.
To reduce the cost of MLMC simulations in particular those based on the Geometric Brownian Motion (GBM), \cite{approximateICDF_Oliver, NestedOliver} have proposed to use approximate random numbers that are generated by applying an approximation of the inverse CDF to uniform random numbers. In \cite{NestedOliver}, the authors proposed a way to integrate these lower precision random variables into a \textit{nested} MLMC framework and completed a numerical analysis to bound the resulting error at each MC level by a product of the time step and the error in the random number approximation. The same authors show in \cite{approximateICDF_Oliver} that using approximate random variables reduces the cost of path generation by a factor 7.


In this paper we propose a nested MLMC framework that combines the use of approximate random normal variables and lower precision calculations to reduce the computational cost of MLMC even further than \cite{brugger2014mixed,NestedOliver}. We illustrate the efficiency of our framework in Matlab, after making several assumptions on the cost of operations and size of the errors that we carefully justify. We focus on the case of GBM and use the approximate RNG methods presented in \cite{approximateICDF_Oliver} as well as a new slightly modified method that combines CDF inversion and the central limit theorem. To choose the precision of the variables in the low precision path generation, we introduce a novel method to optimise the bit-widths. This optimisation is performed before the main path generation loop is executed and is based on a linear model of the payoff error  
due to rounding when computing in low precision. The error model relies on algorithmic differentiation in a similar manner to \cite{unifying-bwoptim,bitwidth-AD,ADAPT}. The bit-width optimisation procedure can be performed off-line, so this stage can be excluded from the on-line time complexity of our framework. The user specified desired accuracy is then enforced by calculating on-line the number of samples that need to be generated.

In terms of hardware design, we suggest implementing the low precision path generation on FPGAs and the full-precision ones on a CPU or GPU. 
The FPGA offers enough flexibility to define a separate bit-width for every variable in the low precision path generation, and can be reconfigured periodically to update the bit-widths when the market parameters have changed considerably. 


The paper is organized as follows : \Cref{sec:MLMC} introduces MLMC and nested MLMC to make clear the estimator that is implemented in our framework. Then in \Cref{sec:RNG} we detail the methods that could be used to obtain approximate random normally distributed numbers very cheaply for the low precision path generation. In \Cref{sec:error_model} and \Cref{sec:costModel} we propose an error model and a cost model (resp.) that we then use to formulate the optimisation problem that is solved to obtain the optimal bit-widths of fixed point variables in \Cref{sec:optimisation}. Finally we summarise our results and future directions in \Cref{sec:conclusion}.




\begin{figure*}[t]
\vskip 0.2in
\begin{center}
\centerline{\includegraphics[width=\textwidth]{Figures/pipeline-vlm-v4.pdf}}
\caption{Overview of our data-aware preference optimization. For each preference instance: (1) We first break the preferred and rejected response into sub-sentences by prompting a large language model (LLM); 
(2) Next, we estimate the similarity scores between each sub-sentence and the given image using the CLIP classifier, and then calculate the differences between the preferred and rejected response as the hardness of the data; 
(3) Finally, we incorporate the estimated hardness into the preference optimization process by modifying $\beta$ in Equ~\eqref{equ:dpo}, allowing the model to adjust based on the data hardness.}
\label{fig:pipleine-vlm}
\end{center}
\vskip -0.2in
\end{figure*}


\section{Preliminary}
\label{sec:preliminary}
In this section, we briefly review the MLLM preference learning procedure, which starts by sampling pairwise preference data with a supervised fine-turned (SFT) model, and then optimizes on such preference data. Specifically, we categorize this process into the following aspects:

\noindent \textbf{Supervised Fine-Tuning (SFT).}
Preference learning of an MLLM $\bm{\pi}$ begins with an SFT model $\bm{\pi}_{\text{SFT}}$. Concretely, the SFT process fine-tunes the pre-trained MLLM model with millions of multi-modal question-answer pairs to align LLM with multi-modal tasks. 
After this process, we construct preference data by sampling pair-wise preference responses from $\bm{\pi}_{\mathrm{SFT}}$, formalized as $(y_w, y_l) \sim \bm{\pi}_{\mathrm{SFT}}(y|x,\mathcal{I})$, where $(\mathcal{I}$ denotes the image and $x$ is the prompt question. 
Meanwhile, $(y_w, y_l)$ are labeled as preferred and less preferred responses by humans, formalized as $(y_w \succ  y_l | \mathcal{I}, x)$.

\noindent \textbf{RLHF with Reward Models.}
Given pair-wise preference data $(y_w, y_l) \sim \bm{\pi}_{\mathrm{SFT}}(y|x,\mathcal{I})$, the preference learning process can be described in 2 stages: reward modeling and preference optimization. 
Specifically, the reward model $r_{\bm{\theta}}(y|\mathcal{I}, x)$ is defined to rank the model responses by learning to distinguish $y_w$ from $y_l$, and the preference optimization aims to distill the preference knowledge into MLLM. 
To learn a reward model, pioneering work \cite{rlhf} employs the Bradley-Terry model \cite{BT_model} to model the pair-wise preference distribution as:
\begin{equation}
\resizebox{.9\hsize}{!}{
\begin{math}
\begin{aligned}
    \mathrm{P}(y_w \succ  y_l|\mathcal{I}, x) & =  \sigma(r^{*}(y_w|\mathcal{I}, x)- (r^{*}(y_l|\mathcal{I}, x)) \\
     & = \frac{\mathrm{exp}(r^{*}(y_w|\mathcal{I}, x))}{\mathrm{exp}(r^{*}(y_w|\mathcal{I}, x))+\mathrm{exp}(r^{*}(y_l|\mathcal{I}, x))}.
\end{aligned}
\end{math}
}
\end{equation}

Thus, the learning process can be achieved by minimizing the negative log-likelihood $-\mathrm{logP}(y_w \succ y_l|\mathcal{I}, x)$ over the preference data with the parametrized reward model $r_{\bm{\phi}}(y_w|\mathcal{I}, x)$ initialized as $\bm{\pi}_{\mathrm{SFT}}$ with a simple linear layer to produce reward prediction. 
With the well-optimized reward model $r_{\phi}^{*}(y|\mathcal{I}, x)$, prior work \cite{rlhf} proposes to employ policy optimization algorithms in RL such as PPO \cite{PPO} to maximize the learned reward with KL-penalty, which can be formalized as:
\begin{equation}
\label{equ:ppo}
\begin{aligned}
    \underset{\bm{\pi}_{\theta}}{\text{max}} & \  \mathbf{E}_{(\mathcal{I},x) \sim \mathcal{D}, y \sim \bm{\pi}_{\theta}(\cdot|\mathcal{I}, x)} [r_{\phi}^{*}(y|\mathcal{I}, x)] \\
    & -\beta \mathbb{D}_{\mathbf{KL}}[\bm{\pi}_{\theta}(y|\mathcal{I},x)||\bm{\pi}_{\text{ref}}(y|\mathcal{I},x)], 
\end{aligned}
\end{equation}
where the fixed reference model $\bm{\pi}_{\text{ref}}$ is parameterized as $\bm{\pi}_{\text{SFT}}$, and the hyper-parameter $\beta$ controls the deviation of $\bm{\pi}_{\theta}$ from $\bm{\pi}_{\text{ref}}$ during the optimization process.

\noindent \textbf{Direct Preference Optimization (DPO).}
To relieve the high computational complexity of reward training in RLHF, DPO \cite{DPO} is proposed, which provides a simple way to directly optimize $\bm{\pi}_{\theta}$ with the pair-wise preference data, without parametrized reward model. Specifically, the DPO loss can be described as:
\begin{equation}
\label{equ:dpo}
\begin{aligned}
    \mathcal{L}_{\mathrm{dpo}} = - \bm{\mathrm{E}}_{(\mathcal{I},x, y_{w}, y_{l})} [ {\log \sigma}( & \beta \log \frac{{\pi}_{\bm{\theta}}(y_{w}|\mathcal{I},x)}{{\pi}_{\mathrm{ref}}(y_{w}|\mathcal{I},x)} \\
    - & \beta \log \frac{{\pi}_{\bm{\theta}}(y_{l}|\mathcal{I},x)}{{\pi}_{\mathrm{ref}}(y_{l}|\mathcal{I},x)}) ].
\end{aligned}
\end{equation}



\section{Methodology}
\paragraph{Preliminaries.}
We primarily focus on the homologous model merging, in which $\boldsymbol{\theta}_i$ all come from the same base model $\boldsymbol{\theta}_{\rm{base}}$. Given $K$ tasks $\{T_1,T_2,\cdots,T_K\}$ and $K$ corresponding fine-tuned models with parameters $\{\boldsymbol{\theta}_1,\boldsymbol{\theta}_2,\cdots,\boldsymbol{\theta}_K\}$, model merging aims to combine $K$ fine-tuned models into one single model simultaneously performing on $\{T_1,T_2,\cdots,T_K\}$ without post-training~\cite{method_p1_1,method_p1_2}.
Task vector~\cite{ilharco2023editing,yang2024adamerging} is a key element in merging method which could enhances the base model‘s ability or enable the model to handle other tasks. Specifically, for task $T_i$, the task vector $\boldsymbol\tau_i\in \mathbb{R}^D$ is defined as the vector obtained by subtracting the SFT weights $\boldsymbol{\theta}_i$ from the base model weight
$\boldsymbol{\theta}_{\rm{base}}$, \emph{i.e.}, $\boldsymbol\tau_i=\boldsymbol{\theta}_i-\boldsymbol{\theta}_{\rm{base}}$. The merged model could be denoted as $\boldsymbol{\theta}_m=\boldsymbol{\theta}_{\rm{base}}+\sum_i \lambda_i\boldsymbol{\tau}_i$, which $\lambda_i$ is the scaling factor measuring the importance of task vector. For clarification, we also denote the neuron set in $\boldsymbol{\theta}_i$ as $\mathcal{N}_i$, the neuron set in $\boldsymbol{\tau}_i$ as $\mathcal{T}_i$.



\begin{algorithm}[!ht]
    \caption{LED-Merging}
    \label{alg1}
    \begin{algorithmic}[1]
        \REQUIRE  base model $\boldsymbol{\theta}_{\rm{base}}$, SFT models $\{\boldsymbol{\theta}_{i}\mid i\in [K]\}$, mask ratios \{$r_{i} \mid i\in [K]\}$, scaling factors $\{\lambda_i\mid i\in[K]\}$, location datasets $\{\mathcal{X}_{i}\mid i\in[K]\}$
        \ENSURE merged parameter $\boldsymbol{\theta}_{m}$
        \STATE $\mathcal{M}\leftarrow\phi$
        \STATE $\boldsymbol{\theta}_{m}\leftarrow \boldsymbol{\theta}_{\rm{base}}$
        \FOR{$i\in [K]$}
        \STATE $I(\boldsymbol{\theta}_i)=\mathbb{E}_{x\sim \mathcal{X}_i}|\boldsymbol{\theta}_{i}\odot \nabla_{\boldsymbol{\theta}_i}\mathcal{L}(x)|$
        \STATE $I(\boldsymbol{\theta}_{\rm{base}})=\mathbb{E}_{x\sim \mathcal{X}_i}|\boldsymbol{\theta}_{\rm{base}}\odot \nabla_{\boldsymbol{\theta}_{\rm{base}}}\mathcal{L}(x)|$
        
        \STATE calculate $\mathcal{T}^{r_i}_{i}$ following Equation \ref{vote}
        \STATE  $\mathcal{M}\leftarrow \mathcal{M}\cup\{\mathcal{T}^{r_i}_i\}$
       
        
   
        
        
        \ENDFOR  
        \FOR{$i\in [K]$}
        
        \STATE calculate $\text{Disjoint}(\mathcal{T}_i^{r_i})$ use Equation~\ref{disjoint_safety}
        \STATE $\boldsymbol{m}_i \leftarrow \boldsymbol{0}$
        \FOR{$d\in \mathcal{T}_i^{r_i}$}
        \STATE $\boldsymbol{m}_{i,d}=1$
        \ENDFOR
        \STATE $\boldsymbol{\theta}_{m}\leftarrow \boldsymbol{\theta}_{m}+\lambda_i \boldsymbol{\tau}_i\odot \boldsymbol{m}_{i}$
        \ENDFOR
    \end{algorithmic}
\end{algorithm}
    %\vspace{-5pt}
\begin{figure*}[h!]
    \centering
    \includegraphics[width=\linewidth]{figs/pipeline_v2.pdf}
    \vspace{-40mm}
    \caption{Overview of our two-stage training pipeline {\ours}.}
    \label{fig:pipeline}
\end{figure*}


\paragraph{LED-Merging: Location, Election, and Disjoint Merging}
To address the neuron misidentification and interference issues in existing model merging methods, we propose LED-Merging (Location, Election, and Disjoint Merging). Specifically, previous studies \cite{modelstock, ilharco2023editing, tiesmerging} fail to accurately identify safety-related neurons in task vectors with a single magnitude score, namely \textit{neuron misidentification}. Meanwhile, there exists an interference between safety-related and utility-related task vector neurons during the merging process, namely \textit{neuron interference}. To address neuron misidentification, we first locate important neurons both in the base and fine-tuned models and then elect neurons from the task vector considering these two scores together. Subsequently, to mitigate the interference, we introduce a disjoint step, isolating these important neurons so that they influence different base neurons. The whole process is illustrated in Figure~\ref{fig:method}. 




In the location and election step, we consider the importance score from base and fine-tuned models simultaneously to locate task-specific neurons. In this way, it is more accurate than relying on the magnitude score alone because task-specific neurons with high importance score in the fine-tuned model may not necessarily score high in the base model, and vice versa.

{\textbf{Location}}.  We first calculate importance scores for each neuron in a base/fine-tuned model. Given a location dataset $\mathcal{X}_i=\{(x,y)_k\}$, where $x$ is the question and $y$ is the answer, we calculate the importance scores for the weight $\boldsymbol{\theta}_i\in\mathbb{R}^D$ in any  layer as follows~\cite{snip,spareseGPT,sun2024a}:
\begin{equation}
    I(\boldsymbol{\theta}_i)=\mathbb{E}_{x\sim \mathcal{X}_i}[\boldsymbol{\theta}_i\odot \nabla _{\boldsymbol{\theta}_i}\mathcal{L}(x)],
    \label{location}
\end{equation}
which $\mathcal{L}(x)=-\log p(y\mid x)$ is the conditional negative log-likelihood loss. We choose the SNIP score~\cite{snip} because it balances computational efficiency and performance~\cite{cq}. Please refer to Sec.~\ref{sec:ablation} for the comparison between different location methods. After computing importance scores, we choose top-$r_i$ neurons as the important neuron subset $\mathcal{N}_{i}^{r_i}$ from $I(\boldsymbol{\theta}_i)$.
 
 % After computing locating scores, we select the neurons scoring both high in base and fine-tuned models as important neurons in task vectors. Then in the disjoint step,  with preventing  polysemantic neurons  from receiving gradient updates towards different directions,
 % we use set difference to isolate the safety   and utility-related neurons  and construct corresponding masks for merging process,

{\textbf{Election}}. A natural question is how to select important neurons in the task vector $\boldsymbol{\tau}_i$ based on $I(\boldsymbol{\theta}_{\rm{base}})$ and $I(\boldsymbol{\theta}_{i})$. The important neurons in the base model may be different from neurons in the fine-tuned model. Therefore, we introduce the following election strategy to select neurons with high scores in both base and fine-tuned models:
\begin{equation}
    \mathcal{T}_i^{r_i}=\mathcal{N}_i^{r_i}\cap \mathcal{N}_{\rm{base}}^{r_i}.
    \label{vote}
\end{equation}
\emph{Remark}. We compare different choosing methods, including scoring low or high in base or fine-tuned model in Section~\ref{sec:ablation} and find that Equation \ref{vote} achieves the best performance.





{\textbf{Disjoint}}. As important neurons from different task vectors may conflict with each other at the same position, we use the set difference to disjoint the neurons from others to prevent interference:
\begin{equation}
    \text{Disjoint}(\mathcal{T}^{r_i}_{i})=\mathcal{T}^{r_i}_{i}-\mathop{\cup}\limits_{{J}\subsetneqq [K],|J|\geq 2}\mathop{\cap}\limits_{j\in {J}}\mathcal{T}^{r_j}_{j}.
    \label{disjoint_safety}
\end{equation}

Next, we construct a mask $\boldsymbol{m}_i\in\mathbb{R}^D$ to implement disjoint in the merging process. Specifically, this mask $\boldsymbol{m}_i$ is used to select neurons from $\mathcal{T}_i$. The mask ratio is $r_i$, where $r\in(0,1]$. The mask $\boldsymbol{m}_i$ can be derived from:
\begin{equation}
    \boldsymbol{m}_{i,d}=\begin{aligned} &\left\{ \begin{array}{ll} 1, & \text{if } d\in \text{Disjoint}(\mathcal{T}_{i}^{r_i}), \\ 0, & \text{otherwise}. \end{array} \right. \end{aligned}
    \label{mask_safety}
\end{equation}


% \subsection{Merging Models with Masks}
{\textbf{Merging}}. The final
merged task vector $\boldsymbol{\tau}_m$ is as follows:
\begin{equation}
    \boldsymbol{\tau}_m= \sum_i \lambda_i\boldsymbol{\tau}_{i}\odot\boldsymbol{m}_i.
    \label{merged_task_vector}
\end{equation}
We summarize the workflow in Algorithm \ref{alg1}.




\section{Experiments}
\label{sec:experiments}

\begin{figure*}[t]
\vspace{-6mm}
    \centering
    \includegraphics[width=0.8\linewidth]{figs/compare.pdf}
    \vspace{-4mm}
    \caption{\textbf{Qualitative comparison} with the baseline for generating a sequence of novel view images.  
    The results demonstrate that our method synthesizes more consistent multi-view images compared to our baseline model (Zero123). In addition, compared to SyncDreamer, our method visually maintains better similarity to the conditioned image and appears more natural.}
    \label{fig:sota_compare}
\vspace{-5mm}
\end{figure*}

\subsection{Experimental Setups}
\textbf{Dataset.}
Following previous work~\cite{zero123, SyncDreamer}, we evaluate our work on the Google Scanned Object (GSO)~\cite{GSO} dataset to verify the zero-shot novel view image synthesis capability. 
We also provide results for additional datasets in the Supplementary Material.
Specifically, we randomly select 30 objects from the GSO dataset with various object categories. 
Unlike recent approaches~\cite{mvdream, SyncDreamer} that aim to enhance the consistency of novel view synthesis models by generating multiple fixed-view images, our method can generate images from any camera pose and any number of views. Therefore, we conduct experiments under different camera pose settings to validate our approach:
specifically, 
1) \textit{16-views with free camera pose}: for each object, we circularly render 16 views with the elevation angles ranging in $[-10\degree, 40\degree]$ and the azimuth angles are evenly distributed in $[0\degree, 360\degree]$. 
2) \textit{16-views with fixed camera pose}: We maintain a constant elevation angle of $30\degree$ and uniformly sample azimuth angles (same as SyncDreamer~\cite{SyncDreamer}).
3) \textit{32-views with free camera pose}: Similar to the first setting, but we sample 32 views.
It's important to note that our method does not require additional training or fine-tuning on any datasets.

\noindent\textbf{Metrics.}
To validate the effectiveness of our method, we mainly evaluate it based on three criteria:
1) \textit{Quality Score}. We evaluate the image quality of synthesized multi-view images by measuring their similarity with ground truth images. Following prior research~\cite{zero123, sparsefusion}, we report the similarity between the synthesized images and the ground truth images with standard metrics: PSNR, SSIM~\cite{ssim}, and LPIPS~\cite{lpips}.
2) \textit{Multi-view Consistency Score}. As the primary goal of our work is to improve the consistency of generated images, we also employ the 3D consistency score~\cite{3dim} to verify the consistency among the synthesized images. Specifically, we train an Instant-NGP~\cite{instant_ngp} with the input image and part of the synthesized novel view images of our model and evaluate the similarity between the remaining synthesized images and the rendered images of Instant-NGP. For the synthesized multi-view images of each object, we allocate $3/4$ for training and reserve the remaining $1/4$ for validation.
Intuitively, if the consistency of synthesized images is improved, the NeRF-like model will train a better object representation, and the re-rendered images will agree more with the validation images.
3) \textit{Input Consistency Score}. To assess the faithfulness of synthesized images in preserving the identity of the input condition image, we introduce the input consistency score. This score calculates the similarity of each synthesized image with the input condition image, utilizing the LPIPS metric.

In addition, we use synthesized multi-view images to train a neural 3D reconstruction model (NeuS~\cite{neus}) and report commonly used Chamfer Distances (CD) and Volume IoUs between the trained 3D model and the ground truth.

\noindent\textbf{Baselines.}
Given that our main goal is to improve the consistency of the trained baseline model without further fine-tuning, we mainly compare our approach with the used baseline model Zero123~\cite{zero123}. Additionally, we compare our method to the SOTA approaches such as PGD~\cite{tseng2023consistent} and SyncDreamer~\cite{SyncDreamer} using the same Zero123 base model.

\noindent\textbf{Implementation Details.}
We use the official checkpoint provided by Zero123~\cite{zero123}, which is trained on objaverse~\cite{objaverse} for 165,000 steps. We inject our epipolar attention layer after step $T=4$ and layer $L=10$ by default. We find that feature fusion weight $\alpha=0.5$, and the number of context views $M=2$ work better.

\begin{table}[t]
\centering
\caption{Comparison of multi-view consistency, image quality, and input consistency of synthesized multi-view images at the 16-view setting with free camera pose.}
\label{tab:view16_free_compare}
\vspace{-2mm}
\scalebox{0.6}{
\begin{tabular}{c ccc ccc c}
\toprule
              & \multicolumn{3}{c}{Multi-view Consistency} & \multicolumn{3}{c}{Quality Score} & \multicolumn{1}{c}{Input Consis.} \\
              \cmidrule(lr){2-4} \cmidrule(lr){5-7} \cmidrule(lr){8-8}
              & PSNR$\uparrow$  & SSIM$\uparrow$ & LPIPS$\downarrow$ 
              & PSNR$\uparrow$  & SSIM$\uparrow$ & LPIPS$\downarrow$ 
              & LPIPS$\downarrow$ 
              \\ \midrule

Zero123
& 15.225        & 0.645       & 0.408
& 14.255        & 0.747       &	0.208
& 0.303         
\\
SyncDreamer
& 14.830        & 0.626       & 0.434
& 12.650        & 0.713       &	0.254
& 0.317         
\\
Ours 
& \best{18.300}	& \best{0.734}	& \best{0.355}
& \best{14.947}	& \best{0.763}	& \best{0.191}
& \best{0.282}
\\

\bottomrule
\end{tabular}
}
\end{table}

\begin{table}[t]
\vspace{-1mm}
\centering
\caption{Comparison of multi-view consistency, image quality, and input consistency at the 16-view setting with fixed camera pose as SyncDreamer~\cite{SyncDreamer}.}
\label{tab:view16_fxied_compare}
\vspace{-3mm}
\scalebox{0.6}{
\begin{tabular}{c ccc ccc c}
\toprule
              & \multicolumn{3}{c}{Multi-view Consistency} & \multicolumn{3}{c}{Quality Score} & \multicolumn{1}{c}{Input Consis.} \\
              \cmidrule(lr){2-4} \cmidrule(lr){5-7} \cmidrule(lr){8-8}
              & PSNR$\uparrow$  & SSIM$\uparrow$ & LPIPS$\downarrow$ 
              & PSNR$\uparrow$  & SSIM$\uparrow$ & LPIPS$\downarrow$ 
              & LPIPS$\downarrow$ 
              \\ \midrule

Zero123
& 16.556        & 0.682       & 0.378
& 14.592        & 0.750       &	0.207
& 0.305         
\\
SyncDreamer
& \best{22.424}        & \best{0.812}       & \best{0.268}
& 15.269        & 0.749       &	0.196
& 0.300         
\\
Ours 
& 21.151	& 0.780	& 0.302
& \best{15.293}	& \best{0.764}	& \best{0.184}
& \best{0.287}
\\

\bottomrule
\end{tabular}
}
\vspace{-4mm}
\end{table}


\subsection{Comparison With Baseline Models}
The quantitative comparison on three settings are shown in Tab.~\ref{tab:view16_free_compare}, Tab.~\ref{tab:view16_fxied_compare}, and Tab.~\ref{tab:view32_free_compare}. The qualitative comparison is shown in Fig.~\ref{fig:sota_compare}.

\begin{table}[t]
\centering
\caption{Comparison of multi-view consistency and image quality scores of synthesized multi-view images at the 32-view setting with free camera pose.}
\vspace{-3mm}
\label{tab:view32_free_compare}
\scalebox{0.7}{
\begin{tabular}{c ccc ccc}
\toprule
              & \multicolumn{3}{c}{Multi-view Consistency} & \multicolumn{3}{c}{Quality Score} \\
              \cmidrule(lr){2-4} \cmidrule(lr){5-7}
              & PSNR$\uparrow$  & SSIM$\uparrow$ & LPIPS$\downarrow$ 
              & PSNR$\uparrow$  & SSIM$\uparrow$ & LPIPS$\downarrow$ 
              \\ \midrule

Zero123
& 16.515        & 0.694       & 0.378
& 15.142        & 0.733       &	0.211
\\
PGD~\cite{tseng2023consistent}
& 18.481        & 0.720       & 0.343
& 15.281        & 0.739       &	0.205
\\
Ours 
& \best{20.655}	& \best{0.792}	& \best{0.305}
& \best{15.268}	& \best{0.742}	& \best{0.203}
\\

\bottomrule
\end{tabular}
}
\vspace{-3mm}
\end{table}

\begin{table*}
  [t]
  \centering
  \resizebox{\textwidth}{!}{%
  \begin{tabular}{cccccccccccc}
    \toprule \multicolumn{2}{c}{Components}                                                             & \multicolumn{5}{c}{Re-executability Rate (\%)} & \multicolumn{5}{c}{Readability (\#)} \\
    \cmidrule(lr){1-2} \cmidrule(lr){3-7} \cmidrule(lr){8-12}        \hspace{8pt}\labelemoji\hspace{8pt}                                                                & \hspace{8pt}\toolemoji\hspace{8pt}                                      & O0                                 & O1             & O2             & O3             & AVG            & O0             & O1             & O2             & O3             & AVG            \\
    \hline
    \rowcolor[rgb]{0.93,0.93,0.93}\multicolumn{12}{c}{\textbf{Initialize with LLM4Decompile-End-6.7B~\citep{llm4decompile}}}   \\
    \xmark                                                                                              & \xmark                                    & 69.51                              & 46.95          & 50.61          & 46.34          & 53.35          & 3.98 & 3.41 & 3.44 & 3.38 & 3.55 \\
    \cmark                                                                                              & \xmark                                    & 75.61                              & 50.61          & 50.00          & 50.00          & 56.55          & 4.01 & 3.44 & 3.39 & \textbf{3.49} & 3.58 \\
    \xmark                                                                                              & \cmark                                    & 83.54                     & \textbf{56.10}          & 51.22          & 50.61 & 60.37 & 4.05 & 3.51 & 3.51 & 3.42 & 3.62 \\
    \cmark                                                                                              & \cmark                                    & \textbf{85.37}                            & \textbf{56.10}                     & \textbf{51.83} & \textbf{52.43}          & \textbf{61.43} & \textbf{4.13} & \textbf{3.60} & \textbf{3.54} & \textbf{3.49} & \textbf{3.69} \\

    \rowcolor[rgb]{0.93,0.93,0.93}\multicolumn{12}{c}{\textbf{Initialize with Deepseek-Coder-6.7B-base~\citep{deepseekcoder}}} \\
    \xmark                                                                                              & \xmark                                    & 59.15                              & 35.98          & 39.02          & 37.80          & 42.99          & 3.71 & 3.05 & 3.16 & 3.05 & 3.24 \\
    \cmark                                                                                              & \xmark                                    & 66.46                              & 41.46          & 38.41          & 36.59          & 45.73          & 3.76 & 3.17 & \textbf{3.21} & 3.08 & 3.31 \\
    \xmark                                                                                              & \cmark                                    & 70.73                              & 39.63          & 39.02          & 40.24          & 47.41          & 3.90 & 3.17 & 3.08 & 3.11 & 3.31 \\
    \cmark                                                                                              & \cmark                                    & \textbf{79.88}                     & \textbf{45.73} & \textbf{43.90} & \textbf{42.68} & \textbf{53.05} & \textbf{3.96} & \textbf{3.21} & 3.18 & \textbf{3.19} & \textbf{3.38} \\
    \bottomrule
  \end{tabular}%
  }
  \caption{The ablation study of different methods across four optimization levels
  (O0, O1, O2, O3), as well as their average scores (AVG). The results in bold represent the optimal performance. The ~\labelemoji~ and ~\toolemoji~ means Relabedling and Function Call. \textbf{Bold} denotes the best performance.}
  \label{tab:ablation}
\end{table*}



\begin{figure*}[ht]
    \centering
    \begin{minipage}{0.65\textwidth}
        \centering
        \includegraphics[width=0.95\linewidth]{figs/ablation.pdf}
        \vspace{-2mm}
        \captionof{figure}{Qualitative Comparison for different design choices. Our method, employing multi-view epipolar attention, demonstrates the best consistency.}
        \label{fig:ablation}
    \end{minipage}\hfill
    \begin{minipage}{0.33\textwidth}
        \centering
        \includegraphics[width=0.8\linewidth]{figs/neus_ver.pdf}
        \vspace{-3mm}
        \caption{Our method shows better direct 3D reconstruction~\cite{neus}.}
        \label{fig:neus}
    \end{minipage}
    \vspace{-5mm}
\end{figure*}

\noindent\textbf{Multi-view Consistency.}
Tab.~\ref{tab:view16_fxied_compare} presents the 3D consistency scores compared to our baseline model (Zero123) and SyncDreamer. The results indicate a significant improvement across all three metrics achieved by our method when compared with Zero123.
While our method exhibits a marginally lower numerical consistency score compared to SyncDreamer, it enables the synthesis of images with arbitrary camera poses.	
This capability is illustrated in Tab.~\ref{tab:view16_free_compare}, where our method consistently enhances consistency with changes in camera pose settings, whereas SyncDreamer fails to do so and exhibits inferior results compared to Zero123.
Furthermore, our method facilitates the synthesis of multi-view images with any number of camera views. This versatility is demonstrated in Tab.~\ref{tab:view32_free_compare}, where our method continues to achieve significant improvements in consistency scores, while SyncDreamer is unable to operate under such conditions.	

Meanwhile, Fig.~\ref{fig:sota_compare} provides a qualitative comparison with the baseline. While both our method and SyncDreamer enhance consistency, our method visually preserves better similarity to the input image, including color and texture details. The input consistency score further corroborates this.

\noindent\textbf{Image Quality.}
While our primary goal centers around enhancing the consistency of synthesized multi-view images, we also evaluate the image quality by comparing the similarity with the ground truth images. The results shown in Tab.~\ref{tab:view16_free_compare}, Tab.~\ref{tab:view16_fxied_compare}, and Tab.~\ref{tab:view32_free_compare} indicate that our method also enhances the image quality under different settings besides improving the consistency.
Moreover, our method shows better image quality compared with SyncDreamer even in the 16-view setting with fixed camera pose.

\noindent\textbf{Input Consistency.}
Input consistency terms whether the results align with the input image.
Fig.~\ref{fig:sota_compare} illustrates that both our method and SyncDreamer enhance multi-view consistency. However, the color and texture details of SyncDreamer's results diverge from the input image and appear visually unnatural.
This discrepancy is evident in the input consistency score presented in Tab.~\ref{tab:view16_fxied_compare}, indicating lower similarity with the condition image in the SyncDreamer results.	

\subsection{Ablation Study}
The overall quantitative results are shown in Tab.~\ref{tab:ablation}, and the qualitative comparisons are shown in Fig.~\ref{fig:ablation}.

\noindent \textbf{Full Attention \vs Epipolar Attention.}
The results presented in Tab.\ref{tab:ablation} and Fig.\ref{fig:ablation} demonstrate that our epipolar attention mechanism can synthesize more consistent multi-view images compared with full attention. Furthermore, our epipolar attention achieves a greater performance improvement compared to full attention when using multiple reference images. This could be attributed to the fact that our epipolar attention more effectively localizes target information, as depicted in Fig.~\ref{fig:full_attn_compare}, thereby reducing noise from the reference images. In the multi-view setting, where multiple reference images are utilized, this noise reduction becomes particularly crucial.
Moreover, it is noteworthy that the epipolar attention mechanism consumes less GPU memory compared to our baseline, as discussed in Sec.~\ref{sec:attn_analysis}.

\noindent \textbf{Attending Single-View \vs Multi-View.}
Applying the epipolar attention significantly improves the consistency between the input and target views. However, the consistency between different views in the unobserved regions of the input view is not well preserved.
After implementing our epipolar attention in the multi-view setting, the consistency across the generated multi-view images is further improved. The last row in Tab.~\ref{tab:ablation} shows that after applying our multi-view epipolar attention, the consistency score is further improved compared with the single-view setting. Besides, the qualitative result in Fig.~\ref{fig:ablation} also shows better consistency among different target views.



\begin{table}[t]
\centering
\vspace{-1mm}
\caption{Comparison of 3D reconstruction results. Our method significantly improves the reconstruction quality.}
\vspace{-3mm}
\label{tab:neus}
\scalebox{0.7}{
\begin{tabular}{c cc}
\toprule
              &  Chamfer Dist.$\downarrow$  & Volume IoU$\uparrow$
\\ \midrule

            Zero123         & 0.017         & 0.819    \\
            SyncDreamer     & \best{0.013}         & \best{0.847}    \\
            Ours            & 0.014	& 0.842 \\

\bottomrule
\end{tabular}
}
\vspace{-5mm}
\end{table}


\vspace{-2mm}
\subsection{Downstream Application}
\vspace{-2mm}
To demonstrate the effectiveness of our method, we also applied it to the downstream 3D reconstruction task. Specifically, we trained the NeuS model~\cite{neus} directly using images synthesized by our method, Zero123, and SyncDreamer, respectively.
The quantitative results in Tab.~\ref{tab:neus} show that the consistent multi-view images synthesized by our method can significantly improve the 3D reconstruction quality.
Additionally, our method exhibits similar performance to SyncDreamer which requires time-consuming re-training.
The qualitative results in Fig.~\ref{fig:neus} show that it is challenging to train the NeuS model directly due to the lack of consistency in the images generated by Zero123. In contrast, our method generates more consistent multi-view images and, therefore, better reconstructs the geometry and texture details.
We show improvements on other downstream applications such as image-to-3D in the Supplementary Material.



In this section, we propose to analyze the effect of undesired responses generated by our unfaithful sampling method on the overall performances of \scope. By varying the value of $\alpha$ in the noisy data generation process (\Cref{alg:preferencedatasetgen}), we can simulate different degrees of hallucinations due to the influence of $\plm$. In this analysis, we examine the impact of negative samples on preference learning.

%\hil{
%\subsection{Effect of number of samples used for fine-tuning phase}
%TODO: Ablation 0.25 / 0.5 / 0.75 / 1.0 -> 0.25 / 0.5 / 0.75 / 1.0 sur ToTTo
%}

%\hil{
%\subsection{Effect of preference tuning method}
%Although we experimented with DPO, the generation of the noisy samples does not depend on the choice of the preference tuning framework. We considered the ORPO framework and report the results in Table.
%TODO: Insert ORPO results.
%}

%\hil{\subsection{Effect of preference tuning without targets $y$}
%TODO: Use self-generated samples from $\psft$ instead of  provided targets $y$ from the dataset in \scope.
%}

\paragraph{How does the value of $\alpha$ affect the training dynamics?}
\label{sec:alpha-effect}
%As shown above, increasing $\alpha$ introduces both intrinsic and extrinsic hallucinations due to the effect of the unconditional model. 
The choice of $\alpha$ is critical. When $\alpha$ is low, the negative samples are too close to the model's own approximation of the underlying data distribution. During the preference-tuning stage, the model struggles to maximize the gap between the likelihood of the clean and noisy samples while maintaining the high likelihood of the clean ones. This causes the model to downweight the likelihood of both samples, leading to degeneracies, see \Cref{fig:train_a01}.
Conversely, when $\alpha$ is high, the generated noisy samples are barely grounded in the input context, making it easy to distinguish between $y$ and $\ylose$ under $p_\theta(\cdot \mid c)$. In this case, the model learns very little compared to its fine-tuned counterpart, see \Cref{fig:train_a07}. Therefore, $\alpha$ should be chosen to balance the noisy generation between being too similar to the reference texts and too easy to discriminate. This scenario is illustrated on \Cref{fig:train_a05} for $\alpha=0.5$. The likelihood of the references does not decrease, and the likelihood of the noisy samples diverges less abruptly than in \Cref{fig:train_a07}, providing a more effective learning signal.

\begin{figure*}
  \centering
  \subfloat[Training with $\alpha = 0.1$.]{\includegraphics[width=0.25\textwidth]{images/logprobs_a0.1_totto.pdf}\label{fig:train_a01}}\hspace{1em}
  \subfloat[Training with $\alpha = 0.5$.]{\includegraphics[width=0.25\textwidth]{images/logprobs_a0.5_totto.pdf}\label{fig:train_a05}}\hspace{1em}
\subfloat[Training with $\alpha = 0.7$.]{\includegraphics[width=0.25\textwidth]{images/logprobs_a0.7_totto.pdf}\label{fig:train_a07}}
  \caption{Preference training dynamics with \textsc{Llama-2-7b} as noise level $\alpha$ increases on ToTTo dataset. Illustration of the three different regimes during preference training. Blue (resp.\ red) curve corresponds the log probability of the reference labels (resp.\ of the synthetic unfaithful samples).}
  
  \label{fig:train_alpha}
\end{figure*}

\paragraph{How does the negative samples construction affect generation quality?}
For low values of $\alpha$, we observe noticeable degeneracies, evidenced by text repetitions. This is shown in \Cref{fig:bleu_alpha}, where BLEU scores decrease abruptly with lower values of $\alpha$. As dicussed in \Cref{sec:results}, in the $[0.4, 0.6]$ interval, the decrease in BLEU appears to be more closely related to the generated outputs diverging from standard fine-tuning patterns, rather than a noticeable decline in fluency.
Regarding optimization efficiency, the three regimes observed in \Cref{fig:train_alpha} can also be identified in \Cref{fig:nli_alpha}, that describes the evolution of the NLI score as a function of $\alpha$. Below a certain level of noise, degeneracies also impact the NLI score. Increasing $\alpha$ beyond a certain point yields no further improvement, as both BLEU and NLI scores converge to the results of standard fine-tuning. As a result, searching for $\alpha$ in the interval $[0.4, 0.6]$ seems to yield the best performances. We observe similar patterns in text summarization tasks, see \Cref{app:scope-analysis}. A quantitative and qualitative analysis of the noisy samples can be found in \Cref{app:noisy-samples}.
%In the extreme case where $\alpha = 0$, \scope is equivalent to one iteration of SPIN \citep{spin}, a method designed to enhance instruction-tuned models for chat. Within the specific context of data-to-text generation and text summarization, we found that this approach is ineffective. We hypothesize that this ineffectiveness is due to the distinct requirements of data-to-text generation, which may not align well with the mechanisms that SPIN leverages for instruction tuning.

% \red{where the model exhibits the greatest boost in faithfulness without showing degeneracies. The hyperparameter $\alpha$.  $\alpha$ is tuned for each dataset based on the NLI score on the validation set after preference-tuning. Since the NLI score does not penalize non-fluent behavior that appears when $\alpha$ is low, we select ranges of $\alpha$ on which values of BLEU do not decrease significantly. In practice, we found that for $\alpha \in \{0.4, 0.5, 0.6\}$, the model exhibits the greatest boost  More details on the effect of $\alpha$ can be found in \Cref{sec:analysis}.}

\begin{figure*}
  \centering
  \subfloat[]{\includegraphics[width=0.30\textwidth]{images/nli_score_totto.pdf}\label{fig:nli_alpha}}\hspace{2em}
  \subfloat[]{\includegraphics[width=0.30\textwidth]{images/bleu_totto.pdf}\label{fig:bleu_alpha}}
  \caption{Evolution of NLI Score and BLEU with $\alpha$ on ToTTo validation set with \textsc{Llama-2-7b}.}
  \vspace{-0.5cm}
\end{figure*}

%\begin{figure}
%  \centering
%
%  \includegraphics[width=0.35\columnwidth]{images/nli_score_totto.pdf}
%\caption{Evolution of NLI Score on ToTTo validation set as a function of $\alpha$ %with \textsc{Llama-2-7b}.}
%  \label{fig:nli_alpha}
%\end{figure}
% \subsection{Noisy samples when varying $\alpha$}
%\red{Hallucinations happen when the generation is not enough grounded in the input context. To simulate hallucinated content, we perturb the decoding process using the pre-trained model without context $\plm(\cdot)$ and varying its effect with a scaling factor $\alpha$. Samples with different values of $\alpha$ can be visualized in \red{Table}. As highlighted, our noisy generation procedure incorporates both intrinsic ( in \red{red}) and extrinsic (in blue) hallucinations.}


\section{Related Work}

\paragraph{Commonsense Reasoning Evaluation} 
There are numerous benchmarks and datasets for commonsense reasoning, most of which are in English. 
%Some work focused on evaluating general commonsense knowledge, such as HellaSwag \cite{zellers2019hellaswag}, CommonsenseQA \cite{talmor2019commonsenseqa}, OpenBookQA \cite{OpenBookQA2018}, and WSC \cite{levesque2012winograd}. 
Some studies focus on evaluating general commonsense knowledge \cite{zellers2019hellaswag,talmor2019commonsenseqa,OpenBookQA2018}. 
%Others target specific aspects of commonsense reasoning, including temporal commonsense with MCTACO \cite{zhou2019going}, physical commonsense with PIQA \cite{bisk2020piqa}, social commonsense with SocialIQA \cite{sap2019socialiqa}, numerical commonsense with NumerSense \cite{lin2020birds}, and scientific commonsense with ARC \cite{clark2018think} and QASC \cite{khot2020qasc}. Notably, most of these datasets are in English. 
Others target specific aspects of commonsense reasoning\cite{zhou2019going,bisk2020piqa,sap2019socialiqa,lin2020birds,clark2018think,khot2020qasc}.
There are some Chinese datasets for commonsense reasoning \cite{sun2024benchmarking,shi2024corecode}. 
For instance, CHARM \cite{sun2024benchmarking} distinguishes between global commonsense and Chinese-specific commonsense but includes only a limited number of everyday commonsense cases. 
However, evaluations aimed at assessing the robustness of commonsense reasoning are still understudied. 

\paragraph{Datasets on Different Reasoning Forms}
There are several datasets relevant to our variant design. For reverse reasoning, ART \cite{DBLP:conf/iclr/BhagavatulaBMSH20}, $\delta$-NLI \cite{DBLP:conf/emnlp/RudingerSHBFBSC20}, and CLUTRR \cite{DBLP:conf/emnlp/SinhaSDPH19} explore different reasoning directions. FCR \cite{DBLP:journals/corr/abs-2204-07408} and NatQuest \cite{ceraolo2024analyzinghumanquestioningbehavior} evaluate causal reasoning, while TimeTravel \cite{DBLP:conf/emnlp/QinBHBCC19} focuses on counterfactual scenario refinement. Additionally, PoE \cite{balepur2024s} assesses reasoning involving negation. 
However, not all these datasets focus on commonsense reasoning, nor are they structured by original questions and their variants. Furthermore, they typically target limited reasoning types. Lastly, our dataset is large-scale and covers diverse commonsense knowledge. 

\paragraph{Robustness and Consistency in LLMs} 
Early work focuses on adversarial attacks, with developing evaluation methods for reading comprehension systems \cite{jia2017adversarial}, followed by universal adversarial triggers \cite{wallace2019universal}. The field then expands to examine spurious correlations, with revealing how models often exploit superficial patterns rather than engaging in genuine reasoning \cite{branco2021shortcutted,geirhos2020shortcut}. And \citealp{ross2022does} investigates whether self-explanation can mitigate these spurious correlations. Coherence and consistency evaluation advances through classifier assessment methods \cite{storks2021beyond} and analysis of accuracy-consistency trade-offs \cite{johnson2023much}. While these studies primarily address model robustness against adversarial attacks or spurious correlations, our work takes a novel approach by examining robustness in reasoning forms.
%, specifically focusing on how models maintain consistent reasoning when presented with different reasoning forms of the same commonsense knowledge.
%\paragraph{Dataset Construction by LLM} 
%Research indicates that when LLMs are utilized for dataset generation, the resulting datasets are more accurate and fluent \cite{lu2022fantastically, min-etal-2022-rethinking} than those created by crowd-sourced annotators. Furthermore, generating datasets with LLMs is significantly more cost-effective than using crowd-sourced annotations \cite{liu2022wanli, wiegreffe2022reframing, west2022symbolic}. Hence, we generate our benchmark by LLM in-context learning.

% \paragraph{In-Context Learning} 
% As LLMs become more widely used, in-context learning (\citealp{brown2020language}; \citealp{ouyang2022training}; \citealp{min-etal-2022-rethinking}) has emerged as the primary approach for executing various tasks. This method involves supplying LLMs with textual instructions and examples and removes the necessity for parameter modifications. Research indicates that when LLMs are utilized for dataset generation, the resulting datasets are more accurate and fluent (\citealp{lu2022fantastically}; \citealp{min-etal-2022-rethinking}) than those created by crowd-sourced annotators. Furthermore, generating datasets with LLMs is significantly more cost-effective than using crowd-sourced annotations (\citealp{liu2022wanli}; \citealp{wiegreffe2022reframing}; \citealp{west2022symbolic}). Hence, we have decided to construct our benchmark by over-generating data using in-context learning and employing human annotators for filtering to ensure high efficiency and high quality.
% Moxin: 这部分应该改成用LLM 生成dataset的工作?
% 


\section{Conclusion and future directions} \label{sec:conclusion}

In this paper we proposed a nested MLMC framework that offers important computational savings by performing most calculations in low precision and exploiting approximate random normal variables for the low precision path calculations. The low precision calculations could be performed in fixed precision on an FPGA for greater efficiency, and we suggested a procedure to optimise the bit-widths of every variable at each Monte Carlo level. This is an important improvement over previous mixed precision MLMC frameworks which held the lower precision fixed \cite{Rounding_error_oliver} or defined uniform bit-width at every level heuristically \cite{brugger2014mixed}. Our numerical results suggest that for the first levels our procedure reduces the cost at these levels by a factor 5 or 7. Hence the overall savings are significant since most paths are calculated on the first levels. Our approach would be even more efficient for the Milstein scheme because its higher order strong convergence leads to a greater proportion of the computational costs being on the coarsest levels.

The next stage of the research project will be to implement the RNG methods and the nested framework on FPGAs to determine the hardware requirements and confirm the extent of the computational savings. It would also be good to compare the performance benefits to using half-precision floating point arithmetic on GPUs or CPUs for the low-accuracy computations.




\normalem
\bibliography{arxiv}
\appendix

\section{Appendix: Prompt}
\label{sec:appendix}
``Here is a sketch of an image. 
$\{input\_color\_mask\}$, while the rest of the white space is the background. 
I need you to infer details of the image based on the given sketch.
The details should include the possible background likely to be present with the $\{input\_color\_mask\}$, the attribute of each object (like wearing, texture, color etc.), the state (including action, posture, etc.) of each object, the direction of each object and the relationships between objects.

You should first analyze the mask carefully, considering the size, location, and relative position of each object mask. Ensure that specific actions are analyzed based on the mask, and infer each aspect with a reasoning process before providing the final output.
The final output format should be: $\{format\_example\}$, and you should refer to the example: $\{few\_shot\}$. You are going to complete the "" in each item, you need to complete them in multiple short phrases based on your above reasoning.

The state and relationship should be as detailed as possible while ensuring they align with the mask, formatted as: objectA action/spatial relation objectB, with both objectA and objectB included.
You should properly refer to some examples of attributes of object $\{attributes\}$ and relationships $\{relationships\}$.
Do not include words like `or', `possibly' in your final output, there should no ambiguity in your output.
Make sure all aspects of given mask is filled.''
% \appendix

\end{document}
