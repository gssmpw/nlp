\section{Summary table showing our taxonomy of computer security research}
\label{sec:table}
Table \ref{tab:research_taxonomy} contains a table which summarizes our taxonomy of computer security research methods that could be considered deceptive, from \Cref{sec:taxonomy}.


\begin{table*}[ht]
    \def\arraystretch{1.4}
    \footnotesize
    \centering
    \caption{A summary of our taxonomy of computer security research methods that could be considered deception (see \Cref{sec:taxonomy})}
        \begin{tabular}{p{2cm}|p{6.0cm}p{6.0cm}p{2cm}} 
%     \hline
        Research method & Method description & Uses of deception in research & Exemplar(s) \\
        \midrule \midrule
        (\ref{sec:pentesting}) Penetration testing \& red teaming  & Hacking software systems to identify and report vulnerabilities in the target system. These functions are often performed by whitehat hackers acting independently or on behalf of the target entity (e.g., a government agency or company). & Researchers spoof network information or incorrect credentials. \cite{gallowayPracticalAttacksDNS2024} game the DNS reputation system to artificially increase the reputation score of a suspicious domain.  & \cite{simons2001, gallowayPracticalAttacksDNS2024, maFakeBehalfImperceptibleEmail} \\
        \hline
        (\ref{sec:passive_mmt}) Web scraping, snapshotting, measurement & Conducting automated or manual collection of publicly visible web content. & Researchers perform passive measurement or observation of online communities; they may provide only partial disclosure of their identities, may refrain from divulging their presence entirely, or may redact or spoof network information in their interactions. \cite{decary2014policing} use bots to infiltrate IRC chat rooms and scrape chat logs without disclosing the presence of these bots to chatroom participants. & \cite{decary2014policing, kreibichSpamCampaignTrail2008} \\
        
        \hline
        (\ref{sec:scripted}) Scripted user studies & Recruiting study participants perform a pre-specified series of actions using their own identities and/or account credentials. This method---and its attendant risks to the surrogate---is often preferable to (e.g.) sock puppets for reasons of ecological validity. & Researchers instruct study surrogates to misrepresent themselves or their intentions to a target system. \cite{haeder2016secret} instruct ``secret shoppers'' to request healthcare provider information from insurance marketplaces. & \cite{haeder2016secret, dimkov2010two} \\
        \hline
        (\ref{sec:sock_puppets}) Sock puppets & Creating and operating fake accounts. Though deception is a core element of this work---the accounts are not operated by the person(s) whose name(s) they bear---the \textit{degree} of deception varies with the parties (e.g., a platform algorithm, other users) most likely to interact with these fake accounts. & Researchers generate fake accounts, oftentimes from multiple IP addresses or devices, in order to circumvent bot detection mechanisms; they might use these accounts to engage with real (human) users who are unaware that they are interacting with a sock puppet. \cite{boshmaf2011socialbot} infiltrate Facebook social graphs with a botnet comprising about a hundred bot accounts. & \cite{boshmaf2011socialbot, srba2023auditing, bandy2021more} \\
        \hline
        (\ref{sec:soc_eng}) Social engineering and Interactive Studies & Interacting with real human users to measure user or system responses to perturbations. & Researchers send false or misrepresentative email communications via spam or phishing campaigns, or perturb online platforms to observe user and platform responses. \cite{royChatbotsPhishbotsPhishing2024} deploy LLM-generated phishing attacks. & \cite{di2022revisiting, acharyaConningCryptoConman2024, jakobssonWhyHowPerform2008, royChatbotsPhishbotsPhishing2024} %\sstodo{JM Radboud, PETS 2022, our age verif work?} 
        \\
        \hline
        (\ref{sec:audits}) Audits & Understanding algorithm functionality under-the-hood via selective manipulation of inputs and observation of corresponding outputs.  & Researchers submit false or misrepresentative data points to a system in order to observe its outputs. \cite{kaplanMeasurementAnalysisImplied2022} posed as Facebook advertisers, submitted multiple job postings, and observed potential biases in the distribution of these ads. & \cite{kaplanMeasurementAnalysisImplied2022, westPictureWorth5002024, chengALIFLowCostAdversarial2024, jiangCanHearYour, kimScoresTellEverything2024} \\
        \hline
        % \sstodo{Does it have to be black box? See e.g. \cite{wanBounceAttackQueryEfficientDecisionbased2024}.}
        (\ref{sec:adversarial_ml}) Attacks on ML & Developing attacks on ML systems in order to test their robustness, infer training set contents or guardrails, or poison training data. & Researchers submit false or misrepresentative prompts to a system and observe its outputs. \cite{perez2022red} use a language model to generate adversarial attacks on an LLM. & \cite{perez2022red, yanLLMAssistedEasyTriggerBackdoor, liuAFGenWholeFunctionFuzzing2024, yangSneakyPromptJailbreakingTextimage2024, yuDontListenMe2024} \\ 
        \hline \hline
        (\ref{sec:id_work}) ID Research & Research involving presentation of real or prop ID documentation to an AI/ML-driven verification system. Prop IDs might be modified forms of genuine ID documents. & Researchers submit false or misrepresentative images to a black box system and observe its outputs. \cite{di2019personal} modify photocopies of ID images and observe the efficacy of ID verification systems on these modified documents.  & \cite{di2022revisiting, di2019personal, zhao2021deep, engelbertzSecurityAnalysisEIDAS} 
        % \sstodo{Our age verif work, PETS 2022} \sstodo{Find some more} 
        \\
        \hline \hline

        \end{tabular}
\label{tab:research_taxonomy}
\end{table*}