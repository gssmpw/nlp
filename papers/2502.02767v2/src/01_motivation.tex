\textbf{Motivation for this work.}
\label{sec:motivation}

This work was inspired by difficulties encountered in the course of planning a study to investigate commercial Internet age verification systems. 
The original study vision included a series of tests intended to determine the effectiveness of simple baseline attacks on these systems: for instance, by having a study participant don simple disguises and use various ID cards to measure the efficacy of the age verification functions. We quickly realized that this line of work would require significant legal and ethical due diligence.

Our research team began designing the study primarily with ethical study design and the CFAA in mind, and worked with a legal clinic to understand the issues. For example, we took significant measures to ensure that our infrastructure never gained ``unauthorized access'' to any system, and we took very strict steps to protect study participants' data. However, once unauthorized access questions were resolved, there were still deceptive elements in the study design: researchers would need to present falsified data to the system to measure its response. 

One recurring theme that emerged from the discussions between the researchers and the legal clinic team was that it was not CFAA that was posing a challenge for our study design.  Rather, it was anti-fraud laws, partly but not entirely arising from our use of government-issued ID documents.
Moreover, as we investigated these questions further, we began to notice that our analyses were
also relevant to significant other swathes of computer security research as well. In particular, the legal and ethical issues were also relevant to most security work that involved deception in investigations of platforms, interactions with ML-driven systems, or interactions with human subjects.



 
