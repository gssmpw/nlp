\section{Long-discussed legal concerns in computer security research}
\label{sec:cfaa-dmca}

%\subsection{Long-discussed legal concerns}
Research that engages in deception must of course also confront the longstanding issues that all software and platform researchers face under laws like the CFAA, DMCA, and claims rooted in breaches of terms of service. While there remain many aspects of these laws that still present legal risk, the risks have worked to become at least more well understood, and in some respects considerably improved. This subsection reviews the current understanding within these laws.

\subsection{CFAA}
%\subsubsection{CFAA}
The CFAA has long been an obstacle for those seeking to do software and platform research \cite{parkResearchersGuideLegal, baranetsky2018}. The law contains close to a dozen independent theories of liability, all circling around the notion that to be actionable one must access a computer or transmit information to a computer either "without authorization," or by "exceeding authorized access." (18 U.S.C. §1030.) There is no research exemption to the CFAA, \cite{calo2018} although experts have called for a safe harbor that would encompass this and other laws \cite{abdo_safe_2022, longpre_safe_2024}.

Nearly all of the techniques discussed above have had to consider risk of civil or criminal liability under the CFAA, with courts struggling for well over a decade to apply the law to common information gathering techniques such as web scraping \cite{sellars2018}. But the scope of the CFAA was helpfully narrowed in 2021 in the Supreme Court's decision \textit{Van Buren v. United States}. In that case, the Supreme Court addressed how to interpret when one "exceeds authorized access," and in particular whether one violates the law when one \textit{has} access to a computer, but uses it for an \textit{impermissible purpose}. The court said no; it did not go as far as to say it would only consider technical restrictions on access to computers, but made clear that those who have at least some access to computer systems do not face CFAA liability for using their access in ways the computer owner dislikes.\footnote{Van Buren v. United States, 593 U.S. 374, 391–92 (2021).} 

The approach taken in \textit{Van Buren} has eased, but not removed, concerns about research colliding with the CFAA \cite{bhandari2024}. As summarized by Park and Albert, "research that only accesses your own devices does not violate the CFAA," and "[i]f you are working with devices that you don't own, but all resources that you access on those devices are either publicly available or you have valid credentials (issued to you) that provide access to them, there is no CFAA violation" \cite{parkResearchersGuideLegal}. The law is still present in experiments rooted in penetration testing and red teaming, and application of the CFAA to adversarial attacks on AI remains a significant open question \cite{albert_ignore_2024}. But other forms of research have seen their criminal risk mitigated substantially, even while other legal theories like breach of contract claims or trespass to chattel have begun to fill in this gap \cite{sobel2021}.

And even where there remains CFAA risk, the risk of federal prosecution has apparently eased as well. The Department of Justice in 2022 updated its Justice Manual to instruct prosecutors to "decline prosecution if available evidence shows the defendant's conduct consisted of, and the defendant intended, good-faith security research," borrowing its definition from the most recent triennial DMCA exemption rulemaking, discussed in the next section \cite{OfficePublicAffairs2022}. It is not clear that this represents a practical change in enforcement strategy \cite{pfefferkorn2022}, but it can be used as an effective legitimizing signal for this form of work.\footnote{Indeed, the fact that the prior version of this policy \textit{was not} binding allowed a prior constitutional challenge to the CFAA proceed. \textit{See} Sandvig v. Sessions, 315 F. Supp. 3d 1, 20 (D.D.C. 2018).} 

Finally, some courts have cut off a threat for civil action under the CFAA by picking up on dicta in the \textit{Van Buren} decision that purports to limit "loss" to "technological harms" only,\footnote{\textit{Van Buren}, 593 U.S. at 376.} and using that deny civil claims when the alleged intrusion only resulted in costs related to investigation of the incident and consulting with technology lawyers.\footnote{\textit{See, e.g.}, X Corp. v. Ctr. for Countering Digital Hate, Inc., 724 F. Supp. 3d 948, 983 (N.D. Cal. 2024), \textit{appeal docketed}, No. 24-2643 (9th Cir. April 25, 2024) (denying CFAA "loss" for attempting to conduct investigations into the extent of unauthorized access and legal expenses).} This approach has not been universal -- \cite{pfefferkorn2022} illustrates an emerging split between courts in how to internalize this indirect instruction from \textit{Van Buren} -- but a move in that direction would provide some of the same practical risk mitigation discussed in relation to fraud above.

\subsection{DMCA} 
%\subsubsection{DMCA} 
Another common legal concern for platform research stems from the "anticircumvention" provisions which were introduced to United States copyright law in the Digital Millennium Copyright Act, enacted in the late 1990s to address concerns about digital piracy. The provisions prohibit the "circumvention" of a "technical protection measure" that controls access to a copyright-protected work. Because software programs and websites are typically protected under copyright, any bypassing of a technical hurdle to access a work at least arguably implicates the law. Legal risk under the DMCA most frequently arises when the research involves bypassing controls like encryption, CAPTCHAs, authentication handshakes, and other technical gates. Some forms of auditing and scraping, as well as attacks on machine learning systems and penetration testing, tend to be the areas of greatest risk under the DMCA.\footnote{\textit{See, e.g.}, Yout, LLC v. RIAA, 633 F. Supp. 3d 650 (D. Conn. 2022), \textit{appeal docketed}, No. 22-2760 (2d Cir. argued Feb. 5, 2024) (finding that bypassing YouTube's technical provisions against downloading videos was likely a DMCA violation).}

Here, too, the law has gradually moved into a more tolerable state for research, albeit with many remaining concerns. As reviewed by Park and Albert, there are both permanent exemptions written into the statute and expanded temporary (but frequently renewed) exemptions for a broader set of research activity that may require the researcher to bypass a technical protection measure \cite{parkResearchersGuideLegal}. There is also a good argument that much of the work of bias research does not implicate the DMCA, as it does not require the researcher to bypass any technical measures. Web scraping, creation of sock puppet accounts, audits, and scripted user studies that use the platforms as designed and intended similarly do not require a researcher to bypass a technical measure, unless there are technical safeguards in place to specifically control that practice (by, for example, presenting a CAPTCHA after a certain number of accounts are created or pages are accessed). \textit{Tooling} for research remains a significant issue, as the regulatory exemptions of the DMCA apply only to individual acts of circumvention, and not the distribution of tools that are used to facilitate such circumvention, which are separately prohibited in 17 U.S.C. §§1201(a)(2), 1201(b)(1) \cite{parkResearchersGuideLegal}.

\subsection{Terms of Service.} 
%\subsubsection{Terms of Service.} 
\label{sec:tos}
Courts that narrowed the scope of the CFAA to activities like web scraping were quick to observe that they were still leaving platform owners recourse in other areas, including lawsuits rooted in the website's terms of service.\footnote{\textit{See} hiQ Labs, Inc. v. LinkedIn Corp., 31 F.4th 1180, 1201 (9th Cir. 2022) ("Entities that view themselves as victims of data scraping are not without resort, even if the CFAA does not apply: state law trespass to chattels claims may still be available. […] And other causes of action, such as copyright infringement, misappropriation, unjust enrichment, conversion, breach of contract, or breach of privacy, may also lie.").} And indeed, present fights around website access gravitate around these contracts---despite the near-unanimous view that these contracts are rarely read or understood and often contain ambiguous or inconsistent language \cite{fiesler_no_2020}. Terms of Service are drafted to press every advantage for the online platform, and routinely prohibit the sorts of activity essential for platform research \cite{bhandari2024}. The past few years have seen an uptick in this form of enforcement, ranging from a platform directly suing a critic for alleged violations on its terms against web scraping,\footnote{\textit{Ctr. for Countering Digital Hate}, 724 F. Supp. 3d 1.} to use of terms of service to justify suspension of accounts, perhaps most famously the accounts of Laura Edelson and others at New York University who gathered data that examined Meta's handling of political advertising in 2020 and 2021 \cite{longpre_safe_2024, gilbert2024risks}. 

Here, too, researchers and others who act in ways in tension with platform terms have begun some effective lines of defense, inspired by cautions by courts about application of the law in ways that "risk[] the possible creation of information monopolies that would disserve the public interest."\footnote{\textit{hiQ Labs}, 31 F.4th at 1202.} This can include very technical and careful reading of terms and more advanced argumentation around when those terms are binding on the user,\footnote{\textit{See, e.g.}, Meta Platforms, Inc. v. Bright Data Ltd., No. 23-cv-77, 2024 WL 251406 at **10, 15 (N.D. Cal. Jan. 23, 2024) (because defendant was not acting as a "user" of Facebook at the time of their scraping of publicly-available information, the Facebook terms of service did not apply to the activity); \textit{but see} X Corp. v. Bright Data Ltd., 733 F. Supp. 3d 832, 847 (N.D. Cal. 2024) (because X's terms of service purported to reach nonusers as well and defendant was aware of that, agreement was validly formed)} arguments that confine the recoverable damages related to the breach to zero or near enough to zero to be a tolerable risk,\footnote{\textit{Ctr. for Countering Digital Hate}, 724 F. Supp. 3d at 969–70 (finding that X could recover neither direct nor special damages from an alleged breach by a research organization); \textit{Bright Data}, 733 F. Supp. 3d at 847–48. Similar approaches have been used to limit claims under the theory of "trespass to chattel," which has become a popular theory by which to pursue web scraping defendants given the new confines of the CFAA. \textit{See Bright Data}, 733 F. Supp. 3d at 843.} and arguments that certain forms of terms of service violations are preempted by copyright law.\footnote{\textit{Bright Data}, 733 F. Supp. 3d at 852–53; Genius Media Group Inc. v. Google LLC, No. 19-cv-7279, 2020 WL 5553639 (E.D.N.Y. Aug. 10, 2020), \textit{aff'd sub nom.} ML Genius Holdings LLC v. Google, No. 20-3113 (2d Cir. March 10, 2022) (unpublished).} The Federal Trade Commission has also brought actions under the Consumer Review Fairness Act (15 U.S.C. §45b) against companies that attempt to use a terms of service to stifle consumer assessments and reviews \cite{ensor2024}. What remains an issue, however, is how to respond to platforms that enforce terms through technological self-help, such as blocking users or IP addresses. Because there is little \textit{right} to be on a given platform, response to such suspension of accounts can be especially challenging \cite{longpre_safe_2024}.
