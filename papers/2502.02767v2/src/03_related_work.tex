\section{Related Work}
\label{sec:related_work}


The works related to this one largely fall into two main categories: Analyses of computer security or computer science research and their relationship to the law in the U.S., and works that address fraud in computer science research.  We describe each, and we then list other legal concerns we encountered in the literature.

As we described in \Cref{sec:motivation}, there is a fairly robust literature describing the intersection of law and computer science research (especially in computer security and networks).
\cite{parkResearchersGuideLegal} provides the most complete and current review of legal risks for security research.
\cite{longpre_safe_2024} reviews legal and practical dangers to researchers examining AI systems, and calls for both legal protection and mechanisms to ensure technical access by researchers to AI systems.  \cite{kumarLegalRisksAdversarial} also examines legal risks in adversarial machine learning research, a subfield of examining AI systems.
\cite{thomasEthicalIssuesResearch2017} discusses ethical and legal issues arising when researchers use datasets of ``illicit origin'' (e.g. leaked, not necessarily reflecting illicit or deceptive actions by the researchers themselves); they identify many legal issues (computer misuse, copyright, data privacy, use of illicit content like terrorist materials or indecent images, national security issues arising from use of confidential information, and contract violation), but fraud is not among them.
\cite{pfefferkorn2022} reviews how the modern interpretation of the CFAA still remains a vector of concern for cybersecurity research.
\cite{phishing-book} and \cite{soghoian2008legal} concern legal risks of phishing research.
\cite{feltenDigitalMillenniumCopyright2002} discusses the difficulties faced by computer security researchers regarding the Digital Millenium Copyright Act in 2002.
\cite{calo2018} describes the increasingly blurry line between ``hacking'' and ``tricking'' and the intriguing boundaries of the Computer Fraud and Abuse Act.
\cite{hantkeWhereAreRed2024} also provides legal and ethical guidance for network security researchers primarily within the German legal environment.


Another category of related work focuses on the chilling effects that arise in research as a result of legal uncertainty or threat, and discusses the negative consequences to society and individuals arising from that difficulty.
\cite{bhandari2024} connects modern attempts to audit platforms for discrimination with the longstanding auditing done in the offline world and describes negative impact on individuals from lack of access to this research; \cite{baranetsky2018} looks at analogous concerns for data journalists.
\cite{marwick2016} reviews best practices for researchers to avoid harassment related to their work on online platforms.
\cite{mulligan_cybersecurity_2015} is the result of a workshop funded by the National Science Foundation meant to address these challenges.

\label{sec:related-fraud}
Although fraud specifically is not mentioned much in modern computer security research, it does arise in the context of phishing research around 2008.  We highlight three works in particular.


\cite{soghoian2008legal} warns that some ``in-the-wild'' research on phishing, including that of \cite{jakobssonWhyHowPerform2008}, could be considered fraud. He provides best practices for researchers, but the article focuses on other aspects of phishing research that are legally fraught and does not analyze fraud outside the context of the CFAA.

Section 17.5 of \cite{phishing-book} discusses legal considerations in phishing research, including fraud. The analysis relies on a definition of ``defraud'' as meaning ``to take advantage of someone,'' and suggests that since researchers do not have ``intent'' to benefit from the advantage, the case is closed.  We see the legal issue as significantly deeper, as we will discuss in \Cref{sec:law}. \cite{phishing-book} also highlights the difficulty of research involving legal IDs or other ``authentication features,'' which we discuss in \Cref{sec:ids} and \Cref{sec:law}.

\cite{jakobssonWhyHowPerform2008}, entitled ``Why and How to Perform Fraud Experiments,'' discusses performing naturalistic ``in-the-wild'' experiments on fraud compared to laboratory experiments, and describes studies that, in essence, measures users' response to phishing attempts by phishing them (but throwing out the results afterward).  As the work itself points out, ``such experiments pose a thorny ethical issue: if a study is identical to reality, then the study itself constitutes a real-world fraud attempt.''  We will examine ethical issues in \Cref{sec:ethics}.


We also encountered many other U.S. laws and regulations that might impact computer security or computer science research to some extent.  These include
the Electronic Communications Privacy Act (ECPA), trade secret law, export controls, \cite{parkResearchersGuideLegal}
the federal CAN-SPAM Act, the FTC Act, \cite{phishing-book}
trademark law, various state laws \cite{phishing-book,soghoian2008legal}, the federal Wiretap Act \cite{ohmLegalIssuesSurrounding2007}, and
trespass to chattels \cite{soghoian2008legal}.
However, we do not discuss these further in this work.

