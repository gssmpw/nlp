\section{Legal and policy considerations}
\label{sec:legal}
\label{sec:law}

In this section we review how the increased use of deception in research can translate to legal risk. As has been long discussed, and summarized most recently by \cite{parkResearchersGuideLegal}, researchers conducting studies into software systems and platforms must routinely consider how their activity may face civil or criminal threats.

Before turning to specific laws and their application, it is important to emphasize how uncertainty can affect academic researchers in ways unlike other actors. Commercial actors also experience some of the concerns, such as those related to web scraping. But those actors are comparatively better equipped to take risks around liability. This is perhaps best evidenced by the litany of artificial intelligence companies that built their systems on scraped data, only to now spend the next several years litigating how lawful that was \cite{EthicalTech2023} .

Researchers can rarely take that risk. Beyond their comparatively fewer resources, the consequences of having work being labeled "unlawful" can provide a deeper chill. Universities and other institutions may pressure researchers to back off of work, lest the institution become the "test case" for a new theory of liability \cite{parkResearchersGuideLegal}. Funders may cut support out of concerns of follow-on liability \cite{gilbert2024risks}. Publishers may be disinclined to publish work if the work is deemed unlawful, even if the claim is dubious \cite{mulligan_cybersecurity_2015}. In short, as Bhandari recently summarized, "[f]or every researcher or journalist willing to conduct important research in the face of such uncertainty, there are likely untold numbers of others who would engage in such work but cannot in light of the risks attached" \cite{bhandari2024}.

\subsection{The rise of fraud}
\label{sec:rise_of_fraud}
Classic legal concerns for software and platform research in the United States usually amount to a combination of the Computer Fraud and Abuse Act (CFAA), the Digital Millennium Copyright Act (DMCA), and concerns about claims rooted in breaches of terms of service. Years of litigation and policy debates have gradually developed some degree of consistency or at least common lines of argumentation within those domains, which are reviewed in \Cref{sec:cfaa-dmca}.

An increased reliance on deception as a research tactic has put a focus on an entirely different domain, for which there has been little to no line drawing: claims rooted in fraud.\footnote{This article considers fraud as a standalone legal cause of action, as opposed how fraud is used as an aggravating factor in other laws or as a defense in certain civil claims \cite{podgor_criminal_1999}. It also does not address the ways in which statutes can sound in fraud but be applied in broader ways, such as the law against a conspiracy to "defraud the United States," which includes monetary fraud where the United States is the victim, alongside broader notions of obstruction of government action. 18 U.S.C. § 371.} There is no single statute that encompasses all of the concept of "fraud." It is instead a collection of civil and criminal statutes that all generally operate around the notion that when a person deceives a person and thereby deprives them of something of value, that should be the basis of liability. The branches of modern fraud law owe their origin to the common law tort of deceit, established well over a century prior and with roots that stretch back considerably further \cite{henricksen2021}. Yet despite their history, the scope of these laws is notoriously unclear and constantly evolving \cite{podgor_criminal_1999}.

While there are countless state and federal laws that can be brought against alleged fraud, the most relevant at the federal level are the federal wire fraud statute (18 U.S.C. § 1343), the identity fraud and theft statutes (18 U.S.C. §§ 1028 and 1028A), the "access device" fraud statute (18 U.S.C. § 1029), and the fraud-specific provisions of the CFAA (18 U.S.C. § 1030(a)(4)). Each has unique aspects, but the most capacious is wire fraud, which requires a prosecutor to show that the defendant made (1) a material deception; (2) with an intent to defraud; (3) through the use of interstate wire, radio, or television communication; (4) which, if successful, would have resulted in a loss of money or property \cite{desantis2018}. Its most famous application to research activity was likely the prosecution of Aaron Swartz, where it was asserted alongside the CFAA on the theory that Swartz's scraping of the JSTOR database was wire fraud due to the efforts he undertook to conceal his identity and appear as other web traffic \cite{sellars_impact_2013}. At the federal level most of these require some sort of nexus to interstate communication or commerce, but that has not proved to be a meaningful limit on claims for online activity. Some courts infer interstate communications by the absence of a relevant server in the parties' state, and other courts allow the mere fact that the communication was sent over the Internet to suffice to show requisite interstate wire communication \cite{malkiel2023}.

The rise of risk in fraud also re-introduces the government as a potential threat actor. With the ebb of the CFAA and DMCA reviewed below, and the lack of criminal enforcement of terms of service generally, the risk of \textit{criminal} prosecution for platform research has reduced considerably over time. Most of the concern has shifted to civil actions by the platforms themselves. This is not to say the government has not been a more general threat; in fact, the past few years have underscored just how much government actors can deter unwanted research through congressional subpoenas, cuts to or conditions on funding, and attempts by the government to more directly control the actions of universities \cite{calo_american_2024}. But the rise of claims of fraud would provide even more government actors further ways to surveil, pressure, and otherwise coerce the actions of research in the name of law enforcement. For example, a state seeking to impose mandatory age verification on certain websites might be tempted to use criminal law to target researchers and advocates who question the legitimacy of age estimation tools. A similar situation arose in 2021, when the Governor of Missouri threatened criminal action against a researcher who discovered and disclosed how a government website leaked private data \cite{brodkin_missouri_2021}. Attorneys general and prosecutors critical of social media misinformation research (which is often perceived by conservatives as a threat against their political speech) risk finding new ways to exert pressure on such research when it involves some form of deception. Criminal fraud laws have long been viewed as a substantial risk for prosecutorial overreach \cite{podgor_criminal_1999}. The rise of deception-based research in areas where the government actors may have strong feelings provides both means and motive to threaten such research.

But past fraud litigation against another public interest group---undercover journalists---may provide some insights on how courts can effectively balance claims of fraud against socially valuable newsgathering. Courts have not been consistent in how they approach claims against undercover journalists,\footnote{\textit{Compare} Dietemann v. Time, Inc., 449 F.2d 245 (9th Cir. 1971) (upholding a breach of privacy claim against an undercover journalist who brought a hidden camera into the home office of dubious homeopathic therapies), \textit{with} Desnick v. ABC, Inc., 44 F.3d 1345 (7th Cir. 1995) (denying claims for trespass, wiretapping, fraud, and breach of privacy against undercover journalist who brought a hidden camera into ophthalmologist's office). \textit{Desnick} itself distinguished \textit{Dietemann} by drawing a distinction between use of fraud to enter a business versus use of fraud to enter a home office. \textit{Desnick}, 44 F.3d at 1352–53.} but have provided useful instruction in how to consider their liability against possible claims of fraud which will be relevant to deception research as well.

The following subsections will chart out some common questions or issues in fraud litigation, how they may apply to the research reviewed above, and what past cases against journalists can provide as insights on how to reconcile the inherent tensions present.

\subsubsection{Information as property}
While not all fraud laws explicitly require this \cite{podgor_criminal_1999}, for federal wire fraud one must show that the defendant's scheme worked to deprive (or, if successful, would have deprived) its victim of money or property. The Supreme Court recently reaffirmed that this is tied to "traditional property interests" of the victim, and not a broader set of informational or dignitary harms.\footnote{Ciminelli v. United States, 143 S. Ct. 1121, 1128 (2023).} But in Carpenter v. United States, a case concerning a journalist who leaked news stories to investors ahead of publication, the Court has also said that confidential business information can be the sort of property that fits that definition.\footnote{Carpenter v. United States, 484 U.S. 19, 25 (1987); \textit{see also} United States v. Abouammo, No. 19-cr-621, 2021 WL 718842 at *4 (N.D. Cal. Feb. 24, 2021) (Twitter's confidential user account information was "property" for purposes of § 1343).} It is therefore insufficient for a researcher to evade liability by arguing they did not benefit monetarily. The fact that the researcher learned company secrets through the deception may keep wire fraud and related laws in play. In this manner, the propertization of company secrets presents a similar danger as it does for efforts to mandate algorithmic transparency through legislation or regulation. As reviewed by \cite{kapczynski2020}, the recognition of confidential business information as property in \textit{Carpenter} and in the Takings Clause case Ruckelshaus v. Monsanto Co.\footnote{467 U.S. 986 (1984).} allows companies to challenge mandated transparency by arguing it amounts to a taking to which they are entitled compensation.

What may save deception-based research from these sorts of claims is an insight from cases examining fraud in undercover reporting. In Desnick v. ABC, Inc, a federal appeals court rejected a theory of civil fraud asserted against a reporter in part because "[t]he only scheme here was a scheme to expose publicly any bad practices the investigative team discovered, and that is not a fraudulent scheme."\footnote{\textit{Desnick}, 44 F.3d at 1355.} In other words, if the information the deception uncovers shows that the company is engaged in some sort of inappropriate behavior, some courts appear less likely to uphold a fraud claim for obtaining that information.

\subsubsection{Deception and materiality}
Further complications arise around the "how" and "to whom" of deception. Within wire fraud, the Supreme Court has already established that false or misleading representation must be \textit{material} to the victim,\footnote{Neder v. United States, 527 U.S. 1 (1999).} Civil claims of fraud often instead emphasize not just materiality, but \textit{reliance} \cite{goldberg_place_2006}. That is, that the tort plaintiff in a civil lawsuit must show that they personally depended upon the false or misleading statement to their detriment. Different forms of platform research will perform differently under these distinct tests, but it is certainly possible that deception done more for the purposes of evading detection of ongoing research, as opposed to a direct representation to another, may fare better under at least the civil formulation. On the outer edges of materiality, courts split on whether a victim who gets exactly what they hoped to out of a transaction is actionably defrauded if they would not have initiated the transaction but for the deception \cite{frohock2020}.  For research where a researcher acts as a customer on a website and indeed completes a transaction, this difference in approaches may be significant.

A noteworthy distinction between deception in platform research and most forms of conventional fraud is that very often the object deceived is not a person at all. It is a computational system. \cite{calo2018} considers the question of whether "tricking" a computational system should be thought of as hacking under the CFAA, but their analysis does not examine whether this is actually \textit{fraud} under the relevant provision of the CFAA (§ 1030(a)(4)), instead focusing on the \textit{damage} provisions of §1030(a)(5). Deceiving a computer was core to the arguments in the Aaron Swartz case, where the prosecution alleged that the techniques Swartz used to download files from JSTOR---including switching his MAC and IP addresses and using IP addresses assigned from MIT---was actionable deception under wire fraud, despite the fact that IP addresses are dynamically assigned and nothing technically requires a computer to keep a consistent MAC address \cite{sellars_impact_2013}.\footnote{A similar point was made by X against a commercial scraper, and rejected by the court that considered it. Meta Platforms, Inc. v. Bright Data Ltd., No. 23-cv-77, 2024 WL 251406 at *7 (N.D. Cal. Jan. 23, 2024) (use of IP proxies not actionable under California's unfair business acts law as an unlawful deception, because IP addresses are inherently dynamic).}

\subsubsection{Artifacts of fraud}
Laws like the federal identity fraud and identity theft statues (18 U.S.C. §§ 1028, 1028A) and the access device fraud statute (18 U.S.C. § 1029) regulate the production, use, and trafficking in specific forms of fraudulent documentation--fake IDs, passports, credit card numbers, serial numbers, and so on. The natural question arises how one can design a study to test how good tools are at detecting inauthentic access documents, if making an artifact that is too successful as a forgery is itself unlawful. How, then, can one conduct such research?

The clues may lie not in undercover journalism,\footnote{Indeed, use of fake IDs for uncover research has led to liability. \textit{See} Planned Parenthood Fed. of Am., Inc. v. Ctr. for Medical Progress, 402 F. Supp. 3d 615 (N.D. Cal. 2019) (allowing a RICO claim to proceed with § 1028 as a predicate act, for an advocacy organization who used a fake ID to entered a Planned Parenthood conference).} but in the world of film and television. Prop makers for films also have no general exemption from these sorts of laws, but regularly make convincing replicas for purposes of film and television. The key in that environment has been a combination of careful tailoring for the use case and effort to mitigate any possible uses of the item that would lead to the social harms these laws are meant to address \cite{geaghan-breiner_how_2022}. In effect, the goal for a prop maker is to create an artifact that is convincing \textit{solely and specifically} for the shot, but wholly unconvincing in other contexts. Collection and destruction of the items after the study can also serve as a practical mitigation. Exercising additional due diligence to ensure that the names used do not correlate to real persons (another common step in film and television production) can also help to mitigate these concerns.

\subsubsection{The First Amendment and Fraud}
Of course, researchers are engaging in their deception for the purpose of producing some sort of academic expression, instead of for financial gain like the typical fraud defendant. Does the First Amendment provide them any help? The short answer is the lawyerly "it depends." Advocates have helped courts to recognize that this form of research has First Amendment value and have protected both its methods of information gathering and dissemination, but what that exactly means is more complicated than one might initially believe.

Prior advocates have helped establish that the First Amendment does in fact extend to information gathering activity done in the course of academic research, and the publication of material like source code in order to illustrate aspects of research \cite{parkResearchersGuideLegal}. Most critically, a court considering a constitutional challenge to the CFAA brought on behalf of academic researchers and a media outlet found that "plaintiffs have a First Amendment interest in harmlessly misrepresenting their identities to target websites."\footnote{Sandvig v. Sessions, 315 F. Supp. 3d 1, 16 (D.D.C. 2018). The court eventually dismissed the case, finding that the CFAA did not reach this activity. Sandvig v. Barr, 451 F. Supp. 3d 73, 92 (D.D.C. 2020).} Cases have also found a First Amendment interest in scraping websites for purposes of research and advocacy.\footnote{\textit{See Sandvig}, 315 F. Supp. 3d at 15; S.C. State Conf. of NAACP v. Kohn, No. 22-cv-1007, 2023 WL 144447 at *6–7 (D.S.C. Jan. 10, 2023) (scraping of government records that have been historically made public is protected by the First Amendment).} This benefits academics and journalists alike \cite{baranetsky2018}. On the publication side, scholars have explained why "crime-facilitating speech," or speech that allows others to understand how to commit certain crimes has public value and should be protected in many contexts, even though it could be put to nefarious use \cite{volokhCrimeFacilitatingSpeech2005, matwyshyn2013}. 

In terms of false speech specifically, the Supreme Court's 2012 decision in United States v. Alvarez\footnote{567 U.S. 709 (2012) (plurality opinion).} is instructive. There, the court struck a "stolen valor" statute that punished those who falsely claimed to receive certain military honors. Its reasoning was rooted in the history by which the Supreme Court approached content-based laws of this nature. Absent a specifically enumerated list "historical and traditional" categories where the state may punish speech, any attempt to control even factually false speech must satisfy strict scrutiny.\footnote{\textit{Id.} at 718.} But, in so doing it upheld past cases that denied First Amendment protection for speech used in certain frauds, and repeatedly stated that speech used in a fraud remained unprotected.\footnote{\textit{Id.} at 719 (citing Illinois ex rel. Madigan v. Telemarketing Assocs., Inc., 538 U.S. 600 (2003)).} What that means precisely remains subject to debate. For its part, the prior case that \textit{Alvarez} cites to make this point emphasizes that "[s]imply labeling an action one for `fraud,' of course, will not carry the day," but that intentional misleading for the purpose of receiving monetary donations would still be actionable.\footnote{\textit{Madigan}, 538 U.S. at 617.} The United States Court of Appeals for the Ninth Circuit began to reconcile this tension in a constitutional challenge brought against Idaho's "Ag-Gag" law, which prohibited use of misrepresentations to enter a food production facility, or to obtain records or employment from a facility. The court in that case used \textit{Alvarez} to strike the provisions that criminalized deception to enter a facility, but not those that proscribed deception to secure employment.\footnote{Animal Legal Defense Fund v. Wasden, 878 F.3d 1184, 1190 (9th Cir. 2018).} 

But even if research methods that use deception are protected by the First Amendment, what that protection actually \textit{means} remains ambiguous. Courts are quick to observe that engaging in journalism or other protected activity does not excuse generally applicable laws that are unrelated to the newsgathering activity.\footnote{\textit{See, e.g.}, \textit{Desnick}, 44 F.3d at 1351 ("[T]here is no journalists' privilege to trespass.")} But for laws that attempt to punish certain types of research activity, however, this allows courts to critically examine whether those laws affect a free speech interest, and if they do, whether they meet requisite scrutiny. For an example outside of the deception context, in 2010 the United States Court of Appeals for the Fourth Circuit considered a free speech challenge to a Virginia law that prohibited the publication of another's Social Security Number. The challenge was brought by an advocate who published the Social Security Numbers of several prominent residents that she obtained from Virginia's public land records, to illustrate the state's poor data handling practices. The court held that punishing the advocate under that law would be unconstitutional.\footnote{Ostergren v. Cuccinelli, 615 F.3d 263, 271 (4th Cir. 2010) (noting that posting the numbers themselves was "integral to her message. Indeed, they \textit{are} her message" (emphasis in original)).} 

Recognizing a free speech interest can help a researcher in other ways as well. It may help inform challenges against government actors, including those that attempt to use pretextual allegations of criminality to chill research activity. Free speech principles animate defenses rooted in "anti-SLAPP" law, or statutes designed to achieve both quick dismissal and attorney's fees for defendants when they receive a legal threat related to some form of public advocacy or petitioning.\footnote{\textit{See, e.g.}, X Corp. v. Ctr. for Countering Digital Hate, Inc., 724 F. Supp. 3d 948, 987 (N.D. Cal. 2024), \textit{appeal docketed}, No. 24-2643 (9th Cir. April 25, 2024) (granting a motion to strike under the California anti-SLAPP law against X's attempt to chill research onto its platform).} It also helps courts push back against attempts to recover damages for reputational harms in non-speech-related tort claims, under a line of cases extending back to the Supreme Court's decision in Hustler Magazine, Inc. v. Falwell.\footnote{485 U.S. 46 (1988); \textit{see, e.g.}, \textit{Ctr. for Countering Digital Hate}, 724 F. Supp. 3d at 976–77 (applying this line of cases to deny recovery of reputational damages against a research-based scraper).} This critical attention to which harms are recoverable has been useful in limiting the overall risk to research that breaches a website's terms of service, as typically the amount of recoverable damages in those cases (assuming the researcher did not seriously damage the website in the process) is at or near zero.

\subsection{Long-discussed legal concerns}
\Cref{sec:cfaa-dmca} describes how computer security resesarch has previously come to work in the context of the CFAA, DMCA, and potential terms of service violations. 

The net effect is a mix for researchers. While these claims remain a source of concern and can chill research in the ways reviewed at the beginning of this section, advocates have found ways around these obstacles that can allow research to proceed.

\subsection{Navigating legal barriers}
Despite considerable barriers to their work, academics conducting security, platform auditing, and adversarial ML studies have found ways to continue their research. Many academics simply do not consider the legal risk, or assume that the ethical analysis conducted by an Institutional Review Board (IRB) also handles any legal concerns. Academic researchers tend to place ethical considerations above legal ones when performing risk analyses \cite{gilbert2024risks}. But a more appropriate path is to consider law and ethics each as important but distinct inquiries, and work to craft a study that addresses both concerns. To that end, we provide here some practical guidance to consider when conducting research that uses deception. Of course, anti-fraud laws are only one of several areas of potential legal risks, and the law can vary considerably in different jurisdictions. When doing deception-based work one should consult with a lawyer that with expertise in computer science research who can help further. \cite{parkResearchersGuideLegal} provides further information about engaging with an attorney for research.

\subsubsection{Appreciate that research involves risk.}
Researchers in this area may need to borrow a mentality from the world of journalism, which has long understood that provocative work in their space inherently involves some risk. The only true risk-free path for a newspaper is not to publish at all. Lawyers, similarly, should think of their role not as risk elimination, but rather risk management. A lawyer is bound by their own rules of ethics not to counsel a client to engage in activity that they know to be criminal or fraudulent,\footnote{ABA Model Rule of Professional Conduct 1.2(d)} but they may provide a candid assessment as to the scope of those laws. And outside of crimes and frauds, a lawyer is permitted to advise a client to, for example, engage in "efficient breach" of a contract when the consequences of doing so are favorable to performing as contracted. This is most relevant here when considering whether a researcher should breach a terms of service, which almost always prohibits any attempt to use the website for something other than the proposed business transaction. A lawyer and their research client might rightly consider the fact that 1) the terms may be unenforceable, 2) companies are unlikely or unwilling to enforce them, or 3) the value of the expected research outcome is worth the potential consequence of enforcement. 

It is also important to remember that different people experience risk differently. Researchers may be more or less able to take a risk depending on their identity and status. A tenured professor stands in very different shoes than a graduate student attending school on a student visa, who may risk having their visa revoked if found guilty of a fraud.

\subsubsection{Consider whether permission is feasible.}
Most of the legal risks under fraud, as well as those under the CFAA and other research-related laws, can be neutralized with the consent of the research target. When the work is not inherently adversarial (or when the platform or vendor welcomes adversarial research as a way for them to test their safeguards) a researcher may be able to obtain consent. Such consent should of course by carefully documented, and use of deception to \textit{obtain} the consent will likely lead to more trouble. Working directly with a vendor may also risk having the vendor steer or influence the research in inappropriate ways, and researchers should hold firm to their independence in such negotiations.

Permission can also be indirect. Depending on the research, the researcher may be able to take advantage of a "bug bounty" or similar open call by a platform or vendor to engage in certain types of research. The researcher should be aware, however, that those bounties will likely come with conditions that may frustrate the goals of the research, including non-disclosure obligations or long embargo windows. Violating the terms of the bounty will make it considerably more difficult to rely on the bounty's permission for the activity.

\subsubsection{Analyze the law carefully and avoid unnecessary deception.}
Careful analysis of the relevant law may also reveal paths forward. In the same vein as our own ID study, \cite{di2019personal} presents modified identification documents to online service providers in order to request personal data collected by the service. Di Martino et al. took pains to ensure that their modifications would not run afoul of anti-forgery and identity theft statutes. The authors found that sensitive information, including handwritten signatures and dates of expiry, could be censored in cases where ID images were required for authentication. As such, they took care to alter only those fields when generating modified ID documents. Additionally, modifications were made to \textit{photocopies} of ID documents only. Similar strategies can help when using the law as a research tool. For example, \cite{borradaile2020whose} used public records law to study social media monitoring done by the Oregon State Police. Knowing that records requests for source code were unlikely to be successful, they instead requested access to code \textit{logs}, which they used for their analysis. 

One should also limit any deception to solely what is needed in order to conduct the study. Borrowing from the world of film and television, any objects that are created in order to further the deception should be tailored to work solely for the research and for no other purpose. Use of one-sided IDs, items of a strikingly different weight than an authentic version, or items with obviously fictitious information can be an effective balance between utility for a study and ineffectiveness in other contexts. But one should be mindful that not all laws or courts agree that a fraudulent item needs to be convincing for a person to be liable.\footnote{\textit{Compare Planned Parenthood}, 402 F. Supp. 3d at 650 (declining to adopt a "sufficient quality" element in a charge for identity fraud under § 1028), \textit{with} United States v. Gomes, 969 F.2d 1290, 1293 (1st Cir. 1992) (a charge for possession of a counterfeit Social Security card could only be met if the counterfeit "possesses enough verisimilitude to deceive an ordinary person")} 

\subsubsection{Avoid obtaining things of value}
Finally, because many laws turn on whether the deception was used to obtain something of value (including, in some cases, confidential information), a researcher should try to come up with ways that cut off any benefit for the deception. The use of sandboxed environments for research are helpful; if the researcher is able to construct the environment such that the \textit{sole} information that is received by the researcher is whether the system \textit{would have} let you in if connected to a live website or system, the researcher can cut off an argument that they actually obtained anything of value.\footnote{It is also possible that access to confidential information that is \textit{never put to use} by the researcher would not amount to wire fraud, but this would be practically challenging to prove and may not be found in all jurisdictions. \textit{See} United States v. Czubinski, 106 F.3d 1069, 1075–76 (1st Cir. 1997) (browsing confidential information out of curiosity, but not using it in any way, does not amount to a scheme to defraud).} Along a similar line, a researcher should budget to pay the customary price for access to any paid system, and not use deception to avoid payment. 

