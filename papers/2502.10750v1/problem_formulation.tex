\section{Problem Formulation}
\label{sec:problem formulation}

An HASN graph is denoted as $G(V, E)$, where $\forall v \in V$ is a set of vertices comprising the sets $H$ (human users) and $AI$ (AI entities), such that $|V| = |H| + |AI|$, and $\forall e \in E$ represents the set of edges between humans, AIs, and human-AI connections. 

\textbf{The \problem\ clustering problem} aims to partition an HASN graph into $K$ disjoint subgraphs $C_i(V_i, E_i)$, where $\bigcup_{i=1}^K V_i \subseteq V$ (since AI nodes and their connected edges may be removed during the clustering process) and $V_i \bigcap V_j = \emptyset$, with prior knowledge of which nodes in the network are AI nodes. The goal of \problem\ is to discover a set of clusters (subgraphs) $P = \{ C_i \}_1^K = \{ C_1, C_2, \ldots, C_K \}$ that can maximize human closeness with minimal AI presence. Concretely, a desirable clustering result of an HASN should achieve two key objectives simultaneously: (1) maximizing human closeness and (2) minimizing the presence of AI nodes for each cluster. 

\subsection{Objective Function of \problem}
\label{subsec:objective_function}

To achieve the goal of \problem, we employ a modularity function introduced in a seminal work by Newman as our objective function \cite{newman2004finding}:

\begin{equation}
Q(P=\{C_i\}_{i=1}^K) = \frac{1}{2|E|} \left( \sum_{i=1}^K \sum_{v_p, v_q \in C_i}\left( A_{pq} - \frac{d_p d_q}{2|E|} \right) \right)
\end{equation}
\vspace{0.5em}

Modularity $Q$ measures clustering quality in networks by comparing the density within clusters to the density between clusters. It ranges from -0.5 to 1, with higher scores indicating better clustering. Here, $A$ is the adjacency matrix, $A_{pq}$ indicates the presence of a connection between nodes $p$ and $q$, and $d_p$ is the degree of node $p$. 

To encourage the clustering algorithm to generate cohesive communities with minimal AI presence, we modify the vanilla modularity by infusing a reward-penalty function. This function reweights the clustering quality based on the ratio of humans (and AIs) presence in each cluster $C_i$, defined by:

\begin{equation}
W(C_i) = \beta \cdot \frac{\sum_{v \in C_i} L_v}{|C_i|} - \gamma \cdot \frac{\sum_{v \in C_i} (1 - L_v)}{|C_i|}
\end{equation}
\vspace{0.5em}

\noindent where 
\begin{equation}
L_v =
\begin{cases} 
1, & \text{if node } v \in H \\
0, & \text{if node } v \in AI
\end{cases}
\end{equation}
\vspace{0.5em}

\noindent This leads to a human-centric modularity $HQ$:

\begin{equation}
HQ(P) = \frac{1}{2|E|} \left( \sum_{i=1}^K \alpha \cdot W(C_i) \cdot \left( \sum_{v_p, v_q \in C_i}\left( A_{pq} - \frac{d_p d_q}{2|E|} \right) \right) \right)
\end{equation}
\vspace{0.5em}

\noindent Note that $\beta$ is the weight for rewarding human nodes, $\gamma$ is the weight for penalizing AI nodes, and $\alpha$ is the weight for adjusting the emphasis on human nodes in the objective function \footnote{For simplicity, we set $\alpha$, $\beta$, and $\gamma$ to 1 in our experiments to observe the proposed algorithm’s core behavior without the added complexity of multiple parameters.}. Accordingly, the purpose of \problem\ is to discover a set of clusters (subgraphs) $P = \{ C_i \}_1^K$ that maximizes $HQ$:

\begin{equation}
P^* = \arg \max_{\{C_i\}_{i=1}^k} HQ(\{C_i\}_{i=1}^K)
\end{equation}
\vspace{0.5em}

This objective function promotes the generation of tight-knit communities with minimal AI presence. Since certain AI entities can aid in the formation of these human-centric communities, it is crucial to identify and preserve AI nodes that can promote human closeness while removing those that can not.

