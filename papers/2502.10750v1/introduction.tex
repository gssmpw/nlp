\begin{figure}[t]
    \centering
    \includegraphics[width=0.76\linewidth]{image/HASN.png}
    \caption{A schematic depiction of HASN, illustrating the interconnected relationships between human users and AI entities. This paper seeks to identify \textit{human-centric communities} within this hybrid network.}
    \label{figure:hasn}
\end{figure}


\section{Introduction}
\label{sec:introduction}

Community detection is a fundamental problem in social network analysis (SNA), focusing on identifying tightly-knit communities with minimal external connections \cite{mcpherson2001birds}\cite{su2022comprehensive}\cite{jin2021survey}. This task is critical for analyzing social relationships and holds practical applications in marketing and personalized services, where insights into community structures enable targeted strategies and enhance user engagement \cite{du2007community}\cite{umrawal2023community}.


The integration of generative AI and Metaverse is transforming traditional social networks, creating human-AI social networks (HASNs) (Figure \ref{figure:hasn}) that blend human and AI entities within shared digital spaces. In Metaverse environments, which combine virtual reality (VR) and mixed reality (MR) components, users engage with both humans and AI-driven entities, such as avatars and virtual assistants, forming communities that span the physical and digital worlds. For example, social Metaverse platforms like Microsoft AltspaceVR \cite{meijers2021globalxr} and Meta Horizon Workrooms \cite{meta2023workrooms} enable immersive, large-scale virtual gatherings, allowing users to interact with AI alongside human participants through customizable 3D environments, spatial audio, and interactive tools.

\begin{figure*}[t]
    \centering
    \includegraphics[height=0.2275\linewidth]{image/4_illustrative.pdf}
    \caption{Illustrative examples of community detection in an HASN.}
    \label{illustrative}
\end{figure*}

In these evolving HASNs, a fundamental shift occurs: traditional social clusters now include AI nodes, making human-centric community detection crucial. Unlike conventional networks, HASNs often aim to prioritize human interactions while accommodating AI entities that provide social support or assistance. For instance, platforms such as Nomi AI \cite{nomiAI2024} create AI companions for users, enhancing social engagement and addressing issues like loneliness. Similarly, platforms like Engage \cite{engage2023} and Mozilla Hubs \cite{hubs2023mozilla} host virtual events and social gatherings, integrating AI to foster rich, interactive experiences. These hybrid networks demand new community detection methods that discover human-centric communities by selectively retaining AI nodes that strengthen community cohesion and removing those that do not. This approach supports applications focused on authentic human engagement, including marketing, user recommendations, and digital companionship.

However, applying traditional community detection methods to these hybrid networks presents several challenges. For instance, (1) methods that ignore AI entities may result in communities with an excess of AIs, which are less effective for human-centric applications such as advertising and recommendation, as these efforts are not relevant for AI entities. Moreover, this approach can also lead to incorrect community assignments for human members. (2) Alternatively, one might consider removing all AI nodes before conducting community detection. However, this could disrupt connections between AI and human nodes, as well as among AI nodes themselves. Since community detection relies heavily on graph topology, these disruptions may lead to inaccurate community assignments for human users. (3) A third approach could involve using anomaly detection techniques to classify certain AI nodes as anomalies and remove them before clustering. However, AI node behavior often diverges from conventional “anomalies” in traditional social networks \cite{ma2021comprehensive}\cite{lamichhane2024anomaly}. Unlike typical outliers, AI nodes may mimic human behavior, support human interactions, and integrate into communities, making this strategy likely to yield unsatisfactory results.

In this work, we introduce a novel community detection problem tailored to scenarios where humans and numerous AI entities are intertwined within social networks\footnote{To the best of our knowledge, this is the first study to explore community detection in the hybrid human-AI scenario. While focusing on structural information, our approach is designed to be compatible with future integration of semantic features, enabling more refined human-centric clustering based on shared interests or interaction patterns.}, emphasizing human closeness derived from the graph structure. We envision an \XR\ environment where a social network comprises both human users and AI entities, forming what we term the Human-AI Social Network (HASN), as illustrated in Figure \ref{figure:hasn}. Accordingly, we envisage four scenarios of HASNs existing in \XR\ \cite{techcrunch_meta_2024}: (1) Random Interaction, (2) Introverted Humans Prioritized for Interaction, (3) Distinct AI Types, and (4) Dual-personality AIs. These proposed scenarios are grounded in the versatility of AI, allowing it to interact widely and adaptively with diverse users \cite{wang2024survey}. (Details are provided in the experimental setup, Section \ref{subsec:experimental_setup}.)


This work focuses on community detection in HASNs, especially when the primary focus is on human users (human-centric communities). Ideally, clustering should produce clusters with high human closeness and minimal AI presence, balancing the removal of AI nodes while maintaining community integrity. Let us explore some illustrative examples as follows. Figure \ref{illustrative} shows several approaches to cluster an HASN depending on how they consider the AI nodes: (1) \textit{AI-complete clustering}: Figure \ref{illustrative} (a) shows a clustering result from a method that ignores AI nodes and performs clustering directly. However, as observed, each community contains an excessive number of AI entities, making it impractical for human-centric applications such as advertising and recommendation. In addition, this approach may result in inaccurate community assignments for human members (e.g., A and D, C and E). 
(2) \textit{AI-blind clustering}: Figure \ref{illustrative} (b) shows a clustering result from a method that treats all AI nodes as outliers and removes them before clustering. Human nodes A and D are initially connected to node F through AI node AI-3, suggesting that A, D, and F may belong to the same community. However, this approach can disrupt connections between AI and human nodes, as well as among AI nodes, resulting in the loss of important links and leading to unsatisfactory community results. (3) \textit{AI-anomaly clustering}: Figure \ref{illustrative} (c) shows a clustering result where certain AI nodes are identified as anomalies through anomaly detection techniques and removed before clustering. However, unlike typical anomalies in traditional social networks, which often link different communities or form dense connections \cite{ma2021comprehensive}\cite{lamichhane2024anomaly}, AI nodes may mimic human behavior, support human interactions, and integrate into communities. Consequently, as shown, traditional anomaly detection would remove AI-2 and AI-3, resulting in inaccurate community outcomes for human users (e.g., A and F). (4) \textit{AI-aware clustering}: Figure \ref{illustrative} (d) illustrates clustering results achieved by identifying AI nodes that serve as bridges between humans or connect key human nodes. For instance, AI-3 is an AI node retained during the clustering process because it effectively links humans A, D, and F. This connectivity does not occur in the aforementioned three methods (i.e., Figure \ref{illustrative} (a), (b), and (c)). In other words, this approach preserves AI nodes that positively impact the community while removing those that do not, thereby enhancing human closeness and fostering potential human-centric communities.


Considering the factors depicted above face the following challenges: (i) \textit{Evaluation of AI nodes}: The behavior patterns of AI nodes may not resemble the typical ”anomalies” found in traditional social networks \cite{ma2021comprehensive}. AI may mimic human behavior, assist humans, and integrate into communities. In other words, some AI nodes are helpful for the formation of potential communities, while some are redundant. Accordingly, traditional anomaly detection methods are inadequate for evaluating AI nodes in emerging HASN graphs. (ii) \textit{Tradeoff between AI removal and community integrity}: Removing AI nodes could lead to disconnections between AI and humans, as well as among AIs themselves. Therefore, it is crucial to identify and preserve AI nodes that enhance human closeness while removing those that do not contribute positively. For instance, some AI nodes act as bridges, enhancing human connections and facilitating the formation of potential communities. (iii) \textit{Avoidance of local optima in AI-aware clustering}: The search space for finding the optimal combination of AI preservation/removal expands exponentially as the number of AI increases. This complexity makes this problem hard to solve and may result in solutions that settle at the local optima.

In this paper, we formulate a new problem, termed \textbf{\textit{human-centric community detection in hybrid networks (HASNs) of \XR}} (denoted by \problem). Unlike previous studies focusing on community detection in purely human networks without considering AI involvement \cite{su2022comprehensive}\cite{jin2021survey}, our approach explores community detection within hybrid networks, such as HASNs (illustrated in Figure \ref{figure:hasn}), composed of both human users and AI entities.  Given an HASN with prior knowledge of which nodes in the network are AI nodes, our goal is to generate clusters that can \textit{maximize human closeness with minimal AI involvement}, by balancing the removal of AI nodes and maintaining community integrity. Specifically, a desirable clustering result of an HASN should achieve two key objectives simultaneously: (1) maximizing human closeness and (2) minimizing the presence of AI nodes within each cluster \footnote{For more information on potential application scenarios of MetaCD, please refer to Appendix \ref{sec: Potential Application Scenarios}.}.

To this end, we design a novel algorithm called \textit{\underline{Cu}stomized AI-aware \underline{S}imulated \underline{A}nnealing for Clustering} (denoted by CUSA), to tackle the above challenges of \problem. \textbf{(1)} For the first challenge in the evaluation of AI nodes, CUSA incorporates \textit{AI Scoring} to evaluate the "\score" of an AI node. Specifically, if an AI node acts to a greater extent as a bridge between humans or connects important human nodes, its \score\ is higher, and vice versa. This is because it can enhance human closeness and potentially lead to the formation of new human communities or the migration of humans to different communities due to the influence of such AI nodes. \textbf{(2)} For the second challenge in the tradeoff between AI removal and community integrity, we develop the \textit{AI-aware Louvain clustering algorithm} into CUSA. This algorithm adaptively groups nodes based on human closeness gain while also taking into account the proportion of AI presence in each community during clustering. \textbf{(3)} For the third challenge in avoidance of local optima in AI-aware clustering, CUSA infuses \textit{AI-aware adaptive clustering (3AC) framework with a probability-based escape strategy}. This framework adaptively partitions an HASN by removing the AI node with the lowest \score. In addition to removing (or preserving) AI nodes based on their \score, we employ a pre-defined probability distribution to escape local optima. We evaluate the performance of CUSA on benchmark real-world social networks (i.e., Cora,  CiteSeer, and PubMed) transformed into HASNs (Figure \ref{figure:hasn}) with our proposed generation strategies. The contributions of this work include:

\begin{itemize}
    \item To the best of our knowledge, \problem\ is the first attempt to study the community detection problem under the new scenarios where humans and numerous AI entities are intertwined in social networks (denoted by HASNs), especially focusing on human-centric communities. In addition, we propose four HASN scenarios, each with specifically designed generation strategies.
    
    \item To effectively address \problem, we develop a novel algorithm called CUSA, incorporating AI-aware clustering techniques that navigate the delicate trade-off between removing AI nodes and maintaining community structure by selectively retaining AI nodes that contribute to community integrity. 
    
    \item Empirical evaluations on real social networks (reconfigured as HASNs) demonstrate the effectiveness of CUSA in forming human-centric communities compared to competitive baselines.

    \item The experiments show that carefully designed generation strategies can improve clustering results, enhancing human closeness and revealing potential communities.
    
\end{itemize}

%The remainder of this paper is organized as follows. Section \ref{sec:related_work} introduces the related works. Section \ref{sec:problem formulation} formulates \problem\ and designs CUSA algorithm in Section \ref{sec:method}. The experimental setup and results are presented in Section \ref{sec:experiments}. Section \ref{sec:conclusion} concludes this paper.