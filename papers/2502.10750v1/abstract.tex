\begin{abstract}
Community detection is a cornerstone problem in social network analysis (SNA), aimed at identifying cohesive communities with minimal external links. However, the rise of generative AI and Metaverse introduce complexities by creating hybrid human-AI social networks (denoted by HASNs), where traditional methods fall short, especially in human-centric settings. This paper introduces a novel community detection problem in HASNs (denoted by MetaCD), which seeks to enhance human connectivity within communities while reducing the presence of AI nodes. Effective processing of MetaCD poses challenges due to the delicate trade-off between excluding certain AI nodes and maintaining community structure. To address this, we propose CUSA, an innovative framework incorporating AI-aware clustering techniques that navigate this trade-off by selectively retaining AI nodes that contribute to community integrity. Furthermore, given the scarcity of real-world HASNs, we devise four strategies for synthesizing these networks under various hypothetical scenarios. Empirical evaluations on real social networks, reconfigured as HASNs, demonstrate the effectiveness and practicality of our approach compared to traditional non-deep learning and graph neural network (GNN)-based methods.
\end{abstract}
