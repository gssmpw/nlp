\section{Related work}
\label{sec:related_work}

 %從問題的面向著手還是方法。 %傳統方法,然後之後DNN效果更好 ,Although 這些方法 perform well for CD 於舊網路的一些議題, 但沒考慮人類與AI交織的新network問題。解決之道。
\noindent\textbf{Community Detection.} Community detection is the process of grouping nodes into densely connected communities with sparse interconnections, depending on the structure of the graph. Early traditional non-deep learning methods, such as the spectral clustering algorithm \cite{amini2013pseudo}, optimize ratio and normalized-cut criteria. Louvain \cite{blondel2008fast} is a well-known optimization algorithm that uses a node-moving strategy to optimize modularity. Extensions of greedy optimizations include simulated annealing \cite{kirkpatrick1983optimization}, extremal optimization \cite{boettcher2002optimization}, and spectral optimization \cite{newman2013spectral}.Recently, deep learning-based community detection techniques \cite{su2022comprehensive}\cite{lecun2015deep} have gained traction, with graph neural networks (GNNs) emerging as a key trend. These methods learn lower-dimensional representations from high-dimensional structural data \cite{khoshraftar2024survey} \cite{tsitsulin2023graph}, capturing diverse information from nodes \cite{ge2024unsupervised}, edges \cite{chikwendu2023comprehensive}, neighborhoods \cite{wang2023overview}, or multigraphs \cite{pan2021multi}. Moreover, deep learning effectively uncovers community structures in large-scale \cite{fang2020survey}, sparse \cite{wu2020deep}, complex \cite{shao2024distributed}, and dynamic networks \cite{pan2024identification}. Despite the extensive exploration of various community detection methods and their application in different real-world scenarios, most existing approaches generally operate under the common assumption that social networks are solely composed of human users. In this work, we investigate a novel community detection problem tailored for social networks where humans and numerous AI entities coexist and interact. To address this, we propose a new method that prioritizes human-centric communities while effectively addressing the complexities introduced by AI entities within human-AI social networks (HASNs).

\noindent\textbf{Graph Anomaly Detection.} Graph Anomaly Detection (GAD) is designed to identify unusual patterns, outliers, or unexpected behaviors in graph-structured data \cite{ma2021comprehensive}\cite{lamichhane2024anomaly}. GAD techniques have proven effective in various applications, including computer network intrusion detection \cite{bilot2023graph}, fraud detection \cite{xiang2023semi}, and anomaly detection in social networks \cite{yu2016survey}\cite{roy2024gad}. In contrast, our work addresses a novel issue by examining interactions between humans and AI in a newly defined social network (HASN). Unlike traditional social networks, where “anomalies” are typically detected through unusual connection patterns—such as bridging distinct communities or establishing dense links with other nodes \cite{ma2021comprehensive}—AI behavior may not inherently appear abnormal. AI can mimic human behavior, assist humans, and seamlessly integrate into communities. Consequently, this distinction makes existing GAD methods ill-suited for addressing our new problem.

\noindent\textbf{Generative Artificial Intelligence.} Generative Artificial Intelligence (GAI) is a form of AI capable of autonomously generating new content, including text, images, audio, and video. The current mainstream approach to realizing GAI involves training large language models (LLMs) \cite{guo2024large}. Various applications of LLMs have rapidly emerged, including ChatGPT \cite{achiam2023gpt}, Gemini \cite{team2023gemini}, and Claude \cite{AnthropicAI2023}. These LLM applications have significantly transformed our lives by adding convenience in areas such as file summarization, code generation, learning assistance, providing inspiration, and even offering life advice and psychological counseling. Building on these observations, we envision a future where social networks are seamlessly integrated with both humans and AIs. We aim to perform community detection within this hybrid network. As far as we are concerned, this concept has not been explored in previous studies, making our work a pioneering effort in this field.