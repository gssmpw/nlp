\section{Algorithm Design for \problem}
\label{sec:method}

\begin{figure*}[ht]
    \centering
    \includegraphics[height=0.335\linewidth]{image/CUSA_flowchart.pdf}
    \caption{The framework of the proposed CUSA algorithm.}
    \label{figure:CUSA flowchart}
\end{figure*}

To effectively address \problem, we develop a novel algorithm, named \textit{\underline{Cu}stomized AI-aware \underline{S}imulated \underline{A}nnealing} (denoted by CUSA), which integrates three key components (\textit{cf.} Figure \ref{figure:CUSA flowchart}).

\textbf{(i)} To tackle the challenge in the evaluation of AI nodes, CUSA incorporates an \textit{\textbf{AI Scoring}} mechanism to assess the \score\ score of each AI node in a human-AI social network (HASN). CUSA achieves this by assigning distinct weights to edges based on neighboring relationships. Specifically, weights are assigned as follows: human-human edges are denoted by $hh$, human-AI edges by $ha$, and AI-AI edges by $aa$, with the relationship $hh > ha > aa$ (defaults set to 3, 2, and 1, respectively)\footnote{The sensitivity of $hh, ha, aa$ is discussed in Appendix \ref{Sensitivity to hh, ha, aa}.}. CUSA then applies three representative approaches commonly used in social networks—eigenvector centrality, betweenness centrality, and clustering coefficient—to evaluate AI node scores. This encourages the identification and preservation of AI nodes that play significant roles in human-centric communities, aligning with MetaCD’s objectives. Each approach serves a specific purpose:

\noindent (1) Eigenvector Centrality (EC) \cite{freeman2002centrality} considers a node important if it connects to other highly influential nodes. We apply this measure, assuming that AI nodes in the HASN are linked to prominent individuals (or vice versa). Thus, prioritizing AI nodes with high eigenvector centrality in clustering is vital for shaping cohesive community structures and enhancing human closeness.

\noindent (2) Betweenness Centrality (BC) \cite{brandes2001faster} identifies a node as crucial if it serves as a bridge or intermediary within the network. This measure aligns with the objectives of the \problem\ problem, as a beneficial AI node often connects disparate human nodes, potentially reshaping community boundaries and reinforcing human closeness. Therefore, we prioritize preserving AI nodes with high betweenness centrality during clustering \footnote{To mitigate the computational cost of the shortest-path calculations in BC, we utilize an optimized version of the BC evaluation method \cite{brandes2001faster}.}.

\noindent (3) Clustering Coefficient (CC) \cite{watts1998collective}  regards a node as important if its neighbors are densely interconnected. However, in our approach, we prioritize retaining AI nodes with low clustering coefficient scores. This preference is based on the observation that a high clustering coefficient suggests ample interconnectivity within the subnetwork, thereby reducing the necessity for the AI node to act as a bridge among other nodes.

As shown in Table \ref{AI node scoring methods} in the experimental section, the combined use of EC, BC, and CC through a linear combination of their normalized scores yields the most effective results. This indicates that each measure (EC, BC, and CC) captures a distinct aspect of information regarding AI nodes, providing a more comprehensive perspective on their significance within the HASN.

\textbf{(ii)} For the tradeoff between AI removal and community integrity, we develop \textit{\textbf{AI-aware Louvain Clustering Algorithm}} into CUSA. The Louvain clustering algorithm \cite{blondel2008fast} optimizes modularity to detect community structures in networks. Specifically, each node is initially treated as an individual community. For each node $i$ (or community $C_i$), we calculate the modularity gains $\Delta Q$ by moving node $i$ to its neighboring communities $C_j$, is defined as:

\begin{align}
\Delta Q(i \rightarrow C_j) = &\left[ \frac{\Sigma in + k_{i,in}}{2|E|} - \left(\frac{\Sigma tot + k_i}{2|E|}\right)^2 \right] \notag \\
&- \left[ \frac{\Sigma in }{2|E|} - \left(\frac{\Sigma tot}{2|E|}\right)^2 - \left(\frac{k_i}{2|E|}\right)^2 \right]
\end{align}
\vspace{0.5em}

\noindent where $\Sigma in$ is the sum of the weights of the links inside the community $C_j$, $\Sigma tot$ is the sum of the weights of the links incident to nodes in the community $C_j$, $k_i$ is the sum of the weights of the links incident to node $i$, and $k_{i, in}$ is the sum of the weights of the links from node $i$ to nodes in the community $C_j$. At each iteration, communities are aggregated into new super-nodes, condensing into separate communities. This iterative process of evaluating node movements and merging communities continues until no further increase in modularity can be achieved\footnote{The derivation of modularity gain $\Delta Q$ are provided in Appendix \ref{sec: Derivation of delta Q}}.

\begin{algorithm}[t]
\caption{The algorithm of CUSA}
\label{alg:CUSA}
\begin{algorithmic}[1]
\Statex \textbf{Input:}  An HASN $G = (V, E)$ where $|V| = |H| + |AI|$
\Statex \textbf{Output:} $P_{best} = \{ C_i \}_1^K$, the $K$ disjoint clusters (subgraphs) $C_i$, each achieving high human closeness with minimal AI presence (namely "human-centric communities")
\STATE $T \gets T_{initial}$
\WHILE{($T > T_{min}$) \textbf{and} $|AI| > 0$}
    \STATE $rnode \gets \texttt{AIScoring($G$, $AI$)}$
    \STATE $G_{new} \gets \texttt{AIRemove($G, rnode$)}$
    \STATE $P_{new} \gets \texttt{Clustering($G_{new}$)}$
    \STATE $P \gets \texttt{Clustering($G$)}$
    \STATE $DE = HQ(P_{new}) - HQ(P)$
    \IF{$DE > 0$} 
            \STATE $G \gets G_{new}$
    \ELSE 
        \IF{$p = e^{DE / T} > U$ where $U \sim \text{Uniform}(0,1)$}
            \STATE $G \gets G_{new}$ 
        \ENDIF
    \ENDIF
    \IF{$HQ(P) \geq HQ(P_{best})$}
        \STATE $P_{best} \gets P$
        \STATE $T \gets f(T)$
    \ENDIF
\ENDWHILE
\RETURN {} $P_{best}$

\end{algorithmic}
\end{algorithm}


To make the Louvain clustering algorithm more human-specific, we incorporate two reweighting terms into the modularity calculation, $W(C_j)^{before}$ and $W(C_j)^{after}$. These terms (defined in Section 3.1) are based on the human-to-AI ratio within the community $C_j$ before and after merging node $i$ (or community $i$), respectively. This approach makes the AI-aware Louvain clustering algorithm more human-specific by not only considering structural information in each merge step but also emphasizing the proportion of human members within each community. When node $i$ (or community $i$) is merged into community $C_j$, these reweighting terms assess whether the merge increases the proportion of human members, leading to a human-centric modularity gain $\Delta HQ$. This ensures that the resulting communities display a higher human proportion and cohesiveness, fostering a more human-centered clustering outcome. Mathematically, this is achieved with the following formula:

\begin{align}
\Delta HQ(i \rightarrow C_j) = 
&\ \alpha \cdot W(C_j)^{\text{after}} \cdot \left[ \frac{\Sigma \text{in} + k_{i,\text{in}}}{2|E|} 
- \left(\frac{\Sigma \text{tot} + k_i}{2|E|}\right)^2 \right] \nonumber \\
&- \alpha \cdot W(C_j)^{\text{before}} \cdot \left[ \frac{\Sigma \text{in}}{2|E|} 
- \left(\frac{\Sigma \text{tot}}{2|E|}\right)^2 
- \left(\frac{k_i}{2|E|}\right)^2 \right]
\end{align}

\noindent where $\alpha$ is a scaling factor that can adjust the emphasis on the presence of human nodes in the cluster (default set to 1). By iteratively applying this formula, the algorithm effectively identifies a high-modularity partitioning of the network, ensuring that clusters contain a higher proportion of humans.

\textbf{(iii)} To prevent getting trapped in local optima during AI-aware clustering, CUSA infuses the \textit{\textbf{AI-aware Adaptive Clustering (3AC) Framework with a Probability-based Escape Strategy}}. This framework adaptively partitions an HASN by iteratively removing the AI node with the lowest \score, leveraging a probability-based escape strategy to enhance search flexibility. The 3AC framework is central to CUSA’s process, and its steps are as follows:

\begin{enumerate}
    \item Evaluate \score\ for all AI nodes of the HASN graph and rank them.
    \item Apply the probability-based escape strategy to remove the AI node with the lowest \score.
    \item Cluster all remaining nodes to obtain partition $P$.
    \item Calculate the $HQ$ for partition $P$.
    \item Re-evaluate \score\ for all remaining AI nodes and rank them again.
    \item Repeat 2-5 until the $HQ$ of $P$ converges to its highest value.
\end{enumerate}

CUSA precisely implements the core steps of the 3AC framework, advancing through each stage to achieve the framework’s objectives. Specifically, CUSA comprises three major components outlined in the 3AC framework: (1) an evaluation of AI nodes’ importance (referred to as \score), informing the selective retention or removal of AI nodes based on their relevance to human-centric communities; (2) the AI-aware Louvain clustering algorithm, which performs clustering while accounting for the human-to-AI ratio within each community to ensure a human-centric structure; and (3) a probability-based escape strategy that dynamically adjusts the likelihood of removing nodes, helping the algorithm escape local optima and explore the solution space more effectively. Through these integrated components, CUSA iteratively adjusts the community structure by evaluating and balancing the human and AI nodes in each iteration. The pseudo-code for CUSA is presented in Algorithm \ref{alg:CUSA}, while the overall flowchart of the CUSA algorithm is illustrated in Figure \ref{figure:CUSA flowchart} for further understanding. \\

\noindent \textbf{Theorem 4.1.} The time complexity of CUSA is $O(|AI|(|V||E| + |V|^2 \log |V|))$, while the space complexity is $O(|V| + |E|)$.
\begin{proof}
Detailed derivations are provided in Appendix \ref{sec: Computational Complexity of CUSA}.
\end{proof}


