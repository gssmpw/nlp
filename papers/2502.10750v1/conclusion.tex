\section{Conclusion}
\label{sec:conclusion}
To the best of our knowledge, this paper is the first to investigate human-centric community detection in hybrid networks (HASNs) that contain human and AI nodes. Based on four proposed HASN scenarios, we introduce a new problem, \problem, which aims to identify clusters that maximize human closeness while minimizing AI presence. To address \problem, we develop a novel algorithm, CUSA, which leverages AI-aware clustering techniques to balance the trade-off between AI removal and community integrity. Empirical evaluations on real social networks, reconfigured as HASNs, demonstrate the effectiveness of CUSA compared with state-of-the-art methods. Furthermore, we observe that tailored generation strategies can further enhance clustering outcomes, providing valuable insights for enterprises developing AIs to foster human connections and uncover latent communities.

\section*{Acknowledgments}
This work is supported in part by the NSTC, Taiwan, under grants 113-2223-E-002-011, 113-2221-E-001-016-MY3, 112-2221-E-001-010-MY3, and 111-2221-E-002-135-MY3, by the Ministry of Education, Taiwan, under grant MOE 113L9009, and by Academia Sinica, Taiwan, under Academia Sinica Investigator Project Grant AS-IV-114-M06. We thank the NCHC for providing computational and storage resources.

