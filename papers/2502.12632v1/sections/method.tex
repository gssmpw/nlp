\section{\sname Diffusion with \lname}
\label{sec:method}
Consider a dataset $\mathcal{D}$, where each example $(\bx, \bc) \in \mathcal{D}$ consists of a video $\bx$ and corresponding conditions $\bc$ (\eg, text captions). Our goal is to train a model using $\mathcal{D}$ to learn a model distribution $p_{\text{model}}(\bx | \bc)$ that matches a ground-truth conditional distribution $p_{\text{data}} (\bx | \bc)$. In particular, we are interested in the situation where each $\bx$ is a \emph{long} video, and video lengths are much larger than conventional methods that often use $\sim$20-128 frames for both training and inference~\citep{he2022lvdm}.

To efficiently model long video distribution, we adopt a ``memory'' that encodes previous long context in our latent diffusion transformer. Specifically, we aim to train a single model capable of: (a) encoding previous context of the long video as a compact memory latent vector, and (b) generating a future clip conditioned on the memory and $\bc$. 

In the rest of this section, we explain \sname Diffusion in detail. In Section~\ref{subsec:ldm}, we provide a brief overview of latent diffusion models. In Section~\ref{subsec:obj}, we describe problem formulation and how we design a training objective. Finally, in Section~\ref{subsec:arch}, we explain the architecture that we used.

\vspace{0.02in}
\noindent\textbf{Notation.}
We write a sequence of vectors $[\bx^{a} \ldots, \bx^{b}]$ with $a<b$ as $\bx^{a:b}$. 

\subsection{Latent diffusion models}
\label{subsec:ldm}
In order to generate data, 
diffusion models learn the \emph{reverse} process of a forward diffusion, where the forward diffusion diffuses an 
example $\bx_{0} \sim p_{\text{data}}(\bx)$ to a (simple) prior distribution $\bx_{T} \sim \mathcal{N}(\mathbf{0}, \sigma_{\mathrm{max}}^2 \mathbf{I})$ (with pre-defined $\sigma_{\mathrm{max}}>0$) with the following stochastic differential equation (SDE):
\begin{align}
    d\bx = \mathbf{f} (\bx, t) dt + g(t) d\mathbf{w},
    \label{eq:forwardsde}
\end{align}
where $\mathbf{f}$, $g$, $\mathbf{w}$ are pre-defined drift and diffusion coefficients, and standard Wiener process (respectively) with $t \in [0, T]$ with $T>0$. 
With this forward process, data sampling can be done with the following reverse SDE of Eq.~\eqref{eq:forwardsde}: 
\begin{align}
    d\bx = \Big[ \mathbf{f} (\bx, t) - \frac{1}{2} g(t)^2 \nabla_{\bx} \log p_t (\bx) \Big] dt + g(t) d\mathbf{\bar{w}},
\end{align}
where $\mathbf{\bar{w}}$ is a standard reverse-time Wiener process, and $\nabla_{\bx} \log p_t(\bx)$ is a score function of the marginal density from Eq.~\eqref{eq:forwardsde} at time $t$.
\citet{song2021scorebased} shows there exists a \emph{probability flow ordinary differential equation (PF ODE)}
whose marginal $p_t (\bx)$ is identical for the SDE:
\begin{align}
    d\bx = \Big[ \mathbf{f} (\bx, t) - \frac{1}{2} g(t)^2 \nabla_{\bx} \log p_t (\bx) \Big] dt.
    \label{eq:pfode}
\end{align}
Following previous diffusion model methods~\citep{lee2024dreamflow,zheng2024fast}, we use the setup in EDM~\citep{karras2022edm} with $\mathbf{f} (\bx, t) \coloneqq \mathbf{0}$, $g(t)\coloneqq \sqrt{2\dot\sigma(t)\sigma(t)}$ and a decreasing noise schedule $\sigma: [0, T] \to \mathbb{R}_{+}$. In this case, the PF ODE in Eq.~\eqref{eq:pfode} can be written:
\begin{align}
    d\bx = -\dot\sigma(t)\sigma(t)\nabla_{\bx}\log p (\bx; \sigma(t)) dt,
\end{align}
where  $p(\bx; \sigma)$ is
the smoothed distribution by adding i.i.d Gaussian noise $\bm{\epsilon}\sim\mathcal{N}(\mathbf{0}, \sigma^2 \mathbf{I})$ with standard deviation $\sigma > 0$. To learn the score function $\nabla_{\bx}\log p (\bx; \sigma(t))$, we train a denoising network $D_{\bm{\theta}} (\bx, t)$ with the denoising score matching (DSM)~\citep{song2019generative} objective for all $t \in [0, T]$:
\begin{align*}
\mathbb{E}_{\bx, \bm{\epsilon}, t}\Big[ \lambda({t}) || D_{\bm{\theta}}(\bx_0 + \bm{\epsilon}, t)  - \mathbf{\bx}_0||_2^2 \Big],
\end{align*}
where $\lambda(\cdot)$ assigns a non-negative weight.

However, training $D_{\bm{\theta}}$ directly with raw high-dimensional $\bx$ is computation and memory expensive. To solve this problem, latent diffusion models~\citep{rombach2021highresolution} first learn a 
lower dimensional latent representation of $\bx$ by training an autoencoder (with encoder $F(\bx) = \bz$ and 
decoder $G(\bz) = \bx)$ to reconstruct $\bx$ from 
the low-dimensional vector $\bz$, and then train $D_{\bm{\theta}}$ to generate in this latent space instead.
Specifically, latent diffusion models use the following denoising objective defined in the latent space:
\begin{align*}
    \mathbb{E}_{\bx, \bm{\epsilon}, t}\Big[ \lambda(t) || D_{\bm{\theta}}(\bz_0 + \bm{\epsilon}, t)  - \mathbf{\bz}_0||_2^2 \Big],
\end{align*}
where $\bz_0=F(\bx_0)$. After training the model in latent space, we sample a latent vector $\bz$ through an ODE or SDE solver~\citep{song2021denoising,song2021scorebased,karras2022edm} starting from random noises and then decode the result using $G$ to generate
a final sample.

\subsection{Modeling long videos via blockwise diffusion}
\label{subsec:obj}

Given a ``long'' video $\bx^{1:S} \in \mathbb{R}^{S \times H \times W \times 3}$ with a resolution $H \times W$ and number of frames $S>0$, we divide the video into $N$ \emph{segments} of length $L$: $\bx^{(i-1)L+1:iL}$ for $1\leq i \leq N$ (thus $S=NL$). \sname will autoregressively generate
these segments one-by-one.

\vspace{0.02in}
\noindent\textbf{Autoencoder.}
Following the standard latent diffusion approach, we would like to be able to encode and decode 
videos using a trained autoencoder, which is typically more lightweight 
compared to the diffusion model.  However for very long videos, even running the autoencoder
can be infeasible---thus we encode and decode videos in \emph{chunks} of $m<N$ contiguous segments at a time.  Specifically, for $1 \leq i \leq N/m$, we encode $\bx^{(i-1)mL+1:imL}$ as a latent vector $\bz^{im:(i+1)m}$ and decode it as:
\begin{gather*}
    \bz^{im:(i+1)m}\coloneqq F(\bx^{(i-1)mL+1:imL}) \in \mathbb{R}^{m\cdot l \times h \times w \times c}, \\
    G(\bz^{im:(i+1)m})\approx \bx^{(i-1)mL+1:imL},
\end{gather*}
where $F(\cdot)$ is an encoder network that maps the original video segments to their corresponding latent vectors with a spatial downsampling factor $d_s = H/h = W/w > 1$ and a temporal downsampling factor $d_l = L/l > 1$, and $G(\cdot)$ is a decoder network. We use these latent segments $\bz^{1}, \ldots, \bz^{n}$ of the original $\bx$ obtained from the autoencoder for modeling the long video distribution.

\vspace{0.02in}
\noindent\textbf{Diffusion model.}
We now directly model the joint distribution of latent segments,  $p(\bz^{1:N}|\bc)$, autoregressively as 
\begin{align}
p (\bz^{1:N} | \bc) = \prod_{n=0}^{N-1} p(\bz^{n+1} | \bz^{1:n}, \bc) \text{with}\,\,\, \bz^{1:0}\coloneqq \mathbf{0}
\end{align}
where we learn all 
$p(\bz^{n+1} | \bz^{1:n}, \bc)$ for $0\leq n \leq N-1$ using a single diffusion model $D_{\bm{\theta}}$.

A na\"ive approach would be to use $\bz^{1:n}$ directly as a condition to the model; however, if the number of segments, $N$, is large, $\bz^{1:n}$ easily becomes extremely high-dimensional and prohibitive with respect to memory and compute. To mitigate this bottleneck, we instead introduce a \emph {fixed-size} hidden state $\bh^{i} \coloneqq [\bh^{i}_1, \ldots, \bh^{i}_{d}]$ recurrently computed from $D_{\bm{\theta}}$. This hidden state serves as a memory vector to encode the context (i.e., the previous sequence of segments) $\bz^{1:n}$, where $d>0$ is a number of hidden states that are used as memory vectors (\ie, $i$ is a segment index and $d$ refers to the number of layers of the model). 

Specifically, for $1 \leq i \leq n$, we compute $\bh^n$ with the following recurrent mechanism:
\begin{equation}
    \bh^{i} = \mathrm{HiddenState}\big(D_{\bm{\theta}} (\bz_0^{i}, 0;\, \small{\mathtt{sg}}(\bh^{i-1}), \bc)\big),
\end{equation}
where $\bh^{0} = [\mathbf{0}, \ldots, \mathbf{0}]$ and $\small{\mathtt{sg}}$ denotes a stop-grad operation.
Note that we use a clean segment $\bz^{i}$ without added noise so we set $t=0$ here, which has not been used in conventional diffusion model training~\citep{ho2021denoising,song2021denoising,song2021scorebased}.

To illustrate equilibria and dynamics of performative prediction games, we focus on a scenario in which a \emph{duopoly} of mortgage companies, i.e. banks, compete to sell loans to customers.

\paragraph{Customer Model:} In our game, each bank is trying to attract customers from a given population $\mathcal{P}$. We model this population as comprised of individuals with a single-dimensional type: we denote individual $j$'s type as $y_j \in [0,1]$. For simplicity, we assume that \(y\) represents the customer’s probability of repaying the loan\footnote{In practice, a customer's (normalized) credit score can be interpreted as a noisy observation of $y_j$. This also corresponds to credit scores being \emph{calibrated}.}, i.e., $y_j := \P[Y_j = 1]$, where $Y_j$ is a random variable such that $Y_j = 0$ means that $j$ defaults on their loan, and $Y_j = 1$ means they repay their loan. Customer types in the population are drawn from a known distribution $D_y$ supported on $[0,1]$. 

\paragraph{Game between Banks:} Each Bank \(i \in \{1, 2\}\) selects two parameters \( (\tau_i, \gamma_i) := \theta_i\), where:
\begin{itemize}
    \item \(\tau_i \in \{\tau_l,\tau_h\}\) is the credit score threshold for approving a customer\footnote{We restrict the bank to only pick between two thresholds, $\tau_l$ and $\tau_h$. However, we highlight how our results are affected when we expand the strategy space to $n > 2$ actions in our experiments of Appendix \ref{app:3gamma}.}. Specifically, a customer $j$ with credit score \(y_j\) is approved by Bank $i$ if and only if \(y_j \geq \tau_i\);
    \item \(\gamma_i \in \{\gamma_l, \gamma_h\}\) is the interest rate offered to approved customers.
\end{itemize}
We denote as shorthand the space of allowable thresholds by $\Gamma := [0,1]$ and allowable interests rates by $\Lambda := [0,1]$. %The latter is set without loss of generality---we simply normalize the rates to be at most $1$. 
% {\color{red} Vidya: just thinking about this but is it natural to restrict interest rate to $1$? I don't think it would affect the equilibrium structure of the game but theoretically I think the interest rate could be anything in $[0,\infty)$.} {\color{green} Guanghui: Could we say something like this is without loss of generality} \gua{changed.}\juba{I think we repeated this twice, the next sentence already had this}
The loan amount is normalized to $1$ in the entire paper, without loss of generality; in this case, if a customer chooses Bank $i$, and the customer is approved by the bank at an interest rate of $\gamma_i$, the expected utility for the bank is equal to
\[
(1+\gamma_i)\cdot \P[Y_i = 1]-\P[Y_i = 0] = (1+\gamma_i)y_i-(1-y_i).
\]


%In practice, the credit score \(y\) serves as a noisy observation of the true likelihood of the customer's repayment. 

\paragraph{Banks' Utilities:} For given parameter choices \(\theta_1 = (\tau_1, \gamma_1)\) by Bank 1 and \(\theta_2 = (\tau_2, \gamma_2)\) by Bank 2, a \emph{rational} customer with credit score $y$ acts as follows:

\begin{enumerate}
    \item \textbf{Qualified for a single bank}: 
        \begin{itemize}
        \item If \(\tau_1 \leq y < \tau_2\), the customer goes to Bank 1, as the score qualifies for Bank 1 but not Bank 2. Conversely, if \(\tau_2 \leq y < \tau_1\), the customer chooses Bank 2.
    \end{itemize}
    \item \textbf{Qualified for both banks}:
     \begin{itemize}
        \item If \(\tau_1, \tau_2 \leq y\) and \(\gamma_1 < \gamma_2\), the customer selects Bank 1 for its lower interest rate. Conversely, if \(\gamma_1 > \gamma_2\), the customer chooses Bank 2.
        \item If \(\gamma_1 = \gamma_2\), the customer picks each bank with probability $1/2$. 
    \end{itemize}
    \item \textbf{Unqualified for both banks}:
    \begin{itemize}
        \item If \(y < \tau_1\) and \(y < \tau_2\), the customer is rejected by both banks.
    \end{itemize}
\end{enumerate}

The expected reward for Bank 1, denoted as \(u_1(\theta_1, \theta_2)\), can then be expressed as:
\begin{align}\label{eq:utility}
    u_1(\theta_1, \theta_2) 
    &=  \mathbb{E}_{y \sim D_y} \left[ \mathbb{I}\{\underbrace{\tau_1 \leq y < \tau_2 \ \cup \ (\tau_1, \tau_2 \leq y \ \cap \ \gamma_1 < \gamma_2)}_{\text{accepted by Bank 1}}\} \cdot \big((1+\gamma_1)y - (1-y)\big) \right] \nonumber\\
    & + \frac{1}{2} \mathbb{E}_{y \sim D_y} \left[ \mathbb{I}\{\underbrace{\tau_1, \tau_2 \leq y \ \cap \ \gamma_1 = \gamma_2}_{\text{accepted by both Banks}}\} \cdot \big((1+\gamma_1)y - (1-y)\big) \right].
\end{align}
Note that the problem is \emph{symmetric}, i.e., the utility function for Bank 2 can be derived by swapping the roles of \(\theta_1\) and \(\theta_2\). I.e., $u_2(\theta_1, \theta_2) = u_1(\theta_2, \theta_1)$. 

% If a bank only attracts customers between thresholds $\tau_a$ and $\tau_b$, for $\tau_a<\tau_b$, we call $[\tau_a,\tau_b]$ the \emph{threshold} range for that bank. For example, if Bank $1$ sets a threshold of $\tau_1$, Bank $2$ a threshold of $\tau_2 > \tau_1$, and $\gamma_1 > \gamma_2$, then Bank 1 has a threshold range of $[\tau_1,\tau_2]$, while bank $2$ has a threshold range of $[\tau_2,1]$.
% Note that the parameters set by \emph{both} banks, i.e. $(\theta_1,\theta_2)$ both influence the threshold range for each of Bank 1 and 2.  If $\tau_1>\tau_2$, $\gamma_1>\gamma_2$, then $\tau_a>\tau_b$, and the bank does not attract any customers. 
% {\color{red} is it possible for $\tau_a > \tau_b$, leading to the bank never attracting customers?} \gua{if $\gamma_1>\gamma_2$, $\tau_1>\tau_2$, then it gets no customer. I think it also makes sense.}\juba{I think we said we wanted to delete the discussion of the threshold range, no?}

% \noindent \textbf{Discrete Model}   
% We now present the discrete version of our model, where the interest rates and thresholds are selected from finite sets \(\Gamma\) and \(\Lambda\), respectively, with $\tau\in[0,1], \gamma\in[0,1]$,  for all $\tau\in\Lambda$ and $\gamma\in\Gamma$, \(|\Gamma| = n\) and \(|\Lambda| = m\). Let \(p_1, p_2 \in \Delta(\Gamma \times \Lambda)\) represent the mixed strategies of the two banks, where \(\Delta(\Gamma \times \Lambda)\) denotes the set of probability distributions over the discrete decision space \(\Gamma \times \Lambda\).


% \begin{Remark}
%    Note that our proposed problem can be reformulated as a standard multi-player performative prediction problem \citep{narang2023multiplayer}. However, in our problem, the data distribution faced by each learner breaks the Lipschitzness assumption of previous work~\citep{hardt2023performative,narang2023multiplayer}. A small modification in one of the learner's thresholds can completely change how demand is allocated across both learners, as is often the case in Bertrand-style games. 
% \end{Remark} 

% \gua{I made some changes to Remark 1, please have a look}
\begin{Remark}
   Previous works in multi-learner performative prediction~\citep{narang2023multiplayer} resort to an insensitivity assumption, i.e., the data distribution faced by each player can only changes slightly when the parameters also change slightly; formally, the data distribution faced by each player is Lipschitz in their decisions. This is immediately not true in our setting: the bank slightly changing its parameters can completely changes the demand distribution of customers it faces. Intuitively, this is because of Bertrand-competition-style effects, where if two banks have similar rates, one bank that lowers their rate by a small amount suddenly captures the entire customer demand that is eligible for that rate.%\juba{made further light edits adding intuition}
   
   In Appendix \ref{Appendix:refumulation}, we discuss this problem more carefully by reformulating our problem in the standard multi-learner performative prediction form given by~\citep{narang2023multiplayer}. We show the distribution is not Lipschitz with respect to the parameters, and thus does not satisfy the insensitivity assumption. 
%Prior work~\citep{hardt2023performative,narang2023multiplayer} showed that, for a general multi-agent performative prediction framework to work, insensitivity assumptions are needed: in the \textbf{worst case}, they can construct settings where the insensitivity assumption does not hold and simple dynamics do not converge anymore. We add nuance to this picture. We will show that our dynamics often converge, even absent insensitivity assumptions, highlighting that while the impossibility results of previous work hold in the worst case, they may not hold in the ``average case'' and especially not in problems motivated by applications. In particular, we will show convergence to a variety of equilibria of our game, and often to symmetric Nash equilibria where insensitivity is immediately violated.
     
\end{Remark}



% \paragraph{Relationship to Performative Prediction} A central point of our work is to highlight that \textcolor{red}{needs writing from intro}. We highlight how our work specifically ties to ``Performative Prediction'' below:


%\textcolor{red}{needs a definition environment}



%Here, \(\E_{\theta_1, \theta_2}\) represents the expected utility of the banks over their respective strategies \((\theta_1, \theta_2)\). These inequalities ensure that neither bank can unilaterally improve its expected utility by deviating from its mixed strategy in the equilibrium.



%and  for all $\tau\in\Gamma$, we have $\tau\in\Lambda$, $(\tau,\gamma)\in[0,1]^2$. Let $\Gamma\times\Lambda$
%In this paper, we focus on the most fundamental case, where there are two choices for each parameter: $0\leq\tau_{\ell}<\tau_{h}\leq 1$, and $0\leq \gamma_{\ell}< \gamma_{h}\leq 1$. In this case, the utility for each pair of decisions forms a $4\times4$ matrix (given in Table \ref{tab:my-table}). We consider the canonical case where $\tau_{\ell}=\frac{1}{2+\gamma_{h}}$, and $\tau_{h}=\frac{1}{2+\gamma_{\ell}}.$ Note that these are natural choices for the thresholds, in the sense that, if there is only one bank and the interest rate is set to be $\gamma$, then $\frac{1}{2+\gamma}$ is the optimal threshold corresponding to the fixed $\gamma$.


%and the thresholds are chosen in $\Lambda=\{\tau^{(1)},\dots,\tau^{(m)}\}$. Here, we only assume that, for each $\gamma\in\Gamma$, there at least exist one $\tau\in\Lambda$ such that $f(\gamma,\tau,1)>0$. Note that this is a very minor assumption, in the sense that, if for a $\gamma$ such that $f(\gamma,\tau,1)<0$ for all $\tau\in\Lambda$, then adopting this decision will lead to negative utility regardless of the opponent's decision, and thus is not an interesting case. 

%\textcolor{red}{The model section is missing the dynamic version of the game. We should clearly define the one-shot and the dynamic game}
% we only considered one-shot case in our paper




To sum up, we train $D_{\bm{\theta}}$ with the following denoising autoencoder objective with the memory $\bh^{n}$:
\begin{align}
    \mathbb{E}
    \Big[
    \lambda(t)||D_{\bm{\theta}}(\bz_t^{n+1}, t;\, \bh^{n}, \bc) - \bz_0^{n+1}||_2^2 
    \Big],
\label{eq:pseudo-obj}
\end{align}
where each data is sampled from the video dataset $\mathcal{D}$, $\epsilon \sim p(\bm{\epsilon})$, $t \sim [1, T]$, and $n \sim P(n)$ with pre-defined prior distributions $p(\bm{\epsilon})$ and $P(n)$. 
Specifically, we set $P(n)$ as $P(n$ {$=$} $0)=\nicefrac{1}{2}$ and $P(n$ {$=$} $k)=\nicefrac{1}{2(N-1)}$ for $k>0$, as generating videos without memory (\ie, $n=0$) is more difficult than continuation with a given memory vector (\ie, $n>0$).
Note that in Eq.~\eqref{eq:pseudo-obj} we do not use the stop-grad operation on $\bh^{n}$ itself; as we mentioned, diffusion model training uses $t \geq 1$ in the common diffusion model objective because they only consider noisy inputs but our memory vector computation uses a clean sample ($t=0$), which cannot be optimized without a backpropagation to $\bh^{n}$. 
Instead, we use the stop-grad operation on the $\bh^{i}$ for $i<n$ which are used to compute $\bh^{n}$ in
order to reduce memory requirements with respect to the number of segments used during training.

\vspace{0.02in}
\noindent\textbf{Training for long term stability.}
Unfortunately, the video quality from the model $D_{\bm{\theta}}$ trained with Eq.~\eqref{eq:pseudo-obj} is usually not satisfactory, because of frame quality degradation caused by error accumulation. We hypothesize that the reason why this happens is because there exists a discrepancy between training and inference: in training, we use ground-truth latent vector $\bz^{n}$ for computation of the memory $\bh^{n}$, but at inference, the model instead uses generated latent vectors, 
where small errors during generation can compound over a long sequence of segments.

To mitigate this discrepancy, we use a noisy version of $\bh^{n}$ at training time, denoted by $\tilde{\bh}^{n}$,  where
\begin{gather}
    \tilde{\bh}^{n} \coloneqq \mathrm{HiddenState}\big(D_{\bm{\theta}} (\bz^{n} + \bm{\xi}, 0;\, \small{\mathtt{sg}}(\bh^{n-1}), \bc)\big),\\ \text{where}\,\,\bh^{0} = [\mathbf{0}, \ldots, \mathbf{0}],
\end{gather}
and $\bm{\xi} \sim p(\bm{\xi})$ with a pre-defined prior distribution $p(\bm{\xi})$. Since $\tilde{\bh}^{n}$ is computed with $\bh^{n-1}$ and a \emph{noisy} latent vector $\bz^{n} + \bm{\xi}$, the model is trained to be robust to small errors and reduces the train-test discrepancy between the memory computed at training and inference. 

Summing up all of these components, our final training objective $\mathcal{L}(\bm{\theta})$ becomes:
\begin{align}
    \mathcal{L}(\bm{\theta}) \coloneqq \mathbb{E} 
    \Big[
    \lambda(t)|| D_{\bm{\theta}}(\bz_t^{n+1}, t;\, \tilde{\bh}^{n}, \bc) - \bz_0^{n+1}||_2^2 
    \Big],
\label{eq:obj}
\end{align}
with prior distributions $p(\bm{\epsilon}), p(n), p(\bm{\xi})$.

For $p(\bm{\epsilon})$, we use a progressively correlated Gaussian distribution proposed in \citet{ge2023preserve} to further mitigate error accumulation. Next, we set $p(\bm{\xi})$ to be a Gaussian $\mathcal{N}(\mathbf{0}$, $\sigma_{\mathrm{mem}}^2\mathbf{I})$ with small $\sigma_{\mathrm{mem}}>0$.

\vspace{0.02in}
\noindent\textbf{Inference.}
After training, we synthesize a long video by autoregressively generating one segment at a time. Specifically, we start from generating a first segment $\bz^{1}$ conditioned on $\bc$, and then iteratively generate $\bz^{n+1}$ for $n>0$ by computing memory $\bh^{n}$ and performing conditional generation from $\bh^{n}$ and $\bc$. We provide detailed pseudocode in Appendix~\ref{appen:sampling}.

\subsection{Architecture}
\label{subsec:arch}

\textbf{Autoencoder.}
Similar to the encoding scheme used in a recent latent video diffusion model, W.A.L.T \citep{gupta2023photorealistic}, we use a
causal 3D CNN encoder-decoder architecture for the video autoencoder based on the  
MAGVIT-2 tokenizer~\citep{yu2024language} without quantization (so that latent vectors lie in a continuous space). We train the autoencoder with a sum of pixel-level reconstruction loss (\eg, mean-squared error), perceptual loss (\eg, LPIPS~\citep{zhang2018perceptual}), and adversarial loss~\citep{goodfellow2014generative} similar to prior image and video generation methods. Recall that both training and inference are not done directly on long videos; they are done after splitting long videos into short segments.

\vspace{0.02in}
\noindent\textbf{Diffusion model.}
As outlined in Figure~\ref{fig:model}, our model architecture is based on the recent diffusion transformer (DiT) architecture~\citep{Peebles2022DiT, ma2024latte, yu2024representation}. 
Thus, given a latent vector $\bz^n \in \mathbb{R}^{l \times h \times w \times c}$ of a video clip $\bx^n$, we patchify it with patch size $p_l \times p_s \times p_s$ to form a flattened latent vector $\mathtt{patchify}(\bz^n) \in \mathbb{R}^{(lhw / p_lp_s^2) \times c}$ with a sequence length $lhw / p_lp_s^2$ and use it as inputs to the model. In particular, we choose W.A.L.T~\citep{gupta2023photorealistic} as backbone, a variant of DiT which employs efficient spatiotemporal windowed attention instead of full attention between large numbers of video patches.

To enable training with long videos with DiT architectures, we introduce a memory-augmented attention layer and insert this layer to the beginning of every Transformer block. Specifically, we design this layer as a cross-attention layer between the previous memory latent vector and the current hidden state, similar to memory-augmented attention~\citep{dai2019transformer} in the language domain. Hence, query, key, and value of a $d$-th memory layer with the segment $\bz^{n}$ and the memory latent vector $\bh^{n-1}$ become:
\begin{gather*}
    \text{query}\coloneqq \bh_{d}^{n},\,\,\text{key} \coloneqq [\bh_{d}^{n-1}, \bh_{d}^{n}],\,\,\text{value}\coloneqq [\bh_{d}^{n-1}, \bh_{d}^{n}], \\ \bh_{d}^{n-1}, \bh_{d}^{n} \in \mathbb{R}^{(hw / p_s^2) \times (l/p_l) \times c'},
\end{gather*}
where $c'$ denotes the hidden dimension of the model and $\bh_{d}^{n-1}, \bh_{d}^{n}$ are \emph{reshaped} as a sequence length $l/p_l$ and a batch dimension size $hw/p_s^2$, similar to previous space-time factorized attention~\citep{bertasius2021space}. 
We also use relative positional encodings that are widely used to handle longer context length.

With this formulation, memory-augmented attentions are only computed together with each of $l/p_l$ patches that have the same spatial location (\ie, temporal attention in video transformers~\citep{bertasius2021space}). This increase is not significant because the computation of attention is restricted only to the sample spatial locations. Our 
memory augmented attentions have $O(L^2HW)$ computational complexity whereas full attention would scale as $O((LHW)^2)$.

Finally, recall that we build our architecture on W.A.L.T, but our general approach of using memory-augmented latent transformers can be applied more broadly to any video diffusion transformer architectures, such as \citet{lu2023vdt} and \citet{ma2024latte}. We provide a detailed illustration of the architecture combined with W.A.L.T in Appendix~\ref{appen:archi}.