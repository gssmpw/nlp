\begin{figure*}[!ht]
    \centering
    \includegraphics[width=\linewidth]{figures-src/concept.pdf}
    \caption{SliderSpace decomposes the visual variation of diffusion model's knowledge corresponding to a concept. These directions can be perceived as interpretable directions of the model's hierarchical knowledge. We show the decomposed slider direction for a concept using SliderSpace and the corresponding labels generated by Claude 3.5 Sonnet.}
    \label{fig:concept}
\end{figure*}

\section{\sname: \lname}
\label{sec:method}
We first formalize our problem as follows. Consider a dataset $\mathcal{D}$, where each data $(\bx, \bc) \in \mathcal{D}$ consists of a video $\bx$ and corresponding conditions $\bc$ (\eg, text captions). Our goal is to train a model using $\mathcal{D}$ to learn a model distribution $p_{\text{model}}(\bx | \bc)$ that matches a ground-truth conditional distribution $p_{\text{data}} (\bx | \bc)$. In particular, we are interested in the situation where each $\bx$ is a \emph{long} video, where target video lengths are much larger than conventional choices of $<$100 for both training and inference~\citep{he2022lvdm}.

To efficiently model the distribution of long videos, we adopt ``memory'' that encodes previous long context in our latent diffusion transformer. Specifically, we aim to train a single model capable of: (a) encoding previous context of the long video as a compact memory latent vector, and (b) generating a future clip conditioned on the memory and a given condition $\bc$. 

In the rest of this section, we explain our \lname (\sname) in detail. In Section~\ref{subsec:ldm}, we provide a brief overview of latent diffusion models. In Section~\ref{subsec:obj}, we describe how we formulate the problem and how we design a training objective. Finally, in Section~\ref{subsec:arch}, we explain the architecture that we used in our framework. %\jon{can delete this if we need space}

\textbf{Notation.}
We write a sequence of vectors $[\bx^{a} \ldots, \bx^{b}]$ with $a<b$ as $\bx^{a:b}$. 

\subsection{Latent diffusion models}
\label{subsec:ldm}
In order to generate data, 
diffusion models learn the \emph{reverse} process of a
%\jon{destructive?}
forward diffusion, where the forward diffusion diffuses a data $\bx_{0} \sim p_{\text{data}}(\bx)$ to a (simple) prior distribution $\bx_{T} \sim \mathcal{N}(\mathbf{0}, \sigma_{\mathrm{max}} \mathbf{I})$ (with pre-defined $\sigma_{\mathrm{max}}>0$) with the following stochastic differential equation (SDE):
\begin{align}
    d\bx = \mathbf{f} (\bx, t) dt + g(t) d\mathbf{w},
    \label{eq:forwardsde}
\end{align}
where $\mathbf{f}$, $g$, and $\mathbf{w}$ are pre-defined drift coefficient, diffusion coefficient, and standard Wiener process (respectively) with $t \in [0, T]$ and pre-defined $T>0$. 
With this forward process, data sampling can be done with the following reverse SDE of Eq.~\eqref{eq:forwardsde}: 
\begin{align}
    d\bx = \Big[ \mathbf{f} (\bx, t) - \frac{1}{2} g(t)^2 \nabla_{\bx} \log p_t (\bx) \Big] dt + g(t) d\mathbf{\bar{w}},
\end{align}
where $\mathbf{\bar{w}}$ is a standard reverse-time Wiener process, and $\nabla_{\bx} \log p_t(\bx)$ is a score function of the marginal density from Eq.~\eqref{eq:forwardsde} at time $t$.
% \jon{in the above eqn, $p_t(x)$ is undefined.  maybe a good place to define it is directly after eqn 1 and say that it has the property $p_0$ is the data distribution and $p_T$ is the prior. and probably good to say here that nabla log pt is the score function}
\citet{song2021scorebased} shows there exists a \emph{probability flow ordinary differential equation (PF ODE)}
whose marginal $p_t (\bx)$ is identical for the SDE with $t \in [0, T]$:
\begin{align}
    d\bx = \Big[ \mathbf{f} (\bx, t) - \frac{1}{2} g(t)^2 \nabla_{\bx} \log p_t (\bx) \Big] dt.
    \label{eq:pfode}
\end{align}
Following previous diffusion model methods~\citep{lee2024dreamflow,zheng2024fast}, we use the setup in EDM~\citep{karras2022edm} with $\mathbf{f} (\bx, t) \coloneqq \mathbf{0}$, $g(t)\coloneqq \sqrt{2\dot\sigma(t)\sigma(t)}$ and a decreasing noise schedule $\sigma: [0, T] \to \mathbb{R}_{+}$. In this case, the PF ODE in Eq.~\eqref{eq:pfode} can be written:
\begin{align}
    d\bx = -\dot\sigma(t)\sigma(t)\nabla_{\bx}\log p (\bx; \sigma(t)) dt,
\end{align}
where we denote $p(\bx; \sigma)$ be
the smoothed distribution by adding i.i.d Gaussian noise $\bm{\epsilon}\sim\mathcal{N}(\mathbf{0}, \sigma \mathbf{I})$ with standard deviation $\sigma > 0$. To learn the score function $\nabla_{\bx}\log p (\bx; \sigma(t))$, we train a denoising network $D_{\bm{\theta}} (\bx, t)$ with the denoising score matching (DSM)~\citep{song2019generative} objective for all $t \in [0, T]$:
\begin{align*}
\mathbb{E}_{\bx, \bm{\epsilon}, t}\Big[ \lambda({t}) || D_{\bm{\theta}}(\bx_0 + \bm{\epsilon}, t)  - \mathbf{\bx}_0||_2^2 \Big],
\end{align*}
where $\lambda(\cdot)$ assigns a non-negative weight.

However, training $D_{\bm{\theta}}$ directly with raw high-dimensional $\bx$ is computation and memory expensive. To solve this problem, latent diffusion models~\citep{rombach2021highresolution} first learn a 
lower dimensional latent representation of $\bx$ by training an autoencoder (with encoder $F(\bx) = \bz$ and 
decoder $G(\bz) = \bx)$ to reconstruct $\bx$ from 
the low-dimensional vector $\bz$, and then train $D_{\bm{\theta}}$ to generate in this latent space instead.
Specifically, latent diffusion models use the following denoising objective defined in the latent space:
\begin{align*}
    \mathbb{E}_{\bx, \bm{\epsilon}, t}\Big[ \lambda(t) || D_{\bm{\theta}}(\bz_0 + \bm{\epsilon}, t)  - \mathbf{\bz}_0||_2^2 \Big],
\end{align*}
where $\bz_0=F(\bx_0)$. After training the model in latent space, we sample a latent vector $\bz$ through an ODE/SDE solver~\citep{song2021denoising,song2021scorebased,karras2022edm} and then decode the result using $G$ to generate
a final sample.

\subsection{Modeling long sequences via blockwise latent diffusion}
\label{subsec:obj}
\textbf{Autoencoder.}
Given a long video $\bx^{1:NL} \in \mathbb{R}^{NL \times H \times W \times 3}$ with a resolution $H \times W$ and a length $NL>0$, we first divide it to $N$ segments $\bx^{(i-1)L+1:iL}$ for $1\leq i \leq N$. This is because to avoid encoder and decoder to compute the entire very long sequence at once to reduce memory and computation constraints. After that, we encode and decode $m<N$ segments joint at a time; for $1 \leq i \leq N/m$, we encode and $\bx^{(i-1)mL+1:imL}$ as a latent vector $\bz^{im:(i+1)m}$ and decode it as:
\begin{align*}
    \bz^{im:(i+1)m}\coloneqq F(\bx^{(i-1)mL+1:imL}) \in \mathbb{R}^{m\cdot l \times h \times w \times c},\quad G(\bz^{im:(i+1)m})\approx \bx^{(i-1)mL+1:imL},
\end{align*}
where $F(\cdot)$ is an encoder network that maps the original video segments to the corresponding latent vectors with a spatial downsampling factor $d_s = H/h = W/w > 1$ and a temporal downsampling factor $d_l = L/l > 1$, and $G(\cdot)$ is a decoder network. We use these latent segments $\bz^{1}, \ldots, \bz^{n}$ of the original $\bx$ obtained from the autoencoder for modeling the long video distribution.

\textbf{Diffusion model.}
We now directly model $p(\bz^{1:L}|\bc)$ with the blockwise-autoregressive modeling with $\bz^{1}, \ldots, \bz^{N}$ of a long video $\bx^{1:L}$, namely $p (\bz^{1:L} | \bc) = \prod_{n=0}^{N-1} p(\bz^{n+1} | \bz^{1:n}, \bc)$ with $\bz^{1:0}\coloneqq \mathbf{0}$,
% \begin{align}
%     p (\bz^{1:L} | \bc) = \prod_{n=0}^{N-1} p(\bz^{n+1} | \bz^{1:n}, \bc), \,\, \text{where}\,\, \bz^{1:0} \coloneqq \mathbf{0},
% \end{align}
where we model all of $p(\bz^{n+1} | \bz^{1:n}, \bc)$ for $0\leq n \leq N-1$ using a single diffusion model $D_{\bm{\theta}}$.

However, if $N$ is large, $\bz^{1:n}$ becomes very high-dimensional, so using $\bz^{1:n}$ directly as a condition to $D_{\bm{\theta}}$ can easily require extreme memory and computation costs. To mitigate this bottleneck, we instead introduce a \emph {fixed-size} hidden state $\bh^{i} \coloneqq [\bh^{i}_1, \ldots, \bh^{i}_{d}]$ recurrently computed from $D_{\bm{\theta}}$ as a memory vector to encode previous contexts $\bz^{1:n}$, where $d>0$ is a number of hidden states that are used as memory vectors (\ie, $i$ is a segment index and $d$ refers to the number of layers of the model). 
%Hence, our diffusion model $D_{\bm{\theta}}(\bz_t^{n+1}, t;\,\bh^{n}, \bc)$ is trained to denoise $\bz_t^{n+1}$ using $t$, $\bc$, and the memory vector $\bh^n$ that amortizes $\bz^{1:n}$.

Specifically, for $1 \leq i \leq n$, we compute $\bh^n$ with the following recurrent mechanism:
\begin{align}
    \bh^{i} = \mathrm{HiddenState}\big(D_{\bm{\theta}} (\bz_0^{i}, 0;\, \small{\mathtt{sg}}(\bh^{i-1}), \bc)\big), \,\, \bh^{0} = [\mathbf{0}, \ldots, \mathbf{0}],
\end{align}
where $\small{\mathtt{sg}}$ denotes a stop-grad operation. Note that we use a clean segment $\bz^{i}$ without noise so we set $t=0$ here, which has not been used in conventional diffusion model training~\citep{ho2021denoising,song2021denoising,song2021scorebased}.

To sum up, we train $D_{\bm{\theta}}$ with the following denoising autoencoder objective with the memory $\bh^{n}$:
\begin{align}
    \mathbb{E}_{(\bx_0, \bc), \bm{\epsilon}, t, n} 
    \Big[
    \lambda(t)||D_{\bm{\theta}}(\bz_t^{n+1}, t;\, \bh^{n}, \bc) - \bz_0||_2^2 
    \Big],
\label{eq:pseudo-obj}
\end{align}
where $(\bx_0, \bc)$ is sampled from the video dataset $\mathcal{D}$, $\epsilon \sim p(\bm{\epsilon})$, $t \sim [1, T]$, and $n \sim p(n)$ with pre-defined prior distributions $p(\bm{\epsilon})$ and $p(n)$. Note that we do not use stop-grad operation to $\bh^{n}$ in Eq.~\eqref{eq:pseudo-obj}; as we mentioned, diffusion model training uses $t \geq 1$ in the common diffusion model objective because they only consider noisy inputs but our memory vector computation uses a clean sample ($t=0$), which cannot be optimized without a backpropagation to $\bh^{n}$. 
%\textcolor{magenta}{Note that while we use $t=0$ for previous latents and $t\geq1$ for the current segment, but one can assign any different latent timesteps prior to the current timestep depending on how we formulate this distribution learning problem.}\jon{this is not really understandable to me}
Moreover, recall that we use the stop-grad operation in the computation of $\bh^{n}$, so the memory cost does not increase rapidly with respect to the number of segments used in training.

\textbf{Inference.}
After training, we synthesize a long video by autoregressively generating short video clips. Specifically, we start from generating a first segment $\bz^{1}$ conditioned on $\bc$, and then iteratively generate $\bz^{n+1}$ for $n>0$ by computing memory $\bh^{n}$ and performing conditional generation from $\bh^{n}$ and $\bc$. We provide the detailed algorithm in Appendix~\ref{appen:sampling}.

\section{Model}
\label{sec:model}
Let $[N] = \{1, 2, \dots, N \}$ be a set of $N$ agents.
We examine an environment in which a system interacts with the agents over $T$ rounds.
Every round $t\leq T$ comprises $N$ \emph{sessions}, each session represents an encounter of the system with exactly one agent, and each agent interacts exactly once with the system every round.
I.e., in each round $t$ the agents arrive sequentially. 


\paragraph{Arrival order} The \emph{arrival order} of round $t$, denoted as $\ordv_t=(\ord_t(1),\dots, \ord_t(N))$, is an element from set of all permutations of $[N]$. Each entry $q$ in $\ordv_t$ is the index of the agent that arrives in the $q^{\text{th}}$ session of round $t$.
For example, if $\ord_t(1) = 2$, then agent $2$ arrives in the first session of round $t$.
Correspondingly, $\ord_t^{-1}(i)=q$ implies that agent $i$ arrives in the $q^{\text{th}}$ session of round $t$. 

As we demonstrate later, the arrival order has an immediate impact on agent rewards. We call the mechanism by which the arrival order is set \emph{arrival function} and denote it by $\ordname$. Throughout the paper, we consider several arrival functions such as the \emph{uniform arrival} function, denoted by $\uniord$, and the \emph{nudged arrival} $\sugord$; we introduce those formally in Sections~\ref{sec:uniform} and~\ref{sec:nudge}, respectively.

%We elaborate more on this concept in Section~\ref{sec: arrival}.


\paragraph{Arms} We consider a set of $K \geq 2$ arms, $A = \{a_1, \ldots, a_K\}$. The reward of arm $a_i$ in round $t$ is a random variable $X_i^t \sim \mathcal{D}^t_i$, where the rewards $(X_i^t)_{i,t}$ are mutually independent and bounded within the interval $[0,1]$. The reward distribution $\mathcal{D}^t_i$ of arm $a_i$, $i\in [K]$ at round $t\in T$ is assumed to be non-stationary but independent across arms and rounds. We denote the realized reward of arm $a_i$ in round $t$ by $x_i^t$. We assume \emph{reward consistency}, meaning that rewards may vary between rounds but remain constant within the sessions of a single round. Specifically, if an arm $a_i$ is selected multiple times during round~$t$, each selection yields the same reward $x_i^t$, where the superscript $t$ indicates its dependence on the round rather than the session. This consistency enables the system to leverage information obtained from earlier sessions to make more informed decisions in later sessions within the same round. We provide further details on this principle in Subsection~\ref{subsec:information}.


\paragraph{Algorithms} An algorithm is a mapping from histories to actions. We typically expect algorithms to maximize some aggregated agent metric like social welfare. Let $\mathcal H^{t,q}$ denote the information observed during all sessions of rounds $1$ to $t-1$ and sessions $1$ to $q-1$ in round $t$.  The history $\mathcal H^{t,q}$ is an element from $(A \times [0,1])^{(t-1) \cdot N +q-1}$, consisting of pairs of the form (pulled arm, realized reward). Notice that we restrict our attention to \emph{anonymous} algorithms, i.e., algorithms that do not distinguish between agents based on their identities. Instead, they only respond to the history of arms pulled and rewards observed, without conditioning on which specific agent performed each action.
%In the most general case, algorithms make decisions at session $q$ of round $t$  based on the entire history $\mathcal H^{t,q}$ and the index of the arriving agent $\ord_t(q)$. %Furthermore, we sometimes assume that algorithms have Bayesian information, i.e., algorithms are aware of the distributions $(\mathcal D_i)^K_{i=1}$. 
Furthermore, we sometimes assume that algorithms have Bayesian information, meaning they are aware of the reward distributions $(\mathcal{D}^t_i)_{i,t}$. If such an assumption is required to derive a result, we make it explicit. %Otherwise, we do not assume any additional knowledge about the algorithm’s information. %This distinction allows us to analyze both general algorithms without prior distributional knowledge and specialized algorithms that leverage Bayesian information.


\paragraph{Rewards} Let $\rt{i}$ denote the reward received by agent $i \in [N]$ at round $t$, and let $\Rt{i}$ denote her cumulative reward at the end of round $t$, i.e., $\Rt{i} = \sum_{\tau=1}^{t}{r^{\tau}_{i}}$. We further denote the \emph{social welfare} as the sum of the rewards all agents receive after $T$ rounds. Formally, $\sw=\sum^{N}_{i=1}{R^T_i}$. We emphasize that social welfare is independent of the arrival order. 


\paragraph{Envy}
We denote by $\adift{i}{j}$ the reward discrepancy of agents $i$ and $j$ in round $t$; namely, $\adift{i}{j}= \rt{i} - \rt{j}$. %We call this term \omer{name??} reward discrepancy in round $t$. 
The (cumulative) \emph{envy} between two agents at round $t$ is the difference in their cumulative rewards. Formally, $\env_{i,j}^t= \Rt{i} - \Rt{j}$ is the envy after $t$ rounds between agent $i$ and $j$. We can also formulate envy as the sum of reward discrepancies, $\env_{i,j}^t= \sum^{t}_{\tau=1}{\adif{i}{j}^\tau}$. Notice that envy is a signed quantity and can be either positive or negative. Specifically, if $\env_{i,j}^t < 0$, we say that agent $i$ envies agent $j$, and if $\env_{i,j}^t > 0$, agent $j$ envies agent $i$. The main goal of this paper is to investigate the behavior of the \emph{maximal envy}, defined as
\[
\env^t = \max_{i,j \in [N]} \env^t_{i,j}.
\]
For clarity, the term \emph{envy} will refer to the maximal envy.\footnote{ We address alternative definitions of envy in Section~\ref{sec:discussion}.} % Envy can also be defined in alternative ways, such as by averaging pairwise envy across all agents. We address average envy in Section~\ref{sec:avg_envy}.}
Note that $\env_{i,j}^t$ are random variables that depend on the decision-making algorithm, realized rewards, and the arrival order, and therefore, so is $\env^t$. If a result we obtain regarding envy depends on the arrival order $\ordname$, we write $\env^t(\ordname)$. Similarly, to ease notation, if $\ordname$ can be understood from the context, it is omitted.



\paragraph{Further Notation} We use the subscript $(q)$ to address elements of the $q^{\text{th}}$ session, for $q\in [N]$.
That is, we use the notation $\rt{(q)}$ to denote the reward granted to the agent that arrives in the $q^{\text{th}}$ session of round $t$ and $\Rt{(q)}$ to denote her cumulative reward. %Additionally, we introduce the notation $\at{(q)}$ to denote the arm pulled in that session.
Correspondingly, $\sdift{q}{w} = \rt{(q)} - \rt{(w)}$ is the reward discrepancy of the agents arriving in the $q^{\text{th}}$ and $w^{\text{th}}$ sessions of round $t$, respectively. 
To distinguish agents, arms, sessions and rounds, we use the letters $i,j$ to mark agents and arms, $q,w$ for sessions, and $t,\tau$ for rounds.


\subsection{Example}
\label{sec: example}
To illustrate the proposed setting and notation, we present the following example, which serves as a running example throughout the paper.

\begin{table}[t]
\centering
\begin{tabular}{|c|c|c|c|}
\hline
$t$ (round) & $\ordv_t$ (arrival order) & $x_1^t$ & $x_2^t$ \\ \hline
1           & 2, 1                     & 0.6     & 0.92    \\ \hline
2           & 1, 2                     & 0.48    & 0.1     \\ \hline
3           & 2, 1                     & 0.15    & 0.8     \\ \hline
\end{tabular}
\caption{
    Data for Example~\ref{example 1}.
}
\label{tbl: example}
\end{table}

\begin{algorithm}[t]
\caption{Algorithm for Example~\ref{example 1}}
\label{alguni}
\DontPrintSemicolon 
\For{round $t = 1$ to $T$}{
    pull $a_{1}$ in the first session\label{alguniexample: first}\\
    \lIf{$x^t_1 \geq \frac{1}{2}$}{pull $a_{1}$ again in second session \label{alguniexample: pulling a again}}
    \lElse{pull $a_{2}$ in second session \label{alguniexample: sopt else}}
}
\end{algorithm}


\begin{example}\label{example 1}
We consider $K=2$ uniform arms, $X_1,X_2 \sim \uni{0,1}$, and $N=2$ for some $T\geq 3$. We shall assume arm decision are made by Algorithm~\ref{alguni}: In the first session, the algorithm pulls $a_1$; if it yields a reward greater than $\nicefrac{1}{2}$, the algorithm pulls it again in the second session (the ``if'' clause). Otherwise, it pulls $a_2$.



We further assume that the arrival orders and rewards are as specified in Table~\ref{tbl: example}. Specifically, agent 2 arrives in the first session of round $t=1$, and pulling arm $a_2$ in this round would yield a reward of $x^1_2 = 0.92$. Importantly, \emph{this information is not available to the decision-making algorithm in advance} and is only revealed when or if the corresponding arms are pulled.




In the first round, $\boldsymbol{\eta}^1 = \left(2,1\right)$; thus, agent 2 arrives in the first session.
The algorithm pulls arm $a_1$, which means, $a^1_{(1)} = a_1$, and the agent receives $r_{2}^1=r_{(1)}^1=x_1^1=0.6$.
Later that round, in the second session, agent 1 arrives, and the algorithm pulls the same arm again since $x^1_1 = 0.6 \geq \nicefrac{1}{2}$ due to the ``if'' clause.
I.e., $a^1_{(2)} = a_1$ and $r_{1}^1 = r_{(2)}^1 = x_1^1 = 0.6$.
Even though the realized reward of arm $a_2$ in that round is higher ($0.92$), the algorithm is not aware of that value.
At the end of the first round, $R^1_1 = R^1_{(2)} = R^1_2 = R^1_{(1)} = 0.6$. The reward discrepancy is thus $\adif{1}{2}^1 = \adif{2}{1}^1= \sdif{2}{1}^1 = 0.6 - 0.6 =0$. 



In the second round, agent 1 arrives first, followed by agent 2.
Firstly, the algorithm pulls arm $a_1$ and agent 1 receives a reward of $r_{1}^2 = r_{(1)}^2 = x_1^2 = 0.48$.
Because the reward is lower than $\nicefrac{1}{2}$, in the second session the algorithm pulls the other arm ($a^2_{(2)} = a_2$), granting agent 2 a reward of $r_{2}^2 = r_{(2)}^2 = x_2^2 = 0.1$.
At the end of the second round, $R^2_1 = R^2_{(1)} = 0.6 + 0.48 = 1.08$ and $R^2_2 = R^2_{(2)} = 0.6 + 0.1 = 0.7$. Furthermore, $\sdif{2}{1}^2 = \adif{2}{1}^2 = r^2_{2} - r^2_{1} = 0.1 - 0.48 = -0.38$.

In the third and final round, agent 2 arrives first again, and receives a reward  of $0.15$ from $a_1$. When agent 1 arrives in the second session, the algorithm pulls arm $a_2$, and she receives a reward of $0.8$. As for the reward discrepancy, $\sdif{2}{1}^3 = \adif{2}{1}^3 = r^3_{2} - r^3_{1} = 0.15 - 0.8 = -0.75$. 

Finally, agent 1 has a cumulative reward of $R^3_1 = R^3_{(2)} = 0.6 + 0.48 + 0.8 = 1.88$, whereas agent~2 has a cumulative reward of $R^3_2 = R^3_{(1)} = 0.6 + 0.1 + 0.15 = 0.85$. In terms of envy, $\env^1_{1,2}= \adif{1}{2}^1 =0$, $\env^2_{1,2}=\adif{1}{2}^1+\adif{1}{2}^2= 0.38$, and $\env^3_{1,2} = -\env^3_{2,1} = R^3_1-R^3_2 = 1.88-0.85 = 1.03$, and consequently the envy in round 3 is $\env^3 = 1.03$.
\end{example}


\subsection{Information Exploitation}
\label{subsec:information}

In this subsection, we explain how algorithms can exploit intra-round information.
Since rewards are consistent in the sessions of each round, acquiring information in each session can be used to increase the reward of the following sessions.
In other words, the earlier sessions can be used for exploration, and we generally expect agents arriving in later sessions to receive higher rewards.
Taken to the extreme, an agent that arrives after all arms have been pulled could potentially obtain the highest reward of that round, depending on how the algorithm operates.

To further demonstrate the advantage of late arrival, we reconsider Example~\ref{example 1} and Algorithm~\ref{alguni}. 
The expected reward for the agent in the first session of round $t$ is $\E{\rt{(1)}}=\mu_1=\frac{1}{2}$, yet the expected reward of the agent in the second session is
\begin{align*}
\E{\rt{(2)}}=\E{\rt{(2)}\mid X^t_1 \geq \frac{1}{2} }\prb{X^t_1 \geq \frac{1}{2}} + \E{\rt{(2)}\mid X^t_1 < \frac{1}{2} }\prb{X^t_1 < \frac{1}{2}};
\end{align*}
thus, $\E{\rt{(2)}} =\E{X^t_1\mid X^t_1 \geq \frac{1}{2} }\cdot \frac{1}{2} + \mu_2\cdot\frac{1}{2} = \frac{5}{8}$.
Consequently, the expected welfare per round is $\E{\rt{(1)}+\rt{(2)}}=1+\frac{1}{8}$, and the benefit of arriving in the second session of any round $t$ is $\E{\rt{(2)} - \rt{(1)}} = \frac{1}{8}$. This gap creates envy over time, which we aim to measure and understand.
%This discrepancy generates envy over time, and our paper aims to better understand it.
\subsection{Socially Optimal Algorithms}
\label{sec: sw}
Since our model is novel, particularly in its focus on the reward consistency element, studying social welfare maximizing algorithms represents an important extension of our work. While the primary focus of this paper is to analyze envy under minimal assumptions about algorithmic operations, we also make progress in the direction of social welfare optimization. See more details in Section~\ref{sec:discussion}.%Due to space limitations, we defer the discussion on socially optimal algorithms to  \ifnum\Includeappendix=0{the appendix}\else{Section~\ref{appendix:sociallyopt}}\fi.




% Since our model is novel and specifically the reward consistency element, it might be interesting to study social welfare optimization. While the main focus of our paper is to study envy under minimal assumptions on how the algorithm operates, we take steps toward this direction as well. Due to space limitations, we defer the discussion on socially optimal algorithms to  \ifnum\Includeappendix=0{the appendix}\else{Section~\ref{appendix:sociallyopt}}\fi.  We devise a socially optimal algorithm for the two-agent case, offer efficient and optimal algorithms for special cases of $N>2$ agents, and an inefficient and approximately optimal algorithm for any instance with $N>2$. Moreover, we address the welfare-envy tradeoff in Section~\ref{sec:extensions}.


% Social welfare, unlike envy, is entirely independent of the arrival order. While the main focus of our paper is to study envy under minimal assumptions on how the algorithm operates, socially optimal algorithms might also be of interest. Due to space limitations, we defer the discussion on socially optimal algorithms to  \ifnum\Includeappendix=0{the appendix}\else{Section~\ref{appendix:sociallyopt}}\fi. We devise a socially optimal algorithm for the two-agent case, offer efficient and optimal algorithms for special cases of $N>2$ agents, and an inefficient and approximately optimal algorithm for any instance with $N>2$. %Furthermore, we treat the welfare-envy tradeoff of the special case of Example~\ref{example 1}.



\subsection{Architecture}
\label{subsec:arch}

We now discuss specific architectural designs that we used, but note that the general approach outlined above can apply broadly to many choices of autoencoder and diffusion model architectures.

\textbf{Autoencoder.}
Similar to the encoding scheme used in a recent latent video diffusion, W.A.L.T~\citep{gupta2023photorealistic}, we use a
causal 3D CNN encoder-decoder architecture for the video autoencoder based on the recent MAGVIT-2 tokenizer~\citep{yu2024language}. We train the autoencoder with a sum of pixel-level reconstruction loss (\eg, mean-squared error), perceptual loss (\eg, LPIPS~\citep{zhang2018perceptual}), and adversarial loss~\citep{goodfellow2014generative} similar to existing image and video latent diffusion model approaches. Recall that both training and inference are not done directly on long videos; they are done after splitting long videos into short segments.

\textbf{Diffusion model.}
As outlined in Figure~\ref{fig:model}, we design our model architecture upon recent diffusion transformer (DiT) architecture~\citep{Peebles2022DiT}. Thus, given a latent vector $\bz^n \in \mathbb{R}^{l \times h \times w \times c}$ of a video clip $\bx^n$, we patchify it with a patch size $p_l \times p_s \times p_s$ to have a flattened latent vector $\mathtt{patchify}(\bz^n) \in \mathbb{R}^{(lhw / p_lp_s^2) \times c}$ with a sequence length $lhw / p_lp_s^2$ and use it as inputs to the model. In particular, we choose W.A.L.T~\citep{gupta2023photorealistic} as backbone, a variant of DiT for videos by introducing compute-efficient spatiotemporal window attention instead of full attention between large number of video patches.

To enable training with long videos with DiT architectures, we introduce a memory-augmented attention layer and insert this layer to every beginning of the Transformer block. Specifically, we design this layer as a cross-attention layer between the previous memory latent vector and the current hidden state, similar to memory-augmented attention~\citep{dai2019transformer} in language domain. Hence, query, key, and value of a $d$-th memory layer with the segment $\bz^{n}$ and the memory latent vector $\bh^{n-1}$ become:
\begin{align*}
    \text{query}\coloneqq \bh_{d}^{n},\quad\text{key} \coloneqq [\bh_{d}^{n-1}, \bh_{d}^{n}],\quad \,\,\text{value}\coloneqq [\bh_{d}^{n-1}, \bh_{d}^{n}], \quad \bh_{d}^{n-1}, \bh_{d}^{n} \in \mathbb{R}^{(hw / p_s^2) \times (l/p_l) \times c'},
\end{align*}
where $c'>0$ denotes the hidden dimension of the model and $\bh_{d}^{n-1}, \bh_{d}^{n}$ are \emph{reshaped} as a sequence length $l/p_l$ and a batch dimension size $hw/p_s^2$, similar to space-time factorized attention in previous video Transformers~\citep{arnab2021vivit,bertasius2021space}. Thus, memory-augmented attentions are only computed together with each of $l/p_l$ patches that have the same spatial location (\ie, temporal attention in video transformers). Because the computation of attention is restricted only to the sample spatial locations, the computation increase from our attention layers does not become significant; as the former has $O(L^2HW)$ computation complexity but the latter has the $O((LHW)^2)$ computation complexity.
We also use relative positional encoding that is widely used in transformer model to handle longer context.
Finally, recall that we build our architecture on W.A.L.T, but this memory-augmented layer idea can be applied to any video diffusion transformer architectures, such as \citep{lu2023vdt,ma2024latte}. We provide an detailed illustration of the architecture combined with W.A.L.T in Appendix~\ref{appen:archi}.