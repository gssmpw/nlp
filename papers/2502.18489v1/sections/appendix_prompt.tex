% \definecolor{gray}{RGB}{50,50,50}
% \definecolor{lightgrey}{RGB}{200, 200, 200}
% \definecolor{lightbrown}{RGB}{180, 150, 100}
% \definecolor{darkgrey}{RGB}{100, 100, 100}
% \definecolor{lightbluegrey}{RGB}{150, 180, 200}
\definecolor{darkbrown}{RGB}{100, 80, 50}
% \definecolor{lightgreengrey}{RGB}{150, 200, 180}
% \definecolor{silver}{RGB}{192, 192, 192}
% \definecolor{sand}{RGB}{194, 178, 128}
% \definecolor{bluegrey}{RGB}{128, 158, 162}
% \definecolor{olive}{RGB}{128, 128, 0}
% \definecolor{lightblue}{RGB}{173, 216, 230}
% \definecolor{cornflowerblue}{RGB}{100, 149, 237}
% \definecolor{deepskyblue}{RGB}{0, 191, 255}
% \definecolor{mediumaquamarine}{RGB}{102, 205, 170}
% \definecolor{limegreen}{RGB}{50, 205, 50}
% \definecolor{gold}{RGB}{255, 215, 0}
\definecolor{darkorange}{RGB}{255, 140, 0}
% \definecolor{orangered}{RGB}{255, 69, 0}
% \definecolor{firebrick}{RGB}{178, 34, 34}
\definecolor{darkred}{RGB}{139, 0, 0}
\definecolor{softblue}{RGB}{140, 160, 200}
\definecolor{lightgreen}{RGB}{145, 204, 117}
\definecolor{lightyellow}{RGB}{250, 200, 88}
\definecolor{lightred}{RGB}{238, 102, 102}
\definecolor{lightblue}{RGB}{115, 192, 222}


% colframe=backgroundcolor,
\newtcolorbox{promptbox}[2][Prompt]{
colback=black!5!white,
arc=5pt, 
boxrule=0.5pt,
fonttitle=\bfseries,
title=#1, 
before upper={\small}, fontupper=\fontfamily{ptm}\selectfont,
colframe=#2, 
}

\lstdefinestyle{pythonstyle}{
    language=Python,
    basicstyle=\ttfamily\footnotesize,
    keywordstyle=\color{blue},
    commentstyle=\color{green},
    stringstyle=\color{red},
    backgroundcolor=\color{lightgray!20},
    frame=single,
    breaklines=true,
    postbreak=\mbox{\textcolor{red}{$\hookrightarrow$}\space}
}


\renewcommand{\lstlistingname}{File}

\lstset{
    autogobble,
    columns=fullflexible,
    showspaces=false,
    showtabs=false,
    breaklines=true,
    showstringspaces=false,
    breakatwhitespace=true,
    breaklines=true,
    backgroundcolor=\color{lightgray!20},
    escapeinside={(*@}{@*)},
    commentstyle=\color{greencomments},
    keywordstyle=\color{bluekeywords},
    stringstyle=\color{redstrings},
    numberstyle=\color{graynumbers},
    basicstyle=\ttfamily\footnotesize,
    frame=l,
    framesep=12pt,
    xleftmargin=12pt,
    tabsize=4,
    captionpos=b
}

\section{Appendix of Prompts.}
\label{sec:ap-method}

\subsection{Prompts of \tool.}
\label{sec:eg-feature-extraction}

\begin{figure}[H]
\centering
\begin{promptbox}[Task Formalization]{softblue}
\textbf{System:} \\
As a professional algorithm engineer, please analyze the given algorithm problem according to the following categories. Do not provide any example implementation:
\begin{itemize}
    \item Entry Point Function Name\item Input/Output Conditions\item Edge Cases and Parameter Types (Int, String, etc.)\item Expected Behavior
\end{itemize}
\textbf{User:} \\
The algorithm problem description is as follows:\\ <natural language description>
\end{promptbox}
\caption{Task Formalization.}
\end{figure}

\begin{figure}[H]
\centering
\begin{promptbox}[Task Formalization Check]{softblue}
\textbf{System:} \\
As an excellent algorithm engineer, please analyze whether the explanation of the problem matches the original requirements. If they are consistent, output “Yes”. If they are not consistent, output “No” and provide the reason, as shown below:
\{"Yes":"NULL"\} \\
\{"No":"The reason is"\}

\textbf{User:} \\
<natural language description> \\
<task description>
\end{promptbox}
\caption{Checking the Task Formalization Result.}
\end{figure}

\begin{figure}[H]
\centering
\begin{promptbox}[Synthesize Test Case Inputs]{softblue}
\textbf{System:} \\
As a tester, your task is to create comprehensive test inputs for the function based on its definition and docstring. These inputs should focus on edge scenarios to ensure the code’s robustness and reliability. Please output all test cases in a single line, starting with input. \\
\textbf{User:} \\
EXAMPLES: \\
Function: 
\begin{lstlisting}[breaklines=true]
from typing import *
def find_the_median(arr: List[int]) -> float:
    Given an unsorted array of integers `arr`, find the median of the array.
    The median is the middle value in an ordered list of numbers.
    If the length of the array is even, then the median is the average of the two middle numbers.
\end{lstlisting}
Test Inputs (OUTPUT format): \\
input: [1] \\
input: [-1, -2, -3, 4, 5] \\
input: [4, 4, 4] \\
input: [....] \\
input: [....] \\
END OF EXAMPLES. \\
Function: \\
<task description>
\end{promptbox}
\caption{Synthesize Test Case Inputs.}
\end{figure}

\begin{figure}[H]
\centering
\begin{promptbox}[Implementation Optimization in Code Domain]{softblue}
\textbf{System:} \\
As a professional Python algorithm programming expert, please provide suggestions for improving code efficiency based on the potential inefficiencies mentioned above. For example: \\
1. Using xxx instead of xxx can significantly improve code efficiency. \\
Please provide at least 20 suggestions.\\
\textbf{User:} \\
<algorithm description>
\end{promptbox}
\caption{Implementation Optimization in Code Domain.}
\end{figure}

\begin{figure}[H]
\centering
\begin{promptbox}[Complete Test Case Generation]{softblue}
\textbf{System:} \\
As a programmer, your task is to calculate all test outputs and write the test case statement corresponding to the test input for the function, given its definition and docstring. Write one test case as a single-line assert statement. \\
\textbf{User:} \\
EXAMPLES: \\
Function: 
\begin{lstlisting}[breaklines=true]
from typing import List
def find_the_median(arr: List[int]) -> float:
    Given an unsorted array of integers `arr`, find the median of the array. The median is the middle value in an ordered list of numbers.
    If the length of the array is even, then the median is the average of the two middle numbers.
\end{lstlisting}
Test Input: \\
input: [1, 3, 2, 5]\\
Test Case: 
\begin{verbatim}
assert find_the_median([1, 3, 2, 5]) == 2.5
\end{verbatim}
END OF EXAMPLES. \\
FUNCTION: \\
<task description>
<input case>
\end{promptbox}
\caption{Complete Test Case Generation.}
\end{figure} 


\begin{figure}[H]
\centering
\begin{promptbox}[Algorithmic Exploration in Logic Domain]{softblue}
\textbf{System:} \\
As a professional algorithm engineer, you can effectively design multiple algorithms to solve the problem with low time complexity and output them in pseudo algorithm format. A pseudo algorithm is a nonlinear, high-level programming language for algorithmic logic. It combines natural language and programming structures to express the steps and sums of algorithms. The main purpose of process algorithms is to clearly display the core ideas and logic of the algorithm without relying on specific programming language syntax. Please design 5 excellent algorithm solutions based on the problem description provided. The time complexity of the algorithm needs to be as small as possible, and try to output 5 algorithms in the form of a pseudo-algorithm in the following format:
PS: DO NOT provide implementation examples!
\begin{lstlisting}[breaklines=true]
```algorithm1
{algorithm key description: this algorithm using xxx, the key is to make sure xxx}
{pseudo algorithm: ..}

{algorithm key description: this algorithm using xxx, the key is to make sure xxx}
{pseudo algorithm: ..}

{algorithm key description: this algorithm using xxx, the key is to make sure xxx}
{pseudo algorithm: ..}

{algorithm key description: this algorithm using xxx, the key is to make sure xxx}
{pseudo algorithm: ..}

{algorithm key description: this algorithm using xxx, the key is to make sure xxx}
{pseudo algorithm: ..}
\end{lstlisting}
\textbf{User:} \\
<task description>
\end{promptbox}
\caption{Algorithmic Exploration in Logic Domain.}
\end{figure}

\begin{figure}[H]
\centering
\begin{promptbox}[Code Candidates Generation]{softblue}
\textbf{System:} \\
As a professional algorithm engineer, please convert the selected algorithm into corresponding code. Ensure the code is complete and well-formatted. When converting to a standardized format, be sure to follow the guidelines specified in the “original question format”: \\
1. Use the same function name as given in the original question format; do not rename it. \\
2. You may incorporate practical optimization details drawn from the knowledge base. \\
The final output format should be as follows:
\begin{verbatim}
```python
{<code>
```
\end{verbatim}
\textbf{User:} \\
<task description> \\
<algorithm description> \\
<efficiency optimization suggestions>
\end{promptbox}
\caption{Code Candidates Generation.}
\end{figure}


\begin{figure}[H]
\centering
\begin{promptbox}[Code Refinement for Correctness]{softblue}
\textbf{System:} \\
As a professional code programming algorithm expert, your task is to correct the code and ensure that the code is fixed without impacting its time complexity or practical efficiency. Then I will provide you with specific code and test cases. \\
Important Notes: \\
1. Do not alter the algorithm itself \\
2. Do not change the format, such as the function name. \\
3. Please output in the specified format. \\
4. Ensure there are no syntax errors. \\
Please output in this format:
\begin{verbatim}
```python
{code}
```
\end{verbatim}

\textbf{User:} \\
<task description> \\
<algorithm description> \\
<efficiency optimization suggestions>
\end{promptbox}
\caption{Code Refinement for Correctness.}
\end{figure}


\begin{figure}[H]
\centering
\begin{promptbox}[Final Results Selection on Code Candidates]{softblue}
\textbf{System:} \\
As a professional algorithm engineer, please help me choose the most efficient code from the following codes. It is worth mentioning that it is necessary to consider the time complexity and practical level comprehensively: \\
INPUT: \\
\{
"1":"def ...()....", \\
"2": "def ...()..." \\
\} \\
OUTPUT:
\begin{verbatim}
```text
{key}
```
\end{verbatim}
EXAMPLE: \\
INPUT: \\
\{
"1":"def ...()....", \\
"2": "def ...()..." \\
\} \\
OUTPUT:
\begin{verbatim}
```text
1
```
\end{verbatim}
\textbf{User:} \\
<corrected code candidate>
\end{promptbox}
\caption{Final Results Selection on Code Candidates (Optional).}
\end{figure}


\begin{figure}[H]
\centering
\begin{promptbox}[Direct Code Generation Prompt for Variant-1]{softblue}
\textbf{System:} \\
As a professional Python algorithm engineer, please solve the algorithms problem and generate a solution code. The final output format should be as follows:
\begin{verbatim}
```python
{code}
```
\end{verbatim}
\textbf{User:} \\
<task description>
\end{promptbox}
\caption{Direct Code Generation Prompt for Variant-1.}
\end{figure}

\begin{figure}[H]
\centering
\begin{promptbox}[Direct Code Generation Prompt for w/o Uniqueness-1\&w/o Uniqueness-2]{softblue}
\textbf{System:} \\
As a professional Python algorithm engineer, please solve the algorithm problem and generate 5 solution codes. Please improve the efficiency of the code as much as possible while ensuring the correctness of the code. The final output format should be as follows:
\begin{verbatim}
```python1
{code}
```
```python2
{code}
```
```python3
{code}
```
```python4
{code}
```
```python5
{code}
```
\end{verbatim}
\textbf{User:} \\
<task description>
\end{promptbox}
\caption{Direct Code Generation Prompt for w/o Uniqueness-1\&w/o Uniqueness-2.}
\end{figure}


% \begin{figure}[H]
% \centering
% \begin{promptbox}[Code Optimization]{softblue}
% \textbf{System:} \\
% As a professional Python engineer, you are familiar with the time cost of various codes in Python and can effectively modify the code to avoid unnecessary time cost. Please  optimize the following code step by step from the implementation level to make it more efficient and output the results in the form of:
% \begin{verbatim}
% ```python
% {code}
% ```.
% \end{verbatim}
% \textbf{User:} \\
% <Code>
% \end{promptbox}
% \caption{Code Optimization for w/o Uniqueness Study-2.}
% \end{figure}

\subsection{Prompts of Effi-Learner.}
\begin{figure}[H]
\centering
\begin{promptbox}[Original Code Generation Prompt in Effi-Learner]{softblue}
Please complete Python code based on the task description.\\
\# Task description:<Task description>\\
\#Solution:
\end{promptbox} 
\caption{Original Code Generation Prompt in Effi-Learner.}
\end{figure}

\begin{figure}[H]
\centering
\begin{promptbox}[Efficiency Optimization Prompt in Effi-Learner.]{softblue}
Optimize the efficiency of the following Python code based on the task, test case, and overhead analysis provided. Ensure the optimized code can pass the given test case.\\
Task Description:\\
<task description>\\
Test Case:\\
<test case>\\
Original Code:\\
\begin{verbatim}
```python
<original code>
```
\end{verbatim}
Overhead Analysis:\\
<profile of original code>\\
Optimization Rules:\\
- Encapsulate the optimized code within a Python code block (i.e., python[Your Code Here]).\\
- Do not include the test case within the code block.\\
- Focus solely on code optimization; test cases are already provided.\\
- Ensure the provided test case passes with your optimized solution.\\
\end{promptbox} 
\caption{Efficiency Optimization Prompt in Effi-Learner.}
\end{figure}


\subsection{Prompts of ECCO.}

\begin{figure}[H]
\centering
\begin{promptbox}[Original Code Generation Prompt in ECCO]{softblue}
Write a python code which is efficient in terms of runtime and memory usage for the following problem description. Wrap the optimized code in a block of 3 backticks
\end{promptbox} 
\caption{Original Code Generation Prompt in ECCO.}
\end{figure}

\begin{figure}[H]
\centering
\begin{promptbox}[Feedback Generation Prompt in ECCO]{softblue}
Give feedback in english for why the code solution below is incorrect or inefficient and how the program can be fixed based on the problem description.\\
<original code>
\end{promptbox} 
\caption{Feedback Generation Prompt in ECCO.}
\end{figure}

\begin{figure}[H]
\centering
\begin{promptbox}[Refine Prompt in ECCO]{softblue}
Refine the given incorrect or sub-optimal code solution based on the feedback specified below. Wrap the refined code in a block of 3 backticks\\
<optimization suggestion>\\
<original code>\\
\end{promptbox} 
\caption{Refine Prompt in ECCO.}
\end{figure}

\subsection{Prompts of Instruct.}

\begin{figure}[H]
\centering
\begin{promptbox}[Prompt for \textbf{Instruction} Baseline]{softblue}
Please generate an efficient and correct code directly
\end{promptbox} 
\caption{Prompt for \textbf{Instruction} Baseline.}
\end{figure}
