\section{Related Work}
\subsection{Molecular Relational Learning}
MRL is an essential task in molecular representation, encompassing various applications ____. 
In this paper, we focus on MI and DDI predictions. 
In molecular interaction prediction, the model primarily predicts changes in chemical properties caused by chemical reactions. CIGIN ____ employed a message-passing network and collaborative attention to encode intra-molecular atoms and predict solvation-free energy. 
% Their interpretability study highlights the importance of atomic interactions in molecular relationships. 
CMRL ____ combined molecular representation with GNNs and causal relationships, identifying substructures causally related to chemical reactions. MMGNN____ explicitly models chemical bonds between atoms by adding connections within the molecule, ultimately predicting solvent-free energy by attention-based aggregation.
DDI prediction focuses on identifying the interaction relationships between drug molecules, which is critical for co-medication in clinical settings ____. 
Substructure-based DDI prediction is currently a research hotspot. 
SA-DDI ____, SSI-DDI ____, and DSN-DDI ____ extract substructure information based on GNNs from drugs in various ways to better represent drug molecules. Additionally, TIGER ____ uses a Transformer architecture with a relation-aware attention mechanism to construct semantic relationships between drugs. From the data perspective, solutions targeting sparse and imbalanced data in DDI tasks have been progressively proposed____. DDIPrompt____ combines hierarchical pre-training with prompt learning to encourage an understanding of the drug's molecular properties. Moreover, researchers have advanced the interpretability of DDI by leveraging extensive drug evaluation____ and hierarchical views____. In summary, the current mainstream of MRL is to utilize graph-based molecular representations while incorporating attention mechanisms to capture interactions between molecular substructures.
% MeTDDI____ conducts an in-depth interpretation of 13,786 DDI interactions involving 73 drugs, aiming to validate its understanding of complex DDI mechanisms.
% 综上所述,目前MRL的主流是以Graph为分子表示基础,并结合注意力机制来捕获子结构间的交互关系。

% MMGNN通过在分子内添加连接的方式显式建模原子之间的化学键,最后结合注意力聚集实现溶剂自由能的预测。从数据层面来看,面向稀缺数据和不平衡数据的DDI解决方案被陆续提出。不少研究者通过广泛药物评估和分层视图等途径,更进一步的开展了DDI的可解释性工作。
% 预测通过最初在分子间原子之间形成不加区分的连接来显式模拟氢键等原子相互作用,然后使用基于注意力的聚集方法进行细化,针对特定的溶质-溶剂对进行定制。
% Although existing MRL methods have advantages in predictive performance, they do not adequately consider the difficulty in accurately identifying response sites due to the spuriousness of attention mechanisms, which significantly reduces the interpretability of the model.


% \textbf{Molecular Relational Learning}.
% MRL is an essential task in molecular representation____. 
% In this paper, we focus on MI and DDI prediction. 
% In MI prediction, CIGIN____ employed a message-passing network to encode molecular atoms. CMRL____ combined molecular representation with causal relationships. 
% In DDI prediction, SA-DDI____, SSI-DDI____, and DSN-DDI____ extracted substructure information in various ways to better represent drug molecules.

\subsection{Representational Alignment}
Representational alignment refines embedding representations to guarantee that input data is coherently and precisely captured within the representation space____. Alignment enhances model performance and reliability in complex environments. In the era of large language models (LLMs), representational alignment has become a critical technology for building LLMs that meet specific human needs____.
Existing studies primarily introduce alignment through inductive bias, behavior imitation, environmental feedback, and model feedback____. In molecular representations, researchers predominantly adopt attention mechanism-based inductive bias to realize representational alignment____.
Molecular representational alignment is a critical step in models that accurately capture molecules' structural and functional properties____.
Graphormer____ achieves molecular-level representational alignment through message passing, while CGIB____ focuses on aligning molecular representations at the substructure level. The Dual-Graph Framework____ combines molecular-level and substructure-level alignments, integrating the strengths of both approaches.
% 分子表示对齐。分子表示对齐是准确反映分子结构和功能特性的模型中关键的一步,是反应建模、关系预测的基础。Graphormer基于消息传递实现分子层面表示对齐,而CGIB则从子结构层面对齐分子表示。Dual-Graph Framework将分子层级和亚结构层级对齐相结合。DDIPrompt将分层预训练与提示学习相结合来鼓励模型全面的理解药物分子的特性。MeTDDI通过对73种药物涉及的13786个DDI进行全面解释以验证其对复杂DDI机制的理解。
% 表示对齐是指通过优化模型嵌入表示使模型输入在表示空间中得到一致且准确的反映。对齐能够提升模型在复杂环境中的表现和可信度。尤其是在大语言模型时代,表示对齐已成为构建满足人类特定需求的大语言模型的关键技术。根据表示对齐信号的不同,现有的研究主要通过归纳偏差、行为模仿、环境反馈和模型反馈的方式引入对齐信号。在分子表示中,研究者大多采用基于注意力机制的归纳偏差实现表示对齐。


Existing molecular representational alignment methods rely on attention mechanisms to compute inductive bias, resulting in results that reflect statistical correlations and lack dynamic adaptability. Therefore, we incorporate chemical domain knowledge into ReAlignFit to dynamically align the representations between substructures based on simulated molecular conformational changes to identify substructure pairs with high functional compatibility.
% 现有的分子表示对齐方法大多基于注意力机制计算归纳偏差,导致结果反映的是统计相关性且缺乏动态适应性。因此,我们将领域知识融入ReAlignFit,在模拟分子构象变化的基础上动态的对齐子结构间的表示以识别具有高功能兼容性的子结构。