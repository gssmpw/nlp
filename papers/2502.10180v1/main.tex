%%%%%%%%%%%%%%%%%%%%%%%%%%%%%%%%%%%%%%%%%%%%%%%%%%%%%%%%%%%%%%%%%%%%%%%%%%%%%%%%
%2345678901234567890123456789012345678901234567890123456789012345678901234567890
%        1         2         3         4         5         6         7         8

\documentclass[letterpaper, 10 pt, conference]{ieeeconf}  % Comment this line out if you need a4paper

%\documentclass[a4paper, 10pt, conference]{ieeeconf}      % Use this line for a4 paper

\IEEEoverridecommandlockouts                              % This command is only needed if 
                                                          % you want to use the \thanks command

\overrideIEEEmargins                                      % Needed to meet printer requirements.

%In case you encounter the following error:
%Error 1010 The PDF file may be corrupt (unable to open PDF file) OR
%Error 1000 An error occurred while parsing a contents stream. Unable to analyze the PDF file.
%This is a known problem with pdfLaTeX conversion filter. The file cannot be opened with acrobat reader
%Please use one of the alternatives below to circumvent this error by uncommenting one or the other
%\pdfobjcompresslevel=0
%\pdfminorversion=4

% See the \addtolength command later in the file to balance the column lengths
% on the last page of the document

% The following packages can be found on http:\\www.ctan.org
%\usepackage{graphics} % for pdf, bitmapped graphics files
%\usepackage{epsfig} % for postscript graphics files
%\usepackage{mathptmx} % assumes new font selection scheme installed
%\usepackage{times} % assumes new font selection scheme installed
%\usepackage{amsmath} % assumes amsmath package installed
%\usepackage{amssymb}  % assumes amsmath package installed
\usepackage{cite}
\usepackage{graphics} % for pdf, bitmapped graphics files
\usepackage{epsfig} % for postscript graphics files
%\usepackage{mathptmx} % assumes new font selection scheme installed
\usepackage{times} % assumes new font selection scheme installed
\usepackage{amsmath,bm}
\usepackage{amssymb}  % assumes amsmath package installed
\usepackage{cases}

\DeclareMathOperator{\sign}{sign}

\usepackage{mathtools}

\usepackage{caption}
\usepackage{subcaption}

\usepackage[usenames, dvipsnames]{color}
\newtheorem{assumption}{Assumption}
\newtheorem{thm}{Theorem}
\newtheorem{lem}{Lemma}
\newtheorem{prop}[thm]{Proposition}
\newtheorem{cor}{Corollary}
\newtheorem{conj}{Conjecture}[section]
\newtheorem{defn}{Definition}[section]
\newtheorem{exmp}{Example}[section]
\newtheorem{rem}{Remark}
\newtheorem{prob}{Problem}

\usepackage{graphicx}
\usepackage{float}
\usepackage{xcolor}

\usepackage{siunitx}


\usepackage{hyperref}
\hypersetup{
    colorlinks=true,
    linkcolor=black,
    citecolor=black,
    urlcolor=blue,
}

\newcommand{\TODO}[1]{{\textcolor{purple}{[\textbf{TODO:} #1]}}}
\newcommand{\lt}[1]{{\color{red} #1}}
\newcommand\tang[1]{{\color{magenta} #1}}
\title{\LARGE \bf
Safe platooning control of connected and autonomous vehicles on curved multi-lane roads  }


\author{Xiao Chen$^{1}$, Zhiqi Tang$^{1}$, Karl H. Johansson$^{1}$ and Jonas Mårtensson$^{1}$% <-this % stops a space
\thanks{$^{1}$ Division of Decision and Control Systems, School of Electrical Engineering and Computer Science, KTH Royal Institute of Technology, Sweden. Emails: {\tt\small \{xiao2, ztang2, kallej, jonas1\}@kth.se}}%
}


\begin{document}



\maketitle
\thispagestyle{empty}
\pagestyle{empty}


%%%%%%%%%%%%%%%%%%%%%%%%%%%%%%%%%%%%%%%%%%%%%%%%%%%%%%%%%%%%%%%%%%%%%%%%%%%%%%%%
\begin{abstract}
% In this paper, we tackle the challenge of vehicle platoon formation, considering both longitudinal and lateral dynamics within a multi-lane highway context. Starting with a kinematic bicycle model, we derive a transformation to the double integrator model at the front axle of each vehicle. Under the assumption of an acyclic digraph with a single spanning tree representing the assigned vehicle topology, we present a safe platoon formation controller at the vehicle's front axel that leverages constructive barrier feedback to achieve both formation tracking and collision avoidance simultaneously. Here, a dissipative term using divergent flow is added as the reactive collision avoidance component. Through stability analysis, we establish the convergence of vehicles to the desired formation and ensure collision avoidance between vehicles and with road boundaries during the formation process under the guidance of the proposed controller. Finally, we evaluate the performance of our method through simulation studies, demonstrating its effectiveness in achieving multi-vehicle platoon formation in on-road scenarios while maintaining safety. 

% This paper investigates the multi-vehicle platoon formation problem on curved roads. We propose a nominal formation control approach that considers the bicycle model within the path-aligned Frenet frame. This framework facilitates the separation of the control objective into two distinct parts: a path-following component for the actual vehicle and a longitudinal formation component for the virtual vehicle obtained through the orthogonal projection of the ego vehicle onto the reference path. To ensure safety, we augment the nominal tracking control with constructive barrier feedback, which supports both lateral and longitudinal collision avoidance. This feedback mechanism acts as a dissipative term, slowing down the vehicle when approaching a potential collision, without compromising the convergence properties of the nominal controller. Consequently, our proposed control method enables the safe platoon formation of vehicles on curved roads, with theoretical guarantees of safety and stability. Simulation results from various initial settings further validate the applicability and performance of the proposed controller for achieving the desired formation objectives.

This paper investigates the safe platoon formation tracking and merging control problem of connected and automated vehicles (CAVs) on curved multi-lane  roads. The first novelty is the separation of the control designs into two distinct parts: a lateral control law that ensures a geometrical convergence towards the reference path regardless of the translational velocity, and a longitudinal control design for each vehicle
to achieve the desired relative arc length and velocity with
respect to its neighboring vehicle. The second novelty is exploiting the constructive barrier feedback as an additive term to the nominal tracking control, ensuring both lateral and longitudinal collision avoidance. This constructive barrier feedback acts as a dissipative term, slowing down the relative velocity toward obstacles without affecting the nominal controller's performance. Consequently, our proposed control method enables safe platoon formation of vehicles on curved multi-lane roads, with theoretical guarantees for safety invariance and stability analysis. Simulation and experimental results on connected vehicles are provided to further validate the effectiveness of the proposed method.
\end{abstract}


%%%%%%%%%%%%%%%%%%%%%%%%%%%%%%%%%%%%%%%%%%%%%%%%%%%%%%%%%%%%%%%%%%%%%%%%%%%%%%%%

\section{Introduction}
Interest in connected and automated vehicles (CAVs) has surged in recent years, driven by advancements in automation and communication technologies. CAVs offer the potential for cooperative on-road operations, promising significant benefits to the transportation sector. Vehicle platooning has emerged as a popular concept among the various forms of cooperative driving technology. Platooning involves a train of CAVs operating with minimal inter-vehicle distances and synchronized speeds. The exploration of vehicle platooning traces back to the early 2000s, initiated by pioneering projects such as PATH \cite{Path} and Sartre \cite{Sartre}, which aimed to assess the potential benefits of this technology. Subsequent studies such as \cite{platoonbeneanalysis,platooningBenefit,turri2016cooperative} have reinforced that vehicle platooning is promising to improve traffic efficiency, enhance safety, and reduce energy consumption.

Earlier works on designing control techniques to support vehicle platooning mainly focused on the simplified one-dimensional longitudinal control of vehicles for distance keeping and velocity synchronization by guaranteeing string stability through the design \cite{platoonlinearstability, Slidingpractical, slidningintegral}. In a more general multi-lane traffic setting, muti-vehicle platooning requires adaptive formation adjustments of vehicles from different lanes while accommodating the road’s shape. The key challenge here is to design efficient platoon control strategies for multi-vehicle systems in structured road environments, preventing vehicle-to-vehicle collisions and encroachment on road boundaries. 

 Most of the existing work in the literature about safe platooning and merging control in multi-lane scenarios is restricted to the simplified type of roads with straight lines, and a majority of them typically focus only on lateral or longitude controller design. For instance, the work in \cite{lyapunov2Dcruise} adopts potential function-based control strategies for a group of vehicles autonomously driving on the straight-line road with velocity consensus. Some other work uses optimization-based controllers, such as model predictive control and barrier function-based optimization controllers. In the work of \cite{MPCdistributed}, a safety-assuring MPC framework is developed, however, focusing only on safe lateral merging maneuvers of vehicles on straight-line roads.  Another approach in \cite{CBFlongi}  exploits barrier function-based optimization controllers but only for longitudinal merging control. %\lt{Since the CBF is handcrafted in practice, the corresponding invariant safe set does not coincide with the maximal control invariant set, which can be traditionally computed through Hamilton Jacobi reachability analysis \ref{} that provides formal safety guarantee under worst case scenario.}
 It is worth noting that the use of optimization-based controllers poses challenges in explicitly
analyzing the equilibrium and convergence of the multi-vehicle
system, in addition to potential computational complexity and feasibility issues. To the best of our knowledge, the control problems of safe platooning and merging for CAVs on multi-laned and curved roads remain open.


Motivated by the above-mentioned open problems, in this paper, we propose a novel decentralized control strategy for the safe platooning and merging problem oncurved multi-lane roads, covering both longitudinal and lateral design. Each vehicle is modeled as a nonholonomic second-order kinematic bicycle model. To handle the platoon formation tracking control on curved roads,  we decouple the control design strategy into path following and formation control problems. Specifically, on one side, lateral control laws are designed for each vehicle ensuring a geometrical convergence towards the reference path regardless of the translational velocity; on the other side, longitudinal control laws are proposed for each vehicle to achieve the desired relative arc length and velocity with respect to its neighboring vehicle.

One of the key distinctions of the proposed work is that we adopt a novel concept of constructive barrier feedback, first presented in \cite{Zhiqi2023Constructive}, for reactive collision avoidance. The constructive barrier feedback exploits \textit{divergent flow} \cite{bhagavatula2011optic}, a natural feature inspired by insects and birds, to prevent collisions while effectively achieving the primary control objective. %The controllers proposed in this paper are designed as the sum of two parts. One is the nominal controllers that drive the group of vehicles to the desired platoon on the reference path. Another is the 
In this work as well as in our preliminary work \cite{XiaoPlatoonform}, constructive barrier feedback is exploited as an additive term to the nominal lateral and longitudinal controllers, which effectively avoids collision between neighboring vehicles and the road edges without compromising the nominal control objectives. Building on our preliminary results in \cite{XiaoPlatoonform}, which considered simplified vehicle dynamics (second-order systems) and straight roads, this work extends the previous findings by addressing a group of CAVs modeled as nonholonomic second-order kinematic bicycle systems operating on curved roads. The incorporation of constructive barrier feedback enables the design to take an explicit state-feedback form, offering a simple and elegant structure with low computational complexity, while providing formal guarantees for safety invariance and stability.

Under the proposed control methods, all vehicles accurately track the desired reference path and converge to the desired formation while maintaining safe distances from both their platoon predecessor and road boundaries, provided the initial conditions are safe. The effectiveness of the proposed approach is demonstrated through theoretical analysis, simulation studies, and experimental validation using connected miniature vehicles.


%\cite{distributedformation,consensusformation,lyapunov2Dcruise}
% Various methods are commonly employed to provide collision avoidance, including potential function-based strategies \cite{Distformation,Doubleflocking,distributedformation,consensusformation,lyapunov2Dcruise}, model predictive control (MPC) \cite{MPCdistributed,MPChierarchical,MPCcooperativelongitud}, and control barrier function (CBF) approaches \cite{CBFQP,CBFmulti,CBFobstacle,CBFlongi,CBFsafetyplatoon}. For potential function-based strategies, methods have been proposed for collision avoidance during formation for both single integrator dynamics \cite{Distformation} and double integrator dynamics \cite{Doubleflocking}. Several studies \cite{distributedformation,consensusformation,lyapunov2Dcruise} have adapted these approaches for vehicle platoon formation. Of note, the work in \cite{lyapunov2Dcruise} bears resemblance to the problem addressed in this paper. Collision avoidance and convergence are achieved by combining the potential function approach with a Lyapunov-based method. However, it's worth noting that this method does not pursue convergence towards a fixed final formation, and its operation is limited to straight road sections. In the MPC-based approach, collision avoidance is attained by integrating safety constraints directly into the receding horizon optimization problem. However, ensuring safety at all times necessitates recursive feasibility, which can only be accomplished through dedicated design of terminal constraints and cost functions. In \cite{MPCdistributed}, a safety-assuring MPC framework is developed to facilitate safe lateral merging maneuvers of vehicles into a platoon. On the other hand, in \cite{MPCcooperativelongitud}, the emphasis is on achieving collision-free longitudinal formation.
% CBF-based method \cite{CBFQP,CBFmulti} is an emerging 
% research direction to guarantee invariant safety given safe initial states. Control objective and safety can be achieved simultaneously by solving a Quadratic Programming (QP) problem that combines the control Lyapunov function (CLF) and the CBF into the constraints. Its application in vehicle control has mainly focused on lane changing and collision avoidance as was done in \cite{CBFobstacle}. For platoon related research, the approach has been adapted in longitudinal merging control in \cite{CBFlongi}, and for lane change of platoon through rule-based CBF design in \cite{CBFsafetyplatoon}. Common to the CBF-based methods, scenario-specific barrier function candidates need to be designed to ensure safety. In addition, due to the nature of the optimization problem, ensuring feasibility and therefore convergence property is a recurring issue for the approach. 


% Recently, Tang et al. proposed in \cite{Zhiqi2023Constructive} a novel concept of constructive barrier feedback which exploits \textit{divergent flow} \cite{bhagavatula2011optic} to prevent collisions while maintaining primary formation objective. Following this design direction, in \cite{XiaoPlatoonform}, we proposed a safe platoon formation control based on the double integrator model. The resulting controller is effective in guiding vehicles safely into a platoon formation in a constrained road segment of a straight shape. In this paper, we extend the applicability of the controller to general curved roads. For this, a kinematic bicycle model over the reference path aligned Frenet frame is considered, facilitating decoupled nominal tracking control between lateral and longitudinal directions. For safety, constructive barrier feedback is added to the nominal tracking control, where a variant form of divergent flow apart from the one used in \cite{Zhiqi2023Constructive} is introduced to enable the decoupling. The divergent flow acts as a dissipative force that slows down the vehicle against potential collision without compromising the nominal formation objective. As a result, under safe initial conditions, all vehicles can track the desired reference path and converge towards the desired formation while maintaining a safe distance from their platoon predecessor and road boundaries. The proposed controller is in the explicit state feedback form which is simple and elegant in design and possesses low computational complexity. The performance and safety properties of the proposed method are both validated in theoretical analysis and through simulation studies.     


The rest of the paper is structured as follows. Section \ref{sec:Pre} provides preliminary results about Constructive barrier feedback. In Section \ref{sec.problem}, we define the platoon formation problem and present the vehicle model used for control design in this study. Section \ref{sec.control} outlines the formulation of the proposed method. The stability property of the method is further stated in Section \ref{sec.stability}. Section \ref{sec.result} presents simulation results aimed at validating the effectiveness of the proposed method. Additionally, we compare these results with a baseline, represented by the nominal control without the constructive barrier feedback component. Experimental studies based on miniature vehicles are conducted and presented in Section \ref{sec.experiment}. Finally, concluding remarks and insights for future research are summarized in Section \ref{sec.conclusion}. 

\section{Preliminary} \label{sec:Pre}
\subsection{Constructive barrier feedback for collision avoidance}
In this section, we will recall the concept of constructive barrier feedback proposed in \cite{Zhiqi2023Constructive} for collision avoidance of a leader-follower structure in which each agent dynamics is described as a double integrator as follows
	\begin{equation}\label{eq:double integrator}
		\left\{
		\begin{aligned}
			\dot{\mathbf{p}}_i&=\mathbf{v}_i\\
			\dot{\mathbf{v}}_i&=\mathbf{u}_i%,\;i=1,...,n.
		\end{aligned}
		\right.
	\end{equation}
where  $ \mathbf{p}_i\in\mathbb{R}^m (m\ge 2)$ is the position,  $ \mathbf{v}_i\in\mathbb{R}^m$ is the velocity of each agent $i$, respectively, and $\mathbf{u}_i\in\mathbb{R}^m$ is the input acceleration. 
%  The relative position vectors between two neighboring agents $i$ and $i-1$ is defined as:
% 	\begin{equation}
% 		\label{eq:eij}
% 		e_{i}:=p_{i-1}-p_{i}, \ i\ge 2.
% 	\end{equation}
	
 
%  Similarly, $\nu_{i}:=\dot e_{i}=v_{i-1}-v_{i}$ denotes the relative velocity between agent $i$ and $i-1$. As long as $\|e_{i}\|\ne 0$, one can define  direction vector from $i$ to $i-1$ as:
% 	$$g_{i}=\frac{e_{i}}{\|e_{i}\|}\in\mathbb{S}^1 $$
% 	where $\mathbb{S}^1:=\{y\in\mathbb{R}^2:\|y\|=1\}$ denote the 1-Sphere.
 
%  Let $r$ be a positive constant that we term the safety distance and define 
% \begin{equation}
% d_{i}:=\|e_{i}\|-r=\|p_{i-1}-p_i\|-r.
% \end{equation}
% A straightforward computation shows that $\dot d_{i}=g_{i}^\top \nu_{i}$.

% To prevent collisions between neighboring agents, the key principle is controlling the relative velocity along the direction $g_{i}$, i.e. $\dot d_{i}=g_{i}^\top \nu_{i}$. 
To get an effective reactive collision avoidance without affecting the stability property of the nominal controller, feedback controller $u_i$ is designed in \cite{Zhiqi2023Constructive} as:
	\begin{equation}\label{eq:ui}
		\mathbf{u}_i=\mathbf{u}_i^n+\mathbf{u}_i^c, 
	\end{equation}
	where $\mathbf{u}_i^n$ is the nominal control input ensuring the asymptotic (or the exponential) convergence of the states $(\mathbf{p}_i,\mathbf{v}_i)$ to the desired trajectory. $\mathbf{u}_i^c$, is a dissipative \textit{control barrier feedback}  slowing down the relative velocity of agent $i$ in the direction of the neighbor agent $j$, $\mathbf{g}_{ij}:=\frac{\mathbf{p}_i-\mathbf{p}_j}{\|\mathbf{p}_i-\mathbf{p}_j\|}$, without compromising the stability nature of the nominal control action
 \begin{equation}\label{eq:cbf}
		\mathbf{u}_i^c=\mathbf{g}_{ij}\frac{\dot d_{ij}}{d_{ij}}
	\end{equation}
 where $d_{ij}:=\|\mathbf{p}_i-\mathbf{p}_j\|-r$ and the divergent flow
 $\frac{\dot d_{ij}}{d_{ij}}$
can be obtained directly from the optical flow using visual information \cite{rosa2014opticalflow}, or estimated from the measure of $d_{ij}$ \cite{hua2010telemetric_measurements}.
	
	
To illustrate the obstacle avoidance principle employed in this context, let's consider a 2-agent system. Using the above definitions of $d=d_{ij}$ and $\dot{d}=\dot d_{ij}=\mathbf{g}_{ij}^\top (\mathbf v_i-\mathbf v_j)$, it is straightforward to verify that:
	\begin{equation} \label{ddot_d}
		\ddot d=-k_o\frac{\dot d }{d} -\alpha_i(t) 
	\end{equation}
	with $\alpha_i(t)=-\frac{\|\mathbf{\pi}_{\mathbf{g}_{ij}}(\mathbf{v}_i-\mathbf{v}_j)\|^2}{d+r} -\mathbf{g}_{ij} ^\top(\mathbf{u}_i-\mathbf{u}_{j})$.
	The barrier effect of $\mathbf{u}_i^c$, is announced  in the following technical lemma:
	\begin{lem}\label{lem:boundness of OF}
			Given the dynamics \eqref{ddot_d}
			with $k_o$ a positive gain and $\alpha_i(t)$ a continuous and bounded function. Then for any initial condition satisfying $d(0)>0$ and $\phi(0)=\frac{\dot d(0)} {d(0)}$ bounded, the following assertions hold:
			\begin{enumerate}
				\item $d$ remains positive, $\forall t\ge 0$.
				\item $d$ converges to zero as $t\to \infty$ if and only if $\lim_{t \to \infty} \int^\top_0 \alpha(\tau) d\tau \to +\infty$. 
				\item If $d$ converges to zero, then $\dot{d}$ is bounded and converges to zero, and $\phi(t)$ remains bounded, $\forall t\ge 0$. Furthermore, if $\alpha_i(t)$ converges to a positive constant $\alpha^0>\epsilon >0$, then $\frac{\dot d}{d}\to -\frac{\alpha^0}{k_o}$ and hence $\ddot d$ converges to zero.
			\end{enumerate}
	\end{lem}

Proof of the lemma is given in \cite{Zhiqi2023Constructive}. This lemma shows the safety invariance property, such that, as long as the initial distance $d(0)$ is positive, $d$ will never cross zero for all times as long as the nominal controller $\mathbf{u}^n_i$, the neighboring agent input $\mathbf{u}_{j}$, and the relative velocity $\mathbf{v}_i-\mathbf{v}_j$ are continuous and bounded.
\section{problem formulation}\label{sec.problem}
This paper considers a platoon formation tracking control problem for $n$ vehicles on a curved multi-lane road. The concept of constructive barrier feedback is exploited in the control design to ensure collision avoidance between neighboring vehicles and the road edges during the platoon formation process. Each vehicle in the platoon is assumed to be connected to its neighboring agents under a directed graph topology as described in the following assumption:
\begin{assumption}\label{ass:topology}
		The topology $\mathcal G$ is fixed and described by an acyclic digraph with a single directed spanning tree, as shown in Fig. \ref{fig:topology}.
		Without loss of generality, agents are numbered (or can be renumbered) such that agent $1$ is the leader, i.e.,  $\mathcal{N}_1= \varnothing$,  all other agents $i, \ i\ge 2$ are followers whose neighboring set is $\mathcal{N}_i = \{i-1\}$.
\end{assumption}

For vehicle dynamics, instead of a simple second-order integrator model considered as in the previous work \cite{Zhiqi2023Constructive}, the motion of each vehicle $i$ is described by a nonholonomic second-order kinematic bicycle model%a kinematic bicycle model as follows:
\begin{equation}\label{eq.kinematic}
    \left\{
    \begin{aligned}
        &\dot{p}_{xi} = v_i\cos\theta_i\\
        &\dot{p}_{yi} = v_i\sin\theta_i\\
        &\dot{v}_i = a_i\\
        &\dot{\theta}_i = v_i\chi_i
    \end{aligned}
    \right.
\end{equation}
where $\mathbf{p}_i = [p_{xi}\; p_{yi}]^\top \in \mathbb R^2$ is the center position of the rear axle for vehicle $i$ expressed in a common fixed world frame $\mathcal I$. $\theta_i$ and $v_i$ indicate the orientation and speed of vehicle $i$, respectively. The control input are acceleration $a_i$ and input curvature $\chi_i=\frac{\tan\delta_i}{L_i}$, where $\delta_i$ and $L_i$ denote steering angle and wheel base of vehicle $i$. 

% The platoon formation tracking control on a curved road requires the simultaneous fulfillment of path-following and platoon formation objectives. The control approach taken in this paper is to effectively decouple the problem as i) finding lateral control laws in terms of $\chi_i$ for follower vehicles that ensure geometric convergence to the reference path regardless of the translational velocity $v_i$; ii) finding longitudinal control laws in terms of $a_i$ for follower vehicles to achieve the desired platoon distance with its neighboring vehicle on the reference path while maintaining the same desired speed. 
The platoon formation tracking control on a curved road can be considered a combination of path-following and formation-control problems. The control design strategy, hence, can be decoupled as i) designing lateral control laws for follower vehicles, ensuring a geometrical convergence towards the reference path regardless of the translational velocity $v_i$; ii) deriving longitudinal control laws for the follower vehicles to achieve desired relative arc length with its neighboring vehicle %$i-1\  (i\ge 2)$, $e_i^*:=s_{i-1}^*-s_{i}^*>0,$ 
on the reference path while maintaining the same desired speed as its neighboring vehicle $i-1$. 
% \lt{For path-following in detail, consider a given reference path on the road and let it be parameterized by its arc length $s$ according to $\mathbf{\Gamma}(s)\in \mathbb{R}^2$ in the world frame $\mathcal I$ as in Fig. \ref{fig.frame}. In addition, let $\chi^r(s)$ and $\theta^r(s)$ denote the curvature and heading of the reference path at $s$, respectively.}
% To conveniently handle the vehicle control problem along the given reference path, we transform the expression of vehicle dynamics \eqref{eq.kinematic} from the fixed world frame $\mathcal I$ to the path-aligned Frenet frame $\mathcal F_i$ similar to the path following control problems in \cite{Frenet}. \lt{The path-aligned Frenet frame $\mathcal{F}_i$ together with its longitudinal and lateral axes $\rho_i$ and $\eta_i$ is shown in Fig. \ref{fig.frame}. It can be seen as a kinematically equivalent fictitious vehicle's virtual frame obtained through orthogonal projection from vehicle $i$'s position on the path point $\Gamma(s_i)$ with $s_i$ the corresponding distance progression for vehicle $i$. As a result of the orthogonal projection, the velocity $v_i^r$ of the virtual vehicle is equivalent to $\dot{s}_i$ and satisfies the following condition}

To conveniently handle the vehicle control problem along the given reference path, we transform the expression of the vehicle dynamics from the fixed world frame  $\mathcal I$ to a reference frame aligned with the path $\mathcal F_i$ similarly to the path following control problems in \cite{Frenet}. The path-aligned frame is analogous to the Frenet frame with the distinction that the normal direction of the path is not necessarily oriented toward the path's curve. It can be seen as a kinematically equivalent fictitious vehicle's virtual frame on the path with longitudinal and lateral axes $\boldsymbol{\rho}_i$ and $\boldsymbol \eta_i$, respectively, as shown in Fig. \ref{fig.frame}. Since the virtual vehicle $i$ is defined by the orthogonal projection of the actual vehicle $i$'s position onto the reference path, using the result from \cite{Frenet}, the velocity of the virtual vehicle satisfies the following condition
\begin{equation}\label{eq.vr}
v_i^r=\frac{v_i\cos\Tilde{\theta}_i}{1-\chi_i^r\Tilde{y}_i}\\
\end{equation}
% where $\chi_i^r$ shorthand for $\chi_i^r(s_i)$ is curvature of the virtual vehicle on the reference path, $\tilde{y}_i$ and $\tilde{\theta}_i = \theta_i - \theta^r_i$ denote the lateral displacement error and orientation error of vehicle $i$ with respect to the virtual frame $\mathcal F_i$ with following dynamics
where $\tilde{y}_i$ and $\tilde{\theta}_i = \theta_i - \theta^r_i$ denote the lateral displacement error and orientation error of vehicle $i$ with respect to the virtual frame $\mathcal F_i$ with following dynamics

\begin{equation}\label{eq.dyna_frenet}
    \left\{
    \begin{aligned}
        &\dot{\Tilde{y}}_i = v_i\sin\Tilde{\theta}_i\\
        &\dot{\Tilde{\theta}}_i = v_i\left(\chi_i - \frac{\chi_i^r\cos\Tilde{\theta}_i}{1-\chi_i^r\Tilde{y}_i}\right).
    \end{aligned}
    \right.  
\end{equation}
% Here, the input curvature $\chi_i$ is to be designed as the lateral controller to drive $(\tilde y_i,\tilde \theta_i)$ to zero. \lt{Note that equations \eqref{eq.vr} and \eqref{eq.dyna_frenet} are well defined under condition
% $|\tilde y_i|<\frac 1 {|\chi_i^r|}$, which will be guaranteed through lateral control design in the next section. In addition, since $v_i^r = \dot{s}_i$, satisfaction of $|\tilde y_i|<\frac 1 {|\chi_i^r|}$ also guarantees the uniqueness of the orthogonal projection between frames. }
where $\chi_i^r$ is curvature of the virtual vehicle $i$ on reference path. The curvature $\chi_i$ is the control input to be designed as the lateral controller to drive $(\tilde y_i,\tilde \theta_i)$ to zero. Note that equation \eqref{eq.vr} and \eqref{eq.dyna_frenet} are well defined as long as the projection onto the path is unique, that is  
$|\tilde y_i|<\frac 1 {\chi_i^r}$.

Note that in the classical path-following problem, translational and orientation control are typically decoupled to stabilize the equilibrium $(\tilde{y},\tilde{\theta})=(0,0)$ independently from the translational motion. In the proposed approach, longitudinal control is an additional objective aimed at tracking the preceding vehicle. In this context and in contrast to the path-following literature, we reverse the constraint in \eqref{eq.vr} by first imposing the desired dynamics of the virtual vehicle $v_i^r$ and then use this constraint to determine the control action for the actual vehicle:
\begin{equation}\label{eq.vi}
v_i=v_i^r\frac{1-\chi_i^r\Tilde{y}_i}{\cos\Tilde{\theta}_i}\\
\end{equation}
The reversed constraint is well-defined given $|\Tilde{\theta}_i|<\frac{\pi}{2}$. Hence, the design strategy focuses first on the virtual acceleration control input for the virtual follower vehicle $i$. Analyze the dynamics of arc length and velocity $(s_i,v_i^r)$ of the virtual vehicle $i$, 

% For platoon formation, the design strategy is to decouple the longitudinal control from the lateral control. We first focus on the virtual acceleration control input for the virtual follower vehicle $i$.  Since the virtual follower vehicles are already on the reference path, we analyze the dynamics of arc length and velocity $(s_i,v_i^r)$ of the virtual vehicle $i$, 
\begin{equation}\label{eq.dyna_frenet_lon_ref}
    \left\{
    \begin{aligned}
        &\dot{s}_i = v_i^r\\
        &\dot{v}^r_i = a^r_i
    \end{aligned}
    \right.  
\end{equation}
where $a_i^r$ is the virtual acceleration control to be designed to drive $(s_i,v_i^r)$ to the desired arc length and velocity $(s_i^*,v_i^*)$. 
Note that as long as  the constraints $|\tilde y_i|<\frac 1 {\chi_i^r}$ and $|\tilde{\theta}_i|<\frac \pi 2$ associated with equations \eqref{eq.vr} and \eqref{eq.vi} are fullfilled respectively, one can recover the actual input $a_i$ from $a_i^r$, $\chi_i$, the state variables $(v_i,\tilde \theta_i, \tilde y_i)$ and the reference path information $(\chi_i^r,\dot \chi_i^r)$ using the derivative of \eqref{eq.vi}:

% Recall \eqref{eq.vr} and \eqref{eq.dyna_frenet}, the actual acceleration input $a_i$ can be reconstructed from $a_i^r$, $\chi_i$, the state variables $(v_i,\tilde \theta_i, \tilde y_i)$ and information of reference path $(\chi_i^r,\dot \chi_i^r)$, by the following relationship
\begin{equation}\label{eq.ar2a1}
\begin{aligned}
    a_i = & \frac{1}{\cos\Tilde{\theta}_i}\biggl(
    a_i^r(1-\chi_i^r\Tilde{y}_i)+v_i\sin\Tilde{\theta}_i \dot{\Tilde{\theta}}_i\\
    &-v_i^r\left(\dot \chi_i^r \Tilde{y}_i+\chi_i^r\dot{\Tilde{y}}_i\right)\biggr)
\end{aligned}
\end{equation}
Now, since for any continuous function $f(t)$ and $g(t)$ having $\dot{f}(t)=\dot{g}(t)$ does not imply that $f(t)=g(t)$, one suggests the following modification to satisfy the constraint \eqref{eq.vi} :
\begin{equation}\label{eq.ar2a}
\begin{aligned}
    a_i = & \frac{1}{\cos\Tilde{\theta}_i}\biggl(
    a_i^r(1-\chi_i^r\Tilde{y}_i)+v_i\sin\Tilde{\theta}_i \dot{\Tilde{\theta}}_i\\
    &-v_i^r\left(\dot \chi_i^r \Tilde{y}_i+\chi_i^r\dot{\Tilde{y}}_i\right)\biggr)-k\left (v_i^r\frac{1-\chi_i^r\Tilde{y}_i}{\cos\Tilde{\theta}_i}-v_i \right)
\end{aligned}
\end{equation}
with $k$ a positive real number.
% which is well defined given that $|\Tilde{\theta}_i|<\frac{\pi}{2}$, $|\tilde y_i|<\frac 1 {|\chi_i^r|}$, and $\chi_i^r$ is a continuous and differentiable function of $s$. Note that when $\Tilde{y}_i = 0$ and $\Tilde{\theta}_i = 0$, one has $a_i = a_i^r$.

% \lt{The desired platoon formation along the reference path is encoded by the desired vehicle index ordering $i$ and the corresponding desired state $[s_i^* \; v_i^*]^\top$. The following assumption assigns the order of the vehicles' index according to their initial distance progression along the reference path.}
% \begin{assumption}\label{ass.vehicle_index}
%     Provided $s_i(0)-s_j(0) \ne 0, \forall i,j\in \mathcal{V} = {1,2,...,n}$, the vertex set of $n$-vehicle system is assigned such that $s_{i-1}(0)-s_i(0)>0,\forall i\geq 2$.
% \end{assumption}
With the above description, the desired platoon formation is defined as follows (see Fig. \ref{fig.formation}):
% The desired state for platoon formation is defined as follows. Fig. \ref{fig.formation} also illustrates the desired formation graphically.
\begin{assumption}\label{ass.platoon_form}
    The leader vehicle, agent 1, is independently controlled such that it is already in its desired configuration: $\tilde y_1=0$, $\tilde{\theta}_1=0$, $v_1=v_1^*=v^*>0$, and $s_1=s_1^*$. The desired configuration of the follower vehicle $i$, $i\geq2$ is on the reference path (i.e., $\tilde y_i=0$ and $\tilde{\theta}_i=0)$ while following its platoon predecessor $i-1$ with desired velocity $v_i^*=v_{i-1}^*>0$ and desired relative arc length $e_i^* := s^*_{i-1} - s^*_{i}>L_i+\epsilon>0$, where $\epsilon>0$ is a predefined safe margin.
\end{assumption}

% In addition to the desired platoon formation, we impose the following assumption on the road environment and the corresponding lateral and longitudinal safety constraints to avoid collisions with neighboring vehicles and road edges.
Besides achieving the desired platoon formation, the group of vehicles has to follow the traffic rules and, hence, should not cross the road boundaries. The assumption below is to describe the road environment and its relationship with the reference path. 
\begin{assumption}\label{ass.road_shape}
     The reference path is parameterized by a smooth arc length $s$ such that the curvature of the path $\chi^r(s)$ is continuous and differentiable with respect to $s$. Without loss of generality, the platoon's reference path coincides with the road's center line. As shown in Fig. \ref{fig.safety}, The left and right road edges are parallel to the road center line with a constant lateral width offset distance $w$ such that $\frac{1}{\chi^r}>w>\epsilon_w>0$ for all $s$, where $\epsilon_w>0$ is a predefined lateral safe margin. In addition, for any two distinct points $\mathbf{p}_i$ and $\mathbf{p}_j$ projecting to the reference path with arc length $s_i$ and $s_j$,  $|s_i-s_j|>  \epsilon>0$ implies $||\mathbf{p}_i - \mathbf{p}_j||_2 >  \epsilon>0$
     % \lt{the left and right road edges are parallel to the road center line with a constant offset width $w$ such that $0<w<\frac{1}{|\chi^r(s)|}$ for all $s$. Given a lateral safety margin $\epsilon_w>0$, to avoid crossing the road edges, we require $|\tilde{y}_i|<w-\epsilon_w<\frac{1}{|\chi^r(s)|}$. Regarding collision avoidance with neighboring vehicles, for any two distinct points $p_i$ and $p_j$ projecting to the reference path with arc length $s_i$ and $s_j$, the shape of the reference path is restricted so that longitudinal safety $|s_i-s_j|>  \epsilon>0$ implies global safety $||p_i - p_j||_2 >  \epsilon>0.$}
\end{assumption}


The following last assumption assigns the order of the vehicles' index according to their initial arc length projecting on the reference path. This condition will be used later for collision avoidance 
 control design between neighboring vehicles.
\begin{assumption}\label{ass.vehicle_index}
    Provided $s_i(0)-s_j(0) \ne 0, \forall i,j\in \mathcal{V} = {1,2,...,n}$, the vertex set of $n$-vehicle system is assigned such that $s_{i-1}(0)-s_i(0)>0,\forall i\geq 2$.
\end{assumption}

Given the above ingredients, the safe platoon formation tracking control problem is formally formulated as follows: 
\begin{prob}\label{prob.1}
    Under Assumptions \ref{ass:topology} - \ref{ass.vehicle_index}, design lateral and longitudinal controller $\chi_i$ and $a_i^r$, respectively for follower vehicles $i\ge 2$, under which the multi-vehicle system achieves the desired platoon formation described in Assumption \ref{ass.platoon_form} while guaranteeing collision-free to road edges and between neighboring vehicles for all the time %as specified in Assumption \ref{ass.road_shape} for all the time. 
\end{prob}

\begin{figure}[t]
	\centering	\centerline{\includegraphics[trim={10cm 7cm 10cm 9cm},clip,width=1\linewidth]{figure/graph.pdf}}
	%\vspace{-0.3cm}
	\caption{Interaction topology for a 4-agent platoon formation scenario. The arrow indicates the information access for each agent $i$ from its neighboring agent $i-1$.}
	\label{fig:topology}
\end{figure}

 \begin{figure}[t]
	\centering	\centerline{\includegraphics[trim={7cm 3cm 5.5cm 4cm},clip,width=1\linewidth]{figure/frenet.pdf}}
	%\vspace{-0.3cm}
	\caption{Kinematic bicycle model for 2-dimensional vehicle motion in both global and Frenet frame. The rectangle of the solid blue line indicates the actual vehicle $i$ and the rectangle of the dashed blue line represents the corresponding virtual vehicle on the path. }
	\label{fig.frame}
\end{figure}

\begin{figure}[t]
	\centering	\centerline{\includegraphics[trim={7cm 3cm 5cm 4cm},clip,width=1\linewidth]{figure/formation.pdf}}
	%\vspace{-0.3cm}
	\caption{Road layout to illustrate the desired platoon formation.}
	\label{fig.formation}
\end{figure}

\section{Safe platooning and merging controller design}\label{sec.control}
\subsection{Lateral controller design}
% Recall Section \ref{sec.problem}, the objective of the lateral control design is to ensure the follower vehicle a geometrical convergence towards the reference path while guaranteeing collision-free with the road edges. 
From the previous problem formulation in Section \ref{sec.problem}, the lateral control design aims to ensure the follower vehicle a geometrical convergence towards the reference path while guaranteeing collision-free with the road edges. 
Inspired by the work \cite{Zhiqi2023Constructive}, the safe lateral controller for each follower vehicle $i\ge 2$ is proposed as two parts:
\begin{equation}\label{eq.control_lat}
\begin{aligned}
    \chi_i =& \chi_i^n+\chi_i^c
\end{aligned}
\end{equation}
where $\chi_i^n$ is the nominal controller that ensures the stabilization of the origin of error variable $(\tilde y_i,\tilde \theta_i)$ and $\chi_i^c$ is the constructive barrier feedback to be designed for avoiding collision with the road edges.

Hence, the nominal lateral controller for the follower vehicle $i\geq2$ is designed as 
\begin{equation}\label{eq.control_lat_n}
\begin{aligned}
    \chi_i^n = -k_1\frac{\sin\Tilde{\theta}_i}{\Tilde{\theta}_i}\Tilde{y}_i - k_2\sign(v_i)\Tilde{\theta}_i
    +  \frac{\chi_i^r\cos\Tilde{\theta}_i}{1-\chi_i^r\Tilde{y}_i}    
\end{aligned}
\end{equation}
where $k_1$ and $k_2$ are positive gains, the first two terms are designed to correct lateral error $\tilde{y}_i$ and orientation error $\tilde{\theta}_i$ respectively, and the last is the feedforward term used to track the reference curvature $\chi_i^r$. 

To prevent vehicles from crossing the road edges, we define the lateral safety distance to the left and right road edges, respectively, as
% \begin{equation}\label{eq.deta}
% d^{\eta}_i=\left\{
% \begin{aligned}
% &w-\epsilon_w, \ &|\tilde y_i|\le\epsilon_1\\
% &\phi(|\tilde y_i|), \  &\epsilon_1<|\tilde y|<\epsilon_2\\
% &w-|\Tilde{y}_i|-\epsilon_w, \ &|\tilde y_i|\ge\epsilon_2
% \end{aligned}
% \right.
% \end{equation}
\begin{equation}\label{eq.deta}
    d^{\eta_L}_i = w-\Tilde{y}_i-\epsilon_w, \ d^{\eta_R}_i = w+\Tilde{y}_i-\epsilon_w
\end{equation}
% \begin{equation}\label{eq.deta}
%     d^{\eta}_i = w-\sign(\tilde{y}_i)\Tilde{y}_i-\epsilon
% \end{equation}
where $w$ is the distance to the road edge from the reference path and $\epsilon_w>0$ is a predefined safe margin as seen in Fig. \ref{fig.safety}. 

It is straightforward to verify that $ d^{\eta_L}_i> 0$ and  $ d^{\eta_R}_i> 0$ implies that i) vehicle $i$ stays within the left and right road edges with a predefined safety margin $\epsilon_w>0$, ii)$|\Tilde{y}_i|< w-\epsilon_w < \frac{1}{|\chi_i^r|}$, based on Assumption \ref{ass.road_shape}, which ensures that $v_i^r$ in \eqref{eq.vr} is well defined.

Analogous to the divergent flow used in \eqref{eq:cbf}, the lateral constructive barrier feedback $\chi_i^c$ is designed as: 
\begin{equation}
\begin{aligned}
    \chi_i^c =- k_3 \left (\frac{{d^{\eta_R}_i}^{'}}{d^{\eta_R}_i}-\frac{{d^{\eta_L}_i}^{'}}{d^{\eta_L}_i}\right)
\end{aligned}
\end{equation}
% \begin{equation}\label{eq.control_lat_o}
% \begin{aligned}
%     \chi_i^c =- k_3 \gamma(|\tilde{y}_i|)\sign(v_i)\sin\Tilde{\theta}_i. 
% \end{aligned}
% \end{equation}}
% The function $\gamma (.)$ is chosen as 
% \begin{equation}
% \gamma(|\tilde{y}_i|)=\left\{
% \begin{aligned}
% &0, \ &|\tilde y_i|\le\epsilon_1\\
% &\phi(|\tilde y_i|), \  &\epsilon_1<|\tilde y_i|<\epsilon_2\\
% &\frac 1 {d_i^\eta(|\tilde y_i|)}, \ &|\tilde y_i|\ge\epsilon_2
% \end{aligned}
% \right.
% \end{equation}
% where $0<\epsilon_1<\epsilon_2<w-\epsilon_w$, $\phi>0$ is a bounded and smooth enough function chosen such that $\gamma$ is a continuously differentiable function for all $|y_i|$. Note that $\chi_i^c$ can be represented as 
% \begin{equation}
%     \chi_i^c= \left\{
% \begin{aligned}
% &0, \ &|\tilde y_i|\le\epsilon_1\\
% &- k_3\phi \sign(v_i)\sin\Tilde{\theta}_i, \  &\epsilon_1<|\tilde y_i|<\epsilon_2\\
% & k_3 \sign(\tilde{y}_i) \frac {{d^{\eta}_i}^{'}} {d_i^\eta}, \ &|\tilde y_i|\ge\epsilon_2
% \end{aligned}
% \right.
% \end{equation}
where the notation $(.)^{'}$ denotes derivative with respect to the curvilinear abscissa $\sigma =\int_0^t |v_i(\tau)| d\tau$, i.e., ${d_i^{\eta_L}}^{'} =\frac{d}{d\sigma} d^{\eta_L}_i=-\sign(v_i)\sin\Tilde{\theta}_i$ and ${d_i^{\eta_R}}^{'} =\sign(v_i)\sin\Tilde{\theta}_i$. Hence $\chi_i^c$ can be represented as
\begin{equation}\label{eq.control_lat_o}
\begin{aligned}
    \chi_i^c =-k_3\left(\frac 1 {d^{\eta_L}_i}+\frac 1 {d^{\eta_R}_i}\right)\sign(v_i)\sin\Tilde{\theta}_i
\end{aligned}
\end{equation}
% ${d_i^{\eta}}^{'} : = \frac{d}{d\sigma}d^\perp_i =\frac{1}{v_i}\dot{d_i^{\eta}} =-\sign(\tilde{y}_i)\sin\Tilde{\theta}_i$.
Instead of using the time derivative of the distance as in \cite{Zhiqi2023Constructive}, the innovation here is to use the derivative of the distance with respect to the new variable $\sigma$ to avoid longitudinal translational velocity $v_i$ appearing in the divergent flow, decoupling the lateral and longitudinal control design.


 
\
% For $ \frac{{d^{\eta}_i}^{'}}{d^{\eta}_i}$, Recall definition \eqref{eq.deta} of $d_i^{\eta}$, we obtain its time derivative as $\dot{d}_i^{\eta} = \alpha_i\dot{\Tilde{y}}_i$. Through a change of variable $d\sigma = v_i dt$, ${d^{\eta}_i}^{'}$ is obtained according to ${d^{\eta}_i}^{'} = \frac{d( d^\eta_i)}{d\sigma} =  \frac{d( d^\eta_i)}{v_idt} = \frac{\alpha_i\dot{\Tilde{y}}_i}{v_i} =\alpha_i \sin\Tilde{\theta}_i$ .





%{\color{blue}Alternative: 
%\begin{equation}
%    \chi = -k_1\Tilde{y} - k_2\sign(v)\sin\Tilde{\theta}-k_3 \sign(v)\frac{d_{\Tilde{y}}^{'}}{d_{\Tilde{y}}} + \frac{\chi(s)v\cos\Tilde{\theta}}{1-\chi(s)\Tilde{y}}
%\end{equation}

%$k_3 \sign(v)$ positive and bounded is not required for the proof of $d_{\Tilde{y}}(t) \geq 0$? 
%}


\subsection{Longitudinal controller design}
\begin{figure}[t]
	\centering	\centerline{\includegraphics[trim={6cm 4cm 4.5cm 3cm},clip,width=1\linewidth]{figure/safedist.pdf}}
	%\vspace{-0.3cm}
 	\caption{The lateral safety distance $d_i^{\eta_L}$ of vehicle $i$ with respect to road edges and the longitudinal safety distance $d_i^\rho$ of vehicle $i$ with respect to its predecessor $i-1$. }
	\label{fig.safety}
\end{figure}

For the longitudinal control, the design is initiated over the virtual vehicle. The virtual acceleration control input $a_i^r$ in \eqref{eq.dyna_frenet_lon_ref} is designed as two parts similar to the lateral controller \eqref{eq.control_lat}
\begin{equation}\label{eq.lon_control_vir}
    a_i^r = a_i^{n}+a_i^{c}.
\end{equation}
The nominal controller $a_i^{n}$ is responsible for driving the virtual vehicle’s longitudinal state $(s_i,v_i^r)$ toward the desired platoon configuration $(s_i^*,v_i^*)$. It is defined as
\begin{equation}\label{eq.lon_control_vir_n}
    a_i^{n} = k_4\Tilde{e}_i+k_5\nu_i+a_{i-1}^r
\end{equation}
where $\Tilde{e}_i: = e_i - e_i^*$ represents the relative longitudinal position error, with $e_i = s_{i-1}-s_{i}$ denoting the difference in arc length between virtual agent $i$ and $i-1$. The desired inter-vehicle spacing is given by $e_i^*$. The term $\nu_i := (v^r_{i-1} - v^r_i)$ is the relative velocity error, and $k_4$ and $k_5$ are positive gains. 

To prevent collision between vehicle $i$ and $i-1$, the longitudinal safety distance $d_i^{\rho}$ between virtual vehicles $i$ and $i-1$ is introduced as 
\begin{equation}
    d_i^{\rho}: = e_i -\epsilon
\end{equation}
where $\epsilon>0$ is the safety margin. See Fig. \ref{fig.safety} for a graphic view of the safety distance. Recall Assumption \ref{ass.road_shape}, one concludes that collision-free between actual vehicles $i$ and $i-1$ is guaranteed, i.e., $||\mathbf{p}_{i-1}-\mathbf{p}_{i}||_2  >\epsilon> 0$, as long as the collision-free between virtual vehicles holds, i.e., $ d_i^{\rho}>0$.

Finally, the longitudinal constructive barrier feedback for collision avoidance between neighboring virtual vehicles is designed as 
\begin{equation}\label{eq.lon_control_vir_o}
    a_i^{c} = k_6\frac{\dot d_i^{\rho}}{ d_i^{\rho}}
\end{equation}
where $\dot{ d_i^{\rho}} = \nu_i$, and $k_6$ is a positive control gain.




\section{Stability analysis}\label{sec.stability}
% Under the proposed control design, the following lemma provides a theoretical analysis for safety invariance and the performance of the longitudinal controller applied for the virtual vehicles.
Under the proposed control design, the following lemma shows safety guarantees and equilibrium analysis of the longitudinal controller.
\begin{lem}\label{lem.boundedv}
Consider a $n$-vehicle system with the system dynamic \eqref{eq.kinematic} and the virtual longitudinal control \eqref{eq.lon_control_vir}, \eqref{eq.lon_control_vir_n} and \eqref{eq.lon_control_vir_o}. If Assumption \ref{ass:topology} - \ref{ass.vehicle_index} are satisfied, then for any bounded initial conditions such that $d_i^{\rho}(0)>0$ and $a_i^c(0)$ bounded, the following assertions hold $\forall i\in\mathcal{V}/\{1\}$, $\forall t \geq 0$:
    \begin{enumerate}
        \item vehicle $i$ remains collision-free with its neighboring vehicle $i-1$, i.e., $d_i^{\rho}(t)$ remains positive, and the controller $a_i^r$ is well-defined  and  bounded;
        \item the origin of $(\tilde e_i,\nu_i)$ is asymptotically stable;
        % \item $v_i^r$ remains positive and bounded as long as $\sum_{j=2}^i 2^{i-j+1}\bigl(k_4\Tilde{e}_j^2(0)+\nu_j^2(0)\bigr)< {v^*}^2$.
    \end{enumerate}
\end{lem}
The proof of this lemma is provided in Appendix \ref{app.1}. 

Under both lateral and longitudinal controllers, the multi-vehicle systems' safety invariant properties and stability analysis are provided by the following theorem.
\begin{thm}\label{thmNvehicle}
Consider a $n$-vehicle system with the dynamics \eqref{eq.kinematic} along with the longitudinal controller defined by \eqref {eq.ar2a}, \eqref{eq.lon_control_vir}, \eqref{eq.lon_control_vir_n} and \eqref{eq.lon_control_vir_o} and lateral controller  \eqref{eq.control_lat}, \eqref{eq.control_lat_n} and \eqref{eq.control_lat_o}. If Assumption \ref{ass:topology} - \ref{ass.vehicle_index} are satisfied, then under any bounded initial condition satisfying,  $d_i^{\rho}(0) > 0$, $d^{\eta}_i(0) > 0$, $k_1\tilde y_i^2(0)+\Tilde{\theta}_i^2(0)<(\frac{\pi}{2}-\epsilon_1)^2$, (with $\epsilon_1<<1$ a positive number), $a_i^c(0)$ and $\chi_i^c(0)$ bounded, 
% \begin{equation}\label{eq.initial-v}
% \sum_{j=2}^i 2^{i-j+1}\bigl(k_4\Tilde{e}_j^2(0)+\nu_j^2(0)\bigr)< {v^*}^2\end{equation}
the following assertions hold $\forall i\in\mathcal{V}/\{1\}$, $\forall t \geq 0$:
\begin{enumerate}
    \item $|\tilde \theta_i|<\frac \pi 2$ and $|\tilde y_i|<\frac 1 {\chi_i^r}$, and the $n$-vehicle system remains safe, i.e., $d_i^{\rho}(t)$, $d^{\eta_L}_i(t)$ and $d^{\eta_R}_i(t)$ remain positive;
    \item the control law $\chi_i$ and $a_i$ are well-defined and bounded;
    \item the origins of $(\Tilde{y}_i,\Tilde{\theta}_i) $ and  $(\tilde s_i,\tilde v_i):=(s_i-s_i^*, v_i-v_i^*)$ are asymptotically stable;
    
\end{enumerate}
\end{thm}
The proof of this theorem is provided in Appendix \ref{app.2}.
% \begin{rem}
% Note that the initial condition \eqref{eq.initial-v} is a conservative condition for the proof, mainly devoted to ensuring the positiveness of $v_i$. Simulation results in the later section show that the positiveness of $v_i$ can still be guaranteed with much relaxed initial conditions. %A large initial domain for the formal proof will be exploited in future work.
% \end{rem}

\section{Simulation results}\label{sec.result}
To demonstrate the performance of the proposed control method, we design a multilane highway scenario where the shape of the road is generated using the Matlab built-in function $referencePathfrenet()$ with fixed waypoints. In particular, a performance comparison between the proposed safe platoon controller \eqref{eq.control_lat}, \eqref{eq.control_lat_n}, \eqref{eq.control_lat_o}, \eqref{eq.lon_control_vir}, \eqref{eq.lon_control_vir_n}, and \eqref{eq.lon_control_vir_o}. and the nominal platoon controller \eqref{eq.control_lat_n} and \eqref{eq.lon_control_vir_n} (which serves as a baseline control strategy) is provided to emphasize that collision avoidance is ensured without affecting the nominal control objectives. The simulation study is performed for a platoon of five vehicles with two different initial settings, constituting two test scenarios referred to as scenario (A) and scenario (B). The initial conditions are summarized in Table. \ref{tb:merge} and \ref{tb:form} and is visualized in Fig. \ref{fig:traj_res}. %Observe that the initial states of vehicles in scenario (A) satisfy condition \eqref{eq.initial-v} while the initial states in (B) do not.

In both scenarios, the controller gains are set as $k_1 = 0.01$, $k_2 = 0.1$, $k_3 = 0.1$, $k_4 = 0.4$, $k_5 = 0.1$, $k_6 = 2$, and the constant parameters are set as $w = \SI{10}{\metre}$, $\epsilon_w = \SI{1.2}{\metre}$, $\epsilon = \SI{5}{\metre}$, $L_i = \SI{4}{\metre}$, $e^*_i = \SI{14}{\metre}$. $v^*_i = \SI{10}{\metre\per\second}$.

In scenario (A), the reference path is placed on the road centerline as in Fig. \ref{fig:traj_merg}, vehicle states are initiated such that vehicles $1$, $3$, and $5$ are already on the path, while vehicles $2$ and $3$ are placed to merge into the platoon from the left and right side of the path respectively. 

In scenario (B), the reference path is placed parallel to the road centerline with a 2 m offset distance from the left road boundary as in Fig. \ref{fig:traj_form}. Here, vehicles are scattered over the road initially. This setup allows us to evaluate the controller's performance in steering vehicles from different lanes into the desired platoon formation while demonstrating the capability of the controller to keep vehicles on the road and avoid collisions with each other. 

The formation evolution under the proposed control method is shown in Fig. \ref{fig:traj_res}, for both scenarios, all vehicles smoothly converge to the desired formation by tracking the reference path and keeping the desired distance with their platoon predecessor. The convergence is more clearly evident in Fig. \ref{fig:error_res}, where the time evolution of the error states $\Tilde{y}_i$, $\Tilde{\theta}_i$, $\Tilde{v}_i$, and $\Tilde{e}_i$ all converge to zero after approximately 20 seconds for both the proposed controller and the baseline. The simulation results indicate that the convergence property of the baseline controller is preserved by the proposed controller. 

The safety can be measured by $d_i^\rho$ and $d_i^\eta$ as $d_i^\rho\leq 0$ and $d_i^\eta\leq 0 $ indicate vehicle-to-vehicle and vehicle-to-road boundary collision (for simplicity of notation here we denote $d_i^\eta = \min(d_i^{\eta_L}, d_i^{\eta_R})$ for measure of the safety to both sides). As shown in Fig \ref{fig:dist_res} with solid lines, the multi-vehicle system successfully avoids collisions under the proposed controller. In contrast, the multi-vehicle system under the baseline control collides with either neighboring vehicles or the road edges. As shown in Fig. \ref{fig:dist_merg} with dashed lines, for scenario (A), the baseline controller failed to keep $d_4^\rho >0$, where a collision happened between vehicle 4 and vehicle 3. For scenario (B) as shown in Fig. \ref{fig:dist_form}, the baseline failed to keep $d_2^\rho>0$, $d_2^\eta>0$, $d_4^\rho>0$, $d_4^\eta>0$, and $d_5^\eta>0$, indicating that vehicles 2, 4, and 5 crossed the road boundary. In addition, vehicle 2 collided with vehicle 1 and vehicle 4 collided with vehicle 3 in scenario (B). 

%\lt{The evolution of vehicle input acceleration and steering angle under both the proposed controller and the baseline controller are shown in Fig. \ref{fig:input_res}, which indicates that control inputs $a_i$ and $\delta_i$ are smooth and vary within reasonable bounds. Note that Lemma \ref{lem.boundedv} and Theorem \ref{thmNvehicle} provide a theoretical safety guarantee with theoretical boundedness on the feedback control signals. In practical applications where the physical limitation on the input signal exists, saturation is required to respect the physical control bound, this will practically affect the initial safe set under which our proposed controller can still guarantee safety and such an initial safe set depends clearly on the choice of the controller gains. Since our proposed controller is in explicit feedback form, a numerical approximation of the initial safe set can be computed through Hamilton Jacobi reachability analysis. It is a formal method that solves a zero-sum game between two players through dynamic programming, in this case between a vehicle $i$ and its platoon predecessor $i-1$. The resulting zero-superlevel set of the value function describes the maximal control invariant set respecting the control input bounds, within which following the proposed controller, vehicle $i$ is guaranteed to stay safe given the worst-case behavior of its predecessor $i-1$. For simplicity, we show the resulting invariant set for the longitudinal case. The acceleration bound is assumed to be $[-6, 3] \si{\metre\per\second^2}$ which is also used in the simulation studies. In addition, to prevent velocity from going below zero, we set $a_i = 0$ if $v_i = 0$. Slices of the numerical safe invariant set is shown in Fig. \ref{fig.BRS}. The curves show the zero-level set of the value function in terms of relative distance $e_i$ and velocity $v_i$ of the follower vehicle $i$ for given velocity $v_{i-1}$ of the leader $i-1$. The area to the right of the curves is the initial safe set.}  \TODO{This paragraph is not clear}

% The evolution of vehicle input acceleration and steering angle under both the proposed controller and the baseline controller are shown in Fig. \ref{fig:input_res}, which indicates that control inputs $a_i$ and $\delta_i$ are smooth and vary within reasonable bounds. \tang{Note that
% Theorem \ref{thmNvehicle} shows the proposed controller is bounded and guarantees safety invariance.} In practical applications where specific physical limitations on the input signal exist, the feedback control needs to be saturated outside the physical bounds, this will practically affect the safe set of initial states under which our proposed controller can still guarantee safety. Such a safe set depends clearly on the choice of the controller gains. Since our proposed controller is in explicit feedback form, a numerical approximation of the safe set of initial states can be computed through Hamilton Jacobi reachability analysis \TODO{add ref}. It is a formal method that solves a zero-sum game between two players through dynamic programming, in this case between a vehicle $i$ and its platoon predecessor $i-1$. The resulting zero-superlevel set of the value function describes the maximal control invariant set respecting the control input bounds, within which following the proposed controller, vehicle $i$ is guaranteed to stay safe given the worst-case behavior of its predecessor $i-1$. For simplicity, we show the resulting invariant set for the longitudinal virtual vehicle \eqref{eq.dyna_frenet_lon_ref} following the proposed control \eqref{eq.lon_control_vir},\eqref{eq.lon_control_vir_n} and \eqref{eq.lon_control_vir_o} with the same gains as in the simulation studies. The acceleration bound is assumed to be $[-6, 3]$ \si{\metre\per\square\second}. In addition, to prevent velocity from going below zero, we set $a^r_i = 0$ if $v^r_i = 0$. Slices of the numerical safe invariant set are shown in Fig. \ref{fig.BRS}. The color-filled areas show the safe initial states for vehicle $i$ in terms of relative distance $e_i$ and velocity $v^r_i$ for different velocity $v^r_{i-1}$ of the leader $i-1$. This safe set shows the initial states from which the proposed controller provides safety under practical physical bounds on input signals. 

An animation of the simulation results can be found at \url{https://bit.ly/platoon_formation_curved_road}., where the performance between the proposed controller and the baseline is compared and visualized. In summary, from the simulation comparisons, one concludes that the proposed safe controller efficiently achieves the nominal formation tracking objective while avoiding collision between neighboring vehicles and road edges with reasonable control inputs.

\begin{table}[t]
\renewcommand{\arraystretch}{1.3}
\begin{center}
\caption{Initial vehicle states for scenario (A)}\label{tb:merge}
\begin{tabular}{cccccc}
\hline \hline
Vehicle & $s_{i}$ & $\Tilde{y}_{i}$ & $\Tilde{\theta}_i$ & $v_{i}$ \\
 & [$\si{\metre}$] & [$\si{\metre}$] & [$\si{\radian}$] & [$\si{\metre\per\second}$]\\
\hline
1 & 50 & 0 & 0 & 10\\
\hline
2 & 42 & 4 & 0 & 13\\
\hline
3 & 36 & 0 & 0 & 10\\
\hline
4 & 28 & -4 & 0 & 16\\
\hline
5 & 22 & 0 & 0 & 10\\
\hline
\end{tabular}
\end{center}
\end{table}

\begin{table}[t]
\renewcommand{\arraystretch}{1.3}
\begin{center}
\caption{Initial vehicle states for scenario (B)}\label{tb:form}
\begin{tabular}{cccccc}
\hline \hline
Vehicle & $s_{i}$ & $\Tilde{y}_{i}$ & $\Tilde{\theta}_i$ & $v_{i}$ \\
& [$\si{\metre}$] & [$\si{\metre}$] & [$\si{\radian}$] & [$\si{\metre\per\second}$]\\
\hline
1 & 50 & 0 & 0 & 10\\
\hline
2 & 40 & -10 & 0 & 12\\
\hline
3 & 29 & -2.5 & 0 & 10\\
\hline
4 & 22 & -12 & 0 & 12\\
\hline
5 & 12 & -5 & 0 & 10\\
\hline
\end{tabular}
\end{center}
\end{table}

\begin{figure*}[t]
     \centering
     \begin{subfigure}[b]{0.48\textwidth}
        \centering
            \includegraphics[trim=5.5cm 7.5cm 5.5cm 7.5cm,clip,width=\linewidth]{figure/center_timelap.pdf}
        \caption{Scenario (A).}
        \label{fig:traj_merg}
     \end{subfigure}
     \begin{subfigure}[b]{0.48\textwidth}
        \centering
            \includegraphics[trim=5.5cm 7.5cm 5.5cm 7.5cm,clip,width=\linewidth]{figure/left_timelap.pdf}
        \caption{Scenario (B).}
        \label{fig:traj_form}
     \end{subfigure}
    \caption{Time evolution snapshot of the platoon formation process at distinct time points. Two parallel red solid lines are the road edges, the black solid line indicates the desired lane for the platoon, and the color solid lines indicate the vehicle trajectory during the formation process. The left sub-figure shows the result for the merging scenario and the right sub-figure shows the result for the formation scenario.}
        \label{fig:traj_res}
\end{figure*}

\begin{figure*}[t]
     \centering
     \begin{subfigure}[b]{0.48\textwidth}
        \centering
            \includegraphics[trim=4cm 4cm 4cm 3cm,clip,width=\linewidth]{figure/center_conve.pdf}
        \caption{Scenario (A)}
        \label{fig:error_merg}
     \end{subfigure}
     \begin{subfigure}[b]{0.48\textwidth}
        \centering
            \includegraphics[trim=4cm 4cm 4cm 3cm,clip,width=\linewidth]{figure/left_conve.pdf}
        \caption{Scenario (B)}
        \label{fig:error_form}
     \end{subfigure}
    \caption{Time evolution of the lateral displacement error $\tilde{y}_i(t)$, the angular orientation error $\tilde{\theta}_i(t)$, the velocity error $\tilde{v}_i(t)$, and the longitudinal displacement error $\tilde{e}_i(t)$ for all $i = \{2,3,4,5\}$ with respective color-coded lines. The solid lines indicate results under the proposed controller \eqref{eq.control_lat}, \eqref{eq.control_lat_n}, \eqref{eq.control_lat_o}, \eqref{eq.lon_control_vir}, \eqref{eq.lon_control_vir_n}, and \eqref{eq.lon_control_vir_o}. The transparent dashed lines indicate the result under the baseline controller \eqref{eq.control_lat_n} and \eqref{eq.lon_control_vir_n}.}
        \label{fig:error_res}
\end{figure*}

\begin{figure*}[t]
     \centering
     \begin{subfigure}[b]{0.48\textwidth}
        \centering
            \includegraphics[trim=4cm 4cm 4cm 3cm,clip,width=\linewidth]{figure/center_dist_error.pdf}
        \caption{Scenario (A)}
        \label{fig:dist_merg}
     \end{subfigure}
     \begin{subfigure}[b]{0.48\textwidth}
        \centering
            \includegraphics[trim=4cm 4cm 4cm 3cm,clip,width=\linewidth]{figure/left_dist_error.pdf}
        \caption{Scenario (B)}
        \label{fig:dist_form}
     \end{subfigure}
    \caption{Time evolution of the safety distance $d_i^\rho(t)$ and $d_{i}^\eta(t)$ for all $i\in\{2,3,4,5\}$. The solid lines indicate the evolution of $d_i^g(t)$ and $d_{i}^\eta(t)$ under the proposed controller \eqref{eq.control_lat}, \eqref{eq.control_lat_n}, \eqref{eq.control_lat_o}, \eqref{eq.lon_control_vir}, \eqref{eq.lon_control_vir_n}, and \eqref{eq.lon_control_vir_o}. The dashed lines indicate the result under the baseline controller \eqref{eq.control_lat_n} and \eqref{eq.lon_control_vir_n}.}
        \label{fig:dist_res}
\end{figure*}

\begin{figure*}[t]
     \centering
     \begin{subfigure}[b]{0.48\textwidth}
        \centering
            \includegraphics[trim=4cm 8cm 4cm 8cm,clip,width=\linewidth]{figure/merg_input.pdf}
        \caption{Scenario (A)}
        \label{fig:merg_input}
     \end{subfigure}
     \begin{subfigure}[b]{0.48\textwidth}
        \centering
            \includegraphics[trim=4cm 8cm 4cm 8cm,clip,width=\linewidth]{figure/form_input.pdf}
        \caption{Scenario (B)}
        \label{fig:form_input}
     \end{subfigure}
    \caption{The vehicle input acceleration $a_i(t)$ and steering angle $\delta_i(t)$ for all $i = \{2,3,4,5\}$. The solid lines indicate results under the proposed controller \eqref{eq.control_lat}, \eqref{eq.control_lat_n}, \eqref{eq.control_lat_o}, \eqref{eq.lon_control_vir}, \eqref{eq.lon_control_vir_n}, and \eqref{eq.lon_control_vir_o}. The transparent dashed lines indicate the result under the baseline controller \eqref{eq.control_lat_n} and \eqref{eq.lon_control_vir_n}.}
        \label{fig:input_res}
\end{figure*}

% \begin{figure}[t]
% 	\centering	\centerline{\includegraphics[trim={4cm 8.5cm 4cm 8.5cm},clip,width=1\linewidth]{figure/BRS.pdf}}
% 	%\vspace{-0.3cm}
% 	\caption{The numerically computed initial safe set in the longitudinal direction for vehicle $i$ under the proposed controller \eqref{eq.lon_control_vir},\eqref{eq.lon_control_vir_n} and \eqref{eq.lon_control_vir_o} and given acceleration bound $[-6, 3]$ \si{\metre\per\square\second}.}
% 	\label{fig.BRS}
% \end{figure}


\section{Experimental Results}\label{sec.experiment}
This section presents the experimental validation of the proposed methods using connected vehicles. The control algorithms are designed for full-scale autonomous cars with acceleration inputs. However, due to limitations of the available testbed, the experiments were conducted indoors on a fleet of scale-model SVEA vehicles—1:10 scale miniature platforms with velocity control inputs. While the SVEA platforms differ from full-scale cars, they are sufficient to demonstrate the effectiveness of the proposed methods as a proof of concept.

Each SVEA is equipped with a Zed Box powered by an NVIDIA Jetson-embedded computer where the proposed controllers run onboard. Qualisys motion capture system (MOCAP) is used to provide accurate real-time state estimation $(\bold p_i, v_i, \theta_i)$ to each vehicle through communication. Each vehicle $i$ shares its state information $(\bold p_i, v_i)$ to its neighbor vehicle $i-1$ according to the communication topology defined in Assumption \ref{fig:topology}. Both the MOCAP-to-vehicle and vehicle-to-vehicle communication are enabled through a local WiFi network using the ROS framework and a tailored communication protocol. The control algorithms $\chi_i$ and $a_i$ are implemented in Python onboard each SVEA with an update frequency fixed at 10 Hz. Since the DC motor on the SVEA platform receives pulse-width modulation (PWM) signals from the electronic speed controller (ESC) to control its shaft speed, the input acceleration $a_i$ is integrated and processed to generate the corresponding PWM signal for accurate speed tracking. The experimental study's overall control and communication stack is shown in Fig. \ref{fig:exp_control_layout}.

% To evaluate the proposed safe platoon formation control approach in a practical setting, an experiment is designed using three 1:10 scale miniature vehicle platforms SVEA developed at KTH \ref{}. Each SVEA is equipped with a Zed Box powered by an NVIDIA Jetson-embedded computer. The experiment is designed in an indoor lab environment with a Qualisys motion capture system (MOCAP) to provide accurate real-time state estimation to each vehicle. \lt{The MOCAP is chosen for its reliability. It can be replaced given reliable local sensing device for relative distance and velocity measurement is available. The global state estimation from MOCAP is mapped locally and transformed into the relative state onboard each SVEA to be communicated. } The relative vehicle state and control are shared within the platoon according to the communication graph as in Assumption \ref{fig:topology}. Both the MOCAP-to-vehicle and vehicle-to-vehicle communication are enabled through a local WiFi network using the ROS framework and a tailored communication protocol. The control is implemented in Python onboard each SVEA with an update frequency fixed at 10 Hz. \lt{Unlike a real-world car taking acceleration input for which our controller is designed.} The DC motor on the SVEA platform receives pulse-width modulation (PWM) from the electrical speed controller (ESC) which commands its shaft speed. As a result, the input acceleration from the controller is integrated and fed through a simple PID to generate the corresponding PWM signal for speed tracking. The overall control and communication stack for the experimental study is shown in Fig. \ref{fig:exp_control_layout}.
 \begin{figure}[t]
	\centering	\centerline{\includegraphics[trim={4cm 5cm 6cm 3cm},clip,width=1\linewidth]{figure/SVEA.pdf}}
	%\vspace{-0.3cm}
	\caption{Control and communication layout of the experimental studies.}
	\label{fig:exp_control_layout}
\end{figure}


 \begin{figure}[t]
	\centering	\centerline{\includegraphics[trim={7cm 3cm 6cm 3cm},clip,width=1\linewidth]{figure/exp_scenario.pdf}}
	%\vspace{-0.3cm}
	\caption{The platoon merging scenario of 3 SVEAs for the experimental studies.}
	\label{fig:exp_scenario}
\end{figure}

% The layout of the experiment is shown in Fig. \ref{fig:exp_scenario}. Due to the limitation of lab space, we designed the layout to be a circular-shaped path for continuous driving and testing. As shown in Fig. \ref{fig:exp_scenario}, the red lines indicate the road boundaries, while the blue lines are the reference paths. The reference paths are placed 0.6 \si{\metre} apart from each other and 0.3 \si{\metre} away from the left respective right road edges. The vehicles are configured to resemble a platoon merging scenario. Here, SVEA 1 and SVEA 3 are designed to drive in a platoon on the inner reference path. SVEA 2 is initially placed on the outer reference path with a higher velocity and is designed to merge between the two vehicles given a merging command. The merging command is activated when the longitudinal distance between SVEA 1 and SVEA 2 is less than \SI{0.8}{\cm}. 

% The controller gains are tuned for the experiment scenario and are set as $k_1 = 5$, $k_2 = 2$, $k_3 = 0.5$, $k_4 = 1$, $k_5 = 3$, $k_6 = 0.2$. A virtual platoon leader on the inner reference path is implemented on SVEA 1 with constant velocity $\SI{0.7}{\metre\per\second}$. So $v_1^* = v_3^* = \SI{0.7}{\metre\per\second}$ as SVEA 1 and SVEA 3 are traveling in a platoon initially. $v_2^*=\SI{1.2}{\metre\per\second}$ is set for SVEA 2 initially and is changed to $v_2^*=\SI{0.7}{\metre\per\second}$ after the merging command is activated. In addition, we have $\epsilon = \SI{0.6}{\metre}$, $\epsilon_w = \SI{0.15}{\metre}$, $L_i = \SI{0.324}{\metre}$, $e_i^* = \SI{1.5}{\metre}$ for all vehicles.
The layout of the experiment is shown in Fig. \ref{fig:exp_scenario}. Due to the limitation of lab space, we designed the layout to be a circular-shaped path for continuous driving and testing. As shown in Fig. \ref{fig:exp_scenario}, the red lines indicate the road boundaries, while the blue lines are the reference paths. The reference paths are placed 0.6 \si{\ meter} apart and 0.3 \si{\ meter} away from the left respective right road edges. The vehicles are configured to resemble a platoon merging scenario. Here, SVEA 1 and SVEA 3 are designed to drive in a platoon on the inner reference path following a virtual platoon leader simulated onboard SVEA 1, which travels at constant velocity 0.7 \si{\metre\per\second}. SVEA 2 is initially placed on the outer reference path with a higher velocity and is designed to merge between the two vehicles given a merging command. The merging command is activated when the longitudinal distance between SVEA 1 and SVEA 2 is less than \SI{0.8}{\cm}. 

The controller gains tuned for the experiment scenario are $k_1 = 5$, $k_2 = 2$, $k_3 = 0.5$, $k_4 = 1$, $k_5 = 3$, $k_6 = 0.2$. The desired velocity is $v^*=v_i^*=\SI{0.7}{\metre\per\second}$ and the desired relative arc length $e_i^*=\SI{1.5}{\metre}$. The wheel base for each vehicle is $L_i = \SI{0.324}{\metre}$ and the safe margin is chosen as $\epsilon = \SI{0.6}{\metre}$, $\epsilon_w = \SI{0.15}{\metre}$. The initial velocity of SVEA 2 before merging the platoon is $v_2(0)=\SI{1.2}{\metre\per\second}$.
% The controller gains are tuned for the experiment scenario and are set as $k_1 = 5$, $k_2 = 2$, $k_3 = 0.5$, $k_4 = 1$, $k_5 = 3$, $k_6 = 0.2$. A virtual platoon leader on the inner reference path is implemented on SVEA 1 with constant velocity $\SI{0.7}{\metre\per\second}$. So $v_1^* = v_3^* = \SI{0.7}{\metre\per\second}$ as SVEA 1 and SVEA 3 are traveling in a platoon initially. $v_2^*=\SI{1.2}{\metre\per\second}$ is set for SVEA 2 initially and is changed to $v_2^*=\SI{0.7}{\metre\per\second}$ after the merging command is activated. In addition, we have $\epsilon = \SI{0.6}{\metre}$, $\epsilon_w = \SI{0.15}{\metre}$, $L_i = \SI{0.324}{\metre}$, $e_i^* = \SI{1.5}{\metre}$ for all vehicles.

To analyze the effectiveness of the proposed method, we again compare the proposed safe platoon controller with the nominal controller as the baseline. The experimental results are presented from the time when the merging command is activated. %\TODO{??? Due to the fact that feedback controllers are implemented onboard each SVEA, there are synchronization issues in displaying the logged data on one single graph. Here, we chose to display the result for SVEA 2 as it is of most interest.} 
As shown in Fig. \ref{fig:error_res_exp}, SVEA vehicles' states converge to desired values under both the safe platoon controller and baseline controller.
%both the proposed controller and the baseline were able to bring SVEA 2 into the desired formation as all error states converged around zero shortly after the merging command at around $\SI{18}{\second}$. 
It demonstrates the robustness of the proposed controller under inaccurate actuation of the miniature vehicle. For safety, in Fig. \ref{fig:dist_safe_exp}, the proposed controller guarantees both $d^\rho_2$ and $d^\eta_2$ positive, resulting in a safe merging process. In comparison, the baseline controller failed to keep $d^\eta_2$ above zero as shown in Fig. \ref{fig:dist_nom_exp}, resulting in SVEA 2 driving over the road boundary during merging. For longitudinal safety distance $d^\rho_i$ between vehicles, due to safety, we did not tune the experimental setting to stress test the difference in terms of inter-vehicle safety between the proposed controller and the baseline controller. But the smaller value of $d^\rho_i$ in Fig. \ref{fig:dist_nom_exp} indicates a higher collision risk of the baseline controller. Finally, the input acceleration and steering are shown in Fig. \ref{fig:exp_input}. The proposed safe platoon controller and the baseline generate similar input signals with reasonable magnitude.

The experiments are recorded and the video can be found at \url{https://bit.ly/platoon_formation_experiment}. In summary, we conclude that the proposed safe platoon controller is applicable in practical implementations.

\begin{figure*}[t]
     \centering
     \begin{subfigure}[b]{0.48\textwidth}
        \centering
            \includegraphics[trim=4cm 4.5cm 4cm 4cm,clip,width=\linewidth]{figure/exp_Error.pdf}
        \caption{Experiment with the safe platoon controller}
        \label{fig:error_safe_exp}
     \end{subfigure}
     \begin{subfigure}[b]{0.48\textwidth}
        \centering
            \includegraphics[trim=4cm 4.5cm 4cm 4cm,clip,width=\linewidth]{figure/exp_ErrorN.pdf}
        \caption{Experiment with the baseline controller}
        \label{fig:error_nom_exp}
     \end{subfigure}
    \caption{Time evolution of the lateral position error $\tilde{y}_i(t)$, the angular error $\tilde{\theta}_i(t)$, the velocity error $\tilde{v}_i(t)$, and the relative formation error $\tilde{e}_i(t)$ for SVEA vehicles in the experiment.}
        \label{fig:error_res_exp}
\end{figure*}

\begin{figure*}[t]
     \centering
     \begin{subfigure}[b]{0.48\textwidth}
        \centering
            \includegraphics[trim=4cm 8.5cm 4cm 8.5cm,clip,width=\linewidth]{figure/exp_Safe.pdf}
        \caption{Experiment with the safe platoon controller}
        \label{fig:dist_safe_exp}
     \end{subfigure}
     \begin{subfigure}[b]{0.48\textwidth}
        \centering
            \includegraphics[trim=4cm 8.5cm 4cm 8.5cm,clip,width=\linewidth]{figure/exp_SafeN.pdf}
        \caption{Experiment with the baseline controller}
        \label{fig:dist_nom_exp}
     \end{subfigure}
    \caption{Time evolution of the safety distance $d^{\rho}_i(t)$ and $d^\eta_i(t)$ for SVEA vehicles in the experiment.}
        \label{fig:dist_res_exp}
\end{figure*}

\begin{figure*}[t]
     \centering
     \begin{subfigure}[b]{0.48\textwidth}
        \centering
            \includegraphics[trim=3.5cm 8.5cm 4cm 8cm,clip,width=\linewidth]{figure/exp_Control.pdf}
        \caption{Experiment with the safe platoon controller}
        \label{fig:exp_input_safe}
     \end{subfigure}
     \begin{subfigure}[b]{0.48\textwidth}
        \centering
            \includegraphics[trim=3.5cm 8.5cm 4cm 8cm,clip,width=\linewidth]{figure/exp_ControlN.pdf}
        \caption{Experiment with the baseline controller}
        \label{fig:exp_input_nom}
     \end{subfigure}
    \caption{The vehicle input acceleration $a_i(t)$ and steering angle $\delta_i(t)$ for SVEA vehicles in the experiment. The input acceleration $a_i(t)$ is displayed in separate plots for clarity. }
        \label{fig:exp_input}
\end{figure*}

\section{Conclusion}\label{sec.conclusion}
This study addresses the problem of platoon formation on multi-lane roads with a general curved shape. The controller design is structured into two key components: path-following and longitudinal formation. To ensure safety, a constructive barrier feedback mechanism is introduced for both components, enabling lateral avoidance of road boundaries and longitudinal avoidance of preceding vehicles. Theoretical and numerical analyses are conducted to evaluate the performance and safety properties of the proposed method. Additionally, the method is experimentally validated using miniature vehicles, demonstrating its practical applicability in real-world scenarios.

Future research will focus on enhancing the controller’s capabilities by incorporating on-road obstacle avoidance during formation. Furthermore, a deeper analysis of the method’s robustness against model uncertainties and measurement errors is essential to further strengthen its reliability in practical implementations. 

\appendix
\renewcommand{\thesubsection}{\Alph{subsection}}
\subsection{Proof of Lemma \ref{lem.boundedv}}\label{app.1}
\begin{proof}
Recall \eqref{eq.dyna_frenet_lon_ref}, \eqref{eq.lon_control_vir}, \eqref{eq.lon_control_vir_n}, and \eqref{eq.lon_control_vir_o}, the closed-loop dynamics of the state $(\tilde e_i,\nu_i)$ is expressed as
\begin{equation}\label{eq.ei.nui}
    \left\{
    \begin{aligned}
        &\dot{\tilde e}_i = \nu_i\\
        &\dot{\nu}_i = -k_4\tilde e_i-k_5\nu_i-k_6\frac{\dot d_i^{\rho}}{d_i^{\rho}}
    \end{aligned}
    \right.  
\end{equation}
Consider the following Lyapunov function
    \begin{equation}\label{eq.Llong}
        \mathcal{L}_i = \frac{1}{2}k_4\Tilde{e}_i^2+\frac{1}{2}\nu_i^2
    \end{equation}
Recall \eqref{eq.ei.nui} and use the fact that $\dot{d}_i^{\rho} = \nu_i$, one has
\begin{equation}
    \dot{\mathcal{L}}_i = -k_5\nu_i^2 - k_6\frac{\nu_i^2}{d_i^{\rho}}
\end{equation}
which is negative semi-definite as long as $d_i^{\rho}>0$. Hence one concludes that $(\tilde e_i,\nu_i)$ is bounded $\forall t\ge 0$ provided $d_i^{\rho}>0$ for all $t>0$.

Proof of item (1):\\
Analyze the derivatives of $\dot d_i^{\rho}=\nu_i$, one has
\begin{equation}\label{eq.ddot_dg}
    \ddot{d}_i^{\rho} = -k_6\frac{\dot{d}_i^{\rho}}{d_i^{\rho}}-k_4\Tilde{e}_i-k_5\nu_i
\end{equation}
We will prove that $d_i^{\rho}$ remains positive using proof by contradiction. 
Assume there is a finite time $T>0$ such that $d_i^{\rho}(T)$ approaches zero. Take the integral of \eqref{eq.ddot_dg} from 0 to $T$, we get
\begin{equation}
\begin{aligned}
     k_6(\ln d_i^{\rho}(T)-\ln d_i^{\rho}(0)) = & \dot{d}_i^{\rho}(0) - \dot{d}_i^{\rho}(T)\\
     &-\int_0^T(k_4\Tilde{e}_i+k_5\nu_i)d\tau
\end{aligned}
\end{equation}
The left-hand side of the above equation tends to negative infinity, while the right-hand side is either bounded or tends to positive infinity since $\Tilde{e}_i$ and $\nu_i$ are bounded for all $0<t<T$, and $\dot{d}_i^{\rho}$ is either bounded or negative infinity. This leads to a contradiction, hence, $d_i^{\rho}$ remains positive for all time. Therefore, one concludes that $(\tilde e_i,\nu_i)$ are bounded. From there, direct application of \cite[Lemma 2]{Zhiqi2023Constructive}, one ensures that  $\frac{\dot{d}_i^{\rho}}{d_i^{\rho}}$ remains bounded for all the time. This in turn, implies that $a_i^r$ \eqref{eq.lon_control_vir} is also bounded for all the time.



Proof of item (2):

Using a similar argument as the proof  of item (2)-\cite[Lemma 2]{Zhiqi2023Constructive} along with Assumption \ref{ass.vehicle_index},   one concludes that the unique equilibrium point $(\tilde{e}_i, \nu_i)=(0,0)$ is asymptotically stable. %The undesired equilibrium point described in \cite[Lemma 2 - item (2)]{Zhiqi2023Constructive} does not appear here due to the Assumption \ref{ass.vehicle_index}.



% Proof of item (2):

% Since $d_i^g$ remains positive for all time, one concludes that the equilibrium point $(\Tilde{e}_i, \nu_i) = (0,0)$ is asymptotically stable. Let $\Tilde{s}_i := s_i-s_i^*$, then $\Tilde{e}_i = e_i-e_i^*$ can be written alternatively as $\Tilde{e}_i= \Tilde{s}_{i-1} - \Tilde{s}_i$.

% According to Assumption \ref{ass.platoon_form}, since $\Tilde{s}_1 = s_1-s_1^* = 0$, $\Tilde{e}_2 = 0$ asymptotically stable implies $\Tilde{s}_2 = 0$ asymptotically stable. By induction, one concludes that $\Tilde{s}_i = 0$ is asymptotically stable for all $i\in\mathcal{V}/\{1\}$.\TODO{Proper Citation}

% Similarly, for $\nu_i = v_{i-1}^r-v_i^r$, since $v_1^r = v_1^* = v^*$, $v_2^r = v^* = v_2^*$ is asymptotically stable as a result of $\nu_2 = 0$ is asymptotically stable. By induction, one concludes that $v_i^r = v_i^*$ is asymptotically stable.

\iffalse
We define error state $(\Tilde{s}_i,\Tilde{v}_i^r)$ for vehicle $i$ as $\Tilde{s}_i = s_i-s_i^*$ and $\Tilde{v}_i^r = v_i^r -v_i^*$.  Consider first for vehicle $i=2$, according to Assumption \ref{ass.platoon_form} we have $\Tilde{e}_2 = \Tilde{s}_1 - \Tilde{s}_2 = -\Tilde{s}_2$ and $\nu_2 = v^*-v_2^r = -\Tilde{v}_2^r$ since $s_1=s_1^*$ and $v_1 = v_1^* = v^*$. The Lyapunov function as defined in \eqref{eq.Llong} then becomes
\begin{equation}
    \mathcal{L}_i = \frac{1}{2}k_4\Tilde{s}_2^2+\frac{1}{2}\mbox{$\Tilde{v}^{r}_2$}^2
\end{equation}



which is zero only when $s_2 = s_2^*$ and $v_2=v^* = v_2^*$. Since $d_2^{\rho}(t)>0$ for all time according to the proof for item (1), we conclude that $\dot{L}_2$ is negative semi-definite and $(\Tilde{s}_2,\Tilde{v}_2) = (0,0)$ is asymptotically stable.  

For vehicle $i = 3$, consider the dynamic for its error state $(\Tilde{s}_3,\Tilde{v}_3^r)$ as
\begin{equation}\label{eq.lonPerturbed}
\left\{
\begin{aligned}
    \dot{\Tilde{s}}_3 &= \Tilde{v}_3^r\\
    \dot{\Tilde{v}}_3^r &= a_{2}^r+ k_4\Tilde{e}_3+k_5\nu_3+k_6\frac{\dot{d^{\rho}_3}}{d^{\rho}_3} -\dot{v}_3^*
\end{aligned}
\right.
\end{equation}
Since $\Tilde{e}_3 = \Tilde{s}_2-\Tilde{s}_3$ and $\nu_3 =\Tilde{v}_2^r-\Tilde{v}_3^r$. \eqref{eq.lonPerturbed} can be viewed as a system perturbed by $\Tilde{s}_2$ and $\Tilde{v}_2^r$ to the following unforced system
\begin{equation}\label{eq.lonUnforced}
    \left\{
    \begin{aligned}
    \dot{\Tilde{s}}_3 &= \Tilde{v}_3^r\\
    \dot{\Tilde{v}}_3^r &= -k_4\Tilde{s}_3-k_5\Tilde{v}_3^r+k_6\frac{\dot{d^{\rho}_3}}{d^{\rho}_3}
    \end{aligned}
    \right.
\end{equation}
which is obtained by setting $\Tilde{e}_3 = -\Tilde{s}_3$, $\nu_3 =-\Tilde{v}_3^r$, and $a_{2}^r = {\dot{v}_2}^{*}={\dot{v}}^{*}={\dot{v}_3}^{*}$

Consider the following Lyapunov candidate for the unforced system \eqref{eq.lonUnforced}
\begin{equation}
    \mathcal{L}_3 = \frac{1}{2}k_4  \Tilde{s}_3^2+\frac{1}{2}\mbox{$\Tilde{v}^{r}_3$}^2
\end{equation}
Its derivative is obtained as:
\begin{equation}
    \dot{\mathcal{L}}_3 = -k_5\mbox{$\Tilde{v}^{r}_3$}^2-k_6\frac{\mbox{$\Tilde{v}^{r}_3$}^2}{d^{\rho}_3}
\end{equation}
which is negative semi-definite since $d^{\rho}_3>0$ as a result of item (1). We conclude that $(\Tilde{s}_3,\Tilde{v}_3) = (0,0)$ is asymptotically stable for the unforced system \eqref{eq.lonUnforced}. Finally, since $(\Tilde{s}_2,\Tilde{v}_2) = (0,0)$ is asymptotically stable, we claim that $(\Tilde{s}_3,\Tilde{v}_3) = (0,0)$ is asymptotically stable for the original system \eqref{eq.lonPerturbed}.

For $i\geq4$, assume that $(\Tilde{s}_{i-1},\Tilde{v}_{i-1}) = (0,0)$ is asymptotically stable for system of $i-1$, then using the same proof as for $i=3$, one concludes that $(\Tilde{s}_{i},\Tilde{v}_{i}) = (0,0)$ is asymptotically stable. By mathematical induction, we conclude that item (2) is true for all $i\in\mathcal{V}/\{1\}$.
\fi


% Proof of item (3):\\
% Let $\Tilde{v}_i^r := v_i^r- v_i^*$, then $\nu_i = v_{i-1}^r -v_i^r=\Tilde{v}^{r}_{i-1}-\Tilde{v}^{r}_{i}$ since $v_{i-1}^* -v_i^* = 0$ according to Assumption \ref{ass.platoon_form}, we can obtain the following inequality
% \begin{equation}\label{eq.ineq1}
%     \mbox{$\Tilde{v}^{r}_i$}^2 \leq 2\mbox{$\Tilde{v}^{r}_{i-1}$}^2 +2\nu_i^2 
% \end{equation}
% By induction, we can rewrite \eqref{eq.ineq1} as 
% \begin{equation}\label{eq.ineq2}
%     \mbox{$\Tilde{v}^{r}_i$}^2 \leq 2^{i-1}\mbox{$\Tilde{v}^{r}_1$}^2+\sum_{j=2}^i 2^{i-j+1} \nu_j^2 = \sum_{j=2}^i 2^{i-j+1} \nu_j^2, \; \forall i\geq2
% \end{equation}
% where the last step is due to the fact that $\Tilde{v}^{r}_1 = 0$.

% Since the Lyapunov function $\mathcal{L}_i = \frac{1}{2}k_4\Tilde{e}_i^2+\frac{1}{2}\nu_i^2$  \eqref{eq.Llong} is a non-decreasing function, we claim the following inequality:
% \begin{equation}\label{eq.ineq3}
%     \nu_i^2(t) \leq k_4\Tilde{e}_i^2(t)+\nu_i^2(t) \leq k_4\Tilde{e}_i^2(0)+\nu_i^2(0)
% \end{equation}
% By combining \eqref{eq.ineq2} and \eqref{eq.ineq3} together with the initial condition $\sum_{j=2}^i 2^{i-j+1}\bigl( k_4\Tilde{e}_j^2(0)+\nu_j^2(0)\bigr)\leq {v^*}^2$, one concludes $0< v_i^r < 2v^*$. 

\end{proof}


%\begin{lem}
%    Consider any two consecutive vehicles, $i$ and $i-1$, and let the acceleration input $a_i$ of the follower vehicle $i$ be given by \eqref{eq.control_lon}. For initial states $v_i(0),s_i(0)$ such that $\Tilde{v}(0) \leq $ 
%\end{lem}

\subsection{Proof of theorem 1}\label{app.2}

\begin{proof}
Consider first the lateral error state $(\Tilde{y}_i, \Tilde{\theta}_i)$, by using \eqref{eq.dyna_frenet},\eqref{eq.control_lat}, \eqref{eq.control_lat_n}, and \eqref{eq.control_lat_o}, its closed-loop dynamics can be expressed as
\begin{equation}\label{eq.closed_lat_error}
    \left\{
    \begin{aligned}
        \dot{\Tilde{y}}_i &= v_i\sin\Tilde{\theta}_i\\
        \dot{\Tilde{\theta}}_i &= -k_1v_i\Tilde{y}_i\frac{\sin\Tilde{\theta}_i}{\Tilde{\theta}_i} - k_2|v_i|\Tilde{\theta}_i\\
        &-k_3 |v_i|\left (\frac 1 {d^{\eta_L}_i}+\frac 1 {d^{\eta_R}_i}\right)\sin\tilde \theta_i
    \end{aligned}
    \right.
\end{equation}
Define the following Lyapunov function:
 \begin{equation}\label{eq.Llat}
    \mathcal{L} = \frac{1}{2}k_1\Tilde{y}_i^2+\frac{1}{2}\Tilde{\theta}_i^2
\end{equation}
Together with \eqref{eq.closed_lat_error}, the derivative of \eqref{eq.Llat} is given by: 
\begin{equation}\label{eq:dotL}
    \dot{\mathcal{L}} = -k_2 |v_i|\Tilde{\theta}_i^2-k_3 |v_i|\left (\frac 1 {d^{\eta_L}_i}+\frac 1 {d^{\eta_R}_i}\right )\Tilde{\theta}_i\sin\Tilde{\theta}_i
\end{equation}
which is negative semi-definite given that $d^{\eta_R}_i>0$, $d^{\eta_R}_i>0$, and $|\Tilde{\theta}_i|\leq \pi/2-\epsilon_1$ (and even more so when $|\Tilde{\theta}_i|\leq \pi$). One verifies that the state $(\tilde y_i,\tilde \theta_i)$ is bounded as long as $d^{\eta_R}_i>0$, and $d^{\eta_R}_i>0$. %In other words, given that $d_i^\eta > 0$, $|\Tilde{\theta}_i|\leq \pi$ describes an invariant set for $\Tilde{\theta}_i$. 

Proof of item (1):\\
Since Lemma \ref{lem.boundedv}- item 1 concludes that $d_i^{\rho}$ remains positive all the time. We will focus on showing the bounds of $|\tilde \theta_i|$ and $|\tilde y_i|$ as well as the positiveness of $d^{\eta_L}_i$ and $d^{\eta_R}_i$. Since the Lyapunov function \eqref{eq.Llat} is nonincreasing,  the initial condition $k_1 \tilde y_i^2(0)+\tilde \theta_i^2(0)<(\frac{\pi}{2}-\epsilon_1)^2$ implies that $|\tilde \theta_i(t)|<\pi/2-\epsilon_1$ as long $d^{\eta_L}_i>0$ and $d^{\eta_R}_i>0$. Recall that from \eqref{eq.deta} and Assumption \ref{ass.road_shape}, if $d^{\eta_L}_i>0$ and $d^{\eta_R}_i>0$ one ensures that $|\Tilde{y}_i|< \frac{1}{\chi_i^r}$.
%The positiveness of $d_i^{\rho}$ is already shown in Lemma 1 - item 1. We will focuses on the proof for $v_i$ and $d^{\eta}_i$ here.
% \TODO{For $v_i$, recall \eqref{eq.vr}, one guarantees that $v_i$ remains positive and bounded if $v_i^r$ remain positive and bounded, $|\Tilde{\theta}_i|<\frac{\pi}{2}$, and $|\Tilde{y}_i|< \frac{1}{\chi_i^r}$. The latter condition is equivalent to $d^{\eta}_i>0$ from \eqref{eq.deta}. Due to condition \eqref{eq.initial-v}, the positiveness and boundness of $v_i$ are guaranteed from the conclusion of Lemma - 1, item (3).}

Due to the convex combination $d^{\eta_R}_i+d^{\eta_L}_i=2w-2\epsilon_w$ and the symmetry of the problem at hand, proving $d^{\eta_R}_i>0$ is similar to prove $d^{\eta_L}_i>0$. From now on, we will only show the proof for $d^{\eta_L}_i$ and denote $d^{\eta}_i=d^{\eta_L}_i$ for the sake of simplicity. Take the derivative of ${d^{\eta}_i}^{'} =- \sign(v_i)\sin\Tilde{\theta}_i$ with respect to $\sigma$ to obtain\footnote{The derivative of $\sign(v_i)$ with respect to $\sigma$ is set to zero because when $v_i=0$ the system is at rest and $\sigma$ cannot be used as a substitute for time to describe changes.}:
\begin{equation}
\begin{aligned}
      {d^{\eta}_i}^{''} =&- \sign(v_i)\frac d {d\sigma}\sin \tilde\theta_i
\end{aligned}
\end{equation}
since $\frac d {d\sigma}\sin \tilde\theta_i=\cos\tilde \theta \sign(v_i)\left(\chi_i - \frac{\chi_i^r\cos\Tilde{\theta}_i}{1-\chi_i^r\Tilde{y}_i}\right)$
\begin{equation}\label{eq.ddot_deta}
\begin{aligned}
      {d^{\eta}_i}^{''}
      = &-k_3\cos\Tilde{\theta}_i\frac{{d^{\eta}_i}^{'}}{d^{\eta}_i}-\alpha_i
\end{aligned}
\end{equation}
where 
\begin{align*}
\alpha_i(\sigma(t))=&-\cos\Tilde{\theta}_i ( -k_1\frac{\sin\Tilde{\theta}_i}{\Tilde{\theta}_i}\Tilde{y}_i- k_2\sign(v_i)\Tilde{\theta}_i\\
&-k_3\frac{\sign(v_i)\sin\tilde \theta_i}{d_i^{\eta_R}})
\end{align*}


%where $\alpha_i(\sigma(t))=-\cos\Tilde{\theta}_i (\chi_i^n-k_3\frac{\sign(v_i)\sin\tilde \theta_i}{d_i^{\eta_R}})$. 
Take integral of the above equation with respect to $\sigma$ from $\sigma_0=\sigma(0)$ to $\sigma_T=\sigma(T)$ with a finite $\sigma_T$ (or equivalently finite $T$), one has:
 \begin{align*}
    & k_3\int_{\sigma_0}^{\sigma_T}\cos\Tilde{\theta}_i\frac{{d^{\eta}_i}^{'}}{d^{\eta}_i} d \sigma={d^{\eta}_i}^{'}(T)-{d^{\eta}_i}^{'}(0)-\int_{\sigma_0}^{\sigma_T}\alpha_id\sigma
    \end{align*}
Using the fact that $|\Tilde{\theta}|<\frac{\pi}{2}$ in the interval $[\sigma_0,\sigma_T)$, and $\cos\Tilde{\theta} >0$ with a bounded derive, one ensures that there exists $\bar k_3\in (0,k_3]$ a positive and bounded scalar such that $\int_{\sigma_0}^{\sigma_T}k_3\cos\Tilde{\theta}_i\frac{{d^{\eta}_i}^{'}}{d^{\eta}_i}d\sigma=\bar k_3\ln \frac{d^{\eta}_i(\sigma_T)}{d^{\eta}_i(\sigma_0)}$ and hence: 
\begin{equation}\label{eq.detaint}
    \begin{aligned}
    &\bar k_3\ln \frac{d^{\eta}_i(\sigma_T)}{d^{\eta}_i(\sigma_0)}={d^{\eta}_i}^{'}(\sigma_T)-{d^{\eta}_i}^{'}(\sigma_0)-\int_{\sigma_0}^{\sigma_T}\alpha_id\sigma
    \end{aligned}
\end{equation}
From this and using the fact that $\frac{{d^{\eta}_i}^{'}(\sigma_0)}{d^{\eta}_i(\sigma_0)}$, ${d^{\eta}_i}^{'}(\sigma_T)$, and ${d^{\eta}_i}^{'}(\sigma_0)$ are bounded and that $\alpha_i$ is bounded in  $[\sigma_0,\sigma_T)$, one concludes that $d^{\eta}_i$ remains positive by following the same reasoning as in the proof of Lemma 1, Item 1.

%\tang{$|y_i(t)|<\frac 1 {\chi_i^r}$ for all the time, is it needed ?}.
%and   \tang{Similar to the proof of Lemma 1 - item (1), we will show that $d^{\eta}_i$ remains positive using proof by contradiction. Assume that in finite time $T$, $d^{\eta}_i(T)$ approaches zero, which implies the left-hand side of equation \eqref{eq.detaint} tends to negative infinity. 

%Since $d^{\eta}_i>0$ from time $0$ to $T$, one has $|\tilde y_i|<1/\chi_i^r$, in addition, the Lyapunov function $\mathcal L_i$ is always decreasing, indicating $|\tilde \theta|<\pi/2$. This implies that $v_i$ is also bounded and positive. 

%However, the right hand side of the equality is bounded since $|\tilde \theta|<\pi/2$ and $\chi_i^n$ are bounded and $d_i^{\eta_R}>0$.
%This contradiction proves that $d^{\eta}_i$ remains positive, hence $|y_i(t)|<\frac 1 {\chi_i^r}$ for all the time.

% \iffalse
% \TODO{THIS IS WRONG: Here, for $0<t<T$, $|\Tilde{\theta}_i(t)|<\frac{\pi}{2}$ given initial condition $|\Tilde{\theta}_i(0)|<\frac{\pi}{2}$, this in turn implies that $0\leq\cos\Tilde{\theta}_i\leq 1$. As a result, the following inequality is obtained from \eqref{eq.detaint}
% \begin{equation}
%     \begin{aligned}
%     k_3&\int_{\sigma(0)}^{\sigma(T)} \frac{{d^{\eta}_i}^{'}}{d^{\eta}_i}d\sigma \geq-\int_{\sigma(0)}^{\sigma(T)}\sign(v_i){d^{\eta}_i}^{''} d\sigma \\
%     &-k_1\int_{\sigma(0)}^{\sigma(T)}\alpha_i\sign(v_i)\cos\Tilde{\theta}_i\Tilde{y}_i\frac{\sin\Tilde{\theta}_i}{\Tilde{\theta}_i}d\sigma\\
%     &- k_2\int_{\sigma(0)}^{\sigma(T)}\alpha_i\cos\Tilde{\theta}_i\Tilde{\theta}_id\sigma
%     \end{aligned}    
% \end{equation}
% where the left side of the above inequality can be expressed as $k_3(\ln(d^{\eta}_i(T))-\ln(d^{\eta}_i(0)))$, which tends to negative infinity as $d^{\eta}_i(T) = 0$. \lt{The first integral on the right side of the inequality is bounded, since ${d_i^\eta}^{'} = \alpha_i\sin\Tilde{\theta}_i$ is bounded and $v_i$ does not exhibit infinite oscillation as a result of $v_i^r$ is asymptotically stable according to Lemma \ref{lem.boundedv} and the fact that $|\Tilde{\theta}_i(t)|<\frac{\pi}{2}$ and $|\Tilde{y}_i|< w-r_\eta < \frac{1}{\chi_i^r}$ given $d^{\eta}_i(T) > 0$. }The second and third integral in the above inequality are bounded since $\sigma(T)$, $\cos\Tilde{\theta}_i$, $\Tilde{y}_i$, $\frac{\sin\Tilde{\theta}_i}{\Tilde{\theta}}$, and $\Tilde{\theta}_i$ are all bounded for $0<t<T$. This leads to a contradiction, hence $d^{\eta}_i>0$ for all finite time. In addition, one concludes that $\Tilde{y}_i$ and $\Tilde{\theta}_i$ are bounded. A direct application of \cite[Lemma 2]{Zhiqi2023Constructive}, one has $\frac{{d^{\eta}_i}^{'}}{d^{\eta}_i}$ is bounded for all the time which implies that $\chi_i$ is bounded for all time \lt{is this true?}. }
% \fi

Proof of item (2):\\
Since $\chi_i^n$ is bounded because $\tilde y_i$ and $\tilde \theta_i$ are bounded, to show that controller $\chi_i$ is bounded, it suffices to show $\chi_i^c$ is bounded when $d_i^{\eta}$ converges to zero as $\sigma(t)$ tends to infinity. The Proof follows a similar structure of \cite[Lemma 2]{Zhiqi2023Constructive}. We will first show that ${d_i^{\eta}}^{'}$ converges to zero.
Define a new variable $\mu$ such that $\mu=\int_{\sigma_0}^{\sigma_t}\frac{1}{d^{\eta}_i}d \sigma$ and rewrite \eqref{eq.ddot_deta} as
\begin{equation}
\begin{aligned}
      \frac{d}{d\mu}{d^{\eta}_i}^{'} 
      = &-k_3\cos\Tilde{\theta}_i{d^{\eta}_i}^{'}-o(\mu)
\end{aligned}
\end{equation}
with $k_3\cos\Tilde{\theta}_i>0$ and $o(\mu)=d_i^\eta \alpha_i$ a perturbation term that tends to zero whens $\mu$ goes to infinity (or equivalently when $d_i^{\eta}$ converges to zero). One recognizes that this is the dynamics of a first-order system (with ${d^{\eta}_i}^{'}$ the state) perturbed by a vanishing perturbation. From there, one concludes that ${d^{\eta}_i}^{'}$ is bounded and also converges to zero as $d_i^{\eta}$ converges to zero. To prove that is $\chi_i^c$ is bounded when $({d^{\eta}_i}^{'}, d^{\eta}_i)\to(0,0)$, it suffices to show that $\frac{{d^{\eta}_i}^{'}}{d^{\eta}_i}$ is bounded. Hence, differentiating $\frac{{d^{\eta}_i}^{'}}{d^{\eta}_i}$ with respect to $\mu$, one verifies that:
\begin{equation}
\begin{aligned}
      \frac{d}{d\mu}\frac{{d^{\eta}_i}^{'}}{d^{\eta}_i} = -(k_3\cos\Tilde{\theta}_i+ {d^{\eta}_i}^{'})\frac{{d^{\eta}_i}^{'}}{d^{\eta}_i}-\alpha_i
\end{aligned}
\end{equation}
Using the fact that $\alpha_i$ is bounded and ${d^{\eta}_i}^{'}$ is converging to zero, one ensures that there exists a $\mu^*$ or equivalently a $\sigma^*$ (and therefore a time instant $T$), such that $({d^{\eta}_i}^{'} +k_3 \cos \Tilde\theta) >0, \; \forall \mu \geq \mu^*$. From there, one guarantees that $\frac{{d^{\eta}_i}^{'}}{d^{\eta}_i}$  is bounded, hence, $\chi_i$ is bounded and well-defined for all the time.

To prove $a_i$ is bounded and well-defined, we recall equation \eqref{eq.ar2a}, \eqref{eq.vi}, and \eqref{eq.dyna_frenet}. One concludes that $a_i$ is well-defined and bounded since $|\Tilde{\theta}_i|<\frac{\pi}{2}-\epsilon_1$, $|\Tilde{y}_i|<\frac{1}{\chi_i^r}$, and $a_i^r$ and $v_i^r$ are bounded and well-defined (Lemma \ref{lem.boundedv}).



Proof of item (3): %Since $\dot{\mathcal{L}}(t)\le 0,\forall t>0$, $\mathcal{L}$ is bounded.Now, let us analyze the second derivatives of the function $\mathcal{L}$

Using Barbalat’s Lemma along with a similar argument as the proof of Lemma 2 - item (2) in \cite{Zhiqi2023Constructive}, one concludes that the unique equilibrium point $(\tilde{y}_i, \Tilde{\theta}_i)=(0,0)$  for the lateral subsystem is asymptotically stable. 

%Similarly to the longitudinal control part, the kind of undesired equilibrium point described in \cite[Lemma 2 - item (2)]{Zhiqi2023Constructive} does not exists here \tang{since both the initial state $y_i(0)$ desired state $y_i^*$ are in the same free space on the road.

From Lemma \ref{lem.boundedv} item (2), one concludes that the origin of $(\tilde e_i,\nu_i)$ is asymptotically stable. Recall equation \eqref{eq.vr}, one has $\tilde v_i^r=\tilde v_i$ provided $(\tilde{y}_i, \Tilde{\theta}_i)=(0,0)$. From Assumption \ref{ass.platoon_form}, the leader vehicle is already in its desired states, i.e., $\tilde s_1=0$ and $\tilde v_1=\tilde v_1^r=0$, hence, the asymptotic stability of  $(\tilde e_i,\nu_i)=(\tilde s_{i-1}-\tilde s_i,\tilde v_{i-1}^r-\tilde v_i^r)=(0,0)$ and $(\tilde{y}_i, \Tilde{\theta}_i)=(0,0)$  implies that the origin of $(\tilde s_i,\tilde v_i)$ is also asymptotically stable.


\iffalse
Using the feedback for $\chi_i$ as in \eqref{eq.control_lat}, we obtain
\begin{equation}\label{eq.ddoubledeta}
\begin{aligned}
     {d^{\eta}_i}^{''} = &-k_1\alpha_i\cos\Tilde{\theta}_i\Tilde{y}_i\frac{\sin\Tilde{\theta}_i}{\Tilde{\theta}_i} - k_2\alpha_i\cos\Tilde{\theta}_i\sign(v_i)\Tilde{\theta}_i\\
     &-k_3 \cos\Tilde{\theta}_i\sign(v_i) \frac{{d^{\eta}_i}^{'}}{d^{\eta}_i} \biggl)
\end{aligned}
\end{equation}
Assume there is a finite time $T>0$ such that $d^{\eta}_i(T) = 0$. Taking the integral over \eqref{eq.ddoubledeta} with respect to $\sigma$ from $\sigma(0)$ to $\sigma(T)$, where $\sigma(T) = \sigma(0)+\int_0^T v_i dt$, we get
\begin{equation}\label{eq.detaint}
    \begin{aligned}
    k_3&\int_{\sigma(0)}^{\sigma(T)}\cos\Tilde{\theta}_i\sign(v_i)  \frac{{d^{\eta}_i}^{'}}{d^{\eta}_i}d\sigma =-\int_{\sigma(0)}^{\sigma(T)}{d^{\eta}_i}^{''} d\sigma \\
    &-k_1\int_{\sigma(0)}^{\sigma(T)}\alpha_i\cos\Tilde{\theta}_i\Tilde{y}_i\frac{\sin\Tilde{\theta}_i}{\Tilde{\theta}_i}d\sigma\\
    &- k_2\int_{\sigma(0)}^{\sigma(T)}\alpha_i\cos\Tilde{\theta}_i\sign(v_i)\Tilde{\theta}_id\sigma
    \end{aligned}
\end{equation}
Here, for $0<t<T$, we have that $d_i^{\eta}>0$, which means $|\Tilde{y}_i(t)|<w-r_{\eta}<\frac{1}{\chi_i^r}$. In addition, $|\Tilde{\theta}_i(t)|<\frac{\pi}{2}$ given initial condition $|\Tilde{\theta}_i(0)|<\frac{\pi}{2}$, this in turn implies that $0\leq\cos\Tilde{\theta}_i\leq 1$. Recall Lemma \ref{lem.boundedv}, we conclude that $v_i(t)\geq0$ and is bounded for $0<t<T$, which in turn implies that $\sign(v_i) = 1$ and $\sigma(T)$ being bounded. Since $\sign(v_i) = 1$ and $0\leq\cos\Tilde{\theta}_i\leq 1$, we first conclude the following inequality from \eqref{eq.detaint}
\begin{equation}
    \begin{aligned}
    k_3&\int_{\sigma(0)}^{\sigma(T)} \frac{{d^{\eta}_i}^{'}}{d^{\eta}_i}d\sigma \geq -\int_{\sigma(0)}^{\sigma(T)}{d^{\eta}_i}^{''} d\sigma \\
    &-k_1\int_{\sigma(0)}^{\sigma(T)}\alpha_i\cos\Tilde{\theta}_i\Tilde{y}_i\frac{\sin\Tilde{\theta}_i}{\Tilde{\theta}_i}d\sigma- k_2\int_{\sigma(0)}^{\sigma(T)}\alpha_i\cos\Tilde{\theta}_i\Tilde{\theta}_id\sigma
    \end{aligned}    
\end{equation}
Further simplification gives
\begin{equation}
    \begin{aligned}
    k_3&(\ln(d^{\eta}_i(T))-\ln(d^{\eta}_i(0))) \geq -({d^{\eta}_i}^{'}(T)-{d^{\eta}_i}^{'}(0)) \\
    &-k_1\int_{\sigma(0)}^{\sigma(T)}\alpha_i\cos\Tilde{\theta}_i\Tilde{y}_i\frac{\sin\Tilde{\theta}_i}{\Tilde{\theta}_i}d\sigma- k_2\int_{\sigma(0)}^{\sigma(T)}\alpha_i\cos\Tilde{\theta}_i\Tilde{\theta}_id\sigma
    \end{aligned}    
\end{equation}
where the left side of the inequality tends to negative infinity as $d^{\eta}_i(T))$ tends to 0, while the right side of the equation is bounded resulting in a contradiction. Recall that ${d^{\eta}_i}^{'} =\alpha_i \sin\Tilde{\theta}_i$, so the first term on the right side is bounded. The second and third term are bounded since $\sigma(T)$, $\cos\Tilde{\theta}_i$, $\Tilde{y}_i$, $\frac{\sin\Tilde{\theta}_i}{\Tilde{\theta}}$, and $\Tilde{\theta}_i$ are all bounded for $0<t<T$. In conclusion, we have that $d_i^{\eta}>0$, as a result, $(\Tilde{y}_i,\Tilde{\theta}_i) = (0,0)$ is asymptotically stable and the input $\chi_i$ is bounded. Furthermore, from Lemma \ref{lem.boundedv} we have $d_i^{\rho}>0$ and $a_i^r$ is bounded, consequently, $a_i$ obtained through \eqref{eq.ar2a} is bounded since $|\Tilde{\theta}_i|<\frac{\pi}{2}$, $\Tilde{y}_i<\frac{1}{\chi_i^r}$ and all other terms are bounded.    


For the longitudinal direction, consider 
\begin{equation}\label{eq.original}
    \left\{
    \begin{aligned}
        &\dot{s}_i = \frac{v_i\cos\Tilde{\theta}_i}{1-\chi_i^r\Tilde{y}_i}\\
        &\dot{v}_i = a_i
    \end{aligned}
    \right.  
\end{equation}
which can be considered as a system perturbed by $\Tilde{\theta}_i$ and $\Tilde{y}_i$ to the unforced system:
\begin{equation}\label{eq.unforced}
    \left\{
    \begin{aligned}
        &\dot{s}_i = v_i\\
        &\dot{v}_i = a_i
    \end{aligned}
    \right.      
\end{equation}

For \eqref{eq.unforced}, since $\Tilde{\theta}_i = 0$ and $\Tilde{y}_i = 0$, we have that $v_i = v_i^r$ and $a_i = a_i^r$. As a result of item (2) in Lemma \ref{lem.boundedv}, we conclude that $(s_i,v_i)$ converges asymptotically to $(s_i^*,v_i^*)$. Since $(\Tilde{\theta}_i,\Tilde{y}_i) = (0,0)$ is asymptotically stable, we conclude that $(s_i,v_i)$ converges to $(s_i^*,v_i^*)$ asymptotically for the original system \eqref{eq.original}.
\fi

\end{proof}

% \lt{To handle the ramp merging situation, where the vehicle on the acceleration lane is forced to merge into the platoon before the lane ends, assume the road boundary with varying width as a function of the longitudinal distance $s$, defined as sigmoid function 
% \begin{equation}
% w(s) = tanh(s)    
% \end{equation}
% or
% \begin{equation}
% w(s) = \frac{1}{1+e^{-s}}
% \end{equation}

% The safety distance to the road boundary with varying a width is given as 
% \begin{equation}
%     d^{\eta}_i = w(s)-|\Tilde{y}_i|-\epsilon_w
% \end{equation}

% Its derivative is
% \begin{equation}
%     {d^{\eta}_i}^{'} = \frac{\partial w}{\partial s}\frac{\dot{s}}{v_i} -\sign(\tilde{y}_i) \sin(\tilde{\theta}_i)
% \end{equation}

% where 
% $\frac{\partial w}{\partial s} = 0, \forall s \in (-\infty, s_a] \cup [s_b, \infty)$

% Consider first the lateral error state $(\Tilde{y}_i, \Tilde{\theta}_i)$, by using \eqref{eq.dyna_frenet},\eqref{eq.control_lat}, \eqref{eq.control_lat_n}, and \eqref{eq.control_lat_o}, its closed-loop dynamics can be expressed as
% \begin{equation}\label{eq.closed_lat_error}
%     \left\{
%     \begin{aligned}
%         &\dot{\Tilde{y}}_i = v_i\sin\Tilde{\theta}_i\\
%         &\dot{\Tilde{\theta}}_i = -k_1v_i\Tilde{y}_i\frac{\sin\Tilde{\theta}_i}{\Tilde{\theta}_i} - k_2|v_i|\Tilde{\theta}_i +k_3 \sign(\tilde{y}_i)|v_i| \frac{{d^{\eta}_i}^{'}}{d^{\eta}_i}
%     \end{aligned}
%     \right.
% \end{equation}
% Define the following Lyapunov function:
%  \begin{equation}\label{eq.Llat}
%     \mathcal{L} = \frac{1}{2}k_1\Tilde{y}_i^2+\frac{1}{2}\Tilde{\theta}_i^2
% \end{equation}
% Its derivative is obtained as: 
% \begin{equation}
%     \dot{\mathcal{L}} = -k_2 |v_i|\Tilde{\theta}_i^2-k_3 |v_i| \frac{\Tilde{\theta}_i\sin\Tilde{\theta}_i}{d^{\eta}_i} + k_3 \sign(\tilde{y}_i) \frac{\partial w}{\partial s}|v_i| \frac{\tilde{\theta}_i\cos\tilde{\theta}_i}{1-\chi_i^r\tilde{y}_i}   
% \end{equation}

% Given that $\dot{s}_i = v^r_i > 0, \forall t$, then there is a time $t_b$ such that $s_i \geq s_b$, so for $t\geq t_b$, the derivative of the Lyapunov function reduce to 
% \begin{equation}
%     \dot{\mathcal{L}} = -k_2 |v_i|\Tilde{\theta}_i^2-k_3 |v_i| \frac{\Tilde{\theta}_i\sin\Tilde{\theta}_i}{d^{\eta}_i}
% \end{equation}

% which is negative semi-definite given that $|\Tilde{\theta}_i|\leq \pi$ and $d^{\eta}_i>0$. Using the initial condition $|\tilde \theta_i(0)|<\pi/2$, one verifies that the state $(\tilde y_i,\tilde \theta_i)$ is bounded as long as $d_i^\eta>0$.

% Possibly, Since $\dot{s} = v_i^r$, $v_i$ is positive and bounded when $|\Tilde{\theta}_i|<\frac{\pi}{2}$ and $|\Tilde{y}_i|<\frac{1}{\chi}_i^r$. so $\sign(v_i)\dot{s} \geq 0$. then the derivative is negative semi-definite given that $|\Tilde{\theta}_i|\leq \pi$ and $d^{\eta}_i>0$. To find the initial condition that ensures the positiveness of $v_i^r$ and $v_i$, Hamilton Jacobi reachability analysis could be used to compute the maximal backward reachable set numerically given unsafe set $v_i^r\leq 0$. 


% To show that $d_i^\eta>0$ for all the time, take the derivative of ${d^{\eta}_i}^{'} =\alpha_i \sin\Tilde{\theta}_i$ with respect to $\sigma$ to obtain:
% \begin{equation}\label{eq.ddot_deta}
% \begin{aligned}
%       {d^{\eta}_i}^{''} = &\frac{d({d^{\eta}_i}^{'})}{dt}\frac{dt}{d\sigma} 
%       \\
%       = &\alpha_i\cos\Tilde{\theta}_i\biggl(-k_1\Tilde{y}_i\frac{\sin\Tilde{\theta}_i}{\Tilde{\theta}_i}- k_2\sign(v_i)\Tilde{\theta}_i\\
%      &-k_3 \alpha_i\sign(v_i)\frac{{d^{\eta}_i}^{'}}{d^{\eta}_i}\biggr)+\frac{\partial^2 w}{\partial s^2}\frac{\dot{s}_i^2}{v_i} + \frac{\partial w}{\partial s}(\frac{\ddot{s}_iv_i-\dot{s}_i\dot{v}_i}{v_i^2})
% \end{aligned}
% \end{equation}

% need to check if the additional terms are bounded when taking the integral.
% }



\addtolength{\textheight}{0cm}   % This command serves to balance the column lengths
                                  % on the last page of the document manually. It shortens
                                  % the textheight of the last page by a suitable amount.
                                  % This command does not take effect until the next page
                                  % so it should come on the page before the last. Make
                                  % sure that you do not shorten the textheight too much.

%%%%%%%%%%%%%%%%%%%%%%%%%%%%%%%%%%%%%%%%%%%%%%%%%%%%%%%%%%%%%%%%%%%%%%%%%%%%%%%%



%%%%%%%%%%%%%%%%%%%%%%%%%%%%%%%%%%%%%%%%%%%%%%%%%%%%%%%%%%%%%%%%%%%%%%%%%%%%%%%%



%%%%%%%%%%%%%%%%%%%%%%%%%%%%%%%%%%%%%%%%%%%%%%%%%%%%%%%%%%%%%%%%%%%%%%%%%%%%%%%%



%%%%%%%%%%%%%%%%%%%%%%%%%%%%%%%%%%%%%%%%%%%%%%%%%%%%%%%%%%%%%%%%%%%%%%%%%%%%%%%%

\input{IEEEreference.bbl}


\end{document}
