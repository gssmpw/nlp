In this work, we proposed a novel soft link-based sampling method namely MixDec Sampling for GNN-based recommendation models.
Instead of being restricted to using hard positive pairs and hard negative pairs, MixDec Sampling provides a soft link to measure the proximity relationship between nodes, and the synthesized nodes also provide data augmentation for nodes with few neighbors.
Extensive experiments demonstrate the proposed MixDec Sampling can improve the recommendation performance of
several representative GNN-based models significantly and consistently on various recommendation benchmarks.
We hope this work would inspire the future soft link sampling method for GNN-based recommendation systems  for
efficient and effective utilizing graph information.
%In conventional negative sampling, the link between nodes is simply separated into positive and negative ones, which limits the full utilization of the structural information of the graph, hence preventing the model from achieving a high accuracy during the training phase.
%To address this problem, we propose MixDec Sampling, which is a new sampling method. Based on negative sampling, this method introduces the concept of soft link and decay item. The distribution of node features is enriched via MixDec Sampling, allowing the model to fully utilize the structural information of the graph.
%Studies demonstrate its superior performance, and experiments on various graph models support the method's generalizability.
%In future work, we aim to find better ways to determine the weight of soft link and to explore the effectiveness of Mix Sampling in other areas.