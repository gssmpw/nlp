\documentclass[conference,final]{IEEEtran}
\IEEEoverridecommandlockouts
% The preceding line is only needed to identify funding in the first footnote. If that is unneeded, please comment it out.
\usepackage{cite}
\usepackage{amsmath,amssymb,amsfonts}
\usepackage{algorithmic}
\usepackage{graphicx}
\usepackage{textcomp}
\usepackage{xcolor}
\usepackage{enumerate}
\usepackage{booktabs}
\usepackage{xspace}
\usepackage{multirow}
% \usepackage{hyperref}
\usepackage{url}
\urlstyle{rm}


% \usepackage{algpseudocode}

\usepackage{algorithm}
\makeatletter
\newcommand{\removelatexerror}{\let\@latex@error\@gobble}
\makeatother


\makeatletter
\DeclareRobustCommand\onedot{\futurelet\@let@token\@onedot}
\def\@onedot{\ifx\@let@token.\else.\null\fi\xspace}

\def\eg{\emph{e.g}\onedot} \def\Eg{\emph{E.g}\onedot}
\def\ie{\emph{i.e}\onedot} \def\Ie{\emph{I.e}\onedot}
\def\cf{\emph{c.f}\onedot} \def\Cf{\emph{C.f}\onedot}
\def\etc{\emph{etc}\onedot} \def\vs{\emph{vs}\onedot}
\def\wrt{w.r.t\onedot} \def\dof{d.o.f\onedot}
\def\etal{\emph{et al}\onedot}
\makeatother


\def\BibTeX{{\rm B\kern-.05em{\sc i\kern-.025em b}\kern-.08em
    T\kern-.1667em\lower.7ex\hbox{E}\kern-.125emX}}
\begin{document}

\title{{\huge MixDec Sampling: A Soft Link-based Sampling Method of Graph Neural Network for Recommendation}\\
\thanks{$\star$ Xiangjin Xie, Yuxin Chen, Ruipeng Wang and Xianli Zhang contributed equally to this research.}
\thanks{$\dagger$ Hai-Tao Zheng and Buyue Qian is the corresponding authors.}
}
\author{Xiangjin Xie$^{1\star}$, Yuxin Chen$^{2\star}$, Ruipeng Wang$^{3\star}$, Xianli Zhang$^{4\star}$, Shilei Cao$^2$, Kai Ouyang$^1$,\\
  Zihan Zhang$^5$, Hai-Tao Zheng$^{1,7\dagger}$, Buyue Qian$^{6\dagger}$, Hansen Zheng$^4$, Bo Hu$^2$, Chengxiang Zhuo$^2$, Zang Li$^2$\\
$^1$Shenzhen International Graduate School, Tsinghua University, $^2$Tencent PCG,\\
$^3$School of Mathematical Science, University of Electronic Science and Technology of China,\\
$^4$School of Computer Science and Technology, Xi’an Jiaotong University,\\ 
$^5$Beihang University, $^6$Beijing Chaoyang Hospital, Capital Medical University, $^7$Pengcheng Laboratory\\
$^1${\tt \{xxj20, oyk20\}@mails.tsinghua.edu.cn}, $^1${\tt zheng.haitao@sz.tsinghua.edu.cn},\\
$^2${\tt \{danikachen, eliasslcao, harryyfhu, felixzhuo, gavinzli\}@tencent.com},\\
$^3${\tt gavinwang@std.uestc.edu.cn}, $^4${\tt \{xlbryant, Hansen\}@stu.xjtu.edu.cn},\\
$^5${\tt zihanzhang@buaa.edu.cn}, $^6${\tt qianbuyue@bjcyh.com}
}



% \author{\IEEEauthorblockN{1\textsuperscript{st} Xiangjin Xie}
% \IEEEauthorblockA{
% \textit{Tsinghua University}\\
% xxj20@mails.tsinghua.edu.cn}
% \and
% \IEEEauthorblockN{2\textsuperscript{nd} Yuxin Chen}
% \IEEEauthorblockA{
% \textit{Tencent}\\
% email address or ORCID}
% \and
% \IEEEauthorblockN{3\textsuperscript{rd} Ruipeng Wang}
% \IEEEauthorblockA{
% \textit{University of Electronic Science and Technology of China}\\
% gavinwang@std.uestc.edu.cn}
% \and
% \IEEEauthorblockN{4\textsuperscript{th} Xianli Zhang}
% \IEEEauthorblockA{\textit{dept. name of organization (of Aff.)} \\
% \textit{name of organization (of Aff.)}\\
% email address or ORCID}
% \and
% \IEEEauthorblockN{5\textsuperscript{th} Shilei Cao}
% \IEEEauthorblockA{\textit{dept. name of organization (of Aff.)} \\
% \textit{name of organization (of Aff.)}\\
% email address or ORCID}
% \and
% \IEEEauthorblockN{6\textsuperscript{th} Kai Ouyang}
% \IEEEauthorblockA{\textit{dept. name of organization (of Aff.)} \\
% \textit{name of organization (of Aff.)}\\
% email address or ORCID}
% \and
% \IEEEauthorblockN{6\textsuperscript{th} Zihan zhang}
% \IEEEauthorblockA{\textit{dept. name of organization (of Aff.)} \\
% \textit{name of organization (of Aff.)}\\
% email address or ORCID}
% \and
% \IEEEauthorblockN{6\textsuperscript{th}  Hai-Tao Zheng}
% \IEEEauthorblockA{\textit{dept. name of organization (of Aff.)} \\
% \textit{name of organization (of Aff.)}\\
% email address or ORCID}
% \and
% \IEEEauthorblockN{6\textsuperscript{th} Buyue Qian}
% \IEEEauthorblockA{\textit{dept. name of organization (of Aff.)} \\
% \textit{name of organization (of Aff.)}\\
% email address or ORCID}
% \and
% \IEEEauthorblockN{6\textsuperscript{th} Bo Hu}
% \IEEEauthorblockA{\textit{dept. name of organization (of Aff.)} \\
% \textit{name of organization (of Aff.)}\\
% email address or ORCID}
% \and
% \IEEEauthorblockN{6\textsuperscript{th} Chengxiang Zhuo}
% \IEEEauthorblockA{\textit{dept. name of organization (of Aff.)} \\
% \textit{name of organization (of Aff.)}\\
% email address or ORCID}
% \and
% \IEEEauthorblockN{6\textsuperscript{th} Zang Li}
% \IEEEauthorblockA{\textit{dept. name of organization (of Aff.)} \\
% \textit{name of organization (of Aff.)}\\
% email address or ORCID}
% }

\maketitle

\begin{abstract}
Graph neural networks have been widely used in recent recommender systems, where negative sampling plays an important role.
Existing negative sampling methods restrict the relationship between nodes as either hard positive pairs or hard negative pairs. This leads to the loss of structural information, and lacks the mechanism to generate positive pairs for nodes with few neighbors.
To overcome limitations, we propose a novel soft link-based sampling method, namely MixDec Sampling, which consists of Mixup Sampling module and Decay Sampling module.
The Mixup Sampling augments node features by synthesizing new nodes and soft links, which provides sufficient number of samples for nodes with few neighbors.  
The Decay Sampling strengthens the digestion of graph structure information by generating soft links for node embedding learning.
To the best of our knowledge, we are the first to model sampling relationships between nodes by soft links in GNN-based recommender systems.
Extensive experiments demonstrate that the proposed MixDec Sampling can significantly and consistently improve the recommendation performance of
several representative GNN-based models on various recommendation benchmarks.


%negative sampling usually acts as a critical strategy to obtain negative supervision for training.
%Negative sampling is a way to obtain regular embedding representations for learning high-quality distributed vector representations. In graph representation learning, negative sampling is also a critical part of training high-quality node representations. However, negative sampling simply divides other nodes for a node into binary of positive and negative samples, which makes it difficult to obtain hierarchical representation of nodes embedding.
%However, negative sampling simply binaries the links between nodes into positive or negative, which makes it difficult to model the hierarchical distribution of nodes.

%In this paper, we propose \textbf{MixDec Sampling}, a sampling method that converts the links of nodes from hard to soft to obtain a more hierarchical distribution of nodes includes several methods to optimize the negative sampling of soft links, including: Mixup Samplig, Decay Sampling, MixDec Sampling, which convert the hard links between nodes into soft links so that the sampling and optimization process can be more accurate. These methods construct synthetic samples and corresponding links by setting links for all node pairs on the basis of BFS and Mixup of different links.

%Empirical results using GraphSAGE GCN, GAT on three public benchmarks show that mix sampling significantly outperforms negative sampling. We also conducted ablation experiments for the appropriate loss function of this soft link, and we matched the corresponding loss function for the soft link based on the experiments.
\end{abstract}
\begin{IEEEkeywords} Graph Neural Network, Negative Sampling, Recommendation System. \end{IEEEkeywords}


\section{Introduction}
\label{sec:intro}
\documentclass[../main.tex]{subfiles}
\graphicspath{{../images/}}
\makeatletter
\def\input@path{{../images/}}
\makeatother
\begin{document}
\section{Introduction}
\begin{figure}
\centering
\begin{tikzpicture}
\node[inner sep=0pt] (ws) at (0, 0) {
\includegraphics[height=.4\textwidth, trim={10cm 0 10cm 0},clip]{world_space.png}};
\node[inner sep=0pt] (cs) at (6,0) {\includegraphics[height=.4\textwidth, trim={10cm 1cm 10cm 4cm},clip]{conf_space.png}};
\end{tikzpicture}
\vspace{-5pt}
\label{fig:pbrm_intro}
\caption{\textbf{Left}: Shows world space obstacles as grey spheres. Robots start and goal configuration is colored red and green, respectively. Configurations along the computed path are colored transparent blue. \textbf{Right:} Mapped world space scenario to configuration space. Obstacle region is the grey mesh. Red spheres are collision-free regions computed by the neural SCDF. The optimized shortest path in the convex corridor is the blue curve.}
\vspace{-25pt}
\end{figure}
Motion planning is the problem of finding a collision-free trajectory that connects a given start and goal configuration. The planning takes place in the configuration space of the robot. For single body robots, like mobile robots or drones, the configuration space and the world space are usually the same. This simplifies the planning, since explicit obstacle representations are available which enables geometrical tools like separating hyperplanes, smallest distance to obstacles etc., to be used when designing motion planning algorithms. For multi-body robots like manipulators, the situation is completely different. The world space obstacles are usually mapped to non-convex regions, and to make the problem even harder, the mapping is usually not known. Forming explicit representations of the obstacle region in the configuration space is usually too expensive or intractable. Despite all of this, sampling based planners are used with great success, which mainly is due to their use of implicit representations of the obstacle region. The basic idea is to construct a graph in the configuration space that covers and connects the collision-free region. From this graph, a path can be extracted that connects a given start and goal configuration. The approach is computationally expensive, since the graph is constructed with the smallest geometrical building block available, points, which represents a collision-check. Furthermore, the extracted paths from the graph are non-smooth and jagged due to the stochastic nature of the approach. This adds an additional post-processing step to the process, where the paths are shortcutted and smoothened, before the path can be used for tracking. Clearly a lot of time is invested to form this graph and produce smooth paths. Thus, if the obstacles start to move, then all of this work is done in no use, since all points that make up this graph need to be re-verified, which is simply too time consuming to be done in real time.
\\\\
In this work, we want to address the existing drawbacks of the sampling based planners. Our main contribution is an improved motion planner where each vertex in the graph covers a collision-free region in the form of a sphere instead of a point and where the edges are formed with neighboring intersecting spheres. This representation has the advantage of instead of returning piecewise linear paths, returning a sequence of overlapping spheres, i.e. a convex corridor, that connects a given start and goal configuration, illustrated in Figure \ref{fig:pbrm_intro}. This convex corridor allows us to use convex optimization to produce smooth trajectories, instead of computationally expensive post-processing methods. The representation further allows us to estimate the coverage of the collision-free space, which gives us awareness and feedback in the offline roadmap construction phase. Finally, our representation is simple to adapt to moving obstacles, simply requery for the new radii and recheck for intersections. 
\\\\
The spherical collision-free regions are formed using a signed distance function (SDF), which is a function that returns the smallest distance from an arbitrary point to the boundary of an obstacle. As the name implies, the distance is signed, thus if the point is inside the obstacle it is negative otherwise positive. If the distance is positive, a sphere with radius equal to the distance is guaranteed to cover a collision-free region. Using an SDF in motion planning is not new, but what is novel about our approach is that we express the distance in the configuration space instead of the world space and by doing so allows us to form these convex collision-free regions. We refer to the resulting SDF as a signed configuration distance function (SCDF). Computing an SCDF analytically is non-trivial, our approach is therefore to parameterize the SCDF with a deep neural network and learn the mapping by supervised learning. Our resulting neural SCDF can compute distances for different parameter values of obstacle shapes and we also show how multiple distances can be combined, thus making our approach flexible.
\section{Related work}
Motion planning algorithms can roughly be divided into three families, grid-based, sampling based and optimization based methods. Grid-based methods (GBM) discretize the planning space from which a graph is then compiled. A standard search method is A$^\star$ \citep{a_star}, which is classified as an \textit{informed} search method, since it employs a heuristic function to speed up the search. A$^\star$ guarantees to return an optimal path at the level of discretization used. GBMs usually discretize the planning space by a regular lattice and this limits the GBMs to problems with low dimensionality due to the curse of dimensionality. Thus, GBMs are usually limited to single-body robots where the degrees of freedom (DOF) are low. To overcome the inherent scaling problem with the GBMs, stochastic methods are usually used for multi-body robots. These methods are termed as sampling-based methods (SBM) and core members within this family are the rapidly-exploring random trees (RRT) \citep{rrt} and the probabilistic roadmap (PRM) \citep{prm}. RRT grows a tree from the start configuration and explores the collision-free region in a rapid way until it is able to connect to the goal region. RRT is usually improved by bi-directional planning \citep{rrt_connect}, i.e. an additional tree is grown from the goal configuration and the trees are tested for connection after any tree has been expanded. RRT is a single-query method, thus it searches for a path from scratch each time it is queried. Contrary to this, PRM is a multi-query method, which solves for multiple queries without starting from scratch. PRM does this by creating a roadmap (graph) that covers the collision-free space as an offline step. The graph is then used to solve for multiple queries. PRMs are used in cases where the environment does not change since the extra offline step is too computationally costly and needs to be re-done if the environment is changed. In our work, we address this inherent issue by using a different roadmap representation. Our vertices in the graph cover a collision-free region in the form of spheres and we form the edges by checking for intersecting spheres. If something in the environment changes, we recompute the spheres radii and recheck the intersections, without relying on collision detection. We use a trained neural network to compute the sphere radius, therefore querying for the radius can be done fast, hence our representation enables the PRM for dynamic environments.
\\\\
In the recent decades, optimization based methods (OBM) \citep{chomp, schulman, itomp, stomp} have been introduced as an alternative to SBM for multi-body robots. Like the SBM, the OBMs scale well to higher dimensional problems and produce smoother motion. It is common to use a SDF in the optimization since it is a smooth function, thus enabling gradient-based methods. However, the standard way of expressing the SDF is in world space. The distance therefore needs to be mapped to the configuration space by the forward kinematics. This mapping makes the optimization problem a non-linear program (NLP), which is computationally expensive to solve. Recently, a different approach has been proposed. In \cite{mp_gcs} motion planning is formulated as a convex optimization problem by using the graph of convex sets framework \citep{gcs}. The underlying idea is to decompose the collision-free space into intersecting convex sets from which a convex optimization problem is formulated. In cases where an explicit representation of the obstacles in the configuration space exists, like for single-body robots, creating collision-free convex regions can be done fast \citep{iris}. For multi-body robots, this is non-trivial. Existing work does this successfully \citep{iris_nlp, iris_c} by an optimization based approach, but the methods are still too time consuming to be used in the presence of moving obstacles. Our approach is instead to use deep learning to learn an SDF expressed in the configuration space. With this, we can query for shortest distances to the collision boundary, which allows us to expand spherical regions which are collision-free. Our approach is fast and therefore enables our suggested roadmap planner to be used in dynamic environments.
\\\\
Recent research has focused on learning collision detection \citep{fk_kernel_distance, diffco, graphdistnet} by predicting the signed distance between the robot links and the surrounding obstacles in the world space. The learned SDF is used in trajectory optimization but since the distance is expressed in the world space, the problem becomes an NLP and therefore takes a long time to solve. We take a novel approach and suggest to instead express the signed distance in the configuration space. This allows us to improve the PRM at the same time as it enables convex optimization for trajectory optimization, which runs faster and is more reliable than NLP solvers. In \cite{cspf} a learned signed distance function in the configuration space is proposed similar to our approach. However, their approach is restricted to point cloud representations, while we propose to represent the obstacles as parameterized geometric shapes, e.g. spheres. Furthermore, we also show how to use our learned SCDF to improve an existing roadmap planner.
\section{Problem formulation}
A robot is located in the world space, $\W \subset \R^3 $. The unique location of the robot is given by its configuration $\q \in \C$, where $\C$ is the configuration space. The set of points covered by the robots bodies at a certain configuration is expressed as $\B(\q) \subset \W$. The robot is surrounded by $\NrObst$ obstacles $\O = \bigcup_{i=1}^{\NrObst} \O_i$, where  $\O_i \subset \W$. The representation of the obstacle in the configuration space is the set $\C\O_i = \{\q \in \C \: |\: \B(\q) \cap \O_i \neq \emptyset \}$. The obstacle space is formed as $\Co = \bigcup_{i=1}^{\NrObst} \C \O_i$. The complement is referred to as the free space, $\Cf = \C \setminus \Co$. The path planning problem is a tuple, ($\Cf$, $\qStart$, $\qGoal$), where we want to connect a query pair, consisting of a start, $\qStart$, and goal configuration, $\qGoal$, with a geometric path, $\q(s): [0, 1] \mapsto \Cf$, such that $\q(0)=\qStart$ and $\q(1)=\qGoal$, or report correctly when such a path does not exist.
\end{document}



\section{Preliminaries}
\label{sec:preliminaries}
In this section,  we introduce related concepts including GNN for recommendation, negative sampling, and Mixup \cite{mixup}.

\subsection{Graph Neural Networks for Recommendation}
 Recommendation is the most important technology in many e-commerce platforms, which has evolved from collaborative filtering to graph-based models. Graph-based recommendation represents all users and items by embedding and recommending items with maximum similarity score (by a inner product operation) for a given user. Here, we briefly describe the pipeline of GNN-based representation learning, including aggregation and optimization with negative sampling.

GNNs learn distributed vectors of nodes by leveraging node features and the graph structure. 
The neighborhood aggregation follows the ``message passing'' mechanism, which iteratively updates a node's embedding $h$ by aggregating the embeddings of its neighbors. Formally, the embedding $h_i^l$ of node $i$ in the $l$-th layer of GNN is defined as:
% GNNs use graph structures and node features to learn distributed vectors to represent graph information. Learning follows the "message passing" mechanism of neighborhood aggregation by iteratively updating a node's embedding $h$ by aggregating the embeddings of its neighbors. Formally, the representation $h_k^i$ of node $i$ in the $k$th layer of GNN is defined as:
 \begin{equation} \label{agg}
     \small
     h_i^l= \sigma\left(\text{AGG}\left(h_{i}^{l-1}, h_{j}^{l-1} \mid j \in N_{(i)},W_l\right)\right),
 \end{equation}
where the \(\sigma\) is activation function, $W_l$ denotes the trainable weights at layer l, $N_{(i)}$ denotes all nodes adjacent to $i$, $\text{AGG}$ is an aggregation function implemented by specific GNN model (\eg GraphSAGE, GCN, GAT, \etc), and $h_i^0$ is typically initialized as the input node feature $v_i$.

 
\subsection{Negative Sampling}

Negative sampling \cite{negsamp} is firstly proposed to serve as a simplified version of Noise Contrastive Estimation\cite{NCE}, which is an efficient way to compute the partition function of an unnormalized distribution to accelerate the training of Word2Vec\cite{word2vec}. The GNN has different non-Euclidean encoder layers with the following negative sampling objective:

\begin{equation}\label{negsampling}
    \mathcal{L} = \log(\sigma (e_{v_i}^Te_{v_p}))+\sum^{c}_{j=1}\mathbb{E}_{v_j\sim P_n(v)}  \log(1-\sigma (e_{v_i}^Te_{v_j})),
\end{equation}
where $v_i$ is a node in the graph, $v_p$ is sampled from the positive distribution of node $v_i$, $v_j$ is sampled from the negative distribution of node $v_i$, $e$ represents the embedding of the node, $\sigma$ represents the sigmoid function, $c$ represents the number of negative samples for each positive sample pair. 
%So Negative Sampling are free to simplify NCE as long as the vector representations retain their quality, it is an effective method to calculate the partition function of unnormalized distribution.


\subsection{Mixup}


\textbf{Mixup\cite{mixup}} is an simple yet effective data augmentation method that is originally proposed for image classification tasks. 
Mathematically, let $(x, y)$ denotes a sample of training data, where $x$ is the raw input samples and $y$ represents the one-hot label of $x$, the Mixup generates synthetic training samples $(\tilde{x}, \tilde{y})$  as follows:
\begin{equation}
% \vspace{-0.2cm}
\begin{split}
% \setlength\abovedisplayskip{0cm}
& \tilde{x}=\lambda x_{i}+(1-\lambda) x_{j}, \\
% \setlength\belowdisplayskip{0cm}
% \setlength\abovedisplayskip{0cm}
& \tilde{y}=\lambda y_{i}+(1-\lambda) y_{j}. \\
% \setlength\belowdisplayskip{0cm}
\end{split}
% \vspace{-0.2cm}
\end{equation}
It generates new samples by using linear interpolations to mix different images and their labels.

\section{Method}
\label{sec:method}

\begin{figure*}
	\centering
	\includegraphics[width = \linewidth]{figure/AgentArena.pdf}
	\caption{\textbf{Stock Trading Workflow in \textit{Agent Trading Arena}.} 
	\textbf{Top:} Workflow of a trading day, including preparation, trading, and post-trading reflection. Agents discuss insights in the chat pool, analyze market trends, execute trades, and refine strategies based on performance.  
	\textbf{Bottom:} Example of agents' interactions in the chat pool and dynamic strategy updates.}
	\label{fig:AgentArena}
	\vspace{-3pt}
\end{figure*}

\section{Proposed Method}

% 核心部分visual representation,

To mitigate the influence of human prior knowledge and memory, we designed a closed-loop economic system~\citep{guo2024economics} called the \textit{Agent Trading Arena}, a zero-sum game simulating complex, quantitative real-world scenarios. The simulation workflow is illustrated in \autoref{fig:AgentArena} and further detailed in \autoref{appendix_arena}. In the \textit{Agent Trading Arena}, agents can invest in assets, earn dividends from holding assets, and pay daily expenses using virtual currency. The agent with the highest total return wins the game.

\subsection{Agent Trading Arena}

\paragraph{Structure of Agent Trading Arena.} 

To eliminate external knowledge biases, asset prices are determined by a bid-ask system, reflecting the prices at which buyers and sellers are willing to transact. The system evolves solely based on agents' actions and interactions, without external influences. This design ensures that the outcomes of agents' actions are not immediately apparent but unfold gradually, influenced by other agents' decisions.

To encourage active participation, a dividend mechanism is introduced. There are two primary sources of income in this system: capital gains from asset price differentials and dividends from holding assets. Dividends for each asset are distributed according to a predefined ratio, serving as an implicit anchor for asset prices. Agents holding more low-cost assets receive higher dividends. To prevent passive asset holding until the end of the game, agents must pay a daily capital cost proportional to their total wealth. These expenses are offset by asset dividends, and only agents with sufficient low-cost assets can cover costs. Under the pressure of significant daily expenses, agents must act swiftly and strategically, triggering frequent trades and price fluctuations to stimulate market activity. This dynamic mechanism ensures fairness in the zero-sum game while preventing agents from relying on fixed strategies to find optimal solutions.

\vspace{-3pt}

\paragraph{Agents Learn and Compete in Arena.}

The zero-sum game structure is crucial to eliminating the possibility of a universally optimal strategy. In fixed scenarios with a static optimal solution, agents could rely on predefined rules or memory-based approaches, bypassing adaptive decision-making. The zero-sum game ensures that there is no universally correct solution, with outcomes evolving dynamically based on agent interactions and competition. This design forces agents to continually adapt, learn from feedback, and develop context-dependent strategies, promoting deeper environmental exploration and preventing reliance on static or memory-driven solutions.

In the \textit{Agent Trading Arena}, agents are unaware of implicit rules, except for the objective to maximize their virtual wealth throughout the simulation. To win this zero-sum game, agents must effectively learn from experience, decipher hidden game rules, and develop strategies to counter competitors. This requires the ability to comprehend numerical feedback, formulate enduring strategies, and make informed decisions. Unlike other mathematical reasoning problems, the results of their actions unfold gradually and dynamically. Moreover, agents are easily misled by erroneous information from competitors, hindering their ability to discern strategic cues from competitors' textual data. Importantly, agents remain unaware of these implicit rules, so applying real-world knowledge does not benefit their performance. Therefore, agents must rely on experiential learning to decipher the hidden game rules and ultimately achieve victory.

\subsection{Types of Numerical Data Input}

\paragraph{Limitations of Textual Numerical Data.}

In the \textit{Agent Trading Arena}, the generated stock data is stored in numerical format. When used directly as input to an LLM, the models often struggle to interpret numerical data accurately or make sound decisions. To mitigate this, we convert the data into textual formats~\citep{numerical_text, long_text}, enhancing semantic features and clarifying output requirements to improve the models' understanding. During interactions, the LLMs process stock prices, trading volumes, and market indices presented as textual numerical data.

\begin{figure*}
	\centering
	\includegraphics[width = \linewidth]{figure/v_t.pdf}
	\caption{\textbf{Textual and Visual Representations of Corresponding Inputs and Outputs.} The left images display the agent’s Buy and Sell trading records, daily trade prices, and K-line charts for three stocks. The output from visual inputs (bottom right) captures overall stock trends and long-term behavior, while the output from textual inputs (top right) focuses on specific current prices.}
	\label{textual_visualized}
	\vspace{-3pt}
\end{figure*}

However, this textual approach reveals significant limitations. While the data is presented clearly, LLMs tend to focus excessively on specific values rather than identifying long-term trends or global patterns. They also struggle with understanding correlative relations and percentage changes, limiting their ability to assess differences and identify connections between data points. When analyzing time-series data with complex patterns, LLMs often fixate on individual data points, overlooking overarching relations. This issue is evident in the analysis output in the top-right corner of \autoref{textual_visualized}, where LLMs' focus on individual values impedes their ability to generalize, reducing their capacity to extract meaningful global insights.

Additionally, LLMs often overemphasize recent data while undervaluing historical information, even when prompted to consider its importance. This prevents them from effectively integrating past data and recognizing long-term patterns, complicating their understanding of numerical relations and trends. These challenges highlight the need for improved mechanisms to process numerical relations, identify global trends, and derive deeper insights from textual numerical data.

\vspace{-3pt}

\paragraph{Potential of Visual Numerical Data.}

Since textual numerical data often leads LLMs to focus on local details while neglecting broader relations, we investigated whether visual representations, such as scatter plots, line charts, and bar charts, could help LLMs better understand overall trends, similar to human reasoning. Thus, we transition from textual numerical data inputs to visualized formats ~\citep{storyllava}. As demonstrated in the bottom-right corner of \autoref{textual_visualized}, visual representations enable LLMs to more effectively grasp global trends, patterns, and relations that are often difficult to discern from textual numerical data alone.

These findings highlight the advantages of structured, visual numerical data, indicating that this format allows LLMs to more intuitively and comprehensively understand complex data, better capturing overall fluctuations, whereas text tends to focus on local details. By combining visualization and textual representations, LLMs not only overcome the challenges of relations in time-series data but also demonstrate better performance in identifying long-term trends and global patterns, while still attending to local details.

\subsection{Reflection Module}

We propose a strategy distillation method, illustrated in \autoref{fig:reflection}, that delivers real-time feedback to LLMs by analyzing both descriptive textual and visual numerical data. This enables the generation of new strategies and optimization of action plans. The approach allows agents to evaluate their results, refine strategies, and adapt continuously based on feedback. The process begins with assessing the day’s trajectory memory and associated strategies using an evaluation function. The strategic generation process leverages contrastive analysis of peak and nadir performers from the evaluation phase, creating bidirectional learning signals that inform subsequent iterations. This iterative cycle ensures continuous strategy evolution, fostering sustained improvement in decision-making.

\begin{figure}[t]
	\centering
	\includegraphics[width = \linewidth]{figure/reflection.pdf}
	\caption{\textbf{Design of the Reflection Module.} The process evaluates daily trajectory memory and strategies (top right), then generates new strategies (center) based on evaluation, environmental feedback (bottom right), and feedback from the 5 top- and bottom-performing strategies. Stock visualization (bottom left) enhances reflection, driving continuous improvement.}
	%The process evaluates daily trajectory memory and strategies, generating new strategies based on positive and negative feedback from the top- and bottom-performing strategies. Stock visualizations (bottom left) further enhance the reflection process, reinforcing continuous strategy refinement.}
	\label{fig:reflection}
	\vspace{-3pt}
\end{figure}

% We propose a strategy distillation method, illustrated in \autoref{fig:reflection}, that provides real-time feedback to LLMs by analyzing both descriptive textual and visualized numerical data. This enables the generation of new strategies and the optimization of action plans. The approach allows agents to assess their results, refine strategies, and continuously adapt based on feedback. The process begins by evaluating the day's trajectory memory and associated strategies using an evaluation function. From this assessment, new strategies are generated by selecting the top-performing and lowest-performing strategies, offering both positive and negative feedback. This iterative cycle ensures continuous strategy evolution, driving sustained improvement in decision-making.

The reflection module plays a crucial role in refining strategies by offering real-time feedback. It analyzes both descriptive textual and visual numerical data to generate new strategies and optimize action plans. Within the \textit{Agent Trading Arena}, the reflection module is triggered regularly to consolidate daily trading records and evaluate the effectiveness of strategies, refining both successful and unsuccessful experiences to guide future decisions. Ineffective strategies are stored in a strategy library for future reference, allowing agents to review and learn from past experiences. Further details can be found in \autoref{appendix_arena}.






\section{Experiments}
\label{sec:experiments}
In this section, we empirically compare the proposed algorithm on both sequence windows and time windows with existing methods.
\paragraph{Datasets} For the sequence-based model, we used two synthetic datasets and two cross-language datasets. The statistics of the datasets are provided in Table \ref{table:statistics}:

\begin{table}[t]
    \centering
    \caption{The statistics of the datasets. The datasets satisfy $1 \leq \|\vx\|\|\vy\| \leq R $.}
    \label{table:statistics}
    \begin{tabular}{|c|c|c|c|c|c|}
    \hline
        Dataset & $n$ & $m_x$ & $m_y$ & $N$ & $R$ \\ \hline
        SYNTHETIC(1) & 100,000 & 1,000 & 2,000 & 50,000 & 65 \\ \hline
        SYNTHETIC(2) & 100,000 & 1,000 & 2,000 & 50,000 & 724 \\ \hline
        APR & 23,235 & 28,017 & 42,833 & 10,000 & 773 \\ \hline
        PAN11 & 88,977 & 5,121 & 9,959 & 10,000 & 5,548 \\ \hline
        EURO & 475,834 & 7,247 & 8,768 & 100,000 & 107,840 \\ \hline
    \end{tabular}
\end{table}

\begin{itemize}
    \item Synthetic: The elements of the two synthetic datasets are initially uniformly sampled from the range (0,1), then multiplied by a coefficient to adjust the maximum column squared norm $R$. The X matrix has 1,000 rows, and the Y matrix has 2,000 rows, each with 100,000 columns. The window size is set to 50,000.
    \item APR: The Amazon Product Reviews (APR) dataset is a publicly available collection containing product reviews and related information from the Amazon website. This dataset consists of millions of sentences in both English and French. We structured it into a review matrix where the X matrix has 28,017 rows, and the Y matrix has 42,833 rows, with both matrices sharing 23,235 columns. The window size is 10,000.
    \item PAN11: PANPC-11 (PAN11) is a dataset designed for text analysis, particularly for tasks such as plagiarism detection, author identification, and near-duplicate detection. The dataset includes texts in English and French. The X and Y matrices contain 5,121 and 9,959 rows, respectively, with both matrices having 88,977 columns. The window size is 10,000.
\end{itemize}
We evaluate the time-based model on another real-world dataset:
\begin{itemize}
    \item EURO: The Europarl (EURO) dataset is a widely used multilingual parallel corpus, comprising the proceedings of the European Parliament. We selected a subset of its English and French text portions. The X and Y matrices contain 7,247 and 8,768 rows, respectively, and both matrices share 475,834 columns. Timestamps are generated using the $Poisson$ $Arrival$ $Process$ with a rate parameter of $\lambda=2$. The window size is set to 100,000, with approximately 30,000 columns of data on average in each window.
\end{itemize}

\paragraph{Setup} For the sequence-based model, we compare the proposed hDS-COD and  aDS-COD with EH-COD~\cite{yao2024approximate} and DI-COD~\cite{yao2024approximate}. We do not consider the Sampling algorithm as a baseline, as its performance is inferior to that of EH-COD and DI-CID, as demonstrated in \cite{yao2024approximate}. %The hDS-COD is adjusted by the parameter $\ell$ and the maximum number of levels $L = \log{R}$, where $R$ is the prior estimate of the maximum squared column norm of the dataset. DI-COD similarly requires a prior estimate of $R$ to limit the maximum number of levels $L = \log{(R/\varepsilon})$. In contrast, aDS-COD and EH-COD do not require an estimate of $R$; their error-space balance is controlled by the parameter $\ell = \frac{1}{\varepsilon}$. 
For the time-based model, we compare the proposed hDS-COD and  aDS-COD with EH-COD and the Sampling algorithm since DI-COD cannot be applied to time-based sliding window model. To achieve the same error bound, the maximum number of levels for hDS-COD is set to $L = \log{(\varepsilon NR)}$, and the initial threshold for aDS-COD is set to $1$.

Our experiments aim to illustrate the trade-offs between space and approximation errors. The x-axis represents two metrics for space: final sketch size and total space cost. The final sketch size refers to the number of columns in the result sketches $\mA$ and $\mB$ generated by the algorithm, representing a compression ratio. The total space cost refers to the maximum space required during the algorithm's execution, measured by the number of columns.We evaluate the approximation performance of all algorithms based on correlation errors $\operatorname{corr-err}(\mathbf{X}_W \mathbf{Y}_W^\top, \mathbf{A} \mathbf{B}^\top)$, which is reflected on the y-axis. Every 1,000 iterations, all algorithms query the window and record the average and maximum errors across all sampled windows.

The experiments for all algorithms were conducted using MATLAB (R2023a), with all algorithms running on a Windows server equipped with 32GB of memory and a single processor of Intel i9-13900K.

\paragraph{Performance} Figure \ref{fig:error vs l} and Figure \ref{fig:error vs space} illustrate the space efficiency comparison of the algorithms on sequence-based datasets. Panels (a-d) show the average errors across all sampled windows, while panels (e-h) display the maximum errors.

Figure \ref{fig:error vs l} evaluates the compression effect of the final sketch. The hDS-COD, aDS-COD, and EH-COD show similar compression performances. But the DS series is more stable, particularly on the synthetic datasets, where they significantly outperform EH-COD and DI-COD. The performance of hDS-COD and aDS-COD is nearly the same, indicating that the adaptive threshold trick in aDS-COD does not have a noticeable negative impact on it, maintaining the same error as hDS-COD.

Figure \ref{fig:error vs space} measures the total space cost of the algorithms. hDS-COD and aDS-COD show a significant advantage over existing methods, as they can achieve the  $\varepsilon$-approximation error with much less space. For the same space cost, the correlation errors of hDS-COD and aDS-COD are much smaller than those of EH-COD and DI-COD. Also, aDS-COD has better space efficiency than hDS-COD because aDS only uses a single-level structure while hDS requires $\log R+1$ levels. We find that hDS-COD requires more space on  SYNTHETIC(2) dataset compared to SYNTHETIC(1) dataset. This phenomenon occurs because SYNTHETIC(2) dataset has a larger $R$, which confirms the dependence on $R$ as stated in Theorem~\ref{thm:hds}. 

Figure \ref{fig:time-based} compares the performance of algorithms on time-based windows. Panels (a) and (b) present the error against the final sketch size, which show that our aDS-COD and hDS-COD algorithms enjoy similar performance as EH-COD and significantly outperform the sampling algorithm. On the other hand, as shown in panels (c) and (d), our methods outperform baselines in terms of total space cost.



\section{Related Work}
\label{sec:Background}
\section{Related Work}

\subsection{View-Dependent Control}
View-dependent representations have a long history in computer graphics.
In his pioneering work, Rademacher proposed interpolating between \textit{key viewpoints} and associated \textit{key deformations} to manipulate 3D models~\cite{rademacher1999view}.
Other researchers have extended the idea to create 3D animation systems~\cite{10.1111:j.1467-8659.2004.00772.x}, streamline the modeling process~\cite{DBLP:journals/corr/abs-2103-15472}, and integrate physical simulation\cite{koyama2013view}.
Of particular note, Rivers et al.~\cite{rivers25Dcartoonmodels} introduced \textit{2.5D Cartoon Models}, a combination of planar meshes transformed, based upon view angle, so as to appears three dimensional.
Our work draws upon these works but is, to our knowledge, the first work to attempt to use view-dependent techniques to retarget 3D motion onto 2D characters.   

\subsection{Animation from 2D Images}

% output is still 2D
Many researchers have proposed different methods for creating animations from 2D images. Hornung et al.~\cite{Hornung2007anim2Dpicmotion} presented a method to deform a character from a photograph given user-provided joint annotations.
\textit{Toonsynth}~\cite{Dvoroznak18-SIG} and \textit{Neural Puppet}~\cite{poursaeed2020neural} both present methods to create new images of hand-drawn characters from examples.
% output is 3D model
Other researchers have proposed methods of obtaining 3D geometry from 2D sketches~\cite{igarashi2006teddy, Dvoroznak20-SA} and images~\cite{ArtiSketch,weng2019photo}.
% focus on sketches specifically
A number of works have specifically focused on childlike drawings.
Lingens et al.~\cite{lingens2020towards} proposed an evolutionary algorithm for animating children's drawings. 
\textit{MagicToon}~\cite{feng2017magictoon} creates a 3D model from childlike drawings for AR applications.
Similar to our work, Smith et al.~\cite{SmithHodgins} proposed a method for animating childlike drawings using 3D skeletal motion. 
However, the resulting animations are only suitable for use in 2D applications and the type of motions it supports are limited.

Unlike these previous works, our solution can be used in 3D contexts but does not create a 3D model. We instead relying upon a view-dependent formulation of the animated character.



\section{Conclusion}
\label{sec:conclusion}
\section*{Conclusion}
This paper aims to enhance our understanding of the computational complexity of computing various Shapley value variants. We found that for various ML models --- including decision trees, regression tree ensembles, weighted automata, and linear regression --- both local and global interventional and baseline SHAP can be computed in polynomial time under HMM modeled distributions. This extends popular algorithms, such as TreeSHAP, beyond their empirical distributional scope. We also establish strict complexity gaps between the various SHAP variants (baseline, interventional, and conditional) and prove the intractability of computing SHAP for tree ensembles and neural networks in simplified scenarios. Overall, we present SHAP as a versatile framework whose complexity depends on four key factors: \begin{inparaenum}[(i)] \item model type, \item SHAP variant, \item distribution modeling approach, \item and local vs. global explanations\end{inparaenum}. We believe this perspective provides deeper insight into the computational complexity of SHAP, paving the way for future work.




%We believe that our framework provides a more intricate understanding of SHAP computation complexity across different models, distributions, and variants, paving the way for further research.

Our work opens promising directions for future research. First, expanding our computational analysis to other SHAP-related metrics, such as asymmetric SHAP~\citep{frye20} and SAGE~\citep{covert2020understanding}, would be valuable. Additionally, we aim to explore more expressive distribution classes and relaxed assumptions beyond those in Section \ref{sec:tractable} while maintaining tractable SHAP computation. Finally, when exact computation is intractable (Section \ref{sec:intractable}), investigating the approximability of SHAP metrics through approximation and parameterized complexity theory~\citep{downey2012parameterized} is an important direction.

%Our work opens several promising avenues for future research on the computational properties of explainable AI methods, with a particular focus on SHAP. First, it would be interesting to broaden the computational analysis conducted in this work to include other popular SHAP-related metrics in the literature, such as asymmetric SHAP \cite{frye20} and SAGE \cite{covert2020understanding}. Also, in the future, we aim to explore more expressive distribution classes and relaxed distributional assumptions—extending beyond those examined in Section \ref{sec:tractable} —that still yield tractable SHAP computation. Finally, when exact computation proves intractable (Section \ref{sec:intractable}), it is worthwhile to theoretically investigate the question of the approximability of computing the SHAP metrics across various configurations, through the lens of approximation and parametrized complexity theory \cite{arora2009computational}.

%This paper aims to deepen our understanding of the computational complexity involved in obtaining different Shapley value variants. We found that for a variety of ML models, including decision trees, tree ensembles for regression, weighted automata, and linear regression models — computing both local and global interventional and baseline SHAP can be done in polynomial time when distributions are modeled by HMMs. This extends the distributional scope of popular algorithms like TreeSHAP, which is limited to empirical distributions. Additionally, we demonstrate a strict complexity gap between SHAP variants, showing that interventional and baseline SHAP can be strictly easier to compute than conditional SHAP. Despite these positive results, we uncovered intractability for various SHAP variants in neural networks and tree ensembles. Finally, we provided generalized complexity relations across SHAP variants. We believe that our framework offers a deeper understanding of the complexity involved in computing SHAP across various variants, models, distributions, as well as in both local and global computations, laying the groundwork for future research.

\section{Acknowledge}
This research is supported by National Natural Science Foundation of China (Grant No.62276154 and 62011540405), Beijing Academy of Artificial Intelligence (BAAI), the Natural Science Foundation of Guangdong Province (Grant No. 2021A1515012640), Basic Research Fund of Shenzhen City (Grant No. JCYJ20210324120012033 and JCYJ20190813165003837), and Overseas Cooperation Research Fund of Tsinghua Shenzhen International Graduate School (Grant No. HW2021008).


\bibliographystyle{ieeetr}
\bibliography{ref}

\end{document}
