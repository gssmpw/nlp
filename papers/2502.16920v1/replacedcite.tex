\section{Related Work}
\subsection{Multi-Party Dialogue Structural Analysis}
The task of predicting relationships between speakers to analyze the structure of MPC began in 2016. ____ first proposed the addressee prediction and utterance selection task. To study this task, they first hand-created a dataset of MPC using log transcripts from Ubuntu IRC channels, and then utilized RNNs to perform the proposed task. Later, ____ proposed SI-RNN, which updates speaker embeddings based on roles for addressee prediction. ____ also proposed a speaker classification task to model the relationships between speakers. Meanwhile, for the MPC response selection task, ____ proposed to track dynamic topics, and then a who-to-whom (W2W) model ____ was proposed to predict the addressees of all utterances in a conversation. ____ proposed the MPC-BERT model, which utilizes multiple MPC learning methods to learn the complex interactions between recent utterances and interlocutors, and it performs post-training for MPC tasks. In addition, ____ proposed the GIFT model to help fine-tuning for MPC tasks with only simple scalar parameters on the attentions.

However, all the methodologies proposed in the above works have the limitation that they utilize the utterances to predict the addressee information of each utterance. This makes it difficult to utilize these methodologies for response generation models in real-world MPC environments.

\subsection{Multi-Party Dialogue Response Generation}
Along with these MPC tasks, there has been a parallel research on MPC response generation, which is the task of generating responses to a multi-party dialog. ____ proposed a graph structure network (GSN) to model the graphical information flow for response generation. Later, Heter-MPC ____ was proposed to model complex interactions between utterances and interlocutors as graphs. This paper used graphs with two types of nodes and six types of edges to model the structure of multi-party conversations. ____ utilized the Expectation-Maximization (EM) algorithm in pre-training to predict the missing addressee information in the dataset. However, they still suffer from the drawback that the fine-tuning process only allows for an ideal setup where all addressees are labeled. To overcome this, MADnet ____ utilizes the EM algorithm in the model of ____ to directly predict and supplement the missing addressee information for training and response generation.

\begin{figure*}[t!]
    \centering
    \includegraphics[width=\textwidth]{figures/overview.png}
    \caption{The overview of the SS-MPC. The encoder part is expected to analyze the dialogue and predict the structural information in dialogue. The decoder part is expected to generate the final response with using the information analyzed in encoder.}
    \label{fig:overview}
\end{figure*}