
\begin{lem}%
    Let $\Pi$ be a set of constraints in the variables $x_1, \ldots, x_i$. $\Pi_i = \Pi$ and $\Pi_{i-1}$ are equisatisfiable, and each assignment to the variables $x_1, \ldots, x_{i-1}$ satisfying $\Pi_{i-1}$ can be extended in order to satisfy $\Pi_i$. 
\end{lem}

\begin{proof}

    Clearly, given the soundness of the CP resolution rule, if $\Pi_i$ is satisfiable, then also $\Pi_{i-1}$ is satisfiable (each constraint in $\Pi_{i-1}$ and not in $\Pi_i$ is entailed by $\Pi_i$). 
    
    It remains to show that if $\sample^{:i}$ is an assignment to the variables $x_1,\ldots,x_{i-1}$ satisfying $\Pi_{i-1}$, the set of constraints $\sample^{:i}(\Pi_i)$ is satisfiable. Similarly to the notation used in the proof of lemma~\ref{lemma:soundness} in Appendix~\ref{app:proof_soundness}, given a set of constraints $\Pi$, the expression $\sample^{:i}(\Pi)$ denotes the set of constraints in the variable $x_i$ obtained by substituting each variable $x_j$ ($j < i$) with the corresponding value $\sample^{:i}_j$ in the constraints in $\Pi$.

    Assume $\sample^{:i}(\Pi_i)$ is not satisfiable. Then, there exist 
two constraints $\Psi$ and $\Psi'$ in $\sample^{:i}(\Pi_i)$ equivalent to $(x_i \ge r_i)$ and $(x_i \le l_i)$, respectively, and 
    \begin{enumerate}
        \item either $l_i < r_i$, 
        \item or $l_i \ge r_i$ and there exists $n \ge 1$ constraints $\{\Psi_1,\Psi_2,\ldots,\Psi_n\}$ in $\sample^{:i}(\Pi_i)$ with each $\Psi_j$ equivalent to
        $(x_i \le l^{\Psi_j}_i) \vee (x_i \ge r^{\Psi_j}_i)$ and $l^{\Psi_1}_i,l^{\Psi_2}_i,\ldots,l^{\Psi_n}_i, r^{\Psi_1}_i, r^{\Psi_2}_i,\ldots,r^{\Psi_n}_i$ 
        such that $l^{\Psi_1}_i < r_i \le r^{\Psi_1}_i$, $l^{\Psi_2}_i < r^{\Psi_1}_i \le r^{\Psi_2}_i$, \ldots, $l^{\Psi_n}_i < r^{\Psi_{n-1}}_i \le l_i < r^{\Psi_n}_i$ and thus $l^{\Psi_1}_i < r_i \le l_i < r^{\Psi_n}_i$. 
    \end{enumerate}
    However, $l_i < r_i$ is not possible because $\CPres_i(\Psi,\Psi')$ belongs to $\sample^{:i}(\Pi_{i-1})$
    and is equivalent to $(r_i \le l_i)$. 
    Regarding the second case, 
    $\sample^i(\Pi_i^\plusplus)$ contains the constraints ( $\equiv$ denotes logical equivalence)
    \begin{gather*}
    \Upsilon_1 = \CPres_i(\Psi,\Psi_1) \equiv (x_i \ge r^{\Psi_1}_i) \vee (r_i \le l_i^{\Psi_1})  \equiv  x_i \ge r^{\Psi_1}_i,\\
    \Upsilon_2 = \CPres_i(\Upsilon_1,\Psi_2) \equiv (x_i \ge r^{\Psi_2}_i) \vee (r_i^{\Psi_1} \le l_i^{\Psi_2}) \equiv x_i \ge r^{\Psi_2}_i, \\
    \ldots, \\
    \Upsilon_n = \CPres_i(\Upsilon_{n-1},\Psi_n)  \equiv x_i \ge r^{\Psi_n}_i,
    \end{gather*}
    and thus $\sample^{:i}(\Pi_{i-1})$ contains  $\CPres_i(\Upsilon_{n},\Psi') \equiv r^{\Psi_n}_i \le l_i$, thus reaching a contradiction.
\end{proof}
