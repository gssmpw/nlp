
\newtheorem*{lem}{Lemma}

\begin{lem}%
    Let $\Pi$ be a finite and satisfiable set of constraints in a single variable $x_i$. For every sample $\sample$, $\lsymb(\sample)$ satisfies $\Pi$ and is optimal wrt $\sample$.
\end{lem}

\begin{proof}\rm
    We first prove that for every sample $\sample$, $\lsymb(\sample)_i$ always satisfies $\Pi$, and then that for every sample $\sample$, $\lsymb(\sample)_i$ is the solution of $\Pi$ with minimal Euclidean distance from $\sample$.
    
    Suppose there exists a sample $\sample$ such that $\lsymb(\sample) \not \in \Omega(\Pi)$. This entails (i) that $\Pi \not = \emptyset$ and (ii) that $\sample \not\in \Omega(\Pi)$. Since $\Pi \not = \emptyset$ and $\Pi$ is satisfiable, $\leftp(\sample) \neq -\infty$ or $\rightp(\sample) \neq + \infty$, and $\lsymb(\sample)_i = \leftp(\sample)$ or $\lsymb(\sample)_i = \rightp(\sample)$. Since by definition  $\leftp(\sample)$ and $\rightp(\sample)$ satisfy $\Pi$ we reached a contradiction.

    Assume  
     $\sample \not \in \Omega(\Pi)$ (otherwise we would have again  $\lsymb(\sample) = \sample$ and the thesis would trivially hold). 
    Let $d$ be the minimum Euclidean distance between any point in $\Omega(\Pi)$ and $\sample$. Let $r$ and $l$ be the two samples with $r_k = l_k = \sample_k = \lsymb(\sample)_k$ when $k \not = i$ and $k \in \{1, \dots, D\}$, 
    $r_i = \sample_i + d$ and $l_i = \sample_i - d$. Either $r$ or $l$ or both belong to $\Omega(\Pi)$. Let $v$ be $r$ if $r \in \Omega(\Pi)$, and $l$ otherwise.
    By definition, $v \in \Omega(\Pi)$ and is optimal wrt $\sample$.
    Assume $v = l$.
     Then, from the optimality of $v$, we have that for every $v'$ with
    $v'_i \in (v_i, \sample_i + d)$, $v' \not \in \Omega(\Pi)$. Hence, there must exist a constraint $\Psi$ such that $v_i = \leftb$ and thus $v_i = \lsymb(\sample)_i$.
    Analogously for the case $v = r$.


    
    


\end{proof}