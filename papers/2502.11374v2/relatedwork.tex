\section{Related Work}
\subsection{Social Recommendation}
To improve the accuracy of recommendation results, numerous social recommendation methods have been developed, incorporating online social relationships between users as side information~\cite{guo2015trustsvd,li2021spex}. Early models like SoReg~\cite{ma2011recommender} and SocialMF~\cite{jamali2010matrix} integrate social connections as regularization terms or utilize trust relationships to project users into latent representations. In recent years, graph neural networks (GNNs)~\cite{kipf2016semi} have achieved great success in deep learning, owing to their strong capability on modeling graph data. DiffNet~\cite{wu2019neural} and its extension DiffNet++~\cite{wu2020diffnet++} model the information diffusion process in social graphs to enlarge the users’ influence scope. Multi-channel hypergraph convolutional network is employed on MHCN~\cite{yu2021self} to enhance social recommendation by leveraging high-order user relations. Self-supervised learning (SSL) is utilized in SEPT~\cite{yu2021socially} to improve social recommendation by uncovering supervisory signals from two complementary views of raw data. DESIGN~\cite{tao2022revisiting} introduces knowledge distillation between models that rely on different data sources to leverage social information effectively. A universal denoised self-augmented learning framework (DSL)~\cite{wang2023denoised} incorporates social influence to decipher user preferences while mitigating noisy effects. Nevertheless, these methods mostly aim to improve accuracy while neglecting diversity. Our work contributes to achieving a balance between accuracy and diversity in social recommendation.
 
\subsection{Diversified Recommendation}
Diversified recommendation aims to provide users with a more varied set of items, enabling users to discover new and unexplored interests~\cite{chen2020improving,liang2021enhancing}. The accuracy-diversity dilemma, pointing higher accuracy often means losing diversity to some extent and vice versa. A classical re-ranking work to enhance diversity is maximal marginal relevance (MMR)~\cite{carbonell1998use}, which uses the notion of marginal relevance to combine relevance and diversity with a trade-off parameter. Determinantal point process (DPP)~\cite{chen2018fast} re-ranks items to achieve the largest determinant on the item’s similarity matrix. DGCN~\cite{zheng2021dgcn} selects node neighbors based on the inverse category frequency for diverse aggregation and further utilizes category-boosted negative sampling and adversarial learning to diverse items in the embedding space. DGRec~\cite{yang2023dgrec} targets diversifying GNN-based recommender systems with diversified embedding generation. Different from these methods, our work facilitates seamless integration with various social recommendation systems, effectively enhancing accuracy while maintaining high diversity.