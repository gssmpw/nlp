
\documentclass[a4paper,USenglish,cleveref, autoref, thm-restate]{lipics-v2021}
%This is a template for producing LIPIcs articles. 
%See lipics-v2021-authors-guidelines.pdf for further information.
%for A4 paper format use option "a4paper", for US-letter use option "letterpaper"
%for british hyphenation rules use option "UKenglish", for american hyphenation rules use option "USenglish"
%for section-numbered lemmas etc., use "numberwithinsect"
%for enabling cleveref support, use "cleveref"
%for enabling autoref support, use "autoref"
%for anonymousing the authors (e.g. for double-blind review), add "anonymous"
%for enabling thm-restate support, use "thm-restate"
%for enabling a two-column layout for the author/affilation part (only applicable for > 6 authors), use "authorcolumns"
%for producing a PDF according the PDF/A standard, add "pdfa"

%\pdfoutput=1 %uncomment to ensure pdflatex processing (mandatatory e.g. to submit to arXiv)
%\hideLIPIcs  %uncomment to remove references to LIPIcs series (logo, DOI, ...), e.g. when preparing a pre-final version to be uploaded to arXiv or another public repository

%\graphicspath{{./graphics/}}%helpful if your graphic files are in another directory

\usepackage{todonotes}
\usepackage{xfrac}
\usepackage{tikz}
\usepackage[normalem]{ulem}
\usetikzlibrary{arrows}
\usetikzlibrary{arrows.meta}

\bibliographystyle{plainurl}% the mandatory bibstyle

\title{Generalized De Bruijn Words, Invertible Necklaces, and the Burrows--Wheeler Transform} %TODO Please add

\titlerunning{Generalized De Bruijn Words, Invertible Necklaces, and the BWT} %TODO optional, please use if title is longer than one line


%\author{Anonymous}{ }{}{}{}

 \author{Gabriele Fici}{Dipartimento di Matematica e Informatica, Università di Palermo, Palermo, Italy \and \url{https://www.unipa.it/persone/docenti/f/gabriele.fici/en/} }{gabriele.fici@unipa.it}{https://orcid.org/0000-0002-3536-327X}{}

 \author{Estéban Gabory}{Dipartimento di Matematica e Informatica, Università di Palermo, Palermo, Italy \and \url{https://www.unipa.it/persone/docenti/g/esteban.gabory/en} }{esteban.gabory@unipa.it}{https://orcid.org/0000-0002-9897-1512}{}%TODO mandatory, please use full name; only 1 author per \author macro; first two parameters are mandatory, other parameters can be empty. Please provide at least the name of the affiliation and the country. The full address is optional. Use additional curly braces to indicate the correct name splitting when the last name consists of multiple name parts.


%\authorrunning{Anonymous}

 \authorrunning{Gabriele Fici and Estéban Gabory} %TODO mandatory. First: Use abbreviated first/middle names. Second (only in severe cases): Use first author plus 'et al.'

\Copyright{Gabriele Fici and Estéban Gabory} %TODO mandatory, please use full first names. LIPIcs license is "CC-BY";  http://creativecommons.org/licenses/by/3.0/

\ccsdesc[100]{Mathematics of computing → Combinatorics; Mathematics of computing → Combinatorial algorithms} %TODO mandatory: Please choose ACM 2012 classifications from https://dl.acm.org/ccs/ccs_flat.cfm 

\keywords{Burrows--Wheeler Transform, Generalized de Bruijn Word, Generalized de Bruijn Graph, Circulant Matrix, Invertible Necklace, Sandpile Group, Reutenauer Group.} %TODO mandatory; please add comma-separated list of keywords

\category{} %optional, e.g. invited paper

\relatedversion{} %optional, e.g. full version hosted on arXiv, HAL, or other respository/website
%\relatedversiondetails[linktext={opt. text shown instead of the URL}, cite=DBLP:books/mk/GrayR93]{Classification (e.g. Full Version, Extended Version, Previous Version}{URL to related version} %linktext and cite are optional

%\supplement{}%optional, e.g. related research data, source code, ... hosted on a repository like zenodo, figshare, GitHub, ...
%\supplementdetails[linktext={opt. text shown instead of the URL}, cite=DBLP:books/mk/GrayR93, subcategory={Description, Subcategory}, swhid={Software Heritage Identifier}]{General Classification (e.g. Software, Dataset, Model, ...)}{URL to related version} %linktext, cite, and subcategory are optional

%\funding{(Optional) general funding statement \dots}%optional, to capture a funding statement, which applies to all authors. Please enter author specific funding statements as fifth argument of the \author macro.

%\acknowledgements{I want to thank \dots}%optional

%\nolinenumbers %uncomment to disable line numbering
\newcommand{\red}[1]{\textcolor{red}{#1}}
%\newcommand{\red}[1]{#1} 

%Editor-only macros:: begin (do not touch as author)%%%%%%%%%%%%%%%%%%%%%%%%%%%%%%%%%%
\EventEditors{John Q. Open and Joan R. Access}
\EventNoEds{2}
\EventLongTitle{42nd Conference on Very Important Topics (CVIT 2016)}
\EventShortTitle{CVIT 2016}
\EventAcronym{CVIT}
\EventYear{2016}
\EventDate{December 24--27, 2016}
\EventLocation{Little Whinging, United Kingdom}
\EventLogo{}
\SeriesVolume{42}
\ArticleNo{23}


\def\dd{\mathinner{.\,.}}
\newcommand{\BWT}{\textsc{BWT}}
\newcommand{\gDBw}{ generalized de Bruijn word}
\newcommand{\bin}{\Sigma_2}
\newcommand{\RG}{RG}
\newcommand{\Z}{\mathbb{Z}}
\DeclareMathOperator{\DB}{DB}
\DeclareMathOperator{\GF}{\mathbb{F}}
\DeclareMathOperator{\ord}{ord}
\DeclareMathOperator{\SNF}{SNF}
\DeclareMathOperator{\Lyn}{\overline{N}}
\DeclareMathOperator{\Neck}{N}
\DeclareMathOperator{\InvNeck}{\hat{N}}
\DeclareMathOperator{\DBW}{DBW}
\DeclareMathOperator{\CM}{\mathcal{C}}
\newcommand{\fctn}[5] {
	#1: 
	\left\{
	\begin{array}{rcl}
		#2 & \to & #3 \\ 
		#4 & \mapsto & #5
	\end{array} 
	\right.
}

\newcommand{\anfctn}[4] {
	\left\{
	\begin{array}{rcl}
		#1 & \to & #2 \\ 
		#3 & \mapsto & #4
	\end{array} 
	\right.
}

\DeclareMathOperator{\Perm}{\mathfrak{S}}

\begin{document}

\nolinenumbers

\maketitle

%TODO mandatory: add short abstract of the document
\begin{abstract}
We define generalized de Bruijn words, as those words having a  Burrows--Wheeler transform that is a concatenation of permutations of the alphabet. We show how to interpret generalized de Bruijn words in terms of Hamiltonian cycles in the  generalized de Bruijn graphs introduced in the early '80s in the context of network design. When the size of the alphabet is a prime, we give relations between generalized de Bruijn words, normal bases of finite fields, invertible circulant matrices, and Reutenauer groups. In particular, we highlight a correspondence between binary de Bruijn words of order $d+1$, binary necklaces of length $2^{d}$ having an odd number of $1$s, invertible BWT matrices of size $2^{d}\times 2^{d}$, and normal bases of the finite field $\GF_{2^{2^{d}}}$.
\end{abstract}

\section{Introduction}
\label{sec:intro}

The Burrows--Wheeler matrix of a word $w$ is the matrix whose rows are the conjugates of $w$ in ascending lexicographic order. Its last column is called the  Burrows--Wheeler transform of $w$, and has the property that it results easier to compress when the input word is a repetitive text. For this reason, it is largely used in textual data compression. But the  Burrows--Wheeler transform has also many interesting combinatorial properties (see~\cite{DBLP:conf/cie/RosoneS13} for a survey). Since the Burrows--Wheeler transform is the same for any conjugate of $w$, it can be viewed as a map from the set of necklaces (conjugacy classes of words) to the set of words.

A \emph{de Bruijn word} of order $d$ over an alphabet $\Sigma$ is a necklace of length $|\Sigma|^d$ such that each of the $|\Sigma|^d$ distinct words of length $d$ occurs exactly once in it. Since each of the $|\Sigma|^{d-1}$ distinct words of length $d-1$ occurs in $|\Sigma|$ consecutive rows of the Burrows--Wheeler matrix of a de Bruijn word, the  Burrows--Wheeler transform of a de Bruijn word is characterized (among words that are images under the Burrows--Wheeler transform) by the fact that it is a concatenation of $|\Sigma|^{d-1}$ alphabet-permutations (de Bruijn words of order $1$). This motivates us to define a \emph{generalized de Bruijn word} as one whose Burrows--Wheeler transform is a concatenation of (any number of) alphabet-permutations.

%\red{As it is well known, de Bruijn words of order $d$ over an alphabet $\Sigma$ are in $1$-to-$1$ correspondence with Hamiltonian cycles of the \emph{de Bruijn graph} of order $d$ over $\Sigma$ (or equivalently, Hamiltonian cycles of the de Bruijn graph of order $d-1$ over $\Sigma$), which is the directed graph having vertices set $V=\Sigma^d$ and such that there is an edge from $u$ to $v$ if $u[1,\dd d-1]=v[0,\dd d-2]$. Equivalently, this graph can be defined on the set of vertices $V=\{0,\dd |\Sigma|^{d}-1\}$, having an edge from $v$ to $u$ if and only if $u=|\Sigma|v+i$ with $i\in [0,\dd |\Sigma|-1]$.}

As it is well known, de Bruijn words are in $1$-to-$1$ correspondence with Hamiltonian cycles in de Bruijn graphs. We show that, analogously,  generalized de Bruijn words are in $1$-to-$1$ correspondence with Hamiltonian cycles in \emph{generalized de Bruijn graphs}, introduced in the early $80$'s independently by Imase and Itoh~\cite{DBLP:journals/tc/ImaseI81}, and by Reddy, Pradhan, and Kuhl~\cite{Reddy} (see also~\cite{DBLP:journals/networks/DuH88}) in the context of network design. The generalized de Bruijn graph $\DB(k,n)$  has vertices $\{0,1,\ldots,n-1\}$ and for every vertex $m$ there is an edge from $m$ to $km+i \mod (n)$ for every $i=0,1,\ldots,k-1$.

In particular, we provide a simple interpretation of the well-known correspondence between de Bruijn words and Hamiltonian cycles in de Bruijn graphs in terms of the \emph{inverse standard permutation} of the  Burrows--Wheeler transform, and we show that this interpretation extends to the generalized case.

%show that the Hamiltonian cycles of the generalized de Bruijn graphs are precisely the inverse standard permutations, written as cycles, of the Burrows--Wheeler transforms of the generalized de Bruijn words.}

%We show that generalized de Bruijn words are in $1$-to-$1$ correspondence with Hamiltonian cycles in  \emph{generalized de Bruijn graphs}, introduced in the early $80$'s independently by Imase and Itoh~\cite{DBLP:journals/tc/ImaseI81}, and by Reddy, Pradhan, and Kuhl~\cite{Reddy}.  For every $k$ and $n$, the generalized de Bruijn graph $\DB(k,n)$ has vertices $\{0,1,\ldots,n-1\}$ and for every vertex $m$ there is an edge from $m$ to $km+i \mod (n)$ for every $i=0,1,\ldots,k-1$ (see Fig.~\ref{fig:DB36} for an example). We show that the Hamiltonian cycles of the generalized de Bruijn graphs are precisely the inverse standard permutations, written as cycles, of the Burrows--Wheeler transforms of the generalized de Bruijn words.

When the size of the alphabet is a prime $p$, we make a bridge between generalized de Bruijn words  and \emph{invertible necklaces}. We call an aperiodic necklace over an alphabet of size $p$ invertible if its Burrows--Wheeler matrix is nonsingular, i.e., has determinant nonzero modulo $p$. Invertible necklaces can also be represented by their circulant matrix, and they form an Abelian group, called the \emph{Reutenauer group} $\RG_p^{n}$. Using a known result in abstract algebra~\cite{CTR01}, we show that if every aperiodic necklace of length $n$ with non-zero weight modulo $p$ (the weight of a word is the sum of its digits) is invertible, then $n$ is either a power of $p$ or a $p$-rooted prime. As a consequence, the famous Artin's conjecture on $p$-rooted primes can be restated as follows: There are infinitely many lengths $n$, different from a power of $p$, for which every aperiodic necklace of length $n$ with non-zero weight modulo $p$ has invertible Burrows--Wheeler matrix.

Chan, Hollmann and Pasechnik~\cite{ECCGTA13} proved that for every prime $p$ and any $n$, one has  $\RG_p^{n} \cong K(\DB(p,n)) \oplus \Z_{p-1}$, where $K(G)$ denotes the sandpile group of the graph $G$. Since the structure of the sandpile group of an Eulerian graph is determined by the invariant factors of the Smith Normal Form of its Laplacian matrix, Chan et al.~managed to give the precise structure of Reutenauer groups and sandpile groups of generalized de Bruijn graphs.

In particular, when $p=2$ and $n=2^d$, we have a bijection between de Bruijn words of order $d$ and binary necklaces of length $2^{d-1}$ with an odd number of $1$s; and, moreover, the two sets have the same structure as Abelian groups. Our hope is that this can shed some light  on questions about words with invertible Burrows--Wheeler matrix that emerged recently~\cite{DBLP:journals/corr/abs-2409-07974,DBLP:journals/corr/abs-2409-09824}.

%%%%%%%%%%%%%%%%%%%%%%%%%%%%%%%%%%%%%%%%%%%%%%%%%%%%%%%%%%%%%%%%%%%%%%%%%%%%%%%%%%%%%%
\section{Preliminaries}
%%%%%%%%%%%%%%%%%%%%%%%%%%%%%%%%%%%%%%%%%%%%%%%%%%%%%%%%%%%%%%%%%%%%%%%%%%%%%%%%%%%%%%


Let $\Sigma_k=\{0,1,\ldots,k-1\}$, $k>1$.  A \emph{word} over  $\Sigma_k$ is a concatenation of elements of $\Sigma_k$. The \emph{length} of a word $w$ is denoted by $|w|$. %The empty word $\varepsilon$ has length $0$. 
% The set of all words 
%  % (resp., all nonempty words)
%  over  $\Sigma_k$ is denoted  $\Sigma_k^*$
%  % (resp.,~by $\Sigma^+$)
%  and is a free monoid
%  % (resp.,~a free semigroup)
%  with respect to concatenation. %, while the set of all words of length equal to $n\geq 0$ over  $\Sigma$ is denoted by $\Sigma^n$. 
 For a letter $i\in\Sigma_k$, $|w|_i$ denotes the number of occurrences of $i$ in $w$. The vector $(|w|_{0},\ldots,|w|_{k-1})$ is the \emph{Parikh vector} of $w$.

Let  $w=w_0\cdots w_{n-2}w_{n-1}$ be a word of length $n>0$. The \emph{weight} of $w$ is  $\sum_{j=0}^{n-1}w_j$. When $n>1$, the \textit{shift} of $w$ is the word $\sigma(w)=w_{n-1}w_0\cdots w_{n-2}$.

% Let $w=uv$, with $u,v\in \Sigma^*$. We say that $u$ is a \emph{prefix} of $w$ and that $v$ is a \emph{suffix} of $w$. A \emph{factor} of $w$ is a prefix of a suffix (or, equivalently, a suffix of a prefix) of $w$. %The sets of prefixes, suffixes, factors of a word $w$ are denoted, respectively, by $\Pref(w),\Suff(w),\Fact(w)$. We also use  $\Pref_k(w),\Suff_k(w)$ to denote the prefix and the suffix of length $k$ of $w$, respectively.
% If $w=w_0w_1\cdots w_{n-1}$, with $w_j\in\Sigma_k$ for all $j$, we let $w[i..j]$ denote the nonempty factor $w_iw_{i+1}\cdots w_j$ of $w$, whenever $0\leq i\leq j\leq n-1$.
% A factor $u$ of a word $w\neq u$ is a \emph{border} of $w$ if $u$ is both a prefix and a suffix of $w$; in this case, $w$ has \emph{period} $|w|-|u|$. A word $w$ is \emph{unbordered} if its longest border is $\varepsilon$; i.e., if its smallest period is $|w|$. %The prefix $\rho_w$ whose length equals the smallest period of a nonempty $w$ is called its \emph{fractional root}.
For a word $w$, the \emph{$n$-th power} of $w$ is the word $w^n$ obtained by concatenating $n$ copies of $w$. %; furthermore, we let $w^\omega$ denote the infinite periodic word $www\cdots$.

\begin{definition}
    We call \emph{alphabet-permutation}  a word over $\Sigma_k$ that contains each letter of $\Sigma_k$ exactly once, and \emph{alphabet-permutation power} a concatenation of one or more alphabet-permutations over $\Sigma_k$.
\end{definition}

\begin{example}
    The word $w=210201102102120$ is an alphabet-permutation power over $\Sigma_3$.
\end{example}

Notice that an alphabet-permutation power has a  balanced Parikh vector, i.e., a Parikh vector of the form $(\frac{n}{k},\frac{n}{k},\ldots,\frac{n}{k})$.

In the binary case, a word $w$ is an alphabet-permutation power if and only if $w=\tau(v)$, where $v$ is a binary word of length $|w|/2$ and $\tau$ is the \emph{Thue--Morse morphism}, the substitution that maps $0$ to $01$ and $1$ to $10$.


Two words $w$ and $w'$ are \emph{conjugates} if $w=uv$ and $w'=vu$ for some words $u$ and $v$. The conjugacy class of a word $w$ can be obtained by repeatedly applying the shift operator, and contains $|w|$ distinct elements if and only if $w$ is \emph{primitive}, i.e., $w\neq v^n$ for any nonempty word $v$ and $n>1$.
A \emph{necklace} (resp.~\emph{aperiodic necklace}) $[w]$ is a conjugacy class of words (resp.~of \emph{primitive words}). 

\begin{remark}
    All along this paper, we chose as representative of a necklace its lexicographically biggest element.
\end{remark}


For example, $[1100]=\{1100,0110,0011,1001\}$ is an aperiodic necklace, while $[1010]=\{1010,0101\}$ is not aperiodic.

The number of aperiodic necklaces of length $n$ over $\Sigma_k$ is given by
\[\Lyn(k,n)=\frac{1}{n}\sum_{d | n}\mu\left(\frac{n}{d}\right)k^d\]
where $\mu(n)$ is the M\"obius function, i.e., the function: $\mu(1)=1$, $\mu(n)=(-1)^j$ if $n$ is the product of $j$ distinct primes or $0$ otherwise, i.e., if $n$ is divisible by the square of a prime number.

The number of  necklaces of length $n$ over $\Sigma_k$ is given by
\[\Neck(k,n)=\sum_{d|n}\Lyn(k,d)=\frac{1}{n}\sum_{d | n}\phi\left(\frac{n}{d}\right)k^d\]
where $\phi(n)$ is the Euler totient function. Recall that the Euler totient function $\phi(n)$ counts the number of positive integers smaller than $n$ and coprime with $n$, i.e., the order of the multiplicative group $\mathbb{Z}_n^*$, and can be computed using the formula
\[\phi(n)=\prod_{p|n}\left(1-\dfrac{1}{p}\right)\]
for $n>1$, and $\phi(1)=1$, where the product is over the distinct primes dividing $n$.

\bigskip



 The \textit{Burrows--Wheeler matrix} (BWT matrix) of a necklace $[w]$ is the matrix  whose rows are the elements of $[w]$ in ascending\footnote{This is the standard convention in the literature. Of course, one can also chose the descending lexicographic order, the properties are symmetric.} lexicographic order. Let us denote by $F$ and $L$, respectively, the first and the last column of the BWT matrix of $[w]$. We have that $F=0^{n_0}1^{n_1}\cdots (k-1)^{n_{k-1}}$, where $(n_0,\ldots, n_{k-1})$ is the Parikh vector of $[w]$; $L$, instead, is the \emph{Burrows--Wheeler Transform} (BWT) of $[w]$. The BWT is therefore a map from the set of  necklaces to the set of words. As it is well known, it is an injective map (we describe below how to invert it). A word is a \emph{BWT image} if it is the BWT of some  necklace. 

The \emph{standard permutation}  of a word $u=u_0u_1\cdots u_{n-1}$ over $\Sigma_k$ is the permutation $\pi_u$ of $\{0,1,\ldots,n-1\}$ such that $\pi_u(i)<\pi_u(j)$ if and only if $u_i<u_j$ or  $u_i=u_j$ and $i<j$. In other words, $\pi_u$ orders distinct letters of $u$ lexicographically, and equal letters by occurrence order, starting from $0$. %~\footnote{In the literature, the standard permutation is often defined as a permutation of $\{1,\ldots,n\}$. Here we start from $0$ for a technical reason.}. 
When $u$ is the BWT image of some necklace, the standard permutation is also called $LF$-mapping. As it is well known, a word $u$ is a BWT image if and only if $\pi_{u}$ is a length-$n$ cycle.

The \emph{inverse standard permutation} $\pi^{-1}_u$ of a word $u$ can be obtained by listing in left-to-right order the positions of $0$ in $u$, then the positions of $1$, and so on. The inverse standard permutation of a BWT image is also called $FL$-mapping.

In fact, a necklace can be uniquely reconstructed from its BWT, using either of the standard permutation or its inverse, written as cycles:
% Let $(n_0,n_1,\ldots, n_{k-1})$ be the Parikh vector of the BWT of $[w]$ (which is the same as the Parikh vector of each word in $[w]$). Let $\pi_u$ be the standard permutation of the BWT $u$ of $[w]$.
If $L$ is the BWT of $[w]$, and $\pi_L=(j_0\, j_1\, \cdots \, j_{n-1})$ is the standard permutation of $L$ written as cycle, with $j_0=0$,  then $L_{j_{n-1}}L_{j_{n-2}}\cdots L_{j_0}$ is the first row of the BWT matrix of $[w]$, i.e., the lexicographically least element of the necklace $[w]$, and this is also equal to $F_{i_0}F_{i_1}\cdots F_{i_{n-1}}$, where $\pi^{-1}_L=(i_0\, i_1\, \cdots \, i_{n-1})$ is the inverse standard permutation of $L$ written as cycle,  with $i_0=0$.

% \begin{example}
% The BWT matrix of $[10100]$ is 
% \[\BM_{[10100]}=\begin{pmatrix}
% 0\, 0\, 1\, 0\, 1 \\
% 0\, 1\, 0\, 0\, 1 \\
% 0\, 1\, 0\, 1\, 0 \\
% 1\, 0\, 0\, 1\, 0 \\
% 1\, 0\, 1\, 0\, 0
% \end{pmatrix}\] 
% so $11000$ is a BWT image, since it is the BWT of $[10100]$. The standard permutation of $11000$ is $\pi_u=\begin{pmatrix}
% 0\, 1\, 2\, 3\, 4 \\
% 3\, 4\, 0\, 1\, 2 \\
% \end{pmatrix}$, i.e., in cycle form, $\pi_u=(0\, 3\, 1\, 4\, 2)$.
% Since the word  $u=11000$ is the BWT of $[w]=[10100]$, we have  that $u_2u_4u_1u_3u_0=00101$ is the lexicographically least element in $[w]$.
% \end{example}

In the case of a  balanced Parikh vector $(\frac{n}{k},\frac{n}{k},\ldots,\frac{n}{k})$, then, one can  reverse the BWT from its inverse standard permutation $\pi_u^{-1}=(i_0\, i_1\, \cdots \, i_{n-1})$, $i_0=0$, by mapping $i_{\ell}$ to  $\left \lfloor \dfrac{i_{\ell}}{n/k}\right \rfloor$ for $\ell=0,1,\ldots,n-1$. 

% \begin{align}\label{eq:FLtoneck}
% \left \lfloor \frac{i_0}{n/k}\right \rfloor \cdots \left \lfloor \frac{i_{n-1}}{n/k}\right \rfloor
% \end{align}
% % It's the preimage of u, not u!

For example, the BWT  of $[w]=[220120011]$ is the word $u=202001121$, whose 
 inverse standard permutation in cycle form is $\pi_u^{-1}=(0\, 1\, 3\, 5\, 8\, 7\, 2\, 4\, 6)$. Mapping: $0,1,2$ to $0$; $3,4,5$ to $1$; $6,7,8$ to $2$, one obtains the word $001122012$, the first row of the BWT matrix of $[w]$.

 Let $ G = (V , E)$ be a finite directed graph, which may have loops and multiple edges. Each edge
$e \in E$ is directed from its source vertex $s(e)$ to its target vertex $t(e)$. 

The directed \emph{line graph} (or \emph{edge graph}) $\mathcal{L} G =(E, E')$ of $G$ has as vertices the edges of $G$, and as edges the set
$$E' = \{(e_1, e_2) \in E \times E \mid s(e_2) = t(e_1)\}.$$

An oriented \emph{spanning tree} of $G$ is a subgraph containing all of the vertices of $G$, having no directed cycles, in which one vertex, the root, has outdegree $0$, and every other vertex has outdegree $1$. The number $\kappa(G)$ of oriented spanning trees of $G$ is sometimes called the \textit{complexity} of $G$. 

A directed graph is \emph{Hamiltonian} if it has a \emph{Hamiltonian cycle}, i.e., one that traverses each node exactly once, while it is \emph{Eulerian} if it has an \emph{Eulerian cycle}, i.e., one that traverses each edge exactly once. As it is well known, a directed graph is Eulerian if and only if $indeg(v) = outdeg(v)$ for all vertices $v$. If a graph is Eulerian, its line graph is Hamiltonian.

\section{Generalized de Bruijn graphs and generalized de Bruijn words}\label{sec:db}
%%%%%%%%%%%%%%%%%%%%%%%%%%%%%%%%%%%%%%%%%%%%%%%%%%%%%%%%%%%%%%%%%%%%%%%%%%%%%%%%%

We start by briefly discussing (ordinary) de Bruijn graphs and de Bruijn words, and relating them to the inverse standard permutation of the Burrows--Wheeler transform.

% A \emph{de Bruijn word} of order $d$ over an alphabet $\Sigma$ is a necklace of length $|\Sigma|^d$ such that each of the $|\Sigma|^d$ distinct words of length $d$ occurs exactly once in it. 
% The \emph{de Bruijn graph} of order $d$ over $\Sigma$ %(or equivalently, Hamiltonian cycles of the de Bruijn graph of order $d-1$ over $\Sigma$), which I
% is defined as the directed graph $\DB(\Sigma,\Sigma^d)$ having vertices set $\Sigma^d$ and such that there is an edge from $u=u_0u_1\cdots u_{d-1}$ to $v=v_0v_1\cdots v_{d-1}$ if and only if
% $u_1\cdots u_{d-1}=v_0\cdots v_{d-2}$.
% %$u[1,\dd d-1]=v[0,\dd d-2]$.
% % WE USED A DIFFERENT NOTATION FOR WORDS!}

% As it is well known, de Bruijn words of order $d$ over $\Sigma$ are in $1$-to-$1$ correspondence with Eulerian cycles of the de Bruijn graph of order $d-1$ over $\Sigma$. Since the line graph of $\DB(\Sigma,\Sigma^d)$ is $\DB(\Sigma,\Sigma^{d+1})$, they are also in correspondence with the Hamiltonian cycles of the graph $\DB(\Sigma,\Sigma^{d+1})$.

% This correspondence is obtained by considering constructing a word by following a Hamiltonian cycle and concatenating the last letter of each traversed node. Note that one can also read the first letter of each node (or the $i$th letter for any fixed $1\le i \le d$), and obtain the same word up to a rotation, resulting in the same necklace.

% \red{Alternatively, de Bruijn graphs can be defined on the set of vertices $V=\{0,\dd |\Sigma|^{d}-1\}$, having an edge from $v$ to $u$ if and only if $u=|\Sigma|v+i$ with $i\in [0,\dd |\Sigma|-1]$. In that case, we denote it $DB(|\Sigma|,|\Sigma|^d)$. Those two definitions have a canonical correspondence, obtained by associating an integer $i$ to its string representation $\text{st}(i)$ in base $|\Sigma|$. Through this correspondence we can also associate Hamiltonian cycles in $DB(\Sigma,\Sigma^d)$ to Hamiltonian cycles in $DB(|\Sigma|,|\Sigma|^d)$}

% We show that de Bruijn words can also be constructed directly from the graph constructed over $[0,\dd \Sigma^{d}-1]$, obtaining the inverse standard permutation of the BWT of the de Bruijn word from each Hamiltonian cycle. Although this result only relies on a simple observation, we did not find any mention of it in the existing literature.

Recall that the de Bruijn graph $\DB(\Sigma_k,k^d)$ of order $d$ is the directed graph whose vertices are the words of length $d$ over $\Sigma_k$ and there is an edge from $a_iw$ to $v$ if and only if $v=wa_j$ for some letter $a_j$. As it is well known, de Bruijn graphs are Eulerian and Hamiltonian. One has $\mathcal{L}\DB(\Sigma_k,k^d) = \DB(\Sigma_k,k^{d+1})$. 

A Hamiltonian path in $\DB(\Sigma_k,k^d)$ can be represented by concatenating the first letter of the labels of the vertices it traverses, and with such representation is called a \emph{de Bruijn word} of order $d$ over $\Sigma_k$. As it is well known, a de Bruijn word of order $d$ over $\Sigma_k$ is a necklace over $\Sigma_k$ containing as factor each of the $\Sigma_k^d$ words of length $d$ over $\Sigma^k$ exactly once. 

Now, since the vertex labels of $\DB(\Sigma_k,k^d)$ are the base-$k$ representations of the integers in $\{0,1,\ldots,k^d-1\}$ on $d$ digits, the de Bruijn graph of order $d$ over $\Sigma_k$ can be equivalently defined as the one having vertex set $\{0,1,\ldots,k^d-1\}$ and there is an edge from $m$ to $km+i \mod (k^d)$ for every $i=0,1,\ldots,k-1$. We let $\DB(k,k^d)$ denote this equivalent representation of the de Bruijn graph of order $d$.
A Hamiltonian cycle in $\DB(k,k^d)$ is then a cyclic permutation of $\{0,1,\ldots,k^d-1\}$. 
The relation between a de Bruijn word of order $d$ over $\Sigma_k$, i.e., a  Hamiltonian cycle in $\DB(\Sigma_k,k^d)$, and the corresponding Hamiltonian cycle in $\DB(k,k^d)$, is given in the following lemma.


\begin{lemma}\label{lem:dbw as FL}
Let $[w]=[w_0\cdots w_{k^d-1}]$ be the necklace encoding a Hamiltonian cycle in $\DB(\Sigma_k,k^d)$, and let $[u]=[u_0\cdots u_{k^d-1}]$ be the corresponding Hamiltonian cycle in $\DB(k,k^d)$. Then $[u]$ is  the inverse standard permutation of the Burrows--Wheeler transform of $[w]$, written as cycle. In particular, one has $w_i=\lfloor u_i/k^{d-1} \rfloor$ for every $0\leq i\leq k^d-1$.
    % Let $DB(k,k^d)$ be a de Bruijn graph, defined on integer vertices. Let $u_1,\dd u_{k^d}$ be a Hamiltonian cycle on $DB(k,k^d)$, and let $w$ be the de Bruijn word obtained by reading the first letter of each node along the Hamiltonian path $\text{st}(u_1),\dd,\text{st}(u_{k^d})$ in the de Bruijn graph $DB(\Sigma_k,\Sigma_k^d)$. Then, the inverse standard permutation $\pi^{-1}_{BWT(w)}$, written as a cycle, is equal to $u_1,\dd u_{k^d}$. In particular, one has $[w]=[\lfloor \frac{u_1}{k^{d-1}} \rfloor,\dd,\lfloor \frac{u_{k^d}}{k^{d-1}} \rfloor]$
\end{lemma}

\begin{proof}
For a given integer $n$ in $\{0,\ldots,  k^d-1\}$, the first digit of $n$ in its base-$k$ representation on $d$ digits is $\lfloor \frac{n}{k^{d-1}} \rfloor$. The claim then follows directly from the definition of inverse standard permutation and the fact that de Bruijn words have balanced Parikh vector.
% \red{We use the following simple key observation: for a given integer $u$ in $\{0,\dd k^d-1\}$, one has $\text{st}(u)[0]=\lfloor \frac{u}{k^{d-1}} \rfloor$, since it is the leading digit of $u$ in its base $k$ representation. The results follow directly from the fact that de Bruijn words have balanced Parikh vector and from Equation~\ref{eq:FLtoneck}}
\end{proof}


We now introduce generalized de Bruijn words.  In~\cite{DBLP:journals/tcs/Higgins12}, Higgins showed that a word $w$ of length $k^n$ over $\Sigma_k=\{0,1,\ldots,k-1\}$  is a (ordinary) de Bruijn word if and only if its BWT is an alphabet-permutation power. This motivates us to introduce the following definition:

\begin{definition}
A necklace of length $kn$ over $\Sigma_k$ is a \emph{generalized de Bruijn word} if its BWT is an alphabet-permutation power.
\end{definition}

For example, $[02201331]$ is a generalized de Bruijn word since its BWT is $21302031$.

% The generalized de Bruijn words of length $6$ over $\Sigma_3$, i.e., the Eulerian cycles of $\DB(3,6)$ read mod $3$, are: $001221$, $002121$, $010122$, and $010212$. The number of generalized de Bruijn words of length $3n$ over $\Sigma_3$ is A192513 in the OEIS. Notice that A192513(n)=A094678(n)$*2^{n-1}$= A003474(n)$/n*2^{n-1}$, where A003474 is the Generalized Euler phi function (for p=3) and A094678 is the Number of normal bases for $GF(3^n)$ over $GF(3)$.
% A formula for the number of generalized de Bruijn words of length $n$ over $\Sigma_k$ exists when $k$ is a prime (as a consequence of  Sandpile groups of generalized de Bruijn and Kautz graphs and circulant matrices over finite fields, Journal of Algebra (2015), pp. 268-295). 

Notice that, in the literature, there already exist several other definitions of generalized de Bruijn words~(see, e.g.,~\cite{Blum83,DBLP:journals/dm/Au15,DBLP:conf/cpm/NelloreW22}).

The \emph{generalized de Bruijn graph} $\DB(k,n)$ has vertices $\{0,1,\ldots,n-1\}$ and for every vertex $m$ there is an edge from $m$ to $km+i \mod (n)$ for every $i=0,1,\ldots,k-1$. They have been introduced independently by Imase and Itoh~\cite{DBLP:journals/tc/ImaseI81}, and by Reddy, Pradhan, and Kuhl~\cite{Reddy} (see also~\cite{DBLP:journals/networks/DuH88}). Generalized de Bruijn graphs are Eulerian and Hamiltonian. By definition, a generalized de Bruijn graph $\DB(k,n)$ is an ordinary de Bruijn graph when $n=k^d$ for some $d>0$. As an example, the generalized de Bruijn graph $\DB(3,6)$ is displayed in Fig.~\ref{fig:DB36}.


\begin{figure}
    \centering
  \includegraphics[width=13cm]{DB36-ipe.pdf}
    \caption{The generalized de Bruijn graph $\DB(3,6)$. }
    \label{fig:DB36}
\end{figure}


The line graph of $\DB(k,n)$ is $\DB(k,kn)$~\cite{DBLP:journals/dm/LiZ91}. Therefore, the Eulerian cycles of $\DB(k,n)$ correspond to the Hamiltonian cycles of $\DB(k,kn)$.

In next Theorem~\ref{thm:charGen}, we give a characterization of generalized de Bruijn words in terms of generalized de Bruijn graphs, which generalizes Lemma~\ref{lem:dbw as FL}.


\begin{lemma}\label{lem:APW-char}
Let $u$ be a word of length $kn$ over $\Sigma_k$, with Parikh vector $(n,\dd, n)$. The following statements are equivalent:
\begin{enumerate}
    \item\label{APW-char1} The word $u$ is an alphabet-permutation power;
    \item\label{APW-char2} For each $j\le k$, $i\le n$, one has $\pi^{-1}_u(i+jn)\in[ki,k(i+1)-1]$; % ($j,\pi_u(j)$) is an edge of the graph $\DB(k,kn)$.
   \item\label{APW-char3} For each $0\le \ell< kn$, there is an edge from $\ell$ to $\pi^{-1}_u(\ell)$ in $\DB(k,kn)$. 
\end{enumerate}
\end{lemma}

\begin{proof}
We first observe that $(\ref{APW-char3})\iff (\ref{APW-char2})$, as the edges of $\DB(k,kn)$ are exactly the ones of the form $(j,kj+i\mod kn)$ with $0\le i \le k-1$, and that $k(i+jn)+i'=ki+i'\mod kn$ for any $i,i'<k$. 
%We now prove that $\ref{APW-char1}\iff \ref{APW-char2}$.

We then make the following remark: $\pi^{-1}_u(\ell)=j$ if $j$ is the position of the $\ell$th letter in $u$, according to the order defined by $\pi^{-1}_u$. In particular, if $u$ has Parikh vector $(n,\dots , n)$, then the position $\pi^{-1}_u(i+jn)$ is the position of the $(i+1)$th occurrence of the letter $j$. In particular, if $u$ is an alphabet-permutation power, the position of the $(i+1)$th occurrence of the letter $j$ occurs in the $(i+1)$th block of length $k$, which means that $\pi^{-1}_u(i+jn)\in [ki,k(i+1)-1]$. This proves that $(\ref{APW-char1})\implies (\ref{APW-char2})$.

%Assume that $u$ is an alphabet-permutation power, in other words, it comprises $n$ blocks of length $k$, where each letter in $\Sigma_k$ appears exactly once in each block. Now, for any $j\le k$, $i\le n$, the position $\pi^{-1}_u(jn+i)$ is the position of the $i+1$th occurrence of the letter $j$, which occurs in the $i+1$th block, which means that $\pi^{-1}_u(jn+i)\in [ik,ik+k-1]$.

Conversely, if (\ref{APW-char2}) is satisfied, then for a given fixed $i\le n$, one has $\pi^{-1}_u(jn+i)\in[ki,k(i+1)-1]$ for \emph{every} $j\le k$. Since $\pi^{-1}_u$ is a permutation, hence one-to-one, we have $[ki,k(i+1)-1]=\{\pi^{-1}_u(i),\pi^{-1}_u(n+i),\dots,\pi^{-1}_u((k-1)n+i)\}$. Again, from the remark that $\pi^{-1}_u(jn+i)$ is the position of the $(i+1)$th occurrence of the letter $j$, one obtains $\{
u[\pi^{-1}_u(i)],u[\pi^{-1}_u(n+i)],\dots,[\pi^{-1}_u((k-1)n+i)]\}=\{0,\dots, k-1\}=\Sigma_k$. Plugging this into the previous identity, we deduce that $[u[ki],u[k(i+1)-1]]$ is a permutation of $\Sigma_k$ for every $i< n$, i.e., $u$ is an alphabet-permutation power. Hence, $(\ref{APW-char2})\implies (\ref{APW-char1})$, and the proof is complete.
%for each $i\in[0,k-1]$, $j\in [0,n-1]$, we have $u[\pi^{-1}_u(jn+i)]=j$. By replacing each position $\ell$ by the letter $u[\ell]$, one can reformulate condition \ref{APW-char3} as follows: For each $0\le j< k$, one has $\{0,\dots, k-1\}=\{u[kn],\dots,u[kn+k-1]\}$. This is clearly equivalent to \ref{APW-char1}, hence \ref{APW-char3}$\implies$\ref{APW-char1}. For the converse, assume the condition~\ref{APW-char1}. This means that for each $j\in [0,n-1]$, the substring $u[nj,\dots,nj+k-1]$ is a permutation of the alphabet $\{0,\dots,k-1\}$.  Hence, the $i$th occurrence of the letter $j$ occurs at one of the positions in $[ni,\dots,ni+k-1]$ in $u$ This exactly means that $\{\pi^{-1}_u(i),\dots,\pi^{-1}_u(n+i)\}=\{ni,\dots,ni+k-1\}$ for every $i\in [1,n]$, hence condition \ref{APW-char3} is satisfied. 
\end{proof}

% \begin{lemma}
% Let $w$ be a word of length $kn$ over $\Sigma_k$, with Parikh vector $(n,\dd, n)$. The following conditions are equivalent:\begin{itemize}
%     \item $[w]$ is a \gDBw;
%     \item The inverse standard permutation of the BWT of $[w]$, $\pi^{-1}_{BWT(w)}$, written as a $2n$-length cycle, spells the nodes visited by a Hamiltonian cycle of $\DB(k,kn)$.
% \end{itemize}
% \end{lemma}

% \begin{proof}
% This follows directly from Lemma~\ref{lem:APW-char}, and from Theorem~\ref{th:BWT cycle}.
% \end{proof}

As a direct consequence of the previous lemma, we have:

\begin{theorem}\label{thm:charGen}
 A necklace is the label of a Hamiltonian cycle of a generalized de Bruijn graph $\DB(k,kn)$ if and only if it is the inverse standard permutation, written as cycle, of the BWT of a generalized de Bruijn word of length $kn$ over $\Sigma_k$.
\end{theorem}

% \begin{proof}
% \todo{To Do}
% \end{proof}

Therefore, the generalized de Bruijn words of length $kn$ over the alphabet $\Sigma_k$ can be constructed by mapping $i\mapsto \lfloor i/n\rfloor$ for every $i=0,\ldots,  kn-1$ to the  Hamiltonian cycles of the generalized de Bruijn graph $\DB(k,kn)$.

\begin{example}
    The Hamiltonian cycles in $\DB(3,6)$ are: $013542$, $015342$, $021354$, and $021534$. Applying the coding $i\mapsto \lfloor i/2\rfloor$ to them, one obtains the ternary generalized de Bruijn words of length $6$:  $001221$, $002121$, $010122$, and $010212$.

The Hamiltonian cycles in $\DB(3,9)$ are: $013274586$, $013287456$, $014327586$, $014328756$, $014562873$, $014573286$, $014586273$, $014587326$, $015628743$, $015743286$, 
$015862743$, $015874326$, $026145873$, $026158743$, $027314586$, 
$027431586$, $027458613$, $027586143$, $028614573$, $028615743$, 
$028731456$, $028743156$, $028745613$, and $028756143$. Applying the coding $i \mapsto \lfloor i/3\rfloor$ to them, one obtains the  (ordinary) ternary de Bruijn words of length $9$: $001021122$, $001022112$, $001102122$, $001102212$, $001120221$, 
$001121022$, $001122021$, $001122102$, $001202211$, $001211022$, 
$001220211$, $001221102$, $002011221$, $002012211$, $002101122$, 
$002110122$, $002112201$, $002122011$, $002201121$, $002201211$, 
$002210112$, $002211012$, $002211201$, and $002212011$.
\end{example}


Thanks to the characterization given in Theorem~\ref{thm:charGen}, we can give a formula for counting  generalized de Bruijn words. We make use of the BEST theorem for directed graphs, together with the fact that the number of Hamiltonian cycles in $\DB(k,kn)$ is equal to the number of Eulerian cycles in $\DB(k,n)$.

\begin{theorem}[BEST theorem]
 The number of distinct Eulerian cycles starting at node $v_i$ in a Eulerian directed graph $G=(V,E)$ is given by
\begin{equation}\label{BEST}
  e(G)=\kappa(v_i)\prod_{j=1}^n(d_{v_j}-1)!
\end{equation}
 where $d(v_j)$ is the outdegree of $v_j$ and $\kappa(v_i)$ is the number of spanning trees oriented towards $v_i$ (the number is the same for any $i$).
\end{theorem}

Since generalized de Bruijn graphs are Eulerian, we have that the outdegree of every node is always $k$ and therefore $\prod_{j=1}^n(d_{v_j}-1)!=((k-1)!)^n$
 
\begin{theorem}\label{thm:counting}
For every $n\geq k-1$, the number of generalized de Bruijn words of length $kn$ over $\Sigma_k$ is 
\begin{equation}\label{eqDBW}
    \DBW_k(kn)=((k-1)!)^n \kappa(\DB(k,n))
\end{equation}
\end{theorem}

\begin{table}[ht]
\begin{center}
\begin{tabular}{cccccccccccccccccccc}
$n$    & 2 & 3 & 4 & 5 & 6  & 7  & 8  & 9  & 10 & 11  & 12  \\ \hline
$\DBW_2(2n)$  & 1& 1& 2& 3& 4& 7& 16& 21& 48& 93& 128\\ \hline
$\DBW_3(3n)$  & 4& 24& 64& 512& 1728& 13312& 32768& 373248& 1310720& 10903552& 
35831808 
\end{tabular}
\end{center}
\caption{First few values of $\DBW_2(2n)$  and $\DBW_3(3n)$ (resp.~sequence A027362 and A192513 in~\cite{oeis}). \label{tab:DBW}}
\end{table}


In turns, the number of spanning trees can be computed by means of a determinant, as an application of the matrix-tree theorem below. % (which holds in general for any directed graph, not necessarily Eulerian --- the number of spanning trees may differ from a node to another in general, but it is the same for all nodes in an Eulerian graph). 
Recall that the \emph{Laplacian matrix} of a directed graph $G$ is the $n\times n$ matrix $L_G=D_G-A_G$, where $D_G$ and $A_G$ are the degree matrix and the adjacency matrix of $G$, respectively. It can be defined by
$$L_{ij}=\begin{cases}
                    d_{v_i} - d_{v_iv_i}  & \mbox{ if }   i = j \\
                    - d_{v_iv_j} & \mbox{ if }   i \neq j
                 \end{cases}$$
where $d_{v_iv_j}$ is the number of edges directed from $v_i$ to $v_j$ (that is, $0$ or $1$), and $d_{v_i} = \sum_j d_{v_iv_j}$ is the outdegree of $v_i$, so that  $d_{v_i} - d_{v_iv_i}$ is the outdegree of $v_i$ minus the number of its self loops.

\begin{theorem}[Matrix-Tree Theorem]\label{K}
 The number $\kappa(v_i)$ of spanning trees oriented towards node $v_i$ is equal to the determinant of the matrix obtained from the Laplacian matrix of $G$ after removing the $i$-th row and the $i$-th column. 
\end{theorem}

Let $L_G$ be the Laplacian matrix of an Eulerian  graph $G$ (since the graph is Eulerian, the matrix is the same for any vertex) and $L_G^0$ be the matrix obtained by removing the last  row and column, sometimes called the \textit{reduced Laplacian matrix} of $G$. By the  matrix-tree theorem, we have $\kappa(G)=\det (L_G^0)$. 


%\subsection{de Bruijn Graphs  and de Bruijn Words}
%%%%%%%%%%%%%%%%%%%%%%%%%%%%%%%%%%%%%%%%%%%%%%
%\bigskip 

\begin{comment}Let $\Sigma_k=\{0,1,\ldots,k-1\}$. The \emph{de Bruijn graph} $\DB(k,k^d)$ of order $d$ over $\Sigma_k$ is the directed graph whose vertices are the words of length $d$ over $\Sigma_k$ and there is an edge from $aw$ to $v$ labeled by $a$ if and only if $v=wb$ for some letter $b$. Equivalently, one can think of vertex labels as the base-$k$ representations of integers in $\{0,1,\ldots,k^d-1\}$ and there is an edge from $m$ to $km+i \mod (k^d)$ for every $i=0,1,\ldots,k-1$. de Bruijn graphs are Eulerian and Hamiltonian. One has $\mathcal{L}\DB(k,k^d) = \DB(k,k^{d+1})$.
A \emph{de Bruijn word} of order $d$ over $\Sigma_k$ is a necklace that labels a Hamiltonian cycle in a de Bruijn graph of order $d$ over $\Sigma_k$, or, equivalently, an Eulerian cycle in a de Bruijn graph of order $d-1$. It therefore has length $k^d$.
\end{comment}

%\subsection{Smith Normal Form and Sandpile Groups}
%%%%%%%%%%%%%%%%%%%%%%%%%%%%%%%%%%%%%%%%%%%%%%
We now discuss the relations between generalized de Bruijn words and \emph{sandpile groups}, which are algebraic objects capturing important properties of directed Eulerian graphs, and can be defined in terms of the reduced Laplacian matrix of the graph. We start by giving some standard results and definitions from~\cite{DBLP:journals/jct/Stanley16}.
Recall that any integer matrix $A$ can be written in a canonical form $A=PDQ$, where both matrices $P$, $Q$ are integer matrices and $D$ is an integer diagonal matrix 
with the property that if $d_1,\ldots, d_n$ are the nonzero entries of the main diagonal of $D$, then $d_i | d_{i+1}$ for every $i$. The matrix $D$ is called the \emph{Smith Normal Form} of $A$. The $d_i$ are called the \emph{invariant factors} of the Smith Normal Form.
%See~\cite{DBLP:journals/jct/Stanley16} for more details.

Let $G=(V,E)$ be a finite strongly connected directed graph, that is, for any $v, w \in V$ there are directed paths in $G$ from $v$ to $w$ and from $w$ to $v$. Then associated to any vertex $v$ of $G$ is an Abelian group $K(G, v)$, the \textit{sandpile group} (also known as critical
group, Picard group, Jacobian, and group of components). When $G$ is Eulerian  the groups $K(G, v)$ and $K(G, u)$ are isomorphic for any $v,u\in V$, and we  denote the \emph{sandpile group of $G$} by $K(G)$, which can be defined as follows:
\begin{equation}\label{isomorphism}
    K(G)= \Z^{n-1} /L_G^0 \cong \bigoplus_{i=1}^{n-1} \Z_{d_i}
\end{equation}
where $d_1,\ldots, d_{n-1}$ are the invariant factors of the Smith Normal Form of $L_G^0$, or equivalently, $d_1,\ldots, d_{n-1},0$ are  the invariant factors of $L_G$.

%The sandpile group can be obtained from the Laplacian matrix $L_G$. It carries the same information as the Smith Normal Form of $L_G$. 
By the matrix-tree theorem, the order of the sandpile group is the number of oriented spanning trees of $G$% rooted at $v$
, $|K(G)|=\kappa(G)$, %whereas its order is  $\kappa(G)=
also equal to  $\det(L_G^0)$, the determinant of the reduced Laplacian matrix of $G$.

For ordinary de Bruijn graphs over a binary alphabet, Levine~\cite{DBLP:journals/jct/Levine11} proved that
 \begin{equation}\label{Levine}
     K(\DB(2,2^d))\cong \bigoplus_{i=1}^{d-1}(\Z_{2^i})^{2^{d-1-i}}.
 \end{equation}


\begin{example}
Consider the generalized de Bruijn graph $\DB(3,6)$ (Fig.~\ref{fig:DB36}). %\todo{put vertex labels inside} 
Its Laplacian matrix is 
    \[L_{\DB(3,6)}=\begin{pmatrix}
 2& -1& 0& -1& 0& 0 \\
 0& 3& -1& 0& -1& -1 \\
 0& 0& 2& 0& -1& -1 \\
 -1& -1& 0& 2& 0& 0 \\
 -1& -1& 0& -1& 3& 0 \\
 0& 0& -1& 0& -1& 2
\end{pmatrix}\]
and its Smith Normal Form is
    \[\SNF(L_{\DB(3,6)})=\begin{pmatrix}
 1& 0& 0& 0& 0& 0 \\
 0& 1& 0& 0& 0& 0 \\
 0& 0& 3& 0& 0& 0 \\
 0& 0& 0& 3& 0& 0 \\
 0& 0& 0& 0& 3& 0 \\
 0& 0& 0& 0& 0& 0
\end{pmatrix}\]

  Hence, by \eqref{isomorphism}, we have $K(\DB(3,6))\cong \Z_3^3$. The matrix $\SNF(L_{\DB(3,6)})$ has determinant $3^3=27$, so $\DB(3,6)$ has $27$ distinct Eulerian cycles --- and $\DB(3,18)$ has $27$ distinct Hamiltonian cycles. Thus, by Theorem~\ref{thm:counting}, there are $27\cdot 2^6=1728$ ternary generalized de Bruijn words of length $18$.

As another example, the Laplacian matrix of $\DB(2,6)$ is 
    \[L_{\DB(2,6)}=\begin{pmatrix}
 1& -1& 0& 0& 0& 0 \\
 0& 2& -1& 0& -1& 0 \\
 0& 0& 2& -1& 0& -1 \\
 0& 0& -1& 2& -1& 0 \\
 -1& -1& 0& 0& 2& 0 \\
 0& 0& 0& -1& 0& 1
\end{pmatrix}\]
Its Smith Normal Form is 
    \[\SNF(L_{\DB(2,6)})=\begin{pmatrix}
 1& 0& 0& 0& 0& 0 \\
 0& 1& 0& 0& 0& 0 \\
 0& 0& 1& 0& 0& 0 \\
 0& 0& 0& 2& 0& 0 \\
 0& 0& 0& 0& 2& 0 \\
 0& 0& 0& 0& 0& 0
\end{pmatrix}\]
 Hence, $K(\DB(2,6))\cong \Z_2^2$, and in fact there are $4$ binary generalized de Bruijn words of length $12$, namely $000010111101$, $000011101101$, $000100101111$, and $000100111011$.
 \end{example}
 


%Now, $d_1,\ldots, d_{n-1}$ are the invariant factors of the Smith Normal Form of $L_G^0$ if and only if $d_1,\ldots, d_{n-1},0$ are  the invariant factors of $L_G$~(see~\cite{DBLP:journals/jct/Stanley16}).
\begin{comment}The sandpile group of $G$ is therefore
\begin{equation}\label{isomorphism}
    K(G)\cong \Z^{n-1} /L_G^0 \cong \bigoplus_{i=1}^{n-1} \Z_{d_i}. 
\end{equation}
\end{comment}

In the next sections, we will deepen our analysis in the case when the alphabet size is a prime number $p$, allowing us to connect generalized de Bruijn words to necklaces having a BWT matrix invertible over $\Z_p$.

\section{Invertible necklaces and Reutenauer groups}\label{sec:invertible}
%%%%%%%%%%%%%%%%%%%%%%%%%%%%%%%%%%%%%%%%%%%%%%

In this section, we define invertible necklace and highlight their connections with abstract algebra.



In \cite{DBLP:journals/corr/abs-2409-07974} (see also the related paper \cite{DBLP:journals/corr/abs-2409-09824}), the authors used BWT matrix multiplication to obtain a group structure on important classes of binary words (namely, \emph{Christoffel words} and \emph{balanced} words) that have an invertible BWT matrix. Following this idea, we define \emph{invertible} necklaces: %. 

\begin{definition}
    Let $p$ be a prime. A necklace $[w]$ over the alphabet $\Sigma_p=\{0,1,\ldots,p-1\}$ is called \emph{invertible} if its BWT matrix has nonzero determinant modulo $p$.
\end{definition}

As observed in \cite{DBLP:journals/corr/abs-2409-07974}, the product of two BWT matrices is not, in general, a BWT matrix; but the author suggests to replace  BWT matrices by \emph{circulant matrices}. As we will see, using circulant matrices we can define the product between any two invertible necklaces.

Let $w$ be a word over $\Sigma_k$. The \textit{circulant matrix} of $w$ is the square matrix $\CM_w$ whose $(i+1)$th row is $\sigma^{i}(w)$. For example, the circulant matrix of $011$ is $\CM_{011}=\begin{pmatrix}
0\, 1\, 1 \\
1\, 0\, 1 \\
1\, 1\, 0
\end{pmatrix}$.

One can immediately observe that, since the circulant matrix of a word can be obtained by permuting the rows of its BWT matrix, a necklace is invertible if and only if any (and then, every) of its associated circulant matrices is invertible.

%In the following, we consider the alphabet $\Sigma_p$, where $p$ is a prime number.

Recall that for every prime $p$ and any positive integer $n$, there is a unique finite field with $p^n$ elements, denoted by $\GF_{p^n}$, and its multiplicative group $\GF_{p^n}^*$ is cyclic. The set of invertible $n \times n$-circulant matrices over $\GF_p$ forms, with respect to matrix multiplication, an Abelian group $C(p, n)$. We consider the group $C(p,n)/\langle Q_n \rangle$, where $Q_n$ is the permutation matrix of the cycle  $(0\, 1\, \ldots \, n-1)$. This group is called the \emph{Reutenauer group} $\RG_p^n$ in \cite{DuzhinJMS}. Intuitively, an element of the Reutenauer group corresponds to a circulant matrix up to rotation of the rows, or, equivalently, of the associated word. Hence, it is in natural bijection with the set of invertible necklaces of length $n$ over $\Sigma_p$, and one can write $\CM_{[w]}=\{\CM_v,~v\in[w]\}\in \RG_p^n$ for the class of circulant matrices associated to a given invertible necklace $[w]$ of length $n$ over $\Sigma_p$. In particular, this induces a group structure on the set of such invertible necklaces, isomorphic to $\RG_p^n$, namely $[u]\cdot [v]=[w]$ if one has $\CM_{[u]}\cdot \CM_{[v]}=\CM_{[w]}$. In particular, this multiplication does not depend on the choice of representatives for each necklace (equivalently, for each matrix).

As shown in~\cite{DuzhinJMS}, the Reutenauer group acts on the set of all aperiodic necklaces %\red{(in particular, this is not restricted to the invertible ones)} 
as follows:
Let $[v]$ be an aperiodic necklace of length $n$ over $\Sigma_p$.  For every $\CM_{[w]}\in \RG_p^n$, we take the necklace of $\CM_{[w]}\cdot v^T$. %In particular, this operation does not depend on the choice of a representative. ALREADY SAID

For example, $\CM_{[11010]}\cdot [11000]=[11110]$ since $\begin{pmatrix}
1\, 1\, 0\, 1\, 0 \\
0\, 1\, 1\, 0\, 1 \\
1\, 0\, 1\, 1\, 0\\
0\, 1\, 0\, 1\, 1\\
1\, 0\, 1\, 0\, 1
\end{pmatrix}\cdot
\begin{pmatrix}
1 \\
1 \\
0\\
0\\
0
\end{pmatrix}= 01111
$.

The orbit of an aperiodic necklace $[v]$ is therefore $O_{[v]}=\{[\CM_{[w]}\cdot v^T] \mid w\in \RG_p^n\}$. 


 % An element $\alpha$ of $\GF_{p^n}$ is called \emph{primitive} if its multiplicative order is $p^n-1$. So, if $\alpha$ is a primitive element of  $\GF_{p^n}$, we have $\GF_{p^n}=\{0,1=\alpha^0,\alpha^1,\alpha^2,\ldots,\alpha^{p^n-2}\}$.

%  Primitive elements are roots of \emph{primitive polynomials}, which are the monic polynomials in $\GF_p[x]$ that divide $x^{p^n}-1$ but not $x^m-1$ for any $m<p^n$. The field $\GF_{p^n}$ can therefore be constructed by choosing any primitive polynomial $f(x)\in \GF_p[x]$ and representing the nonzero elements of the field either  by powers of $\alpha$ modulo $f(\alpha)$ or by the corresponding words of length $n$ over $\Sigma_p$ (see Table~\ref{tableGF}), in such a way that the shift corresponds to the multiplication by $\alpha$ modulo $f(\alpha)$. 
% \begin{table}[h!]
%   \begin{center}
%     \caption{The nonzero elements of the finite field $\GF_{2^4}$ constructed using the primitive polynomial $x^4+x+1$.}
%     \label{tableGF}
%     \begin{tabular}{l|l|l} % <-- Alignments: 1st column left, 2nd middle and 3rd right, with vertical lines in between
%       %\textbf{Value 1} & \textbf{Value 2} & \textbf{Value 3}\\
%       power of $\alpha$ & polynomial in $\alpha$ & word \\
%       \hline
%     $\alpha^0$ & $1$ & $0001$\\
%       $\alpha^1$ & $\alpha$ & $0010$\\
%       $\alpha^2$  & $\alpha^2$ & $0100$\\
%       $\alpha^3$  & $\alpha^3$ & $1000$\\
%     $\alpha^4$  & $\alpha+1$ & $0011$\\
%         $\alpha^5$ & $\alpha^2+\alpha$ & $0110$\\
%       $\alpha^6$ & $\alpha^3+\alpha^2$ & $1100$\\
%       $\alpha^7$  & $\alpha^3+\alpha+1$ & $1011$\\
%       $\alpha^8$  & $\alpha^2+1$ & $0101$\\
%     $\alpha^9$  & $\alpha^3+\alpha$ & $1010$\\
%     $\alpha^{10}$ & $\alpha^2+\alpha+1$ & $0111$\\
%       $\alpha^{11}$ & $\alpha^3+\alpha^2+\alpha$ & $1110$\\
%       $\alpha^{12}$  & $\alpha^3+\alpha^2+\alpha+1$ & $1111$\\
%       $\alpha^{13}$  & $\alpha^3+\alpha^2+1$ & $1101$\\
%     $\alpha^{14}$  & $\alpha^3+1$ & $1001$\\
%     \end{tabular}
%   \end{center}
% \end{table}
We now recall some classical results and definitions in abstract algebra (see, for example, Chapters V and VI of~\cite{lang02} for a more detailed presentation), which have a strong connection with circulant matrices and  invertible necklaces.

As it is well known, the number of irreducible polynomials over $\GF_p$ of degree $n$ is equal to the number of aperiodic necklaces of length $n$ over $\Sigma_p$, but there is no known canonical bijection between the two sets. 
% The \emph{minimum polynomial} of a nonzero element $\alpha^i$ of $\GF_{p^n}$ is the (unique) polynomial $f(x)\in\GF_p[x]$ with least degree of which $\alpha^i$ is a root. This polynomial has as roots all the $m$ distinct elements of the form $\alpha^i, \alpha^{ip},\ldots, \alpha^{ip^{m-1}}$, where $m$ is the least integer such that $ \alpha^{ip^{m}}=\alpha^i$. Thus, $f(x)$ has degree $m$ and can be factored as
% \[f(x)=(x-\alpha^i)(x-\alpha^{ip})\cdots (x-\alpha^{ip^{m-1}}).\]
% %The sum of all roots of $f(x)$ is called the \emph{trace} of $f(x)$. %, and of $\alpha$ as well. 
% For example, the minimum polynomial of $\alpha^7$ in $\GF_{2^4}$ is $x^4+x^3+1$, which has roots also $\alpha^{14}$, $\alpha^{13}$, and $\alpha^{11}$. %; its trace is therefore \todo{what??}.
The \emph{minimum polynomial} of a nonzero element $\alpha$ of $\GF_{p^n}$ is the (unique) polynomial $f\in\GF_p[X]$ with least degree of which $\alpha$ is a root. This polynomial has as roots all the $m$ distinct elements of the form $\alpha, \alpha^{p},\ldots, \alpha^{p^{m-1}}$, where $m$ is the least integer such that $\alpha^{p^{m}}=\alpha$, i.e., the distinct roots of $f$ are the orbit of $\alpha$ under the application of the Frobenius automorphism $\alpha\mapsto \alpha^p$. Thus, $f$ has degree $m$ and can be factored as
\[f=(x-\alpha)(x-\alpha^{p})\cdots (x-\alpha^{p^{m-1}}).\]
The sum of all roots of $f$ is called the \emph{trace} of $f$. %, and of $\alpha$ as well. 

    
But $\GF_{p^n}$ is also a vector space over $\GF_{p}$. An element $\gamma$ of $\GF_{p^n}$ is called \emph{normal} if  $\{\gamma, \gamma^{p},\ldots,\gamma^{p^{n-1}}\}$ forms a vector space basis for $\GF_{p^n}$ over $\GF_{p}$, called a \emph{normal basis}. 

%For example, $\{\alpha^7,\alpha^{14},\alpha^{13},\alpha^{11}\}$ is a normal basis of $\GF_{2^4}$ over $\GF_{2}$, since any element of $\GF_{2^4}$ is a linear combination of elements in the basis. 

The minimum polynomial of a normal element of $\GF_{p^n}$ over $\GF_{p}$ is called a \emph{normal polynomial}. A monic irreducible polynomial of degree $n$ over $\GF_p$ is normal if and only if it is coprime with $X^{p^n}-1$.   
Normal polynomials have non-zero trace.
%If $\alpha\in \GF_{p^n}$ is a root of a monic, irreducible polynomial $f(x)$ of degree $n$, then the elements $\alpha, \alpha^p,\ldots,\alpha^{p^{n-1}}$ are roots of $f(x)$ and \[f(x)=(x-\alpha)\cdots (x-\alpha^{p^n-1})=x^n-(\alpha + \alpha^p +\ldots +\alpha^{p^{n-1}})x^{n-1}+\ldots +(-1)^n(\alpha \cdot \alpha^p \cdots \alpha^{p^{n-1}}).\]

If $\gamma$ is a normal element of $\GF_{p^n}$, we can write all the elements of $\GF_{p^n}$ as linear combinations of elements in the normal basis $\{\gamma, \gamma^{p},\ldots,\gamma^{p^{n-1}}\}$. That is, we can write all the elements of $\GF_{p^n}$ as $n$-length words, or  $n$-length  vectors, corresponding to the (reversals of) the $p$-ary expansions on $n$ digits of the integers in $\{0,1,\ldots,p^n-1\}$. More precisely, if $\alpha=w_0\gamma+w_1\gamma^p+\ldots +w_{n-1}\gamma^{p^{n-1}}$, $w_j\in\GF_p$ for every $j=0,\ldots,n-1$, we can represent $\alpha$ by the word $w=w_0w_1\cdots w_{n-1}$. 
The application of the Frobenius automorphism to an element of the field corresponds to the shift of the corresponding word, and an orbit to a conjugacy class, i.e., to a necklace.  This necklace is aperiodic if and only if the orbit is maximal, i.e., has size $n$.

% \begin{table}[ht]
%   \begin{center}
%     \caption{The finite field $\GF_{2^4}$ as a vector space over $\GF_2$, where the elements are grouped by orbits with respect to a normal element $\gamma$.}
%     \label{tableGFV}
%     \begin{tabular}{l|l|l|l} % <-- Alignments: 1st column left, 2nd middle and 3rd right, with vertical lines in between
%       %\textbf{Value 1} & \textbf{Value 2} & \textbf{Value 3}\\
%      orbit & orbit as vectors (words) & orbit as numbers & normal \\
%       \hline
%     $\{0\}$ & $\{0000\}$ & $\{0\}$ & no \\
%     $\{\gamma,\gamma^2,\gamma^4,\gamma^8\}$ & $\{1000,0100,0010,0001\}$ & $\{1,2,4,8\}$ & yes\\
%     $\{\gamma+\gamma^2,\gamma^2+\gamma^4,\gamma^4+\gamma^8,\gamma+\gamma^8\}$ & $\{1100,0110,0011,1001\}$ & $\{3,6,12,9\}$ & no\\
%     $\{\gamma^2+\gamma^8,\gamma+\gamma^4\}$ & $\{0101,1010\}$ & $\{5,10\}$ & no\\
%     $\{\gamma+\gamma^2+\gamma^4,\gamma^2+\gamma^4+\gamma^8,\gamma+\gamma^4+\gamma^8,\gamma+\gamma^2+\gamma^8\}$ & $\{1110,0111,1011,1101\}$ & $\{7,14,13,11\}$ & yes\\
%     $\{\gamma+\gamma^2+\gamma^4+\gamma^8\}$ & $\{1111\}$ & $\{15\}$ & no \\
%     \end{tabular}
%   \end{center}
% \end{table}

\begin{table}[ht]
  \begin{center}
    \caption{The finite field $\GF_{2^4}$ as a vector space over $\GF_2$, where the elements are grouped by orbits with respect to a normal element $\gamma$.}
    \label{tableGFV}
    \begin{tabular}{l|l|l|l} 
     orbit & orbit as vectors (words)  & normal \\
      \hline
    $\{0\}$ & $\{0000\}$ & no \\
    $\{\gamma,\gamma^2,\gamma^4,\gamma^8\}$ & $\{1000,0100,0010,0001\}$  & yes\\
    $\{\gamma+\gamma^2,\gamma^2+\gamma^4,\gamma^4+\gamma^8,\gamma+\gamma^8\}$ & $\{1100,0110,0011,1001\}$  & no\\
    $\{\gamma^2+\gamma^8,\gamma+\gamma^4\}$ & $\{0101,1010\}$  & no\\
    $\{\gamma+\gamma^2+\gamma^4,\gamma^2+\gamma^4+\gamma^8,\gamma+\gamma^4+\gamma^8,\gamma+\gamma^2+\gamma^8\}$ & $\{1110,0111,1011,1101\}$ &  yes\\
    $\{\gamma+\gamma^2+\gamma^4+\gamma^8\}$  & $\{1111\}$ & no \\
    \end{tabular}
  \end{center}
\end{table}

From now on, we suppose that a normal element $\gamma$ has been chosen, and we represent any element of the field $\GF_{p^n}$ as a word (vector) written in the corresponding normal basis. In this representation, orbits correspond to necklaces.
The orbit of $\gamma$ corresponds to the necklace $[10^{n-1}]$. There may be other  orbits constituted by normal elements. For example, the orbit corresponding to the necklace $[1110]$ is another orbit of normal elements in $\GF_{2^4}$ (see Table~\ref{tableGFV}).
%\todo{the orbit of $1^n$ corresponds to the multiplicative identity}

% \textcolor{red}{
% If $\alpha^i$ is a normal element, and $w=w_0w_1\cdots w_{n-1}$ is the associated word,  the polynomial $w_0+w_1x+\ldots +w_{n-1}x^{n-1}$ is a normal polynomial. FALSE!
% With the invertible necklace $[w]=[w_0w_1\cdots w_{n-1}]$ whose associated normal basis is $\alpha, \alpha^p,\ldots,\alpha^{p^{n-1}}$, we associate  the polynomial $w_0+w_1x+\ldots +w_{n-1}x^{n-1}$, which will therefore be a normal polynomial. 
% The sum of the rows of an invertible circulant matrix is non-zero iff the associated polynomial has non-zero trace.
% }

% For example, since the binary expansion of $7$ with $4$ digits is $0111$ the normal basis $\{\alpha^7,\alpha^{14},\alpha^{13},\alpha^{11}\}$ corresponds to the necklace $[1110]=\{1110,0111,1011,1101\}$.

%Every finite field has a normal basis. 
The number of normal bases of $\GF_{p^n}$ is equal to the number of normal polynomials of degree $n$ over $\GF_p$ and is counted by $\Phi_p(n)$, the generalized Euler totient function, i.e., the function that counts the number of polynomials over $\GF_p$ of degree smaller than $n$ and coprime with $X^n-1$, which can be computed using the formula
\[\Phi_p(n)=p^n\prod_{d | (n/\nu(n))}\left(1-\dfrac{1}{p^{\ord_p(d)}}\right)^{\tfrac{\phi(d)}{\ord_p(d)}}\]
for $n>1$, 
%with $m$ coprime with $p$ such that $n=mp^r$ for some $r$, i.e., $m$ is the quotient of the division of $n$ by the largest power of $p$ dividing $n$, 
where $\nu(n)$ is the exponent of the largest power of $p$ that divides $n$,
and $\ord_p(d)$ is the multiplicative order of $d$ modulo $p$. Although we were not able to find an explicit mention of this formula, it can be simply derived, for example, from~\cite{10.5555/1941953}.

The first few values of the sequences $\Phi_2(n)$ and $\Phi_3(n)$ are presented in Table~\ref{tab:Phi}.

\begin{table}[ht]
\begin{center}
\begin{tabular}{cccccccccccccccccccc}
$n$    & 1 & 2 & 3 & 4 & 5 & 6  & 7  & 8  & 9  & 10 & 11  & 12  \\ \hline
$\Phi_2(n)$  & 1& 2& 3& 8& 15& 24& 49& 128& 189& 480& 1023& 1536\\ \hline
$\Phi_3(n)$  & 1& 4& 18& 32& 160& 324& 1456& 2048& 13122& 25600& 117128 &209952
\end{tabular}
\end{center}
\caption{First few values of $\Phi_2(n)$  and $\Phi_3(n)$ (resp.~sequence A003473 and A003474 in~\cite{oeis}).\label{tab:Phi}}
\end{table}

Let $p$ be a prime, and consider words in $\Sigma_p^n$ as vectors in the vector space $\GF_{p^n}$. %Remember that the addition in $\GF_2$ is the XOR. So, for example, in $\GF_{2^3}$ we have $010+111=101$. 
%The \textit{circulant matrix} of a word $w$ is the square matrix $\CM_w$ whose $(i+1)$th row is $\sigma^{i}(w)$. For example, the circulant matrix of $011$ is $\CM_{011}=\begin{pmatrix}
%0\, 1\, 1 \\
%1\, 0\, 1 \\
%1\, 1\, 0
%\end{pmatrix}$.

An $n \times n$-circulant matrix over $\GF_p$ is  in $C(n,p)$ if and only if it is invertible (or nonsingular), if and only if its determinant is non-zero modulo $p$, or, equivalently, if 
%the polynomial $\sum_{i=0}^{n-1}w_ix^i$ is coprime with $x^n-1$ \todo{check}. Or, equivalently, if and only if 
its rows form a normal basis of $\GF_{p^n}$.

Since every element of a normal basis can be chosen as the first row of a corresponding invertible circulant matrix, the order of $C(p,n)$ is $\Phi_p(n)$, and the order of $\RG_p^n$ (hence, the number of invertible necklaces) is $\Phi_p(n)/n$.
%the set of invertible $n \times n$-circulant matrices over $\GF_p$ forms, with respect to matrix multiplication, an Abelian group $C(p, n)$, whose order is $\Phi_p(n)$ (since any element of the normal basis can be chosen as the first row of the matrix). However, there are $\Phi_p(n)/n$ ``canonical'' elements if one considers only matrices whose first row is the lexicographically biggest element in the conjugacy class, i.e., if one considers the group $C(p,n)/\langle Q_n \rangle$, where $Q_n$ is the permutation matrix of the cycle  $(0 \, 1\, \ldots \, n-1)$. This group is called the \emph{Reutenauer group} $\RG_p^n$ in \cite{DuzhinJMS}. %The elements of the Reutenauer group can therefore be viewed as basis change matrices.  

\begin{comment}
The cardinality of the largest orbit is equal to the order of the whole Reutenauer group $\RG_p^n$, i.e., to $\Phi_p(n)$. FALSE for p>2!
 For every $\CM_w\in \RG_p^n$ and every primitive word $v$ of length $n$, one has that $\CM_w\cdot v^T$ is a conjugate of $v$.
For example, $\CM_{1110}\cdot (0110)^T=\begin{pmatrix}
1110 \\
0111 \\
1011\\
1101
\end{pmatrix}\cdot
\begin{pmatrix}
0 \\
1 \\
1\\
0
\end{pmatrix}=(0011)
$.
\todo[inline]{This way it FIXES necklaces!}
\begin{lemma}
The group $RG_2^n$ acts on $PN_n$ by the group action as follows: given an invertible necklace $[r]$ and an aperiodic necklace $[v]$, one defines $[r]\cdot [v]$ as the necklace associated with the matrix $\CM_r\cdot \CM_v$.
\end{lemma}
\begin{proof}
It is clear that invertible words act on words through circulant matrix multiplication. This action preserves word rotation: 
indeed, for every $C(w)\in N_n$ and every word $v$ of length $n$, one has that $C(w)\cdot v^T$ is a conjugate of $v$. This induces an action of $RG_2^n$ on necklaces. Lemma~\ref{} implies that this action preserves the set of primitive necklaces: \todo{actually I'm not sure anymore of the argument that would not use field theory. I wanted to say that the product of two invertible matrices is invertible and use the lemma saying that invertible matrices are primitive words. But the converse is not true and we would need the converse here, for the cases where the primitive necklace that we multiply is not invertible.}
\end{proof}
Clearly, since $10^{n-1}$ is invertible for every $n$, and its associated circulant matrix is the identity matrix $Id$, the first orbit of the action of $\RG_2^n$ coincides with the set of invertible elements of length $n$.
The action of $\RG_2^n$ on the orbit of $10^{n-1}$ therefore reveals the structure of the group $\RG_2^n$.
One has, for example, $\RG_2^4 \cong \mathbb{Z}_2$ and $\RG_2^8 \cong \mathbb{Z}_2^2 \oplus \mathbb{Z}_4$.
\end{comment}




%\subsection{Invertible Words}

We therefore arrive at the following characterization:

\begin{theorem}\label{thm:main-iso}
     Let $[w]$ be an aperiodic necklace  over $\Sigma_p$, $p$ prime, and let $w$ be the lexicographically biggest word in $[w]$. TFAE:
    \begin{enumerate}
        \item The necklace $[w]$ is invertible;
        \item Any word in $[w]$ has invertible circulant matrix over $\GF_p$;
        %\item Any word in $[w]$ has invertible BWT matrix over $\GF_p$;
        \item $\CM_{[w]}$ belongs to the Reutenauer group $\RG_p^n$;
        \item Any word in $[w]$ is a vector corresponding to a normal element of $\GF_{p^n}$;
        \item Every word of length $n$  over $\Sigma_p$ can be written in a unique way as sum (modulo $p$) of words in $[w]$.
    \end{enumerate}
\end{theorem}

%We call such an aperiodic necklace an \emph{invertible} necklace.

% Since the conjugacy class of a word is the same as the cyclotomic orbit of the corresponding element in the field, an invertible necklace of length $n$ corresponds to a normal basis of the vector space $\GF_{p^n}$. 

\begin{example}
Let $p=2$ and $n=4$. We have three aperiodic binary necklaces, namely $[1000]$, $[1100]$ and $[1110]$. Out of them, the first and the last are invertible, while the second is not, since $\det \begin{pmatrix}
1\, 1\, 0\, 0\\
0\, 1\, 1\, 0 \\
0\, 0\, 1\, 1\\
1\, 0\, 0\, 1
\end{pmatrix}=0$. Indeed, $1100+0110+0011=1001$ in $\GF_{2^4}$, so the elements are not linearly independent as vectors, and therefore do not form a basis of $\GF_{2^4}$.

The Reutenauer group $\RG_2^4$ is therefore constituted by the matrices $Id=\CM_{[1000]}=\begin{pmatrix}
1\, 0\, 0\, 0 \\
0\, 1\, 0\, 0 \\
0\, 0\, 1\, 0\\
0\, 0\, 0\, 1
\end{pmatrix}$
and
$\CM_{[1110]}=\begin{pmatrix}
1\, 1\, 1\, 0 \\
0\, 1\, 1\, 1 \\
1\, 0\, 1\, 1\\
1\, 1\, 0\, 1
\end{pmatrix}$.
\end{example}

So, the set of aperiodic necklaces of length $n$ over $\Sigma_p$ with invertible BWT matrix forms an abelian group isomorphic to the Reutenauer group $\RG_p^n$, but the multiplication has to be carried out with the circulant matrices rather than with BWT matrices. This was also speculated in a remark we recently found at the end of \cite{DBLP:journals/corr/abs-2409-07974} (see also the related paper \cite{DBLP:journals/corr/abs-2409-09824}).

\begin{remark}\label{rk:weight}
The weight of a word being invariant under shift, the sum of the elements in any given row (resp.~column) of $\CM_{[w]}$ equals the weight of $w$. 
 Therefore, if $[w]$ is an invertible necklace, then by Theorem~\ref{thm:main-iso} the words in $[w]$ are the roots of a normal polynomial. Hence, the sum of the words (as vectors) in $[w]$ is the trace of a normal polynomial. In particular, this trace is a vector having all components equal to the weight of $w$ modulo $p$. Since the matrix $\CM_{[w]}$ is invertible, this trace is nonzero. In particular, binary invertible necklaces have an odd number of $1$s. However, in general, having non-zero trace is not a sufficient condition for an element to be normal.
 % \red{Furthermore, given an invertible necklace $[w]$, the weight of $[w]$ is equal to the sum of each row of $\CM_{[w]}$. In turn, from Theorem~\ref{thm:main-iso}, $[w]$ corresponds to a normal basis of the field $\GF_{p^n}$, whose elements correspond to, by definition, both the rows of $\CM_{[w]}$ and the roots of a normal polynomial $f$. Hence, the trace of the normal polynomial $f$ can be written as a vector whose components are sums of rows of $\CM_{[w]}$, which are all equal to the weight of $[w]$.}
\end{remark}

Recall that a prime $q$ is called \emph{$p$-rooted} if $p$ is a generator of the cyclic group $\GF_{q}^*$. For example, $5$ is $2$-rooted since 
% $\{2\mod 5,2^2\mod 5,2^3\mod 5,2^4\mod 5\}=\{2,4,3,1\}=(\GF_{5})^*$, 
$\{2^i\mod 5 \mid i=1,\ldots,4\}=\GF_{5}^*$, 
while $7$ is not $2$-rooted since 
%$\{2\mod 7,2^2\mod 7,2^3\mod 7,2^4\mod 7, 2^5\mod 7, 2^6\mod 7\}=\{2,4,1\}$.
$\{2^i\mod 7\mid i=1,\ldots,6\}=\{1,2,4\}$.


    When $q$ is a $p$-rooted prime, an irreducible polynomial of degree $q$ over $\GF_{2^q}$ is normal if and only if its trace is non-zero~\cite{DBLP:journals/tit/PeiWO86}. In \cite{CTR01}, the authors obtained the following remarkable result:

\begin{theorem}\label{thm:remarkable}
    If every degree $n$ irreducible polynomial over $\GF_p$ with non-zero trace is normal, then $n$ is either a power of $p$ or a $p$-rooted prime.
\end{theorem}

The latter result, together with Remark~\ref{rk:weight}, immediately implies the following

\begin{corollary}
     If every aperiodic necklace of length $n$ over $\Sigma_p$ with non-zero weight modulo $p$ is invertible, then $n$ is either a power of $p$ or a $p$-rooted prime.
\end{corollary}

% So, the sequence of $n$'s for which every degree $n$ irreducible polynomial over $\GF_p$ with nonzero trace is normal (i.e., for which every aperiodic necklace with non-zero weight modulo $p$ is invertible) is  the union of powers of $p$ with the $p$-rooted primes. 

For example, the sequence of $2$-rooted primes is A001122 in OEIS~\cite{oeis}. This latter is also the sequence of numbers of the form $2m+1$ such that $(10)^m$ is a BWT image (see \cite{Aulicino01061969,DBLP:journals/tcs/MantaciRRSV17}), i.e., lengths for which one can have a BWT that is completely anti-clustered, that is, has the largest possible number of runs. 
\begin{comment}
    This follows from the fact that the inverse standard permutation of $(10)^m$, written as a permutation of $\{1,\ldots,2m\}$, is equal to $(1\ 2\ 2^2\ \ldots\ 2^{2m}) \mod (2m+1)$, 
%$(2^i \mod (2m+1))_{i=0,\ldots, 2m}$, 
i.e., $2$ generates $\Z_{2m+1}$.
\end{comment}

A famous conjecture of Emil Artin states that the sequence of $p$-rooted primes is infinite for every $p$. So, one can restate Artin's conjecture as follows:

\begin{conjecture}
    There are infinitely many lengths $n$, different from a power of $p$, for which every aperiodic necklace of length $n$ over $\Sigma_p$ with non-zero weight modulo $p$ has invertible BWT matrix.
\end{conjecture}



\section{Generalized de Bruijn words and invertible necklaces}\label{sec:final}
%%%%%%%%%%%%%%%%%%%%%%%%%%%%%%%%%%%%%%%%%%%%%%%%%%%%%%%%%%%%%%%%%%%%%%%%%%%%%%%%%%%%

Every finite abelian group $G$ is isomorphic to the direct sum of cyclic groups, i.e., $G\cong \bigoplus_i \Z_{d_i}$. One has, for example, $\RG_2^4 \cong \mathbb{Z}_2$ and $\RG_2^8 \cong \mathbb{Z}_2^2 \oplus \mathbb{Z}_4$ (remember that the order of  $\RG_p^n$ is $\Phi_p(n)/n$).
%Notice that the order of $\RG_p^n$ is $\Phi_p(n)/n$; so for example, the order of  $\RG_2^8$ is $\Phi_2(8)/8=16$.

 Chan, Hollmann and Pasechnik~\cite{ECCGTA13} proved that the structure of Reutenauer groups can be found by means of an isomorphism with the sandpile groups of generalized de Bruijn graphs, as reported in the next theorem.
 
% Chan, Hollmann and Pasechnik~\cite{ECCGTA13} studied the sandpile group of generalized de Bruijn graphs  $\DB(p, n)$ for every prime $p$ and any $n$, and proved the following result, that connects the objects defined in Sections~\ref{sec:db} and~\ref{sec:invertible}:
 
 \begin{theorem}\label{thm:KDB-RG}
    Let $K(\DB(p,n))$ be the  sandpile group of the generalized de Bruijn graph $\DB(p,n)$, $p$ prime. Then 
    \[ K(\DB(p,n)) \oplus \Z_{p-1} \cong \RG_p^{n},\] 
    where $\RG_p^n\cong C(p,n)/\langle Q_n \rangle$ is the $n$-th Reutenauer group over $\GF_p$.
 \end{theorem}

As a consequence, for $p$ prime,  $K(\DB(p,n))\cong  C(p, n)/(\Z_{p-1}\times \Z_n)$ and therefore  $\kappa(DB(p,n))=\dfrac{\Phi_p(n)}{n(p-1)}$.
%%%%%%%%%%%%%%%%%%%%%%%%%%%%%%%%%%%%%%%%%%%%%%

%\subsection{Graphs}
%%%%%%%%%%%%%%%%%%%%%%%%%%%%%%%%%%%%%%%%%%%%%%
\begin{comment}
Let $ G = (V , E)$ be a finite directed graph, which may have loops and multiple edges. Each edge
$e \in E$ is directed from its source vertex $s(e)$ to its target vertex $t(e)$. 

The directed \emph{line graph} (or \emph{edge graph}) $\mathcal{L} G =(E, E')$ of $G$ has as vertices the edges of $G$, and as edges the set
$$E' = \{(e_1, e_2) \in E \times E \mid s(e_2) = t(e_1)\}.$$

An oriented \emph{spanning tree} of $G$ is a subgraph containing all of the vertices of $G$, having no directed cycles, in which one vertex, the root, has outdegree $0$, and every other vertex has outdegree $1$. The number $\kappa(G)$ of oriented spanning trees of $G$ is sometimes called the \textit{complexity} of $G$. 

A directed graph is \emph{Hamiltonian} if it has a \emph{Hamiltonian cycle}, i.e., one that traverses each node exactly once, while it is \emph{Eulerian} if it has an \emph{Eulerian cycle}, i.e., one that traverses each edge exactly once. As it is well known, a directed graph is Eulerian if and only if $indeg(v) = outdeg(v)$ for all vertices $v$. If a graph is Eulerian, its line graph is Hamiltonian.
\end{comment}

\begin{comment}
\begin{theorem}[BEST theorem]
 The number of distinct Eulerian cycles starting at node $v_i$ in a Eulerian directed graph $G=(V,E)$ is given by
\begin{equation}\label{BEST}
  e(G)=\kappa(v_i)\prod_{j=1}^n(d_{v_j}-1)!
\end{equation}
 where $d(v_j)$ is the outdegree of $v_j$ and $\kappa(v_i)$ is the number of spanning trees oriented towards $v_i$ (the number is the same for any $i$).
\end{theorem}

The number of spanning trees can be computed by means of a determinant, as an application of the matrix-tree theorem below. % (which holds in general for any directed graph, not necessarily Eulerian --- the number of spanning trees may differ from a node to another in general, but it is the same for all nodes in an Eulerian graph). 

The \emph{Laplacian matrix} of a directed graph $G$ is the $n\times n$ matrix $L_G=D_G-A_G$, where $D_G$ and $A_G$ are the degree matrix and the adjacency matrix of $G$, respectively. It can be defined by
$$L_{ij}=\begin{cases}
                    d_{v_i} - d_{v_iv_i}  & \mbox{ if }   i = j \\
                    - d_{v_iv_j} & \mbox{ if }   i \neq j
                 \end{cases}$$
where $d_{v_iv_j}$ is the number of edges directed from $v_i$ to $v_j$ (that is, $0$ or $1$), and $d_{v_i} = \sum_j d_{v_iv_j}$ is the outdegree of $v_i$, so that  $d_{v_i} - d_{v_iv_i}$ is the outdegree of $v_i$ minus the number of its self loops.

\begin{theorem}[Matrix-Tree Theorem]\label{K}
 The number $\kappa(v_i)$ of spanning trees oriented towards node $v_i$ is equal to the determinant of the matrix $L_{-i}$ obtained from the Laplacian matrix of $G$ by removing the $i$-th row and the $i$-th column. 
\end{theorem}

Let $L_G$ be the Laplacian matrix of an Eulerian  graph $G$ (since the graph is Eulerian, the matrix is the same for any vertex) and $L_G^0$ be the matrix obtained by removing the last  row and column, sometimes called the \textit{reduced Laplacian matrix} of $G$. By the  matrix-tree theorem, we have $\kappa(G)=\det (L_G^0)$. 


%\subsection{de Bruijn Graphs  and de Bruijn Words}
%%%%%%%%%%%%%%%%%%%%%%%%%%%%%%%%%%%%%%%%%%%%%%
%\bigskip 

Let $\Sigma_k=\{0,1,\ldots,k-1\}$. The \emph{de Bruijn graph} $\DB(k,k^d)$ of order $d$ over $\Sigma_k$ is the directed graph whose vertices are the words of length $d$ over $\Sigma_k$ and there is an edge from $aw$ to $v$ labeled by $a$ if and only if $v=wb$ for some letter $b$. Equivalently, one can think of vertex labels as the base-$k$ representations of integers in $\{0,1,\ldots,k^d-1\}$ and there is an edge from $m$ to $km+i \mod (k^d)$ for every $i=0,1,\ldots,k-1$. de Bruijn graphs are Eulerian and Hamiltonian. One has $\mathcal{L}\DB(k,k^d) = \DB(k,k^{d+1})$.

A \emph{de Bruijn word} of order $d$ over $\Sigma_k$ is a necklace that labels a Hamiltonian cycle in a de Bruijn graph of order $d$ over $\Sigma_k$, or, equivalently, an Eulerian cycle in a de Bruijn graph of order $d-1$. It therefore has length $k^d$.

%\subsection{Smith Normal Form and Sandpile Groups}
%%%%%%%%%%%%%%%%%%%%%%%%%%%%%%%%%%%%%%%%%%%%%%

Recall that any integer matrix $A$ can be \red{written into a canonical form $A=PDQ$, where both matrices $P$, $Q$ are integer matrices and $D$ is an integer diagonal matrix}%diagonalized into a matrix with nonzero entries only on the main diagonal 

with the property that if $d_1,\ldots, d_n$ are the nonzero entries of the main diagonal \red{of $D$}, then $d_i | d_{i+1}$ for every $i$. \red{The matrix $Q$} is called the \emph{Smith Normal Form} of $A$. The $d_i$ are called the \emph{invariant factors} of the Smith Normal Form.
See~\cite{DBLP:journals/jct/Stanley16} for more details.

Let $G=(V,E)$ be a finite strongly connected directed graph, that is, for any $v, w \in V$ there are directed paths in $G$ from $v$ to $w$ and from $w$ to $v$. Then associated to any vertex $v$ of $G$ is an Abelian group $K(G, v)$, the \textit{sandpile group} (also known as critical
group, Picard group, Jacobian, and group of components). The sandpile group can be obtained from the Laplacian matrix $L_G$. It carries the same information as the Smith Normal Form of $L_G$. By the matrix-tree theorem, its order is the number of oriented spanning trees of $G$ rooted at $v$, $\kappa(G,v)$. When $G$ is Eulerian  the groups $K(G, v)$ and $K(G, u)$ are isomorphic for any $v,u\in V$, and we  denote the sandpile group of $G$ just by $K(G)$, whereas its order is  $\kappa(G)=\det(L_G^0)$, the determinant of the reduced Laplacian matrix of $G$.

Now, $d_1,\ldots, d_{n-1}$ are the invariant factors of the Smith Normal Form of $L_G^0$ if and only if $d_1,\ldots, d_{n-1},0$ are  the invariant factors of $L_G$~(see~\cite{DBLP:journals/jct/Stanley16}). The sandpile group of $G$ is therefore
\begin{equation}\label{isomorphism}
    K(G)\cong \Z^{n-1} /L_G^0 \cong \bigoplus_{i=1}^{n-1} \Z_{d_i}. 
\end{equation}
\end{comment}



%\subsection{Generalized de Bruijn Graphs}
%%%%%%%%%%%%%%%%%%%%%%%%%%%%%%%%%%%%%%%%%%%%%%

\begin{comment}
The \emph{generalized de Bruijn graph} $\DB(k,n)$ has vertices $\{0,1,\ldots,n-1\}$ and for every vertex $m$ there is an edge from $m$ to $km+i \mod (n)$ for every $i=0,1,\ldots,k-1$. They have been introduced independently by Imase and Itoh~\cite{DBLP:journals/tc/ImaseI81}, and by Reddy, Pradhan, and Kuhl~\cite{Reddy} (see also~\cite{DBLP:journals/networks/DuH88}). Generalized de Bruijn graphs are Eulerian and Hamiltonian. By definition, a generalized de Bruijn graph $\DB(k,n)$ is an ordinary de Bruijn graph when $n=k^d$ for some $d>0$.

\begin{figure}
    \centering
  \includegraphics[width=13cm]{DB(3,6)-ipe.pdf}
    \caption{The generalized de Bruijn graph $\DB(3,6)$. }
    \label{fig:DB36}
\end{figure}


The line graph of $\DB(k,n)$ is $\DB(k,kn)$~\cite{DBLP:journals/dm/LiZ91}. Therefore, the Eulerian cycles of $\DB(k,n)$ correspond to the Hamiltonian cycles of $\DB(k,kn)$.

As an example, the generalized de Bruijn graph $\DB(3,6)$ is displayed in Fig.~\ref{fig:DB36}.%\todo{put vertex labels inside} 
Its Laplacian matrix is 
    \[L_{\DB(3,6)}=\begin{pmatrix}
 2& -1& 0& -1& 0& 0 \\
 0& 3& -1& 0& -1& -1 \\
 0& 0& 2& 0& -1& -1 \\
 -1& -1& 0& 2& 0& 0 \\
 -1& -1& 0& -1& 3& 0 \\
 0& 0& -1& 0& -1& 2
\end{pmatrix}\]
and its Smith Normal Form is
    \[\SNF(L_{\DB(3,6)})=\begin{pmatrix}
 1& 0& 0& 0& 0& 0 \\
 0& 1& 0& 0& 0& 0 \\
 0& 0& 3& 0& 0& 0 \\
 0& 0& 0& 3& 0& 0 \\
 0& 0& 0& 0& 3& 0 \\
 0& 0& 0& 0& 0& 0
\end{pmatrix}\]

  Hence, by \eqref{isomorphism}, we have $K(\DB(3,6))\cong \Z_3^3$.

As another example, the Laplacian matrix of $\DB(2,6)$ is 
    \[L_{\DB(2,6)}=\begin{pmatrix}
 1& -1& 0& 0& 0& 0 \\
 0& 2& -1& 0& -1& 0 \\
 0& 0& 2& -1& 0& -1 \\
 0& 0& -1& 2& -1& 0 \\
 -1& -1& 0& 0& 2& 0 \\
 0& 0& 0& -1& 0& 1
\end{pmatrix}\]
Its Smith Normal Form is 
    \[\SNF(L_{\DB(2,6)})=\begin{pmatrix}
 1& 0& 0& 0& 0& 0 \\
 0& 1& 0& 0& 0& 0 \\
 0& 0& 1& 0& 0& 0 \\
 0& 0& 0& 2& 0& 0 \\
 0& 0& 0& 0& 2& 0 \\
 0& 0& 0& 0& 0& 0
\end{pmatrix}\]
 
 Hence, $K(\DB(2,6))\cong \Z_2^2$.
\end{comment}

\begin{table}[ht]
  \begin{center}
    \caption{  The invariant factors of the reduced Laplacian matrix of the generalized de Bruijn graphs for $k=3$, which give the structure of the sandpile groups $K(\DB(3,n))$ and Reutenauer groups $\RG_3^n$.  \label{tabledi}}
    \begin{tabular}{l|l|l|l} % <-- Alignments: 1st column left, 2nd middle and 3rd right, with vertical lines in between
      %\textbf{Value 1} & \textbf{Value 2} & \textbf{Value 3}\\
     $n$ & $d_i$ & $K(\DB(3,n))$ & $\RG_3^n$\\
      \hline
    $3$ & $(1,3)$ & $\Z_3$  & $\Z_3\oplus \Z_2$\\
    $4$ & $(1,1,4)$ & $\Z_4$ & $\Z_4\oplus \Z_2$ \\
    $5$ & $(1, 1, 1, 16)$ & $\Z_{16}$ & $\Z_{16}\oplus \Z_2$\\
    $6$ & $(1, 1, 3, 3, 3)$ & $\Z_3^3$  & $\Z_3^3\oplus \Z_2$\\
    $7$ & $(1, 1, 1, 1, 1, 104)$  & $\Z_{104}$ & $\Z_{104}\oplus \Z_2$\\
    $8$ & $(1, 1, 1, 1, 2, 8, 8)$  & $\Z_{8}^2\oplus \Z_2$  & $\Z_{8}^2\oplus \Z_2^2$\\
    $9$ & $(1, 1, 1, 3, 3, 3, 3, 9)$ & $\Z_{9}\oplus \Z_3^3$ & $\Z_{9}\oplus \Z_3^3\oplus \Z_2$ \\
    $10$ & $(1, 1, 1, 1, 1, 1, 1, 16, 80)$ & $\Z_{80}\oplus \Z_{16}$ & $\Z_{80}\oplus \Z_{16}\oplus \Z_2$\\
    $11$ & $(1, 1, 1, 1, 1, 1, 1, 1, 22, 242)$ & $\Z_{242}\oplus \Z_{22}$ & $\Z_{242}\oplus \Z_{22}\oplus \Z_2$\\
    $12$ & $(1, 1, 1, 1, 3, 3, 3, 3, 3, 3, 12)$ & $\Z_{12}\oplus \Z_{3}^6$ & $\Z_{12}\oplus \Z_{3}^6\oplus \Z_2$
    \end{tabular}
  \end{center}
\end{table}
\begin{comment}
 Chan, Hollmann and Pasechnik~\cite{ECCGTA13} studied the sandpile group of generalized de Bruijn graphs  $\DB(p, n)$ for every prime $p$ and any $n$,  and proved the following result:
 
 \begin{theorem}\label{thm:KDB-RG}
    Let $K(\DB(p,n))$ be the  sandpile group of the generalized de Bruijn graph $\DB(p,n)$, $p$ prime. Then 
    \[ K(\DB(p,n)) \oplus \Z_{p-1} \cong \RG_p^{n},\] 
    where $\RG_p^n\cong C(p,n)/\langle Q_n \rangle$ is the $n$-th Reutenauer group over $\GF_p$.
 \end{theorem}

As a consequence, for $p$ prime,  $K(\DB(p,n))\cong  C(p, n)/(\Z_{p-1}\times \Z_n)$ and therefore  $\kappa(DB(p,n))=\dfrac{\Phi_p(n)}{n(p-1)}$.
\end{comment}
When $n$ is a power of $p$, the precise structure of the sandpile group can be easily given in terms of $p$ and $n$. Indeed, from the results in~\cite{ECCGTA13}, one can derive the following

 \begin{lemma}
For every prime $p$ and any $n\geq 1$ one has:
\begin{equation}\label{eq:SandpilePrime}
    K(\DB(p,p^n))\cong \big[ \bigoplus_{i=1}^{n-1}(\Z_{p^i})^{p^{n-1-i}(p^2-2p+1)}\big]\oplus \Z_{p^n}^{p-2}.
\end{equation}
 \end{lemma}
 
\begin{proof}
We use the formula from~\cite{ECCGTA13} (Theorem 3.1), that describes the group $K(\DB(d,k))$ for arbitrary $d,k$, in the case where $d=p$ is prime and $k=p^n$ for $n\ge 1$:

\begin{align*}
 K(\DB(p,p^n))%&\cong \bigoplus_{i=0}^{n-1}\big[\Z_{p^{i}}\oplus \Z_{p^{i+1}}^{n_i-2n_{i+1}+n_{i+2}-1} \big]\\
 &\cong \big[ \bigoplus_{i=1}^{n-1}\Z_{p^i}\big]\oplus \big[ \bigoplus_{i=0}^{n-2}\Z_{p^{i+1}}^{p^{n-i}-2p^{n-i-1}+p^{n-i-2}-1} \big]\oplus \Z_{p^n}^{p^{n-(n-1)}-2p^{n-(n-1)-1}+p^{n-(n-1)-1}-1} \\             &\cong \big[ \bigoplus_{i=1}^{n-1}\Z_{p^i}\big]\oplus \big[ \bigoplus_{i=0}^{n-2}\Z_{p^{i+1}}^{p^{n-i-2}(p^2-2p+1)-1} \big]\oplus \Z_{p^n}^{p-2} \\
 &\cong \big[ \bigoplus_{i=1}^{n-1}\Z_{p^i}\big]\oplus \big[ \bigoplus_{i=1}^{n-1}\Z_{p^{i}}^{p^{n-i-1}(p^2-2p+1)-1} \big]\oplus \Z_{p^n}^{p-2} \\
 &\cong \big[ \bigoplus_{i=1}^{n-1}\Z_{p^{i}}^{p^{n-i-1}(p^2-2p+1)} \big]\oplus \Z_{p^n}^{p-2} 
 \end{align*}
\end{proof}
 
Note that, as expected, for $p=2$ we obtain \eqref{Levine}.


 % Chan, Hollmann and Pasechnik~\cite{CHAN2015268} extended this result to generalized de Bruijn graphs  $\DB(p, n)$ for every prime $p$,  and proved that  the sandpile group of $\DB(p, n)$ is isomorphic to $C(p, n)/\langle Q_n\rangle \times \GF^*_p$, where $Q_n$ denotes the
% permutation matrix of the $n$-cycle,  $C(p, n)$ the group of invertible $n \times n$-circulant matrices over $\GF_p$, the finite field with $p$ elements, and $\GF^*_p$ is its (cyclic) multiplicative group.

% They also proved that $C(p, n)/\Z_n \cong \RG(p,n)$, and therefore \[\RG(p,n)\cong K(\DB(p,n))\oplus \Z_{p-1}.\]

Actually, Chan, Hollmann and Pasechnik gave the exact structure of $K(\DB(k, n))$ for arbitrary $k$ and $n$~\cite[Theorem 3.1]{ECCGTA13}. We refer the reader to~\cite{ECCGTA13,CHAN2015268} for further details.


 
%%%%%%%%%%%%%%%%%%%%%%%%%%%%%%%%%%%%%%%%%%%%%%%%%%%%%%%%%%%%%%%%%%%%%%%%%%%%%%%%%%%%%%%%
%\section{Generalized de Bruijn Words}
%%%%%%%%%%%%%%%%%%%%%%%%%%%%%%%%%%%%%%%%%%%%%%%%%%%%%%%%%%%%%%%%%%%%%%%%%%%%%%%%%%%%%%%%


% Using the previous formula, one can count the number of generalized de Bruijn words in polynomial time without actually generating the generalized de Bruijn graphs (while A191744, the sequence for $p=5$, in OEIS~\cite{oeis} is marked as hard!).
\begin{comment}
\begin{table}[ht]
\begin{center}
\begin{tabular}{cccccccccccccccccccc}
$n$    & 2 & 3 & 4 & 5 & 6  & 7  & 8  & 9  & 10 & 11  & 12  \\ \hline
$\DBW_2(2n)$  & 1& 1& 2& 3& 4& 7& 16& 21& 48& 93& 128\\ \hline
$\DBW_3(3n)$  & 4& 24& 64& 512& 1728& 13312& 32768& 373248& 1310720& 10903552& 
35831808 
\end{tabular}
\end{center}
\caption{First few values of $\DBW_2(2n)$  and $\DBW_3(3n)$ (resp.~sequence A027362 and A192513 in~\cite{oeis}). \label{tab:DBW}}
\end{table}
\end{comment}
\begin{comment}
\begin{example}
    Let us compute the number of generalized de Bruijn words of length $18=3\cdot 6$ over $\Sigma_3$. The generalized de Bruijn graph $\DB(3,6)$ is displayed in Fig.~\ref{fig:DB36}. Its reduced Laplacian matrix $L_{\DB(3,6)}^0$ has Smith Normal Form
\[\SNF(L_{\DB(3,6)}^0)=\begin{pmatrix}
 1& 0& 0& 0& 0 \\
 0& 1& 0& 0& 0 \\
 0& 0& 3& 0& 0 \\
 0& 0& 0& 3& 0 \\
 0& 0& 0& 0& 3 
\end{pmatrix}\]
and hence determinant $3^3=27$, so $\DB(3,6)$ has $27$ distinct Eulerian cycles (and $\DB(3,18)$ has $27$ distinct Hamiltonian cycles). Thus, there are $27\cdot 2^6=1728$ generalized de Bruijn words of length $18$.
\end{example}
\end{comment}

Hence, we obtain the following formula for generalized de Bruijn words over an alphabet of prime size:
%In the special case when $k$ is a prime, we have the following

\begin{theorem}
Let $p$ be a prime. For every $n\geq 2$, the number of generalized de Bruijn words of length $pn$ over $\Sigma_p=\{0,1,\ldots,p-1\}$ is 
\begin{equation}\label{eq:DBWprime}
    \DBW_p(pn)=\dfrac{((p-1)!)^n}{p-1}\dfrac{\Phi_p(n)}{n}=\dfrac{((p-1)!)^n}{p-1}\InvNeck_p(n)
\end{equation}
where $\InvNeck_p(n)$ is the number of invertible  necklaces of length $n$ over $\Sigma_p$.
\end{theorem}

\begin{proof}
% By the result of Chan, Hollmann and Pasechnik,   $\kappa(\DB(p,n))$ is equal to the order of $K(\DB(p,n))\cong C(p, n)/\langle Q_n\rangle \times \GF^*_p$. Since the order of  $C(p,n)$ is $\Phi_p(n)$, the order of $\langle Q_n\rangle$ is $n$ and the order of  $\GF^*_p$ is $p-1$, we have $\kappa(\DB(p,n))=\dfrac{\Phi_p(n)}{n(p-1)}$. The result then follows from Theorem~\ref{thm:counting}.
We know that $\kappa(\DB(p,n))$ is equal to the order of $K(\DB(p,n))$. By Theorem~\ref{thm:KDB-RG}, $K(\DB(p,n))\cong  C(p, n)/(\Z_{p-1}\times \Z_n)$. Since the order of  $C(p,n)$ is $\Phi_p(n)$, we have $\kappa(\DB(p,n))=\Phi_p(n)/(n(p-1))$. The result then follows from \eqref{eqDBW}.
\end{proof}

\begin{remark}
Let $pn=p^d$, $d> 1$, i.e., $n=p^{d-1}$. Since $\Phi_p(p^{d-1})=p^{p^{d-1}}\dfrac{p-1}{p}$, from \eqref{eq:DBWprime} we get 
\[\DBW_p(p^d)=\dfrac{((p-1)!)^{p^{d-1}}p^{p^{d-1}}}{pp^{d-1}}=\dfrac{(p!)^{p^{d-1}}}{p^d}\]
 i.e., the number of (ordinary) de Bruijn words of order $d$ over $\Sigma_p$.
\end{remark}

In the special case $p=2$ and $n=2^d$, one can summarize Theorems \ref{thm:remarkable} and \ref{thm:KDB-RG} as follows:
%The following theorem, which is a direct consequence of  Theorems \ref{thm:remarkable} and \ref{thm:KDB-RG}, summarizes the results presented in this paper in the case  $p=2$ and $n=2^d$.

\begin{theorem}
    The following sets are in bijection: \begin{itemize}
        \item The set of binary de Bruijn words of length $2^{d+1}$;
        \item The set of necklaces of length $2^d$ having odd number of $1$s, which coincides with the set of invertible necklaces of length $2^d$;
        \item The set of normal bases of $\GF_{2^{2^d}}$;
        \item The set of binary polynomials $f\in \GF_2[x]$, having degree smaller than $2^d$, and coprime with $X^{2^d}-1$.
    \end{itemize}
\end{theorem}


\begin{remark}
Any constructive bijection would allow one to derive algorithms for interchanging  between these objects. One could for example use such a  bijection to generate every  binary de Bruijn word with uniform probability, a problem that is still open --- in \cite{DBLP:conf/latin/LiptakP24} the authors presented an efficient algorithm to generate every  binary de Bruijn word with positive probability.
Moreover, this could be made efficient in practice since the sum in $\GF_2$ is nothing else than the XOR.
\end{remark}



%The previous formula allows one to compute fast (in linear time?) the number of generalized de Bruijn words of length $pn$ over $\Sigma_p=\{0,1,\ldots,p-1\}$ when $p$ is a prime.

% \subsection{old part}
% %%%%%%%%%%%%%%%%%%%%%%%%%%%%%%%%%%%%%%%%%%%%%%%%%%%%%%%%%%%%%%%%%%%%%%%%%%%%%%%%%%%%%%%%

% We say that $x\in \bin^{2n}$ is a \emph{generalized de Bruijn word} if there exists a word $y\in \bin^n$ such that $\BWT(x)=\tau (y)$. 

% We call $2$-\emph{generalized de Bruijn graph} of order $n$ the graph $B_n=(V,E)$ with $V=[1,n]$ and $(i,j)\in E$ if and only if $j\in \{2i,2i+1\} \mod n$.


% \begin{lemma}\label{lem:TM-char}
% Let $u\in\Sigma_a^{an}$ be a word having Parikh vector $(n,\dd, n)$. The following statements are equivalent:
% \begin{enumerate}
%     \item\label{} One has $u\in \{\Perm(\Sigma)\}^n$
%     \item\label{} For each $i\le an$, ($i,\pi_u(i)$) is an edge of the graph $B_{an}$.
%     \item\label{} For each $i\le n$, one has $\{\pi_u(i),\pi_u(n+i),\dd ,\pi_u((a-1)n+i)\}=\{ai,ai+1,\dd,ai+a-1\}$
% \end{enumerate}
% \end{lemma}

% \todo[inline]{The above Lemma generalizes Lemma 8 to non-binary alphabets. We lose the characterization with morphism because every permutation of the alphabet is allowed within a block (not only rotations). This is still interesting in terms of compression of the BWT. In the case of an alphabet of prime length, can we still say something interesting? Another remark: "Every permutation of the alphabet is allowed within a block" - in theory, yes, but if one wants an Eulerian walk on the graph, is there some restriction of what could actually happen? Maybe then one could save the morphism thing and consider the preimages of those words - analogously to the definition of "strange words". todo: look in the literature if the BWT of de Bruijn words can be characterized as we do for length 2 and Tue-Morse}


% \begin{lemma}\label{lem:TM-char}
% Let $u\in\bin^{2n}$ be a binary word having Parikh vector $(n,n)$. The following statements are equivalent:
% \begin{enumerate}
%     \item\label{lem:TM-char1} There exists $y\in\bin^n$ such that $\tau(y)=u$
%     \item\label{lem:TM-char2} One has $u\in \{01,10\}^n$
%     \item\label{lem:TM-char3} For each $i\le 2n$, ($i,\pi_u(i)$) is an edge of the graph $B_{2n}$.
%     \item\label{lem:TM-char4} For each $i\le n$, one has $\{\pi_u(i),\pi_u(n+i)\}=\{2i,2i+1\}$
% \end{enumerate}
% \end{lemma}

% \begin{proof}
% It is clear that \ref{lem:TM-char1}$\iff$ \ref{lem:TM-char2}, and that \ref{lem:TM-char4}$\implies$\ref{lem:TM-char3}. The converse \ref{lem:TM-char3}$\implies\ref{lem:TM-char4}$ follows from the fact that $\pi_u(i)$ is a permutation, hence for every $i\in [1,n]$,  $\pi_u(i)\neq \pi_u(n+i)$. We can deduce observing that, modulo $n$, one has $2(n+i)=2i$ and $(2(n+i)+1)=2i+1$.

% We are left with proving \ref{lem:TM-char2}$\iff$\ref{lem:TM-char4}. 

% Intuitively, $\pi_u(i)=p$ if $p$ is the position of the $i$th letter in $u$, according to the order defined by $\pi_u$. Since $u$ has Parikh vector $(n,n)$, we know that for each $i\in [1,n]$, we have $u[\pi_u(i)]=0$ and $u[\pi_u(n+i)]=1$. One can then reformulate condition \ref{lem:TM-char4} to the following: For each $i\le n$, one has $\{0,1\}=\{u[2i],u[2i+1]\}$. This is clearly equivalent to \ref{lem:TM-char2}, hence \ref{lem:TM-char4}$\implies$\ref{lem:TM-char2}. For the converse, assume the condition~\ref{lem:TM-char2}. This means that for each $i\in [1,n]$, one has either  $u[2i,2i+1]=01$, or $u[2i,2i+1]=10$. Hence, the $i$th occurence of the letter $0$ (and the $i$th occurence of the letter $1$) in $u$ is at position $2i$ or $2i+1$. This exactly means that $\{\pi_u(i),\pi_u(n+i)\}=\{2i,2i+1\}$ for every $i\in [1,n]$, hence condition \ref{lem:TM-char4} is satisfied. 
% \end{proof}

% \begin{lemma}
% Let $x\in\bin^{2n}$ with Parikh vector $(n,n)$ and $u=\BWT(x)$. The following conditions are equivalent:\begin{itemize}
%     \item $x$ is a \gDBw
%     \item The permutation $\pi_u$, written as a $2n$-length cycle, spells the nodes visited by a Hamiltonian cycle of $B_{2n}$.
% \end{itemize}
% \end{lemma}

% %\begin{theorem}
% %Let $n\ge 1$. There is a bijection between the set of generalized de Bruijn words of length $2n$ and the set of Hamiltonian walks on $B_n$.
% %\end{theorem}

% \begin{proof}
% Let $x\in\bin^{2n}$ with Parikh vector $(n,n)$, and $u=\BWT(x)$.%, and let us assume that $x$ is a \gDBw.
% %Let $x$ be a \gDBw of length $2n$, with $u=\BWT(x)$ and let $y\in\bin^n$ such that $\tau(y)=u$.

% From Lemma~\ref{lem:TM-char}, we know that $u=\tau(y)$ for some $y\in \bin^n$ (and equivalently, $x$ is a \gDBw) if and only if ($i,\pi_u(i)$) is an edge of the graph $B_{2n}$ for every $i\in [1,2n]$. Since $u=\BWT(x)$, we deduce from Theorem~\ref{th:BWT cycle} that $\pi_u$, and hence also $\pi_u$, are length-$2n$ cycles - which, by definition of a permutation, contain each integer $i\le 2n$ exactly once. The result follows.

% \begin{comment}
% Let us now assume that $x$ is a \gDBw, namely that there exists $y\in\bin^n$ with $u=\tau(y)$. This means that $u\in\{01,10\}^n$, which gives us more information on the structure of $\pi_u$. Indeed, consider $i\in[1,n]$. One either has $y[i]=0$, in which case $u[2i,2i+1]=01$, either $y[i]=1$, in which case $u[2i,2i+1]=10$. This means that the $i$th occurence of the letter $0$ (and similarly, of the letter $1$) in $u$ is at position $2i$ or $2i+1$. In particular, one has $\{\pi_u(i),\pi_u(n+i)\}=\{2i,2i+1\}$ for every $i\in [1,n]$ - and this exactly means, by definition, that for every $i\in [1,2n]$, the pair $(i,\pi_u(i))\in E(B_{2n})$.

% We can then consider the permutation $\pi_u$ as a $2n$-length cycle on $B_{2n}$. Since it is also a permutation, it visits each vertices exactly once, hence, it is a Hamiltonian cycle. 

% Reciprocally, assume that $\pi_u$ is a Hamiltonian cycle on $B_{2n}$.
% Observe that, for $i\in[1,n]$, one has $\pi_u(i)\in \{2i,2i+1\}$ (by definition of $B_{2n}$), but since one has $2(n+i)=2i \mod n$ and $2(n+i)+1=2i+1 \mod n$, and that the cycle defined by $\pi_u$ is Hamiltonian, we deduce that $\pi_u(i)=2i$ (resp. $\pi_u(i)=2i+1$) if and only if $\pi_u(n+i)=2i+1$ (resp. $\pi_u(n+i)=2i$). This means that, for each $i\in [1,n]$, one has $u[2i,2i+1]\in\{01,10\}$, which means that $u=\tau(y)$ for some string $y\in \bin^n$. Hence, $x$ is a \gDBw. %We can now construct a string $y$ from $\pi$ by setting $y[2i,2i+1]=01$ if $\pi(i)=2i$ and $\pi(n+i)=2i+1$, and $y[2i,2i+1]=10$ otherwise, and have $\pi=\pi_y^{-1}$. 

% %and define a permutation $\pi\in\mathfrak{S}_{2n}$ such that $\pi(i)=p$ if $\gamma$ visits the edge $(i,p)$ (the fact that $\pi$ is indeed a well defined permutation is equivalent to the fact that $\gamma$ is Hamiltonian).  

% \end{comment}
% \end{proof}

% \begin{comment}

% \section{Construction for standard de Bruijn Words}

% \begin{lemma}

% \end{lemma}

% \begin{lemma}
% Let $u$ be a de Bruijn word of length $n=2^k$, and let $x=\BWT(u)$. The standard permutation $\pi_x$ can be written as a unique cycle $(p_1\dd p_n)$. We write $v$ for the word obtained by replacing each $0\le p_i < 2^k-1$ by $0$ and each $2^k-1\le p_i < 2^k$ by $1$, and $w$ for the word obtained by replacing each $p_i$ by $p_i \mod 2$. We have $[u]=[v]=[w]$.
% \end{lemma}

% \begin{proof}
% \todo[inline]{$[u]=[v]$ follows from the known result (this corresponds to the LF mapping). $[u]=[w]$ is more mysterious. We can prove that doing that in terms of the whole set works, because $[v]$ is how we construct generalized \DBW, in the case of a classical \DBW, and $[w]$ corresponds to reading the \DBW along the walk. Experimentally the equality seems to work element-wise as well. Conjecture: This equality could characterize \DBW. In particular, it seems that it never holds for generalized \DBW.

% Update: After a few experiments, it characterizes another set (we can call it quasi \DBW), that seems to contain \DBW, and (few) other words. For example for length 16 we have 17 words (16 \DBW, and 1 extra, 0000011011111001). Some of those new words are generalized \DBW in our sense, some are generalized \DBW in other's articles sense, but nothing seems to always hold. Up to length 20, there are only 12 words that are quasi \DBW but not \DBW: 1 has length 10, 1 has length 12, 3 have length 14, 1 has length 16, 5 have length 18 and 1 has length 20 }
    
% \end{proof}
% \end{comment}





% %%%%%%%%%%%%%%%%%%%%%%%%%%%%%%%%%%%%%%%%%%%%%%%%%%%%%%%%%%%%%%%%%%%%%%%%%%%%%%%%%%%%%%%%
% \section{The Binary Case}
% %%%%%%%%%%%%%%%%%%%%%%%%%%%%%%%%%%%%%%%%%%%%%%%%%%%%%%%%%%%%%%%%%%%%%%%%%%%%%%%%%%%%%%%%

% As a consequence of the previous results, in the binary case we have a bijection between invertible words of length $n$ and generalized de Bruijn words of length $2n$.

% Since generalized de Bruijn words of length $2n$ have a BWT that is a Thue--Morse image, it is natural to compare Thue--Morse images of binary invertible words with binary generalized de Bruijn words. However, the former always start with $1$ and end with $0$, while the latter always start with $01$ and end with $10$. We could not find an explicit bijection between the two classes of binary words.



% %%%%%%%%%%%%%%%%%%%%%%%%%%%%%%%%%%%%%%%%%%%%%%%%%%%%%%%%%%%%%%%%%%%%%%%%%%%%%%%%%%%%%%%%
% \section{Conclusions and Open Problems}
% %%%%%%%%%%%%%%%%%%%%%%%%%%%%%%%%%%%%%%%%%%%%%%%%%%%%%%%%%%%%%%%%%%%%%%%%%%%%%%%%%%%%%%%%

% Explain how important could be to understand better these connections.

%%
%% Bibliography
%%

%% Please use bibtex, 

%\bibliography{ref}
\begin{thebibliography}{10}

\bibitem{10.5555/1941953}
J\"org Arndt.
\newblock {\em Matters Computational: Ideas, Algorithms, Source Code}.
\newblock Springer-Verlag, Berlin, Heidelberg, 1st edition, 2010.

\bibitem{DBLP:journals/dm/Au15}
Yu~Hin Au.
\newblock {Generalized de Bruijn words for primitive words and powers}.
\newblock {\em Discret. Math.}, 338(12):2320--2331, 2015.
\newblock URL: \url{https://doi.org/10.1016/j.disc.2015.05.025}, \href
  {https://doi.org/10.1016/J.DISC.2015.05.025}
  {\path{doi:10.1016/J.DISC.2015.05.025}}.

\bibitem{Aulicino01061969}
D.~J. Aulicino and Morris Goldfeld.
\newblock A new relation between primitive roots and permutations.
\newblock {\em The American Mathematical Monthly}, 76(6):664--666, 1969.
\newblock \href {https://doi.org/10.1080/00029890.1969.12000297}
  {\path{doi:10.1080/00029890.1969.12000297}}.

\bibitem{Blum83}
Lenore Blum, Manuel Blum, and Michael Shub.
\newblock {Comparison of Two Pseudo-Random Number Generators}.
\newblock In David Chaum, Ronald~L. Rivest, and Alan~T. Sherman, editors, {\em
  Advances in Cryptology}, pages 61--78, Boston, MA, 1983. Springer US.

\bibitem{ECCGTA13}
Swee~Hong Chan, Henk D.~L. Hollmann, and Dmitrii~V. Pasechnik.
\newblock {Critical groups of generalized de Bruijn and Kautz graphs and
  circulant matrices over finite fields}.
\newblock In Jaroslav Ne{\v{s}}et{\v{r}}il and Marco Pellegrini, editors, {\em
  The Seventh European Conference on Combinatorics, Graph Theory and
  Applications}, pages 97--104, Pisa, 2013. Scuola Normale Superiore.

\bibitem{CHAN2015268}
Swee~Hong Chan, Henk~D.L. Hollmann, and Dmitrii~V. Pasechnik.
\newblock {Sandpile groups of generalized de Bruijn and Kautz graphs and
  circulant matrices over finite fields}.
\newblock {\em Journal of Algebra}, 421:268--295, 2015.
\newblock Special issue in memory of Ákos Seress.
\newblock URL:
  \url{https://www.sciencedirect.com/science/article/pii/S002186931400475X},
  \href {https://doi.org/10.1016/j.jalgebra.2014.08.029}
  {\path{doi:10.1016/j.jalgebra.2014.08.029}}.

\bibitem{CTR01}
Yaotsu Chang, T.~K. Truong, and I.~S. Reed.
\newblock {Normal bases over $GF(q)$}.
\newblock {\em Journal of Algebra}, 241:89--101, 2001.

\bibitem{DBLP:journals/networks/DuH88}
Ding{-}Zhu Du and Frank~K. Hwang.
\newblock {Generalized de Bruijn digraphs}.
\newblock {\em Networks}, 18(1):27--38, 1988.
\newblock URL: \url{https://doi.org/10.1002/net.3230180105}, \href
  {https://doi.org/10.1002/NET.3230180105} {\path{doi:10.1002/NET.3230180105}}.

\bibitem{DuzhinJMS}
S.~V. Duzhin and D.~V. Pasechnik.
\newblock Groups acting on necklaces and sandpile groups.
\newblock {\em Journal of Mathematical Sciences}, 200(6):690--697, 2014.
\newblock \href {https://doi.org/10.1007/s10958-014-1960-6}
  {\path{doi:10.1007/s10958-014-1960-6}}.

\bibitem{DBLP:journals/tcs/Higgins12}
Peter~M. Higgins.
\newblock {Burrows-Wheeler transformations and de Bruijn words}.
\newblock {\em Theor. Comput. Sci.}, 457:128--136, 2012.
\newblock URL: \url{https://doi.org/10.1016/j.tcs.2012.07.019}, \href
  {https://doi.org/10.1016/J.TCS.2012.07.019}
  {\path{doi:10.1016/J.TCS.2012.07.019}}.

\bibitem{DBLP:journals/tc/ImaseI81}
Makoto Imase and Masaki Itoh.
\newblock Design to minimize diameter on building-block network.
\newblock {\em {IEEE} Trans. Computers}, 30(6):439--442, 1981.
\newblock \href {https://doi.org/10.1109/TC.1981.1675809}
  {\path{doi:10.1109/TC.1981.1675809}}.

\bibitem{lang02}
Serge Lang.
\newblock {\em Algebra}.
\newblock Springer, New York, NY, 2002.

\bibitem{DBLP:journals/jct/Levine11}
Lionel Levine.
\newblock Sandpile groups and spanning trees of directed line graphs.
\newblock {\em J. Comb. Theory {A}}, 118(2):350--364, 2011.
\newblock URL: \url{https://doi.org/10.1016/j.jcta.2010.04.001}, \href
  {https://doi.org/10.1016/J.JCTA.2010.04.001}
  {\path{doi:10.1016/J.JCTA.2010.04.001}}.

\bibitem{DBLP:journals/dm/LiZ91}
Xueliang Li and Fuji Zhang.
\newblock {On the numbers of spanning trees and Eulerian tours in generalized
  de Bruijn graphs}.
\newblock {\em Discret. Math.}, 94(3):189--197, 1991.
\newblock \href {https://doi.org/10.1016/0012-365X(91)90024-V}
  {\path{doi:10.1016/0012-365X(91)90024-V}}.

\bibitem{DBLP:conf/latin/LiptakP24}
Zsuzsanna Lipt{\'{a}}k and Luca Parmigiani.
\newblock {A BWT-Based Algorithm for Random de Bruijn Sequence Construction}.
\newblock In Jos{\'{e}}~A. Soto and Andreas Wiese, editors, {\em {LATIN} 2024:
  Theoretical Informatics - 16th Latin American Symposium, Puerto Varas, Chile,
  March 18-22, 2024, Proceedings, Part {I}}, volume 14578 of {\em Lecture Notes
  in Computer Science}, pages 130--145. Springer, 2024.
\newblock \href {https://doi.org/10.1007/978-3-031-55598-5\_9}
  {\path{doi:10.1007/978-3-031-55598-5\_9}}.

\bibitem{DBLP:journals/tcs/MantaciRRSV17}
Sabrina Mantaci, Antonio Restivo, Giovanna Rosone, Marinella Sciortino, and
  Luca Versari.
\newblock Measuring the clustering effect of {BWT} via {RLE}.
\newblock {\em Theor. Comput. Sci.}, 698:79--87, 2017.
\newblock URL: \url{https://doi.org/10.1016/j.tcs.2017.07.015}, \href
  {https://doi.org/10.1016/J.TCS.2017.07.015}
  {\path{doi:10.1016/J.TCS.2017.07.015}}.

\bibitem{DBLP:conf/cpm/NelloreW22}
Abhinav Nellore and Rachel~A. Ward.
\newblock {Arbitrary-Length Analogs to de Bruijn Sequences}.
\newblock In Hideo Bannai and Jan Holub, editors, {\em 33rd Annual Symposium on
  Combinatorial Pattern Matching, {CPM} 2022, June 27-29, 2022, Prague, Czech
  Republic}, volume 223 of {\em LIPIcs}, pages 9:1--9:20. Schloss Dagstuhl -
  Leibniz-Zentrum f{\"{u}}r Informatik, 2022.
\newblock URL: \url{https://doi.org/10.4230/LIPIcs.CPM.2022.9}, \href
  {https://doi.org/10.4230/LIPICS.CPM.2022.9}
  {\path{doi:10.4230/LIPICS.CPM.2022.9}}.

\bibitem{DBLP:journals/tit/PeiWO86}
Din~Y. Pei, Charles~C. Wang, and Jim~K. Omura.
\newblock Normal basis of finite field gf(2\({}^{\mbox{m}}\)).
\newblock {\em {IEEE} Trans. Inf. Theory}, 32(2):285--287, 1986.
\newblock \href {https://doi.org/10.1109/TIT.1986.1057152}
  {\path{doi:10.1109/TIT.1986.1057152}}.

\bibitem{Reddy}
S.M. Reddy, D.K. Pradhan, and J.G. Kuhl.
\newblock {Directed graphs with minimum diameter and maximal connectivity}.
\newblock Technical report, School of Engineering, Oakland University, July
  1980.

\bibitem{DBLP:journals/corr/abs-2409-09824}
Christophe Reutenauer and Jeffrey~O. Shallit.
\newblock {Christoffel Matrices and Sturmian Determinants}.
\newblock {\em CoRR}, abs/2409.09824, 2024.
\newblock URL: \url{https://doi.org/10.48550/arXiv.2409.09824}, \href
  {https://arxiv.org/abs/2409.09824} {\path{arXiv:2409.09824}}, \href
  {https://doi.org/10.48550/ARXIV.2409.09824}
  {\path{doi:10.48550/ARXIV.2409.09824}}.

\bibitem{DBLP:conf/cie/RosoneS13}
Giovanna Rosone and Marinella Sciortino.
\newblock {The Burrows-Wheeler Transform between Data Compression and
  Combinatorics on Words}.
\newblock In Paola Bonizzoni, Vasco Brattka, and Benedikt L{\"{o}}we, editors,
  {\em The Nature of Computation. Logic, Algorithms, Applications - 9th
  Conference on Computability in Europe, CiE 2013, Milan, Italy, July 1-5,
  2013. Proceedings}, volume 7921 of {\em Lecture Notes in Computer Science},
  pages 353--364. Springer, 2013.
\newblock \href {https://doi.org/10.1007/978-3-642-39053-1\_42}
  {\path{doi:10.1007/978-3-642-39053-1\_42}}.

\bibitem{oeis}
N.~J.~A. Sloane.
\newblock The {O}n-{L}ine {E}ncyclopedia of {I}nteger {S}equences.
\newblock Available electronically at \url{http://oeis.org}.

\bibitem{DBLP:journals/jct/Stanley16}
Richard~P. Stanley.
\newblock {Smith normal form in combinatorics}.
\newblock {\em J. Comb. Theory {A}}, 144:476--495, 2016.
\newblock URL: \url{https://doi.org/10.1016/j.jcta.2016.06.013}, \href
  {https://doi.org/10.1016/J.JCTA.2016.06.013}
  {\path{doi:10.1016/J.JCTA.2016.06.013}}.

\bibitem{DBLP:journals/corr/abs-2409-07974}
Luca~Q. Zamboni.
\newblock {On Christoffel words and their lexicographic array}.
\newblock {\em CoRR}, abs/2409.07974, 2024.
\newblock URL: \url{https://doi.org/10.48550/arXiv.2409.07974}, \href
  {https://arxiv.org/abs/2409.07974} {\path{arXiv:2409.07974}}, \href
  {https://doi.org/10.48550/ARXIV.2409.07974}
  {\path{doi:10.48550/ARXIV.2409.07974}}.

\end{thebibliography}


\end{document}
