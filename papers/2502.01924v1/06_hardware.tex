\section{\label{case_rc_car}Hardware Experiments - RC Car}

Finally, we consider a real-world a miniature RC car with dynamics modeled as \eqref{eq:dyn_hw}, with $L=23.5 \text{ cm}$, controls $\ctrl=[V,\delta]$ with ranges $V \in [0.7,1.4]\text{ m/s}$ and $\delta \in [-25^{\circ},25^{\circ}]$, and disturbances $d_x,d_y \in [-0.1,0.1]$ to account for model mismatches and state estimation error. The vehicle is tasked with completing laps over the racetrack shown in Fig.~\ref{fig:exp_main_result}.
%
\setlength{\arraycolsep}{2pt} % Adjust spacing as needed
\begin{equation}\label{eq:dyn_hw}
{\fontsize{8.5}{10}\selectfont
\dot{\state}
= \begin{bmatrix} \dot{x} & \dot{y} & \dot{\theta} \end{bmatrix}
= \begin{bmatrix} V \cos(\theta)+ d_x, & V \sin(\theta)+ d_y, & V \tan(\delta) / L \end{bmatrix}
}
\end{equation}
\setlength{\arraycolsep}{5pt} % Reset spacing

We adapt our evaluation metrics for this hardware study to better reflect real-world, single-run applicability. Instead of batch statistics, we measure the \textit{CompTime} for each method -- the time taken to generate and evaluate potential samples within a multiple-lap run. This metric reflects how well-suited each technique is for real-time control. We also report the car's average \textit{Speed} over three laps to measure how aggressive the policy is. The \textit{RelCost} metric remains consistent with the simulations, providing a normalized cost relative to the proposed approach for the methods that managed to maintain safety.

As cost function we use (\ref{eq:cost_mppi_car}), where the first term penalizes going slower than $V_\text{max}=1.4\text{ m/s}$, the second term penalizes the distance from the track's center line, the third term $P(\state)$ penalizes going into the obstacle set, BRT, or decrease in safety, depending on the method.
%, and the last term penalizes going into the BRT. The penalty weights $K_{Obs}$ and $K_{BRT}$ alternate between $(K_{Obs},K_{BRT})=(50,0)$ for the 'Obs costs' and 'Obs costs+ LR filter' cases and $(K_{Obs},K_{BRT})=(0,50)$  for the 'BRT costs' and 'BRT costs+ LR filter' cases, for the proposed method their value is irrelevant as hallucinations are guaranteed not to enter the BRT or obstacle sets. 
%with $l_{center}$ the constant distance between the center of the lane and the nearest edge,
\begin{equation}\small
\label{eq:cost_mppi_car}
S = (V_{max}-V)^2 + K_{c} (l_{center}-\targetfunc(x)) +  P(\state)
\end{equation}
%
The controllers were implemented using JAX \cite{jax} on a laptop equipped with an NVIDIA GeForce RTX 4060. 
% for its parallelization capabilities; the GPU used is a Laptop NVIDIA GeForce RTX 4060 where 
We generate $1000$ parallel hallucinations (with $100$ time steps each) in a loop running at $50Hz$. Results are summarized in Table~\ref{tab:hw_results}, and trajectories for the first lap are shown in Fig~\ref{fig:hw_traj}.

First, we highlight the need for hard safety constraints as the methods that only rely on safety penalties fail to clear the top-left tight turn in the track as shown in Fig~\ref{fig:hw_traj}. Fine-tuning the cost function and MPPI parameters might allow unfiltered methods to complete laps. Still, we want to consider and compare methods that provably allow for safe executions.
%
\begin{table}[t]
\caption{Hardware experiments results summary.}
\centering
\renewcommand{\arraystretch}{1.2} % Adjust row height for vertical centering
\begin{tabularx}{\columnwidth}{|>{\centering\arraybackslash}p{2.5cm}|>{\centering\arraybackslash}X|>{\centering\arraybackslash}p{1.4cm}|>{\centering\arraybackslash}p{2.1cm}|}
\hline
\textbf{Method} & \textbf{CompTime (ms)} & \textbf{RelCost} & \textbf{Speed (m/s)} \\
\hline
Obs costs & 1.8 ± 0.3 & fail & 1.00 ± 0.05 \\
BRT costs & 1.8 ± 0.3  & fail & 1.01 ± 0.06 \\
Obs costs + LRF & 1.7 ± 0.4 & 1.1874 & 1.03 ± 0.12\\
BRT costs + LRF & 1.8 ± 0.4 & 1.1626 & 1.04 ± 0.12 \\
Shield-MPPI & 1.7 ± 0.2 & 1.1038 & 1.04 ± 0.08\\
DualGuard (Ours) & 2.5 ± 0.4 & 1.0000 & 1.10 ± 0.11 \\
\hline
\end{tabularx}
\label{tab:hw_results}
\end{table}
%
The proposed method leads to faster and more performant trajectories than the other safe baselines. A direct comparison with the baselines that also use an output LRF illustrates that the proposed safe hallucination step improved the quality of the samples as exemplified in Fig.~\ref{fig:exp_main_result}(B)(C), leading to a better overall performance and a higher average speed. Also, the proposed method outperforms the Shield-MPPI baseline even after tuning its hyperparameters to the best of our capabilities so that it maintains safety without an excessive impact on performance.

The computation times are nearly identical across all baselines, as each method fundamentally involves calculating performant terms of the cost function and querying the obstacle set or BRT for safety-related penalties. The proposed method introduces an additional LRF step for each sample along hallucinated trajectories, resulting in a slight increase in computational time. Nevertheless, all methods, including the proposed one, operate well within the $20ms$ time budget, leaving ample time for the control loop to handle state estimation, communications, and actuation.
%
\begin{figure}[b] 
\begin{center} 
\vspace{0.0em}
\includegraphics[width=0.925\columnwidth]{fig/hw_traj_v2.png}
\vspace{-0.5em}
\caption{Top view of the RC car's trajectories under each method.}
\label{fig:hw_traj}
\end{center}
\end{figure}

