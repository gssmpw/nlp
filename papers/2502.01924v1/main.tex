\documentclass[letterpaper, 10 pt, conference]{ieeeconf}  
% Comment this line out if you need a4paper

%\documentclass[a4paper, 10pt, conference]{ieeeconf}      % Use this line for a4 paper

\IEEEoverridecommandlockouts      

\setlength{\topmargin}{-10pt}     % Moves content down (increase value to push it down)
\setlength{\textheight}{9.5in}    % Adjust text height (reduce if bottom has extra space)
\setlength{\oddsidemargin}{-10pt} % Left margin for odd pages
\setlength{\evensidemargin}{-10pt} % Left margin for even pages
\setlength{\textwidth}{7in}        % Adjust width (default is 7in for IEEE)
% This command is only needed if 
% you want to use the \thanks command

%\overrideIEEEmargins                                      
% Needed to meet printer requirements.

%In case you encounter the following error:
%Error 1010 The PDF file may be corrupt (unable to open PDF file) OR
%Error 1000 An error occurred while parsing a contents stream. Unable to analyze the PDF file.
%This is a known problem with pdfLaTeX conversion filter. The file cannot be opened with acrobat reader
%Please use one of the alternatives below to circumvent this error by uncommenting one or the other
%\pdfobjcompresslevel=0
%\pdfminorversion=4

% See the \addtolength command later in the file to balance the column lengths
% on the last page of the document

% The following packages can be found on http:\\www.ctan.org
%\usepackage{graphics} % for pdf, bitmapped graphics files
%\usepackage{epsfig} % for postscript graphics files
%\usepackage{mathptmx} % assumes new font selection scheme installed
%\usepackage{times} % assumes new font selection scheme installed
\usepackage{amsmath} % assumes amsmath package installed
%\usepackage{amssymb}  % assumes amsmath package installed
\usepackage{mathtools}
\usepackage{hyperref}
\usepackage{color}
\usepackage[dvipsnames]{xcolor}

\usepackage{graphicx} % For adding images and graphics
\usepackage{array} % For advanced table formatting
\usepackage{tabularx} % For tables that adjust to column width
\usepackage{caption} % For better control of table captions

%\usepackage{showframe}


\newcommand{\thought}[1]{{\color[rgb]{0.2,0.39,0.66}(#1)}}
\newcommand{\todo}[1]{{\color[rgb]{1.0,0.0,0.0}(#1)}}
\newcommand{\hsh}[1]{{\color{green!50!black} Henrik: #1}}
\newcommand{\st}[1]{{\color{red!50!black} Sebastian: #1}}

\newcommand{\ulm}[1]{_{\scaleto{\mathrm{#1}}{3pt}}}
\newcommand\at[2]{\left.#1\right|_{#2}}











\newtheorem{assumption}{Assumption}

\DeclareMathOperator*{\argmax}{arg\,max}
\DeclareMathOperator*{\argmin}{arg\,min}

\newcommand{\swname}[1]{\texttt{#1}}
\newcommand{\ie}{i\/.\/e\/.,\/~}
\newcommand{\eg}{e\/.\/g\/.,\/~}
\newcommand{\cf}{cf\/.\/~}

\newcommand{\fig}{Fig\/.\/~}
\newcommand{\defn}{Def\/.\/~}
\newcommand{\sect}{Sec\/.\/~}
\newcommand{\tabl}{Tab\/.\/~}
\newcommand{\algo}{Algorithm~}
\newcommand{\theo}{Theorem~}

\newcommand{\bnnl}{3 hidden layers}
\newcommand{\bnnn}{50 neurons}
\newcommand{\bnna}{tanh activations}

\newcommand{\capt}[1]{\mdseries{\emph{#1}}}

\newcommand{\videolink}{at \url{https://youtu.be/_d7AqTRjz6g}}
\newcommand{\codelink}{\url{https://github.com/wheelbot/mini-wheelbot}}

\newcommand{\fakepar}[1]{\vspace{0mm}\noindent\textbf{#1.}}

\newcommand{\needref}{\textcolor{red}{[REF]}}

\newcommand{\plotfontsize}{9pt}

\newcommand{\note}{\textcolor{blue}}
\newcommand\ucnote[1]{\textcolor{orange}{[UC: #1]}}

\usepackage{mdframed}
\mdfdefinestyle{MyFrame}{%
    linecolor=black,
    outerlinewidth=2pt,
    %roundcorner=20pt,
    innertopmargin=4pt,
    innerbottommargin=4pt,
    innerrightmargin=4pt,
    innerleftmargin=4pt,
    leftmargin = 4pt,
    rightmargin = 4pt,
    skipabove=10pt,  % Space above the box
    skipbelow=10pt  % Space below the box
    %backgroundcolor=gray!50!white}
        }


        
\usepackage[font=footnotesize,labelfont=bf]{caption}

\title{\LARGE \bf DualGuard MPPI: Safe and Performant Optimal Control by Combining Sampling-Based MPC and Hamilton-Jacobi Reachability }

\author{Javier Borquez$^{1}$, Luke Raus$^{2}$, Yusuf Umut Ciftci$^{1}$, and Somil Bansal$^{3}$ % <-this % stops a space
\thanks{This work is supported by the University of Santiago de Chile, the NSF CAREER Program under award 2240163 and the DARPA ANSR program.}%
\thanks{$^{1}$University of Southern California. \{javierbo, yciftci\}@usc.edu.}%
\thanks{$^{2}$Olin College of Engineering. lraus@olin.edu.}\thanks{$^{3}$Stanford University. somil@stanford.edu.}
}

\begin{document}

\colorlet{usercolorname}{red!20} %make nonzero to highlight
\sethlcolor{usercolorname}

\makeatletter
\let\@oldmaketitle\@maketitle
    \renewcommand{\@maketitle}{\@oldmaketitle
    \centering
    \vspace{1em}
    \includegraphics[width=0.99\textwidth]{fig/hw_banner_safe_mppi_v5.png}
    % \vspace{-0.5cm}
    \captionof{figure}{We propose DualGuard MPPI - a framework to solve optimal control problems for robots subjected to hard safety constraints. Our method integrates safety filtering during the sampling process in MPPI to generate safe hallucinations, ensuring safe executions while improving the exploration and sample efficiency in MPPI algorithms. An output least restrictive filter is used to ensure safe executions on the system, despite potential multimodality in the sampling process. (A) We apply the proposed framework to an RC car experiment where the vehicle completes laps without leaving the track (breaching safety), while trying to stay centered in the lane and going as fast as possible. (B) Hallucinated trajectories in classical MPPI generate only high-cost unsafe executions near a tight corner, which results in breaching the boundary of the track. (C) DualGuard MPPI safe hallucinations generate only collision-free executions with mild costs depending only on performance criteria, resulting in safe and performant behavior. Details on this experiment are provided in Section \ref{case_rc_car}.}
    \vspace{-0.2cm} % this command is important for spacing!!! dont comment or delete it -- zc after long tuning.
    \label{fig:exp_main_result}
    \setcounter{figure}{1}
  % \bigskip
  }
\makeatother

\maketitle


%%%%%%%%%%%%%%%%%%%%%%%%%%%%%%%%%%%%%%%%%%%%%%%%%%%%%%%%%%%%%%%%%%%%%%%%%%%%%%%%
\begin{abstract}
Designing controllers that are both safe and performant is inherently challenging.
This co-optimization can be formulated as a constrained optimal control problem, where the cost function represents the performance criterion and safety is specified as a constraint.
While sampling-based methods, such as Model Predictive Path Integral (MPPI) control, have shown great promise in tackling complex optimal control problems, they often struggle to enforce safety constraints. 
To address this limitation, we propose DualGuard-MPPI, a novel framework for solving safety-constrained optimal control problems.
Our approach integrates Hamilton-Jacobi reachability analysis within the MPPI sampling process to ensure that all generated samples are provably safe for the system. 
On the one hand, this integration allows DualGuard-MPPI to enforce strict safety constraints; at the same time, it facilitates a more effective exploration of the environment with the same number of samples, reducing the effective sampling variance and leading to better performance optimization.
Through several simulations and hardware experiments, we demonstrate that the proposed approach achieves much higher performance compared to existing MPPI methods, without compromising safety.
\end{abstract}

% 
\begin{figure}[ht]
    \centering
    \includegraphics[width=0.8\linewidth]{graphs/greater_than_naive.pdf}
    \vspace{0.5cm}
    \includegraphics[width=0.8\linewidth]{graphs/p1_bottom.png}
    \vspace{-5pt}
    \caption{\textcolor{positional}{Positional} vs.\ \textcolor{nonpositional}{non-positional} circuits. In a \textcolor{nonpositional}{non-positional} circuit, the same edges must be included at all positions. A \textcolor{positional}{positional} circuit can distinguish between the same edge at different positions. This specificity yields better trade-offs between circuit size and faithfulness. It can also increase both precision and recall.}
    \label{fig:p1}
    \vspace{-5pt}
\end{figure}

\section{Introduction}

\looseness=-1
A primary goal of interpretability research is to characterize the internal mechanisms in language models (LMs) and other NLP models. 
A core approach in this area is \textbf{circuit discovery}---identifying the minimal subgraph within the model's computation graph that performs a specific task \citep{olah2021framework,olah-mech}.
Typically, the nodes of a circuit represent model components (e.g., attention heads, neurons, or layers).
While manual circuit discovery methods can yield position-specific insights \citep{wanginterpretability,goldowskydill2023localizingmodelbehaviorpath}, \emph{automatic methods often overlook positional information}, treating components as uniformly relevant across all input token positions \citep{conmytowards,syed2023attribution}. 
For instance, if an attention head is included in a circuit, it is assumed to contribute equally to the computation for every position in the input sequence.
The assumption that circuits are position-invariant ignores the fact that different positions often require distinct computations.
By ignoring positions, current methods limit their ability to capture mechanisms that operate across positions, such as interactions between attention heads across positions.

In this study, we start by demonstrating that positional agnosticism is a significant limitation (\S\ref{sec:motivating}). Then, to address these limitations, we introduce a new approach: position-aware edge attribution patching (PEAP; \S\ref{sec:full_circ_discovery}; Figure~\ref{fig:p1}). Current approaches  assume that if an edge is in a circuit, then the same edge will be in the circuit at all positions, thus leading to low precision. It is also assumed that an edge's importance should be aggregated across positions before deciding whether it should be included in the circuit; this can lead to cancellation effects, and thus low recall. PEAP instead allows us to compute the importance of cross-positional edges, and separately evaluates edge importance at each position. We show that this leads to smaller and more accurate circuits; see Figure~\ref{fig:p1}.

Incorporating positional information into circuit discovery is straightforward when inputs have the same length and structure across examples.

However, realistic datasets are not nearly this templatic.
How, then, can we incorporate positional information into automatic circuit discovery?
To address this challenge, we propose \textbf{schemas} (\S\ref{sec:schema}). 
Schemas assign semantic labels to spans of tokens, enabling information aggregation across examples even when the spans differ in length.

For example, in the input ``The \textcolor{positional}{war} lasted from 1453 to 14\underline{\hspace{1em}},'' the span ``\textcolor{positional}{war}'' could be labeled as ``\emph{Subject}''.
This enables handling spans with varying lengths: the phrase ``\textcolor{positional}{Black Plague}'' in another example can be treated as a single positional span with the same role as ``\textcolor{positional}{war}''.
In experiments with two LMs and three tasks, we find that circuits discovered using schemas achieve a better trade-off between circuit size and faithfulness to the model's behavior than position-agnostic circuits.
Importantly, position-aware circuits offer a more precise representation of the underlying mechanisms, providing a more concise foundation for mechanistic explanations.

We also present a fully automated pipeline for schema generation and application (\S\ref{sec:schema-generation}) using large language models (LLMs). 
We evaluate the quality of the generated schemas and their utility in discovering position-aware circuits (\S\ref{sec:schema-eval}).
Notably, circuits derived using automatically generated and applied schemas achieve comparable faithfulness scores to circuits discovered with human-designed and manually applied schemas.

We summarize our contributions as follows:
\begin{itemize}[noitemsep,leftmargin=*,topsep=1pt,parsep=1pt]
    \item Introduce a position-aware circuit discovery method, which obtains better faithfulness than position-agnostic discovery.  
    \item Introduce dataset schemas,  facilitating positional circuit discovery in more naturalistic settings. 
    \item Develop an automated schema generation and application pipeline with LLMs, yielding schemas that are comparable to manually-annotated ones.
\end{itemize}

 % 
\section{Problem Formulation}
\label{sec:problem_statement}

%

\makeatletter
\renewcommand*{\coloneq}{\mathrel{\rlap{%
  \raisebox{0.3ex}{$\m@th\cdot$}}%
  \raisebox{-0.3ex}{$\m@th\cdot$}}%
  =}
\makeatother


\ifx\coloneqq\undefined
  \makeatletter
  \newcommand*{\coloneqq}{\mathrel{{%
  \raisebox{0.110ex}{$\m@th::$}}}
  =}
  \makeatother
\else
  \makeatletter
  \renewcommand*{\coloneqq}{\mathrel{{%
  \raisebox{0.110ex}{$\m@th::$}}}
  =}
  \makeatother
\fi


\noindent \textbf{Charger and Time Intervals}: Consider the building has $N$ heterogeneous chargers $\mathcal{C} = \{C_1, C_2, \dots,C_N\}$. Each charger $C_i$ has limits on the charging rate, minimum $C_i^{min}$ and maximum $C_i^{max}$; $C_i^{min} < 0$ implies the charger $C_i$ is bi-directional and can discharge and $C_i^{min} = 0$ represents a unidirectional charger with no discharging. We assume that all chargers are designed to be able to charge at maximum rates simultaneously, i.e., $\sum_{i=1}^{i=N} C_i^{max} < \text{maximum rated capacity of the building} $.  
The planning horizon is one billing period, usually a month, which we divide into equal-sized fixed time intervals $\mathcal{T} = \{T_1, T_2, \dots\, T_{end}\}$, where $T_{j}-T_{j-1}=\delta$ (we use  $\delta$ = 0.25 hours). The choice of $\delta$ is user-specific and provides a stable decision epoch, preventing rapid changes in the charging rate.

\noindent \textbf{Charging Power}: Let us assume that the function $\mathcal{P}:  \mathcal{C} \times \mathcal{T}  \rightarrow \Re$ specifies the power consumed by the charger $C_i$ at time $T_j$. If the power is zero, the charger is not active, and if the power is negative, the charger discharges, acting as an energy source. Note that by construction $P(C_i,T_j) \in [C_i^{min},C_i^{max}]$. Let us also assume that function $\mathcal{B}: \mathcal{T}  \rightarrow \Re^{+} $ specifies the average building power consumed in $\delta$ time interval. 
Given the charger and the building power consumption, we can calculate the total cost for the billing period. The parts of the total cost are based on the property type, time of day, and state of the power grid and are based upon the rules and regulations set by the local transmission system operator (TSO) and distribution system operator (DSO). These parts include energy expenses for building power and charging, which vary with peak and off-peak hours, as well as demand charges based on the peak power draw over a longer-term period. 

Let the price of the energy consumed is given by $\theta_E : \mathcal{T}  \rightarrow \Re^{+}$ (in \$/kWh). In practice, the Time-of-Use (TOU) electricity rates do not vary continuously and are rather divided into two parts each day, i.e., a peak and a non-peak period. 
Then, the total cost of the energy consumed is  $\Theta_E(\mathcal{P})= \sum_{j=1}^{j=end} \left(\sum_{i=1}^{i=N} (P(C_i,T_j)) + \Building (T_j)\right) \times \theta_E  (T_j) \times \delta$. Effectively, $\Theta_E$ is a function of charging power  $\mathcal{P}=\{P(C_i, T_j)| C_i\in \mathcal{C}, T_j\in \mathcal{T}\}$.  
 
\noindent \textbf{Demand Charge}: The demand charge is calculated using the maximum (peak) power consumed during any time interval in the billing period, with the demand price denoted as $\theta_D$ (in \$/kW).
Let $P^{max} = \max_{j=1}^{j=end} (\sum_{i=1}^{i=N}$ $P(C_i,T_j)) + \Building(T_j)$ denote the maximum power consumed. The demand charge is given by $\Theta_D(\mathcal{P})= \theta_D \times P^{max} \times \delta$, which is a function of charging power  $\mathcal{P}$. Hence, the total cost of energy bought from the power grid is  $\Theta_E(\mathcal{P})+\Theta_D(\mathcal{P})$. To minimize the cost, we must reduce the net power usage when the cost $\theta_E$ is high and manage the power peaks to ensure $P^{max}$ remains as low as possible. Often, the demand charge is levied to ensure that the industrial buildings do not put excess burden on the power grid. In our problem, we use estimates of peak power and denote it by $\hat{P}^{max}$. It is important to note that the demand charge is typically applied during peak hours of the TOU electricity rate, as reflected in our formulation.
%The challenge with being able to lower the demand charge is the coupled temporal complexity. Assume if along a billing period of one month, it is not possible to reduce $P^{max}$, which will occur in the last week of the month, we may as well utilize high charging powers in early part of the month. 

%Let us now describe the decision problem given the structure above. 
\noindent \textbf{Electric Vehicle Sessions}: Assume that during the billing period $\mathcal{T}$, a set of electric vehicles, denoted as $\mathcal{V}$, are serviced at the building. Each EV $V$ is characterized by its arrival time $\mathcal{A}: \mathcal{V} \rightarrow  \mathcal{T}$ and departure time $\mathcal{D}: \mathcal{V} \rightarrow  \mathcal{T}$. Note that if the same vehicle arrives more than once, we will treat it as a separate session. If the EV arrives between time slots $[T_{i-1}, T_{i}]$, we consider its effective arrival time as $\mathcal{A}(V) = T_i$. Similarly, if the vehicle departs between $[T_{j}, T_{j+1}]$, we consider its effective departure time as $\mathcal{D}(V) = T_j$. EV sessions are contiguous, i.e., EV is expected to remain at the site between $\mathcal{A}(V)$ and $\mathcal{D}(V)$, for $\forall V \in \mathcal{V}$. 
%Additionally, it is important to emphasize that we may know the estimated arrival time $\hat{\mathcal{A}}(V)$ and departure time $\hat{\mathcal{D}}(V)$ for each session, but true arrival and departure times are unknown ahead of time and can only be observed once they happen. 
For each  $V$, we know the initial state of charge   $\SOCI: \mathcal{V} \rightarrow \Re^+$ and the required final state of charge (measured as a percentage of the battery capacity)   $\SOCR: \mathcal{V} \rightarrow \Re^+$ upon arrival. $\SOCMIN: \mathcal{V} \rightarrow \Re^+$ is the minimum allowed SoC for the car i.e., the car cannot be discharged below this value, and $\SOCMAX: \mathcal{V} \rightarrow \Re^+$  is the maximum allowed SoC for the car. The minimum and maximum bounds are specified by the EV manufacturer, considering the impact of charging and discharging on battery health. ${\it CAP}:\mathcal{V} \rightarrow \Re^+$ denotes the vehicle's battery capacity in kWh. We track the current SoC of the EV using ${\it SOC}$, where ${\it SOC}: \mathcal{V} \times \mathcal{T} \rightarrow \Re^+$ and it is defined later.



% For each vehicle session, we record 



% and are assigned to available chargers in $\mathcal{C}$, with each EV represented by $V$.   
% Each EV $V$ is characterized by its arrival time $\mathcal{A}(V)$, its scheduled departure time $\mathcal{D}(V)$, and its energy capacity $C(V)$. Additionally, each EV has a required state of charge, $\SOCR(V)$, and an initial state of charge, $\SOCI(V)$.
% We assume that EVs arrive at continuous time. If an EV arrives between time slots $[T_{j-1}, T_{j}]$, we consider its effective arrival time as $\mathcal{A}(V) = T_j$. Similarly, if an EV departs between $[T_{j}, T_{j+1}]$, we consider its effective departure time as $\mathcal{D}(V) = T_j$.



%lowering the demand charge though is that if we know in future there is no possibility of reducing demand charge then we can 


%We describe the problem as a series of problems starting from the basic V2B problem and expanding to the more complex problem of optimizing charging policy across a month. ~\Cref{table:notations} summarizes the key symbols utilized in the paper.
%
%These costs vary based on the property type, time of day, state of grid, and are based upon the rules and regulations set by the local transmission system operator (TSO) and distribution system operator (DSO). These costs include energy expenses for building power and charging, which vary with peak and off-peak hours, as well as demand charges based on the peak power draw over a longer-term period (e.g., one month). 


%To formalize this problem, we first provide~\Cref{table:notations} summarizing all notations used in this paper. 

% \subsection{Basic Vehicle-to-Building Problem} 
% % For any V2B problem we suppose that there are a set of mixed-mode chargers denoted as $\mathcal{C} = \{C^1, C^2, \dots\}$. Each charger $C_i$ is characterized by available charging powers within the range $[P^i_{\text{min}}, P^i_{\text{max}}]$, where $P^i_{\text{min}} < 0$ if charger $C_i$ is bi-directional. 
% %
% % Simple V2B 
% \textbf{Basic V2B Problem}: The objective of the Vehicle-to-Building (V2B) problem is to optimally schedule Electric Vehicle (EV) charging and discharging throughout a billing period to minimize the total electricity bill, while ensuring that each EV reaches its required state of charge (SoC) (\% of its capacity) by the time of departure. This required SoC represents the portion of the EV's total energy capacity that must be met before it leaves.
 
% %We focus on controlling the charging powers of EV-connected chargers online during the billing period $\mathcal{T}$. We adjust the charging power $P^i_t$ (in kilowatts) of each charger $C_i$ at each time slot $t$, indexed by $\{0, 1, \dots, t_{\text{end}}\}$, corresponding to a fixed time interval $\Delta t$ (in hours). The values $0$ and $t_{\text{end}}$ denote the start and end of the billing period, respectively. 

% % EVs arriving during the billing period $\mathcal{T}$ are assigned available chargers. We let $v$ represent an EV arriving during $\mathcal{T}$, which would be assigned to a charger in $ \mathcal{C}$. 
% %${\it V2C}(v)\in \mathcal{C}$. 
% % Inthe problem, each EV $v$ is specified by its arrival time $T_\mathcal{A}(V)$, predefined departure time $T_\mathcal{D}(V)$, and a power capacity ${\it Cap}(v)$. Additionally, each EV has a required State of Charge (SoC) ${\it SOC}^{req}(v)$ that indicates the desired energy ratio of its power capacity that must be reached before departure, and an initial SoC ${\it SoC^{in}}(v)$. 
% % Here we use $v_t^i$ to indicate the EV connected to charger $C_i$ at time slot $t$. 
% Assume that during the billing period $\mathcal{T}$, a set of EVs, denoted as $\mathcal{V}$, arrive and are assigned to available chargers in $\mathcal{C}$, with each EV represented by $V$.   
% Each EV $V$ is characterized by its arrival time $\mathcal{A}(V)$, its scheduled departure time $\mathcal{D}(V)$, and its energy capacity $C(V)$. Additionally, each EV has a required state of charge, $\SOCR(V)$, and an initial state of charge, $\SOCI(V)$.
% We assume that EVs arrive at continuous time. If an EV arrives between time slots $[T_{j-1}, T_{j}]$, we consider its effective arrival time as $\mathcal{A}(V) = T_j$. Similarly, if an EV departs between $[T_{j}, T_{j+1}]$, we consider its effective departure time as $\mathcal{D}(V) = T_j$.

%which indicates the target energy level (as a percentage of its capacity) that must be reached before its departure, 
%We use $v_t^i$ to represent the EV connected to charger $C_i$ at time slot $t$. The SoC of each charger-connected EV is tracked at each time slot, ${\it SOC}_t(v^i_t)$, based on the charging power of its connected charger, and initialized by ${\it SoC^{in}}(v)$. It is updated according to the following equation: 
\noindent \textbf{Charger Assignment}: 
{
Our approach employs a two-layer decision-making process for EV charging optimization. First, a heuristic assigns EVs to chargers upon arrival. Second, an RL-based policy optimizes charging rates at fixed intervals. 
}
We define an EV assignment function $\eta: \mathcal{V} \rightarrow \mathcal{C}$, where ($V \in \mathcal{V}$) $\eta(V) = C_i$ indicates the charger assigned to EV $V$. Correspondingly, we also maintain a charger-EV occupancy function $\phi: \mathcal{C} \times \mathcal{T} \rightarrow \mathcal{V}$, where $\phi(C_i, T_j) = V$, representing the connection of charger $C_i$ with EV $V$ at time $T_j$.  
The correlation of these two functions can be expressed as $\phi(\eta(V), T_j) = V, \ \text{s.t.}\ \mathcal{A}(V) \leq T_j \leq \mathcal{D}(V) $
indicating that if EV $V$ is assigned to charger $C_i$ through the function $\eta$, then at any time slot within its stay duration, it is confirmed that EV $V$ is connected to charger $C_i$. If no EV is connected to the charger at time $T_j$, the function may return a $\emptyset$ denoting an inactive state, expressed as $\phi(C_i, T_j) = \emptyset$. 
% This highlights the dynamic nature of charger assignments, ensuring that no two EVs share a charger simultaneously. Our FIFO policy prioritizes bidirectional chargers as the optimal strategy (see Table 5 in the Appendix), assigning EVs accordingly to optimize charging efficiency.  
This underscores the dynamic nature of charger assignments, which ensures that no two electric vehicles share a charger simultaneously. Our FIFO policy prioritizes bidirectional chargers as the optimal strategy (see ~\Cref{table:charger_assignment_policies} in the appendix\footnote{The full paper, including the appendix, is available on arXiv.}), enhancing charging efficiency. 
% emphasizing the dynamic nature of the connection function. Note that two vehicles cannot be connected to a charger simultaneously. We consider a first-in, first-out policy that assigns EVs to bidirectional chargers first. 
% , breaking ties assigning to later departing cars, 
% It is also important to emphasize that if the chargers are homogeneous and their count is greater than the number of vehicles, then the assignment problem will be trivial. Otherwise, the assignment problem is part of the decision process, as is our case. 
We also maintain the connection between the assigned charger and the EV until departure. For EV charging, we approximate a linear charging profile, following prior work~\cite{sundstrom2010optimization}. The SoC is updated at each time slot  $T_j$  using the following equation: 
\begin{equation}
{\it SOC}(V, T_{j+1}) = {\it SOC}(V, T_j) + \textstyle\frac{P(\eta(V), T_j)\times \delta} {{\it CAP}(V)}
\label{eq: soc}
\end{equation} 


\iffalse
For the EV assignment approach, we utilize a FIFO (First In, First Out) procedure that prioritizes bi-directional charging and charger ID. This ensures that the earliest arriving EVs are charged or discharged first while maximizing the use of bi-directional chargers, which offers the potential for peak shaving and demand charge reduction. Here, we use the function $\text{ID}(C_i)$ to indicate the charger ID of $C_i$, prioritizing EV assignment to bi-directional chargers by using smaller ID numbers. To enable this FIFO procedure, we set constraints to ensure the FIFO procedure: 
\begin{equation}
    \mathcal{A}(V) < A(V_{k+1}) \implies \text{ID}(\eta(V)) < \text{ID}(\eta(V_{k+1}))
\end{equation} 
This constraint indicates that if the arrival time of EV $V$ is earlier than that of EV $V_{k+1}$, then the ID of the charger assigned to $V$ must also be smaller than the ID of the charger assigned to $V_{k+1}$.  


We then track the state of charge (SoC) of EVs after they connect to chargers based on the charging power of the chargers. We define the function $SoC: \mathcal{V} \times \mathcal{T} \rightarrow [0, 1]$. The SoC update function is given by  
\begin{equation}
SoC(V, T_{I+1}) = SoC(V, T_j) + \frac{P(\eta(V), T_j) \times\delta }{C(V)},
\end{equation} 
with time slot $T_j \geq \mathcal{A}(V)$ and $T_{j+1} \leq \mathcal{D}(V)$, and $P(\eta(V), T_j)$ represents the charging power of the charger assigned to EV $V$ at time $T_j$, and $\Delta $ is the time interval.  
% We track the SoC of each charger-connected EV at each time slot, ${\it SOC}_t(v^i_t)$, based on its connected charger's charging power, initialized by ${\it SoC^{in}}(v)$. It is updated according to: 
% \begin{equation}
%     % {\it SOC}_{t+1}(v) = {\it SOC}_t(v) + P^{{\it V2C}(v)}_{t}\delta t / {\it Cap}(v), 
%     {\it SOC}_{t+1}(v_{t+1}^i) = {\it SOC}_{t}(v_t^i) + (P^i_t \times\delta t)/{\it Cap}(v_t^i)
% \label{eq: SoC}
% \end{equation}
% Here we use $v_t^i$ to indicate the EV connected to charger $C_i$ at time slot $t$.  
\fi 



\noindent \textbf{Feasibility}:
The set \textit{Feasible} indicates the feasible solutions that satisfy the following constraints:
\begin{align}
    & \forall C_i \in \mathcal{C}, \forall T_j \in \mathcal{T}: C_i^{min} \leq P(C_i, T_j) \leq C_i^{max} \label{eq:charging_rate} \\
    & \forall C_i \in \mathcal{C}, \forall T_j \in \mathcal{T}, \forall V \in \mathcal{V}: {\it SOC}(V, T_j) \geq \SOCMIN(V)\label{eq:soc_min} \\
    & \forall C_i \in \mathcal{C}, \forall T_j \in \mathcal{T}, \forall V \in \mathcal{V}: {\it SOC}(V, T_j)\leq \SOCMAX(V)\label{eq:soc_max} \\
    & \forall T_j \in \mathcal{T}: \textstyle\sum_{C_i \in \mathcal{C}} P(C_i, T_j) + \mathcal{B}(T_j) \geq 0  \label{eq:building_power}     
\end{align} 
Here, Constraint~(\ref{eq:charging_rate}) guarantees a valid charging action range, Constraints~(\ref{eq:soc_min} and \ref{eq:soc_max}) ensures that each EV's SoC remains within an acceptable range, and Constraint~(\ref{eq:building_power}) ensures that discharging power does not exceed building power. 




\noindent \textbf{Objectives}: 
% \SOCMIN(V) \geq \text{SoC}(V, T_j) \leq \SOCMAX(V), \forall C_i \in \mathcal{C}, \forall T_j \in \mathcal{T} \label{eq:soc_range} \\ 
% & \mathcal{B}(T_j) + \sum_{C_i \in \mathcal{C}} P(C_i, T_j) \geq 0, \quad \forall T_j \in \mathcal{T} \label{eq:building_power} 
% One of the primary objectives is to ensure the vehicles are charged to the requirement by the time they leave. The decision variable is the charger assignment and charging power per interval. % Thus, if $\eta$ is the charger assignment and $\mathcal{P}$ is the charging power decision per charger per interval, then % The objective of the energy trading problem is to maximize the amount of energy traded.
% Formally, an optimal solution to the energy trading problem is
% To fulfill charging requirements, one objective of this V2B problem is to decide the charger charging powers across all time slots, denoted as $\mathcal{P} = \{P(C_i, T_j)$| $C_i \in \mathcal{C}$ and $T_j \in \mathcal{T}\}$. The goal is to minimize the discrepancy between each EV's SoC at departure and its required SoC, denoted as $\Delta_{{\it SOC}}(\mathcal{P})$, computed by:
% \begin{equation}
% \arg\min_{\mathcal{P}}\delta_{{\it SOC}}(\mathcal{P}) = \sum_{V \in \mathcal{V}} \max\left( {\it SOC}^R(V) - {\it SOC}(V, \mathcal{D}(V)), 0 \right)
% \label{eq: soc}
% \end{equation}  
One of our objectives for the V2B problem is to minimize the total cost over the billing period, incorporating the Time-Of-Use (TOU) electricity rates and demand charges. This objective is expressed as:
% $\min_{(\eta,\mathcal{P}) \in \textit{Feasible}}$
% %(\mathcal{B}, \theta_E, \theta_D, \mathcal{V}, \SOCR, \SOCI, \SOCMIN,\SOCMAX, \mathcal{C})} \\  
% $\left( \Theta_E (\mathcal{P}) + \Theta_D(\mathcal{P}) \right)$
\begin{align}
\label{eq: billing}
\begin{split}
\min_{(\eta,\mathcal{P}) \, \in \textit{Feasible}}%(\mathcal{B}, \theta_E, \theta_D, \mathcal{V}, \SOCR, \SOCI, \SOCMIN,\SOCMAX, \mathcal{C})} \\  
\left( \Theta_E (\mathcal{P}) + \Theta_D(\mathcal{P}) \right)
\end{split}
\end{align}

The second objective ensures that vehicles are charged to their requirement, $\SOCR$, by the time they leave.
\begin{align}
\label{eq: soc_penalty}
\begin{split}
\min_{(\eta,\mathcal{P}) \in \textit{Feasible}} \textstyle\sum_{V \in \mathcal{V}} \max(\SOCR(V) - {\it SOC}(V,\mathcal{D}(V)), 0)
\end{split}
\end{align}
% The inner \texttt{max} function guarantees that all EV users' energy requirements are satisfied, even if this leads to overcharging the vehicle, as dictated by the problem's conditions.
The inner \texttt{max} function ensures EV users' energy requirements are met, even if overcharging occurs.
However, in practical scenarios, short stays may make meeting the SoC requirement impossible. To address this, we reformulate the objectives into a multi-weighted framework.
The optimal charger assignment and actions are then determined by optimizing these combined objectives.




% \begin{equation}
% \arg\min_{\mathcal{P}} \left( \Theta_E (\mathcal{P}) + \Theta_D(\mathcal{P}) \right)
% \label{eq: billing}
% \end{equation} 



% We consider two approaches to address the multi-objective aspect of this problem. First, we can set the SoC objective (as shown in Equation~(\ref{eq: soc})) as a constraint, ensuring that the SoC discrepancy is zero at the time of EV departure by requiring $\Delta_{{\it SoC}}(\mathcal{P}) = 0$. However, in practical scenarios, meeting the SoC requirement may be impossible due to short stay durations and high demand. Therefore, we opt for the second approach, which involves using a weighted sum of both objective functions:
% \begin{equation}
% \arg\min_{\mathcal{P}} \alpha \times \left( \Theta_E (\mathcal{P}) + \Theta_D(\mathcal{P}) \right) + \beta \times\delta_{{\it SoC}}(\mathcal{P}) 
% \label{eq: billing} 
% \end{equation} 
% In this method, we assign a high penalty coefficient $\beta$ for missing SoC to prioritize the SoC charging requirements.


% Moreover, we impose the following constraints on  $\mathcal{P}$:  \begin{subequations}

% Here, the set \(\{v_t^i \mid t \in \mathcal{T}, C_i \in \mathcal{C}\}\) represents all arriving EVs connected to chargers during the billing period \(\mathcal{T}\).


% V2B for Industrial Profiles
%Commercial and industrial complexes are often billed differently that residential houses. Their electricity rates often vary across a single day based on a rate structure used by electric utilities. 
% Our V2B scenario for commercial and industrial complexes
% work under Time-of-use (TOU) rates policies electricity bill. TOU rates incentives customers to use electricity during off-peak hours and reduce usage during peak hours. This pricing model reflects the varying cost of generating and delivering electricity at different times of the day, which is influenced by the overall demand on the electrical grid. 
%By adjusting their energy consumption patterns—such as running heavy machinery or charging electric vehicles during off-peak hours—industrial and commercial businesses can take advantage of lower rates and significantly reduce their overall electricity costs.

\iffalse
% \textbf{V2B for Commercial Buildings}: 
% This work is specifically designed for commercial and industrial smart buildings with EV chargers. By optimally controlling the charger charging power $\mathcal{P}$, the secondary objective of the V2B problem is to minimize the total electricity bill, considering both building power consumption and EV charging, {\color{black} which is governed by Time-of-Use (TOU) rate and demand charge policies.}
% TOU rates encourage customers to shift electricity usage to off-peak hours and reduce consumption during peak periods, reflecting the fluctuating costs of electricity generation and delivery based on grid demand.
% We denote the peak hour time slots during the billing period $\mathcal{T}$ as $\mathcal{T}_P$. 


% In addition to kilowatt-hour (kWh) consumption charges, businesses incur {\it demand charge} based on their peak power demand (measured in kilowatts, kW) during the billing period. %These charges ensure utilities maintain the capacity to meet maximum electricity needs. 
% For industrial businesses, managing demand charges is crucial for controlling energy costs. the V2B problem focuses on reducing peak demand by shifting loads, such as discharging EVs to supply building energy needs, thereby lowering overall grid power consumption. Here, we denote the demand price as  ${\it Pr}^{D}$ (in \$/kW).  

% Above all, we design the second objective of the V2B problem to optimize the charging power sequence $\mathcal{P}$ to minimize the total bill over the billing period, incorporating TOU rates and demand charges. This total cost, denoted as $\hat{Cost}(\mathcal{P})$, is computed by: 
% \begin{equation}
% \displaystyle
% \arg\min \hat{\Theta}(\mathcal{P}) = \Theta_E(\mathcal{P}) + \Theta^{D}(\mathcal{P})
% \label{eq:objective_1}
% \end{equation}

% where ${\it B}(T_j)$ denotes the building power at time slot $t$.  %and ${\it Pr}^E_t$ represents the electricity price, which varies over time and is typically higher during peak hours and lower during off-peak hours.  
% and ${\it Cost}^{DC}(\mathcal{P})$ denotes the demand charge, is determined based on the peak power consumption from both building power and charging. This charge can be calculated by: 
% \begin{equation}
%     \Theta_D(\mathcal{P}) =\max_{T_j\in\mathcal{T}_{P}} (\mathcal{B}(T_j)+\sum_{C_i\in\mathcal{C}}{\it P}(C_i, T_j))\times \theta_D 
% \end{equation}. 

% % V2B under uncertainty
% \textbf{V2B Under Uncertainty}: Uncertainty in the V2B problem arises from various factors. User behavior, such as the actual arrival and departure times of EVs, can deviate from the planned schedule due to traffic conditions or individual preferences, complicating predictions. The SoC required by EV users also varies among individuals under different circumstances. Additionally, building power demand can fluctuate significantly across seasons and may experience sudden changes, prompting Transmission System Operators (TSOs) to activate emergency load reduction programs to alleviate grid stress. Each of these uncertainties impact the overall objective, making their consideration essential in the V2B problem. 

%  \subsection{\nissan{} Use Case} 
% Our work aims to collaborate with our industry partner, \nissan{}, an EV manufacturer with a smart building equipped with both unidirectional and bidirectional chargers, as shown in ~\Cref{fig: EV chargers}. We are developing an online EV charging system to minimize the total electricity bill over a billing period of one month under a Time-of-Use (TOU) rate policy for their headquarters. 

% \noindent \textbf{Charger and EV availability}: Currently, {\color{black} Nissan Advanced Technology Center - Silicon Valley (NATCSV), is a smart building that offers 8 unidirectional and 5 bidirectional chargers for employees. Additionally, they manufacture EVs with bidirectional capabilities such as the Nissan Leaf.} These vehicles have battery sizes ranging from 42 kWh to 60 kWh. Thus, several of them are capable of providing substantial power back to the building when needed.

% \noindent \textbf{Service Classification:} Their headquarters fall within the bounds of Silicon Valley Power (SVP), a not-for-profit municipal electric utility owned and operated by the City of Santa Clara, California, United States. Due to their size and energy consumption, NATCSV is classified as a small industrial service under SVP.
% % For commercial and industrial customers whose energy use exceeds 8,000 kWh per month, but whose maximum electric demand does not exceed 4,000 (kW),
% This means that aside from the monthly electricity bills, they are also subject to time-of-use (TOU) rates and demand charges. The presence of these factors make NATCSV a candidate for V2B optimization. TOU introduces variations in the electricity rates, billing them higher during peak hours (6:00AM to 10:00PM) compared to off-peak hours. While demand charge adds a flat rate multiplier to their total bill based on the highest average kW delivery of the 15-minutes interval in which
% such delivery is greater than in any other 15-minute interval in the month.
% %
% The goal is to leverage the combination of bidirectional chargers, EVs, and a smart charging policy to potentially reduce electricity costs and demand charges. 

% \noindent \textbf{Data}:To leverage this opportunity, \nissan{} has collected building demand and EV telemetry data related to vehicle arrival/departure schedules and SoC needs from their headquarters and EVs. They have amassed data from May 2023 to January 2024 (detailed in Section~\ref{ssec:data}), which will be utilized in this paper to ensure the authenticity of the results. 

\fi
% 

\section{Background}
\label{sec:background}

\paragraph{Causal ordering}
In the following, we consider a matrix $\mB \in \R^{p \times p}$ that encodes the structure of a directed acyclic graph (DAG), which means that the non-zero entries of $\mB$ represent the edges of the graph.
It follows that there exist a strictly lower triangular matrix $\mT$ and a permutation matrix $\mP$, referred to as the \emph{causal ordering}, such that $\mB = \mP^\top \mT \mP$.
In general, such an ordering is not unique.

\paragraph{LiNGAM, a model for causal discovery}
Let $\vx \in \R^p$ be a random vector of observations and let $\mB \in \R^{p \times p}$ represent a DAG.
We consider a structural equation model, also known as a functional causal model, where the data follows
\begin{align}
    \label{eq:causal_model_original}
    \vx = \mB \vx + \vs
\end{align}
where the entries $s_1,\ldots,s_p$ of the vector $\vs \in \R^p$ are independent noise terms, called disturbances.
Causal discovery consists in inferring the parameters $\mB$ of the model, from observations of $\vx$.
Yet, the identifiability of such a model and therefore the uniqueness of the inferred $\mB$ is not straightforward.
In fact, it is well-known that Gaussian noises make the model unidentifiable, in general~\citep{richardson2002identifiability,genin2021identifiability}.
A major advance was that \citet{shimizu2006linear} assumed the noise variables to be
\textit{non-Gaussian} leading to their Linear Non-Gaussian Acyclic Model (LiNGAM), which leads to identifiability, as discussed later.
In the model, we can then interpret $\mP$ as representing a reordering of the observations, such that the permuted entries $\mP \vx$ satisfy
\begin{align}
    \label{eq:causal_model}
    \mP \vx = \mT \mP \vx + \mP \vs
\end{align}
where the causal matrix between the $\mP\vx$ is now $\mT$ and thus strictly lower triangular.
This means any entry $(\mP \vx)_j$ is a weighted sum of previous entries $(\mP \vx)_{<j}$ and noise $(\mP \vs)_{j}$.
Because $(\mP \vx)_j$ does not depend on future entries $(\mP \vx)_{>j}$, we say that $\vx$ follows a causal ordering given by $\mP$.

\paragraph{Relation of LiNGAM to ICA}
The LiNGAM model, or any similar model based on~\eqref{eq:causal_model_original}, can be rewritten as a latent variable model, in particular an Independent Component Analysis (ICA) model \citep{Hyvabook} as
\begin{align}
    \label{eq:ica_model}
    \vx = \mA \vs
\end{align}
where, again, the entries in $\vs$ are independent and non-Gaussian, and the ``mixing" matrix $\mA$ expresses how the data is generated from the latents, and is given by $\mA = (\mI - \mB)^{-1}$, where $\mB$ is a DAG. Now, many methods developed for ICA can be used to estimate the matrix $\mA$, but it is important to take this special structure into account~\citep{shimizu2006linear}.
In particular, any ICA algorithm does not directly return the correct matrix $\mA$ but rather a related matrix where the columns of $\mA$ may appear in an arbitrary order.

\paragraph{Identifiability of LiNGAM} 
\citet{shimizu2006linear} showed that the LiNGAM model is identifiable in terms of the matrix $\mB$, with no indeterminacies unlike in basic ICA. A rigorous re-statement of this result is given in the following theorem which we prove for completeness in Appendix~\ref{app:ssec:lingam}.
\begin{theorem}[Identifiability of LiNGAM]
\label{theorem:lingam}
In the statistical model defined by~\eqref{eq:causal_model_original}, the parameter $\mB$ is identifiable, provided that the entries in $\vs$ are mutually independent, that at most one of them is Gaussian, and that $\mB$ is a DAG.
\end{theorem}
A further question that has received less attention is whether the model is identifiable in terms of the causal ordering $\mP$. In fact, it is not in general: specifically, there may exist many permutation matrices $\mP$ and strictly lower triangular matrices $\mT$ such that the generated data has the same distribution and the generating permutation cannot be identified.
For instance, in the degenerate case $\mB = \mT = \vzero$, any permutation matrix $\mP$ is equally valid and gives the same data distribution. As is well-known, a DAG in general defines only a partial order in the sense that for some pairs of variables, we cannot necessarily say which is ``earlier" and which is ``later". Thus, to make the causal ordering well-defined, we need further assumptions, as will be considered below.\footnote{One would argue that in some cases, $\mP$ is not even well-defined, and thus it cannot be identifiable, and it is not appropriate to use the concept of identiability. We prefer here to confound the concepts in the sense that we talk about identifiability of $\mP$ if it is uniquely defined and can be uniquely recovered from the data.
A more rigorous justification could be developed by assuming a hierarchical data generation process, where the $\mP$ and $\mT$ are generated first, and the $\mB$ is generated based on them. In that case, if the decomposition of $\mB$ is unique, and $\mB$ is identifiable, also $\mP$ is identifiable in the conventional statistical sense.\label{footnote:1}}


\paragraph{Multi-view ICA}
A multi-view version of ICA is of great practical interest.  One might obtain a number of views of the same data that might be, for example, different subjects in a biomedical context, different users in more technological applications, or different measurement systems of the same physical phenomenon. 
A multi-view extension of ICA
can then be defined in various ways. Here, we consider the case where the components are, at least partly, shared over views, while the mixing matrices (as well as optional noise terms) are view-dependent. 

This leads to the definition of Shared ICA \citep{richard2021sharedica,anderson2013multiviewidentifiability}:
\begin{align}
    \label{eq:multiview_ica_model}
    \vx^i = \mA^i (\vs + \vn^i)
\end{align}
where $\vx^i$ are the different views indexed by the view index $i$.
Recently, identifiability conditions have been explored for such multi-view ICA models. 
On the one hand, it is obvious that if the components are non-Gaussian, the Shared ICA model reduces to a standard ICA model when stacking the different views in a single vector, and hence identifiable. But \citet{richard2021sharedica,anderson2013multiviewidentifiability} showed the surprising result that even if more than one component is Gaussian, the model can still be identifiable if the variances of the noises $\vn^i$ are sufficiently diverse.

% 
\section{Approach}
\label{sec:approach} 



% \begin{figure*}[t]
%     \centering
%     \begin{subfigure}[b]{0.68\linewidth}
%         % \centering
%         \includegraphics[height=2.2in,keepaspectratio,trim=0.2cm 0.2cm 0.2cm 0.2cm,clip]{figures/framework.pdf}
%         % \vspace{-0.1in}
%         \caption{Reinforcement Learning Framework.}
%        \label{fig:framework}
%     \end{subfigure}
%     % \hfill
%     \begin{subfigure}[b]{0.28\linewidth}
%         % \centering
%         \includegraphics[height=2.2in,keepaspectratio,trim=0cm 0.5cm 0cm 0.5cm,clip]{figures/inference_pipeline.pdf}
%         \vspace{-0.15in}
%         \caption{Pipeline for Inference.}
%         \label{fig:pipeline}
%     \end{subfigure}
%     \caption{\color{black}(a) Our framework relies on daily samples along with an estimated monthly peak power. A reinforcement learning policy based on DDPG is trained with policy guidance and action masking. These techniques improve the performance of the model. Finally, it generates charger actions for every given state. (b) At inference time $T_j$, the model requires data of connected cars, charger states and building load reading. It then uses a monthly peak power estimator to forecast the peak power for the entire billing period. The result are the charger actions for time $T_{j+1}$.}
%     \label{fig:framework_and_pipeline}
% \end{figure*}

% \begin{figure*}[t]
% \centering
% % \includegraphics[width=0.95\textwidth]{figures/framework_only.pdf}
% \includegraphics[width=0.95\textwidth]{figures/framework_v2.pdf}
% \caption{(Left) Data Processing Pipeline, (right) Complete Framework of our RLApproach.}
% \label{fig:framework_and_pipeline}
% \Description{}{}
% \end{figure*}


%We employ the Deep Deterministic Policy Gradient (DDPG) algorithm~\cite{lillicrap2015continuous} to train the policy $\pi(S(T_j))$ for the V2B problem. DDPG is ideal for handling our continuous action space and supports off-policy training, allowing the RL model to learn from diverse experiences across various scenarios, thereby enhancing model generalization. To enhance the policy performance, we propose integrating action masking and policy guidance techniques alongside DDPG, as outlined below. Figure~\ref{fig:framework_and_pipeline}.  

%. This framework enables optimal decision-making regarding charger power rates and load management, allowing for effective responses to uncertainties in user behaviors and building loads in our working scenario. We aim to derive a policy for each time slot that minimizes total costs while ensuring that the SoC requirements of the EVs are met throughout the billing period. 

%Considering the uncertainty in the V2B problem, we model it as a sequential decision-making problem using a Markov Decision Process (MDP) to simulate state transitions and utilize an RL-based approach to train a policy for this MDP. 

In this section, we first outline the data pipeline, focusing on how we handle the monthly training data, generate input features, and execute the policy (Figure~\ref{fig:framework_and_pipeline}) and then detail the MDP formulation. \ad{while the figure talks about delayed update, its discussion clearly in a highlighted paragraph is missing. This is an important part and hence must be emphasized.}
%, then discuss the steps of the data processing pipeline, 
 

\subsection{Data Pipeline}
\label{ssec:pipeline}
%To generate training and testing data for the RL models in the V2B problem, we follow these steps for data collection and processing. 
% % \begin{enumerate}[leftmargin=*] 
% \textbf{Data Collection}: We gather data from \nissan{}'s EV fleet, including arrival and departure times, initial SoC, and required SoC. We also collect historical smart building load demand across different seasons from May 2023 to January 2024. Additionally, data from EV chargers is collected, including charging/discharging rates.
    
% \textbf{Modeling for User Behavior and Building Load Fluctuation}: Based on the collected data, we model multiple features. For instance, we use the Poisson distribution to model EV arrivals at each hour, considering inputs such as current time, month, and weather information. Similarly, we train models using random forest to generate estimateed EV stay duration, initial SoC, and required SoC, effectively modeling EV driver behaviors. To model the building load fluctuation, we utilize 9 months of real building energy consumption data collected from \nissan{}'s smart building, recorded at 15-minute intervals. The building load is modeled using a normal distribution, with the real building load as the mean and introducing a standard error of 2 kW.

% \textbf{Generating Sample Chains for Training and Testing}: We then generate training samples for one-month billing period episodes, including EV arrival/departure schedules along with their initial SoC and required SoC, using the developed models for user behaviors. Building load data is generated at every 15-minute interval over a month. Additionally, we incorporate data for 15 \nissan{} EV chargers, consisting of 5 bidirectional and 10 unidirectional chargers. Time-of-Use (ToU) electricity rate schedules and demand charge data from utility providers are also integrated to simulate realistic billing scenarios. We generate 110 samples for each month from May 2023 to January 2025. Each sample contains all the above information for each day of the month, and we use 60 samples for training and 50 for testing.  
{\color{black} 
We outline the data pipeline in Figure~\ref{fig:framework_and_pipeline}, which encompasses the preliminary processes for RL training, detailing data collection, sample generation, peak power estimation, and training episode segmentation. \ad{clarify how much data exists. Also, clarify how we reduce the monthly problem to daily problem}
% These are all essential steps for effective RL training. 
 %  \textbf{Data Collection.} We first gather comprehensive data from \nissan{}'s EV fleet, including arrival and departure times, initial and required State of Charge (SoC), as well as historical smart building load demand across seasons from May 2023 to January 2024. Additionally, we collect charging and discharging rates from EV chargers. Multiple features are modeled to capture these distributions using the Poisson distribution for EV arrivals, SoC requirements, and building fluctuations based on historical data.

% \textbf{Generating Sample Chains for Training and Testing}. We create training samples for one-month billing episodes that encompass EV schedules, initial SoC, and required SoC. Building load flows are generated at 15-minute intervals based on historical data. We also integrate Time-of-Use (ToU) electricity rate schedules and demand charge data from utility providers to simulate realistic billing scenarios. 
% , as well as historical smart building load demand across seasons from May 2023 to January 2024. 
% Additionally, we collect telemetry data from EVs and EV chargers.
{\bf Sample generation.}
Data was collected and provided by \nissan{} from their fleet and chargers at their office building over the period from May 2023 to January 2024. The dataset includes EV arrival and departure times and initial and required State of Charge (SoC) as well as building power demand readings at 15 minute intervals. Since the location is categorized as an industrial building, it is subject to time-of-use (TOU) electricity rates and demand charges. TOU introduces variations in the electricity rates, with higher prices during peak hours (6:00 AM to 10:00 PM). Demand charge adds a flat rate multiplier to the highest average power delivery across 15-minute intervals over the billing period, and is added to the total bill.
To capture the distributions of these features, we utilize the Poisson distribution for modeling EV arrivals, SoC requirements, and building fluctuations based on historical data. We then empirically sampled $1000$ one-month billing episodes samples for each month from May 2023 to January 2024.  
% The building did not have data before May 2023 and data collection stopped in Jan 2024.

%Additionally, we integrate Time-of-Use (ToU) electricity rate schedules and demand charge data from utility providers to simulate realistic billing scenarios. Using these data models, we generate 1,000 samples for each month from May 2023 to January 2024. 

\textbf{Splitting data into daily episodes.}
\jpt{Can we explain a bit more on how splitting monthly to daily help with the training?}
To enhance training efficacy, we address the challenge of lengthy state-action episodes by splitting the monthly training dataset into daily episodes.
We focus only on weekdays since employees are often not present at the office, resulting in minimal EV arrivals and low building loads, both of which do not significantly impact the overall monthly demand charge. We incorporate the estimated monthly peak power as an input feature and penalize only those peak power values exceeding this estimated value in each daily episode. This approach incentivizes the policy to minimize the monthly peak power while effectively training with daily episodes.

{{\bf Down-sampling Strategy}.
During training, we varied the amount of samples used for training and observed that utilizing a larger number of training episodes results in longer convergence time and overall worse performance. We show the results of this in~\Cref{ssec:ablation}.
% During training, we implement a down-sampling strategy from the 1,000 monthly samples. We observe that utilizing the complete dataset results in excessive training iterations, significantly increasing the training duration (up to 3 days on our computing systems) and complicating convergence, motivating us to perform down-sampling. 
Thus, we utilize a k-means clustering approach to down-sample. We used $k=5$ and clustered using the optimal demand charge derived from the MILP solution for each sample. 
% This optimal demand charge is closely correlated with sample characteristics, such as building load distribution and EV user's behavior, and significantly influences the total bill. 
% Specifically, We clustered 1,000 samples into five groups based on optimal demand charge using k-means. 
From each cluster, we select 60 and 50 samples for the final training and testing datasets respectively, ensuring that these datasets are mutually exclusive. 
% This method allows us to train and evaluate our RL model using fewer but representative training samples, thereby mitigating the loss of diversity and distribution among the samples.
}

\textbf{Peak power estimation.} 
% A key aspect of our state definition is the monthly peak power estimation. 
We employ a MILP solver to determine the minimum demand charge derived from optimal action sequences over the one-month billing period across all training samples. From the distribution of peak power from the MILP solutions, we set the lower bound of the 95\% confidence interval as the estimated peak power for the month, providing a conservative initial estimate.



%{\bf Data Collection}: We gather data from \nissan{}'s EV fleet, including arrival and departure times, initial and required State of Charge (SoC). Historical smart building load demand is collected across seasons from May 2023 to January 2024, along with charging and discharging rates from EV chargers. Multiple features are modeled to capture these distributions, employing the Poisson distribution for EV arrivals, SoC requirements, and building fluctuations based on historical data.
%{\bf Generating Sample Chains for Training and Testing}: We create training samples for one-month billing episodes, encompassing EV schedules, initial SoC, and required SoC, using the developed models. Building load flows are generated at 15-minute intervals based on historical data. Additionally, Time-of-Use (ToU) electricity rate schedules and demand charge data from utility providers are integrated to simulate realistic billing scenarios.

% \textbf{Feature Generation}: Based on the cleaned data, we generate input features such as building load profiles, EV arrival schedules, initial SoC, and departure SoC targets. These features serve as the input states for the RL model training and testing phases. 
% This process ensures that the RL models are trained on accurate and diverse data, providing robustness and generalization across various V2B scenarios.

%{\bf Generating samples.} To assess our RL-based V2B charging optimization approach, we used real-world data collected {\color{black} from an \nissan{}'s office building, along with charger meter readings in Santa Clara, California.} This data was used to derive statistics and develop generative models for generating samples for training and testing the RL model, which also serves as input to our simulator.

%\textbf{Monthly Peak Power estimation:} Our state definition includes an initial peak power estimation over the whole billing period as an input feature. We employ a Mixed-Integer Linear Programming (MILP) solver to determine the minimum demand charge derived from optimal action sequences over a one-month period across all training samples. For each month, we analyze the distribution of peak power from the MILP solutions and use the lower bound of the 95\% confidence interval as the estimated monthly demand charge, providing a conservative initial estimate. 
 
%{\bf Splitting Training Data into Daily Episodes:} During our training process, we encountered a challenge: utilizing a one-month billing period resulted in lengthy state-action episodes, hindering the policy's ability to learn the Q-value effectively. To address this, we reduced the training episodes to daily segments, focusing on weekdays. This decision was informed by our observation that weekends often have infrequent EV arrivals and low buidling load, which do not significantly impact the overall monthly demand charge.  To ensure that monthly peak power considerations are integrated, we incorporate the estimated monthly demand charge as an input feature. Additionally, the reward function penalizes only those demand charges that exceed this estimated value in each daily episode. This approach incentivizes the policy to minimize the monthly demand charge while effectively training with daily episodes.

{\bf Model inference.} Our RL-based policy operates at $delta$ intervals to determine current policy actions. To generate these actions, we require a set of input data, including the charger status (connected EV's current SoC, expected departure time, current building load, and charger information), the charging rate limits for each charger, and the estimated peak power. 
%{which can be derived from historical data or machine learning-based predictive models}.
In this paper, we utilize the estimated peak power based on training samples. With this current information, we abstract the input features for the RL model, enabling it to determine the power rate control actions for the upcoming time interval.
} 



\subsection{Environment Simulator}
Our approach uses stateless discrete event simulator that serves as the digital twin of our target environment. It holds a state that represents the entirety of the world. This includes information on EVs, building, and the grid. This allows us to investigate how any action or decision can potentially impact the real world. Decisions are taken at the end of each set of events for any given time period. There are two main decisions that must be taken when solving the V2B charging problem. (1) Charger assignments and (2) Charger actions. We address the charger assignment decision below and provide information on the charger action in~\Cref{sec:RL}.

\textbf{Environment updates.} The input episodes serve as the entirety of the simulator's world view. Each event includes an event type and time, matching their real world trigger and occurrence. We identify several critical events in the episodes to serve as the triggers for the simulator. These include (1) EV arrivals, (2) EV departures, (3) building power readings, and (4) TOU rate changes. Events are placed in a queue, with each event triggering an update to the environment which modifies the state. Updates to the state, which include the charging or discharging of EVs, are based on the elapsed time between events. Only EV arrival events trigger charger assignment decisions, while all events trigger a charger action decision.

\input{results/charger_assignments}
\textbf{Charger assignment.} In our approach we consider a first-in, first-out policy that prioritizes the assignment of EVs to bidirectional chargers. If multiple EVs arrive at the same time, then we break the ties randomly. \Cref{table:charger_assignment_policies} shows the different charger assignment and tie breaking policies tested. Bidirectional charging assignments outperform any other assignment policy. Tie breaking policies that favor latest departing cars have marginal advantage over others. While the assignment policies can be further optimized, we elected to follow this heuristic, focusing instead on the second decision problem of charger action.

\textbf{Charger actions.} We provide several policies with our simulator to contrast and compare with our proposed approach. Charger action policies receive a state of the environment for a particular time and generate actions based on this. The simulator is stateless. Thus, it provides only a current representation of the world at that specific time to each policy.


\subsection{Markov Decision Process Model}
\label{ssec:MDP}

%We model the V2B problem as a Markov Decision Process (MDP). This framework enables optimal decision-making regarding charger power rates and load management, allowing for effective responses to uncertainties in user behaviors and building loads in our working scenario. 
 % \color{black} We define the MDP as a tuple $(\mathcal{S}, \mathcal{A}, {\it Trans}(S(T_j),A(T_j))$, ${\it Reward}(S(T_j)$, $A(T_j))$, where $\mathcal{S} = \{S(T_j) \mid T_j \in \mathcal{T}\}$ is the set of states representing the system's state at the beginning of each time slot $T_j$, $\mathcal{A} = \{A(T_j) \mid T_j \in \mathcal{T}\}$ is the set of actions, with each action $A(T_j) \in \mathcal{A}$ representing a decision made at time slot $T_j$ and including continuous values indicating the power rate of all chargers. The state transition function $\text{Trans}(S(T_j), A(T_j))$ describes the transition from state $S(T_j)$ to the next state based on action $A(T_j)$, while the reward function $\text{Reward}(S(T_j), A(T_j))$ assigns a reward for taking action $A(T_j)$ in state $S(T_j)$. The key notations in the MDP are provided as follows: 

%We briefly describe the state, actions and transitions for the problem. 
% The key notations in the MDP for the V2B problem are as follows:
% \rishav{add small \\description of MDP, \\as a tuple\\ (S,A,P,T,$\gamma$, \\describing each \\in few words}
%Our objective is to determine the optimal power rate sequence $\mathcal{P}$ across all time slots $T_j \in \mathcal{T}$. 
%To address the uncertainly in our working scenario, we model the V2B problem as a Markov Decision Process (MDP), which effectively tracks changing conditions like fluctuating SoC, building load, and unpredictable EV arrivals. MDPs provide a framework for making optimal decisions regarding charger power rates and load management, ensuring effective responses to variations in user behavior and demand.    We model the V2B charging control process as a MDP, aiming to find the optimal power rate sequence \(\mathcal{P}\) for all time slots \(T_j \in \mathcal{T}\). Our goal is to derive the optimal policy for determining $P(C_i, T_j)$ for each time slot, minimizing the total cost while ensuring that the EVs' SoC requirements are met throughout the billing period.     




{\bf State.}
%At each time slot $T_j\in\mathcal{T}$, we define state $S(T_j) \in \mathcal{S}$ by using a combination of features that we identified through a mix of domain expertise and experiments.
The complete state space for the problem can be described using features that provide historical, current, and future estimation at a given time $T_j$.
%These features provide historical as well as current context and include estimated value of peak power that is used to calculate demand charge.  The complete state features of our V2B problem at each decision-making time 
This includes key input parameters for each vehicle such as the current SoC, required SoC, departure time, and battery capacity for each EV, along with SoC boundaries across 15 chargers. Additionally, the current building load, time slot, day of the week, past historical building load, and long-term peak power estimation value are included, resulting in approximately 100 features. While this state space is complete, it is not tractable to be used for the learning process. Therefore, we had to reduce the state space. For this,
%For improving practical performance, 
we leveraged domain-specific knowledge to abstract key information, reducing the state space to 37 essential state elements without compromising crucial data. We describe them below.
%abstracting the following features providing historical and current building load and arrival EVs, as well as features for future estimation:
    % \item The current building load is $\Building(T_j)$. 
    % \item The power gap between the current builidng load to the estimated peak power for the billing period is $\PrdPeak(T_j) - \Building(T_j)$. By incorporating this long-term peak power estimation, we provide the RL model with a feature that aids in estimating the optimal peak power for demand charge reduction.
    % \item The historical building load information which constrains the mean peak building load$\mu(B^H(T_j))$ and variance $\sigma^2(B^H(T_j))$  over the previous 7 days. This data can also inform estimations of future building load.  
    % \item The current time slot $T_j$ and day of the week are included to help the RL model distinguish daily patterns and manage feature diversity, enhancing generalization across different weekdays. 
    % \item Number of EV arrivals upto time slot $T_j$ for the current day, represented as $|\{V| V\in \mathcal{V}, A(V)\leq T_j \}|$, tracking the current EVs and their status. 
    % \item We include the status of all chargers $\CS(T_j) = \{(\PowerNeed(C_i, T_j)$, $\ReTime(C_i, T_j))\}_{C_i \in \mathcal{C}}$, where each charger status is defined by two components: (i) $\PowerNeed(C_i, T_j)$ representing the energy gap between the required SoC and the current SoC of the EV, where, 
    % \begin{equation*} 
    %  \PowerNeed(C_i, T_j) = ( \SOCR(V) - \SOC(V, T_j) ) \times {\it CAP}(V)
    %  \label{eq:chargerstate}
    % \end{equation*} 
    % with $V=\phi(C_i, T_j)$ indicating the EV connected to $C_i$ at time slot $T_j$
    % and, (ii) $\ReTime(C_i, T_j) = D(\phi(C_i, T_j)) - T_j$, specifying the time remaining till the departure of the EV.
   
    \begin{enumerate}[leftmargin=*]
    \item The current time slot, $T_j$.
    \item The current building load, denoted as ${B}(T_j)$.
    \item The power gap between the current building load and the estimated peak power for the billing period, $ \PrdPeak(T_j) - {B}(T_j)$. It aids the RL model in estimating optimal peak power for demand charge reduction.
    \item The mean peak building load over the previous 7 days, $\mu(B^H(T_j))$.
    \item The variance of the peak building load over the previous 7 days, $\sigma^2(B^H(T_j))$. It informs the model about the future building load.  
    \item The day of the week for the current time slot, $T_j$. It helps the RL model distinguish daily patterns and enhance generalization.
    \item The number of EV arrivals up to time slot $T_j$, represented as $|\{V | V \in \mathcal{V}, A(V) \leq T_j \}|$. This helps in tracking the number of EVs currently present.
    \item The energy needed by each EV that is connected to a charger at time slot $T_j$, given by $[\PowerNeed(C_i, T_j)]_{C_i \in \mathcal{C}}$, and is initialized to $0$. This represents the energy gap between the required SoC and the current SoC of the EV connected to charger $C_i$ at time slot $T_j$, defined as: 
    \begin{equation} 
     \PowerNeed(C_i, T_j) = (\SOCR(V) - \SOC(V, T_j)) \times \text{CAP}(V)
     \label{eq:chargerstate}
    \end{equation}
    where $V=\phi(C_i, T_j)$ indicating the EV connected to $C_i$ at time slot $T_j$. 
    \item The remaining time until the departure of each EV connected to the chargers is given by $[\ReTime(C_i, T_j)]_{C_i \in \mathcal{C}}$, and is set to 0 when no cars are connected. Each term is computed as $\ReTime(C_i, T_j) = \DepartureTime(\phi(C_i, T_j)) - T_j$. 
\end{enumerate} 
   % where $RT(C_i,T_j)$ represents the remaining time before the departure of the EV, i.e.,
    %The energy gap $\PowerNeed(C_i, T_j)$ is calculated based on the difference between the target SoC and the current SoC at time $T_j$.
   % Status of all chargers is given by $\CS(T_j) = \{ CS(C_i, T_j) \}_{C_i \in \mathcal{C}, T_j \in \mathcal{T}}$. Each charger status is specified by $CS(C_i, T_j) = (\PowerNeed(C_i, T_j)$ , $RT(C_i, T_j))$. Here, $\PowerNeed(C_i, T_j)$ represents the energy gap between the required SoC and the current SoC for the EV connected to $C_i$, calculated by: 
 %We set $CS_t^i = (0,0)$ if no EV is connected to charger $C^i$.

% {\bf Initial State.} The initial state $S(T_0)$ is defined at the beginning of the billing period. It consists of the starting building load $B(T_0)= 0$, the estimated peak power $\PrdPeak(T_0)$, which may be derived from historical data or forecasts (as detailed in Section~\ref{ssec:pipeline}), and historical load values from the preceding 7 days, represented as $B^{H}(T_0)$. The initial time is set to $T_0$, corresponding to the day of the week. At this initial time, the number of EV arrivals is $0$. Additionally, all parameters of the initial status of all chargers $\CS(T_0)$ are set to 0.  
%the initial status of all chargers, denoted as $\CS(T_0) = \{ CS(C_i, T_0) \}_{C^i \in \mathcal{C}}$, is initialized with $CS(C_i, T_0) = (0,0)$ for all chargers.  

{\bf Actions.} The actions $A(T_j) \in \mathcal{A}$ in this MDP are continuous and specify the power rates of all chargers at time $T_j$, where $A(T_j) = [P(C_i, T_j)]_{C^i \in \mathcal{C}}$.

{\bf State Transition.} We utilize a discrete event simulator to track the state transitions. \ad{Need to write a bit about the simulator.} The simulator is given a ``chain'' empirically sampled from the data provided by our partner representing a day in the monthly billing period. 
\jpt{Will connect it to the new subsection in Approach}
Each chain is accompanied by the estimate of peak power over the entire billing period, $\PrdPeak(T_0)$, based on the full sequence of daily chains across a month. This peak is generated by solving a MILP program that gives all the monthly chains as an input, and during training, this ``optimal peak'' across the month is used as an input for a given daily chain \ad{the whole concept of daily and monthly chains is a bit confusing and has to be cleared up. Basically how we reduce the monthly problem into a sequence of daily problems has to be described here clearly.}.


% Given the complexity of modeling state transition probabilities in the V2B problem, we conduct simulations for each billing period, referred to as a ``sample''. The simulator includes:
% \begin{itemize}[leftmargin=*] 
%     \item {The estimated peak power over entire billing period, $\PrdPeak(T_0)$ based on training samples.}
%     \item Electricity prices, $\theta_E(T_j)$ for $T_j \in \mathcal{T}$.
%     \item Building load, ${B}(T_j)$ for $T_j \in \mathcal{T}$.
%     \item EV arrival and departure schedules with SoC requirements, $\forall V \in \mathcal{V}: \SOCI, \SOCR, SOC, CAP$.
%     \item Available chargers and charging limits $\{C_i \in \mathcal{C}\}$.
% \end{itemize} 
The simulator updates the status of the building and chargers based on actions taken at each time slot. 
%The state transition function is defined as ${\it Trans}: \mathcal{S} \times \mathcal{A} \leftarrow \mathcal{S}$, with $(S(T_j), A(T_j)) \mapsto S(T_{j+1})$,  
%\rishav{change \\this and simulator\\ steps to use \\ ${T_j}$ \&$T_{j-1}$} 
%indicating how state features are updated.
This process includes the following steps: 

\begin{enumerate}[leftmargin=*]
    \item Initialize the estimated peak power, $\PrdPeak(T_0)$, which can be derived from historical data \ad{does not make sense. We need to clearly describe the process of how this is estimated for training as well as inference. Please update. There is no mention in section 4.2} (detailed in Section~\ref{ssec:pipeline}),
    , and update it by
    $
    \PrdPeak(T_{j+1}) = \max(\PrdPeak(T_j)$, $ \Building(T_j) + \sum_{C^i \in \mathcal{C}} P(C_i, T_j)),
    $
    which updates the estimated peak power depending on the previous estimate and the current peak power.
    \item Update SoC of EVs connected to all chargers: $\SOC(\phi(C_i,T_j), T_{j+1})$ using action $A(T_j)$ according to Equation~(\ref{eq: soc}). 
    %Additionally, we apply action post-processing to keep EV SoCs within valid boundaries by adjusting the power rate for stopping charging or discharging when they exceed $SoC^{\text{max}}(\phi(C_i, T_j))$ or drop below $SoC^{\text{min}}(\phi(C_i, T_j))$. 
   \item Update the EV charger assignment $\phi(C_i, T_j)$ and $\eta(V_k)$ by first releasing chargers with departing EVs in the next time slot $T_{j+1}$ and then assigning new arrival EVs to idle chargers, following the FIFO procedure and prioritizing the bi-directional chargers first. 
   \item Update the energy requirement of all EVs connected to a charger: $[\PowerNeed(C_i, T_{j+1})]_{C_i \in \mathcal{C}}$ (by Equation~(\ref{eq:chargerstate})) based on EV's current SoCs.
   \item Update the remaining time of all EVs connected to chargers: $[\ReTime(C_i, T_{j+1})]_{C_i \in \mathcal{C}}$ at time slot $T_{j+1}$.   
\end{enumerate}


{\bf Action Reward.} We define the function ${\it Reward}: \mathcal{S} \times \mathcal{A} \rightarrow \Re$, where ${\it Reward}(S(T_j), A(T_j))$ evaluates the reward for actions taken in a specific state, focusing on minimizing the total bill while satisfying SoC requirements. This function is expressed by: \ad{is the subscript index for actions is correct? }
\begin{align}
   & \mathit{Reward}(S(T_j), A(T_j)) = \lambda_{S} \cdot \mathit{Reward}_1 + \lambda_{E} \cdot \mathit{Reward}_2 + \lambda_{D} \cdot \mathit{Reward}_3
\end{align}
where, 
\begin{align*}
\begin{aligned}
    \mathit{Reward}_1 &=  \sum\limits_{C^i\in\mathcal{C}} \max(0, \min(\PowerNeed(C_i, T_j), P(C_i, T_j) \cdot \delta)) \\
    \mathit{Reward}_2 &= - P(C_i, T_j) \cdot \delta  \cdot \theta_E(T_j) \\
    \mathit{Reward}_3 &= - \max(0, \Building(T_j) + \sum\limits_{C^i \in \mathcal{C}} P(C_i, T_j) - \PrdPeak(T_j)) \cdot \theta_D
\end{aligned}
\end{align*}
% \begin{align}
%    & \mathit{Reward}(S(T_j), A(T_j)) = \nonumber \\
%    & \sum\limits_{C^i\in\mathcal{C}} \max(0, \min(\PowerNeed(C_i, T_j), P(C_i, T_j) \times\delta)) \times \lambda_{S} \nonumber \\
%    & - P(C_i, T_j) \times\delta  \times \theta_E(T_j) \times \lambda_{E} \nonumber \\
%    & - \max(0, \Building(T_j) + \sum\limits_{C^i \in \mathcal{C}} P(C_i, T_j) - \PrdPeak(T_j)) \times \theta_D \times \lambda_{D} 
% \end{align}
where $\mathit{Reward}_1$ promotes actions that charge EVs to reach their required SoC, as outlined in Eq. (\ref{eq: soc}). $\mathit{Reward}_2$ penalizes the energy costs generated by this action, and $\mathit{Reward}_3$ penalizes the increase in demand charges caused by rising peak power, aligning with our objective in Eq. (\ref{eq: billing}). These functions use three reward coefficients, $\lambda_{S}$, $\lambda_{E}$, and $\lambda_{D}$ to balance trade-offs between these reward factors. 
%The coefficients are adjustable during training to optimize model performance. 

% {\bf Policy}: Our objective is to develop the policy $\pi: \mathcal{S} \rightarrow \mathcal{A}$, where $\pi(S(T_j))$ generates the optimal charging actions for a given state, maximizing the overall action reward function throughout the billing period $T_j \in \mathcal{T}$.
 
% \rishav{Add discount factor \\short description}

% 
\begin{table*}[t]
    \centering
    \resizebox{\textwidth}{!}{
\begin{tabular}{l|rrllrrll}
\toprule
\textbf{Dataset} & \multicolumn{4}{c}{\textbf{GSM8K}} & \multicolumn{4}{c}{\textbf{MATH}} \\
\cmidrule(lr){1-1} \cmidrule(lr){2-5} \cmidrule(lr){6-9}
\textbf{Method} & Acc & Len & Rel. Acc & Rel. Len & Acc & Len & Rel. Acc & Rel. Len \\
\midrule
\multicolumn{9}{l}{\textit{Zero-Shot Prompting}} \\
\midrule
\hspace{12pt}Baseline & 78.06 & 241.87 & 100.00 \small{(0.00)} & 100.00 \small{(0.00)} & 46.40 & 480.37 & 100.00 \small{(0.00)} & 100.00 \small{(0.00)} \\
\hspace{12pt}Be Concise & 77.98 & 214.87 & 99.85 \small{(1.18)} & 88.46 \small{(10.37)} & 47.76 & 446.09 & 102.71 \small{(7.59)} & 92.66 \small{(7.46)} \\
\hspace{12pt}Hand Crafted 2 (ours) & 76.72 & 184.13 & 98.27 \small{(3.67)} & 77.10 \small{(22.27)} & 46.84 & 404.85 & 101.62 \small{(4.79)} & 85.26 \small{(15.97)} \\
\midrule
\multicolumn{9}{l}{\textit{FT - External Data}} \\
\midrule
\hspace{12pt}Direct Answer & 19.70 & 3.17 & 24.88 \small{(5.03)} & 1.36 \small{(0.40)} & 15.08 & 6.98 & 35.16 \small{(10.34)} & 1.44 \small{(0.73)} \\
\hspace{12pt}Human CoT & 65.73 & 127.85 & 83.82 \small{(7.28)} & 54.95 \small{(13.17)} & 33.88 & 243.54 & 75.61 \small{(13.56)} & 53.14 \small{(13.87)} \\
\hspace{12pt}GPT4o CoT & 76.36 & 156.24 & 97.65 \small{(3.63)} & 67.60 \small{(16.70)} & 40.44 & 399.80 & 90.52 \small{(15.07)} & 87.21 \small{(22.22)} \\
\midrule
\multicolumn{9}{l}{\textit{FT - Best-of-N Self-Generation}} \\
\midrule
\hspace{12pt}Naive BoN & 77.12 & 214.22 & 98.79 \small{(1.64)} & 87.17 \small{(8.79)} & 47.64 & 433.26 & 101.74 \small{(7.04)} & 89.89 \small{(3.99)} \\
\hspace{12pt}Rational Metareasoning & 76.15 & 207.49 & 97.21 \small{(5.74)} & 84.93 \small{(5.09)} & 47.56 & 432.56 & 103.02 \small{(6.56)} & 90.56 \small{(5.25)} \\
\midrule
\multicolumn{9}{l}{\textit{FT - Few-Shot Conditioned Self-Generation (ours)}} \\
\midrule
\hspace{12pt}FS-Human & 76.66 & 161.72 & 98.06 \small{(3.28)} & 67.96 \small{(16.62)} & 46.44 & 421.54 & 99.69 \small{(6.97)} & 87.78 \small{(5.98)} \\
\hspace{12pt}FS-GPT4o & 78.07 & 175.54 & 99.94 \small{(1.69)} & 73.15 \small{(13.49)} & 47.36 & 421.21 & 101.87 \small{(5.33)} & 87.58 \small{(6.60)} \\
\hspace{12pt}FS-Self & 77.27 & 190.03 & 98.86 \small{(2.51)} & 77.51 \small{(9.18)} & 48.00 & 426.67 & 102.67 \small{(5.24)} & 88.50 \small{(4.49)} \\
\midrule
\multicolumn{9}{l}{\textit{FT - Few-Shot Conditioned Best-of-N Self-Generation (ours)}} \\
\midrule
% GPT4o Best-of-16 (Naive) & 75.48 & 153.51 & 96.56 \small{(3.79)} & 64.12 \small{(16.35)} & 47.28 & 367.49 & 101.50 \small{(9.81)} & 76.96 \small{(11.42)} \\
\hspace{12pt}FS-GPT4o-BoN & 75.88 & 153.38 & 97.00 \small{(4.11)} & 64.25 \small{(16.66)} & 47.36 & 364.33 & 102.56 \small{(6.24)} & 76.30 \small{(10.56)} \\
\hspace{24pt}\raisebox{0.5ex}{$\llcorner$}\hspace{4pt}\textit{Budget-Matched} & 76.24 & 160.59 & 97.44 \small{(3.67)} & 67.15 \small{(16.41)} & 47.52 & 384.43 & 101.58 \small{(9.53)} & 80.43 \small{(9.04)} \\

\bottomrule
\end{tabular}
    }
    \caption{
        Evaluation of zero-shot prompting, fine-tuning (FT), and our proposed methods on GSM8K and MATH datasets, averaged across five main models.
        We report absolute accuracy (\%) and length (tokens), along with relative accuracy (\%) and length (\%) compared to baseline prompting. Values in parentheses show standard deviations.
    }
    \label{tab:main_results}
\end{table*}

\section{Results}
\label{results}

% \begin{table*}[t]
% \resizebox{\textwidth}{!}{
% \centering
% \small
% \begin{tabular}{m{2.5cm}|m{15cm}}
% \toprule
% \textbf{Method} & \textbf{Content} \\ %& \# \textbf{Tokens}\\
% \midrule
% \midrule[0.05em]
% \multicolumn{2}{c}{Question} \\ 
% \midrule
% - & A robe takes 2 bolts of blue fiber and half that much white fiber.  How many bolts in total does it take?  \\
% \midrule
% \midrule[0.05em]
% \multicolumn{2}{c}{Reasoning Paths} \\
% \midrule
% Zero-Shot & To find the total number of bolts needed, we need to calculate the amount of white fiber first, since it's half the amount of blue fiber. Step 1: Determine the amount of blue fiber needed. The robe takes 2 bolts of blue fiber. Step 2: Calculate the amount of white fiber needed. The robe takes half the amount of white fiber as blue fiber. Half of 2 bolts is $2/2$ = 1 bolt. Step 3: Add the amount of blue fiber and white fiber to find the total. Total bolts = blue fiber + white fiber. Total bolts = 2 bolts + 1 bolt. Total bolts = 3 bolts. The answer is \textbf{\textit{3}}. \\ %& 213 \\
% \midrule
% FS-GPT4o-BoN & The robe takes 2 bolts of blue fiber. It takes half that much white fiber, which is $2 / 2$ = 1 bolt. Add the blue and white fiber together: 2 + 1 = 3 bolts. The answer is \textbf{\textit{3}}. \\ %57 \\
% \bottomrule
% \end{tabular}
% }
% \caption{Example rationales generated from the original Llama-3.1-8B model and the model fine-tuned using our method. The question is randomly sampled from GSM8K. For the same question, one can observe that our rationale is much more concise, removing redundant parts that explain the question.}
% \label{tab:samples}
% \vspace{-0.1in}
% \end{table*}


\begin{table*}[t]
\resizebox{\textwidth}{!}{
\centering
\normalsize
\begin{tabular}{m{17.5cm}}
\toprule
\multicolumn{1}{c}{\textbf{Question}} \\ 
\midrule
A robe takes 2 bolts of blue fiber and half that much white fiber.  How many bolts in total does it take? \\
\midrule
\multicolumn{1}{c}{\textbf{Reasoning Paths}} \\
\midrule
\textbf{[Zero-Shot]} 
To find the total number of bolts needed, we need to calculate the amount of white fiber first, since it's half the amount of blue fiber. Step 1: Determine the amount of blue fiber needed. The robe takes 2 bolts of blue fiber. Step 2: Calculate the amount of white fiber needed. The robe takes half the amount of white fiber as blue fiber. Half of 2 bolts is $2/2$ = 1 bolt. Step 3: Add the amount of blue fiber and white fiber to find the total. Total bolts = blue fiber + white fiber. Total bolts = 2 bolts + 1 bolt. Total bolts = 3 bolts. The answer is 3. \\
\midrule
\textbf{[FS-GPT4o-BoN]} 
The robe takes 2 bolts of blue fiber. It takes half that much white fiber, which is $2 / 2$ = 1 bolt. Add the blue and white fiber together: 2 + 1 = 3 bolts. The answer is 3. \\
\bottomrule
\end{tabular}
}
\caption{Example rationales generated from the original Llama-3.1-8B model (\textbf{Zero-Shot}) and the model fine-tuned using our method (\textbf{FS-GPT4o-BoN}). The question is randomly sampled from GSM8K. For the same question, one can observe that our rationale is much more concise, removing redundant parts that explain the question.}
\label{tab:samples}
\vspace{-0.1in}
\end{table*}


\subsection{Main results}

Our main results, presented in \autoref{tab:main_results} and \autoref{fig:main_methods_comparison}, demonstrate the performance of our self-training methods against baseline approaches.
% We highlight key observations from these results below.

\paragraph{Naive BoN fine-tuning is effective but sample inefficient.}
Naive BoN fine-tuning effectively reduces output length without significantly degrading model performance. 
This also holds true for Qwen2.5-Math-1.5B and DeepSeekMath-7B (\autoref{tab:main_results_full_gsm8k} and \autoref{tab:main_results_full_math}), which failed to achieve length reduction through zero-shot prompting.
% However, while naive BoN does reduce output length, the reduction is limited to 12\%.
However, the length reduction from naive BoN with $N=16$ is limited to 12\% on average.
Furthermore, as illustrated in Figure~\ref{fig:bon_sample_efficiency}, achieving more compression with BoN becomes progressively less efficient.

\paragraph{Iterative baseline yields similar results as naive BoN fine-tuning.}
% We compare our single-step naive BoN approach with Rational Metareasoning \cite{de2024rational}, an iterative approach using expert iteration \cite{zelikman2022star}  which incorporates an additional \textit{utility reward} to balance efficiency and accuracy in BoN sampling.
Rational Metareasoning, an iterative baseline, yields similar relative length reduction and relative accuracy to BoN fine-tuning. 
This suggests that the utility reward proposed by \citet{de2024rational} may not effectively achieve both accuracy gains and token length reduction.

\begin{figure}[t] % "h" places the figure roughly here
    \centering
    \includegraphics[width=\columnwidth]{figures/main_methods_comparison.pdf} % Adjust width as needed
    \caption{Tradeoff between relative accuracy and length reduction for main methods. Results are averaged over GSM8K and MATH across five main models. Matching colors and shapes indicate the same FS prompt. FS conditioning without augmentation (†) are marked with lighter colors. 
    Relative length reduction refers to 100 - relative length (\%).}
    \label{fig:main_methods_comparison} % Label for referencing in text
\end{figure}
% \red{TODO - shorten this}

\paragraph{Few-shot conditioning outperforms BoN in length reduction.}
The results demonstrate that few-shot conditioning achieves a greater relative length reduction compared to naive BoN, including math-specialized models (\autoref{tab:main_results_full_gsm8k} and \autoref{tab:main_results_full_math}).
% This reduction is attributed to the fact that the fine-tuning datasets generated through few-shot conditioning contain shorter reasoning paths compared to those generated by naive BoN, as illustrated in \autoref{fig:bon_sample_efficiency}.  % too long
This is in line with the superior length reduction of few-shot conditioning, compared to naive BoN as shown in \autoref{fig:bon_sample_efficiency}.
Notably, self-training on generations conditioned on human-annotated examples (FS-Human) achieves an average relative length of 67.96\% on GSM8K, compared to 87.17\% with naive BoN.  % good to have some specific numbers in the text
% We further analyze the effect of fine-tuning on length reduction in \autoref{analysis}.



\paragraph{Self-training better preserves accuracy than training with external data.} 
\autoref{tab:main_results} shows fine-tuning with external data (\textit{FT-External Data}) leads to a significant reduction in relative length but causes a severe drop in relative accuracy. 
% \autoref{fig:main_methods_comparison} further highlights that while fine-tuning with GPT-4o CoT (FT-GPT4o) achieves slightly greater reduction in relative length than fine-tuning with self-generated data using few-shots from GPT-4o (FS-GPT4o), it results in substantially lower relative accuracy.  % a bit complicated / not concrete (conrete evidence = one where we beat external FT in both accuracy and reduction)
\autoref{fig:main_methods_comparison} further highlights the accuracy preservation of self-training: fine-tuning with external concise reasoning supervision from GPT-4o (FT-GPT4o) lies below the Pareto-curve of relative accuracy and relative length reduction, established by our self-training methods.
% NAMGYU - TODO add some commentary

\paragraph{Few-shot conditioned BoN achieves best length reduction while maintaining accuracy.}
% Few-shot conditioned BoN enables substantial length reduction compared to all other BoN and few-shot methods while maintaining relative accuracy.
FS-BoN elicits the largest length reduction among our self-training methods, while maintaining relative accuracy, on average.
Notably, for math-specialized models, FS-GPT4o-BoN achieves the greatest reduction among all methods, except those fine-tuned on external data which greatly sacrifice the accuracy (\autoref{tab:main_results_full_gsm8k} and \autoref{tab:main_results_full_math}). 
% This result reflects how applying BoN to few-shot conditioning further reduces the relative length of the training data while also increasing the proportion of correct samples.  % unnecessary

\paragraph{Augmentation boosts accuracy for few-shot conditioning.}
\autoref{fig:main_methods_comparison} compares few-shot conditioning, i.e., FS and FS-BoN, with and without augmentation (†). 
Augmentation improves accuracy by providing solutions for previously unsolvable hard questions as discussed in \autoref{sample_augmentation}. 
While augmentation may slightly affect reduction rates, they remain superior to naive BoN and RM.
% Similar effect is observed for augmentation in FS-BoN.
% Even when matching the budget (\textit{Budget-Matched}) with other fine-tuning methods using self-generated data in \autoref{tab:main_results}, it achieves the greatest length reduction among them with minimal accuracy degradation.
Even when matching the budget (\textit{Budget-Matched}) with other self-training methods in \autoref{tab:main_results}, it achieves the greatest length reduction among them with minimal accuracy degradation.
The effect of augmentation on training data length is analyzed in \autoref{appx_augmentation_length}.
% Furthermore, as shown in Figure \ref{fig:main_methods_comparison}, augmentation on few-shot conditioned BoN enhances accuracy similar to naive BoN and Meta-Reasoning while achieving greater length reduction.

\begin{figure}[t]
    \centering
    \includegraphics[width=\columnwidth]{figures/length_by_difficulty.pdf} % Adjust width as needed
    \caption{\textbf{Tokens are reduced adaptively according to question difficulty.} 
    Token reduction rate for each difficulty level on MATH, for 4 models individually and averaged.
    % Higher difficulty levels show lower reduction rates.
    Relative length reduction refers to 100 - relative length (\%).
    }
    \label{fig:length_difficulty} % Label for referencing in text
\end{figure}

\subsection{Analysis}
\label{analysis}
% This section analyzes length reduction: transfer from generation to fine-tuning, reduction by question difficulty, qualitative analysis, and consistency across model sizes. DeepSeekMath-7B is excluded from quantitative analysis due to cost.
% let's keep this short
In this section, we analyze the length reduction effects in depth.
We exclude DeepSeekMath-7B from quantiative analysis due to cost.


% \paragraph{Analysis on sample efficiency}
% As shown in \autoref{fig:bon_sample_efficiency}, best-of-n (BoN) sampling requires a substantial number of samples to be generated to achieve a level of reasoning length reduction comparable to that achievable through few-shot conditioning.
% In other words, it is infeasible to reach the reasoning length reduction performance of few-shot conditioning using BoN alone, without generating a prohibitively large number of samples.
% However, our experiments consistently demonstrate that combining few-shot conditioning with BoN sampling is more effective in reducing reasoning length than using either technique in isolation.
% Specifically, few-shot conditioning helps to guide the model towards generating more concise reasoning paths, while BoN sampling allows us to select the shortest and most accurate path from a diverse set of candidates.
% This synergistic effect results in a more efficient and effective approach to concise reasoning.


% \paragraph{FT can reduce generation length effectively.}
% As shown in \autoref{fig:ft_length_scatter}, after fine-tuning, the models tend to follow the length of the training data, suggesting that reasoning length reduction can be achieved through simple supervised fine-tuning on short reasoning samples.
% Note that test generation length is relatively longer than the training data length, as the models can generate lengthy incorrect answers, while the training data consists of correct answers.
% Correctly generated answers align more closely with training data length as shown in (Appendix~\ref{appx_length_scatter_correct}).

% \paragraph{Length reduction through generation and fine-tuning}
% Our method reduces reasoning length in two stages: generation and fine-tuning.
% First, as shown in \autoref{fig:ft_length_scatter}, 
% % generation length for training data varies depending on the method. 
% few-shot conditioning methods produce shorter outputs than naive BoN, with few-shot conditioned BoN achieving the shortest. 
% Second, fine-tuning with shorter rationales results in shorter model outputs, showing a strong correlation between test and training lengths\footnote{Test generation lengths are generally longer than training data lengths due to the possibility of lengthy incorrect answers during testing. Test outputs that are correct align more closely with training data lengths, as shown in Appendix~\ref{appx_length_scatter_correct}.}.
% Overall, FS-GPT4o-BoN effectively generates and trains for shorter reasoning paths.
% Additionally, unlike zero-shot methods, our approach significantly reduces token length in math-tuned models like Qwen2.5-Math-1.5B with FS-GPT4o-BoN, achieving 54.7\% relative length after fine-tuning. (See \autoref{tab:main_results_full_gsm8k} and \autoref{tab:main_results_full_math}).

\paragraph{Tokens are reduced adaptively according to question complexity.} 
The MATH dataset's difficulty levels range from 1 (basic algebra) to 5 (advanced calculus and complex mathematical reasoning).
As shown in \autoref{fig:length_difficulty}, our method adaptively reduces tokens based on question difficulty, with higher difficulty leading to less reduction.
% Most models achieve their peak reduction (around 20\%--40\%) at difficulty levels 1-2, where simple concepts allow for more concise explanations.
% The reduction rate gradually declines at levels 3-5, indicating our method's ability to preserve necessary details for complex problems automatically.
%  -> not precise. simple concepts allow for more concise explanations *in absolute terms*, but this does not necessarily mean that length reduction *relative to the default* should be high. E.g., if the model already uses very few tokens for easy questions, then relative reduction would be low.
The higher reduction (20\%--40\%) at easier difficulty levels (1--2) suggests that the original model outputs for these easier questions contained unnecessary tokens.
This reveals a gap in current models' ability to tailor their inference budget to problem complexity.
Our method effectively closes this gap by reducing redundancy, allowing for more precise token allocation based on question difficulty.

\begin{figure}[t] % "h" places the figure roughly here
    \centering
    \includegraphics[width=\columnwidth]{figures/scaling_methods_comparison.pdf} % Adjust width as needed
    \caption{Scaling study on baseline and few-shot conditioned self-training methods. Results are averaged over GSM8K and MATH for Llama 1B, 3B, and 8B.
    % Accuracy tends to be maintained, with greater length reduction using our FS-GPT4o(-BoN) method.
    Relative length reduction refers to 100 - relative length (\%).
    }
    \label{fig:scaling_methods_comparison} % Label for referencing in text
\end{figure}

\paragraph{Self-training maintains consistency across model scales.}
We conduct a scaling study on Llama-3.2-1B, 3B, and Llama-3.1-8B to examine consistency across different model sizes (\autoref{fig:scaling_methods_comparison}). 
Overall, token reduction increases as the model size increases, while the maintenance of accuracy does not show a strong correlation with model size. 
RM exhibits lower reduction rates compared to our few-shot conditioned self-training methods across all models and shows a decrease in accuracy with increasing model size. 
% The few-shot method also shows a similar trend in length reduction, but it achieves the best relative accuracy in the 3B model.
Our standalone few-shot conditioning method (FS-GPT4o) also shows a similar trend in length reduction, but better preserves accuracy.
Our joint FS-GPT4o-BoN method achieves the greatest reduction across all models, maintaining relative accuracy across different model sizes, especially in the largest 8B model.



\paragraph{Sample study}
\autoref{tab:samples} presents qualitative examples of reasoning paths generated by the model before and after fine-tuning with our method. 
The original reasoning exhibits verbosity, containing redundant processes such as question confirmation and repeated instructions. 
In contrast, the reasoning generated by our method includes only the necessary steps, significantly reducing the number of tokens while still arriving at the correct answer. 
% These examples demonstrate the effectiveness of our method in reducing token count. 
More examples are provided in the \autoref{appx_sample_studies}.

\begin{figure}[t]
    \centering
    \includegraphics[width=\columnwidth]{figures/both_length_scatter.pdf} % Adjust width as needed
    \caption{\textbf{Fine-tuning effectively transfers the length reduction to the model.} Correlation between the relative length of train data and test output averaged over GSM8K and MATH across 4 models. Training length includes only correct solutions. Solid points represent test lengths including all generated outputs, while lighter points indicate test lengths of correct solutions only.}
    \label{fig:ft_length_scatter} % Label for referencing in text
\end{figure}

\paragraph{Length reduction is transferred through fine-tuning.}
As shown in \autoref{fig:ft_length_scatter}, fine-tuning with shorter rationales results in shorter model outputs, showing a strong correlation between test and training lengths.
% Test generation lengths (solid datapoints) are generally longer than training data lengths due to the possibility of lengthy incorrect answers during testing.
% However, when comparing with test generation lengths that are correct (lighter datapoints), they align more closely with training data lengths.
We note that the length of test outputs (incorrect and correct) are longer than the length of training samples (only correct) on average.
This is mainly because incorrect paths are generally longer than correct ones.
We find a closer correspondence between train length and test length of correct samples only, indicated by the lighter datapoints.
This discrepancy suggests the need to terminate incorrect paths early to minimize redundant inference overhead.
We consider this for future work.

%
\section{\label{case_rc_car}Hardware Experiments - RC Car}

Finally, we consider a real-world a miniature RC car with dynamics modeled as \eqref{eq:dyn_hw}, with $L=23.5 \text{ cm}$, controls $\ctrl=[V,\delta]$ with ranges $V \in [0.7,1.4]\text{ m/s}$ and $\delta \in [-25^{\circ},25^{\circ}]$, and disturbances $d_x,d_y \in [-0.1,0.1]$ to account for model mismatches and state estimation error. The vehicle is tasked with completing laps over the racetrack shown in Fig.~\ref{fig:exp_main_result}.
%
\setlength{\arraycolsep}{2pt} % Adjust spacing as needed
\begin{equation}\label{eq:dyn_hw}
{\fontsize{8.5}{10}\selectfont
\dot{\state}
= \begin{bmatrix} \dot{x} & \dot{y} & \dot{\theta} \end{bmatrix}
= \begin{bmatrix} V \cos(\theta)+ d_x, & V \sin(\theta)+ d_y, & V \tan(\delta) / L \end{bmatrix}
}
\end{equation}
\setlength{\arraycolsep}{5pt} % Reset spacing

We adapt our evaluation metrics for this hardware study to better reflect real-world, single-run applicability. Instead of batch statistics, we measure the \textit{CompTime} for each method -- the time taken to generate and evaluate potential samples within a multiple-lap run. This metric reflects how well-suited each technique is for real-time control. We also report the car's average \textit{Speed} over three laps to measure how aggressive the policy is. The \textit{RelCost} metric remains consistent with the simulations, providing a normalized cost relative to the proposed approach for the methods that managed to maintain safety.

As cost function we use (\ref{eq:cost_mppi_car}), where the first term penalizes going slower than $V_\text{max}=1.4\text{ m/s}$, the second term penalizes the distance from the track's center line, the third term $P(\state)$ penalizes going into the obstacle set, BRT, or decrease in safety, depending on the method.
%, and the last term penalizes going into the BRT. The penalty weights $K_{Obs}$ and $K_{BRT}$ alternate between $(K_{Obs},K_{BRT})=(50,0)$ for the 'Obs costs' and 'Obs costs+ LR filter' cases and $(K_{Obs},K_{BRT})=(0,50)$  for the 'BRT costs' and 'BRT costs+ LR filter' cases, for the proposed method their value is irrelevant as hallucinations are guaranteed not to enter the BRT or obstacle sets. 
%with $l_{center}$ the constant distance between the center of the lane and the nearest edge,
\begin{equation}\small
\label{eq:cost_mppi_car}
S = (V_{max}-V)^2 + K_{c} (l_{center}-\targetfunc(x)) +  P(\state)
\end{equation}
%
The controllers were implemented using JAX \cite{jax} on a laptop equipped with an NVIDIA GeForce RTX 4060. 
% for its parallelization capabilities; the GPU used is a Laptop NVIDIA GeForce RTX 4060 where 
We generate $1000$ parallel hallucinations (with $100$ time steps each) in a loop running at $50Hz$. Results are summarized in Table~\ref{tab:hw_results}, and trajectories for the first lap are shown in Fig~\ref{fig:hw_traj}.

First, we highlight the need for hard safety constraints as the methods that only rely on safety penalties fail to clear the top-left tight turn in the track as shown in Fig~\ref{fig:hw_traj}. Fine-tuning the cost function and MPPI parameters might allow unfiltered methods to complete laps. Still, we want to consider and compare methods that provably allow for safe executions.
%
\begin{table}[t]
\caption{Hardware experiments results summary.}
\centering
\renewcommand{\arraystretch}{1.2} % Adjust row height for vertical centering
\begin{tabularx}{\columnwidth}{|>{\centering\arraybackslash}p{2.5cm}|>{\centering\arraybackslash}X|>{\centering\arraybackslash}p{1.4cm}|>{\centering\arraybackslash}p{2.1cm}|}
\hline
\textbf{Method} & \textbf{CompTime (ms)} & \textbf{RelCost} & \textbf{Speed (m/s)} \\
\hline
Obs costs & 1.8 ± 0.3 & fail & 1.00 ± 0.05 \\
BRT costs & 1.8 ± 0.3  & fail & 1.01 ± 0.06 \\
Obs costs + LRF & 1.7 ± 0.4 & 1.1874 & 1.03 ± 0.12\\
BRT costs + LRF & 1.8 ± 0.4 & 1.1626 & 1.04 ± 0.12 \\
Shield-MPPI & 1.7 ± 0.2 & 1.1038 & 1.04 ± 0.08\\
DualGuard (Ours) & 2.5 ± 0.4 & 1.0000 & 1.10 ± 0.11 \\
\hline
\end{tabularx}
\label{tab:hw_results}
\end{table}
%
The proposed method leads to faster and more performant trajectories than the other safe baselines. A direct comparison with the baselines that also use an output LRF illustrates that the proposed safe hallucination step improved the quality of the samples as exemplified in Fig.~\ref{fig:exp_main_result}(B)(C), leading to a better overall performance and a higher average speed. Also, the proposed method outperforms the Shield-MPPI baseline even after tuning its hyperparameters to the best of our capabilities so that it maintains safety without an excessive impact on performance.

The computation times are nearly identical across all baselines, as each method fundamentally involves calculating performant terms of the cost function and querying the obstacle set or BRT for safety-related penalties. The proposed method introduces an additional LRF step for each sample along hallucinated trajectories, resulting in a slight increase in computational time. Nevertheless, all methods, including the proposed one, operate well within the $20ms$ time budget, leaving ample time for the control loop to handle state estimation, communications, and actuation.
%
\begin{figure}[b] 
\begin{center} 
\vspace{0.0em}
\includegraphics[width=0.925\columnwidth]{fig/hw_traj_v2.png}
\vspace{-0.5em}
\caption{Top view of the RC car's trajectories under each method.}
\label{fig:hw_traj}
\end{center}
\end{figure}


%
\section{Discussion and Conclusion}
\label{sec:discuss}

We presented \bench, the first framework  and experimental platform to benchmark AI Agents for IT automation tasks. \bench strives to capture the complexity of modern IT systems and the diversity of IT tasks. The reproducibility of \bench ensures the community-driven effort despite inherent nondeterminism of large-scale IT systems. 

One of the key design principles of \bench is ensuring its flexibility to support diverse areas of different IT systems
and its extensibility to new scenarios. While current scope of \bench is comprehensive and representative, we plan to further enrich the benchmark suites by adding other important processes essential to modern IT automation. Furthermore, we plan to expand our benchmarking beyond event-triggered scenarios. 
We are actively working to expand scenario coverage for the supported processes and promote growth through open-community contributions.
 We invite the community to reproduce their real-world-inspired incidents in a synthetic sandboxed environment leveraging the \bench. We expect that everyone contributing can bring their expertise to the table.

We expect \bench to drive the innovations of AI agent-based techniques with a direct impact on the safety, efficiency, and intelligence of today’s IT infrastructures. 
With \bench, we are starting to explore many deep, exciting open problems: How to develop domain-specific AI agents that specialize in certain types of IT tasks? How to orchestrate multiple agents with various expertise to collaborate on bigger projects? How can we ensure safety of agent-driven solutions? How can we effectively use human-in-the-loop while developing diverse adaptive agents? We invite everyone to participate in answering these questions and realizing the vision of using AI agents to automate critical IT tasks.




%%%%%%%%%%%%%%%%%%%%%%%%%%%%%%%%%%%%%%%%%%%%%%%%%%%%%%%%%%%%%%%%%%%%%%%%%%%%%%%%

\addtolength{\textheight}{-1cm}   % This command serves to balance the column lengths
% on the last page of the document manually. It shortens         % the textheight of the last page by a suitable amount.
% This command does not take effect until the next page
% so it should come on the page before the last. Make
% sure that you do not shorten the textheight too much.

%%%%%%%%%%%%%%%%%%%%%%%%%%%%%%%%%%%%%%%%%%%%%%%%%%%%%%%%%%%%%%%%%%%%%%%%%%%%%%%%

% \section*{APPENDIX}

% Appendixes should appear before the acknowledgment.

% \section*{ACKNOWLEDGMENT}

% Thanks to ...
%%%%%%%%%%%%%%%%%%%%%%%%%%%%%%%%%%%%%%%%%%%%%%%%%%%%%%%%%%%%%%%%%%%%%%%%%%%%%%%%

\bibliographystyle{IEEEtran}
\bibliography{  ./Bib/bib_reach,
                ./Bib/bib_SIAL,
                ./Bib/bib_MPC,
                ./Bib/bib_cbf,
                ./Bib/bib_local,
                ./Bib/bib_fltr
}

\end{document}
