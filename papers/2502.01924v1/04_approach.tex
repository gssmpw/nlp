\section{\label{approach}Dualguard MPPI}

As discussed in Sec. \ref{background_mppi}, MPPI solves the optimal control problem in \eqref{eq:opt_problem} using a sampling-based method. While MPPI may encourage the satisfaction of safety constraints in \eqref{eq:opt_problem} via penalizing safety violations in the cost function, ensuring safety constraints remains challenging. Moreover, by construction, the high-cost safety breaching sequences are mostly ignored during the optimal control sequence computation, wasting computational resources that could have been used to refine the system performance further. While the safety filtering mechanism discussed in \ref{background_lr} can provide a safety layer after MPPI to enforce the safety constraint, this approach is myopic in nature and might lead to performance impairment in favor of safety.

To overcome these challenges, we propose DualGuard MPPI, a two-layered safety filtering approach. First, the safety filtering is incorporated during the sampling process itself, where a least restrictive filter is applied along the sampled control sequence rollouts using a pre-computed safety value function. This ensures that all hallucinations satisfy the safety constraint and can contribute to performance optimization. To ensure that the resultant optimal control sequence also satisfies the safety constraint, we filter the output optimal control sequence by a safety filter as well. In addition to ensuring the safety constraints, the proposed framework leads to an increased sample efficiency as the safe hallucinations keep all sampled trajectories safe and relevant to the performance objective, thereby avoiding "sample wastage" due to safety constraint violation. The proposed algorithm is presented in Alg. 1. Details on the proposed filtering stages are discussed next.

\begin{figure}[t] 
\begin{center} 
\vspace{0.0em}
\includegraphics[width=0.9\columnwidth]{fig/alg_mppi.png}
\vspace{-1em}
\end{center}
\end{figure}
%
\subsection{\label{safe_hallucinations}Generating Safe hallucinations}
%
As in classical MPPI, we consider $K$ sequences of random perturbations $\delta_j^k$, that modify a nominal control sequence $u_j$. The perturbed sequences are applied to the dynamic model of the system while being filtered at each time step along the horizon, using  the LRF in eq. (\ref{eq:lst_restrict_safety_ctrl}), resulting in hallucinated trajectories that are guaranteed to maintain safety. The cost-to-go $S^k$ for the filtered trajectory is calculated over the safety-filtered control perturbation sequence $\Delta_j^k$.
% obtained as the difference between the filtered control and the nominal sequence.

The nominal control sequence $u_j$, the filtered control perturbations $\Delta_j^k$ and the cost over the safe hallucinated trajectories $S^k$ are used to calculate the optimal control sequence $u^*_j$ using the update rule defined in (\ref{eq:update_law}). As we weigh over controls that only produce safe trajectories, all K sequences contribute to the performance optimization, reducing the variance of MPPI algorithm and leading to better performance.

\subsection{\label{output_filter}Output Least restrictive filtering}

Even though the optimal control sequence $u^*_j$ was obtained by weighing control perturbations that individually resulted in safe trajectories, the safety guarantees may not hold for $u_j^*$. For this reason, before applying the first control in the sequence $u^*_0$ to the system, we perform one last LRF step to guarantee the safe operation of the complete system. The rest of the sequence is used as the nominal control for the next time step, and the entire process is repeated.

One case that encourages this last filtering stage corresponds to scenarios where the safe hallucinations present multimodality. We can imagine a vehicle trying to swerve around an obstacle by turning right or left; even if both swerving maneuvers result in safe behavior, a weighting over hallucinations split between these two modalities could result in maintaining a straight trajectory, causing a collision. Such issues are resolved by this additional filtering stage which picks one of the safe modes.

\begin{mdframed}[style=MyFrame,nobreak=false]

\textbf{Running example \textit{(Safe Planar Navigation)}:}

Considering our running example we observe the proposed steps of DualGuard MPPI in Fig.~\ref{fig:safe_mppi_steps}. First, we visualize the unmodified hallucination step in the left panel. Next, using the same control perturbations, we show how the safe hallucination step renders all the samples safe. Hallucinated trajectories are presented in a blue-to-red scale, corresponding to the associated low-to-high costs.

The right panel of Fig.~\ref{fig:safe_mppi_steps} shows how the safe hallucinations might present themselves in a multimodal fashion for some obstacle configurations; this indicates a possible failure mode where the control sequence given by the update rule in \eqref{eq:update_law} could drive the vehicle directly into the obstacles, leading to safety violations. The proposed output filtering stage safeguards against this possibility.  

\vspace{1em}
{\centering      \includegraphics[width=1.0\columnwidth]{fig/safe_mppi_steps_v2.png}
      \captionof{figure}{Unmodified hallucinations (Left). Safe hallucinations (Center). Possible multimodality on safe hallucinations encourages the use of output least restrictive filtering (Right).}
      \label{fig:safe_mppi_steps}\par} 

\end{mdframed}





