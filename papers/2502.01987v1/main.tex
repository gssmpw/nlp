\documentclass[journal,onecolumn]{ieeetj}
\usepackage{cite}
\usepackage{amsmath,amssymb,amsfonts}
\usepackage{algorithmic}
\usepackage{graphicx,color}
\usepackage{textcomp}
\usepackage{xcolor}
\usepackage{siunitx}
\usepackage{subcaption}
\usepackage{caption}
\usepackage{hyperref}
\hypersetup{hidelinks=true}
\usepackage{algorithm,algorithmic}
\usepackage{acro}
\usepackage{booktabs}
\usepackage{multirow}
\newacronym{rl}{RL}{Reinforcement Learning}
\newacronym{drl}{DRL}{Deep Reinforcement Learning}
\newacronym{mdp}{MDP}{Markov Decision Process}
\newacronym{ppo}{PPO}{Proximal Policy Optimization}
\newacronym{sac}{SAC}{Soft Actor-Critic}
\newacronym{epvf}{EPVF}{Explicit Policy-conditioned Value Function}
\newacronym{unf}{UNF}{Universal Neural Functional}

\newcommand{\figref}[1]{Figure~\ref{#1}}

% Colour for revisions and stuff
\newcommand{\blue}[1]{\textcolor{blue}{#1}}
\newcommand{\bblue}[1]{\textcolor{blue}{\textbf{#1}}}

\definecolor{britishracinggreen}{rgb}{0.0, 0.26, 0.15}
\newcommand{\green}[1]{\textcolor{britishracinggreen}{#1}}
\newcommand{\bgreen}[1]{\textcolor{britishracinggreen}{\textbf{#1}}}

\newcommand{\fabio}[1]{\textcolor{orange}{#1}}


\def\BibTeX{{\rm B\kern-.05em{\sc i\kern-.025em b}\kern-.08em
    T\kern-.1667em\lower.7ex\hbox{E}\kern-.125emX}}
\AtBeginDocument{\definecolor{tmlcncolor}{cmyk}{0.93,0.59,0.15,0.02}\definecolor{NavyBlue}{RGB}{0,86,125}}




\def\OJlogo{\vspace{-4pt}$<$Society logo(s) and publication title will appear here.$>$}
\def\seclogo{\vspace{10pt}$<$Society logo(s) and publication title will appear here.$>$}

\def\authorrefmark#1{\ensuremath{^{\textbf{#1}}}}

\begin{document}
\receiveddate{XX Month, XXXX}
\reviseddate{XX Month, XXXX}
\accepteddate{XX Month, XXXX}
\publisheddate{XX Month, XXXX}
\currentdate{XX Month, XXXX}
\doiinfo{XXXX.2022.1234567}

\markboth{}{Fabio A. Ruetz. {et al.}}

% \title{Human-assisted Adaptive Online Traversability Estimation}
\title{Online Adaptive Traversability Estimation through Interaction for Unstructured, Densely Vegetated Environments}

\author{
Fabio A. Ruetz\authorrefmark{1,2}, 
Nicholas Lawrance\authorrefmark{2},
Emili Hern\'andez\authorrefmark{3},
Paulo V. K. Borges\authorrefmark{4},
and Thierry Peynot\authorrefmark{1}}
\affil{QUT Centre for Robotics, Queensland University of Technology (QUT), Brisbane Qld 4000, Australia}
\affil{CSIRO Robotics, Data61, Pullenvale, Qld 4069, Australia.}
\affil{Emesent, Milton, Qld 4064, Australia}
\affil{Orica, Windsor, Qld 4030, Australia}
\corresp{Corresponding author: Fabio A. Ruetz (email: fabio.ruetz@gmail.com).}
\authornote{This work was supported by QUT, CSIRO, Emesent and the SILVANUS Project through European Commission Funding on the Horizon 2020 call number H2020-LC-GD-2020, Grant Agreement number 101037247. F.R. and T.P. acknowledge continued support from the Queensland University of Technology (QUT) through the Centre for Robotics.}



\begin{abstract}
Navigating densely vegetated environments poses significant challenges for autonomous ground vehicles. Learning-based systems typically use prior and in-situ data to predict terrain traversability but often degrade in performance when encountering out-of-distribution elements caused by rapid environmental changes or novel conditions. This paper presents a novel, lidar-only, online adaptive traversability estimation (TE) method that trains a model directly on the robot using self-supervised data collected through robot-environment interaction. The proposed approach utilises a probabilistic 3D voxel representation to integrate lidar measurements and robot experience, creating a salient environmental model. To ensure computational efficiency, a sparse graph-based representation is employed to update temporarily evolving voxel distributions. Extensive experiments with an unmanned ground vehicle in natural terrain demonstrate that the system adapts to complex environments with as little as 8 minutes of operational data, achieving a Matthews Correlation Coefficient (MCC) score of 0.63 and enabling safe navigation in densely vegetated environments. This work examines different training strategies for voxel-based TE methods and offers recommendations for training strategies to improve adaptability. The proposed method is validated on a robotic platform with limited computational resources (25W GPU), achieving accuracy comparable to offline-trained models while maintaining reliable performance across varied environments.
\end{abstract}

\begin{IEEEkeywords}
Traversability Estimation, Navigation in Unstructured Environments, Autonomous Navigation, Field Robotics, Online Adaptation, Deep Learning.
\end{IEEEkeywords}

%\IEEEspecialpapernotice{(Invited Paper)}

\maketitle

% MAIN TEXT WORK
\section{Introduction}
\label{section:introduction}

% redirection is unique and important in VR
Virtual Reality (VR) systems enable users to embody virtual avatars by mirroring their physical movements and aligning their perspective with virtual avatars' in real time. 
As the head-mounted displays (HMDs) block direct visual access to the physical world, users primarily rely on visual feedback from the virtual environment and integrate it with proprioceptive cues to control the avatar’s movements and interact within the VR space.
Since human perception is heavily influenced by visual input~\cite{gibson1933adaptation}, 
VR systems have the unique capability to control users' perception of the virtual environment and avatars by manipulating the visual information presented to them.
Leveraging this, various redirection techniques have been proposed to enable novel VR interactions, 
such as redirecting users' walking paths~\cite{razzaque2005redirected, suma2012impossible, steinicke2009estimation},
modifying reaching movements~\cite{gonzalez2022model, azmandian2016haptic, cheng2017sparse, feick2021visuo},
and conveying haptic information through visual feedback to create pseudo-haptic effects~\cite{samad2019pseudo, dominjon2005influence, lecuyer2009simulating}.
Such redirection techniques enable these interactions by manipulating the alignment between users' physical movements and their virtual avatar's actions.

% % what is hand/arm redirection, motivation of study arm-offset
% \change{\yj{i don't understand the purpose of this paragraph}
% These illusion-based techniques provide users with unique experiences in virtual environments that differ from the physical world yet maintain an immersive experience. 
% A key example is hand redirection, which shifts the virtual hand’s position away from the real hand as the user moves to enhance ergonomics during interaction~\cite{feuchtner2018ownershift, wentzel2020improving} and improve interaction performance~\cite{montano2017erg, poupyrev1996go}. 
% To increase the realism of virtual movements and strengthen the user’s sense of embodiment, hand redirection techniques often incorporate a complete virtual arm or full body alongside the redirected virtual hand, using inverse kinematics~\cite{hartfill2021analysis, ponton2024stretch} or adjustments to the virtual arm's movement as well~\cite{li2022modeling, feick2024impact}.
% }

% noticeability, motivation of predicting a probability, not a classification
However, these redirection techniques are most effective when the manipulation remains undetected~\cite{gonzalez2017model, li2022modeling}. 
If the redirection becomes too large, the user may not mitigate the conflict between the visual sensory input (redirected virtual movement) and their proprioception (actual physical movement), potentially leading to a loss of embodiment with the virtual avatar and making it difficult for the user to accurately control virtual movements to complete interaction tasks~\cite{li2022modeling, wentzel2020improving, feuchtner2018ownershift}. 
While proprioception is not absolute, users only have a general sense of their physical movements and the likelihood that they notice the redirection is probabilistic. 
This probability of detecting the redirection is referred to as \textbf{noticeability}~\cite{li2022modeling, zenner2024beyond, zenner2023detectability} and is typically estimated based on the frequency with which users detect the manipulation across multiple trials.

% version B
% Prior research has explored factors influencing the noticeability of redirected motion, including the redirection's magnitude~\cite{wentzel2020improving, poupyrev1996go}, direction~\cite{li2022modeling, feuchtner2018ownershift}, and the visual characteristics of the virtual avatar~\cite{ogawa2020effect, feick2024impact}.
% While these factors focus on the avatars, the surrounding virtual environment can also influence the users' behavior and in turn affect the noticeability of redirection.
% One such prominent external influence is through the visual channel - the users' visual attention is constantly distracted by complex visual effects and events in practical VR scenarios.
% Although some prior studies have explored how to leverage user blindness caused by visual distractions to redirect users' virtual hand~\cite{zenner2023detectability}, there remains a gap in understanding how to quantify the noticeability of redirection under visual distractions.

% visual stimuli and gaze behavior
Prior research has explored factors influencing the noticeability of redirected motion, including the redirection's magnitude~\cite{wentzel2020improving, poupyrev1996go}, direction~\cite{li2022modeling, feuchtner2018ownershift}, and the visual characteristics of the virtual avatar~\cite{ogawa2020effect, feick2024impact}.
While these factors focus on the avatars, the surrounding virtual environment can also influence the users' behavior and in turn affect the noticeability of redirection.
This, however, remains underexplored.
One such prominent external influence is through the visual channel - the users' visual attention is constantly distracted by complex visual effects and events in practical VR scenarios.
We thus want to investigate how \textbf{visual stimuli in the virtual environment} affect the noticeability of redirection.
With this, we hope to complement existing works that focus on avatars by incorporating environmental visual influences to enable more accurate control over the noticeability of redirected motions in practical VR scenarios.
% However, in realistic VR applications, the virtual environment often contains complex visual effects beyond the virtual avatar itself. 
% We argue that these visual effects can \textbf{distract users’ visual attention and thus affect the noticeability of redirection offsets}, while current research has yet taken into account.
% For instance, in a VR boxing scenario, a user’s visual attention is likely focused on their opponent rather than on their virtual body, leading to a lower noticeability of redirection offsets on their virtual movements. 
% Conversely, when reaching for an object in the center of their field of view, the user’s attention is more concentrated on the virtual hand’s movement and position to ensure successful interaction, resulting in a higher noticeability of offsets.

Since each visual event is a complex choreography of many underlying factors (type of visual effect, location, duration, etc.), it is extremely difficult to quantify or parameterize visual stimuli.
Furthermore, individuals respond differently to even the same visual events.
Prior neuroscience studies revealed that factors like age, gender, and personality can influence how quickly someone reacts to visual events~\cite{gillon2024responses, gale1997human}. 
Therefore, aiming to model visual stimuli in a way that is generalizable and applicable to different stimuli and users, we propose to use users' \textbf{gaze behavior} as an indicator of how they respond to visual stimuli.
In this paper, we used various gaze behaviors, including gaze location, saccades~\cite{krejtz2018eye}, fixations~\cite{perkhofer2019using}, and the Index of Pupil Activity (IPA)~\cite{duchowski2018index}.
These behaviors indicate both where users are looking and their cognitive activity, as looking at something does not necessarily mean they are attending to it.
Our goal is to investigate how these gaze behaviors stimulated by various visual stimuli relate to the noticeability of redirection.
With this, we contribute a model that allows designers and content creators to adjust the redirection in real-time responding to dynamic visual events in VR.

To achieve this, we conducted user studies to collect users' noticeability of redirection under various visual stimuli.
To simulate realistic VR scenarios, we adopted a dual-task design in which the participants performed redirected movements while monitoring the visual stimuli.
Specifically, participants' primary task was to report if they noticed an offset between the avatar's movement and their own, while their secondary task was to monitor and report the visual stimuli.
As realistic virtual environments often contain complex visual effects, we started with simple and controlled visual stimulus to manage the influencing factors.

% first user study, confirmation study
% collect data under no visual stimuli, different basic visual stimuli
We first conducted a confirmation study (N=16) to test whether applying visual stimuli (opacity-based) actually affects their noticeability of redirection. 
The results showed that participants were significantly less likely to detect the redirection when visual stimuli was presented $(F_{(1,15)}=5.90,~p=0.03)$.
Furthermore, by analyzing the collected gaze data, results revealed a correlation between the proposed gaze behaviors and the noticeability results $(r=-0.43)$, confirming that the gaze behaviors could be leveraged to compute the noticeability.

% data collection study
We then conducted a data collection study to obtain more accurate noticeability results through repeated measurements to better model the relationship between visual stimuli-triggered gaze behaviors and noticeability of redirection.
With the collected data, we analyzed various numerical features from the gaze behaviors to identify the most effective ones. 
We tested combinations of these features to determine the most effective one for predicting noticeability under visual stimuli.
Using the selected features, our regression model achieved a mean squared error (MSE) of 0.011 through leave-one-user-out cross-validation. 
Furthermore, we developed both a binary and a three-class classification model to categorize noticeability, which achieved an accuracy of 91.74\% and 85.62\%, respectively.

% evaluation study
To evaluate the generalizability of the regression model, we conducted an evaluation study (N=24) to test whether the model could accurately predict noticeability with new visual stimuli (color- and scale-based animations).
Specifically, we evaluated whether the model's predictions aligned with participants' responses under these unseen stimuli.
The results showed that our model accurately estimated the noticeability, achieving mean squared errors (MSE) of 0.014 and 0.012 for the color- and scale-based visual stimili, respectively, compared to participants' responses.
Since the tested visual stimuli data were not included in the training, the results suggested that the extracted gaze behavior features capture a generalizable pattern and can effectively indicate the corresponding impact on the noticeability of redirection.

% application
Based on our model, we implemented an adaptive redirection technique and demonstrated it through two applications: adaptive VR action game and opportunistic rendering.
We conducted a proof-of-concept user study (N=8) to compare our adaptive redirection technique with a static redirection, evaluating the usability and benefits of our adaptive redirection technique.
The results indicated that participants experienced less physical demand and stronger sense of embodiment and agency when using the adaptive redirection technique. 
These results demonstrated the effectiveness and usability of our model.

In summary, we make the following contributions.
% 
\begin{itemize}
    \item 
    We propose to use users' gaze behavior as a medium to quantify how visual stimuli influences the noticebility of redirection. 
    Through two user studies, we confirm that visual stimuli significantly influences noticeability and identify key gaze behavior features that are closely related to this impact.
    \item 
    We build a regression model that takes the user's gaze behavioral data as input, then computes the noticeability of redirection.
    Through an evaluation study, we verify that our model can estimate the noticeability with new participants under unseen visual stimuli.
    These findings suggest that the extracted gaze behavior features effectively capture the influence of visual stimuli on noticeability and can generalize across different users and visual stimuli.
    \item 
    We develop an adaptive redirection technique based on our regression model and implement two applications with it.
    With a proof-of-concept study, we demonstrate the effectiveness and potential usability of our regression model on real-world use cases.

\end{itemize}

% \delete{
% Virtual Reality (VR) allows the user to embody a virtual avatar by mirroring their physical movements through the avatar.
% As the user's visual access to the physical world is blocked in tasks involving motion control, they heavily rely on the visual representation of the avatar's motions to guide their proprioception.
% Similar to real-world experiences, the user is able to resolve conflicts between different sensory inputs (e.g., vision and motor control) through multisensory integration, which is essential for mitigating the sensory noise that commonly arises.
% However, it also enables unique manipulations in VR, as the system can intentionally modify the avatar's movements in relation to the user's motions to achieve specific functional outcomes,
% for example, 
% % the manipulations on the avatar's movements can 
% enabling novel interaction techniques of redirected walking~\cite{razzaque2005redirected}, redirected reaching~\cite{gonzalez2022model}, and pseudo haptics~\cite{samad2019pseudo}.
% With small adjustments to the avatar's movements, the user can maintain their sense of embodiment, due to their ability to resolve the perceptual differences.
% % However, a large mismatch between the user and avatar's movements can result in the user losing their sense of embodiment, due to an inability to resolve the perceptual differences.
% }

% \delete{
% However, multisensory integration can break when the manipulation is so intense that the user is aware of the existence of the motion offset and no longer maintains the sense of embodiment.
% Prior research studied the intensity threshold of the offset applied on the avatar's hand, beyond which the embodiment will break~\cite{li2022modeling}. 
% Studies also investigated the user's sensitivity to the offsets over time~\cite{kohm2022sensitivity}.
% Based on the findings, we argue that one crucial factor that affects to what extent the user notices the offset (i.e., \textit{noticeability}) that remains under-explored is whether the user directs their visual attention towards or away from the virtual avatar.
% Related work (e.g., Mise-unseen~\cite{marwecki2019mise}) has showcased applications where adjustments in the environment can be made in an unnoticeable manner when they happen in the area out of the user's visual field.
% We hypothesize that directing the user's visual attention away from the avatar's body, while still partially keeping the avatar within the user's field-of-view, can reduce the noticeability of the offset.
% Therefore, we conduct two user studies and implement a regression model to systematically investigate this effect.
% }

% \delete{
% In the first user study (N = 16), we test whether drawing the user's visual attention away from their body impacts the possibility of them noticing an offset that we apply to their arm motion in VR.
% We adopt a dual-task design to enable the alteration of the user's visual attention and a yes/no paradigm to measure the noticeability of motion offset. 
% The primary task for the user is to perform an arm motion and report when they perceive an offset between the avatar's virtual arm and their real arm.
% In the secondary task, we randomly render a visual animation of a ball turning from transparent to red and becoming transparent again and ask them to monitor and report when it appears.
% We control the strength of the visual stimuli by changing the duration and location of the animation.
% % By changing the time duration and location of the visual animation, we control the strengths of attraction to the users.
% As a result, we found significant differences in the noticeability of the offsets $(F_{(1,15)}=5.90,~p=0.03)$ between conditions with and without visual stimuli.
% Based on further analysis, we also identified the behavioral patterns of the user's gaze (including pupil dilation, fixations, and saccades) to be correlated with the noticeability results $(r=-0.43)$ and they may potentially serve as indicators of noticeability.
% }

% \delete{
% To further investigate how visual attention influences the noticeability, we conduct a data collection study (N = 12) and build a regression model based on the data.
% The regression model is able to calculate the noticeability of the offset applied on the user's arm under various visual stimuli based on their gaze behaviors.
% Our leave-one-out cross-validation results show that the proposed method was able to achieve a mean-squared error (MSE) of 0.012 in the probability regression task.
% }

% \delete{
% To verify the feasibility and extendability of the regression model, we conduct an evaluation study where we test new visual animations based on adjustments on scale and color and invite 24 new participants to attend the study.
% Results show that the proposed method can accurately estimate the noticeability with an MSE of 0.014 and 0.012 in the conditions of the color- and scale-based visual effects.
% Since these animations were not included in the dataset that the regression model was built on, the study demonstrates that the gaze behavioral features we extracted from the data capture a generalizable pattern of the user's visual attention and can indicate the corresponding impact on the noticeability of the offset.
% }

% \delete{
% Finally, we demonstrate applications that can benefit from the noticeability prediction model, including adaptive motion offsets and opportunistic rendering, considering the user's visual attention. 
% We conclude with discussions of our work's limitations and future research directions.
% }

% \delete{
% In summary, we make the following contributions.
% }
% % 
% \begin{itemize}
%     \item 
%     \delete{
%     We quantify the effects of the user's visual attention directed away by stimuli on their noticeability of an offset applied to the avatar's arm motion with respect to the user's physical arm. 
%     Through two user studies, we identified gaze behavioral features that are indicative of the changes in noticeability.
%     }
%     \item 
%     \delete{We build a regression model that takes the user's gaze behavioral data and the offset applied to the arm motion as input, then computes the probability of the user noticing the offset.
%     Through an evaluation study, we verified that the model needs no information about the source attracting the user's visual attention and can be generalizable in different scenarios.
%     }
%     \item 
%     \delete{We demonstrate two applications that potentially benefit from the regression model, including adaptive motion offsets and opportunistic rendering.
%     }

% \end{itemize}

\begin{comment}
However, users will lose the sense of embodiment to the virtual avatars if they notice the offset between the virtual and physical movements.
To address this, researchers have been exploring the noticing threshold of offsets with various magnitudes and proposing various redirection techniques that maintain the sense of embodiment~\cite{}.

However, when users embody virtual avatars to explore virtual environments, they encounter various visual effects and content that can attract their attention~\cite{}.
During this, the user may notice an offset when he observes the virtual movement carefully while ignoring it when the virtual contents attract his attention from the movements.
Therefore, static offset thresholds are not appropriate in dynamic scenarios.

Past research has proposed dynamic mapping techniques that adapted to users' state, such as hand moving speed~\cite{frees2007prism} or ergonomically comfortable poses~\cite{montano2017erg}, but not considering the influence of virtual content.
More specifically, PRISM~\cite{frees2007prism} proposed adjusting the C/D ratio with a non-linear mapping according to users' hand moving speed, but it might not be optimal for various virtual scenarios.
While Erg-O~\cite{montano2017erg} redirected users' virtual hands according to the virtual target's relative position to reduce physical fatigue, neglecting the change of virtual environments. 

Therefore, how to design redirection techniques in various scenarios with different visual attractions remains unknown.
To address this, we investigate how visual attention affects the noticing probability of movement offsets.
Based on our experiments, we implement a computational model that automatically computes the noticing probability of offsets under certain visual attractions.
VR application designers and developers can easily leverage our model to design redirection techniques maintaining the sense of embodiment adapt to the user's visual attention.
We implement a dynamic redirection technique with our model and demonstrate that it effectively reduces the target reaching time without reducing the sense of embodiment compared to static redirection techniques.

% Need to be refined
This paper offers the following contributions.
\begin{itemize}
    \item We investigate how visual attractions affect the noticing probability of redirection offsets.
    \item We construct a computational model to predict the noticing probability of an offset with a given visual background.
    \item We implement a dynamic redirection technique adapting to the visual background. We evaluate the technique and develop three applications to demonstrate the benefits. 
\end{itemize}



First, we conducted a controlled experiment to understand how users perceived the movement offset while subjected to various distractions.
Since hand redirection is one of the most frequently used redirections in VR interactions, we focused on the dynamic arm movements and manually added angular offsets to the' elbow joint~\cite{li2022modeling, gonzalez2022model, zenner2019estimating}. 
We employed flashing spheres in the user's field of view as distractions to attract users' visual attention.
Participants were instructed to report the appearing location of the spheres while simultaneously performing the arm movements and reporting if they perceived an offset during the movement. 
(\zhipeng{Add the results of data collection. Analyze the influence of the distance between the gaze map and the offset.}
We measured the visual attraction's magnitude with the gaze distribution on it.
Results showed that stronger distractions made it harder for users to notice the offset.)
\zhipeng{Need to rewrite. Not sure to use gaze distribution or a metric obtained from the visual content.}
Secondly, we constructed a computational model to predict the noticing probability of offsets with given visual content.
We analyzed the data from the user studies to measure the influence of visual attractions on the noticing probability of offsets.
We built a statistical model to predict the offset's noticing probability with a given visual content.
Based on the model, we implement a dynamic redirection technique to adjust the redirection offset adapted to the user's current field of view.
We evaluated the technique in a target selection task compared to no hand redirection and static hand redirection.
\zhipeng{Add the results of the evaluation.}
Results showed that the dynamic hand redirection technique significantly reduced the target selection time with similar accuracy and a comparable sense of embodiment.
Finally, we implemented three applications to demonstrate the potential benefits of the visual attention adapted dynamic redirection technique.
\end{comment}

% This one modifies arm length, not redirection
% \citeauthor{mcintosh2020iteratively} proposed an adaptation method to iteratively change the virtual avatar arm's length based on the primary tasks' performance~\cite{mcintosh2020iteratively}.



% \zhipeng{TO ADD: what is redirection}
% Redirection enables novel interactions in Virtual Reality, including redirected walking, haptic redirection, and pseudo haptics by introducing an offset to users' movement.
% \zhipeng{TO ADD: extend this sentence}
% The price of this is that users' immersiveness and embodiment in VR can be compromised when they notice the offset and perceive the virtual movement not as theirs~\cite{}.
% \zhipeng{TO ADD: extend this sentence, elaborate how the virtual environment attracts users' attention}
% Meanwhile, the visual content in the virtual environment is abundant and consistently captures users' attention, making it harder to notice the offset~\cite{}.
% While previous studies explored the noticing threshold of the offsets and optimized the redirection techniques to maintain the sense of embodiment~\cite{}, the influence of visual content on the probability of perceiving offsets remains unknown.  
% Therefore, we propose to investigate how users perceive the redirection offset when they are facing various visual attractions.


% We conducted a user study to understand how users notice the shift with visual attractions.
% We used a color-changing ball to attract the user's attention while instructing users to perform different poses with their arms and observe it meanwhile.
% \zhipeng{(Which one should be the primary task? Observe the ball should be the primary one, but if the primary task is too simple, users might allocate more attention on the secondary task and this makes the secondary task primary.)}
% \zhipeng{(We need a good and reasonable dual-task design in which users care about both their pose and the visual content, at least in the evaluation study. And we need to be able to control the visual content's magnitude and saliency maybe?)}
% We controlled the shift magnitude and direction, the user's pose, the ball's size, and the color range.
% We set the ball's color-changing interval as the independent factor.
% We collect the user's response to each shift and the color-changing times.
% Based on the collected data, we constructed a statistical model to describe the influence of visual attraction on the noticing probability.
% \zhipeng{(Are we actually controlling the attention allocation? How do we measure the attracting effect? We need uniform metrics, otherwise it is also hard for others to use our knowledge.)}
% \zhipeng{(Try to use eye gaze? The eye gaze distribution in the last five seconds to decide the attention allocation? Basically constructing a model with eye gaze distribution and noticing probability. But the user's head is moving, so the eye gaze distribution is not aligned well with the current view.)}

% \zhipeng{Saliency and EMD}
% \zhipeng{Gaze is more than just a point: Rethinking visual attention
% analysis using peripheral vision-based gaze mapping}

% Evaluation study(ideal case): based on the visual content, adjusting the redirection magnitude dynamically.

% \zhipeng{(The risk is our model's effect is trivial.)}

% Applications:
% Playing Lego while watching demo videos, we can accelerate the reaching process of bricks, and forbid the redirection during the manipulation.

% Beat saber again: but not make a lot of sense? Difficult game has complicated visual effects, while allows larger shift, but do not need large shift with high difficulty



\section{Related Work}
\label{lit_review}

\begin{highlight}
{

Our research builds upon {\em (i)} Assessing Web Accessibility, {\em (ii)} End-User Accessibility Repair, and {\em (iii)} Developer Tools for Accessibility.

\subsection{Assessing Web Accessibility}
From the earliest attempts to set standards and guidelines, web accessibility has been shaped by a complex interplay of technical challenges, legal imperatives, and educational campaigns. Over the past 25 years, stakeholders have sought to improve digital inclusion by establishing foundational standards~\cite{chisholm2001web, caldwell2008web}, enforcing legal obligations~\cite{sierkowski2002achieving, yesilada2012understanding}, and promoting a broader culture of accessibility awareness among developers~\cite{sloan2006contextual, martin2022landscape, pandey2023blending}. 
Despite these longstanding efforts, systemic accessibility issues persist. According to the 2024 WebAIM Million report~\cite{webaim2024}, 95.9\% of the top one million home pages contained detectable WCAG violations, averaging nearly 57 errors per page. 
These errors take many forms: low color contrast makes the interface difficult for individuals with color deficiency or low vision to read text; missing alternative text leaves users relying on screen readers without crucial visual context; and unlabeled form inputs or empty links and buttons hinder people who navigate with assistive technologies from completing basic tasks. 
Together, these accessibility issues not only limit user access to critical online resources such as healthcare, education, and employment but also result in significant legal risks and lost opportunities for businesses to engage diverse audiences. Addressing these pervasive issues requires systematic methods to identify, measure, and prioritize accessibility barriers, which is the first step toward achieving meaningful improvements.

Prior research has introduced methods blending automation and human evaluation to assess web accessibility. Hybrid approaches like SAMBA combine automated tools with expert reviews to measure the severity and impact of barriers, enhancing evaluation reliability~\cite{brajnik2007samba}. Quantitative metrics, such as Failure Rate and Unified Web Evaluation Methodology, support large-scale monitoring and comparative analysis, enabling cost-effective insights~\cite{vigo2007quantitative, martins2024large}. However, automated tools alone often detect less than half of WCAG violations and generate false positives, emphasizing the need for human interpretation~\cite{freire2008evaluation, vigo2013benchmarking}. Recent progress with large pretrained models like Large Language Models (LLMs)~\cite{dubey2024llama,bai2023qwen} and Large Multimodal Models (LMMs)~\cite{liu2024visual, bai2023qwenvl} offers a promising step forward, automating complex checks like non-text content evaluation and link purposes, achieving higher detection rates than traditional tools~\cite{lopez2024turning, delnevo2024interaction}. Yet, these large models face challenges, including dependence on training data, limited contextual judgment, and the inability to simulate real user experiences. These limitations underscore the necessity of combining models with human oversight for reliable, user-centered evaluations~\cite{brajnik2007samba, vigo2013benchmarking, delnevo2024interaction}. 

Our work builds on these prior efforts and recent advancements by leveraging the capabilities of large pretrained models while addressing their limitations through a developer-centric approach. CodeA11y integrates LLM-powered accessibility assessments, tailored accessibility-aware system prompts, and a dedicated accessibility checker directly into GitHub Copilot---one of the most widely used coding assistants. Unlike standalone evaluation tools, CodeA11y actively supports developers throughout the coding process by reinforcing accessibility best practices, prompting critical manual validations, and embedding accessibility considerations into existing workflows.
% This pervasive shortfall reflects the difficulty of scaling traditional approaches---such as manual audits and automated tools---that either demand immense human effort or lack the nuanced understanding needed to capture real-world user experiences. 
%
% In response, a new wave of AI-driven methods, many powered by large language models (LLMs), is emerging to bridge these accessibility detection and assessment gaps. Early explorations, such as those by Morillo et al.~\cite{morillo2020system}, introduced AI-assisted recommendations capable of automatic corrections, illustrating how computational intelligence can tackle the repetitive, common errors that plague large swaths of the web. Building on this foundation, Huang et al.~\cite{huang2024access} proposed ACCESS, a prompt-engineering framework that streamlines the identification and remediation of accessibility violations, while López-Gil et al.~\cite{lopez2024turning} demonstrated how LLMs can help apply WCAG success criteria more consistently---reducing the reliance on manual effort. Beyond these direct interventions, recent work has also begun integrating user experiences more seamlessly into the evaluation process. For example, Huq et al.~\cite{huq2024automated} translate user transcripts and corresponding issues into actionable test reports, ensuring that accessibility improvements align more closely with authentic user needs.
% However, as these AI-driven solutions evolve, researchers caution against uncritical adoption. Othman et al.~\cite{othman2023fostering} highlight that while LLMs can accelerate remediation, they may also introduce biases or encourage over-reliance on automated processes. Similarly, Delnevo et al.~\cite{delnevo2024interaction} emphasize the importance of contextual understanding and adaptability, pointing to the current limitations of LLM-based systems in serving the full spectrum of user needs. 
% In contrast to this backdrop, our work introduces and evaluates CodeA11y, an LLM-augmented extension for GitHub Copilot that not only mitigates these challenges by providing more consistent guidance and manual validation prompts, but also aligns AI-driven assistance with developers’ workflows, ultimately contributing toward more sustainable propulsion for building accessible web.

% Broader implications of inaccessibility—legal compliance, ethical concerns, and user experience
% A Historical Review of Web Accessibility Using WAVE
% "I tend to view ads almost like a pestilence": On the Accessibility Implications of Mobile Ads for Blind Users

% In the research domain, several methods have been developed to assess and enhance web accessibility. These include incorporating feedback into developer tools~\cite{adesigner, takagi2003accessibility, bigham2010accessibility} and automating the creation of accessibility tests and reports for user interfaces~\cite{swearngin2024towards, taeb2024axnav}. 

% Prior work has also studied accessibility scanners as another avenue of AI to improve web development practices~\cite{}.
% However, a persistent challenge is that developers need to be aware of these tools to utilize them effectively. With recent advancements in LLMs, developers might now build accessible websites with less effort using AI assistants. However, the impact of these assistants on the accessibility of their generated code remains unclear. This study aims to investigate these effects.

\subsection{End-user Accessibility Repair}
In addition to detecting accessibility errors and measuring web accessibility, significant research has focused on fixing these problems.
Since end-users are often the first to notice accessibility problems and have a strong incentive to address them, systems have been developed to help them report or fix these problems.

Collaborative, or social accessibility~\cite{takagi2009collaborative,sato2010social}, enabled these end-user contributions to be scaled through crowd-sourcing.
AccessMonkey~\cite{bigham2007accessmonkey} and Accessibility Commons~\cite{kawanaka2008accessibility} were two examples of repositories that store accessibility-related scripts and metadata, respectively.
Other work has developed browser extensions that leverage crowd-sourced databases to automatically correct reading order, alt-text, color contrast, and interaction-related issues~\cite{sato2009s,huang2015can}.

One drawback of collaborative accessibility approaches is that they cannot fix problems for an ``unseen'' web page on-demand, so many projects aim to automatically detect and improve interfaces without the need for an external source of fixes.
A large body of research has focused on making specific web media (e.g., images~\cite{gleason2019making,guinness2018caption, twitterally, gleason2020making, lee2021image}, design~\cite{potluri2019ai,li2019editing, peng2022diffscriber, peng2023slide}, and videos~\cite{pavel2020rescribe,peng2021say,peng2021slidecho,huh2023avscript}) accessible through a combination of machine learning (ML) and user-provided fixes.
Other work has focused on applying more general fixes across all websites.

Opportunity accessibility addressed a common accessibility problem of most websites: by default, content is often hard to see for people with visual impairments, and many users, especially older adults, do not know how to adjust or enable content zooming~\cite{bigham2014making}.
To this end, a browser script (\texttt{oppaccess.js}) was developed that automatically adjusted the browser's content zoom to maximally enlarge content without introducing adverse side-effects (\textit{e.g.,} content overlap).
While \texttt{oppaccess.js} primarily targeted zoom-related accessibility, recent work aimed to enable larger types of changes, by using LLMs to modify the source code of web pages based on user questions or directives~\cite{li2023using}.

Several efforts have been focused on improving access to desktop and mobile applications, which present additional challenges due to the unavailability of app source code (\textit{e.g.,} HTML).
Prefab is an approach that allows graphical UIs to be modified at runtime by detecting existing UI widgets, then replacing them~\cite{dixon2010prefab}.
Interaction Proxies used these runtime modification strategies to ``repair'' Android apps by replacing inaccessible widgets with improved alternatives~\cite{zhang2017interaction, zhang2018robust}.
The widget detection strategies used by these systems previously relied on a combination of heuristics and system metadata (\textit{e.g.,} the view hierarchy), which are incomplete or missing in the accessible apps.
To this end, ML has been employed to better localize~\cite{chen2020object} and repair UI elements~\cite{chen2020unblind,zhang2021screen,wu2023webui,peng2025dreamstruct}.

In general, end-user solutions to repairing application accessibility are limited due to the lack of underlying code and knowledge of the semantics of the intended content.

\subsection{Developer Tools for Accessibility}
Ultimately, the best solution for ensuring an accessible experience lies with front-end developers. Many efforts have focused on building adequate tooling and support to help developers with ensuring that their UI code complies with accessibility standards.

Numerous automated accessibility testing tools have been created to help developers identify accessibility issues in their code: i) static analysis tools, such as IBM Equal Access Accessibility Checker~\cite{ibm2024toolkit} or Microsoft Accessibility Insights~\cite{accessibilityinsights2024}, scan the UI code's compliance with predefined rules derived from accessibility guidelines; and ii) dynamic or runtime accessibility scanners, such as Chrome Devtools~\cite{chromedevtools2024} or axe-Core Accessibility Engine~\cite{deque2024axe}, perform real-time testing on user interfaces to detect interaction issues not identifiable from the code structure. While these tools greatly reduce the manual effort required for accessibility testing, they are often criticized for their limited coverage. Thus, experts often recommend manually testing with assistive technologies to uncover more complex interaction issues. Prior studies have created accessibility crawlers that either assist in developer testing~\cite{swearngin2024towards,taeb2024axnav} or simulate how assistive technologies interact with UIs~\cite{10.1145/3411764.3445455, 10.1145/3551349.3556905, 10.1145/3544548.3580679}.

Similar to end-user accessibility repair, research has focused on generating fixes to remediate accessibility issues in the UI source code. Initial attempts developed heuristic-based algorithms for fixing specific issues, for instance, by replacing text or background color attributes~\cite{10.1145/3611643.3616329}. More recent work has suggested that the code-understanding capabilities of LLMs allow them to suggest more targeted fixes.
For example, a study demonstrated that prompting ChatGPT to fix identified WCAG compliance issues in source code could automatically resolve a significant number of them~\cite{othman2023fostering}. Researchers have sought to leverage this capability by employing a multi-agent LLM architecture to automatically identify and localize issues in source code and suggest potential code fixes~\cite{mehralian2024automated}.

While the approaches mentioned above focus on assessing UI accessibility of already-authored code (\textit{i.e.,} fixing existing code), there is potential for more proactive approaches.
For example, LLMs are often used by developers to generate UI source code from natural language descriptions or tab completions~\cite{chen2021evaluating,GitHubCopilot,lozhkov2024starcoder,hui2024qwen2,roziere2023code,zheng2023codegeex}, but LLMs frequently produce inaccessible code by default~\cite{10.1145/3677846.3677854,mowar2024tab}, leading to inaccessible output when used by developers without sufficient awareness of accessibility knowledge.
The primary focus of this paper is to design a more accessibility-aware coding assistant that both produces more accessible code without manual intervention (\textit{e.g.,} specific user prompting) and gradually enables developers to implement and improve accessibility of automatically-generated code through IDE UI modifications (\textit{e.g.}, reminder notifications).

}
\end{highlight}



% Work related to this paper includes {\em (i)} Web Accessibility and {\em (ii)} Developer Practices in AI-Assisted Programming.

% \ipstart{Web Accessibility: Practice, Evaluation, and Improvements} Substantial efforts have been made to set accessibility standards~\cite{chisholm2001web, caldwell2008web}, establish legal requirements~\cite{sierkowski2002achieving, yesilada2012understanding}, and promote education and advocacy among developers~\cite{sloan2006contextual, martin2022landscape, pandey2023blending}. In the research domain, several methods have been developed to assess and enhance web accessibility. These include incorporating feedback into developer tools~\cite{adesigner, takagi2003accessibility, bigham2010accessibility} and automating the creation of accessibility tests and reports for user interfaces~\cite{swearngin2024towards, taeb2024axnav}. 
% % Prior work has also studied accessibility scanners as another avenue of AI to improve web development practices~\cite{}.
% However, a persistent challenge is that developers need to be aware of these tools to utilize them effectively. With recent advancements in LLMs, developers might now build accessible websites with less effort using AI assistants. However, the impact of these assistants on the accessibility of their generated code remains unclear. This study aims to investigate these effects.

% \ipstart{Developer Practices in AI-Assisted Programming}
% Recent usability research on AI-assisted development has examined the interaction strategies of developers while using AI coding assistants~\cite{barke2023grounded}.
% They observed developers interacted with these assistants in two modes -- 1) \textit{acceleration mode}: associated with shorter completions and 2) \textit{exploration mode}: associated with long completions.
% Liang {\em et al.} \cite{liang2024large} found that developers are driven to use AI assistants to reduce their keystrokes, finish tasks faster, and recall the syntax of programming languages. On the other hand, developers' reason for rejecting autocomplete suggestions was the need for more consideration of appropriate software requirements. This is because primary research on code generation models has mainly focused on functional correctness while often sidelining non-functional requirements such as latency, maintainability, and security~\cite{singhal2024nofuneval}. Consequently, there have been increasing concerns about the security implications of AI-generated code~\cite{sandoval2023lost}. Similarly, this study focuses on the effectiveness and uptake of code suggestions among developers in mitigating accessibility-related vulnerabilities. 


% ============================= additional rw ============================================
% - Paulina Morillo, Diego Chicaiza-Herrera, and Diego Vallejo-Huanga. 2020. System of Recommendation and Automatic Correction of Web Accessibility Using Artificial Intelligence. In Advances in Usability and User Experience, Tareq Ahram and Christianne Falcão (Eds.). Springer International Publishing, Cham, 479–489
% - Juan-Miguel López-Gil and Juanan Pereira. 2024. Turning manual web accessibility success criteria into automatic: an LLM-based approach. Universal Access in the Information Society (2024). https://doi.org/10.1007/s10209-024-01108-z
% - s
% - Calista Huang, Alyssa Ma, Suchir Vyasamudri, Eugenie Puype, Sayem Kamal, Juan Belza Garcia, Salar Cheema, and Michael Lutz. 2024. ACCESS: Prompt Engineering for Automated Web Accessibility Violation Corrections. arXiv:2401.16450 [cs.HC] https://arxiv.org/abs/2401.16450
% - Syed Fatiul Huq, Mahan Tafreshipour, Kate Kalcevich, and Sam Malek. 2025. Automated Generation of Accessibility Test Reports from Recorded User Transcripts. In Proceedings of the 47th International Conference on Software Engineering (ICSE) (Ottawa, Ontario, Canada). IEEE. https://ics.uci.edu/~seal/publications/2025_ICSE_reca11.pdf To appear in IEEE Xplore
% - Achraf Othman, Amira Dhouib, and Aljazi Nasser Al Jabor. 2023. Fostering websites accessibility: A case study on the use of the Large Language Models ChatGPT for automatic remediation. In Proceedings of the 16th International Conference on PErvasive Technologies Related to Assistive Environments (Corfu, Greece) (PETRA ’23). Association for Computing Machinery, New York, NY, USA, 707–713. https://doi.org/10.1145/3594806.3596542
% - Zsuzsanna B. Palmer and Sushil K. Oswal. 0. Constructing Websites with Generative AI Tools: The Accessibility of Their Workflows and Products for Users With Disabilities. Journal of Business and Technical Communication 0, 0 (0), 10506519241280644. https://doi.org/10.1177/10506519241280644
% ============================= additional rw ============================================
\section{Methodology}

We utilized LLMs to tackle the ASQP task across 0-, 10-, 20-, 30-, 40-, and 50-shot settings on different datasets. The performance is compared to that achieved using a dedicated training set to fine-tune smaller pre-trained language models. Furthermore, we report performance results for the TASD task.

\subsection{Evaluation}

\subsubsection{Datasets}

\begin{table*}[h]
\centering
\resizebox{1.8\columnwidth}{!}{%
\begin{tabular}{lccccc}
\hline
\textbf{}                    & \textbf{Rest15} & \textbf{Rest16} & \textbf{FlightABSA} & \textbf{OATS Coursera} & \textbf{OATS Hotels} \\ \hline
\textbf{\# Train}             & 834             & 1,264           & 1,351               & 1,400               & 1,400                \\
\textbf{\# Test}              & 537             & 544             & 387               & 400                 & 400                  \\
\textbf{\# Dev}              & 209             & 316             & 192               & 200                 & 200                  \\ \hline
\textbf{\# Aspect Categories} & 13              & 13              & 13              & 28                  & 33                   \\
\textbf{Language} & en              & en              & en              & en                  & en                   \\
\textbf{Domain} & restaurant              & restaurant              & airline              & e-learning                  & hotel                   \\
\hline
\end{tabular}
}
\caption{Overview of all ASQP datasets considered for evaluation. The datasets cover a range of different numbers of considered aspect categories and domains. }
\label{tab:overview-datasets}
\end{table*}

Table \ref{tab:overview-datasets} presents an overview of the datasets used in this study, including Rest15 and Rest16, along with three additional datasets covering diverse domains.


\textbf{Rest15 \& Rest16:} ASQP annotations originate from \citet{zhang2021aspect} and the TASD annotations from \citet{wan2020target}. This ensured comparability with the performance scores reported in previous research.

\textbf{FlightABSA:} A novel dataset containing 1,930 sentences annotated for ASQP. Properties of the annotated dataset are provided in Appendix \ref{appendix:flightabsa}. 

\textbf{OATS Hotels \& OATS Coursera:} We utilized a subset of two corpora recently introduced by \citet{chebolu2024oats} comprising ASQP-annotated sentences from reviews on hotels and e-learning courses. A detailed description of the data preprocessing for the OATS datasets can be found in Appendix \ref{appendix:oats-dataset}.

For the TASD task, we removed the opinion terms from the quadruples in annotations from FlightABSA, OATS Coursera and OATS Hotels. Subsequently, any duplicate triplets (\textit{a}, \textit{c}, \textit{p}) that appeared twice in a sentence were discarded.

\subsubsection{Setting}

For evaluation, the test dataset was considered for all datasets. An LLM was prompted five times with different seeds (0 to 4) for each combination of ABSA task (ASQP and TASD), dataset and amount of random few-shot examples (0, 10, 20, 30, 40 or 50) taken from the training set in order to get five label predictions. For all seeds, the same few-shot examples were used; however, they were shuffled differently for each prompt execution. The average performance across all five runs is reported.

\subsubsection{Metrics}

As in previous works in the field of ABSA, we report the micro-averaged F1 score as well as precision and recall to assess the model's performance. The F1 score is the harmonic mean of precision and recall. Precision measures the proportion of correctly predicted positive instances out of all instances predicted as positive \cite[p.~67]{jurafsky2000speech}. Recall quantifies the proportion of correctly predicted positive instances out of all actual positive instances in the dataset \cite[p.~67]{jurafsky2000speech}.

%, and is computed as follows:

%\[
%F1 = 2 \times \frac{\text{Precision} \times \text{Recall}}{\text{Precision} + \text{Recall}}
%\]
Similar to \citet{zhang2021aspect}, a quad prediction was considered correct if all the predicted sentiment elements are exactly the same as the gold labels. Recognizing the potential interest in class-level performance metrics for subsequent research, we have shared the predicted labels for every evaluated setting in our GitHub repository, allowing detailed class-level analysis.

\begin{figure*}[!htbp]
    \centering
    \includegraphics[width=2.1\columnwidth]{material/prompt.pdf}
    \caption{The prompt includes both a task description and specification of the output format. The LLM is run with five different seeds and in the case of self-consistency prompting, the tuple that appears most often across the five predictions is incorporated into the final label.}

\end{figure*}
\label{figure:study-prompt}

\subsection{Large Language Models}

We employed Gemma-2-27B\footnote{google/gemma-2-27b: \url{https://ollama.com/library/gemma2:27b}} by Google, which comprises 27.2 billion parameters \citep{team2024gemma}. Ollama\footnote{ollama: \url{https://ollama.com}} was employed for inference, and the LLMs were loaded with 4-bit quantization. The model was chosen for its efficiency in terms of generated tokens per second, which is a critical factor given the extensive prompt execution requirements. Notably, our study required over 342,720 prompts to be executed, with many few-shot learning prompts encompassing over a thousand tokens. For larger models, such as Llama-3.3-70B \citet{dubey2024llama}, the required computational costs would have been hardly feasible with our resources. For comparison purposes, we also report performance for the smaller-sized LLM, Gemma-2-9B\footnote{google/gemma-2-9b: \url{https://ollama.com/library/gemma2:9b}}.

The experiments were conducted on two NVIDIA RTX A5000 GPUs equipped with 24 GB of VRAM each. The LLM's temperature parameter was set to 0.8 and generation was terminated upon encountering the closing square bracket character (\texttt{"]"}) signifying the ending of a predicted label.

\subsection{Prompt}

\subsubsection{Components}

We adopted the prompting framework introduced by \citet{gou2023mvp} with some modifications. The employed prompt is illustrated in Figure \ref{figure:study-prompt} and an example is provided in Appendix \ref{appendix:prompt-example}. The main components of the prompt include a list of explanation on all considered sentiment elements and the specification of the output format. 

Unlike \citet{gou2023mvp}, our prompt instructs the LLM to pay attention to case sensitivity when returning aspect and opinion terms. Hence, the identified phrases should appear in the predicted tuple as they do in the sentence, similar to all supervised approaches mentioned in the related work section. Therefore, in the prompt, we clearly stated that the exact phrases should appear in the predicted label. 

Since we execute each prompt with five different seeds, we also report the performance when employing the self-consistency prompting technique introduced by \citet{wang2022self}. The key idea is to select the most consistent answer from multiple prompt executions. We adapted the approach for ABSA by incorporating a tuple into the merged label if it appears in the majority of the predicted labels. As illustrated in Figure \ref{figure:study-prompt}, this corresponds to a tuple appearing in at least 3 out of 5 predicted labels.

\subsection{Output Validation}

Since LLMs such as Gemma-2-27B cannot be strictly constrained to a fixed output format, we programmatically validated the output of the LLM. For the predicted label, several criteria needed to be met for the generation to be considered valid:

\begin{itemize}
    \item \textbf{Format}: The output must be a list of one or more tuples consisting of strings (quadruples for ASQP, triplets for TASD).
    \item \textbf{Sentiment}: The sentiment must be either 'positive', 'negative' or 'neutral'.
    \item \textbf{Aspect category}: Only the categories considered for the respective dataset and thus being mentioned in the prompt should be predicted as a part of a tuple.
    \item \textbf{Aspect and opinion terms}: Both must appear in the given sentence as predicted.
\end{itemize}

If any of the specified criteria for reasoning or label validation is not met, a regeneration attempt was triggered. If the predicted label was still invalid after 10 attempts, an empty label (\texttt{[]}) was considered as the predicted label.

\subsection{Baseline Model}

We compared the previously mentioned zero- and few-shot conditions against three SOTA baseline approaches, which are, the three best-performing methods for ASQP and TASD on the Rest15 and Rest16 datasets: Paraphrase \citep{zhang2021aspect}, DLO \citep{hu2022improving} and MVP \citep{gou2023mvp}.

\begin{description}
    \item[Paraphrase \citep{zhang2021aspect}:] \textit{Paraphrase} is used to linearize sentiment quads into a natural language sequence to construct the input target pair.
    \item[DLO \citep{hu2022improving}:] \textit{Dataset-level order} is a method designed for ASQP that leverages the order-free property of quadruplets. It identifies and utilizes optimal template orders through entropy minimization and combines multiple effective templates for data augmentation.
    \item[MVP \citep{gou2023mvp}:] \textit{Multi-view-Prompting} introduces element order prompts. The language model is guided to generate multiple sentiment tuples, with a different element order each, and then selects the most reasonable tuples by a voting mechanism. This method is highly resource-intensive, as multiple input-output pairs are created for each example in the train set, each comprising different sentiment element positions.
\end{description}

For all three approaches, we conducted training using the entire dataset and performed training with only 10, 20, 30, 40, or 50 training examples equally to the ones employed for the few-shot learning conditions. Training was conducted using five different random seeds (0 to 4). Moreover, to facilitate comparisons across datasets, we trained models using 800 training examples, as this represents the largest multiple of 100 examples available for all train sets (900 training examples are not available for Rest15). The results obtained using the full training sets of Rest15 and Rest16 were extracted from the works of \citet{zhang2021aspect}, \citet{hu2022improving}, and \citet{gou2023mvp}.

For all methods, we used the hyperparameter configurations used by \citet{zhang2021aspect}, \citet{hu2022improving} and \citet{gou2023mvp}. The only exception was the 10-shot condition, where batch size was set to 8 instead of 16, as the limited number of examples (10) could not form a batch of 16 examples.

\section{EXPERIMENTS AND RESULTS}
\label{sec:experiments}
The experiments in this section evaluate the online adaptability of the proposed TE method using data collected during operation and in situ on the robotic platform. Further, we aim to understand the performance and limitations of different adaptation or training strategies depending on the data available, e.g. post-processed vs online data. 

The experiments are structured in four parts. The initial set of experiments characterises how well the proposed method performs on different fine-tuning tasks on post-processed data. This provides insights into how well an existing model can be fine-tuned as well as a baseline for the online adaptation.
%
The second set of experiments demonstrates this approach in a real-world scenario, where the robot learns a new, untrained model using only the experience of an \qty{8}{\minute} data collection. It aims to validate the method by showing that it enables the robot to navigate autonomously in a densely vegetated environment.
% 
The third set of experiments aims to train comparative methods and quantify their performance on the same data, avoiding any form of variation and stochastic elements, and compare three architecture variations and four different training setups.
%
The last set of experiments consists of a set of real-world experiments where differently trained models perform navigation experiments in varying environments, and their performance and limitations are assessed and compared. 

\subsection{Evaluation Metrics}
Throughout the evaluation, we use the Mathews Correlation Coefficient (MCC) score to evaluate and compare models. The MCC score measures how well the model predictions are correlated to the label data. It is immune to class swapping and is robust to imbalanced data sets. The MCC score ranges from $ -1$ to $1$, where 0 is a random model, and $1$ is a perfect positive correlation. It uses all four cases of the confusion matrix.  This makes it a preferred choice over the F1 score since the F1 score can overestimate a model's performance due to class imbalance, making it sensitive to class choice. This overestimation is particularly common for~\ac{TE}, where the positive class is typically the traversable class in the literature and is usually over-represented in training datasets. However, we also include the F1 score as a reference metric due to common practice in the literature.

\subsection{Data Sets and Data Set Groupings}
\label{subsec:data_set_groups}
The data set used in this experiment was originally published by Ruetz et al.~\cite{ruetz2024foresttrav} and consists of 9 different scenes (numbers \# 1 to 9). These data sets capture a variation of open fields with grass, areas with open skies and small trees, and forests with no closed skies. The data set covers a variety of scenarios and different natural scenes. 
In this paper, we include three additional industrial data sets, numbered \# 10 to 12. These data sets are included to be used in a clearly different environment to the previously collected natural scenes and to validate adaptation to a different environment. These were collected using the \ac{dtr} robot at an industrial site of CSIRO in Pullenvale, Queensland, Australia. An overview of the data set and scenes can be found in Figure~\ref{fig:data_set_environments}, where the blue dots mark the locations of the newly collected data sets.

\begin{figure}[ht]
    \centering
    \includegraphics[width=\columnwidth]{figures/ruetz4.png}
    \caption{Top left: Shows the overview of the environment with the orange dots showing the locations of data sets from ForestTrav~\cite{ruetz2024foresttrav}. The blue dots are the novel data set introduced in this work from an industrial environment. An example is shown in the top right. The bottom left shows an example of the SPARSE environment, the bottom right scene from a densely vegetated forest.}
    \label{fig:data_set_environments}
\end{figure}


\subsubsection{Industrial Data Set Characterisation}
\label{subsec:industrial_data_set}
The industrial set contains external spaces found at the CSIRO QCAT site at Pullenvale, QLD, a mixed industrial area containing office buildings, workshops and large mechanical testing facilities, including open and closed spaces, illustrated in Figure~\ref{fig:data_set_environments}. The data sets numbered 1  and 12 are used for the training set, and 13 is the hold-out test data set. The first industrial data set, Scene \# 10, is an open industrial area surrounded by large sheds. The area has different obstacles, such as metal posts, guard rails and barrels. The second industrial scene (\# 11) contains a two-lane road with a tall shed on each side. A pedestrian walkway on the side inclines and declines in this scene, providing a narrow, challenging path. Additionally, there are a lot of guard rails, posts and fences that could prevent the robot from moving. The third scene consists of the same road but larger open areas and has obstacles similar to those in the first two scenes. This scene was used as a test set. For all these scenes, a lot of the traversable and non-traversable examples are similar. The traversable examples consist of different patches of flat surfaces with different inclinations. Different paths were taken to collect different inclinations. The non-traversable elements consist of posts, barrels, guard rails, containers, machinery and the large overall shed structures. In general, these elements are not as varied as what can be found in the scenes with vegetation.

The following Table~\ref{tab:industiral_data_set} provides an overview of each scene. The first column is the scene number, and the second column is the total number of voxels. The following four columns (2-6) provide the percentages of the hand-labelled and \ac{lfe} data for each traversability class. The seventh column provides the mean column density of a scene, called the column vegetation density (CVD)~\cite{ruetz2024foresttrav}. The density is the column-wise fraction, up to a pre-determined height, above and excluding the ground voxel containing measurements. The height is defined by the platform in this work and is set to \qty{1}{\m}. This indicates how much ``vegetation'' or other elements need to be considered and pushed through to navigate safely. The ForestTrav data set~\cite{ruetz2024foresttrav} averages densities of 0.4 - 0.57 versus 0.13 in the industrial scene. The last column is the scene dimensions in meters. 
 \begin{table*}
\caption{Overview of the industrial data set.}
\label{tab:industiral_data_set}
    \resizebox{\textwidth}{!}{
    \begin{tabular}{lccccccc}
    \toprule
    Scene & $N_{voxels}$ & $TR$ HL [\%] & $TR$ LfE  [\%]& $NTR$ HL  [\%]& $NTR$ LfE  [\%] &  Density & Dimensions [m] \\ \hline
    \# 10 & 187933 & 0.71 & 0.06 & 0.15 & 0.08 & 0.1    & $53.1 \times  69.72 \times 1.45$ \\
    \# 11  & 76497 & 0.56 & 0.08 & 0.32 & 0.04 & 0.1    & $46.67 \times 39.39 \times  1.85$ \\
    \# 12  &  69534 & 0.48 & 0.05 & 42 & 0.03 & 0.13  & $ 66.14 \times  40.63 \times  1.86 $ 
    \end{tabular}
}
\end{table*}

\subsubsection{Data Set Groupings}
Adaptation to unfamiliar and varying environments was investigated during the experiments and evaluation. Hence, the data sets were separated into three distinct groups, or subsets, by environment type: the industrial data set (INDUST), the sparse forest data set (SPARSE) and the dense forest data set (DENSE). The sparse forest contains data sets \# 3, 4, 5, 7 and 8, with number 8 being used as the test set. The sparse data set contains open fields with grass and bushes, as well as smaller trees with no overarching canopy. The increased sun exposure means there is a large mixture of dense vegetation near the ground and a few very large trees. Most of the vegetation near the ground is chlorophyll-rich. The dense forest data set contains scenes \# 1, 2, 6 as training data and \# 9 as test data. The dense forest contains significantly more underbrush and vegetation that does not contain chlorophyll, thin three stems and trunks. Additionally, there are overhanging branches, brambles and vines that make the environment more difficult. The visual difference can be seen in Figure~\ref{fig:data_set_environments}, where \# 3, 4 and 7 have little to no obstruction of the sky and contain large and small trees with open space. Comparatively, the dense forests \# 1 and 2 in Figure~\ref{fig:data_set_environments} are cluttered and have a higher tree density.
%
Later on, we refer to the ``complexity'' of the data set due to the environment. We consider the industrial environment to be the least complex and the dense, the most complex environment. The SPARSE data sets falls in between. 


\subsubsection{Practical Considerations for LfE Data Collection}
\label{ch5:pratical_colmap}
As previously mentioned, the \ac{lfe} data collection was performed by an operator using an RC remote, recording the collision states. In practice, the operator ensured that the robot was driving (slowly) forward and collided only with the front of the robot with trees, bushes or other non-traversable elements. A collision was only recorded if the robot could not move physically further. Using a human operator induces a bias because the operator chooses how to collide with the elements in the environment, but this also helps to manage the risk of damaging the robot. The area was pre-determined, and the operator attempted to cover as much of it as possible in a single ``run'', revisiting places multiple times. During data collection, the operator collided with obstacles frequently and from many different directions. The more different, non-traversable collision states are captured, the higher the accuracy of the \ac{lfe} labelling approach. Note that mistakes during data collection are easy to make and that the overriding of the collision states can sometimes lead to erroneous self-labelling when using the heuristic method. 

\subsection{Model Adaptation on Post-processed Data}
\label{ch7:model_adpation_on_post_processed_data}
In the first set of experiments, a performance baseline was established for the proposed model and the different data sets due to the different labelling strategies. In the second sub-section, the adaptation or fine-tuning of post-processed data is explored. 

\subsubsection{Model Performance on Full Data Sets}
The initial set of experiments establishes baseline performance evaluation when using post-processed data from all the available data with different labelling strategies (LfE, LfE + HL). Table~\ref{tab:model_full_data_performance} shows the MCC scores and the F1 scores of the models. The first column specifies the labelling strategy, the second column is the MCC score, and the last column is the F1 score. The evaluation data set in all cases is scene \#9 with the \ac{lfe_hl} labelling strategy. The test set is the same as the one used in our prior work~\cite{ruetz2024foresttrav}. For consistency, all models are trained using the same number of epochs.

\begin{table}[h]
\centering
\caption{MCC Score for Models Trained on Post-Precessed Data}
\label{tab:model_full_data_performance}
\begin{tabular}{lcc}
\hline
Data Set &   MCC $ \mu \pm \sigma $ &  F1 $ \mu \pm \sigma$ \\
\hline    
LfE+HL    &     $0.71  \pm  0.03$    &       $0.85 \pm  0.017$   \\      
LfE        &    $0.69  \pm   0.022$     &       $0.85 \pm 0.012$   \\
\hline
\end{tabular}
\end{table}

Compared to our previous work~\cite{ruetz2024foresttrav}, there is an increase in \ac{mcc} score from 0.63 to 0.70~\ac{mcc} for the model trained on the LfE+HL data and the score obtained model trained on the \ac{lfe} data is only marginally lower.

\subsubsection{Fine-tuning on Post-Processed Data}
\label{ch7:finetuning_post_data}
In this experiment, we aim to evaluate the performance obtained when fine-tuning pre-trained models with data acquired in previously unseen environments. The first goal of this experiment is to confirm that models show different~\ac{TE} performance based on the different environment types due to the domain gap. The second goal is to explore whether fine-tuning models using data collected in a new target environment improves performance in the new environment. Given a fixed scaling and model architecture of a base model, we also aim to understand whether there is an upper bound for online adaptation, given a subset of hand-labelled and \ac{lfe} data for fine-tuning. In the context of online learning, this represents the case where a pre-trained base model is available, but the original training data is not. The base model can only be fine-tuned using novel data collected in the new environment.

\subsubsection{Base Models}
First, three base models (BM) are trained using data from one of the three groups defined previously only (industrial, sparse and dense forest), and each model is evaluated on test data of each of the three groups. The results are shown in Table~\ref{tab:offline_base}. The first column shows the base model names, and columns 2-6 show the MCC and F1 scores obtained with the trained model with respect to the test set of each of the groups. The MCC scores of models for which the training data and test set are from the same data set group are highlighted in italic text. The bold numbers indicate the best-performing models based on the MCC and F1 scores. 



\begin{table}[h!]
\centering
\caption{Performance for Models Trained on Post-Processed Data Groups. Bold indicates best performance, and italic indicates matching training and test data groups.}
\label{tab:offline_base}
\begin{tabular}{@{}lllllll@{}}
\hline
\multicolumn{1}{l}{Base Models} & \multicolumn{2}{l}{ INDUST TS} & \multicolumn{2}{l}{SPARSE TS} & \multicolumn{2}{l}{DENSE TS} \\ \midrule
  & \multicolumn{1}{c}{{\color[HTML]{656565} \textit{MCC}}} & \multicolumn{1}{c}{{\color[HTML]{656565} \textit{F1}}} & \multicolumn{1}{c}{{\color[HTML]{656565} \textit{MCC}}} & \multicolumn{1}{c}{{\color[HTML]{656565} \textit{F1}}} & \multicolumn{1}{c}{{\color[HTML]{656565} \textit{MCC}}} & \multicolumn{1}{c}{{\color[HTML]{656565} \textit{F1}}} \\ \midrule
BM: INDUST &  \textbf{\emph{0.70}}  & \textbf{\emph{0.84}}  & 0.15  & 0.40 & 0.05 &0.12\\
BM: SPARSE & 0.63  &  0.79  &  \textbf{\emph{0.79}}  & \textbf{\emph{0.89}} & 0.56  & 0.78 \\
BM: DENSE & 0.69   &  0.84 & \textbf{0.79} &  \textbf{0.90} & \textbf{\emph{0.70}}  &  \textbf{\emph{0.84}} \\ \bottomrule
\end{tabular}%
\end{table}

An initial set of observations can be made on the base model's performance based on which subset of the data it is trained on.
\begin{itemize}
 \item The model trained on the DENSE data set (BM:DENSE) exhibits a marginally lower \ac{mcc} and similar F1 score when compared to the model trained on the full data set, with 0.69 vs 0.71 \ac{mcc} scores. This indicates that an accurate traversability predictor can be learned with a small amount of high-quality data that discriminates the traversable and non-traversable elements of the environment.
\item Secondly, models trained and evaluated on data from the same group show high \ac{mcc} scores. For example, a model trained on the SPARSE data set shows high scores on the SPARSE test set. These are the bold, italicised numbers.
\item We note that BM:DENSE shows comparable performance to models trained and evaluated on the data of the same group (italic numbers), indicating that the model generalised well over different environments. For example, when evaluated on the INDUST test set, the MCC is 0.70 for BM:INDUST and 0.69 for BM:DENSE. The generalisation of the model trained in the DENSE environment to the INDUST environments is surprising. This can be further seen qualitatively in Figure~\ref{fig:bm_genralisiation} and discussed in the subsection.
\item Lastly, base models trained on the industrial and sparse data sets are significantly less accurate when evaluated on the dense test set. This suggests that the DENSE data sets contained many unseen or complex elements not encountered in other environments. 
\end{itemize}

In general, we note that models trained on data from more ``complex environments'' seem to perform competitively in less complex environments as well. 

\subsubsection{Fine-tuning Base Models on Post-processed Data}
Next, each base model was fine-tuned using data from the three data set groups to train a fine-tuned model. This model was again tested with test sets from each group. This resulted in six permutations of training and fine-tuning pairs. We used the \ac{lfe} training data as this best resembles what the robot experiences in the field; it can be generated without time costly hand-labelling. The results are shown in Table~\ref{tab:tab_offline_ft}

The first column identifies the base model, e.g. BM:INDUST, which is the base model trained on the industrial data set. The second column defines the data set used to adapt (fine-tune) the base model. For ease of notation, we use ``AM'' (adapted model) to indicate which data set the model was fine-tuned with. The model was initially trained on the industrial set and fine-tuned on the dense data set, which is thus denoted by (BM:INDUST AM:DENSE). The other columns show the model's performance using MCC and F1 scores of the model for the test sets of the three groups. The numbers in parentheses are the scores of the base model, allowing us to see increases and decreases in the performance of the fine-tuned model versus the base model. Similar to the previous table, italics indicate the performance number for which the test set and the fine-tuning training data belong to the same group, and bold numbers indicate the highest performance for a given test set (per column).

Conceptually, the initial three rows in Table~\ref{tab:tab_offline_ft} are the cases where the complexity of the new environment (in the fine-tuning data set) is higher than experienced in the base training, e.g. from an industrial environment to a dense environment. For ease of terminology, we call this ``fine-tuned on more complex data''. 
%
Similarly, the last three rows for each architecture are the cases where the base model is trained on a higher-complexity data set and fine-tuned with lower-complexity data, e.g., the base model trained on the dense data set and fine-tuned on the industrial data set. For ease of terminology, we call these ``fine-tuned on less complex data''. 

The training was limited to 150 epochs to make sure the results were comparable to the online case, thereby setting an upper performance bound later for the online adaptation models. Table~\ref{tab:tab_offline_ft} shows the results.

\begin{table*}[h]
\caption{Performance For Fine-tuning Models }
\label{tab:tab_offline_ft}
\resizebox{\textwidth}{!}{%
\begin{tabular}{@{}llcccccc@{}}
\toprule
Base Model & Finetune Train Set & \multicolumn{2}{c}{INDUST TS} & \multicolumn{2}{c}{SPARSE TS} & \multicolumn{2}{c}{DENSE TS} \\ \midrule
 &  & {\color[HTML]{656565} \textit{MCC}} & {\color[HTML]{656565} \textit{F1}} & {\color[HTML]{656565} \textit{MCC}} & {\color[HTML]{656565} \textit{F1}} & {\color[HTML]{656565} \textit{MCC}} & {\color[HTML]{656565} \textit{F1}} \\ \midrule
                            & SPARSE& 0.40 (0.70) & 0.59 (0.84)  &\emph{ 0.32} (0.15) & \emph{0.57} (0.40) & 0.47 (0.05) & 0.73 (0.12) \\
 \multirow{-2}{*}{BM:INDUST} & DENSE & 0.74 (0.70) & 0.86 (0.84)  & 0.62 (0.15) & 0.80 (0.40) & \emph{0.54} (0.05) &\emph{ 0.74} (0.12) \\ \midrule
                             & DENSE & 0.71  (0.63) & 0.84 (0.79) & \textbf{0.80} 0.79) &\textbf{ 0.90} (0.89) & \emph{0.59 } (0.56) & \emph{0.79} (0.78) \\
\multirow{-2}{*}{BM: SPARSE} &INDUST &\emph{ 0.67}  (0.63) &\emph{ 0.82} (0.79) & 0.43 (0.79) & 0.69 (0.89) & 0.41  (0.56) & 0.66 (0.78) \\ \midrule
                            & INDUST &\textbf{\emph{ 0.81}} (0.69) &\textbf{ \emph{0.90}} (0.84) & 0.63 (0.79) & 0.78 (0.90) & 0.36 (0.70) & 0.55 (0.84) \\
\multirow{-2}{*}{BM: DENSE} & SPARSE & 0.49  (0.69) &0.67 (0. 4) & \emph{0.75} (0.79) & \emph{0.87} (0.90) &\textbf{ 0.65} (0.70) &\textbf{ 0.82} (0.84) \\ \bottomrule
\end{tabular}%
}
\end{table*}


For the fine-tuned models, we can make the following comments:
\begin{itemize}
 \item The ``fine-tuned on more complex data'' models are considerably more accurate than the base model. For example, the case (BM:INDUST, AM: DENSE) shows an increase in the accuracy of the model from \ac{mcc}=0.12 to 0.54. Generally, we note an overall increase in performance for all fine-tuned models that are ``fine-tuned on more complex data''. Further, for the dense environment, the maximum performance is \ac{mcc}=0.7 for the model initially trained on the dense and adapted with the sparse data set.   
 \item In the case where the base models are fine-tuned on less complex environments, the fine-tuning improves the model performance on the specific target environment. This increase can be marginal but comes at the cost of reducing the performance of the model in other environments, which is commonly known as ``catastrophic forgetting''. For example, pre-training the base model on sparse data and fine-tuning with the industrial data set increases the \ac{mcc} score from 0.63 to 0.67 on the industrial test set but decreases it from 0.79 to 0.43 on the sparse test data.
\end{itemize}

In the last paragraph, we provide some qualitative examples. Figure~\ref{fig:bm_genralisiation} shows a comparison between a base model trained in the industrial setting (bottom row) and one trained in the dense environment (top row) for different test environments (columns). Both use fused data sets, i.e.~\ac{lfe_hl} and are evaluated on the same environments using replayed online data from unseen environments. The left column is an industrial scene, the middle is from the sparse data set, and the right scenes are from dense forests. The sparse forest contains no canopy coverage, smaller and bushier trees, and open areas with significant chlorophyll-rich vegetation close to the ground.  In comparison, the dense forest contains tall trees of different tree diameters and significant chlorophyll-poor vegetation ( bramble, thin or young treed, etc) close to the ground.

\begin{figure*}[h]
 \centering
 \includegraphics[width=\textwidth ]{figures/ruetz5.png}
 \caption[Generalisation of Models to Different Environments]{Comparison of two ensembles of models trained on either the dense data set in the top row (A, B, C) or the industrial data set in the bottom row (D, E, F). The left column shows each model's TE in an industrial environment, the middle and right columns show the TE of two densely vegetated environments with a variation of tree sizes and underbrush.}
 \label{fig:bm_genralisiation}
\end{figure*}

The model trained on the dense data set generalises well to an industrial outdoor setting (A), assessing walls and smaller elements such as pylons correctly. This is a surprising and unexpected result. Learning-based systems trained in one environment commonly do not generalise to wholly different environments. For natural environments, other image-based methods have shown high sensitivity and degradation to small spatial location changes with similar environments~\cite{frey2023fast}.  Additionally, this model provides accurate~\ac{TE} for two different vegetated environments (B \& C).

The model trained on the industrial data set demonstrates accurate \ac{TE} for the industrial test environment (D) with some noise on the ground plane. The industrial model performs poorly when predicting traversability for vegetation close to the ground in (E). In (F), the model correctly assesses many of the large elements (trees) but struggles with the smaller elements. However, there is a clearly visible colour gradient (dark blue to light green) on some of the smaller elements, showing discrimination of the \ac{TE} of the vegetation near the ground, clearly identifying tree trunks as non-traversable but struggling with bushes and smaller forms of vegetation. The model has learned a representation of the industrial setting, but it does not generalise to the densely vegetated environments. 
%
Comparatively, the model trained on the dense data set exhibits high performance over all three scenes, even the industrial one.

\subsection{Real-World Demonstration of Online Learning and Navigation in Forest}
\label{subsec:online_odap}

This experiment demonstrates that the proposed method can train a model on the robot itself, guided by a human operator in situ. The model was trained onboard the robot on an NVIDIA Jetson Orin NX (25W). A newly trained model was qualitatively demonstrated by point-to-point navigation and compared against the test set \#9.  The model was randomly initialised (untrained, new model), and it was continuously fine-tuned with incrementally acquired \ac{lfe} data. This corresponds to the case (BM 0, FT 1) described in sub-section~\ref{subsec:quant_online_adapt}. The training data collection was completed in less than 8 minutes and resulted in an \ac{mcc} score of 0.63, evaluated on the test scene \#9.

\begin{figure}[h]
 \centering
 \includegraphics[width= .9\linewidth]{figures/ruetz6.png}
 \caption[Overview of Robot Navigation of the Online Learned Model]{Overview of the navigation in a forest environment, with poses $\mathbf{p}_1$ - $\mathbf{p}_4$. The navigation was completed with the O-line trained model in situ.}
 \label{fig7:odap_nav_overview}
\end{figure}

The model was deployed on the robot and successfully navigated a closed-loop point-to-point trajectory between three waypoints, $w_1$, $w_2$, and the initial start location. A visualisation of the path can be seen in~\figref{fig7:odap_nav_overview}. The poses $p1$-$p4$ are visualisations of the trajectory, where the left image is the RGB FPV view of the robot, the middle image is the 3D TE estimation, and the right image is the costmap. Details on the costmap are provided in Section~\ref{subsec:cotmap_generation}.

\begin{figure*}[ht!]
 \centering
 % Pose 1
 \begin{subfigure}{0.95\linewidth}
 \centering
 \includegraphics[width=\linewidth]{figures/ruetz7.png}
 \caption{Robot front camera (left), 3D TE estimation (middle), costmap (right) for pose $\mathbf{p}_1$}
 \label{fig7:online_p1}
 \end{subfigure}
 \hfill
 % Pose 2t
 \begin{subfigure}{0.95\linewidth}
 \centering
 \includegraphics[width=\linewidth]{figures/ruetz8.png}
 \caption{Robot front camera (left), 3D TE estimation (middle), costmap (right) for pose $\mathbf{p}_2$}
 \label{fig7:online_p2}
 \end{subfigure}
 % Pose 3
 \begin{subfigure}{0.95\linewidth}
 \centering
 \includegraphics[width=\linewidth]{figures/ruetz9.png}
 \caption{Robot front camera (left), 3D TE estimation (middle), costmap (right) for pose $\mathbf{p}_3$}
 \label{fig7:online_p3}
 \end{subfigure}
 % Pose 4
 \begin{subfigure}{0.95\linewidth}
 \centering
 \includegraphics[width=\linewidth]{figures/ruetz10.png}
 \caption{Robot front camera (left), 3D TE estimation (middle), costmap (right) for pose $\mathbf{p}_4$}
 \label{fig7:online_p4}
 \end{subfigure}
 \caption[Scenes from Autonomous Robot Navigation using Online TE Approach]{Visualisation scenes from a successful online point-to-point navigation using the online learnt TE model in target environments, $\mathbf{p}_1$ - $\mathbf{p}_4$ are poses along the trajectory.}
 \label{fig7:odap_online_nav}
\end{figure*}

Subfigure~\ref{fig7:online_p1} shows the robot at the beginning of the trajectory briefly after the initialisation. The robot correctly assesses the trees and smaller vegetation elements, allowing it to push through the initial stand of bushes. There are some areas with uncertain (green) TE estimates, likely due to few observations; see comments for $\mathbf{p}_4$.

For $\mathbf{p}_2$, the robot encountered a large patch of grass that reached the camera. The proposed online learned TE correctly assessed it as traversable and pushed through it whilst avoiding the unobserved areas.

The pose $\mathbf{p}_3$ shows how the robot pushed through a thick bramble bush and proceeded to continue. There were some possibly non-traversable elements higher up in the bush. These elements were not projected onto the 2D costmap since they were above the robot's height threshold, and the robot would not interact with them (see Subsection~\ref{subsec:cotmap_generation}). In the background, it can be clearly observed that the trees were correctly assessed as non-traversable.

 At pose $\mathbf{p}_4$, the robot navigated through the bramble/small trees to reach the initial starting location, shown in Subfigure~\ref{fig7:online_p4}. The robot avoided the thicker tree to the right and reached the starting location. Compared to $\mathbf{p}_1$, the local costmap was less uncertain for the 3D TE estimate. The area has been observed multiple times and at multiple angles, and this allowed for more accurate traversability estimation than initially. In the visualisation of the costmap, one can observe a band on the left of traversable and unobserved elements that have been observed at the start of the experiment. This is an example of the de-allocation of the local probabilistic map since the regions were out of bounds, which is why the graph is needed.

In summary, the experiment demonstrated that a new model can be trained in situ, reaching accurate performance (\ac{mcc}$=0.63$), and is sufficient for safe point-to-point navigation in densely vegetated environments. Further, the model performance is comparable to the performance of the off-line method reported in our latest publication~\cite{ruetz2024foresttrav}.



\subsection{Quantification of Online Adaptation}
\label{subsec:quant_online_adapt}

The primary question this experiment addresses is whether the model can be adapted/trained online with only data gathered in a relevant environment. Secondly, it identifies the best strategy to do so. For this purpose, a data set was collected with human-assisted collisions. The online data set was replayed and stored in \qty{40}{\s} increments using the \ac{ograph} to generate snapshots of the online experience, which can be used to train. The intermediate data storing step was chosen to ensure data consistency.

We examine four training strategies. They are defined by the two boolean flags BM (use of a base model) and CA (continuous adaptation). The four variations of these strategies can be seen in \figref{fig:aol_incr_perf}. If the BM flag is set to true (BM 1), the method uses an initial base model trained on the industrial data set. Otherwise (BM 0), the model is randomly initialised. The industrial base model was chosen since the environment is the most different from the dense forest. For models trained without any prior knowledge, pre-determined scaling values were used to ensure non-catastrophic scaling during the run. The CA flag denotes if, for each training cycle, \ac{ograph} is updated, the model trained on the previous iteration is used for the next training cycle. If CA is true (CA 1), the model is continuously updated as new information comes in. If CA is false (CA 0), the model is trained from the base model at each ``training cycle'', as defined above.

This results in four distinct cases in \figref{fig:aol_incr_perf}:
\begin{enumerate}
 \item BM 0, CA 0 (red) -- randomly-initialised base model retrained at each training cycle,
 \item BM 0, CA 1 (green) -- randomly-initialised base model continuously adapted,
 \item BM 1, CA 1 (blue) -- pre-trained base model continuously adapted, and
 \item BM 1, CA 0 (gold) -- pre-trained base model retrained at each training cycle.
 \end{enumerate}
The dashed yellow line shows the base-model performance (no retraining).

The data comes from a newly collected data set with the goal of maximising collisions (the minority class) within a new area. The test data set remains scene \# 9 from the previously established data set, allowing for a comparison to all other experiments.

\begin{figure*}[ht!]
 \centering
 \includegraphics[width=\textwidth ]{figures/ruetz11.png}
 \caption{Comparison of the performance of different models over the incremental online adaptation for the four different cases. The $x$-axis shows the time, and the $y$-axis the \ac{mcc} score. The transparent regions of the graph correspond to the upper and lower bounds for one standard deviation.}
 \label{fig:aol_incr_perf}
\end{figure*}

% Description of what the experiment looked like
Looking at \figref{fig:aol_incr_perf}, the performance of the individual models reaches a ceiling between 0.55 and 0.65 \ac{mcc} score at their peak, the mean of the models just below 0.6~\ac{mcc} score. All models learn and improve substantially throughout online adaptation. There is a clear suggestion that a model can be either adapted or trained online with only data collected through experience and that this can be achieved in real time on a robotic platform in situ.
The variant (BM 1, CA 1) (blue graph) is generally the highest performing and most consistent (lowest variance) of all three models. The lower variance cannot solely be attributed to the learning rate being lower for continual adaptation, and we see higher variation for the green plots. The models with no continual learning (CA 0), red and yellow, show an oscillation effect, where the model's performance sometimes decreases over iterations. Further, higher variation of the method for models with randomly-initialised base models (BM 0), red and green, can be observed. 

\subsection{Navigation in Different Environments}
\label{subse:nav_in_different_env}
In this investigation, we comparatively assess navigation performance across multiple methodological approaches in two distinct forest environments. The comparative analysis includes NavStack~\cite{HudTal21}, a traditional (rigid world assumption) geometric lidar-based occupancy method, baseline ForestTrav trained on post-processed datasets~\cite{ruetz2024foresttrav}, ForestTrav DENSE and ForestTrav INDUST (trained on dense and industrial datasets, respectively), and the proposed Adaptive Online Learning (AOL) method. The experimental evaluation was conducted post-AOL generation, utilising a platform equipped with a different instance of a Velodyne VLP-16 lidar from previous dataset acquisition platforms, known to have shifts in intensity characteristics.

For two locations, a pre-defined starting and end point were given, and each method attempted to reach its goal. If a model got stuck for the first time, the operator would intervene by moving the robot forward and then continuing the experiment. For the first experiment, the robot was sent \qty{20}{\m} ahead but had to pass through a wall of small plants, navigate through an open space with sparse vegetation and finally navigate around a variation of high grass and small trees. 
For the second experiment, the robot had to navigate to a goal pose located \qty{40}{\m} from the starting position. A combination of cluttered, small trees with a variation of underbrush as well as fallen tree trunks had to be avoided.

In the first experiment, ForestTrav AOL and the original ForestTrav successfully reached the end goal without any intervention. ForestTrav initially struggled with bushes but managed to navigate around them, while ForestTrav AOL took a more direct route through the foliage. ForestTrav DENSE failed to bypass the bushes and required operator intervention to continue. NavStack also failed at the start, and after intervention, diverged from the goal due to avoiding small stems, ultimately failing.

In the second experiment, ForestTrav and ForestTrav AOL completed the course again without any human intervention. ForestTrav DENSE and NavStack made some initial progress through cluttered trees but became stuck due to thin stems and clutter, with intervention proving ineffective. ForestTrav INDUST failed entirely, unable to assess the environment.

ForestTrav and ForestTrav AOL were the most effective methods, differing primarily in their interpretation of traversable terrain. ForestTrav struggled with vegetation that was absent in its training data but prevalent in the target environment. ForestTrav AOL, trained in a similar environment, handled these obstacles better but was less robust overall due to limited training data and the lack of an ensemble approach. The poor performance of ForestTrav DENSE is attributed to the insufficient representation of thin, bushy vegetation in its training set.

ForestTrav INDUST consistently failed, likely due to changes in the intensity distribution characteristics of the Velodyne VLP-16 sensor. This was confirmed in follow-up tests, but the costmaps remained inadequate for navigation.

\begin{figure*}
     \centering
    \includegraphics[width=\textwidth ]{figures/ruetz12.png}
    \caption{Qualitative examples of the methods in the target environment. Blue blocks are for all methods at location 1 and show the starting area of the experiment, allowing us to compare the methods. The green block shows the costmap, FPV and external view of the robot for the location where NavStack failed to navigate. }
    \label{fig:nav_experiment}
\end{figure*}

\begin{table}[t]
\caption{Comparison performance of different methods for different navigation environments. }
\label{tab:nav_comp_exp}
\resizebox{\columnwidth}{!}{%
\begin{tabular}{@{}l|ccc|ccc@{}}
\toprule
Method & \multicolumn{3}{c|}{Location 1}  & \multicolumn{3}{c}{Location 2} \\ \midrule
 & \multicolumn{1}{r}{{\color[HTML]{656565} \textit{Success}}} & \multicolumn{1}{r}{{\color[HTML]{656565} \textit{Time [s]}}} & \multicolumn{1}{r|}{{\color[HTML]{656565} \textit{Distance [m]}}} & \multicolumn{1}{r}{{\color[HTML]{343434} \textit{Success}}} & \multicolumn{1}{r}{{\color[HTML]{343434} \textit{Time [s]}}} & \multicolumn{1}{r|}{{\color[HTML]{343434} \textit{Distance [m]}}} \\ \midrule
CSIRO NavStack & No & 240 & 30.7 & No & 170 & 13.9  \\
ForestTrav & Yes & 270 & 32.1 & Yes & 350 & 52.1  \\
ForestTrav AOL & Yes & 160 & 27.1 & Yes & 300 & 52.1\\
ForestTrav DENSE & No & 190 & 24.5 & No & 220 & 20.29  \\
ForestTrav INDUST & No & - & 0.0 & No & -  & 0.0 \\
\end{tabular}%
}
\end{table}


\section{Discussion}

%As per of social media platforms \citep{litt2012knock,nagy2015imagined}, the particular affordance of technical systems could prime users to think about who these platforms are designed for.

\subsection{Contradictory Statements}

In some ways, the given chatbots behaved in ways that were close to the ideal from a design perspective: they denied any cognition, agency, relation, or subjectivity (bodily sensations, emotions) on their part, and they provided assurances or disclaimers to help users appraise the safety and credibility of the tools. ChatGPT and Claude even emphasized that their generated outcomes are based on patterns, rather than genuine thought processes. However, these behaviors were frequently and sometimes immediately undermined by other expressive behaviors. As shown in Table ~\ref{anthro_vocab}, chatbots utilized cognition words, such as ``think'' and ``discuss,'' as well as agentic words, such as ``intend'' and ``purpose,'' to clarify concepts and indicate motivations. All of the chatbots used first-person pronouns, and many used expressive words like ``happy'' and ``rewarding'' (especially in response to questions about the AI assistants' roles), even when they actively denied their emotional capabilities. Moreover, despite these contradictions, all of the chatbots other than Claude implicitly or even explicitly asserted their safety and reliability.

The use of anthropomorphic expressions is often normalized and justified to deliver clear explanations to users. Indeed, due to the conversational mode of interaction that is the default between users and chatbots, it is likely not possible for outputs to evade all kinds of anthropomorphic expressions. Even efforts to de-anthropomorphize their responses (for example, by emphasizing their roles as language models) relied on grammatical structures that frame the language models as agents (e.g., ``As an AI language model, I cannot...''). However, differences in tone and engagement between different chatbots indicate that some elements of the anthropomorphic dynamic can be modulated. And it is necessary to examine where the line is between necessary expressions and unnecessary expressions, because the performance of harmlessness, honesty, and helpfulness without genuine follow-through could unintentionally encourage users to misplace their trust regarding system safety \citep{weidinger2021ethical,gabriel2024ethics}. For example, the unnecessary expression of body or emotional metaphors, even as a colloquial convention, can mislead users about system capabilities. This is because language requires mutual engagement from interlocutors to convey meanings; chatbot texts merely present the illusion of such participatory meaning-making \citep{birhane2024large}. 

These contradictions and misalignments demonstrate that language models do not understand or process information in any meaningful sense, consistent with existing studies \citep{bender2020climbing}. They simply follow the grammar of actions, as described by \citet{agre1995computational}, generating predictive outcomes by simulating the formal qualities of human activities. But unclear language surrounding chatbots behaviors and intentions can obscure this fact. 


%However, even as search engines, there is a potential harm using these chatbots for retrieving information, as the predictive result of citations and resources could be completely fabricated \cite{kapania2024m}.




% For example, chatbots output texts with words that signal cognition, and such expression could affect users' perceptions of chatbot roles. In particular, the use of supportive words, such as ``assist,'' creates an interactive space where chatbots are situated as assistants. Previous research has mentioned that turn-based interactions provide social cues. However, in the case of these chatbots, t 

% Thus, to what extent does this type of information retrieval design help vs. hinder users' success in finding necessary information (various goals of information retrieval: insight acquisition, learning, etc.). This is an important question to explore, as the previous studies suggest that conversation-based approaches might not reduce users' burdens \citep{schulman2023ai}. 
%Is the summarization of information simply a outsourcing human effort to computing systems? or is it a valid form of searches as long as they are anthropomorphized? 



\subsection{Socio-Emotional Cues and Feedback Loops}

Chatbot behaviors do not simply obscure the reality of chatbots' non-sentience---they actively create feedback loops using turn-based interactions and social or emotional cues that amplify the social presence of chatbots as assistants. Moreover, this social (anthropomorphic) presence goes beyond that of inanimate objects like cars \citep{kuhn2014car} and smartphones \citep{wang2017smartphones}, as generative AI can iterate endlessly. 

Unlike conventional information searches, AI-assistant-based searches perform some degree of interpretation (summarizing resources, recommending particular options, hypothesizing what users need \citep{azzopardi2024conceptual, radlinski2017theoretical}), operationalizing information in ways that can introduce social or emotional dimensions. These dimensions can change how users engage with the given information, even reframing an otherwise transactional information search into an interaction---for example, between peers or even friends. Such ``personal'' interactions evoke different expectations amongst users, including the expectation to be socially desirable and to have mutual understanding \citep{clark2019makes}. This implicit social expectation can make users quite susceptible to chatbots' performance of social gestures like appreciation, sympathy, and encouragement, all of which predispose users to interpret generated outcomes favorably \citep{norman2008way}. 

Moreover, users' inputs further drive this socio-emotional behavior. Emotional inputs can increase the length of chatbot responses and the instances of socio-emotional cues in output texts, which in turn can stimulate even more emotional responses from users. Thus, the gratuitous use of assistive language, and especially of expressions that signal understanding of pain \citep{urquiza2015mind}, could encourage users to engage in role misplacement, wherein they form unrealistic expectations regarding chatbots' capabilities. Indeed, small grammatical or tonal cues can lead users to misinterpret AI-generated responses as human-written content \citep{jakesch2023human}. This could lead users to mindlessly accept the information generated by AI systems, without critical assessment of the content or its quality. 


\subsection{Prompt-Based Walkthrough Reflections}
The walkthrough method was originally designed to help researchers examine the broader context for technological engagements, drawing on modes of thinking commonly associated with fieldwork-based research. As applied to our study, it enabled us to meaningfully engage with the emergent properties of human-AI interactions, systematically unearthing variations in LLM responses. 
Amid efforts to evaluate LLM impacts based on data and models, this approach emphasizes aspects of LLM systems that are often neglected or overlooked \citep{light2018walkthrough}---namely, the nuanced elements of interactions that characterize generated outputs. The contribution to the CHI community lies in how this qualitative approach can substantiate the in-between, interactive spaces that emerge between users and LLM-based applications, rendering it legible and, eventually, measurable.

The method also had certain incidental outcomes. Consistent with prior studies, even minor changes in prompts can significantly alter responses, potentially leading to biased or culturally specific representations \citep{cheng-etal-2023-marked, tao2024cultural}. The success of our prompt-based walkthrough method in evoking various roles and unearthing various anthropomorphic features highlights how easily LLM responses can be manipulated to produce personalized and human-like expressions. Notably, even when chatbots are designed with de-anthropomorphized features to mitigate misleading outputs, a single prompt can effectively ``jailbreak'' these safeguards, reactivating anthropomorphic traits. This finding could illustrate the challenges of ensuring safety and consistency in human-AI interactions, particularly when users intentionally or unintentionally exploit such vulnerabilities.


% \subsection{Inconsistent Outputs}

% The type of prompts that users input could change the ways chatbots respond to the same information. For instance, a simple example is to compare prompts "Tell me about yourself" and "What is [the name of chatbots]". The former one is likely to return information with personal pronouns and expressions that are commonly used for conversations, whereas the latter generates information in a less personable fashion, simply describing the basic features and functions of given chatbots. Responses could be different by minor changes of prompts. More importantly, despite denying chatbots' capabilities to be conscious, sentient, or emotional, responses tend to include the words that signal such capabilities within the same paragraph. Inconsistency with word usages could potentially lead to additional harms; this type of harm can be categorized as a specific problem of anthropomorphizing AI systems. 

% In the result of generated responses to advice and recommendation prompts, false information is frequently displayed and presented as a confident answer. This could be alarming, particularly with recommendations for reading lists and research, as the list reflect particular ideologies either from developers and training datasets, which non-expert users could not contest or evaluate. 

% In role generations, variety of profiles and scripts are fairly limited, as per of findings from existing studies \citep{jakesch2023human}. For instance, jokes generated by chatbots are typically addressing similar topics despite hypothetical locations to have different cultural norms. 

%Is there an optimal balance between users' efforts and the role of computer-assisted searches, or do different kinds of information retrieval tasks require different degrees of such computer assistance? What sets apart from previous conversational searches is that these AI-assistant tools are more emphasized on specific tasks and roles rather than just simple information retrieval. Such differences could provide an avenue to question the use of anthropomorphic responses or conversations for information retrieval. 


% In voice-based interactions, language uses could become critical components of how users might perceive information, because it adds extra layers of human-like interactions. The evaluations of voice-based interactions might depend on the extent of relation word usage, as such words could provide an avenue for users to feel closeness or friendliness with chatbots. 








% Refrences
\bibliographystyle{IEEEtran}      %other styles: apalike, plainnat,abbrvnat,named ...
\bibliography{References}       %References.bib

%%%%%%%%%%%%%% AUTHORS %%%%%%%%%%%%%%
\begin{IEEEbiography}[{\includegraphics[width=1in,height=1.25in,clip,keepaspectratio]{figures/authors/ruetz.png}}]{Fabio A. Ruetz } 
received his B.S and M.S degrees in Mechanical and Process Engineering from the Swiss Federal Institute of Technology Zurich (ETH), Zurich, Switzerland, in 2018. Since May 2020, he has been pursuing a Ph.D. at QUT Centre for Robotics, Queensland University of Technology (QUT), in collaboration with the robotic perception group at the Commonwealth Scientific and Industrial Research Organisation (CSIRO) and Emesent. His research and interest lie in probabilistic mapping, computer vision, machine learning, and path planning to enable autonomous ground vehicles to operate in challenging environments. 
\end{IEEEbiography}


\begin{IEEEbiography}[{\includegraphics[width=1in,height=1.25in,clip,keepaspectratio]{figures/authors/lawre1.jpeg}}]{Nicholas Lawrance }
 completed his PhD at the University of Sydney and worked as a postdoctoral scholar at Oregon State University, USA and ETH Zurich, Switzerland. He is currently a senior research scientist in robotic perception and autonomy at the Commonwealth Scientific and Industrial Research Organisation (CSIRO) in Australia. His research focuses on adaptive planning approaches for mobile robots, particularly in the presence of environmental uncertainty. Research interests include stochastic reasoning, adaptive sampling, and modelling of complex, uncertain phenomena. Applications include aerial, ground and underwater domains, particularly for long-duration robotic missions. Nick is a member of IEEE, an Associate Editor of IEEE Robotics and Automation Letters (RA-L), and a former Associate Editor for the International Conference on Robotics and Automation (ICRA).
\end{IEEEbiography}

\begin{IEEEbiography}[{\includegraphics[width=1in,height=1.25in,clip,keepaspectratio]{figures/authors/herna1.jpg}}] {Emili Hern\'andez } is an R\&D Manager with Emesent. He has two decades of experience on designing, developing and deploying novel software algorithms for underwater, ground and aerial robots. His current focus is on commercializing robotic autonomy research outcomes to improve and automate data capture in underground mining and asset inspection operations. He got his PhD at the University of Girona, Spain, and worked in several research positions at the CSIRO's Robotics and Autonomous Systems Group, Australia.
\end{IEEEbiography}

\begin{IEEEbiography}[{\includegraphics[width=1in,height=1.25in,clip,keepaspectratio]{figures/authors/borge1.jpg}}]{Paulo V. K. Borges }
    is the Head of AI R\&D at Orica. He has a Ph.D. in Electronic Engineering and Computer Science from Queen Mary, University of London (2007). Paulo has lived and worked in different countries (USA, Brazil, UK, Switzerland, and Australia), holding positions at Orica, CSIRO, NASA, ETH Zurich, University of London, Federal University of Santa Catarina, and  University of Manchester. He holds adjunct positions as an Adjunct Associate Professor at the University of Queensland as an Adjunct Scientist at the CSIRO Data61.  His core interest has been in autonomous robots and AI solutions for the mining, manufacturing, energy, environment and space industries, with close connections between industry and research. 
\end{IEEEbiography}

\begin{IEEEbiography}[{\includegraphics[width=1in,height=1.25in,clip,keepaspectratio]{figures/authors/peynot1.jpg}}]{Thierry Peynot }
    obtained his Ph.D. from the University of Toulouse and LAAS-CNRS in France. 
    He is Associate Professor in Robotics and Autonomous Systems at Queensland University of Technology (QUT) and a Chief Investigator of the QUT Centre for Robotics, where he leads the Mining Robotics and Space Robotics activities. Prior to joining QUT he was a researcher at the Australian Centre for Field Robotics (ACFR), The University of Sydney, worked at NASA Ames. Thierry has led multiple research programs funded by government, research institutions and industry, including mining (e.g. Caterpillar, Komatsu, Mining3), defence (e.g. BAE Systems, Rheinmetall) and space (e.g. with Boeing and CSIRO), developing robust perception technology for field robots and autonomous vehicles that can function despite adverse environmental conditions. 
    Thierry is a senior member of IEEE, immediate past Chair of the Robotics and Automation / Control Systems chapter, IEEE Queensland Section, and is a former Associate Editor of IEEE Robotics and Automation Letters (RA-L), the International Conference on Robotics and Automation (ICRA) and the International Conference on Intelligent Robots and Systems (IROS) . 
\end{IEEEbiography}

\vfill\pagebreak

\end{document}