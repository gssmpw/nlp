\documentclass[journal,onecolumn]{ieeetj}
\usepackage{cite}
\usepackage{amsmath,amssymb,amsfonts}
\usepackage{algorithmic}
\usepackage{graphicx,color}
\usepackage{textcomp}
\usepackage{xcolor}
\usepackage{siunitx}
\usepackage{subcaption}
\usepackage{caption}
\usepackage{hyperref}
\hypersetup{hidelinks=true}
\usepackage{algorithm,algorithmic}
\usepackage{acro}
\usepackage{booktabs}
\usepackage{multirow}
\begin{acronym}
\acro{gan}[GANs]{Generative Adversarial Networks}
\acro{rl}[RL]{Reinforcement Learning}
\acro{pae}[PAE]{Periodic Autoencoder}
\acro{fld}[FLD]{Fourier Latent Dynamics}
\acro{ppo}[PPO]{Proximal Policy Optimization}
\acro{fft}[FFT]{Fast Fourier Transform}
\acro{pca}[PCA]{Principal Component Analysis}
\acro{dfm}[DFM]{Deep Fourier Mimic}
\acro{dof}[DoF]{Degrees of Freedom}
\acro{mlp}[MLPs]{Multi-Layer Perceptrons}
\end{acronym}



\newcommand{\figref}[1]{Figure~\ref{#1}}

% Colour for revisions and stuff
\newcommand{\blue}[1]{\textcolor{blue}{#1}}
\newcommand{\bblue}[1]{\textcolor{blue}{\textbf{#1}}}

\definecolor{britishracinggreen}{rgb}{0.0, 0.26, 0.15}
\newcommand{\green}[1]{\textcolor{britishracinggreen}{#1}}
\newcommand{\bgreen}[1]{\textcolor{britishracinggreen}{\textbf{#1}}}

\newcommand{\fabio}[1]{\textcolor{orange}{#1}}


\def\BibTeX{{\rm B\kern-.05em{\sc i\kern-.025em b}\kern-.08em
    T\kern-.1667em\lower.7ex\hbox{E}\kern-.125emX}}
\AtBeginDocument{\definecolor{tmlcncolor}{cmyk}{0.93,0.59,0.15,0.02}\definecolor{NavyBlue}{RGB}{0,86,125}}




\def\OJlogo{\vspace{-4pt}$<$Society logo(s) and publication title will appear here.$>$}
\def\seclogo{\vspace{10pt}$<$Society logo(s) and publication title will appear here.$>$}

\def\authorrefmark#1{\ensuremath{^{\textbf{#1}}}}

\begin{document}
\receiveddate{XX Month, XXXX}
\reviseddate{XX Month, XXXX}
\accepteddate{XX Month, XXXX}
\publisheddate{XX Month, XXXX}
\currentdate{XX Month, XXXX}
\doiinfo{XXXX.2022.1234567}

\markboth{}{Fabio A. Ruetz. {et al.}}

% \title{Human-assisted Adaptive Online Traversability Estimation}
\title{Online Adaptive Traversability Estimation through Interaction for Unstructured, Densely Vegetated Environments}

\author{
Fabio A. Ruetz\authorrefmark{1,2}, 
Nicholas Lawrance\authorrefmark{2},
Emili Hern\'andez\authorrefmark{3},
Paulo V. K. Borges\authorrefmark{4},
and Thierry Peynot\authorrefmark{1}}
\affil{QUT Centre for Robotics, Queensland University of Technology (QUT), Brisbane Qld 4000, Australia}
\affil{CSIRO Robotics, Data61, Pullenvale, Qld 4069, Australia.}
\affil{Emesent, Milton, Qld 4064, Australia}
\affil{Orica, Windsor, Qld 4030, Australia}
\corresp{Corresponding author: Fabio A. Ruetz (email: fabio.ruetz@gmail.com).}
\authornote{This work was supported by QUT, CSIRO, Emesent and the SILVANUS Project through European Commission Funding on the Horizon 2020 call number H2020-LC-GD-2020, Grant Agreement number 101037247. F.R. and T.P. acknowledge continued support from the Queensland University of Technology (QUT) through the Centre for Robotics.}



\begin{abstract}
Navigating densely vegetated environments poses significant challenges for autonomous ground vehicles. Learning-based systems typically use prior and in-situ data to predict terrain traversability but often degrade in performance when encountering out-of-distribution elements caused by rapid environmental changes or novel conditions. This paper presents a novel, lidar-only, online adaptive traversability estimation (TE) method that trains a model directly on the robot using self-supervised data collected through robot-environment interaction. The proposed approach utilises a probabilistic 3D voxel representation to integrate lidar measurements and robot experience, creating a salient environmental model. To ensure computational efficiency, a sparse graph-based representation is employed to update temporarily evolving voxel distributions. Extensive experiments with an unmanned ground vehicle in natural terrain demonstrate that the system adapts to complex environments with as little as 8 minutes of operational data, achieving a Matthews Correlation Coefficient (MCC) score of 0.63 and enabling safe navigation in densely vegetated environments. This work examines different training strategies for voxel-based TE methods and offers recommendations for training strategies to improve adaptability. The proposed method is validated on a robotic platform with limited computational resources (25W GPU), achieving accuracy comparable to offline-trained models while maintaining reliable performance across varied environments.
\end{abstract}

\begin{IEEEkeywords}
Traversability Estimation, Navigation in Unstructured Environments, Autonomous Navigation, Field Robotics, Online Adaptation, Deep Learning.
\end{IEEEkeywords}

%\IEEEspecialpapernotice{(Invited Paper)}

\maketitle

% MAIN TEXT WORK
\begin{figure}[ht]
    \centering
    \includegraphics[width=0.8\linewidth]{graphs/greater_than_naive.pdf}
    \vspace{0.5cm}
    \includegraphics[width=0.8\linewidth]{graphs/p1_bottom.png}
    \vspace{-5pt}
    \caption{\textcolor{positional}{Positional} vs.\ \textcolor{nonpositional}{non-positional} circuits. In a \textcolor{nonpositional}{non-positional} circuit, the same edges must be included at all positions. A \textcolor{positional}{positional} circuit can distinguish between the same edge at different positions. This specificity yields better trade-offs between circuit size and faithfulness. It can also increase both precision and recall.}
    \label{fig:p1}
    \vspace{-5pt}
\end{figure}

\section{Introduction}

\looseness=-1
A primary goal of interpretability research is to characterize the internal mechanisms in language models (LMs) and other NLP models. 
A core approach in this area is \textbf{circuit discovery}---identifying the minimal subgraph within the model's computation graph that performs a specific task \citep{olah2021framework,olah-mech}.
Typically, the nodes of a circuit represent model components (e.g., attention heads, neurons, or layers).
While manual circuit discovery methods can yield position-specific insights \citep{wanginterpretability,goldowskydill2023localizingmodelbehaviorpath}, \emph{automatic methods often overlook positional information}, treating components as uniformly relevant across all input token positions \citep{conmytowards,syed2023attribution}. 
For instance, if an attention head is included in a circuit, it is assumed to contribute equally to the computation for every position in the input sequence.
The assumption that circuits are position-invariant ignores the fact that different positions often require distinct computations.
By ignoring positions, current methods limit their ability to capture mechanisms that operate across positions, such as interactions between attention heads across positions.

In this study, we start by demonstrating that positional agnosticism is a significant limitation (\S\ref{sec:motivating}). Then, to address these limitations, we introduce a new approach: position-aware edge attribution patching (PEAP; \S\ref{sec:full_circ_discovery}; Figure~\ref{fig:p1}). Current approaches  assume that if an edge is in a circuit, then the same edge will be in the circuit at all positions, thus leading to low precision. It is also assumed that an edge's importance should be aggregated across positions before deciding whether it should be included in the circuit; this can lead to cancellation effects, and thus low recall. PEAP instead allows us to compute the importance of cross-positional edges, and separately evaluates edge importance at each position. We show that this leads to smaller and more accurate circuits; see Figure~\ref{fig:p1}.

Incorporating positional information into circuit discovery is straightforward when inputs have the same length and structure across examples.

However, realistic datasets are not nearly this templatic.
How, then, can we incorporate positional information into automatic circuit discovery?
To address this challenge, we propose \textbf{schemas} (\S\ref{sec:schema}). 
Schemas assign semantic labels to spans of tokens, enabling information aggregation across examples even when the spans differ in length.

For example, in the input ``The \textcolor{positional}{war} lasted from 1453 to 14\underline{\hspace{1em}},'' the span ``\textcolor{positional}{war}'' could be labeled as ``\emph{Subject}''.
This enables handling spans with varying lengths: the phrase ``\textcolor{positional}{Black Plague}'' in another example can be treated as a single positional span with the same role as ``\textcolor{positional}{war}''.
In experiments with two LMs and three tasks, we find that circuits discovered using schemas achieve a better trade-off between circuit size and faithfulness to the model's behavior than position-agnostic circuits.
Importantly, position-aware circuits offer a more precise representation of the underlying mechanisms, providing a more concise foundation for mechanistic explanations.

We also present a fully automated pipeline for schema generation and application (\S\ref{sec:schema-generation}) using large language models (LLMs). 
We evaluate the quality of the generated schemas and their utility in discovering position-aware circuits (\S\ref{sec:schema-eval}).
Notably, circuits derived using automatically generated and applied schemas achieve comparable faithfulness scores to circuits discovered with human-designed and manually applied schemas.

We summarize our contributions as follows:
\begin{itemize}[noitemsep,leftmargin=*,topsep=1pt,parsep=1pt]
    \item Introduce a position-aware circuit discovery method, which obtains better faithfulness than position-agnostic discovery.  
    \item Introduce dataset schemas,  facilitating positional circuit discovery in more naturalistic settings. 
    \item Develop an automated schema generation and application pipeline with LLMs, yielding schemas that are comparable to manually-annotated ones.
\end{itemize}


\section{Related work}


Recent advances in single-image animatable head avatar generation can be categorized into mainly 2D-based and 3D-based approaches. 

\paragraph{\bf Image to 2D Animatable Avatar.}
2D-based methods, leveraging the power of convolutional neural networks (CNNs)~\cite{DBLP:conf/cvpr/KarrasLAHLA20,DBLP:conf/cvpr/IsolaZZE17,DBLP:conf/nips/GoodfellowPMXWOCB14}, often employ generative adversarial networks (GANs)~\cite{DBLP:conf/cvpr/StyleGAN} for direct image synthesis. Early approaches~\cite{DBLP:conf/cvpr/WangDYSW23,DBLP:conf/cvpr/BurkovPGL20,DBLP:conf/iccv/ZakharovSBL19} focus on injecting expression and pose features into the generator network, often utilizing architectures like U-Net or StyleGAN~\cite{DBLP:conf/cvpr/StyleGAN}.
Some other 2D methods~\cite{DBLP:journals/corr/abs-2407-03168,DBLP:conf/cvpr/ZhangQZZW0CW023,DBLP:conf/cvpr/HongZS022,DBLP:conf/mm/DrobyshevCKILZ22,DBLP:conf/cvpr/BurkovPGL20,DBLP:conf/nips/SiarohinLT0S19} represent expressions and poses as warping fields applied to the source image. 
Benefiting from advances in image and video diffusion networks, more recent 2D-based works~\cite{DBLP:journals/corr/abs-2410-07718,DBLP:journals/corr/abs-2406-08801,DBLP:conf/eccv/TianWZB24} get improved results with diffusion techniques. 
However, these methods still face challenges related to long generation times and significant computational resource demands. Audio-driven 2D control methods~\cite{DBLP:conf/cvpr/ZhangCWZSGSW23,DBLP:journals/corr/abs-2211-12368,DBLP:conf/iccv/GuoCLLBZ21} are easy to use but cannot explicitly control facial expressions and poses. 2D-based techniques often struggle with large pose or expression variations due to the lack of an explicit 3D structure, sometimes producing unrealistic distortions or identity changes. While some 2D methods~\cite{SadTalker,StyleHEAT,Pirenderer,DBLP:conf/cvpr/WangM021,MegaPortraits} incorporate 3D Morphable Models (3DMMs)~\cite{DBLP:conf/fgr/GerigMBELSV18,DBLP:journals/tog/LiBBL017,DBLP:conf/avss/PaysanKARV09,DBLP:conf/siggraph/BlanzV99} to mitigate these issues, they typically cannot achieve free-viewpoint rendering. 

\vspace{-0.1in}

\begin{figure*}[h]
    \centering
    \includegraphics[width=0.9\linewidth]{images/framework.pdf}
    \caption{\textbf{Overall Framework.} Our framework utilizes learnable query features attached to FLAME vertices to perform cross-attention with the extracted multi-level image features. The extracted features are then decoded to reconstruct the Gaussian avatar in the canonical space, which can be animated utilizing standard linear blend skinning (LBS) and corrective blendshapes as the FLAME model did and rendered in real-time on various platforms.}
    \label{fig:framework}
\end{figure*}

\paragraph{\bf Image to 3D Animatable Avatar.}
3D-aware methods offer improved geometric consistency and free-viewpoint rendering capabilities. Early 3D approaches~\cite{DBLP:conf/eccv/KhakhulinSLZ22,DBLP:conf/cvpr/XuYCWDJT20} utilize 3DMMs for head avatar reconstruction. With the advent of Neural Radiance Fields (NeRFs)~\cite{DBLP:conf/eccv/MildenhallSTBRN20}, many recent methods~\cite{DBLP:conf/siggraph/YuFZWYBCSWSW23,DBLP:conf/cvpr/MaZQLZ23,DBLP:conf/cvpr/LiZWZ0CZWB023,GPAvatar,ye2024real3d,deng2024portrait4d,deng2024portrait4d2,DBLP:conf/eccv/KiMC24,DBLP:conf/cvpr/BaiFWZSYS23,PointAvatar,Nerfies,INSTA} have adopted this representation for higher fidelity, particularly in modeling fine details like hair. However, NeRF-based~\cite{DBLP:conf/cvpr/ZhangZLHLWGCL024,HAvatar,DBLP:conf/cvpr/BaiTHSTQMDDOPTB23,AD-NeRF,DBLP:journals/tog/GaoZXHGZ22,DBLP:journals/tog/ParkSHBBGMS21,DBLP:conf/cvpr/AtharXSSS22,DBLP:journals/corr/abs-2112-05637,DBLP:conf/iccv/TretschkTGZLT21,DBLP:conf/cvpr/GafniTZN21,DBLP:conf/eccv/KiMC24,DBLP:conf/cvpr/BaiFWZSYS23,PointAvatar,Nerfies,DBLP:conf/siggraph/YuFZWYBCSWSW23,DBLP:conf/cvpr/MaZQLZ23,DBLP:conf/cvpr/LiZWZ0CZWB023} approaches often require extensive training data, including multi-view or single-view videos, raising privacy concerns and limiting generalization to unseen identities. Some methods~\cite{DBLP:conf/cvpr/SunWWLZZL23,DBLP:conf/3dim/ZhuangMKS22,DBLP:journals/pami/SunWZHWL24,DBLP:journals/tvcg/TangZYZCMW24,DBLP:conf/iclr/XuZLZBFS23} bypass this data requirement by training generators with random noise and then inverting them for identity-specific reconstruction, but inversion accuracy remains a challenge. Test-time optimization offers another alternative, but its computational cost limits practical applications. Several recent works~\cite{goha2023,hidenerf2023,gpavatar2024,ye2024real3d,ma2024cvthead,deng2024portrait4d,deng2024portrait4d2,GGHead} have explored one-shot 3D head reconstruction to address the limitations of data requirements and computational cost. These methods employ various techniques, such as tri-plane features, deformation fields, point-based expression fields, and vertex-feature transformers. Despite these advancements, NeRF-based methods often struggle with real-time rendering. 
Recently, 3D Gaussian Splatting~\cite{GaussianSplatting} has emerged as a promising alternative, offering both high-quality results and fast rendering speeds. However, existing Gaussian Splatting methods~\cite{GaussianAvatar,DBLP:conf/cvpr/XuCL00ZL24} typically rely on video data for training for each person, limiting their ability to generalize to new identities. Instead, the most recent work, GAGAvatar~\cite{GAGAvatar}, proposes a one-shot 3D Gaussian-based head avatar generation method. However, it still relies heavily on complex 2D neural post-processing to achieve optimal animation outcomes, thus it is not a pure 3D solution and the extra neural network hinders its application on various platforms. In contrast, our work generates Gaussian heads that are immediately animatable and renderable without additional networks or post-processing steps, enabling seamless integration into existing rendering pipelines for real-time animation and rendering across a wide range of platforms, including mobile phones. 
\section{Methodology}
 

\begin{figure*}[t]
    \centering
    \scalebox{0.9}{
    % Define a style for the special labels (for 1, 2, 3)
\begin{tikzpicture}[
    >=Stealth,     % arrow tips
    thick,         % line thickness
    node distance=2.5cm
]

%--- User (person) on the left ---
\node[
  person,
  minimum size=1.25cm,
  label=below:User
] (user) {};

%--- Rectangle in the middle: "1" inside, "operations" below ---
\node[
  draw,
  rectangle,
  minimum width=2.0cm,
  minimum height=1.2cm,
  label=below:operations,
  right=of user
] (box) {\textcircled{1}};

%--- Cylinder (DB) on the right ---
\node[
  draw,
  cylinder,
  shape border rotate=90,
  aspect=1.5,
  minimum height=1.5cm,
  minimum width=1cm,
  label=below:DB,
  right=of box
] (db) {};

%--- Bidirectional arrows with special labels ---
% Arrows between user and box labeled "3"
\draw[->] (user) to[bend left=10] node[above]{\textcircled{3}} (box);
\draw[->] (box)  to[bend left=10]  (user);

% Arrows between box and DB labeled "2"
\draw[->] (box)  to[bend left=10] node[above]{\textcircled{2}} (db);
\draw[->] (db)   to[bend left=10]  (box);

\end{tikzpicture}}
    \caption{An overview of {\sc Demotic} is shown. {\sc Demotic} takes a Verilog instance describing a combinational circuit and parse it into its corresponding probabilistic model described in PyTorch. The embedding layer converts the learnable real-value inputs into probabilities. The $\ell_2$-loss function is calculated in each training iteration and the input variables are updated using GD.}
    \label{fig1}
\end{figure*}


In this section, we describe our differentiable solver/sampler for multi-level digital circuits. While the common approach in solving CircuitSAT typically involves converting the underlying circuit into CNF and employing a SAT solver to find the satisfying solution, we take a completely different approach. Instead, we re-frame the CircuitSAT problem as a multi-output regression task, transforming it into a learning problem. Digital circuits are inherently discrete and non-differentiable. Therefore, we first need to relax the CircuitSAT problem into a continuous form while accurately capturing the structure and behavior of the circuit. To accomplish this, we leverage the probability model of digital gates, as shown in Table \ref{tab1}. This probability model is commonly used in different domains such as stochastic computing \cite{Ardakani2017SC} and dynamic power estimation of digital circuits \cite{harris2010cmos}. We use these probabilities to model each gate in the circuit. The result of such modeling is a differentiable formulation of the underlying circuit that accurately describes its functionality while preserving its spatial structure. Of course, the outcome of this model remains identical to the original circuit in its discrete form for any binary input valuations.




\begin{table}[t]
    \centering
    \caption{Probability model of logic gates.}
    \vspace{-0.25cm}
    \begin{table*}[!th]
\centering
\resizebox{\textwidth}{!}{%
\begin{tabular}{@{}llcccccccccc@{}}
\toprule
& & \multicolumn{2}{c}{\textbf{Intent Detection}} & \multicolumn{2}{c}{\textbf{Topic Mining}} & \multicolumn{2}{c}{\textbf{Domain Discovery}} & \multicolumn{1}{c}{\textbf{Type}} & \multicolumn{1}{c}{\textbf{Emotion}} & \\
\cmidrule(lr){3-4} \cmidrule(lr){5-6} \cmidrule(lr){7-8} \cmidrule(lr){9-9} \cmidrule(lr){10-10}  %\cmidrule(lr){11-11}
\textbf{Model} & \textbf{Method} & \textbf{BANKING} & \textbf{CLINC} & \textbf{Reddit} & \textbf{StackEx} & \textbf{MTOP} & \textbf{CLINC(D)} & \textbf{FewEvent} & \textbf{GoEmotion} & \textbf{AVG} \\ \midrule \midrule
GPT-4o-mini & Standard Prompting & 0.652 & 0.792 & 0.534 & 0.482 & 0.896 & 0.536 & 0.630 & 0.378 & 0.613 \\
& Self-Consistency & 0.666 & 0.802 & 0.586 & 0.494 & 0.902 & 0.530 & 0.640 & 0.382 & 0.625 \\
& TestNUC & 0.712 & 0.858 & 0.614 & 0.528 & 0.936 & 0.544 & 0.674 & 0.410 & 0.660 \\
& \cellcolor{gray!18}TestNUC\textdagger & \cellcolor{gray!18}\textbf{0.764} & \cellcolor{gray!18}\textbf{0.864} & \cellcolor{gray!18}\textbf{0.646} & \cellcolor{gray!18}\textbf{0.540} & \cellcolor{gray!18}\textbf{0.948} & \cellcolor{gray!18}\textbf{0.554} & \cellcolor{gray!18}\textbf{0.680} & \cellcolor{gray!18}\textbf{0.414} & \cellcolor{gray!18}\textbf{0.676} \\ \midrule \midrule
Llama-3.1-8B & Standard Prompting & 0.572 & 0.726 & 0.502 & 0.492 & 0.892 & 0.528 & 0.530 & 0.332 & 0.572 \\
& Self-Consistency & 0.620 & 0.774 & 0.564 & 0.526 & 0.902 & 0.518 & 0.564 & 0.340 & 0.601 \\
& TestNUC & 0.694 & 0.806 & 0.618 & 0.558 & 0.934 & 0.528 & 0.596 & 0.356 & 0.636 \\
& \cellcolor{gray!18}TestNUC\textdagger & \cellcolor{gray!18}\textbf{0.724} & \cellcolor{gray!18}\textbf{0.812} & \cellcolor{gray!18}\textbf{0.646} & \cellcolor{gray!18}\textbf{0.576} & \cellcolor{gray!18}\textbf{0.940} & \cellcolor{gray!18}\textbf{0.542} & \cellcolor{gray!18}\textbf{0.614} & \cellcolor{gray!18}\textbf{0.360} & \cellcolor{gray!18}\textbf{0.652} \\ \midrule \midrule
Claude-3-Haiku & Standard Prompting & 0.680 & 0.848 & 0.486 & 0.564 & 0.892 & 0.552 & 0.594 & 0.336 & 0.619 \\
& Self-Consistency & 0.702 & 0.870 & 0.510 & 0.578 & 0.904 & 0.564 & 0.568 & 0.350 & 0.631 \\
& TestNUC & 0.762 & 0.894 & 0.596 & 0.588 & 0.940 & 0.590 & 0.620 & 0.348 & 0.667 \\
& \cellcolor{gray!18}TestNUC\textdagger & \cellcolor{gray!18}\textbf{0.804} & \cellcolor{gray!18}\textbf{0.902} & \cellcolor{gray!18}\textbf{0.612} & \cellcolor{gray!18}\textbf{0.600} & \cellcolor{gray!18}\textbf{0.946} & \cellcolor{gray!18}\textbf{0.622} & \cellcolor{gray!18}\textbf{0.660} & \cellcolor{gray!18}\textbf{0.368} & \cellcolor{gray!18}\textbf{0.689} \\ \midrule \midrule
GPT-4o & Standard Prompting & 0.746 & 0.924 & 0.712 & 0.674 & 0.962 & 0.614 & 0.682 & 0.406 & 0.715 \\
& Self-Consistency & 0.758 & 0.922 & 0.720 & 0.688 & 0.958 & 0.624 & 0.696 & 0.426 & 0.724 \\
&TestNUC & 0.804 & 0.934 & 0.744 & \textbf{0.710} & 0.974 & 0.644 & 0.692 & 0.446 & 0.744 \\
& \cellcolor{gray!18}TestNUC\textdagger & \cellcolor{gray!18}\textbf{0.824} & \cellcolor{gray!18}\textbf{0.940} & \cellcolor{gray!18}\textbf{0.750} & \cellcolor{gray!18}\textbf{0.710} & \cellcolor{gray!18}\textbf{0.978} & \cellcolor{gray!18}\textbf{0.654} & \cellcolor{gray!18}\textbf{0.708} & \cellcolor{gray!18}\textbf{0.464} & \cellcolor{gray!18}\textbf{0.754} \\
\bottomrule
\end{tabular}%
}
\caption{Accuracy comparison with Standard Prompting and Self-Consistency across four diverse LLMs. TestNUC consistently improves the inference performance on all benchmark datasets. $\dagger$ denotes that 50 neighbors are utilized.}
\label{tab:main_compare_sc}
\end{table*}
    \label{tab1}
    \vspace{-0.25cm}
\end{table}






Given the differentiable model of the circuit obtained by replacing its discrete logic gates with their corresponding probability model, our objective now is to generate a set of inputs that satisfy a desired constraint. This constraint could pertain to any desired valuation of intermediate signals or outputs. To generate satisfying solutions to the CircuitSAT problem, we represent the input variables to the circuit as $\textbf{V} \in \mathbb{R}^{b\times n}$, where $n$ represents the number of variables and $b$ denotes the batch size. We define the matrix $\textbf{V}$ as the parameters of an embedding layer in our circuit model, which will be updated during the learning process. It is worth mentioning that the number of variables in our sampling method is significantly fewer than that of SAT samplers, remaining the same as the number of inputs in the circuit. This discrepancy arises because SAT samplers deal with the CNF of the circuit, where each gate or component introduces additional variables. The embedding layer converts the real-value input variables of the circuit into probabilities in the range from $0$ to $1$ using the sigmoid function $\sigma(\cdot)$, expressed as:
\begin{equation}
    \textbf{P} = \sigma(\textbf{V}) = \dfrac{1}{1 + e^{-\textbf{V}}},
\end{equation}
where $\textbf{P} \in [0, 1]^{b\times n}$ represents the input probabilities to the underlying circuit. The circuit functionality is then computed as:
\begin{equation}
    \textbf{Y} = \mathcal{F}(\textbf{P}),
\end{equation}
where $\mathcal{F}:[0, 1]^{b \times n} \rightarrow [0, 1]^{b \times m}$ denotes the probabilistic model of the circuit. The matrix $\textbf{Y} \in [0, 1]^{b \times m}$ denotes the $m$ outputs across $b$ data batches. The $\ell_2$-loss function $\mathcal{L}$ can be constructed by measuring the distance between $\textbf{Y}$ and the target output valuation matrix $\textbf{T} \in \{0, 1\}^{b \times m}$ as follows:
\begin{equation}
    \mathcal{L} = \sum_{b,m} \left|\left| \textbf{Y} - \textbf{T} \right|\right|^2_2.
\end{equation}
The above loss function can be minimized, and the input variables (i.e., $\textbf{V}$) can be updated using GD in an iterative manner. Upon convergence, the $b$ solutions to the CircuitSAT problem are obtained by converting the soft input values (i.e., $\textbf{V}$) into hard values (i.e., $\widetilde{\textbf{V}} \in \{0, 1\}^{b\times n}$).

Fig. \ref{fig1} illustrates the overview of {\sc Demotic}. {\sc Demotic} is equipped with a parser to covert the circuit described in either bit-blasted Verilog or Berkeley Logic Interchange Format (BLIF) into its corresponding probabilistic model. Consequently, {\sc Demotic} can describe combinational circuits and generate satisfying solutions for any arbitrary constraint on the circuit. Such a sampling paradigm can also benefit from GPU acceleration due to the parallel independent computations across the data batches, enabling a high-throughput sampling procedure. 

To better understand our methodology, let us consider a quantitative example using the module ``c$15$'' shown in Fig. \ref{fig1}. We set the output node $G19$ to $1$ as an output constraint, while the output node $G22$ can take any value of either $0$ or $1$. Therefore, the goal in this example is to find a pair of inputs such that the output node $G19$ is equal to $1$. In this example, the input nodes contributing to our output constraint are $G3$, $G6$, and $G7$. These inputs are learned iteratively using gradient descent. The remaining input nodes, $G1$, $G2$, and $G3$, will not be updated and can take any arbitrary binary values. During each training iteration, each input node is updated by computing the derivative of the loss function with respect to each input node.

To illustrate the process, we generate two samples. In the first step, we randomly assign two values to each input node as follows:
\begin{equation}
    \textbf{v}_{G3} = \begin{bmatrix}
           0.1 \\
           -0.2 
         \end{bmatrix}, \textbf{v}_{G6} = \begin{bmatrix}
           0.5 \\
           -0.4 
         \end{bmatrix}, \textbf{v}_{G7} = \begin{bmatrix}
           -0.7 \\
           -0.8 
         \end{bmatrix},
\end{equation}
where the concatenation of the above vectors forms the matrix $\textbf{V}$. Next, the input probabilities to the circuit are calculated using the sigmoid function:
\begin{equation}
    \textbf{p}_{G3} = \begin{bmatrix}
           0.5250 \\
           0.4502
         \end{bmatrix}, \textbf{p}_{G6} = \begin{bmatrix}
           0.6225 \\
           0.4013
         \end{bmatrix}, \textbf{p}_{G7} = \begin{bmatrix}
           0.3318 \\
           0.3100
         \end{bmatrix}.
\end{equation}
Using the probability model of each gate shown in Table \ref{tab1}, the probabilities of the intermediate node $G11$ and the output node $G19$ are calculated as follows:
\begin{equation}
    \textbf{p}_{G11} = \begin{bmatrix}
           0.4939 \\
           0.4902
         \end{bmatrix}, \textbf{p}_{G19} = \begin{bmatrix}
           0.1639 \\
           0.1520
         \end{bmatrix}.
\end{equation}
Given the target value of 1 for the output node $G19$, the loss is calculated as:
\begin{equation}
    \mathcal{L} = (\textbf{p}_{G19} - 1)^2 = \begin{bmatrix}
           (0.1639 - 1)^2  \\
           (0.1520 - 1)^2 
         \end{bmatrix} = \begin{bmatrix}
           0.6991  \\
           0.7192 
         \end{bmatrix}.
\end{equation}


The above computations are commonly referred to as forward computations. To update the value of the input variables, we need to calculate the derivative of the loss with respect to each input variable, which is referred to as backward computations. This involves using the derivatives of each gate (as shown in Table \ref{tab1}) and applying the chain rule. The process is derived as follows:
\begin{align}
    \dfrac{\partial \mathcal{L}}{\partial \textbf{v}_{G3}} &= \dfrac{\partial \mathcal{L}}{\partial \textbf{p}_{G19}} \dfrac{\partial \textbf{p}_{G19}}{\partial \textbf{p}_{G11}} \dfrac{\partial \textbf{p}_{G11}} {\partial \textbf{p}_{G3}}
    \dfrac{\partial \textbf{p}_{G3}} {\partial \textbf{v}_{G3}} = 2\textbf{p}_{G19} \cdot \textbf{p}_{G7} \cdot (1 - 2\textbf{p}_{G6}) \nonumber \\ 
    &\cdot \sigma(\textbf{v}_{G3})\cdot (1 - \sigma(\textbf{v}_{G3})) = \begin{bmatrix}
           0.0339  \\
           -0.0257 
         \end{bmatrix}, \nonumber 
\end{align}
\begin{align}
    \dfrac{\partial \mathcal{L}}{\partial \textbf{v}_{G6}} &= \dfrac{\partial \mathcal{L}}{\partial \textbf{p}_{G19}} \dfrac{\partial \textbf{p}_{G19}}{\partial \textbf{p}_{G11}} \dfrac{\partial \textbf{p}_{G11}} {\partial \textbf{p}_{G6}}
    \dfrac{\partial \textbf{p}_{G6}} {\partial \textbf{v}_{G6}} = 2\textbf{p}_{G19} \cdot \textbf{p}_{G7} \cdot (1 - 2\textbf{p}_{G3})  \nonumber 
 \\ 
    & \cdot \sigma(\textbf{v}_{G6}) \cdot (1 - \sigma(\textbf{v}_{G6}))  = \begin{bmatrix}
           0.0065   \\
           -0.0126 
         \end{bmatrix}, \nonumber 
\end{align}
\begin{align}
    \dfrac{\partial \mathcal{L}}{\partial \textbf{v}_{G7}} &= \dfrac{\partial \mathcal{L}}{\partial \textbf{p}_{G19}} \dfrac{\partial \textbf{p}_{G19}}{\partial \textbf{p}_{G7}} \dfrac{\partial \textbf{p}_{G7}} {\partial \textbf{v}_{G7}} = 2\textbf{p}_{G19} \cdot \textbf{p}_{G11} \nonumber 
 \\ 
    & \cdot \sigma(\textbf{v}_{G7})\cdot (1 - \sigma(\textbf{v}_{G7})) = \begin{bmatrix}
           -0.1831   \\
           -0.1778 
         \end{bmatrix},
\end{align}
where ``$\cdot$'' denotes element-wise multiplication.


At this point, each variable is updated using the gradient descent update rule. This involves subtracting the derivative of the loss, scaled by the learning rate, from the corresponding input variables. Given a learning rate of $\gamma = 10$, the new values of the input variables at the end of this iteration are obtained as follows:
\begin{align}
    \textbf{v}_{G3} &= \textbf{v}_{G3} - \gamma \dfrac{\partial \mathcal{L}}{\partial \textbf{v}_{G3}} =  \begin{bmatrix}
           -0.2389 \\
           0.0569
         \end{bmatrix}, \textbf{v}_{G6} = \begin{bmatrix}
           0.4349 \\
           -0.2741
         \end{bmatrix}, \nonumber \\ \textbf{v}_{G7} & = \begin{bmatrix}
           1.1311 \\
           0.9783
         \end{bmatrix}.
\end{align}
This process can be repeated multiple times until convergence. However, even after one iteration in this specific example, we obtain two valid and distinct solutions by rounding the input variables to their nearest discrete values after applying the sigmoid function. In this example, the two input pairs of $(v_{G3} = -0.2389, v_{G6} = 0.4349, v_{G7} = 1.1311)$ and $(v_{G3} = 0.0569, v_{G6} = -0.2741, v_{G7} = 0.9783)$ are rounded to $(\widetilde{v}_{G3} = 0, \widetilde{v}_{G6} = 1, \widetilde{v}_{G7} = 1)$ and $(\widetilde{v}_{G3} = 1, \widetilde{v}_{G6} = 0, \widetilde{v}_{G7} = 1)$, respectively. As demonstrated through this example, the forward and backward computations of the two samples are independent of each other. This allows for the parallel execution of the learning process across multiple samples (i.e., batches), enabling GPU acceleration.


% \begin{figure*}[t]
%     \centering
%      \begin{subfigure}[b]{0.6\textwidth}
%          \centering
%          \scalebox{0.9}{\begin{tikzpicture}[auto, node distance=2cm,>=latex']
    % Combinational block 1
    \node [draw, fill = dateblue!30, shape=rectangle, minimum width=2.5cm, minimum height=1.5cm, text width=2cm, align = center, line width=1.5pt] (comb1) {Combinational Circuit ($\mathcal{F}_h$)};
    % Flip-Flop block
    \node [draw, fill = datemagenta!40, shape=rectangle, right = 1.25cm of comb1, align = center, minimum height=2cm, line width=1.5pt] (ff) {Flip-Flops};
    % Combinational block 2
    \node [draw, fill = dateblue!30,shape=rectangle, minimum width=2.5cm, minimum height=1.5cm, right =1.25cm of ff, text width=2cm, align = center, yshift=0.5cm, line width=1.5pt] (comb2) {Combinational Circuit ($\mathcal{F}_o$)};
    % Input nodes
    \node [left =1cm of comb1, coordinate] (input1) {};
    % Output node
    \node [right of=comb2, coordinate] (output) {};
    % Connection arrows
    \draw [->, >=stealth, line width=1.5pt] (input1) -- node[left, xshift = -0.5cm] {$\mathbf P_{t}$} (comb1);
    \draw [->, >=stealth, line width=1.5pt] (comb1) -- node {$\mathbf H_{t}$} (ff);
    \draw [->, >=stealth, line width=1.5pt] (comb1.east) -- node {} (ff);
    \draw [->, >=stealth, line width=1.5pt] (ff) -- node {$\mathbf H_{t-1}$} ([yshift=-0.5cm]comb2.west);
    \draw [->, >=stealth, line width=1.5pt] ([xshift=0.5cm]ff.east) |-  ([yshift=-1.5cm,xshift = -0.5cm]comb1.west) |-  ([yshift=-0.25cm, xshift = -0.15cm]comb1.west) |- ([yshift=-0.25cm]comb1.west);
    \draw [->, >=stealth, line width=1.5pt] (comb2) -- node[right, xshift = 0.5cm] {$\mathbf Y_t$} (output);
    \draw [->, >=stealth, line width=1.5pt, dashed] ([yshift=0cm, xshift = -0.5cm]comb1.west) |- ([yshift=1cm, xshift = -0.75cm]comb2.west) |- ([yshift=0.5cm, xshift =0cm]comb2.west);
\end{tikzpicture}}
%          \vspace{-0.5cm}
%          \caption{}
%          \label{fig2a}
%      \end{subfigure}
%      \hfill
%      \begin{subfigure}[b]{0.39\textwidth}
%          \centering
%          \scalebox{0.9}{\begin{tikzpicture}[scale=1, transform shape];
                \node [nnlayer]                     at ( 0,     0)    (sig1)          {$\mathcal{F}_o$};
                \node [nnlayer]                     at ( 1.5,   0)    (sig2)          {$\mathcal{F}_h$};

                \node [anchor=east]                 at ( -1,  -0.5) (hiddenlast)    {$\mathbf H_{t-1}$};
                \node [anchor=west]                 at ( 3.5,     1.) (hiddennext)    {$\mathbf H_{t}$};

                \node [anchor=west]                 at ( 0,     2.25) (hiddennext1)    {$\mathbf Y_{t}$};

                \node [anchor=east]                 at ( -1, -1.5)  (input)         {$\mathbf P_{t}$};

                \draw [ultra thick, rounded corners=0.2cm] (hiddenlast) -| (sig1);
                \draw [ultra thick, rounded corners=0.2cm] (hiddenlast) -| (sig2);


                \draw [->, >=stealth, ultra thick, rounded corners=0.2cm] (sig1.north) -- ([yshift = 1.5cm]sig1.north);;

                % \draw [->, >=stealth, ultra thick, rounded corners=0.2cm] (sig2.north) |- ([xshift=2cm, yshift=0.5cm]sig2.north) |- ( 6,  -0.5);

                \draw [->, >=stealth, ultra thick, rounded corners=0.2cm] (sig2.north) |- ([xshift=1.5cm, yshift=0.5cm]sig2.north);

                \draw [ultra thick, rounded corners=0.2cm] (input) -- ++(1.2,0) |- (0.5,-0.5);

                \begin{pgfonlayer}{background}
                \draw [fill=dateblue!10, rounded corners=.5cm] (-.5, -1) rectangle (3,1.5);
                \end{pgfonlayer}


            \end{tikzpicture}}
%          \vspace{-0.5cm}
%          \caption{}
%          \label{fig2b}
%      \end{subfigure}
%     \caption{The general form of a sequential circuit is shown in (a), and the recurrent cell for sequential circuits is depicted in (b).}
%     \label{fig2}
%     \vspace{-0.5cm}
% \end{figure*}



% \section{Sequential Circuits}
% So far, we have described how combinational circuits can be modeled and analyzed using {\sc Demotic}. In contrast to combinational circuits, where outputs are determined solely by their present inputs, the output in sequential circuits depends on both the past behavior of the circuit and the present values of inputs. The temporal operations of sequential circuits are controlled by a clock signal. The contents of memory elements (i.e., flip-flops) represent the past behavior of such a circuit, which is commonly referred to as the \textit{state} of the circuit. 


% Solving CircuitSAT problems for sequential circuits presents a unique challenge as it requires finding a sequence of inputs that satisfies the target constraint over a series of clock cycles. To tackle such problems, we can leverage a novel technique inspired by recurrent neural networks (RNNs). In RNNs, backpropagation through time is utilized during the learning process, allowing for updates to the network's hidden state at each time step. Similarly, in the context of solving CircuitSAT problems, we perform forward computations to iteratively update the state values at each clock cycle. During backward computations, gradients are backpropagated through time, extending back to the initial time step (i.e., the first clock cycle), to adjust the input sequence accordingly. While this approach draws parallels to RNN training, it is tailored to the unique challenges posed by solving CircuitSAT problems for sequential circuits. 



% Fig. \ref{fig2a} shows the general structure of a sequential circuit. We use this structure to formulate the CircuitSAT problem for sequential circuits to find satisfying solutions using {\sc Demotic}. In this structure, there are two combination circuits: one to update the state of the circuit (i.e., the content values of flip-flops) and the other one to generate the output. It is worth mentioning that both of these combinational circuits take the present values of flip-flops and primary inputs at the current time step as their inputs. Let us represent the primary input variables at time step $t$ as $\textbf{V}_t \in \mathbb{R}^{b\times n}$. We encode the primary input variables at each time step as learnable parameters to an embedding layer followed by the sigmoid function to provide input probabilities at time $t$ as $\textbf{P}_t \in [0, 1]^{p\times n}$ to the combinational circuits, i.e.,
% \begin{equation}
%     \textbf{P}_t = \sigma(\textbf{V}_t).
% \end{equation}
% The present output of the circuit (i.e., $\textbf{Y}_t \in [0, 1]^{b\times m}$) is computed as: 
% \begin{equation}
%     \textbf{Y}_{t} = \mathcal{F}_o(\textbf{P}_t, \textbf{H}_{t-1}),
% \end{equation}
% where $\mathcal{F}_o$ and $\textbf{H}_t \in [0, 1]^{b \times r}$ denote the functionality of the combinational circuit generating outputs and the present values of flip-flops at each time step, respectively. The number of flip-flops in the circuit is represented by $r$. The state of the circuit for the next time step is obtained as:
% \begin{equation}
%     \textbf{H}_{t} = \mathcal{F}_h(\textbf{P}_t, \textbf{H}_{t-1}),
% \end{equation}
% where the functionality of the combinational circuit updating the values of flip-flops is denoted by $\mathcal{F}_h$. The $\ell_2$-loss function $\mathcal{L}$ can then be constructed by measuring the distance between $\textbf{Y}_t$ at the desired time step $N$ and the target output valuation matrix $\textbf{T} \in \{0, 1\}^{b \times m}$ as follows:
% \begin{equation}
%     \mathcal{L} = \sum_{b,m} \left|\left| \textbf{T} - \textbf{Y}_N \right|\right|^2_2.
% \end{equation}
% With such a formulation for sequential circuits, {\sc Demotic} can solve the CircuitSAT problem and provide $b$ solutions. The general form of the recurrent cell for sequential circuits is shown in Fig. \ref{fig2b}, which is analogous to the RNN cell.


\section{EXPERIMENTS AND RESULTS}
\label{sec:experiments}
The experiments in this section evaluate the online adaptability of the proposed TE method using data collected during operation and in situ on the robotic platform. Further, we aim to understand the performance and limitations of different adaptation or training strategies depending on the data available, e.g. post-processed vs online data. 

The experiments are structured in four parts. The initial set of experiments characterises how well the proposed method performs on different fine-tuning tasks on post-processed data. This provides insights into how well an existing model can be fine-tuned as well as a baseline for the online adaptation.
%
The second set of experiments demonstrates this approach in a real-world scenario, where the robot learns a new, untrained model using only the experience of an \qty{8}{\minute} data collection. It aims to validate the method by showing that it enables the robot to navigate autonomously in a densely vegetated environment.
% 
The third set of experiments aims to train comparative methods and quantify their performance on the same data, avoiding any form of variation and stochastic elements, and compare three architecture variations and four different training setups.
%
The last set of experiments consists of a set of real-world experiments where differently trained models perform navigation experiments in varying environments, and their performance and limitations are assessed and compared. 

\subsection{Evaluation Metrics}
Throughout the evaluation, we use the Mathews Correlation Coefficient (MCC) score to evaluate and compare models. The MCC score measures how well the model predictions are correlated to the label data. It is immune to class swapping and is robust to imbalanced data sets. The MCC score ranges from $ -1$ to $1$, where 0 is a random model, and $1$ is a perfect positive correlation. It uses all four cases of the confusion matrix.  This makes it a preferred choice over the F1 score since the F1 score can overestimate a model's performance due to class imbalance, making it sensitive to class choice. This overestimation is particularly common for~\ac{TE}, where the positive class is typically the traversable class in the literature and is usually over-represented in training datasets. However, we also include the F1 score as a reference metric due to common practice in the literature.

\subsection{Data Sets and Data Set Groupings}
\label{subsec:data_set_groups}
The data set used in this experiment was originally published by Ruetz et al.~\cite{ruetz2024foresttrav} and consists of 9 different scenes (numbers \# 1 to 9). These data sets capture a variation of open fields with grass, areas with open skies and small trees, and forests with no closed skies. The data set covers a variety of scenarios and different natural scenes. 
In this paper, we include three additional industrial data sets, numbered \# 10 to 12. These data sets are included to be used in a clearly different environment to the previously collected natural scenes and to validate adaptation to a different environment. These were collected using the \ac{dtr} robot at an industrial site of CSIRO in Pullenvale, Queensland, Australia. An overview of the data set and scenes can be found in Figure~\ref{fig:data_set_environments}, where the blue dots mark the locations of the newly collected data sets.

\begin{figure}[ht]
    \centering
    \includegraphics[width=\columnwidth]{figures/ruetz4.png}
    \caption{Top left: Shows the overview of the environment with the orange dots showing the locations of data sets from ForestTrav~\cite{ruetz2024foresttrav}. The blue dots are the novel data set introduced in this work from an industrial environment. An example is shown in the top right. The bottom left shows an example of the SPARSE environment, the bottom right scene from a densely vegetated forest.}
    \label{fig:data_set_environments}
\end{figure}


\subsubsection{Industrial Data Set Characterisation}
\label{subsec:industrial_data_set}
The industrial set contains external spaces found at the CSIRO QCAT site at Pullenvale, QLD, a mixed industrial area containing office buildings, workshops and large mechanical testing facilities, including open and closed spaces, illustrated in Figure~\ref{fig:data_set_environments}. The data sets numbered 1  and 12 are used for the training set, and 13 is the hold-out test data set. The first industrial data set, Scene \# 10, is an open industrial area surrounded by large sheds. The area has different obstacles, such as metal posts, guard rails and barrels. The second industrial scene (\# 11) contains a two-lane road with a tall shed on each side. A pedestrian walkway on the side inclines and declines in this scene, providing a narrow, challenging path. Additionally, there are a lot of guard rails, posts and fences that could prevent the robot from moving. The third scene consists of the same road but larger open areas and has obstacles similar to those in the first two scenes. This scene was used as a test set. For all these scenes, a lot of the traversable and non-traversable examples are similar. The traversable examples consist of different patches of flat surfaces with different inclinations. Different paths were taken to collect different inclinations. The non-traversable elements consist of posts, barrels, guard rails, containers, machinery and the large overall shed structures. In general, these elements are not as varied as what can be found in the scenes with vegetation.

The following Table~\ref{tab:industiral_data_set} provides an overview of each scene. The first column is the scene number, and the second column is the total number of voxels. The following four columns (2-6) provide the percentages of the hand-labelled and \ac{lfe} data for each traversability class. The seventh column provides the mean column density of a scene, called the column vegetation density (CVD)~\cite{ruetz2024foresttrav}. The density is the column-wise fraction, up to a pre-determined height, above and excluding the ground voxel containing measurements. The height is defined by the platform in this work and is set to \qty{1}{\m}. This indicates how much ``vegetation'' or other elements need to be considered and pushed through to navigate safely. The ForestTrav data set~\cite{ruetz2024foresttrav} averages densities of 0.4 - 0.57 versus 0.13 in the industrial scene. The last column is the scene dimensions in meters. 
 \begin{table*}
\caption{Overview of the industrial data set.}
\label{tab:industiral_data_set}
    \resizebox{\textwidth}{!}{
    \begin{tabular}{lccccccc}
    \toprule
    Scene & $N_{voxels}$ & $TR$ HL [\%] & $TR$ LfE  [\%]& $NTR$ HL  [\%]& $NTR$ LfE  [\%] &  Density & Dimensions [m] \\ \hline
    \# 10 & 187933 & 0.71 & 0.06 & 0.15 & 0.08 & 0.1    & $53.1 \times  69.72 \times 1.45$ \\
    \# 11  & 76497 & 0.56 & 0.08 & 0.32 & 0.04 & 0.1    & $46.67 \times 39.39 \times  1.85$ \\
    \# 12  &  69534 & 0.48 & 0.05 & 42 & 0.03 & 0.13  & $ 66.14 \times  40.63 \times  1.86 $ 
    \end{tabular}
}
\end{table*}

\subsubsection{Data Set Groupings}
Adaptation to unfamiliar and varying environments was investigated during the experiments and evaluation. Hence, the data sets were separated into three distinct groups, or subsets, by environment type: the industrial data set (INDUST), the sparse forest data set (SPARSE) and the dense forest data set (DENSE). The sparse forest contains data sets \# 3, 4, 5, 7 and 8, with number 8 being used as the test set. The sparse data set contains open fields with grass and bushes, as well as smaller trees with no overarching canopy. The increased sun exposure means there is a large mixture of dense vegetation near the ground and a few very large trees. Most of the vegetation near the ground is chlorophyll-rich. The dense forest data set contains scenes \# 1, 2, 6 as training data and \# 9 as test data. The dense forest contains significantly more underbrush and vegetation that does not contain chlorophyll, thin three stems and trunks. Additionally, there are overhanging branches, brambles and vines that make the environment more difficult. The visual difference can be seen in Figure~\ref{fig:data_set_environments}, where \# 3, 4 and 7 have little to no obstruction of the sky and contain large and small trees with open space. Comparatively, the dense forests \# 1 and 2 in Figure~\ref{fig:data_set_environments} are cluttered and have a higher tree density.
%
Later on, we refer to the ``complexity'' of the data set due to the environment. We consider the industrial environment to be the least complex and the dense, the most complex environment. The SPARSE data sets falls in between. 


\subsubsection{Practical Considerations for LfE Data Collection}
\label{ch5:pratical_colmap}
As previously mentioned, the \ac{lfe} data collection was performed by an operator using an RC remote, recording the collision states. In practice, the operator ensured that the robot was driving (slowly) forward and collided only with the front of the robot with trees, bushes or other non-traversable elements. A collision was only recorded if the robot could not move physically further. Using a human operator induces a bias because the operator chooses how to collide with the elements in the environment, but this also helps to manage the risk of damaging the robot. The area was pre-determined, and the operator attempted to cover as much of it as possible in a single ``run'', revisiting places multiple times. During data collection, the operator collided with obstacles frequently and from many different directions. The more different, non-traversable collision states are captured, the higher the accuracy of the \ac{lfe} labelling approach. Note that mistakes during data collection are easy to make and that the overriding of the collision states can sometimes lead to erroneous self-labelling when using the heuristic method. 

\subsection{Model Adaptation on Post-processed Data}
\label{ch7:model_adpation_on_post_processed_data}
In the first set of experiments, a performance baseline was established for the proposed model and the different data sets due to the different labelling strategies. In the second sub-section, the adaptation or fine-tuning of post-processed data is explored. 

\subsubsection{Model Performance on Full Data Sets}
The initial set of experiments establishes baseline performance evaluation when using post-processed data from all the available data with different labelling strategies (LfE, LfE + HL). Table~\ref{tab:model_full_data_performance} shows the MCC scores and the F1 scores of the models. The first column specifies the labelling strategy, the second column is the MCC score, and the last column is the F1 score. The evaluation data set in all cases is scene \#9 with the \ac{lfe_hl} labelling strategy. The test set is the same as the one used in our prior work~\cite{ruetz2024foresttrav}. For consistency, all models are trained using the same number of epochs.

\begin{table}[h]
\centering
\caption{MCC Score for Models Trained on Post-Precessed Data}
\label{tab:model_full_data_performance}
\begin{tabular}{lcc}
\hline
Data Set &   MCC $ \mu \pm \sigma $ &  F1 $ \mu \pm \sigma$ \\
\hline    
LfE+HL    &     $0.71  \pm  0.03$    &       $0.85 \pm  0.017$   \\      
LfE        &    $0.69  \pm   0.022$     &       $0.85 \pm 0.012$   \\
\hline
\end{tabular}
\end{table}

Compared to our previous work~\cite{ruetz2024foresttrav}, there is an increase in \ac{mcc} score from 0.63 to 0.70~\ac{mcc} for the model trained on the LfE+HL data and the score obtained model trained on the \ac{lfe} data is only marginally lower.

\subsubsection{Fine-tuning on Post-Processed Data}
\label{ch7:finetuning_post_data}
In this experiment, we aim to evaluate the performance obtained when fine-tuning pre-trained models with data acquired in previously unseen environments. The first goal of this experiment is to confirm that models show different~\ac{TE} performance based on the different environment types due to the domain gap. The second goal is to explore whether fine-tuning models using data collected in a new target environment improves performance in the new environment. Given a fixed scaling and model architecture of a base model, we also aim to understand whether there is an upper bound for online adaptation, given a subset of hand-labelled and \ac{lfe} data for fine-tuning. In the context of online learning, this represents the case where a pre-trained base model is available, but the original training data is not. The base model can only be fine-tuned using novel data collected in the new environment.

\subsubsection{Base Models}
First, three base models (BM) are trained using data from one of the three groups defined previously only (industrial, sparse and dense forest), and each model is evaluated on test data of each of the three groups. The results are shown in Table~\ref{tab:offline_base}. The first column shows the base model names, and columns 2-6 show the MCC and F1 scores obtained with the trained model with respect to the test set of each of the groups. The MCC scores of models for which the training data and test set are from the same data set group are highlighted in italic text. The bold numbers indicate the best-performing models based on the MCC and F1 scores. 



\begin{table}[h!]
\centering
\caption{Performance for Models Trained on Post-Processed Data Groups. Bold indicates best performance, and italic indicates matching training and test data groups.}
\label{tab:offline_base}
\begin{tabular}{@{}lllllll@{}}
\hline
\multicolumn{1}{l}{Base Models} & \multicolumn{2}{l}{ INDUST TS} & \multicolumn{2}{l}{SPARSE TS} & \multicolumn{2}{l}{DENSE TS} \\ \midrule
  & \multicolumn{1}{c}{{\color[HTML]{656565} \textit{MCC}}} & \multicolumn{1}{c}{{\color[HTML]{656565} \textit{F1}}} & \multicolumn{1}{c}{{\color[HTML]{656565} \textit{MCC}}} & \multicolumn{1}{c}{{\color[HTML]{656565} \textit{F1}}} & \multicolumn{1}{c}{{\color[HTML]{656565} \textit{MCC}}} & \multicolumn{1}{c}{{\color[HTML]{656565} \textit{F1}}} \\ \midrule
BM: INDUST &  \textbf{\emph{0.70}}  & \textbf{\emph{0.84}}  & 0.15  & 0.40 & 0.05 &0.12\\
BM: SPARSE & 0.63  &  0.79  &  \textbf{\emph{0.79}}  & \textbf{\emph{0.89}} & 0.56  & 0.78 \\
BM: DENSE & 0.69   &  0.84 & \textbf{0.79} &  \textbf{0.90} & \textbf{\emph{0.70}}  &  \textbf{\emph{0.84}} \\ \bottomrule
\end{tabular}%
\end{table}

An initial set of observations can be made on the base model's performance based on which subset of the data it is trained on.
\begin{itemize}
 \item The model trained on the DENSE data set (BM:DENSE) exhibits a marginally lower \ac{mcc} and similar F1 score when compared to the model trained on the full data set, with 0.69 vs 0.71 \ac{mcc} scores. This indicates that an accurate traversability predictor can be learned with a small amount of high-quality data that discriminates the traversable and non-traversable elements of the environment.
\item Secondly, models trained and evaluated on data from the same group show high \ac{mcc} scores. For example, a model trained on the SPARSE data set shows high scores on the SPARSE test set. These are the bold, italicised numbers.
\item We note that BM:DENSE shows comparable performance to models trained and evaluated on the data of the same group (italic numbers), indicating that the model generalised well over different environments. For example, when evaluated on the INDUST test set, the MCC is 0.70 for BM:INDUST and 0.69 for BM:DENSE. The generalisation of the model trained in the DENSE environment to the INDUST environments is surprising. This can be further seen qualitatively in Figure~\ref{fig:bm_genralisiation} and discussed in the subsection.
\item Lastly, base models trained on the industrial and sparse data sets are significantly less accurate when evaluated on the dense test set. This suggests that the DENSE data sets contained many unseen or complex elements not encountered in other environments. 
\end{itemize}

In general, we note that models trained on data from more ``complex environments'' seem to perform competitively in less complex environments as well. 

\subsubsection{Fine-tuning Base Models on Post-processed Data}
Next, each base model was fine-tuned using data from the three data set groups to train a fine-tuned model. This model was again tested with test sets from each group. This resulted in six permutations of training and fine-tuning pairs. We used the \ac{lfe} training data as this best resembles what the robot experiences in the field; it can be generated without time costly hand-labelling. The results are shown in Table~\ref{tab:tab_offline_ft}

The first column identifies the base model, e.g. BM:INDUST, which is the base model trained on the industrial data set. The second column defines the data set used to adapt (fine-tune) the base model. For ease of notation, we use ``AM'' (adapted model) to indicate which data set the model was fine-tuned with. The model was initially trained on the industrial set and fine-tuned on the dense data set, which is thus denoted by (BM:INDUST AM:DENSE). The other columns show the model's performance using MCC and F1 scores of the model for the test sets of the three groups. The numbers in parentheses are the scores of the base model, allowing us to see increases and decreases in the performance of the fine-tuned model versus the base model. Similar to the previous table, italics indicate the performance number for which the test set and the fine-tuning training data belong to the same group, and bold numbers indicate the highest performance for a given test set (per column).

Conceptually, the initial three rows in Table~\ref{tab:tab_offline_ft} are the cases where the complexity of the new environment (in the fine-tuning data set) is higher than experienced in the base training, e.g. from an industrial environment to a dense environment. For ease of terminology, we call this ``fine-tuned on more complex data''. 
%
Similarly, the last three rows for each architecture are the cases where the base model is trained on a higher-complexity data set and fine-tuned with lower-complexity data, e.g., the base model trained on the dense data set and fine-tuned on the industrial data set. For ease of terminology, we call these ``fine-tuned on less complex data''. 

The training was limited to 150 epochs to make sure the results were comparable to the online case, thereby setting an upper performance bound later for the online adaptation models. Table~\ref{tab:tab_offline_ft} shows the results.

\begin{table*}[h]
\caption{Performance For Fine-tuning Models }
\label{tab:tab_offline_ft}
\resizebox{\textwidth}{!}{%
\begin{tabular}{@{}llcccccc@{}}
\toprule
Base Model & Finetune Train Set & \multicolumn{2}{c}{INDUST TS} & \multicolumn{2}{c}{SPARSE TS} & \multicolumn{2}{c}{DENSE TS} \\ \midrule
 &  & {\color[HTML]{656565} \textit{MCC}} & {\color[HTML]{656565} \textit{F1}} & {\color[HTML]{656565} \textit{MCC}} & {\color[HTML]{656565} \textit{F1}} & {\color[HTML]{656565} \textit{MCC}} & {\color[HTML]{656565} \textit{F1}} \\ \midrule
                            & SPARSE& 0.40 (0.70) & 0.59 (0.84)  &\emph{ 0.32} (0.15) & \emph{0.57} (0.40) & 0.47 (0.05) & 0.73 (0.12) \\
 \multirow{-2}{*}{BM:INDUST} & DENSE & 0.74 (0.70) & 0.86 (0.84)  & 0.62 (0.15) & 0.80 (0.40) & \emph{0.54} (0.05) &\emph{ 0.74} (0.12) \\ \midrule
                             & DENSE & 0.71  (0.63) & 0.84 (0.79) & \textbf{0.80} 0.79) &\textbf{ 0.90} (0.89) & \emph{0.59 } (0.56) & \emph{0.79} (0.78) \\
\multirow{-2}{*}{BM: SPARSE} &INDUST &\emph{ 0.67}  (0.63) &\emph{ 0.82} (0.79) & 0.43 (0.79) & 0.69 (0.89) & 0.41  (0.56) & 0.66 (0.78) \\ \midrule
                            & INDUST &\textbf{\emph{ 0.81}} (0.69) &\textbf{ \emph{0.90}} (0.84) & 0.63 (0.79) & 0.78 (0.90) & 0.36 (0.70) & 0.55 (0.84) \\
\multirow{-2}{*}{BM: DENSE} & SPARSE & 0.49  (0.69) &0.67 (0. 4) & \emph{0.75} (0.79) & \emph{0.87} (0.90) &\textbf{ 0.65} (0.70) &\textbf{ 0.82} (0.84) \\ \bottomrule
\end{tabular}%
}
\end{table*}


For the fine-tuned models, we can make the following comments:
\begin{itemize}
 \item The ``fine-tuned on more complex data'' models are considerably more accurate than the base model. For example, the case (BM:INDUST, AM: DENSE) shows an increase in the accuracy of the model from \ac{mcc}=0.12 to 0.54. Generally, we note an overall increase in performance for all fine-tuned models that are ``fine-tuned on more complex data''. Further, for the dense environment, the maximum performance is \ac{mcc}=0.7 for the model initially trained on the dense and adapted with the sparse data set.   
 \item In the case where the base models are fine-tuned on less complex environments, the fine-tuning improves the model performance on the specific target environment. This increase can be marginal but comes at the cost of reducing the performance of the model in other environments, which is commonly known as ``catastrophic forgetting''. For example, pre-training the base model on sparse data and fine-tuning with the industrial data set increases the \ac{mcc} score from 0.63 to 0.67 on the industrial test set but decreases it from 0.79 to 0.43 on the sparse test data.
\end{itemize}

In the last paragraph, we provide some qualitative examples. Figure~\ref{fig:bm_genralisiation} shows a comparison between a base model trained in the industrial setting (bottom row) and one trained in the dense environment (top row) for different test environments (columns). Both use fused data sets, i.e.~\ac{lfe_hl} and are evaluated on the same environments using replayed online data from unseen environments. The left column is an industrial scene, the middle is from the sparse data set, and the right scenes are from dense forests. The sparse forest contains no canopy coverage, smaller and bushier trees, and open areas with significant chlorophyll-rich vegetation close to the ground.  In comparison, the dense forest contains tall trees of different tree diameters and significant chlorophyll-poor vegetation ( bramble, thin or young treed, etc) close to the ground.

\begin{figure*}[h]
 \centering
 \includegraphics[width=\textwidth ]{figures/ruetz5.png}
 \caption[Generalisation of Models to Different Environments]{Comparison of two ensembles of models trained on either the dense data set in the top row (A, B, C) or the industrial data set in the bottom row (D, E, F). The left column shows each model's TE in an industrial environment, the middle and right columns show the TE of two densely vegetated environments with a variation of tree sizes and underbrush.}
 \label{fig:bm_genralisiation}
\end{figure*}

The model trained on the dense data set generalises well to an industrial outdoor setting (A), assessing walls and smaller elements such as pylons correctly. This is a surprising and unexpected result. Learning-based systems trained in one environment commonly do not generalise to wholly different environments. For natural environments, other image-based methods have shown high sensitivity and degradation to small spatial location changes with similar environments~\cite{frey2023fast}.  Additionally, this model provides accurate~\ac{TE} for two different vegetated environments (B \& C).

The model trained on the industrial data set demonstrates accurate \ac{TE} for the industrial test environment (D) with some noise on the ground plane. The industrial model performs poorly when predicting traversability for vegetation close to the ground in (E). In (F), the model correctly assesses many of the large elements (trees) but struggles with the smaller elements. However, there is a clearly visible colour gradient (dark blue to light green) on some of the smaller elements, showing discrimination of the \ac{TE} of the vegetation near the ground, clearly identifying tree trunks as non-traversable but struggling with bushes and smaller forms of vegetation. The model has learned a representation of the industrial setting, but it does not generalise to the densely vegetated environments. 
%
Comparatively, the model trained on the dense data set exhibits high performance over all three scenes, even the industrial one.

\subsection{Real-World Demonstration of Online Learning and Navigation in Forest}
\label{subsec:online_odap}

This experiment demonstrates that the proposed method can train a model on the robot itself, guided by a human operator in situ. The model was trained onboard the robot on an NVIDIA Jetson Orin NX (25W). A newly trained model was qualitatively demonstrated by point-to-point navigation and compared against the test set \#9.  The model was randomly initialised (untrained, new model), and it was continuously fine-tuned with incrementally acquired \ac{lfe} data. This corresponds to the case (BM 0, FT 1) described in sub-section~\ref{subsec:quant_online_adapt}. The training data collection was completed in less than 8 minutes and resulted in an \ac{mcc} score of 0.63, evaluated on the test scene \#9.

\begin{figure}[h]
 \centering
 \includegraphics[width= .9\linewidth]{figures/ruetz6.png}
 \caption[Overview of Robot Navigation of the Online Learned Model]{Overview of the navigation in a forest environment, with poses $\mathbf{p}_1$ - $\mathbf{p}_4$. The navigation was completed with the O-line trained model in situ.}
 \label{fig7:odap_nav_overview}
\end{figure}

The model was deployed on the robot and successfully navigated a closed-loop point-to-point trajectory between three waypoints, $w_1$, $w_2$, and the initial start location. A visualisation of the path can be seen in~\figref{fig7:odap_nav_overview}. The poses $p1$-$p4$ are visualisations of the trajectory, where the left image is the RGB FPV view of the robot, the middle image is the 3D TE estimation, and the right image is the costmap. Details on the costmap are provided in Section~\ref{subsec:cotmap_generation}.

\begin{figure*}[ht!]
 \centering
 % Pose 1
 \begin{subfigure}{0.95\linewidth}
 \centering
 \includegraphics[width=\linewidth]{figures/ruetz7.png}
 \caption{Robot front camera (left), 3D TE estimation (middle), costmap (right) for pose $\mathbf{p}_1$}
 \label{fig7:online_p1}
 \end{subfigure}
 \hfill
 % Pose 2t
 \begin{subfigure}{0.95\linewidth}
 \centering
 \includegraphics[width=\linewidth]{figures/ruetz8.png}
 \caption{Robot front camera (left), 3D TE estimation (middle), costmap (right) for pose $\mathbf{p}_2$}
 \label{fig7:online_p2}
 \end{subfigure}
 % Pose 3
 \begin{subfigure}{0.95\linewidth}
 \centering
 \includegraphics[width=\linewidth]{figures/ruetz9.png}
 \caption{Robot front camera (left), 3D TE estimation (middle), costmap (right) for pose $\mathbf{p}_3$}
 \label{fig7:online_p3}
 \end{subfigure}
 % Pose 4
 \begin{subfigure}{0.95\linewidth}
 \centering
 \includegraphics[width=\linewidth]{figures/ruetz10.png}
 \caption{Robot front camera (left), 3D TE estimation (middle), costmap (right) for pose $\mathbf{p}_4$}
 \label{fig7:online_p4}
 \end{subfigure}
 \caption[Scenes from Autonomous Robot Navigation using Online TE Approach]{Visualisation scenes from a successful online point-to-point navigation using the online learnt TE model in target environments, $\mathbf{p}_1$ - $\mathbf{p}_4$ are poses along the trajectory.}
 \label{fig7:odap_online_nav}
\end{figure*}

Subfigure~\ref{fig7:online_p1} shows the robot at the beginning of the trajectory briefly after the initialisation. The robot correctly assesses the trees and smaller vegetation elements, allowing it to push through the initial stand of bushes. There are some areas with uncertain (green) TE estimates, likely due to few observations; see comments for $\mathbf{p}_4$.

For $\mathbf{p}_2$, the robot encountered a large patch of grass that reached the camera. The proposed online learned TE correctly assessed it as traversable and pushed through it whilst avoiding the unobserved areas.

The pose $\mathbf{p}_3$ shows how the robot pushed through a thick bramble bush and proceeded to continue. There were some possibly non-traversable elements higher up in the bush. These elements were not projected onto the 2D costmap since they were above the robot's height threshold, and the robot would not interact with them (see Subsection~\ref{subsec:cotmap_generation}). In the background, it can be clearly observed that the trees were correctly assessed as non-traversable.

 At pose $\mathbf{p}_4$, the robot navigated through the bramble/small trees to reach the initial starting location, shown in Subfigure~\ref{fig7:online_p4}. The robot avoided the thicker tree to the right and reached the starting location. Compared to $\mathbf{p}_1$, the local costmap was less uncertain for the 3D TE estimate. The area has been observed multiple times and at multiple angles, and this allowed for more accurate traversability estimation than initially. In the visualisation of the costmap, one can observe a band on the left of traversable and unobserved elements that have been observed at the start of the experiment. This is an example of the de-allocation of the local probabilistic map since the regions were out of bounds, which is why the graph is needed.

In summary, the experiment demonstrated that a new model can be trained in situ, reaching accurate performance (\ac{mcc}$=0.63$), and is sufficient for safe point-to-point navigation in densely vegetated environments. Further, the model performance is comparable to the performance of the off-line method reported in our latest publication~\cite{ruetz2024foresttrav}.



\subsection{Quantification of Online Adaptation}
\label{subsec:quant_online_adapt}

The primary question this experiment addresses is whether the model can be adapted/trained online with only data gathered in a relevant environment. Secondly, it identifies the best strategy to do so. For this purpose, a data set was collected with human-assisted collisions. The online data set was replayed and stored in \qty{40}{\s} increments using the \ac{ograph} to generate snapshots of the online experience, which can be used to train. The intermediate data storing step was chosen to ensure data consistency.

We examine four training strategies. They are defined by the two boolean flags BM (use of a base model) and CA (continuous adaptation). The four variations of these strategies can be seen in \figref{fig:aol_incr_perf}. If the BM flag is set to true (BM 1), the method uses an initial base model trained on the industrial data set. Otherwise (BM 0), the model is randomly initialised. The industrial base model was chosen since the environment is the most different from the dense forest. For models trained without any prior knowledge, pre-determined scaling values were used to ensure non-catastrophic scaling during the run. The CA flag denotes if, for each training cycle, \ac{ograph} is updated, the model trained on the previous iteration is used for the next training cycle. If CA is true (CA 1), the model is continuously updated as new information comes in. If CA is false (CA 0), the model is trained from the base model at each ``training cycle'', as defined above.

This results in four distinct cases in \figref{fig:aol_incr_perf}:
\begin{enumerate}
 \item BM 0, CA 0 (red) -- randomly-initialised base model retrained at each training cycle,
 \item BM 0, CA 1 (green) -- randomly-initialised base model continuously adapted,
 \item BM 1, CA 1 (blue) -- pre-trained base model continuously adapted, and
 \item BM 1, CA 0 (gold) -- pre-trained base model retrained at each training cycle.
 \end{enumerate}
The dashed yellow line shows the base-model performance (no retraining).

The data comes from a newly collected data set with the goal of maximising collisions (the minority class) within a new area. The test data set remains scene \# 9 from the previously established data set, allowing for a comparison to all other experiments.

\begin{figure*}[ht!]
 \centering
 \includegraphics[width=\textwidth ]{figures/ruetz11.png}
 \caption{Comparison of the performance of different models over the incremental online adaptation for the four different cases. The $x$-axis shows the time, and the $y$-axis the \ac{mcc} score. The transparent regions of the graph correspond to the upper and lower bounds for one standard deviation.}
 \label{fig:aol_incr_perf}
\end{figure*}

% Description of what the experiment looked like
Looking at \figref{fig:aol_incr_perf}, the performance of the individual models reaches a ceiling between 0.55 and 0.65 \ac{mcc} score at their peak, the mean of the models just below 0.6~\ac{mcc} score. All models learn and improve substantially throughout online adaptation. There is a clear suggestion that a model can be either adapted or trained online with only data collected through experience and that this can be achieved in real time on a robotic platform in situ.
The variant (BM 1, CA 1) (blue graph) is generally the highest performing and most consistent (lowest variance) of all three models. The lower variance cannot solely be attributed to the learning rate being lower for continual adaptation, and we see higher variation for the green plots. The models with no continual learning (CA 0), red and yellow, show an oscillation effect, where the model's performance sometimes decreases over iterations. Further, higher variation of the method for models with randomly-initialised base models (BM 0), red and green, can be observed. 

\subsection{Navigation in Different Environments}
\label{subse:nav_in_different_env}
In this investigation, we comparatively assess navigation performance across multiple methodological approaches in two distinct forest environments. The comparative analysis includes NavStack~\cite{HudTal21}, a traditional (rigid world assumption) geometric lidar-based occupancy method, baseline ForestTrav trained on post-processed datasets~\cite{ruetz2024foresttrav}, ForestTrav DENSE and ForestTrav INDUST (trained on dense and industrial datasets, respectively), and the proposed Adaptive Online Learning (AOL) method. The experimental evaluation was conducted post-AOL generation, utilising a platform equipped with a different instance of a Velodyne VLP-16 lidar from previous dataset acquisition platforms, known to have shifts in intensity characteristics.

For two locations, a pre-defined starting and end point were given, and each method attempted to reach its goal. If a model got stuck for the first time, the operator would intervene by moving the robot forward and then continuing the experiment. For the first experiment, the robot was sent \qty{20}{\m} ahead but had to pass through a wall of small plants, navigate through an open space with sparse vegetation and finally navigate around a variation of high grass and small trees. 
For the second experiment, the robot had to navigate to a goal pose located \qty{40}{\m} from the starting position. A combination of cluttered, small trees with a variation of underbrush as well as fallen tree trunks had to be avoided.

In the first experiment, ForestTrav AOL and the original ForestTrav successfully reached the end goal without any intervention. ForestTrav initially struggled with bushes but managed to navigate around them, while ForestTrav AOL took a more direct route through the foliage. ForestTrav DENSE failed to bypass the bushes and required operator intervention to continue. NavStack also failed at the start, and after intervention, diverged from the goal due to avoiding small stems, ultimately failing.

In the second experiment, ForestTrav and ForestTrav AOL completed the course again without any human intervention. ForestTrav DENSE and NavStack made some initial progress through cluttered trees but became stuck due to thin stems and clutter, with intervention proving ineffective. ForestTrav INDUST failed entirely, unable to assess the environment.

ForestTrav and ForestTrav AOL were the most effective methods, differing primarily in their interpretation of traversable terrain. ForestTrav struggled with vegetation that was absent in its training data but prevalent in the target environment. ForestTrav AOL, trained in a similar environment, handled these obstacles better but was less robust overall due to limited training data and the lack of an ensemble approach. The poor performance of ForestTrav DENSE is attributed to the insufficient representation of thin, bushy vegetation in its training set.

ForestTrav INDUST consistently failed, likely due to changes in the intensity distribution characteristics of the Velodyne VLP-16 sensor. This was confirmed in follow-up tests, but the costmaps remained inadequate for navigation.

\begin{figure*}
     \centering
    \includegraphics[width=\textwidth ]{figures/ruetz12.png}
    \caption{Qualitative examples of the methods in the target environment. Blue blocks are for all methods at location 1 and show the starting area of the experiment, allowing us to compare the methods. The green block shows the costmap, FPV and external view of the robot for the location where NavStack failed to navigate. }
    \label{fig:nav_experiment}
\end{figure*}

\begin{table}[t]
\caption{Comparison performance of different methods for different navigation environments. }
\label{tab:nav_comp_exp}
\resizebox{\columnwidth}{!}{%
\begin{tabular}{@{}l|ccc|ccc@{}}
\toprule
Method & \multicolumn{3}{c|}{Location 1}  & \multicolumn{3}{c}{Location 2} \\ \midrule
 & \multicolumn{1}{r}{{\color[HTML]{656565} \textit{Success}}} & \multicolumn{1}{r}{{\color[HTML]{656565} \textit{Time [s]}}} & \multicolumn{1}{r|}{{\color[HTML]{656565} \textit{Distance [m]}}} & \multicolumn{1}{r}{{\color[HTML]{343434} \textit{Success}}} & \multicolumn{1}{r}{{\color[HTML]{343434} \textit{Time [s]}}} & \multicolumn{1}{r|}{{\color[HTML]{343434} \textit{Distance [m]}}} \\ \midrule
CSIRO NavStack & No & 240 & 30.7 & No & 170 & 13.9  \\
ForestTrav & Yes & 270 & 32.1 & Yes & 350 & 52.1  \\
ForestTrav AOL & Yes & 160 & 27.1 & Yes & 300 & 52.1\\
ForestTrav DENSE & No & 190 & 24.5 & No & 220 & 20.29  \\
ForestTrav INDUST & No & - & 0.0 & No & -  & 0.0 \\
\end{tabular}%
}
\end{table}


\section{Discussion}
\subsection{Case Study}


Fig. \ref{fig:casestudy} shows 2-D UMAP \cite{mcinnes2020umapuniformmanifoldapproximation} projections of embedding vectors for PetClinic Microservices \cite{microapps2024petclinic} using VoyageAI, ME-unixcoder-340K, and ME-llm2vec-340K. ME-unixcoder and ME-llm2vec show clearer microservice clusters compared to VoyageAI and Fig. \ref{fig:mexample}. For instance, \textit{API-Gateway} service classes are split in VoyageAI's representation but closer in the other models. ME-llm2vec demonstrates the closest grouping within microservices and clearest separation between them. In fact, ME-llm2vec's figure shows only 6 clear outliers which we review in detail and display their names and neighbors.



The two \textit{MetricConfig} classes, \textit{ResourceNotFoundException} and \textit{CacheConfig} lack domain-specific terms since they are utility classes, which highlights the importance of separating them from domain-related ones during the decomposition. However, ME-llm2vec was able correctly represent classes with even slight domain hints. For instance, most models struggle to differentiate between the nearly identical entry-point classes (e.g. \textit{ConfigServerApplication}), as seen in Fig. \ref{fig:mexample} and \ref{fig:casestudy} while ME-llm2vec managed to correctly place them within their services. On the other hand, the class \textit{PetRequest}, which was closer to \textit{API-Gateway} instead of \textit{Customers}, shows an intriguing outlier. Despite ME-llm2vec correctly matching the "Pet" related classes, it failed with \textit{PetRequest}. its function as a Request object, which is typically associated with the Gateway pattern, is a potential reason. Notably, ME-llm2vec successfully identified \textit{API-Gateway} classes, differentiating them from \textit{Customers}. We find this interesting because \textit{API-Gateway} includes classes representing various bounded contexts, often causing confusion in other models. ME-llm2vec recognized these classes' distinct purpose, grouping them together despite their diverse domains.

% Both \textit{API-Gateway} and \textit{Customers} services contain a "PetType" class. But in \textit{Customers}'s case, this class was closer to the "Specialty" class from \textit{Vets}, which is likely due to nearly identical source code they have. 

\subsection{Discussion}


We designed the analysis component to be as abstract as possible to accommodate the rapidly evolving representation learning landscape. As new and improved embedding models are published, they can be integrated with minimal effort. While our evaluation results show that with ME-LLM2Vec, we can generate highly cohesive and consistent decompositions, one of our objectives is to highlight the potential of Language Models in generating more efficient representations than traditional approaches for the decomposition problem. In fact, MonoEmbed is both a decomposition approach (when considering the full approach) and an embedding model (when using models such as ME-LLM2Vec). These models can be used to enrich existing decomposition approaches. For example, MicroMiner's CodeBERT \cite{trabelsi2023microminer} can be replaced with ME-LLM2Vec and the GNN based methods \cite{desai2021cogcn,yedida2023deeply,mathai2022chgnn,qian2023gdcdvf} can be extended by using ME-LLM2Vec as the encoder. In fact, it can be used as an additional representation type in approaches such as \cite{khaled2022hydecomp,qian2023gdcdvf}. These models can be even extended further by incorporating unstructured inputs (e.g. resources, configurations, documentation) and different PLs.




\subsection{Threats to Validity}
\subsubsection{Internal Validity}
Clustering algorithms and decomposition approaches have hyper-parameters that can affect performance on evaluation benchmarks. To mitigate this threat, we compared their performance with different hyper-parameter inputs across a varied set of evaluation applications.

\subsubsection{External Validity}
To address the threat of our approach to generalize on monolithic applications and PLs, we used a large set of monolithic and microservices applications from related work \cite{kalia2021mono2micro,khaled2022hydecomp,yedida2023deeply,jin2021fosci} to benchmark decomposition approaches. 

\subsubsection{Construct Validity}
This threat can potentially be in the form of the evaluation metrics used in our experiments. In order to mitigate this threat, we employ established metrics in supervised learning tasks (RQ1-3) and different metrics from decomposition research \cite{khaled2022hydecomp,kalia2021mono2micro,jin2021fosci,yedida2023deeply,mathai2022chgnn} (RQ4). 


% Refrences
\bibliographystyle{IEEEtran}      %other styles: apalike, plainnat,abbrvnat,named ...
\bibliography{References}       %References.bib

%%%%%%%%%%%%%% AUTHORS %%%%%%%%%%%%%%
\begin{IEEEbiography}[{\includegraphics[width=1in,height=1.25in,clip,keepaspectratio]{figures/authors/ruetz.png}}]{Fabio A. Ruetz } 
received his B.S and M.S degrees in Mechanical and Process Engineering from the Swiss Federal Institute of Technology Zurich (ETH), Zurich, Switzerland, in 2018. Since May 2020, he has been pursuing a Ph.D. at QUT Centre for Robotics, Queensland University of Technology (QUT), in collaboration with the robotic perception group at the Commonwealth Scientific and Industrial Research Organisation (CSIRO) and Emesent. His research and interest lie in probabilistic mapping, computer vision, machine learning, and path planning to enable autonomous ground vehicles to operate in challenging environments. 
\end{IEEEbiography}


\begin{IEEEbiography}[{\includegraphics[width=1in,height=1.25in,clip,keepaspectratio]{figures/authors/lawre1.jpeg}}]{Nicholas Lawrance }
 completed his PhD at the University of Sydney and worked as a postdoctoral scholar at Oregon State University, USA and ETH Zurich, Switzerland. He is currently a senior research scientist in robotic perception and autonomy at the Commonwealth Scientific and Industrial Research Organisation (CSIRO) in Australia. His research focuses on adaptive planning approaches for mobile robots, particularly in the presence of environmental uncertainty. Research interests include stochastic reasoning, adaptive sampling, and modelling of complex, uncertain phenomena. Applications include aerial, ground and underwater domains, particularly for long-duration robotic missions. Nick is a member of IEEE, an Associate Editor of IEEE Robotics and Automation Letters (RA-L), and a former Associate Editor for the International Conference on Robotics and Automation (ICRA).
\end{IEEEbiography}

\begin{IEEEbiography}[{\includegraphics[width=1in,height=1.25in,clip,keepaspectratio]{figures/authors/herna1.jpg}}] {Emili Hern\'andez } is an R\&D Manager with Emesent. He has two decades of experience on designing, developing and deploying novel software algorithms for underwater, ground and aerial robots. His current focus is on commercializing robotic autonomy research outcomes to improve and automate data capture in underground mining and asset inspection operations. He got his PhD at the University of Girona, Spain, and worked in several research positions at the CSIRO's Robotics and Autonomous Systems Group, Australia.
\end{IEEEbiography}

\begin{IEEEbiography}[{\includegraphics[width=1in,height=1.25in,clip,keepaspectratio]{figures/authors/borge1.jpg}}]{Paulo V. K. Borges }
    is the Head of AI R\&D at Orica. He has a Ph.D. in Electronic Engineering and Computer Science from Queen Mary, University of London (2007). Paulo has lived and worked in different countries (USA, Brazil, UK, Switzerland, and Australia), holding positions at Orica, CSIRO, NASA, ETH Zurich, University of London, Federal University of Santa Catarina, and  University of Manchester. He holds adjunct positions as an Adjunct Associate Professor at the University of Queensland as an Adjunct Scientist at the CSIRO Data61.  His core interest has been in autonomous robots and AI solutions for the mining, manufacturing, energy, environment and space industries, with close connections between industry and research. 
\end{IEEEbiography}

\begin{IEEEbiography}[{\includegraphics[width=1in,height=1.25in,clip,keepaspectratio]{figures/authors/peynot1.jpg}}]{Thierry Peynot }
    obtained his Ph.D. from the University of Toulouse and LAAS-CNRS in France. 
    He is Associate Professor in Robotics and Autonomous Systems at Queensland University of Technology (QUT) and a Chief Investigator of the QUT Centre for Robotics, where he leads the Mining Robotics and Space Robotics activities. Prior to joining QUT he was a researcher at the Australian Centre for Field Robotics (ACFR), The University of Sydney, worked at NASA Ames. Thierry has led multiple research programs funded by government, research institutions and industry, including mining (e.g. Caterpillar, Komatsu, Mining3), defence (e.g. BAE Systems, Rheinmetall) and space (e.g. with Boeing and CSIRO), developing robust perception technology for field robots and autonomous vehicles that can function despite adverse environmental conditions. 
    Thierry is a senior member of IEEE, immediate past Chair of the Robotics and Automation / Control Systems chapter, IEEE Queensland Section, and is a former Associate Editor of IEEE Robotics and Automation Letters (RA-L), the International Conference on Robotics and Automation (ICRA) and the International Conference on Intelligent Robots and Systems (IROS) . 
\end{IEEEbiography}

\vfill\pagebreak

\end{document}