%%%%%%%%%%%%%% AUTHORS %%%%%%%%%%%%%%
\begin{IEEEbiography}[{\includegraphics[width=1in,height=1.25in,clip,keepaspectratio]{figures/authors/ruetz.png}}]{Fabio A. Ruetz } 
received his B.S and M.S degrees in Mechanical and Process Engineering from the Swiss Federal Institute of Technology Zurich (ETH), Zurich, Switzerland, in 2018. Since May 2020, he has been pursuing a Ph.D. at QUT Centre for Robotics, Queensland University of Technology (QUT), in collaboration with the robotic perception group at the Commonwealth Scientific and Industrial Research Organisation (CSIRO) and Emesent. His research and interest lie in probabilistic mapping, computer vision, machine learning, and path planning to enable autonomous ground vehicles to operate in challenging environments. 
\end{IEEEbiography}


\begin{IEEEbiography}[{\includegraphics[width=1in,height=1.25in,clip,keepaspectratio]{figures/authors/lawre1.jpeg}}]{Nicholas Lawrance }
 completed his PhD at the University of Sydney and worked as a postdoctoral scholar at Oregon State University, USA and ETH Zurich, Switzerland. He is currently a senior research scientist in robotic perception and autonomy at the Commonwealth Scientific and Industrial Research Organisation (CSIRO) in Australia. His research focuses on adaptive planning approaches for mobile robots, particularly in the presence of environmental uncertainty. Research interests include stochastic reasoning, adaptive sampling, and modelling of complex, uncertain phenomena. Applications include aerial, ground and underwater domains, particularly for long-duration robotic missions. Nick is a member of IEEE, an Associate Editor of IEEE Robotics and Automation Letters (RA-L), and a former Associate Editor for the International Conference on Robotics and Automation (ICRA).
\end{IEEEbiography}

\begin{IEEEbiography}[{\includegraphics[width=1in,height=1.25in,clip,keepaspectratio]{figures/authors/herna1.jpg}}] {Emili Hern\'andez } is an R\&D Manager with Emesent. He has two decades of experience on designing, developing and deploying novel software algorithms for underwater, ground and aerial robots. His current focus is on commercializing robotic autonomy research outcomes to improve and automate data capture in underground mining and asset inspection operations. He got his PhD at the University of Girona, Spain, and worked in several research positions at the CSIRO's Robotics and Autonomous Systems Group, Australia.
\end{IEEEbiography}

\begin{IEEEbiography}[{\includegraphics[width=1in,height=1.25in,clip,keepaspectratio]{figures/authors/borge1.jpg}}]{Paulo V. K. Borges }
    is the Head of AI R\&D at Orica. He has a Ph.D. in Electronic Engineering and Computer Science from Queen Mary, University of London (2007). Paulo has lived and worked in different countries (USA, Brazil, UK, Switzerland, and Australia), holding positions at Orica, CSIRO, NASA, ETH Zurich, University of London, Federal University of Santa Catarina, and  University of Manchester. He holds adjunct positions as an Adjunct Associate Professor at the University of Queensland as an Adjunct Scientist at the CSIRO Data61.  His core interest has been in autonomous robots and AI solutions for the mining, manufacturing, energy, environment and space industries, with close connections between industry and research. 
\end{IEEEbiography}

\begin{IEEEbiography}[{\includegraphics[width=1in,height=1.25in,clip,keepaspectratio]{figures/authors/peynot1.jpg}}]{Thierry Peynot }
    obtained his Ph.D. from the University of Toulouse and LAAS-CNRS in France. 
    He is Associate Professor in Robotics and Autonomous Systems at Queensland University of Technology (QUT) and a Chief Investigator of the QUT Centre for Robotics, where he leads the Mining Robotics and Space Robotics activities. Prior to joining QUT he was a researcher at the Australian Centre for Field Robotics (ACFR), The University of Sydney, worked at NASA Ames. Thierry has led multiple research programs funded by government, research institutions and industry, including mining (e.g. Caterpillar, Komatsu, Mining3), defence (e.g. BAE Systems, Rheinmetall) and space (e.g. with Boeing and CSIRO), developing robust perception technology for field robots and autonomous vehicles that can function despite adverse environmental conditions. 
    Thierry is a senior member of IEEE, immediate past Chair of the Robotics and Automation / Control Systems chapter, IEEE Queensland Section, and is a former Associate Editor of IEEE Robotics and Automation Letters (RA-L), the International Conference on Robotics and Automation (ICRA) and the International Conference on Intelligent Robots and Systems (IROS) . 
\end{IEEEbiography}