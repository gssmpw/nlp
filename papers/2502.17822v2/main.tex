%% The first command in your LaTeX source must be the \documentclass command.
% \documentclass[sigconf, anonymous]{acmart}
\documentclass[sigconf]{acmart}
% extra add
\usepackage{multirow}
\usepackage{booktabs}
\usepackage{amsthm}
% \usepackage {verbatim}
\usepackage{graphicx}
\usepackage{amsmath}
\usepackage{color,xcolor}
\usepackage{hyperref}

\definecolor{mygreen}{RGB}{34,200,34}
\definecolor{myyellow}{RGB}{255,200,0}
\usepackage[linesnumbered,ruled,vlined]{algorithm2e}

%%
%% \BibTeX command to typeset BibTeX logo in the docs
\AtBeginDocument{%
  \providecommand\BibTeX{{%
    \normalfont B\kern-0.5em{\scshape i\kern-0.25em b}\kern-0.8em\TeX}}}

%% Rights management information.  This information is sent to you
%% when you complete the rights form.  These commands have SAMPLE
%% values in them; it is your responsibility as an author to replace
%% the commands and values with those provided to you when you
%% complete the rights form.



\setcopyright{acmcopyright}
\copyrightyear{2025}
\acmYear{2025}
\setcopyright{acmlicensed}\acmConference[ICMR '25]{Proceedings of the 2025 International Conference on Multimedia Retrieval}{June 30-July 3, 2025}{Chicago, USA}

% \acmBooktitle{Proceedings of the 2025 International Conference on Multimedia Retrieval (ICMR '25), June 30--July 3, 2025, Chicago, USA}
% \acmDOI{10.1145/3652583.3658038}
% \acmISBN{979-8-4007-0619-6/24/06}

% 通讯作者标识

% \settopmatter{printacmref=true}
\begin{document}

%%
%% The "title" command has an optional parameter,
%% allowing the author to define a "short title" to be used in page headers.
% \title{MSVD: Multimodal Semantics Summarization of Video Data}
\title{Easy-Poly: A Easy Polyhedral Framework For 3D Multi-Object Tracking}

%%
%% The "author" command and its associated commands are used to define
%% the authors and their affiliations.
%% Of note is the shared affiliation of the first two authors, and the
%% "authornote" and "authornotemark" commands
%% used to denote shared contribution to the research.

% ################author###########
 

\author{Peng Zhang }
\email{52205901027@stu.ecnu.edu.cn}
% \orcid{0009-0009-6036-8687}
\affiliation{
  \institution{East China Normal University}
  \city{Shanghai}
  \country{China}
}

\author{Xin Li \footnotemark[1]}
\email{lx_cs@sjtu.edu.cn}
% \orcid{0009-0009-6036-8687}
% \authornote{corresponding author}
\affiliation{
  \institution{Shanghai Jiao Tong University}
  \city{Shanghai}
  \country{China}
}

\author{Xin Lin \footnotemark[1]}
\email{xlin@cs.ecnu.edu.cn}
% \orcid{0009-0009-6036-8687}
% \authornote{corresponding author}
\affiliation{
  \institution{East China Normal University}
  \city{Shanghai}
  \country{China}
}

\author{Liang He}
\email{lhe@cs.ecnu.edu.cn}
% \orcid{0009-0009-6036-8687}
\affiliation{
  \institution{East China Normal University}
  \city{Shanghai}
  \country{China}
}



%%
%% By default, the full list of authors will be used in the page
%% headers. Often, this list is too long, and will overlap
%% other information printed in the page headers. This command allows
%% the author to define a more concise list
%% of authors' names for this purpose.
% \renewcommand{\shortauthors}{Trovato and Tobin, et al.}

%%
%% The abstract is a short summary of the work to be presented in the
%% article.
Large language model (LLM)-based agents have shown promise in tackling complex tasks by interacting dynamically with the environment. 
Existing work primarily focuses on behavior cloning from expert demonstrations and preference learning through exploratory trajectory sampling. However, these methods often struggle in long-horizon tasks, where suboptimal actions accumulate step by step, causing agents to deviate from correct task trajectories.
To address this, we highlight the importance of \textit{timely calibration} and the need to automatically construct calibration trajectories for training agents. We propose \textbf{S}tep-Level \textbf{T}raj\textbf{e}ctory \textbf{Ca}libration (\textbf{\model}), a novel framework for LLM agent learning. 
Specifically, \model identifies suboptimal actions through a step-level reward comparison during exploration. It constructs calibrated trajectories using LLM-driven reflection, enabling agents to learn from improved decision-making processes. These calibrated trajectories, together with successful trajectory data, are utilized for reinforced training.
Extensive experiments demonstrate that \model significantly outperforms existing methods. Further analysis highlights that step-level calibration enables agents to complete tasks with greater robustness. 
Our code and data are available at \url{https://github.com/WangHanLinHenry/STeCa}.

%%
%% The code below is generated by the tool at http://dl.acm.org/ccs.cfm.
%% Please copy and paste the code instead of the example below.
%%
\begin{CCSXML}
<ccs2012>
   <concept>
       <concept_id>10010147.10010178.10010224.10010225.10010230</concept_id>
       <concept_desc>Computing methodologies~Video summarization</concept_desc>
       <concept_significance>500</concept_significance>
       </concept>
   <concept>
       <concept_id>10010147.10010178.10010179.10010182</concept_id>
       <concept_desc>Computing methodologies~Natural language generation</concept_desc>
       <concept_significance>300</concept_significance>
       </concept>
 </ccs2012>
\end{CCSXML}

% \ccsdesc[500]{Human-centered computing~Information visualization}
\ccsdesc[500]{Computing methodologies~Tracking}

% \ccsdesc[500]{Computing methodologies~Video summarization}
% \ccsdesc[300]{Computing methodologies~Natural language generation}
%%
%% Keywords. The author(s) should pick words that accurately describe
%% the work being presented. Separate the keywords with commas.
\keywords{Computer vision; Autonomous driving; 3D object detection; 3D multi object tracking; Deep learning; Kalman filter}

%%
%% This command processes the author and affiliation and title
%% information and builds the first part of the formatted document.
\maketitle
\renewcommand{\thefootnote}{\fnsymbol{footnote}} 
\footnotetext[1] {Corresponding author}
% \renewcommand{\thefootnote}{\fnsymbol{footnote}} 
% \footnotetext[1] {Corresponding author}
% \begin{abstract}
Recent advancements in 3D multi-object tracking (3D MOT) have predominantly relied on tracking-by-detection pipelines. However, these approaches often neglect potential enhancements in 3D detection processes, leading to high false positives (FP), missed detections (FN), and identity switches (IDS), particularly in challenging scenarios such as crowded scenes, small-object configurations, and adverse weather conditions. Furthermore, limitations in data preprocessing, association mechanisms, motion modeling, and life-cycle management hinder overall tracking robustness. To address these issues, we present \textbf{Easy-Poly}, a real-time, filter-based 3D MOT framework for multiple object categories. Our contributions include: (1) An \textit{Augmented Proposal Generator} utilizing multi-modal data augmentation and refined SpConv operations, significantly improving mAP and NDS on nuScenes; (2) A \textbf{Dynamic Track-Oriented (DTO)} data association algorithm that effectively manages uncertainties and occlusions through optimal assignment and multiple hypothesis handling; (3) A \textbf{Dynamic Motion Modeling (DMM)} incorporating a confidence-weighted Kalman filter and adaptive noise covariances, enhancing MOTA and AMOTA in challenging conditions; and (4) An extended life-cycle management system with adjustive thresholds to reduce ID switches and false terminations. Experimental results show that Easy-Poly outperforms state-of-the-art methods such as Poly-MOT and Fast-Poly~\cite{li2024fast}, achieving notable gains in mAP (e.g., from 63.30\% to 64.96\% with LargeKernel3D) and AMOTA (e.g., from 73.1\% to 74.5\%), while also running in real-time. These findings highlight Easy-Poly's adaptability and robustness in diverse scenarios, making it a compelling choice for autonomous driving and related 3D MOT applications. The source code of this paper will be published upon acceptance.

% 3D Multi-Object Tracking (MOT) is essential for autonomous driving systems, contributing significantly to vehicle safety and navigation. Despite recent advancements, existing 3D tracking methods still face significant challenges in accuracy, particularly when dealing with small objects, crowded environments, and adverse weather conditions. To overcome these challenges, we propose \textbf{Easy-Poly}, a novel and efficient multi-modal 3D MOT framework. \textbf{Easy-Poly} employs the Focal Sparse Convolution (\textbf{FocalsConv}) model for object detection. By optimizing convolution operations and augmenting data with multiple modalities, we significantly enhance detection precision.
% \textbf{Easy-Poly} introduces several key innovations: (1) an optimized Kalman filter in the pre-processing stage, (2) integration of the Dynamic Track-Oriented (\textbf{DTO}) Data Association algorithm with confidence-weighted motion models for data association, (3) Dynamic Motion Modeling (\textbf{DMM}) with Adaptive Noise Covariances, and (4) enhanced trajectory management throughout the tracking life-cycle. These improvements increase the robustness and efficiency of tracking, especially in complex scenarios such as crowded scenes and challenging weather conditions. Experimental results on the \textbf{nuScenes} dataset demonstrate that in the pre-processing stage of \textbf{Easy-Poly}, the optimized \textbf{FocalsConv} model achieves a mean Average Precision (mAP) of \textbf{64.96\%} for object detection. Furthermore, the multi-object tracking performance reaches a high AMOTA of \textbf{75.0\%}, surpassing existing methods across multiple performance metrics.
 
% Code and data are available at \textcolor{blue}{\textit{\url{https://github.com/zhangpengtom/FocalsConvPlus}}} and  \textcolor{blue}
%  \textit{\url{https://github.com/zhangpengtom/EasyPoly}.}
%  } 

\end{abstract}
\section{Introduction}

Deep Reinforcement Learning (DRL) has emerged as a transformative paradigm for solving complex sequential decision-making problems. By enabling autonomous agents to interact with an environment, receive feedback in the form of rewards, and iteratively refine their policies, DRL has demonstrated remarkable success across a diverse range of domains including games (\eg Atari~\citep{mnih2013playing,kaiser2020model}, Go~\citep{silver2018general,silver2017mastering}, and StarCraft II~\citep{vinyals2019grandmaster,vinyals2017starcraft}), robotics~\citep{kalashnikov2018scalable}, communication networks~\citep{feriani2021single}, and finance~\citep{liu2024dynamic}. These successes underscore DRL's capability to surpass traditional rule-based systems, particularly in high-dimensional and dynamically evolving environments.

Despite these advances, a fundamental challenge remains: DRL agents typically rely on deep neural networks, which operate as black-box models, obscuring the rationale behind their decision-making processes. This opacity poses significant barriers to adoption in safety-critical and high-stakes applications, where interpretability is crucial for trust, compliance, and debugging. The lack of transparency in DRL can lead to unreliable decision-making, rendering it unsuitable for domains where explainability is a prerequisite, such as healthcare, autonomous driving, and financial risk assessment.

To address these concerns, the field of Explainable Deep Reinforcement Learning (XRL) has emerged, aiming to develop techniques that enhance the interpretability of DRL policies. XRL seeks to provide insights into an agent’s decision-making process, enabling researchers, practitioners, and end-users to understand, validate, and refine learned policies. By facilitating greater transparency, XRL contributes to the development of safer, more robust, and ethically aligned AI systems.

Furthermore, the increasing integration of Reinforcement Learning (RL) with Large Language Models (LLMs) has placed RL at the forefront of natural language processing (NLP) advancements. Methods such as Reinforcement Learning from Human Feedback (RLHF)~\citep{bai2022training,ouyang2022training} have become essential for aligning LLM outputs with human preferences and ethical guidelines. By treating language generation as a sequential decision-making process, RL-based fine-tuning enables LLMs to optimize for attributes such as factual accuracy, coherence, and user satisfaction, surpassing conventional supervised learning techniques. However, the application of RL in LLM alignment further amplifies the explainability challenge, as the complex interactions between RL updates and neural representations remain poorly understood.

This survey provides a systematic review of explainability methods in DRL, with a particular focus on their integration with LLMs and human-in-the-loop systems. We first introduce fundamental RL concepts and highlight key advances in DRL. We then categorize and analyze existing explanation techniques, encompassing feature-level, state-level, dataset-level, and model-level approaches. Additionally, we discuss methods for evaluating XRL techniques, considering both qualitative and quantitative assessment criteria. Finally, we explore real-world applications of XRL, including policy refinement, adversarial attack mitigation, and emerging challenges in ensuring interpretability in modern AI systems. Through this survey, we aim to provide a comprehensive perspective on the current state of XRL and outline future research directions to advance the development of interpretable and trustworthy DRL models.

\section{Related Work}  \label{sec:related_work}


Depending on how the sparsity mask is constructed, sparse attention methods can be divided into three types:
\textbf{(1) Pattern required methods}
rely on some fixed patterns of the attention map, such as sliding windows or attention sinks~\cite{xiao2023efficient}. H2O~\cite{zhang2023h2o}, InfLLM~\cite{xiao2024infllm}, and DUOAttention~\cite{xiao2024duoattention} rely on sliding window pattern. SampleAttention~\cite{zhu2024sampleattention}, MOA~\cite{moaattention}, and StreamingLLM~\cite{xiao2023efficient} rely on sliding window and attention sink pattern. DitFastAttn~\cite{yuan2024ditfastattn} relies on sliding window patterns and similarities between different attention maps. Moreover, DitFastAttn is restricted to simple diffusion transformers, showing incompatibility with language models and MMDiT models like Flux~\cite{flux}, Stable Diffusion3 and 3.5~\cite{stable_diffusion_3_5}, and CogVideoX~\cite{yang2024cogvideox}.
As the pattern varies across models, these methods may not universally work for different models.
\textbf{(2) Dynamic sparse methods}
 dynamically construct the sparse mask based on the input without the need of preset patterns, and are thus potentially more universal. 
Existing works can be further categorized into channel compression and token compression. Channel compression methods include SparQAttn~\cite{ribar2023sparq} and LokiAttn~\cite{singhania2024loki}. They construct the mask by carrying full attention with reduced dimensionality. However, as the dimension is already small, e.g., 64, 128, in commonly used attention, the speedup potential might be limited. 
Token compression methods include MInference~\cite{jiang2407minference} and FlexPrefill~\cite{FlexPrefill}. They construct the mask by compressing each block of tokens to a single token and compute attention on this shorter sequence. However, this approximation is too aggressive: missing important blocks of $P$ is possible if they do not have a large attention score on the compressed sequence. SeerAttention~\cite{gao2024seerattention} requires training of additional parameters for attention, which is expensive to use. Moreover, they are all designed for language models, and their applicability to other model types, such as diffusion models, remains uncertain.
\textbf{(3) Training-based methods}
 modify the attention computation logic, requiring retraining the entire model, such as Reformer~\cite{kitaev2020reformer} and FastAttention~\cite{pagliardini2023fast}. These methods are much more expensive to use than training-free methods. 

There are other ways to accelerate attention, such as optimizing the kernel implementation~\cite{dao2022flashattention,dao2023flashattention,shah2024flashattention}, quantization~\cite{2024sageattention}, distributing the workload~\cite{liu2023ringattentionblockwisetransformers}, and designing linear time attention~\cite{wang2020linformer,choromanski2020rethinking,yu2022metaformer,katharopoulos2020transformers}. They are orthogonal to our approach. 

% % \section{BIDS Dataset}
% \label{sec: dataset}

% % 数据集概括
% As it is very costly to build a bimodal summarization dataset from scratch, we, therefore, leverage the QVHighlights dataset~\cite{lei2021detecting} to construct a \textbf{B}imodal V\textbf{ID}eo \textbf{S}ummarization dataset (\textbf{BIDS}) to support the investigation of the BiSSV task. The constructed BIDS dataset finally contains 8130 videos with corresponding ground-truth Visual-Modal (VM) and Textual-Modal (TM) Summaries and saliency scores annotated for each 2-second clip, indicating its significance. Following the restrictions of traditional video summarization~\cite{gygli2014creating}, we ensure that the length of the VM-Summary does not exceed 15\% of the original video's duration. We describe the data processing and analysis in detail in the following subsections.

% \subsection{Data Processing}
% \label{sec: datapro}

% % 数据处理概括
% We aim to build a bimodal video summarization dataset with triplet data samples (video, TM-Summary, VM-summary), where the TM-Summary is a concise text description, and the VM-summary contains highlighted segments within the video. Firstly, we merge text-related segments from the original videos to guarantee that the TM-Summary accurately captures the main content of the video. 
% Secondly, we design a ranking-based extraction algorithm to preserve the most salient visual content as VM-Summary. 
% Lastly, we perform data cleaning to remove unsuitable videos that lack a clear focus for summarization. The overview of the BIDS building process is illustrated in Figure \ref{fig: data_construct}.

% % 步骤
% % 合并标注和视频
% \noindent \textbf{Data merging.} QVHighlights~\cite{lei2021detecting} is a video dataset that supports query-based moment retrieval and highlight detection, with annotations of natural language query, segments relevant to the query, and saliency scores for each 2s-clip within the segments. Taking the query as the TM-Summary, we merge the relevant segments chronologically as original videos in our dataset. In this way, we obtain a (video, TM-Summary) pair, for which we subsequently extract the VM-Summary. 

% % 提取 VM-Summary
% \noindent \textbf{VM-Summary extraction.} We utilize the annotated 2s-saliency scores for VM-Summary extraction. 
% Unlike the Knapsack algorithm utilized by previous video summarization datasets~\cite{song2015tvsum,gygli2014creating}, our extraction algorithm retains salient visual content within long segments and avoids favoring short segments. An illustration of this algorithm is presented in Figure ~\ref{fig: data_construct}. 
% We also provide a  pseudo-code in Appendix \ref{sec: pseudo code}.

% (a) \textit{Ranking}. We first merge adjacent 2s-clips with the same saliency scores into segments. Then, we rank all the candidate segments according to their saliency scores. The candidate segments are subsequently selected for VM-Summary in descending order. To comply with the length limit of VM-Summary (15\% video duration in our case), we may need to scale some candidate segments.

% (b) \textit{Scaling}. As the candidate segments vary in length, the purpose of scaling is to preserve informative parts within segments while guaranteeing conciseness. Specifically, candidate segments with the same score will be appended to the VM-Summary if it does not surpass the length constraint. Otherwise, these segments are proportionally scaled. 
% We assume that the parts closer to higher-scored segments usually contain more valuable information.
% Therefore, if the segment has higher-scored neighbors, adjacent parts closer to those neighbors are preserved (colored in \textcolor{red}{red} and \textcolor{myyellow}{yellow}, indicating two and one higher-scored neighbors, respectively); otherwise, its central part is preserved (colored in \textcolor{mygreen}{green}). The scaled segments are appended to the VM-Summary, and the segments with lower ranks are all rejected.

% % 数据清洗
% \noindent \textbf{Data cleaning.} Finally, we remove segments shorter than 2 seconds and videos with VM-Summary occupying less than 5\% of the video duration since they lack clear focal points for summarization. Finally, of 8,172 videos, only 42 (0.51\%) videos are removed.

% \subsection{Data Analysis}
% \label{sec: dataana}

% % 传统数据处理方式的缺陷
% Traditional video summarization datasets use the Knapsack algorithm to generate VM-Summary~\cite{gygli2014creating,song2015tvsum}. 
% However, Otami M et al.~\cite{otani2019rethinking} point out that their segmentation-selection pipeline favors short segments since selecting long segments costs more. 
% However, long and visually consistent segments can also contain informative moments. For example, when watching a video of \textit{someone playing basketball}, most of the visual content is similar, but we can still identify key moments, such as \textit{shooting}.
% Inspired by humans' ability to distinguish important moments in long videos, we choose to scale the candidate segments instead of rejecting them entirely. As a result, our VM-Summary shows a stronger correlation between the saliency scores and the selected segments.  

% % 相关系数比较
% We use Spearman's correlation coefficient $\rho$~\cite{zwillinger1999crc} to validate the effectiveness of our VM-Summary extraction algorithm. A higher coefficient between the saliency scores $S$ and the frame-level selection sequence $F$ (1 for the frame being selected into the VM-Summary and 0 for otherwise) indicates more salient content is preserved, which is the goal of summarization. 
% As presented in Table ~\ref{tab: dataset comparison}, BIDS has the highest Spearman's $\rho$ compared to traditional datasets. Moreover, Spearman's $\rho$ between $S$ and $F$ (generated by annotators) surpasses the $\rho$ between $S$ and GT-$F$ (obtained by applying Knapsack algorithm over the annotated saliency scores) in SumMe~\cite{gygli2014creating}, which further demonstrates that Knapsack algorithm can not effectively preserve salient parts within long segments. 

% % 统计数据
% After removing invalid and duplicate videos, BIDS contains 8130 videos, with 5854/650/1626 videos for training/validation/test set. 
% We ensure that the original videos between different sets do not overlap to avoid data leakage.  
% The data statistics of BIDS are presented in Table \ref{tab: dataset statistics}. As presented in Figure \ref{fig: distribution}, our algorithm is able to generate VM-summaries within a strict length constraint, with the majority occupying 14-15\% of the video's duration. Furthermore, the segments in a VM-Summary are evenly distributed throughout the corresponding video.

% \begin{table}[t]
%     \centering
%     \small  
%     \caption{Comparison with traditional video summarization datasets.
%     $\rho$: Average Spearman's correlation coefficient. 
%     Sig.: Significance (p < 0.05). 
%     $S$: Saliency score.
%     $F$: Frame-level sequence indicating each frame is selected (1) or not selected (0) into the VM-Summary. 
%     GT-$F$: the $F$ is calculated by averaging human annotated scores for each video in SumMe~\cite{gygli2014creating} and TVSum~\cite{song2015tvsum}.
%     dp: the $F$ is obtained by the Knapsack algorithm.
%     } 
    
%     \vspace{-8pt}
%     \begin{tabular}{ccccc}
%     \toprule
%     \textbf{Dataset}                & \textbf{Set of Variables}          & $\boldsymbol{\rho}$ & \textbf{Sig.}  & \textbf{\# of Videos}    \\
%     \midrule
%     \multirow{2}{*}{SumMe~\cite{gygli2014creating}} &($S$, GT-$F_{dp}$) & 0.34                               & \checkmark & \multirow{2}{*}{25} \\
%                            &($S$, $F$)        &  \underline{0.44}                        & \checkmark &                     \\
%                            \midrule
%     \multirow{2}{*}{TVSum~\cite{song2015tvsum}} &($S$, GT-$F_{dp}$)& 0.31                               & \checkmark & \multirow{2}{*}{50} \\
%                            &($S$, $F_{dp}$)    & 0.24                               & $\times$   &                     \\
%                            \midrule
%     BIDS(ours)           &($S$, GT-$F$)     & \textbf{0.52}                      & \checkmark & \textbf{8130}     \\ 
%     \bottomrule
%     \end{tabular}
%     \label{tab: dataset comparison}
% \end{table}


% \begin{figure}[t]
%     \centering
%     \includegraphics[width=0.95\linewidth]{Images/Distribution.pdf}
%     \vspace{-6pt}
%     \caption{(a) Distribution of duration ratio between VM-Summary and original video; (b) Distribution of temporal positions of the segments selected into the VM-Summary in the original video.
%     }
%     \label{fig: distribution}
% \end{figure}

% \begin{figure*}[t]
%     \centering
%     \includegraphics[width=0.8\linewidth]{Images/Framework.pdf}
%     \vspace{-8pt}
%     \caption{Model Architecture of UBiSS. 
%     }
%     \label{fig: framework}
% \end{figure*}

% \begin{table*}[t]
%     \centering
%      \small
%     \captionsetup{skip=10pt}
%     % \renewcommand{\arraystretch}{1.2}
%     \caption{Statistics of BIDS. 
%     {VM: Visual-Modal Summary. TM: Textual-Modal Summary.}
%     \vspace{-8pt}
%     }
%     \begin{tabular}{ccccccc}
%            \toprule
%            & \textbf{Avg. Video Len(s)} & \textbf{Total Video Len(h)} & \textbf{Avg. VM Len(s)} & \textbf{Avg. VM proportion(\%)} & \textbf{Avg. TM Len(word)} & \textbf{\# of Videos} \\
%            \midrule
%             Training   & 43.55           & 70.82             & 6.05         & 14.07               & 10.52            & 5854             \\
%             Validation & 40.05           & 7.23              & 5.57         & 14.07               & 10.41            & 650              \\
%             Test       & 44.83           & 20.25             & 6.19         & 14.12               & 10.42            & 1626             \\
%             All        & 43.53           & 98.3              & 6.04         & 14.08               & 10.49            & 8130      \\
%             \bottomrule
%     \end{tabular}
%     \label{tab: dataset statistics}
% \end{table*}

\section{GistVis}
\label{sec:gistvis-method}
The following sections describe the design of GistVis. We first define the uniform data representation schema \textit{data fact} (\textbf{DG1}, Sec.~\ref{sec:gistvis-gistfact}). Then, we provide a detailed description of the purpose of each module in the GistVis computational pipeline, how it could support plug-and-play property (\textbf{DG2}, Sec.~\ref{sec:computational-pipeline}), and the design choices of our word-scale visualizations (\textbf{DG3}). Lastly, we introduce the implementation details of the current iteration of GistVis (Sec.~\ref{sec:implementation-detail}).

\subsection{Data Fact}
\label{sec:gistvis-gistfact}

We define \textit{data fact} as the uniform data structure to encode all text content, either with or without data insights. Under this definition, we characterize \textit{data fact} as a declarative intermediate data structure that encapsulates all the key information required to generate word-scale visualizations. We formulate \textit{data fact} as a 2-tuple:
\begin{equation}
\mathTextItalics{fact}\, \coloneq \left \{ \mathTextItalics{unitSegmentSpec, dataSpec?} \right \} \nonumber
\end{equation}
where \textit{unitSegmentSpec} records key information related to the \textbf{unit segment}. Meanwhile, \textit{dataSpec} is a list that contains the data restored from the information provided by the unit segment. Because not all unit segments contain data insights, \textit{dataSpec} is an optional attribute in the tuple and could be used to distinguish between plain text segments and data insight segments. For clarity, we use the \textit{Typescript} notation (question mark) to represent optional elements. In the following, we explain the design of \textbf{unit segment specification} (\textit{unitSegmentSpec}) and \textbf{data specification} (\textit{dataSpec}).

\subsubsection{Unit Segment Specification}
\label{sec:gistvis-unitSegmentSpec}
We define unit segment specification as a 4-tuple:
\begin{equation}
    \mathTextItalics{unitSegmentSpec}\, \coloneq \left \{
        \mathTextItalics{type, context, attribute?, position?}
    \right \} \nonumber
\end{equation}

\paragraph{Type}
Type contains seven candidate types, including the six data insight types (i.e., value, proportion, comparison, trend, rank, and extreme) we selected from the facts taxonomy (see Sec.~\ref{sec:formative-study}), and one ``no type'' to represent plain text. 

\paragraph{Context}
Content is where we store the original text snippets that are unit segments. In this work, we define a unit segment as one or multiple sentences that collectively convey one data insight or are similar in semantics. We implement this definition to support situations where data insights span multiple adjacent sentences.

However, it is worth noting that we defined unit segments based on a strong assumption that relevant information of one data insight is described in sequential order, proximate in position, and contained within the same paragraph. Cases might exist where the same data insight could occur in numerous places across the document~\cite{goffin2020Interaction}. We argue that word-scale visualizations designed for document-centric analysis might not be the optimal solution to convey such information. Thus, we only focus on performing paragraph-level segmentation.

\smallskip
The above two entries (\textit{Type}, \textit{Context}) record contextual information directly related to the text and are ubiquitous for all unit segments. Meanwhile, \textit{Attribute} and \textit{Position} are data-related auxiliary information of the text and relevant to specific \textit{Types}. Referencing the four-level model of the semantic content of visualization proposed by \citet{lundgard2022accessible}, we viewed \textit{Attribute} and \textit{Position} to represent L2 - L3 (statistical concepts and relations, perceptual and cognitive phenomena) information and L1 (elemental and encoded properties) information respectively. Because \textit{Attribute} and \textit{Position} convey key semantic insight about the data and are closely related to text descriptions, we leave those attributes in the unit segment specification.

\paragraph{Attribute}
Attribute is an optional entry explicitly designed for data types extreme and trend. The candidate options for attributes include ``increasing'' and ``decreasing'' for the trend type and ``maximum'' and ``minimum'' for the extreme type. The attribute information is a supplementary constraint in the visualization generation process to correct potential errors in the generated visualizations and reflect the semantics of the text description.

\paragraph{Position}
The position is an optional entry to handle the extreme data fact type. It represents the original text description that should be highlighted to provide contextual information about the maximum or minimum values. For example, if we want to augment "\textit{the maximum of sales for company A,}" it is more informative to label the entire phrase rather than just "company A" (the standard highlight practice for other data fact types). If position is available in the extreme data fact type, the position attribute would override the default practice and ensure the entire phrase is highlighted.

\subsubsection{Data Specification}
\label{sec:gistvis-dataspec}
While Unit Segment Specification characterizes the textual content and higher-level data insights, data specification represents the raw data elements reconstructed from the textual content. Data Specification is designed to be an analogy of tabular datasets (Fig.~\ref{fig:dataspec-tableview}), which we define in a four-tuple:
\begin{equation}
    \mathTextItalics{dataSpec}\, \coloneq \left \{ 
        \mathTextItalics{space, breakdown, feature, value}
    \right \}[] \nonumber
\end{equation}

Specifically, \textbf{space} is a facet of analysis with a given text description. For example, if a sentence describes the market share of different car manufacturers, the analysis space would be ``car manufacture'' (Fig.~\ref{fig:dataspec-tableview} \ding{182}). Meanwhile, \textbf{breakdown} is a set of temporal or categorical data fields in which data are further divided under the space. For example, the brand name, like ``Brand A'' (Fig.~\ref{fig:dataspec-tableview} \ding{183}), would be the breakdown for ``car manufacture''. \textbf{Feature} is the measurement of breakdown. For example, we could measure the sales percentage for each manufacturer (Fig.~\ref{fig:dataspec-tableview} \ding{184}), a feature derived from annual sales of car manufacturers. Lastly, \textbf{value} is a numerical data field that could be retrieved from a combination of breakdown and feature. For example, the ``sales percentage'' of ``Brand A'' is 0.5 (Fig.~\ref{fig:dataspec-tableview} \ding{185}). All data attributes are required for each data specification entry, with the only exception being the ``not a number'' (NaN) value attributes. Cases exist when the unit segment describes a semantic data insight (e.g., increasing or decreasing for the trend type), and we make ``not a number cases'' a special condition for GistVis to process.

\begin{figure}[tb]
  \centering
  \includegraphics[width=0.8\linewidth]{figures/GistFact-Definition.pdf}
  \caption{Each element in data specification consists a four-tuple, space \ding{182}, breakdown \ding{183}, feature \ding{184} and value \ding{185}.}
  \label{fig:dataspec-tableview}
\end{figure}

In the following section, we describe how the fields in the data fact specification are filled using the GistVis pipeline.

\subsection{Computational Pipeline}
\label{sec:computational-pipeline}
Based on the data requirement of data fact, we proposed the GistVis pipeline to automatically transform data-rich text descriptions to word-scale visualizations. We utilized both LLM-based and design knowledge-driven approaches to achieve word-scale visualization generation. Specifically, we decomposed the generation process into four stages: gist discovery, fact type annotation, fact specification extraction, and fact visualization. We capitalized on LLMs' natural language understanding capability for the first three stages and applied the prompt chaining strategy~\cite{wu2022ai} to transform natural language descriptions into data facts. Meanwhile, the fact visualization stage uses a simple heuristic-driven approach to map data facts to interactive visualization components. As an automated process, all prompts and visualization heuristic rules are readily coded into the pipeline (see Supplementary Material for detailed prompt design). Fig.~\ref{fig:algorithm-pipeline} shows the GistVis computational pipeline, which consists of four modules respective to the four stages above: Discoverer (\textbf{M1}), Annotator (\textbf{M2}), Extractor (\textbf{M3}) and Visualizer \textbf{(M4)}.

\begin{figure*}
    \includegraphics[width=1\textwidth]{figures/GistVis-Pipeline.pdf}
    \caption{The GistVis pipeline consists of four modules: Discoverer (\textbf{M1}), Annotator (\textbf{M2}), Extractor (\textbf{M3}), and Visualizer (\textbf{M4}). Data flows through the four modules sequentially, where a large language model captures the insight of the data-rich document (\textbf{M1-M3}). Visualizer (\textbf{M4}) maps the captured insight into interactive visualizations, populated in situ in the text document at word scale.}
    \label{fig:algorithm-pipeline}
\end{figure*}

\subsubsection{Discoverer}
The first step in the GistVis pipeline is to divide paragraphs into unit segments. Discoverer (Fig.~\ref{fig:algorithm-pipeline}.~\textbf{M1}) leverages the zero-shot capability of LLMs~\cite{brown2020language} to perform the segmentation process. We restrain LLMs from identifying the shortest unit segment possible to better pair the text descriptions with in situ word scale visualizations. The prompt contains the six fact types to provide LLMs with more detailed segmentation requirements. Meanwhile, although LLMs are informed of the data fact types in the instructions, we do not label the data fact type at this stage to keep the task simple.
Moreover, the segmentation process shortens the context length, filtering out excess information for the subsequent data fact type labeling step. Additionally, to reduce hallucinations from LLMs during the segmentation process, we instruct the LLM to keep the text description ``as is'' without modifying the original text description or punctuation.

\subsubsection{Annotator}
The Annotator module (Fig.~\ref{fig:algorithm-pipeline}.~\textbf{M2}) aims to fill in the \textit{Type} field in the data fact specification for each segment from the prior module. Since LLMs are well calibrated to answer multiple choice and true/false questions~\cite{kadavath2022language}, we formulate the data fact type annotation as a two-stage question-answering (QA) problem. In the first stage, we ask LLMs to make a true/false judgment on whether the given segment belongs to a specific data fact type (\textbf{Type Checker}). Then, since one segment can be classified into multiple data fact types, we perform another round of prompting (\textbf{Type Moderator}) in a multiple-choice format to determine the most appropriate data fact type for each segment.

\paragraph{Type checker}
Type Checker identifies all possible fact types for each segment. To incorporate visualization knowledge into LLMs, we applied the few-shot in-context learning prompting paradigm~\cite{brown2020language}. We constructed seven individual reasoners for each of the seven fact types. Specifically, following a task description informing LLMs to return whether or not the segment can be classified as a given fact type, we define the fact type followed by three examples (two positive, one negative), allowing the LLMs to capture the definition of data fact type and forming a 3-shot prompt. After running through all seven reasoners, we record all the possible data fact types in a list for further moderation. If all reasoners responded false, the segment would be labeled a text-only segment (i.e., no type) and not proceed for further analysis.


\paragraph{Type moderator}
Taking the output from the Type Checker, the Type Moderator determines the most suitable data fact type for a given segment. We formulate the prompt in a multiple-choice pattern, with the options drawn from the previous type-checking step. Due to the varying number of possible candidates, we do not include examples and only reiterate the definition for the candidates. After the moderation step, we narrow the data fact label to only one label and fill the type field in the data fact specification in preparation for the data extraction phase.

We annotated data fact types in two instead of one round for the following reasons. First, we limit the input length and avoid excessive length prompts that could potentially harm inference capability~\cite{li2024longcontext}. Splitting the annotation task into two rounds would save space—we avoided all the samples for each data type occurring in the same prompt. Secondly, we consider the extensibility of the GistVis pipeline. Splitting the process into two stages would ensure that adding custom data fact types requires no more than writing additional prompts for the new data fact type. We also justified this design decision with an ablation study in our technical evaluation (Sec.~\ref{subsec:quant-eval-annotator}).

\subsubsection{Extractor}
Since each data fact type has different requirements, the Extractor module (Fig.~\ref{fig:algorithm-pipeline}.~\textbf{M3}) applies a case-by-case extraction strategy based on the data fact type generated from the Annotator module when extracting the data specifications. We formulate the prompts based on the visualization specification requirement described in Sec.~\ref{sec:gistvis-dataspec}.

Compared with the methods that applied regular expression to identify numbers in a text description, LLMs allow more flexibility in how the data is presented. For example, when numbers are not expressed numerically (e.g., 10 thousand, ten thousand), regular expressions would fail in extracting the correct underlying data. Instead, we prompt LLMs to convert non-numerical expressions of numbers to their numerical form (i.e., 10000), thus expanding data extraction capability. We also use a number parser to further transform all extracted data into a numerical form.


\subsubsection{Visualizer}

\begin{figure*}[tb]
  \centering
  \includegraphics[width=\linewidth]{figures/design-space.pdf}
  \caption{A collection of 14 candidate visualizations and the corresponding chart type for each data fact type. The \textbf{Example} column shows the effect of the appearance of word-scale visualization in data-rich documents. We present the examples when the mouse hovers over the word-scale visualization of focus.}
  \label{fig:visualization-design}
\end{figure*}

Visualizer (Fig.~\ref{fig:algorithm-pipeline}.~\textbf{M4}) applies a chart-based approach (as opposed to the grammar of graphics~\cite{wilkinson2012grammar} approach) in visualizing data facts. We took the chart-based approach based on the observation that various business visualization toolkits~\cite{g2plot-2024, echarts-2024} use this paradigm. 

\paragraph{Visualization Design}
We implement three basic chart types (bar chart, line chart, and icon array) and use their variants with relevant icons to tailor to the needs of different data fact types. The variants of the three chart types and icons constructed a design space including 14 candidate visualizations to represent different conditions of data insights (Fig.~\ref{fig:visualization-design}). Notably, we use horizontal stacked Bar Charts to represent proportion. Although prior works have applied Pie Charts (e.g.,~\cite{goffin2017Exploratory, huth2024eye}), we argue that the limited height could make discerning the angle difference between categories difficult. Thus, in this work, we attempt to capitalize on the relatively ample space on the horizontal dimension and use length, a more effective visual channel than angle~\cite{munzner2014visualization}, to encode proportional data. We also made several hard constraints to facilitate the readability of word-scale visualizations. For example, we limit the maximum rank for visualization to 10, avoiding situations of ultra-long vertical bar charts if the value of the ranking is large. We made this decision based on the fact that only one in 35 rank insights we identified in our corpus (Sec.~\ref{sec:formative-study}) included a rank above 10. Moreover, the definition of the rank fact type implied a sorting process on the dataset such that the rank is relative, leading to the inclination of reporting ranking with single-digit numbers. Meanwhile, to avoid situations such as rank overflow, we also designed a fallback condition, presenting a question mark icon to indicate the unit segments that might contain data insights not properly presented by GistVis. We expect that users could better perform analytical activities~\cite{amar2005lowlevel} with data-rich documents through the locally aggregated views of data in the form of word-scale visualizations.

\begin{table*}[tbp]
\small
  \centering
  \caption{The visualization knowledge the Visualizer (M4) module assumes when parsing data fact specification. C and T represent categorical and temporal data types, whereas N represents the numerical data type. The fallback condition will be hit if data is missing in the data fact specification.}
    \begin{tabularx}{\linewidth}{cccccX}
    \toprule
    \textbf{Fact Type} ($t$) & \textbf{Breakdown} ($b$) & \textbf{Feature} ($f$) & \textbf{Attribute} ($a$) & \textbf{Position} ($p$) & \textbf{Tooltip Syntax} \\
    \midrule
    Proportion & C/T   & N     & \ding{53}     & \ding{53}     & The proportion of $\{b_i\}$ is $\{v_i\}$. \\
    Value & C/T   & N     & \ding{53}     & $\geq$1 & The value of $\{b_i\}$ is $\{v_i\}$. \\
    Extreme & C/T   & N     & \makecell[c]{maximum/\\minimum} & $=$1  & The $\{a_i\}$ of $\{b_i\}$. \\
    Comparison & C/T   & N     & \ding{53}     & \ding{53}     & The difference between $\{b_i\}$ and $\{b_j\}$ is $\{| v_i - v_j|\}$. \\
    Rank  & C     & N     & \ding{53}     & \ding{53}     & Rank $\{v_i\}$: $\{b_i\}$ \\
    Trend & T     & N     & \makecell[c]{positive/\\negative} & \ding{53}     & \makecell[l]{$\{a_i\}$ \\ $\{f\}$ of $\{b_i\}$ is $\{v_i\}$. \\ The $\{a_i\}$ is $\{| v_i - v_j|\}$} \\
    \midrule
    (Fallback cond.)  & \ding{53}     & \ding{53}     & \ding{53}     & \ding{53}     & May contain data insight of $\{t\}$. \\
    \bottomrule
    \end{tabularx}
  \label{tab:gistvis-implementation}
\end{table*}


\paragraph{Linking Word-scale Visualizations with Text}
We also added interactive features to word-scale visualizations. The motivation behind making word-scale visualizations interactive is twofold: 1) to pack more information in word-scale visualizations, and 2) to enhance the reading experience by coupling visualization with text (Sec.~\ref{subsec:relatedwork-vistext}). 

Firstly, to pack more information within word-scale visualizations, we designed a drill-down interaction that pops up a tooltip every time users hover over the visualization. The tooltip contains basic data descriptions about the word-scale visualization, allowing users to grasp key information even when viewing the visualization standalone. We define a set of default syntax to describe the data insights for each data fact type according to the value types in the data fact specification (Table.~\ref{tab:gistvis-implementation}). Crucially, users access the tooltips on demand, minimizing their impact in obstructing a normal document reading process.


Secondly, to enhance the reading experience, we designed bidirectional interactions between text and word-scale visualizations. Specifically,
we highlight the ``entities'' of a sentence via matching document text with the set of extracted breakdowns in the data fact specification. When a specific visual element is selected, the corresponding entity synchronously lights up to show the correspondence between the entity and its value. Conversely, when a specific entity is selected, the corresponding visual element in the word-scale visualization would also light up to help users retrieve the entity related to the value (see Fig.~\ref{fig:visualization-design} Example column).

\subsection{Implementation}
\label{sec:implementation-detail}
We implement the GistVis pipeline using a typical web stack. For rendering the data-rich document augmented by GistVis, we use open-source libraries, including React\footnote{\url{https://react.dev/}} as the UI framework and D3.js~\cite{bostock2011datadriven} for rendering the word-scale visualizations in svg format. We took inspiration from prior jQuery-based word-scale visualization packages such as Sparklificator~\cite{sparklificator-package} and Piety~\cite{piety-package} and expanded their functionality using the React framework.

As for LLMs, we chose \texttt{DeepSeek-V2.5}~\cite{deepseek-ai2024deepseekv2}, an open-source\footnote{\url{https://huggingface.co/deepseek-ai/DeepSeek-V2.5}} Mixture-of-Experts (MoE) language model released by \texttt{DeepSeek}\footnote{\url{https://www.deepseek.com/en}}. We made this decision based on its decent performance at a low cost (overall cost less than 0.28\$ per 1M tokens). However, since we did not have the computation device to run such a large model, we resorted to the commercial API release of the exact DeepSeek model in our implementation. 
We then chained the processing steps with LangChain.js v0.1~\cite{langchainjs}, the JavaScript implementation of a popular framework for developing LLM-powered applications.

\section{Experiments}
\label{sec: experiments}

\begin{table*}
        % \centering
        \caption{A comparison between our proposed method with other advanced methods on the nuScenes test set.
        % This leaderboard is available at \href{https://www.nuscenes.org/tracking?externalData=all&mapData=all&modalities=Any}{nuScenes official benchmark}.
        $\ddagger$ means the GPU device.
        \textcolor{black}{The reported runtimes of all methods exclude the detection time.}
        Poly-MOT~\cite{li2023poly}, Fast-Poly~\cite{li2024fast} and Easy-Poly rely entirely on the detector input, as they do not utilize any visual or deep features \textcolor{black}{during tracking}.}
        \label{table:nu_test}
        % \renewcommand{\arraystretch}{0.7}
        \setlength{\tabcolsep}{1.6mm}
        {
        \begin{tabular}{cccc|ccc|ccc}
        \toprule
        \multicolumn{1}{c}{\textbf{Method}} & \textbf{Device} & \textbf{Detector} & \textbf{Input} & \textbf{AMOTA}$\uparrow$ & \textbf{MOTA}$\uparrow$ & \textbf{FPS}$\uparrow$ & \textbf{IDS}$\downarrow$ & \textbf{FN}$\downarrow$ & \textbf{FP}$\downarrow$ \\ \midrule
                                     EagerMOT~\cite{kim2021eagermot}               & \textbf{\text{--}}       & CenterPoint~\cite{yin2021center}\&Cascade R-CNN~\cite{cai2018cascade}         & 2D+3D       & 67.7      & 56.8     & 4    & 1156    & 24925   & 17705   \\
                                   CBMOT~\cite{benbarka2021score}     & I7-9700          & CenterPoint~\cite{yin2021center}\&CenterTrack~\cite{zhou2020tracking}          & 2D+3D         & 67.6      & 53.9      & \textcolor{red}{80.5}    & 709    & 22828   & 21604   \\
                                   % ShaSTA~\cite{sadjadpour2023shasta} & A100$\ddagger$       & CenterPoint~\cite{yin2021center}         & 3D      & 69.6      & 57.8     & 10     & 473    & 21293   & 16746   \\
                                   Minkowski~\cite{gwak2022minkowski} & TITAN$\ddagger$  & Minkowski~\cite{gwak2022minkowski}         & 3D      & 69.8      & 57.8     & 3.5    & 325    & 21200   & 19340   \\
                                   ByteTrackv2~\cite{zhang2023bytetrackv2} & \textbf{\text{--}}       & TransFusion-L~\cite{bai2022transfusion}         & 3D      & 70.1      & 58     & \textbf{\text{--}}    & 488    & 21836   & 18682   \\
                                   3DMOTFormer~\cite{ding20233dmotformer}& 2080Ti$\ddagger$       & BEVFuison~\cite{liu2023bevfusion}       & 2D+3D      & 72.5      & 60.9     & \textcolor{blue}{54.7}    & 593    & 20996   & \textcolor{blue}{17530}   \\  
                                   %CAMO-MOT~\cite{li2023camo}& 3090Ti$\ddagger$   & BEVFuison~\cite{liu2023bevfusion}\&FocalsConv~\cite{chen2022focal}   & 2D+3D      & 75.3      & \textbf{63.5}     & \textbf{\text{--}}    & 324    & 18192   & 17269   \\
                                   Poly-MOT~\cite{li2023poly}               & 9940X       & LargeKernel3D~\cite{chen2022scaling}         &2D+3D       & \textcolor{blue}{75.4}      & \textcolor{blue}{62.1}     & 3    & \textcolor{blue}{292}    & \textcolor{blue}{17956}   & 19673   \\ 
                                  Fast-Poly~\cite{li2024fast}               & 7945HX       & LargeKernel3D~\cite{chen2022scaling}         &2D+3D       & \textcolor{blue}{75.8}      & \textcolor{blue}{62.8}     & 34.2    & \textcolor{blue}{326}    & \textcolor{blue}{18415}   &  \textcolor{blue}{17098}  \\ \midrule

                                  \textbf{Easy-Poly (Ours)}               & 4090Ti$\ddagger$       & LargeKernel3D~\cite{chen2022scaling}         &2D+3D       &  \textcolor{red}{75.9}      & \textcolor{red}{63.0}     & \textcolor{blue}{34.9}    & \textcolor{red}{287}    & \textcolor{red}{17620}   &   \textcolor{red}{16718}  \\
        \bottomrule
        \end{tabular}}
        % \vspace{-1.5em}
 \end{table*}


\begin{table*}
\vspace{0.5em}
\begin{center}
\caption{
{A comparison of existing methods applied to the nuScenes val set.}}
\label{table:nu_val}
% \renewcommand{\arraystretch}{0.7}
\setlength{\tabcolsep}{2.4mm}
{
\begin{tabular}{cccccccc}
\toprule
\bf{Method} & \bf{Detector} & \bf{Input Data} & \bf{MOTA$\uparrow$} & \bf{AMOTA$\uparrow$} & \bf{AMOTP$\downarrow$} & \bf{FPS$\uparrow$} & \bf{IDS$\downarrow$}  \\ \hline
CBMOT~\cite{benbarka2021score}   & CenterPoint~\cite{yin2021center} \& CenterTrack~\cite{zhou2020tracking} & 2D + 3D & -- & 72.0 & \textbf{\textcolor{red}{48.7}}  & -- & 479   \\
EagerMOT~\cite{kim2021eagermot}  & CenterPoint~\cite{yin2021center} \& Cascade R-CNN~\cite{cai2018cascade} & 2D + 3D  & -- & 71.2   & 56.9 & 13  & 899    \\
SimpleTrack~\cite{pang2022simpletrack}  & CenterPoint~\cite{yin2021center} & 3D & 60.2 & 69.6  & 54.7 & 0.5  & 405  \\
CenterPoint~\cite{yin2021center}   & CenterPoint~\cite{yin2021center} & 3D & -- & 66.5  & 56.7 & -- & 562 \\ 
OGR3MOT~\cite{zaech2022learnable}  & CenterPoint~\cite{yin2021center} &3D & 60.2 & 69.3  & 62.7 & 12.3 & \textbf{\textcolor{blue}{262}}  \\ \hline

\textbf{Poly-MOT}~\cite{li2023poly}      & CenterPoint~\cite{yin2021center} & 3D & 61.9 & 73.1   & \textbf{\textcolor{blue}{52.1}} & 5.6 & 281   \\ 
% \textbf{Poly-MOT}~\cite{li2023poly}      & LargeKernel3D-L~\cite{chen2022scaling}  & 3D   & \textbf{\textcolor{red}{75.2}}    & 54.1   & \textbf{\textcolor{red}{252}} \\ 

\textbf{Poly-MOT}~\cite{li2023poly}      & LargeKernel3D-L~\cite{chen2022scaling}  & 3D & 54.1 & \textbf{\textcolor{red}{75.2}}    & 54.1  & 8.6 & 252 \\

\textbf{Fast-Poly}~\cite{li2024fast}      & CenterPoint~\cite{yin2021center} & 3D & \textbf{\textcolor{blue}{63.2}}  & 73.7   &  -- & \textbf{\textcolor{blue}{28.9}} & 414   \\ \hline
 
\textbf{Easy-Poly (Ours)}       & CenterPoint~\cite{yin2021center} & 2D + 3D & \textbf{\textcolor{blue}{64.4}} & \textbf{\textcolor{blue}{74.5}}   & 54.9  & \textbf{\textcolor{blue}{34.6}} & \textbf{\textcolor{blue}{272}}   \\
\textbf{Easy-Poly (Ours)}       & LargeKernel3D~\cite{chen2022scaling}  & 2D + 3D  & \textbf{\textcolor{red}{64.8}} & \textbf{\textcolor{blue}{75.0}}    & \textbf{\textcolor{blue}{53.6}}  & \textbf{\textcolor{red}{34.9}}  & \textbf{\textcolor{red}{242}} \\ \hline
% \vspace{-4.5em}
% \setlength{\abovecaptionskip}{3pt}
% \setlength{\belowcaptionskip}{3pt}
\end{tabular}}
\end{center}
\end{table*}

\subsection{Datasets}

% The nuScenes dataset~\cite{caesar2020nuscenes} consists of 850 training and 150 test sequences, capturing a wide range of driving scenarios, including challenging weather conditions and nighttime environments. Each sequence contains approximately 40 frames, with keyframes sampled at 2Hz and fully annotated. In addition to this, it provides annotations for object-level attributes such as visibility, activity, pose, and more. It includes a large volume of RGB and point-cloud data (in PCD format). The official evaluation protocol utilizes \textbf{AMOTA}, \textbf{MOTA}, and \textbf{sAMOTA}~\cite{weng20203d} as primary metrics, evaluating performance across seven object categories: Car (\textit{Car}), Bicycle (\textit{Bic}), Motorcycle (\textit{Moto}), Pedestrian (\textit{Ped}), Bus (\textit{Bus}), Trailer (\textit{Tra}), and Truck (\textit{Tru}). Given the substantial size of the complete nuScenes dataset, users often prefer the nuScenes-mini dataset. Notably, Poly-MOT, Fast-Poly, and our proposed Easy-Poly methods exclusively utilize keyframes for tracking tasks.


The nuScenes dataset~\cite{caesar2020nuscenes} consists of 850 training and 150 test sequences, capturing a wide range of driving scenarios, including challenging weather conditions and nighttime environments. Each sequence contains approximately 40 frames, with keyframes sampled at 2Hz and fully annotated. The official evaluation protocol utilizes \textbf{AMOTA}, \textbf{MOTA}, and \textbf{sAMOTA}~\cite{weng20203d} as primary metrics, evaluating performance across seven object categories: Car (\textit{Car}), Bicycle (\textit{Bic}), Motorcycle (\textit{Moto}), Pedestrian (\textit{Ped}), Bus (\textit{Bus}), Trailer (\textit{Tra}), and Truck (\textit{Tru}). Notably, Poly-MOT, Fast-Poly, and our proposed Easy-Poly methods exclusively utilize keyframes for tracking tasks.

\subsection{Implementation Details}

% Our tracking method is implemented in Python under the Nvidia 4090X GPU.  Hyperparameters are chosen based on the best AMOTA identified in the validation set. SF thresholds are category-specific and detector-specific, which are (\textit{Bic}: 0.15; are (\textit{Car}: 0.16; are (\textit{Moto}: 0.16; \textit{Bus}: 0.12; \textit{Tra}: 0.13; \textit{Tru}: 0; \textit{Ped}: 0.13) on nuScenes on Waymo. The NMS thresholds are 0.08 on all categories and datasets. We also employ Scale-NMS~\cite{huang2021bevdet} on (\textit{Bic}, \textit{Ped}) on nuScenes. With default $IoU_{bev}$ in NMS, we additionally utilize our proposed $A\text{-}gIoU_{bev}$ to describe similarity for (\textit{Bic}, \textit{Ped}, \textit{Bus}, \textit{Tru}) on nuScenes. The motion models and filters are consistent with~\cite{li2023poly}. The lightweight filter is implemented by the median filter with $l_{lw}=5$ on all datasets. The association metrics are all implemented by $A\text{-}gIoU$ on all datasets. The first association thresholds $\theta_{fm}$ are category-specific, which are (\textit{Bic, Moto, Bus}: 1.6; \textit{Car, Tru}: 1.2; \textit{Tra}: 1.16;\textit{Ped}: 1.78) on nuScenes. Voxel mask size $\theta_{vm}$ is 5\textit{m} on nuScenes.  The count-based and output file strategies are consistent with~\cite{li2023poly}. In the confidence-based part, decay rates $\sigma$ are category-specific, which are (\textit{Ped}: 0.18; \textit{Car}: 0.26; \textit{Tru, Moto}: 0.28; \textit{Tra}: 0.22; \textit{Bic,Bus}: 0.24) on nuScenes. The delete threshold $\theta_{dl}$ are (\textit{Bus}: 0.08,  \textit{Ped}: 0.1 and 0.04 for other categories) on nuScenes.

Our tracking framework is implemented in Python and executed on an Nvidia 4090X GPU. Hyperparameters are optimized based on the highest AMOTA achieved on the validation set. The following category-specific and SF thresholds are employed for nuScenes: (\textit{Bic}: 0.15; \textit{Car}: 0.16; \textit{Moto}: 0.16; \textit{Bus}: 0.12; \textit{Tra}: 0.13; \textit{Tru}: 0; \textit{Ped}: 0.13). NMS thresholds are uniformly set to 0.08 across all categories and datasets. Additionally, we implement Scale-NMS~\cite{huang2021bevdet} for (\textit{Bic}, \textit{Ped}) categories on nuScenes.
In conjunction with the default in NMS, we introduce our novel  metric to enhance similarity assessment for (\textit{Bic}, \textit{Ped}, \textit{Bus}, \textit{Tru}) categories on nuScenes. Motion models and filters are consistent with those described in~\cite{li2023poly}. A lightweight filter, implemented as a median filter with , is applied across all datasets. Association metrics universally employ  across all datasets.
Category-specific first association thresholds  for nuScenes are as follows: (\textit{Bic, Moto, Bus}: 1.6; \textit{Car, Tru}: 1.2; \textit{Tra}: 1.16; \textit{Ped}: 1.78). The voxel mask size  is set to 5\textit{m} on nuScenes. Count-based and output file strategies align with those presented in~\cite{li2023poly}.
In the confidence-based component, category-specific decay rates for nuScenes are: (\textit{Ped}: 0.18; \textit{Car}: 0.26; \textit{Tru, Moto}: 0.28; \textit{Tra}: 0.22; \textit{Bic, Bus}: 0.24). The delete thresholds  are set as follows: (\textit{Bus}: 0.08, \textit{Ped}: 0.1, and 0.04 for all other categories) on nuScenes.

In the object tracking phase of 3D MOT, Easy-Poly exhibits exceptional performance following a series of optimizations. These enhancements include pre-processing, Kalman filtering, motion modeling, and tracking cycle refinements. The integration of these techniques significantly improves the algorithm's effectiveness in complex 3D environments.

Easy-Poly exhibits exceptional performance on the test set, achieving a \textbf{75.9\%} AMOTA score, surpassing the majority of existing 3D MOT methods. As shown in Table \ref{table:nu_test}, Easy-Poly attains a remarkably low IDS count of \textbf{287} while maintaining the highest AMOTA (\textbf{75.9\%}) among all modal methods. This underscores Easy-Poly's ability to maintain stable tracking without compromising recall. Notably, Easy-Poly achieves state-of-the-art performance without relying on additional image data input. Easy-Poly significantly outperforms competing algorithms in the critical 'Car' category. With minimal computational overhead, it delivers impressive results, highlighting its potential for integration into real-world autonomous driving systems. The False Negative and False Positive metrics in Table \ref{table:nu_test} further demonstrate Easy-Poly's robust continuous tracking capability while maintaining high recall.

For validation set experiments in Table \ref{table:nu_val}, we utilize CenterPoint~\cite{yin2021center} as the detector to ensure fair comparisons. As illustrated in Table \ref{table:nu_val}, Easy-Poly significantly outperforms most deep learning-based methods in both tracking accuracy (\textbf{75.0\%} AMOTA, \textbf{64.8\%} MOTA) and computational efficiency (\textbf{34.9} FPS). Compared to the baseline FastPoly~\cite{li2024fast}, Easy-Poly achieves substantial improvements of \textbf{+1.3\%} in MOTA and \textbf{+1.6\%} in AMOTA, while operating \textbf{1.5x} faster under identical conditions. When integrated with the high-performance LargeKernel3D~\cite{chen2022scaling} detector, Easy-Poly demonstrates even more impressive detection and tracking capabilities. Furthermore, when employing the multi-camera detector \textcolor{black}{DETR3D}~\cite{DETR3D} with constrained performance, Easy-Poly exhibits robust real-time performance without compromising accuracy. Furthermore, the lower AMOTP and IDS metrics demonstrate Easy-Poly's exceptional capability in tracking small objects and maintaining performance in complex scenarios and adverse weather conditions. These results underscore the algorithm's robustness across diverse and challenging environments.
% It is noteworthy that achieving optimal AMOTA necessitates a stringent score filter threshold, which marginally reduces the latency advantage of our method. Nevertheless, Easy-Poly maintains a favorable balance between accuracy and computational efficiency.

\begin{table}
% \vspace{0.5em}
        \caption{Comparing different data association algorithms using CenterPoint and (lines 1-7) and LargeKernel3D (lines 8-11) methods on nuScenes val set. Among them, the algorithms lines 1-3 are the Fast-Poly framework and in lines 4-11 are the latest our Easy-Poly framework.}

        \label{table:nus_assoc}
        % \renewcommand{\arraystretch}{0.7}
        \setlength{\tabcolsep}{0.1mm}
        \begin{tabular}{cccccc}
        \toprule
        \multicolumn{1}{c}{\textbf{Algorithms}} & \textbf{MOTA}$\uparrow$ & \textbf{AMOTA}$\uparrow$ & \textbf{AMOTP}$\downarrow$ & \textbf{IDS}$\downarrow$ & \textbf{FN}$\downarrow$\\ 
        
        \midrule
         %  Mutual Nearest Neighbor (MNN)
         MNN  & 62.2  & 72.5 & 52.4  & 433 & 16644  \\  
         Greedy    & 62.3  & 72.7 &  53.4  & 428  & 17647 \\
         Hungarian & \textbf{\textcolor{blue}{63.2}}  & \textbf{\textcolor{blue}{73.7}} & \textbf{\textcolor{blue}{52.1}} & \textbf{\textcolor{blue}{414}}  & \textbf{\textcolor{blue}{15996}}  \\
        
        \midrule  
        % 多模态+数据增强
         %  Mutual Nearest Neighbor (MNN)
        \textbf{MNN (Ours)}  & 63.7  & 73.6 & 54.8  & 406 &  \textbf{\textcolor{red}{15873 }}\\  
        \textbf{Greedy (Ours)}    &  64.0  & 73.7  &  54.6  & 368  & 16736 \\
        \textbf{Hungarian (Ours)}  & 64.3  & 74.3 & \textbf{\textcolor{red}{54.3}} &  335 & 16892  \\

        \textbf{DTO (Ours)}   & \textbf{\textcolor{red}{64.4}}  &   \textbf{\textcolor{red}{74.5}} & 54.9 &  \textbf{\textcolor{red}{272}} & 16982  \\

        
         \midrule
         \textbf{MNN (Ours)}  & 64.1  & 73.9 & 54.0  & 370 & 15865  \\  
         \textbf{Greedy (Ours)}    & 64.5  & 74.3 & 53.7  & 307  & 16014 \\
        \textbf{Hungarian (Ours)}  & 64.7  & 74.8 & 53.9  & 291  & 15923 \\

        \textbf{DTO (Ours)}  & \textbf{\textcolor{red}{64.8}}  & \textbf{\textcolor{red}{75.0}} & \textbf{\textcolor{red}{53.6}}  & \textbf{\textcolor{red}{242}}  & \textbf{\textcolor{red}{15488}} \\

    \bottomrule
\end{tabular}
\end{table}



% \begin{table}
% \caption{The ablation study of whether or not to use Score Filter and Non-Maximum Suppression, including the Run-Time, which represents the execution time of the Pre-processing Module. We compared Poly-MOT~\cite{li2023poly} (rows 1-3) with our proposed Easy-Poly method (rows 4-6).}
% \label{table:nu_NMSsf}
% % \renewcommand{\arraystretch}{0.7}
% \setlength{\tabcolsep}{2.7mm}
% {
% \begin{tabular}{cccc}
% \toprule

% \textbf{Variable} & \textbf{AMOTA$\uparrow$} & \textbf{IDS$\downarrow$} &
% \textbf{Run-Time (s) $\downarrow$}
% \\ 
% \midrule
% NMS + SF & \textbf{\textcolor{blue}{73.1}}   & \textbf{\textcolor{blue}{281}}  & 0.055    \\
% NMS     & 71.8  & 320  & 0.093     \\
% SF       & 68.6  & 354  & \textbf{\textcolor{blue}{0.008}}     \\ 
% \midrule
% \textbf{NMS + SF (Ours)} & \textbf{\textcolor{red}{75.0}}   & \textbf{\textcolor{red}{242}}  & 0.037    \\
% \textbf{NMS (Ours)}      & 73.6   & 273  & 0.068    \\
% \textbf{SF (Ours)}       & 71.2   & 308  & \textbf{\textcolor{red}{0.008}}     \\\hline
% % \vspace{-3.2em}
% \end{tabular}}
% \end{table}



\subsection{Comparative Evaluations} 
\label{sec: Comparative}

% Our proposed method, Easy-Poly, achieves state-of-the-art performance on the nuScenes validation set, demonstrating \textbf{75.0\%} AMOTA at \textbf{34.9} FPS, surpassing existing approaches. Utilizing an identical detector, Easy-Poly outperforms Fast-Poly \cite{li2024fast} across nearly all key metrics, with notable improvements in accuracy (\textbf{+1.3\%} AMOTA, \textbf{+1.6\%} MOTA) and speed (\textbf{+6.0 FPS}). While marginally slower than CBMOT \cite{benbarka2021score} and 3DMOTFormer \cite{ding20233dmotformer}, Easy-Poly significantly exceeds their accuracy while maintaining robust real-time performance. Our open-source implementation establishes a strong baseline for 3D MOT, providing a solid foundation for future advancements in the field.

In this study, we conduct a comprehensive evaluation of the association stage, focusing on four algorithms: Hungarian, Greedy, MNN, and the novel DTO. Our extensive experiments, summarized in Table~\ref{table:nus_assoc}, reveal that the Easy-Poly consistently outperforms Fast-Poly across both CenterPoint and LargeKernel3D frameworks. Notably, LargeKernel3D demonstrates superior performance over CenterPoint, particularly in complex tracking scenarios. Among the association algorithms, Hungarian and DTO consistently yield superior results, underscoring their robustness and efficacy in diverse multi-object tracking contexts. Compared to the Hungarian algorithm, DTO not only achieves similarly excellent AMOTA values but also provides more robust and accurate tracking performance, especially in challenging scenarios involving occlusions, missed detections, or false positives. These findings highlight the critical role of algorithm selection and model optimization in advancing the state-of-the-art in 3D object tracking.

\begin{figure*}[t]
    \centering
    \includegraphics[width=0.95\linewidth]{Images/Ablation_study_line_chart_for_NMS.pdf}
    \vspace{-8pt}
    \caption
    {
      The ablation study of whether or not to use Score Filter and Non-Maximum Suppression, including the Run-Time, which represents the execution time of the Pre-processing Module. We compared Poly-MOT with our proposed Easy-Poly method.
     }
     \Description{}
    \label{fig: NMSFS}
\end{figure*}


\textcolor{black}{Table \ref{table:nu_life} demonstrates the efficacy of our proposed methods. The Fast-Poly tracklet termination strategy (line 3) significantly outperforms baseline score refinement~\cite{benbarka2021score} (line 2), yielding a \textbf{2.7\%} improvement in AMOTA and reducing FN by \textbf{2366}. This enhancement mitigates tracker vulnerabilities in mismatch scenarios, including occlusions. Further performance gains are achieved through smoother score prediction (line 4), resulting in additional improvements of \textbf{0.4\%} AMOTA, \textbf{0.1\%} MOTA, and a reduction of \textbf{926 FN}.}
Our Easy-Poly model exhibits even more substantial performance enhancements. The tracklet termination strategy (line 7) surpasses the baseline score refinement (line 6) by \textbf{2.8\%} in AMOTA while decreasing FN by \textbf{1844}. Furthermore, the integration of smoother score prediction (line 8) further boosts the tracking performance, resulting in improvements of \textbf{0.4\%} in AMOTA and \textbf{0.6\%} in MOTA, along with a reduction of \textbf{575} FN.


\begin{table}
% \vspace{0.5em}
        \caption{A comparison on distinct life-cycle modules on nuScenes val set.
        \textbf{Average} means using the online average score to delete.
        \textbf{Latest} means using the latest score to delete.
        \textbf{Max-age} means using the continuous mismatch time to delete.
        Other settings are under the best performance. Methods in lines 1-4 and lines 5-8 use Fast-Poly~\cite{li2024fast} and Our Easy-Poly.
        }
        \label{table:nu_life}
        % \renewcommand{\arraystretch}{0.7}
        \setlength{\tabcolsep}{0.1mm}
        \begin{tabular}{ccccc}
        \toprule
        \multicolumn{1}{c}{\textbf{Strategy}} & \textbf{AMOTA}$\uparrow$ & \textbf{MOTA}$\uparrow$ & \textbf{FPS}$\uparrow$ & \textbf{FN}$\downarrow$\\ \midrule
 
         Count                \& Max-age            & 73.3      & 62.9     & 23.0  & 16523    \\
         %predict:normal update:normal+delete = -100
         Confidence~\cite{benbarka2021score} \& Latest       & 70.6      & 63.2     & \textbf{\textcolor{blue}{45.8}}  & 19192   \\%predict:minus update:multi+latest
         Confidence~\cite{benbarka2021score} \& Average      & 73.3      & 63.1     & 28.3 & 16826   \\%predict:minus update:multi+average
         Confidence  \& Average      & \textbf{\textcolor{blue}{73.7}}      & \textbf{\textcolor{blue}{63.2}}     & 28.9  & \textbf{\textcolor{blue}{15900}}   \\%predict:normal update:multi+average 27.7or28.9
         \midrule
         \textbf{Count \& Max-age (Ours)}          & 74.3      & 63.8     & 34.3  & 16024    \\
         %predict:normal update:normal+delete = -100
         \textbf{Confidence~\cite{benbarka2021score} \& Latest (Ours)}        & 71.8      & 63.7     & \textbf{\textcolor{red}{50.2}}  & 17907   \\%predict:minus update:multi+latest
         \textbf{Confidence~\cite{benbarka2021score} \& Average (Ours)}       & 74.6      & 64.2     & 35.6 & 16063   \\%predict:minus update:multi+average
         \textbf{Confidence \& Average (Ours)}      & \textbf{\textcolor{red}{75.0}}      & \textbf{\textcolor{red}{64.8}}     & 36.9  & \textbf{\textcolor{red}{15488}}   \\%predict:normal update:multi+average 36.9
        \bottomrule
        \end{tabular}
    \end{table}


\subsection{Ablation Studies}

% The Sf can filter out low-score bounding boxes while NMS can remove duplicate bounding boxes with high confidence, which makes the remaining bounding boxes have superior quality. For the Poly-MOT, using SF before NMS brings inference 40\% reduction in pre-processing inference time while boosting AMOTA by 1.3\% compared with only using NMS. The Fast-Poly has better performance, using SF before NMS brings inference 50\% reduction in pre-processing inference time while boosting AMOTA by 1.4\% compared with only using NMS, as demonstrate in Table~\ref{table:nu_NMSsf}.  It is clear from the table that when only SF is used without NMS, although the running time is fast, both AMOTA and IDS values become lower and the values drop very significantly.

We optimize two-stage filtering approach that synergistically combines SF and NMS to significantly enhance bounding box quality in multi-object tracking scenarios. This novel method effectively integrates SF to eliminate low-score detections and NMS to remove high-confidence duplicates, resulting in a set of superior quality bounding boxes. Our comprehensive experimental results, presented in Figure~\ref{fig: NMSFS}, demonstrate substantial improvements in both computational efficiency and tracking accuracy across multiple state-of-the-art models. For the Poly-MOT model, our approach of applying SF before NMS yields a remarkable \textbf{40\%} reduction in pre-processing inference time while simultaneously improving AMOTA by \textbf{1.3\%} compared to using NMS alone. These results highlight the significant potential of our method in improving real-time tracking capabilities without compromising accuracy. The Easy-Poly model exhibits even more impressive performance gains, further validating the scalability and effectiveness of our approach. By applying SF before NMS, we achieve a substantial \textbf{50\%} reduction in pre-processing time coupled with a \textbf{1.4\%} improvement in AMOTA. This notable improve in both speed and accuracy underscores the robustness of our method across different model architectures. Importantly, our analysis reveals critical insight into the interplay between SF and NMS. Although using SF without NMS accelerates processing, it leads to significant degradation in both AMOTA and Identity Switches IDS metrics. This observation underscores the crucial importance of our combined SF-NMS approach in maintaining an optimal balance between processing speed and tracking accuracy. The synergistic effect of SF and NMS not only enhances the quality of bounding boxes but also optimizes the trade-off between computational efficiency and tracking performance. This balance is particularly vital in real-world applications where both speed and accuracy are paramount, such as autonomous driving and surveillance systems. 
% The consistent improvements observed across different models suggest broad applicability and potential for integration into various tracking frameworks, paving the way for more efficient and accurate tracking systems in complex, real-world environments.


% \subsection{Visualization}

\section{Conclusion}
In this study, we introduce \ours, a novel framework designed to achieve lossless acceleration in generating ultra-long sequences with \acp{llm}. By analyzing and addressing three challenges, \ours significantly enhances the efficiency of the generation process. Our experimental results demonstrate that \ours achieves over $3\times$ acceleration across various model scales and architectures. Furthermore, \ours effectively mitigates issues related to repetitive content, ensuring the quality and coherence of the generated sequences. These advancements position \ours as a scalable and effective solution for ultra-long sequence generation tasks.

% 
\onecolumn

\section{Additional Experimental Settings}
\label{sec:appendix:exp_setting}

\subsection{Hyper-parameters}
The default sampling parameters for LLM generation are \{temperature=0.75, top\_k=40, top\_p=0.95\}.
Across all MCTS experiments, we set the number of rollouts to 8, the number of sub-questions per node to 4, the number of candidate sub-answers per sub-question to 4, the maximum depth allowed to 5.

\subsection{Dataset Statistics}
Since we make use of the rStar code base~\footnote{\url{https://github.com/zhentingqi/rStar/}} to implement \mosa, we directly adopt the data files released in their git repository.
There are 1,319 instances in GSM8K, 1,000 instances in SVAMP, 500 instances in MATH-500, and 687 instances in StrategyQA.

\section{Additional Implementation Details for Aggregators}
\label{sec:appendix:prompt}

In this section, we will show the basic instruction and several in-context learning demonstrations for aggregators.

% The aggregating operation that was introduced in the main content of this paper is mainly conducted within each search step, which is termed as Action Aggregation.
% Apart from it, we also aggregate all search trajectories after the entire search process terminates, which is named as Final Aggregation.
% In this section, we first give an ablation for the two types of aggregations and show that Action Aggregation is more essential than the other.
% Then, we will show the aggregating prompts and several in-context learning demonstrations.

% \subsection{Ablation on Aggregating}

% \begin{table*}[t]
% \caption{
% }
% \label{tab:appendix:action_or_final_agg}
% \vskip 0.15in
% \begin{center}
% \begin{small}
% % \begin{sc}
% \begin{tabular}{l|cccc|cr}
% \toprule
% Method & GSM8K & SVAMP & MATH & StrategyQA & Avg. \\
% \midrule
% \textsc{MoSA} as Proposers \& Aggregators  &  91.96 &  94.90 &  56.60 &  76.42 & 79.97  \\
% \hspace{10pt} $\circ$ \emph{without} Action Aggregation & 91.51 & 95.20 & 54.40 & 75.69 & 79.20 \\
% \hspace{10pt} $\circ$ \emph{without} Final Aggregation & 91.96 & 94.60 & 56.20 & 75.69 & 79.61 \\
% \bottomrule
% \end{tabular}
% % \end{sc}
% \end{small}
% \end{center}
% \vskip -0.1in
% \end{table*}

% As shwon in Table~\ref{tab:appendix:action_or_final_agg}, removing Action Aggregation leads to an average decrease of 0.77\% while that of removing Final Aggregation is only 0.36\%.
% This suggests that aggregation during each search step is more essential than aggregation after the search terminates.

% \subsection{Prompt Templates for Aggregators}

% % \paragraph{Basic Prompt}


% \textbf{Basic Prompt for Action Aggregation}

% \begin{lstlisting}
% **TASK**:\nYou are an intelligent and supportive AI assistant. You will receive a collection of responses from various AI assistants regarding a query. Your goal is to synthesize these responses into a single, high-quality response. You should first write down your thoughts as \"\#\#\# Thoughts: \textless{}THOUGHTS\textgreater{}.\", and then write down the synthesized response in the format of \"\#\#\# Synthesized Response: \textless{}SYNTHESIZED RESPONSE\textgreater{}.\". If \"The answer is: \textless{}ANSWER\textgreater{}\" appears in the set of responses, you should also write down \"The answer is: \textless{}ANSWER\textgreater{}\"; otherwise, do not write it. \n1. Critically assess the information in the responses, as some may be biased, incorrect, or nonsensical. \n2. Your synthesized response should not merely echo the provided answers but should present a refined, accurate, and thorough response to the prompt. \n3.Ensure that your response is well-organized, coherent, and meets the highest standards of accuracy and reliability.\n
% \end{lstlisting}

% \textbf{Basic Prompt for Final Aggregation}

% \begin{lstlisting}
% **TASK**:\nYou are an intelligent and supportive AI assistant. You will receive a collection of responses from various AI assistants regarding a user's query. Your goal is to synthesize these responses into a single, high-quality response. You should write down your thoughts, and then write down the final answer in the format of \"The answer is: \textless{}ANSWER\textgreater{}.\".\n1. Critically assess the information in the collection of responses, as some may be biased, incorrect, or nonsensical. \n2. Your synthesized response should not merely echo the provided responses but should present a refined, accurate, and thorough response to the user's query. \n3.Ensure that your response is well-organized, coherent, and meets the highest standards of accuracy and reliability.\n
% \end{lstlisting}

% \subsection{ICL Demonstrations for Aggregators}

% \textbf{An ICL Demonstration for Action Aggregation}

% \begin{lstlisting}
% \end{lstlisting}

\newpage

\begin{center}
	\small
	\begin{tcolorbox}[width=1\linewidth,title={\textbf{Basic Instruction for Aggregators}}]
		**TASK**:
        
        You are an intelligent and supportive AI assistant. You will receive a collection of responses from various AI assistants regarding a query. Your goal is to synthesize these responses into a single, high-quality response. You should first write down your thoughts as "\#\#\# Thoughts: \textless{}THOUGHTS\textgreater{}.", and then write down the synthesized response in the format of "\#\#\# Synthesized Response: \textless{}SYNTHESIZED RESPONSE\textgreater{}.". If "The answer is: \textless{}ANSWER\textgreater{}" appears in the set of responses, you should also write down "The answer is: \textless{}ANSWER\textgreater{}"; otherwise, do not write it. 
        
        1. Critically assess the information in the responses, as some may be biased, incorrect, or nonsensical. 
        
        2. Your synthesized response should not merely echo the provided answers but should present a refined, accurate, and thorough response to the prompt. 
        
        3.Ensure that your response is well-organized, coherent, and meets the highest standards of accuracy and reliability.
	\end{tcolorbox}
\end{center}

\newpage


\begin{center}
	\small
	\begin{tcolorbox}[width=1\linewidth,title={\textbf{An Example ICL Demonstration of MATH-500 for Aggregators: User Query}}]

\textbf{User:}


\textbf{[Here is the user query that was fed into those AI assistants]:}

Given a question, please decompose it into sub-questions. For each sub-question, please answer it in a complete sentence, ending with "The answer is \textless a numeric answer \textgreater". When the original question is answerable, please start the subquestion with "Now we can answer the question: \textless{}original question\textgreater{}".

Question 5: Find the product $CD$ of the integers $C$ and $D$ for which

\[\frac{C}{x-3}+\frac{D}{x+8}=\frac{4x-23}{x^2+5x-24}\]for all real values of $x$ except $-8$ and $3$.

Question 5.1: What is the equation obtained by multiplying both sides of the given equation by $(x - 3)(x + 8)$?

Answer 5.1: $(x - 3)(x + 8) \cdot \frac{C}{x-3} + (x - 3)(x + 8) \cdot \frac{D}{x+8} = (x - 3)(x + 8) \cdot \frac{4x-23}{x^2+5x-24}$

This simplifies to $C(x + 8) + D(x - 3) = 4x - 23$.

The answer is $C(x + 8) + D(x - 3) = 4x - 23$.

Question 5.2: What are the values of $C$ and $D$ that satisfy the equation $C(x + 8) + D(x - 3) = 4x - 23$ for all $x$ except $-8$ and $3$?

Answer 5.2:


\textbf{[Here are the responses from various AI assistants]:}

Response 1:

To find the values of $C$ and $D$, we can substitute specific values of $x$ into the equation $C(x + 8) + D(x - 3) = 4x - 23$ to obtain a system of equations. 

Question 5.2.1: What happens if we substitute $x = 3$ into the equation $C(x + 8) + D(x - 3) = 4x - 23$?

Answer 5.2.1: Substituting $x = 3$ gives $C(3 + 8) + D(3 - 3) = 4 \cdot 3 - 23$, which simplifies to $11C = -11$. The answer is $11C = -11$.

Question 5.2.2: What happens if we substitute $x = -8$ into the equation $C(x + 8) + D(x - 3) = 4x - 23$?

Answer 5.2.2: Substituting $x = -8$ gives $C(-8 + 8) + D(-8 - 3) = 4 \cdot (-8) - 23$, which simplifies to $-11D = -59$. The answer is $-11D = -59$.

Question 5.2.3: Now we can answer the question: What are the values of $C$ and $D$?

Answer 5.2.3: Solving the system of equations $11C = -11$ and $-11D = -59$ gives $C = -1$ and $D = 7/11 \cdot 11 = 7$. The answer is -1 and 7 respectively.

Response 2:

To find the values of $C$ and $D$, we can compare the coefficients of the corresponding terms on both sides of the equation.

For the $x^2$ terms, we have $C + D = 0$.

For the $x$ terms, we have $8C - 3D = 4$.

Solving these equations simultaneously, we find that $C = 3$ and $D = -3$.

Therefore, the product $CD = 3 \cdot (-3) = \boxed{-9}$.

Response 3:

To find $C$ and $D$, we can substitute $x = -8$ and $x = 3$ into the equation $C(x + 8) + D(x - 3) = 4x - 23$.

For $x = -8$:
\[C(-8 + 8) + D(-8 - 3) = 4(-8) - 23\]
\[-11D = -55\]
\[D = 5\]

For $x = 3$:
\[C(3 + 8) + D(3 - 3) = 4(3) - 23\]
\[11C = -13\]
\[C = -\frac{13}{11}\]

Now we can answer the question: What is the product $CD$ of the integers $C$ and $D$?
\[CD = \left(-\frac{13}{11}\right)(5) = -\frac{65}{11}\]
The answer is $-\frac{65}{11}$.

Response 4:

To find $C$ and $D$, we can set $x = 3$ and $x = -8$ to get two equations:

1. $C(3 + 8) + D(3 - 3) = 4(3) - 23 \Rightarrow 11C = -11 \Rightarrow C = -1$

2. $C(-8 + 8) + D(-8 - 3) = 4(-8) - 23 \Rightarrow -11D = -55 \Rightarrow D = 5$

The answer is $C = -1$ and $D = 5$.


	\end{tcolorbox}
\end{center}



\newpage


\begin{center}
	\small
	\begin{tcolorbox}[width=1\linewidth,title={\textbf{An Example ICL Demonstration of MATH-500 for Aggregators: Assistant Response}}]

\textbf{Assistant:}

\#\#\# Thoughts:

1. The responses generally attempt to solve the equation \(C(x + 8) + D(x - 3) = 4x - 23\) by substituting specific values for \(x\), namely \(x = 3\) and \(x = -8\).

2. The correct method involves creating a system of linear equations by substituting these values and solving for \(C\) and \(D\).

3. There is a discrepancy in the values of \(C\) and \(D\) obtained in the responses:

   - Some responses correctly find \(C = -1\) and \(D = 5\).
   
   - Another response incorrectly finds \(C = -13/11\) and \(D = 5\).
   
   - Another response finds \(C = 3\) and \(D = -3\), which is not consistent with the given equation.

4. The correct values should satisfy both equations derived from the substitutions.

\#\#\# Synthesized Response:
 
 The equation \(C(x + 8) + D(x - 3) = 4x - 23\) can be solved by substituting specific values for \(x\):

1. Substituting \(x = 3\):
\[C(3 + 8) + D(3 - 3) = 4(3) - 23\]
\[11C = -11 \]
\[C = -1\]

2. Substituting \(x = -8\):
\[C(-8 + 8) + D(-8 - 3) = 4(-8) - 23\]
\[-11D = -55 \]
\[D = 5\]

Thus, the values of \(C\) and \(D\) are \(C = -1\) and \(D = 5\). The product \(CD\) is:
\[CD = (-1) \cdot 5 = -5\]

The answer is: -5

	\end{tcolorbox}
\end{center}




\newpage





\begin{center}
	\small
	\begin{tcolorbox}[width=1\linewidth,title={\textbf{An Example ICL Demonstration of StrategyQA for Aggregators: User Query}}]

\textbf{User:}


\textbf{[Here is the user query that was fed into those AI assistants]:}

Given a question, please decompose it into sub-questions. For each sub-question, please answer it in a complete sentence. When the original question is answerable, please start the sub-question with "Now we can answer the question: \textless{}original question\textgreater{}".

Question 5: Johnny Carson: American talk show host and comedian. Could Johnny Carson's children fill out a water polo team?

Question 5.1: How many children did Johnny Carson have?

Answer 5.1: Johnny Carson had three children: two daughters, Catherine and Christie, and a son, Jeff.

Question 5.2: How many players are needed to form a water polo team?

Answer 5.2:

\textbf{[Here are the responses from various AI assistants]:}

Response 1:

A standard water polo team consists of 13 players.

Response 2:

A water polo team consists of seven players, including one goalkeeper and six field players.

Response 3:

A water polo team consists of 13 players: 1 goalkeeper and 12 field players (7 field players and 5 substitutes).

Response 4:

A standard water polo team consists of 13 players, including 7 field players and 6 substitutes.

	\end{tcolorbox}
\end{center}


\newpage



\begin{center}
	\small
	\begin{tcolorbox}[width=1\linewidth,title={\textbf{An Example ICL Demonstration of StrategyQA for Aggregators: Assistant Response}}]

\textbf{Assistant:}

\#\#\# Thoughts:

1. The responses contain valuable information about the composition of a water polo team. 

2. There is a consensus that a water polo team consists of 13 players in total. However, there is some variation in how these 13 players are categorized.

3. The key point to clarify is how many players are in the water at one time and how many are substitutes.

4. It appears that in the water during play, there are 7 players (including the goalkeeper), and the remaining players are substitutes.

\#\#\# Synthesized Response:

The synthesized response is: 

To form a water polo team, a total of 13 players are required. This team includes 7 players who are in the water during play, comprising one goalkeeper and six field players. The remaining 6 players serve as substitutes.

	\end{tcolorbox}
\end{center}
% % \acks{Acknowledgements go here.}

\balance
\bibliographystyle{ACM-Reference-Format}
\bibliography{reference}

% \appendix
% 
\onecolumn

\section{Additional Experimental Settings}
\label{sec:appendix:exp_setting}

\subsection{Hyper-parameters}
The default sampling parameters for LLM generation are \{temperature=0.75, top\_k=40, top\_p=0.95\}.
Across all MCTS experiments, we set the number of rollouts to 8, the number of sub-questions per node to 4, the number of candidate sub-answers per sub-question to 4, the maximum depth allowed to 5.

\subsection{Dataset Statistics}
Since we make use of the rStar code base~\footnote{\url{https://github.com/zhentingqi/rStar/}} to implement \mosa, we directly adopt the data files released in their git repository.
There are 1,319 instances in GSM8K, 1,000 instances in SVAMP, 500 instances in MATH-500, and 687 instances in StrategyQA.

\section{Additional Implementation Details for Aggregators}
\label{sec:appendix:prompt}

In this section, we will show the basic instruction and several in-context learning demonstrations for aggregators.

% The aggregating operation that was introduced in the main content of this paper is mainly conducted within each search step, which is termed as Action Aggregation.
% Apart from it, we also aggregate all search trajectories after the entire search process terminates, which is named as Final Aggregation.
% In this section, we first give an ablation for the two types of aggregations and show that Action Aggregation is more essential than the other.
% Then, we will show the aggregating prompts and several in-context learning demonstrations.

% \subsection{Ablation on Aggregating}

% \begin{table*}[t]
% \caption{
% }
% \label{tab:appendix:action_or_final_agg}
% \vskip 0.15in
% \begin{center}
% \begin{small}
% % \begin{sc}
% \begin{tabular}{l|cccc|cr}
% \toprule
% Method & GSM8K & SVAMP & MATH & StrategyQA & Avg. \\
% \midrule
% \textsc{MoSA} as Proposers \& Aggregators  &  91.96 &  94.90 &  56.60 &  76.42 & 79.97  \\
% \hspace{10pt} $\circ$ \emph{without} Action Aggregation & 91.51 & 95.20 & 54.40 & 75.69 & 79.20 \\
% \hspace{10pt} $\circ$ \emph{without} Final Aggregation & 91.96 & 94.60 & 56.20 & 75.69 & 79.61 \\
% \bottomrule
% \end{tabular}
% % \end{sc}
% \end{small}
% \end{center}
% \vskip -0.1in
% \end{table*}

% As shwon in Table~\ref{tab:appendix:action_or_final_agg}, removing Action Aggregation leads to an average decrease of 0.77\% while that of removing Final Aggregation is only 0.36\%.
% This suggests that aggregation during each search step is more essential than aggregation after the search terminates.

% \subsection{Prompt Templates for Aggregators}

% % \paragraph{Basic Prompt}


% \textbf{Basic Prompt for Action Aggregation}

% \begin{lstlisting}
% **TASK**:\nYou are an intelligent and supportive AI assistant. You will receive a collection of responses from various AI assistants regarding a query. Your goal is to synthesize these responses into a single, high-quality response. You should first write down your thoughts as \"\#\#\# Thoughts: \textless{}THOUGHTS\textgreater{}.\", and then write down the synthesized response in the format of \"\#\#\# Synthesized Response: \textless{}SYNTHESIZED RESPONSE\textgreater{}.\". If \"The answer is: \textless{}ANSWER\textgreater{}\" appears in the set of responses, you should also write down \"The answer is: \textless{}ANSWER\textgreater{}\"; otherwise, do not write it. \n1. Critically assess the information in the responses, as some may be biased, incorrect, or nonsensical. \n2. Your synthesized response should not merely echo the provided answers but should present a refined, accurate, and thorough response to the prompt. \n3.Ensure that your response is well-organized, coherent, and meets the highest standards of accuracy and reliability.\n
% \end{lstlisting}

% \textbf{Basic Prompt for Final Aggregation}

% \begin{lstlisting}
% **TASK**:\nYou are an intelligent and supportive AI assistant. You will receive a collection of responses from various AI assistants regarding a user's query. Your goal is to synthesize these responses into a single, high-quality response. You should write down your thoughts, and then write down the final answer in the format of \"The answer is: \textless{}ANSWER\textgreater{}.\".\n1. Critically assess the information in the collection of responses, as some may be biased, incorrect, or nonsensical. \n2. Your synthesized response should not merely echo the provided responses but should present a refined, accurate, and thorough response to the user's query. \n3.Ensure that your response is well-organized, coherent, and meets the highest standards of accuracy and reliability.\n
% \end{lstlisting}

% \subsection{ICL Demonstrations for Aggregators}

% \textbf{An ICL Demonstration for Action Aggregation}

% \begin{lstlisting}
% \end{lstlisting}

\newpage

\begin{center}
	\small
	\begin{tcolorbox}[width=1\linewidth,title={\textbf{Basic Instruction for Aggregators}}]
		**TASK**:
        
        You are an intelligent and supportive AI assistant. You will receive a collection of responses from various AI assistants regarding a query. Your goal is to synthesize these responses into a single, high-quality response. You should first write down your thoughts as "\#\#\# Thoughts: \textless{}THOUGHTS\textgreater{}.", and then write down the synthesized response in the format of "\#\#\# Synthesized Response: \textless{}SYNTHESIZED RESPONSE\textgreater{}.". If "The answer is: \textless{}ANSWER\textgreater{}" appears in the set of responses, you should also write down "The answer is: \textless{}ANSWER\textgreater{}"; otherwise, do not write it. 
        
        1. Critically assess the information in the responses, as some may be biased, incorrect, or nonsensical. 
        
        2. Your synthesized response should not merely echo the provided answers but should present a refined, accurate, and thorough response to the prompt. 
        
        3.Ensure that your response is well-organized, coherent, and meets the highest standards of accuracy and reliability.
	\end{tcolorbox}
\end{center}

\newpage


\begin{center}
	\small
	\begin{tcolorbox}[width=1\linewidth,title={\textbf{An Example ICL Demonstration of MATH-500 for Aggregators: User Query}}]

\textbf{User:}


\textbf{[Here is the user query that was fed into those AI assistants]:}

Given a question, please decompose it into sub-questions. For each sub-question, please answer it in a complete sentence, ending with "The answer is \textless a numeric answer \textgreater". When the original question is answerable, please start the subquestion with "Now we can answer the question: \textless{}original question\textgreater{}".

Question 5: Find the product $CD$ of the integers $C$ and $D$ for which

\[\frac{C}{x-3}+\frac{D}{x+8}=\frac{4x-23}{x^2+5x-24}\]for all real values of $x$ except $-8$ and $3$.

Question 5.1: What is the equation obtained by multiplying both sides of the given equation by $(x - 3)(x + 8)$?

Answer 5.1: $(x - 3)(x + 8) \cdot \frac{C}{x-3} + (x - 3)(x + 8) \cdot \frac{D}{x+8} = (x - 3)(x + 8) \cdot \frac{4x-23}{x^2+5x-24}$

This simplifies to $C(x + 8) + D(x - 3) = 4x - 23$.

The answer is $C(x + 8) + D(x - 3) = 4x - 23$.

Question 5.2: What are the values of $C$ and $D$ that satisfy the equation $C(x + 8) + D(x - 3) = 4x - 23$ for all $x$ except $-8$ and $3$?

Answer 5.2:


\textbf{[Here are the responses from various AI assistants]:}

Response 1:

To find the values of $C$ and $D$, we can substitute specific values of $x$ into the equation $C(x + 8) + D(x - 3) = 4x - 23$ to obtain a system of equations. 

Question 5.2.1: What happens if we substitute $x = 3$ into the equation $C(x + 8) + D(x - 3) = 4x - 23$?

Answer 5.2.1: Substituting $x = 3$ gives $C(3 + 8) + D(3 - 3) = 4 \cdot 3 - 23$, which simplifies to $11C = -11$. The answer is $11C = -11$.

Question 5.2.2: What happens if we substitute $x = -8$ into the equation $C(x + 8) + D(x - 3) = 4x - 23$?

Answer 5.2.2: Substituting $x = -8$ gives $C(-8 + 8) + D(-8 - 3) = 4 \cdot (-8) - 23$, which simplifies to $-11D = -59$. The answer is $-11D = -59$.

Question 5.2.3: Now we can answer the question: What are the values of $C$ and $D$?

Answer 5.2.3: Solving the system of equations $11C = -11$ and $-11D = -59$ gives $C = -1$ and $D = 7/11 \cdot 11 = 7$. The answer is -1 and 7 respectively.

Response 2:

To find the values of $C$ and $D$, we can compare the coefficients of the corresponding terms on both sides of the equation.

For the $x^2$ terms, we have $C + D = 0$.

For the $x$ terms, we have $8C - 3D = 4$.

Solving these equations simultaneously, we find that $C = 3$ and $D = -3$.

Therefore, the product $CD = 3 \cdot (-3) = \boxed{-9}$.

Response 3:

To find $C$ and $D$, we can substitute $x = -8$ and $x = 3$ into the equation $C(x + 8) + D(x - 3) = 4x - 23$.

For $x = -8$:
\[C(-8 + 8) + D(-8 - 3) = 4(-8) - 23\]
\[-11D = -55\]
\[D = 5\]

For $x = 3$:
\[C(3 + 8) + D(3 - 3) = 4(3) - 23\]
\[11C = -13\]
\[C = -\frac{13}{11}\]

Now we can answer the question: What is the product $CD$ of the integers $C$ and $D$?
\[CD = \left(-\frac{13}{11}\right)(5) = -\frac{65}{11}\]
The answer is $-\frac{65}{11}$.

Response 4:

To find $C$ and $D$, we can set $x = 3$ and $x = -8$ to get two equations:

1. $C(3 + 8) + D(3 - 3) = 4(3) - 23 \Rightarrow 11C = -11 \Rightarrow C = -1$

2. $C(-8 + 8) + D(-8 - 3) = 4(-8) - 23 \Rightarrow -11D = -55 \Rightarrow D = 5$

The answer is $C = -1$ and $D = 5$.


	\end{tcolorbox}
\end{center}



\newpage


\begin{center}
	\small
	\begin{tcolorbox}[width=1\linewidth,title={\textbf{An Example ICL Demonstration of MATH-500 for Aggregators: Assistant Response}}]

\textbf{Assistant:}

\#\#\# Thoughts:

1. The responses generally attempt to solve the equation \(C(x + 8) + D(x - 3) = 4x - 23\) by substituting specific values for \(x\), namely \(x = 3\) and \(x = -8\).

2. The correct method involves creating a system of linear equations by substituting these values and solving for \(C\) and \(D\).

3. There is a discrepancy in the values of \(C\) and \(D\) obtained in the responses:

   - Some responses correctly find \(C = -1\) and \(D = 5\).
   
   - Another response incorrectly finds \(C = -13/11\) and \(D = 5\).
   
   - Another response finds \(C = 3\) and \(D = -3\), which is not consistent with the given equation.

4. The correct values should satisfy both equations derived from the substitutions.

\#\#\# Synthesized Response:
 
 The equation \(C(x + 8) + D(x - 3) = 4x - 23\) can be solved by substituting specific values for \(x\):

1. Substituting \(x = 3\):
\[C(3 + 8) + D(3 - 3) = 4(3) - 23\]
\[11C = -11 \]
\[C = -1\]

2. Substituting \(x = -8\):
\[C(-8 + 8) + D(-8 - 3) = 4(-8) - 23\]
\[-11D = -55 \]
\[D = 5\]

Thus, the values of \(C\) and \(D\) are \(C = -1\) and \(D = 5\). The product \(CD\) is:
\[CD = (-1) \cdot 5 = -5\]

The answer is: -5

	\end{tcolorbox}
\end{center}




\newpage





\begin{center}
	\small
	\begin{tcolorbox}[width=1\linewidth,title={\textbf{An Example ICL Demonstration of StrategyQA for Aggregators: User Query}}]

\textbf{User:}


\textbf{[Here is the user query that was fed into those AI assistants]:}

Given a question, please decompose it into sub-questions. For each sub-question, please answer it in a complete sentence. When the original question is answerable, please start the sub-question with "Now we can answer the question: \textless{}original question\textgreater{}".

Question 5: Johnny Carson: American talk show host and comedian. Could Johnny Carson's children fill out a water polo team?

Question 5.1: How many children did Johnny Carson have?

Answer 5.1: Johnny Carson had three children: two daughters, Catherine and Christie, and a son, Jeff.

Question 5.2: How many players are needed to form a water polo team?

Answer 5.2:

\textbf{[Here are the responses from various AI assistants]:}

Response 1:

A standard water polo team consists of 13 players.

Response 2:

A water polo team consists of seven players, including one goalkeeper and six field players.

Response 3:

A water polo team consists of 13 players: 1 goalkeeper and 12 field players (7 field players and 5 substitutes).

Response 4:

A standard water polo team consists of 13 players, including 7 field players and 6 substitutes.

	\end{tcolorbox}
\end{center}


\newpage



\begin{center}
	\small
	\begin{tcolorbox}[width=1\linewidth,title={\textbf{An Example ICL Demonstration of StrategyQA for Aggregators: Assistant Response}}]

\textbf{Assistant:}

\#\#\# Thoughts:

1. The responses contain valuable information about the composition of a water polo team. 

2. There is a consensus that a water polo team consists of 13 players in total. However, there is some variation in how these 13 players are categorized.

3. The key point to clarify is how many players are in the water at one time and how many are substitutes.

4. It appears that in the water during play, there are 7 players (including the goalkeeper), and the remaining players are substitutes.

\#\#\# Synthesized Response:

The synthesized response is: 

To form a water polo team, a total of 13 players are required. This team includes 7 players who are in the water during play, comprising one goalkeeper and six field players. The remaining 6 players serve as substitutes.

	\end{tcolorbox}
\end{center}


\end{document}
