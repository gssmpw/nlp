%% The first command in your LaTeX source must be the \documentclass command.
% \documentclass[sigconf, anonymous]{acmart}
\documentclass[sigconf]{acmart}
% extra add
\usepackage{multirow}
\usepackage{booktabs}
\usepackage{amsthm}
% \usepackage {verbatim}
\usepackage{graphicx}
\usepackage{amsmath}
\usepackage{color,xcolor}
\usepackage{hyperref}

\definecolor{mygreen}{RGB}{34,200,34}
\definecolor{myyellow}{RGB}{255,200,0}
\usepackage[linesnumbered,ruled,vlined]{algorithm2e}

%%
%% \BibTeX command to typeset BibTeX logo in the docs
\AtBeginDocument{%
  \providecommand\BibTeX{{%
    \normalfont B\kern-0.5em{\scshape i\kern-0.25em b}\kern-0.8em\TeX}}}

%% Rights management information.  This information is sent to you
%% when you complete the rights form.  These commands have SAMPLE
%% values in them; it is your responsibility as an author to replace
%% the commands and values with those provided to you when you
%% complete the rights form.



\setcopyright{acmcopyright}
\copyrightyear{2025}
\acmYear{2025}
\setcopyright{acmlicensed}\acmConference[ICMR '25]{Proceedings of the 2025 International Conference on Multimedia Retrieval}{June 30-July 3, 2025}{Chicago, USA}

% \acmBooktitle{Proceedings of the 2025 International Conference on Multimedia Retrieval (ICMR '25), June 30--July 3, 2025, Chicago, USA}
% \acmDOI{10.1145/3652583.3658038}
% \acmISBN{979-8-4007-0619-6/24/06}

% 通讯作者标识

% \settopmatter{printacmref=true}
\begin{document}

%%
%% The "title" command has an optional parameter,
%% allowing the author to define a "short title" to be used in page headers.
% \title{MSVD: Multimodal Semantics Summarization of Video Data}
\title{Easy-Poly: A Easy Polyhedral Framework For 3D Multi-Object Tracking}

%%
%% The "author" command and its associated commands are used to define
%% the authors and their affiliations.
%% Of note is the shared affiliation of the first two authors, and the
%% "authornote" and "authornotemark" commands
%% used to denote shared contribution to the research.

% ################author###########
 

\author{Peng Zhang }
\email{52205901027@stu.ecnu.edu.cn}
% \orcid{0009-0009-6036-8687}
\affiliation{
  \institution{East China Normal University}
  \city{Shanghai}
  \country{China}
}

\author{Xin Li \footnotemark[1]}
\email{lx_cs@sjtu.edu.cn}
% \orcid{0009-0009-6036-8687}
% \authornote{corresponding author}
\affiliation{
  \institution{Shanghai Jiao Tong University}
  \city{Shanghai}
  \country{China}
}

\author{Xin Lin \footnotemark[1]}
\email{xlin@cs.ecnu.edu.cn}
% \orcid{0009-0009-6036-8687}
% \authornote{corresponding author}
\affiliation{
  \institution{East China Normal University}
  \city{Shanghai}
  \country{China}
}

\author{Liang He}
\email{lhe@cs.ecnu.edu.cn}
% \orcid{0009-0009-6036-8687}
\affiliation{
  \institution{East China Normal University}
  \city{Shanghai}
  \country{China}
}



%%
%% By default, the full list of authors will be used in the page
%% headers. Often, this list is too long, and will overlap
%% other information printed in the page headers. This command allows
%% the author to define a more concise list
%% of authors' names for this purpose.
% \renewcommand{\shortauthors}{Trovato and Tobin, et al.}

%%
%% The abstract is a short summary of the work to be presented in the
%% article.
\begin{abstract}
Out-of-distribution (OOD) detection and OOD generalization are widely studied in Deep Neural Networks (DNNs), yet their relationship remains poorly understood. We empirically show that the degree of Neural Collapse (NC) in a network layer is inversely related with these objectives: stronger NC improves OOD detection but degrades generalization, while weaker NC enhances generalization at the cost of detection. This trade-off suggests that a single feature space cannot simultaneously achieve both tasks. To address this, we develop a theoretical framework linking NC to OOD detection and generalization. We show that entropy regularization mitigates NC to improve generalization, while a fixed Simplex Equiangular Tight Frame (ETF) projector enforces NC for better detection. Based on these insights, we propose a method to control NC at different DNN layers. In experiments, our method excels at both tasks across OOD datasets and DNN architectures. 

\begin{comment}   

Out-of-distribution (OOD) detection and OOD generalization are critical for deploying machine learning models in real-world scenarios. While substantial progress has been made in addressing these problems independently, few works have attempted to tackle them jointly. However, existing methods often rely on auxiliary OOD training data and primarily focus on covariate-shifted OOD data that share labels with in-distribution (ID) data. In contrast, we tackle the more realistic and challenging task of jointly detecting and generalizing to semantic OOD data with disjoint labels from the ID data, without auxiliary OOD training data.
Achieving both objectives simultaneously is inherently difficult due to a fundamental conflict — OOD generalization requires enhanced transferability, while OOD detection necessitates the inhibition of transfer.
To address this, we leverage insights from neural collapse (NC) — a phenomenon in deep networks where top-layer representations suppress feature variability and adopt a Simplex Equiangular Tight Frame (ETF) structure, impairing transferability. By controlling NC, we unify OOD detection and generalization: preventing NC enhances OOD transfer while inducing NC improves OOD detection.
Our proposed method excels at both tasks across various OOD datasets and architectures. 

\end{comment}


\end{abstract}

%%
%% The code below is generated by the tool at http://dl.acm.org/ccs.cfm.
%% Please copy and paste the code instead of the example below.
%%
\begin{CCSXML}
<ccs2012>
   <concept>
       <concept_id>10010147.10010178.10010224.10010225.10010230</concept_id>
       <concept_desc>Computing methodologies~Video summarization</concept_desc>
       <concept_significance>500</concept_significance>
       </concept>
   <concept>
       <concept_id>10010147.10010178.10010179.10010182</concept_id>
       <concept_desc>Computing methodologies~Natural language generation</concept_desc>
       <concept_significance>300</concept_significance>
       </concept>
 </ccs2012>
\end{CCSXML}

% \ccsdesc[500]{Human-centered computing~Information visualization}
\ccsdesc[500]{Computing methodologies~Tracking}

% \ccsdesc[500]{Computing methodologies~Video summarization}
% \ccsdesc[300]{Computing methodologies~Natural language generation}
%%
%% Keywords. The author(s) should pick words that accurately describe
%% the work being presented. Separate the keywords with commas.
\keywords{Computer vision; Autonomous driving; 3D object detection; 3D multi object tracking; Deep learning; Kalman filter}

%%
%% This command processes the author and affiliation and title
%% information and builds the first part of the formatted document.
\maketitle
\renewcommand{\thefootnote}{\fnsymbol{footnote}} 
\footnotetext[1] {Corresponding author}
% \renewcommand{\thefootnote}{\fnsymbol{footnote}} 
% \footnotetext[1] {Corresponding author}
% \begin{abstract}
Advancements in DNA sequencing technologies have significantly improved our ability to decode genomic sequences. However, the prediction and interpretation of these sequences remain challenging due to the intricate nature of genetic material. Large language models (LLMs) have introduced new opportunities for biological sequence analysis. Recent developments in genomic language models have underscored the potential of LLMs in deciphering DNA sequences. Nonetheless, existing models often face limitations in robustness and application scope, primarily due to constraints in model structure and training data scale. To address these limitations, we present \textbf{Gener}\textit{ator}, a generative genomic foundation model featuring a context length of 98k base pairs (bp) and 1.2B parameters. Trained on an expansive dataset comprising 386B bp of eukaryotic DNA, the \textbf{Gener}\textit{ator} demonstrates state-of-the-art performance across both established and newly proposed benchmarks. The model adheres to the central dogma of molecular biology, accurately generating protein-coding sequences that translate into proteins structurally analogous to known families. It also shows significant promise in sequence optimization, particularly through the prompt-responsive generation of enhancer sequences with specific activity profiles. These capabilities position the \textbf{Gener}\textit{ator} as a pivotal tool for genomic research and biotechnological advancement, enhancing our ability to interpret and predict complex biological systems and enabling precise genomic interventions. Implementation details and supplementary resources are available at \url{https://github.com/GenerTeam/GENERator}.
\keywords{DNA, Genomics, Foundation model, Generative model}
\vspace{12pt}
\end{abstract}



\section{Introduction}

Chain-of-Thought (CoT) prompting~\cite{Nye:2021, cot, Kojima:2022cotzero} has emerged as a cornerstone strategy for enhancing Large Language Models (LLMs) in complex reasoning tasks. By eliciting step-by-step inference, CoT enables LLMs to decompose intricate problems into manageable subtasks, thereby improving their problem-solving performance~\cite{Yao:2023tot, Wang:2023self-consistency, Zhou:2023least, Shinn:2023Reflexion}. Recent advancements, such as OpenAI's o1~\cite{o1} and DeepSeek-R1~\cite{deepseekr1}, further demonstrate that scaling up CoT lengths from hundreds to thousands of reasoning steps could continuously improve LLM reasoning. These breakthroughs have underscored CoT’s potential to advance LLM capabilities, expanding the boundaries of AI-driven problem-solving.

\begin{figure}[t]
\centering
    \includegraphics[width=0.95\columnwidth]{fig/intro.pdf}
    \caption{In contrast to vanilla CoT that generates all reasoning tokens sequentially, \method enables LLMs to \textit{skip} tokens with less semantic importance (\textit{e.g.,} \includegraphics[width=7pt]{fig/token.pdf}~) and learn shortcuts between critical reasoning tokens, facilitating controllable CoT compression.}
    \label{fig:intro}
\end{figure}

Despite its effectiveness, the increased length of CoT sequences introduces substantial computational overhead. Due to the autoregressive nature of LLM decoding, longer CoT outputs lead to proportional increases in both inference latency and memory footprints of key-value cache. Additionally, the quadratic computational cost of attention layers further exacerbates this burden. These issues become particularly pronounced when CoT sequences extend into thousands of reasoning steps, resulting in significant computational costs and prolonged response times. While prior research has explored methods for selectively skipping reasoning steps~\cite{Ding:2024cotshortcut, liu2024skipstep}, recent findings~\cite{jin:2024cotlength, Merrill:2024cotlength} suggest that such reductions may conflict with test-time scaling~\cite{o1-blog, snell2025scaling}, ultimately impairing LLM reasoning performance. Therefore, striking an optimal balance between CoT efficiency and reasoning accuracy remains a critical open challenge.

In this work, we delve into CoT efficiency and seek the answer to an important question: \textit{``Does every token in the CoT output contribute equally to deriving the answer?''} We empirically analyze the semantic importance of tokens within CoT outputs and reveal that their contributions to the reasoning performance vary, as depicted in Figure 2. Building on this insight, we introduce \method, a simple yet effective approach that enables LLMs to \textit{skip} less important tokens within CoT sequences and learn shortcuts between critical reasoning tokens, thereby allowing for controllable CoT compression with adjustable ratios. Specifically, as shown in Figure~\ref{fig:intro}, \method constructs compressed CoT training data with various compression ratios, by pruning unimportance tokens from original LLM CoT trajectories. Then, it conducts a general supervised fine-tuning process on target LLMs with this training data, facilitating LLMs to automatically trim redundant tokens during reasoning.

We conduct extensive experiments across various models, including LLaMA-3.1-8B-Instruct and the Qwen2.5-Instruct series, using two widely recognized math reasoning benchmarks: GSM8K and MATH-500. The results validate the effectiveness of \method in compressing CoT outputs while maintaining robust reasoning performance. Notably, Qwen2.5-14B-Instruct exhibits almost \textbf{NO} performance drop (less than $0.4\%$) with a $\bm{40\%}$ reduction in token usage on GSM8K. On the challenging MATH-500 dataset, LLaMA-3.1-8B-Instruct effectively reduces CoT token usage by $\bm{30}\%$ with a performance decline of less than $4\%$, resulting in a $\bm{1.4}\times$ inference speedup. Further analysis underscores the coherence of \method in specified compression ratios and its potential scalability with stronger compression techniques.

\method is distinguished by its low training cost. For Qwen2.5-14B-Instruct, \method fine-tunes only 0.2\% of the model's parameters using LoRA. The size of the compressed CoT training data is no larger than that of the original training set, with 7,473 examples in GSM8K and 7,500 in MATH. The training is completed in approximately 2 hours for the 7B model and 2.5 hours for the 14B model on two 3090 GPUs. These characteristics make \method an efficient and reproducible approach, suitable for use in efficient and cost-effective LLM deployment.

To sum up, our key contributions are:
\begin{enumerate}
    \item To the best of our knowledge, this work is the \textit{first} to investigate the potential of enhancing CoT efficiency through \textit{token skipping}, inspired by the varying semantic importance of tokens in CoT trajectories of LLMs.
    \item We introduce \method, a simple yet effective approach that enables LLMs to skip redundant tokens within CoTs and learn shortcuts between critical tokens, facilitating CoT compression with adjustable ratios.
    \item Our experiments validate the effectiveness of \method. When applied to Qwen2.5-14B-Instruct, \method reduces reasoning tokens by $40\%$ (from 313 to 181) on GSM8K, with less than a $0.4\%$ performance drop.
\end{enumerate}

\section{Related Work}

\subsection{LoRA and its Variants}
PEFT methods aim to update a small proportion of parameters to adapt LLMs for downstream tasks \cite{hou_adapter,Li_prompt,zaken_bitfit, wu_moslora, hu_lora}.
Among these methods, LoRA \cite{hu_lora}, which injects trainable low-rank branches to approximate the weight updates, has become increasingly popular as introducing no latency during inference.
In the vanilla LoRA method, the authors introduce two linear projection layers and initialize them as Kaiming uniform and zero matrices \cite{hu_lora}. 
The following variants can be categorized into: 1) searching ranks \cite{zhang_adalora}; 2) introducing training skills such as setting different learning rates \cite{Hayou_lora+}; and 3) designing new structures, such as \cite{wu_moslora}.
However, all these variants focus on improving performance for the ideal scenarios without weight noise.
In this paper, we propose HaLoRA, which is customized for hardware deployment.




\subsection{Hybrid CIM Architecture}
Hybrid CIM architectures combine different memory devices to achieve capabilities beyond what pure single-memory-device architectures can offer. Among them, RRAM-SRAM hybrid architectures have attracted significant attention by combining the high energy efficiency of RRAM with accurate computation of SRAM \cite{wen_science,vlsi_minotaur,liu_hardsea,krishnan_hybrid}.
These hybrid designs typically partition computational tasks based on the characteristics of each memory device: deploying high-precision, frequently updated operations on SRAM while allocating computation-intensive yet structurally simple operations to RRAM \cite{wen_science}.
This strategy has facilitated the efficient implementation of convolutional neural networks (CNNs) and lightweight neural architectures \cite{vlsi_minotaur,cnn_cim,chimera}, enabling their deployment in edge computing applications such as robotic localization \cite{slam}, target tracking\cite{tracking}, and recommendation systems\cite{edge_nlp}.
However, existing hybrid architectures primarily focus on implementing small-scale models, with limited exploration of large language models. 
In this work, we explore efficient LLM deployment with hybrid CIM architecture.





\subsection{Robustness Methods against Hardware Non-idealities}

The robustness methods against the RRAM non-idealities have been a hot topic for the past decade.
Specifically, these robust methods can be categorized into 1) noise-aware training, which typically incorporates noise during the training process or introduces robust loss functions \cite{kd_rram, bayes, bayesft}, and 2) hardware compensation strategies, such as mapping critical weights to low-variation areas~\cite{tfix,victor}.
However, these methods mainly focus on the robustness of convolutional neural networks (CNNs) and are hard to generalize to LLMs.
Considering noise-aware training methods, the key is to continuously train the models to improve their robustness including knowledge distillation \cite{kd_rram} and Bayesian neural network training \cite{bayes, bayesft}.
Due to the massive size of the LLM model, such as 3 billion parameters \cite{llama_report}, the cost of continuous training is unaffordable.
Meanwhile, the hardware compensation strategies are impractical for LLMs since pre-testing and correcting each layer through input regularization and column-shared factors introduce substantial additional hardware overhead.
In this paper, we focus on improving the robustness of LoRA-finetuned LLMs at the finetuning stage. 

% % \section{BIDS Dataset}
% \label{sec: dataset}

% % 数据集概括
% As it is very costly to build a bimodal summarization dataset from scratch, we, therefore, leverage the QVHighlights dataset~\cite{lei2021detecting} to construct a \textbf{B}imodal V\textbf{ID}eo \textbf{S}ummarization dataset (\textbf{BIDS}) to support the investigation of the BiSSV task. The constructed BIDS dataset finally contains 8130 videos with corresponding ground-truth Visual-Modal (VM) and Textual-Modal (TM) Summaries and saliency scores annotated for each 2-second clip, indicating its significance. Following the restrictions of traditional video summarization~\cite{gygli2014creating}, we ensure that the length of the VM-Summary does not exceed 15\% of the original video's duration. We describe the data processing and analysis in detail in the following subsections.

% \subsection{Data Processing}
% \label{sec: datapro}

% % 数据处理概括
% We aim to build a bimodal video summarization dataset with triplet data samples (video, TM-Summary, VM-summary), where the TM-Summary is a concise text description, and the VM-summary contains highlighted segments within the video. Firstly, we merge text-related segments from the original videos to guarantee that the TM-Summary accurately captures the main content of the video. 
% Secondly, we design a ranking-based extraction algorithm to preserve the most salient visual content as VM-Summary. 
% Lastly, we perform data cleaning to remove unsuitable videos that lack a clear focus for summarization. The overview of the BIDS building process is illustrated in Figure \ref{fig: data_construct}.

% % 步骤
% % 合并标注和视频
% \noindent \textbf{Data merging.} QVHighlights~\cite{lei2021detecting} is a video dataset that supports query-based moment retrieval and highlight detection, with annotations of natural language query, segments relevant to the query, and saliency scores for each 2s-clip within the segments. Taking the query as the TM-Summary, we merge the relevant segments chronologically as original videos in our dataset. In this way, we obtain a (video, TM-Summary) pair, for which we subsequently extract the VM-Summary. 

% % 提取 VM-Summary
% \noindent \textbf{VM-Summary extraction.} We utilize the annotated 2s-saliency scores for VM-Summary extraction. 
% Unlike the Knapsack algorithm utilized by previous video summarization datasets~\cite{song2015tvsum,gygli2014creating}, our extraction algorithm retains salient visual content within long segments and avoids favoring short segments. An illustration of this algorithm is presented in Figure ~\ref{fig: data_construct}. 
% We also provide a  pseudo-code in Appendix \ref{sec: pseudo code}.

% (a) \textit{Ranking}. We first merge adjacent 2s-clips with the same saliency scores into segments. Then, we rank all the candidate segments according to their saliency scores. The candidate segments are subsequently selected for VM-Summary in descending order. To comply with the length limit of VM-Summary (15\% video duration in our case), we may need to scale some candidate segments.

% (b) \textit{Scaling}. As the candidate segments vary in length, the purpose of scaling is to preserve informative parts within segments while guaranteeing conciseness. Specifically, candidate segments with the same score will be appended to the VM-Summary if it does not surpass the length constraint. Otherwise, these segments are proportionally scaled. 
% We assume that the parts closer to higher-scored segments usually contain more valuable information.
% Therefore, if the segment has higher-scored neighbors, adjacent parts closer to those neighbors are preserved (colored in \textcolor{red}{red} and \textcolor{myyellow}{yellow}, indicating two and one higher-scored neighbors, respectively); otherwise, its central part is preserved (colored in \textcolor{mygreen}{green}). The scaled segments are appended to the VM-Summary, and the segments with lower ranks are all rejected.

% % 数据清洗
% \noindent \textbf{Data cleaning.} Finally, we remove segments shorter than 2 seconds and videos with VM-Summary occupying less than 5\% of the video duration since they lack clear focal points for summarization. Finally, of 8,172 videos, only 42 (0.51\%) videos are removed.

% \subsection{Data Analysis}
% \label{sec: dataana}

% % 传统数据处理方式的缺陷
% Traditional video summarization datasets use the Knapsack algorithm to generate VM-Summary~\cite{gygli2014creating,song2015tvsum}. 
% However, Otami M et al.~\cite{otani2019rethinking} point out that their segmentation-selection pipeline favors short segments since selecting long segments costs more. 
% However, long and visually consistent segments can also contain informative moments. For example, when watching a video of \textit{someone playing basketball}, most of the visual content is similar, but we can still identify key moments, such as \textit{shooting}.
% Inspired by humans' ability to distinguish important moments in long videos, we choose to scale the candidate segments instead of rejecting them entirely. As a result, our VM-Summary shows a stronger correlation between the saliency scores and the selected segments.  

% % 相关系数比较
% We use Spearman's correlation coefficient $\rho$~\cite{zwillinger1999crc} to validate the effectiveness of our VM-Summary extraction algorithm. A higher coefficient between the saliency scores $S$ and the frame-level selection sequence $F$ (1 for the frame being selected into the VM-Summary and 0 for otherwise) indicates more salient content is preserved, which is the goal of summarization. 
% As presented in Table ~\ref{tab: dataset comparison}, BIDS has the highest Spearman's $\rho$ compared to traditional datasets. Moreover, Spearman's $\rho$ between $S$ and $F$ (generated by annotators) surpasses the $\rho$ between $S$ and GT-$F$ (obtained by applying Knapsack algorithm over the annotated saliency scores) in SumMe~\cite{gygli2014creating}, which further demonstrates that Knapsack algorithm can not effectively preserve salient parts within long segments. 

% % 统计数据
% After removing invalid and duplicate videos, BIDS contains 8130 videos, with 5854/650/1626 videos for training/validation/test set. 
% We ensure that the original videos between different sets do not overlap to avoid data leakage.  
% The data statistics of BIDS are presented in Table \ref{tab: dataset statistics}. As presented in Figure \ref{fig: distribution}, our algorithm is able to generate VM-summaries within a strict length constraint, with the majority occupying 14-15\% of the video's duration. Furthermore, the segments in a VM-Summary are evenly distributed throughout the corresponding video.

% \begin{table}[t]
%     \centering
%     \small  
%     \caption{Comparison with traditional video summarization datasets.
%     $\rho$: Average Spearman's correlation coefficient. 
%     Sig.: Significance (p < 0.05). 
%     $S$: Saliency score.
%     $F$: Frame-level sequence indicating each frame is selected (1) or not selected (0) into the VM-Summary. 
%     GT-$F$: the $F$ is calculated by averaging human annotated scores for each video in SumMe~\cite{gygli2014creating} and TVSum~\cite{song2015tvsum}.
%     dp: the $F$ is obtained by the Knapsack algorithm.
%     } 
    
%     \vspace{-8pt}
%     \begin{tabular}{ccccc}
%     \toprule
%     \textbf{Dataset}                & \textbf{Set of Variables}          & $\boldsymbol{\rho}$ & \textbf{Sig.}  & \textbf{\# of Videos}    \\
%     \midrule
%     \multirow{2}{*}{SumMe~\cite{gygli2014creating}} &($S$, GT-$F_{dp}$) & 0.34                               & \checkmark & \multirow{2}{*}{25} \\
%                            &($S$, $F$)        &  \underline{0.44}                        & \checkmark &                     \\
%                            \midrule
%     \multirow{2}{*}{TVSum~\cite{song2015tvsum}} &($S$, GT-$F_{dp}$)& 0.31                               & \checkmark & \multirow{2}{*}{50} \\
%                            &($S$, $F_{dp}$)    & 0.24                               & $\times$   &                     \\
%                            \midrule
%     BIDS(ours)           &($S$, GT-$F$)     & \textbf{0.52}                      & \checkmark & \textbf{8130}     \\ 
%     \bottomrule
%     \end{tabular}
%     \label{tab: dataset comparison}
% \end{table}


% \begin{figure}[t]
%     \centering
%     \includegraphics[width=0.95\linewidth]{Images/Distribution.pdf}
%     \vspace{-6pt}
%     \caption{(a) Distribution of duration ratio between VM-Summary and original video; (b) Distribution of temporal positions of the segments selected into the VM-Summary in the original video.
%     }
%     \label{fig: distribution}
% \end{figure}

% \begin{figure*}[t]
%     \centering
%     \includegraphics[width=0.8\linewidth]{Images/Framework.pdf}
%     \vspace{-8pt}
%     \caption{Model Architecture of UBiSS. 
%     }
%     \label{fig: framework}
% \end{figure*}

% \begin{table*}[t]
%     \centering
%      \small
%     \captionsetup{skip=10pt}
%     % \renewcommand{\arraystretch}{1.2}
%     \caption{Statistics of BIDS. 
%     {VM: Visual-Modal Summary. TM: Textual-Modal Summary.}
%     \vspace{-8pt}
%     }
%     \begin{tabular}{ccccccc}
%            \toprule
%            & \textbf{Avg. Video Len(s)} & \textbf{Total Video Len(h)} & \textbf{Avg. VM Len(s)} & \textbf{Avg. VM proportion(\%)} & \textbf{Avg. TM Len(word)} & \textbf{\# of Videos} \\
%            \midrule
%             Training   & 43.55           & 70.82             & 6.05         & 14.07               & 10.52            & 5854             \\
%             Validation & 40.05           & 7.23              & 5.57         & 14.07               & 10.41            & 650              \\
%             Test       & 44.83           & 20.25             & 6.19         & 14.12               & 10.42            & 1626             \\
%             All        & 43.53           & 98.3              & 6.04         & 14.08               & 10.49            & 8130      \\
%             \bottomrule
%     \end{tabular}
%     \label{tab: dataset statistics}
% \end{table*}

To address the limitations of current denoising techniques~\cite{wang2021denoising, he2024double, gao2022selfguided, lin2023autodenoise}, we conduct a thorough investigation into the underlying causes of the overlap between normal and noisy interactions in the overall loss distribution. Then, we refine existing denoising criteria and introduce a novel resampling strategy for denoising based on users' personal loss distributions, called PLD. Furthermore, we enhance the denoising capability of PLD through rigorous theoretical analysis, resulting in a more robust and effective denoising methodology.


\subsection{Motivation}
To investigate the causes of the overlap between normal and noisy interactions in the overall loss distribution, we conduct an experimental analysis. Using the MIND dataset~\cite{wu2020mind} as a case study, we introduce additional noise ratios of 10\%, 20\%, 30\%, and 40\% into user interactions and evaluate the impact on LightGCN~\cite{he2020lightgcn}\footnote{Similar experiments are conducted on different datasets and models, yielding consistent results. Due to space constraints, we present only the results for this configuration here.}. Detailed description of the experimental setup can be found in Section~\ref{sec:exp_setup}.

To facilitate this analysis, we introduce the following notation:
\begin{itemize}[leftmargin=*]
    \item $\mathcal{I}_{\text{normal}}$: the set of normal interactions.
    \item $\mathcal{I}_{\text{noise}}$: the set of noisy interactions.
    \item $l_{u,v}$: the loss corresponding to the interaction between user $u$ and item $v$.
    \item $\mathcal{O}$: the overlap region containing noisy interactions with lower losses and normal interactions with higher losses in the \textbf{overall loss distribution}, as depicted in Figure~\ref{fig:user_loss_org}.
    \item $\mathcal{O}_{u}$: the overlap region in user $u$'s \textbf{personal loss distribution}.
\end{itemize}
\textbf{Note that} Quartiles are used instead of max-min values to mitigate the influence of extreme values when determining overlap regions.
We further define the following sets to analyze interactions within the overlap regions:
\begin{itemize}[leftmargin=*]
    \item $\mathcal{I}^{\mathcal{G}}_{\text{normal}} = \{ (u, v) \mid (u, v) \in \mathcal{I}_{\text{normal}} \land l_{u,v} \in \mathcal{O}\}$: the set of normal interactions that fall within the overlap region of the overall loss distribution.
    % $\mathcal{I}^{\mathcal{G}}_{\text{normal}} \coloneqq \mathcal{I}_{\text{normal}} \cap \mathcal{O}$: the set of normal interactions that fall within the overlap region in the overall loss distribution.
    \item $\mathcal{I}^{\mathcal{G}}_{\text{noise}} = \{ (u, v) \mid (u, v) \in \mathcal{I}_{\text{noise}} \land l_{u,v} \in \mathcal{O} \}$: the set of noisy interactions that fall within the overlap region of the overall loss distribution.
    \item $\mathcal{I}^{\mathcal{P}}_{\text{normal}} = \{ (u, v) \mid (u, v) \in \mathcal{I}_{\text{normal}} \land l_{u,v} \in \mathcal{O}_{u} \}$: the set of normal interactions that fall within the overlap region of the personal loss distribution.
    \item $\mathcal{I}^{\mathcal{P}}_{\text{noise}} = \{ (u, v) \mid (u, v) \in \mathcal{I}_{\text{noise}} \land l_{u,v} \in \mathcal{O}_{u} \}$: the set of noisy interactions that fall within the overlap region of the personal loss distribution.
\end{itemize}



\begin{figure}
    \centering
    \includegraphics[width=0.485\textwidth]{Imgs/user_loss_org.pdf}
    \caption{Probability Distribution of losses.}
    \label{fig:user_loss_org}
\end{figure}

% \begin{table}[t]
%   \centering
%     \caption{Interaction-wise loss statistics}
%     \resizebox{0.47\textwidth}{!}{

% \begin{tabular}{cccc}
%     \toprule
%     \textbf{ Noise Ration } & \textbf{ $\vert \mathcal{U}^{\text{any}, +}_{\text{normal}} \vert$ } & \textbf{ $\vert \mathcal{U}^{\text{any}, -}_{\text{noise}} \vert$  } & \textbf{$\vert \mathcal{U}^{\text{any}, +}_{\text{normal}} \cap \mathcal{U}^{\text{any}, -}_{\text{noise}} \vert$} \\
%     \midrule
%     0.1 &19,312 &11,250 &5,943 \\
%     0.2 &23,354 &20,092 &10,254 \\
%     0.3 &23,512 &24,813 &13,181 \\
%     0.4 &23,588 &27,452 &15,079 \\
%     \bottomrule
%     \end{tabular}
%     }
%   \label{tab:inter_user}%
% \end{table}%


\begin{table}[t]
  \centering
    \caption{Statistics of overall loss distribution}
    \resizebox{0.475\textwidth}{!}{

\begin{tabular}{ccccc}
    \toprule
    \textbf{ Noise Ratio } & $\vert \mathcal{I}^{\mathcal{G}}_{\text{normal}} \vert$  & $\vert \mathcal{I}^{\mathcal{G}}_{\text{normal}} \vert / \vert \mathcal{I}_{\text{normal}} \vert $ & $\vert \mathcal{I}^{\mathcal{G}}_{\text{noise}} \vert$    & $\vert \mathcal{I}^{\mathcal{G}}_{\text{noise}} \vert / \vert \mathcal{I}_{\text{noise}} \vert$\\
    \midrule
    0.1 &152,892 &18.40\% &12,976 &15.61\% \\
    0.2 &162,700 &19.58\% &29,130 &17.52\% \\
    0.3 &162,763 &19.58\% &45,275 &18.16\% \\
    0.4 &159,454 &19.19\% &59,486 &17.89\% \\
    \bottomrule
    \end{tabular}
    }
  \label{tab:inter_user}%
\end{table}%


\textbf{Overall Loss Distribution.} The overall loss distribution consists of the loss of all interactions. For clarity, we separate the overall loss distribution into normal and noisy interaction loss distributions. Figure~\ref{fig:user_loss_org} illustrates that normal and noisy interactions exhibit significant overlap in the overall loss distribution across varying noise ratios. As shown in Table~\ref{tab:inter_user}, across different noise ratios, the values of $\vert \mathcal{I}^{\mathcal{G}}_{\text{normal}} \vert / \vert \mathcal{I}_{\text{normal}} \vert$ and $\vert \mathcal{I}^{\mathcal{G}}_{\text{noise}} \vert / \vert \mathcal{I}_{\text{noise}} \vert$ are generally high. This makes it difficult to distinguish between normal and noisy interactions based on the overall loss distribution, increasing the likelihood of denoising errors in existing methods~\cite{wang2021denoising, he2024double, gao2022selfguided, lin2023autodenoise}. Consequently, relying solely on the overall loss distribution may not be an effective approach for differentiating between normal and noisy interactions.

\textbf{User Case in Personal Loss Distribution.} To further analyze these interactions, we randomly select five users from the dataset and display their personal loss distributions across varying noise ratios. As shown in Figure~\ref{fig:user_loss}, for each user, the losses of normal interactions consistently remain lower than those of noisy interactions. However, due to significant variance in users' personal loss distributions, the overlap depicted in Figure~\ref{fig:user_loss_org} is primarily attributed to certain users exhibiting normal interaction losses that exceed other users' noisy interaction losses. For example, at a noise ratio of 0.2, the noisy interaction losses of user 2 differ from corresponding normal interaction losses but are similar to the normal interaction losses of user 5.


\begin{figure}
    \centering
    \includegraphics[width=0.485\textwidth]{Imgs/user_loss_user.pdf}
    \caption{Personal loss distribution for five users.}
    \label{fig:user_loss}
\end{figure}

\textbf{Statistics of Personal Loss Distribution.} Building on the previous user case analysis, we propose that noisy interactions can be more effectively identified by analyzing users' personal loss distributions. To further illustrate this, we examine the statistical differences between users' normal and noisy interaction losses. For each user, we compute the difference between the lower quartile of their normal interaction losses and the upper quartile of their noisy interaction losses. As depicted in Figure~\ref{fig:user_diff}, most users exhibit higher noisy interaction losses compared to their normal interaction losses, a trend that persists across all noise levels. As shown in Table~\ref{tab:inter_person}, compared to $\vert \mathcal{I}^{\mathcal{G}}_{\text{normal}} \vert$ and $\vert \mathcal{I}^{\mathcal{G}}_{\text{noise}} \vert$, $\vert \mathcal{I}^{\mathcal{P}}_{\text{normal}} \vert$ and $\vert \mathcal{I}^{\mathcal{P}}_{\text{noise}} \vert$ decrease significantly, further validating the effectiveness of personal loss distributions for distinguishing normal interactions from noisy ones. This observation offers valuable insights for potential improvements in denoising strategies.



\subsection{PLD Methodology}
Based on the above insights, a straightforward denoising method would treat higher-loss interactions within the personal loss distribution as noise. However, the sparsity of user interactions causes significant fluctuations in personal loss distributions. As a result, reweight-based methods may cause drastic changes in the weight assigned to the same interaction across consecutive epochs, undermining training stability. Additionally, due to variations in the presence and amount of noise, dropping the highest-loss interactions could negatively affect users with little or no noisy interactions. For instance, with a fixed drop rate (e.g., 10\%), a user without noisy interactions would still experience a 10\% drop in normal interactions during training, which would degrade the user's experience.


\begin{figure}
    \centering
    \includegraphics[width=0.485\textwidth]{Imgs/user_diff.pdf}
    \caption{Difference between normal and noisy interactions in personal loss distributions across all users.}
    \label{fig:user_diff}
\end{figure}

\begin{table}[t]
  \centering
    \caption{Statistics of personal loss distribution}
    \resizebox{0.475\textwidth}{!}{

\begin{tabular}{ccccc}
    \toprule
    \textbf{ Noise Ratio } & $\vert \mathcal{I}^{\mathcal{P}}_{\text{normal}} \vert$  & $\vert \mathcal{I}^{\mathcal{P}}_{\text{normal}} \vert / \vert \mathcal{I}_{\text{normal}} \vert $ & $\vert \mathcal{I}^{\mathcal{P}}_{\text{noise}} \vert$    & $\vert \mathcal{I}^{\mathcal{P}}_{\text{noise}} \vert / \vert \mathcal{I}_{\text{noise}} \vert$\\
    \midrule
    0.1 &38,998 &5.13\% &2,125 &2.79\% \\
    0.2 &36,971 &4.44\% &4,979 &2.99\% \\
    0.3 &35,571 &4.28\% &7,969 &3.19\% \\
    0.4 &35,511 &4.27\% &10,771 &3.24\% \\
    \bottomrule
    \end{tabular}
    }
  \label{tab:inter_person}%
\end{table}%

To address these issues, we propose solving this problem through probabilistic sampling. Specifically, we aim to reduce the probability of noisy interactions being optimized while ensuring that users without noise remain unaffected. To this end, we propose a resampling strategy named PLD, which consists of two parts: Candidate Pool Construction and Item Resampling.

\textbf{Candidate Pool Construction.} To prevent items with extremely small losses from being repeatedly sampled, we pre-construct a candidate item pool, $\mathcal{C}_{u}^{k}$ of size $k$ for each user $u$. Items in $\mathcal{C}_{u}^{k}$ are randomly sampled from the user's interacted items, $\mathcal{V}_{u}$.

\textbf{Item Resampling.} Next, we calculate the loss $l_{u,v}$ for each of the $k$ items in the candidate pool. We then perform resampling based on the computed loss values. Specifically, for user $u$, the sampling probability for item $v$ in the candidate pool $\mathcal{C}_{u}^{k}$ is determined by:
\begin{equation}
\label{eq:p_i}
    P_{u, v} = \frac{\exp(-l_{u,v})}{\sum_{j \in \mathcal{C}_{u}^{k}} \exp(-l_{u,j})}.
\end{equation}
Finally, the resampled item is selected as the positive interaction for the current optimization step.

This method ensures that variances in personal loss distributions do not adversely affect the sampling process. Moreover, this approach ensures that normal interactions are optimized, even for users without noisy interactions—unlike previous methods, which always drop a subset of interactions~\cite{wang2021denoising, he2024double}.




\subsection{Theoretical Analysis}
To analyze the effectiveness of the PLD method, we examine the probability that PLD samples both normal and noisy interactions.
\begin{theorem}
\label{the:p_i_j}
    % For a user \( u \), there are \( n \) items with normal interactions and \( m \) items with noisy interactions. Assume that the loss of the user's normal interactions follows a Gaussian distribution \( \mathcal{N}(\mu_1, \sigma^2) \) and the loss of noisy interactions follows a Gaussian distribution \( \mathcal{N}(\mu_2, \sigma^2) \), where \( \mu_1 < \mu_2 \) and \( \mu_1, \mu_2 > \sigma \). From these \( m+n \) interactions, we first randomly select $k$ interactions, and then resample one positive interaction according to Equation~\ref{eq:p_i}. Let \( \Lambda_{\text{normal}} \) denote the sum of sampling probabilities for normal interactions, and \( \Lambda_{\text{noise}} \) denote the sum of sampling probabilities for noisy interactions. Define the following:
    % alpha &= \exp\left(-\mu_1 + \frac{\sigma^2}{2}\right), 
    %              &\beta = \exp\left(-\mu_2 + \frac{\sigma^2}{2}\right),\\
    For a user \( u \), there are \( n \) items with normal interactions and \( m \) items with noisy interactions. 
    Suppose the loss of each normal interaction follows a distribution with mean \(\mu_1\) and variance \(\sigma^2\), and the loss of each noisy interaction follows a distribution with mean \(\mu_2\) and variance \(\sigma^2\). We assume \(\mu_1 < \mu_2\) and \(\mu_1, \mu_2 > \sigma\). From these \( m+n \) interactions, we first randomly select $k$ interactions, and then resample one positive interaction according to Equation~\ref{eq:p_i}. Let \( \Lambda_{\text{normal}} \) denote the sum of sampling probabilities for normal interactions, and \( \Lambda_{\text{noise}} \) denote the sum of sampling probabilities for noisy interactions. 
    Let \(\alpha\) and \(\beta\) represent the expectations of the normal and noisy interaction losses, respectively, where the expectation is taken over the exponential of the loss. Define the following:
    \begin{equation*}
        \begin{aligned}
            & \begin{aligned}
                \gamma &= \exp(\sigma^2) - 1, 
                 % &\eta = \frac{n}{n+m} \alpha + \frac{m}{n+m} \beta,\\
                 &\eta = \frac{n\alpha + m\beta}{n+m},\\
            \end{aligned}\\
            & \begin{aligned}
                \Gamma &= \frac{(n\alpha - m\beta)}{m+n} \cdot \frac{(\alpha^2 + \beta^2)(\gamma + \frac{m}{n+m}) + \beta^2 }{\eta^3},\\
                \chi &= \frac{\gamma}{(n+m)} \left[ n\alpha^2 - m\beta^2 \right]
            \end{aligned} 
        \end{aligned}
    \end{equation*}
    we have:
    \begin{equation}
    \label{eq:the}
        \begin{aligned}
            \mathbb{E}[\Lambda_{\text{normal}} - \Lambda_{\text{noise}}] = \left \{
                \begin{aligned}
                    &\frac{n - m}{n + m} ~~ & k = 1 \\
                    & \begin{aligned}
                        & \frac{n\alpha - m\beta}{(m+n)\eta} + \underbrace{\frac{\Gamma}{k} - \frac{\chi}{C^2} \frac{k}{(k-1)^2}}_{\mathrm{Fluctuation}~~\mathrm{term}}
                    \end{aligned}
                     ~~ & k > 1 \\
                \end{aligned}
            \right. ,
        \end{aligned}
    \end{equation}
    where \( C \in [\beta, \alpha] \) is a constant term.
\end{theorem}

The proof of Theorem~\ref{the:p_i_j} is detailed in Appendix~\ref{prf:the_p}. The term \(\frac{\Gamma}{k} - \frac{\chi}{C^2} \frac{k}{(k-1)^2}\) arises from the variance component in the denominator of the softmax function, exhibiting larger fluctuations when \(k\) is small, while stabilizing as \(k\) increases.

% \renewcommand{\algorithmicrequire}{ \textbf{Input:}}     % Use 'Input' in the format of Algorithm
% \renewcommand{\algorithmicensure}{ \textbf{Output:}}    % Use 'Output' in the format of Algorithm
% \algnewcommand{\LineComment}[1]{\Statex \(\triangleright\) #1}
% \begin{algorithm}[t]
%     \caption{Training Procedure with PLD} % Algorithm name
%     \label{al:dtr}
%     \begin{algorithmic}[1]
%     \renewcommand{\baselinestretch}{1.5}
%         \Require{Training set $\mathcal{D}$, pool size $k$, temperature coefficient $\tau$, batch size $\mathbb{B}$, loss function $\mathcal{L}(u, i, j)$}
%         \Ensure{Model parameters $\Theta$.}
%         \While{stopping criteria not met}
%             \LineComment \textit{PLD}
%             \State Draw $\mathbb{B}$ triples $(u, \mathcal{C}_{u}^{k}, j)$ from $\mathcal{D}$. 
%             \State Initialize the batch set $\mathcal{D}_{\mathbb{B}} = \emptyset$
%             \For{each $(u, \mathcal{C}_{u}^{k}, j)$}
%                 \State Calculate $l_i$ for $i \in \mathcal{C}_{u}^{k}$ using $\mathcal{L}(u, i, j)$.
%                 \State Resample $i^*$ based on Equation~\ref{eq:p_tau} within $\mathcal{C}_{u}^{k}$.
%                 \State Add $(u, i^*, j)$ to the batch set $\mathcal{D}_{\mathbb{B}}$.
%             \EndFor
%             \LineComment \textit{Standard Training}
%             \State Update $\Theta$ according to $\mathcal{L}(u, i, j)$ for each $(u, i^*, j)$ in $\mathcal{D}_{\mathbb{B}}$.
%         \EndWhile
%         \State \Return $\Theta$
%     \end{algorithmic}
% \end{algorithm}

According to Theorem~\ref{the:p_i_j}, when \(k=1\), PLD reduces to standard training with \(\mathbb{E}[\Lambda_{\text{normal}} - \Lambda_{\text{noise}}] = \frac{n - m}{n + m}\). For \(k>1\), given \(\alpha, \beta, \gamma > 0\), with \(\alpha > \beta\) and \(n \gg m\), we find that \(\Gamma > \frac{\chi}{C^2}\). Thus, \(\mathbb{E}[\Lambda_{\text{normal}} - \Lambda_{\text{noise}}] > \frac{n - m}{n + m}\). This indicates that \textbf{PLD outperforms standard training, demonstrating superior denoising capabilities.}

% To further enhance the effectiveness of the PLD method, we can increase \(\frac{n\alpha - m\beta}{(m+n)\eta}\). Specifically, let \(\xi = \frac{\beta}{\alpha} = \exp(\mu_1 - \mu_2) < 1\)~(see derivations in Appendix~\ref{App:prof}). Then, we can express \(\frac{n\alpha - m\beta}{(m+n)\eta} = \frac{n - \xi m}{n + \xi m}\). Notably, since \(\frac{\partial \frac{n\alpha - m\beta}{(m+n)\eta}}{\partial \xi} < 0\), we can decrease \(\xi\) to amplify \(\frac{n\alpha - m\beta}{(m+n)\eta}\), thus enlarging \(\mathbb{E}[\Lambda_{\text{normal}} - \Lambda_{\text{noise}}]\).
To further enhance the effectiveness of the PLD method, we can increase \(\frac{n\alpha - m\beta}{(m+n)\eta}\). Specifically, let \(\xi = \frac{\beta}{\alpha} = \exp\left(g\left(\mu_1 - \mu_2\right)\right) < 1\), where $g(\cdot)$ is a monotonically increasing function. We can express \(\frac{n\alpha - m\beta}{(m+n)\eta} = \frac{n - \xi m}{n + \xi m}\). Notably, since \(\frac{\partial \frac{n\alpha - m\beta}{(m+n)\eta}}{\partial \xi} < 0\), we can decrease \(\xi\) to amplify \(\frac{n\alpha - m\beta}{(m+n)\eta}\), thus enlarging \(\mathbb{E}[\Lambda_{\text{normal}} - \Lambda_{\text{noise}}]\).
Based on this idea, we introduce a temperature coefficient \(\tau\) into Equation~\ref{eq:p_i}:
\begin{equation}
\label{eq:p_tau}
    P_{u, v} = \frac{\exp(-l_{u,v} / \tau)}{\sum_{j \in \mathcal{C}_{u}^{k}} \exp(-l_{u,j} / \tau)}.
\end{equation}
% In this manner, the new \(\xi'\) can be considered as \(\xi' = \exp\left((\mu_1 - \mu_2)/\tau\right)\). By reducing \(\tau\), we can further enlarge \(\frac{n\alpha - m\beta}{(m+n)\eta}\). The algorithmic flow of PLD is outlined in Appendix~\ref{sec:app_methods} (Algorithm~\ref{al:dtr}).

In this manner, the new \(\xi'\) can be considered as \(\xi' = \exp\left(g\left(\left(\mu_1 - \mu_2\right)/\tau\right)\right)\). By reducing \(\tau\), we can further enlarge \(\frac{n\alpha - m\beta}{(m+n)\eta}\). The algorithmic flow of PLD is outlined in Appendix~\ref{sec:app_methods} (Algorithm~\ref{al:dtr}).

Additionally, we perform an in-depth analysis and comparison of the time and space complexity of PLD and baseline methods. For further details, please refer to Appendix~\ref{sec:dis}.

\section{Experiments}

\subsection{Models}
To comprehensively evaluate on LMMs, we conducted zero-shot inference across both commercial and open-source models. Our evaluation suite includes leading commercial models GPT-4o~\cite{hurst2024gpt40} and Gemini1.5-Pro~\cite{Gemini} alongside state-of-the-art open-source alternatives of varying scales: Qwen2.5-VL~\cite{qwen2.5-VL}, Qwen2-VL~\cite{wang2024qwen2}, LLaVA-v1.6~\cite{liu2023llava}, CogVLM~\cite{wang2023cogvlm}, MiniCPM-o-2.6~\cite{yao2024minicpm}, mPlug-Owl2~\cite{ye2023mplugowl2}, InternVL2v5~\cite{chen2024internvl},LLaVA-NEXT-Video~\cite{zhang2024llavanextvideo} and Cambrian~\cite{tong2024cambrian1}. Besides, Janus-Pro~\cite{chen2025januspro}, which unifies multimodal understanding and generation, is included to test the abilities between Unified Model and Vision Language Model.  This diverse selection enables us to analyze how model scale, architecture, and training approaches influence comic comprehensive capabilities. 


% Detailed specifications and inference configurations for each model are provided in Appendix~\ref{appendix:model}.

\subsection{Experimental Details}
% 隐含含义理解和预测帧内容这两个任务都是选择题,因为他们的标准答案是不唯一的很难衡量。如果模型选择了正确的答案选项,则认为是正确的,也就是说accuracy是主要的metric。 而对于排序任务,这一个任务形式非常新颖,对于一个comic strip,其输入顺序可以随便打乱,且答案是确切的。所以这个任务我们既采用了选择题的题型又使用了问答题。
The task prompts is displayed in Table ~\ref{prompt}. For visual narrative comprehension task, model is provided with the whole image. But for next-frame prediction and multi-frame sequence reordering task, LMMs infer with image sequences.
The hyper-parameters for each LMMs in the experiments including possible settings are detailed in Appendix~\ref{appendix:hyper-param}. Furthermore, to assess human capabilities in these tasks, we randomly select 100 questions from the dataset for each task and instruct human evaluators to answer. This allows us to benchmark the performance of human participants against our models, offering a thorough comparison of both human and LMMs proficiency in these specific tasks. 


\subsection{Main Results}
Our comprehensive evaluation reveals that while LMMs show promising capabilities in comprehension and prediction tasks, they significantly underperformed in sequence reordering tasks. Moreover, there remains a substantial performance gap between current models and human performance across all tasks. Unified Model underperformed than Vision Language Model.

\paragraph{Contextual Frame Prediction}
The frame prediction task appears to be the most tractable among the three tasks. GPT-4o achieves the highest score of $69.95\%$, followed closely by Qwen2-VL at $64.00\%$.This demonstrates that the performance gap between closed and open-source models is relatively small for this task. However, Janus-Pro perform notably below expectations ($27.50\%$), possibly due to its unified model architectural.

\paragraph{Visual Narrative Comprehension}
For visual narrative comprehension, we observe a similar pattern but with generally lower scores. GPT-4o leads with $61.60\%$, while other models show varying degrees of capability. 

\paragraph{Temporal Narrative Reordering}
The frame reordering task proves to be the most challenging, with all models performing significantly below human capability. Even the best-performing models struggle to exceed $30\%$ accuracy, with many achieving scores around $25-26\%$, which is slightly higher than random selection. 
Notably, several models (marked with *) are unable to perform this task due to their architectural limitations in processing multiple images simultaneously. For these models, we attempted to accommodate their single-image constraint by concatenating multiple frames horizontally into a single image, with white margins serving as frame boundaries. However, this workaround appears to be suboptimal, as these models likely struggle to properly distinguish individual frame boundaries and maintain the semantic independence of each frame, ultimately leading to their poor performance on the reordering task. 

The poor performance on reordering task suggests that current LMMs, regardless of their scale or architecture, have not yet developed robust capabilities for understanding temporal relationships and sequential logic in visual narratives.




\section{Conclusion}

This work analysed the results of evolutionary wrapper approaches using decision tree based models as proxies and compared them with common \gls{FE} techniques on a \gls{HL} detection problem. Three experiments were conducted using the proposed framework, each employing different proxy models.

When comparing the three experiments, an interesting behaviour of the framework was discovered, when changing the proxy model. The \gls{DT} experiment drastically reduced the number of features, while the other models did not. To further reduce the number of features, one could bias the grammar or apply some penalty in the fitness function for the individuals that use a large number of features. This might not change the behaviour when using different models other than a \gls{DT}, but it forcefully reduces the number of features.  

The results confirm that FEDORA can reduce the dimensionality of the data while statistically maintaining baseline performance, in every experiment. The framework consistently outperforms the remaining \gls{FE} methods, with statistical significance and large effect sizes, proving itself as a viable alternative.

The best result obtained is 76.2\% balanced accuracy using an individual from the \gls{RF} experiment, and a \gls{XGB} algorithm as the testing model, using 57 total features (45 Original, 6 Engineered and 6 Complex) out of the 60 original ones. When using the least amount of features, the best result is 72,8\% balanced accuracy using an individual from the \gls{DT} experiment and a \gls{RF} algorithm as the testing model, using a single complex feature.

In future work, exploring the above-mentioned behaviours might be relevant to better understanding them, namely when biasing the grammar or penalizing the use of many features in the fitness function. Concerning the explainability of the FEDORA transformations, researching meaningful grammar operators might prove useful in addressing problem-specific needs. In this case, having logical operators for the boolean features, which have values of "yes" or "no", and the choice of a simple decision algorithm as the proxy, may increase explainability. Additionally, the previous study has identified several areas for future research, yet to be addressed. For instance, comparing the framework with other common and more complex methods and completing the full \gls{ML} pipeline through the use of a method that addresses the \gls{CASH}, such as \cite{assunccao2020evolution}, and comparing it to other full pipeline frameworks, could be beneficial for contextualizing and evaluating the framework within the \gls{AutoML} and \gls{EC} domains. The framework still needs to be analysed with different datasets to properly assess its generalization capabilities.
% \section{} 
\label{reduction-P}

To prove our argument, we apply the splitting property of the Poisson process. Let \( N(t) \) be a Poisson process with rate parameter \( \lambda \). If events are split into two groups with probabilities \( p \) and \( 1-p \), then the resulting processes \( N_1(t) \) and \( N_2(t) \) are independent Poisson processes with rate parameters \( p\lambda \) and \( (1-p)\lambda \) respectively \cite{splitting_poisson}.

From process \( j \)'s perspective, we can split arrivals from sensor \( i \) into two groups: informative and uninformative arrivals with probabilities \( \nc_{ij} \) and \( 1-\nc_{ij} \), respectively. The rate of arrivals from sensor \( i \) is \( \lambda_i \), so the rate of informative arrivals for process \( j \) from sensor \( i \) is \( \nc_{ij}\lambda_i \). Additionally, we can further split the informative arrivals based on whether they can preempt ongoing service. The rate of informative arrivals that can preempt ongoing service for process \( j \) from sensor \( i \) is \( \np_{i}\nc_{ij}\lambda_i \) and the rate of informative arrivals that can not preempt ongoing service for process \( j \) from sensor \( i \) is \( (1-\np_{i})\nc_{ij}\lambda_i \). Since all these arrivals are Poisson, we can merge them into a single process. The total arrival rate of informative packets that can preempt ongoing service for process \( j \) is given by

\begin{equation}
\Tilde{\lambda}_j = \sum_{i=1}^{N} \np_{i}\nc_{ij}\lambda_i
\end{equation}

Similarly, the total arrival rate of informative packets that can not preempt ongoing service for process \( j \) is

\begin{equation}
\Tilde{\lambda}_j = \sum_{i=1}^{N} (1-\np_{i})\nc_{ij}\lambda_i
\end{equation}

We can express these rates in vector form as follows:

\begin{equation}
\boldsymbol{\Tilde{\lambda}}^T = \begin{bmatrix}
\Tilde{\lambda}_{1} & \Tilde{\lambda}_{2} & \dots & \Tilde{\lambda}_{M}
\end{bmatrix} = (\boldsymbol{\lambda}^T \odot \bfp^T) \bfc,
\end{equation}
\begin{equation}
\boldsymbol{\dot{\lambda}}^T = \begin{bmatrix}
\dot{\lambda}_{1} & \dot{\lambda}_{2} & \dots & \dot{\lambda}_{M}
\end{bmatrix} = (\boldsymbol{\lambda}^T \odot (1-\bfp^T)) \bfc,
\end{equation}


The importance of the packet is whether it has information of process $j$ so  we can say that The system with $N$ sensors and arrival rates $\boldsymbol{\lambda}$ shown in Figure \ref{fig:system_model} equivalents to the system with two sources as shown in Figure \ref{fig:equiv_model} from process $j$'s perspective.


\section{}\label{spv-appendix}


We adopt the stochastic hybrid system (SHS) model as defined in \cite{yates2019}, with a key distinction: our model incorporates probabilistic preemption. The system dynamics are depicted in Figure \ref{fig:equiv_model} so we can analyze the AoI for any process $i$ and generalize it. First, the discrete state is denoted as $q(t) = q \in Q = \{0, 1, 2\}$, where $q = 0$ represents an idle server, and $q \in \{1, 2\}$ signifies that an update packet is currently being serviced. The continuous state is described as $x(t) = [x_0(t), x_1(t)]$, where $x_0(t)$ represents the current age of the process, and $x_1(t)$ captures the potential age if the packet in service is successfully delivered. Notably, $x_1(t)$ is irrelevant in state $0$ since no packet is in service. In state $1$, $x_1(t)$ corresponds to the age of the informative update being serviced. Conversely, in state $2$, where an uninformative update is in service, the completion of this update does not affect the process age, rendering $x_1(t)$ irrelevant in this state as well.

\begin{table}[h]
\centering
\caption{Table of Transitions for the Markov Chain in Figure \ref{fig:shs}.}
\begin{tabular}{c c c c c c}
\toprule
$l$ & $q_l \rightarrow q'_l$ & $\lambda^{(l)}$ & $\mathbf{xA}_l$ & $\mathbf{A}_l$ & $\mathbf{v}_{q_l}\mathbf{A}_l$ \\
\midrule
1 & $0 \rightarrow 1$ & $\Tilde{\lambda}_{1}+\dot{\lambda}_{1}$ & $\begin{bmatrix} x_0 & 0 \end{bmatrix}$ & \small $\begin{bmatrix} 1 & 0 \\ 0 & 0 \end{bmatrix}$ \normalsize & $\begin{bmatrix} v_{00} & 0 \end{bmatrix}$ \\
2 & $0 \rightarrow 2$ & $\lambda_{C}-\Tilde{\lambda}_{1}-\dot{\lambda}_{1}$ & $\begin{bmatrix} x_0 & 0 \end{bmatrix}$ & \small $\begin{bmatrix} 1 & 0 \\ 0 & 0 \end{bmatrix}$ \normalsize & $\begin{bmatrix} v_{00} & 0 \end{bmatrix}$ \\
3 & $1 \rightarrow 0$ & $\mu$        & $\begin{bmatrix} x_1 & 0 \end{bmatrix}$ & \small$\begin{bmatrix} 0 & 0 \\ 1 & 0 \end{bmatrix}$ \normalsize & $\begin{bmatrix} v_{11} & 0 \end{bmatrix}$ \\
4 & $1 \rightarrow 1$ & $\Tilde{\lambda}_{1}
$  & $\begin{bmatrix} x_0 & 0 \end{bmatrix}$ & \small$\begin{bmatrix} 1 & 0 \\ 0 & 0 \end{bmatrix}$\normalsize & $\begin{bmatrix} v_{10} & 0 \end{bmatrix}$ \\
5 & $1 \rightarrow 2$ & $\Tilde{\lambda}_{C}-\Tilde{\lambda}_{1}$  & $\begin{bmatrix} x_0 & 0 \end{bmatrix}$ & \small$\begin{bmatrix} 1 & 0 \\ 0 & 0 \end{bmatrix}$ \normalsize & $\begin{bmatrix} v_{10} & 0 \end{bmatrix}$ \\
6 & $2 \rightarrow 0$ & $\mu$        & $\begin{bmatrix} x_0 & 0 \end{bmatrix}$ & \small$\begin{bmatrix} 0 & 0 \\ 1 & 0 \end{bmatrix}$ \normalsize & $\begin{bmatrix} v_{20} & 0 \end{bmatrix}$ \\
7 & $2 \rightarrow 1$ & $\Tilde{\lambda}_{1}$  & $\begin{bmatrix} x_0 & 0 \end{bmatrix}$ & \small$\begin{bmatrix} 1 & 0 \\ 0 & 0 \end{bmatrix}$ \normalsize & $\begin{bmatrix} v_{20} & 0 \end{bmatrix}$ \\
\bottomrule
\end{tabular}
\label{shs_table}
\end{table}


\begin{figure}
    \centering
    \includegraphics[width=0.5\linewidth]{figures/shs.png}
    \caption{The Markov chain for updates.}
    \label{fig:shs}
\end{figure}

A Markov chain representing the discrete state $q(t)$ is depicted in Figure~\ref{fig:shs}. The corresponding transitions of the SHS at state $q_l$ are detailed in Table~\ref{shs_table}. In the figure, a directed edge $l$ from node $q$ to node $q'$ indicates that transitions from state $q$ to state $q'$ occur at an exponential rate $\lambda^{(l)}$, as specified in the table. 





%\suresh{Incomplete.}

We first show that the stationary probability vector $\pi$ satisfies $
\mathbf{\pi D} = \mathbf{\pi Q} \quad \text{with}$ 
\begin{align}
\quad
\mathbf{D} = \text{diag}[\lambda_{C}, \mu + \Tilde{\lambda}_{C}, \mu + \Tilde{\lambda}_{1}], \quad  \\ \mathbf{Q} = 
\begin{bmatrix}
0 & \Tilde{\lambda}_{1}+\dot{\lambda}_{1} & \lambda_{C}-\Tilde{\lambda}_{1}-\dot{\lambda}_{1} \\
\mu & \Tilde{\lambda}_{1} & \Tilde{\lambda}_{C}-\Tilde{\lambda}_{1} \\
\mu & \Tilde{\lambda}_{1} & 0
\end{bmatrix}.
\end{align}
Applying $\sum_{i=0}^{2} \pi_i = 1$, the stationary probabilities are 
\begin{equation}
\pi_0 = \frac{\mu}{(\lambda_C + \mu)}, \label{pi0}
\end{equation}
\begin{equation}
\pi_1 = \frac{\lambda_C\Tilde{\lambda}_{1} + \dot{\lambda}_{1}\mu + \Tilde{\lambda}_{1}\mu}{(\lambda_C + \mu)(\Tilde{\lambda}_{C} + \mu)}, \label{pi1}
\end{equation}
\begin{equation}
\pi_2 = \frac{\Tilde{\lambda}_{C}\lambda_C + \lambda_C\mu -\lambda_C\Tilde{\lambda}_{1}  - \dot{\lambda}_{1}\mu - \Tilde{\lambda}_{1}\mu}{(\lambda_C + \mu)(\Tilde{\lambda}_{C} + \mu)} . \label{pi2}
\end{equation}

\section{}\label{aoi-appendix}





Given the SHS model and $\pi$ in Appendix \ref{spv-appendix}, we can evaluate $\bar{v}$ to find the AoI. Let 
\begin{equation}
\mathbf{\bar{v}} = [\mathbf{\bar{v}_0} \ \mathbf{\bar{v}_1} \ \mathbf{\bar{v}_2}] = [\bar{v}_{00} \ \bar{v}_{01} \ \bar{v}_{10} \ \bar{v}_{11} \ \bar{v}_{20} \ \bar{v}_{21}].   
\end{equation}
It follows that
\begin{equation}
\mathbf{\bar{v}D} = \mathbf{\pi B} +  \mathbf{\bar{v}R},
\end{equation}
where 
\begin{equation}
\mathbf{D} = \text{diag}[\lambda_C, \lambda_C, \mu + \Tilde{\lambda}_{C}, \mu + \Tilde{\lambda}_{C}, \mu + \Tilde{\lambda}_{1}, \mu + \Tilde{\lambda}_{1}],
\end{equation}
\begin{equation}
\mathbf{B} =
\begin{bmatrix}
1 & 0 & 0 & 0 & 0 & 0 \\
0 & 0 & 1 & 1 & 0 & 0 \\
0 & 0 & 0 & 0 & 1 & 0
\end{bmatrix},
\end{equation}
and
\begin{equation}
\mathbf{R} = 
\begin{bmatrix}
0 & 0 & \Tilde{\lambda}_{1}+\dot{\lambda}_{1}  & 0 & \lambda_{C}-\Tilde{\lambda}_{1}-\dot{\lambda}_{1} & 0 \\
0 & 0 & 0 & 0 & 0 & 0 \\
0 & 0 & \Tilde{\lambda}_{1} & 0 & \Tilde{\lambda}_{C}-\Tilde{\lambda}_{1} & 0 \\
\mu & 0 & 0 & 0 & 0 & 0 \\
\mu & 0 & \Tilde{\lambda}_{1} & 0 & 0 & 0 \\
0 & 0 & 0 & 0 & 0 & 0
\end{bmatrix}.
\end{equation}

Then, we obtain $\bar{v}_{01}=\bar{v}_{21} = 0 $ and 
\begin{align}
&
\label{pi_v}
\begin{bmatrix}
\bar{\pi}_0 & \bar{\pi}_1 & \bar{\pi}_1 & \bar{\pi}_2
\end{bmatrix}
= \\ \nonumber \hat{\mathbf{v}}&
\begin{bmatrix}
\lambda_{C} & -\Tilde{\lambda}_{1}-\dot{\lambda}_{1} & 0 & \Tilde{\lambda}_{1}+\dot{\lambda}_{1}-\lambda_{C} \\
0 & \mu + \Tilde{\lambda}_{C}-\Tilde{\lambda}_{1} & 0 & \Tilde{\lambda}_{1} - \Tilde{\lambda}_{C} \\
-\mu & 0 & \mu + \Tilde{\lambda}_{C} & 0 \\
-\mu & -\Tilde{\lambda}_{1} & 0 & \mu + \Tilde{\lambda}_{1}
\end{bmatrix}, \\
\text{where } 
\hat{\mathbf{v}} &= 
\begin{bmatrix}
\bar{v}_{00} & \bar{v}_{10} & \bar{v}_{11} & \bar{v}_{20}
\end{bmatrix}. \nonumber
\end{align}

After solving eq. (\ref{pi_v}) using eqs. (\ref{pi0}), (\ref{pi1}), and (\ref{pi2}), we determine $\mathbf{\bar{v}}$. Later, we find the average age of information using the formula for a single process $j$ $\Delta_j = \sum_{q=0}^2 \bar{v}_{10}$ as follows:

%\suresh{Is this what you defined as $\Delta_{\rm sum}$ earlier?}

\footnotesize
\begin{align}
\Delta_j = \frac{\lambda_{C}^{2} \tilde{\lambda}_C + \lambda_{C}^{2} \mu + \lambda_{C} \dot{\lambda}_1 \mu + 2 \lambda_{C} \tilde{\lambda}_C \mu + 2 \lambda_{C} \mu^{2} + \tilde{\lambda}_C \mu^{2} + \mu^{3}}{\mu \left(\lambda_{C}^{2} \tilde{\lambda}_1 + \lambda_{C} \dot{\lambda}_1 \mu + 2 \lambda_{C} \tilde{\lambda}_1 \mu + \dot{\lambda}_1 \mu^{2} + \tilde{\lambda}_1 \mu^{2}\right)}
\end{align}
\normalsize

\section{}\label{iteration-appendix}


In this section, we discuss the upper bound on the number of iterations required by the outer space accelerating branch-and-bound algorithm to achieve a global $\epsilon_0$-optimal solution. According to Theorem 5 in \cite{JIAO2022112701}, for any given positive error $\epsilon_0 \in (0, 1)$, the algorithm converges to the desired solution in at most
\begin{equation}
p \cdot \left\lceil \log_2 \frac{p\tau \delta(\Omega)}{\epsilon_0} \right\rceil 
\end{equation}
iterations.

Here, the symbols used in the theorem are defined as follows:

\begin{itemize}
    \item $\Omega \subseteq \mathbf{R}^p$ is a compact hyper-subrectangle, and $\delta(\Omega)$ is defined as:
    \begin{equation}
    \delta(\Omega) = \max_{i=1,2,\dots,p} \{ \bar{U}_i - \bar{L}_i \},    
    \end{equation}
    where $\bar{U}_i$ and $\bar{L}_i$ represent the upper and lower bounds of the $i$-th dimension of the rectangle $\Omega$.

    \item $\tau$ is defined as:
    \begin{equation}\label{tau_eq}
    \tau = \max_{i=1,\dots,p} \frac{4 \max\{|\bar{l}_i|, |\bar{u}_i|\}}{\min\{\bar{L}_i, \bar{U}_i, \bar{L}_i^2, \bar{U}_i^2\}},
    \end{equation}
    where the terms are determined as follows:
    \begin{align}
        \bar{l}_i &= \min_{y \in \Theta} n_i(y), \quad \bar{u}_i = \max_{y \in \Theta} n_i(y), \nonumber \\
        \bar{L}_i &= \min_{y \in \Theta} d_i(y), \quad \bar{U}_i = \max_{y \in \Theta} d_i(y).
    \end{align}

    \item The terms $n_i(y)$ and $d_i(y)$ come from the problem defined as:
    \begin{align}
    \quad \min f(y) = \sum_{i=1}^p \frac{n_i(y)}{d_i(y)}, \quad \nonumber \\ \text{s.t.} \; y \in \Theta = \{y \in \mathbf{R}^n \mid Ay \leq b \}. 
    \end{align}
    \end{itemize}


We can reformulate our problem to determine the upper bound using these definitions. The variable in our problem is $\bfp$, and the objective is specified in (\ref{objective_func}). There are $M$ different linear fractions in the objective. The numerators of these fractions increase as any element of $\bfp$ increases. Consequently, we obtain $\bar{l}_i$ when $\bfp = 0$ and $\bar{u}_i$ when $\bfp = 1$ as follows:
\begin{align}
            \bar{l}_i &= \mu(\mu + \lambda_C)^2 + \sum_{i=1}^{N} \lambda_{i}\mu\lambda_C \nc_{ij},\quad \bar{u}_i = (\mu + \lambda_{C})^3
\end{align}

 Similarly, the denominators of these fractions decrease as any element of $\bfp$ increases, leading to $\bar{L}_i$ when $\bfp = 0$ and $\bar{U}_i$ when $\bfp = 1$.
 \begin{align}
            \bar{L}_i &=  (\mu + \lambda_C) \mu^2 \sum_{i=1}^{N} \nc_{ij} \lambda_{i}, \quad \bar{U}_i =  (\mu + \lambda_C)^2 \mu \sum_{i=1}^{N} \nc_{ij} \lambda_{i}.
\end{align}

After that, $\delta(\Omega)$ becomes: 

\begin{align}
        \delta(\Omega) = \max_{i=1,2,\dots,M} \{(\mu + \lambda_C)\lambda_C \mu \sum_{i=1}^{N} \nc_{ij} \lambda_{i}\} \leq (\mu + \lambda_C)\lambda_C^2 \mu , 
\end{align}

Last, we find $\tau$. In our problem, all parameters and variables are positive, so both the nominators and the denominators are positive, which can help us simplify eq. (\ref{tau_eq}) and obtain $\tau$ as follows:

    \begin{align}
    \tau = \max_{i=1,\dots,M} \frac{4 \bar{u}_i}{\bar{L}_i^2} = \frac{4 (\mu + \lambda_{C})}{\mu^4 \hat{\lambda}_{\min}^2},\\ \nonumber
    \text{where } \hat{\lambda}_{\min} = \min(\boldsymbol{\lambda}^T \bfc)
    \end{align}

Putting all together, for any given positive error $\epsilon_0 \in (0, 1)$, the outer space accelerating branch-and-bound algorithm can seek out a global $\epsilon_0$-optimum solution in at most 
\begin{equation}
M \cdot \left\lceil \log_2 \frac{4M (\mu+\lambda_C)^2\lambda_C^2}{\epsilon_0\mu^3\hat{\lambda}_{\min}^{2}} \right\rceil
\end{equation}
iterations as shown in Theorem \ref{Theo2}.
% \input{Sections/8-Acknowledgements}

\balance
\bibliographystyle{ACM-Reference-Format}
\bibliography{reference}

% \appendix
% \section{} 
\label{reduction-P}

To prove our argument, we apply the splitting property of the Poisson process. Let \( N(t) \) be a Poisson process with rate parameter \( \lambda \). If events are split into two groups with probabilities \( p \) and \( 1-p \), then the resulting processes \( N_1(t) \) and \( N_2(t) \) are independent Poisson processes with rate parameters \( p\lambda \) and \( (1-p)\lambda \) respectively \cite{splitting_poisson}.

From process \( j \)'s perspective, we can split arrivals from sensor \( i \) into two groups: informative and uninformative arrivals with probabilities \( \nc_{ij} \) and \( 1-\nc_{ij} \), respectively. The rate of arrivals from sensor \( i \) is \( \lambda_i \), so the rate of informative arrivals for process \( j \) from sensor \( i \) is \( \nc_{ij}\lambda_i \). Additionally, we can further split the informative arrivals based on whether they can preempt ongoing service. The rate of informative arrivals that can preempt ongoing service for process \( j \) from sensor \( i \) is \( \np_{i}\nc_{ij}\lambda_i \) and the rate of informative arrivals that can not preempt ongoing service for process \( j \) from sensor \( i \) is \( (1-\np_{i})\nc_{ij}\lambda_i \). Since all these arrivals are Poisson, we can merge them into a single process. The total arrival rate of informative packets that can preempt ongoing service for process \( j \) is given by

\begin{equation}
\Tilde{\lambda}_j = \sum_{i=1}^{N} \np_{i}\nc_{ij}\lambda_i
\end{equation}

Similarly, the total arrival rate of informative packets that can not preempt ongoing service for process \( j \) is

\begin{equation}
\Tilde{\lambda}_j = \sum_{i=1}^{N} (1-\np_{i})\nc_{ij}\lambda_i
\end{equation}

We can express these rates in vector form as follows:

\begin{equation}
\boldsymbol{\Tilde{\lambda}}^T = \begin{bmatrix}
\Tilde{\lambda}_{1} & \Tilde{\lambda}_{2} & \dots & \Tilde{\lambda}_{M}
\end{bmatrix} = (\boldsymbol{\lambda}^T \odot \bfp^T) \bfc,
\end{equation}
\begin{equation}
\boldsymbol{\dot{\lambda}}^T = \begin{bmatrix}
\dot{\lambda}_{1} & \dot{\lambda}_{2} & \dots & \dot{\lambda}_{M}
\end{bmatrix} = (\boldsymbol{\lambda}^T \odot (1-\bfp^T)) \bfc,
\end{equation}


The importance of the packet is whether it has information of process $j$ so  we can say that The system with $N$ sensors and arrival rates $\boldsymbol{\lambda}$ shown in Figure \ref{fig:system_model} equivalents to the system with two sources as shown in Figure \ref{fig:equiv_model} from process $j$'s perspective.


\section{}\label{spv-appendix}


We adopt the stochastic hybrid system (SHS) model as defined in \cite{yates2019}, with a key distinction: our model incorporates probabilistic preemption. The system dynamics are depicted in Figure \ref{fig:equiv_model} so we can analyze the AoI for any process $i$ and generalize it. First, the discrete state is denoted as $q(t) = q \in Q = \{0, 1, 2\}$, where $q = 0$ represents an idle server, and $q \in \{1, 2\}$ signifies that an update packet is currently being serviced. The continuous state is described as $x(t) = [x_0(t), x_1(t)]$, where $x_0(t)$ represents the current age of the process, and $x_1(t)$ captures the potential age if the packet in service is successfully delivered. Notably, $x_1(t)$ is irrelevant in state $0$ since no packet is in service. In state $1$, $x_1(t)$ corresponds to the age of the informative update being serviced. Conversely, in state $2$, where an uninformative update is in service, the completion of this update does not affect the process age, rendering $x_1(t)$ irrelevant in this state as well.

\begin{table}[h]
\centering
\caption{Table of Transitions for the Markov Chain in Figure \ref{fig:shs}.}
\begin{tabular}{c c c c c c}
\toprule
$l$ & $q_l \rightarrow q'_l$ & $\lambda^{(l)}$ & $\mathbf{xA}_l$ & $\mathbf{A}_l$ & $\mathbf{v}_{q_l}\mathbf{A}_l$ \\
\midrule
1 & $0 \rightarrow 1$ & $\Tilde{\lambda}_{1}+\dot{\lambda}_{1}$ & $\begin{bmatrix} x_0 & 0 \end{bmatrix}$ & \small $\begin{bmatrix} 1 & 0 \\ 0 & 0 \end{bmatrix}$ \normalsize & $\begin{bmatrix} v_{00} & 0 \end{bmatrix}$ \\
2 & $0 \rightarrow 2$ & $\lambda_{C}-\Tilde{\lambda}_{1}-\dot{\lambda}_{1}$ & $\begin{bmatrix} x_0 & 0 \end{bmatrix}$ & \small $\begin{bmatrix} 1 & 0 \\ 0 & 0 \end{bmatrix}$ \normalsize & $\begin{bmatrix} v_{00} & 0 \end{bmatrix}$ \\
3 & $1 \rightarrow 0$ & $\mu$        & $\begin{bmatrix} x_1 & 0 \end{bmatrix}$ & \small$\begin{bmatrix} 0 & 0 \\ 1 & 0 \end{bmatrix}$ \normalsize & $\begin{bmatrix} v_{11} & 0 \end{bmatrix}$ \\
4 & $1 \rightarrow 1$ & $\Tilde{\lambda}_{1}
$  & $\begin{bmatrix} x_0 & 0 \end{bmatrix}$ & \small$\begin{bmatrix} 1 & 0 \\ 0 & 0 \end{bmatrix}$\normalsize & $\begin{bmatrix} v_{10} & 0 \end{bmatrix}$ \\
5 & $1 \rightarrow 2$ & $\Tilde{\lambda}_{C}-\Tilde{\lambda}_{1}$  & $\begin{bmatrix} x_0 & 0 \end{bmatrix}$ & \small$\begin{bmatrix} 1 & 0 \\ 0 & 0 \end{bmatrix}$ \normalsize & $\begin{bmatrix} v_{10} & 0 \end{bmatrix}$ \\
6 & $2 \rightarrow 0$ & $\mu$        & $\begin{bmatrix} x_0 & 0 \end{bmatrix}$ & \small$\begin{bmatrix} 0 & 0 \\ 1 & 0 \end{bmatrix}$ \normalsize & $\begin{bmatrix} v_{20} & 0 \end{bmatrix}$ \\
7 & $2 \rightarrow 1$ & $\Tilde{\lambda}_{1}$  & $\begin{bmatrix} x_0 & 0 \end{bmatrix}$ & \small$\begin{bmatrix} 1 & 0 \\ 0 & 0 \end{bmatrix}$ \normalsize & $\begin{bmatrix} v_{20} & 0 \end{bmatrix}$ \\
\bottomrule
\end{tabular}
\label{shs_table}
\end{table}


\begin{figure}
    \centering
    \includegraphics[width=0.5\linewidth]{figures/shs.png}
    \caption{The Markov chain for updates.}
    \label{fig:shs}
\end{figure}

A Markov chain representing the discrete state $q(t)$ is depicted in Figure~\ref{fig:shs}. The corresponding transitions of the SHS at state $q_l$ are detailed in Table~\ref{shs_table}. In the figure, a directed edge $l$ from node $q$ to node $q'$ indicates that transitions from state $q$ to state $q'$ occur at an exponential rate $\lambda^{(l)}$, as specified in the table. 





%\suresh{Incomplete.}

We first show that the stationary probability vector $\pi$ satisfies $
\mathbf{\pi D} = \mathbf{\pi Q} \quad \text{with}$ 
\begin{align}
\quad
\mathbf{D} = \text{diag}[\lambda_{C}, \mu + \Tilde{\lambda}_{C}, \mu + \Tilde{\lambda}_{1}], \quad  \\ \mathbf{Q} = 
\begin{bmatrix}
0 & \Tilde{\lambda}_{1}+\dot{\lambda}_{1} & \lambda_{C}-\Tilde{\lambda}_{1}-\dot{\lambda}_{1} \\
\mu & \Tilde{\lambda}_{1} & \Tilde{\lambda}_{C}-\Tilde{\lambda}_{1} \\
\mu & \Tilde{\lambda}_{1} & 0
\end{bmatrix}.
\end{align}
Applying $\sum_{i=0}^{2} \pi_i = 1$, the stationary probabilities are 
\begin{equation}
\pi_0 = \frac{\mu}{(\lambda_C + \mu)}, \label{pi0}
\end{equation}
\begin{equation}
\pi_1 = \frac{\lambda_C\Tilde{\lambda}_{1} + \dot{\lambda}_{1}\mu + \Tilde{\lambda}_{1}\mu}{(\lambda_C + \mu)(\Tilde{\lambda}_{C} + \mu)}, \label{pi1}
\end{equation}
\begin{equation}
\pi_2 = \frac{\Tilde{\lambda}_{C}\lambda_C + \lambda_C\mu -\lambda_C\Tilde{\lambda}_{1}  - \dot{\lambda}_{1}\mu - \Tilde{\lambda}_{1}\mu}{(\lambda_C + \mu)(\Tilde{\lambda}_{C} + \mu)} . \label{pi2}
\end{equation}

\section{}\label{aoi-appendix}





Given the SHS model and $\pi$ in Appendix \ref{spv-appendix}, we can evaluate $\bar{v}$ to find the AoI. Let 
\begin{equation}
\mathbf{\bar{v}} = [\mathbf{\bar{v}_0} \ \mathbf{\bar{v}_1} \ \mathbf{\bar{v}_2}] = [\bar{v}_{00} \ \bar{v}_{01} \ \bar{v}_{10} \ \bar{v}_{11} \ \bar{v}_{20} \ \bar{v}_{21}].   
\end{equation}
It follows that
\begin{equation}
\mathbf{\bar{v}D} = \mathbf{\pi B} +  \mathbf{\bar{v}R},
\end{equation}
where 
\begin{equation}
\mathbf{D} = \text{diag}[\lambda_C, \lambda_C, \mu + \Tilde{\lambda}_{C}, \mu + \Tilde{\lambda}_{C}, \mu + \Tilde{\lambda}_{1}, \mu + \Tilde{\lambda}_{1}],
\end{equation}
\begin{equation}
\mathbf{B} =
\begin{bmatrix}
1 & 0 & 0 & 0 & 0 & 0 \\
0 & 0 & 1 & 1 & 0 & 0 \\
0 & 0 & 0 & 0 & 1 & 0
\end{bmatrix},
\end{equation}
and
\begin{equation}
\mathbf{R} = 
\begin{bmatrix}
0 & 0 & \Tilde{\lambda}_{1}+\dot{\lambda}_{1}  & 0 & \lambda_{C}-\Tilde{\lambda}_{1}-\dot{\lambda}_{1} & 0 \\
0 & 0 & 0 & 0 & 0 & 0 \\
0 & 0 & \Tilde{\lambda}_{1} & 0 & \Tilde{\lambda}_{C}-\Tilde{\lambda}_{1} & 0 \\
\mu & 0 & 0 & 0 & 0 & 0 \\
\mu & 0 & \Tilde{\lambda}_{1} & 0 & 0 & 0 \\
0 & 0 & 0 & 0 & 0 & 0
\end{bmatrix}.
\end{equation}

Then, we obtain $\bar{v}_{01}=\bar{v}_{21} = 0 $ and 
\begin{align}
&
\label{pi_v}
\begin{bmatrix}
\bar{\pi}_0 & \bar{\pi}_1 & \bar{\pi}_1 & \bar{\pi}_2
\end{bmatrix}
= \\ \nonumber \hat{\mathbf{v}}&
\begin{bmatrix}
\lambda_{C} & -\Tilde{\lambda}_{1}-\dot{\lambda}_{1} & 0 & \Tilde{\lambda}_{1}+\dot{\lambda}_{1}-\lambda_{C} \\
0 & \mu + \Tilde{\lambda}_{C}-\Tilde{\lambda}_{1} & 0 & \Tilde{\lambda}_{1} - \Tilde{\lambda}_{C} \\
-\mu & 0 & \mu + \Tilde{\lambda}_{C} & 0 \\
-\mu & -\Tilde{\lambda}_{1} & 0 & \mu + \Tilde{\lambda}_{1}
\end{bmatrix}, \\
\text{where } 
\hat{\mathbf{v}} &= 
\begin{bmatrix}
\bar{v}_{00} & \bar{v}_{10} & \bar{v}_{11} & \bar{v}_{20}
\end{bmatrix}. \nonumber
\end{align}

After solving eq. (\ref{pi_v}) using eqs. (\ref{pi0}), (\ref{pi1}), and (\ref{pi2}), we determine $\mathbf{\bar{v}}$. Later, we find the average age of information using the formula for a single process $j$ $\Delta_j = \sum_{q=0}^2 \bar{v}_{10}$ as follows:

%\suresh{Is this what you defined as $\Delta_{\rm sum}$ earlier?}

\footnotesize
\begin{align}
\Delta_j = \frac{\lambda_{C}^{2} \tilde{\lambda}_C + \lambda_{C}^{2} \mu + \lambda_{C} \dot{\lambda}_1 \mu + 2 \lambda_{C} \tilde{\lambda}_C \mu + 2 \lambda_{C} \mu^{2} + \tilde{\lambda}_C \mu^{2} + \mu^{3}}{\mu \left(\lambda_{C}^{2} \tilde{\lambda}_1 + \lambda_{C} \dot{\lambda}_1 \mu + 2 \lambda_{C} \tilde{\lambda}_1 \mu + \dot{\lambda}_1 \mu^{2} + \tilde{\lambda}_1 \mu^{2}\right)}
\end{align}
\normalsize

\section{}\label{iteration-appendix}


In this section, we discuss the upper bound on the number of iterations required by the outer space accelerating branch-and-bound algorithm to achieve a global $\epsilon_0$-optimal solution. According to Theorem 5 in \cite{JIAO2022112701}, for any given positive error $\epsilon_0 \in (0, 1)$, the algorithm converges to the desired solution in at most
\begin{equation}
p \cdot \left\lceil \log_2 \frac{p\tau \delta(\Omega)}{\epsilon_0} \right\rceil 
\end{equation}
iterations.

Here, the symbols used in the theorem are defined as follows:

\begin{itemize}
    \item $\Omega \subseteq \mathbf{R}^p$ is a compact hyper-subrectangle, and $\delta(\Omega)$ is defined as:
    \begin{equation}
    \delta(\Omega) = \max_{i=1,2,\dots,p} \{ \bar{U}_i - \bar{L}_i \},    
    \end{equation}
    where $\bar{U}_i$ and $\bar{L}_i$ represent the upper and lower bounds of the $i$-th dimension of the rectangle $\Omega$.

    \item $\tau$ is defined as:
    \begin{equation}\label{tau_eq}
    \tau = \max_{i=1,\dots,p} \frac{4 \max\{|\bar{l}_i|, |\bar{u}_i|\}}{\min\{\bar{L}_i, \bar{U}_i, \bar{L}_i^2, \bar{U}_i^2\}},
    \end{equation}
    where the terms are determined as follows:
    \begin{align}
        \bar{l}_i &= \min_{y \in \Theta} n_i(y), \quad \bar{u}_i = \max_{y \in \Theta} n_i(y), \nonumber \\
        \bar{L}_i &= \min_{y \in \Theta} d_i(y), \quad \bar{U}_i = \max_{y \in \Theta} d_i(y).
    \end{align}

    \item The terms $n_i(y)$ and $d_i(y)$ come from the problem defined as:
    \begin{align}
    \quad \min f(y) = \sum_{i=1}^p \frac{n_i(y)}{d_i(y)}, \quad \nonumber \\ \text{s.t.} \; y \in \Theta = \{y \in \mathbf{R}^n \mid Ay \leq b \}. 
    \end{align}
    \end{itemize}


We can reformulate our problem to determine the upper bound using these definitions. The variable in our problem is $\bfp$, and the objective is specified in (\ref{objective_func}). There are $M$ different linear fractions in the objective. The numerators of these fractions increase as any element of $\bfp$ increases. Consequently, we obtain $\bar{l}_i$ when $\bfp = 0$ and $\bar{u}_i$ when $\bfp = 1$ as follows:
\begin{align}
            \bar{l}_i &= \mu(\mu + \lambda_C)^2 + \sum_{i=1}^{N} \lambda_{i}\mu\lambda_C \nc_{ij},\quad \bar{u}_i = (\mu + \lambda_{C})^3
\end{align}

 Similarly, the denominators of these fractions decrease as any element of $\bfp$ increases, leading to $\bar{L}_i$ when $\bfp = 0$ and $\bar{U}_i$ when $\bfp = 1$.
 \begin{align}
            \bar{L}_i &=  (\mu + \lambda_C) \mu^2 \sum_{i=1}^{N} \nc_{ij} \lambda_{i}, \quad \bar{U}_i =  (\mu + \lambda_C)^2 \mu \sum_{i=1}^{N} \nc_{ij} \lambda_{i}.
\end{align}

After that, $\delta(\Omega)$ becomes: 

\begin{align}
        \delta(\Omega) = \max_{i=1,2,\dots,M} \{(\mu + \lambda_C)\lambda_C \mu \sum_{i=1}^{N} \nc_{ij} \lambda_{i}\} \leq (\mu + \lambda_C)\lambda_C^2 \mu , 
\end{align}

Last, we find $\tau$. In our problem, all parameters and variables are positive, so both the nominators and the denominators are positive, which can help us simplify eq. (\ref{tau_eq}) and obtain $\tau$ as follows:

    \begin{align}
    \tau = \max_{i=1,\dots,M} \frac{4 \bar{u}_i}{\bar{L}_i^2} = \frac{4 (\mu + \lambda_{C})}{\mu^4 \hat{\lambda}_{\min}^2},\\ \nonumber
    \text{where } \hat{\lambda}_{\min} = \min(\boldsymbol{\lambda}^T \bfc)
    \end{align}

Putting all together, for any given positive error $\epsilon_0 \in (0, 1)$, the outer space accelerating branch-and-bound algorithm can seek out a global $\epsilon_0$-optimum solution in at most 
\begin{equation}
M \cdot \left\lceil \log_2 \frac{4M (\mu+\lambda_C)^2\lambda_C^2}{\epsilon_0\mu^3\hat{\lambda}_{\min}^{2}} \right\rceil
\end{equation}
iterations as shown in Theorem \ref{Theo2}.


\end{document}
