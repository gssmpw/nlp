%!TEX root = catuscia.tex

%% file conclusion.tex

\section{Concluding remarks}

\label{sec-concl}

We presented an axiomatization of rooted branching probabilistic
{\bisimilarity} and proved its soundness and completeness. In doing
so, we aimed to stay close to a straightforward completeness proof for
the axiomatization of rooted branching {\bisimilarity} for
non-deterministic processes that employed concrete processes, which is
also presented in this paper. In particular, the route via concrete
processes guided us to find the right formulation of the axioms
\hyperlink{BP}{BP} and~\hyperlink{G}{G} for branching {\bisimilarity}
in the probabilistic case.

Future work will include the study of the extension
of the setting of the present paper with a parallel
operator~\cite{DPP05:klop}. In particular a congruence result for the
parallel operator should be obtained, which for the mixed
non-deterministic and probabilistic setting can be challenging.
\hspace{-.4pt}Also the inclusion of recursion\,\cite{DP07:tcs,FG19:jlamp}
is a clear direction for further research. 

The present conditional form of axioms \hyperlink{BP}{BP}
and~\hyperlink{G}{G} is only semantically motivated. However, the
axiom~\hyperlink{G}{G} has a purely syntactic counterpart of the form%
\footnote{The extended abstract of this paper \cite{GGV19} also proposed
 a purely syntactic counterpart of axiom B; this turned out to be incorrect,
 however.}
\begin{displaymath}
  \begin{array}{l}
    \alpha \pref \big( \,
    \partial\dg( \, \sumiinI \: \tau \pref 
    \mg(P_i \probc{r_i} \partial \dor(E+\sumiinI \tau \pref P_i\dor)
    \mg)
    + E + \sumiinI \: \tau \pref P_i\dg) \oplusr Q \, \big) 
    \smallskip \\
     \hspace{134pt}
    \qquad {} = \;
    \alpha \pref ( \, \partial \dor( E + \sumiinI \: \tau \pref P_i \dor)
    \oplusr Q \, )\;.
  \end{array}
\end{displaymath}
Admittedly, this form is a bit complicated to work with. An alternative
approach could be to axiomatize the relation~$\sqsubseteq$, or perhaps to
introduce and axiomatize an auxiliary process operator~$+'$ such that
$E \sqsubseteq P$ can be translated into the condition $E +' P = P$
or similar.

Also, we want to develop a minimization algorithm for probabilistic
processes modulo branching probabilistic {\bisimilarity}.  Eisentraut
et al.\ propose in~\cite{EHKTZ13:qest} an algorithm for deciding
equivalence with respect to weak distribution {\bisimilarity} relying
on a state-based characterization, a result presently not available in
our setting.  Other work and proposals for weak {\bisimilarity}
include~\cite{CS02:concur,FHHT16:fac,TH15:ic}, but these do not fit
well with the installed base of our toolset~\cite{Bun19:tacas}. For
the case of strong probabilistic {\bisimilarity} without combined
transitions we recently developed in~\cite{GRV18:algorithms} an
algorithm improving upon the early results of~\cite{BEM00:jcss}. In
\cite{TH15:ic} a polynomial algorithm for Segala's probabilistic
branching {\bisimilarity}, which differs from our notion of
probabilistic branching {\bisimilarity}, is defined. We hope to arrive
at an efficient algorithm by combining ideas
from~\cite{Val10:fi,VF10:tacas,TH15:ic} and
of~\cite{GV90:icalp,GJKW17:tcl}.
