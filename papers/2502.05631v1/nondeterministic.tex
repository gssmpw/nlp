%!TEX root = catuscia.tex

%% file non-deterministic.tex

\section{Completeness: the non-deterministic case}

\label{sec-nondet}

In this section we present 
an approach to prove completeness of an axiomatic theory for branching
{\bisimilarity} exploiting the notion of a concrete process in the setting
of a basic process language. In the remainder of the paper we
extend the approach to a process language involving probabilistic choice.

\blankline

\noindent
We assume to be given a set of actions~$\calA$ including the so-called
silent action~$\tau$. The process language we consider is called a
Minimal Process Language in~\cite{BBR10:cup}. It provides
inaction~$\bfzero$, a prefix construct for each action $a \in \calA$,
and non-deterministic choice.

\begin{definition}[Syntax]
  The class~$\calE$ of non-deterministic processes over~$\calA$, with
  typical element~$E$, is given by\vspace{-.5ex}
  \[
    E \bnfeq \bfzero \mid \alpha \pref E \mid E + E\vspace{-1.5ex}
  \]
  with actions~$\alpha$ from~$\calA$.
\pagebreak[3]
\end{definition}

\noindent
The process $\bfzero$ cannot perform any action, $\alpha \pref E$ can
perform action $\alpha$ and subsequently behave as $E$, and $E_1+E_2$
represents the choice in behavior between $E_1$ and~$E_2$.

For $E \in \calE$ we define its complexity~$c(E)$ by $c(\bfzero) = 0$,
$c(\alpha \pref E) = c(E) + 1$, and $c(E + F) = c(E) + c(F)$.

The behavior of processes in~$\calE$ is given by a structured
operational semantics going back to~\cite{HM80:icalp}.

\begin{definition}[Operational semantics]
  \label{def-nd-transition-relation}
  The transition relation ${\rightarrow} \subseteq \calE \times \calA
  \times \calE$ is
  given by
  \halflineup
  \begin{displaymath}
    \begin{array}{c}
      \sosrule{}{\alpha \pref E \arrow{\alpha} E}
      \: \textsc{\small (pref)}
     \medskip \\
      \sosrule{E_1 \arrow{\alpha} E_1}{E_1 + E_2 \arrow{\alpha} E_1}
      \: \textsc{\small (nd-choice\,1)}
      \qquad
      \sosrule{E_2 \arrow{\alpha} E_2}{E_1 + E_2 \arrow{\alpha} E_2}
      \: \textsc{\small (nd-choice\,2)}
   \end{array}
  \end{displaymath}
\end{definition}

\medskip

\noindent
We have auxiliary definitions and relations derived from the
transition relation of Definition~\ref{def-nd-transition-relation}.
%
A process $E' \in \calE$ is called a derivative of a process~$E \in
\calE$ iff $E_0, \ldots, E_n \in \calE$ and $\alpha_1, \ldots,
\alpha_n$ exist such that $E \equiv E_0$, $E_{i{-}1} \arrow{\alpha_i}
E_i$, and $E_n \equiv E'$.
%
We define $\der(E) = \lc E' \in \calE \mid \mbox{$E'$ derivative of~$E$}
\rc$.
%
Furthermore, for $E, E' \in \calE$ and $\alpha \in \calA$ we write $E
\alphahidearrow E'$ iff $E \alphaarrow E'$, or $\alpha = \tau$ and $E
= E'$. We use $\Arrow{}$ to denote the reflexive transitive closure of
$\arrow{(\tau)}$.

The definitions of strong and branching {\bisimilarity} for~$\calE$ are
standard and adapted from~\cite{GW96:jacm,Mil89:phi}.

\begin{definition}[Strong and branching {\bisimilarity}]
  \label{def-branching-bisimilar}
  \begin{itemize}
  \item [(a)] A symmetric relation $\calR \subseteq \calE \times
    \calE$ is called a \textit{strong bisimulation relation} iff for all
    $E, E', F \in \calE$ if $E \mopcalR F$ and $E
    \arrow{\alpha} E'$ then there is an $F' \in \calE$ such that
    \begin{displaymath}
      F \alphaarrow F' \ \text{and} \ 
      E' \mopcalR F'.
    \end{displaymath}
  \item [(b)] A symmetric relation $\calR \subseteq \calE \times
    \calE$ is called a \textit{branching bisimulation relation} iff for
    all $E, E', F \in \calE$ if $E \mopcalR F$ and $E
    \arrow{\alpha} E'$, then there are $\Fbar, F' \in \calE$ such that
    \begin{displaymath}
      F \Arrow{} \Fbar, \ 
      \Fbar \alphahidearrow F', \ 
      E \mopcalR \Fbar, \ \text{and} \ 
      E' \mopcalR F'.
    \end{displaymath}
  \item [(c)] Strong {\bisimilarity}, denoted by~${\bisim} \subseteq \calE
    \times \calE$, and branching {\bisimilarity}, written as~${\brbisim}
    \subseteq \calE \times \calE$, are defined as the largest strong
    bisimulation relation on~$\calE$ and the largest branching
    bisimulation relation on~$\calE$, respectively.
  \end{itemize}
\end{definition}

\noindent
Clearly, in view of the definitions, strong {\bisimilarity} between two
processes implies branching {\bisimilarity} between the two processes.

If for a transition $E \arrow{\tau} E'$ we have that $E \brbisim E'$,
the transition is called \textit{inert}. A process~$\Ebar$ is called
\textit{concrete} iff it has no inert transitions, i.e., if
$E'\in\der(\Ebar)$ and $E' \arrow{\tau} E''$, then~$E' \nbrbisim
E''$. We write $\calEconc = \lc \Ebar \in \calE \mid \mbox{$\Ebar$
  concrete} \rc$.

\blankline

\noindent
Next we introduce a restricted form of branching {\bisimilarity}, called
rooted branching {\bisimilarity}, instigated by the fact that branching
{\bisimilarity} itself is not a congruence for the choice operator. This
makes branching {\bisimilarity} unsuitable for equational reasoning where
it is natural to replace subterms by equivalent terms. Note that weak
{\bisimilarity} has the same problem~\cite{Mil89:phi}.

For example, we have for any process~$E$ that $E$ and~$\tau \pref E$
are branching bisimilar, but in the context of a non-deterministic
alternative they may not, i.e., it is not necessarily the case $E + F
\brbisim \tau \pref E + F$. More concretely, although $\bfzero
\brbisim \tau \pref \bfzero$, it does not hold that $\bfzero + b \pref
\bfzero \brbisim \tau \pref \bfzero + b \pref \bfzero$. The
$\tau$-move of $\tau \pref \bfzero + b \pref \bfzero$ to~$\bfzero$ has
no counterpart in $\bfzero + b \pref \bfzero$ because $\bfzero + b \pref
\bfzero \nbrbisim \bfzero$.

\begin{definition}
  \label{def-branching-bisimilarity}
  % Two processes $E, F \in \calE$ are called \emph{rooted} branching
  % bisimilar, notation~$E \rootedbrbisim F$, iff
  % \begin{itemize}
  % \item[(a)]
  % $E \arrow{\alpha} E'$
  % implies $F \arrow{\alpha} F'$ and~$E' \brbisim F'$ for some $F' \in
  % \calE$ and, vice versa, 
  % \item[(b)]
  % $F \arrow{\alpha} F'$ implies $E
  % \arrow{\alpha} E'$ and~$E' \brbisim F'$ for some $E' \in \calE$.
  % \end{itemize}
  A symmetric $\calR \mathbin\subseteq \calE \times \calE$ is called a
  rooted branching bisimulation relation iff for all $E, F \in \calE$
  such that $E \mopcalR F$ it holds that if $E \arrow{\alpha} E'$
  for~$\alpha \in \calA$, $E' \in \calE$ then $F \arrow{\alpha} F'$
  and $E' \brbisim F'$ for some~$F' \in \calE$. Rooted branching
  bisimilarity, denoted by
  ${\rootedbrbisim} \subseteq \calE \times \calE$, is defined as the
  largest rooted branching bisimulation relation.
\end{definition}

\noindent
The definition of rooted branching bisimilarity boils down to calling
processes $E, F \in \calE$ rooted branching bisimilar,
notation~$E \rootedbrbisim F$, iff (i)~$E \arrow{\alpha} E'$ implies
$F \arrow{\alpha} F'$ and~$E' \brbisim F'$ for some $F' \in \calE$
and, vice versa, (ii)~$F \arrow{\alpha} F'$ implies
$E \arrow{\alpha} E'$ and~$E' \brbisim F'$ for some $E' \in
\calE$. The formulation of Definition~\ref{def-branching-bisimilarity}
for the nondeterministic processes of this section corresponds
directly to the definition of rooted branching \emph{probabilistic}
bisimulation that we will introduce in Section~\ref{sec-bpb}, see
Definition~\ref{rooted-bpb}. 
 
\blankline

\noindent
Direct from the definitions we see ${\bisim} \subseteq
{\rootedbrbisim} \subseteq {\brbisim}$. As implicitly announced we
have a congruence result for rooted branching {\bisimilarity}.

\begin{lemma}[\cite{GW96:jacm}]
  $\rootedbrbisim$ is a congruence on~$\calE$ for the operators $\pref$ and $+$.
\end{lemma}

\noindent
It is well-known that strong and branching {\bisimilarity} for~$\calE$
can be equationally characterized~\cite{BBR10:cup,GW96:jacm,Mil89:phi}.

\begin{definition}[Axiomatization of $\bisim$ and~$\rootedbrbisim$]
  The theory~$\AXd$ is given by the axioms A1 to~A4 listed in
  Table~\ref{table-axiomatization-of-branching-bisimilarity}.
  The theory~$\AXdb$ contains in addition the axiom~\hyperlink{B}{B}.
\end{definition}

\begin{table}
  \vspace{-4ex}
  \centering
  \def\arraystretch{1.1}
  \begin{tabular}{|@{\;}l@{\ \;}l|}
    \hline
    A1 & $E + F = F + E$ \rule{0pt}{12pt} \\
    A2 & $(E + F) + G = E + ( F + G)$ \\
    A3 & $E + E = E$ \\
    A4 & $E + \bfzero = E$ 
    \rule[-5pt]{0pt}{12pt}
%%
    \\ \hline
    \hypertarget{B}{B} & $\alpha \pref ( \, F + \tau \pref ( E + F )
    \, ) = \alpha 
    \pref ( E + F )$
    \rule{0pt}{12pt}\rule[-5pt]{0pt}{12pt} \\ 
    \hline
  \end{tabular}
  \def\arraystretch{1.0}

  \medskip

  \caption{Axioms for strong and branching {\bisimilarity}}
  \label{table-axiomatization-of-branching-bisimilarity}
  \vspace{-4ex}
\end{table}

\noindent
If two processes are provably equal, they are rooted branching
bisimilar.

\begin{lemma} [Soundness]
  \label{lemma-soundness-base}
  For all $E,F \in \calE$, if $\AXdb \vdash E = F$ then $E
  \rootedbrbisim F$.
\end{lemma}

\begin{proof}[Sketch]
  First one shows that the left-hand side and the right-hand
  side of the axioms of~$\AXdb$ are rooted branching bisimilar.
%%
  Next, one observes that rooted branching {\bisimilarity} is a
  congruence.
\end{proof}


\noindent
It is well-known that strong {\bisimilarity} is equationally
characterized by the axioms A1 to~A4 of
Table~\ref{table-axiomatization-of-branching-bisimilarity}.

\begin{theorem}[$\AXd$ sound and complete for~$\bisim$]
  \label{theorem-AXd-sound-and-complete}
  For all processes $E, F \in \calE$ it holds that $\AXd \vdash E =
  F$ iff $E \bisim F$.
\end{theorem}

\begin{proof}
  See for example~\cite[Section~7.4]{Mil89:phi}.
\end{proof}

\noindent
For concrete processes that have no inert transitions, branching
{\bisimilarity} and strong {\bisimilarity} coincide. Hence, in view of
Theorem~\ref{theorem-AXd-sound-and-complete}, branching {\bisimilarity}
implies equality for~$\AXd$.

\begin{lemma}
  \label{lemma-branching-implies-equal-for-concrete-processes}
  For all concrete $\Ebar,\Fbar \in \calEconc$, if $\Ebar \brbisim
  \Fbar$ then both $\Ebar \bisim \Fbar$ and $\AXd \vdash \Ebar = \Fbar$.
\end{lemma}

\begin{proof}[Sketch]
  Consider $\Ebar, \Fbar \in \calEconc$ such that $\Ebar \brbisim
  \Fbar$. Let~$\calR$ be a branching bisimulation relation relating
  $\Ebar$ and~$\Fbar$. Define $\calR'$ as the restriction of~$\calR$
  to the derivatives of $\Ebar$ and~$\Fbar$, i.e., $\calR' = \calR
  \cap ( (\der(\Ebar) \times \der(\Fbar)) \cup (\der(\Fbar) \times
  \der(\Ebar)) )$. Then $\calR'$ is a strong bisimulation relation,
  since none of the processes involved admits an inert
  $\tau$-transition. By the completeness of~$\AXd$, see
  Theorem~\ref{theorem-AXd-sound-and-complete}, it follows that $\AXd
  \vdash \Ebar = \Fbar$.
\end{proof}

\noindent
We are now in a position to prove the main technical result of this
section, viz.\ that branching {\bisimilarity} implies equality under a
prefix. In the proof the notion of a concrete process plays a central
role. 

\begin{lemma}
  \label{lemma-branching-bisim-vs-equal-under-prefix}
  \mbox{}
  \begin{itemize}
  \item [(a)] For all processes $E \in \calE$, a
    concrete process $\Ebar \in \calEconc$ exists such that $E \brbisim
    \Ebar$ and $\AXdb \vdash { \alpha \pref E = \alpha \pref \Ebar }$
    for all $\alpha \in \calA$.
  \item [(b)] For all processes $F,G \in \calE$, if $F \brbisim
    G$ then $\AXdb \vdash { \alpha \pref F = \alpha \pref G }$ for all
    $\alpha \in \calA$.%
  \pagebreak[3]
 \end{itemize}
\end{lemma}

\begin{proof}
  We prove statements (a) and~(b) by simultaneously induction on 
  $c(E)$ and $\max \lbrace c(F), c(G) \rbrace$, respectively.

  Basis, $c(E) = 0$. We have that $E= \bfzero+\cdots+\bfzero$. Hence,
  take $\Ebar=\bfzero$. Clearly, part (a) of the lemma holds as
  $\bfzero$ is concrete, $E \brbisim \bfzero$ and $\AXdb \vdash\alpha
  \pref E = \alpha \pref \bfzero$ for all $\alpha \in \calA$.

  Induction step for (a): $c(E) > 0$. The process $E$ can be written as 
  ${ \sumiinI \: \alpha_i \pref E_i }$
  for some finite $I$ and suitable~$\alpha_i\in\calA$ and $E_i\in\calE$.

  First suppose that for some $i_0\in I$ we have $\alpha_{i_0} = \tau$
  and $E_{i_0} \brbisim E$.  Then $\AXd \vdash E = H + \tau.E_{i_0}$,
  where \plat{$H := \sum_{i\in I\setminus\{i_o\}} \alpha_i \pref
    E_i$}.
  \vspace{1pt} By the induction hypothesis~(a), there is a term
  $\Ebar_{i_0} \in \calEconc$ such that $E_{i_0} \brbisim
  \Ebar_{i_0}$. We claim that $\Ebar_{i_0} \brbisim E_{i_0}+H$.

  For suppose $\Ebar_{i_0} \alphaarrow F$.
  \vspace*{1pt}
  Then $E_{i_0} \Arrow{} E'_{i_0} \alphahidearrow G$ where
  $\Ebar_{i_0}\brbisim E'_{i_0}$ 
  and $F \brbisim G$. In case $E_{i_0} = E'_{i_0}$ it follows that $E_{i_0}
  \alphahidearrow G$. 
  Since $\Ebar_{i_o}$ is concrete, either $\alpha \neq \tau$ or $F
  \nbrbisim \Ebar_{i_0}$. 
  Hence, $\alpha \neq \tau$ or $G \nbrbisim E_{i_0}$. So $E_{i_0}
  \alphaarrow G$. 
  Consequently, $E_{i_0}+H \alphaarrow G$.\linebreak[3]
  In case $E_{i_0} \neq E'_{i_0}$ we have $E_{i_0}+H \Arrow{} E'_{i_0}
  \alphahidearrow G$. 

  Now suppose $E_{i_0} + H \alphaarrow F$.  Then either
  $E_{i_0} \alphaarrow F$ or $H \alphaarrow F$.  \vspace*{1pt} In the
  first case we have
  $\Ebar_{i_0} \Arrow{} \Ebar'_{i_0} \alphahidearrow G$ where
  $E_{i_0}\brbisim \Ebar'_{i_0}$ and $F \brbisim G$, \vspace{1pt}
  while in the latter case $E \alphaarrow F$, and since
  $E \brbisim E_{i_0} \brbisim \Ebar_{i_0}$ we have
  $\Ebar_{i_0} \Arrow{} \Ebar'_{i_0} \alphahidearrow G$ where
  $E\brbisim \Ebar'_{i_0}$ and $F \brbisim G$. Because $\Ebar_{i_0}$
  is concrete, $\Ebar'_{i_o} = \Ebar_{i_0}$.  Thus
  $\Ebar_{i_0} \alphahidearrow G$ with $F \brbisim G$, which was
  to be shown.

  Hence $E_{i_0} \brbisim \Ebar_{i_0} \brbisim E_{i_0}+H$. Clearly
  $c(E_{i_0}), c(E_{i_0}+H) < c(E)$.  Therefore, by the induction
  hypothesis (b),
  $\AXdb\vdash \tau \pref E_{i_0} = \tau \pref ( E_{i_0} + H )$.  By
  the induction hypothesis (a), there is a term $\Ebar \in \calEconc$
  such that $\Ebar \brbisim E_{i_0} + H$ \vspace{1pt} and
  $\AXdb\vdash \alpha \pref \Ebar = \alpha \pref ( E_{i_0} + H )$. Now
  we have $E \brbisim E_{i_0} \brbisim E_{i_0} + H \brbisim
  \Ebar$. Therefore,
  \begin{align*}
    \AXdb\vdash\alpha \pref E 
    & = \alpha \pref ( H + \tau \pref E_{i_0} ) 
    && \\
    & = \alpha \pref ( H + \tau \pref ( E_{i_0} + H ) ) 
    && (\text{since $\AXdb\vdash\tau \pref E_{i_0} = \tau \pref
      (E_{i_0} + H)$}) \\ 
    & = \alpha \pref ( E_{i_0} + H ) 
    && (\text{use axiom \hyperlink{B}{B}}) \\
    & = \alpha \pref \Ebar
    && (\text{by the choice of $\Ebar$}).
  \end{align*}
  Hence, we have shown the existence of a desired process $\Ebar$ with
  the required properties.

  Now suppose, for all $i \in I$ we have $\alpha_i \neq \tau$ or
  $E_i\nbrbisim E$. Clearly $c(E_i) < c(E)$ for all $i \in I$.  By
  the induction hypothesis we can find, for all $i \in I$, concrete
  $\Ebar_i$ such that $\Ebar_i \brbisim E_i$ and $\AXdb \vdash \alpha
  \pref \Ebar_i = \alpha \pref E_i$ for all $\alpha \in \calA$. Define
  $\Ebar = { \sumiinI \: \alpha_i \pref \Ebar_i }$. Then $\Ebar
  \brbisim E$ and $\Ebar$~is concrete too, since $\Ebar_i \brbisim E_i
  \nbrbisim E \brbisim \Ebar$ for~$i \in I$ in case $\alpha_i = \tau$.
  Moreover, $\AXdb\vdash E = \Ebar$, since $E = \sumiinI \: \alpha_i
  \pref E_i = \sumiinI \: \alpha_i \pref \Ebar_i = \Ebar$. Hence, for
  $\alpha \in \calA$, $\AXdb \vdash \alpha \pref E = \alpha \pref
  \Ebar$.

  Both the base and the induction step for (b): $\max \lbrace c(F),
  c(G) \rbrace \geqslant 0$. Suppose $F \brbisim G$. Pick $\Fbar,
  \Gbar \in \calEconc$ such that $F \brbisim \Fbar$ and
  $\AXdb\vdash\alpha \pref F = \alpha \pref \Fbar$ for all~$\alpha \in
  \calA$, and similarly for $G$ and~$\Gbar$. Then we have $\Fbar
  \brbisim \Gbar$. Since $\Fbar$ and~$\Gbar$ are concrete it follows
  that $\AXd\vdash\Fbar = \Gbar$, see
  Lemma~\ref{lemma-branching-implies-equal-for-concrete-processes}. Now
  pick any $\alpha \in \calA$. Then we have $\AXdb\vdash\alpha \pref F
  = \alpha \pref \Fbar = \alpha \pref \Gbar = \alpha \pref G$.
\end{proof}

\noindent
By now we have gathered sufficient building blocks to prove the main
result of this section.

\begin{theorem}[$\AXdb$ sound and complete for $\rootedbrbisim$]
  \label{theorem-completeness}
  For all processes $E, F \in \calE$ it holds that $E \rootedbrbisim
  F$ iff $\AXdb \vdash E = F$.
\end{theorem}

\begin{proof}
  In view of Lemma~\ref{lemma-soundness-base} we only need to prove
  completeness of~$\AXdb$ for rooted branching
  {\bisimilarity}. Suppose $E, F \in \calE$ and $E \rootedbrbisim
  F$. Let $E = \sumiinI \: \alpha_i \pref E_i$ and $F= \sumjinJ \:
  \beta_j \pref F_j$ for suitable index sets $I$ and~$J$, $\alpha_i,
  \beta_j \in \calA$, $E_i, F_j \in \calE$. Since $E \rootedbrbisim F$
  we have (i)~for all $i \in I$ there is a $j \in J$ such that
  $\alpha_i = \beta_j $ and $E_i \brbisim F_j$, and, symmetrically,
  (ii)~for all $j \in J$ there is an $i \in I$ such that $\alpha_i =
  \beta_j$ and $E_i \brbisim F_j$.  Put $K = \lc (i,j) \in I \times J
  \mid ( \alpha_i = \beta_j ) \land ( E_i \brbisim F_j ) \rc$. Define
  the processes $G, H \in \calE$ by
  \begin{displaymath}
    G = \sumkinK \: \gamma_k \pref G_k
    \quad \text{and} \quad
    H = \sumkinK \: \zeta_k \pref H_k
  \end{displaymath}
  where, for $i \in I$, $\gamma_k = \alpha_i$ and $G_k \equiv E_i$ if
  $k = (i,j)$ for some $j \in J$, and, similarly for $j \in J$,
  $\zeta_k = \beta_j$ and $H_k \equiv F_j$ if $k = (i,j)$ for some $i
  \in I$. Then $G$ and~$H$ are well-defined. Moreover, $\AXd \vdash E
  = G$ and $\AXd \vdash F = H$.

  For $k \in K$, say $k = (i,j)$, it holds that $\gamma_k = \alpha_i =
  \beta_j = \zeta_k$ and $G_k \equiv E_i \brbisim F_j \equiv H_k$, by
  definition of~$K$. By
  Lemma~\ref{lemma-branching-bisim-vs-equal-under-prefix}b we obtain,
  for all $k \in K$, $\AXdb \vdash { \gamma_k \pref G_k = \zeta_k
    \pref H_k }$. From this we get
  \begin{displaymath}
    \AXdb \vdash E 
    = \sumiinI \: \alpha_i \pref E_i
    = \sumkinK \: \gamma_k \pref G_k
    = \sumkinK \: \zeta_k \pref H_k
    = \sumjinJ \: \beta_j \pref F_j
    = F
  \end{displaymath}
  which concludes the proof of the theorem.
\end{proof}
