%!TEX root = catuscia.tex


%% file intro.tex

\section{Introduction}

In~\cite{DP07:tcs}, in a setting of a process language featuring both
non-deterministic and probabilistic choice, Yuxin Deng and Catuscia
Palamidessi propose an equational theory for a notion of weak
{\bisimilarity} and prove its soundness and completeness. Not
surprisingly, the axioms dealing with a silent step are reminiscent to
the well-known $\tau$-laws of Milner~\cite{Mil89:phi,Mil89:ic}. The process
language treated in~\cite{DP07:tcs} includes recursion, thereby
extending the calculus and axiomatization of~\cite{BS01:icalp}. While
the weak transitions of~\cite{DP07:tcs} can be characterized as
finitary, infinitary semantics is treated in~\cite{FG19:jlamp},
providing a sound and complete axiomatization also building
on the seminal work of Milner~\cite{Mil89:ic}.

In this paper we focus on branching {\bisimilarity} in the sense
of~\cite{GW96:jacm}, rather than on weak {\bisimilarity} as
in~\cite{BS01:icalp,DP07:tcs,FG19:jlamp}. 
%%
In the non-probabilistic setting branching {\bisimilarity} has the
advantage over weak {\bisimilarity} that it has far more efficient
algorithms~\cite{GJKW17:tcl,GV90:icalp}. Furthermore, it has a strong
logical underpinning~\cite{DV95:jacm}. It would be very attractive to
have these advantages available also in the probabilistic case, where
model checking is more demanding. See also the initial work reported
in~\cite{GV17:festschrift}.

For a similarly basic process language as in~\cite{DP07:tcs}, without
recursion though, we propose a notion of branching probabilistic {\bisimilarity}
as well as a sound and complete equational axiomatization. Hence,
instead of lifting 
all $\tau$-laws to the probabilistic setting, we only need to do this
for the B-axiom of~\cite{GW96:jacm}, the single axiom capturing inert
silent steps. For what is referred to as the alternating
model~\cite{HJ90:rtss}, branching probabilistic {\bisimilarity} has been
studied in~\cite{AGT12:tcs,AW06:tcs}. Also~\cite{Seg95:thesis}
discusses branching probabilistic {\bisimilarity}. However, the proposed
notions of branching {\bisimilarity} are either no congruence for the
parallel operator, or they invalidate the identities below which we
desire. The paper~\cite{AG09:sofsem} proposes a complete theory for a
variant of branching {\bisimilarity} that is not consistent with the
first $\tau$-law unfortunately.

Our investigation is led by the wish to identify the three processes
below, involving as a subprocess a probabilistic choice between $P$
and~$Q$. Essentially, ignoring the occurrence of the action~$a$
involved, the three processes represent (i)~a probabilistic choice of
weight~$\frac34$ between to instances of the subprocess mentioned,
(ii)~the subprocess on its own, and (iii)~a probabilistic choice of
weight~$\frac13$ for the subprocess and a rescaling of the subprocess,
in part overlapping.

\medskip

\scalebox{0.95}{%
\!\!\!%!TEX root = catuscia.tex

%% file example01.tex

\hbox{%
\begin{tikzpicture}[> = stealth', semithick, inner sep = 1pt, scale=0.45]
  \tikzstyle{state} = [circle, draw=Black, fill=LightYellow, minimum size=4.0mm]
  \tikzstyle{split} = [circle, draw=black, fill=black]
  \node [state] (t0) at (2.5,9) {$s_0$}; 
  \node [split] (y0) at (2.5,7.25) {};
  \node [state] (t1) at (1.0,6) {$s_1$}; 
  \node [split] (y1) at (1.0,4.25) {}; 
  \coordinate (t3) at (-0.25,3); 
  \coordinate (t4) at (2.0,3); 
  \node [state] (t2) at (4.0,6) {$s_2$}; 
  \node [split] (y2) at (4.0,4.25) {}; 
  \coordinate (t5) at (3.0,3); 
  \coordinate (t6) at (5.25,3); 
%%
  \path [>=] (t0) edge node [midway, right=1pt] {$a$} (y0);
  \path [->] (y0) edge node [midway, above left] {$\frac{3}{4}$} (t1);
  \path [->] (y0) edge node [midway, above right] {$\frac{1}{4}$} (t2);
  \draw [>=] (y0) ++(220:.3) arc (217.5:322.5:.3);
%%
  \path [>=] (t1) edge node [midway, left=0.5pt] {$\tau$} (y1);
  \path [->, shorten >=0.25mm] (y1) edge node [midway, above left] {$\frac{1}{2}$} (t3);
  \path [->, shorten >=0.25mm] (y1) edge node [midway, above right] {$\frac{1}{2}$} (t4);
  \draw [>=] (y1) ++(220:.3) arc (217.5:315:.3);
  \fill [fill=Pink, draw=Black, >=] (t3) -- ++(-1.75,-2.0) -- ++(2.0,0.0)
  -- (t3);  
  \node at ([shift={(-0.35,-1.4)}] t3) {P};
  \fill [fill=LightBlue, draw=Black, >=] (t4) -- ++(-1.75,-2.0) --
  ++(2.0,0.0) -- (t4); 
  \node at ([shift={(-0.45,-1.4)}] t4) {Q};
%%
  \path [>=] (t2) edge node [midway, left=0.5pt] {$\tau$} (y2);
  \path [->, shorten >=0.25mm] (y2) edge node [midway, above left] {$\frac{1}{2}$} (t5);
  \path [->, shorten >=0.25mm] (y2) edge node [midway, above right] {$\frac{1}{2}$} (t6);
  \draw [>=] (y2) ++(227.5:.3) arc (217.5:320:.3);
  \fill [fill=Pink, draw=Black, >=] (t5) -- ++(-0.25,-2.0) -- ++(2.0,0.0)
  -- (t5);  
  \node at ([shift={(+0.45,-1.4)}] t5) {P};
  \fill [fill=LightBlue, draw=Black, >=] (t6) -- ++(-0.25,-2.0) --
  ++(2.0,0.0) -- (t6); 
  \node at ([shift={(+0.45,-1.4)}] t6) {Q};
%%
  \node (S) at (2.5,-0.25) {\small
    $a \pref 
    \Bigl( 
    \partial \bigl( \tau \pref (P \probc{\frac12} Q) \bigr) 
    \probc{\frac34} 
    \partial \bigl( \tau \pref (P \probc{\frac12} Q) \bigr)
    \mkern-2mu \Bigr)$} ;
\end{tikzpicture}
%%
\hspace*{-0.75cm}
%%
\begin{tikzpicture}[> = stealth' , semithick, inner sep = 1pt, scale=0.45]
  \tikzstyle{state} = [circle, draw=Black, fill=LightYellow, minimum size=4.0mm]
  \tikzstyle{split} = [circle, draw=black, fill=black]
  \node [state] (u0) at (2.5,9) {$t_0$}; 
  \node [split] (x0) at (2.5,7.25) {};
  \coordinate (u1) at (1.0,6); 
  \coordinate (u2) at (4.0,6); 
  \coordinate (u3) at (4.0,1); 
  \path (u2) -- (u3);
  \path [>=] (u0) edge node [midway, right=1pt] {$a$} (x0);
  \path [->, shorten >=0.25mm] (x0) edge node [midway, above left] {$\frac{1}{2}$} (u1);
  \path [->, shorten >=0.25mm] (x0) edge node [midway, above right] {$\frac{1}{2}$} (u2);
  \draw [>=] (x0) ++(217.5:.3) arc (217.5:322.5:.3);
  \fill [fill=Pink, draw=Black, >=] (u1) -- ++(-1.0,-2.0) --
  ++(2.0,0.0) -- (u1);  
  \node at ([shift={(0,-1.4)}] u1) {P};
  \fill[fill=LightBlue, draw=Black, >=] (u2) -- ++(-1.0,-2.0) -- ++(2.0,0.0) -- (u2);
  \node at ([shift={(0,-1.4)}] u2) {Q};
%%
  \node (T) at (2.5,2.75) {\small
    $a \pref (P \probc{\frac12} Q)$} ; 
  \node (T') at (2.5,-0.5) {} ; 
\end{tikzpicture}
%%
  \hspace*{-0.75cm}
%%
\begin{tikzpicture}[> = stealth', semithick, inner sep = 1pt, scale=0.45]
  \tikzstyle{state} = [circle, draw=Black, fill=LightYellow, minimum size=4.0mm]
  \tikzstyle{split} = [circle, draw=Black, fill=black]
  \node [state] (s0) at (2.5,9) {$u_0$}; 
  \node [split] (x0) at (2.5,7.25) {};
  \node [state] (s1) at (1.0,6) {$u_1$}; 
  \node [split] (x1) at (1.0,4.25) {}; 
  \coordinate (s2) at (3,6); 
  \coordinate (s3) at (5.5,6); 
  \coordinate (s4) at (-0.5,3); 
  \coordinate (s5) at (2.5,3); 
%%
  \path [>=] (s0) edge node [midway, right=1pt] {$a$} (x0);
  \path [->] (x0) edge node [midway, above left] {$\frac{1}{3}$} (s1);
  \path [->] (x0) edge node [midway, right=2pt] {$\frac{1}{3}$} (s2);
  \path [->] (x0) edge node [midway, above right] {$\frac{1}{3}$} (s3);
  \draw [>=] (x0) ++(217.5:.3) arc (217.5:340:.3);
  \path [>=] (s1) edge node [midway, left =0.5pt] {$\tau$} (x1);
  \fill [fill=Pink, draw=Black, >=] (s2) -- ++(-0.5,-2.0) -- ++(2.0,0.0) -- (s2); 
  \node at ([shift={(+0.35,-1.4)}] s2) {P};
  \fill [fill=LightBlue, draw=Black, >=] (s3) -- ++(-0.5,-2.0) --
  ++(2.0,0.0) -- (s3); 
  \node at ([shift={(+0.35,-1.4)}] s3) {Q};
%%
  \draw [>=] (x1) ++(217.5:.3) arc (217.5:322.5:.3);
  \path [->] (x1) edge node [midway, above left] {$\frac{1}{2}$} (s4);
  \path [->] (x1) edge node [midway, above right] {$\frac{1}{2}$} (s5);
  \fill [fill=Pink, draw=Black, >=] (s4) -- ++(-1.0,-2.0) -- ++(2.0,0.0) -- (s4); 
  \node at ([shift={(0,-1.4)}] s4) {P};
  \fill [fill=LightBlue, draw=Black, >=] (s5) -- ++(-1.0,-2.0) --
  ++(2.0,0.0) -- (s5); 
  \node at ([shift={(0,-1.4)}] s5) {Q};
%%
  \node (U) at (2.5,-0.25) {\small
    $a \pref 
    \Bigl( 
    \partial \bigl( \tau \pref (P \probc{\frac12} Q) \bigr)
    \probc{\frac13} 
    (P \probc{\frac12} Q)
    \Bigr)$} ;
\end{tikzpicture}
} %% hbox

\!\!%
}%

\medskip

\noindent
In our view, all three processes starting from $s_0$, $t_0$, and $u_0$
are equivalent. The behavior that can be observed from them when
ignoring $\tau$-steps and coin tosses to resolve probabilistic choices
is the same. This leads to a definition of probabilistic branching
{\bisimilarity} that hitherto was not proposed in the literature and
appears to be the pendant of weak distribution {\bisimilarity} defined
by~\cite{EHKTZ13:qest}.

As for~\cite{DP07:tcs} we seek to stay close to the treatment of the
non-deterministic fragment of the process calculus at hand. However,
as an alternate route in proving completeness, we rely on the
definition of a concrete process.  We first apply the approach for
strictly non-deterministic processes and \emph{mutatis mutandis} for
the whole language allowing processes that involve both
non-deterministic and probabilistic choice. For now, let's call a
process concrete if it doesn't exhibit inert transitions,
i.e.\ $\tau$-transitions that don't change the potential behavior of
the process essentially. The approach we follow first establishes
soundness for branching (probabilistic) {\bisimilarity} and soundness and
completeness for strong (probabilistic) {\bisimilarity}. Because of the
non-inertness of the silent steps involved, strong and branching
{\bisimilarity} coincide for concrete processes. The trick then is to
relate a pair of branching (probabilistically) bisimilar processes to
a corresponding pair of concrete processes. Since these are also
branching (probabilistically) bisimilar as argued, they are
consequently strongly (probabilistically) bisimilar, and, voil\`a,
provably equal by the completeness result for strong (probabilistic)
{\bisimilarity}.

The remainder of the paper is organized as follows. In
Section~\ref{sec-prelim} we gather some notation regarding probability
distributions. For illustration purposes Section~\ref{sec-nondet}
treats the simpler setting of non-deterministic processes reiterating
the completeness proof for the equational theory of~\cite{GW96:jacm}
for rooted branching {\bisimilarity}. Next, after introducing
branching probabilistic {\bisimilarity} and some of its fundamental
properties in Sections~\ref{sec-bpb} and~\ref{fundamental},
respectively, in Section~\ref{sec-prob} we prove the main result,
viz.\ the completeness of an equational theory for rooted branching
probabilistic {\bisimilarity}, following the same lines set out in
Section~\ref{sec-nondet}. In Section~\ref{sec-concl} we wrap up and
make concluding remarks.

