%!TEX root = catuscia.tex

%% file commands.tex
\usepackage{ifthen} % For handling the optional arguments in defintion, etc. environments. 
\newcommand{\optionalargumentinbrackets}[1]{\ifthenelse{\equal {#1} {} }{}{~(#1)}}

\newcounter{theoremcnt}[section]
\renewcommand{\thetheoremcnt}{\thesection.\arabic{theoremcnt}}

\renewenvironment{definition}[1][]%   One optional argument. 
{\begin{trivlist}\refstepcounter{theoremcnt}
\item[]\textbf{Definition~\thetheoremcnt \optionalargumentinbrackets{#1}.} }{\end{trivlist}}

\renewenvironment{example}[1][]%
{\begin{trivlist}\refstepcounter{theoremcnt}
\item[]\textbf{Example~\thetheoremcnt \optionalargumentinbrackets{#1}.} }{\end{trivlist}}

\renewenvironment{lemma}[1][]%
{\begin{trivlist}\refstepcounter{theoremcnt}
\item[]\textbf{Lemma~\thetheoremcnt \optionalargumentinbrackets{#1}.} }{\end{trivlist}}

\renewenvironment{proof}[1][]%
{\begin{trivlist}\item[]\textbf{Proof\optionalargumentinbrackets{#1}.} }{\qed\end{trivlist}}

\renewenvironment{theorem}[1][]%
{\begin{trivlist}\refstepcounter{theoremcnt}
\item[]\textbf{Theorem~\thetheoremcnt \optionalargumentinbrackets{#1}.} }{\end{trivlist}}

\renewenvironment{corollary}[1][]%
{\begin{trivlist}\refstepcounter{theoremcnt}
\item[]\textbf{Corollary~\thetheoremcnt \optionalargumentinbrackets{#1}.} }{\end{trivlist}}

\makeatletter
\newcommand{\xRightarrow}[2][]{\ext@arrow
  0359\Rightarrowfill@{#1}{#2}}
\makeatother

\newcommand{\Arrow}[1]{\xRightarrow{\, {#1} \; \,}}
\newcommand{\AXd}{\mathit{A \mkern-2mu X}}
\newcommand{\AXdb}{\mathit{A \mkern-2mu X}^b}
\newcommand{\AXp}{\mathit{A \mkern-2mu X}_{\mkern-3mu p}}
\newcommand{\AXpb}{\mathit{A \mkern-2mu X}^b_{\mkern-3mu p}}
\newcommand{\DistrE}{\Distr(\calE)}
\newcommand{\DistrEconc}{\Distr(\calE_{\mathit{cc}})}
\newcommand{\DistrX}{\Distr(X)}
\newcommand{\Phat}{\hat{P}}
\newcommand{\Qhat}{\hat{Q}}

\newcommand{\alphaarrow}{\arrow{\alpha}}
\newcommand{\alphahidearrow}{\arrow{(\alpha)}}
% \newcommand{\arrow}[1]{\xrightarrow{\,#1\,}}
% Adapted the command below, to avoid odd spacing between lines if 
% objects above the arrow are slightly larger. 
\newcommand{\arrow}[1]{\xrightarrow{\raisebox{0ex}[0.1ex][-0.1ex]{\scriptsize $\,#1\,$}}}
\newcommand{\astop}{a \pref \partial(\bfzero)}
\newcommand{\bfzero}{\normalfont{\textbf{0}}}
\newcommand{\bigoplusiinI}{\textstyle{\bigoplus_{i \mathord{\in} I}}}
\newcommand{\bigoplusjinJ}{\textstyle{\bigoplus_{j \mathord{\in} J}}}
\newcommand{\bigoplusjinJprime}{\bigoplus_{j \mathord{\in} J'}}
\newcommand{\bigopluskinK}{\textstyle{\bigoplus_{k \mathord{\in} K}}}
\newcommand{\bisim}{\mathrel{\,%
  \raisebox{.3ex}{$\underline{\makebox[.7em]{$\leftrightarrow$}}$}\,}}
\newcommand{\bnfeq}{\mathrel{{:}{:}{=}}}
\newcommand{\brbisim}{\mathrel{\,%
  \raisebox{.3ex}{$\underline{\makebox[.8em]{$\leftrightarrow$}}_{\,
      b}$}\mkern-1mu}}
\newcommand{\bstop}{b \pref \partial(\bfzero)}
\newcommand{\der}{\textsl{der} \mkern1mu}  
\newcommand{\dfeq}{\stackrel{\mathrm{df}}{=}}
\newcommand{\half}{\textstyle{\frac12}}
\newcommand{\la}{\langle}
\newcommand{\lc}{\lbrace \:}
\newcommand{\mopcalR}{\,{\calR}\,}   % het gebruik van \mathop laat het symbool zakken.
\newcommand{\mycdot}{\mathop{\cdot}}
\newcommand{\myfrac}[2]{\textstyle{\frac{#1}{#2}}}
\newcommand{\myoplus}{\mathop{\oplus}}
\newcommand{\mystop}{\partial(\bfzero)}
\newcommand{\nbrbisim}{\mkern4mu {\not\mkern-2.3mu\brbisim} \mkern4mu}
\newcommand{\nrootedbrbisim}{\mkern4mu {\not\mkern-1.5mu\rootedbrbisim} \mkern4mu}
\newcommand{\oplusr}{\oplus_r}
\newcommand{\pref}{\mathop{.}}
\newcommand{\ra}{\rangle}
\newcommand{\rc}{\: \rbrace}
\newcommand{\rootedbrbisim}{\mathrel{\,%
  \raisebox{.3ex}{$\underline{\makebox[.7em]{$\leftrightarrow$}}_{\,
      rb}$}\mkern-1mu}}
\newcommand{\singleton}[1]{\lbrace {#1} \rbrace}
\newcommand{\sosrule}[2]{%
  \def\arraystretch{0.50}
  \begin{array}{c} {#1}\rule[-4pt]{0pt}{11pt} \\ \hline
    \rule{0pt}{13pt}{#2} \end{array}
  \def\arraystretch{1.0}}
\newcommand{\sumiinI}{\textstyle\sum_{i {\in} I}}
\newcommand{\sumjinJ}{\textstyle\sum_{j {\in} J}}
\newcommand{\sumkinK}{\textstyle\sum_{k {\in} K}}
\newcommand{\sumiinIJ}{\textstyle\sum_{i {\in} I\cup J}}
\newcommand{\tauarrow}{\arrow{\tau}}
\newcommand{\tauhidearrow}{\arrow{(\tau)}}
\newcommand{\vdashAXpb}{\vdash_{\AXpb}}


%% decorations
\newcommand{\Cbar}{\bar{C}}
\newcommand{\Ebar}{\bar{E}}
\newcommand{\Ehat}{\hat{E}}
\newcommand{\Fbar}{\bar{F}}
\newcommand{\Gbar}{\bar{G}}
\newcommand{\Pbar}{\bar{P}}
\newcommand{\Qbar}{\bar{Q}}
\newcommand{\etabar}{\bar{\eta}}
\newcommand{\mubar}{\bar{\mu}}
\newcommand{\nubar}{\bar{\nu}}
\newcommand{\rbar}{\bar{r}}

%% slanted letters
\newcommand{\Distr}{\textsl{Distr}\mkern1mu}


%% caligraphic letters
\newcommand{\calA}{\mathcal{A}}
\newcommand{\calAtau}{\mathcal{A}_\tau}
\newcommand{\calE}{\mathcal{E}}
\newcommand{\calEconc}{\mathcal{E}_{\mathit{cc}}}
\newcommand{\calI}{\mathcal{I}}
\newcommand{\calP}{\mathcal{P}}
\newcommand{\calPconc}{\mathcal{P}_{\!\mathit{cc}}}
\newcommand{\calR}{\mathcal{R}}


%% spacing
\newcommand{\blankline}{\vspace*{1.0\baselineskip}}
\newcommand{\halflineup}{\vspace*{-0.5\baselineskip}}
\newcommand{\quarterlineup}{\vspace*{-0.35\baselineskip}}
\newcommand{\myrule}{\rule[-5pt]{0pt}{16pt}}
\newcommand{\todo}[1]{\textcolor{red}{#1}}
\newenvironment{bluetext}{\color{blue}}{}


\colorlet{colorA}{LightYellow}
\colorlet{colorB}{Pink}
\colorlet{colorC}{LightBlue}
\colorlet{colorD}{Plum}
\colorlet{colorE}{Gold!50}
\colorlet{colorF}{Orange!50}
\colorlet{colorG}{Red!50}
\colorlet{colorH}{Green!50}
\colorlet{colorI}{NavyBlue!50}
\colorlet{colorJ}{Brown!50}
\colorlet{colorK}{Coral}
\colorlet{colorL}{RoyalBlue!50}
\colorlet{colorM}{OliveDrab!50}
\colorlet{colorX}{Gray!50}

%% Rob's proposals
\newcommand{\bisimilarity}{bisimilarity}
\newcommand{\probc}[1]{\mathbin{\mbox{${}_{\scriptscriptstyle #1 \mkern1mu} \hspace{-.8pt}\oplus\,$}}}
\renewcommand{\oplusr}{\probc{r}}
\newcommand{\den}[1]{\mbox{$[\hspace{-1.6pt}[$}#1\mbox{$]\hspace{-1.6pt}]$}}  % denotation
\newcommand{\plat}[1]{\raisebox{0pt}[0pt][0pt]{#1}}
\definecolor[named]{darkgreen}{cmyk}{1,0,.7,.5}
\definecolor[named]{darkorange}{cmyk}{0,0.82,1,0.01}
\definecolor[named]{purpleblue}{cmyk}{.42,.85,0,.29}
\newcommand{\dg}[1]{\textcolor{darkgreen}#1}
\newcommand{\dor}[1]{\textcolor{darkorange}#1}
\newcommand{\mg}[1]{\textcolor{purpleblue}#1}
\newcommand{\weg}[1]{}
