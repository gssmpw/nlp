
%With technological \begin{abstract}  
Test time scaling is currently one of the most active research areas that shows promise after training time scaling has reached its limits.
Deep-thinking (DT) models are a class of recurrent models that can perform easy-to-hard generalization by assigning more compute to harder test samples.
However, due to their inability to determine the complexity of a test sample, DT models have to use a large amount of computation for both easy and hard test samples.
Excessive test time computation is wasteful and can cause the ``overthinking'' problem where more test time computation leads to worse results.
In this paper, we introduce a test time training method for determining the optimal amount of computation needed for each sample during test time.
We also propose Conv-LiGRU, a novel recurrent architecture for efficient and robust visual reasoning. 
Extensive experiments demonstrate that Conv-LiGRU is more stable than DT, effectively mitigates the ``overthinking'' phenomenon, and achieves superior accuracy.
\end{abstract}  advancements, n
New regulations are continuously introduced to ensure that software development complies with the ethical concerns and prioritizes public safety. %applications comply with legal frameworks. 
A prerequisite for demonstrating compliance involves tracing software requirements to legal provisions. \textit{Requirements traceability} is a fundamental task where requirements engineers are supposed to analyze technical requirements against target artifacts,  %hundreds of technical requirements for complex systems, 
often under  %is not feasible for , who typically have 
limited time budget. Doing this analysis manually for complex systems with hundreds of requirements is infeasible. The legal dimension introduces additional challenges that only exacerbate manual effort. 

In this paper, we investigate two automated solutions based on large language models (LLMs) to predict trace links between requirements and legal provisions. 
The first solution, \kashif, is a classifier that leverages sentence transformers and semantic similarity. The second solution prompts a recent generative LLM based on \RICE, a prompt engineering framework.

On a benchmark dataset, we empirically evaluate \kashif and compare it against a baseline classifier from the literature.
%Compared to a baseline classifier
%that was originally developed using this dataset, 
\kashif can identify trace links with an average recall of $\approx$67\%, %and precision of $\approx$49\%
outperforming the baseline with a substantial gain of 54 percentage points (pp) in recall. %, but falling behind with 8.5 pp in terms of  precision. 
However, on unseen, more complex requirements documents traced to the European general data protection regulation (GDPR), \kashif performs poorly, yielding an average recall of 15\%. %accuracy of 44.1\%. 
On the same documents, however, our \RICE-based solution yields an average recall of 84\%, % accuracy 
with a remarkable gain of about 69 pp over \kashif. 
Our results suggest that requirements traceability in the legal context cannot be simply addressed by building classifiers, as such solutions do not generalize and fail to perform well on complex regulations and requirements. 
%Recent LLMs offer a more promising alternative that can resolve this issue. 
Resorting to generative LLMs, with careful prompt engineering, is thus a more promising alternative. % to address legal traceability than traditional classification. 
%The requirements in these documents are traced to the European general data protection regulation (GDPR). 
%Unlike the baseline, \kashif can transfer knowledge, to some extent, albeit with poor performance: reaching an average accuracy of $\approx$27\%. 

%, we investigate leverage the more recent large language models (e.g., GPT), by providing only few examples in the prompt (few shot learning),  which 
%yields an accuracy of $\approx$71\%. , leading to a gain of $\approx$44 pp over \kashif, thus suggesting this is a promising approach.  
