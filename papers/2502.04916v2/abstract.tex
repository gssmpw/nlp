
%With technological \begin{abstract}
Retrieval-Augmented Generation (RAG) is often used with Large Language Models (LLMs) to infuse domain knowledge or user-specific information. In RAG, given a user query, a retriever extracts chunks of relevant text from a knowledge base. These chunks are sent to an LLM as part of the input prompt. Typically, any given chunk is repeatedly retrieved across user questions. However, currently, for every question, attention-layers in LLMs fully compute the key values (KVs) repeatedly for the input chunks, as state-of-the-art methods cannot reuse KV-caches when chunks appear at arbitrary locations with arbitrary contexts. Naive reuse leads to output quality degradation.  This leads to potentially redundant computations on expensive GPUs and increases latency. In this work, we propose \sys, a system for managing and reusing precomputed KVs corresponding to the text chunks (we call \textit{chunk-caches}) in RAG-based systems. We present how to identify \hl{\textit{chunk-caches} that are reusable}, how to efficiently perform a small fraction of recomputation to \textit{fix} the cache to maintain output quality, and how to efficiently store and evict \textit{chunk-caches} in the hardware for maximizing reuse while masking any overheads. With real production workloads as well as synthetic datasets, we show that \sys reduces redundant computation by \textbf{51\%} over SOTA prefix-caching and \textbf{75\%} over full recomputation.
\hl{Additionally, with continuous batching on a real production workload, we get a \textbf{1.6$\times$} speedup in throughput and a \textbf{2$\times$} reduction in end-to-end response latency over prefix-caching while maintaining quality, for both the \llama-3-8B and \llama-3-70B models. 
}
\end{abstract}




advancements, n
New regulations are continuously introduced to ensure that software development complies with the ethical concerns and prioritizes public safety. %applications comply with legal frameworks. 
A prerequisite for demonstrating compliance involves tracing software requirements to legal provisions. \textit{Requirements traceability} is a fundamental task where requirements engineers are supposed to analyze technical requirements against target artifacts,  %hundreds of technical requirements for complex systems, 
often under  %is not feasible for , who typically have 
limited time budget. Doing this analysis manually for complex systems with hundreds of requirements is infeasible. The legal dimension introduces additional challenges that only exacerbate manual effort. 

In this paper, we investigate two automated solutions based on large language models (LLMs) to predict trace links between requirements and legal provisions. 
The first solution, \kashif, is a classifier that leverages sentence transformers and semantic similarity. The second solution prompts a recent generative LLM based on \RICE, a prompt engineering framework.

On a benchmark dataset, we empirically evaluate \kashif and compare it against a baseline classifier from the literature.
%Compared to a baseline classifier
%that was originally developed using this dataset, 
\kashif can identify trace links with an average recall of $\approx$67\%, %and precision of $\approx$49\%
outperforming the baseline with a substantial gain of 54 percentage points (pp) in recall. %, but falling behind with 8.5 pp in terms of  precision. 
However, on unseen, more complex requirements documents traced to the European general data protection regulation (GDPR), \kashif performs poorly, yielding an average recall of 15\%. %accuracy of 44.1\%. 
On the same documents, however, our \RICE-based solution yields an average recall of 84\%, % accuracy 
with a remarkable gain of about 69 pp over \kashif. 
Our results suggest that requirements traceability in the legal context cannot be simply addressed by building classifiers, as such solutions do not generalize and fail to perform well on complex regulations and requirements. 
%Recent LLMs offer a more promising alternative that can resolve this issue. 
Resorting to generative LLMs, with careful prompt engineering, is thus a more promising alternative. % to address legal traceability than traditional classification. 
%The requirements in these documents are traced to the European general data protection regulation (GDPR). 
%Unlike the baseline, \kashif can transfer knowledge, to some extent, albeit with poor performance: reaching an average accuracy of $\approx$27\%. 

%, we investigate leverage the more recent large language models (e.g., GPT), by providing only few examples in the prompt (few shot learning),  which 
%yields an accuracy of $\approx$71\%. , leading to a gain of $\approx$44 pp over \kashif, thus suggesting this is a promising approach.  
