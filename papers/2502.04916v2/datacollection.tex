\subsection{The \texttt{HIPAA} Dataset}\label{subsec:datacol}

In this work, we develop our approach and base our initial evaluation on the \texttt{HIPAA} dataset, a publicly available dataset, created and released in 2010~\cite{cleland:2010} and reused in 2017~\cite{Guo:17}. 
The dataset was manually created by identifying trace links of requirements against the regulatory statements elicited from the the USA government's Health Insurance Privacy and Portability Act (HIPAA) regulation. The provisions are the following:  access control (\texttt{AC}), audit controls (\texttt{AUD}), person or entity authentication (\texttt{PA}), transmission security (\texttt{TS}), unique user identification (\texttt{UUI}), emergency access procedure (\texttt{EAP}), automatic logoff (\texttt{AL}), encryption and decryption (\texttt{SED}), encryption (\texttt{TED}), and integrity controls (\texttt{IC}).
\texttt{HIPAA} consists of 10 requirements documents, all are shall-requirements, from the healthcare domain. In total, the dataset contains 1,891 requirements, of which 243 have trace links. Table~\ref{tab:hipaa-dataset} summarizes the different documents (rows) in \texttt{HIPAA}, their description, and the distribution of the trace links across provisions (columns). 

% As we elaborate in Section~\ref{subsec:evalProcedure}, our evaluation is based on the leave-one-out (LOO) method, i.e., we hold out a document to be used as a test set, and train our approach \kashif on the remaining documents. However, to ensure a reasonable balance between the training and test sets, we exclude one document (\texttt{CCHIT}, labeled H2 in the table) from the LOO process since it contains 1,064 requirements, i.e., more than a half of the dataset. 

\begin{table*}
\footnotesize
\centering
\caption{Statistics of the HIPAA dataset~\cite{cleland:2010}. Rows list the documents in \texttt{HIPAA}, and  columns  provide their description and the distribution of the trace links across provisions in each document. }\label{tab:hipaa-dataset}
\begin{tabularx}{0.98\textwidth}
{@{} p{0.05\textwidth} @{\hskip 0.5em} p{0.2\textwidth} @{\hskip 0.5em} p{0.05\textwidth} @{\hskip 0.8em} *{11}{>{\centering\arraybackslash}X}@{}}
\toprule
\textbf{ID} & \textbf{Description} & \textbf{All} & \texttt{\textbf{AC}} & \texttt{\textbf{AUD}} & \texttt{\textbf{AL}} & \texttt{\textbf{EAP}} & \texttt{\textbf{PA}} & \texttt{\textbf{SED}} & \texttt{\textbf{TED}} & \texttt{\textbf{TS}} & \texttt{\textbf{IC}} & \texttt{\textbf{UUI}} \\ 
\midrule
H1 & Care2x: Hospital Info. System. & 44 & 1 & 1 & 1 & 0 & 1 & 1 & 1 & 0 & 0 & 0 \\ 
H2 & CCHIT: Certification Commission for HCT. & 1064 & 17 & 33 & 1 & 1 & 12 & 2 & 2 & 2 & 5 & 3 \\ 
H3 & ClearHealth: EMR System. %\url{http://www.clear-health.com} 
& 44 & 1 & 4 & 1 & 0 & 0 & 1 & 1 & 0 & 2 & 1 \\ 
H4 & Physician: Electronic Info. Exchange between Clinicians. %\url{http://hmss.org/content/files/CTC.use-Case.pdf} 
& 147 & 7 & 2 & 0 & 2 & 0 & 0 & 0 & 1 & 3 & 0 \\ 
H5 & iTrust: Role-based HCT Web app. %\url{http://agile.csc.ncsu.edu/itrust/wiki/doku.php} 
& 184 & 2 & 35 & 1 & 0 & 6 & 0 & 0 & 0 & 0 & 2 \\ 
H6 & Trial Implementations: National Coordinator for Health IT & 100 & 4 & 0 & 0 & 0 & 13 & 0 & 0 & 2 & 4 & 2 \\ 
H6 & PatientOS: HCT Info. System. & 91 & 1 & 2 & 3 & 1 & 0 & 3 & 1 & 1 & 0 & 1 \\ 
H8 & PracticeOne: A Suite of HCT Info. Systems. & 34 & 3 & 1 & 0 & 0 & 1 & 0 & 0 & 1 & 1 & 0 \\ 
H9 & Lauesen: Sample EMR System. & 66 & 11 & 0 & 1 & 0 & 5 & 0 & 0 & 0 & 3 & 1 \\ 
H10 & WorldVistA: Veteran Administrations EMR. & 117 & 6 & 2 & 2 & 0 & 4 & 0 & 0 & 0 & 0 & 1 \\ 
\midrule
& \textbf{Total counts} & \textbf{1891} & \textbf{53} & \textbf{86} & \textbf{10} & \textbf{4} & \textbf{42} & \textbf{7} & \textbf{5} & \textbf{7} & \textbf{18} & \textbf{11} \\ 
\bottomrule
\end{tabularx}
\begin{tablenotes}
     \vspace*{.5em}
 \it    \item[*] 
     \texttt{EMR}: Electronic Medical Record. 
     \texttt{HCT}: Healthcare Technology. 
\end{tablenotes}
\end{table*}

