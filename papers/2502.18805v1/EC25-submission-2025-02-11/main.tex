% \def\DRAFT{} 

\PassOptionsToPackage{prologue,dvipsnames}{xcolor}  % EREL: required to prevent option clash: see https://github.com/borisveytsman/acmart/issues/406 

\documentclass[format=acmsmall, review=false]{acmart}


\usepackage{acm-ec-25}
\usepackage{booktabs} % For formal tables
\usepackage[ruled]{algorithm2e} % For algorithms
\renewcommand{\algorithmcfname}{ALGORITHM}
\SetAlFnt{\small}
\SetAlCapFnt{\small}
\SetAlCapNameFnt{\small}
\SetAlCapHSkip{0pt}
\IncMargin{-\parindent}

% ==== EDEN: added
\usepackage{cleveref}

% ==== EDEN: added "bibliography=common"
\usepackage[bibliography=common]{apxproof}

% ==== EDEN: added
\renewcommand{\appendixprelim}[1]{%
  %\clearpage % <--- uncomment if you want a new page
}

\newtheoremrep{theorem}{Theorem}[section]
\newtheoremrep{proposition}[theorem]{Proposition}
\newtheoremrep{lemma}[theorem]{Lemma}
\newtheoremrep{claim}[theorem]{Claim}
\newtheoremrep{observation}[theorem]{Observation}
\newtheoremrep{open}[theorem]{Open Question}

\theoremstyle{definition}
\newtheoremrep{remark}[theorem]{Remark}


\usepackage[dvipsnames]{xcolor}  % Option Clash
\ifdefined\DRAFT
    \newcommand{\er}[1]{\textcolor{blue}{#1}}
    \newcommand{\erel}[1]{\er{(Erel says: #1)}}
    \newcommand{\alex}[1]{\textcolor{red}{(Alex says: #1)}}
    \newcommand{\eden}[1]{\textcolor{Green}{(Eden says: #1)}}
    \newcommand{\rmark}[1]{\textcolor{BrickRed}{#1}}
    \newcommand{\biaoshuai}[1]{\textcolor{purple}{(Biaoshuai says: #1)}}
\else
    \newcommand{\er}[1]{#1}
    \newcommand{\erel}[1]{}
    \newcommand{\alex}[1]{}
    \newcommand{\eden}[1]{}
    \newcommand{\rmark}[1]{#1}
    \newcommand{\biaoshuai}[1]{}
\fi


\newcommand{\valT}[2]{v_{#1,#2}}
\newcommand{\repT}[2]{r_{#1,#2}}

\newcommand{\prefs}{\mathbf{P}}
\newcommand{\prefsExcI}{\prefs_{-i}}
\newcommand{\domain}{\mathcal{D}}
\newcommand{\domains}{\mathbf{\domain}}
\newcommand{\domainsExcI}{\domains_{-i}}
\newcommand{\notK}{\widebar{K}}

\usepackage{wrapfig}

\DeclareMathOperator*{\argmax}{arg\,max}
\DeclareMathOperator*{\argmin}{arg\,min}

\newcommand{\ceil}[1]{\left\lceil #1 \right\rceil}
\newcommand{\floor}[1]{\left\lfloor #1 \right\rfloor}

\newcommand{\propallocations}{\mathcal{X}^{\mathrm{PROP}}}
\newcommand{\efallocations}{\mathcal{X}^{\mathrm{EF1}}}
\newcommand{\yefallocations}{\mathcal{Y}^{\mathrm{EF1}}}


\usepackage{multirow}
\usepackage{array}
\usepackage{tabulary}
\newcolumntype{K}[1]{>{\centering\arraybackslash}m{#1}}
% ====


% Choose a citation style by commenting/uncommenting the appropriate line:
%\setcitestyle{acmnumeric}
\setcitestyle{authoryear}

% Title. Note the optional short title for running heads. In the interest of anonymization, please do not include any acknowledgements.
\title[Degree of Risk-Averse Truthfulness]{It’s Not All Black and White: Degree of Truthfulness for Risk-Avoiding Agents}

% Anonymized submission.
\author{Submission 1843}


% % ===== for sub 
% \author{Eden Hartman}
% \affiliation{%
% \institution{Bar-Ilan University}
% \country{Israel}
% }
% \author{Erel Segal-Halevi}
% \affiliation{%
% \institution{Ariel University}
% \country{Israel}
% }
% \author{Biaoshuai Tao}
% \affiliation{%
% \institution{Shanghai Jiao Tong University}
% \country{China}
% }

% % =========



% Abstract. Note that this must come before \maketitle.
\begin{abstract}
    
The classic notion of \emph{truthfulness} requires that no agent has a profitable manipulation -- an untruthful report that, for \emph{some} combination of reports of the other agents, increases her utility. 
This strong notion implicitly assumes that the manipulating agent either knows what all other agents are going to report, or is willing to take the risk and act as-if she knows their reports.


Without knowledge of the others' reports, most manipulations are \emph{risky} -- they might decrease the manipulator's utility for some other combinations of reports by the other agents.
Accordingly, a recent paper (Bu, Song and Tao, ``On the existence of truthful fair cake cutting mechanisms'', Artificial Intelligence 319 (2023), 103904) suggests a relaxed notion, which we refer to as \emph{risk-avoiding truthfulness (RAT)}, which requires only that no agent can gain from a \emph{safe} manipulation -- one that is sometimes beneficial and never harmful.


Truthfulness and RAT are two extremes: the former considers manipulators with complete knowledge of others, whereas the latter considers manipulators with no knowledge at all. 
In reality, agents often know about some --- but not all --- of the other agents.
This paper introduces the \emph{RAT-degree} of a mechanism, 
defined as the smallest number of agents whose reports, if known, may allow another agent to safely manipulate, or $n$ if there is no such number. 
This notion interpolates between classic truthfulness (degree $n$) and RAT (degree at least $1$): a mechanism with a higher RAT-degree is harder to manipulate safely. 

To illustrate the generality and applicability of this concept, we analyze the RAT-degree of prominent mechanisms across various social choice settings, including auctions, indivisible goods allocations, cake-cutting, voting, and stable matchings.
\end{abstract}

\begin{document}

% Title page for title and abstract only.
\begin{titlepage}

\maketitle

% Optionally include a table of contents
\vspace{1cm}
\setcounter{tocdepth}{1} % adjust to 1 if desired
\tableofcontents

\end{titlepage}

% Paper body

\section{Introduction}


\begin{figure}[t]
\centering
\includegraphics[width=0.6\columnwidth]{figures/evaluation_desiderata_V5.pdf}
\vspace{-0.5cm}
\caption{\systemName is a platform for conducting realistic evaluations of code LLMs, collecting human preferences of coding models with real users, real tasks, and in realistic environments, aimed at addressing the limitations of existing evaluations.
}
\label{fig:motivation}
\end{figure}

\begin{figure*}[t]
\centering
\includegraphics[width=\textwidth]{figures/system_design_v2.png}
\caption{We introduce \systemName, a VSCode extension to collect human preferences of code directly in a developer's IDE. \systemName enables developers to use code completions from various models. The system comprises a) the interface in the user's IDE which presents paired completions to users (left), b) a sampling strategy that picks model pairs to reduce latency (right, top), and c) a prompting scheme that allows diverse LLMs to perform code completions with high fidelity.
Users can select between the top completion (green box) using \texttt{tab} or the bottom completion (blue box) using \texttt{shift+tab}.}
\label{fig:overview}
\end{figure*}

As model capabilities improve, large language models (LLMs) are increasingly integrated into user environments and workflows.
For example, software developers code with AI in integrated developer environments (IDEs)~\citep{peng2023impact}, doctors rely on notes generated through ambient listening~\citep{oberst2024science}, and lawyers consider case evidence identified by electronic discovery systems~\citep{yang2024beyond}.
Increasing deployment of models in productivity tools demands evaluation that more closely reflects real-world circumstances~\citep{hutchinson2022evaluation, saxon2024benchmarks, kapoor2024ai}.
While newer benchmarks and live platforms incorporate human feedback to capture real-world usage, they almost exclusively focus on evaluating LLMs in chat conversations~\citep{zheng2023judging,dubois2023alpacafarm,chiang2024chatbot, kirk2024the}.
Model evaluation must move beyond chat-based interactions and into specialized user environments.



 

In this work, we focus on evaluating LLM-based coding assistants. 
Despite the popularity of these tools---millions of developers use Github Copilot~\citep{Copilot}---existing
evaluations of the coding capabilities of new models exhibit multiple limitations (Figure~\ref{fig:motivation}, bottom).
Traditional ML benchmarks evaluate LLM capabilities by measuring how well a model can complete static, interview-style coding tasks~\citep{chen2021evaluating,austin2021program,jain2024livecodebench, white2024livebench} and lack \emph{real users}. 
User studies recruit real users to evaluate the effectiveness of LLMs as coding assistants, but are often limited to simple programming tasks as opposed to \emph{real tasks}~\citep{vaithilingam2022expectation,ross2023programmer, mozannar2024realhumaneval}.
Recent efforts to collect human feedback such as Chatbot Arena~\citep{chiang2024chatbot} are still removed from a \emph{realistic environment}, resulting in users and data that deviate from typical software development processes.
We introduce \systemName to address these limitations (Figure~\ref{fig:motivation}, top), and we describe our three main contributions below.


\textbf{We deploy \systemName in-the-wild to collect human preferences on code.} 
\systemName is a Visual Studio Code extension, collecting preferences directly in a developer's IDE within their actual workflow (Figure~\ref{fig:overview}).
\systemName provides developers with code completions, akin to the type of support provided by Github Copilot~\citep{Copilot}. 
Over the past 3 months, \systemName has served over~\completions suggestions from 10 state-of-the-art LLMs, 
gathering \sampleCount~votes from \userCount~users.
To collect user preferences,
\systemName presents a novel interface that shows users paired code completions from two different LLMs, which are determined based on a sampling strategy that aims to 
mitigate latency while preserving coverage across model comparisons.
Additionally, we devise a prompting scheme that allows a diverse set of models to perform code completions with high fidelity.
See Section~\ref{sec:system} and Section~\ref{sec:deployment} for details about system design and deployment respectively.



\textbf{We construct a leaderboard of user preferences and find notable differences from existing static benchmarks and human preference leaderboards.}
In general, we observe that smaller models seem to overperform in static benchmarks compared to our leaderboard, while performance among larger models is mixed (Section~\ref{sec:leaderboard_calculation}).
We attribute these differences to the fact that \systemName is exposed to users and tasks that differ drastically from code evaluations in the past. 
Our data spans 103 programming languages and 24 natural languages as well as a variety of real-world applications and code structures, while static benchmarks tend to focus on a specific programming and natural language and task (e.g. coding competition problems).
Additionally, while all of \systemName interactions contain code contexts and the majority involve infilling tasks, a much smaller fraction of Chatbot Arena's coding tasks contain code context, with infilling tasks appearing even more rarely. 
We analyze our data in depth in Section~\ref{subsec:comparison}.



\textbf{We derive new insights into user preferences of code by analyzing \systemName's diverse and distinct data distribution.}
We compare user preferences across different stratifications of input data (e.g., common versus rare languages) and observe which affect observed preferences most (Section~\ref{sec:analysis}).
For example, while user preferences stay relatively consistent across various programming languages, they differ drastically between different task categories (e.g. frontend/backend versus algorithm design).
We also observe variations in user preference due to different features related to code structure 
(e.g., context length and completion patterns).
We open-source \systemName and release a curated subset of code contexts.
Altogether, our results highlight the necessity of model evaluation in realistic and domain-specific settings.





\subsection{Related Word}
There is a vast body of work on truthfulness relaxations and alternative measurements of manipulability. Due to space constraints, we provide only a brief overview here; a more in-depth discussion of these works and their relation to RAT-degree can be found in \Cref{apx:related}.


\paragraph{Truthfulness Relaxations.} Various truthfulness relaxations focus on a certain subset of all possible manipulations, which are considered more ``likely''. It requires that none of the manipulations from this subset is profitable. Different relaxations consider different subsets of ``likely'' manipulations.

\citet{brams2006better} propose \emph{maximin strategy-proofness}, where an agent manipulates only if it is always beneficial.
\er{\citet{waxman2021manipulation} were the first (as far as we know) to use the term \emph{safe manipulation}.%
\footnote{
They use ``safe manipulation'' for a manipulation that is both safe and profitable.
}
They examined the possible manipulations of agents with three different levels of knowledge on the social networks.
}

\erel{Interesting and closely-related paper; should read in depth. Maybe also send our paper to the authors.}

\citet{BU2023Rat} called a mechanism for cake-cutting, in which no agent has a safe-and-profitable manipulation, \emph{risk-averse truthful (RAT)};
we prefer to call such a mechanism \emph{risk-avoiding truthful}, as the definition assumes that agents completely avoid any risk.
We extend their work by generalizing RAT to any social choice problem, and by suggesting a quantitative measure of the robustness of a mechanism to such manipulations.

\citet{troyan2020obvious} introduce \emph{not-obvious manipulability (NOM)}, which assumes agents consider only extreme best or worst cases. RAT and NOM are independent notions. \citet{regret2018Fernandez} define \emph{regret-free truth-telling (RFTT)}, where agents never regret truth-telling after observing the outcome. RAT and RFTT do not imply each other. Additionally, \citet{slinko2008nondictatorial,slinko2014ever,hazon2010complexity} study "safe manipulations" in voting, but they consider coalition of voters and a different type of risk - that too many or too few participants will perform the exact safe manipulation.


\paragraph{Alternative Measurements.}
There are many approaches for quantifying manipulability from different perspectives. 
One approach considers the computational complexity of finding a profitable manipulation — e.g., \cite{bartholdi1989computational,bartholdi1991single} (see \cite{faliszewski2010ai,veselova2016computational} for surveys). 
Another measurement is the number of bits
an agent needs to know in order to have a safe manipulation, in the spirit of communication complexity,  e.g., \cite{nisan2002communication, grigorieva2006communication, Communication2019Branzei,Babichenko2019communication} and compilation complexity - e.g., \citep{chevaleyre2009compiling,xia2010compilation,karia2021compilation}.
A third approach evaluates the probability that a profitable manipulation exists —e.g., \cite{barrot2017manipulation,lackner2018approval,lackner2023free}.
The \emph{incentive ratio}, which measures how much an agent can improve their utility by manipulating, is also widely studied—e.g., \cite{chen2011profitable,chen2022incentive,li2024bounding,cheng2022tight,cheng2019improved}. Other metrics include assessing the average and maximum gain per manipulation \cite{aleskerov1999degree} and counting the number of agents who benefit from manipulating \cite{andersson2014budget,andersson2014least}.

\er{Please refer to \Cref{apx:related} for a more thorough discussion of these alternative notions, and their relation to our notion of RAT-degree.}




\iffalse

\eden{maybe to put somewhere:
\begin{itemize}
    \item Strategy-Proofness: ? \eden{I'm not sure what the difference is. Wiki says that it means that each player has a weakly-dominant strategy (so that no player can gain by "spying" over the other players), but in \url{https://www.cs.cmu.edu/~sandholm/cs15-892F13/algorithmic-game-theory.pdf} page 218: "incentive compatibility also called strategy-proofness or truthfulness".}
    \item 
    \item Truthfulness: 
    \begin{itemize}
        \item No profitable manipulations.
        \item Telling the truth is a weakly-dominant strategy. 
    \end{itemize} 
    \item RAT: 
    \begin{itemize}
        \item No safe-and-profitable manipulations.
        \item No strategy weakly-dominates telling the truth.
    \end{itemize} 
    \item Maximin Strategy-Proofness: 
    \begin{itemize}
        \item No always-profitable manipulations. 
        \item No strategy strictly-dominates telling the truth.
    \end{itemize} 
    \item NOM: No obvious manipulations. 
\end{itemize}
Also:
\begin{itemize}
    \item A safe-and-profitable manipulation: a strategy that weakly-dominates telling the truth.
    \item An always-profitable manipulation: strategy that strictly-dominates telling the truth.
\end{itemize}
}

\newpage
\fi
\section{Preliminaries}\label{sec:preliminaries}


We consider a generic social choice setting, with a set of $n$ \emph{agents} $N = \{a_1, \ldots, a_n\}$, and a set of potential \emph{outcomes} $X$.
%
% EREL: safe-profitable-manipulation: see https://english.stackexchange.com/a/1159/24024
%
% \paragraph{Preferences.} % Types.?
Each agent, $a_i \in N$, has preferences over the set of outcomes $X$, that can be described in one of two ways: (1) a linear ordering of the outcomes, or (2) a utility function from $X$ to $\mathbb{R}$.
%
The set of all possible preferences for agent~$a_i$ is denoted by $\domain_i$, and is referred to as the agent's \emph{domain}. 
We denote the agent's \emph{true} preferences by $T_i \in \domain_i$.
%
Unless otherwise stated, when agent~$a_i$ weakly prefers the outcome $x_1$ over $x_2$, it is denoted by $x_1 \succeq_i x_2$; and when she strictly prefers $x_1$ over $x_2$, it is denoted by $x_1 \succ_i x_2$.

% \paragraph{Profile.} For any subset of the $n$ agents, a profile for this set is a list of types, one for each agent in the set. 

% \paragraph{Aggregation.} 
% Let $\mathbf{R} := (R_1,\ldots,R_n) \in \domain_1\times\cdots \times \domain_n$ be a profile for the $n$ agents.
% For each agent $i$, we denote by $\mathbf{R}_{-i}$ the profile for all the agents except $i$ in which their types are as in $\mathbf{R}$: 
% $$ \mathbf{R}_{-i} := (R_1, 
% \ldots, R_{i-1}, R_{i+1},\ldots,R_n)$$
% Given any type for agent $i$, $R'_i \in \domain_i$, we denote by $(R'_i, \mathbf{R}_{-i})$ the profile for all agents in which agent $i$'s type is $R'_i$ and all other agents' types are as in $\mathbf{R}_{-i}$:
% $$ (R'_i, \mathbf{R}_{-i}) := (R_1, 
% \ldots, R_{i-1}, R'_i, R_{i+1},\ldots,R_n)$$

% \paragraph{Mechanism.}
A \emph{mechanism} or \emph{rule}  is a function $f: \domain_1\times\cdots \times \domain_n \to X$, which takes as input a list of reported preferences $P_1,\ldots,P_n$ (which may differ from the true preferences), and returns the chosen outcome.
In this paper, we focus on deterministic and single-valued mechanisms. 
% \eden{single-valued? deterministic?}

% \paragraph{Preferences Aggregation: Single Agent.}
For any agent $a_i \in N$, we denote by $(P_i, \prefsExcI)$ the preference profile in which agent~$a_i$ reports $P_i$, and the other agents report $\prefsExcI$.

% % ===== more formal too long ======
% % ======================================
% % We denote the set of all possible combinations of preferences, $\domain_1\times\cdots \times \domain_n$, by $\domains_N$.
% For each agent~$i$, let $\domainsExcI := \domain_1 \times \cdots \times D_{i-1} \times D_{i+1} \times \cdots \times D_{n}$ denote the set of all combinations of preferences of all agents except $i$.

% Given preferences for agent~$i$, $P_i \in \domain_i$, and preferences for all agents except $i$, $\prefsExcI \in \domainsExcI$, we use $(P_i, \prefsExcI)$ to denote the preferences profile in which agent~$i$ has the preferences $P_i$ and the preferences of all other agents are as in $\prefsExcI$.
% % ======================================




% \subsection{Truthfulness and Risk-Avoiding Truthfulness (RAT)}
% \eden{new}
% \eden{I think it would be clear if we'll use $P'_i$}

\paragraph{Truthfulness}
A \emph{manipulation} for a mechanism $f$ and agent~$a_i \in N$ is an untruthful report $P_i \in \domain_i \setminus \{T_i\}$.
A manipulation is \emph{profitable} if there exists a set of preferences of the other agents for which it increases the manipulator's utility:
\begin{align}
\label{eq:manipulation}
    &\exists \prefsExcI\in \domainsExcI : \quad f(P_i,\prefsExcI) \succ_i f(T_i,\prefsExcI)
\end{align}
% In words:  agent~$i$ \emph{strictly} prefers the outcome resulting from reporting $P_i$ over the outcome resulting from reporting the truth $T_i$.
%
% In this case, we say that agent~$i$ has a manipulation.
%
% \paragraph{Truthful.}
%
%
% A mechanism $f$ is called \emph{truthful} if it has no profitable manipulations for any agent~$a_i\in N$; and \emph{manipulable} otherwise. 
%
A mechanism $f$ is called \emph{manipulable} if some agent~$a_i$ has a profitable manipulation; otherwise $f$ is called \emph{truthful}. 



% \paragraph{Safe Manipulation.}
\paragraph{RAT}
A manipulation is \emph{safe} if it never harms the manipulator's utility -- it is weakly preferred over telling the truth for any possible preferences of the other agents:
\begin{align}
    &\forall \prefsExcI\in \domainsExcI : \quad f(P_i,\prefsExcI) \succeq_i f(T_i,\prefsExcI)
\end{align}
% \eqref{eq:manipulation} says, as above, that $P_i$ is a profitable manipulation for $a_i$. \eqref{eq:safe-manipulation} says, in words, that for any possible preferences of the other agents, agent~$a_i$ weakly prefers the outcome resulting from reporting $P_i$ over the outcome resulting from reporting the truth $T_i$.
%
% In this case, we say that agent~$i$ has a safe-manipulation.
%
% \paragraph{Risk-Averse Truthful (RAT)}
%
%
% A mechanism $f$ is called \emph{risk-averse truthful (RAT)} if it 
% % is \emph{not} safely manipulable \cite{BU2023Rat}.
% has no profitable-and-safe manipulations for any agent $a_i\in N$; and \emph{safely-manipulable} otherwise. 
%
A mechanism $f$ is called \emph{safely-manipulable} if some agent~$a_i$ has a manipulations that is profitable and safe; otherwise $f$ is called \emph{risk-avoiding truthful (RAT)}. 




\section{The RAT-Degree}\label{sec:RAT-degree}


\newcommand{\prefsOf}[1]{\mathbf{P}_{#1}}
\newcommand{\domainsOf}[1]{\mathbf{D}_{#1}}

% \paragraph{Preferences Aggregation: Multiple Agents.}
% Fix an agent $a_i \in N$. 
Let $k \in \{0,\ldots, n-1\}$, $K \subseteq N \setminus \{a_i\}$ with $|K| = k$ and $\notK := N \setminus (\{a_i\} \cup K)$.
We denote by $(P_i, \prefsOf{K}, \prefsOf{\notK})$ the preference profile in which the preferences of agent $a_i$ are $P_i$, the preferences of the agents in $K$ are $\prefsOf{K}$, and the preferences of the agents in $\notK$ are $\prefsOf{\notK}$.

\eden{not sure about the name.. too many '-'}
\begin{definition}
A manipulation, $P_i$, is \emph{profitable-and-safe-given-$k$-known-agents} if for some subset $K \subseteq N \setminus \{a_i\}$ with $|K| = k$ and some preferences for them $\prefsOf{K} \in \domainsOf{K}$, the following holds:
    \begin{align}
    \label{eq:k-manipulation}
            \quad \exists \prefsOf{\notK} \in \domainsOf{\notK} \colon \quad f(P_i,\prefsOf{K}, \prefsOf{\notK}) \succ_i f(T_i,\prefsOf{K}, \prefsOf{\notK})
            \\
    \label{eq:k-safe-manipulation}
            \text{ and } \quad   \forall \prefsOf{\notK} \in \domainsOf{\notK} \colon \quad f(P_i,\prefsOf{K}, \prefsOf{\notK}) \succeq_i f(T_i,\prefsOf{K}, \prefsOf{\notK})
    \end{align}
\end{definition}
In words: The agents in $K$ are those  whose preferences are \emph{Known} to $a_i$; the agents in $\notK$ are those whose preferences are unknown to $a_i$.
Given that the preferences of the known-agents are $\prefsOf{K}$, \eqref{eq:k-manipulation} says that there exist a preference profile of the unknown-agents that makes the manipulation profitable for agent~$a_i$; while \eqref{eq:k-safe-manipulation} says that the manipulation is safe -- it is weakly preferred over telling the truth for any preference profile of the unknown-agents.

%
% (a) , (b) , (c) ; and (3) ; such that agent~$i$ has a safe manipulation given $\prefsOf{K}$.
% That is, agent~$i$ has an alternative report $P_i \in \domain_i$,  $P_i\neq T_i$, such that:
%


% In this case, we say that agent~$i$ has a safe-manipulation given $\prefsOf{K}$.
%
% In this case, we say that agent~$i$ has $k$-known-agents safe-manipulation.
A profitable-and-safe manipulation (with no known-agents) is a special case in which $K=\emptyset$.

\begin{definition}
A mechanism $f$ is called \emph{$k$-known-agents safely-manipulable} if some agent~$a_i$ has a profitable-and-safe-manipulation-given-$k$-known-agents.
\end{definition}

\begin{propositionrep}
    Let $k \in \{0, \ldots, n-2\}$.
    If a mechanism is $k$-known-agents safely-manipulable, then it is also $(k+1)$-known-agents safely-manipulable.
\end{propositionrep}
\begin{proof}
    By definition, some agent~$a_i$ has a profitable-and-safe-manipulation-given-$k$-known-agents. 
    That is, there exists a subset $K \subseteq N \setminus \{a_i\}$ with $|K| = k$ and some preference profile for them $\prefsOf{K} \in \domainsOf{K}$, such that \eqref{eq:k-manipulation} and \eqref{eq:k-safe-manipulation} hold.
    Let $a_j \in \notK$.
    Consider the preferences $P_j$ that $a_j$ has in some profile satisfying \eqref{eq:k-manipulation} (profitable). 
    Define  $K^+ := K \cup \{a_j\}$ and construct a preference profile where the preferences of the agents in $K$ remain $\prefsOf{K}$, and $a_j$'s preferences are set to $P_j$.
    Since \eqref{eq:k-manipulation} holds for $P_j$, the same manipulation remains profitable given the new set of known-agents.
    Moreover, \eqref{eq:k-safe-manipulation} continues to hold, as the set of unknown agents has only shrunk.
    Thus, the mechanism is also $(k+1)$-known-agents safely manipulable.
\end{proof}


% \paragraph{The RAT-Degree.}
\begin{definition}%[The RAT-Degree]
    The \emph{RAT-degree} of a mechanism $f$ is the minimum $k$ for which the mechanism is $k$-known-agent safely manipulable, or $n$ if there is no such $k$.
\end{definition}


% \paragraph{Hierarchy.}

\begin{observation}
A mechanism is truthful~ if-and-only-if its RAT-degree is $n$.
\end{observation}

\begin{observation}
 A mechanism is RAT~ if-and-only-if its RAT-degree is at least $1$.
\end{observation}

\Cref{fig:hierarchy-RAT-Degree} illustrates the relation between classes of different RAT-degree. 
% \eden{Not sure about the diagram, I added another option.}

% \begin{figure}[h]
%     \centering
%     \includegraphics[width=\linewidth]{images/hierarchy-diagram-RAT-Degree-w.png}
%     \caption{Hierarchy of the Manipulability and RAT-Degree Classes}
%     \label{fig:hierarchy-RAT-Degree}
% \end{figure}
% % \eden{to add somewhere that all the mechanism that have degrees $1, \ldots, n$ are RAT}

\begin{figure}[t]
    \centering
    \includegraphics{images/sub-hierarchy-diagram-RAT-Degree.png}
    \caption{Hierarchy of the Manipulability and RAT-Degree Classes. KA stands for Known-Agents.}
    \label{fig:hierarchy-RAT-Degree}
\end{figure}



\begin{wraptable}{r}{5cm}
    \centering
% \begin{table}[h]
    \begin{tabular}{c|c|c|c|c}
         & $\prefsExcI^1$ & $\prefsExcI^2$ & $\prefsExcI^3$ &... \\
         \hline
         $T_i$ & & &\\
         \hline
         $P_i \neq T_i$ & & &\\
    \label{tab:safe-manip-i}
    \end{tabular}
    \caption{A Safe-And-Profitable Manipulation from an Agent Perspective.}
% \end{table}
\end{wraptable}
\subsection{An Intuitive Point of View}
Consider \Cref{tab:safe-manip-i}. 
When the risk-avoiding agent has no information ($0$-known-agents), a profitable-and-safe manipulation is a row in the table that represents an alternative report $P_i \neq T_i$, that dominates $T_i$ -- this means that for \emph{each} one of the columns, the outcome in the corresponding row is better than the outcome of the first row.
When the risk-avoiding agent has more information ($k$-known-agents, when $k >0$), it is equivalent to considering a strict subset of the columns.
Lastly, when the risk-avoiding agent has a full information ($(n-1)$-known-agents), it is equivalent to consider only one column.

% For clarity, \Cref{apx:not-KNAM} provides the formal definition of \emph{not} $k$-known-agents safely manipulable.

% \eden{for us, maybe to put in the appendix..?}
% \ifdefined\DRAFT 
% For $0 < k < n$, in order to prove that the degree is $k$ we need to show that it is $k$-known-agents safely manipulable, but also that it is \emph{not} $(k-1)$-known-agents safely manipulable.

% For clarity, we provide the direct definition of \emph{not} $k$-known-agents safely manipulable.


% % \eden{$=$ (1) telling the truth is a weak dominant strategy given any strategy profile for $k$-known-agents, or (2) lying is not a weak dominant strategy}
% \fi


% \eden{Equivalent, maybe more intuitive?} NOT TRUE!!!

% \begin{definition}[\textbf{Not} $k$-Known-Agent Safely Manipulable]
%     Let $k \in \{0,\ldots, n-1\}$. A mechanism $f$ is \emph{Not} $k$-known-agent safely manipulable if for each agent $i \in N$ and an alternative report for agent~$i$, $P_i \in \domain_i$,  $P_i\neq T_i$, \emph{at least one of} the following holds:
%     \begin{enumerate}
%         \item For any set of preferences for the other agents $\prefsExcI\in \domainsExcI$: 
%         \begin{align*}
%             f(T_i,\prefsExcI) \succeq_i f(P_i,\prefsExcI)
%         \end{align*} 
%         Agent~$i$ weakly prefers the outcome resulting from reporting the truth $T_i$ over the outcome resulting from reporting $P_i$.

%         \item There exists a set $\notK \subseteq N \setminus \{i\}$ with $|\notK| = (n-k-1)$, and preferences for the agents in $\notK$, $\prefsOf{\notK} \in \domainsOf{\notK}$, such that the following holds for $K:= N \setminus (\{i\} \cup K)$:
%         \begin{align*}
%             \forall \prefsOf{K} \in \domainsOf{K} \colon \quad f(T_i,\prefsOf{K}, \prefsOf{\notK})  \succ_i f(P_i,\prefsOf{K}, \prefsOf{\notK})
%         \end{align*}

%         There exists a set of preferences of the agents in $\notK$, for which agent~$i$ weakly prefers the outcome resulting from reporting the truth $T_i$ over the outcome resulting from reporting $P_i$.
%     \end{enumerate}
% \end{definition}
% \newpage
\section{Summary of Results}
\eden{I think it would be nice if we mark the new mechanisms - I used $\dagger$ }

\renewcommand{\arraystretch}{1.1}

% =============================================

\subsection*{Auctions for a Single Good (\Cref{sec:single-item-auction})}


\begin{tabular}{|K{1.8cm}|K{1.8cm}|K{1.8cm}|K{1.4cm}|K{2.5cm}|K{1.8cm}|}
\hline
RAT-degree:  &  0 & 1 & $\ldots$ &$n-1$ & $n$ \\
\hline
Mechanism: & 1st-price & 1st-price w.discount & & $\dagger$ Avg. price 1st-and-2nd  &  2nd-price\\
\hline
\end{tabular}

% \begin{itemize}
%     \item First-Price Auction 
    
%     RAT-degree: $0$.
    
%     \item First-Price Auction with Positive Discount

%     RAT-degree: $1$.
    
%     \item Average-Between-First-And-Second-Price Auction
    
%     RAT-degree: $n-1$.
    
%     \item Second-Price Auction  (Known)
    
%     RAT-degree: $n$.
% \end{itemize}

% =============================================


\subsection*{Indivisible Goods Allocation (\Cref{sec:indivisible-good-aloc})}


\begin{tabular}{|K{1.8cm}|K{1.8cm}|K{2.52cm}|K{1.4cm}|K{1.8cm}|K{1.8cm}|}
\hline
RAT-degree:  &  0 & 1 & $\ldots$ & $n-1$& $n$ \\
\hline
\multirow{2}{*}{Mechanism:} & \multirow{2}{*}{Utilitarian} & Norm. Utilitarian & & $\dagger$ New \rmark{name}  &  \multirow{2}{*}{Dictatorship}\\
\cline{3-3}
 &  & Round-Robin & & \Cref{sect:indivisible-EF1-n-1} & \\
\hline
\end{tabular}



\eden{TODO:
\begin{itemize}
    \item we only proved that Round Rubin is at most $1$ -- need to add or to change. 
    \item need to give a name to the new mechanism in \Cref{sect:indivisible-EF1-n-1}
\end{itemize}}

% \begin{itemize}
%     \item Utilitarian Goods Allocation
    
%     RAT-degree: $0$.
    
%     \item Normalized Utilitarian Goods Allocation

%     % For two agents and two goods, the RAT-degree is $0$.

%     % For more than two agents, the RAT-degree is at least $1$ (is $1$??).

%     For (one or) more than two goods, the RAT-degree is $1$.
    
%     [the case of two goods is special -- safely manipulable for some agents]


%     \item Round-Robin Goods Allocation 

%     When there are more goods than agents, the RAT-degree is at most $n-2$. \eden{todo}

%     \item Dictatorship (Known) \eden{reference?}
    
%     RAT-degree: $n$. [impossibility]
% \end{itemize}


% =============================================

\subsection*{Cake Cutting (\Cref{sec:cake-cutting})}


\begin{tabular}{|K{1.8cm}|K{1.8cm}|K{2.52cm}|K{1.4cm}|K{1.8cm}|K{1.8cm}|}
\hline
RAT-degree:  &  0 & 1 & $\ldots$ & $n-1$& $n$ \\
\hline
\multirow{2}{*}{Mechanism:} & \multirow{2}{*}{Utilitarian} & Norm. Utilitarian & & $\dagger$ New \rmark{name}  &  \multirow{2}{*}{Dictatorship}\\
\cline{3-3}
 &  & \rmark{TODO} & & \Cref{sect:cake-Prop+PO} & \\
\hline
\end{tabular}




% ==============================================


\subsection*{Single-Winner Voting (\Cref{sec:single-winner-voting})}



\begin{tabular}{|K{1.8cm}|K{1.8cm}|K{1.34cm}|K{3cm}|K{1.34cm}|K{1.8cm}|}
\hline
RAT-degree:  &  0 & $\ldots$ & $n/2+1$& $\ldots$ & $n$ \\
\hline
\multirow{2}{*}{Mechanism:} &  &  & Plurality & &  \multirow{2}{*}{Dictatorship}\\
\cline{4-5}
& && \multicolumn{2}{c|}{Positional Scoring Rules} &\\
\hline
\end{tabular}



% \begin{itemize}
%     \item Plurality
%     RAT-Degree: $n/2 + 1$. \eden{todo}

%     \item $\ell$-approval

%     RAT-Degree: $n/(\ell + 1) + 1$. \eden{need to think about it}

%     \item Veto

%     RAT-Degree: $n/m + 1$. \eden{need to verify}

%    \item Dictatorship (Known, Arrow's theorem)
   
%    RAT-degree: $n$.
% \end{itemize}


% =========================================



\subsection*{Stable Matching (\Cref{sec:matching})}

\begin{tabular}{|K{1.8cm}|K{1.8cm}|K{1.34cm}|K{1.34cm}|K{1.34cm}|K{3cm}|K{1.34cm}|K{1.8cm}|}
\hline
RAT-degree:  &  0 & $\ldots$ & $3$ & $4$ & $5$ &  $\ldots$ & $n$ \\
\hline
\multirow{2}{*}{Mechanism:} &  &  & Plurality & &  Dictatorship\\
\cline{4-6}
& && \multicolumn{3}{c|}{Positional Scoring Rules}\\
\hline
\end{tabular}



% \eden{to add section}
% \begin{itemize}
%     \item Deferred acceptance (Gale–Shapley)
    
%     RAT-Degree: 
%     \begin{itemize}
%         \item For the men (proposing): $n$ (known)

%         \item  For the women (recipients of proposals): at most $3$ with "blocking", at most $5$ without.
%     \end{itemize}
% \end{itemize}

% \newpage

\ifdefined\DRAFT
\section{Didn't get there yet...} 
% \eden{for us: to remove}
\begin{itemize}
    \item 
    \textbf{Auctions:} 
    $k$-identical-items auction;
    GSP auction.

    Problems with second price auction: \href{https://citeseerx.ist.psu.edu/document?repid=rep1&type=pdf&doi=4d92838c1357e132987d82ad3c6a426141a7f8cd}{(link)}

    \item \textbf{Goods Allocations:} 
    
    Borda (ranking), Max-Nash-Welfare, Egalitarian,..

    \item \textbf{Single-Winner Voting:} 
    
    Plurality with run-off, Borda, k-approval, Condorcet..

    \item \textbf{Budget-proposal aggregation}

    multi-dimensional

    \item \textbf{Approval-Based Committee Election:} 
    
    need to verify Phragmen’s rule,
    
    \item \textbf{Cake Cutting}

    \item \textbf{Stable Matching}
    
    \item \textbf{Rent division (sharing?)}
\end{itemize}
\fi

% \eden{need to mention somewhere that, for simplicity, we start by single-step mechanisms}


% \newpage
\section{Summary of Results}
\eden{I think it would be nice if we mark the new mechanisms - I used $\dagger$ }

\renewcommand{\arraystretch}{1.1}

% =============================================


\begin{tabular}{|K{1cm}|K{2cm}|K{2cm}|K{2cm}|K{2cm}|K{2cm}|}
\hline
RAT-degree  &  Auctions for a Single Good (\Cref{sec:single-item-auction}) & Indivisible Goods Allocation (\Cref{sec:indivisible-good-aloc}) & $\ldots$ &$n-1$ & $n$ \\
\hline
\hline
$0$ & 1st-price & 
Utilitarian & cake & voting & matching\\
\hline
$1$ & 1st-price w.discount & Norm. Utilitarian, Round-Robin & cake & voting & matching\\
\hline
$3$ &  &  &  &  & DA \\
\hline
$5$ &  & & &  & DA without truncation\\
\hline
% $\vdots$ & \multicolumn{5}{c|}{}\\
% $\vdots$ & & & & &\\
% \hline
$n/2+1$ &  &  &  & voting & matching\\
\hline
% $\vdots$ & & & & &\\
% \hline
$n-1$ & $\dagger$ Avg. price 1st-and-2nd  & $\dagger$ New \rmark{name} (\Cref{sect:indivisible-EF1-n-1}) & cake & voting & matching\\
\hline
$n$ & 2nd-price & Dictatorship & cake & voting & matching \\
\hline
\end{tabular}


\section{Auction for a Single Good}
\label{sec:single-item-auction}

We consider a seller owning a single good, and $n$ potential buyers (the agents). 
% \eden{TODO: use this structure for all sections}
The true preferences $T_i$ are given by real values $v_i \geq 0$, representing the happiness of agent $i$ from receiving the good. 
The reported preferences $P_i$ are the ``bids'' $b_i \geq 0 $.
A mechanism in this context has to determine the \emph{winner} --- the agent who will receive the good, and the \emph{price} --- how much the winner will pay.

We assume that agents are quasi-linear -- meaning their valuations can be interpreted in monetary units. 
Accordingly, the utility of the winning agent is the valuation minus the price; while the utility of the other agents is zero. 

\paragraph{Results.} The two most well-known mechanisms in this context are first-price auction and second-price auctions. 
%
First-price auction is known to be manipulable; moreover, it is easy to show that it is safely-manipulable, so its RAT-degree is $0$.
We show that a first-price auction with a positive discount has RAT-degree $1$.
%
Second-price auction is known to be truthful, so its  RAT-degree is $n$.
However, it has some decisive practical disadvantages \cite{ausubel2006lovely}, in particular, when buyers are risk-averse, the expected revenue of a second-price auction is lower than that of a first-price auction \cite{nisan2007algorithmic};
even when the buyers are risk-neutral, a risk-averse \emph{seller} would prefer the revenue distribution of a first-price auction \citep{krishna2009auction}.%
%\footnote{We are grateful to Michael Greinecker for the information.
% https://economics.stackexchange.com/q/59899/385
% https://economics.stackexchange.com/a/24369/385}

This raises the question of whether it is possible to combine the advantages of both auction types.
Indeed, we prove that 
%Lastly, we establish the following surprising and intriguing result:
% EREL: I do not like to compliment my own results... better to let the readers see for themselves.
any auction that applies a weighted average between the first-price and the second-price achieves a RAT-degree of of $n-1$, which is very close to being fully truthful (RAT-degree $n$). 
This implies that a manipulator agent would need to obtain information about all $n-1$ other agents to safely manipulate -- which is a very challenging task.
Importantly, the seller’s revenue from such auction is higher compared to the second-price auction, giving this mechanism a significant advantage in this context. 
This result opens the door to exploring new mechanisms that are not truthful but come very close it. 
Such mechanisms may enable desirable properties that  are unattainable with truthful mechanisms.


\subsection{First-Price Auction}
In first-price auction, the agent who bids the highest price wins the good and pays her bid; the other agents get and pay nothing. 

It in well-known that the first-price auction is not truthful.
We start by proving that the situation is even worse: first-price auction is safely-manipulable, meaning that its RAT-degree is $0$.

\begin{theoremrep}
First-price auction is safely manipulable (RAT-degree = $0$).
\end{theoremrep}
\begin{proofsketch}
    A truthful agent always gets a utility of $0$,  whether he wins or loses.
    On the other hand, an agent who manipulates by bidding slightly less than $v_i$ gets a utility of at least $0$ when he loses and a strictly positive utility when he wins. 
    Thus, this manipulation is safe and profitable.
\end{proofsketch}
\begin{proof}
To prove the mechanism is safely manipulable, we need to show an agent and an alternative bid, such that the agent always \emph{weakly} prefers the outcome that results from reporting the alternative bid over reporting her true valuation, and \emph{strictly} prefers it in at least one scenario.
We prove that the mechanism is safely-manipulable by all agents with positive valuations.
    
Let $i \in N$ be an agent with valuation $v_i > 0$, we shall now prove that bidding $b_i < v_i$ is a safe manipulation.
    
We need to show that for any combination of bids of the other agents, bidding $b_i$ does not harm agent~$i$, and that there exists a combination where bidding $b_i$ strictly increases her utility. 
    To do so, we consider the following cases according to the maximum bid of the other agents:
    \begin{itemize}
        \item The maximum bid is smaller than $b_i$: if agent~$i$ bids her valuation $v_i$, she wins the good and pays $v_i$, results in utility $0$. 
        However, by bidding $b_i < v_i$, she still wins but pays only $b_i$, yielding a positive utility. 
        
        Thus, in this case, agent~$i$ strictly increases her utility by lying.

        
        \item The maximum bid is between $b_i$ and $v_i$: if agent $i$ bids her valuation $v_i$, she wins the good and pays $v_i$, resulting in utility $0$. 
        By bidding $b_i < v_i$, she loses the good but pays noting, also resulting in utility $0$.
        
        Thus, in this case, bidding $b_i$ does not harm agent~$i$.

        \item  The maximum bid is higher than $v_i$: Regardless of whether agent~$i$ bids her valuation $v_i$ or $b_i < v_i$, she does not win the good, resulting in utility $0$.
        
        Thus, in this case, bidding $b_i$ does not harm agent~$i$.
    \end{itemize}

\end{proof}





% ======================================
\subsection{First-Price Auction with Discount}\label{sec:first-price-w-discount}

In first-price auction with discount, the agent $i$ with the highest bid $b_i$ wins the item and pays $(1-t)b_i$, where $t\in(0,1)$ is a constant. As before, the other agents get and pay nothing. 



We prove that, although this minor change does not make the  mechanism truthful, it increases the degree of trustfulness for risk-averse agents. 
However, it is still quite vulnerable to manipulation as knowing the strategy of one other agent might be sufficient to safely-manipulate it.

\begin{theorem}
\label{auction-with-discount}
The RAT-degree of the First-Price Auction with Discount is $1$.
\end{theorem}

% \eden{to try to shorten the proofs}

To prove that the RAT-degree is $1$, we need to show that the mechanism is (a) \emph{not} safely-manipulable, and (b) 
$1$-known-agents safely-manipulable.
We prove each of these in a lemma.

\begin{lemmarep}
    First-Price Auction with Discount is not safely-manipulable. 
\end{lemmarep}
\begin{proofsketch}
    % The key observation is that h
    Here, unlike in the first-price auction, whenever the agent wins the good, she gains positive utility.  This implies that no manipulation can be safe. If she under-bids her value, she risks losing the item, which strictly decreases her utility. If she over-bids, she might end up paying more than necessary, reducing her utility without increasing her chances of winning.
\end{proofsketch}
\begin{proof}
We need to show that, for each agent $i$ and any bid $b_i\neq v_i$, at least one of the following is true: either (1) for any combination of bids of the other agents, agent $i$ weakly prefers the outcome from bidding $v_i$; or (2) there exists such a combination for which $i$ strictly prefers the outcome from bidding $v_i$. We consider two cases.

Case 1: $v_i = 0$. In this case condition (1) clearly holds, as bidding $v_i$ guarantees the agent a utility of $0$, and no potential outcome of the auction can give $i$ a positive utility.

Case 2: $v_i > 0$. In this case we prove that condition (2) holds. We consider two sub-cases:
\begin{itemize}
\item Under-bidding ($b_i<v_i$): 
    whenever $\displaystyle \max_{j\neq i}b_j \in (b_i,v_i)$, 
%    Consider any combination of bids of the other agents, in which the maximum bid is strictly smaller than $v_i$ and strictly greater than $b_i$.    In this case, 
when agent $i$ bids truthfully she wins the good and pays $(1-t) v_i$, resulting in utility $t v_i > 0$;
but when she bids $b_i$ she does not win, resulting in utility $0$.

\item Over-bidding ($v_i<b_i$): 
    whenever $\displaystyle \max_{j\neq i} < v_i$, 
%    Consider any combination of bids of the other agents, in which the maximum bid is strictly smaller than $v_i$.    In this case, 
when agent $i$ bids truthfully her utility is $t v_i$ as before;
when she bids $b_i$ she wins and pays $(1-t)b_i > (1-t)v_i$, so her utility is less than $t v_i$.
\end{itemize}
In both cases lying may harm agent $i$. Thus, she has no safe manipulation. 
\end{proof}

\begin{lemmarep}
    First-Price Auction with Discount is $1$-known-agents safely-manipulable. 
\end{lemmarep}
\begin{proofsketch}
    Consider the case where the manipulator $a_i$ knows there is another agent $a_j$ bidding $b_j \in (v_i, v_i/(1-t))$.
    When biding truthfully, she loses and gets zero utility. 
    However, by bidding any value $b_i\in (b_j, v_i/(1-t))$, she either wins and gains positive utility or loses and remains with zero utility. Thus it is safe and profitable. 
\end{proofsketch}
\begin{proof}
% Exists, Exists, Forall
we need to identify an agent $i$, for whom there exists another agent $j \neq i$ and a bid $b_j$, such that if $j$ bids $b_j$, then agent $i$ has a safe manipulation.
% (i.e., a bid different from its valuation that its outcome is always weakly preferred and is strictly preferred in at least one scenario).

Indeed, let $i \in N$ be an agent with valuation $v_i >0$ and let $j \neq i$ be another agent who bids some value  $b_j \in (v_i, v_i/(1-t))$. 
We prove that bidding any value $b_i\in (b_j, v_i/(1-t))$ is a safe manipulation for $i$.

If $i$ truthfully bids $v_i$, she loses the good (as $b_j > v_i$), and gets a utility of $0$.

If $i$ manipulates by bidding some $b_i\in (b_j, v_i/(1-t))$, then she gets either the same or a higher utility, depending on the maximum bid among the unknown agents ($b^{\max} := \displaystyle \max_{\ell\neq i, \ell\neq i}b_{\ell}$):
    \begin{itemize}
        \item If $b^{\max} < b_i$, then $i$ wins and pays $(1-t)b_i$, resulting in a utility of $v_i - (1-t)b_i > 0$, as $(1-t)b_i < v_i$. Thus, $i$ strictly gains by lying.

        \item  If $b^{\max} > b_i$, then $i$ does not win the good, resulting in utility $0$. Thus, in this case, bidding $b_i$ does not harm $i$.
        
        \item If $b^{\max} = b_i$, then one of the above two cases happens (depending on the tie-breaking rule).
    \end{itemize}
In all cases $b_i$ is a safe manipulation, as claimed.
\end{proof}



\subsection{Average-First-Second-Price Auction} 
% \eden{I think this name can be a bit confusing because it can be average between all prices}
In the Average-First-Second-Price (AFSP) Auction, the agent $i$ with the highest bid $b_i$ wins the item and pays $w b_i + (1-w) b^{\max}_{-i}$, where $\displaystyle b^{\max}_{-i} := \max_{j\neq i} b_j$ is the second-highest bid, and $w\in(0,1)$ is a fixed constant. That is, the price is a weighted average between the first price and the second price.


We show that this simple change makes a significantly difference -- the RAT-degree increases to $n-1$. 
This means that a manipulator agent would need to obtain information about all other agents to safely manipulate the mechanism --- a very challenging task in practice.

\begin{theorem}
\label{auction-average-price}
The RAT-degree of the Average-Price Auction is $(n-1)$.
\end{theorem}
The theorem is proved using the following two lemmas.

\begin{lemmarep}
The AFSP mechanism is not $(n-2)$-known-agents safely-manipulable.
\end{lemmarep}

\begin{proofsketch}
Let $a_i$ be a potential manipulator, $b_i$ a potential manipulation, $K$ be a set of known agents, with $|K| = n-2$, and $\mathbf{b}_{K}$ be a vector that represents their bids.
We prove that the only unknown-agent $a_j$ can make any manipulation either not profitable or not safe.
We denote by $b^{\max}_{K}$ the maximum bid among the agents in $K$ and consider each of the six possible orderings of $v_i$, $b_i$ and $b^{\max}_{K}$.
In two cases ($v_i < b_i < b^{\max}_{K}$ or $b_i < v_i < b^{\max}_{K}$) the manipulation is not profitable;
in the other four cases, the manipulation is not safe.
\end{proofsketch}

% === prev: (backup):
% \begin{proofsketch}
% \erel{TODO: try to shorten sketch}
% Let $i\in N$ be an agent with true value $v_i>0$, and a manipulation $b_i \neq v_i$. We show that this manipulation is unsafe even when knowing the bids of $n-2$ of the other agents.

% Let $K$ be a subset of $(n-2)$ of the remaining agents (the agents in $K$ are the ``known agents''), and let $\mathbf{b}_{K}$ be a vector that represents their bids.
% Lastly, let $j$ be the only agent in $N\setminus (K\cup \{i\})$.
% We need to prove that at least one of the following is true: either (1) the manipulation is not profitable --- for any possible bid of agent~$j$, agent~$i$ weakly prefers the outcome from bidding $v_i$ 
% % $f(v_i, b_j, b_{K})$
% % rather than bidding $b_i$; 
% over the outcome from bidding $b_i$;
% or (2) the manipulation is not safe --- there exists a bid for agent~$j$, such that agent~$i$ strictly prefers the outcome from bidding $v_i$.

% Let $\displaystyle b^{\max}_{K} := \max_{\ell\in K}b_{\ell}$.
% %\eden{should say somewhere that we neglect equalities}\\
% %\eden{Cases (for us):}
% We consider each of the six possible orderings of $v_i$, $b_i$ and $b^{\max}_{K}$.
% In two cases ($v_i < b_i < b^{\max}_{K}$ or $b_i < v_i < b^{\max}_{K}$) the manipulation is not profitable;
% in the other four cases, the manipulation is not safe.
% \end{proofsketch}



\begin{proof}
Let $i\in N$ be an agent with true value $v_i>0$, and a manipulation $b_i \neq v_i$. We show that this manipulation is unsafe even when knowing the bids of $n-2$ of the other agents.

Let $K$ be a subset of $(n-2)$ of the remaining agents (the agents in $K$ are the ``known agents''), and let $\mathbf{b}_{K}$ be a vector that represents their bids.
Lastly, let $j$ be the only agent in $N\setminus (K\cup \{i\})$.
We need to prove that at least one of the following is true: either (1) the manipulation is not profitable --- for any possible bid of agent~$j$, agent~$i$ weakly prefers the outcome from bidding $v_i$ 
% $f(v_i, b_j, b_{K})$
% rather than bidding $b_i$; 
over the outcome from bidding $b_i$;
or (2) the manipulation is not safe --- there exists a bid for agent~$j$, such that agent~$i$ strictly prefers the outcome from bidding $v_i$.

Let $\displaystyle b^{\max}_{K} := \max_{\ell\in K}b_{\ell}$.
%\eden{should say somewhere that we neglect equalities}\\
%\eden{Cases (for us):}
We consider each of the six possible orderings of $v_i$, $b_i$ and $b^{\max}_{K}$ (cases with equalities are contained in cases with inequalities, according to the tie-breaking rule):


\begin{itemize}
    \item 
%    (Cases 1 and 3) 
    $v_i < b_i < b^{\max}_{K}$ or $b_i < v_i < b^{\max}_{K}$:  
    In these cases (1) holds, as for any bid of agent $j$, agent $i$ never wins. Therefore the manipulation is not profitable.
%    We show that in this case the first condition holds.     Since $b^{\max}_{K}$ is higher than both -- the valuation of agent $i$, $v_i$, and her alternative bid, $b_i$; regardless of what agent $j$ bids, agent $i$ does not win the good and gets a utility $0$.     Thus, in this case, agent~$i$ is indifferent between telling the truth and manipulating.
    
    \item 
%    (Case 2) 
    $v_i < b^{\max}_{K} < b_i$:
We show that (2) holds. Assume that $j$ bids any value $b_j \in (v_i, b_i)$.
When $i$ bids truthfully, she does not win the good so her utility is $0$.
But when $i$ bids $b_i$, she wins and pays a weighted average between $b_i$ and $\max(b_j, b_K^{\max})$. As both these numbers are strictly greater than $v_i$, the payment is larger than $v_i$ as well, resulting in a negative utility. Hence, the manipulation is not safe.
    
    % \item $b_i < v_i < b^{\max}_{K}$: \eden{same as case (1)}     We show that in this case the first condition holds.     Since $b^{\max}_{K}$ is higher than the valuation of agent $i$, $v_i$, and of its and alternative bid, $b_i$; no matter what agent $j$ bids, agent $i$ does not win the item and gets a utility $0$.     Thus, in this case, telling the truth weakly dominants this alternative strategies.
    
\item 
%(Case 4) 
$b^{\max}_{K} < v_i < b_i$: We show that (2) holds.
Assume that $j$ bids any value $b_j < b^{\max}_{K}$.
When $i$ tells the truth, she wins and pays $w v_i + (1-w)b^{\max}_{K}$; 
but when $i$ bids $b_i$, she still wins but pays a higher price, $w b_i + (1-w)b^{\max}_{K}$, so her utility decreases.   
Hence, the manipulation is not safe.
    
\item 
%(Case 5) 
$b_i < b^{\max}_{K} < v_i$: We show that (2) holds.
Assume that agent $j$ bids any value $b_j < b_i$.
When $i$ tells the truth, she wins and pays $w v_i + (1-w)b^{\max}_{K} < v_i$, resulting in a positive utility.
But when $i$ bids $b_i$, she does not win and her utility is $0$.
Hence, the manipulation is not safe.
    
\item 
%(Case 6) 
$b^{\max}_{K} < b_i < v_i$: We show that (2) holds.
Assume that $j$ bids any value $b_j\in(b_i,v_i)$.
When $i$ tells the truth, she wins and pays $w v_i + (1-w)b_j < v_i$, resulting in a positive utility.
But when $i$ bids $b_i$, she does not win and her utility is $0$.
Hence, the manipulation is not safe.


\item 
$b_i < b^{\max}_{K} < v_i$ or $b^{\max}_{K} < b_i < v_i$:
We show that (2) holds. 
Assume that $j$ bids any value $b_j\in(b_i,v_i)$.
When $i$ tells the truth, she wins and pays $w v_i + (1-w)\max(b_j, b^{\max}_{K})$, which is smaller than $v_i$ as both $b_j$ and $b^{\max}_{K}$ are smaller than $v_i$. Therefore, $i$'s utility is positive.
But when $i$ bids $b_i$, she does not win and her utility is $0$.
Hence, the manipulation is not safe.
\end{itemize}
\end{proof}

\begin{lemmarep}
The AFSP mechanism is $(n-1)$-known-agents safely-manipulable.
\end{lemmarep}
\begin{proofsketch}
Consider any combination of bids of the other agents in which all the bids are strictly smaller than $v_i$. 
Let $b^{\max}_{-i}$ be the highest bid among the other agents. 
Then any alternative bid $b_i \in (b^{\max}_{-i}, v_i)$ is a safe manipulation.
\end{proofsketch}
\begin{proof}
% Forall, Forall, Exists
given an agent $i \in N$, we need to show an alternative bid $b_i\neq v_i$ and a combination of $(n-1)$ bids of the other agents, such that the agent strictly prefers the outcome resulting from its untruthful bid over her true valuation.

Consider any combination of bids of the other agents in which all the bids are strictly smaller than $v_i$. 
Let $b^{\max}_{-i}$ be the highest bid among the other agents. 
We prove that any alternative bid $b_i \in (b^{\max}_{-i}, v_i)$ is a safe manipulation.

When agent $i$ bids her valuation $v_i$, she wins the good and pays $w v_i + (1-w) b^{\max}_{-i}$, yielding a (positive) utility of 
\begin{align*}
    v_i- w v_i - (1-w)b^{\max}_{-i}
    =
    (1-w) (v_i - b^{\max}_{-i}).
\end{align*}
But when $i$ bids $b_i$, as $b_i > b^{\max}_{-i}$, she still wins the good but pays $w b_i + (1-w) b^{\max}_{-i}$, which is smaller as $b_i<v_i$; therefore her utility is higher.
% resulting in higher utility of
%\begin{align*}
%    &
%    v_i-w v_i - (1-w)b^{\max}_{-i}
%    > 
%    v_i-\frac{1}{2}(v_i + b^{\max}_{-i}) && \text{(As $b_i < v_i$)}
%\end{align*} 
\end{proof}

\paragraph{Conclusion.}
By choosing a high value for the parameter $w$, the Average-Price Auction becomes similar to the first-price auction, and therefore may attain a similar revenue in practice, but with better strategic properties. 
The average-price auction is sufficiently simple to test in practice; we find it very interesting to check how it fares in comparison to the more standard auction types.


% \newpage
\section{Indivisible Goods Allocations}\label{sec:indivisible-good-aloc}
In this section, we consider several mechanisms to allocate $m$ indivisible goods $G = \{g_1, \ldots, g_m\}$ among the $n$ agents.
Here, the true preferences $T_i$ are given by $m$ real values: $v_{i,\ell}  \geq 0$ for any $g_{\ell} \in G$, representing the happiness of agent $a_i$ from receiving the good $g_{\ell}$. 
The reported preferences $P_i$ are real values $r_{i,\ell} \geq 0 $.
We assume the agents have \emph{additive} valuations over the goods.
Given a bundle $S\subseteq G$, let $v_i(S)=\sum_{g_\ell\in S}v_{i,\ell}$ be agent $a_i$'s utility upon receiving the bundle $S$.
A mechanism in this context gets $n$ (potentially untruthful) reports from all the agents and determines the \emph{allocation} -- a partition $(A_1,\ldots,A_n)$ of $G$, where $A_i$ is the bundle received by agent $a_i$.

% The key difference from auctions discussed in the previous section is that allocations do not involve money. As a result, we cannot rely on monetary incentives to encourage truthful reporting, as we did before. This section explores alternative ways to achieve this behavior.


\paragraph{Results.} We start by considering a simple mechanism, the utilitarian goods allocation -- which assigns each good to the agent who reports the highest value for it.
We prove that this mechanism is safely manipulable (RAT-degree = $0$). 
We then show that the RAT-degree can be increased to $1$ by requiring normalization --- the values reported by each agent are scaled such that the set of all items has the same value for all agents.
The RAT-degree of the famous round-robin mechanism is also at most $1$.
In contrast, we design a new mechanism that satisfies the common fairness notion called EF1 (envy-freeness up to one good), that attains a RAT-degree of $n-1$.



% ===========================================
\subsection{Utilitarian Goods Allocation}\label{sec:utilitarian-alloc}
The \emph{utilitarian} rule aims to maximize the sum of agents' utilities. 
When agents have additive utilities, this goal can be achieved by assigning each good to an agent who reports the highest value for it.
This section analyzes this mechanism.

We make the practical assumption that agents’ reports are bounded from above by some maximum possible value $r_{max} >0$.
%\erel{I think we can remove this assumption, but I kept it since it is common in practice.}

we assume that in cases where multiple agents report the same highest value for some good, the mechanism employs some tie-breaking rule to allocate the good to one of them.
However, the tie-breaking rule must operate independently for each good, meaning that the allocation for one good cannot depend on the tie-breaking outcomes of other goods.

%We further assume that there is at least one agent who has a value different than $0$ and $r_{max}$ for at least one of the goods. Otherwise, for each good, all agents' (true) values are $0$ or $r_{max}$; and in this case the mechanism is trivially truthful.
%\erel{I do not think we need to make this assumption explicitly. To prove manipulation, we can simply take one agent with non-extreme reports.}


\begin{theoremrep}
\label{prop:auction-knownagents}
The Utilitarian allocation rule is safely manipulable (RAT-degree = 0).
\end{theoremrep}

\begin{proofsketch}
    Manipulating by reporting the highest possible value, $r_{max}$, for all goods is both profitable and safe. It is profitable because if the maximum report among the other agents for a given good lies between the manipulator's true value and their alternative bid, $r_{max}$, the manipulator's utility strictly increases. It is safe because, in all other cases, the utility remains at least as high.
\end{proofsketch}

%\erel{ In the cake-cutting section, this idea is explained much more succinctly; see  \Cref{obs:utilitarian-cake-cutting}. Maybe we can do the same here? Then we also do not need the $r_{\max}$.}

\begin{proof}
    To prove the mechanism is safely manipulable, we need to show one agent that has an alternative report, such that the agent always \emph{weakly} prefers the outcome that results from reporting the alternative report over reporting her true valuations, and \emph{strictly} prefers it in at least one scenario.
    
     Let $a_1 \in N$ be an agent who has a value different than $0$ and $r_{max}$ for at least one of the goods. Let $g_1$ be such good -- that is, $0 < v_{1,1} < r_{max}$.
     We prove that reporting the highest possible value, $r_{max}$, for all goods is a safe manipulation.
     Notice that this report is indeed a manipulation as it is different than the true report in at least one place.
     \erel{We can also consider the manipulation that increases only $v_{1,l}$ to $r_{max}$. Then we have fewer cases to consider: for good $l$ the manipulation is safe and profitable, and for the other goods there is no change at all.}
    
    We need to show that for any combination of reports of the other agents, this report does not harm agent~$a_i$, and that there exists a combination where it strictly increases her utility. 

Since the utilities are additive and tie-breaking is performed separately for each good, we can analyze each good independently. 
    The following proves that for all goods, agent $a_1$ always weakly prefers to \er{manipulate}. Case 4 (marked by *) proves that for the good $g_1$, there exists a combination of reports of the others, for which agent~$a_1$ strictly prefers the outcome from \er{manipulating}.
    Which proves that it is indeed a safe profitable manipulation. 

    
    Let $g_{\ell} \in G$ be a good. 
    We consider the following cases according to the value of agent $a_1$ for the good $g_{\ell}$, $v_{1, \ell}$; and the maximum report among the other agents for this good:
    % , denoted by $r_{-i, \ell}^{max}$: 
    \begin{itemize}
        \item $v_{1,\ell} = r_{max}$: In this case, both the truthful and the untruthful reports are the same.

        \item $v_{1,\ell} =0$: Agent~$a_1$ does not care about this good, so regardless of the reports which determines whether or not agent~$a_1$ wins the good, her utility from it is $0$.

        Thus, in this case, agent $a_1$ is indifferent between telling the truth and manipulating.
        
    
        \item $0 < v_{1,\ell} < r_{max}$ and the maximum report of the others is strictly smaller than $v_{1,\ell}$: Agent~$a_1$ wins the good in both cases -- whether she reports her true value or reports $r_{max}$.
        
        Thus, in this case, agent $a_1$ is indifferent between telling the truth and manipulating.

        
        \item (*) $0 < v_{1,\ell} < r_{max}$ and the maximum report of the others is greater than $v_{i,j}$ and smaller than $r_{max}$: when agent~$a_1$ reports her true value for the good $g_{\ell}$, then she does not win it. However, by bidding $r_{max}$, she does. 

        Thus, in this case, agent~$a_1$ strictly increases her utility by lying (as her value for this good is positive).

        

        \item $0 < v_{1,\ell} < r_{max}$ and the maximum report of the others equals $r_{max}$: 
        when agent $a_1$ reports her true value for the good, she does not win it. 
        But by bidding $r_{max}$, she may win the good (depending on the tie-breaking rule).

        Thus, in this case, agent $a_1$ 
        \er{either strictly gain or does not lose from manipulating.}
%        strictly increases her (expected) utility by lying.
    \end{itemize}
\end{proof}

% This mechanism has a zero-information safe manipulation: reporting  the highest possible value for each item increases the agent's chances to get the item, and has no risk.


\subsection{Normalized Utilitarian Goods Allocation}\label{sec:normalized-utilitarian-alloc}

In the \emph{normalized} utilitarian allocation rule, the agents' reports are first normalized such that each agent's values sum to a given constant $V >0$.
Then, each good is given to the agent with the highest \emph{normalized} value.
% As before, if several agents report the same highest normalized value for a particular good, the good will be assigned to one of them according to some tie-breaking rule that operates independently for each good (see \Cref{sec:utilitarian-alloc} for more details).
%
% Notice that when there is only one good, the mechanism allows only one strategy; thus, in particular, there are no manipulations.
% \eden{ The case where there are exactly two goods is special and analyzed separately at the end of the section OR  in the appendix)}.
We focus on the case of at least three goods.

\begin{theorem}
\label{thm:normalized-utilitarian-goods}
For $m \geq 3$ goods,  the RAT-degree of the Normalized Utilitarian allocation rule is $1$.
\end{theorem}

We prove \Cref{thm:normalized-utilitarian-goods} using several lemmas that analyze different cases. 

The first two lemmas prove that the rule is not 
safely-manipulable, so its RAT-degree is at least $1$.
\Cref{claim:normalized-agent-who-likes-single-good} addresses agents who value only one good \er{positively}, while \Cref{claim:normalized-agent-who-likes-at-least-two} covers agents who value at least two goods \er{positively}.



\begin{lemmarep}
\label{claim:normalized-agent-who-likes-single-good}
An agent who values only one good positively cannot safely manipulate the Normalized Utilitarian allocation rule.
\end{lemmarep}

\begin{proofsketch}
Due to the normalization requirement, 
any manipulation by such an agent involves reporting a lower value for the only good she values positively, while reporting a higher values for some goods she values at 0. This reduces her chances of winning her desired good and raises her chances of winning goods she values at zero, ultimately decreasing her utility. Thus, the manipulation is \er{neither profitable nor} safe.
\end{proofsketch}

\begin{proof}
    % We prove that telling the truth is a strictly dominate strategy for such agents.
    %
    Let $a_1$ be an agent who values only one good and let $g_1$ be the only good she likes. That is, her true valuation is $v_{1,1} = V$ and $v_{1,\ell} = 0$ for any $\ell \neq 1$ (any good $g_{\ell}$ different than $g_1$).
    
    To prove that agent $a_1$ does \emph{not} have a safe manipulation, we need to show that for any report for her either (1) for any reports of the other agents, agent $a_1$ weakly prefers the outcome from telling the truth; or (2) there exists a reports of the other agents, for which agent $a_1$ strictly prefers the outcome from telling the truth.

    
    Let $(r_{1,1}, \ldots, r_{1,m})$ be an alternative report for the $m$ goods. We assume that the values are already normalized.
    We shall now prove that the second condition holds (lying may harm the agent). 
    
    First, as the alternative report is different, we can conclude that $r_{1,1} < V$.
    We denote the difference by $\epsilon = V - r_{1,j}$.
    Next, consider the following reports of the other agents (all agents except $a_1$): $V - \frac{1}{2}\epsilon$ for item $g_1$ and $\frac{1}{2}\epsilon$ for some other good. 
    
    When agent $a_1$ reports her true value she wins her desired good $g_1$, which gives her utility $V >0$. 
    However, when she lies, she loses good $g_1$ and her utility decreases to $0$ (winning goods different than $g_1$ does not increases her utility).
    
    That is, lying may harm the agent. 
\end{proof}

% If each agent values only one good, this implies that the mechanism is not safely manipulable.
% Next, we assume that there is at least one agent who values at least two of the goods, and prove that such agent cannot safely manipulate the mechanism as well. 

\begin{lemmarep}
\label{claim:normalized-agent-who-likes-at-least-two}
An agent who values positively at least two goods cannot safely manipulate the Normalized Utilitarian allocation rule.
\end{lemmarep}

% \newcommand{\increasedInd}{\ell^{\uparrow}}
% \newcommand{\decreasedInd}{\ell^{\downarrow}}
% EREL: suggestion for a different notation (my eyes could not notice the difference between u$\uparrow$ and $\downarrow$..)
\newcommand{\increasedInd}{\mathrm{inc}}
\newcommand{\decreasedInd}{\mathrm{dec}}

\begin{proofsketch}
    Since values are normalized, any manipulation by such an agent must involve increasing the reported value of at least one good $g_{\increasedInd}$ while decreasing the value of at least one other good $g_{\decreasedInd}$. We show that such a manipulation is not safe, by considering the case where all other agents report as follows: they assign a value of $0$ to $g_{\increasedInd}$, a value between the manipulator's true and reported value for $g_{\decreasedInd}$, and a value slightly higher than the manipulator’s report for all other goods (it is possible to construct such reports that are non-negative and normalized). 
    
    With these reports, the manipulation causes the manipulator to lose the good $g_{\decreasedInd}$ which has a positive value for her, whereas   
    she wins the good $g_{\increasedInd}$ with or without the manipulation, and does not win any other good. Hence, the manipulation strictly decreases her total value.
\end{proofsketch}


\begin{proof}
    Let $a_1$ be an agent who values at least two good, $\valT{1}{1}, \ldots, \valT{1}{m}$ her true values for the $m$ goods, and $\repT{1}{1}, \ldots, \repT{1}{m}$ a manipulation for $a_1$.
    We need to show that the manipulation is either not \emph{safe} -- there exists a combination of the other agents' reports for which agent $a_1$ strictly prefers the outcome from telling the truth; or not \emph{profitable} -- for any combination of reports of the other agents, agent $a_1$ weakly prefers the outcome from telling the truth. 
    We will show that the manipulation is not safe by providing an explicit combination of other agents' reports.

    First, notice that since the the true values and untruthful report of agent $a_1$ are different and they sum to the same constant $V$, there must be a good $g_{\increasedInd}$ whose value was increased (i.e., $ \valT{1}{\increasedInd} < \repT{1}{\increasedInd}$), and a good $g_{\decreasedInd}$ whose value was decreased (i.e., $ \valT{1}{\decreasedInd} > \repT{1}{\decreasedInd}$).
    
    Next, let $\epsilon := \min \left\{\frac{1}{m-1}\repT{1}{\increasedInd},~ \frac{1}{2}(\valT{1}{\decreasedInd} - \repT{1}{\decreasedInd})\right\}$, notice that $\epsilon > 0$.
    Also, let $c := \repT{1}{\increasedInd} - \epsilon$, notice that $c > 0$ as well \er{(here we use the condition $m\geq 3$).}

    We consider the combination of reports in which all agents except $a_1$ report the following values, denoted by $r(1), \ldots, r(m)$:
    \begin{itemize}
        \item For good $g_{\increasedInd}$ they report $r(\increasedInd) := 0$.

        \item For good $g_{\decreasedInd}$ they report 
%        $\repT{1}{\decreasedInd} + \epsilon$:
        % (a value higher by $\epsilon$ than the report of agent $a_1$).
        $r(\decreasedInd) := \repT{1}{\decreasedInd} + \epsilon$.

        \item For the rest of the goods, $g_\ell \in G \setminus \{g_{\increasedInd}, g_{\decreasedInd}\}$, they report 
%        $ \repT{1}{\ell} + \frac{1}{m-2} c$:
         % (a value higher by $\frac{1}{m-2} c$ than the report of agent $a_1$)
%        $\forall g_\ell \in G \setminus \{g_{\increasedInd}, g_{\decreasedInd}\}\colon \quad 
        $r(\ell) := \repT{1}{\ell} + \frac{1}{m-2} c$.
        
    \end{itemize}

We prove that the above values constitute a legal report --- they are  non-negative and normalized to $V$.

First, we show that the sum of values in this report is $V$:
    \begin{align*}
        \sum_{\ell =1}^m r(\ell) &= r(\increasedInd) + r(\decreasedInd) +\sum_{g_{\ell} \in G \setminus \{g_{\increasedInd}, g_{\decreasedInd}\}} r(\ell) 
        \\
        & = 0 + (\repT{1}{\decreasedInd} + \epsilon) + \sum_{g_{\ell} \in G \setminus \{g_{\increasedInd}, g_{\decreasedInd}\}} \left(\repT{1}{\ell} + \frac{1}{m-2} c\right) 
        \\
        & = (\repT{1}{\decreasedInd} + \epsilon) + \sum_{g_{\ell} \in G \setminus \{g_{\increasedInd}, g_{\decreasedInd}\}} \repT{1}{\ell} + (m-2)\frac{1}{m-2} c
        \\
        & = (\repT{1}{\decreasedInd} + \epsilon) + \sum_{g_{\ell} \in G \setminus \{g_{\increasedInd}, g_{\decreasedInd}\}} \repT{1}{\ell} + (\repT{1}{\increasedInd} - \epsilon) = \sum_{\ell =1}^m \repT{1}{\ell} = V.
    \end{align*}
    
Second, we show that all the values are non-negative:
    \begin{itemize}
        \item Good $g_{\increasedInd}$: it is clear as $r(\increasedInd) =0$.

        \item Good $g_{\decreasedInd}$: since  $r(\decreasedInd)$ is strictly higher than the (non-negative) report of agent $a_1$ by $\epsilon >0 $, it is clearly non-negative.

        \item Rest of the goods, $g_{\ell} \in G \setminus \{g_{\increasedInd}, g_{\decreasedInd}\}$: 
        since $\epsilon = \min \{\frac{1}{m-1}\repT{1}{\increasedInd}, ~\frac{1}{2}(\valT{1}{\decreasedInd} - \repT{1}{\decreasedInd})\}$, it is clear that $\epsilon \leq \frac{1}{m-1}\repT{1}{\increasedInd}$. As $m \geq 3$ and $\repT{1}{\increasedInd} >0$, we get that $c = \repT{1}{\increasedInd} - \epsilon = \frac{m-2}{m-1} \repT{1}{\increasedInd}$ is higher than $0$.
        As $r(\ell)$ is strictly higher than the (non-negative) report of agent $a_1$ by $c >0 $, it is clearly non-negative.
    \end{itemize}


Now, we prove that, given these reports for the $n-1$ unknown agents, agent $a_1$ strictly prefers the outcome from reporting truthfully to the outcome from manipulating.

    We look at the two possible outcomes for each good -- the one from telling and truth and the other from lying, and show that the outcome of telling the truth is always either the same or better, and that for at least one of the goods that agent $a_1$ wants (specifically, $g_\decreasedInd$) it is strictly better.

    % \eden{maybe: we show that the bundle agent $a_1$ gets when she is truthful is a subset (contained in the ?) of the the bundle she gets when she lies}
    \begin{itemize}
        \item For good $g_\increasedInd$ we consider two cases.
        \begin{enumerate}
            \item If $\valT{1}{\increasedInd} = 0$: when agent $a_1$ is truthful we have a tie for this good as $r(\increasedInd) = 0$.
            When agent $a_1$ manipulates, she wins the good (as $\repT{1}{\increasedInd} > \valT{1}{\increasedInd} = 0 = r(\increasedInd)$).
            However, as $\valT{1}{\increasedInd} = 0$, in both cases, her utility from this good is $0$.

            \item If $\valT{1}{\increasedInd} > 0$:
            Whether agent $a_1$ says is truthful or not, she wins the good as $\repT{1}{\increasedInd} > \valT{1}{\increasedInd} > 0 =r(\increasedInd)$.
            Thus, for this good, the agent receives the same utility (of $\valT{1}{\increasedInd}$) when telling the truth or lying.

        \end{enumerate}
        
        
        \item For good $g_\decreasedInd$: when agent $a_1$ is truthful, she wins the good since $r(\decreasedInd) < \valT{1}{\decreasedInd}$:
        \begin{align*}
            r(\decreasedInd) &= \repT{1}{\decreasedInd} + \epsilon \\
            &= \repT{1}{\decreasedInd} + \min \{\frac{1}{m-1}\repT{1}{\increasedInd},~ \frac{1}{2}(\valT{1}{\decreasedInd} - \repT{1}{\decreasedInd})\}\\
            &\leq \repT{1}{\decreasedInd} +  \frac{1}{2}(\valT{1}{\decreasedInd} - \repT{1}{\decreasedInd}) \\
            &= \frac{1}{2}(\valT{1}{\decreasedInd} + \repT{1}{\decreasedInd})< \frac{1}{2}(\valT{1}{\decreasedInd} + \valT{1}{\decreasedInd}) = \valT{1} {\decreasedInd} && \text{(as $\repT{1}{\decreasedInd} < \valT{1}{\decreasedInd}$)}
        \end{align*}
        But when agent $a_1$ manipulates, she loses the good since $r(\decreasedInd) > \repT{1}{\decreasedInd}$ (as $r(\decreasedInd) = \repT{1}{\decreasedInd} + \epsilon$ and $\epsilon > 0$).
        
        As the real value of agent $a_1$ for this good is positive, the agent strictly prefers telling the truth for this good.
%        (telling the truth is strictly better)

        \item Rest of the goods, $g_{\ell} \in G \setminus \{g_{\increasedInd}, g_{\decreasedInd}\}$:
        When agent $a_1$ is truthful, all the outcomes are possible -- the agent either wins or loses or that there is a tie.    
        
        However, as for this set of goods the reports of the other agents are $r(\ell) = \repT{1}{q}+\frac{1}{n-2}c > \repT{1}{\ell}$, when agent $a_1$ manipulates, she always loses the good.
        Thus, her utility from lying is either the same or smaller (since losing the good is the worst outcome).

%        (telling the truth is weakly better)
    \end{itemize}

    Thus, the manipulation may harm the agent.
\end{proof}

% \fi


\iffalse

% ===============================================
% \begin{claim}\label{normalized-more-than-2-agents-at least-2-goods}
%     When there are more than two agents and at least two goods, the normalized utilitarian goods allocation is \emph{not} safely-manipulable.
% \end{claim}
\begin{claim}\label{normalized-more-than-2-agents-at least-2-goods}
   For $n \geq 3$, the normalized utilitarian goods allocation is \emph{not} safely-manipulable.
\end{claim}

\begin{proof}
    Let $a_1$ be an agent who values at least two of the
    goods, and let $\valT{1}{1}, \ldots, \valT{1}{m}$ be her values for the $m$ goods. Assume that the values are normalized.
    
     We need to show that for any alternative preference for agent $a_1$ either (1) for any reports of the other agents, the outcome from telling the truth is always weakly preferred; or (2) there exists a reports of the other agents, for which the outcome from telling the truth is strictly preferred.

     Let $\repT{1}{1}, \ldots, \repT{1}{m}$ be alternative reports for the $m$ goods, also normalized.

     First, notice that since the the true values and untruthful reports for agent $a_1$ are different and they sum to the same constant $V$, there must be a good $\increasedInd$ whose value was increased (i.e., $ \valT{1}{\increasedInd} < \repT{1}{\increasedInd}$), and a good $\decreasedInd$ whose value was decreased (i.e., $ \valT{1}{\decreasedInd} > \repT{1}{\decreasedInd}$).


    % \eden{this proof requires $3$ agents. need to think what happens for 2 agents}
     We shall now prove that the second condition holds (lying may harm the agent). 
     Consider the case where all the other agents (all except $a_1$) report one of the following sets of values: 
     \begin{enumerate}
         \item For good $\decreasedInd$: $\valT{1}{\decreasedInd} - \frac{1}{2}\epsilon$, where  $\epsilon := \frac{1}{2}(\valT{1}{\decreasedInd} - \repT{1}{\decreasedInd})$. Notice that $\epsilon > 0$ as $ \valT{1}{\decreasedInd} > \repT{1}{\decreasedInd}$.

         For good $\increasedInd$: $\left(V - (\valT{1}{\decreasedInd} - \frac{1}{2}\epsilon)\right)$.

         Rest of the goods, $q \in Q \setminus \{\increasedInd, \decreasedInd\}$: $0$.

         \item For good $\increasedInd$: $V$.

         Rest of the goods, $q \in Q \setminus \{\increasedInd\}$: $0$.
     \end{enumerate}
     We consider only the profiles in which at least one agent reports the first set and at least one agent reports the second set.

     It is easy to see that both profiles are legal reports (all values are between $0$ and $V$ and they sum to $V$).

    Claim: agent $a_1$ strictly prefers the outcome from telling the truth.

    We look at the two outcomes per good and show that for each good, the outcome of telling the truth is either the same or better, and that for one of the goods that agent $a_1$ wants ($\decreasedInd$) it is strictly better.


    \begin{itemize}
        \item Good $\increasedInd$: when agent $a_1$ is truthful, she does not win the good --- since she values at least two goods, $\valT{1}{\increasedInd} < V$, which means that one of the agents who report $V$ wins it.
        
        However, when she lies, there are two possible cases.
        \begin{enumerate}
            \item If $\repT{1}{\increasedInd} < V$:
            as before, when agent $a_1$ manipulates, she does not win the good.
            
            (the same)
            
            \item If $\repT{1}{\increasedInd} = V$: when agent $a_1$ manipulates, we have tie for this good. 

            \begin{enumerate}
                \item If $\valT{1}{\increasedInd} = 0$, regardless of whether she receives the good or not, her utility from this good is $0$.
                
                (the same). 

                \item If $\valT{1}{\increasedInd} > 0$:
                when agent $a_1$ is truthful or lies, she wins the good (as $\repT{1}{\increasedInd} > \valT{1}{\increasedInd} > 0 =\repT{2}{\increasedInd}$.
                Thus, for this good, the agent receives the same utility (of $\valT{1}{\increasedInd}$) when telling the truth or lying (the same).
            \end{enumerate}
        \end{enumerate}
        
        
        \item Good $\decreasedInd$: when agent $a_1$ is truthful, she wins the good since $\repT{2}{\decreasedInd} < \valT{1}{\decreasedInd}$:
        \begin{align*}
            \repT{2}{\decreasedInd} &= \repT{1}{\decreasedInd} + \epsilon \\
            &= \repT{1}{\decreasedInd} + \min \{\frac{1}{n-1}\repT{1}{\increasedInd},~ \frac{1}{2}(\valT{1}{\decreasedInd} - \repT{1}{\decreasedInd})\}\\
            &\leq \repT{1}{\decreasedInd} +  \frac{1}{2}(\valT{1}{\decreasedInd} - \repT{1}{\decreasedInd}) \\
            &= \frac{1}{2}(\valT{1}{\decreasedInd} + \repT{1}{\decreasedInd})< \frac{1}{2}(\valT{1}{\decreasedInd} + \valT{1}{\decreasedInd}) = \valT{1} {\decreasedInd} && \text{(as $\repT{1}{\decreasedInd} < \valT{1}{\decreasedInd}$)}
        \end{align*}
        However, when agent $a_1$ manipulates, she loses the good since $\repT{2}{\decreasedInd} > \repT{1}{\decreasedInd}$ (as $\repT{2}{\decreasedInd} = \repT{1}{\decreasedInd} + \epsilon$ and $\epsilon > 0$).
        
        As the real value of agent $a_1$ for this good is positive, the agent strictly prefers telling the truth for this good.

        \item Rest of the goods, $q \in Q \setminus \{\increasedInd, \decreasedInd\}$: 
        When agent $a_1$ is truthful, all the outcomes are possible -- the agent either wins or loses or that there is a tie-breaking.    
        
        However, as for these goods $\repT{2}{q} = \repT{1}{q}+\frac{1}{n-2}c > \repT{1}{q}$, when agent $a_1$ manipulates, she always loses the good.
        Thus, her utility from lying is either the same or smaller (since losing the good is the worst outcome).
    \end{itemize}

    Thus, lying may harm the agent.
\end{proof}




\begin{claim}
    The RAT-degree of the normalized utilitarian goods allocation is $1$.
\end{claim}

\begin{proof}
    To prove that the RAT-degree is $1$, we need to show that the mechanism is:
    (a) \emph{not} safely-manipulable.
    (b) $1$-known-agents safely-manipulable.\\

    (a)
\end{proof}

\fi

The last lemma shows that the RAT-degree is at most $1$, thus completing the proof of the theorem.
\begin{lemmarep}\label{normalized-1-known}
    With $m \geq 3$ goods, Normalized Utilitarian is $1$-known-agent safely-manipulable.
\end{lemmarep}

\begin{proofsketch}
Consider a scenario where there is a known agent who reports $0$ for some good $g$ that the manipulator wants and slightly more than the manipulator’s true values for all other goods. In this case, by telling the truth, the manipulator has no chance to win any good except $g$. Therefore, reporting a 
value of $V$ for $g$ and a value of $0$ for all other goods is a safe manipulation.

The same manipulation is also profitable, since it is possible that the reports of all $n-2$ unknown agents for $g$ are larger than the manipulator’s true value and smaller than $V$. In such a case, the manipulation causes the manipulator to win $g$, which strictly increases her utility.
\end{proofsketch}


\begin{proof}
    Let $a_1$ be an agent and let $\valT{1}{1}, \ldots, \valT{1}{m}$ be her values for the $m$ goods.
    We need to show (1) an alternative report for agent~$a_1$, (2) another agent~$a_2$, and (3) a report agent ~$a_2$; such that 
    for any combination of reports of the remaining $n-2$ (unknown) agents, agent $a_1$ weakly prefers the outcome from lying, and that there exists a combination for which agent $a_1$ strictly prefers the outcome from lying.

    Let $g_{\ell^+}$ be a good that agent $a_1$ values (i.e., $\valT{1}{\ell^+} >0$).
    
    Let $a_2$ be an agent different than $a_1$.
    We consider the following report for agent~$a_2$: first, $\repT{2}{\ell^+} :=0$, and $\repT{2}{\ell} := \repT{1}{\ell} + \epsilon$ for any good $g_{\ell}$ different than $g_{\ell^+}$, where $\epsilon := \frac{1}{m-1} \valT{1}{\ell^+} $.
    Notice that $\epsilon >0$.

    We shall now prove that reporting $V$ for the good $g_{\ell^+}$ (and $0$ for the rest of the goods) is a safe manipulation for $a_1$ given that $a_2$ reports the described above.

    When agent~$a_1$ reports her true values, then she does not win the goods different than $g_{\ell^+}$ -- this is true regardless of the reports of the remaining $n-2$ agents, as agent~$a_2$ reports a higher value $\repT{1}{\ell} < \repT{1}{\ell} + \epsilon = \repT{2}{\ell}$.
    For the good $g_{\ell^+}$,  we only know that $\repT{2}{\ell^+} = 0 > \valT{1}{\ell^+} > \repT{1}{\ell^+} = V$, meaning that it depends on the reports of the $(n-2)$ remaining agents. 
    We consider the following cases, according to the maximum report for $g_{\ell^+}$ among the remaining agents:
    \begin{itemize}
        \item If the maximum is smaller than $\valT{1}{\ell^+}$: agent $a_1$ wins the good $\valT{1}{\ell^+}$ in both cases (when telling the truth or lies). 

        (the same)

        \item If the maximum is greater than $\valT{1}{\ell^+}$ but smaller than $\repT{1}{\ell^+} = V$: when agent $a_1$ tells the truth, she does not win the good. 
        However, when she lies, she does.
        
        (lying is strictly better). 

        \item If the maximum equals $\repT{1}{\ell^+} = V$: when agent $a_1$ tells the truth, she does not win the good. 
        However, when she lies, we have tie for this good.
        Although agent $a_1$ is risk-averse, having a chance to win the good is strictly better than no chance at all.
        
        (lying is strictly better). 
    \end{itemize}
    
    
\end{proof}


% ========================================
\subsection{Round-Robin Item Allocation}
In round-robin item allocation, the agents are arranged according to some predetermined order $\pi$. 
There are $m$ rounds, corresponding to the number of items. 
In each round, the agent whose turn it is (based on $\pi$) takes their most preferred item from the set of items that remain unallocated at that point.
When there are multiple items that are equally most preferred, the agent \er{breaks the tie according to a fixed item-priority ordering.}
Note that normalization is not important for Round-Robin, as it is not affected by scaling the valuations.

If there are at most $n$ items, then the rule is clearly truthful.
Thus, we assume that there are $m\geq n+1$ items.
Even the slight increment to $m=n+1$ makes a significant difference:

\begin{lemma}\label{claim:RR-RAT1-tie}
    With $m\geq n+1$ items, round-robin  is $1$-known-agent safely-manipulable.
\end{lemma}

\begin{proof}
    Let $\pi$ be the order according to agents' indices: $a_1,a_2,\ldots,a_n$.
    Let agent $a_1$'s valuation be such that $v_{1,1}=v_{1,2}=1$ and $v_{1,3}=\cdots = v_{1,m}=0$.
    Suppose agent $a_1$ knows that agent $a_2$ will report the valuation with $v_{2,1}=0$, $v_{2,2}=1$, and $v_{2,3}=\cdots=v_{2,m}=0.5$.
    We show that agent $a_1$ has a safe and profitable manipulation by reporting $v_{1,1}'=0.5$, $v_{1,2}'=1$, and $v_{1,3}'=\cdots=v_{1,m}'=0$.

    Firstly, we note that agent $a_1$'s utility is always $1$ when reporting her valuation truthfully, regardless of the valuations of agents $a_3,\ldots,a_n$.
    This is because agent $a_1$ will receive item $g_1$ (by the item-index tie-breaking rule) and agent $a_2$ will receive item $g_2$ in the first two rounds.
    The allocation of the remaining items does not affect agent $a_1$'s utility.

    Secondly, after misreporting, agent $a_1$ will receive item $g_2$ in the first round, which already secures agent $a_1$ a utility of at least $1$. Therefore, agent $a_1$'s manipulation is safe.

    Lastly, if the remaining $n-2$ agents report the same valuations as agent $a_2$ does, it is easy to verify that agent $a_1$ will receive item $g_1$ in the $(n+1)$-th round.
    In this case, agent $a_1$'s utility is $2$.
    Thus, the manipulation is profitable.
\end{proof}

We show a partial converse.
\begin{lemma}
    With $m \leq 2n$ items, round-robin is not safely-manipulable.
\end{lemma}


\begin{proof}
If $m\leq n$ then Round-Robin is clearly truthful, so we assume $m\geq n+1$.

We first consider manipulations by $a_1$ (the agent who picks an item first).
We order the items by descending order of $a_1$'s true values ($v_{1,1} \geq v_{1,2} \geq \cdots \geq v_{1,m}$), and subject to that, by the fixed item-priority ordering.

If $v_{1,1} = \cdots = v_{1,n+1}$, then $a_1$ always gets the maximum possible value (two highest-valued items) by being truthful, so no manipulation is profitable. Therefore we assume that $v_{1,1} > v_{1,n+1}$.

When $a_1$ is truthful, he gets $g_1$ first. Then the other $n-1$ agents pick items, and then $a_1$ gets the best of the remaining $m-n$ items.
If the manipulation is such that 
%$v'_{1,1}$ is still the highest value, then 
$a_1$ still gets $g_1$ first, then the first round proceeds in exactly the same way with or without the manipulation, so in the second round, $a_1$ still has the same set of $m-n$ items available. Therefore, the manipulation is not profitable.
Hence, we assume that $a_1$ manipulates such that
he picks some other item, say $g_z\neq g_1$, at the first round. 

If $z\geq n+1$, then the manipulation is not profitable, as when $a_1$ is truthful his value is at least $g_{1,1}+g_{1,n+1}$, and when he manipulates his value cannot be higher. Hence, we assume that $z\leq n$. We prove that this manipulation is not safe.

Suppose the $n-1$ unknown agents all have same valuation: they rank the goods the same as $a_1$, except that $g_z$ is moved to be lower than $g_{n}$.

When $a_1$ is truthful, he gets $g_1$ first. 
When the other $n-1$ agents take their items, there are at least $n-1$ items better for them than $g_z$. At the start of the second round $g_z$ is available, so $a_1$ gets an item at least as good, and his value is at least $v_{1,1} + v_{1,z}$.

When $a_1$ manipulates, he gets $g_z$ first. 
Then the other $n-1$ agents take all items $\{g_1,\ldots,g_n\}$ except $g_z$. Then at the start of the second round the best available item for $a_1$ is $g_{n+1}$, so his total value is at most $v_{1,z}+v_{1,n+1}$. This is strictly smaller than $v_{1,1} + v_{1,z}$ by the assumption $v_{1,1}>v_{1,n+1}$, so the manipulation is harmful for $a_1$.

We now consider any other agent $a_j$. Any manipulation by $a_j$ has no effect on the $j-1$ items picked first. Therefore, from the point of view of $a_j$, the situation is as if he picks first from a set of $m-j+1$ items. Therefore, a similar proof applies, and $a_j$ has no safe manipulation either.
\end{proof}

As a corollary, we get:
\begin{theorem}
The RAT-degree of round-robin is $n$ when $m\leq n$, exactly $1$ when $n+1\leq m\leq 2n$, and at most $1$ when $m> 2n$.
\end{theorem}
    
\iffalse
We show that even the slight increase to $m=n+1$ makes a significant difference: in this case, the degree drops to at most $n-2$ -- meaning that there exists an agent who can safely manipulate the outcome, even in the presence of an unknown agent.

\eden{I think the degree depends on the agent's preferences -- on how many of its most preferred items she is willing to risk in order to get her "best-in-the-middle" - something like that.. }

% For agent $a_i$, we denote its $q$-th most-preferred item by $g^i_{[q]}$. We assume that there exists an agents, $a_1$, for whom there exists a bundle of items that is strictly preferred over the following bundle:
% \begin{align*}
%     \{g^1_{[1]}, g^1_{[n+1]}, \ldots, g^1_{[\lfloor \frac{m}{n} \rfloor n+1]}\}
% \end{align*}
%  and completely different (no intersection between thw two sets)
    
%     Formally, let $q^1_{best} := \argmax_{q\in [m]} \valT{1}{q}$ and $q^1_{worst} := \argmin_{q\in [m]} \valT{1}{q}$. Then, there exists $q^1_{mid1} \neq q^1_{mid2} \in ([m] \setminus \{q^1_{best}, q^1_{worst}\})$ such that:
%     \begin{align*}
%         \valT{1}{q^1_{best}} + \valT{1}{q^1_{worst}} < \valT{1}{q^1_{mid1}} + \valT{1}{q^1_{mid2}}
%     \end{align*}


% and prove that the RAT-degree is \eden{???}.

% \begin{claim}
%     When there are more items than agents, $m\geq n+1$; the RAT-degree of the round-robin items allocation is at most $(n-2)$.

%     $n - \lceil \frac{m}{n} \rceil$
% \end{claim}

\fi

The proof of \Cref{claim:RR-RAT1-tie} uses weak preferences (preferences with ties). In particular, if $a_1$'s valuations for the top two items are different, then picking $g_2$ first is risky.
% , as it might result in losing $g_1$.
With strict preferences, we could only prove a much weaker upper bound of $n-2$. This raises the following question.
\begin{open}
    What is the RAT-degree of round-robin when agents  report strict preferences?
\end{open}


\iffalse
\begin{claim}
    For $m = n+1$, the RAT-degree of the round-robin items allocation is at most $(n-2)$.
\end{claim}

\begin{proof}
    To prove that the RAT-degree is at most $(n-2)$, we need to show that the mechanism is $(n-2)$-known-agents manipulable.
    To do so, we need to show that there exists an agent~$a_1$, and a set of $n-2$ known-agents such that agent~$a_1$ can safely manipulate. 

    Let $a_1$ be the agent who choose first. Notice that $a_1$ is the only agent who gets $2$ items in this case. 
    Consider the case where her true valuations are $m$ for her most-preferred until $1$ for her leased-preferred. 
    Assume, without loss of generality, that her most-preferred is $g_m$ and her least-preferred is $g_1$.

    
    
    \eden{maybe this should be explained before}
    We assume that there exists an agents, $a_1$, for whom there are two items that are neither her favorite nor her least-preferred, such that she prefers the bundle of these two items over the bundle of her favorite and least-preferred items.
    Formally, let $q^1_{best} := \argmax_{q\in [m]} \valT{1}{q}$ and $q^1_{worst} := \argmin_{q\in [m]} \valT{1}{q}$. Then, there exists $q^1_{mid1} \neq q^1_{mid2} \in ([m] \setminus \{q^1_{best}, q^1_{worst}\})$ such that:
    \begin{align*}
        \valT{1}{q^1_{best}} + \valT{1}{q^1_{worst}} < \valT{1}{q^1_{mid1}} + \valT{1}{q^1_{mid2}}
    \end{align*}
    Notice that this assumption holds in many cases -- for example, with 
    % $4$ agents and 
    $5$ items, where the agent who picks first values the items by $1, \ldots, 5$. Then, the bundle consisting of her favorite and least-preferred items is worth $5+1 =6$, while the bundle of her second- and third-best items is worth $4+3=7$ (assuming additive utilities).



    
    
    
    % , for additive ranking utilities (i.e., the utility of the $q$-th most-preferred item is $m-q+1$ and the utility from a bundles is the sum) for $n \geq 5$ -- as the utility from the bundle of the first-and-fifth most-preferred items is $(2m-4)$ while the utility from the bundle of the second-and-third most-preferred items is $(2m-3)$

    

    Let $a_1$ be an agent who choose first, and assume her true valuations are $\valT{1}{1} > \valT{1}{2} > \ldots > \valT{1}{m}$.
    We need to show a strategy profile for the other agents, for which agent~$a_1$ can safely manipulate. 

    Consider the case where the values of all other agents are: $\valT{1}{1} > \valT{1}{m} > \valT{1}{m-1} > \ldots > \valT{1}{2}$.

    Telling the truth for agent~$a_1$ means that she takes her favorite item~$g_1$ in the first round. In this case, she ends the second round with the items $g_1$ and $g_2$ as all other agents prefer all the remaining other items over $g_2$.

    Lying for agent~$a_1$ means to take some item different than $g_1$ -- specifically, one that she likes less. We prove that taking the item~$g_2$ first is a safe manipulation for agent $a_1$.
    
    When agent~$a_1$ takes the items $g_2$ in the first round, she ends the second round with the items $g_2$ and $g_3$ as the other agents prefer all the remaining other items over $g_3$. 

    As agent~$a_1$ strictly prefers the item~$g_1$ over $g_3$, her utility from telling the truth is strictly higher. 
    
\end{proof}


\eden{todo}

\eden{need to explain why it is sufficient to consider the first as the potential manipulator and the last as the "bad" agent}
\begin{proof}
    To prove that the RAT-degree is $(n-1)$, we need to show that the mechanism is: (a) not $(n-2)$-known-agents manipulable; and 
    (b) $(n-1)$-known-agents manipulable.

    (a) Let $a_i$ be an agent. Assume without loss of generality that her true valuations are $\valT{i}{1} > \valT{i}{2} > \ldots > \valT{i}{m}$.
    Telling the truth in this case, means to choose the most preferred good that was not already taken be the other agents in each round.

    We consider another agent~$a_j \neq a_i$ with the same valuations, i.e., $\valT{j}{1} > \valT{j}{2} > \ldots > \valT{j}{m}$.
    We shall now prove that, regardless of the valuations of the remaining $n-2$ agents, agent $a_i$ loses when she lies.  

    % Let $\pi(i)$ be agent~$a_i$'s location in the order, l
    Let $G_b$ be the set of goods that were already taken before agent~$a_i$'s first turn
    
    
    
    
    
    
    good $g_1$ in the first round. 
    There are only $n$ possible manipulations -- namely, taking one of the goods $g_2,\ldots,g_n$ first.

     We shall now see that for any strategy profile of $(n-2)$-known-agents $a_2,\ldots, a_{n-1}$, there exists a strategy for the remaining agent, $a_n$, for which lying harm agent~$a_1$.

     Let $Q^T$ be the set of goods that the $(n-2)$-known-agents would have take if agent~$a_1$ tells the truth. Denote agent~$a_1$'s favorite good among the remaining goods -- from $G \setminus (\{q_1\} \cup Q)$ -- by $g^T$.

     Let $Q^L$ be the set of goods that the $(n-2)$-known-agents would have take for some lie of agent~$a_1$. Denote agent~$a_1$'s favorite good among the remaining goods -- from $G \setminus (\{q_1\} \cup Q)$ -- by $g^T$.

    Let $q_1 \neq g_1$ be the good that agent~$a_1$ took, and let $Q$ be the set of goods that the $(n-2)$-known-agents took.
    Denote agent~$a_1$'s favorite good among the remaining goods (from $G \setminus (\{q_1\} \cup Q)$ by $\ell^+$; and her leased-preferred among this set by $q^-$.
    Notice that $\ell^+ \neq q^-$.

    Consider the case where agent~$a_n$'s favorite good is $g_1$ and his second-best, $q^-$.

    When agent~$a_1$ tells the truth, she wins $g_1$ and $\ell^+$, since agent~$a_n$ will choose $q^-$ in this scenario. 
    However, when she lies, she wins $q_1$ and $\ell^+$, since agent~$a_n$ will choose $g_1$ in this scenario. 
 
    

    (b) Let $a_1$ be an agent who choose first, and assume her true valuations are $\valT{1}{1} > \valT{1}{2} > \ldots > \valT{1}{m}$.
    We need to show a strategy profile for the other agents, for which agent~$a_1$ can safely manipulate. 

    Consider the case where the values of all other agents are: $\valT{1}{1} > \valT{1}{m} > \valT{1}{m-1} > \ldots > \valT{1}{2}$.

    Telling the truth for agent~$a_1$ means that she takes her favorite good~$g_1$ in the first round. In this case, she ends the second round with the goods $g_1$ and $g_2$ as all other agents prefer all the remaining other goods over $g_2$.

    Lying for agent~$a_1$ means to take some good different than $g_1$ -- specifically, one that she likes less. We prove that taking the good~$g_2$ first is a safe manipulation for agent $a_1$.
    
    When agent~$a_1$ takes the goods $g_2$ in the first round, she ends the second round with the goods $g_2$ and $g_3$ as the other agents prefer all the remaining other goods over $g_3$. 

    As agent~$a_1$ strictly prefers the good~$g_1$ over $g_3$, her utility from telling the truth is strictly higher. 
    

    
    \end{proof}

% \paragraph{Intuition}
% Suppose there are $m\geq n+1$ goods, agent~$a_1$ plays first, and her true valuations are:
% \begin{align*}
%     v_{1,q} = 
%     \begin{cases}
%     2^{-q}  & q\in \{1,\ldots,n+1\}
%     \\
%     0 & q \in \{n+2,\ldots,m\}
%     \end{cases}
% \end{align*}
% It is possible that after agent $a_1$ truthfully picks good $g_1$ at the first step, her second good will be $n+1$.
% There are $n-1$ potential manipulations, namely taking good $2,\ldots,n$ first.
% However, a manipulation is profitable only if agent $a_1$ also gets the good $g_1$.

\begin{proposition}
With $n+1$ items,
Round-Robin item allocation is 

(a) not $(n-2)$-known-agents manipulable;

(b) $(n-1)$-known-agents manipulable.
\end{proposition}
\begin{proof}
(a) Suppose $i$ knows the preferences of only $n-2$ agents.
If one of them prefers item $1$\eden{ need to define it more accurately}, then $i$ has no profitable manipulation.
If none of them prefers item $1$, then $i$ may have a profitable manipulation, but this depends on the preference of the remaining (unknown) agent. Therefore, there is no safe manipulation.

(b) Knowing the preferences of all $n-1$ agents of course allows $i$ to manipulate optimally.
\end{proof}

(a) Let $a_1$ be an agent with the following true valuations:
    \begin{align*}
        v_{1,q} = 
        \begin{cases}
        2^{-q}  & q\in \{1,\ldots,n+1\}
        \\
        0 & q \in \{n+2,\ldots,m\}
        \end{cases}
    \end{align*}
    That is, she cares only about the goods $g_1,\ldots, g_{n+1}$ and $\valT{1}{1} > \valT{1}{2} > \ldots > \valT{1}{(n+1)}$.

    Telling the truth in this case, means to choose good $g_1$ first. 
    There are only $n$ possible manipulations -- namely, taking one of the goods $g_2,\ldots,g_n$ first.

\fi



   
\subsection{An EF1 Mechanism with RAT-degree $n-1$}
\label{sect:indivisible-EF1-n-1}
% \biaoshuai{new}

In this section, we focus on mechanisms that always output fair allocations (with respect to the reported valuation profile).
We consider the widely used fairness criterion \emph{envy-freeness up to one item} (EF1), which intuitively means that no agent envies another agent if one item is (hypothetically) removed from that agent's bundle.

\begin{definition}
Given a valuation profile $(v_1,\ldots,v_n)$, an allocation $(A_1,\ldots,A_n)$ is \emph{envy-free up to one item} (EF1) if for every pair of $i,j\in [n]$ there exists a good $g$ such that
$v_i(A_i)\geq v_i(A_j\setminus\{g\})$.
\end{definition}
% In words, an allocation is EF1 if any agent $a_i$ does not envy any agent $a_j$ if one good is  removed from agent $a_j$'s bundle.

It is well-known and easy to see that the round-robin mechanism always outputs an EF1 allocation.
% for any pair of $i,j\in[n]$, if agent $a_i$ comes before agent $a_j$ under $\pi$, then agent $a_i$ does not envy agent $a_j$ as agent $a_i$ weakly prefers the good she receives than the good received by agent $a_j$ in every $n$-rounds iteration; if agent $a_i$ comes after agent $a_j$, then agent $a_i$ does not envy agent $a_j$ after removing the first good received by $a_j$, as agent $a_i$ weakly prefers the good she receives in every $n$-rounds iteration than the good received by agent $a_j$ in the next $n$-rounds iteration.
However, as we have seen in the previous section, the round-robin mechanism has a very low RAT-degree.
In this section, we propose a new EF1 mechanism that has RAT-degree $n-1$.

\subsubsection{Description of Mechanism}
\label{sect:EF1Mechanism_description}
The mechanism has two components: an agent selection rule $\Gamma$ and an allocation rule $\Psi$.
The agent selection rule $\Gamma$ takes the valuation profile $(v_1,\ldots,v_n)$ as an input and outputs the indices $i^+,i^-$ of two agents, where agent $a_{i^+}$ is called \emph{the mechanism-favored agent} and agent $a_{i^-}$ is called \emph{the mechanism-unfavored agent} (the reason of both names will be clear soon).
The allocation rule $\Psi$ takes the valuation profile $(v_1,\ldots,v_n)$ and the indices of the two agents $i^+,i^-$ output by $\Gamma$ as inputs and then outputs an EF1 allocation $(A_1,\ldots,A_n)$.
We will complete the description of our mechanism by defining $\Gamma$ and $\Psi$.

% \erel{Is it possible to simplify the description by letting $\Gamma$ return a permutaiton of the agents (as in the cake-cutting section), and running round-robin according to that permutation?}
% \biaoshuai{I have thought about this, and I don't think this is true. The same counterexample in the RR RAT degree 1 proof applies here. In that example, no matter the orders, agent 1's manipulation is always safe.}

We first define $\Psi$.
Given the input valuation profile $(v_1,\ldots,v_n)$, let $\efallocations_{i^-}$ be the set of all EF1 allocations $(A_1,\ldots,A_n)$ in which agent $a_{i^-}$ is not envied by anyone else, i.e., for any agent $a_i$, we have $v_i(A_i)\geq v_i(A_{i^-})$.
Notice that $\efallocations_{i^-}$ is nonempty: 
the allocation output by the round-robin mechanism with $i^-$ being the last agent under $\pi$ satisfies both (1) and (2) above.

The rule $\Psi$ then outputs an allocation $(A_1,\ldots,A_n)$ in $\efallocations_{i^-}$ that maximizes $v_{i^+}(A_{i^+})$.
When there are multiple maximizers, the rule breaks the tie in an arbitrary consistent way.
This finishes the description of $\Psi$.

To describe $\Gamma$, we first state a key property called \emph{volatility} that we want from $\Gamma$, and then show that a volatile rule can be constructed.
Informally, volatility says the following.
If an arbitrary agent $a_i$ changes the reported valuation profile from $v_i$ to $v_i'$, we can construct a valuation profile $v_j$ for another agent $a_j$ such that $v_j$ has a positive value on only one pre-specified good and the two agents output by $\Gamma$ switch from some pre-specified $i^+,i^-$ to some pre-specified $i^{+'},i^{-'}$.

\begin{definition}
A selection rule $\Gamma$ is called \emph{volatile}
if 
for any six indices of agents $i,j,i^+,i^-,i^{+'},i^{-'}$ with $i\neq j$, $i^+\neq i^-$, and $i^{+'}\neq i^{-'}$, any good $g_{\ell^\ast}\in G$, any set of $n-2$ valuation profiles $\{v_k\}_{k\notin \{i,j\}}$, and any two reported valuation profiles $v_i,v_i'$ of agent $a_i$ with $v_i\neq v_i'$ (i.e., $v_{i,\ell}\neq v_{i,\ell}'$ for at least one good $g_{\ell}$), there exists a valuation function $v_j$ of agent $a_j$ such that
\begin{itemize}
    \item $v_{j,\ell^\ast}>0$, and $v_{j,\ell}=0$ for any $\ell\neq\ell^\ast$, and
    \item $\Gamma$ outputs $i^+$ and $i^-$ for the valuation profile $\{v_k\}_{k\notin\{i,j\}}\cup\{v_i\}\cup\{v_j\}$;
    \item $\Gamma$ outputs $i^{+'}$ and $i^{-'}$ for the valuation profile $\{v_k\}_{k\notin\{i,j\}}\cup\{v_i'\}\cup\{v_j\}$.
\end{itemize}
\end{definition}
In other words, a manipulation of agent $i$ from $v_i$ to $v_i'$ can affect the output of $\Gamma$ in any possible way (from any pair $i^+,i^-$ to any other pair $i^{+'},i^{-'}$), depending on the report of agent $j$.

We will use an arbitrary volatile rule $\Gamma$ for our mechanism.
We conclude the description of $\Gamma$ by proving (in the appendix) that such a rule exists.

% \biaoshuai{Probably put the proof of the following proposition in the appendix?}
\begin{proposition}
\label{prop:volatile}
    There exists a volatile agent selection rule $\Gamma$.
\end{proposition}
\begin{proof}
    The rule $\Gamma$  first finds the maximum value among all agents and all goods: $\displaystyle v^\ast := \max_{i\in[n],\ell\in[m]}v_{i,\ell}$.
    It then views the value $v^\ast$ as a binary string that encodes the following information:
    \begin{itemize}
        \item the index $i$ of an agent $a_i$;
        \item a non-negative integer $t$,
        \item two non-negative integers $a,b$, between $0$ and ${n \choose 2}$.
        % \item four non-negative integers $a^+,b^+,a^-,b^-$.
    \end{itemize}
    We append $0$'s as most significant bits to $v^\ast$ if the length of the binary string is not long enough to support the format of the encoding.
   If the encoding of $v^\ast$ is longer than the length enough for encoding the above-mentioned information, we take only the least significant bits in the amount required for the encoding.


   
    The mechanism-favored agent $a_{i^+}$ and the mechanism-unfavored agent $a_{i^-}$ are then decided in the following way.
    Let $s\in\{0,1\}$ be the bit at the $t$-th position of the binary encoding of the value $v_i(g_\ell)$.

    Let $p := (a\cdot s + b) \bmod {n \choose 2}$.
    Each value of $p$ corresponds to a pair of different agents $(i^+, i^-)$.
    
    % Let $i^+ := (a^+\cdot s+b^+ \bmod n) + 1$. 
    % if $a^+\cdot s+b^+\in[n]$, and let $i^+=1$ otherwise.
    % Let $i^- := (a^-\cdot s+b^- \bmod n) + 1$.
    % If $i^+=i^-$, we make them different by setting $i^+:=1, i^-:=n$.
    % if $a^-\cdot s+b^-\in [n]$, and let $i^-=n$ otherwise.
    % \erel{Can we take $i^+ = (a^+\cdot s + b^+) \bmod n$?}

    To see that $\Gamma$ is volatile, suppose $v_i$ and $v_i'$ are different in the $t$-th bits of their binary encoding.
    We construct a value $v^*$ that encodes the integers
    $i,t,a,b$ where
    \begin{enumerate}
        \item the $t$-th bit of $v_{i}$ is $s$ and the $t$-th bit of $v_{i}'$ is $s'$ for $s\neq s'$;
        \item The pair $(i^+,i^-)$ corresponds to the integer $(a\cdot s + b) \bmod {n \choose 2}$.
        \item The pair $(i^{+'},i^{-'})$ corresponds to the integer $(a\cdot s' + b) \bmod {n \choose 2}$.
        % $i^+=a^+\cdot s+b^+$ and $i^{+'}=a^{+}\cdot s'+b^{+}$;
        % \item $i^-=a^-\cdot s+b^-$ and $i^{-'}=a^{-}\cdot s'+b^{-}$.
    \end{enumerate}
    (1) can always be achieved by some encoding rule.
    To see (2) and (3) can always be achieved, assume $s=1$ and $s'=0$ without loss of generality.
We can then take $b := $ the integer corresponding to the pair $(i^{+'},i^{-'})$, and $a := - b + $ the integer corresponding to the pair $(i^{+},i^{-})$, modulo ${n\choose 2}$.
    
    % We can then set $b^+=i^{+'}$ and $a^+=i^+-b^+$.
    % We can then set $b^+=i^{+'}$ and $a^+=i^+-b^+$.
    % Similar construction can show that (3) is always achievable.
    
    We then construct a valuation $v_j$ such that $v_{j,\ell^\ast}$ is the largest and is equal to $v^*$.
    In case $v^*$ is not large enough, we increase it as needed by adding most significant digits.
\end{proof}

\begin{remark}
The proof of \Cref{prop:volatile} requires that the mechanism does \emph{not} normalize the valuations, nor place any upper bound on the reported values. Suppose there were an upper bound $V$ on the value. $V$ encodes some agent $i$, bit number $t$, and integers $a,b$. It is possible that these numbers give the highest priority to agent $i$. In that case, agent $i$ could manipulate by reporting the value $V$.
\end{remark}

\subsubsection{Proving RAT-degree of $n-1$}
Before we proceed to the proof, we first define some additional notions.
We say that $(A_1,\ldots,A_n)$ is a \emph{partial allocation} if $A_i\cap A_j=\emptyset$ for any pair of $i,j\in[n]$ and $\bigcup_{i=1}^nA_i\subseteq G$.
The definition of EF1 can be straightforwardly extended to partial allocations.
Given a possibly partial allocation $(A_1,\ldots,A_n)$, we say that agent $a_i$ \emph{strongly envies} agent $a_j$ if $v_i(A_i)<v_i(A_j\setminus\{g\})$ for any $g\in A_j$, i.e., the EF1 criterion from $a_i$ to $a_j$ fails.
Given $t \in[n]$, we say that a (possibly partial) allocation $(A_1,\ldots,A_n)$ is \emph{EF1 except for $t$} if for any pair $i,j\in[n]$ with $i\neq t$ we have $v_i(A_i)\geq v_i(A_j\setminus \{g\})$ for some $g\in G$.
In words, the allocation is EF1 except that agent $a_t$ is allowed to strongly-envy others.

We first prove some lemmas which will be used later.
\begin{lemma}\label{prop:partialtocomplete}
    Fix a valuation profile. Let $(A_1,\ldots,A_n)$ be a partial EF1 allocation. There exists a complete EF1 allocation $(A_1^+,\ldots,A_n^+)$ such that $v_i(A_i^+)\geq v_i(A_i)$ for each $i\in [n]$.
\end{lemma}
\begin{proof}
    Construct the envy-graph for the partial allocation $(A_1,\ldots,A_n)$ and then perform the \emph{envy-graph procedure} proposed by~\citet{lipton2004approximately} to obtain a complete allocation $(A_1^+,\ldots,A_n^+)$.
    The monotonic property of the procedure directly implies this proposition.
\end{proof}

\begin{lemmarep}\label{prop:maximuminexception}
    Fix a valuation profile and an arbitrary agent $a_{t}$. Let $\efallocations$ be the set of all complete EF1 allocations. Let ${\efallocations}^{-t}$ be the set of all possibly partial allocations that are EF1 except for possibly $a_t$. The allocation in $\efallocations$ that maximizes agent $a_t$'s utility is also the one in ${\efallocations}^{-t}$ that maximizes $a_t$'s utility.
\end{lemmarep}
In other words, if $a_t$ gets the maximum possible value subject to EF1, he cannot get a higher value by agreeing to give up the EF1 guarantee for himself.
This claim is trivially true for share-based fairness notions such as proportionality, but quite challenging to prove for EF1; see appendix.

% \biaoshuai{I think this proof should go to the appendix.}
\begin{proof}
    Assume $t=1$ without loss of generality.
    The allocation space $\efallocations$ is clearly a subset of ${\efallocations}^{-1}$.
    Assume for the sake of contradiction that, for all possibly partial allocations in ${\efallocations}^{-1}$, agent $a_1$ strongly envies someone else.
    Let $(A_1,\ldots,A_n)$ be  an allocation that minimizes $|\bigcup_{i=1}^nA_i|$ (minimizes the total number of goods allocated) among all allocations in ${\efallocations}^{-1}$.

    For each $i\neq 1$, agent $a_i$ will strongly envy some other agent if an arbitrary good is removed from $A_i$, for otherwise, the minimality is violated.
    We say that an agent $a_i$ \emph{champions} agent $a_j$ if the following holds.
    \begin{itemize}
        \item $a_i$ strongly envies $a_j$ for $i=1$;
        \item for $i\neq 1$, let agent $a_i$ removes the most valuable good from each $A_k$ (for $k\neq i$) and let $A_k^-$ be the resultant bundle; then the championed agent, agent $a_j$, is defined by the index $k$ with the maximum $v_i(A_k^-)$.
    \end{itemize}
    We then construct a \emph{champion graph} which is a directed graph with $n$ vertices where the vertices represent the $n$ agents and an edge from $a_i$ to $a_j$ represents that agent $a_i$ champions agent $a_j$.
    By our definition, each vertex in the graph has at least one outgoing edge, so the graph must contain a directed cycle $C$.

    Consider a new allocation $(A_1',\ldots,A_n')$ defined as follows. For every edge $(a_i,a_j)$ in $C$, let agent $a_i$ remove the most valuable good from $A_{j}$ and then take the bundle.
    We will show that $v_1(A_1')\geq v_1(A_1)$ and $(A_1',\ldots,A_n')\in {\efallocations}^{-1}$, which will contradict the minimality of $(A_1,\ldots,A_n)$.

    It is easy to see $v_1(A_1')\geq v_1(A_1)$. If agent $a_1$ is not in the cycle $C$, then her utility is unchanged. Otherwise, she receives a bundle that she previously strongly envies, and one good is then removed from the bundle. The property of strong envy guarantees that $v_1(A_1')>v_1(A_1)$.

    To show that $(A_1',\ldots,A_n')\in {\efallocations}^{-1}$, first consider any agent $a_i$ with $i\neq 1$ that is not in $C$.
    Agent $a_i$'s bundle is unchanged, and she will not strongly envy anyone else as before (as only item-removals happen during the update).
    
    Next consider any agent $a_i$ with $i\neq 1$ that is in $C$.
    Let $a_j$ be the agent such that $(a_i,a_j)$ is an edge in $C$.
    Let $A_j^-$ be the bundle with the most valuable good (according to $v_i$) removed from $A_j$.
    By our definition, agent $a_i$ receives $A_j^-$ in the new allocation.
    We will prove that agent $a_i$, by receiving $A_j^-$, does not strongly-envy any of the original bundles $A_k$, for any $k\in [n]$.
    
    Our definition of championship ensures that this is true for any $k\neq i$, as the new bundle of $a_i$ is at least as valuable for $a_i$ than every other bundle with an item removed.
    
    It remains to show that this holds for $k=i$.
    As we have argued at the beginning, in the original allocation $(A_1,\ldots,A_n)$, removing any item from $A_i$ would make agent $a_i$ strongly envy some other agent.
    By our definition of championship, when one good $g'$ is removed from $A_i$, agent $a_i$ strongly envies $a_j$, which implies $a_i$ thinks $A_j^-$ is more valuable than $A_i\setminus\{g'\}$.
    Therefore, in the new allocation, by receiving $A_j^-$, agent $a_i$ does not strongly envy $A_i$.
    
    We have proved that agent $a_i$, by receiving the bundle $A_j^-$, does not strongly envy any of the $n$ original bundles $A_1,\ldots,A_n$.
    Since the new allocation only involves item removals, agent $a_i$ does not strongly envy anyone else in the new allocation.
    
    Hence, the new allocation is in $\efallocations_{1^-}$, which contradicts the minimality of $(A_1,\ldots,A_n)$.
\end{proof}

\begin{theoremrep}\label{thm:gamma-psi-indivisible}
    The $\Gamma$-$\Psi$ mechanism in Sect.~\ref{sect:EF1Mechanism_description} has a RAT-degree of $n-1$.
\end{theoremrep}
\begin{proofsketch}
    For every profitable manipulation by $a_i$, and for every unknown agent $a_j$, the volatility of $\Gamma$ implies that, for some possible valuation $v_j$, 
    a truthful report by $a_i$ leads to $a_i$ being the favored agent and $a_j$ being the unfavored agent, whereas the manipulation leads to 
    $a_i$ being the unfavored agent and $a_j$ being the favored agent. We use this fact, combined with \Cref{prop:partialtocomplete} and \Cref{prop:maximuminexception}, to prove that the manipulation may be harmful for $a_i$.
\end{proofsketch}
\begin{proof}
Let $i,j$ be two arbitrary agents.
Fix $n-2$ arbitrary valuations $\{v_k\}_{k\notin\{i,j\}}$ for the remaining $n-2$ agents.
Consider two arbitrary valuations for agent $a_i$, $v_i$ and $v_i'$, with $v_i\neq v_i'$, where $v_i$ is $a_i$'s true valuation.
We will show that switching from $v_i$ to $v_i'$ is not a safe manipulation.

Let $\ell^\ast$ be some good that $a_i$ values positively, that is, $v_{i,\ell^\ast}>0$.
By the volatility of $\Gamma$, we can construct the valuation of agent $a_j$ such that 
\begin{itemize}
    \item $v_{j,\ell^\ast}>0$, and $v_{j,\ell}=0$ for any $\ell\neq\ell^\ast$;
    \item if agent $a_i$ truthfully reports $v_i$, then agent $a_i$ is mechanism-favored and agent $a_j$ is mechanism-unfavored; if agent $a_i$ reports $v_i'$ instead, then agent $a_j$ is mechanism-favored and agent $a_i$ is mechanism-unfavored.
\end{itemize}

Let $(A_1,\ldots,A_n)$ be the allocation output by $\Psi$ when agent $a_i$ reports $v_i$ truthfully, and $(A_1',\ldots,A_n')$ be the allocation output by $\Psi$ when agent $a_i$ reports $v_i'$.
Our objective is to show that $v_i(A_i)>v_i(A_i')$.

Let us consider $(A_1',\ldots,A_n')$ first.
We know that $g_{\ell^\ast}\in A_j'$.
To see this, notice that $A_j'$ maximizes agent $a_j$'s utility as long as $g_{\ell^\ast}\in A_j'$.
In addition, there exists a valid allocation $(A_1',\ldots,A_n')$ output by $\Psi$ with $g_{\ell^\ast}\in A_j'$: consider the round-robin mechanism with agent $a_j$ be the first and agent $a_i$ be the last under the order $\pi$.

Consider a new allocation $(A_1'',\ldots,A_n'')$ in which $g_{l^*}$ is moved from $a_j$ to $a_i$, that is,
\begin{itemize}
\item $A_i''=A_i'\cup\{g_{\ell^\ast}\}$,
\item $A_j''=A_j'\setminus\{g_{\ell^\ast}\}$,
\item $A_k''=A_k'$ for $k\notin\{i,j\}$.
\end{itemize}
Notice that $(A_1'',\ldots,A_n'')$ is EF1 except for $i$: 
\begin{itemize}
    \item agent $a_j$ will not envy any agent $a_k$ with $k\neq i$ (as each bundle $A_k''$ has a zero value for $j$), and agent $a_j$ will not envy agent $a_i$ upon removing the item $g_{\ell^\ast}$ from $A_i''$;
    \item no other agent $k$ strongly envies agent $i$: given that no one envies agent $i$ in $(A_1',\ldots,A_n')$ (as agent $i$ is mechanism-unfavored), no one strongly envies agent $i$ in $(A_1'',\ldots,A_n'')$;
    \item no agent strongly envies agent $a_j$, as $A_j''\subsetneq A_j'$;
    \item any two agents in $N\setminus\{a_i,a_j\}$ do not strongly envy each other, as their allocations are not changed.
\end{itemize}

Now, consider $(A_1,\ldots,A_n)$, which is the allocation that favors agent $a_i$ when agent $a_i$ truthfully reports $v_i$.
We can assume $A_j=\emptyset$ without loss of generality.
If not, we can reallocate goods in $A_j$ to the remaining $n-1$ agents while keeping the EF1 property among the remaining $n-1$ agents (\Cref{prop:partialtocomplete}).
Agent $a_j$ will not strongly envy anyone, as removing the good $g_{\ell^\ast}$ kills the envy.
Thus, the resultant allocation is still EF1 and no one envies the empty bundle $A_j$.
In addition, by \Cref{prop:partialtocomplete}, the utility of each agent $a_k$ with $k\neq j$ does not decrease after the reallocation of $A_j$.

Let ${\yefallocations}$ be the set of all EF1 allocations of the item-set $G$ to the agent-set $N\setminus\{a_j\}$.
Let ${\yefallocations}^{-i}$ be the set of all possibly partial allocations of the item-set $G$ to the agent-set $N\setminus\{a_j\}$ that are EF1 except for agent $i$.
The above argument shows that $(A_1,\ldots,A_{j-1},A_{j+1},\ldots,A_n)$ is an allocation in ${\yefallocations}$ that maximizes agent $a_i$'s utility.
We have also proved that $(A_1'',\ldots,A_{j-1}'',A_{j+1}'',\ldots,A_n'')\in {\yefallocations}^{-i}$.
By \Cref{prop:maximuminexception}, we have $v_i(A_i)\geq v_i(A_i'')$.
In addition, we have $A_i''=A_i'\cup\{g_{\ell^\ast}\}$, $g_{\ell^\ast}\notin A_i'$, and $v_{i,\ell^\ast}>0$ (by our assumption), which imply $v_i(A_i'')>v_i(A_i')$.
Therefore, $v_i(A_i)>v_i(A_i')$.
\end{proof}

\begin{remark}
Our $\Gamma$-$\Psi$ algorithm does not run in polynomial time, as it requires to maximize the utility of a certain agent subject to EF1. which is %not known to be polynomial-time solvable.
an NP-hard problem (e.g., the proof in Appendix A.2 of~\citet{barman2019fair} can easily imply this).
We do not know if a polynomial-time EF1 algorithm with a high RAT-degree exists.
\end{remark}
% \newpage
\section{Cake Cutting}
\label{sec:cake-cutting}
In this section, we study the \emph{cake cutting} problem: the allocation of divisible heterogeneous resources to $n$ agents.
The cake cutting problem was proposed by~\citet{Steinhaus48,Steinhaus49}, and it is a widely studied subject in mathematics, computer science, economics, and political science.

In the cake cutting problem, the resource/cake is modeled as an interval $[0,1]$, and it is to be allocated among a set of $n$ agents $N=\{a_1,\ldots,a_n\}$.
An allocation is denoted by $(A_1,\ldots,A_n)$ where $A_i\subseteq[0,1]$ is the share allocated to agent $a_i$.
We require that each $A_i$ is a union of finitely many closed non-intersecting intervals, and, for each pair of $i,j\in[n]$, $A_i$ and $A_j$ can only intersect at interval endpoints, i.e., the measure of $A_i\cap A_j$ is $0$.
We say an allocation is \emph{complete} if $\bigcup_{i=1}^nA_i=[0,1]$.
Otherwise, it is \emph{partial}.

The true preferences $T_i$ of agent $a_i$ are given by  a \emph{value density function} $v_i:[0,1]\to\mathbb{R}_{\geq0}$ that describes agent $a_i$'s preference over the cake.
To enable succinct encoding of the value density function, we adopt the widely considered assumption that each $v_i$ is \emph{piecewise constant}: there exist finitely many points $x_{i0},x_{i1},x_{i2},\ldots,x_{ik_i}$ with $0=x_{i0}<x_{i1}<x_{i2}<\cdots<x_{ik_i}=1$ such that $v_i$ is a constant on every interval $(x_{i\ell},x_{i(\ell+1)})$, $\ell=0,1,\ldots,k_i-1$.
Given a subset $S\subseteq[0,1]$ that is a union of finitely many closed non-intersecting intervals, agent $a_i$'s value for receiving $S$ is then given by
$V_i(S)=\int_Sv_i(x)dx.$

Fairness and efficiency are two natural goals for allocating the cake.
For efficiency, we consider two commonly used criteria: \emph{social welfare} and \emph{Pareto-optimality}.
Given an allocation $(A_1,\ldots,A_n)$, its \emph{social welfare} is given by
$\sum_{i=1}^nV_i(A_i)$.
This is a natural measurement of efficiency that represents the overall happiness of all agents.
Pareto-optimality is a yes-or-no criterion for efficiency.
An allocation $(A_1,\ldots,A_n)$ is \emph{Pareto-optimal} if there does not exist another allocation $(A_1',\ldots,A_n')$ such that $V_i(A_i')\geq V_i(A_i)$ for each agent $a_i$ and at least one of these $n$ inequalities is strict.

For fairness, we study two arguably most important notions: \emph{envy-freeness} and \emph{proportionality}.
An allocation $(A_1,\ldots,A_n)$ is \emph{proportional} if each agent receives her average share, i.e., for each $i\in[n]$, $V_i(A_i)\geq\frac1nV_i([0,1])$.
An allocation $(A_1,\ldots,A_n)$ is \emph{envy-free} is every agent weakly prefers her own allocated share, i.e., for every pair $i,j\in[n]$, $V_i(A_i)\geq V_i(A_j)$.
A complete envy-free allocation is always proportional
\iffalse %EREL: removed to save space
: for each $i\in[n]$, summing up the $n$ inequalities $V_i(A_i)\geq V_i(A_j)$ for $j=1,\ldots,n$ yields
$$nV_i(A_i)\geq \sum_{j=1}^nV_i(A_j)=V_i\left(\bigcup_{j=1}^nA_j\right)=V_i([0,1]),$$
which is $V_i(A_i)\geq\frac1nV_i([0,1])$.
However, this implication does not hold for partial allocation: $A_1=\cdots=A_n=\emptyset$ gives an envy-free allocation, but it is clearly not proportional.
\fi
, but this implication does not hold for partial allocations.

Before we discuss our results, we define an additional notion, \emph{uniform segment}, which will be used throughout this section.
Given $n$ value density functions $v_1,\ldots,v_n$ (that are piecewise constant by our assumptions), we identify the set of points of discontinuity for each $v_i$ and take the union of the $n$ sets.
Sorting these points by ascending order, we let $x_1,\ldots,x_{m-1}$ be all points of discontinuity for all the $n$ value density functions.
Let $x_0=0$ and $x_m=1$.
These points define $k$ intervals, $(x_0,x_1),(x_1,x_2),\ldots,(x_{m-1},x_m)$, such that each $v_i$ is a constant on each of these intervals.
We will call each of these intervals a \emph{uniform segment}, and we will denote $X_t=(x_{t-1},x_{t})$ for each $t=1,\ldots,m$.
For each agent $a_i$, we will slightly abuse the notation by using $v_i(X_t)$ to denote $v_i(x)$ with $x\in X_t$.

Since all agents' valuations on each uniform segment are uniform, it is tempting to think about the cake cutting problem as the problem of allocating $m$ divisible homogeneous goods.
However, this interpretation is inaccurate when concerning agents' strategic behaviors, as, in the cake cutting setting, an agent can manipulate her value density function with a different set of points of discontinuity, which affects how the divisible goods are defined.
To see a significant difference between these two models, in the divisible goods setting, the \emph{equal division rule} that allocates each divisible good evenly to the $n$ agents is truthful (with RAT-degree $n$), envy-free and proportional, while, in the cake cutting setting, it is proved in~\citet{tao2022existence} that truthfulness and proportionality are incompatible even for two agents.


\paragraph{Results}
In Sect.~\ref{sect:cake-msw}, we start by considering the simple mechanism that outputs allocation with the maximum social welfare.
We show that the RAT-degree of this mechanism is $0$.
Similar as it is in the case of indivisible goods, we also consider the normalized variant of this mechanism, and we show that the RAT-degree is $1$.
In Sect.~\ref{sect:cake-fair}, we consider mechanisms that output fair allocations. We review the mechanisms studied in~\citet{BU2023Rat} by studying their RAT-degrees. 
We will see that one of those mechanisms, which always outputs envy-free allocation, has a RAT-degree of $n-1$.
However, this mechanism has a very poor performance on efficiency.
Finally, in Sect.~\ref{sect:cake-Prop+PO}, we propose a new mechanism with RAT-degree $n-1$ that always outputs proportional and Pareto-optimal allocations.

\subsection{Maximum Social Welfare Mechanisms}
\label{sect:cake-msw}
It is easy to find an allocation that maximizes the social welfare: for each uniform segment $X_t$, allocate it to an agent $a_i$ with the maximum $v_i(X_t)$.
When multiple agents have equally largest value of $v_i(X_t)$ on the segment $X_t$, we need to specify a tie-breaking rule.
However, as we will see later, the choice of the tie-breaking rule does not affect the RAT-degree of the mechanism.

%We consider two different tie-breaking rules:
%\begin{enumerate}
%    \item agent with the smallest index takes $X_t$, and
%    \item $X_t$ is evenly distributed among these agents such that the agent with the smallest index takes the left-most piece of $X_t$ and the agent with the largest index takes the right-most piece.
%\end{enumerate}

It is easy to see that, whatever the tie-breaking rule is, the maximum social welfare mechanism is safely manipulable.
It is safe for an agent to report higher values on every uniform segment.
For example, doubling the values on all uniform segments is clearly a safe manipulation.

\begin{observation}
\label{obs:utilitarian-cake-cutting}
Utilitarian Cake-Cutting with any tie-breaking rule has RAT-degree $0$.
\end{observation}

We next consider the following variant of the maximum social welfare mechanism:
first rescale each $v_i$ such that $V_i([0,1])=\int_0^1v_i(x)dx=1$, and then output the allocation with the maximum social welfare.
We will show that the RAT-degree is $1$.
The proof is similar to the one for indivisible items (\Cref{thm:normalized-utilitarian-goods}) and is given in the appendix.

\begin{theoremrep}
When there are at least three agents,
Normalized Utilitarian Cake-Cutting with any tie-breaking rule has RAT-degree $1$.
\end{theoremrep}
\begin{proof}
    We assume without loss of generality that the value density function reported by each agent is normalized (as, otherwise, the mechanism will normalize the function for the agent).
    
    We first show that the mechanism is not $0$-known-agents safely-manipulable.
    Consider an arbitrary agent $a_i$ and let $v_i$ be her true value density function.
    Consider an arbitrary misreport $v_i'$ of agent $a_i$ with $v_i'\neq v_i$.
    Since the value density functions are normalized, there must exist an interval $(a,b)$ where $v_i'$ and $v_i$ are constant and $v_i'(x)<v_i(x)$ for $x\in(a,b)$.
    Choose $\varepsilon>0$ such that $v_i(x)>v_i'(x)+\varepsilon$.
    Consider the following two value density functions (note that both are normalized):
    $$v^{(1)}(x)=\left\{\begin{array}{ll}
        v_i'(x)+\varepsilon & \mbox{if }x\in(a,b) \\
        v_i'(x)-\varepsilon\cdot\frac{b-a}{1+a-b} & \mbox{otherwise}
    \end{array}\right. \quad\mbox{and}\quad v^{(2)}(x)=\left\{\begin{array}{ll}
        v_i'(x)-\varepsilon & \mbox{if }x\in(a,b) \\
        v_i'(x)+\varepsilon\cdot\frac{b-a}{1+a-b} & \mbox{otherwise}
    \end{array}\right..$$
    Suppose the remaining $n-1$ agents' reported value density functions are either $v^{(1)}$ or $v^{(2)}$ and each of $v^{(1)}$ and $v^{(2)}$ is reported by at least one agent (here we use the assumption $n\geq 3$).
    In this case, agent $a_i$ will receive the empty set by reporting $v_i'$.
    On the other hand, when reporting $v_i$, agent $a_i$ will receive an allocation that at least contains $(a,b)$ as a subset.
    Since $v_i$ has a positive value on $(a,b)$, reporting $v_i'$ is not a safe manipulation.

    We next show that the mechanism is $1$-known-agent safely-manipulable.
    \erel{Doesn't this part follow from the analogous result on indivisible items?}\biaoshuai{I think so}
    Suppose agent $a_1$'s true value density function is
    $$v_1(x)=\left\{\begin{array}{ll}
        1.5 & \mbox{if }x\in[0,0.5] \\
        0.5 & \mbox{otherwise}
    \end{array}\right.,$$ 
    and agent $a_1$ knows that agent $a_2$ reports the uniform value density function $v_2(x)=1$ for $x\in[0,1]$.
    We will show that the following manipulation of agent $a_1$ is safe and profitable.
    $$v_1'(x)=\left\{\begin{array}{ll}
        2 & \mbox{if }x\in[0,0.5] \\
        0 & \mbox{otherwise}
    \end{array}\right.$$
    Firstly, regardless of the reports of the remaining $n-2$ agents, the final allocation received by agent $a_1$ must be a subset of $[0,0,5]$, as agent $a_2$'s value is higher on the other half $(0.5,1]$.
    Since $v_1'$ is larger than $v_1$ on $[0,0,5]$, any interval received by agent $a_1$ when reporting $v_1$ will also be received if $v_1'$ were reported.
    Thus, the manipulation is safe.

    Secondly, if the remaining $n-2$ agents' value density functions are
    $$v_3(x)=v_4(x)=\cdots=v_n(x)=\left\{\begin{array}{ll}
        1.75 & \mbox{if }x\in[0,0.5] \\
        0.25 & \mbox{otherwise}
    \end{array}\right.,$$
    it is easy to verify that agent $a_1$ receives the empty set when reporting truthfully and she receives $[0,0.5]$ by reporting $v_1'$.
    Therefore, the manipulation is profitable.
\end{proof}

\subsection{Fair Mechanisms}
\label{sect:cake-fair}
In this section, we focus on mechanisms that always output fair (envy-free or proportional) allocations.
As we have mentioned earlier, it is proved in~\citet{tao2022existence} that truthfulness and proportionality are incompatible even for two agents and even if partial allocations are allowed.
This motivates the search for fair cake-cutting algorithms with a high RAT-degree.

% \erel{To motivate this section, we should cite the impossibility result on truthful fair mechanisms; this motivates the search for the "next-best" option, which is a high RAT-degree.} \biaoshuai{Done. I added the sentence above.}

The mechanisms discussed in this section have been considered in~\citet{BU2023Rat}.
However, they are only studied by whether or not they are risk-averse truthful (in our language, whether the RAT-degree is positive).
With our new notion of RAT-degree, we are now able to provide a more fine-grained view of their performances on strategy-proofness.

One natural envy-free mechanism is to evenly allocate each uniform segment $X_t$ to all agents.
Specifically, each $X_t$ is partitioned into $n$ intervals of equal length, and each agent receives exactly one of them.
It is easy to see that $V_i(A_j)=\frac1nV_i([0,1])$ for any $i,j\in[n]$ under this allocation, so the allocation is envy-free and proportional.

To completely define the mechanism, we need to specify the order of evenly allocating each $X_t=(x_{t-1},x_t)$ to the $n$ agents.
A natural tie-breaking rule is to let agent $a_1$ get the left-most interval and agent $a_n$ get the right-most interval.
Specifically, agent $a_i$ receives the $i$-th interval of $X_t$, which is $[x_{t-1}+\frac{i-1}n(x_t-x_{t-1}),x_{t-1}+\frac{i}n(x_t-x_{t-1})]$.
However, it was proved in~\citet{BU2023Rat} that the equal division mechanism under this ordering rule is safely-manipulable, i.e., its RAT-degree is $0$.
In particular, agent $a_1$, knowing that she will always receive the left-most interval in each $X_t$, can safely manipulate by deleting a point of discontinuity in her value density function if her value on the left-hand side of this point is higher.

To avoid this type of manipulation, a different ordering rule was considered by~\citet{BU2023Rat} (See Mechanism 3 in their paper): at the $t$-th segment, the $n$ equal-length subintervals of $X_t$ are allocated to the $n$ agents with the left-to-right order $a_t,a_{t+1},\ldots,a_n,a_1,a_2,\ldots,a_{t-1}$.
By using this ordering rule, an agent does not know her position in the left-to-right order of $X_t$ without knowing others' value density functions.
Indeed, even if only one agent's value density function is unknown, an agent cannot know the index $t$ of any segment $X_t$.
This suggests that the mechanism has a RAT-degree of $n-1$.

\begin{theoremrep}
    Consider the mechanism that evenly partitions each uniform segment $X_t$ into $n$ equal-length subintervals and allocates these $n$ subintervals to the $n$ agents with the left-to-right order $a_t,a_{t+1},\ldots,a_n,a_1,a_2,\ldots,a_{t-1}$. It has RAT-degree $n-1$ and always outputs envy-free allocations.
\end{theoremrep}
\begin{proofsketch}
    Envy-freeness is trivial: for any $i,j\in[n]$, we have $V_i(A_j)=\frac1nV_i([0,1])$.
    The general impossibility result in~\citet{tao2022existence} shows that no mechanism with the envy-freeness guarantee can be truthful, so the RAT-degree is at most $n-1$.

 To show that the RAT-degree is exactly $n-1$, we show that, if even a single agent is not known to the manipulator, it is possible that this agent's valuation adds discontinuity points in a way that the ordering in each uniform segment is unfavorable for the manipulator.
\end{proofsketch}
\begin{proof}
    Envy-freeness is trivial: for any $i,j\in[n]$, we have $V_i(A_j)=\frac1nV_i([0,1])$.
    The general impossibility result in~\citet{tao2022existence} shows that no mechanism with the envy-freeness guarantee can be truthful, so the RAT-degree is at most $n-1$.

    To show that the RAT-degree is exactly $n-1$, consider an arbitrary agent $a_i$ with true value density function $v_i$ and an arbitrary agent $a_j$ whose report is unknown to agent $a_i$.
    Fix $n-2$ arbitrary value density functions $\{v_k\}_{k\notin\{i,j\}}$ that are known by agent $a_i$ to be the reports of the remaining $n-2$ agents.
    For any $v_i'$, we will show that agent $a_i$'s reporting $v_i'$ is either not safe or not profitable.

    Let $T$ be the set of points of discontinuity for $v_1,v_2,\ldots,v_{j-1},v_{j+1},\ldots,v_n$, and $T'$ be the set of points of discontinuity with $v_i$ replaced by $v_i'$.
    If $T\subseteq T'$ (i.e., the uniform segment partition defined by $T'$ is ``finer'' than the partition defined by $T$), the manipulation is not profitable, as agent $a_i$ will receive her proportional share $\frac1nV_i([0,1])$ in both cases.
    
    It remains to consider the case where there exists a point of discontinuity $y$ of $v_i$ such that $y\in T$ and $y\notin T'$.
    This implies that $y$ is a point of discontinuity in $v_i$, but not in $v_i'$ nor in the valuation of any other agent.
    We will show that the manipulation is not safe in this case.

    Choose a sufficiently small $\varepsilon>0$ such that $(y-\varepsilon, y+\varepsilon)$ is contained in a uniform segment defined by $T'$.
    We consider two cases, depending on whether the ``jump'' of $v_i$ in its discontinuity point $y$ is upwards or downwards.
    
    \underline{Case 1:}  $\lim_{x\to y^-}v_i(x)<\lim_{x\to y^+}v_i(x)$.
    We can construct $v_j$ such that: 1) $y-\varepsilon$ and $y+\varepsilon$ are points of discontinuity of $v_j$, and 2) the uniform segment $(y-\varepsilon, y+\varepsilon)$ under the profile $(v_1,\ldots,v_{i-1},v_i',v_{i+1},\ldots,v_n)$ is the $t$-th segment where $n$ divides $t-i$ (i.e., agent $a_i$ receives the left-most subinterval of this uniform segment).
    Notice that 2) is always achievable by inserting a suitable number of points of discontinuity for $v_j$ before $y-\varepsilon$.
    Given that $\lim_{x\to y^-}v_i(x)<\lim_{x\to y^+}v_i(x)$, agent $a_i$'s allocated subinterval on the segment $(y-\varepsilon, y+\varepsilon)$ has value strictly less than $\frac1nV_i([(y-\varepsilon, y+\varepsilon])$.

    \underline{Case 2:} $\lim_{x\to y^-}v_i(x)>\lim_{x\to y^+}v_i(x)$.
    We can construct $v_j$ such that 1) $(y-\varepsilon, y+\varepsilon)$ is a uniform segment under the profile $(v_1,\ldots,v_{i-1},v_i',v_{i+1},\ldots,v_n)$, and 2) agent $a_i$ receives the right-most subinterval on this segment.
    In this case, agent $a_i$ again receives a value of strictly less than $\frac1nV_i([(y-\varepsilon, y+\varepsilon])$ on the segment $(y-\varepsilon, y+\varepsilon)$.

    We can do this for every point $y$ of discontinuity of $v_i$ that is in $T\setminus T'$.
    By a suitable choice of $v_j$ (with a suitable number of points of discontinuity of $v_j$ inserted in between), we can make sure agent $a_i$ receives a less-than-average value on every such segment $(y-\varepsilon,y+\varepsilon)$.
    Moreover, agent $a_i$ receives exactly the average value on each of the remaining segments, because the remaining discontinuity points of $T$ are contained in $T'$.
    Therefore, the overall utility of $a_i$ by reporting $v_i'$ is strictly less than $\frac1nV_i([0,1])$.
    Given that $a_i$ receives value exactly $\frac1nV_i([0,1])$ for truthfully reporting $v_i$, reporting $v_i'$ is not safe.
\end{proof}

Although the equal division mechanism with the above-mentioned carefully designed ordering rule is envy-free and has a high RAT-degree of $n-1$, it is undesirable in at least two aspects:
\begin{enumerate}
    \item it requires quite many cuts on the cake by making $n-1$ cuts on each uniform segment; this is particularly undesirable if piecewise constant functions are used to approximate more general value density functions.

    \item it is highly inefficient: each agent $a_i$'s utility is never more than her minimum proportionality requirement $\frac1nV_i([0,1])$;
\end{enumerate}

% We handle point (1) in \Cref{sect:cake-connected} and point (2) in \Cref{sect:cake-Prop+PO}.

%\subsection{Connected cake-cutting} \label{sect:cake-connected}

% \biaoshuai{I think the remaining part can be mostly put to the appendix if we do not have enough space. We can just say briefly here that all the known variants of moving-knife have RAT-degree at most $1$.}
% EREL: Done

Regarding point (1), researchers have been looking at allocations with \emph{connected pieces}, i.e., allocations with only $n-1$ cuts on the cake.
A well-known mechanism in this category is \emph{the moving-knife procedure}, which always outputs proportional allocations.
This mechanism was first proposed by~\citet{dubins1961cut}. It always returns a proportional connected allocation.
Unfortunately, it was shown by~\citet{BU2023Rat} that Dubins and Spanier's moving-knife procedure is safely-manipulable for some very subtle reasons.

\citet{BU2023Rat} proposed a variant of the moving-knife procedure that is RAT.
In addition, they showed that another variant of moving-knife procedure proposed by~\citet{ortega2022obvious} is also RAT.\footnote{It should be noticed that, when $v_i$ is allowed to take $0$ value, tie-breaking needs to be handled very properly to ensure RAT. See \citet{BU2023Rat} for more details. Here, for simplicity, we assume $v_i(x)>0$ for each $i\in[n]$ and $x\in[0,1]$.}
In the appendix, we describe both mechanisms and show that both of them have RAT-degree $1$.
%
\begin{toappendix}
\subsection{Moving-knife mechanisms: descriptions and proofs}
Hereafter, we assume $v_i(x)>0$ for each $i\in[n]$ and $x\in[0,1]$.

\paragraph{Dubins and Spanier's moving-knife procedure}
Let $u_i=\frac1nV_i([0,1])$ be the value of agent $a_i$'s proportional share.
In the first iteration, each agent $a_i$ marks a point $x_i^{(1)}$ on $[0,1]$ such that the interval $[0,x_i^{(1)}]$ has value exactly $u_i$ to agent $a_i$.
Take $x^{(1)}=\min_{i\in[n]}x_i^{(1)}$, and the agent $a_{i_1}$ with $x_{i_1}^{(1)}=x^{(1)}$ takes the piece $[0,x^{(1)}]$ and leaves the game.
In the second iteration, let each of the remaining $n-1$ agents $a_i$ marks a point $x_i^{(2)}$ on the cake such that $[x^{(1)},x_i^{(2)}]$ has value exactly $u_i$.
Take $x^{(2)}=\min_{i\in[n]\setminus\{i_1\}}x_i^{(2)}$, and the agent $a_{i_2}$ with $x_{i_2}^{(2)}=x^{(2)}$ takes the piece $[x^{(1)},x^{(2)}]$ and leave the game.
This is done iteratively until $n-1$ agents have left the game with their allocated pieces.
Finally, the only remaining agent takes the remaining part of the cake.
It is easy to see that each of the first $n-1$ agents receives exactly her proportional share, while the last agent receives weakly more than her proportional share; hence the procedure always returns a proportional allocation.

Notice that, although the mechanism is described in an iterative interactive way that resembles an extensive-form game, we will consider the \emph{direct-revelation} mechanisms in this paper, where the $n$ value density functions are reported to the mechanism at the beginning.
In the above description of Dubins and Spanier's moving-knife procedure, as well as its two variants mentioned later, by saying ``asking an agent to mark a point'', we refer to that the mechanism computes such a point based on the reported value density function.
In particular, we do not consider the scenario where agents can adaptively choose the next marks based on the allocations in the previous iterations.

Unfortunately, it was shown by~\citet{BU2023Rat} that Dubins and Spanier's moving-knife procedure is safely-manipulable for some very subtle reasons.


\citet{BU2023Rat} proposed a variant of the moving-knife procedure that is risk-averse truthful.
In addition, \citet{BU2023Rat} shows that another variant of moving-knife procedure proposed by~\citet{ortega2022obvious} is also risk-averse truthful.\footnote{It should be noticed that, when $v_i$ is allowed to take $0$ value, tie-breaking needs to be handled very properly to ensure risk-averse truthfulness. See \citet{BU2023Rat} for more details.}
Below, we will first describe both mechanisms and then show that both of them have RAT-degree $1$.

\paragraph{Ortega and Segal-Halevi's moving knife procedure}
The first iteration of Ortega and Segal-Halevi's moving knife procedure is the same as it is in Dubins and Spanier's.
After that, the interval $[x^{(1)},1]$ is then allocated \emph{recursively} among the $n-1$ agents $[n]\setminus\{a_{i_1}\}$.
That is, in the second iteration, each agent $a_i$ marks a point $x_i^{(2)}$ such that the interval $[x^{(1)}, x_i^{(2)}]$ has value exactly $\frac1{n-1}V_i([x^{(1)},1])$ (instead of $\frac1nV_1([0,1])$ as it is in Dubins and Spanier's moving-knife procedure).
The remaining part is the same: the agent with the left-most mark takes the corresponding piece and leaves the game.
After the second iteration, the remaining part of the cake is again recursively allocated to the remaining $n-2$ agents.
This is continued until the entire cake is allocated.

\paragraph{Bu, Song, and Tao's moving knife procedure}
Each agent $a_i$ is asked to mark all the $n-1$ ``equal-division-points'' $x_i^{(1)},\ldots,x_i^{(n-1)}$ at the beginning such that $V_i([x_i^{(t-1)},x_i^{(t)}])=\frac1nV_i([0,1])$ for each $t=1,\ldots,n$, where we set $x_i^{(0)}=0$ and $x_i^{(n)}=1$.
The remaining part is similar to Dubins and Spanier's moving-knife procedure:
in the first iteration, agent $i_1$ with the minimum $x_{i_1}^{(1)}$ takes $[0,x_{i_1}^{(1)}]$ and leave the game; in the second iteration, agent $i_2$ with the minimum $x_{i_2}^{(2)}$ among the remaining $n-1$ agents takes $[x_{i_1}^{(1)},x_{i_2}^{(2)}]$ and leave the game; and so on.
The difference to Dubins and Spanier's moving-knife procedure is that each $x_i^{(t)}$ is computed at the beginning, instead of depending on the position of the previous cut.


\begin{theorem}
    The RAT-degree of Ortega and Segal-Halevi's moving knife procedure is $1$.
\end{theorem}
\begin{proof}
    It was proved in~\citet{BU2023Rat} that the mechanism is not $0$-known-agent safely-manipulable.
    It remains to show that it is $1$-known-agents safely-manipulable.
    Suppose agent $a_1$'s value density function is uniform, $v_1(x)=1$ for $x\in[0,1]$, and agent $a_1$ knows that agent $a_2$ will report $v_2$ such that $v_2(x)=1$ for $x\in[1-\varepsilon,1]$ and $v_2(x)=0$ for $x\in[0,1-\varepsilon)$ for some very small $\varepsilon>0$ with $\varepsilon\ll\frac1n$.
    We show that the following $v_1'$ is a safe manipulation.
    $$v_1'(x)=\left\{\begin{array}{ll}
        1 & x\in[0,\frac{n-2}n] \\
        \frac2{n\varepsilon} & x\in[1-2\varepsilon,1-\varepsilon]\\
        0 & \mbox{otherwise}
    \end{array}\right.$$
    Before we move on, note an important property of $v_1'$: for any $t\leq\frac{n-2}n$, we have $V_1([t,1])=V_1'([t,1])$.

    Let $[a,b]$ be the piece received by agent $a_1$ when she reports $v_1$ truthfully.
    If $b\leq \frac{n-2}n$, the above-mentioned property implies that she will also receive exactly $[a,b]$ for reporting $v_1'$.
    If $b>\frac{n-2}n$, then we know that agent $a_1$ is the $(n-1)$-th agent in the procedure.
    To see this, we have $V_1([a,b])\geq\frac1n([0,1])=\frac1n$ by the property of Ortega and Segal-Halevi's moving knife procedure, and we also have $V_1([b,1])=1-b<\frac2n$.
    This implies $b$ cannot be the $1/k$ cut point of $[a,1]$ for $k\geq 3$.
    On the other hand, it is obvious that agent $a_2$ takes a piece after agent $a_1$.
    Thus, by the time agent $a_1$ takes $[a,b]$, the only remaining agent is agent $a_2$.

    Since there are exactly two remaining agents in the game before agent $a_1$ takes $[a,b]$, we have $V_1([a,1])\geq\frac2nV_1([0,1])=\frac2n$.
    This implies $a\leq\frac{n-2}n$ and $b=\frac{a+1}2\leq \frac{n-1}n$.
    On the other hand, by reporting $v_1'$, agent $a_1$ can then get the piece $[a,b']$ with $b'\in[1-2\varepsilon,1-\varepsilon]$.
    We see that $b'>b$. Thus, the manipulation is safe and profitable.
\end{proof}


\begin{theorem}
    The RAT-degree of Bu, Song, and Tao's moving knife procedure is $1$.
\end{theorem}
\begin{proof}
    It was proved in~\citet{BU2023Rat} that the mechanism is not $0$-known-agent safely-manipulable.
    The proof that it is $1$-known-agents safely-manipulable is similar to the proof for Ortega and Segal-Halevi's moving knife procedure, with the same $v_1,v_1'$ and $v_2$.
    It suffices to notice that the first $n-2$ equal-division-points are the same for $v_1$ and $v_1'$, where the last equal-division-point of $v_1'$ is to the right of $v_1$'s.
    Given that agent $a_2$ will always receive a piece after agent $a_1$, the same analysis in the previous proof can show that the manipulation is safe and profitable. 
\end{proof}

\subsection{Additional proofs} 
\end{toappendix}
These results invoke the following question.
\begin{open}
    Is there a proportional connected cake-cutting rule with RAT-degree at least  $2$?
    % \erel{Is it indeed open?}\biaoshuai{Yes to the best of my knowledge.}
\end{open}

We handle point (2) from above in the following subsection.

\subsection{A Proportional and Pareto-Optimal Mechanism with RAT-degree $n-1$}
\label{sect:cake-Prop+PO}
In this section, we provide a mechanism with RAT-degree $n-1$ that always outputs proportional and Pareto-optimal allocations.
In addition, we show that the mechanism can be implemented in polynomial time.
The mechanism uses some similar ideas as the one in \Cref{sect:indivisible-EF1-n-1}.

% \erel{How about the following mechanism: Find a proportional allocation that maximizes the value of agent 1; then of agent 2; etc. It is proportional, Pareto-efficient, and apparently truthful - as it is similar to a serial dictatorship (constrained by proportionality). Is it true?}
% \biaoshuai{I think the mechanism you are suggesting is the same as mine? I don't think it is truthful: the agents with larger indices can manipulate and change the set of feasible (proportional) allocations, which may affect agent $1$'s allocation and potentially be beneficial for themselves. In fact, my EC paper shows that truthfulness is incompatible with proportionality.}
% \erel{It is the same as yours, except that we do not need the function $\Gamma$ --- we just use the same order all the time. It is simpler, although it is not anonymous. Is its RAT-degree still $n-1$?}
% \biaoshuai{I am not sure about this.}

\subsubsection{Description of Mechanism}
The mechanism has two components: an order selection rule $\Gamma$ and an allocation rule $\Psi$.
The order selection rule $\Gamma$ takes the valuation profile $(v_1,\ldots,v_n)$ as an input and outputs an order $\pi$ of the $n$ agents.
We use $\pi_i$ to denote the $i$-th agent in the order.
The allocation rule $\Psi$ then outputs an allocation based on $\pi$.

We first define the allocation rule $\Psi$.
Let $\propallocations$ be the set of all proportional allocations.
Then $\Psi$ outputs an allocation in $\propallocations$ in the following ``leximax'' way:
\begin{enumerate}
    \item the allocation maximizes agent $\pi_1$'s utility;
    \item subject to (1), the allocation maximizes agent $\pi_2$'s utility;
    \item subject to (1) and (2), the allocation maximizes agent $\pi_3$'s utility;
    \item $\cdots$
\end{enumerate}

We next define $\Gamma$.
We first adapt the volatility property of $\Gamma$ (defined in Sect.~\ref{sect:indivisible-EF1-n-1}) to the cake-cutting setting.

\begin{definition}
A function $\Gamma$ (from the set of valuation profiles to the set of orders on agents) is called \emph{volatile} if for any two agents $a_i\neq a_j$ and any two orders $\pi$ and $\pi'$, any set of $n-2$ value density functions $\{v_k\}_{k\notin\{i,j\}}$, any value density function $\bar{v}_j$, and any two reported valuation profiles $v_i,v_i'$ of agent $a_i$ with $v_i\neq v_i'$, there exists a valuation function $v_j$ of agent $a_j$ such that
\begin{itemize}

    \item $v_j$ is a rescaled version of $\bar{v}_j$, i.e., there exists $\alpha$ such that $v_j(x)=\alpha\bar{v}_j(x)$ for all $x\in[0,1]$;
% \erel{If we normalize the valuations, maybe we do not need this rescaling?}\biaoshuai{My construction of $\Gamma$ depends on the highest value of value density functions.}
    
    \item $\Gamma$ outputs $\pi$ for the valuation profile $\{v_k\}_{k\notin\{i,j\}}\cup\{v_i\}\cup\{v_j\}$, and $\Gamma$ outputs $\pi'$ for the valuation profile $\{v_k\}_{k\notin\{i,j\}}\cup\{v_i'\}\cup\{v_j\}$.
\end{itemize}

In other words, a manipulation of agent $i$ from $v_i$ to $v_i'$ can affect the output of $\Gamma$ in any possible way (from any order $\pi$ to any order $\pi'$), depending on the report of agent $j$.
\end{definition}


\begin{propositionrep}
    There exists a volatile function $\Gamma$.
\end{propositionrep}
\begin{proof}
    The function does the following.
    It first finds the maximum value among all the value density functions (overall all uniform segments): $v^\ast=\max_{i\in[n],\ell\in[m]}v_i(X_\ell)$.
    It then views $v^\ast$ as a binary string that encodes the following information:
    \begin{itemize}
        \item the index $i$ of an agent $a_i$,
        \item a non-negative integer $t$,
        \item two non-negative integers $a$ and $b$ that are at most $n!-1$.
        % \erel{Should these integers be in the range $0,\ldots,n!$ ?}\biaoshuai{It does not seem matter, as I am taking mod $n!$ at the end. But it does not harm to do it.}
    \end{itemize}
    We append $0$'s as most significant bits to $v^\ast$ if the length of the binary string is not long enough to support the format of the encoding.
    If the encoding of $v^\ast$ is longer than the length enough for encoding the above-mentioned information, we take only the least significant bits in the amount required for the encoding.
% \erel{Can $v^*$ always represent such a long string? What if, incidentally, $v^*=1$?}\biaoshuai{I added the last sentence}

    The order $\pi$ is chosen in the following way.
    Firstly, we use an integer between $0$ and $n!-1$ to encode an order.
    Then, let $s$ be the $t$-th bit that encodes agent $a_i$'s value density function.
    The order is defined to be $as+b\bmod (n!)$.

We now prove that $\Gamma$ is volatile.
Suppose $v_i$ and $v_i'$ differ at their $t$-th bits, 
so that that the $t$-th bit of $v_i$ is $s$ and the $t$-th bit of $v_i'$ is $s'\neq s$.
We construct a number $v^*$ that encodes the index $i$, the integer $t$, and two integers $a,b$ 
such that $as+b\bmod (n!)$ encodes $\pi$ and $as'+b\bmod (n!)$ encodes $\pi'$.


Then, we construct $v_j$ by rescaling $\bar{v}_j$ such that the maximum value among all density functions is attained by $v_j$, and this number is exactly $v^{\ast}$, that is, $v^\ast=v_j(X_\ell)$ for some uniform segment $X_\ell$.
If the encoded $v^*$ is not large enough to be a maximum value, we enlarge it as needed by adding most significant bits.

By definition, $\Gamma$ returns $\pi$ when $a_i$ reports $v_i$ and returns $\pi'$ when $a_i$ reports $v_i'$.

% \erel{What if the encoded $v^*$ is too small, as there are other agents (not $i,j$) that have larger values in some segments?}\biaoshuai{If the encoding of $v^\ast$ is longer than the length enough for encoding the above-mentioned information, we can only take a substring (say, the least significant bits) that is just enough. I added another sentence at the end of the first paragraph.}
\end{proof}

\subsubsection{Properties of the Mechanism}
The mechanism always outputs a proportional allocation by definition.
It is straightforward to check that it outputs a Pareto-efficient allocation.
\begin{propositionrep}
The $\Gamma$-$\Psi$ mechanism for cake-cutting always returns a Pareto-efficient allocation.
\end{propositionrep}
\begin{proof}
Suppose for the sake of contradiction that $(A_1,\ldots,A_n)$ output by the mechanism is Pareto-dominated by $(A_1',\ldots,A_n')$, i.e., we have
\begin{enumerate}
    \item $V_i(A_i')\geq V_i(A_i)$ for each agent $a_i$, and
    \item for at least one agent $a_i$, $V_i(A_i')> V_i(A_i)$.
\end{enumerate}
Property (1) above ensures $(A_1',\ldots,A_n')$ is proportional and thus is also in $\propallocations$: for each $i\in[n]$, $V_i(A_i')\geq V_i(A_i)\geq \frac1nV_i([0,1])$ (as the allocation $(A_1,\ldots,A_n)$ is proportional).
Based on the property (2), find the smallest index $i$ such that $V_{\pi_i}(A_i')>V_{\pi_i}(A_i)$.
We see that $(A_1,\ldots,A_n)$ does not maximize the $i$-th agent in the order $\pi$, which contradicts the definition of the mechanism.
\end{proof}

It then remains to show that the mechanism has RAT-degree $n-1$.
We need the following proposition; it follows from known results on super-proportional cake-cutting \citep{dubins1961cut,woodall1986note}; for completeness we provide a proof in the appendix.

\begin{propositionrep} \label{prop:strictlymorethanproportional}
    Let $\propallocations$ be the set of all proportional allocations for the valuation profile $(v_1,\ldots,v_n)$.
    Let $(A_1,\ldots,A_n)$ be the allocation in $\propallocations$ that maximizes agent $a_i$'s utility.
    If there exists $j\in[n]\setminus\{i\}$ such that $v_i$ and $v_j$ are not identical up to scaling, then $V_i(A_i)>\frac1nV_i([0,1])$.
\end{propositionrep}
\begin{proof}
    We will explicitly construct a proportional allocation $(B_1,\ldots,B_n)$ where $V_i(B_i)>\frac1nV_i([0,1])$ if the pre-condition in the statement is satisfied.
    Notice that this will imply the proposition, as we are finding the allocation maximizing $a_i$'s utility.
    To construct such an allocation, we assume $v_i$ and $v_j$ are normalized without loss of generality (then $v_i\neq v_j$), and consider the equal division allocation where each uniform segment $X_t$ is evenly divided.
    This already guarantees that agent $a_i$ receives a value of $\frac1nV_i([0,1])$.
    Since $v_i$ and $v_j$ are normalized and $v_i\neq v_j$, there exist two uniform segments $X_{t_1}$ and $X_{t_2}$ such that $v_i(X_{t_1})>v_j(X_{t_1})$ and $v_i(X_{t_2})<v_j(X_{t_2})$.
    Agent $a_i$ and $a_j$ can then exchange parts of their allocations on $X_{t_1}$ and $X_{t_2}$ to improve the utility for both of them, which guarantees the resultant allocation is still proportional.
    For example, set $\varepsilon>0$ be a very small number.
    Agent $a_i$ can give a length of $\frac{\varepsilon}{v_i(X_{t_2})+v_j(X_{t_2})}$ from $X_{t_2}$ to agent $a_j$, in exchange of a length of $\frac{\varepsilon}{v_i(X_{t_1})+v_j(X_{t_1})}$ from $X_{t_1}$.
    This describes the allocation $(B_1,\ldots,B_n)$.
\end{proof}

The proof of the following theorem is similar to the one for indivisible goods (\Cref{thm:gamma-psi-indivisible}). 
\begin{theoremrep}
    The $\Gamma$-$\Psi$ mechanism for cake-cutting has RAT-degree $n-1$.
\end{theoremrep}


\begin{proof}
Consider an arbitrary agent $a_i$ with the true value density function $v_i$, and an arbitrary agent $a_j$ whose reported value density function is unknown to $a_i$.
Fix $n-2$ arbitrary value density function $\{v_k\}_{k\notin\{i,j\}}$ for the remaining $n-2$ agents.
Consider an arbitrary manipulation $v_i'\neq v_i$.

Choose a uniform segment $X_t$ with respect to $(v_1,\ldots,v_{j-1},v_{j+1},\ldots,v_n)$,
satisfying $v_i(X_t)>0$.
Choose a very small interval $E\subseteq X_t$, such that the value density function
$$\bar{v}_j=\left\{\begin{array}{ll}
    0 & \mbox{if }x\in E \\
    v_i(x) & \mbox{otherwise}
\end{array}\right.$$
is not a scaled version of some $v_k$ with $k\in[n]\setminus\{i,j\}$.
% \erel{I did not understand this sentence. Do you mean: the normalized $\bar{v}_j$ not equal to any normalized $v_k$?}\biaoshuai{yes, to *some* $v_k$ is sufficient}
Apply the volatility of $\Gamma$ to find a value density function $v_j$ for agent $a_j$ that rescales $\bar{v}_j$ such that
\begin{enumerate}
    \item when agent $a_i$ reports $v_i$, agent $a_i$ is the first in the order output by $\Gamma$;
    \item when agent $a_i$ reports $v_i'$, agent $a_j$ is the first in the order output by $\Gamma$.
\end{enumerate}

Let $(A_1,\ldots,A_n)$ and $(A_1',\ldots,A_n')$ be the output allocation for the profiles $\{v_k\}_{k\notin\{i,j\}}\cup\{v_i\}\cup\{v_j\}$ and $\{v_k\}_{k\notin\{i,j\}}\cup\{v_i'\}\cup\{v_j\}$ respectively.
Since $\bar{v}_j$ is not a scaled version of some $v_k$, its rescaled version $v_j$ is also different.
By Proposition~\ref{prop:strictlymorethanproportional}, $V_j(A_j')>\frac1nV_j([0,1])$,
as $a_j$ is the highest-priority agent when $a_i$ reports $v'_i$.
Let $D$ be some subset of $A_j'$ with $V_j(D)>0$ and $V_j(A_j'\setminus D)\geq\frac1nV_j([0,1])$, and consider the allocation $(A_1^+,\ldots,A_n^+)$ in which $D$ is moved from $a_j$ to $a_i$, that is,
\begin{itemize}
    \item for $k\notin\{i,j\}$, $A_k^+=A_k'$;
    \item $A_i^+=A_i'\cup D$;
    \item $A_j^+=A_j'\setminus D$.
\end{itemize}
It is clear by our construction that the new allocation is still proportional with respect to $\{v_k\}_{k\notin\{i,j\}}\cup\{v_i'\}\cup\{v_j\}$.
In addition, by the relation between $\bar{v}_j$ and $v_i$ (and thus the relation between $v_j$ and $v_i$), we have $V_i(D)>0$ based on agent $a_i$'s true value density function $v_i$.
Therefore, under agent $a_i$'s true valuation, $V_i(A_i^+)>V_i(A_i')$.

If the allocation $(A_1^+,\ldots,A_n^+)$ is not proportional under the profile $\{v_k\}_{k\notin\{i,j\}}\cup\{v_i\}\cup\{v_j\}$ (where $v_i'$ is changed to $v_i$), then the only agent for whom proportionality is violated must be agent $i$, that is,$V_i(A_i^+)<\frac1nV_i([0,1])$.
It then implies $V_i(A_i')<\frac1nV_i([0,1])$.
On the other hand, agent $a_i$ receives at least her proportional share when reporting truthfully her value density function $v_i$.
This already implies the manipulation is not safe.

If the allocation $(A_1^+,\ldots,A_n^+)$ is proportional under the profile $\{v_k\}_{k\notin\{i,j\}}\cup\{v_i\}\cup\{v_j\}$, then it is in $\propallocations$.
Since agent $a_i$ is the first agent in the order when reporting $v_i$ truthfully, we have $V_i(A_i)\geq V_i(A_i^+)$, which further implies $V_i(A_i)>V_i(A_i')$.
Again, the manipulation is not safe.
\end{proof}

Finally, we analyze the run-time of our mechanism.
\begin{propositionrep}
The $\Gamma$-$\Psi$ mechanism for cake-cutting can be computed in polynomial time.
\end{propositionrep}
\begin{proof}
We first note that $\Gamma$ can be computed in polynomial time.
Finding $v^\ast$ and reading the information of $i,t,a$, and $b$ can be performed in linear time, as it mostly only requires reading the input of the instance.
In particular, the lengths of $a$ and $b$ are both less than the input length, so $as+b$ is of at most linear length and can also be computed in linear time.
Finally, the length of $n!$ is $\Theta(n\log n)$, so $as+b \bmod (n!)$ can be computed in polynomial time.
We conclude that $\Gamma$ can be computed in polynomial time.

We next show that $\Psi$ can be computed by solving linear programs.
Let $x_{it}$ be the length of the $t$-th uniform segment allocated to agent $a_i$.
Then an agent $a_i$'s utility is a linear expression $\sum_{t=1}^mv_{i}(X_t)x_{it}$, and requiring an agent's utility is at least some value (e.g., her proportional share) is a linear constraint.
We can use a linear program to find the maximum possible utility $u_{\pi_1}^\ast$ for agent $\pi_1$ among all proportional allocations.
In the second iteration, we write the constraint $\sum_{t=1}^mv_{\pi_1}(X_t)x_{it}\geq u_{\pi_1}^\ast$ for agent $\pi_1$, the proportionality constraints for the $n-2$ agents $[n]\setminus\{\pi_1,\pi_2\}$, and maximize agent $\pi_2$'s utility.
This can be done by another linear program and gives us the maximum possible utility $u_{\pi_2}^\ast$ for agent $\pi_2$.
We can iteratively do this to figure out all of $u_{\pi_1}^\ast,u_{\pi_2}^\ast,\ldots,u_{\pi_n}^\ast$ by linear programs. 
\end{proof}

\subsection{Towards An Envy-Free and Pareto-Optimal Mechanism with RAT-degree $n-1$}
Given the result in the previous section, it is natural to ask if the fairness guarantee can be strengthened to envy-freeness.
A compelling candidate is the mechanism that always outputs allocations with maximum \emph{Nash welfare}.
The Nash welfare of an allocation $(A_1,\ldots,A_n)$ is defined by the product of agents utilities:
$\displaystyle \prod_{i=1}^nV_i(A_i).$

It is well-known that such an allocation is envy-free and Pareto-optimal.
However, computing its RAT-degree turns out to be very challenging for us.
We conjecture the answer is $n-1$.
\begin{open}
    What is the RAT-degree of the maximum Nash welfare mechanism?
\end{open}
% \newpage
\section{Single-Winner Ranked Voting}\label{sec:single-winner-voting}
We consider $n$ voters (the agents) who need to elect one winner from a set $C$ of $m$ \emph{candidates}.
The agents' preferences are given by strict linear orderings $\succ_i$ over the candidates.

When there are only two candidates, the majority rules and its variants (weighted majority rules)  are truthful.
With three or more candidates, the   Gibbard--Satterthwaite Theorem implies that the only truthful rules are dictatorships. 
Our goal is to find non-dictatorial rules with a high RAT-degree.

%\eden{if we have time: maybe to change 'Alice' to 'voter $v_1$'} \erel{I think ``agent'' is better as it is more consistent with the rest of the paper.}

Throughout the analysis, we consider a specific agent Alice, who looks for a safe profitable manipulation. Her true ranking is $c_m \succ_A \cdots \succ_A c_1$.
We assume that, for any $j>i$, Alice strictly prefers a victory of $c_j$ to a tie between $c_j$ and $c_i$, and strictly prefers this tie to a victory of $c_i$.%
\footnote{We could also assume that ties are broken at random, but this would require us to define preferences on lotteries, which we prefer to avoid in this paper.}




\subsection{Positional voting rules: general upper bound}
\newcommand{\scorevector}{\mathbf{s}}
\newcommand{\score}{\operatorname{score}}
A \emph{positional voting rule}
is parameterized by a vector of scores, $\scorevector=(s_1,\ldots,s_m)$, where $s_1\leq \cdots \leq  s_m$ and $s_1 < s_m$.
Each voter reports his entire ranking of the $m$ candidates. Each such ranking is translated to an assignment of a score to each candidate: the lowest-ranked candidate is given a score of $s_1$, the second-lowest candidate is given $s_2$, etc., and the highest-ranked candidate is given a score of $s_m$. 
The total score of each candidate is the sum of scores he received from the rankings of all $n$ voters. The winner is the candidate with the highest total score. 

Formally, for any subset $N'\subseteq N$ and any candidate $c\in C$, we denote by $\score_{N'}(c)$ the total score that $c$ receives from the votes of the agents in $N'$. Then the winner is $\arg\max_{c\in C}\score_N(c)$. If there are several agents with the same maximum score, then the outcome is considered a tie.

Common special cases of positional voting are \emph{plurality voting}, in which $\scorevector = (0,0,0,\ldots,0,1)$, and 
\emph{anti-plurality voting}, in which $\scorevector = (0,1,1,\ldots,1,1)$.
By the Gibbard--Satterthwaite theorem, all positional voting rules are manipulable, so their RAT-degree is smaller than $n$.
But, as we will show next, some positional rules have a higher RAT-degree than others.


In the upcoming lemmas, we identify the manipulations that are safe and profitable for Alice under various conditions on the score vector $\scorevector$. We assume throughout that there are $m\geq 3$ candidates, and that $n$ is sufficiently large.
We allow an agent to abstain, which means that his vote gives the same score to all candidates.%
\footnote{
We need the option to abstain in order to avoid having different constructions for even $n$ and odd $n$; see the proofs for details.
}

\begin{lemmarep}
\label{lem:sm>sm1}
Let $m\geq 3$ and $n\geq 4$.
If $s_m > s_{m-1}$
and there are $k\geq \ceil{n/2}+1$ known agents,
then
switching the top two candidates ($c_m$ and $c_{m-1}$) may be a safe profitable manipulation for Alice.
\end{lemmarep}
\begin{proofsketch}
For some votes by the known agents, the manipulation is safe since $c_m$ has no chance to win, and it is profitable as it may help $c_{m-1}$ win over worse candidates.
\end{proofsketch}
\begin{proof}
Suppose there is a subset $K$ of $\ceil{n/2}+1$ known agents, who vote as follows:
\begin{itemize}
\item $\floor{n/2}-1$ known agents rank $c_{m-2} \succ c_{m-1} \succ \cdots \succ c_m$.
\item One known agent ranks $c_{m-2} \succ c_m \succ
c_{m-1} \succ \cdots $. 
\item One known agent ranks $c_{m-1} \succ c_{m-2} \succ \cdots \succ c_m$. 
\item In case $n$ is odd, the remaining known agent abstains.
\end{itemize}
We first show that $c_m$ cannot win. To this end, we show that the difference in scores between $c_{m-2}$ and $c_m$ is always strictly positive.
\begin{itemize}
\item The difference in scores given by the known agents is 
\begin{align*}
\score_K(c_{m-2})-\score_K(c_m) =
&
(\floor{n/2}-1)(s_m-s_1) 
+ (s_m-s_{m-1})
+ (s_{m-1}-s_1).
%+ (s_{m-1}-s_1) 
%+ (s_m-s_2)
\\
=&
(\floor{n/2})(s_m-s_1) 
%(s_m-s_1)
%+ 
%(s_{m-1}-s_2).
\end{align*}
\item There are
$\floor{n/2}-1$ agents not in $K$ (including Alice).
These agents can reduce the score-difference by at most 
$(\floor{n/2}-1)(s_m-s_1)$.
Therefore, 
\begin{align*}
\score_N(c_{m-2})-\score_N(c_m) \geq (s_m-s_1),
\end{align*}
which is positive for any score vector.
So $c_m$ has no chance to win or even tie.
\end{itemize}
Therefore, switching $c_m$ and $c_{m-1}$ can never harm Alice --- the manipulation is safe.

Next, we show that the manipulation can help $c_{m-1}$ win. We compute the score-difference between $c_{m-1}$ and the other candidates with and without the manipulation. 

Suppose that the agents not in $K$ vote as follows:
\begin{itemize}
\item the $\floor{n/2}-2$ unknown agents%
\footnote{Here we use the assumption $n\geq 4$.}
rank $c_{m-1}\succ c_{m-2}\succ \cdots $.
\item Alice votes truthfully $c_m\succ c_{m-1}\succ c_{m-2} \cdots \succ c_1$.
\end{itemize}
Then,
\begin{align*}
\score_N(c_{m-2}) - \score_N(c_{m-1})
=&
(\floor{n/2}-1)(s_{m}-s_{m-1}) 
+ (s_m-s_{m-2})
+ (s_{m-1}-s_{m})
\\
&+
(\floor{n/2}-2)(s_{m-1}-s_{m}) 
+ (s_{m-2}-s_{m-1})
\\
=&
(s_{m}-s_{m-1}),
\end{align*}
which is positive by the assumption $s_m>s_{m-1}$.
The candidates $c_{j<m-2}$ are ranked even lower than $c_{m-1}$ by all agents. Therefore the winner is $c_{m-2}$.

If Alice switches $c_{m-1}$ and $c_m$, then the score of $c_{m-1}$ increases by $s_m-s_{m-1}$ and the scores of all other candidates except $c_m$ do not change. Therefore, 
$\score_N(c_{m-2}) - \score_N(c_{m-1})$ becomes $0$, and there is a tie between $c_{m-2}$ and $c_{m-1}$, which is better for Alice by assumption.
Therefore, the manipulation is profitable.

\iffalse % EREL: old proof, without abstinence
Specifically, suppose the $\ceil{n/2}+1$ known-agents rank as follows:
\begin{itemize}
\item $\floor{n/2}-1$ known agents rank $c_1 \succ c_{m-1} \succ \cdots \succ c_m$;
\item One known agent ranks $c_{m-1} \succ c_1 \succ \cdots \succ c_m$. 
\item In case $n$ is odd, another known agent ranks $c_{m-1} \succ c_1 \succ \cdots \succ c_m$.
\item One known agent ranks $c_1 \succ  \cdots \succ  c_m \succ c_{m-1}$;
\end{itemize}
We first show that $c_m$ cannot win. To this end, we show that the difference in scores between $c_1$ and $c_m$ is always strictly positive.
\begin{itemize}
\item The difference in scores given by the known agents is 
\begin{align*}
&
(\floor{n/2}-1)(s_m-s_1) + (1+n\bmod 2)(s_{m-1}-s_1) + (s_m-s_2)
\\
=&
(\floor{n/2}-1)(s_m-s_1) + 
(s_m-s_1)
+ 
(s_{m-1}-s_2)
+
(n\bmod 2)(s_{m-1}-s_1)
\end{align*}
\item There are
$\floor{n/2}-1$ agents not in $K$ (including Alice).
These agents can reduce the score-difference by at most 
$(\floor{n/2}-1)(s_m-s_1)$.
Therefore, the score-difference is at least 
$(s_m-s_1)
+ 
(s_{m-1}-s_2)
+
(n\bmod 2)(s_{m-1}-s_1)$,
which is positive for any score-vector when $m\geq 3$. So $c_m$ has no chance to win or even tie for victory.
\end{itemize}
Therefore, switching $c_m$ and $c_{m-1}$ can never harm Alice --- the manipulation is safe.

Next, we show that the manipulation can help $c_{m-1}$ win. To this end, we show that the manipulation may affect the score-difference between $c_{m-1}$ to $c_1$.

We assume that the $\floor{n/2}-2$ unknown agents rank $c_{m-1}\succ c_1\succ \cdots $. Then, without Alice's vote, the score-difference is
\begin{align*}
&
(\floor{n/2}-1)(s_{m-1}-s_m) + (1+n\bmod 2)(s_m-s_{m-1}) + (s_1-s_m)
+
(\floor{n/2}-2)(s_m-s_{m-1})
\\
=&
(n\bmod 2)(s_m-s_{m-1}) 
+ (s_1-s_m).
\end{align*}
Alice's vote can affect the outcome in the following way:
\begin{itemize}
\item If Alice votes truthfully $c_m\succ c_{m-1}\succ \cdots \succ c_1$, then the vote adds $s_{m-1}-s_1$ to the difference, which becomes 
$(n\bmod 2-1)(s_m - s_{m-1})$.

\item If Alice switches $c_{m-1}$ and $c_m$, then 
the vote adds $s_m - s_1$ to the difference, so the difference becomes $(n\bmod 2)(s_m - s_{m-1})$.
\end{itemize}
When $n$ is even, the manipulation changes the difference from $s_{m-1}-s_m$, which is negative by the assumption $s_m > s_{m-1}$, to $0$; so the manipulation replaces a victory for $c_1$ with a tie between $c_{m-1}$ and $c_1$, which by assumption is better for Alice.

When $n$ is odd, the manipulation changes the difference from $0$ to $s_{m}-s_1$, which is positive by the assumption $s_m > s_{m-1}$, so the manipulation replaces a tie between $c_{m-1}$ and $c_1$ with a victory for $c_{m-1}$, which is again better for Alice. In both cases, the manipulation is profitable.
\fi
\end{proof}

%The following lemma is a special case of \Cref{lem:st1>st}; we prove it explicitly as a warm-up. 
\begin{lemmarep}
\label{lem:s2>s1}
Let $m\geq 3$ and $n\geq 2m$.
If $s_2 > s_1$
and there are $k\geq \ceil{n/2}+1$ known agents,
then
switching the bottom two candidates ($c_2$ and $c_1$) may be a safe profitable manipulation for Alice.
\end{lemmarep}
\begin{proofsketch}
For some votes by the known agents,
$c_1$ has no chance to win, so the worst candidate for Alice that could win is $c_2$. Therefore, switching $c_1$ and $c_2$ cannot harm, but may help a better candidate win over $c_2$.
\end{proofsketch}

\iffalse % EREL: tried an alternative proof --- using n-k
\begin{proof}
Suppose there is a subset $K$ of $k$ known agents, who vote as follows (Note that the assumption on $k$ implies $k\geq n-k+2$):
\begin{itemize}
\item $n-k$ known agents rank $c_2 \succ c_m \succ \cdots \succ c_1$.
\item Two known agents rank $c_m \succ c_2 \succ
\cdots \succ c_1$.
\item The remaining known agents (if any) abstain.
\end{itemize}


We first show that $c_1$ cannot win. To this end, we show that the difference in scores between $c_2$ and $c_1$ is always strictly positive.
\begin{itemize}
\item The difference in scores given by the known agents is 
\begin{align*}
\score_K(c_2)-\score_K(c_1) =
&
(n-k)(s_m-s_1) 
+ 2(s_{m-1}-s_1).
\end{align*}
\item The $n-k$ agents not in $K$ (including Alice) 
can reduce the score-difference by at most 
$(n-k)(s_m-s_1)$.
Therefore, 
\begin{align*}
\score_N(c_2)-\score_N(c_1) 
\geq  &
(n-k)(s_m-s_1) 
+ 2(s_{m-1}-s_1)
-(n-k)(s_m-s_1)
\\
= &
2(s_{m-1} - s_1),
\end{align*}
which is positive 
by the assumption $s_2>s_1$.
So $c_1$ has no chance to win or even tie.
\end{itemize}
Therefore, switching $c_2$ and $c_1$ can never harm Alice --- the manipulation is safe.

Next, we show that the manipulation can help $c_m$ win, when the agents not in $K$ vote as follows:
\begin{itemize}
\item $n-k-2$ unknown agents rank $c_m\succ c_2\succ \cdots $,
where each candidate except $c_1,c_2,c_m$ is ranked last by at least one voter (here we use the assumption $n\geq 2m$).
\item One unknown agent ranks 
$c_2\succ \cdots \succ c_m \succ c_1$;
\item Alice votes truthfully $c_m\succ  \cdots \succ c_2 \succ c_1$.  
\end{itemize}
Then,
\begin{align*}
\score_N(c_2) - \score_N(c_m)
=&
(n-k)(s_{m}-s_{m-1}) 
+ 2(s_{m-1}-s_{m})
\\
&+
(n-k-2)(s_{m-1}-s_{m}) 
+ (s_m-s_2)
+ (s_2-s_m)
\\
=&
0.
\end{align*}
Moreover, for any $j\not\in\{1,2,m\}$, the score of $c_j$ is even lower (here we use the assumption that $c_j$ is ranked last by at least one unknown agent):
\begin{align*}
\score_N(c_2) - \score_N(c_j)
\geq &
(n-k)(s_{m}-s_{m-2}) 
+ 2(s_{m-1}-s_{m-2})
\\
&+
(n-k-3)(s_{m-1}-s_{m-2}) 
+ (s_{m-1}-s_1)
+ (s_m-s_{m-1})
+ (s_2-s_{m-1})
\\
\geq & (s_{m-1}-s_{1})
+ (s_2-s_{m-1})
\\
= & s_2 - s_1,
\end{align*}
which is positive by the lemma assumption.
Therefore, when Alice is truthful, the outcome is a tie between $c_m$ and $c_2$.

If Alice switches $c_1$ and $c_2$, then the score of $c_2$ decreases by $s_2-s_1$, which is positive by the lemma assumption, and the scores of all other candidates except $c_1$ do not change. So $c_m$ wins, which is better for Alice than a tie.
Therefore, the manipulation is profitable.
\end{proof}
\fi % alternative proof - using n-k


\begin{proof}
Suppose there is a subset $K$ of $\ceil{n/2}+1$ known agents, who vote as follows:
\begin{itemize}
\item $\floor{n/2}-1$ known agents rank $c_2 \succ c_m \succ \cdots \succ c_1$.
\item Two known agents rank $c_m \succ c_2 \succ
\cdots \succ c_1$. 
\item In case $n$ is odd, the remaining known agent abstains.
\end{itemize}
We first show that $c_1$ cannot win. To this end, we show that the difference in scores between $c_2$ and $c_1$ is always strictly positive.
\begin{itemize}
\item The difference in scores given by the known agents is 
\begin{align*}
\score_K(c_2)-\score_K(c_1) =
&
(\floor{n/2}-1)(s_m-s_1) 
+ 2(s_{m-1}-s_1).
\end{align*}
\item There are
$\floor{n/2}-1$ agents not in $K$ (including Alice).
These agents can reduce the score-difference by at most 
$(\floor{n/2}-1)(s_m-s_1)$.
Therefore, 
\begin{align*}
\score_N(c_2)-\score_N(c_1) \geq 2(s_{m-1}-s_1),
\end{align*}
which is positive 
by the assumption $s_2>s_1$.
So $c_1$ has no chance to win or even tie.
\end{itemize}
Therefore, switching $c_2$ and $c_1$ can never harm Alice --- the manipulation is safe.

Next, we show that the manipulation can help $c_m$ win, when the agents not in $K$ vote as follows:
\begin{itemize}
\item $\floor{n/2}-3$ unknown agents rank $c_m\succ c_2\succ \cdots $,
where each candidate except $c_1,c_2,c_m$ is ranked last by at least one voter (here we use the assumption $n\geq 2m$).
\item One unknown agent ranks 
$c_2\succ \cdots \succ c_m \succ c_1$;
\item Alice votes truthfully $c_m\succ  \cdots \succ c_2 \succ c_1$.  
\end{itemize}
Then,
\begin{align*}
\score_N(c_2) - \score_N(c_m)
=&
(\floor{n/2}-1)(s_{m}-s_{m-1}) 
+ 2(s_{m-1}-s_{m})
\\
&+
(\floor{n/2}-3)(s_{m-1}-s_{m}) 
+ (s_m-s_2)
+ (s_2-s_m)
\\
=&
0.
\end{align*}
Moreover, for any $j\not\in\{1,2,m\}$, the score of $c_j$ is even lower (here we use the assumption that $c_j$ is ranked last by at least one unknown agent):
\begin{align*}
\score_N(c_2) - \score_N(c_j)
\geq &
(\floor{n/2}-1)(s_{m}-s_{m-2}) 
+ 2(s_{m-1}-s_{m-2})
\\
&+
(\floor{n/2}-4)(s_{m-1}-s_{m-2}) 
+ (s_{m-1}-s_1)
+ (s_m-s_{m-1})
+ (s_2-s_{m-1})
\\
\geq & (s_{m-1}-s_{1})
+ (s_2-s_{m-1})
\\
= & s_2 - s_1,
\end{align*}
which is positive by the lemma assumption.
Therefore, when Alice is truthful, the outcome is a tie between $c_m$ and $c_2$.

If Alice switches $c_1$ and $c_2$, then the score of $c_2$ decreases by $s_2-s_1$, which is positive by the lemma assumption, and the scores of all other candidates except $c_1$ do not change. So $c_m$ wins, which is better for Alice than a tie.
Therefore, the manipulation is profitable.
\end{proof}

\iffalse
\begin{lemmarep}
Let $m\geq 4$ and $n\geq 2m$.
If $s_3 > s_2 = s_1$,
and there are $k\geq \ceil{n/2}+1$ known agents,
then
switching $c_3$ and $c_2$ may be a safe profitable manipulation for Alice.
\end{lemmarep}
\begin{proofsketch}
The idea is that, for some votes by the known agents, both $c_2$ and $c_1$ have has no chance to win, 
so the worst candidate that could win is $c_3$.
Therefore, switching $c_2$ and $c_3$ cannot harm, but can help better candidates win over $c_3$.
\end{proofsketch}
\begin{proof}
Suppose there is a subset $K$ of $\ceil{n/2}+1$ known agents, who vote as follows:
\begin{itemize}
\item $\floor{n/2}-1$ known agents rank $c_3 \succ c_m \succ \cdots \succ c_2 \succ c_1$.
\item Two known agents rank $c_m \succ c_3 \succ
\cdots \succ c_2 \succ c_1$. 
\item In case $n$ is odd, the remaining known agent abstains.
\end{itemize}
We first show that both $c_1$ and $c_2$ cannot win. 
Note that, by the lemma assumption $s_1=s_2$, these two candidates receive exactly the same score by all known agents. We show that the difference in scores between $c_3$ and both $c_2$ and $c_1$ is always strictly positive.
\begin{itemize}
\item The difference in scores given by the known agents is 
\begin{align*}
\score_K(c_3)-\score_K(c_2) =
&
(\floor{n/2}-1)(s_m-s_1) 
+ (s_{m-1}-s_1).
\\
%=&
%(\floor{n/2})(s_m-s_1) 
\end{align*}
\item There are
$\floor{n/2}-1$ agents not in $K$ (including Alice).
These agents can reduce the score-difference by at most 
$(\floor{n/2}-1)(s_m-s_1)$.
Therefore, 
\begin{align*}
\score_N(c_3)-\score_N(c_2) \geq (s_{m-1}-s_1),
\end{align*}
which is positive 
by the assumptions $m\geq 4$ and $s_3>s_2$.
So both $c_2$ and $c_1$ have no chance to win or even tie.
\end{itemize}
Therefore, switching $c_3$ and $c_2$ can never harm Alice --- the manipulation is safe.

Next, we show that the manipulation can help $c_m$ win. We compute the score-difference between $c_m$ and the other candidates with and without the manipulation. 

Suppose that the agents not in $K$ vote as follows:
\begin{itemize}
\item $\floor{n/2}-3$ unknown agents rank $c_m\succ c_3\succ \cdots $,
where each candidate except $c_1,c_2,c_3,c_m$ is ranked last by at least one voter (here we use the assumption $n\geq 2m$).
\item One unknown agent ranks 
$c_3\succ \cdots \succ c_m \succ c_2 \succ c_1$;
\item Alice votes truthfully $c_m\succ  \cdots \succ c_3 \succ c_2 \succ c_1$.  
\end{itemize}
Then,
\begin{align*}
\score_N(c_3) - \score_N(c_m)
=&
(\floor{n/2}-1)(s_{m}-s_{m-1}) 
+ 2(s_{m-1}-s_{m})
\\
&+
(\floor{n/2}-3)(s_{m-1}-s_{m}) 
+ (s_m-s_3)
+ (s_3-s_m)
\\
=&
0.
\end{align*}
Moreover, for any $j\not\in\{1,2,3,m\}$, the score of $c_j$ is even lower (here we use the assumption that $c_j$ is ranked last by at least one unknown agent):
\begin{align*}
\score_N(c_3) - \score_N(c_j)
\geq &
(\floor{n/2}-1)(s_{m}-s_{m-2}) 
+ 2(s_{m-1}-s_{m-2})
\\
&+
(\floor{n/2}-4)(s_{m-1}-s_{m-2}) 
+ (s_{m-1}-s_1)
+ (s_m-s_{m-1})
+ (s_3-s_{m-1})
\\
\geq & (s_{m-1}-s_{1})
+ (s_3-s_{m-1})
\\
= & s_3 - s_1,
\end{align*}
which is positive by the lemma assumption.
Therefore, when Alice is truthful, the outcome is a tie between $c_m$ and $c_3$.

If Alice switches $c_2$ and $c_3$, then the score of $c_3$ decreases by $s_3-s_2$, which is positive by the lemma assumption, and the scores of all other candidates except $c_2$ do not change. So $c_m$ wins, which is better for Alice than a tie.
Therefore, the manipulation is profitable.
\end{proof}
\fi

\Cref{lem:s2>s1} can be generalized as follows. 

\begin{lemmarep}
\label{lem:st1>st}
Let $m\geq 3$ and $n\geq 2m$.
For every integer $t \in \{1,\ldots, m-2\}$,
if $s_{t+1} > s_t = \cdots = s_1$,
and there are $k\geq \ceil{n/2}+1$ known agents,
then switching $c_{t+1}$ and $c_t$ may be a safe profitable manipulation.
\end{lemmarep}
%Note that \Cref{lem:s2>s1} is the special case $t=1$.

\begin{proofsketch}
For some votes by the known agents,
all candidates $c_1,\ldots,c_t$ have no chance to win, 
so the worst candidate for Alice that could win is $c_{t+1}$.
Therefore, switching $c_t$ and $c_{t+1}$ cannot harm, but can help better candidates win over $c_{t+1}$.
\end{proofsketch}
\begin{proof}
Suppose there is a subset $K$ of $\ceil{n/2}+1$ known agents, who vote as follows:
\begin{itemize}
\item $\floor{n/2}-1$ known agents rank $c_{t+1} \succ c_m$ first and rank $c_t \succ \cdots \succ c_1$ last.
\item Two known agents rank $c_m \succ c_{t+1}$ first and rank $c_t \succ \cdots \succ c_1$ last.
\item In case $n$ is odd, the remaining known agent abstains.
\end{itemize}
We first show that the $t$ worst candidates for Alice ($c_1,\ldots, c_t$) cannot win. 
Note that, by the lemma assumption $s_t = \cdots = s_1$, all these candidates receive exactly the same score by all known agents. We show that the difference in scores between $c_{t+1}$ and $c_t$ (and hence all $t$ worst candidates) is always strictly positive.
\begin{itemize}
\item The difference in scores given by the known agents is 
\begin{align*}
\score_K(c_{t+1})-\score_K(c_t) =
&
(\floor{n/2}-1)(s_m-s_1) 
+ (s_{m-1}-s_1).
\end{align*}
\item There are
$\floor{n/2}-1$ agents not in $K$ (including Alice).
These agents can reduce the score-difference by at most 
$(\floor{n/2}-1)(s_m-s_1)$.
Therefore, 
\begin{align*}
\score_N(c_{t+1})-\score_N(c_t) \geq (s_{m-1}-s_1),
\end{align*}
which is positive 
by the assumption $m-2 \geq t$ and $s_{t+1}>s_t$.
So no candidate in $c_1,\ldots,c_t$ has a chance to win or even tie.
\end{itemize}
Therefore, switching $c_{t+1}$ and $c_t$ can never harm Alice --- the manipulation is safe.

Next, we show that the manipulation can help $c_m$ win. We compute the score-difference between $c_m$ and the other candidates with and without the manipulation. 

Suppose that the agents not in $K$ vote as follows:
\begin{itemize}
\item $\floor{n/2}-3$ unknown agents rank $c_m\succ c_{t+1}\succ \cdots $,
where each candidate in $c_{t+2},\ldots,c_{m-1}$ is ranked last by at least one voter (here we use the assumption $n\geq 2m$).
\item One unknown agent ranks 
$c_{t+1}\succ \cdots \succ c_m \succ c_t \succ \cdots \succ c_1$;
\item Alice votes truthfully $c_m\succ  \cdots \succ c_{t+1} \succ  c_t \succ \cdots \succ c_1$.  
\end{itemize}
Then,
\begin{align*}
\score_N(c_{t+1}) - \score_N(c_m)
=&
(\floor{n/2}-1)(s_{m}-s_{m-1}) 
+ 2(s_{m-1}-s_{m})
\\
&+
(\floor{n/2}-3)(s_{m-1}-s_{m}) 
+ (s_m-s_{t+1})
+ (s_{t+1}-s_m)
\\
=&
0.
\end{align*}
Moreover, for any $j\in\{t+2,\ldots,m-1\}$, the score of $c_j$ is even lower (here we use the assumption that $c_j$ is ranked last by at least one unknown agent):
\begin{align*}
\score_N(c_{t+1}) - \score_N(c_j)
\geq &
(\floor{n/2}-1)(s_{m}-s_{m-2}) 
+ 2(s_{m-1}-s_{m-2})
\\
&+
(\floor{n/2}-4)(s_{m-1}-s_{m-2}) 
+ (s_{m-1}-s_1)
+ (s_m-s_{m-1})
+ (s_{t+1}-s_{m-1})
\\
\geq & (s_{m-1}-s_{1})
+ (s_{t+1}-s_{m-1})
\\
= & s_{t+1} - s_1,
\end{align*}
which is positive by the lemma assumption.
Therefore, when Alice is truthful, the outcome is a tie between $c_m$ and $c_{t+1}$.

If Alice switches $c_t$ and $c_{t+1}$, then the score of $c_{t+1}$ decreases by $s_{t+1}-s_t$, which is positive by the lemma assumption, and the scores of all other candidates except $c_t$ do not change. As $c_t$ cannot win, $c_m$ wins, which is better for Alice than a tie.
Therefore, the manipulation is profitable.
\end{proof}

Combining the lemmas leads to an upper bound on the RAT-degree of any positional voting rule:
\begin{theorem}
\label{thm:upper-positional}
The RAT-degree of any positional voting rule for $m\geq 3$ candidates and $n\geq 2m$ agents is at most $\ceil{n/2}+1$.
\end{theorem}
\begin{proof}
Consider a positional voting rule with score-vector $\scorevector$. Let $t \in \{1,\ldots,m-1\}$ be the smallest index for which $s_{t+1} > s_t$ (there must be such an index by definition of a score-vector).

If $t\leq m+2$, then \Cref{lem:st1>st} implies that, for some votes by the $\ceil{n/2}+1$ known agents, switching $c_{t+1}$ and $c_t$ may be a safe and profitable manipulation for Alice.

Otherwise, $t=m-1$, and \Cref{lem:sm>sm1} implies the same.

In all cases, Alice has a safe profitable manipulation.
\end{proof}

Next, we show that the \emph{plurality voting rule}, which is the positional voting rule with score-vector $(0,0,0,\ldots,0,1)$, attains the upper bound of \Cref{thm:upper-positional}
(at least when $n$ is even).


\begin{lemmarep}
\label{lem:lower-plurality}
Let $n\geq 4$ be an even number.
For the plurality voting rule,  if there are at most $n/2$ known agents, then Alice has no safe profitable manipulation.
\end{lemmarep}

\begin{proof}
For any potential manipulation by Alice
we have to prove that, 
for any set $K$ of $n/2$ agents and any combination of votes by the agents of $K$, 
either (1) the manipulation is not profitable 
(for any preference profile for the $(n/2-1)$ unknown agents, Alice weakly prefers to tell the truth); or (2) the manipulation is not safe (there exists a preference profile for the unknown agents such that Alice strictly prefers to tell the truth).

\newcommand{\aFav}{c_m}
\newcommand{\aAlt}{c^A_{alt}}
\newcommand{\kAlt}{c^K_{alt}}

If the manipulation does not involve Alice's top candidate $c_m$, then it does not affect the outcome and cannot be profitable. So let us consider a manipulation in which Alice ranks another candidate $\aAlt\neq c_m$ at the top.

Let $\displaystyle \kAlt = \argmax_{j \in [m-1]} \score_K(c_j)$ denote the candidate with the highest number of votes among the known agents, except Alice's top candidate ($c_m$).
Consider the following two cases.

\paragraph{\underline{Case 1:} $\score_K(\kAlt) = 0$.} Since $\kAlt$ is a candidate who got the maximum number of votes from $K$ except $\aFav$, this implies that all $n/2$ known agents either vote for $\aFav$ or abstain,
Then, it is possible that the $n/2-1$ unknown agents vote for $\aAlt$ or abstain, such that the score-difference $\score(c_m)-\score(\aAlt) = 1$.
Then, when Alice tells the truth, her favorite candidate, $\aFav$ wins, as $\score_N(c_m) -\score_N(\aAlt) = 2$ and the scores of all other candidates are $0$. 
But when Alice manipulates and votes for $\aAlt$, the outcome is a tie between $\aFav$ and $\aAlt$, which is worse for Alice.
Hence, Alice's manipulation is not safe.


\paragraph{\underline{Case 2:} $\score_K(\kAlt) \geq 1$.}
Then again the manipulations not safe, as it is possible that the unknown agents vote as follows: 
\begin{itemize}
\item Some $\score_K(\aFav)$ agents vote for $\kAlt$;
\item Some 
$\score_K(\kAlt) -1$ agents vote for $c_m$. 

Note that 
this is possible as both values are non-negative and $\score_K(\aFav) + \score_K(\kAlt) \leq \sum_{j =1}^m \score_K(c_j) \leq  n/2$,
which means that $\score_K(\aFav) + \left(\score_K(\kAlt)-1\right) \leq n/2-1$ (the number of unknown agents);
\item 
The remaining unknown agents (if any) are split evenly between $c_m$ and $\kAlt$; if the number of remaining unknown agents is odd, then the extra agent votes for $c_m$.
\end{itemize}
We now prove that the manipulation is harmful for Alice. We distinguish three sub-cases. Denote by $R$ the set of Remaining unknown agents, mentioned in the third bullet above:
\begin{itemize}
\item If $R=\emptyset$
(which means that $\score_K(\aFav) + \score_K(\kAlt) -1 = n/2 -1$),
then
the scores of $\aFav$ and $\kAlt$ without Alice's vote are $n/2$ and $n/2-1$ respectively, which are at least $2$ and $1$ respectively (as $n\geq 4$).
The scores of all other candidates are $0$.

% There is another candidate with one vote. \erel{???}
% However, as $2 < n/2$, even if Alice was voting for this candidate, it has no chance to win, even as tie. 
% Thus only $\aFav$ and $\kAlt$ can win. 
% \eden{I think we need to assume that $n \geq 6$(?)}
When Alice is truthful, the outcome is a tie between $\aFav$ and $\kAlt$;
when Alice manipulates and votes for $\aAlt$, $\kAlt$ wins, which is worse for Alice.



\item If $|R|>0$ and it is even,
then without Alice's vote, 
$\score(\kAlt)$ is strictly higher than the scores of all other candidates, and higher than $\score(\aFav)$ by exactly $1$.

When Alice is truthful, the outcome is a tie between $\aFav$ and $\kAlt$.
But when Alice manipulates and votes for $\aAlt$, 
either $\kAlt$ wins or there is a tie between $\aAlt$ and $\kAlt$; both outcomes are worse for Alice.

\item If $|R|>0$ and it is odd, 
% (i.e., $\score_K(\aFav) + \score_K(\kAlt) -1 < n/2 -1$ and $n/2 -1 - \left(\score_K(\aFav) + \score_K(\kAlt) -1\right)$ is odd):
then without Alice's vote, $\score(\aFav) = \score(\kAlt)$, and both scores are at least the maximum score among the other candidates.
When Alice is truthful, her favorite candidate $\aFav$ wins. 
But when Alice manipulates and votes for $\aAlt$,
then either $\aAlt$ wins, or there is a tie between $\aFav$ and $\kAlt$ (and possibly some other candidates); both outcomes are worse for Alice.
\end{itemize}
Thus, in all cases, Alice does not have a safe profitable manipulation.
\end{proof}


Combining \Cref{lem:sm>sm1} and 
\Cref{lem:lower-plurality} gives the exact RAT-degree of plurality voting.
\begin{corollary}
When $m\geq 3$ and $n\geq 4$ and $n$ is even, the RAT-degree of plurality voting is $n/2+1$.    
\end{corollary}

\subsection{Positional voting rules: tighter upper bounds}

We now show that positional voting rules may have a RAT-degree substantially lower than plurality.

\iffalse % Special case; kept for didactic purposes.
\begin{lemmarep}
\label{lem:b:s2>s1}
Let $m\geq 3$ and $n\geq 3m$.
If $s_2 > s_1$ and there are $k$ known agents,
where 
\begin{align*}
k > \frac{2 s_m - s_2 - s_1}    {3 s_m + s_{m-1} - s_2 - 3 s_1} n,
\end{align*}
then
switching the bottom two candidates ($c_2$ and $c_1$) may be a safe profitable manipulation for Alice.
\end{lemmarep}
\begin{proofsketch}
Note that the expression on the right-hand side can be as small as $n/3$. 
Still, an adaptation of the construction of \Cref{lem:s2>s1} works: for some votes of the known agents,
the score of $c_1$ is necessarily lower than the \emph{arithmetic mean} of the scores of $c_m$ and $c_2$; hence, it is lower than either $c_m$ or $c_2$.
Therefore ,$c_1$ still cannot win, so switching $c_1$ and $c_2$ is safe.
\end{proofsketch}
\begin{proof}
Suppose there is a subset $K$ of $k$ known agents, who vote as follows:
\begin{itemize}
\item $k-2$ known agents rank $c_2 \succ c_m \succ \cdots \succ c_1$.
\item Two known agents rank $c_m \succ c_2 \succ
\cdots \succ c_1$. 
\end{itemize}
We first show that $c_1$ cannot win. To this end, we show that the difference in scores between $c_2$ and $c_1$, or between $c_m$ and $c_1$, is always strictly positive.
\begin{itemize}
\item The differences in scores given by the known agents is 
\begin{align*}
\score_K(c_2)-\score_K(c_1) =
&
(k-2)(s_m-s_1) 
+ 2(s_{m-1}-s_1).
\\
\score_K(c_m)-\score_K(c_1) =
&
(k-2)(s_{m-1}-s_1) 
+ 2(s_{m}-s_1).
\end{align*}
\item There are $n-k$ agents not in $K$ (including Alice). 
These agents give $c_2$ and $c_m$ together at least $(n-k)(s_1+s_2)$ points, and give $c_1$ at most $(n-k)s_m$ points. Therefore, we can bound the sum of score differences as follows:
\begin{align*}
&
[\score_N(c_2)-\score_N(c_1)]+[\score_N(c_m)-\score_N(c_1)] 
\\
\geq
&
k(s_m+s_{m-1}-2 s_1) 
+ 
(n-k)(s_2 + s_1 - 2 s_m)
\\
=
&
k(3 s_m + s_{m-1}-s_2 -3 s_1) 
- 
n(2 s_m - s_2 - s_1).
\end{align*}
The assumption on $k$ implies that this expression is positive. Therefore, either 
$\score_N(c_2)-\score_N(c_1)$ or $\score_N(c_m)-\score_N(c_1)$ or both are positive.
So $c_1$ has no chance to win or even tie.
Therefore, switching $c_2$ and $c_1$ is a safe manipulation.
\end{itemize}


Next, we show that the manipulation can help $c_m$ win, when the agents not in $K$ vote as follows:
\begin{itemize}
\item $k-4$ unknown agents rank $c_m\succ c_2\succ \cdots $,
where each candidate except $c_1,c_2,c_m$ is ranked last by at least one voter (here we use the assumption $n\geq 3m$: the condition on $k$ implies $k>n/3\geq m$, so $k\geq m+1$ and $k-4\geq m-3$).
\item One unknown agent ranks 
$c_2\succ \cdots \succ c_m \succ c_1$;
\item Alice votes truthfully $c_m\succ  \cdots \succ c_2 \succ c_1$.  
\item All other unknown voters (if any) abstain.
\end{itemize}
Then,
\begin{align*}
\score_N(c_2) - \score_N(c_m)
=&
(k-2)(s_{m}-s_{m-1}) 
+ 2(s_{m-1}-s_{m})
\\
&+
(k-4)(s_{m-1}-s_{m}) 
+ (s_m-s_2)
+ (s_2-s_m)
\\
=&
0.
\end{align*}
Moreover, for any $j\not\in\{1,2,m\}$, the score of $c_j$ is even lower (here we use the assumption that $c_j$ is ranked last by at least one unknown agent):
\begin{align*}
\score_N(c_2) - \score_N(c_j)
\geq &
(k-2)(s_{m}-s_{m-2}) 
+ 2(s_{m-1}-s_{m-2})
\\
&+
(k-5)(s_{m-1}-s_{m-2}) 
+ (s_{m-1}-s_1)
+ (s_m-s_{m-1})
+ (s_2-s_{m-1})
\\
\geq & (s_{m-1}-s_{1})
+ (s_2-s_{m-1})
\\
= & s_2 - s_1,
\end{align*}
which is positive by the lemma assumption.
Therefore, when Alice is truthful, the outcome is a tie between $c_m$ and $c_2$.

If Alice switches $c_1$ and $c_2$, then the score of $c_2$ decreases by $s_2-s_1$, which is positive by the lemma assumption, and the scores of all other candidates except $c_1$ do not change. So $c_m$ wins, which is better for Alice than a tie.
Therefore, the manipulation is profitable.
\end{proof}

In particular, for the anti-plurality rule the condition is $k>n/3$.
\begin{corollary}
The RAT-degree of anti-plurality is at most $\floor{n/3}+1$.
\end{corollary}
\fi

The following lemma strengthens 
\Cref{lem:s2>s1}.

\newcommand{\topLscores}{s_{\mathrm{top:}\ell}}
\newcommand{\botLscores}{s_{\mathrm{bot:}\ell}}
\begin{lemmarep}
\label{lem:z:s2>s1}
Let $\ell \in \{2,\ldots, m-1\}$ be an integer.
Consider a positional voting setting with $m\geq 3$ candidates and $n\geq (\ell+1)m$ agents.
Denote $\topLscores := \sum_{j=m-\ell+1}^m s_j = $  the sum of the $\ell$ highest scores and $\botLscores := \sum_{j=1}^{\ell}s_j = $ the sum of the $\ell$ lowest scores.

If $s_2 > s_1$ and there are $k$ known agents,
where 
\begin{align*}
k > \frac{\ell s_m - \botLscores}{\ell s_m + \topLscores - \botLscores - \ell s_1} n,
\end{align*}
then switching the bottom two candidates ($c_2$ and $c_1$) may be a safe profitable manipulation for Alice.
\end{lemmarep}
\begin{proofsketch}
The proof has a similar structure to that of \Cref{lem:s2>s1}.
Note that the expression at the right-hand side can be as small as $\displaystyle \frac{1}{\ell+1}n$ (for the anti-plurality rule), which is much smaller than the $\ceil{n/2}+1$ known agents required in \Cref{lem:s2>s1}.
Still, we can prove that, for some reports of the known agents, the score of $c_1$ is necessarily lower than the \emph{arithmetic mean} of the scores of the $\ell$ candidates $\{c_m, c_2, \cdots, c_{\ell}\}$. Hence, it is lower than at least one of these scores. Therefore ,$c_1$ still cannot win, so switching $c_1$ and $c_2$ is safe.
\end{proofsketch}
\begin{proof}
Suppose there is a subset $K$ of $k$ known agents, who vote as follows:
\begin{itemize}
\item $k-2$ known agents rank $c_2 \succ c_m$, then all candidates $\{c_3, \cdots , c_{\ell}\}$ in an arbitrary order, then the rest of the candidates in an arbitrary order, and lastly $c_1$.
\item Two known agents rank $c_m \succ c_2$, then all candidates $\{c_3 , \cdots , c_{\ell}\}$ in an arbitrary order, then the rest of the candidates in an arbitrary order, and lastly $c_1$.
\end{itemize}
We first show that $c_1$ cannot win. 
Denote $L := \{c_m, c_2, c_3, \ldots, c_{\ell}\}$.
We show that the difference in scores between some of the $\ell$ candidates in $L$ and $c_1$ is always strictly positive.
\begin{itemize}
\item The known agents rank all candidates in $L$ at the top $\ell$ positions. Therefore, each agent gives all of them together a total score of $\topLscores$. So
\begin{align*}
\sum_{c\in L} (\score_K(c)-\score_K(c_1)) =
&
k(\topLscores - \ell s_1).
\end{align*}
\item There are $n-k$ agents not in $K$ (including Alice). 
Each of these agents gives all candidates in $L$ together at least $\botLscores$, and gives $c_1$ at most $s_m$ points. Therefore, we can bound the sum of score differences as follows:
\begin{align*}
\sum_{c\in L} (\score_N(c)-\score_N(c_1)) \geq
&
k (\topLscores - \ell s_1)
+ (n-k) (\botLscores - \ell s_m)
\\
=&
k (\ell s_m + \topLscores - \botLscores - \ell s_1)
+ n(\botLscores - \ell s_m).
\end{align*}
The assumption on $k$ implies that this expression is positive. Therefore, for at least one $c\in L$, $\score_N(c)-\score_N(c_1) > 0$.
So $c_1$ has no chance to win or even tie.
Therefore, switching $c_2$ and $c_1$ is a safe manipulation.
\end{itemize}


Next, we show that the manipulation can help $c_m$ win, when the agents not in $K$ vote as follows:
\begin{itemize}
\item $k-4$ unknown agents rank $c_m\succ c_2\succ \cdots $,
where each candidate except $c_1,c_2,c_m$ is ranked last by at least one voter (here we use the assumption $n\geq (\ell+1)m$: the condition on $k$ implies $k>n/(\ell+1)\geq m$, so $k\geq m+1$ and $k-4\geq m-3$).
\item One unknown agent ranks 
$c_2\succ \cdots \succ c_m \succ c_1$;
\item Alice votes truthfully $c_m\succ  \cdots \succ c_2 \succ c_1$.  
\item If there are remaining unknown agents, then they are split evenly between 
$c_m\succ c_2\succ \cdots $ and 
$c_2\succ c_m\succ \cdots $ (if the number of remaining agents is odd, then the last one abstains).
\end{itemize}
Then,
\begin{align*}
\score_N(c_2) - \score_N(c_m)
=&
(k-2)(s_{m}-s_{m-1}) 
+ 2(s_{m-1}-s_{m})
\\
&+
(k-4)(s_{m-1}-s_{m}) 
+ (s_m-s_2)
+ (s_2-s_m)
\\
=&
0.
\end{align*}
Moreover, for any $j\not\in\{1,2,m\}$, the score of $c_j$ is even lower (here we use the assumption that $c_j$ is ranked last by at least one unknown agent):
\begin{align*}
\score_N(c_2) - \score_N(c_j)
\geq &
(k-2)(s_{m}-s_{m-2}) 
+ 2(s_{m-1}-s_{m-2})
\\
&+
(k-5)(s_{m-1}-s_{m-2}) 
+ (s_{m-1}-s_1)
+ (s_m-s_{m-1})
+ (s_2-s_{m-1})
\\
\geq & (s_{m-1}-s_{1})
+ (s_2-s_{m-1})
\\
= & s_2 - s_1,
\end{align*}
which is positive by the lemma assumption.
Therefore, when Alice is truthful, the outcome is a tie between $c_m$ and $c_2$.

If Alice switches $c_1$ and $c_2$, then the score of $c_2$ decreases by $s_2-s_1$, which is positive by the lemma assumption, and the scores of all other candidates except $c_1$ do not change. So $c_m$ wins, which is better for Alice than a tie.
Therefore, the manipulation is profitable.
\end{proof}

In particular, for the anti-plurality rule the condition in \Cref{lem:z:s2>s1} for $\ell=m-1$ is $k>n/m$.
\begin{corollary}
Let $m\geq 3$ and $n\geq m^2$.
The RAT-degree of anti-plurality is at most $\floor{n/m}+1$.
\end{corollary}
Intuitively, the reason that anti-plurality fares worse than plurality is that, even with a small number of known agents, it is possible to deduce that some candidate has no chance to win, and therefore there is a safe manipulation.

While we do not yet have a complete characterization of the RAT-degree of positional voting rules, our current results already show the strategic importance of the choice of scores.


\iffalse
The following lemma strengthens \Cref{lem:st1>st}
in a similar way.
\begin{lemma}
\label{lem:z:st1>st}
Let $\ell \in \{2,\ldots, m-1\}$ 
and $t\in\{1,\ldots,m-\ell \}$ be integers.

Consider a positional voting setting with $m\geq 3$ candidates and $n\geq (\ell+1)m$ agents.
Denote $\topLscores := \sum_{j=m-\ell+1}^m s_j = $  the sum of the $\ell$ highest scores and $\botLscores := \sum_{j=1}^{\ell}s_j = $ the sum of the $\ell$ lowest scores.
If $s_{t+1} > s_t = \cdots = s_1$ and there are $k$ known agents,
where 
\begin{align*}
k > \frac{\ell s_m - \botLscores}{\ell s_m + \topLscores - \botLscores - \ell s_1} n,
\end{align*}
then switching $c_{t+1}$ and $c_t$ may be a safe profitable manipulation.
\end{lemma}


\begin{proof}

Suppose there is a subset $K$ of $k$ known agents, who vote as follows:
\begin{itemize}
\item $k-2$ known agents rank $c_{t+1} \succ c_m$ first, then all candidates $\{c_{m-1}, \cdots , c_{m-\ell+1}\}$ in an arbitrary order, then the rest of the candidates in an arbitrary order, and lastly the candidates $\{c_1,\ldots,c_t\}$ in an arbitrary order (note this is possible as $t+\ell\leq m$).
\item Two known agents rank $c_m \succ c_{t+1}$ first, then all candidates $\{c_{m-1}, \cdots , c_{m-\ell+1}\}$ in an arbitrary order, then the rest of the candidates in an arbitrary order, and lastly the candidates $\{c_1,\ldots,c_t\}$ in an arbitrary order.
\end{itemize}

We first show that the $t$ worst candidates for Alice ($c_1,\ldots, c_t$) cannot win. 
Note that, by the lemma assumption $s_t = \cdots = s_1$, all these candidates receive exactly the same score by all known agents. We show that the difference in scores between $c_{t+1}$ and $c_t$ (and hence all $t$ worst candidates) is always strictly positive.
\begin{itemize}
\item The difference in scores given by the known agents is 
\begin{align*}
\score_K(c_{t+1})-\score_K(c_t) =
&
(\floor{n/2}-1)(s_m-s_1) 
+ (s_{m-1}-s_1).
\end{align*}
\item There are
$\floor{n/2}-1$ agents not in $K$ (including Alice).
These agents can reduce the score-difference by at most 
$(\floor{n/2}-1)(s_m-s_1)$.
Therefore, 
\begin{align*}
\score_N(c_{t+1})-\score_N(c_t) \geq (s_{m-1}-s_1),
\end{align*}
which is positive 
by the assumption $m-2 \geq t$ and $s_{t+1}>s_t$.
So no candidate in $c_1,\ldots,c_t$ has a chance to win or even tie.
\end{itemize}
Therefore, switching $c_{t+1}$ and $c_t$ can never harm Alice --- the manipulation is safe.

Next, we show that the manipulation can help $c_m$ win. We compute the score-difference between $c_m$ and the other candidates with and without the manipulation. 

Suppose that the agents not in $K$ vote as follows:
\begin{itemize}
\item $\floor{n/2}-3$ unknown agents rank $c_m\succ c_{t+1}\succ \cdots $,
where each candidate in $c_{t+2},\ldots,c_{m-1}$ is ranked last by at least one voter (here we use the assumption $n\geq 2m$).
\item One unknown agent ranks 
$c_{t+1}\succ \cdots \succ c_m \succ c_t \succ \cdots \succ c_1$;
\item Alice votes truthfully $c_m\succ  \cdots \succ c_{t+1} \succ  c_t \succ \cdots \succ c_1$.  
\end{itemize}
Then,
\begin{align*}
\score_N(c_{t+1}) - \score_N(c_m)
=&
(\floor{n/2}-1)(s_{m}-s_{m-1}) 
+ 2(s_{m-1}-s_{m})
\\
&+
(\floor{n/2}-3)(s_{m-1}-s_{m}) 
+ (s_m-s_{t+1})
+ (s_{t+1}-s_m)
\\
=&
0.
\end{align*}
Moreover, for any $j\in\{t+2,\ldots,m-1\}$, the score of $c_j$ is even lower (here we use the assumption that $c_j$ is ranked last by at least one unknown agent):
\begin{align*}
\score_N(c_{t+1}) - \score_N(c_j)
\geq &
(\floor{n/2}-1)(s_{m}-s_{m-2}) 
+ 2(s_{m-1}-s_{m-2})
\\
&+
(\floor{n/2}-4)(s_{m-1}-s_{m-2}) 
+ (s_{m-1}-s_1)
+ (s_m-s_{m-1})
+ (s_{t+1}-s_{m-1})
\\
\geq & (s_{m-1}-s_{1})
+ (s_{t+1}-s_{m-1})
\\
= & s_{t+1} - s_1,
\end{align*}
which is positive by the lemma assumption.
Therefore, when Alice is truthful, the outcome is a tie between $c_m$ and $c_{t+1}$.

If Alice switches $c_t$ and $c_{t+1}$, then the score of $c_{t+1}$ decreases by $s_{t+1}-s_t$, which is positive by the lemma assumption, and the scores of all other candidates except $c_t$ do not change. As $c_t$ cannot win, $c_m$ wins, which is better for Alice than a tie.
Therefore, the manipulation is profitable.
\end{proof}


The following lemma extends \Cref{lem:sm>sm1}.
\begin{lemma}
\label{lem:z:sm>sm1}
Let $\ell \in \{2,\ldots, m-1\}$ be an integer.
Consider a positional voting setting with $m\geq 3$ candidates and  $n\geq 4$ agents.
If $s_m > s_{m-1}$
and there are $k$ known agents, where
\begin{align*}
k > \frac{\ell s_m - \botLscores}{\ell s_m + \topLscores - \botLscores - \ell s_1} n,
\end{align*}
then switching the top two candidates ($c_m$ and $c_{m-1}$) may be a safe profitable manipulation for Alice.
\end{lemma}

\begin{proof}
Suppose there is a subset $K$ of $k$ known agents, who vote as follows:
\begin{itemize}
\item $k-2$ known agents rank $c_{m-2} \succ c_{m-1}$ first, 
then all candidates $\{c_{m-3},\ldots,c_{m-\ell}\}$,
then the other candidates in an arbitrary order,
and lastly $c_m$.
\item Two known agents rank $c_{m-1} \succ c_{m-2}$ first, 
then all candidates $\{c_{m-3},\ldots,c_{m-\ell}\}$,
then the other candidates in an arbitrary order,
and lastly $c_m$.
\end{itemize}

We first show that $c_m$ cannot win. 
Denote $L := \{c_{m-1}, \ldots, c_{m-\ell}\}$.
We show that the difference in scores between some of the $\ell$ candidates in $L$ and $c_m$ is always strictly positive.
\begin{itemize}
\item The known agents rank all candidates in $L$ at the top $\ell$ positions. Therefore, each agent gives all of them together a total score of $\topLscores$. So
\begin{align*}
\sum_{c\in L} (\score_K(c)-\score_K(c_m)) =
&
k(\topLscores - \ell s_1).
\end{align*}
\item There are $n-k$ agents not in $K$ (including Alice).
Each of these agents gives all candidates in $L$ together at least $\botLscores$, and gives $c_1$ at most $s_m$ points. Therefore, we can bound the sum of score differences as follows:
\begin{align*}
\sum_{c\in L} (\score_N(c)-\score_N(c_m)) \geq
&
k (\topLscores - \ell s_1)
+ (n-k) (\botLscores - \ell s_m)
\\
=&
k (\ell s_m + \topLscores - \botLscores - \ell s_1)
+ n(\botLscores - \ell s_m).
\end{align*}
The assumption on $k$ implies that this expression is positive. Therefore, for at least one $c\in L$, $\score_N(c)-\score_N(c_m) > 0$.
So $c_m$ has no chance to win or even tie.
\end{itemize}
Therefore, switching $c_m$ and $c_{m-1}$ is a safe manipulation.

Next, we show that the manipulation can help $c_{m-1}$ win. We compute the score-difference between $c_{m-1}$ and the other candidates with and without the manipulation. 

Suppose that the agents not in $K$ vote as follows:
\begin{itemize}
\item Some $k-4$ unknown agents
rank $c_{m-1}\succ c_{m-2}\succ \cdots $.
\item One unknown agent ranks $c_m \succ c_{m-2} \succ c_{m-1}\succ \cdots $.
\item If there are remaining unknown
agents, then they split evenly between 
$c_{m-1}\succ c_{m-2}\succ \cdots $ and $c_{m-2}\succ c_{m-1}\succ \cdots $ 
(if the number of these remaining agents is odd, the last one abstains).
\item Alice votes truthfully $c_m\succ c_{m-1}\succ c_{m-2} \cdots \succ c_1$.
\end{itemize}
Then,
\begin{align*}
\score_N(c_{m-2}) - \score_N(c_{m-1})
=&
(k-2)(s_{m}-s_{m-1}) 
+ 2(s_{m-1}-s_{m})
\\
&+
(k-4)(s_{m-1}-s_{m}) 
+ (s_{m-1}-s_{m-2})
+ (s_{m-2}-s_{m-1})
\\
=&
0.
\end{align*}
The candidates $c_{j<m-2}$ are ranked even lower than $c_{m-1}$ by all agents. Therefore the outcome is a tie between $c_{m-1}$ and $c_{m-2}$.

If Alice switches $c_{m-1}$ and $c_m$, then the score of $c_{m-1}$ increases by $s_m-s_{m-1}$ and the scores of all other candidates except $c_m$ do not change. Therefore, 
$\score_N(c_{m-2}) - \score_N(c_{m-1})$ becomes negative, so $c_{m-1}$ wins, which is better for Alice by assumption.
Therefore, the manipulation is profitable.
\end{proof}
\fi


\iffalse
% EREL: interesting, but not enough time to complete..
We now show a partial lower bound for positional voting.


\begin{proposition}\label{lower:score-voting-old}
Suppose there are $m=3$ candidates
and $n$ is odd. 
Then the RAT-degree of any score-vector voting rule is at least $
\floor{(n+1)/6}$.
\end{proposition}
\begin{proof}
Let $K$ be a set of $k$ known agents.
Let $T_1$ be the true preferences of Alice and let $P_1$ be a potential manipulation. We show that it is not a safe manipulation.

If $P_1$ does not change the score given to any candidate, then it has no effect on the outcome and therefore cannot be profitable.
Otherwise, there are some two candidates, say $c_j$ and $c_k$, such that $c_j$ gets a higher score by $T_1$ and $c_k$ gets a higher score by $P_1$. We will show that, for some set of reports by the unknown agents, $c_j$ and $c_k$ are tied for the first place. This means that if Alice reports $T_i$ then $c_j$ wins, but if $i$ manipulates to $P_i$ then $c_k$ wins, which is a worse outcome for Alice.

Specifically, suppose the unknown agents report as follows:
\begin{itemize}
\item Some $k$ unknown agents report exactly as the agents in $K$, except that $c_j$ and $c_k$ switch places (this guarantees that $c_j$ and $c_k$ are tied).
\item Some $2 k + 2$ unknown agents report $c_k > c_j > c_l$, where $c_l$ is the third candidate.
\item Some $2 k + 2$ unknown agents report $c_j > c_k > c_l$.
\end{itemize}
Now, the score of $c_k$ and $c_j$ is equal, and at least $(k s_1 + k s_2) + (2 k+2) s_2 + (2 k+2) s_3 = k s_1 + (3 k +2) s_2 + (2 k + 2)s_3$. The score of $c_l$ is at most $2 k s_3 + (4 k + 4) s_1$.
The difference in scores is at least $2 (s_3-s_1) + (3k + 2)(s_2-s_1)$, which is positive for any score vector. So $c_j$ and $c_k$ tie for the first place. Moreover, the difference remains positive regardless of Alice's vote. Therefore, if Alice votes truthfully then $c_j$ wins; otherwise $c_k$ wins. 

The total number of agents required for the construction (including Alice) is $6 k + 5$. We have shown that $k$ known agents are not sufficient for a safe manipulation; hence the RAT-degree is at least $k+1 = (n+1)/6$.

If $n$ is odd and $k+1 = \floor{(n+1)/6}$, then $n$ can be either $6k+5$ or $6k+7$ or $6k+9$. The above construction covers the first case; for the two latter cases, simply add one or two unknown agents that report $c_k > c_j > c_l$ and $c_j > c_k > c_l$.

Future work: extend to $m\geq 4$ candidates and more general values of $n$. See question here: \url{https://math.stackexchange.com/q/5029581}
\end{proof}

\fi


\iffalse
\subsection{Condorcet rules}
\erel{Just some initial thoughts}
A \emph{Condorcet voting rule} is a rule that always selects a Condorcet winner if one exists. Condorcet rules differ in how they pick the winner when a Condorcet winner does not exist.

A simple Condorcet rule is the MaxMin-Condorcet rule. It works in the following way:
\begin{itemize}
\item For each ordered pair of candidates $(c,c')$, compute $S(c,c')$ as the number of voters who prefer $c$ to $c'$.
\item The score of each candidate $c$ is $\min_{c'} S(c,c')$, that is, the lowest score of $c$ in all pairwise competitions.
\item The candidate with the highest score wins. 
\end{itemize}
Note that, if there is a Condorcet winner, then his score will be larger than $n/2$, and the score of all other candidates will be smaller than $n/2$, so the Condorcet winner will be selected.

This rule potentially has a high RAT-degree, as the winner depends on $m-1$ scores per agent, rather than one. 
\erel{I could not compute its RAT-degree - some brain-storming on this could help.}
\fi


\subsection{Higher RAT-degree?}
\Cref{thm:upper-positional} raises the question of whether some other, non-positional voting rules have RAT-degrees substantially higher than $n/2$.
%
Using ideas similar to those in \Cref{sect:indivisible-EF1-n-1}, 
we could use a selection rule $\Gamma$ to choose a ``dictator'', and implement the dictator's first choice.
This deterministic mechanism has RAT-degree $n-1$, as without knowledge of all other agents' inputs, every manipulation might cause the manipulator to lose the chance of being a dictator. 
However, besides the fact that this is an unnatural mechanism, it suffers from other problems such as the \emph{no-show paradox} (a participating voter might affect the selection rule in a way that will make another agent a dictator, which might be worse than not participating at all).

Our main open problem is therefore to devise natural voting rules with a high RAT-degree.
\begin{open}
Does there exist a non-dictatorial voting rule that satisfies the participation criterion (i.e. does not suffer from the no-show paradox),  with RAT-degree larger than $\ceil{n/2}+1$? 
\end{open}


% \newpage

\newcommand{\men}{M}
\newcommand{\women}{W}
\newcommand{\man}{m}
\newcommand{\woman}{w}

\section{Stable Matchings}\label{sec:matching}

% \eden{I'm not sure if it would be better to use men and women or students and universities}

In this section, we consider mechanisms for stable matchings. 
Here, the $n$ agents are divided into two disjoint subsets, $\men$ and $\women$, that need to be matched to each other. The most common examples are men and women or students and universities. 
Each agent has a strict preference order over the agents in the other set and being unmatched -- for each $\man \in \men$, an order $\succ_{\man} $ over $\women\cup \{\phi\}$; and for each $\woman \in \women$ an order, $\succ_{\woman}$, over $\men\cup \{\phi\}$.  

A \emph{matching} between $\men$ to $\women$ is a mapping $\mu$ from $\men \cup \women$ to $\men \cup \women \cup \{\phi\}$ such that (1) $\mu(\man) \in \women \cup \{\phi\}$ for each $\man \in \men$, (2) and $\mu(\woman) \in \men \cup \{\phi\}$ for each $\woman \in \women$, and (3) $\mu(\man) = \woman$ if and only if $\mu(\woman) = \man$ for any $(\man, \woman) \in \men \times \women$. 
% \begin{align*} 
%     \forall \man \in \men \colon &\quad \mu(\man) \in \women \cup \{\phi\}\\
%     \forall \woman \in \women \colon &\quad \mu(\woman) \in \men \cup \{\phi\}\\
%     \forall \man, \woman \in \men \cup \women \colon &\quad  \mu(\man) = \woman  \iff \mu(\woman) = \man
% \end{align*}
When $\mu(a) = \phi$ it means that agent $a$ is unmatched under $\mu$. 
%
A matching is said to be \emph{stable} if (1) \emph{no} agent prefers being unmatched over their assigned match, and (2) there is \emph{no} pair $(\man, \woman) \in \men \times \women$ such that $\man$ prefers $\woman$ over his assigned match while $\woman$ prefers $\man$ over her assigned match -- $\woman \succ_{\man} \mu(\man)$ and $\man \succ_{\woman} \mu(\woman)$.


A mechanism in this context gets the preference orders of all agents and returns a stable matching.

\paragraph{Results} Our results for this problem are preliminary, so we provide only a brief overview here, with full descriptions and proofs in the appendix. We believe, however, that this is an important problem and that our new definition opens the door to many further questions.

We first analyze the deferred acceptance mechanism and prove that its RAT-degree is at most $3$, showing that it is $3$-known-agents safely manipulable. The proof relies on \emph{truncation}, where an agent in $\women$ falsely reports preferring to remain unmatched over certain options. We further show that even without truncation, the RAT-degree is at most $5$.

Finally, we examine the Boston mechanism and establish an upper bound of $2$ on its RAT-degree.


\subsection{Deferred Acceptance (Gale-Shapley)}\label{sec:deferred-acceptance}

The \emph{deferred acceptance} algorithm \cite{gale1962college} is one of the most well-known mechanisms for computing a stable matching. 
In this algorithm, one side of the market - here, $\men$ - proposes, while the other side - $\women$ - accepts or rejects offers iteratively. 
%
\begin{toappendix}
\subsection{Deferred Acceptance (Gale-Shapley): descriptions and proofs}
The algorithm proceeds as follows:
\begin{enumerate}
    \item Each $\man \in \men$ proposes to his most preferred alternative according to $\succ_{\man}$ that has not reject him yet and that he prefers over being matched. 

    \item Each $\woman \in W$ tentatively accepts her most preferred proposal according to $\succ_{\woman}$ that she prefers over being matched, and rejects the rest.

    \item The rejected agents propose to their next most preferred choice as in step 1.

    \item The process repeats until no one of the rejected agents wishes to make a new proposal.

    \item The final matching is determined by the last set of accepted proposal.
\end{enumerate}



It is well known that the mechanism is truthful for the proposing side ($\men$) but untruthful for the other side ($\women$).
That is, the agents in $\women$ may have an incentive to misreport their preferences to obtain a better match.
\end{toappendix}
%
We prove that: 
 
\begin{theoremrep}\label{prop-def-acc-trunc}
    The RAT-degree of deferred acceptance is at most $3$.
\end{theoremrep}

\begin{proof}
    To prove that the RAT-degree is at most $3$, we show that it is $3$-known-agents manipulable. 

    Let $\woman_1 \in \women$ and assume without loss of generality that $\man_1 \succ_{\woman_1} \man_2 \succ_{\woman_1} \cdots$  (the preferences between the other alternatives are irrelevant).

    Consider the case where the $3$-known-agents are as follows::
    \begin{itemize}
        \item Let $\woman_2 \in \women$ be an agent whose preferences are $\man_2 \succ_{\woman_2} \man_1 \succ_{\woman_2} \cdots $.

        \item The preferences of $\man_1$ are $\woman_2 \succ_{\man_1} \woman_1 \succ_{\man_1} \cdots$.

        \item The preferences of $\man_2$ are $\woman_1 \succ_{\man_2} \woman_2 \succ_{\man_2} \cdots$.
    \end{itemize}



    In this case, when $\woman_1$ tells the truth, the resulting matching includes the pairs $(\man_1, \woman_2)$ and $(\man_2, \woman_1)$, since in this case it proceeds as follows:
    \begin{itemize}
        \item In the first step, all the agents in $\men$ propose to their most preferred option: $\man_1$ proposes to $\woman_2$ and $\man_2$ proposes to $\woman_1$.

        Then, the agents in $\women$ tentatively accept their most preferred proposal among those received, as long as she prefers it to remaining unmatched: 
        $\woman_1$ tentatively accepts $\man_2$ since she prefers him over being unmatched, and since $\man_2$ must be her most preferred option among the proposers as $\man_1$ (her top choice) did not propose to her.
        Similarly, $\woman_2$ tentatively accepts $\man_1$.

        \item In the following steps, more rejected agents in $\men$ might propose to $\woman_1$ and $\woman_2$, but they will not switch their choices, as they prefer $\man_2$ and $\man_1$, respectively. 
        
        Thus, when the algorithm terminates $\man_1$ is matched to $\woman_2$ and $\man_2$ is matched to $\woman_1$. 
    \end{itemize}
    Which means that $\woman_1$ is matched to her second-best option.


    We shall now see that $\woman_1$ can increase her utility by misreporting that her preference order is: $\man_1 \succ'_{\woman_1} \phi \succ'_{\woman_1} \man_2 \succ'_{\woman_1} \cdots $.
    The following shows that in this case, the resulting matching includes the pairs $(\man_1, \woman_1)$ and $(\man_2, \woman_2)$, meaning that $\woman_1$ is matched to her most preferred option (instead of her second-best).
    \begin{itemize}
        \item In the first step, as before, $\man_1$ proposes to $\woman_2$ and $\man_2$ proposes to $\woman_1$.

        However, here,  $\woman_1$ rejects $\man_2$ because, according to her false report, she prefers being unmatched over being matched to $\man_2$.
        As before, $\woman_2$ tentatively accepts $\man_1$.


        \item In the second step, $\man_2$, having been rejected by $\woman_1$, proposes to his second-best choice $\woman_2$.

        Since $\woman_2$ prefers $\man_2$ over $\man_1$,  she rejects $\man_1$ and tentatively accepts $\man_2$.


        \item In the third step, $\man_1$, having been rejected by  $\woman_2$, proposes to his second-best choice $\woman_1$.

        $\woman_1$ now tentatively accepts $\man_1$ since according to her false report, she prefers him over being unmatched.
        
        
        \item In the following steps, more rejected agents in $\men$ might propose to $\woman_1$ and $\woman_2$, but they will not switch their choices, as they prefer $\man_2$ and $\man_1$, respectively. 
        
        
        In the following steps, more rejected agents in $\men$ can propose to $\woman_1$ and $\woman_2$ but they would reject their proposes as both of them have the alternative of being matched to their best-option. 

        Thus, when the algorithm terminates $\man_1$ will be matched to $\woman_1$ and and $\man_2$ will be matched to $\woman_2$. 
    \end{itemize}
    

    Thus, regardless of the reports of the remaining $(n-4)$ remaining (unknown) agents report, $\woman_1$ strictly prefers to misreport her preferences.
\end{proof}

The proof is based on a key type of manipulation in this setting called \emph{truncation} \cite{roth1999truncation,ehlers2008truncation,coles2014optimal} -- where agents in $\women$ falsely report that they prefer being unmatched rather than being matched with certain agents — even though they actually prefer these matches to being unmatched.

However, in some settings, it is reasonable to assume that agents always prefer being matched if possible. In such cases, the mechanism is designed to accept only preferences over agents from the opposite set (or equivalently, orders where being unmatched is always the least preferred option). Clearly, under this restriction, truncation is not a possible manipulation.
We prove that even when truncation is not possible, the RAT-degree is bounded.


\begin{theoremrep}\label{prop-def-acc-no-trunc}
     Without truncation, the RAT-degree of deferred acceptance is at most $5$.
\end{theoremrep}


\begin{proof}
    To prove that the RAT-degree is at most $5$, we show that it is $5$-known-agents manipulable. 

    Let $\woman_1 \in \women$ and assume without loss of generality that $\man_1 \succ_{\woman_1} \man_2 \succ_{\woman_1} \man_3 \succ_{\woman_1} \cdots$  (the preferences between the other alternatives are irrelevant).

    Consider the case where the $5$-known-agents are as follows:
    \begin{itemize}
        \item Let $\woman_2 \in \women$ be an agent whose preferences are $\man_1 \succ_{\woman_2} \man_2 \succ_{\woman_2} \man_3 \succ_{\woman_2} \cdots $.
        
        \item Let $\woman_3 \in \women$ be an agent whose preferences are $\man_2 \succ_{\woman_3} \man_1 \succ_{\woman_3} \man_3 \succ_{\woman_2} \cdots $.

        \item The preferences of $\man_1$ are $\woman_3 \succ_{\man_1} \woman_1 \succ_{\man_1} \woman_2 \succ_{\man_1} \cdots$.

        \item The preferences of $\man_2$ are $\woman_1 \succ_{\man_2} \woman_3 \succ_{\man_2} \woman_2 \succ_{\man_2} \cdots$.

        \item The preferences of $\man_3$ are $\woman_1 \succ_{\man_2} \woman_3 \succ_{\man_2} \woman_2 \succ_{\man_2} \cdots$.
    \end{itemize}



    In this case, when $\woman_1$ tells the truth, the resulting matching includes the pairs $(\man_1, \woman_3)$, $(\man_2, \woman_1)$ and $(\man_3, \woman_2)$, since in this case it proceeds as follows:
    \begin{itemize}
        \item In the first step, all the agents in $\men$ propose to their most preferred option: $\man_1$ proposes to $\woman_3$, while $\man_2$ and $\man_3$ proposes to $\woman_1$.

        Then, the agents in $\women$ tentatively accept their most preferred proposal among those received.
        $\woman_1$ tentatively accepts $\man_2$ since he must be her most preferred option among the proposers -- as he is her second-best and her top choice, $\man_1$, did not propose to her; and rejects $\man_3$.
        Similarly, $\woman_3$ tentatively accepts $\man_1$.
        $\woman_2$ did not get any proposes.  

        \item In the second step, $\man_3$, having been rejected by $\woman_1$, proposes to his second-best choice $\woman_3$.

        Since $\woman_3$ prefers her current match $\man_1$ over $\man_3$,  she rejects $\man_3$.

        \item In the third step, $\man_3$, having been rejected by $\woman_3$, proposes to his third-best choice $\woman_2$.

        Since $\woman_2$ does not have a match, she tentatively accepts $\man_3$.


        \item In the following steps, more rejected agents in $\men$ - that are not $\man_1, \man_2$ and $\man_3$, might propose to $\woman_1, \woman_2$ and $\woman_3$, but they will not switch their choices, as they can only be least preferred than their current match. 
        
        Thus, when the algorithm terminates $\man_1$ is matched to $\woman_3$, $\man_2$ is matched to $\woman_1$, and $\man_3$ is matched to $\woman_2$. 
    \end{itemize}
    Which means that $\woman_1$ is matched to her second-best option.


    However, when $\woman_1$ misreporting that her preference order is: $\man_1 \succ'_{\woman_1} \man_3 \succ'_{\woman_1} \man_2 \succ'_{\woman_1} \cdots$.
    The following shows that in this case, the resulting matching includes the pairs $(\man_1, \woman_1)$, $(\man_2, \woman_3)$ and $(\man_3, \woman_2)$, meaning that $\woman_1$ is matched to her most preferred option (instead of her second-best).
    \begin{itemize}
        \item In the first step, as before, $\man_1$ proposes to $\woman_3$, while $\man_2$ and $\man_3$ proposes to $\woman_1$.

        However, here,  $\woman_1$ tentatively accepts $\man_3$ and rejects $\man_2$.
        As before, $\woman_3$ tentatively accepts $\man_1$ and $\woman_2$ did not get any proposes.


        \item In the second step, $\man_2$, having been rejected by $\woman_1$, proposes to his second-best choice $\woman_3$.

        Since $\woman_3$ prefers $\man_2$ over $\man_1$,  she rejects $\man_1$ and tentatively accepts $\man_2$.


        \item In the third step, $\man_1$, having been rejected by  $\woman_3$, proposes to his second-best choice $\woman_1$.

        $\woman_1$ tentatively accepts $\man_1$ since according to her false report, she prefers him over $\man_3$.

        
        \item In the fourth step, $\man_3$, having been rejected by  $\woman_1$, proposes to his second-best choice $\woman_3$.

        $\woman_3$ prefers her current match $\man_2$, and thus rejects $\man_3$.


        \item In the fifth step, $\man_3$, having been rejected by  $\woman_3$, proposes to his third-best choice $\woman_2$.

        As $\woman_2$ does not have a match, she tentatively accepts $\man_3$.
        
        
        \item In the following steps, more rejected agents in $\men$ - that are not $\man_1, \man_2$ and $\man_3$, might propose to $\woman_1, \woman_2$ and $\woman_3$, but they will not switch their choices, as they can only be least preferred than their current match. 
        
        Thus, when the algorithm terminates $\man_1$ is matched to $\woman_1$, $\man_2$ is matched to $\woman_3$, and $\man_3$ is matched to $\woman_2$. 
    \end{itemize}
    

    Thus, regardless of the reports of the remaining $(n-6)$ remaining (unknown) agents report, $\woman_1$ strictly prefers to misreport her preferences.
\end{proof}




\subsection{Boston Mechanism}
The \emph{Boston} mechanism \cite{abdulkadirouglu2003school} is a widely used mechanism for assigning students or schools. 
% Unlike the Deferred Acceptance algorithm, it  prioritizes higher-ranked choices in a sequential manner.
\begin{toappendix}
\subsection{Boston Mechanism: descriptions and proofs}
The mechanism proceeds in rounds as follows:
\begin{enumerate}
    \item Each $\man \in \men$ proposes to his most preferred alternative according $\succ_{\man}$ that has not yet rejected him and is still available.

    \item Each $\woman \in \women$ (permanently) accepts her most preferred proposal according to $\succ_{\woman}$ and rejects the rest. Those who accept a propose become unavailable. 

    \item The rejected agents propose to their next most preferred choice as in step $1$.

    \item The process repeats until all agents are either assigned or have exhausted their preference lists.
\end{enumerate}

\end{toappendix}
We prove that:

\begin{theoremrep}
    \label{prop-boston}
    The RAT-degree of the Boston mechanism is at most $2$.
\end{theoremrep}

\begin{proof}
    To prove that the RAT-degree is at most $2$, we show that it is $2$-known-agents manipulable.

    Let $\man_1 \in \men$ be an agent with preferences $\woman_1 \succ_{\man_1} \woman_2 \succ_{\man_1} \cdots$.
    Consider the case where the two known-agents are as follows:
    
    \begin{itemize}
        \item Let $\man_2 \in \men$ be an agent whose preferences are similar, $\woman_1 \succ_{\man_2} \woman_2 \succ_{\man_2} \cdots$.
        \item The preferences of $\woman_1$ are $\man_2 \succ_{\woman_1} \man_1 \succ_{\woman_1} \cdots$.
    \end{itemize}

When $\man_1$ reports truthfully, the mechanism proceeds as follows: In the first round, both $\man_1$ and $\man_2$ proposes to $\woman_1$. Since $\woman_1$ prefers $\man_2$, she rejects $\man_1$ and becomes unavailable. Thus, in the second round, $\man_1$ proposes to $\woman_2$.


However, we prove that $\man_1$ has a safe-and-profitable manipulation: misreports his preference as $\woman_2 \succ'_{\man_1} \woman_3 \succ'_{\man_1} \cdots \succ'_{\man_1} \woman_1$.
The manipulation is safe since $\man_1$ never had a chance to be matched with $\woman_1$ (regardless of his report). 
The manipulation is profitable as there exists a case where the manipulation improves $\man_1$’s outcome. Consider the case where the top choice of $\woman_2$ is $\man_1$ and there is another agent, $\man_3$, whose top choice is $\woman_2$. 
Notice that $\woman_2$ prefers $\man_1$ over $\man_3$. 
When $\man_1$ reports truthfully, then in the first round, $\man_3$ proposes to $\woman_2$ and gets accepted, making her unavailable by the time $\man_1$ reaches her in the second round.
However, if $\man_1$ misreports and proposes to $\woman_2$ in the first round, she will accept him (as she prefers him over $\man_3$). This guarantees that $\man_1$ is matched to $\woman_2$, improving his outcome compared to truthful reporting.
\end{proof}

% \eden{\url{https://web.stanford.edu/~alroth/papers/bostonMay182006.pdf}}
% \section{Budget-Proposal Aggregation}\label{sec:budget-proposal-aggregation}
\section{Theoretical Analysis}\label{sec:theoretical}

\textbf{Different correct answers are competitor.}\quad For any LLM trained with cross-entropy loss, different correct answers are competitors in terms of probability \footnote{The ``same question'' refers to questions that are semantically equivalent but do not need to be identical.}. Continuing with the example of proposing a president, suppose $\tau^{a}$ (``\texttt{Barack}'') is the label of a sample whose $\bm{q}$ is ``\texttt{[INST]Could you give me one name of president?[\textbackslash INST]}'' and a generated token vector $\bm{a}_{t-1}$  can be decoded into ``\texttt{Sure, here is a historical American president:**}'', the loss of the next token at this position during supervised fine-tuning can be written as:
\begin{equation}
\begin{aligned}
 &L^{\tau^a} = - \log \frac{\exp(\mathcal{M}({\tau^a}|\bm{q},\bm{a}_{t-1}))}{\sum_{m=1}^{|\bm{Y}|} \exp(\mathcal{M}(\tau^{m}|\bm{q},\bm{a}_{t-1}))} ,
 % \\   &L^{\tau^b} = - \log \frac{\exp(\mathcal{M}(\tau^b|\bm{q},\bm{a}_{t-1}))}{\sum_{m=1}^{|\bm{Y}|} \exp(\mathcal{M}(\tau^{m}|\bm{q},\bm{a}_{t-1}))} ,
\end{aligned}
\end{equation}
where $L^{\tau^a}$ is the loss on the sample with the next token label $\tau^{a}$.
Consider cases where multiple distinct answers to the same question appear in the training set, the situation becomes different. For example, $\tau^{b}$ (``\texttt{George}'') is the label in another sample with the same question. When the model is simultaneously fine-tuned on both samples, the gradient update for the model will be:
\begin{equation}
\begin{aligned}
 & \nabla_{\mathcal{M}} (L^{\tau^a} + L^{\tau^b}) = \nabla_{\mathcal{M}} L^{\tau^a} + \nabla_{\mathcal{M}} L^{\tau^b} \\
% &= -y_a^{\tau^a}\frac{1}{\Omega_a^{\tau^a}}\nabla_{\mathcal{M}}\Omega_a^{\tau^a}-\sum_{m \neq a}^{|\bm{Y}|} y_a^{\tau^m}\frac{1}{\Omega_a^{\tau^m}}\nabla_{\mathcal{M}}\Omega_a^{\tau^m}
% \\
% &\quad -y_b^{\tau^b}\frac{1}{\Omega_b^{\tau^b}}\nabla_{\mathcal{M}}\Omega_b^{\tau^b}-\sum_{m \neq b}^{|\bm{Y}|} y_b^{\tau^m}\frac{1}{\Omega_b^{\tau^m}}\nabla_{\mathcal{M}}\Omega_b^{\tau^m}
% \\
&\quad= \underbrace{-y_a^{\tau^a}\frac{1}{\Omega_a^{\tau^a}}\nabla_{\mathcal{M}}\Omega_a^{\tau^a}-y_b^{\tau^b}\frac{1}{\Omega_b^{\tau^b}}\nabla_{\mathcal{M}}\Omega_b^{\tau^b}}_{\text{(1) maximizing the probability of annotated answer}}\\& \quad \underbrace{-y_a^{\tau^b}\frac{1}{\Omega_a^{\tau^b}}\nabla_{\mathcal{M}}\Omega_a^{\tau^b}-y_b^{\tau^a}\frac{1}{\Omega_b^{\tau^a}}\nabla_{\mathcal{M}}\Omega_b^{\tau^a}}_{{\text{\textbf{(2)} minimizing the probability of the other annotated answer}}}\\& \quad \underbrace{-\sum_{m \neq a,b}^{|\bm{Y}|}y_{a,b}^{\tau^m} \left[ \frac{1}{\Omega_a^{\tau^m}}\nabla_{\mathcal{M}}\Omega_a^{\tau^m} + \frac{1}{\Omega_b^{\tau^m}}\nabla_{\mathcal{M}}\Omega_b^{\tau^m} \right]}_{\text{(3) minimizing the probability of incorrect answers}},
\end{aligned}\label{eq:competitor}
\end{equation}
where $\Omega_a^{\tau^a}=\frac{\exp(\mathcal{M}(\tau^a|\bm{q},\bm{a}_{t-1}))}{\sum_{m=1}^{|\bm{Y}|} \exp(\mathcal{M}(\tau^{m}|\bm{q},\bm{a}_{t-1}))}$, and $y_a^{\tau^m}$ indicates the next token label of a training sample with ground-truth label ${\tau^a}$, that is, we have $y_a^{\tau^a}=1$ and $y_a^{\tau^b}=0$. In particular, when $\mathcal{M}$ is in a certain state during training, we have $\Omega_a^{\tau^a}=\Omega_b^{\tau^a}$, and we make distinctions to facilitate the reader's understanding here. As we can see, for scenarios with multiple answers, the training objective can be divided into three parts:
(1) For each sample, increase the probability of its own annotation in the output distribution.
(2) For each sample, decrease the probability of another sample's annotation in the output distribution. \textit{\textbf{Note:}} This part leads to the issue where probability cannot anymore capture the reliability of LLM responses, as different correct answers tend to reduce the probability of other correct answers, making low probabilities cannot indicates low reliability.
(3) For both samples, decrease the probability of other outputs not present in the annotations in the output distribution.





% Bibliography
\newpage
\bibliographystyle{ACM-Reference-Format}
\bibliography{main}

% Appendix
\appendix
\newpage

\putsec{related}{Related Work}

\noindent \textbf{Efficient Radiance Field Rendering.}
%
The introduction of Neural Radiance Fields (NeRF)~\cite{mil:sri20} has
generated significant interest in efficient 3D scene representation and
rendering for radiance fields.
%
Over the past years, there has been a large amount of research aimed at
accelerating NeRFs through algorithmic or software
optimizations~\cite{mul:eva22,fri:yu22,che:fun23,sun:sun22}, and the
development of hardware
accelerators~\cite{lee:cho23,li:li23,son:wen23,mub:kan23,fen:liu24}.
%
The state-of-the-art method, 3D Gaussian splatting~\cite{ker:kop23}, has
further fueled interest in accelerating radiance field
rendering~\cite{rad:ste24,lee:lee24,nie:stu24,lee:rho24,ham:mel24} as it
employs rasterization primitives that can be rendered much faster than NeRFs.
%
However, previous research focused on software graphics rendering on
programmable cores or building dedicated hardware accelerators. In contrast,
\name{} investigates the potential of efficient radiance field rendering while
utilizing fixed-function units in graphics hardware.
%
To our knowledge, this is the first work that assesses the performance
implications of rendering Gaussian-based radiance fields on the hardware
graphics pipeline with software and hardware optimizations.

%%%%%%%%%%%%%%%%%%%%%%%%%%%%%%%%%%%%%%%%%%%%%%%%%%%%%%%%%%%%%%%%%%%%%%%%%%
\myparagraph{Enhancing Graphics Rendering Hardware.}
%
The performance advantage of executing graphics rendering on either
programmable shader cores or fixed-function units varies depending on the
rendering methods and hardware designs.
%
Previous studies have explored the performance implication of graphics hardware
design by developing simulation infrastructures for graphics
workloads~\cite{bar:gon06,gub:aam19,tin:sax23,arn:par13}.
%
Additionally, several studies have aimed to improve the performance of
special-purpose hardware such as ray tracing units in graphics
hardware~\cite{cho:now23,liu:cha21} and proposed hardware accelerators for
graphics applications~\cite{lu:hua17,ram:gri09}.
%
In contrast to these works, which primarily evaluate traditional graphics
workloads, our work focuses on improving the performance of volume rendering
workloads, such as Gaussian splatting, which require blending a huge number of
fragments per pixel.

%%%%%%%%%%%%%%%%%%%%%%%%%%%%%%%%%%%%%%%%%%%%%%%%%%%%%%%%%%%%%%%%%%%%%%%%%%
%
In the context of multi-sample anti-aliasing, prior work proposed reducing the
amount of redundant shading by merging fragments from adjacent triangles in a
mesh at the quad granularity~\cite{fat:bou10}.
%
While both our work and quad-fragment merging (QFM)~\cite{fat:bou10} aim to
reduce operations by merging quads, our proposed technique differs from QFM in
many aspects.
%
Our method aims to blend \emph{overlapping primitives} along the depth
direction and applies to quads from any primitive. In contrast, QFM merges quad
fragments from small (e.g., pixel-sized) triangles that \emph{share} an edge
(i.e., \emph{connected}, \emph{non-overlapping} triangles).
%
As such, QFM is not applicable to the scenes consisting of a number of
unconnected transparent triangles, such as those in 3D Gaussian splatting.
%
In addition, our method computes the \emph{exact} color for each pixel by
offloading blending operations from ROPs to shader units, whereas QFM
\emph{approximates} pixel colors by using the color from one triangle when
multiple triangles are merged into a single quad.




\end{document}
