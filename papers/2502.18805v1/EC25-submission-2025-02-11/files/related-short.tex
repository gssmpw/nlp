\subsection{Related Word}
There is a vast body of work on truthfulness relaxations and alternative measurements of manipulability. Due to space constraints, we provide only a brief overview here; a more in-depth discussion of these works and their relation to RAT-degree can be found in \Cref{apx:related}.


\paragraph{Truthfulness Relaxations.} Various truthfulness relaxations focus on a certain subset of all possible manipulations, which are considered more ``likely''. It requires that none of the manipulations from this subset is profitable. Different relaxations consider different subsets of ``likely'' manipulations.

\citet{BU2023Rat} introduce the definition on which this paper is build upon: \emph{risk-avoiding truthfulness (RAT)},\footnote{The original term of \citet{BU2023Rat}  is risk-\emph{averse} truthfulness. However, since the definition assumes that agents completely avoid any element of risk, we adopt this new name, aiming to more accurately reflect this assumption.} assuming agents manipulate only when it is sometimes beneficial but never harmful. We extend their work by generalizing RAT from cake cutting to any social choice problem, and by suggesting a quantitative measure of the robustness of a mechanism to such manipulations.

\citet{brams2006better} propose \emph{maximin strategy-proofness}, where an agent manipulates only if it is always beneficial, making it a weaker condition than RAT. 
\citet{troyan2020obvious} introduce \emph{not-obvious manipulability (NOM)}, which assumes agents consider only extreme best or worst cases. RAT and NOM are independent notions. \citet{regret2018Fernandez} define \emph{regret-free truth-telling (RFTT)}, where agents never regret truth-telling after observing the outcome. RAT and RFTT do not imply each other. Additionally, \citet{slinko2008nondictatorial,slinko2014ever,hazon2010complexity} study "safe manipulations" in voting, but they consider coalition of voters and a different type of risk - that too many or too few participants will perform the exact safe manipulation.


\paragraph{Alternative Measurements.}
There are many approaches quantify manipulability from different perspectives. 
One approach considers the computational complexity of finding a profitable manipulation — e.g., \cite{bartholdi1989computational,bartholdi1991single} (see \cite{faliszewski2010ai,veselova2016computational} for surveys). 
Another measurement is the number of bits
an agent needs to know in order to have a safe manipulation - spirit to the concepts of communication complexity - e.g., \cite{nisan2002communication, grigorieva2006communication, Communication2019Branzei,Babichenko2019communication} and compilation complexity - e.g., \citep{chevaleyre2009compiling,xia2010compilation,karia2021compilation}.
A third approach evaluates the probability that a profitable manipulation exists —e.g., \cite{barrot2017manipulation,lackner2018approval,lackner2023free}.
The \emph{incentive ratio}, which measures how much an agent can improve their utility by manipulating, is also widely studied—e.g., \cite{chen2011profitable,chen2022incentive,li2024bounding,cheng2022tight,cheng2019improved}. Other metrics include assessing the average and maximum gain per manipulation \cite{aleskerov1999degree} and counting the number of agents who benefit from manipulating \cite{andersson2014budget,andersson2014least}.




