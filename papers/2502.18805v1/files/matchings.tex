% \newpage

\newcommand{\men}{M}
\newcommand{\women}{W}
\newcommand{\man}{m}
\newcommand{\woman}{w}

\section{Stable Matchings}\label{sec:matching}

\eden{I'm not sure if it would be better to use men and women or students and universities}

\erel{``men and women'' are usually used when it is a one-to-one matching. ``students and universities'' are used when it is one-to-many.}

In this section, we consider mechanisms for stable matchings. 
Here, the $n$ agents are divided into two disjoint subsets, $\men$ and $\women$, that need to be matched to each other. The most common examples are men and women or students and universities. 
Each agent has a strict preference order over the agents in the other set and being unmatched -- for each $\man \in \men$, an order $\succ_{\man} $ over $\women\cup \{\phi\}$; and for each $\woman \in \women$ an order, $\succ_{\woman}$, over $\men\cup \{\phi\}$.  

A \emph{matching} between $\men$ to $\women$ is a mapping $\mu$ from $\men \cup \women$ to $\men \cup \women \cup \{\phi\}$ such that (1) $\mu(\man) \in \women \cup \{\phi\}$ for each $\man \in \men$, (2) and $\mu(\woman) \in \men \cup \{\phi\}$ for each $\woman \in \women$, and (3) $\mu(\man) = \woman$ if and only if $\mu(\woman) = \man$ for any $(\man, \woman) \in \men \times \women$. 
% \begin{align*} 
%     \forall \man \in \men \colon &\quad \mu(\man) \in \women \cup \{\phi\}\\
%     \forall \woman \in \women \colon &\quad \mu(\woman) \in \men \cup \{\phi\}\\
%     \forall \man, \woman \in \men \cup \women \colon &\quad  \mu(\man) = \woman  \iff \mu(\woman) = \man
% \end{align*}
When $\mu(a) = \phi$ it means that agent $a$ is unmatched under $\mu$. 
%
A matching is said to be \emph{stable} if (1) \emph{no} agent prefers being unmatched over their assigned match, and (2) there is \emph{no} pair $(\man, \woman) \in \men \times \women$ such that $\man$ prefers $\woman$ over his assigned match while $\woman$ prefers $\man$ over her assigned match -- $\woman \succ_{\man} \mu(\man)$ and $\man \succ_{\woman} \mu(\woman)$.


A mechanism in this context gets the preference orders of all agents and returns a stable matching.
\er{See \citet{gonczarowski2024structural} for a recent description of the structure of matching mechanisms.}

\erel{Consider citing other matching-related papers, such as  \citet{gonczarowski2014manipulation}, and other recent papers. }

\paragraph{Results} Our results for this problem are preliminary, so we provide only a brief overview here, with full descriptions and proofs in the appendix. We believe, however, that this is an important problem and that our new definition opens the door to many interesting questions.

We first analyze the deferred acceptance mechanism and prove that its RAT-degree is at least $1$ and at most $3$. The proof of the upper bound relies on \emph{truncation}, where an agent in $\women$ falsely reports preferring to remain unmatched over certain options. We further show that even without truncation, the RAT-degree is at most $5$.

Finally, we examine the Boston mechanism and establish an upper bound of $2$ on its RAT-degree.

\subsection{Deferred Acceptance (Gale-Shapley)}\label{sec:deferred-acceptance}

The \emph{deferred acceptance} algorithm \cite{gale1962college} is one of the most well-known mechanisms for computing a stable matching. 
In this algorithm, one side of the market --- here, $\men$ --- proposes, while the other side --- $\women$ --- accepts or rejects offers iteratively. 
%
\begin{toappendix}
\subsection{Deferred Acceptance (Gale-Shapley): descriptions and proofs}
The algorithm proceeds as follows:
\begin{enumerate}
    \item Each $\man \in \men$ proposes to his most preferred alternative according to $\succ_{\man}$ that has not reject him yet and that he prefers over being matched. 

    \item Each $\woman \in W$ tentatively accepts her most preferred proposal according to $\succ_{\woman}$ that she prefers over being matched, and rejects the rest.

    \item The rejected agents propose to their next most preferred choice as in step 1.

    \item The process repeats until no one of the rejected agents wishes to make a new proposal.

    \item The final matching is determined by the last set of accepted proposal.
\end{enumerate}
\end{toappendix}

It is well known that the mechanism is truthful for the proposing side ($\men$) but untruthful for the other side ($\women$).
That is, the agents in $\women$ may have an incentive to misreport their preferences to obtain a better match.
\er{Moreover, there is provably no mechanism for two-sided matching that is truthful for both sides.}
\erel{TODO: find citation for this}

\er{This section provides a more nuanced analysis of the amount of knowledge required by $\women$ agents to manipulate safely. We focus on a specific agent $w_1\in W$, 
with ranking $m_1 \succ_{w_1} m_2 \succ_{w_1} \cdots $.
There are three kinds of potential manipulations for any $w_1\in W$:}
\begin{enumerate}
\item 
Demoting some $m_j\in M$ from above $\phi$ to below $\phi$ (i.e., claiming that an acceptable partner is unacceptable for her). This is equivalent to reporting only a prefix of the ranking, sometimes called \emph{truncation}  \citep{roth1999truncation,ehlers2008truncation,coles2014optimal}.
\item 
Promoting some $m_j\in M$ from below $\phi$ to above $\phi$
(i.e., claiming that an unacceptable partner is acceptable for her). 
\item 
Reordering some $m_i,m_j\in M$ (i.e., claiming that she prefers $m_i$ to $m_j$ where in fact she prefers $m_j$ to $m_i$).
\end{enumerate}

\begin{lemmarep}
\label{lem:da-truncation}
\er{
In Deferred Acceptance with $k\geq 3$ known agents, there may be a safe and profitable truncation manipulation for $w_1$.
}
% EREL: stated as lemmas, to emphasize the kinds of manipulations used (for better understanding of the mechanism).
\end{lemmarep}
 
\begin{proof}
Suppose the $3$ known agents are as follows:
\begin{itemize}
\item Let $\woman_2 \in \women$ be an agent whose preferences are $\man_2 \succ_{\woman_2} \man_1 \succ_{\woman_2} \cdots $.

\item The preferences of $\man_1$ are $\woman_2 \succ_{\man_1} \woman_1 \succ_{\man_1} \cdots$.

\item The preferences of $\man_2$ are $\woman_1 \succ_{\man_2} \woman_2 \succ_{\man_2} \cdots$.
\end{itemize}
When $\woman_1$ is truthful, the resulting matching includes the pairs $(\man_1, \woman_2)$ and $(\man_2, \woman_1)$, since in this case it proceeds as follows:
\begin{itemize}
\item In the first step, all the agents in $\men$ propose to their most preferred option: $\man_1$ proposes to $\woman_2$ and $\man_2$ proposes to $\woman_1$.

Then, the agents in $\women$ tentatively accept their most preferred proposal among those received, as long as she prefers it to remaining unmatched: 
$\woman_1$ tentatively accepts $\man_2$ since she prefers him over being unmatched, and since $\man_2$ must be her most preferred option among the proposers as $\man_1$ (her top choice) did not propose to her.
Similarly, $\woman_2$ tentatively accepts $\man_1$.

\item In the following steps, more rejected agents in $\men$ might propose to $\woman_1$ and $\woman_2$, but they will not switch their choices, as they prefer $\man_2$ and $\man_1$, respectively. 

Thus, when the algorithm terminates $\man_1$ is matched to $\woman_2$ and $\man_2$ is matched to $\woman_1$, which means that $\woman_1$ is matched to her second-best option.
\end{itemize}


We shall now see that $\woman_1$ can increase her utility by truncating her preference order to $\man_1 \succ'_{\woman_1} \phi \succ'_{\woman_1} \man_2 \succ'_{\woman_1} \cdots $.
The following shows that in this case, the resulting matching includes the pairs $(\man_1, \woman_1)$ and $(\man_2, \woman_2)$, meaning that $\woman_1$ is matched to her most preferred option (instead of her second-best).
\begin{itemize}
\item In the first step, as before, $\man_1$ proposes to $\woman_2$ and $\man_2$ proposes to $\woman_1$.

However, here,  $\woman_1$ rejects $\man_2$ because, according to her false report, she prefers being unmatched over being matched to $\man_2$.
As before, $\woman_2$ tentatively accepts $\man_1$.


\item In the second step, $\man_2$, having been rejected by $\woman_1$, proposes to his second-best choice $\woman_2$.

Since $\woman_2$ prefers $\man_2$ over $\man_1$,  she rejects $\man_1$ and tentatively accepts $\man_2$.


\item In the third step, $\man_1$, having been rejected by  $\woman_2$, proposes to his second-best choice $\woman_1$.

$\woman_1$ now tentatively accepts $\man_1$ since according to her false report, she prefers him over being unmatched.


\item In the following steps, more rejected agents in $\men$ might propose to $\woman_1$ and $\woman_2$, but they will not switch their choices, as they prefer $\man_2$ and $\man_1$, respectively. 

Thus, when the algorithm terminates $\man_1$ will be matched to $\woman_1$ and and $\man_2$ will be matched to $\woman_2$. 
\end{itemize}

Thus, regardless of the reports of the remaining $(n-4)$ remaining (unknown) agents, $\woman_1$ strictly prefers to manipulate her preferences.
\end{proof}

In some settings, it is reasonable to assume that agents always prefer being matched if possible. In such cases, the mechanism is designed to accept only preferences over agents from the opposite set (or equivalently, orders where being unmatched is always the least preferred option). Clearly, under this restriction, truncation is not a possible manipulation.
We prove that even when truncation is not possible, the RAT-degree is bounded.

\begin{lemmarep}
\label{lem:da-switch}
\er{
In Deferred Acceptance without truncation, with $k\geq 5$ known agents, there may be a safe and profitable reorder manipulation for $w_1$.
}
\end{lemmarep}

\begin{proof}
Suppose the $5$ known agents are as follows:
\begin{itemize}
\item Let $\woman_2 \in \women$ be an agent whose preferences are $\man_1 \succ_{\woman_2} \man_2 \succ_{\woman_2} \man_3 \succ_{\woman_2} \cdots $.

\item Let $\woman_3 \in \women$ be an agent whose preferences are $\man_2 \succ_{\woman_3} \man_1 \succ_{\woman_3} \man_3 \succ_{\woman_2} \cdots $.

\item The preferences of $\man_1$ are $\woman_3 \succ_{\man_1} \woman_1 \succ_{\man_1} \woman_2 \succ_{\man_1} \cdots$.

\item The preferences of $\man_2$ are $\woman_1 \succ_{\man_2} \woman_3 \succ_{\man_2} \woman_2 \succ_{\man_2} \cdots$.

\item The preferences of $\man_3$ are $\woman_1 \succ_{\man_2} \woman_3 \succ_{\man_2} \woman_2 \succ_{\man_2} \cdots$.
\end{itemize}


When $\woman_1$ is truthful, the resulting matching includes the pairs $(\man_1, \woman_3)$, $(\man_2, \woman_1)$ and $(\man_3, \woman_2)$, since in this case it proceeds as follows:
\begin{itemize}
\item In the first step, all the agents in $\men$ propose to their most preferred option: $\man_1$ proposes to $\woman_3$, while $\man_2$ and $\man_3$ proposes to $\woman_1$.

Then, the agents in $\women$ tentatively accept their most preferred proposal among those received.
$\woman_1$ tentatively accepts $\man_2$ since he must be her most preferred option among the proposers -- as he is her second-best and her top choice, $\man_1$, did not propose to her; and rejects $\man_3$.
Similarly, $\woman_3$ tentatively accepts $\man_1$.
$\woman_2$ did not get any proposes.  

\item In the second step, $\man_3$, having been rejected by $\woman_1$, proposes to his second-best choice $\woman_3$.

Since $\woman_3$ prefers her current match $\man_1$ over $\man_3$,  she rejects $\man_3$.

\item In the third step, $\man_3$, having been rejected by $\woman_3$, proposes to his third-best choice $\woman_2$.

Since $\woman_2$ does not have a match, she tentatively accepts $\man_3$.


\item In the following steps, more rejected agents in $\men$ - that are not $\man_1, \man_2$ and $\man_3$, might propose to $\woman_1, \woman_2$ and $\woman_3$, but they will not switch their choices, as they can only be least preferred than their current match. 

Thus, when the algorithm terminates $\man_1$ is matched to $\woman_3$, $\man_2$ is matched to $\woman_1$, and $\man_3$ is matched to $\woman_2$,
which means that $\woman_1$ is matched to her second-best option.
\end{itemize}

But if $\woman_1$ swaps $m_2$ and $m_3$ and reports $\man_1 \succ'_{\woman_1} \man_3 \succ'_{\woman_1} \man_2 \succ'_{\woman_1} \cdots$, then the resulting matching includes the pairs $(\man_1, \woman_1)$, $(\man_2, \woman_3)$ and $(\man_3, \woman_2)$, meaning that $\woman_1$ is matched to her most preferred option (instead of her second-best).
\begin{itemize}
\item In the first step, as before, $\man_1$ proposes to $\woman_3$, while $\man_2$ and $\man_3$ proposes to $\woman_1$.

However, here,  $\woman_1$ tentatively accepts $\man_3$ and rejects $\man_2$.
As before, $\woman_3$ tentatively accepts $\man_1$ and $\woman_2$ did not get any proposes.


\item In the second step, $\man_2$, having been rejected by $\woman_1$, proposes to his second-best choice $\woman_3$.

Since $\woman_3$ prefers $\man_2$ over $\man_1$,  she rejects $\man_1$ and tentatively accepts $\man_2$.


\item In the third step, $\man_1$, having been rejected by  $\woman_3$, proposes to his second-best choice $\woman_1$.

$\woman_1$ tentatively accepts $\man_1$ since according to her false report, she prefers him over $\man_3$.


\item In the fourth step, $\man_3$, having been rejected by  $\woman_1$, proposes to his second-best choice $\woman_3$.

$\woman_3$ prefers her current match $\man_2$, and thus rejects $\man_3$.


\item In the fifth step, $\man_3$, having been rejected by  $\woman_3$, proposes to his third-best choice $\woman_2$.

As $\woman_2$ does not have a match, she tentatively accepts $\man_3$.


\item In the following steps, more rejected agents in $\men$ - that are not $\man_1, \man_2$ and $\man_3$, might propose to $\woman_1, \woman_2$ and $\woman_3$, but they will not switch their choices, as they can only be least preferred than their current match. 

Thus, when the algorithm terminates $\man_1$ is matched to $\woman_1$, $\man_2$ is matched to $\woman_3$, and $\man_3$ is matched to $\woman_2$. 
\end{itemize}


Thus, regardless of the reports of the remaining $(n-6)$ remaining (unknown) agents report, $\woman_1$ strictly prefers to manipulate her preferences.
\end{proof}

\er{We conjecture that the numbers $3$ and $5$ are tight, but currently have only a weaker lower bound of $1$, which follows from the following lemma:}
\begin{lemmarep}
\er{
In Deferred Acceptance, no agent has a safe manipulation.
}
\end{lemmarep}
\begin{proof}
We consider each of the three possible kinds of manipulations, and show that each of them is not safe for $w_1$.

\paragraph{Demoting some $m_j\in M$ to below $\phi$.}
For some unknown agents' rankings, $m_j$ is the only one who proposes to $w_1$.
Therefore, when $w_1$ demotes $m_j$ she rejects him and remains unmatched, but when she is truthful she is matched to him, which is better for her.

\paragraph{Promoting some $m_j\in M$ to above $\phi$.}
An analogous argument works in this case too.

\paragraph{Reordering some $m_i,m_j\in M$.}
For some unknown agents' rankings, $m_i$ and $m_j$ are the only ones who propose to $w_1$, and they do so simultaneously. Therefore, when $w_1$ is truthful she is matched to the one she prefers, but when she manipulates she rejects him in favor of the less-preferred one.

In all three cases, the manipulation is not safe.
\end{proof}

Combining the above lemmas gives:

\begin{theorem}
\label{prop-def-acc-trunc} 
\label{prop-def-acc-no-trunc}
The RAT-degree of Deferred Acceptance is at least $1$.

It is at most $3$ when truncation is allowed, 
and at most $5$ when truncation is not allowed.
\end{theorem}
 

\subsection{Boston Mechanism}
The \emph{Boston} mechanism \cite{abdulkadirouglu2003school} is a widely used mechanism for assigning students or schools. 
It is not truthful for both sides. Moreover, we show that it can be safely manipulated with little information.
% Unlike the Deferred Acceptance algorithm, it  prioritizes higher-ranked choices in a sequential manner.
\begin{toappendix}
\subsection{Boston Mechanism: descriptions and proofs}
The mechanism proceeds in rounds as follows:
\begin{enumerate}
    \item Each $\man \in \men$ proposes to his most preferred alternative according $\succ_{\man}$ that has not yet rejected him and is still available.

    \item Each $\woman \in \women$ (permanently) accepts her most preferred proposal according to $\succ_{\woman}$ and rejects the rest. Those who accept a proposal become unavailable. 

    \item The rejected agents propose to their next most preferred choice as in step $1$.

    \item The process repeats until all agents are either assigned or have exhausted their preference lists.
\end{enumerate}

\end{toappendix}


\begin{lemmarep}
\label{lem:prop-boston}
\er{
In the Boston Mechanism with $k\geq 2$ known agents, there may be a safe and profitable reorder manipulation for some $m_1\in M$.
}
\end{lemmarep}
\begin{proof}
Let $\man_1 \in \men$ be an agent with preferences $\woman_1 \succ_{\man_1} \woman_2 \succ_{\man_1} \cdots$.
Suppose the two known agents are as follows:
\begin{itemize}
\item Let $\man_2 \in \men$ be an agent whose preferences are similar to $m_1$, $\woman_1 \succ_{\man_2} \woman_2 \succ_{\man_2} \cdots$.
\item The preferences of $\woman_1$ are $\man_2 \succ_{\woman_1} \man_1 \succ_{\woman_1} \cdots$.
\end{itemize}

When $\man_1$ reports truthfully, the mechanism proceeds as follows: In the first round, both $\man_1$ and $\man_2$ proposes to $\woman_1$. Since $\woman_1$ prefers $\man_2$, she rejects $\man_1$ and becomes unavailable. Thus, in the second round, $\man_1$ proposes to $\woman_2$.

We prove that $\man_1$ has a safe-and-profitable manipulation: $\woman_2 \succ'_{\man_1} \woman_3 \succ'_{\man_1} \cdots \succ'_{\man_1} \woman_1$.
The manipulation is safe since $\man_1$ never had a chance to be matched with $\woman_1$ (regardless of his report). 
The manipulation is profitable as there exists a case where the manipulation improves $\man_1$’s outcome. Consider the case where the top choice of $\woman_2$ is $\man_1$ and there is another agent, $\man_3$, whose top choice is $\woman_2$. 
Notice that $\woman_2$ prefers $\man_1$ over $\man_3$. 
When $\man_1$ reports truthfully, then in the first round, $\man_3$ proposes to $\woman_2$ and gets accepted, making her unavailable by the time $\man_1$ reaches her in the second round.
However, if $\man_1$ manipulates, he proposes to $\woman_2$ in the first round, and she will accept him (as she prefers him over $\man_3$). This guarantees that $\man_1$ is matched to $\woman_2$, improving his outcome compared to truthful reporting.
\end{proof}

The lemma implies:
\begin{theoremrep}
    \label{prop-boston}
    The RAT-degree of the Boston mechanism is at most $2$.
\end{theoremrep}

We have seen two matching mechanisms with a low RAT-degree -- at most a small constant independent of $n$. This raises an open question:

\begin{open}
Is there a stable matching mechanism with RAT-degree in $\Omega(n)$?
\end{open}
% \eden{\url{https://web.stanford.edu/~alroth/papers/bostonMay182006.pdf}}