% \newpage
\section{Single-Winner Ranked Voting}\label{sec:single-winner-voting}
We consider $n$ voters (the agents) who need to elect one winner from a set $C$ of $m$ \emph{candidates}.
The agents' preferences are given by strict linear orderings $\succ_i$ over the candidates.

When there are only two candidates, the majority rules and its variants (weighted majority rules)  are truthful.
With three or more candidates, the   Gibbard--Satterthwaite Theorem 
\cite{gibbard1973manipulation,satterthwaite1975strategy}
implies that the only truthful rules are dictatorships. 
Our goal is to find non-dictatorial rules with a high RAT-degree.

%\eden{if we have time: maybe to change 'Alice' to 'voter $v_1$'} \erel{I think ``agent'' is better as it is more consistent with the rest of the paper.}

Throughout the analysis, we consider a specific agent Alice, who looks for a safe profitable manipulation. Her true ranking is $c_m \succ_A \cdots \succ_A c_1$.
We assume that, for any $j>i$, Alice strictly prefers a victory of $c_j$ to a tie between $c_j$ and $c_i$, and strictly prefers this tie to a victory of $c_i$.%
\footnote{We could also assume that ties are broken at random, but this would require us to define preferences on lotteries, which we prefer to avoid in this paper.}




\subsection{Positional voting rules: general bounds}
\newcommand{\scorevector}{\mathbf{s}}
\newcommand{\score}{\operatorname{score}}
A \emph{positional voting rule}
is parameterized by a vector of scores, $\scorevector=(s_1,\ldots,s_m)$, where $s_1\leq \cdots \leq  s_m$ and $s_1 < s_m$.
Each voter reports his entire ranking of the $m$ candidates. Each such ranking is translated to an assignment of a score to each candidate: the lowest-ranked candidate is given a score of $s_1$, the second-lowest candidate is given $s_2$, etc., and the highest-ranked candidate is given a score of $s_m$. 
The total score of each candidate is the sum of scores he received from the rankings of all $n$ voters. The winner is the candidate with the highest total score. 

Formally, for any subset $N'\subseteq N$ and any candidate $c\in C$, we denote by $\score_{N'}(c)$ the total score that $c$ receives from the votes of the agents in $N'$. Then the winner is $\arg\max_{c\in C}\score_N(c)$. If there are several agents with the same maximum score, then the outcome is considered a tie.

Common special cases of positional voting are \emph{plurality voting}, in which $\scorevector = (0,0,0,\ldots,0,1)$, and 
\emph{anti-plurality voting}, in which $\scorevector = (0,1,1,\ldots,1,1)$.
By the Gibbard--Satterthwaite theorem, all positional voting rules are manipulable, so their RAT-degree is smaller than $n$.
But, as we will show next, some positional rules have a higher RAT-degree than others.

\paragraph{Results.} 
\er{
We will show that all positional voting rules have an RAT-degree between $\approx n/m$ and $\approx n/2$. These bounds are almost tight: the 
upper bound is attained by plurality and the lower bound is attained by anti-plurality (up to small additive constants).
}

In the upcoming lemmas, we identify the manipulations that are safe and profitable for Alice under various conditions on the score vector $\scorevector$. We assume throughout that there are $m\geq 3$ candidates, and that $n$ is sufficiently large.
We allow an agent to abstain, which means that his vote gives the same score to all candidates.%
\footnote{
We need the option to abstain in order to avoid having different constructions for even $n$ and odd $n$; see the proofs for details.
}


The first lemma implies a lower bound of $\approx n/m$ on the RAT-degree of positional voting rules.
\begin{lemmarep}
\label{lem:lower-positional}
In any positional voting rule for $m\geq 3$ candidates,
if the number of known agents is at most $(n+1)/m - 2$,
then Alice has no safe profitable manipulation.
\end{lemmarep}
\begin{proofsketch}
For any combination of rankings of the known agents, it is possible that the unknown agents vote in a way that balances out the votes of the known agents, such that all agents have almost the same score; if Alice is truthful, there is a tie between two candidates, and if she manipulates, the worse of these candidates win.
\end{proofsketch}
\begin{proof}
If a manipulation does not change any score, then it is clearly not profitable. So suppose the manipulation changes the score of some candidates. Let $c_i$ be a candidate whose score increases by the largest amount by the manipulation (note that it cannot be Alice's top candidate). If there are several such candidates, choose the one that is ranked highest by Alice. Let $c_j$ be some candidate that Alice prefers to $c_i$.

To show that the manipulation is not safe, suppose the unknown agents vote as follows.
\begin{itemize}
\item For every known agent $a_i$, some $m-1$ unknown agents vote with ``rotated'' variants of $a_i$'s ranking (e.g. if $a_i$ ranks $c_1\succ c_2\succ c_3 \succ c_4$, then three unknown agents rank $c_2\succ c_3 \succ c_4 \succ c_1$,
$c_3 \succ c_4 \succ c_1\succ c_2$ 
and 
$c_4 \succ c_1\succ c_2\succ c_3$).
\item Additional $m-1$ unknown agents vote with rotated variants of Alice's true ranking.

\item Additional $\ceil{m/2-1}$ unknown agents rank 
rank $c_i$ first and $c_j$ second, 
and additional $\ceil{m/2-1}$ unknown agents rank $c_j$ first and $c_i$ second. These $2\ceil{m/2-1}$ unknown agents rank the remaining $m-2$ candidates such that each candidate appears last at least once (note that $2\ceil{m/2-1}\geq m-2$ so this is possible).
\item The other unknown agents, if any, abstain.
\end{itemize}
When Alice is truthful, the scores are as follows:
\begin{align*}
\score_N(c_i) = \score_N(c_j) &= 
(k+1)\cdot \sum_{j=1}^m s_j + 
(\ceil{m/2-1})(s_m+s_{m-1})
\\
\score(c_{\ell}) &= 
(k+1)\cdot \sum_{j=1}^n s_j + 
S_{\ell}
&& \forall \ell\neq i,j
\end{align*}
where $S_{\ell}$ is the sum of some $2\ceil{m/2-1}$ scores; all these scores are at most $s_{m-1}$, and some of them are equal to $s_1$. As $s_1<s_m$ for any score vector, 
$S_{\ell} < (\ceil{m/2-1})(s_m+s_{m-1})$ for all $\ell\neq i,j$. Hence, the two candidates $c_i$ and $c_j$ both have a score strictly higher than every other candidate.

When Alice manipulates, the score of $c_i$ increases by the largest amount, so $c_i$ wins. Since $c_j \succ_A c_i$, this outcome is worse for Alice than the tie.

The number of agents required is 
\begin{align*}
k + (m-1)k + (m-1) + 1 + 2\ceil{m/2-1}
\leq 
m (k+2) - 1.
\end{align*}
The condition on $k$ in the lemma ensures that $n$ is at least as large.
\end{proof}


We now prove an upper bound of $\approx n/2$ on the RAT-degree. We need several lemmas.



%The following lemma is a special case of \Cref{lem:st1>st}; we prove it explicitly as a warm-up. 
\begin{lemmarep}
\label{lem:s2>s1}
Let $m\geq 3$ and $n\geq 2m$.
If $s_2 > s_1$
and there are $k\geq \ceil{n/2}+1$ known agents,
then
switching the bottom two candidates ($c_2$ and $c_1$) may be a safe profitable manipulation for Alice.
\end{lemmarep}
\begin{proofsketch}
For some votes by the known agents,
$c_1$ has no chance to win, so the worst candidate for Alice that could win is $c_2$. Therefore, switching $c_1$ and $c_2$ cannot harm, but may help a better candidate win over $c_2$.
\end{proofsketch}

\iffalse % EREL: tried an alternative proof --- using n-k
\begin{proof}
Suppose there is a subset $K$ of $k$ known agents, who vote as follows (Note that the assumption on $k$ implies $k\geq n-k+2$):
\begin{itemize}
\item $n-k$ known agents rank $c_2 \succ c_m \succ \cdots \succ c_1$.
\item Two known agents rank $c_m \succ c_2 \succ
\cdots \succ c_1$.
\item The remaining known agents (if any) abstain.
\end{itemize}


We first show that $c_1$ cannot win. To this end, we show that the difference in scores between $c_2$ and $c_1$ is always strictly positive.
\begin{itemize}
\item The difference in scores given by the known agents is 
\begin{align*}
\score_K(c_2)-\score_K(c_1) =
&
(n-k)(s_m-s_1) 
+ 2(s_{m-1}-s_1).
\end{align*}
\item The $n-k$ agents not in $K$ (including Alice) 
can reduce the score-difference by at most 
$(n-k)(s_m-s_1)$.
Therefore, 
\begin{align*}
\score_N(c_2)-\score_N(c_1) 
\geq  &
(n-k)(s_m-s_1) 
+ 2(s_{m-1}-s_1)
-(n-k)(s_m-s_1)
\\
= &
2(s_{m-1} - s_1),
\end{align*}
which is positive 
by the assumption $s_2>s_1$.
So $c_1$ has no chance to win or even tie.
\end{itemize}
Therefore, switching $c_2$ and $c_1$ can never harm Alice --- the manipulation is safe.

Next, we show that the manipulation can help $c_m$ win, when the agents not in $K$ vote as follows:
\begin{itemize}
\item $n-k-2$ unknown agents rank $c_m\succ c_2\succ \cdots $,
where each candidate except $c_1,c_2,c_m$ is ranked last by at least one voter (here we use the assumption $n\geq 2m$).
\item One unknown agent ranks 
$c_2\succ \cdots \succ c_m \succ c_1$;
\item Alice votes truthfully $c_m\succ  \cdots \succ c_2 \succ c_1$.  
\end{itemize}
Then,
\begin{align*}
\score_N(c_2) - \score_N(c_m)
=&
(n-k)(s_{m}-s_{m-1}) 
+ 2(s_{m-1}-s_{m})
\\
&+
(n-k-2)(s_{m-1}-s_{m}) 
+ (s_m-s_2)
+ (s_2-s_m)
\\
=&
0.
\end{align*}
Moreover, for any $j\not\in\{1,2,m\}$, the score of $c_j$ is even lower (here we use the assumption that $c_j$ is ranked last by at least one unknown agent):
\begin{align*}
\score_N(c_2) - \score_N(c_j)
\geq &
(n-k)(s_{m}-s_{m-2}) 
+ 2(s_{m-1}-s_{m-2})
\\
&+
(n-k-3)(s_{m-1}-s_{m-2}) 
+ (s_{m-1}-s_1)
+ (s_m-s_{m-1})
+ (s_2-s_{m-1})
\\
\geq & (s_{m-1}-s_{1})
+ (s_2-s_{m-1})
\\
= & s_2 - s_1,
\end{align*}
which is positive by the lemma assumption.
Therefore, when Alice is truthful, the outcome is a tie between $c_m$ and $c_2$.

If Alice switches $c_1$ and $c_2$, then the score of $c_2$ decreases by $s_2-s_1$, which is positive by the lemma assumption, and the scores of all other candidates except $c_1$ do not change. So $c_m$ wins, which is better for Alice than a tie.
Therefore, the manipulation is profitable.
\end{proof}
\fi % alternative proof - using n-k


\begin{proof}
Suppose there is a subset $K$ of $\ceil{n/2}+1$ known agents, who vote as follows:
\begin{itemize}
\item $\floor{n/2}-1$ known agents rank $c_2 \succ c_m \succ \cdots \succ c_1$.
\item Two known agents rank $c_m \succ c_2 \succ
\cdots \succ c_1$. 
\item In case $n$ is odd, the remaining known agent abstains.
\end{itemize}
We first show that $c_1$ cannot win. To this end, we show that the difference in scores between $c_2$ and $c_1$ is always strictly positive.
\begin{itemize}
\item The difference in scores given by the known agents is 
\begin{align*}
\score_K(c_2)-\score_K(c_1) =
&
(\floor{n/2}-1)(s_m-s_1) 
+ 2(s_{m-1}-s_1).
\end{align*}
\item There are
$\floor{n/2}-1$ agents not in $K$ (including Alice).
These agents can reduce the score-difference by at most 
$(\floor{n/2}-1)(s_m-s_1)$.
Therefore, 
\begin{align*}
\score_N(c_2)-\score_N(c_1) \geq 2(s_{m-1}-s_1),
\end{align*}
which is positive 
by the assumption $s_2>s_1$.
So $c_1$ has no chance to win or even tie.
\end{itemize}
Therefore, switching $c_2$ and $c_1$ can never harm Alice --- the manipulation is safe.

Next, we show that the manipulation can help $c_m$ win, when the agents not in $K$ vote as follows:
\begin{itemize}
\item $\floor{n/2}-3$ unknown agents rank $c_m\succ c_2\succ \cdots $,
where each candidate except $c_1,c_2,c_m$ is ranked last by at least one voter (here we use the assumption $n\geq 2m$).
\item One unknown agent ranks 
$c_2\succ \cdots \succ c_m \succ c_1$;
\item Alice votes truthfully $c_m\succ  \cdots \succ c_2 \succ c_1$.  
\end{itemize}
Then,
\begin{align*}
\score_N(c_2) - \score_N(c_m)
=&
(\floor{n/2}-1)(s_{m}-s_{m-1}) 
+ 2(s_{m-1}-s_{m})
\\
&+
(\floor{n/2}-3)(s_{m-1}-s_{m}) 
+ (s_m-s_2)
+ (s_2-s_m)
\\
=&
0.
\end{align*}
Moreover, for any $j\not\in\{1,2,m\}$, the score of $c_j$ is even lower (here we use the assumption that $c_j$ is ranked last by at least one unknown agent):
\begin{align*}
\score_N(c_2) - \score_N(c_j)
\geq &
(\floor{n/2}-1)(s_{m}-s_{m-2}) 
+ 2(s_{m-1}-s_{m-2})
\\
&+
(\floor{n/2}-4)(s_{m-1}-s_{m-2}) 
+ (s_{m-1}-s_1)
+ (s_m-s_{m-1})
+ (s_2-s_{m-1})
\\
\geq & (s_{m-1}-s_{1})
+ (s_2-s_{m-1})
\\
= & s_2 - s_1,
\end{align*}
which is positive by the lemma assumption.
Therefore, when Alice is truthful, the outcome is a tie between $c_m$ and $c_2$.

If Alice switches $c_1$ and $c_2$, then the score of $c_2$ decreases by $s_2-s_1$, which is positive by the lemma assumption, and the scores of all other candidates except $c_1$ do not change. So $c_m$ wins, which is better for Alice than a tie.
Therefore, the manipulation is profitable.
\end{proof}

\iffalse
\begin{lemmarep}
Let $m\geq 4$ and $n\geq 2m$.
If $s_3 > s_2 = s_1$,
and there are $k\geq \ceil{n/2}+1$ known agents,
then
switching $c_3$ and $c_2$ may be a safe profitable manipulation for Alice.
\end{lemmarep}
\begin{proofsketch}
The idea is that, for some votes by the known agents, both $c_2$ and $c_1$ have has no chance to win, 
so the worst candidate that could win is $c_3$.
Therefore, switching $c_2$ and $c_3$ cannot harm, but can help better candidates win over $c_3$.
\end{proofsketch}
\begin{proof}
Suppose there is a subset $K$ of $\ceil{n/2}+1$ known agents, who vote as follows:
\begin{itemize}
\item $\floor{n/2}-1$ known agents rank $c_3 \succ c_m \succ \cdots \succ c_2 \succ c_1$.
\item Two known agents rank $c_m \succ c_3 \succ
\cdots \succ c_2 \succ c_1$. 
\item In case $n$ is odd, the remaining known agent abstains.
\end{itemize}
We first show that both $c_1$ and $c_2$ cannot win. 
Note that, by the lemma assumption $s_1=s_2$, these two candidates receive exactly the same score by all known agents. We show that the difference in scores between $c_3$ and both $c_2$ and $c_1$ is always strictly positive.
\begin{itemize}
\item The difference in scores given by the known agents is 
\begin{align*}
\score_K(c_3)-\score_K(c_2) =
&
(\floor{n/2}-1)(s_m-s_1) 
+ (s_{m-1}-s_1).
\\
%=&
%(\floor{n/2})(s_m-s_1) 
\end{align*}
\item There are
$\floor{n/2}-1$ agents not in $K$ (including Alice).
These agents can reduce the score-difference by at most 
$(\floor{n/2}-1)(s_m-s_1)$.
Therefore, 
\begin{align*}
\score_N(c_3)-\score_N(c_2) \geq (s_{m-1}-s_1),
\end{align*}
which is positive 
by the assumptions $m\geq 4$ and $s_3>s_2$.
So both $c_2$ and $c_1$ have no chance to win or even tie.
\end{itemize}
Therefore, switching $c_3$ and $c_2$ can never harm Alice --- the manipulation is safe.

Next, we show that the manipulation can help $c_m$ win. We compute the score-difference between $c_m$ and the other candidates with and without the manipulation. 

Suppose that the agents not in $K$ vote as follows:
\begin{itemize}
\item $\floor{n/2}-3$ unknown agents rank $c_m\succ c_3\succ \cdots $,
where each candidate except $c_1,c_2,c_3,c_m$ is ranked last by at least one voter (here we use the assumption $n\geq 2m$).
\item One unknown agent ranks 
$c_3\succ \cdots \succ c_m \succ c_2 \succ c_1$;
\item Alice votes truthfully $c_m\succ  \cdots \succ c_3 \succ c_2 \succ c_1$.  
\end{itemize}
Then,
\begin{align*}
\score_N(c_3) - \score_N(c_m)
=&
(\floor{n/2}-1)(s_{m}-s_{m-1}) 
+ 2(s_{m-1}-s_{m})
\\
&+
(\floor{n/2}-3)(s_{m-1}-s_{m}) 
+ (s_m-s_3)
+ (s_3-s_m)
\\
=&
0.
\end{align*}
Moreover, for any $j\not\in\{1,2,3,m\}$, the score of $c_j$ is even lower (here we use the assumption that $c_j$ is ranked last by at least one unknown agent):
\begin{align*}
\score_N(c_3) - \score_N(c_j)
\geq &
(\floor{n/2}-1)(s_{m}-s_{m-2}) 
+ 2(s_{m-1}-s_{m-2})
\\
&+
(\floor{n/2}-4)(s_{m-1}-s_{m-2}) 
+ (s_{m-1}-s_1)
+ (s_m-s_{m-1})
+ (s_3-s_{m-1})
\\
\geq & (s_{m-1}-s_{1})
+ (s_3-s_{m-1})
\\
= & s_3 - s_1,
\end{align*}
which is positive by the lemma assumption.
Therefore, when Alice is truthful, the outcome is a tie between $c_m$ and $c_3$.

If Alice switches $c_2$ and $c_3$, then the score of $c_3$ decreases by $s_3-s_2$, which is positive by the lemma assumption, and the scores of all other candidates except $c_2$ do not change. So $c_m$ wins, which is better for Alice than a tie.
Therefore, the manipulation is profitable.
\end{proof}
\fi

\Cref{lem:s2>s1} can be generalized as follows. 

\begin{lemmarep}
\label{lem:st1>st}
Let $m\geq 3$ and $n\geq 2m$.
For every integer $t \in \{1,\ldots, m-2\}$,
if $s_{t+1} > s_t = \cdots = s_1$,
and there are $k\geq \ceil{n/2}+1$ known agents,
then switching $c_{t+1}$ and $c_t$ may be a safe profitable manipulation.
\end{lemmarep}
%Note that \Cref{lem:s2>s1} is the special case $t=1$.

\begin{proofsketch}
For some votes by the known agents,
all candidates $c_1,\ldots,c_t$ have no chance to win, 
so the worst candidate for Alice that could win is $c_{t+1}$.
Therefore, switching $c_t$ and $c_{t+1}$ cannot harm, but can help better candidates win over $c_{t+1}$.
\end{proofsketch}
\begin{proof}
Suppose there is a subset $K$ of $\ceil{n/2}+1$ known agents, who vote as follows:
\begin{itemize}
\item $\floor{n/2}-1$ known agents rank $c_{t+1} \succ c_m$ first and rank $c_t \succ \cdots \succ c_1$ last.
\item Two known agents rank $c_m \succ c_{t+1}$ first and rank $c_t \succ \cdots \succ c_1$ last.
\item In case $n$ is odd, the remaining known agent abstains.
\end{itemize}
We first show that the $t$ worst candidates for Alice ($c_1,\ldots, c_t$) cannot win. 
Note that, by the lemma assumption $s_t = \cdots = s_1$, all these candidates receive exactly the same score by all known agents. We show that the difference in scores between $c_{t+1}$ and $c_t$ (and hence all $t$ worst candidates) is always strictly positive.
\begin{itemize}
\item The difference in scores given by the known agents is 
\begin{align*}
\score_K(c_{t+1})-\score_K(c_t) =
&
(\floor{n/2}-1)(s_m-s_1) 
+ (s_{m-1}-s_1).
\end{align*}
\item There are
$\floor{n/2}-1$ agents not in $K$ (including Alice).
These agents can reduce the score-difference by at most 
$(\floor{n/2}-1)(s_m-s_1)$.
Therefore, 
\begin{align*}
\score_N(c_{t+1})-\score_N(c_t) \geq (s_{m-1}-s_1),
\end{align*}
which is positive 
by the assumption $m-2 \geq t$ and $s_{t+1}>s_t$.
So no candidate in $c_1,\ldots,c_t$ has a chance to win or even tie.
\end{itemize}
Therefore, switching $c_{t+1}$ and $c_t$ can never harm Alice --- the manipulation is safe.

Next, we show that the manipulation can help $c_m$ win. We compute the score-difference between $c_m$ and the other candidates with and without the manipulation. 

Suppose that the agents not in $K$ vote as follows:
\begin{itemize}
\item $\floor{n/2}-3$ unknown agents rank $c_m\succ c_{t+1}\succ \cdots $,
where each candidate in $c_{t+2},\ldots,c_{m-1}$ is ranked last by at least one voter (here we use the assumption $n\geq 2m$).
\item One unknown agent ranks 
$c_{t+1}\succ \cdots \succ c_m \succ c_t \succ \cdots \succ c_1$;
\item Alice votes truthfully $c_m\succ  \cdots \succ c_{t+1} \succ  c_t \succ \cdots \succ c_1$.  
\end{itemize}
Then,
\begin{align*}
\score_N(c_{t+1}) - \score_N(c_m)
=&
(\floor{n/2}-1)(s_{m}-s_{m-1}) 
+ 2(s_{m-1}-s_{m})
\\
&+
(\floor{n/2}-3)(s_{m-1}-s_{m}) 
+ (s_m-s_{t+1})
+ (s_{t+1}-s_m)
\\
=&
0.
\end{align*}
Moreover, for any $j\in\{t+2,\ldots,m-1\}$, the score of $c_j$ is even lower (here we use the assumption that $c_j$ is ranked last by at least one unknown agent):
\begin{align*}
\score_N(c_{t+1}) - \score_N(c_j)
\geq &
(\floor{n/2}-1)(s_{m}-s_{m-2}) 
+ 2(s_{m-1}-s_{m-2})
\\
&+
(\floor{n/2}-4)(s_{m-1}-s_{m-2}) 
+ (s_{m-1}-s_1)
+ (s_m-s_{m-1})
+ (s_{t+1}-s_{m-1})
\\
\geq & (s_{m-1}-s_{1})
+ (s_{t+1}-s_{m-1})
\\
= & s_{t+1} - s_1,
\end{align*}
which is positive by the lemma assumption.
Therefore, when Alice is truthful, the outcome is a tie between $c_m$ and $c_{t+1}$.

If Alice switches $c_t$ and $c_{t+1}$, then the score of $c_{t+1}$ decreases by $s_{t+1}-s_t$, which is positive by the lemma assumption, and the scores of all other candidates except $c_t$ do not change. As $c_t$ cannot win, $c_m$ wins, which is better for Alice than a tie.
Therefore, the manipulation is profitable.
\end{proof}

\begin{lemmarep}
\label{lem:sm>sm1}
Let $m\geq 3$ and $n\geq 4$.
If $s_m > s_{m-1}$
and there are $k\geq \ceil{n/2}+1$ known agents,
then
switching the top two candidates ($c_m$ and $c_{m-1}$) may be a safe profitable manipulation for Alice.
\end{lemmarep}
\begin{proofsketch}
For some votes by the known agents, the manipulation is safe since $c_m$ has no chance to win, and it is profitable as it may help $c_{m-1}$ win over worse candidates.
\end{proofsketch}
\begin{proof}
Suppose there is a subset $K$ of $\ceil{n/2}+1$ known agents, who vote as follows:
\begin{itemize}
\item $\floor{n/2}-1$ known agents rank $c_{m-2} \succ c_{m-1} \succ \cdots \succ c_m$.
\item One known agent ranks $c_{m-2} \succ c_m \succ
c_{m-1} \succ \cdots $. 
\item One known agent ranks $c_{m-1} \succ c_{m-2} \succ \cdots \succ c_m$. 
\item In case $n$ is odd, the remaining known agent abstains.
\end{itemize}
We first show that $c_m$ cannot win. To this end, we show that the difference in scores between $c_{m-2}$ and $c_m$ is always strictly positive.
\begin{itemize}
\item The difference in scores given by the known agents is 
\begin{align*}
\score_K(c_{m-2})-\score_K(c_m) =
&
(\floor{n/2}-1)(s_m-s_1) 
+ (s_m-s_{m-1})
+ (s_{m-1}-s_1).
%+ (s_{m-1}-s_1) 
%+ (s_m-s_2)
\\
=&
(\floor{n/2})(s_m-s_1) 
%(s_m-s_1)
%+ 
%(s_{m-1}-s_2).
\end{align*}
\item There are
$\floor{n/2}-1$ agents not in $K$ (including Alice).
These agents can reduce the score-difference by at most 
$(\floor{n/2}-1)(s_m-s_1)$.
Therefore, 
\begin{align*}
\score_N(c_{m-2})-\score_N(c_m) \geq (s_m-s_1),
\end{align*}
which is positive for any score vector.
So $c_m$ has no chance to win or even tie.
\end{itemize}
Therefore, switching $c_m$ and $c_{m-1}$ can never harm Alice --- the manipulation is safe.

Next, we show that the manipulation can help $c_{m-1}$ win. We compute the score-difference between $c_{m-1}$ and the other candidates with and without the manipulation. 

Suppose that the agents not in $K$ vote as follows:
\begin{itemize}
\item the $\floor{n/2}-2$ unknown agents%
\footnote{Here we use the assumption $n\geq 4$.}
rank $c_{m-1}\succ c_{m-2}\succ \cdots $.
\item Alice votes truthfully $c_m\succ c_{m-1}\succ c_{m-2} \cdots \succ c_1$.
\end{itemize}
Then,
\begin{align*}
\score_N(c_{m-2}) - \score_N(c_{m-1})
=&
(\floor{n/2}-1)(s_{m}-s_{m-1}) 
+ (s_m-s_{m-2})
+ (s_{m-1}-s_{m})
\\
&+
(\floor{n/2}-2)(s_{m-1}-s_{m}) 
+ (s_{m-2}-s_{m-1})
\\
=&
(s_{m}-s_{m-1}),
\end{align*}
which is positive by the assumption $s_m>s_{m-1}$.
The candidates $c_{j<m-2}$ are ranked even lower than $c_{m-1}$ by all agents. Therefore the winner is $c_{m-2}$.

If Alice switches $c_{m-1}$ and $c_m$, then the score of $c_{m-1}$ increases by $s_m-s_{m-1}$ and the scores of all other candidates except $c_m$ do not change. Therefore, 
$\score_N(c_{m-2}) - \score_N(c_{m-1})$ becomes $0$, and there is a tie between $c_{m-2}$ and $c_{m-1}$, which is better for Alice by assumption.
Therefore, the manipulation is profitable.

\iffalse % EREL: old proof, without abstinence
Specifically, suppose the $\ceil{n/2}+1$ known-agents rank as follows:
\begin{itemize}
\item $\floor{n/2}-1$ known agents rank $c_1 \succ c_{m-1} \succ \cdots \succ c_m$;
\item One known agent ranks $c_{m-1} \succ c_1 \succ \cdots \succ c_m$. 
\item In case $n$ is odd, another known agent ranks $c_{m-1} \succ c_1 \succ \cdots \succ c_m$.
\item One known agent ranks $c_1 \succ  \cdots \succ  c_m \succ c_{m-1}$;
\end{itemize}
We first show that $c_m$ cannot win. To this end, we show that the difference in scores between $c_1$ and $c_m$ is always strictly positive.
\begin{itemize}
\item The difference in scores given by the known agents is 
\begin{align*}
&
(\floor{n/2}-1)(s_m-s_1) + (1+n\bmod 2)(s_{m-1}-s_1) + (s_m-s_2)
\\
=&
(\floor{n/2}-1)(s_m-s_1) + 
(s_m-s_1)
+ 
(s_{m-1}-s_2)
+
(n\bmod 2)(s_{m-1}-s_1)
\end{align*}
\item There are
$\floor{n/2}-1$ agents not in $K$ (including Alice).
These agents can reduce the score-difference by at most 
$(\floor{n/2}-1)(s_m-s_1)$.
Therefore, the score-difference is at least 
$(s_m-s_1)
+ 
(s_{m-1}-s_2)
+
(n\bmod 2)(s_{m-1}-s_1)$,
which is positive for any score-vector when $m\geq 3$. So $c_m$ has no chance to win or even tie for victory.
\end{itemize}
Therefore, switching $c_m$ and $c_{m-1}$ can never harm Alice --- the manipulation is safe.

Next, we show that the manipulation can help $c_{m-1}$ win. To this end, we show that the manipulation may affect the score-difference between $c_{m-1}$ to $c_1$.

We assume that the $\floor{n/2}-2$ unknown agents rank $c_{m-1}\succ c_1\succ \cdots $. Then, without Alice's vote, the score-difference is
\begin{align*}
&
(\floor{n/2}-1)(s_{m-1}-s_m) + (1+n\bmod 2)(s_m-s_{m-1}) + (s_1-s_m)
+
(\floor{n/2}-2)(s_m-s_{m-1})
\\
=&
(n\bmod 2)(s_m-s_{m-1}) 
+ (s_1-s_m).
\end{align*}
Alice's vote can affect the outcome in the following way:
\begin{itemize}
\item If Alice votes truthfully $c_m\succ c_{m-1}\succ \cdots \succ c_1$, then the vote adds $s_{m-1}-s_1$ to the difference, which becomes 
$(n\bmod 2-1)(s_m - s_{m-1})$.

\item If Alice switches $c_{m-1}$ and $c_m$, then 
the vote adds $s_m - s_1$ to the difference, so the difference becomes $(n\bmod 2)(s_m - s_{m-1})$.
\end{itemize}
When $n$ is even, the manipulation changes the difference from $s_{m-1}-s_m$, which is negative by the assumption $s_m > s_{m-1}$, to $0$; so the manipulation replaces a victory for $c_1$ with a tie between $c_{m-1}$ and $c_1$, which by assumption is better for Alice.

When $n$ is odd, the manipulation changes the difference from $0$ to $s_{m}-s_1$, which is positive by the assumption $s_m > s_{m-1}$, so the manipulation replaces a tie between $c_{m-1}$ and $c_1$ with a victory for $c_{m-1}$, which is again better for Alice. In both cases, the manipulation is profitable.
\fi
\end{proof}


Combining the lemmas leads to the following bounds on the RAT-degree:
\begin{theorem}
\label{thm:positional-bounds}
For any positional voting rule with $m\geq 3$ candidates:

(a) The RAT-degree is at least $\floor{(n+1)/m}-1$;

(b) When $n\geq 2m$, the RAT-degree is at most $\ceil{n/2}+1$.
\end{theorem}
\begin{proof}
The lower bound follows immediately from \Cref{lem:lower-positional}.

For the upper bound, consider a positional voting rule with score-vector $\scorevector$. Let $t \in \{1,\ldots,m-1\}$ be the smallest index for which $s_{t+1} > s_t$ (there must be such an index by definition of a score-vector).

If $t\leq m+2$, then \Cref{lem:st1>st} implies that, for some votes by the $\ceil{n/2}+1$ known agents, switching $c_{t+1}$ and $c_t$ may be a safe and profitable manipulation for Alice.

Otherwise, $t=m-1$, and \Cref{lem:sm>sm1} implies the same.

In all cases, Alice has a safe profitable manipulation.
\end{proof}



\subsection{Plurality and anti-plurality}
We now show that the bounds of \Cref{thm:positional-bounds} are tight up to small additive constants.

We first show that the upper bound of $\approx n/2$ is attained by the \emph{plurality voting rule}, which is the positional voting rule with score-vector $(0,0,0,\ldots,0,1)$.


\begin{lemmarep}
\label{lem:lower-plurality}
\er{In the plurality voting rule with $n\geq 5$ agents,
if the number of known agents is at most $n/2$, then Alice has no safe profitable manipulation.
}
\end{lemmarep}
\begin{proofsketch}
When there are at most $n/2$ known agents, there are at least $n/2-1$ unknown agents. For some votes of these unknown agents, the outcome when Alice votes truthfully is a tie between Alice's top candidate and another candidate. But when Alice manipulates, the other candidate wins.
\end{proofsketch}

\begin{proof}
%For any potential manipulation by Alice we have to prove that, for any set $K$ of $n/2$ agents and any combination of votes by the agents of $K$, %either (1) the manipulation is not profitable (for any preference profile for the $(n/2-1)$ unknown agents, Alice weakly prefers to tell the truth); or (2) the manipulation is not safe (there exists a preference profile for the unknown agents such that Alice strictly prefers to tell the truth).

\newcommand{\aFav}{c_m}
\newcommand{\aAlt}{c^A_{alt}}
\newcommand{\kAlt}{c^K_{alt}}

If a manipulation does not involve Alice's top candidate $c_m$, then it does not affect the outcome and cannot be profitable. So let us consider a manipulation in which Alice ranks another candidate $\aAlt\neq c_m$ at the top. We show that the manipulation is not safe.

Note that there are $n-k-1$ unknown agents; the lemma condition implies $n-k-1\geq n-n/2-1 = n/2-1 \geq k-1$.

Let $\displaystyle \kAlt = \argmax_{j \in [m-1]} \score_K(c_j)$ denote the candidate with the highest number of votes among the known agents, except Alice's top candidate ($c_m$).
Consider the following two cases.

\paragraph{\underline{Case 1:} $\score_K(\kAlt) = 0$.} Since $\kAlt$ is a candidate who got the maximum number of votes from $K$ except $\aFav$, this implies that all $k$ known agents either vote for $\aFav$ or abstain.

Suppose that some $k-1$ unknown agents vote for $\aAlt$ or abstain, such that the score-difference $\score(c_m)-\score(\aAlt) = 1$ (if there are additional agents, they abstain).
Then, when Alice is truthful, her favorite candidate, $\aFav$ wins, as $\score_N(c_m) -\score_N(\aAlt) = 2$ and the scores of all other candidates are $0$. 
But when Alice manipulates and votes for $\aAlt$, the outcome is a tie between $\aFav$ and $\aAlt$, which is worse for Alice.



\paragraph{\underline{Case 2:} $\score_K(\kAlt) \geq 1$.}
Then again the manipulations not safe, as it is possible that the unknown agents vote as follows: 
\begin{itemize}
\item Some $\score_K(\aFav)$ agents vote for $\kAlt$;
\item Some 
$\score_K(\kAlt) -1$ agents vote for $c_m$. 

This is possible as both values are non-negative and $\score_K(\aFav) + \score_K(\kAlt) \leq \sum_{j =1}^m \score_K(c_j) \leq  k$, so $\score_K(\aFav) + \left(\score_K(\kAlt)-1\right) \leq k-1\leq $ the number of unknown agents.
\item 
The remaining unknown agents (if any) are split evenly between $c_m$ and $\kAlt$; if the number of remaining unknown agents is odd, then the extra agent abstains. %votes for $c_m$.
\end{itemize}
We now prove that the manipulation is harmful for Alice.

Denote $N' := N\setminus \{Alice\} = $ all agents except Alice. Then
\begin{align*}
\score_{N'}(\kAlt) &= \score_K(\aFav) + \score_K(\kAlt);
\\
\score_{N'}(\aFav) &= \score_K(\aFav) + \score_K(\kAlt) - 1.
\end{align*}
so the score-difference is exactly $1$.

Also, as $\kAlt$ has the largest score among the known agents, this still holds with the unknown agents, as all of them vote for either $\kAlt$ or $\aFav$.

We claim that $\score_{N'}(\kAlt)$ is strictly higher than that of all other candidates. Indeed:
\begin{itemize}
\item If $|R|\geq 2$, then $\kAlt$ receives at least one vote from an unknown agent, whereas all other candidates except $\aFav$ receive none.

\item Otherwise, $|R|\leq 1$, which means that $\score_K(\aFav) + \score_K(\kAlt) - 1 \geq n-k-2$, so
$\score_{N'}(\kAlt) \geq n-k-1\geq n/2-1$, which is larger than $1$ since $n\geq 5$. 
On the other hand, 
$\score_K(\aFav) + \score_K(\kAlt) - 1 \geq k-1$, which  implies that all other candidates together received at most one vote from all known agents.
\end{itemize}
Now, if Alice is truthful, the outcome is a tie between $\kAlt$ and $\aFav$, but when she manipulates and removes her vote from $c_m$, the outcome is a victory for $\kAlt$, which is worse for her.

Thus, in all cases, Alice does not have a safe profitable manipulation.
\end{proof}

Combining \Cref{lem:sm>sm1} and 
\Cref{lem:lower-plurality} gives an almost exact RAT-degree of plurality voting.
\begin{theorem}
With $m\geq 3$ candidates and $n\geq 5$ agents, the RAT-degree of plurality voting is $n/2+1$ when $n$ is even;
it is between $\floor{n/2}+1$ and  $\ceil{n/2}+1$ when $n$ is odd.
\end{theorem}


Next, we show that the lower bound of $\approx n/m$ is attained by the \emph{anti-plurality voting rule}. To this end, we prove several upper bounds on the RAT-degree for more general score-vectors. 


\iffalse % Special case; kept for didactic purposes.
\begin{lemmarep}
\label{lem:b:s2>s1}
Let $m\geq 3$ and $n\geq 3m$.
If $s_2 > s_1$ and there are $k$ known agents,
where 
\begin{align*}
k > \frac{2 s_m - s_2 - s_1}    {3 s_m + s_{m-1} - s_2 - 3 s_1} n,
\end{align*}
then
switching the bottom two candidates ($c_2$ and $c_1$) may be a safe profitable manipulation for Alice.
\end{lemmarep}
\begin{proofsketch}
Note that the expression on the right-hand side can be as small as $n/3$. 
Still, an adaptation of the construction of \Cref{lem:s2>s1} works: for some votes of the known agents,
the score of $c_1$ is necessarily lower than the \emph{arithmetic mean} of the scores of $c_m$ and $c_2$; hence, it is lower than either $c_m$ or $c_2$.
Therefore ,$c_1$ still cannot win, so switching $c_1$ and $c_2$ is safe.
\end{proofsketch}
\begin{proof}
Suppose there is a subset $K$ of $k$ known agents, who vote as follows:
\begin{itemize}
\item $k-2$ known agents rank $c_2 \succ c_m \succ \cdots \succ c_1$.
\item Two known agents rank $c_m \succ c_2 \succ
\cdots \succ c_1$. 
\end{itemize}
We first show that $c_1$ cannot win. To this end, we show that the difference in scores between $c_2$ and $c_1$, or between $c_m$ and $c_1$, is always strictly positive.
\begin{itemize}
\item The differences in scores given by the known agents is 
\begin{align*}
\score_K(c_2)-\score_K(c_1) =
&
(k-2)(s_m-s_1) 
+ 2(s_{m-1}-s_1).
\\
\score_K(c_m)-\score_K(c_1) =
&
(k-2)(s_{m-1}-s_1) 
+ 2(s_{m}-s_1).
\end{align*}
\item There are $n-k$ agents not in $K$ (including Alice). 
These agents give $c_2$ and $c_m$ together at least $(n-k)(s_1+s_2)$ points, and give $c_1$ at most $(n-k)s_m$ points. Therefore, we can bound the sum of score differences as follows:
\begin{align*}
&
[\score_N(c_2)-\score_N(c_1)]+[\score_N(c_m)-\score_N(c_1)] 
\\
\geq
&
k(s_m+s_{m-1}-2 s_1) 
+ 
(n-k)(s_2 + s_1 - 2 s_m)
\\
=
&
k(3 s_m + s_{m-1}-s_2 -3 s_1) 
- 
n(2 s_m - s_2 - s_1).
\end{align*}
The assumption on $k$ implies that this expression is positive. Therefore, either 
$\score_N(c_2)-\score_N(c_1)$ or $\score_N(c_m)-\score_N(c_1)$ or both are positive.
So $c_1$ has no chance to win or even tie.
Therefore, switching $c_2$ and $c_1$ is a safe manipulation.
\end{itemize}


Next, we show that the manipulation can help $c_m$ win, when the agents not in $K$ vote as follows:
\begin{itemize}
\item $k-4$ unknown agents rank $c_m\succ c_2\succ \cdots $,
where each candidate except $c_1,c_2,c_m$ is ranked last by at least one voter (here we use the assumption $n\geq 3m$: the condition on $k$ implies $k>n/3\geq m$, so $k\geq m+1$ and $k-4\geq m-3$).
\item One unknown agent ranks 
$c_2\succ \cdots \succ c_m \succ c_1$;
\item Alice votes truthfully $c_m\succ  \cdots \succ c_2 \succ c_1$.  
\item All other unknown voters (if any) abstain.
\end{itemize}
Then,
\begin{align*}
\score_N(c_2) - \score_N(c_m)
=&
(k-2)(s_{m}-s_{m-1}) 
+ 2(s_{m-1}-s_{m})
\\
&+
(k-4)(s_{m-1}-s_{m}) 
+ (s_m-s_2)
+ (s_2-s_m)
\\
=&
0.
\end{align*}
Moreover, for any $j\not\in\{1,2,m\}$, the score of $c_j$ is even lower (here we use the assumption that $c_j$ is ranked last by at least one unknown agent):
\begin{align*}
\score_N(c_2) - \score_N(c_j)
\geq &
(k-2)(s_{m}-s_{m-2}) 
+ 2(s_{m-1}-s_{m-2})
\\
&+
(k-5)(s_{m-1}-s_{m-2}) 
+ (s_{m-1}-s_1)
+ (s_m-s_{m-1})
+ (s_2-s_{m-1})
\\
\geq & (s_{m-1}-s_{1})
+ (s_2-s_{m-1})
\\
= & s_2 - s_1,
\end{align*}
which is positive by the lemma assumption.
Therefore, when Alice is truthful, the outcome is a tie between $c_m$ and $c_2$.

If Alice switches $c_1$ and $c_2$, then the score of $c_2$ decreases by $s_2-s_1$, which is positive by the lemma assumption, and the scores of all other candidates except $c_1$ do not change. So $c_m$ wins, which is better for Alice than a tie.
Therefore, the manipulation is profitable.
\end{proof}

In particular, for the anti-plurality rule the condition is $k>n/3$.
\begin{corollary}
The RAT-degree of anti-plurality is at most $\floor{n/3}+1$.
\end{corollary}
\fi

The following lemma strengthens 
\Cref{lem:s2>s1}.

\newcommand{\topLscores}{s_{\mathrm{top:}\ell}}
\newcommand{\botLscores}{s_{\mathrm{bot:}\ell}}
\begin{lemmarep}
\label{lem:z:s2>s1}
Let $\ell \in \{2,\ldots, m-1\}$ be an integer.
Consider a positional voting setting with $m\geq 3$ candidates and $n\geq (\ell+1)m$ agents.
Denote $\topLscores := \sum_{j=m-\ell+1}^m s_j = $  the sum of the $\ell$ highest scores and $\botLscores := \sum_{j=1}^{\ell}s_j = $ the sum of the $\ell$ lowest scores.

If $s_2 > s_1$ and there are $k$ known agents,
where 
\begin{align*}
k > \frac{\ell s_m - \botLscores}{\ell s_m + \topLscores - \botLscores - \ell s_1} n,
\end{align*}
then switching the bottom two candidates ($c_2$ and $c_1$) may be a safe profitable manipulation for Alice.
\end{lemmarep}
\begin{proofsketch}
The proof has a similar structure to that of \Cref{lem:s2>s1}.
Note that the expression at the right-hand side can be as small as $\displaystyle \frac{1}{\ell+1}n$ (for the anti-plurality rule), which is much smaller than the $\ceil{n/2}+1$ known agents required in \Cref{lem:s2>s1}.
Still, we can prove that, for some reports of the known agents, the score of $c_1$ is necessarily lower than the \emph{arithmetic mean} of the scores of the $\ell$ candidates $\{c_m, c_2, \cdots, c_{\ell}\}$. Hence, it is lower than at least one of these scores. Therefore ,$c_1$ still cannot win, so switching $c_1$ and $c_2$ is safe.
\end{proofsketch}
\begin{proof}
Suppose there is a subset $K$ of $k$ known agents, who vote as follows:
\begin{itemize}
\item $k-2$ known agents rank $c_2 \succ c_m$, then all candidates $\{c_3, \cdots , c_{\ell}\}$ in an arbitrary order, then the rest of the candidates in an arbitrary order, and lastly $c_1$.
\item Two known agents rank $c_m \succ c_2$, then all candidates $\{c_3 , \cdots , c_{\ell}\}$ in an arbitrary order, then the rest of the candidates in an arbitrary order, and lastly $c_1$.
\end{itemize}
We first show that $c_1$ cannot win. 
Denote $L := \{c_m, c_2, c_3, \ldots, c_{\ell}\}$.
We show that the difference in scores between some of the $\ell$ candidates in $L$ and $c_1$ is always strictly positive.
\begin{itemize}
\item The known agents rank all candidates in $L$ at the top $\ell$ positions. Therefore, each agent gives all of them together a total score of $\topLscores$. So
\begin{align*}
\sum_{c\in L} (\score_K(c)-\score_K(c_1)) =
&
k(\topLscores - \ell s_1).
\end{align*}
\item There are $n-k$ agents not in $K$ (including Alice). 
Each of these agents gives all candidates in $L$ together at least $\botLscores$, and gives $c_1$ at most $s_m$ points. Therefore, we can bound the sum of score differences as follows:
\begin{align*}
\sum_{c\in L} (\score_N(c)-\score_N(c_1)) \geq
&
k (\topLscores - \ell s_1)
+ (n-k) (\botLscores - \ell s_m)
\\
=&
k (\ell s_m + \topLscores - \botLscores - \ell s_1)
+ n(\botLscores - \ell s_m).
\end{align*}
The assumption on $k$ implies that this expression is positive. Therefore, for at least one $c\in L$, $\score_N(c)-\score_N(c_1) > 0$.
So $c_1$ has no chance to win or even tie.
Therefore, switching $c_2$ and $c_1$ is a safe manipulation.
\end{itemize}


Next, we show that the manipulation can help $c_m$ win, when the agents not in $K$ vote as follows:
\begin{itemize}
\item $k-4$ unknown agents rank $c_m\succ c_2\succ \cdots $,
where each candidate except $c_1,c_2,c_m$ is ranked last by at least one voter (here we use the assumption $n\geq (\ell+1)m$: the condition on $k$ implies $k>n/(\ell+1)\geq m$, so $k\geq m+1$ and $k-4\geq m-3$).
\item One unknown agent ranks 
$c_2\succ \cdots \succ c_m \succ c_1$;
\item Alice votes truthfully $c_m\succ  \cdots \succ c_2 \succ c_1$.  
\item If there are remaining unknown agents, then they are split evenly between 
$c_m\succ c_2\succ \cdots $ and 
$c_2\succ c_m\succ \cdots $ (if the number of remaining agents is odd, then the last one abstains).
\end{itemize}
Then,
\begin{align*}
\score_N(c_2) - \score_N(c_m)
=&
(k-2)(s_{m}-s_{m-1}) 
+ 2(s_{m-1}-s_{m})
\\
&+
(k-4)(s_{m-1}-s_{m}) 
+ (s_m-s_2)
+ (s_2-s_m)
\\
=&
0.
\end{align*}
Moreover, for any $j\not\in\{1,2,m\}$, the score of $c_j$ is even lower (here we use the assumption that $c_j$ is ranked last by at least one unknown agent):
\begin{align*}
\score_N(c_2) - \score_N(c_j)
\geq &
(k-2)(s_{m}-s_{m-2}) 
+ 2(s_{m-1}-s_{m-2})
\\
&+
(k-5)(s_{m-1}-s_{m-2}) 
+ (s_{m-1}-s_1)
+ (s_m-s_{m-1})
+ (s_2-s_{m-1})
\\
\geq & (s_{m-1}-s_{1})
+ (s_2-s_{m-1})
\\
= & s_2 - s_1,
\end{align*}
which is positive by the lemma assumption.
Therefore, when Alice is truthful, the outcome is a tie between $c_m$ and $c_2$.

If Alice switches $c_1$ and $c_2$, then the score of $c_2$ decreases by $s_2-s_1$, which is positive by the lemma assumption, and the scores of all other candidates except $c_1$ do not change. So $c_m$ wins, which is better for Alice than a tie.
Therefore, the manipulation is profitable.
\end{proof}

In particular, for the anti-plurality rule the condition in \Cref{lem:z:s2>s1} for $\ell=m-1$ is $k>n/m$, which implies a lower bound of $\floor{n/m}+1$. 
Combined with the general upper bound of \Cref{thm:positional-bounds}, we get:

\begin{theorem}
With $m\geq 3$ candidates and $n\geq m^2$ agents,
The RAT-degree of anti-plurality voting is 
at least $\floor{(n+1)/m}-1$ and
at most $\floor{n/m}+1$.
\end{theorem}
Intuitively, the reason that anti-plurality fares worse than plurality is that, even with a small number of known agents, it is possible to deduce that some candidate has no chance to win, and therefore there is a safe manipulation.

While we do not yet have a complete characterization of the RAT-degree of positional voting rules, our current results already show the strategic importance of the choice of scores.


\iffalse
The following lemma strengthens \Cref{lem:st1>st}
in a similar way.
\begin{lemma}
\label{lem:z:st1>st}
Let $\ell \in \{2,\ldots, m-1\}$ 
and $t\in\{1,\ldots,m-\ell \}$ be integers.

Consider a positional voting setting with $m\geq 3$ candidates and $n\geq (\ell+1)m$ agents.
Denote $\topLscores := \sum_{j=m-\ell+1}^m s_j = $  the sum of the $\ell$ highest scores and $\botLscores := \sum_{j=1}^{\ell}s_j = $ the sum of the $\ell$ lowest scores.
If $s_{t+1} > s_t = \cdots = s_1$ and there are $k$ known agents,
where 
\begin{align*}
k > \frac{\ell s_m - \botLscores}{\ell s_m + \topLscores - \botLscores - \ell s_1} n,
\end{align*}
then switching $c_{t+1}$ and $c_t$ may be a safe profitable manipulation.
\end{lemma}


\begin{proof}

Suppose there is a subset $K$ of $k$ known agents, who vote as follows:
\begin{itemize}
\item $k-2$ known agents rank $c_{t+1} \succ c_m$ first, then all candidates $\{c_{m-1}, \cdots , c_{m-\ell+1}\}$ in an arbitrary order, then the rest of the candidates in an arbitrary order, and lastly the candidates $\{c_1,\ldots,c_t\}$ in an arbitrary order (note this is possible as $t+\ell\leq m$).
\item Two known agents rank $c_m \succ c_{t+1}$ first, then all candidates $\{c_{m-1}, \cdots , c_{m-\ell+1}\}$ in an arbitrary order, then the rest of the candidates in an arbitrary order, and lastly the candidates $\{c_1,\ldots,c_t\}$ in an arbitrary order.
\end{itemize}

We first show that the $t$ worst candidates for Alice ($c_1,\ldots, c_t$) cannot win. 
Note that, by the lemma assumption $s_t = \cdots = s_1$, all these candidates receive exactly the same score by all known agents. We show that the difference in scores between $c_{t+1}$ and $c_t$ (and hence all $t$ worst candidates) is always strictly positive.
\begin{itemize}
\item The difference in scores given by the known agents is 
\begin{align*}
\score_K(c_{t+1})-\score_K(c_t) =
&
(\floor{n/2}-1)(s_m-s_1) 
+ (s_{m-1}-s_1).
\end{align*}
\item There are
$\floor{n/2}-1$ agents not in $K$ (including Alice).
These agents can reduce the score-difference by at most 
$(\floor{n/2}-1)(s_m-s_1)$.
Therefore, 
\begin{align*}
\score_N(c_{t+1})-\score_N(c_t) \geq (s_{m-1}-s_1),
\end{align*}
which is positive 
by the assumption $m-2 \geq t$ and $s_{t+1}>s_t$.
So no candidate in $c_1,\ldots,c_t$ has a chance to win or even tie.
\end{itemize}
Therefore, switching $c_{t+1}$ and $c_t$ can never harm Alice --- the manipulation is safe.

Next, we show that the manipulation can help $c_m$ win. We compute the score-difference between $c_m$ and the other candidates with and without the manipulation. 

Suppose that the agents not in $K$ vote as follows:
\begin{itemize}
\item $\floor{n/2}-3$ unknown agents rank $c_m\succ c_{t+1}\succ \cdots $,
where each candidate in $c_{t+2},\ldots,c_{m-1}$ is ranked last by at least one voter (here we use the assumption $n\geq 2m$).
\item One unknown agent ranks 
$c_{t+1}\succ \cdots \succ c_m \succ c_t \succ \cdots \succ c_1$;
\item Alice votes truthfully $c_m\succ  \cdots \succ c_{t+1} \succ  c_t \succ \cdots \succ c_1$.  
\end{itemize}
Then,
\begin{align*}
\score_N(c_{t+1}) - \score_N(c_m)
=&
(\floor{n/2}-1)(s_{m}-s_{m-1}) 
+ 2(s_{m-1}-s_{m})
\\
&+
(\floor{n/2}-3)(s_{m-1}-s_{m}) 
+ (s_m-s_{t+1})
+ (s_{t+1}-s_m)
\\
=&
0.
\end{align*}
Moreover, for any $j\in\{t+2,\ldots,m-1\}$, the score of $c_j$ is even lower (here we use the assumption that $c_j$ is ranked last by at least one unknown agent):
\begin{align*}
\score_N(c_{t+1}) - \score_N(c_j)
\geq &
(\floor{n/2}-1)(s_{m}-s_{m-2}) 
+ 2(s_{m-1}-s_{m-2})
\\
&+
(\floor{n/2}-4)(s_{m-1}-s_{m-2}) 
+ (s_{m-1}-s_1)
+ (s_m-s_{m-1})
+ (s_{t+1}-s_{m-1})
\\
\geq & (s_{m-1}-s_{1})
+ (s_{t+1}-s_{m-1})
\\
= & s_{t+1} - s_1,
\end{align*}
which is positive by the lemma assumption.
Therefore, when Alice is truthful, the outcome is a tie between $c_m$ and $c_{t+1}$.

If Alice switches $c_t$ and $c_{t+1}$, then the score of $c_{t+1}$ decreases by $s_{t+1}-s_t$, which is positive by the lemma assumption, and the scores of all other candidates except $c_t$ do not change. As $c_t$ cannot win, $c_m$ wins, which is better for Alice than a tie.
Therefore, the manipulation is profitable.
\end{proof}


The following lemma extends \Cref{lem:sm>sm1}.
\begin{lemma}
\label{lem:z:sm>sm1}
Let $\ell \in \{2,\ldots, m-1\}$ be an integer.
Consider a positional voting setting with $m\geq 3$ candidates and  $n\geq 4$ agents.
If $s_m > s_{m-1}$
and there are $k$ known agents, where
\begin{align*}
k > \frac{\ell s_m - \botLscores}{\ell s_m + \topLscores - \botLscores - \ell s_1} n,
\end{align*}
then switching the top two candidates ($c_m$ and $c_{m-1}$) may be a safe profitable manipulation for Alice.
\end{lemma}

\begin{proof}
Suppose there is a subset $K$ of $k$ known agents, who vote as follows:
\begin{itemize}
\item $k-2$ known agents rank $c_{m-2} \succ c_{m-1}$ first, 
then all candidates $\{c_{m-3},\ldots,c_{m-\ell}\}$,
then the other candidates in an arbitrary order,
and lastly $c_m$.
\item Two known agents rank $c_{m-1} \succ c_{m-2}$ first, 
then all candidates $\{c_{m-3},\ldots,c_{m-\ell}\}$,
then the other candidates in an arbitrary order,
and lastly $c_m$.
\end{itemize}

We first show that $c_m$ cannot win. 
Denote $L := \{c_{m-1}, \ldots, c_{m-\ell}\}$.
We show that the difference in scores between some of the $\ell$ candidates in $L$ and $c_m$ is always strictly positive.
\begin{itemize}
\item The known agents rank all candidates in $L$ at the top $\ell$ positions. Therefore, each agent gives all of them together a total score of $\topLscores$. So
\begin{align*}
\sum_{c\in L} (\score_K(c)-\score_K(c_m)) =
&
k(\topLscores - \ell s_1).
\end{align*}
\item There are $n-k$ agents not in $K$ (including Alice).
Each of these agents gives all candidates in $L$ together at least $\botLscores$, and gives $c_1$ at most $s_m$ points. Therefore, we can bound the sum of score differences as follows:
\begin{align*}
\sum_{c\in L} (\score_N(c)-\score_N(c_m)) \geq
&
k (\topLscores - \ell s_1)
+ (n-k) (\botLscores - \ell s_m)
\\
=&
k (\ell s_m + \topLscores - \botLscores - \ell s_1)
+ n(\botLscores - \ell s_m).
\end{align*}
The assumption on $k$ implies that this expression is positive. Therefore, for at least one $c\in L$, $\score_N(c)-\score_N(c_m) > 0$.
So $c_m$ has no chance to win or even tie.
\end{itemize}
Therefore, switching $c_m$ and $c_{m-1}$ is a safe manipulation.

Next, we show that the manipulation can help $c_{m-1}$ win. We compute the score-difference between $c_{m-1}$ and the other candidates with and without the manipulation. 

Suppose that the agents not in $K$ vote as follows:
\begin{itemize}
\item Some $k-4$ unknown agents
rank $c_{m-1}\succ c_{m-2}\succ \cdots $.
\item One unknown agent ranks $c_m \succ c_{m-2} \succ c_{m-1}\succ \cdots $.
\item If there are remaining unknown
agents, then they split evenly between 
$c_{m-1}\succ c_{m-2}\succ \cdots $ and $c_{m-2}\succ c_{m-1}\succ \cdots $ 
(if the number of these remaining agents is odd, the last one abstains).
\item Alice votes truthfully $c_m\succ c_{m-1}\succ c_{m-2} \cdots \succ c_1$.
\end{itemize}
Then,
\begin{align*}
\score_N(c_{m-2}) - \score_N(c_{m-1})
=&
(k-2)(s_{m}-s_{m-1}) 
+ 2(s_{m-1}-s_{m})
\\
&+
(k-4)(s_{m-1}-s_{m}) 
+ (s_{m-1}-s_{m-2})
+ (s_{m-2}-s_{m-1})
\\
=&
0.
\end{align*}
The candidates $c_{j<m-2}$ are ranked even lower than $c_{m-1}$ by all agents. Therefore the outcome is a tie between $c_{m-1}$ and $c_{m-2}$.

If Alice switches $c_{m-1}$ and $c_m$, then the score of $c_{m-1}$ increases by $s_m-s_{m-1}$ and the scores of all other candidates except $c_m$ do not change. Therefore, 
$\score_N(c_{m-2}) - \score_N(c_{m-1})$ becomes negative, so $c_{m-1}$ wins, which is better for Alice by assumption.
Therefore, the manipulation is profitable.
\end{proof}
\fi






\iffalse
\subsection{Condorcet rules}
\erel{Just some initial thoughts}
A \emph{Condorcet voting rule} is a rule that always selects a Condorcet winner if one exists. Condorcet rules differ in how they pick the winner when a Condorcet winner does not exist.

A simple Condorcet rule is the MaxMin-Condorcet rule. It works in the following way:
\begin{itemize}
\item For each ordered pair of candidates $(c,c')$, compute $S(c,c')$ as the number of voters who prefer $c$ to $c'$.
\item The score of each candidate $c$ is $\min_{c'} S(c,c')$, that is, the lowest score of $c$ in all pairwise competitions.
\item The candidate with the highest score wins. 
\end{itemize}
Note that, if there is a Condorcet winner, then his score will be larger than $n/2$, and the score of all other candidates will be smaller than $n/2$, so the Condorcet winner will be selected.

This rule potentially has a high RAT-degree, as the winner depends on $m-1$ scores per agent, rather than one. 
\erel{I could not compute its RAT-degree - some brain-storming on this could help.}
\fi


\subsection{Higher RAT-degree?}
\Cref{thm:positional-bounds} raises the question of whether some other, non-positional voting rules have RAT-degrees substantially higher than $n/2$.
%
Using ideas similar to those in \Cref{sect:indivisible-EF1-n-1}, 
we could use a selection rule $\Gamma$ to choose a ``dictator'', and implement the dictator's first choice.
This deterministic mechanism has RAT-degree $n-1$, as without knowledge of all other agents' inputs, every manipulation might cause the manipulator to lose the chance of being a dictator. 
However, besides the fact that this is an unnatural mechanism, it suffers from other problems such as the \emph{no-show paradox} (a participating voter might affect the selection rule in a way that will make another agent a dictator, which might be worse than not participating at all).

Our main open problem is therefore to devise natural voting rules with a high RAT-degree.
\begin{open}
Does there exist a non-dictatorial voting rule that satisfies the participation criterion (i.e. does not suffer from the no-show paradox),  with RAT-degree larger than $\ceil{n/2}+1$? 
\end{open}

