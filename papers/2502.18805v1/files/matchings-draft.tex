\newpage

\newcommand{\men}{M}
\newcommand{\women}{W}
\newcommand{\man}{m}
\newcommand{\woman}{w}

\section{Stable Matchings}
\eden{I used this model: \cite{coles2014optimal} \url{https://scholar.harvard.edu/files/ran/files/optimal_truncation.pdf}}
In this section, we consider mechanisms for stable matchings. 
Here, the $n$ agents are divided into two disjoint subsets, $\men$ and $\women$, that need to be matched to each other -- the most common examples are men and women or students and universities. 
Each $\man \in \men$ has a strict preference order, $\succ_{\man}$, over $\women \cup \{\man\}$, representing his utility of being matched with $\woman \in \women$ or remaining unmatched.
Similarly, each $\woman \in \women$ has a strict preference order, $\succ_{\woman}$, over $\men \cup \{\woman\}$.  

A \emph{matching} (or pairing) between $\men$ to $\women$ -is a mapping $\mu$ from $\men \cup \women$ to $\men \cup \women$ that satisfies the following:
\begin{align*}
    \forall \man \in \men \colon &\quad \mu(\man) \in \women \cup \{\man\}\\
    \forall \woman \in \women \colon &\quad \mu(\woman) \in \men \cup \{\woman\}\\
    \forall \man, \woman \in \men \cup \women \colon &\quad  \mu(\man) = \woman  \iff \mu(\woman) = \man
\end{align*}

\eden{I'm not sure if it would be better to use men and women or students and universities}

A matching is said to be \emph{stable} if there is \emph{no} pair $(\man, \woman) \in \men \times \women$ such that $\woman \succ_{\man} \mu(\man)$ and $\man \succ_{\woman} \mu(\woman)$ -- that is, there is no pair where $\man$ prefers $\woman$ over his assigned match while $\woman$ prefers $\man$ over her assigned match;  or that one of them prefers being matched.  


A mechanism in this context gets the preference orders of all agents and returns a stable matching.

\subsection{Deferred Acceptance (Gale-Shapley)}
The \emph{Deferred Acceptance} algorithm \cite{} is one of the most well-known mechanisms for computing a stable matching. 
In this algorithm, one side of the market - here, $\men$ - proposes, while the other side - $\women$ accepts or rejects offers iteratively. 
The algorithm proceeds as follows:
\begin{enumerate}
    \item Each $\man \in \men$ proposes to his most preferred alternative according to $\succ_{\man}$ that has not reject him it and that he prefers over being matched. 

    \item Each $\woman \in W$ tentatively accepts her most preferred proposal according to $\succ_{\woman}$ that she prefers over being matched, and rejects the rest.

    \item The rejected agents propose to their next most preferred choice as in step 1.

    \item The process repeats until no one of the rejected agents wishes to make a new proposal.

    \item The final matching is determined by the last set of accepted proposal.
\end{enumerate}


It is well known that the mechanism is truthful for the proposing side ($\men$) but untruthful for the other side ($\women$) -- which may have an incentive to misreport their preferences to obtain a better match.

A key type of manipulation in this setting is \emph{truncation} \cite{coles2014optimal} -- agents in the proposed size can get a better match by falsely reports that they prefer to remain unmatched rather than being matched with some of the agents in the other set, even if, in reality, she would prefer those men over being unmatched. 
% By doing so, she can potentially avoid being matched with lower-ranked options and receive a better match in the final outcome.  
We analyze the \emph{RAT-degree} of the mechanism and show that it is at most $3$. 


We separately analyze the case where we assume that for any agent being unmatched is the worst outcome -- that is, the agents' reference orders are only over the agents on the other set 
 
\begin{proposition}
    The RAT-degree of deferred acceptance is at most $3$. When blocking is not allowed, the degree is at most $5$.
\end{proposition}
