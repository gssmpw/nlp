% \newpage
\section{Cake Cutting}
\label{sec:cake-cutting}
In this section, we study the \emph{cake cutting} problem: the allocation of divisible heterogeneous resources to $n$ agents.
The cake cutting problem was proposed by~\citet{Steinhaus48,Steinhaus49}, and it is a widely studied subject in mathematics, computer science, economics, and political science.

In the cake cutting problem, the resource/cake is modeled as an interval $[0,1]$, and it is to be allocated among a set of $n$ agents $N=\{a_1,\ldots,a_n\}$.
An allocation is denoted by $(A_1,\ldots,A_n)$ where $A_i\subseteq[0,1]$ is the share allocated to agent $a_i$.
We require that each $A_i$ is a union of finitely many closed non-intersecting intervals, and, for each pair of $i,j\in[n]$, $A_i$ and $A_j$ can only intersect at interval endpoints, i.e., the measure of $A_i\cap A_j$ is $0$.
We say an allocation is \emph{complete} if $\bigcup_{i=1}^nA_i=[0,1]$.
Otherwise, it is \emph{partial}.

The true preferences $T_i$ of agent $a_i$ are given by  a \emph{value density function} $v_i:[0,1]\to\mathbb{R}_{\geq0}$ that describes agent $a_i$'s preference over the cake.
To enable succinct encoding of the value density function, we adopt the widely considered assumption that each $v_i$ is \emph{piecewise constant}: there exist finitely many points $x_{i0},x_{i1},x_{i2},\ldots,x_{ik_i}$ with $0=x_{i0}<x_{i1}<x_{i2}<\cdots<x_{ik_i}=1$ such that $v_i$ is a constant on every interval $(x_{i\ell},x_{i(\ell+1)})$, $\ell=0,1,\ldots,k_i-1$.
Given a subset $S\subseteq[0,1]$ that is a union of finitely many closed non-intersecting intervals, agent $a_i$'s value for receiving $S$ is then given by
$V_i(S)=\int_Sv_i(x)dx.$

Fairness and efficiency are two natural goals for allocating the cake.
For efficiency, we consider two commonly used criteria: \emph{social welfare} and \emph{Pareto-optimality}.
Given an allocation $(A_1,\ldots,A_n)$, its \emph{social welfare} is given by
$\sum_{i=1}^nV_i(A_i)$.
This is a natural measurement of efficiency that represents the overall happiness of all agents.
Pareto-optimality is a yes-or-no criterion for efficiency.
An allocation $(A_1,\ldots,A_n)$ is \emph{Pareto-optimal} if there does not exist another allocation $(A_1',\ldots,A_n')$ such that $V_i(A_i')\geq V_i(A_i)$ for each agent $a_i$ and at least one of these $n$ inequalities is strict.

For fairness, we study two arguably most important notions: \emph{envy-freeness} and \emph{proportionality}.
An allocation $(A_1,\ldots,A_n)$ is \emph{proportional} if each agent receives her average share, i.e., for each $i\in[n]$, $V_i(A_i)\geq\frac1nV_i([0,1])$.
An allocation $(A_1,\ldots,A_n)$ is \emph{envy-free} is every agent weakly prefers her own allocated share, i.e., for every pair $i,j\in[n]$, $V_i(A_i)\geq V_i(A_j)$.
A complete envy-free allocation is always proportional
\iffalse %EREL: removed to save space
: for each $i\in[n]$, summing up the $n$ inequalities $V_i(A_i)\geq V_i(A_j)$ for $j=1,\ldots,n$ yields
$$nV_i(A_i)\geq \sum_{j=1}^nV_i(A_j)=V_i\left(\bigcup_{j=1}^nA_j\right)=V_i([0,1]),$$
which is $V_i(A_i)\geq\frac1nV_i([0,1])$.
However, this implication does not hold for partial allocation: $A_1=\cdots=A_n=\emptyset$ gives an envy-free allocation, but it is clearly not proportional.
\fi
, but this implication does not hold for partial allocations.

Before we discuss our results, we define an additional notion, \emph{uniform segment}, which will be used throughout this section.
Given $n$ value density functions $v_1,\ldots,v_n$ (that are piecewise constant by our assumptions), we identify the set of points of discontinuity for each $v_i$ and take the union of the $n$ sets.
Sorting these points by ascending order, we let $x_1,\ldots,x_{m-1}$ be all points of discontinuity for all the $n$ value density functions.
Let $x_0=0$ and $x_m=1$.
These points define $k$ intervals, $(x_0,x_1),(x_1,x_2),\ldots,(x_{m-1},x_m)$, such that each $v_i$ is a constant on each of these intervals.
We will call each of these intervals a \emph{uniform segment}, and we will denote $X_t=(x_{t-1},x_{t})$ for each $t=1,\ldots,m$.
For each agent $a_i$, we will slightly abuse the notation by using $v_i(X_t)$ to denote $v_i(x)$ with $x\in X_t$.

Since all agents' valuations on each uniform segment are uniform, it is tempting to think about the cake cutting problem as the problem of allocating $m$ divisible homogeneous goods.
However, this interpretation is inaccurate when concerning agents' strategic behaviors, as, in the cake cutting setting, an agent can manipulate her value density function with a different set of points of discontinuity, which affects how the divisible goods are defined.
To see a significant difference between these two models, in the divisible goods setting, the \emph{equal division rule} that allocates each divisible good evenly to the $n$ agents is truthful (with RAT-degree $n$), envy-free and proportional, while, in the cake cutting setting, it is proved in~\citet{tao2022existence} that truthfulness and proportionality are incompatible even for two agents.


\paragraph{Results}
In Sect.~\ref{sect:cake-msw}, we start by considering the simple mechanism that outputs allocation with the maximum social welfare.
We show that the RAT-degree of this mechanism is $0$.
Similar as it is in the case of indivisible goods, we also consider the normalized variant of this mechanism, and we show that the RAT-degree is $1$.
In Sect.~\ref{sect:cake-fair}, we consider mechanisms that output fair allocations. We review the mechanisms studied in~\citet{BU2023Rat} by studying their RAT-degrees. 
We will see that one of those mechanisms, which always outputs envy-free allocation, has a RAT-degree of $n-1$.
However, this mechanism has a very poor performance on efficiency.
Finally, in Sect.~\ref{sect:cake-Prop+PO}, we propose a new mechanism with RAT-degree $n-1$ that always outputs proportional and Pareto-optimal allocations.

\subsection{Maximum Social Welfare Mechanisms}
\label{sect:cake-msw}
It is easy to find an allocation that maximizes the social welfare: for each uniform segment $X_t$, allocate it to an agent $a_i$ with the maximum $v_i(X_t)$.
When multiple agents have equally largest value of $v_i(X_t)$ on the segment $X_t$, we need to specify a tie-breaking rule.
However, as we will see later, the choice of the tie-breaking rule does not affect the RAT-degree of the mechanism.

%We consider two different tie-breaking rules:
%\begin{enumerate}
%    \item agent with the smallest index takes $X_t$, and
%    \item $X_t$ is evenly distributed among these agents such that the agent with the smallest index takes the left-most piece of $X_t$ and the agent with the largest index takes the right-most piece.
%\end{enumerate}

It is easy to see that, whatever the tie-breaking rule is, the maximum social welfare mechanism is safely manipulable.
It is safe for an agent to report higher values on every uniform segment.
For example, doubling the values on all uniform segments is clearly a safe manipulation.

\begin{observation}
\label{obs:utilitarian-cake-cutting}
Utilitarian Cake-Cutting with any tie-breaking rule has RAT-degree $0$.
\end{observation}

We next consider the following variant of the maximum social welfare mechanism:
first rescale each $v_i$ such that $V_i([0,1])=\int_0^1v_i(x)dx=1$, and then output the allocation with the maximum social welfare.
We will show that the RAT-degree is $1$.
The proof is similar to the one for indivisible items (\Cref{thm:normalized-utilitarian-goods}) and is given in the appendix.

\begin{theoremrep}
When there are at least three agents,
Normalized Utilitarian Cake-Cutting with any tie-breaking rule has RAT-degree $1$.
\end{theoremrep}
\begin{proof}
    We assume without loss of generality that the value density function reported by each agent is normalized (as, otherwise, the mechanism will normalize the function for the agent).
    
    We first show that the mechanism is not $0$-known-agents safely-manipulable.
    Consider an arbitrary agent $a_i$ and let $v_i$ be her true value density function.
    Consider an arbitrary misreport $v_i'$ of agent $a_i$ with $v_i'\neq v_i$.
    Since the value density functions are normalized, there must exist an interval $(a,b)$ where $v_i'$ and $v_i$ are constant and $v_i'(x)<v_i(x)$ for $x\in(a,b)$.
    Choose $\varepsilon>0$ such that $v_i(x)>v_i'(x)+\varepsilon$.
    Consider the following two value density functions (note that both are normalized):
    $$v^{(1)}(x)=\left\{\begin{array}{ll}
        v_i'(x)+\varepsilon & \mbox{if }x\in(a,b) \\
        v_i'(x)-\varepsilon\cdot\frac{b-a}{1+a-b} & \mbox{otherwise}
    \end{array}\right. \quad\mbox{and}\quad v^{(2)}(x)=\left\{\begin{array}{ll}
        v_i'(x)-\varepsilon & \mbox{if }x\in(a,b) \\
        v_i'(x)+\varepsilon\cdot\frac{b-a}{1+a-b} & \mbox{otherwise}
    \end{array}\right..$$
    Suppose the remaining $n-1$ agents' reported value density functions are either $v^{(1)}$ or $v^{(2)}$ and each of $v^{(1)}$ and $v^{(2)}$ is reported by at least one agent (here we use the assumption $n\geq 3$).
    In this case, agent $a_i$ will receive the empty set by reporting $v_i'$.
    On the other hand, when reporting $v_i$, agent $a_i$ will receive an allocation that at least contains $(a,b)$ as a subset.
    Since $v_i$ has a positive value on $(a,b)$, reporting $v_i'$ is not a safe manipulation.

    We next show that the mechanism is $1$-known-agent safely-manipulable.
    \erel{Doesn't this part follow from the analogous result on indivisible items?}\biaoshuai{I think so}
    Suppose agent $a_1$'s true value density function is
    $$v_1(x)=\left\{\begin{array}{ll}
        1.5 & \mbox{if }x\in[0,0.5] \\
        0.5 & \mbox{otherwise}
    \end{array}\right.,$$ 
    and agent $a_1$ knows that agent $a_2$ reports the uniform value density function $v_2(x)=1$ for $x\in[0,1]$.
    We will show that the following manipulation of agent $a_1$ is safe and profitable.
    $$v_1'(x)=\left\{\begin{array}{ll}
        2 & \mbox{if }x\in[0,0.5] \\
        0 & \mbox{otherwise}
    \end{array}\right.$$
    Firstly, regardless of the reports of the remaining $n-2$ agents, the final allocation received by agent $a_1$ must be a subset of $[0,0,5]$, as agent $a_2$'s value is higher on the other half $(0.5,1]$.
    Since $v_1'$ is larger than $v_1$ on $[0,0,5]$, any interval received by agent $a_1$ when reporting $v_1$ will also be received if $v_1'$ were reported.
    Thus, the manipulation is safe.

    Secondly, if the remaining $n-2$ agents' value density functions are
    $$v_3(x)=v_4(x)=\cdots=v_n(x)=\left\{\begin{array}{ll}
        1.75 & \mbox{if }x\in[0,0.5] \\
        0.25 & \mbox{otherwise}
    \end{array}\right.,$$
    it is easy to verify that agent $a_1$ receives the empty set when reporting truthfully and she receives $[0,0.5]$ by reporting $v_1'$.
    Therefore, the manipulation is profitable.
\end{proof}

\subsection{Fair Mechanisms}
\label{sect:cake-fair}
In this section, we focus on mechanisms that always output fair (envy-free or proportional) allocations.
As we have mentioned earlier, it is proved in~\citet{tao2022existence} that truthfulness and proportionality are incompatible even for two agents and even if partial allocations are allowed.
This motivates the search for fair cake-cutting algorithms with a high RAT-degree.

% \erel{To motivate this section, we should cite the impossibility result on truthful fair mechanisms; this motivates the search for the "next-best" option, which is a high RAT-degree.} \biaoshuai{Done. I added the sentence above.}

The mechanisms discussed in this section have been considered in~\citet{BU2023Rat}.
However, they are only studied by whether or not they are risk-averse truthful (in our language, whether the RAT-degree is positive).
With our new notion of RAT-degree, we are now able to provide a more fine-grained view of their performances on strategy-proofness.

One natural envy-free mechanism is to evenly allocate each uniform segment $X_t$ to all agents.
Specifically, each $X_t$ is partitioned into $n$ intervals of equal length, and each agent receives exactly one of them.
It is easy to see that $V_i(A_j)=\frac1nV_i([0,1])$ for any $i,j\in[n]$ under this allocation, so the allocation is envy-free and proportional.

To completely define the mechanism, we need to specify the order of evenly allocating each $X_t=(x_{t-1},x_t)$ to the $n$ agents.
A natural tie-breaking rule is to let agent $a_1$ get the left-most interval and agent $a_n$ get the right-most interval.
Specifically, agent $a_i$ receives the $i$-th interval of $X_t$, which is $[x_{t-1}+\frac{i-1}n(x_t-x_{t-1}),x_{t-1}+\frac{i}n(x_t-x_{t-1})]$.
However, it was proved in~\citet{BU2023Rat} that the equal division mechanism under this ordering rule is safely-manipulable, i.e., its RAT-degree is $0$.
In particular, agent $a_1$, knowing that she will always receive the left-most interval in each $X_t$, can safely manipulate by deleting a point of discontinuity in her value density function if her value on the left-hand side of this point is higher.

To avoid this type of manipulation, a different ordering rule was considered by~\citet{BU2023Rat} (See Mechanism 3 in their paper): at the $t$-th segment, the $n$ equal-length subintervals of $X_t$ are allocated to the $n$ agents with the left-to-right order $a_t,a_{t+1},\ldots,a_n,a_1,a_2,\ldots,a_{t-1}$.
By using this ordering rule, an agent does not know her position in the left-to-right order of $X_t$ without knowing others' value density functions.
Indeed, even if only one agent's value density function is unknown, an agent cannot know the index $t$ of any segment $X_t$.
This suggests that the mechanism has a RAT-degree of $n-1$.

\begin{theoremrep}
    Consider the mechanism that evenly partitions each uniform segment $X_t$ into $n$ equal-length subintervals and allocates these $n$ subintervals to the $n$ agents with the left-to-right order $a_t,a_{t+1},\ldots,a_n,a_1,a_2,\ldots,a_{t-1}$. It has RAT-degree $n-1$ and always outputs envy-free allocations.
\end{theoremrep}
\begin{proofsketch}
    Envy-freeness is trivial: for any $i,j\in[n]$, we have $V_i(A_j)=\frac1nV_i([0,1])$.
    The general impossibility result in~\citet{tao2022existence} shows that no mechanism with the envy-freeness guarantee can be truthful, so the RAT-degree is at most $n-1$.

 To show that the RAT-degree is exactly $n-1$, we show that, if even a single agent is not known to the manipulator, it is possible that this agent's valuation adds discontinuity points in a way that the ordering in each uniform segment is unfavorable for the manipulator.
\end{proofsketch}
\begin{proof}
    Envy-freeness is trivial: for any $i,j\in[n]$, we have $V_i(A_j)=\frac1nV_i([0,1])$.
    The general impossibility result in~\citet{tao2022existence} shows that no mechanism with the envy-freeness guarantee can be truthful, so the RAT-degree is at most $n-1$.

    To show that the RAT-degree is exactly $n-1$, consider an arbitrary agent $a_i$ with true value density function $v_i$ and an arbitrary agent $a_j$ whose report is unknown to agent $a_i$.
    Fix $n-2$ arbitrary value density functions $\{v_k\}_{k\notin\{i,j\}}$ that are known by agent $a_i$ to be the reports of the remaining $n-2$ agents.
    For any $v_i'$, we will show that agent $a_i$'s reporting $v_i'$ is either not safe or not profitable.

    Let $T$ be the set of points of discontinuity for $v_1,v_2,\ldots,v_{j-1},v_{j+1},\ldots,v_n$, and $T'$ be the set of points of discontinuity with $v_i$ replaced by $v_i'$.
    If $T\subseteq T'$ (i.e., the uniform segment partition defined by $T'$ is ``finer'' than the partition defined by $T$), the manipulation is not profitable, as agent $a_i$ will receive her proportional share $\frac1nV_i([0,1])$ in both cases.
    
    It remains to consider the case where there exists a point of discontinuity $y$ of $v_i$ such that $y\in T$ and $y\notin T'$.
    This implies that $y$ is a point of discontinuity in $v_i$, but not in $v_i'$ nor in the valuation of any other agent.
    We will show that the manipulation is not safe in this case.

    Choose a sufficiently small $\varepsilon>0$ such that $(y-\varepsilon, y+\varepsilon)$ is contained in a uniform segment defined by $T'$.
    We consider two cases, depending on whether the ``jump'' of $v_i$ in its discontinuity point $y$ is upwards or downwards.
    
    \underline{Case 1:}  $\lim_{x\to y^-}v_i(x)<\lim_{x\to y^+}v_i(x)$.
    We can construct $v_j$ such that: 1) $y-\varepsilon$ and $y+\varepsilon$ are points of discontinuity of $v_j$, and 2) the uniform segment $(y-\varepsilon, y+\varepsilon)$ under the profile $(v_1,\ldots,v_{i-1},v_i',v_{i+1},\ldots,v_n)$ is the $t$-th segment where $n$ divides $t-i$ (i.e., agent $a_i$ receives the left-most subinterval of this uniform segment).
    Notice that 2) is always achievable by inserting a suitable number of points of discontinuity for $v_j$ before $y-\varepsilon$.
    Given that $\lim_{x\to y^-}v_i(x)<\lim_{x\to y^+}v_i(x)$, agent $a_i$'s allocated subinterval on the segment $(y-\varepsilon, y+\varepsilon)$ has value strictly less than $\frac1nV_i([(y-\varepsilon, y+\varepsilon])$.

    \underline{Case 2:} $\lim_{x\to y^-}v_i(x)>\lim_{x\to y^+}v_i(x)$.
    We can construct $v_j$ such that 1) $(y-\varepsilon, y+\varepsilon)$ is a uniform segment under the profile $(v_1,\ldots,v_{i-1},v_i',v_{i+1},\ldots,v_n)$, and 2) agent $a_i$ receives the right-most subinterval on this segment.
    In this case, agent $a_i$ again receives a value of strictly less than $\frac1nV_i([(y-\varepsilon, y+\varepsilon])$ on the segment $(y-\varepsilon, y+\varepsilon)$.

    We can do this for every point $y$ of discontinuity of $v_i$ that is in $T\setminus T'$.
    By a suitable choice of $v_j$ (with a suitable number of points of discontinuity of $v_j$ inserted in between), we can make sure agent $a_i$ receives a less-than-average value on every such segment $(y-\varepsilon,y+\varepsilon)$.
    Moreover, agent $a_i$ receives exactly the average value on each of the remaining segments, because the remaining discontinuity points of $T$ are contained in $T'$.
    Therefore, the overall utility of $a_i$ by reporting $v_i'$ is strictly less than $\frac1nV_i([0,1])$.
    Given that $a_i$ receives value exactly $\frac1nV_i([0,1])$ for truthfully reporting $v_i$, reporting $v_i'$ is not safe.
\end{proof}

Although the equal division mechanism with the above-mentioned carefully designed ordering rule is envy-free and has a high RAT-degree of $n-1$, it is undesirable in at least two aspects:
\begin{enumerate}
    \item it requires quite many cuts on the cake by making $n-1$ cuts on each uniform segment; this is particularly undesirable if piecewise constant functions are used to approximate more general value density functions.

    \item it is highly inefficient: each agent $a_i$'s utility is never more than her minimum proportionality requirement $\frac1nV_i([0,1])$;
\end{enumerate}

% We handle point (1) in \Cref{sect:cake-connected} and point (2) in \Cref{sect:cake-Prop+PO}.

%\subsection{Connected cake-cutting} \label{sect:cake-connected}

% \biaoshuai{I think the remaining part can be mostly put to the appendix if we do not have enough space. We can just say briefly here that all the known variants of moving-knife have RAT-degree at most $1$.}
% EREL: Done

Regarding point (1), researchers have been looking at allocations with \emph{connected pieces}, i.e., allocations with only $n-1$ cuts on the cake.
A well-known mechanism in this category is \emph{the moving-knife procedure}, which always outputs proportional allocations.
This mechanism was first proposed by~\citet{dubins1961cut}. It always returns a proportional connected allocation.
Unfortunately, it was shown by~\citet{BU2023Rat} that Dubins and Spanier's moving-knife procedure is safely-manipulable for some very subtle reasons.

\citet{BU2023Rat} proposed a variant of the moving-knife procedure that is RAT.
In addition, they showed that another variant of moving-knife procedure proposed by~\citet{ortega2022obvious} is also RAT.\footnote{It should be noticed that, when $v_i$ is allowed to take $0$ value, tie-breaking needs to be handled very properly to ensure RAT. See \citet{BU2023Rat} for more details. Here, for simplicity, we assume $v_i(x)>0$ for each $i\in[n]$ and $x\in[0,1]$.}
In the appendix, we describe both mechanisms and show that both of them have RAT-degree $1$.
%
\begin{toappendix}
\subsection{Moving-knife mechanisms: descriptions and proofs}
Hereafter, we assume $v_i(x)>0$ for each $i\in[n]$ and $x\in[0,1]$.

\paragraph{Dubins and Spanier's moving-knife procedure}
Let $u_i=\frac1nV_i([0,1])$ be the value of agent $a_i$'s proportional share.
In the first iteration, each agent $a_i$ marks a point $x_i^{(1)}$ on $[0,1]$ such that the interval $[0,x_i^{(1)}]$ has value exactly $u_i$ to agent $a_i$.
Take $x^{(1)}=\min_{i\in[n]}x_i^{(1)}$, and the agent $a_{i_1}$ with $x_{i_1}^{(1)}=x^{(1)}$ takes the piece $[0,x^{(1)}]$ and leaves the game.
In the second iteration, let each of the remaining $n-1$ agents $a_i$ marks a point $x_i^{(2)}$ on the cake such that $[x^{(1)},x_i^{(2)}]$ has value exactly $u_i$.
Take $x^{(2)}=\min_{i\in[n]\setminus\{i_1\}}x_i^{(2)}$, and the agent $a_{i_2}$ with $x_{i_2}^{(2)}=x^{(2)}$ takes the piece $[x^{(1)},x^{(2)}]$ and leave the game.
This is done iteratively until $n-1$ agents have left the game with their allocated pieces.
Finally, the only remaining agent takes the remaining part of the cake.
It is easy to see that each of the first $n-1$ agents receives exactly her proportional share, while the last agent receives weakly more than her proportional share; hence the procedure always returns a proportional allocation.

Notice that, although the mechanism is described in an iterative interactive way that resembles an extensive-form game, we will consider the \emph{direct-revelation} mechanisms in this paper, where the $n$ value density functions are reported to the mechanism at the beginning.
In the above description of Dubins and Spanier's moving-knife procedure, as well as its two variants mentioned later, by saying ``asking an agent to mark a point'', we refer to that the mechanism computes such a point based on the reported value density function.
In particular, we do not consider the scenario where agents can adaptively choose the next marks based on the allocations in the previous iterations.

Unfortunately, it was shown by~\citet{BU2023Rat} that Dubins and Spanier's moving-knife procedure is safely-manipulable for some very subtle reasons.


\citet{BU2023Rat} proposed a variant of the moving-knife procedure that is risk-averse truthful.
In addition, \citet{BU2023Rat} shows that another variant of moving-knife procedure proposed by~\citet{ortega2022obvious} is also risk-averse truthful.\footnote{It should be noticed that, when $v_i$ is allowed to take $0$ value, tie-breaking needs to be handled very properly to ensure risk-averse truthfulness. See \citet{BU2023Rat} for more details.}
Below, we will first describe both mechanisms and then show that both of them have RAT-degree $1$.

\paragraph{Ortega and Segal-Halevi's moving knife procedure}
The first iteration of Ortega and Segal-Halevi's moving knife procedure is the same as it is in Dubins and Spanier's.
After that, the interval $[x^{(1)},1]$ is then allocated \emph{recursively} among the $n-1$ agents $[n]\setminus\{a_{i_1}\}$.
That is, in the second iteration, each agent $a_i$ marks a point $x_i^{(2)}$ such that the interval $[x^{(1)}, x_i^{(2)}]$ has value exactly $\frac1{n-1}V_i([x^{(1)},1])$ (instead of $\frac1nV_1([0,1])$ as it is in Dubins and Spanier's moving-knife procedure).
The remaining part is the same: the agent with the left-most mark takes the corresponding piece and leaves the game.
After the second iteration, the remaining part of the cake is again recursively allocated to the remaining $n-2$ agents.
This is continued until the entire cake is allocated.

\paragraph{Bu, Song, and Tao's moving knife procedure}
Each agent $a_i$ is asked to mark all the $n-1$ ``equal-division-points'' $x_i^{(1)},\ldots,x_i^{(n-1)}$ at the beginning such that $V_i([x_i^{(t-1)},x_i^{(t)}])=\frac1nV_i([0,1])$ for each $t=1,\ldots,n$, where we set $x_i^{(0)}=0$ and $x_i^{(n)}=1$.
The remaining part is similar to Dubins and Spanier's moving-knife procedure:
in the first iteration, agent $i_1$ with the minimum $x_{i_1}^{(1)}$ takes $[0,x_{i_1}^{(1)}]$ and leave the game; in the second iteration, agent $i_2$ with the minimum $x_{i_2}^{(2)}$ among the remaining $n-1$ agents takes $[x_{i_1}^{(1)},x_{i_2}^{(2)}]$ and leave the game; and so on.
The difference to Dubins and Spanier's moving-knife procedure is that each $x_i^{(t)}$ is computed at the beginning, instead of depending on the position of the previous cut.


\begin{theorem}
    The RAT-degree of Ortega and Segal-Halevi's moving knife procedure is $1$.
\end{theorem}
\begin{proof}
    It was proved in~\citet{BU2023Rat} that the mechanism is not $0$-known-agent safely-manipulable.
    It remains to show that it is $1$-known-agents safely-manipulable.
    Suppose agent $a_1$'s value density function is uniform, $v_1(x)=1$ for $x\in[0,1]$, and agent $a_1$ knows that agent $a_2$ will report $v_2$ such that $v_2(x)=1$ for $x\in[1-\varepsilon,1]$ and $v_2(x)=0$ for $x\in[0,1-\varepsilon)$ for some very small $\varepsilon>0$ with $\varepsilon\ll\frac1n$.
    We show that the following $v_1'$ is a safe manipulation.
    $$v_1'(x)=\left\{\begin{array}{ll}
        1 & x\in[0,\frac{n-2}n] \\
        \frac2{n\varepsilon} & x\in[1-2\varepsilon,1-\varepsilon]\\
        0 & \mbox{otherwise}
    \end{array}\right.$$
    Before we move on, note an important property of $v_1'$: for any $t\leq\frac{n-2}n$, we have $V_1([t,1])=V_1'([t,1])$.

    Let $[a,b]$ be the piece received by agent $a_1$ when she reports $v_1$ truthfully.
    If $b\leq \frac{n-2}n$, the above-mentioned property implies that she will also receive exactly $[a,b]$ for reporting $v_1'$.
    If $b>\frac{n-2}n$, then we know that agent $a_1$ is the $(n-1)$-th agent in the procedure.
    To see this, we have $V_1([a,b])\geq\frac1n([0,1])=\frac1n$ by the property of Ortega and Segal-Halevi's moving knife procedure, and we also have $V_1([b,1])=1-b<\frac2n$.
    This implies $b$ cannot be the $1/k$ cut point of $[a,1]$ for $k\geq 3$.
    On the other hand, it is obvious that agent $a_2$ takes a piece after agent $a_1$.
    Thus, by the time agent $a_1$ takes $[a,b]$, the only remaining agent is agent $a_2$.

    Since there are exactly two remaining agents in the game before agent $a_1$ takes $[a,b]$, we have $V_1([a,1])\geq\frac2nV_1([0,1])=\frac2n$.
    This implies $a\leq\frac{n-2}n$ and $b=\frac{a+1}2\leq \frac{n-1}n$.
    On the other hand, by reporting $v_1'$, agent $a_1$ can then get the piece $[a,b']$ with $b'\in[1-2\varepsilon,1-\varepsilon]$.
    We see that $b'>b$. Thus, the manipulation is safe and profitable.
\end{proof}


\begin{theorem}
    The RAT-degree of Bu, Song, and Tao's moving knife procedure is $1$.
\end{theorem}
\begin{proof}
    It was proved in~\citet{BU2023Rat} that the mechanism is not $0$-known-agent safely-manipulable.
    The proof that it is $1$-known-agents safely-manipulable is similar to the proof for Ortega and Segal-Halevi's moving knife procedure, with the same $v_1,v_1'$ and $v_2$.
    It suffices to notice that the first $n-2$ equal-division-points are the same for $v_1$ and $v_1'$, where the last equal-division-point of $v_1'$ is to the right of $v_1$'s.
    Given that agent $a_2$ will always receive a piece after agent $a_1$, the same analysis in the previous proof can show that the manipulation is safe and profitable. 
\end{proof}

\subsection{Additional proofs} 
\end{toappendix}
These results invoke the following question.
\begin{open}
    Is there a proportional connected cake-cutting rule with RAT-degree at least  $2$?
    % \erel{Is it indeed open?}\biaoshuai{Yes to the best of my knowledge.}
\end{open}

We handle point (2) from above in the following subsection.

\subsection{A Proportional and Pareto-Optimal Mechanism with RAT-degree $n-1$}
\label{sect:cake-Prop+PO}
In this section, we provide a mechanism with RAT-degree $n-1$ that always outputs proportional and Pareto-optimal allocations.
In addition, we show that the mechanism can be implemented in polynomial time.
The mechanism uses some similar ideas as the one in \Cref{sect:indivisible-EF1-n-1}.

% \erel{How about the following mechanism: Find a proportional allocation that maximizes the value of agent 1; then of agent 2; etc. It is proportional, Pareto-efficient, and apparently truthful - as it is similar to a serial dictatorship (constrained by proportionality). Is it true?}
% \biaoshuai{I think the mechanism you are suggesting is the same as mine? I don't think it is truthful: the agents with larger indices can manipulate and change the set of feasible (proportional) allocations, which may affect agent $1$'s allocation and potentially be beneficial for themselves. In fact, my EC paper shows that truthfulness is incompatible with proportionality.}
% \erel{It is the same as yours, except that we do not need the function $\Gamma$ --- we just use the same order all the time. It is simpler, although it is not anonymous. Is its RAT-degree still $n-1$?}
% \biaoshuai{I am not sure about this.}

\subsubsection{Description of Mechanism}
The mechanism has two components: an order selection rule $\Gamma$ and an allocation rule $\Psi$.
The order selection rule $\Gamma$ takes the valuation profile $(v_1,\ldots,v_n)$ as an input and outputs an order $\pi$ of the $n$ agents.
We use $\pi_i$ to denote the $i$-th agent in the order.
The allocation rule $\Psi$ then outputs an allocation based on $\pi$.

We first define the allocation rule $\Psi$.
Let $\propallocations$ be the set of all proportional allocations.
Then $\Psi$ outputs an allocation in $\propallocations$ in the following ``leximax'' way:
\begin{enumerate}
    \item the allocation maximizes agent $\pi_1$'s utility;
    \item subject to (1), the allocation maximizes agent $\pi_2$'s utility;
    \item subject to (1) and (2), the allocation maximizes agent $\pi_3$'s utility;
    \item $\cdots$
\end{enumerate}

We next define $\Gamma$.
We first adapt the volatility property of $\Gamma$ (defined in Sect.~\ref{sect:indivisible-EF1-n-1}) to the cake-cutting setting.

\begin{definition}
A function $\Gamma$ (from the set of valuation profiles to the set of orders on agents) is called \emph{volatile} if for any two agents $a_i\neq a_j$ and any two orders $\pi$ and $\pi'$, any set of $n-2$ value density functions $\{v_k\}_{k\notin\{i,j\}}$, any value density function $\bar{v}_j$, and any two reported valuation profiles $v_i,v_i'$ of agent $a_i$ with $v_i\neq v_i'$, there exists a valuation function $v_j$ of agent $a_j$ such that
\begin{itemize}

    \item $v_j$ is a rescaled version of $\bar{v}_j$, i.e., there exists $\alpha$ such that $v_j(x)=\alpha\bar{v}_j(x)$ for all $x\in[0,1]$;
% \erel{If we normalize the valuations, maybe we do not need this rescaling?}\biaoshuai{My construction of $\Gamma$ depends on the highest value of value density functions.}
    
    \item $\Gamma$ outputs $\pi$ for the valuation profile $\{v_k\}_{k\notin\{i,j\}}\cup\{v_i\}\cup\{v_j\}$, and $\Gamma$ outputs $\pi'$ for the valuation profile $\{v_k\}_{k\notin\{i,j\}}\cup\{v_i'\}\cup\{v_j\}$.
\end{itemize}

In other words, a manipulation of agent $i$ from $v_i$ to $v_i'$ can affect the output of $\Gamma$ in any possible way (from any order $\pi$ to any order $\pi'$), depending on the report of agent $j$.
\end{definition}


\begin{propositionrep}
    There exists a volatile function $\Gamma$.
\end{propositionrep}
\begin{proof}
    The function does the following.
    It first finds the maximum value among all the value density functions (overall all uniform segments): $v^\ast=\max_{i\in[n],\ell\in[m]}v_i(X_\ell)$.
    It then views $v^\ast$ as a binary string that encodes the following information:
    \begin{itemize}
        \item the index $i$ of an agent $a_i$,
        \item a non-negative integer $t$,
        \item two non-negative integers $a$ and $b$ that are at most $n!-1$.
        % \erel{Should these integers be in the range $0,\ldots,n!$ ?}\biaoshuai{It does not seem matter, as I am taking mod $n!$ at the end. But it does not harm to do it.}
    \end{itemize}
    We append $0$'s as most significant bits to $v^\ast$ if the length of the binary string is not long enough to support the format of the encoding.
    If the encoding of $v^\ast$ is longer than the length enough for encoding the above-mentioned information, we take only the least significant bits in the amount required for the encoding.
% \erel{Can $v^*$ always represent such a long string? What if, incidentally, $v^*=1$?}\biaoshuai{I added the last sentence}

    The order $\pi$ is chosen in the following way.
    Firstly, we use an integer between $0$ and $n!-1$ to encode an order.
    Then, let $s$ be the $t$-th bit that encodes agent $a_i$'s value density function.
    The order is defined to be $as+b\bmod (n!)$.

We now prove that $\Gamma$ is volatile.
Suppose $v_i$ and $v_i'$ differ at their $t$-th bits, 
so that that the $t$-th bit of $v_i$ is $s$ and the $t$-th bit of $v_i'$ is $s'\neq s$.
We construct a number $v^*$ that encodes the index $i$, the integer $t$, and two integers $a,b$ 
such that $as+b\bmod (n!)$ encodes $\pi$ and $as'+b\bmod (n!)$ encodes $\pi'$.


Then, we construct $v_j$ by rescaling $\bar{v}_j$ such that the maximum value among all density functions is attained by $v_j$, and this number is exactly $v^{\ast}$, that is, $v^\ast=v_j(X_\ell)$ for some uniform segment $X_\ell$.
If the encoded $v^*$ is not large enough to be a maximum value, we enlarge it as needed by adding most significant bits.

By definition, $\Gamma$ returns $\pi$ when $a_i$ reports $v_i$ and returns $\pi'$ when $a_i$ reports $v_i'$.

% \erel{What if the encoded $v^*$ is too small, as there are other agents (not $i,j$) that have larger values in some segments?}\biaoshuai{If the encoding of $v^\ast$ is longer than the length enough for encoding the above-mentioned information, we can only take a substring (say, the least significant bits) that is just enough. I added another sentence at the end of the first paragraph.}
\end{proof}

\subsubsection{Properties of the Mechanism}
The mechanism always outputs a proportional allocation by definition.
It is straightforward to check that it outputs a Pareto-efficient allocation.
\begin{propositionrep}
The $\Gamma$-$\Psi$ mechanism for cake-cutting always returns a Pareto-efficient allocation.
\end{propositionrep}
\begin{proof}
Suppose for the sake of contradiction that $(A_1,\ldots,A_n)$ output by the mechanism is Pareto-dominated by $(A_1',\ldots,A_n')$, i.e., we have
\begin{enumerate}
    \item $V_i(A_i')\geq V_i(A_i)$ for each agent $a_i$, and
    \item for at least one agent $a_i$, $V_i(A_i')> V_i(A_i)$.
\end{enumerate}
Property (1) above ensures $(A_1',\ldots,A_n')$ is proportional and thus is also in $\propallocations$: for each $i\in[n]$, $V_i(A_i')\geq V_i(A_i)\geq \frac1nV_i([0,1])$ (as the allocation $(A_1,\ldots,A_n)$ is proportional).
Based on the property (2), find the smallest index $i$ such that $V_{\pi_i}(A_i')>V_{\pi_i}(A_i)$.
We see that $(A_1,\ldots,A_n)$ does not maximize the $i$-th agent in the order $\pi$, which contradicts the definition of the mechanism.
\end{proof}

It then remains to show that the mechanism has RAT-degree $n-1$.
We need the following proposition; it follows from known results on super-proportional cake-cutting \citep{dubins1961cut,woodall1986note}; for completeness we provide a proof in the appendix.

\begin{propositionrep} \label{prop:strictlymorethanproportional}
    Let $\propallocations$ be the set of all proportional allocations for the valuation profile $(v_1,\ldots,v_n)$.
    Let $(A_1,\ldots,A_n)$ be the allocation in $\propallocations$ that maximizes agent $a_i$'s utility.
    If there exists $j\in[n]\setminus\{i\}$ such that $v_i$ and $v_j$ are not identical up to scaling, then $V_i(A_i)>\frac1nV_i([0,1])$.
\end{propositionrep}
\begin{proof}
    We will explicitly construct a proportional allocation $(B_1,\ldots,B_n)$ where $V_i(B_i)>\frac1nV_i([0,1])$ if the pre-condition in the statement is satisfied.
    Notice that this will imply the proposition, as we are finding the allocation maximizing $a_i$'s utility.
    To construct such an allocation, we assume $v_i$ and $v_j$ are normalized without loss of generality (then $v_i\neq v_j$), and consider the equal division allocation where each uniform segment $X_t$ is evenly divided.
    This already guarantees that agent $a_i$ receives a value of $\frac1nV_i([0,1])$.
    Since $v_i$ and $v_j$ are normalized and $v_i\neq v_j$, there exist two uniform segments $X_{t_1}$ and $X_{t_2}$ such that $v_i(X_{t_1})>v_j(X_{t_1})$ and $v_i(X_{t_2})<v_j(X_{t_2})$.
    Agent $a_i$ and $a_j$ can then exchange parts of their allocations on $X_{t_1}$ and $X_{t_2}$ to improve the utility for both of them, which guarantees the resultant allocation is still proportional.
    For example, set $\varepsilon>0$ be a very small number.
    Agent $a_i$ can give a length of $\frac{\varepsilon}{v_i(X_{t_2})+v_j(X_{t_2})}$ from $X_{t_2}$ to agent $a_j$, in exchange of a length of $\frac{\varepsilon}{v_i(X_{t_1})+v_j(X_{t_1})}$ from $X_{t_1}$.
    This describes the allocation $(B_1,\ldots,B_n)$.
\end{proof}

The proof of the following theorem is similar to the one for indivisible goods (\Cref{thm:gamma-psi-indivisible}). 
\begin{theoremrep}
    The $\Gamma$-$\Psi$ mechanism for cake-cutting has RAT-degree $n-1$.
\end{theoremrep}


\begin{proof}
Consider an arbitrary agent $a_i$ with the true value density function $v_i$, and an arbitrary agent $a_j$ whose reported value density function is unknown to $a_i$.
Fix $n-2$ arbitrary value density function $\{v_k\}_{k\notin\{i,j\}}$ for the remaining $n-2$ agents.
Consider an arbitrary manipulation $v_i'\neq v_i$.

Choose a uniform segment $X_t$ with respect to $(v_1,\ldots,v_{j-1},v_{j+1},\ldots,v_n)$,
satisfying $v_i(X_t)>0$.
Choose a very small interval $E\subseteq X_t$, such that the value density function
$$\bar{v}_j=\left\{\begin{array}{ll}
    0 & \mbox{if }x\in E \\
    v_i(x) & \mbox{otherwise}
\end{array}\right.$$
is not a scaled version of some $v_k$ with $k\in[n]\setminus\{i,j\}$.
% \erel{I did not understand this sentence. Do you mean: the normalized $\bar{v}_j$ not equal to any normalized $v_k$?}\biaoshuai{yes, to *some* $v_k$ is sufficient}
Apply the volatility of $\Gamma$ to find a value density function $v_j$ for agent $a_j$ that rescales $\bar{v}_j$ such that
\begin{enumerate}
    \item when agent $a_i$ reports $v_i$, agent $a_i$ is the first in the order output by $\Gamma$;
    \item when agent $a_i$ reports $v_i'$, agent $a_j$ is the first in the order output by $\Gamma$.
\end{enumerate}

Let $(A_1,\ldots,A_n)$ and $(A_1',\ldots,A_n')$ be the output allocation for the profiles $\{v_k\}_{k\notin\{i,j\}}\cup\{v_i\}\cup\{v_j\}$ and $\{v_k\}_{k\notin\{i,j\}}\cup\{v_i'\}\cup\{v_j\}$ respectively.
Since $\bar{v}_j$ is not a scaled version of some $v_k$, its rescaled version $v_j$ is also different.
By Proposition~\ref{prop:strictlymorethanproportional}, $V_j(A_j')>\frac1nV_j([0,1])$,
as $a_j$ is the highest-priority agent when $a_i$ reports $v'_i$.
Let $D$ be some subset of $A_j'$ with $V_j(D)>0$ and $V_j(A_j'\setminus D)\geq\frac1nV_j([0,1])$, and consider the allocation $(A_1^+,\ldots,A_n^+)$ in which $D$ is moved from $a_j$ to $a_i$, that is,
\begin{itemize}
    \item for $k\notin\{i,j\}$, $A_k^+=A_k'$;
    \item $A_i^+=A_i'\cup D$;
    \item $A_j^+=A_j'\setminus D$.
\end{itemize}
It is clear by our construction that the new allocation is still proportional with respect to $\{v_k\}_{k\notin\{i,j\}}\cup\{v_i'\}\cup\{v_j\}$.
In addition, by the relation between $\bar{v}_j$ and $v_i$ (and thus the relation between $v_j$ and $v_i$), we have $V_i(D)>0$ based on agent $a_i$'s true value density function $v_i$.
Therefore, under agent $a_i$'s true valuation, $V_i(A_i^+)>V_i(A_i')$.

If the allocation $(A_1^+,\ldots,A_n^+)$ is not proportional under the profile $\{v_k\}_{k\notin\{i,j\}}\cup\{v_i\}\cup\{v_j\}$ (where $v_i'$ is changed to $v_i$), then the only agent for whom proportionality is violated must be agent $i$, that is,$V_i(A_i^+)<\frac1nV_i([0,1])$.
It then implies $V_i(A_i')<\frac1nV_i([0,1])$.
On the other hand, agent $a_i$ receives at least her proportional share when reporting truthfully her value density function $v_i$.
This already implies the manipulation is not safe.

If the allocation $(A_1^+,\ldots,A_n^+)$ is proportional under the profile $\{v_k\}_{k\notin\{i,j\}}\cup\{v_i\}\cup\{v_j\}$, then it is in $\propallocations$.
Since agent $a_i$ is the first agent in the order when reporting $v_i$ truthfully, we have $V_i(A_i)\geq V_i(A_i^+)$, which further implies $V_i(A_i)>V_i(A_i')$.
Again, the manipulation is not safe.
\end{proof}

Finally, we analyze the run-time of our mechanism.
\begin{propositionrep}
The $\Gamma$-$\Psi$ mechanism for cake-cutting can be computed in polynomial time.
\end{propositionrep}
\begin{proof}
We first note that $\Gamma$ can be computed in polynomial time.
Finding $v^\ast$ and reading the information of $i,t,a$, and $b$ can be performed in linear time, as it mostly only requires reading the input of the instance.
In particular, the lengths of $a$ and $b$ are both less than the input length, so $as+b$ is of at most linear length and can also be computed in linear time.
Finally, the length of $n!$ is $\Theta(n\log n)$, so $as+b \bmod (n!)$ can be computed in polynomial time.
We conclude that $\Gamma$ can be computed in polynomial time.

We next show that $\Psi$ can be computed by solving linear programs.
Let $x_{it}$ be the length of the $t$-th uniform segment allocated to agent $a_i$.
Then an agent $a_i$'s utility is a linear expression $\sum_{t=1}^mv_{i}(X_t)x_{it}$, and requiring an agent's utility is at least some value (e.g., her proportional share) is a linear constraint.
We can use a linear program to find the maximum possible utility $u_{\pi_1}^\ast$ for agent $\pi_1$ among all proportional allocations.
In the second iteration, we write the constraint $\sum_{t=1}^mv_{\pi_1}(X_t)x_{it}\geq u_{\pi_1}^\ast$ for agent $\pi_1$, the proportionality constraints for the $n-2$ agents $[n]\setminus\{\pi_1,\pi_2\}$, and maximize agent $\pi_2$'s utility.
This can be done by another linear program and gives us the maximum possible utility $u_{\pi_2}^\ast$ for agent $\pi_2$.
We can iteratively do this to figure out all of $u_{\pi_1}^\ast,u_{\pi_2}^\ast,\ldots,u_{\pi_n}^\ast$ by linear programs. 
\end{proof}

\subsection{Towards An Envy-Free and Pareto-Optimal Mechanism with RAT-degree $n-1$}
Given the result in the previous section, it is natural to ask if the fairness guarantee can be strengthened to envy-freeness.
A compelling candidate is the mechanism that always outputs allocations with maximum \emph{Nash welfare}.
The Nash welfare of an allocation $(A_1,\ldots,A_n)$ is defined by the product of agents utilities:
$\displaystyle \prod_{i=1}^nV_i(A_i).$

It is well-known that such an allocation is envy-free and Pareto-optimal.
However, computing its RAT-degree turns out to be very challenging for us.
We conjecture the answer is $n-1$.
\begin{open}
    What is the RAT-degree of the maximum Nash welfare mechanism?
\end{open}