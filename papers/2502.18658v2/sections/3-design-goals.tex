\section{Design Goals}
% A technology probe is an ''\textit{instrument that is deployed to find out about the unknown—returning with useful or interesting data}'' \cite{hutchinson2003techprobe}. Technology probes are a common approach in human-computer interaction research to engage users directly in contextualized studies \cite{gaver1999cultural_probes,graham2008probe_participation}.
To investigate the effects of a proactive AI programming agent on user workflow, we used a technology probe ---  an \textit{instrument that is deployed to find out about the unknown—returning with useful or interesting data} \cite{hutchinson2003techprobe} --- to explore the design space of the \revise{timing, representation, and scope of interaction} (RQ1) with the following design considerations:
% We further analyze relevant literature on human collaboration and interruption in programming and other types of teamwork to identify key design considerations to guide the implementation of the \sys{} prototype:

\begin{itemize}
    % \item \textbf{DG1: Facilitate timely and context-aware proactive AI assistance} by anticipating programmer needs based on their code editor activities and offering suggestions, insights, or corrections to support their tasks.
    \item \textbf{\revise{DG1: Establish heuristics for timely proactive AI assistance}} by anticipating programmer needs based on editor activities and offering suggestions, insights, or corrections to support their tasks.
    % \item \textbf{DG2: Increase AI visibility} by representing the AI agent using visible cues to indicate its actions, intentions, and decision-making processes, enhancing the user's awareness of the agent's assistance
    \item \textbf{\revise{DG2: Represent AI agent's presence}} using visible cues to indicate its actions, intentions, and decision-making processes, enhancing the user's awareness of the agent's assistance.
    % \item \textbf{DG3: Reduce user interaction efforts to adopt proactive AI assistance} by minimizing interruptions, reducing conflicts, and managing past interaction contexts.
    \item \revise{\textbf{DG3: Provide flexible scopes of interaction} by designing interactions at both a global code editor level and at a local code-line level to meet different abstractions of user need and improve context management.}
    \item \textbf{DG4: Support different mechanisms in the probe and creating different versions of the system} to structure comparisons and evaluations of different designs
    
\end{itemize}


