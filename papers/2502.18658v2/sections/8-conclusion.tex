\section{Conclusion}
As AI programming tools increasingly feature intelligent agents that proactively support workflows, we aimed to evaluate the impact of AI-initiated assistance compared to the traditional user-driven approach.
We designed \sys{}, a design probe that analyzes the user's actions and current work state to initiate in-time, contextualized support. 
The AI agent employs defined heuristics derived from human collaboration principles to time the proactive assistance.
\sys{} manifests the AI agent's visual presence in the editor to showcase the interaction process, and enables localized context management using threaded breakout messages and provenance signals.
In a three-condition experiment with 18 programmers, we found that proactivity lowered users' expression effort to convey intent to the AI, but also incurred more workflow disruptions. 
However, our design of \sys{} alleviated disruptions and increased users' awareness of the AI, resulting in a collaboration experience closer to working with a partner than a tool.
From our study, we uncovered different strategies users adopted to create a balanced and productive workflow with proactive AI across different programming processes, but also revealed concerns about over-reliance, potentially leading to a loss of user control, ownership, and code understanding.
Summarizing our findings, we proposed a set of design implications and outlined opportunities and risks for future systems that integrate proactive AI assistance in users' programming workflows.
