\begin{abstract}
    AI programming tools enable powerful code generation, and recent prototypes attempt to reduce user effort with proactive AI agents, but their impact on programming workflows remains unexplored.
    We introduce and evaluate Codellaborator, a design probe LLM agent that initiates programming assistance based on editor activities and task context. 
    We explored three interface variants to assess trade-offs between increasingly salient AI support: prompt-only, proactive agent, and proactive agent with presence and context (Codellaborator). 
    In a within-subject study ($N=18$), we find that proactive agents increase efficiency compared to prompt-only paradigm, but also incur workflow disruptions. 
    However, presence indicators and interaction context support alleviated disruptions and improved users' awareness of AI processes. 
    We underscore trade-offs of Codellaborator on user control, ownership, and code understanding, emphasizing the need to adapt proactivity to programming processes. 
    Our research contributes to the design exploration and evaluation of proactive AI systems, presenting design implications on AI-integrated programming workflow.
    % AI programming tools enable powerful code generation, and recent prototypes attempt to lower use efforts by building proactive AI agents, but their impact on human programming workflows remains unexplored.
    % We introduce and evaluate Codellaborator, a design probe AI agent that initiates programming assistance based on user activities and task context.
    % In a three-condition within-subject experiment ($N=18$), participants reported that compared to a baseline akin to Github Copilot, the proactive agent reduced expression efforts with AI-initiated interactions but also incurred workflow disruptions.
    % The Codellaborator condition with visible agent presence and context management informed by collaboration principles alleviated disruptions and improved users' awareness of AI processes.
    % We also underscore trade-offs Codellaborator could bring to user control, ownership, and code understanding, and the need to adapt proactivity to various programming processes.
    % Our research contributes to the design exploration and evaluation of proactive AI systems, presenting design implications on AI-integrated programming workflow.
\end{abstract}