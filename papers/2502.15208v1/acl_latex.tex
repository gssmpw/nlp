% This must be in the first 5 lines to tell arXiv to use pdfLaTeX, which is strongly recommended.
\pdfoutput=1
% In particular, the hyperref package requires pdfLaTeX in order to break URLs across lines.

\documentclass[11pt]{article}

% Change "review" to "final" to generate the final (sometimes called camera-ready) version.
% Change to "preprint" to generate a non-anonymous version with page numbers.
\usepackage[]{acl}

% Standard package includes
\usepackage{times}
\usepackage{latexsym}   

% For proper rendering and hyphenation of words containing Latin characters (including in bib files)
\usepackage[T1]{fontenc}
% For Vietnamese characters
% \usepackage[T5]{fontenc}
% See https://www.latex-project.org/help/documentation/encguide.pdf for other character sets

% This assumes your files are encoded as UTF8
\usepackage[utf8]{inputenc}

% This is not strictly necessary, and may be commented out,
% but it will improve the layout of the manuscript,
% and will typically save some space.
\usepackage{microtype}

% This is also not strictly necessary, and may be commented out.
% However, it will improve the aesthetics of text in
% the typewriter font.
\usepackage{inconsolata}

%Including images in your LaTeX document requires adding
%additional package(s)

\usepackage{subfigure}
\usepackage{array}
\usepackage{geometry}
\usepackage{booktabs}
\usepackage{amsmath}
\usepackage{graphicx}
\usepackage{hyperref}
\usepackage{url}
\usepackage{wrapfig}
\usepackage{subcaption}
\usepackage{colortbl}
\usepackage{xcolor}
% If the title and author information does not fit in the area allocated, uncomment the following
%
%\setlength\titlebox{<dim>}
%
% and set <dim> to something 5cm or larger.

\title{Unveiling Attractor Cycles in Large Language Models: A Dynamical Systems View of Successive Paraphrasing}
% Glimpsing Generation Momentum through the Lens of Recursive Paraphrasing
% Author information can be set in various styles:
% For several authors from the same institution:
% \author{Author 1 \and ... \and Author n \\
%         Address line \\ ... \\ Address line}
% if the names do not fit well on one line use
%         Author 1 \\ {\bf Author 2} \\ ... \\ {\bf Author n} \\
% For authors from different institutions:
% \author{Author 1 \\ Address line \\  ... \\ Address line
%         \And  ... \And
%         Author n \\ Address line \\ ... \\ Address line}
% To start a separate ``row'' of authors use \AND, as in
% \author{Author 1 \\ Address line \\  ... \\ Address line
%         \AND
%         Author 2 \\ Address line \\ ... \\ Address line \And
%         Author 3 \\ Address line \\ ... \\ Address line}


\author{ 
 Zhilin Wang$^{\spadesuit}\footnotemark[1]$\hspace{0.5mm},
 Yafu Li$^{\spadesuit}\footnotemark[1]$\hspace{0.5mm},
  Jianhao Yan$^{\diamondsuit \clubsuit}$\hspace{0.5mm} \\
\bf{Yu Cheng$^{\heartsuit}$\hspace{0.5mm}\hspace{0.5mm}, 
 Yue Zhang$^{\clubsuit }$}\hspace{0.2mm}\hspace{1.5mm} \\
$^\spadesuit$ Shanghai AI Laboratory\ \ \ \quad$^\clubsuit$Westlake University \\\quad$^\diamondsuit$  Zhejiang University \ \ \quad$^\heartsuit$Chinese University of Hong Kong \\
 % \texttt{yafuly@gmail.com} \quad \texttt{linzwcs@gmail.com} \quad 
 \texttt{\{linzwcs,yafuly,elliottyan37\}@gmail.com}  \\
 \quad\texttt{\{chengyu\}@cse.cuhk.edu.hk} 
 \quad\texttt{\{zhangyue\}@westlake.edu.cn}\\
}

%\author{
%  \textbf{First Author\textsuperscript{1}},
%  \textbf{Second Author\textsuperscript{1,2}},
%  \textbf{Third T. Author\textsuperscript{1}},
%  \textbf{Fourth Author\textsuperscript{1}},
%\\
%  \textbf{Fifth Author\textsuperscript{1,2}},
%  \textbf{Sixth Author\textsuperscript{1}},
%  \textbf{Seventh Author\textsuperscript{1}},
%  \textbf{Eighth Author \textsuperscript{1,2,3,4}},
%\\
%  \textbf{Ninth Author\textsuperscript{1}},
%  \textbf{Tenth Author\textsuperscript{1}},
%  \textbf{Eleventh E. Author\textsuperscript{1,2,3,4,5}},
%  \textbf{Twelfth Author\textsuperscript{1}},
%\\
%  \textbf{Thirteenth Author\textsuperscript{3}},
%  \textbf{Fourteenth F. Author\textsuperscript{2,4}},
%  \textbf{Fifteenth Author\textsuperscript{1}},
%  \textbf{Sixteenth Author\textsuperscript{1}},
%\\
%  \textbf{Seventeenth S. Author\textsuperscript{4,5}},
%  \textbf{Eighteenth Author\textsuperscript{3,4}},
%  \textbf{Nineteenth N. Author\textsuperscript{2,5}},
%  \textbf{Twentieth Author\textsuperscript{1}}
%\\
%\\
%  \textsuperscript{1}Affiliation 1,
%  \textsuperscript{2}Affiliation 2,
%  \textsuperscript{3}Affiliation 3,
%  \textsuperscript{4}Affiliation 4,
%  \textsuperscript{5}Affiliation 5
%\\
%  \small{
%    \textbf{Correspondence:} \href{mailto:email@domain}{email@domain}
%  }
%}

\begin{document}
\maketitle
\begin{abstract}
\renewcommand{\thefootnote}{\fnsymbol{footnote}}
\footnotetext[1]{\ Equal contribution. Work was done during Zhilin Wang's internship at Shanghai AI Laboratory and Westlakenlp Lab.}
%\footnotetext[2]{\ Corresponding authors.}

 
% We demonstrate that periodicity can be mitigated through contextual interventions and perturbation strategies, but fully escaping these attractor states remains challenging. 

% Paraphrasing tools based on large language models (LLMs) have become widely used for rephrasing text. While some researchers focus on single-instance paraphrasing for tasks such as AI text detection and data augmentation, other work explores successive paraphrasing, which involves rephrasing a text multiple times in a sequence. In this study, we conduct an in-depth analysis of the characteristics of successive paraphrasing in current prevalent LLMs. 
% Our findings reveal that LLMs often regenerate paraphrases similar to earlier versions, rather than generating entirely new ones, demonstrating a phenomenon of 2-periodicity. This periodicity can be attributed to the self-reinforcing nature of LLMs, which is represented by three types of convergence in our work.
% This periodicity is not limited to paraphrasing but extends to other sequential tasks with similar characteristics. Additionally, the period can be adjusted by providing LLMs with contextual history.
% The results of our analysis experiments show that periodicity can be alleviated, but not eliminated, while preserving the original meaning. 
% The perturbation experiments further reveal that LLMs tend to perform synonym replacement rather than altering the text structure during successive paraphrasing. 
% Our work offers a comprehensive analysis of successive paraphrasing, identifies the underlying periodicity, and highlights the limited expressive capabilities of LLMs.

Dynamical systems theory provides a framework for analyzing iterative processes and evolution over time. Within such systems, repetitive transformations can lead to stable configurations, known as attractors, including fixed points and limit cycles. Applying this perspective to large language models (LLMs), which iteratively map input text to output text, provides a principled approach to characterizing long-term behaviors. Successive paraphrasing serves as a compelling testbed for exploring such dynamics, as paraphrases re-express the same underlying meaning with linguistic variation. Although LLMs are expected to explore a diverse set of paraphrases in the text space, our study reveals that successive paraphrasing converges to stable periodic states, such as 2-period attractor cycles, limiting linguistic diversity. This phenomenon is attributed to the self-reinforcing nature of LLMs, as they iteratively favour and amplify certain textual forms over others. This pattern persists with increasing generation randomness or alternating prompts and LLMs. These findings underscore inherent constraints in LLM generative capability, while offering a novel dynamical systems perspective for studying their expressive potential.





\end{abstract}



\documentclass[../main.tex]{subfiles}
\graphicspath{{../images/}}
\makeatletter
\def\input@path{{../images/}}
\makeatother
\begin{document}
\section{Introduction}
\begin{figure}
\centering
\begin{tikzpicture}
\node[inner sep=0pt] (ws) at (0, 0) {
\includegraphics[height=.4\textwidth, trim={10cm 0 10cm 0},clip]{world_space.png}};
\node[inner sep=0pt] (cs) at (6,0) {\includegraphics[height=.4\textwidth, trim={10cm 1cm 10cm 4cm},clip]{conf_space.png}};
\end{tikzpicture}
\vspace{-5pt}
\label{fig:pbrm_intro}
\caption{\textbf{Left}: Shows world space obstacles as grey spheres. Robots start and goal configuration is colored red and green, respectively. Configurations along the computed path are colored transparent blue. \textbf{Right:} Mapped world space scenario to configuration space. Obstacle region is the grey mesh. Red spheres are collision-free regions computed by the neural SCDF. The optimized shortest path in the convex corridor is the blue curve.}
\vspace{-25pt}
\end{figure}
Motion planning is the problem of finding a collision-free trajectory that connects a given start and goal configuration. The planning takes place in the configuration space of the robot. For single body robots, like mobile robots or drones, the configuration space and the world space are usually the same. This simplifies the planning, since explicit obstacle representations are available which enables geometrical tools like separating hyperplanes, smallest distance to obstacles etc., to be used when designing motion planning algorithms. For multi-body robots like manipulators, the situation is completely different. The world space obstacles are usually mapped to non-convex regions, and to make the problem even harder, the mapping is usually not known. Forming explicit representations of the obstacle region in the configuration space is usually too expensive or intractable. Despite all of this, sampling based planners are used with great success, which mainly is due to their use of implicit representations of the obstacle region. The basic idea is to construct a graph in the configuration space that covers and connects the collision-free region. From this graph, a path can be extracted that connects a given start and goal configuration. The approach is computationally expensive, since the graph is constructed with the smallest geometrical building block available, points, which represents a collision-check. Furthermore, the extracted paths from the graph are non-smooth and jagged due to the stochastic nature of the approach. This adds an additional post-processing step to the process, where the paths are shortcutted and smoothened, before the path can be used for tracking. Clearly a lot of time is invested to form this graph and produce smooth paths. Thus, if the obstacles start to move, then all of this work is done in no use, since all points that make up this graph need to be re-verified, which is simply too time consuming to be done in real time.
\\\\
In this work, we want to address the existing drawbacks of the sampling based planners. Our main contribution is an improved motion planner where each vertex in the graph covers a collision-free region in the form of a sphere instead of a point and where the edges are formed with neighboring intersecting spheres. This representation has the advantage of instead of returning piecewise linear paths, returning a sequence of overlapping spheres, i.e. a convex corridor, that connects a given start and goal configuration, illustrated in Figure \ref{fig:pbrm_intro}. This convex corridor allows us to use convex optimization to produce smooth trajectories, instead of computationally expensive post-processing methods. The representation further allows us to estimate the coverage of the collision-free space, which gives us awareness and feedback in the offline roadmap construction phase. Finally, our representation is simple to adapt to moving obstacles, simply requery for the new radii and recheck for intersections. 
\\\\
The spherical collision-free regions are formed using a signed distance function (SDF), which is a function that returns the smallest distance from an arbitrary point to the boundary of an obstacle. As the name implies, the distance is signed, thus if the point is inside the obstacle it is negative otherwise positive. If the distance is positive, a sphere with radius equal to the distance is guaranteed to cover a collision-free region. Using an SDF in motion planning is not new, but what is novel about our approach is that we express the distance in the configuration space instead of the world space and by doing so allows us to form these convex collision-free regions. We refer to the resulting SDF as a signed configuration distance function (SCDF). Computing an SCDF analytically is non-trivial, our approach is therefore to parameterize the SCDF with a deep neural network and learn the mapping by supervised learning. Our resulting neural SCDF can compute distances for different parameter values of obstacle shapes and we also show how multiple distances can be combined, thus making our approach flexible.
\section{Related work}
Motion planning algorithms can roughly be divided into three families, grid-based, sampling based and optimization based methods. Grid-based methods (GBM) discretize the planning space from which a graph is then compiled. A standard search method is A$^\star$ \citep{a_star}, which is classified as an \textit{informed} search method, since it employs a heuristic function to speed up the search. A$^\star$ guarantees to return an optimal path at the level of discretization used. GBMs usually discretize the planning space by a regular lattice and this limits the GBMs to problems with low dimensionality due to the curse of dimensionality. Thus, GBMs are usually limited to single-body robots where the degrees of freedom (DOF) are low. To overcome the inherent scaling problem with the GBMs, stochastic methods are usually used for multi-body robots. These methods are termed as sampling-based methods (SBM) and core members within this family are the rapidly-exploring random trees (RRT) \citep{rrt} and the probabilistic roadmap (PRM) \citep{prm}. RRT grows a tree from the start configuration and explores the collision-free region in a rapid way until it is able to connect to the goal region. RRT is usually improved by bi-directional planning \citep{rrt_connect}, i.e. an additional tree is grown from the goal configuration and the trees are tested for connection after any tree has been expanded. RRT is a single-query method, thus it searches for a path from scratch each time it is queried. Contrary to this, PRM is a multi-query method, which solves for multiple queries without starting from scratch. PRM does this by creating a roadmap (graph) that covers the collision-free space as an offline step. The graph is then used to solve for multiple queries. PRMs are used in cases where the environment does not change since the extra offline step is too computationally costly and needs to be re-done if the environment is changed. In our work, we address this inherent issue by using a different roadmap representation. Our vertices in the graph cover a collision-free region in the form of spheres and we form the edges by checking for intersecting spheres. If something in the environment changes, we recompute the spheres radii and recheck the intersections, without relying on collision detection. We use a trained neural network to compute the sphere radius, therefore querying for the radius can be done fast, hence our representation enables the PRM for dynamic environments.
\\\\
In the recent decades, optimization based methods (OBM) \citep{chomp, schulman, itomp, stomp} have been introduced as an alternative to SBM for multi-body robots. Like the SBM, the OBMs scale well to higher dimensional problems and produce smoother motion. It is common to use a SDF in the optimization since it is a smooth function, thus enabling gradient-based methods. However, the standard way of expressing the SDF is in world space. The distance therefore needs to be mapped to the configuration space by the forward kinematics. This mapping makes the optimization problem a non-linear program (NLP), which is computationally expensive to solve. Recently, a different approach has been proposed. In \cite{mp_gcs} motion planning is formulated as a convex optimization problem by using the graph of convex sets framework \citep{gcs}. The underlying idea is to decompose the collision-free space into intersecting convex sets from which a convex optimization problem is formulated. In cases where an explicit representation of the obstacles in the configuration space exists, like for single-body robots, creating collision-free convex regions can be done fast \citep{iris}. For multi-body robots, this is non-trivial. Existing work does this successfully \citep{iris_nlp, iris_c} by an optimization based approach, but the methods are still too time consuming to be used in the presence of moving obstacles. Our approach is instead to use deep learning to learn an SDF expressed in the configuration space. With this, we can query for shortest distances to the collision boundary, which allows us to expand spherical regions which are collision-free. Our approach is fast and therefore enables our suggested roadmap planner to be used in dynamic environments.
\\\\
Recent research has focused on learning collision detection \citep{fk_kernel_distance, diffco, graphdistnet} by predicting the signed distance between the robot links and the surrounding obstacles in the world space. The learned SDF is used in trajectory optimization but since the distance is expressed in the world space, the problem becomes an NLP and therefore takes a long time to solve. We take a novel approach and suggest to instead express the signed distance in the configuration space. This allows us to improve the PRM at the same time as it enables convex optimization for trajectory optimization, which runs faster and is more reliable than NLP solvers. In \cite{cspf} a learned signed distance function in the configuration space is proposed similar to our approach. However, their approach is restricted to point cloud representations, while we propose to represent the obstacles as parameterized geometric shapes, e.g. spheres. Furthermore, we also show how to use our learned SCDF to improve an existing roadmap planner.
\section{Problem formulation}
A robot is located in the world space, $\W \subset \R^3 $. The unique location of the robot is given by its configuration $\q \in \C$, where $\C$ is the configuration space. The set of points covered by the robots bodies at a certain configuration is expressed as $\B(\q) \subset \W$. The robot is surrounded by $\NrObst$ obstacles $\O = \bigcup_{i=1}^{\NrObst} \O_i$, where  $\O_i \subset \W$. The representation of the obstacle in the configuration space is the set $\C\O_i = \{\q \in \C \: |\: \B(\q) \cap \O_i \neq \emptyset \}$. The obstacle space is formed as $\Co = \bigcup_{i=1}^{\NrObst} \C \O_i$. The complement is referred to as the free space, $\Cf = \C \setminus \Co$. The path planning problem is a tuple, ($\Cf$, $\qStart$, $\qGoal$), where we want to connect a query pair, consisting of a start, $\qStart$, and goal configuration, $\qGoal$, with a geometric path, $\q(s): [0, 1] \mapsto \Cf$, such that $\q(0)=\qStart$ and $\q(1)=\qGoal$, or report correctly when such a path does not exist.
\end{document}


\section{Successive Paraphrasing as System Function}

% \subsection{Task Definition}

% Successive paraphrasing involves repeatedly generating variations of a given text while maintaining semantic equivalence, where each iteration builds upon the previous output. 
% The ideal outcome of this process is to produce a diverse set of paraphrases that maintain the original meaning while exploring different linguistic structures, styles, and nuances. 
% Let $\mathcal{T}$ be a compact metric space representing all possible texts, $P: \mathcal{T} \rightarrow \mathcal{T}$ represents the LLM's mapping from a text to its paraphrase.
% Starting from an initial text $T_0 \in \mathcal{T}$, we define the sequence $\{ T_n \}_{n=0}^\infty$ recursively by:
% \begin{equation}
% T_{n+1} = P(T_n), \quad n = 0, 1, 2, \dots    
% \end{equation}
% The set $\mathcal{P}(T)$ denotes the complete text space for valid paraphrases of $T$, which is assumed as a finite space. 
% As the length of the paraphrasing sequence increases, the number of potential paraphrases, i.e., $|\mathcal{P}(T)|$, grows.
% Ideally, each iteration leverages the inherent knowledge embedded in in language models to introduce novel vocabulary, syntax, and phrasing in subsequent generations—thereby enhancing textual diversity.
% the language model draws upon its vast training data to produce varied outputs, which can lead to the discovery of unique perspectives and unexpected formulations.



In this section, we briefly introduce the theoretical framework of dynamical systems and applies it to understand the iterative process of successive paraphrasing. 
By viewing paraphrase generation as the repeated application of a transformation (the LLM’s paraphrasing function), we connect observed phenomena, e.g., periodicity and convergence, to well-studied concepts in systems theory. 
% This theoretical grounding will guide our interpretation of the empirical results in subsequent sections.

\subsection{Systems Theory Foundations}

Systems theory provides a broad mathematical and conceptual framework for analyzing how complex processes evolve over time~\cite{system1}. 
The core idea is modeling the state of a system and its evolution through deterministic or stochastic rules. 
In continuous or discrete time, systems can exhibit distinct behaviors, ranging from stable equilibria to oscillatory dynamics or even chaotic patterns.

A \textbf{dynamical system} is commonly defined as a set of states and a rule describing how those states vary under iteration. 
When a transformation repeatedly maps an initial state to a new state, one of several outcomes often emerges:
\textit{Fixed Points}: States that remain unchanged under the transformation, representing equilibrium;
\textit{Limit Cycles}: Closed loops of states that recur periodically, representing sustained oscillations;
\textit{More Complex Attractors}: Patterns to which the system’s trajectories converge, including chaotic attractors.

These attractors shape the long-term behavior of the system. If an initial state lies within the basin of attraction of a limit cycle, for example, the system will converge to that cycle regardless of small perturbations. Identifying such attractors offers valuable insights into the stability and variability of the system’s evolution.

\subsection{Framing Successive Paraphrasing as a Dynamical System}
Successive paraphrasing involves iteratively generating variations of a given text while maintaining semantic equivalence, where each iteration builds upon the previous output. 
We propose viewing successive paraphrasing as a discrete dynamical system. 
Let $\mathcal{T}$ be the space of all possible texts. 
Consider a large language model that defines a paraphrasing function: $P: T \rightarrow T$,
where $P(T)$ outputs a paraphrase of the input text $T$. 
Given an initial text $T_0 \in \mathcal{T}$, successive paraphrasing generates the sequence $\{ T_n \}_{n=0}^\infty$ recursively by:
\begin{equation}
T_{n+1} = P(T_n), \quad n = 0, 1, 2, \dots    
\end{equation}

The set $\mathcal{P}(T)$ denotes the complete text space for valid paraphrases of $T$, which is assumed as a finite space.
In theory, the space of potential paraphrases $\mathcal{P}(T)$ can be vast, especially as text length grows. 
Each new iteration can potentially explore fresh textual variations, e.g., new syntactic structures, vocabulary choices, and stylistic nuances, while maintaining semantic equivalence.
From a systems perspective, if the mapping $P$ is capable of diversifying output states, one might expect the generated text sequence to spread broadly through the space $\mathcal{P}(T)$, never stuck in repetitive patterns, resembling a system without stable attractors. 
In contrast, if the LLM’s internal biases lead to favouring certain textual forms, the sequence may enter a basin of attraction and converge onto a stable set of states. 
In other words, rather than exhibiting limitless variety, the system might find itself drawn to limit cycles, i.e., periodic attractors in the paraphrase space.



\begin{figure*}[t!]
    \centering
    \includegraphics[width=1\textwidth]{article/figures/periodicity.pdf}
    \caption{The difference confusion matrix for successive paraphrasing, where EN and ZH denotes English and Chinese sentence-level paraphrase generation accordingly. Both the x and y axes represent paraphrases at each step, and the value at the ($i$-th, $j$-th) grid position indicates the difference between the paraphrases at the $i$-th and $j$-th positions. 
    A darker color indicates a smaller difference value between two paraphrases. 
    The black arrow underlines the differences between \(T_i\) and \(T_{i-2}\), and averaging these values and subtracting the result from 1 gives our 2-period degree $\tau$.
    } 
    \label{figs:periodicity}
\end{figure*}



\section{Experiment Setup}
To systematically investigate this pattern, we first build dedicated testbeds and evaluation criteria.
\paragraph{Source Data Collection.}
We consider English and Chinese paraphrasing in this work.
For English paraphrase generation, we collect human-written source documents by sampling instances from the MAGE dataset~\cite{li2023mage}.
Specifically, we uniformly collect 1,000 sentences and 30 paragraphs from each domain in the dataset.
This results in a total of 1,000 sentences and 300 paragraphs for subsequent paraphrasing.
For Chinese, we source 200 sentences from WMT 2019 \citep{barrault-etal-2019-findings} and 200 sentences from Wikipedia \cite{wikidump}. 
Detailed data statistics is presented in Appendix~\ref{app:data_stat}.
The main experiments (Section~\ref{sec:main}) utilize sentence-level paraphrasing datasets, while analytic experiments employ paragraph-level datasets to demonstrate the generality of our findings (Section~\ref{sec:Analysis}).

% model setting
\paragraph{Paraphrase Generation.}
For English paraphrasing, we utilize Mistral-7B-Instruct-v0.3 ~\cite{jiang2023mistral7b}, Meta-Llama-3-8B-Instruct ~\cite{touvron2023llama2openfoundation}, Meta-Llama-3-70B-Instruct~\cite{touvron2023llama2openfoundation}, Qwen2.5-7B-Instruct, Qwen2.5-14B-Instruct, Qwen2.5-72B-Instruct~\cite{qwen2}, GPT-4o-mini and GPT-4o~\cite{openai2024gpt4technicalreport}.
For Chinese, we use  Qwen2.5-7B-Instruct, Qwen2.5-14B-Instruct, Qwen2.5-72B-Instruct, and GPT-4o-mini for paraphrase generation.
% inference hyper-parameter
By default, we set the temperature to 0.6 and p to 0.9 during the decoding process.
% TBD: steps? and beams?
We sample 10 different paraphrases at each step by setting the number of search beams to 10 and sequentially rephrasing each sample for 15 rounds. 
We select the candidate with the highest probability for the next paraphrasing iteration. 

% evaluation metrics
%For evaluation metrics, we employ the normalized edit distance function $d$ to measure the diversity between two paraphrases. 
%We define the costs of operations, including insertion, deletion, and replacement, as 1. 
%The formula for the \textbf{normalized edit distance} can be expressed as:
%\[
%d(A, B) = \frac{E(A, B)}{\max(|A|, |B|)}
%\]
%where \( E(A, B) \) represents the edit distance between strings \( A \) and \( B \), and \( |A| \) and \( |B| \) denote the lengths of strings \( A \) and \( B \), respectively.
\paragraph{Evaluation Metrics.}
We use the normalized Levenshtein edit distance function $d$ to quantify the textual differences between two paraphrases.
To provide a more intuitive of the attractor cycle, we propose a metric termed 2-periodicity degree to quantify and study the cyclic pattern in successive paraphrasing.
The 2-periodicity degree \(\tau\) is defined as $\tau = 1 - \frac{1}{M-2} \sum_{i=3}^{M} d(T_{i}, T_{i-2})$, which captures the average textual similarity between the current paraphrase and that from two steps prior. 
$M$ denotes the total number of paraphrasing iterations. 
A higher $\tau$ indicates stronger periodicity, i.e., similar between two paraphrases. 
For instance, if successive paraphrases exhibit perfect 2-periodicity such that \(d(T_{i}, T_{i-2}) = 0\), then \(\tau = 1\), indicating that the current paraphrase matches exactly with that from two steps earlier. 
To evaluate semantic equivalence, we employ cosine similarity on sentence embeddings~\footnote{https://huggingface.co/sentence-transformers/all-MiniLM-L6-v2}~\cite{sentence_embed}.





%\begin{figure*}
%    \centering
%    \subfigure[\textcolor{blue}{\textbf{GPT-4o-mini}}]{\includegraphics[width=0.3\linewidth]{article/figures/periodicity/gpt4o-mini.pdf}}   
%    \subfigure[\textcolor{blue}{\textbf{Llama3-8B}}]{\includegraphics[width=0.3\linewidth]{article/figures/periodicity/llama3-8b.pdf}}   
%    \subfigure[\textcolor{red}{\textbf{GLM4-9B}}]{\includegraphics[width=0.3\linewidth]{article/figures/periodicity/glm4-zh.pdf}}   
%    \subfigure[\textcolor{blue}{\textbf{GPT-4o}}]{\includegraphics[width=0.3\linewidth]{article/figures/periodicity/gpt4o.pdf}}   
%    \subfigure[\textcolor{blue}{\textbf{Llama3-70B}}]{\includegraphics[width=0.3\linewidth]{article/figures/periodicity/llama3-70b.pdf}}   
%    \subfigure[\textcolor{red}{\textbf{GPT-4o-mini}}]{\includegraphics[width=0.3\linewidth]{article/figures/periodicity/gpt4o-mini-zh.pdf}} 
%    \caption{The diversity confusion matrix of successive paraphrasing. Figures (a), (b), (d), and (e) illustrate the diversity confusion matrix for English successive paraphrasing, whereas figures (c) and (f) display the matrix for Chinese.}  
%    \label{figs:periodicity}
%\end{figure*}

%\begin{table*}[!h]
%    \centering
%    \begin{tabular}{lccccc}
%        \toprule
%        \textbf{Model} &\textcolor{blue}{\textbf{GLM4-9B}} &\textcolor{blue}{\textbf{Llama3-8B}}& \textcolor{blue}{\textbf{Llama3-70B}} & \textcolor{red}{\textbf{GLM-9B}} \\
%        \midrule
%         & 0.25& 0.29 & 0.40 & 0.24 \\
%        \midrule
%        \textbf{Model} & \textcolor{blue}{\textbf{Mistral-7B}} & \textcolor{blue}{\textbf{GPT-4o-mini}} & \textcolor{blue}{\textbf{GPT-4o}} & \textcolor{red}{\textbf{GPT-4o-mini}} \\
%        \midrule
%          & 0.29 & 0.17 & 0.18& 0.12 \\
%        \bottomrule
%    \end{tabular}
    %\begin{tabular}{lccccc}
    %    \toprule
    %    Model & GLM-9B & Llama3-8B &Llama3-70B & \textcolor{red}{GLM-9B} \\
    %    \midrule
    %    Periodicity &0.2511 & 0.2839 &  0.4018 & 0.2418 \\
    %    \midrule
    %    Model & Mistral-7B & GPT-4o-mini & GPT-4o & \textcolor{red}{GPT-4o-mini} \\
    %    \midrule
    %    Periodicity &0.2897 & 0.1673 & 0.1847 & 0.1219 \\
    %    \bottomrule
    %\end{tabular}
%    \caption{The periodicity degree of different LLMs during successive paraphrasing. The models represented in blue indicate the periodicity degree on the English dataset, while those in red indicate it on the Chinese dataset.}
%    \label{table:periodicity}
%\end{table*}


%\begin{table*}[!h]
%    \centering
%    \begin{tabular}{l l l l l}
%        \toprule
%        \textbf{Experiments} & \textbf{4 Other Tasks} & \textbf{Model \& Prompt-Changing} & \textbf{Perturbation} & \textbf{Mitigation} \\ 
%        \midrule
%        \textbf{Iterative Number} & 10 & 15 & 10 & 15 \\ 
%        \textbf{Dataset} & Paragraph & Paragraph & Paragraph & Sentence Level \\ 
%        \textbf{Dataset} & gpt-4o-mini & models set 1 & gpt-4o-mini & en models(en) \\ 
%        \bottomrule
%    \end{tabular}
%    \caption{Summary of ANALYSIS Experiments, prompt-changing: gpt-4o-mini: model-changing: gpt-4o-mini, gpt-4o, glm4, llama3-8b}
%\end{table*}
\begin{table*}[t!]
    \centering
    \small
    \begin{tabular}{cccccc}
        \toprule
        \rowcolor{blue!20} \textbf{Mistral-7B} & \textbf{Llama3-8B} & \textbf{Llama3-70B} & \textbf{GPT-4o-mini} & \textbf{GPT-4o} & \textbf{Qwen2.5-7B} \\
        \midrule
        0.71 & 0.72 & 0.60 & 0.83 & 0.81 & 0.86 \\
        \midrule
        \cellcolor{blue!20} \textbf{Qwen2.5-14B} &\cellcolor{blue!20}  \textbf{Qwen2.5-72B} &  \cellcolor{red!20}  \textbf{Qwen2.5-7B} & \cellcolor{red!20}  \textbf{Qwen2.5-14B} &  \cellcolor{red!20} \textbf{Qwen2.5-72B} & \cellcolor{red!20} \textbf{ GPT-4o-mini} \\
        \midrule
        0.89 & 0.92 & 0.70 & 0.84 & 0.92 & 0.88 \\
        \bottomrule
    \end{tabular}
     \caption{The periodicity degree $\tau$ of different LLMs. The models represented in \colorbox{blue!20}{blue} denotes the English paraphrase generation, while\colorbox{red!20}{red} indicating Chinese paraphrasing.}
    \label{table:periodicity}
    \label{tab:model_comparison}
\end{table*}


\begin{figure*}[h]
    \centering
    \includegraphics[width=1\linewidth]{article/figures/ppl_div_mergev2.pdf}
    \caption{Convergence of perplexity, reverse perplexity, and generation diversity. The left and middle plots show that as the number of steps increases, both perplexity and reverse perplexity decrease steadily until they reach their lower bounds. The right plot shows that generation decreases as perplexity decreases.} 
    \label{figs:convergence}
\end{figure*}


\section{Results}
\label{sec:main}
%We start with our sentence-level dataset because most LLMs perform well on paraphrasing tasks at this level of granularity.
Building on the dynamical systems perspective introduced earlier, we now examine the empirical evidence that successive paraphrasing leads LLMs toward stable attractor cycles. 
We iteratively paraphrase sentences over 15 rounds within the sentence-level dataset and calculate the 2-periodicity degree.

\subsection{Periodicity}
\label{Periodicity}
We calculate the textual difference between $T_i$ and $T_{i-2}$ for paraphrases at each step. 
Arranging these differences into a confusion matrix (Figure~\ref{figs:periodicity}) reveals a pronounced 2-period cycle. 
For all LLMs, the matrix’s alternating light and dark patterns indicate that paraphrases generated at even iterations cluster together, and similarly, those at odd iterations form another cluster. 
This clear partitioning aligns with the behavior of a dynamical system converging onto a 2-period limit cycle—an attractor that draws the iterative process into a stable oscillation between two distinct states.

We also quantify this periodicity across different LLMs, as shown in Table~\ref{table:periodicity}. While all models exhibit some degree of 2-periodicity, Qwen2.5-72B shows a particularly strong and consistent cycle in both English and Chinese, whereas Llama3-70B displays relatively weaker periodic behavior. 
Models with higher periodicity tend to retain more semantic fidelity, suggesting that the recurring attractor states preserve core meaning even as they oscillate between two paraphrastic forms, as shown in Appendix~\ref{app:similarity}. 

While this periodicity can be viewed as an implicit repetition issue, it differs from explicit repetition of previously seen context. 
Instead, the model implicitly cycles through a limited set of paraphrastic forms without directly referencing prior iterations. 
In terms of systems theory, the model’s mapping function $P$ creates a dynamical environment in which the state space is not fully explored, with the trajectories settling into a 2-period attractor.


% Interestingly, even-step paraphrases are often closer to the original text than odd-step paraphrases, reinforcing the notion that the system alternates between two stable states—a hallmark of a limit cycle attractor.


% We measure the difference between paraphrases generated at different iterations from the same original sample during the process. 
% We then average the results across all samples to form an overall difference confusion matrix.
% Figure \ref{figs:periodicity} presents a selection of LLMs and their corresponding difference confusion matrices, highlighting the variation in paraphrases across successive steps during the paraphrasing process.
%Additional results are available in Appendix \ref{App:periodicity}. 

% We measure the semantic similarity between each paraphrase and its original text in our English setting. The results presented in Appendix \ref{figs:similarity} show that LLMs with higher periodicity tend to experience lower information loss. 
% Additionally, most LLMs exhibit 2-periodicity in semantic similarity.
% Notably, even-step paraphrases tend to be more similar to the original text compared to odd-step paraphrases.

% The periodicity of paraphrasing can also be viewed as a repetition problem, as seen in previous research. 
% However, unlike explicit text repetition with prior context, it represents an implicit repetition issue, where LLMs tend to reproduce the previous response without seeing the prior context.


%When we engage in successive paraphrasing, it raises the question of whether the subsequent paraphrases can resemble the original closely, or even mirror it in extreme cases. 
%To figure out this question, we calculate the BLEU score between paraphrases of different steps. 
%Specifically, a successive paraphrasing process can be represented as  $SP = s_0,s_1,s_2,\dots,s_n$, where $s_0$ represents the original text and the $s_i$ is derived from $s_{i-1}$ through paraphrasing.
%We then calculate the BLEU score between $s_i,s_j \in SP$ in a successive paraphrasing (SP) process to obtain the similarity confusion matrix. In our experimental setting, there are 800 original texts, which means 800 SP processes.
%After generating all the confusion matrices, we calculate the mean value to obtain the final similarity confusion matrix. 
%The result is shown in Figure 1.







%Figure 1 shows that for both English and Chinese settings, as the number of paraphrasing steps increases, all LLMs exhibit periodicity, which is reflected by the phenomenon that the $s_i$ is very similar to $s_{i-2}$.

%Figure 1 shows that in both English and Chinese settings, as the number of paraphrasing steps increases, all LLMs exhibit periodicity with a period of 2 in successive paraphrasing.
%This is reflected in the phenomenon where \( s_i \) is very similar to \( s_{i-2} \). 
%However, the different LLMs demonstrate varying degrees of periodicity in figures (a), (b), (c), and (d) for English, and figures (e) and (f) for Chinese. 
%To directly compare the degree of periodicity, we quantify it by measuring the distance between the actual and ideal similarity sequences.
%Ideally, \( s_i \) is equal to \( s_{i+2} \), making the similarity value between \( s_i \) and \( s_{i+2} \) equal to 1.0. 
%However, in reality, \( s_i \) is not always equal to \( s_{i+2} \). This discrepancy can be measured by subtracting the true similarity between \( s_i \) and \( s_{i+2} \) from 1.0.
%Let \( \kappa  \) be the similarity (BLEU) function. 
%We can evaluate the degree of periodicity by using the formula \( \sum_{i=3}^{n} \left(( 1.0 -\kappa(s_{i-2}, s_i))/{(n-2)} \right) \), where a lower value indicates a higher degree of periodicity, and the result is shown in table \ref{table:periodicity}.




\subsection{Convergence to Stable Attractor}
\label{sec:Convergence}

To probe the internal dynamics that lead to these attractor cycles, we explore generation determinism with successive paraphrasing unfolds.
We define a \textbf{conditioned perplexity} $\sigma(T_i \mid T_{i-1})$, reflecting the model’s confidence in generating $T_i$ given $T_{i-1}$, and a \textbf{reverse perplexity} $\hat{\sigma}(T_i \mid T_{i+1})$, indicating how easily $T_i$ could be reconstructed from $T_{i+1}$.

Figure~\ref{figs:convergence} demonstrates that as successive paraphrasing proceeds, both perplexity and reverse perplexity decrease. 
The forward direction (perplexity) quickly converges to a low boundary, while the reverse direction starts high, indicating that initially it is hard to ``go back'' from $T_{i+1}$ to $T_i$.
However, it drops fast as paraphrasing proceeds and aligns with the forward perplexity. 
Finally, the system evolves towards a state where generating $T_{i+1}$ from $T_i$ is nearly as deterministic and predictable as reconstructing $T_i$ from $T_{i+1}$.
This symmetry resembles a \textbf{stable attractor} in a dynamical system, where bidirectional predictability indicates that the system has ``locked in'' to a limit cycle.

We further quantify generation diversity by sampling multiple paraphrases at each iteration and computing the Vendi score~\cite{friedman2022vendi}. 
As shown in Figure~\ref{figs:convergence}, a low perplexity indicates a low generation diversity. 
A Vendi score of one indicates that all paraphrases in the beam are identical to each other.
%We quantify the generation diversity by applying the Vendi score~\cite{friedman2022vendi,pasarkar2023cousins} to each beam.
%The lower the Vendi score, the less diverse the generated paraphrases are, with a score of one indicating that all paraphrases in the beam are identical.
As both forward and reverse perplexity decreases, the model consistently produces similar paraphrases, leaving minimal room for alternative textual trajectories. 
From a systems viewpoint, the collapse into low perplexity and low diversity states corresponds to the model settling into the basin of attraction of a periodic orbit. 
Once inside the basin, the model’s generative behavior becomes nearly deterministic, causing the output sequence to cycle predictably.


The notion of invertibility, where each paraphrase can be treated as a paraphrase of its own paraphrase, further explains the robustness of periodicity. 
Invertibility places constraints on the mapping function $P$, effectively enabling a bidirectional relationship between states which encourages stable cycles. 
This insight suggests that tasks with similar invertible properties, e.g., translation, can also display limit cycle behavior, a hypothesis we will explore in Section~\ref{sec:beyond paraphrasing}.


% —just as a dynamical system settles into a limit cycle where each subsequent state is predetermined by the stable orbit.


%Inspired by previous research on repetitive text generation, we further investigate the inner feature of LLMs during successive paraphrasing.
%To simplify, we refer to all perplexity as conditioned perplexity and give it a function \(\sigma\).
%Thus, the perplexity of \(T_i\) conditioned on \(T_{i-1}\) can be represent as \(\sigma(T_{i}|T_{i-1})\).
%Meanwhile, we measure the perplexity of \(T_{i}\) generated by \(T_{i+1}\), and define it as reverse perplexity \(\hat{\sigma}(T_{i}|T_{i+1})\).
%To measure the generation diversity for each paraphrasing, we sample 10 sequences for each paraphrasing and utilize vendi score~\cite{pasarkar2023cousins,friedman2022vendi} to measure the diversity in the outputs.


%Additionally, we measure the perplexity of \(T_i\) generated by \(T_{i+1}\), defining it as reverse perplexity, represented by \(\hat{\sigma}(T_{i}|T_{i+1})\). 
%To assess the generation diversity of each paraphrasing, we sample 10 sequences for each paraphrase and use the Vendi score~\cite{pasarkar2023cousins,friedman2022vendi} to evaluate the diversity of the outputs.



% Inspired by previous research on repetitive text generation, we further investigate the internal features of LLMs during successive paraphrasing. To simplify, we refer to all perplexity measurements as conditioned perplexity and denote it by a function \(\sigma\). 
% Thus, the perplexity of \(T_i\) conditioned on \(T_{i-1}\) is represented as \(\sigma(T_{i}|T_{i-1})\).
% For each sample, we calculate the \(\sigma(T_{i}|T_{i-1})\) at each iteration, which results in the left plot of Figure \ref{figs:convergence}.
% Additionally, we measure the perplexity of \(T_i\) generated by \(T_{i+1}\), defining it as reverse perplexity, represented by \(\hat{\sigma}(T_{i}|T_{i+1})\). 
% Similar to perplexity, reverse perplexity reflects the probabilities of \(T_{i}\) generated by \(T_{i+1}\). 
% We apply reverse perplexity to each sample at each iteration, which results in the middle plot of Figure \ref{figs:convergence}.
% Furthermore,  we measure the generation diversity for each paraphrasing by sampling 10 paraphrases at each iteration and applying Vendi score~\cite{pasarkar2023cousins,friedman2022vendi} to assess the diversity in the outputs.
% Given a sample \(T_i\), we average the perplexity \(\sigma(T_{i+1}|T_{i})\) across the 10 paraphrases \(T_{i+1}\) and calculate the generation diversity of these 10 responses.
% The samples are then grouped into 20 buckets based on perplexity. For each bucket, we select the median perplexity and the average generation diversity to create the right-hand plot in Figure \ref{figs:convergence}.




%We pair $(T_{i-1}, T_i)$ for each same sample, calculate the average perplexity of $T_i$ conditioned on $T_{i-1}$, and visualize the evolution of perplexity across successive paraphrasing steps, shown in the left figure of Figure \ref{figs:convergence}. 
%Additionally, we measure the perplexity of \(T_i\) generated by \(T_{i+1}\), defining it as reverse perplexity, represented by \(\hat{\sigma}(T_{i}|T_{i+1})\).
%Given a text \( T_i \), we evaluate generation diversity following previous research~\cite{friedman2022vendi,pasarkar2023cousins}. 
%For each paraphrase, we compute the average perplexity of the next step and its corresponding generation diversity. The paraphrases are then grouped into 50 buckets based on perplexity. For each bucket, we select the median perplexity and the average generation diversity to create the right-hand plot in Figure \ref{figs:convergence}.

% As shown in Figure \ref{figs:convergence}, both perplexity and reverse perplexity decrease with the increasing number of paraphrase steps. Compared to reverse perplexity, the perplexity starts at a relatively low value and quickly converges to a lower boundary. In contrast, reverse perplexity begins at a much higher value, indicating the difficulty for LLMs to paraphrase \(T_1\) back to its original text. Over successive paraphrasing steps, reverse perplexity steadily decreases, eventually reaching a value close to that of perplexity.
% It reveals that as the paraphrasing progresses, the probabilities of paraphrasing \(T_i\) back to \(T_{i-1}\) continue to increase, eventually becoming comparable to the generation probability of the next step.
% Meanwhile, perplexity exhibits a positive correlation with generation diversity. 
% When the perplexity of paraphrases in the next step is low, the paraphrases tend to be more similar to each other.
% For models like Qwen2.5-72B, generation diversity can decrease to an extremely low level, representing a high similarity between next-step paraphrases.
% In fact, this reflects a limitation in the generation strategy of LLMs. Current autoregressive language models generate text token by token. 
% When the perplexity of a paraphrase approaches 1, the probabilities of each token converge towards 1, making it difficult for LLMs to sample alternative tokens simultaneously.
% This highlights the limited expressive capabilities of LLMs when the generation is in a low perplexity state. 
% The three characteristics mentioned above are referred to as convergence, and it is this convergence that leads to periodicity.

%In our opinion, the periodicity might be related to the self-enhancement nature of LLMs. 
%As the number of paraphrasing steps increases, LLMs become progressively more confident in their answers, which means their output distribution will become more focused. 
%To support this hypothesis, for an SP process, we utilize the LLMs to generate 10 paraphrases at each step and calculate the average condition perplexity of these paraphrases, which we refer to as sample space perplexity.
%We claim that a lower sample space perplexity value indicates a more centralized output distribution as explained in Appendix \ref{}.
%Figure \ref{figure:PPL}(a) illustrates that as the number of paraphrasing steps increases, the perplexity of the sample space decreases and ultimately converges to a constant value. 
%This phenomenon strongly supports our hypothesis. 



%While LLMs become more confident in their answers, their outputs will be more similar to each other.
%To measure output diversity, we employ the same sampling process used for sample space perplexity. 
%Instead of averaging conditional perplexity, we use the BLEU score to calculate the similarity between each paraphrase at each step. We then subtract the average similarity from 1.0 to represent the diversity of the outputs. 
%Figure \ref{figure:PPL}(b) shows that as the number of paraphrasing steps increases, the output diversity also decreases until it reaches a specific value.
%So, with the progress of the successive paraphrasing p rocess, the generation of LLMs becomes increasingly specific.


%In fact, lower sample space perplexity leads to a lower output diversity. 
%For all paraphrases, we can obtain their sample space perplexity and output diversity during the next-step paraphrasing.
%We divided these paraphrases into 50 buckets based on their sample space perplexity. 
%For each bucket, we calculated the average output diversity and the middle value of the sample space perplexity range.
%Figure \ref{figure:PPL}(c) illustrates that when the sample space perplexity is low, there is a positive correlation between sample space perplexity and output diversity.


%We believe these characteristics may be present in all large language models (LLMs). 
%However, these alone are insufficient to explain the periodicity of SP. 
%In fact, the inherent characteristics of SP itself also play a crucial role in its periodicity.
%If we paraphrase \(s_i\) as \(s_{i+1}\), it also makes sense to paraphrase \(s_{i+1}\) as \(s_i\).
%For LLMs, it means that the likelihood of \(s_{i}\) conditioned on \(s_{i+1}\) is not near zero.
%In an extreme case, let's assume \(s_{i+2}\) is the same as \(s_i\). 
%Therefore, while the perplexity of \(s_{i+2}\) conditioned on \(s_{i+1}\) converges after declining, the perplexity of \(s_i\) conditioned on \(s_{i+1}\) should exhibit the same trend.
%For this reason, we calculate the reverse conditional perplexity of SP at each step. 
%As shown in Figure \ref{figure:PPL}, the reverse conditional perplexity exhibits the same trend as the conditional perplexity, which strongly supports our guess.


%In our opinion, the three convergence characteristics mentioned above are the key factors contributing to the periodicity of SP. 
%We will explain how these convergence characteristics lead to periodicity in section \ref{Convergences2Periodicity}.




% \subsection{Deeper Understanding Periodicity}
\label{sec:Convergence2Periodicity}

%\textbf{Periodicity serves as an outward expression of convergence.}
The inherent characteristics of LLMs enable them to generate sequences with low perplexity, indicating a high probability of occurrence. 
During successive paraphrasing, both perplexity and reverse perplexity consistently decrease, resulting in diminished generation diversity and an increased likelihood of \(T_{i-1}\) appearing in the output space of \(T_i\). 
When the reverse perplexity approaches the perplexity, we can treat \(T_{i-1}\) and \(T_{i+1}\) as essentially equivalent. Additionally, low generation diversity suggests that the texts in \(T_{i+1}\) are highly similar to one another.
Therefore, the text in \(T_{i-1}\) may be virtually indistinguishable from the text in \(T_{i+1}\).

%In extreme cases, when \(T_{i-1}\) functions as a candidate paraphrase for \(T_i\) and the generation diversity of \(T_i\) is sufficiently low, the difference between \(T_{i-1}\) and \(T_{i+1}\) becomes minimal. 
%This is due to the reduction in reverse perplexity; as the reverse perplexity of \(T_{i-1}\) conditioned on \(T_i\) approaches that of \(T_{i+1}\) conditioned on \(T_i\), \(T_{i-1}\) qualifies as a candidate paraphrase for \(T_i\). 
%However, the intrinsic nature of paraphrasing requires LLMs to produce variations of the original text, resulting in \(T_{i+1}\) differing from \(T_i\). Combining these characteristics, LLMs perform a 2-periodicity in successive paraphrasing.




The decrease in reverse perplexity can be attributed to the invertibility of paraphrasing. 
Given a text and one of its paraphrases, we can consider the text as a paraphrase of its paraphrase, reinforcing the notion of invertibility of paraphrasing.
Building on this perspective, there are likely other tasks characterized by invertibility that exhibit similar periodicity to paraphrasing. We will extend our experiments to investigate this further in Section \ref{sec:beyond paraphrasing}.


%The nature of LLMs leads them to generate sequences with low perplexity, indicating a high likelihood at the same time.
%During successive paraphrasing, both perplexity and reverse perplexity consistently decrease, leading to a reduction in generation diversity and an increase in the likelihood of $T_{i-1}$ appearing in the output space of $T_i$.
%In an extreme situation, where the $T_{i-1}$ is a candidate paraphrase of \(T_i\) and the generation diversity of \(T_i\) is also low enough, there will be little difference between $T_{i-1}$ and $T_{i+1}$.
%Due to the decrease in reverse perplexity, when the reverse perplexity of \(T_{i-1}\) conditioned on \(T_i\) is near the perplexity of \(T_{i+1}\) conditioned on \(T_i\), T_{i-1} is in the candidate paraphrases of \(T_i\).



\iffalse

\subsection{The Mathematical Perspective}

We aim to explain that under certain conditions, the successive paraphrasing process using a LLM converges to an approximate period-2 behavior. Specifically, we incorporate the empirical findings that both forward and backward perplexities decrease as the iterative steps increase.
We represent the convergence of forward and backward perplexity as,
\[
\lim_{n\to\infty}\mathcal{L}(T_{n+1}|T_{n})=\mathcal{L}_f \geq 0    
\]
\[
\lim_{n\to\infty}\mathcal{L}(T_{n}|T_{n+1})=\mathcal{L}_b \geq 0    
\]
as both forward and backward perplexity decreases, \(P\) and \(P^{-1}\) becomes increasingly deterministic, 

Meanwhile, there exists \(\delta>0\) such that for all \(n\),
\[
d(T_{n+1},T_n) \geq \delta
\]
This ensures sufficient diversity between consecutive texts.


Under the above definitions and assumptions, the sequence \(\{T_n\}\) exhibits approximate period-2 behavior.
Specifically, there exists \(N \in Z^+\), such that for all \( n \geq N \),
\[
d(T_{n+2},d_{n}) \leq \epsilon,
\]
where \(\epsilon > 0\) is a small positive value indicating low diversity between \(T_n\) and \(T_{n+2}\). 
We will provide a brief explanation below.


\textbf{Step 1: Behavior of \(P\) and \(P^{-1}\) After Perplexity Convergence.}
Given \(P^{-1}\) represents the LLM’s mapping from a paraphrase back to the original text.
As both forward and backward perplexities decrease, the functions 
\(P\) and \(P^{-1} \) become increasingly deterministic and invertible. 
For sufficiently large \(n\), we have:
\[
P(T_n) = T_{n+1}, P^{-1}(T_{n+1}) = T_n.
\]

\textbf{Step 2: Establishing that \(P\) is an Involution.}
We will show that \(P\) is an involution on the set of texts after convergence.
Since \(P^{-1}\) is the inverse of P, and both are deterministic:
\[P^{-1}=P\]
Therefore, \(P(P(T)) = T\), for all T after convergence, which means that \(P\) is an involutive function.



\textbf{Step 3: Implications of \(P\) Being an Involution.}
Because \(P\) is an involution, its mapping consists solely of fixed points and
transpositions (cycles of length 2). 
However, the minimal dissimilarity constraint rules out fixed points (cycles of length 1), since:
\[P(T) = T \rightarrow d(T, T) = 0 < \delta\]

\textbf{Step 4: The Sequence Must Cycle Between Exactly Two Texts.}
Given that \(P\) is an involution without fixed points, the only possible cycles
are of length 2. Let \(T_A\) and \(T_B\) be the texts such that:
\[
P(T_A)=T_B, P(T_B)=T_A
\]
Thus, the sequence alternates between TA and TB:
\begin{equation*}
T_n =
\begin{cases}
    T_A, & \text{if } n \text{ is even}, \\
    T_B, & \text{if } n \text{ is odd}.
\end{cases}
\end{equation*}


\textbf{Step 5: Verifying the Minimal Dissimilarity Constraint.}
The minimal dissimilarity constraint is satisfied since:
\[d(P(Tn), Tn) = d(Tn+1, Tn) \geq \delta\]
This holds for all \(n\), as \(T_n\) and \(T_{n+1}\) are distinct due to the absence of fixed points.

\textbf{Step 6: Bounding the Distance Between \(T_{n+2}\) and \(T_n\).}
In the deterministic case, we have:
\[
T_{n+2} = P(T_{n+1}) = P(P(T_n)) = T_n.
\]
As a result:
\[
d(T_{n+2}, T_n) = d(T_n, T_n) = 0.
\]
However, in practice, residual stochasticity introduces small deviations, so we consider:
\[
d(T_{n+2}, T_n) \leq \epsilon,
\]
where \(\epsilon\) is a small positive value. This adjustment links the theoretical result to empirical observations, such as the low diversity (e.g., low normalized edit distance scores) between \(T_n\) and \(T_{n+2}\).

\textbf{Step 7: Impossibility of Longer Periods.}
Assume, for the sake of contradiction, that there exists a period \(k > 2\) with distinct texts \(T_0, T_1, \dots, T_{k-1}\) such that: \(T_{n+k} = T_n\), and \(P(T_i) = T_{i+1}\) for \(i = 0, 1, \dots, k-1 \mod k\).

Since \(P\) is deterministic and invertible (an involution), we must have:
\[
P(P(T_i)) = T_i \quad \text{for all } i.
\]
This implies:
\[
T_i = P(P(T_i)) = P(T_{i+1}) = T_{i+2}.
\]
Therefore:
\[
T_{i+2} = T_i.
\]
This means the sequence repeats every 2 steps, contradicting the assumption that the period is \(k > 2\). Thus, longer periods are impossible under the given conditions.


As we illustrate above, the only possible behavior is an approximate period-2 cycle between \(T_A\) and \(T_B\). The minimal dissimilarity constraint eliminates
fixed points, and the involutive nature of \(P\) precludes longer cycles. 
\fi

\section{Analysis}
\label{sec:Analysis}
% In this section, we will discuss the generalization of this phenomenon. 
% Considering content limitation, We put the respective confusion matrixes in Appendix \ref{app:generalization}.
In this section, we perform analytical experiments on paragraph-level paraphrase datasets to generalize our findings to longer texts.
We first demonstrate the extension of our findings to other task formats (Section~\ref{sec:beyond paraphrasing}). 
Then we go through a set of methods to try to escape from the attractor cycles in the remaining subsections.

\subsection{Beyond Paraphrase Generation}
\label{sec:beyond paraphrasing}

% \textbf{Effect of text length.} 
% A longer context can naturally be conveyed in many different ways. However, this does not hold true for LLMs.
% We extend our sentence-level experiments to the paraghraph-level.
% We utilize GPT-4o-mini to successively rephrase the text 15 times in our paragraph datasets and measure the periodicity and the differences between paraphrases.

% \textbf{Task Extension.} 

% According to our analysis in Section~\ref{sec:Convergence}, LLMs are expected to exhibit periodicity in invertible tasks. We further propose 4 tasks with the invertible characteristic, including polishing (\textbf{Pol.}), informal-to-formal style transfer (\textbf{I/F.}), clarify (\textbf{Clar.}), and translation (\textbf{Trans.}) using our paragraph dataset.
% The details of these tasks can be accessed in Appendix \ref{App:tasks}.
Our earlier results indicate that successive paraphrasing leads LLMs to settle into periodic attractors—specifically, 2-period limit cycles. According to the systems-theoretic perspective, such cycles should arise whenever the transformation is invertible, enabling a bidirectional mapping that makes prior states easily reproducible. To test this, we examine four additional invertible tasks at the paragraph level: polishing (Pol.), clarification (Clar.), informal-to-formal style transfer (I/F.), and forward/backward translation (Trans.). These tasks are defined in Appendix~\ref{App:tasks}.

\begin{figure}
\centering
\includegraphics[width=0.45\textwidth]{article/figures/app_task_extv2.pdf}
\caption{The difference confusion matrix for four tasks beyond paraphrasing. 
Note that in translations, the difference between texts in two different languages is set to one.}
\label{figs:task_extensions}
\end{figure}


% We plot the difference between \(T_i\) and \(T_{i-2}\) as the number of steps increases, and compare the periodicity of four tasks alongside paraphrasing. 
% As illustrated in Figures \ref{figs:Other_tasks_div_trend}, even at the paragraph level, LLMs tend to become trapped in specific stable states, exhibiting high 2-periodicity degrees. 
Figure~\ref{figs:task_extensions} shows that even for these varied tasks, LLMs repeatedly converge to stable states, exhibiting pronounced 2-periodicity. 
Table~\ref{table:task_extension} shows the degree of 2-periodicity across these tasks, with values ranging from 0.65 to 0.87. 
This finding reinforces the idea that invertibility fosters the emergence of limit cycles, as the model iterates the transformation and settles into an attractor. 
While paraphrasing is our primary lens, these findings confirm that stable attractor cycles are a broader characteristic of LLM behavior in iterative, invertible mappings.
% Confusion matrices of these tasks are presented in Appendix~\ref{App:tasks}. 



\begin{table}[!h]
    \centering
    \small
    \begin{tabular}{cccccc}
        \toprule
        \textbf{Tasks}& \textbf{Para.} & \textbf{Clar.} &\textbf{Pol.} & \textbf{I/F.} & \textbf{Trans.} \\
        \midrule
        \textbf{$\tau$} &0.80 & 0.83 &  0.86 & 0.65 & 0.87 \\
        \bottomrule
    \end{tabular}
    \caption{Impact of perturbations on periodicity compared to the original during paraphrasing.}
    \label{table:task_extension}
\end{table}

%\begin{figure}[!h]
%    \centering
%    \includegraphics[width=1\linewidth]{article/figures/TaskExtentionv2.pdf}
%    \caption{The periodicity of paraphrasing and other tasks on a paragraph-level English dataset using GPT-4o-mini, where a lower value indicates higher periodicity.} 
%    \label{figs:Other_tasks_Periodicity}
%\end{figure}




%
%Similar to paraphrasing, we iteratively perform these tasks on a paragraph-level dataset.
%We plot the difference between \(T_i\) and \(T_{i-2}\), and compare the 2-periodicity of four tasks %alongside paraphrasing.
%Figure \ref{figs:Other_tasks_div_trend} illustrates the decreasing trend of the four tasks, similar to that of paraphrasing, while Figure \ref{figs:Other_tasks_Periodicity} demonstrates their comparable periodicity to paraphrasing.





\subsection{Alternating Models and Prompts}
% We extend our experimental settings by introducing model and prompt variations in the process of successive paraphrasing.

% \textbf{Paraphrasing with prompt variation.}
% We design four distinct prompts that are used to rephrase text while preserving its original meaning, which can be accessed in (Appendix \ref{App:prompt_var}).
% GPT-4o-mini is then employed for successive paraphrasing, with a prompt randomly selected from the set for each iteration. 
% Although the prompt is changed during successive paraphrasing, it maintains a 2-periodicity, experiencing minimal degradation, as shown in Figure \ref{figs:model_prompt_var}.


% \textbf{Paraphrasing with model variation.}
% We selected a model set consisting of GPT-4o-mini, GPT-4o, Llama3-8B, and Qwen2.5-7B. 
% We randomly chose a different LLM from the set for each iteration. 
% Although each LLM exhibits its own paraphrasing style, contributing to the diversity of paraphrases, the 2-periodicity still persists.
% In other aspects, we calculated the perplexity of \(T_i\) conditioned on \(T_{i-1}\) using Llama3-8B, where both \(T_i\) and \(T_{i-1}\) were generated by different LLMs.
% As shown in Figure \ref{figs:PPL_conditioned_Llama3}, the perplexity of \(T_i\) decreases during successive paraphrasing, despite the fact that it was not generated by the LLM performing the calculation. 
% Both the existing convergence and 2-periodicity indicate the presence of statistically optimal patterns that exist across all LLMs.

One intuitive approach to escape an attractor is to introduce perturbations in the transformation itself. We attempt this by varying both models and prompts during successive paraphrasing. 
For \textbf{prompt variation}, we design four different paraphrasing prompts (refer to Appendix~\ref{App:prompt_var}) and randomly select one at each iteration. Despite regularly switching prompts, the 2-period cycle persists, as shown in Figure~\ref{figs:model_prompt_var}.

\begin{figure}[!hb]
    \centering
    \begin{minipage}{0.45\textwidth}
        \centering
        \includegraphics[width=\linewidth]{article/figures/app_model_prompt_var.pdf}
        \caption{The difference confusion matrices for model variation and prompt variation.}
        \label{figs:model_prompt_var}
    \end{minipage}
    \hfill
    \begin{minipage}{0.45\textwidth}
          \centering
    \includegraphics[width=1\linewidth]{article/figures/model_var_conditioned_llama3-7b.pdf}
    \caption{The perplexity of \(T_i\) conditioned on \(T_{i-1}\) calculated by Llama3-8B. Both \(T_i\) and \(T_{i-1}\) are generated by other LLMs. }    
    \label{figs:PPL_conditioned_Llama3}
    \end{minipage}
    \label{fig:overall}
\end{figure}

Similarly, we introduce \textbf{model variation} by alternating among GPT-4o-mini, GPT-4o, Llama3-8B, and Qwen2.5-7B during successive paraphrasing. 
Although each model brings its own stylistic biases, the fundamental attractor cycle remains intact. 
Interestingly, perplexity computed by a single model (e.g., Llama3-8B) on paraphrases generated by other models still decreases over iterations in Figure~\ref{figs:PPL_conditioned_Llama3}. 
This suggests that the attractor states are not confined to a single model’s parameter space.
Instead, they reflect a more general statistical optimum that multiple LLMs gravitate toward.



From a systems perspective, this findings suggest that randomizing the transformation function $P$ does not inherently break the attractor. 
The system remains in a basin of attraction shared across these varied modeling conditions, implying that the stable cycle is a robust property of the iterative transformation rather than a quirk of any particular prompt or model.


\subsection{Increasing Generation Randomness}
\label{sec:temperature}
% When LLMs become entrenched in what appears to be an optimal state, increasing the temperature has little effect in enabling them to escape this condition.
% We repeated our paraphrasing experiments using GPT-4o-mini with temperature settings of  0.6, 0.9, 1.2 and 1.5.

% As shown in Figure \ref{figs:Temperature}, while increasing the temperature from 0.6 to 1.5, The difference between paraphrases increases too.
% Nevertheless, the 2-periodicity phenomenon still remains. 
% Further elevation of the temperature leads to the generation of nonsensical text.
% In other words, LLMs cannot explore more valid expressions during paraphrasing.


\begin{figure}[!h]
    \centering
    \includegraphics[width=\linewidth]{article/figures/Temperature.pdf}
    \caption{
    The difference between \(T_{15}\) and \(T_i\) generated by GPT-4o-mini. 
    By increasing the temperature, randomness is amplified, causing the differences to grow as well.
    }    
    \label{figs:Temperature}
\end{figure}
Another strategy is introducing more stochasticity in the generation process by increasing the generation temperature. 
Higher temperatures expand the immediate token selection space, potentially allowing trajectories to wander away from the attractor. However, as shown in Figure~\ref{figs:Temperature}, while higher temperatures do increase the difference between successive paraphrases, the system still exhibits a 2-period cycle. 
Further increases in temperature lead only to nonsensical outputs.
This outcome aligns with dynamical systems theory: a small increase in stochasticity may create local perturbations, but if the basin of attraction is strong, the system remains near the limit cycle. 
Excessive stochastic forcing can push the system out of meaningful regions of state space entirely, leading to “chaotic” or nonsensical behavior, rather than discovering a new stable attractor with richer linguistic diversity.



\subsection{Experiments with Complex Prompts}
Previous experiments were conducted using a simple paraphrase prompt, leading to existing limitations. To solve this, we experimented with a more complicated prompt, and the results indicated similar periodicity patterns. This prompt forces LLMs to enhance grammatical and syntactical variety. 
We used this prompt to instruct GPT-4o-mini to successively paraphrase the paragraph-level test set for 15 rounds. The empirical evaluation of periodicity (2-periodicity score) and convergence (PPL) of the successive paraphrasing with the complex prompt is listed below. Both the difference confusion matrix and the prompt are shown in  Appendix~\ref{app:complex_prompt}.

\begin{table}[h]
\centering
\small
\begin{tabular}{ccc}
\toprule
\textbf{Model} & \textbf{Periodicity} & \textbf{Convergence} \\ \midrule
Original & 0.80 & 1.19 \\
Complex  & 0.67 & 1.33 \\ \bottomrule
\end{tabular}
\caption{Periodicity and Convergence Table}
\end{table}

Although the sophisticated prompt alleviated the periodicity and convergence in some degree, the pattern of 2-period cycle remained strong. For context, a periodicity score of 0.67 implies an average edit distance of 0.33 between paraphrases two steps apart, whereas direct paraphrase exhibits an edit distance of 0.68. 


\subsection{Incoporating Local Perturbations}


% To account for perturbations that occur in real-world usage, we simulate human interference during successive paraphrasing.
We introduce local perturbations to mitigate the attractor cycle pattern. 
At the end of each iteration, we edit 5\% of the text by introducing perturbations using three methods: synonym replacement (S.R.), word swapping (W.S.), and random insertion or deletion (I./D.).
% According to Figure \ref{figs:3-periodicity}, synonym replacement results in minimal periodicity degradation, followed by random insertion or deletion. In contrast, word swapping leads to significant degradation.
As shown in Table~\ref{table:huamn_interven}, 
among these interventions, synonym replacement barely affects periodicity, suggesting that minor lexical changes do not move the system out of the attractor’s basin. 
It indicates that except during the first paraphrasing, LLMs primarily perform synonym replacements for words or phrases, as shown in Figure \ref{figs:intro}.
Word swapping, however, causes more significant disruption, lowering periodicity more effectively. 
From a dynamical standpoint, large structural perturbations are needed to shift the system’s state out of a stable cycle. 
Local lexical tweaks do not suffice because the attractor’s pull is strong and preserved at a deeper structural level.
\begin{table}[!h]
    \centering
    \small
    \begin{tabular}{cccc}
        \toprule
            \textbf{w/o Perturb.} & \textbf{S.R.} & \textbf{W.S.} & \textbf{I./D.} \\
        \midrule
             0.77 & 0.73 &  0.62 & 0.66 \\
        % \midrule
        % Difference & - & \textbf{-0.04} & -0.15 & -0.11 \\
        \bottomrule
    \end{tabular}
    \caption{Impact of different types of perturbations on 2-periodicity degrees $\tau$, compared to the original text during paraphrasing.}
    \label{table:huamn_interven}
\end{table}

% The minimal impact of synonym replacement aligns with our observations, as shown in Figure \ref{figs:intro}.
% These paraphrases, which involve synonym replacement, demonstrate differences similar to those observed without any perturbation at odd steps.
% It indicates that LLMs primarily perform synonym replacements for words or phrases, except during the first paraphrasing, which significantly alters the text's structure and language use.
% Although random insertion and deletion may lead to information loss, it does not significantly alter the text structure compared to word swapping, which causes substantial changes.



\subsection{Paraphrasing with History Paraphrases}


% We also examine the impact of providing historical context during paraphrasing. 
We consider a scenario where the transformation $\hat{P}$ depends on both $T_i$ and $T_{i-1}$.
This added historical context can alter the equilibrium states. 
In a scenario where we paraphrase \(T_{i}\) based on the reference \(T_{i-1}\), it is essential that \(T_{i+1}\) differs from both \(T_{i}\) and \(T_{i-1}\). This function can be expressed as: $T_{i+1} = \hat{P}(T_{i}, T_{i-1})$.
In this context, \(P_{i-1}\) emerges as a strong candidate for paraphrasing \(P(T_{i+1}, T_{i})\), as it aligns with the distribution of LLMs while maintaining difference from \(\hat{P}(T_{i+1}, T_{i})\), satisfying the task requirement.
As a result, this more complex cycle still represents a stable attractor, albeit of higher order, as shown in Figure~\ref{figs:3-periodicity}.
% Consequently, the periodicity in this scenario will be three.
% We also examine the impact of providing historical context during paraphrasing.
% In a scenario where we paraphrase \(T_{i}\) based on the reference \(T_{i-1}\), it is essential that \(T_{i+1}\) differs from both \(T_{i}\) and \(T_{i-1}\).
% This added historical context can alter the equilibrium states. In fact, incorporating the immediate history leads the system to settle into a 3-period cycle rather than a 2-period cycle, as shown in Figure~\ref{figs:3-periodicity}. 
% This more complex cycle still represents a stable attractor, albeit of higher order.
% It suggests that even with more elaborate conditioning, the model is not freed from limit cycles; instead, it can become locked into more intricate periodic behaviors.



\label{sec:history}
\begin{figure}[!h]
    \centering
    \subfigure{\includegraphics[width=0.6\linewidth]{article/figures/3-periodicity.pdf}} 
    %\includegraphics[width=\linewidth]{article/figures/perturbation_periodicity.pdf}
    \caption{When adding historical paraphrases, LLMs exhibit 3-periodicity in the paraphrasing task.}    
    \label{figs:3-periodicity}
\end{figure}

% To verify this, we utilize \(\hat{P}\) during successive paraphrasing. As shown in Figure \ref{figs:3-periodicity}, it indeed performs 3-periodicity, highlighting the limited expression capabilities of LLMs again.

\subsection{Sample Selection Strategies}

\label{sec:sampling_strat}
%\begin{figure}[h]
%    \centering
%    \includegraphics[width=1\linewidth]{article/figures/sim.pdf}
%    \caption{The similarity between paraphrases and the original texts increases during the paraphrasing process.} 
%    \label{figs:sim}
%\end{figure}



% Finally, we investigate methods to steer the system away from stable attractors at the least cost of generation quality. 
% Increasing temperature alone did not help, but controlling perplexity and selecting among multiple sampled paraphrases at each iteration shows promise. 
% By choosing a paraphrase with a higher perplexity, we partially weaken the attractor’s grip, reducing periodicity. 
% Rather than always taking the one with the highest or lowest perplexity. 


% However, as shown in Figures 8 and 14, this may come at the cost of semantic fidelity.

We investigate methods to steer the system away from stable attractors at the least cost of generation quality. 
Given the correlation between periodicity and perplexity, it is intuitive to mitigate this issue by increasing perplexity while maintaining generation quality.
To achieve this, we can randomly sample multiple paraphrases at each iteration and select the one based on perplexity.
We design three types of strategies: selecting the paraphrase with the maximum or minimum perplexity or randomly choosing one at each iteration.
Figures \ref{figs:strategy} illustrate that selecting a higher perplexity can reduce periodicity.
However, such diversity comes at the cost of semantic equivalence (Appendix~\ref{app:sample_selection}).
Considering both periodicity and meaning preservation, we recommend the random strategy, which effectively reduces periodicity while incurring minimal information loss compared to selecting the option with the lowest perplexity.


\begin{figure}[h]
    \centering
    \includegraphics[width=1\linewidth]{article/figures/strategy.pdf}
    \caption{The periodicity of three strategies using different LLMs. } 
    \label{figs:strategy}
\end{figure}

\subsection{The Benefit of Mitigating Small-Size Cycles}
As a data augmentation method, paraphrase diversity should impact downstream tasks.
To figure out this, we conducted an experiment on domain classification using successive paraphrasing for data augmentation. We selected a commonly used dataset, AG News~\cite{zhang2016characterlevelconvolutionalnetworkstext} as the testbed, and trained BERT-based models using different data. For data augmentation, we conducted 5 rounds of successive paraphrasing under two different settings—min-strategy (Min.~Strat.) and max-strategy (Max.~Strat.), as detailed in Section~\ref{sec:sampling_strat}. Below are our results:

\begin{table}[h]
\centering
\small
\begin{tabular}{cccc}
\toprule
\textbf{Metric} & \textbf{w/o Aug.} & \textbf{Min. Strat.} & \textbf{ Max. Strat.} \\ \midrule
    Accuracy↑                       & 83.10        & 83.80        & 84.41        \\ 
2-periodicity↓    &       -     & 3.38          & 4.15          \\ \bottomrule
\end{tabular}
\caption{Performance on AG News across different data augmentation strategies.}
\end{table}

The table shows that our max-strategy, which more effectively mitigates the 2-periodicity cycle, yields more diverse paraphrases for data augmentation and therefore achieves higher classification accuracy (84.4\%) compared to both no augmentation and the min-strategy. We will add these findings to our revision to illustrate the beneficial impact of mitigating small-size cycles on downstream tasks.
\section{Related Work}
% \subsection{Vision Language Model}
% 시각장애인에서 상황을 설명할 DB가 없으니 만들었다. 그리고 이를 VLM에 튜닝했다.
\subsection{Technical approaches for assisting the visually-impaired}


\subsection{Datasets for visual instruction tuning}

Software development is increasingly conceived as a collaboration activity between developers and AIs. Indeed, IDEs already implement features to enable interactive development, with AI suggesting implementations that are reused by developers.

Although multiple studies show this interaction can be successful, there is still limited understanding of how the models must be configured and used in the context of code generation tasks. This study addresses this gap, systematically investigating the impact of several key parameters, including the repeated submission of a prompt to accommodate for the non-deterministic nature of the models.

Our study reveals several key findings about the usage of ChatGPT. In particular, we discovered how creativity, although up to a limited extent, is useful to increase the range of methods whose code can be generated correctly. A major role is played by parameter top-p, which is commonly underrated, and instead has a major impact on the correctness of the results, with lower values producing better results. Finally, prompts should be submitted multiple times, with $5$ repetitions combined with a temperature of $1.2$ resulting in an effective configuration in our experiments.  

Future work concerns two main research directions. One is about replicating this experiment with other AI assistants, to validate our findings in multiple contexts. The second research direction concerns finding strategies to deal with the need to submit the same prompt multiple times to obtain a useful result, and thus developing approaches able to select or merge multiple responses automatically. 


\bibliography{acl_latex}

\appendix
\label{sec:appendix}


\section{Data Statistics}
\label{app:data_stat}
We provide source information of our data in table \ref{table:dataset} and statistic information of data length in \ref{figs:data distribution}.
\begin{table*}[!h]
    \centering
    \begin{tabular}{lcccccc}
        \toprule
        \textbf{Dataset} & \textbf{TLDR} & \textbf{SQuAD} & \textbf{ROCT} & \textbf{Yelp} & \textbf{ELI5} & \textbf{Sci\_Gen} \\
        \midrule
        \textbf{Sentence/Paragraph} & 100/30 & 100/30 & 100/30 & 100/30 & 100/30 & 100/30 \\
        \midrule
        \textbf{Dataset} & \textbf{XSum} & \textbf{CMV} & \textbf{HSWAG} & \textbf{WP} & \textcolor{red}{\textbf{Wiki}} & \textcolor{red}{\textbf{WMT}} \\
        \midrule
        \textbf{Sentence/Paragraph} & 100/30 & 100/30 & 100/30 & 100/30 & 200/0 & 200/0 \\
        \bottomrule
    \end{tabular}
    \caption{Dataset Setup: Datasets marked in red indicate Chinese datasets, while others represent English datasets. The value indicates the number of extracted samples. For example, we extract 100 sentences and 30 paragraphs from the TLDR dataset.}
    \label{table:dataset}
\end{table*}
\begin{figure}[!h]

\centering
\includegraphics[width=0.45\textwidth]{article/figures/data_dist.pdf}
\caption{Statistical patterns of data length distribution.}
\label{figs:data distribution}
\end{figure}

\section{Change in similarity}
\label{app:similarity}
We measure the change in similarity between \(T_i\) and \(T_0\) across successive paraphrasing steps. The results are presented in Figure \ref{figs:similarity}.
As the number of paraphrasing steps increases, most LLMs maintain the similarity between paraphrases and their corresponding original texts, with the exception of an initial drop in similarity.
Meanwhile, it also exhibits aslight 2-periodicity in similarity.
By combining Figure \ref{figs:similarity} and Table \ref{figs:periodicity}, we found that models with higher periodicity also exhibit higher similarity.



\label{App:periodicity}
\begin{figure}[!h]
\centering
\includegraphics[width=0.5\textwidth]{article/figures/similarity.pdf}
\caption{Similarity changes during successive paraphrasing. Qwen2.5-72B is the best at preserving meaning, while all other LLMs experience slight degradation in similarity, except during the first paraphrasing step.}
\label{figs:similarity}
\end{figure}

\section{Generalization}
\label{app:generalization}
\label{App:Temperature}


\subsection{Task Extentions}
We propose four additional tasks beyond paraphrasing: polishing (\textbf{Pol.}), clarification (\textbf{Clar.}), informal-to-formal style transfer (\textbf{I/F.}), and forward/backward translation (\textbf{Trans.}).
The detailed prompts for these tasks are listed in Table \ref{tab:task_extension_prompts}. 
We perform these tasks on our paragraph dataset, calculate the textual difference of the paraphrase at each iteration with the initial text, and plot the results in Figure~\ref{figs:Other_tasks_div_trend}.
As the number of paraphrasing steps increases, the difference between \(T_i\) and \(T_{i-2}\) decreases. 
After 7 steps, there is little difference between \(T_i\) and \(T_{i-2}\).

% All of these tasks exhibit 2-periodicity, and the corresponding difference confusion matrices are shown in Figure \ref{figs:task_extensions}.




\begin{figure}[h]
    \centering
    \includegraphics[width=1\linewidth]{article/figures/TaskExtentionTrend.pdf}
    \caption{The trend in normalized edit distance between $T_i$ and $T_{i-2}$ across various tasks during the repetition process using GPT-4o-mini.}    
    \label{figs:Other_tasks_div_trend}
\end{figure}


\label{App:tasks}
\begin{table}[!h]
    \centering
    \small
    \begin{tabular}{p{0.12\linewidth}p{0.78\linewidth}}
    \toprule
        \textbf{Pol.} &  Please polish the following text: \{text\}  \\
    \midrule
        \textbf{Clar.} &  Please rewrite the following text in a way that is simpler and easier to understand, using clear language and shorter sentences without losing the original meaning: \{text\}  \\
    \midrule
        \textbf{I/F.} & Transform the following text into an informal style:  \{text\} / Rewrite the following text in a formal style:  \{text\}  \\
    \midrule
        \textbf{Trans.} & Please translate the following English text into Chinese: \{text\} / Please translate the following English text into Chinese: \{text\}  \\
    \bottomrule
    \end{tabular}
    \caption{Four types of prompts for extension tasks. The last two tasks involve switching between different languages and styles, separated by a semicolon.}
    \label{tab:task_extension_prompts}
\end{table}


\subsection{Model and Prompt variation}

\label{App:prompt_var}

We continue to modify the models and prompts during paraphrasing.
The chosen model set includes GPT-4o-mini, GPT-4o, Qwen2.5-7B, and Llama3-8B.
Four variations of the paraphrasing prompts are provided in Table \ref{tab:prompts_var}. 


\begin{table}[!h]
    \centering
    \small
    \begin{tabular}{p{0.05\linewidth}p{0.88\linewidth}}
    \toprule
        \textbf{A:} &  Please paraphrase the following text: \{text\}  \\
    \midrule
        \textbf{B:} &  Please rephrase the text below: \{text\}  \\
    \midrule
        \textbf{C:} & Please rewrite the following text:  \{text\}  \\
    \midrule
        \textbf{D:} & Please polish the text below: \{text\}  \\
    \bottomrule
    \end{tabular}
    \caption{Four variations of paraphrasing prompts. In the prompt variation experiments, a prompt is randomly selected at each step to perform the paraphrasing.}
    \label{tab:prompts_var}
\end{table}

\subsection{Experiment with A Complex Prompt}
\label{app:complex_prompt}
We conduct experiments on our paragraph-level dataset using a more complex prompt, as presented in Table~\ref{tab:more_complex_prompt}. The resulting difference confusion matrix is shown in Figure~\ref{fig:more_complex_prompt}.

\begin{figure}
    \centering
    \includegraphics[width=0.75\linewidth]{article/figures/complex.pdf}
    \caption{Difference confusion matrix for the complex prompt.}
    \label{fig:more_complex_prompt}
\end{figure}

\begin{table}[!h]
    \centering
    \small
    \begin{tabular}{p{0.95\linewidth}}
    \toprule
    Please rewrite the following paragraph with the goal of enhancing lexical and syntactical variety without changing the original meaning. Pay attention to employing diverse vocabulary, increasing the complexity and variation of sentence structures, using different conjunctions and clause constructions to make the expression more diverse and rich, while maintaining the core information and logical coherence of the original text. Specifically, avoid repetitive sentence patterns and try to express the same ideas in different ways.\\
    \bottomrule
    \end{tabular}
    \caption{A more complex prompt enhancing lexical and syntactical variety.}
    \label{tab:more_complex_prompt}
\end{table}

\subsection{Increasing Randomness}

We measure the impact of increasing randomness on periodicity by adjusting the generation temperature. We select four temperature values: 0.6, 0.9, 1.2, and 1.5. 
The results are shown in Figure \ref{figs:app_temperature}.
Although the temperature increases to a very high level, the 2-periodicity still persists.
Further increasing temperature will cause nonsense responses.

\begin{figure}[t!]
\centering
\includegraphics[width=0.45\textwidth]{article/figures/app_temperaturev2.pdf}
\caption{The difference confusion matrix for successive paraphrasing at different temperature settings, conducted by GPT-4o-mini.}
\label{figs:app_temperature}
\end{figure}


\begin{figure}[t!]
    \centering
    \begin{minipage}{0.45\textwidth}
        \centering
        \includegraphics[width=\linewidth]{article/figures/sim.pdf}
        \caption{The similarity between paraphrases and the original texts increases during the paraphrasing process.}
        \label{figs:sim}
    \end{minipage}
    \label{fig:overall}
\end{figure}

\subsection{Sample Selection Strategies}
\label{app:sample_selection}
We propose three strategies for successive paraphrasing and evaluate them across different LLMs. 
To assess the impact of these strategies on meaning preservation, we measure the similarity between the paraphrases \(T_i\) and their corresponding original texts \(T_0\) and demonstrate the result in Figure \ref{figs:sim}.
By combining Figure \ref{figs:sim} and Figure \ref{figs:strategy}, we suggest that the random strategy preserves meaning significantly better than the max strategy, while also effectively alleviating periodicity.



\section{Case Study}
We present part of the first case from our paragraph dataset for successive paraphrasing in Figure \ref{tab:case_study_1}.


%\begin{figure*}
%    \centering
%    \begin{minipage}{\textwidth}
%        \centering
%        \includegraphics[width=1\textwidth]{article/figures/app_task_ext.pdf}
%        \caption{The difference confusion matrix for four tasks beyond paraphrasing. 
%        Note that in translations, the difference between texts in two different languages is set to one.}
%        \label{figs:task_extensions}
%    \end{minipage}
%    \hfill
%    \begin{minipage}{1\textwidth}
%        \centering
%        \includegraphics[width=\linewidth]{article/figures/app_temperature.pdf}
%        \caption{The difference confusion matrix for successive paraphrasing at different temperature settings, conducted by GPT-4o-mini.}
%        \label{figs:app_temperature}
%    \end{minipage}
%    \label{fig:overall}
%\end{figure*}


%%%%%%%%%%%%%%%%%%%%% 代替换
\begin{table*}[t!]
    \setlength{\belowcaptionskip}{-0.cm}
    \centering
    \small
    \renewcommand{\arraystretch}{1.2} % Adds padding to the rows for better visualization
    \begin{tabular}{lp{0.8\linewidth}} % Adjust the 0.7\linewidth as needed
    \toprule
    \multicolumn{2}{c}{\textbf{Sample of successive paraphrasing}} \\
    \midrule
      $T_0$ &Australia has a strong tradition of an honest days pay for an honest days work. We have a strong union presence which leads to high wages backed by a strong resource economy. We also have a very multicultural society with many immigrants choosing to settle in Australia this leads to diverse shopping areas where goods originally made all over the world are made right here on our shores. The main reason Australia is the best country in the world is our heavily ingrained concept of a "fair go" meaning you give anybody a chance to prove themselves no matter who they are. This is not even mentioning our spectacular landscapes and beaches. All in all Australia is the best country in the world.  \\
    \midrule
      $ T_1$ & Australia has a robust tradition of fair compensation for hard work. The strong presence of unions contributes to high wages, supported by a solid resource-based economy. Additionally, our multicultural society, enriched by many immigrants, creates diverse shopping areas where products from around the globe are available right here in Australia. A key reason why Australia stands out as the best country in the world is our deep-rooted belief in a "fair go," which emphasizes giving everyone the opportunity to demonstrate their abilities, regardless of their background. Not to mention, our stunning landscapes and beautiful beaches add to our appeal. Overall, Australia truly is the best country in the world. \\
      \midrule
     $ T_2$ & Australia has a strong tradition of ensuring fair pay for hard work. The significant influence of unions leads to high wages, bolstered by a stable resource-driven economy. Furthermore, our multicultural society, enriched by a diverse population of immigrants, fosters vibrant shopping districts where global products are readily available. One of the main reasons Australia is considered the best country in the world is our fundamental belief in a "fair go," which highlights the importance of providing everyone with the chance to showcase their talents, regardless of their background. Additionally, our breathtaking landscapes and beautiful beaches enhance our country's appeal. Overall, Australia is truly the best place in the world.  \\
    \midrule
    $ T_3$ & Australia has a strong history of promoting fair compensation for hard work. The powerful presence of unions contributes to high wages, supported by a stable economy driven by natural resources. In addition, our multicultural society, enriched by a diverse immigrant population, creates lively shopping areas filled with international products. A key reason why Australia is often regarded as the best country in the world is our core belief in a "fair go," which emphasizes the importance of giving everyone the opportunity to demonstrate their abilities, regardless of their background. Moreover, our stunning landscapes and beautiful beaches add to the allure of our nation. Overall, Australia is undoubtedly one of the finest places in the world.  \\
    \midrule
   $T_4$ & Australia has a rich tradition of advocating for fair pay for hard work. The strong influence of unions helps maintain high wages, bolstered by a robust economy fueled by natural resources. Additionally, our multicultural society, enhanced by a varied immigrant community, fosters vibrant shopping districts filled with international goods. A significant reason Australia is often viewed as one of the best countries globally is our fundamental belief in a "fair go," which highlights the necessity of providing everyone with the chance to showcase their talents, regardless of their background. Furthermore, our breathtaking landscapes and beautiful beaches contribute to the appeal of our nation. Overall, Australia is truly one of the most exceptional places in the world.  \\
     \midrule
    $T_5$ & Australia has a rich history of promoting fair compensation for hard work. The strong presence of unions plays a crucial role in ensuring high wages, supported by a thriving economy driven by natural resources. Moreover, our diverse society, enriched by a wide-ranging immigrant population, creates lively shopping areas filled with products from around the globe. A key reason Australia is often regarded as one of the best countries in the world is our core belief in a "fair go," which emphasizes the importance of giving everyone the opportunity to demonstrate their abilities, regardless of their background. Additionally, our stunning landscapes and beautiful beaches enhance the allure of our nation. In summary, Australia is truly one of the most remarkable places in the world. \\
     \midrule
    $T_6$ & Australia has a notable history of advocating for fair pay for hard work. The significant influence of unions is essential in maintaining high wages, bolstered by a robust economy fueled by natural resources. Furthermore, our diverse society, enriched by a broad immigrant community, fosters vibrant shopping districts filled with international products. A major reason Australia is frequently seen as one of the best countries globally is our fundamental belief in a "fair go," which underscores the importance of providing everyone the chance to showcase their talents, irrespective of their background. Additionally, our breathtaking landscapes and beautiful beaches add to the appeal of our nation. In conclusion, Australia is indeed one of the most extraordinary places in the world.  \\
     \midrule
    $ T_7$ & Australia has a rich history of promoting fair compensation for hard work. The strong presence of unions plays a crucial role in sustaining high wages, supported by a thriving economy that benefits from its natural resources. Moreover, our multicultural society, enhanced by a diverse immigrant population, creates lively shopping areas filled with global products. A key reason Australia is often regarded as one of the best countries in the world is our core belief in a "fair go," which highlights the importance of giving everyone the opportunity to demonstrate their abilities, regardless of their background. Additionally, our stunning landscapes and picturesque beaches contribute to the allure of our nation. In summary, Australia is truly one of the most remarkable places on Earth.  \\
    \bottomrule
    \end{tabular}
    \caption{
    \label{tab:case_study_1}
    Case illustration of the first sample in our paragraph dataset. Due to the limited content, we just display \( T_0... T_{7}\).
    }


    
\end{table*}
\end{document}

