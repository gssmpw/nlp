%%%%%%%%%%%%%%%%%%%%%%%%%%%%%%
\section{Proofs}
\subsection{Proof of Theorem~\ref{thm:epsisymmgreedy}}
We start by proving the approximation ratio of Algorithm~\ref{algo:greedydmin}.
For a point $u \in U$ and $r>0$, let $B(u,r) = \{v \in U | \dmin(v,u) < r\}$ denote the $\dmin$-ball around $u$ with radius $r$.
We first prove the Property~\ref{prop:ballsgreedy}.
\begin{prop}
\label{prop:ballsgreedy}
If a pseudometric space $(U,d)$ is $\epsilon$-symmetric, then for any $x,y \in B(z,r)$ it holds that $\dmin(x,y) < (2+\epsilon)r$.
\end{prop}
\begin{proof}
It holds that
\begin{align*}
\dmin(x,y) &= \min\{d(x,y),d(y,x)\} \\
&\leq \min\{d(x,z)+d(z,y),d(y,z)+d(z,x)\}.
\end{align*}
If $d(x,z) \leq d(z,x)$, then 
\begin{align*}
\dmin(x,y) \leq d(x,z)+d(z,y) < r + (1+\epsilon)r = (2+\epsilon)r.
\end{align*}
Otherwise $d(x,z) > d(z,x)$, and then
\begin{align*}
\dmin(x,y) \leq d(y,z)+d(z,x) < (1+\epsilon)r + r = (2+\epsilon)r,
\end{align*}
and Property~\ref{prop:ballsgreedy} follows.
\end{proof}
Now we continue with the proof of Theorem~\ref{thm:epsisymmgreedy}.
Consider an optimum solution $O=\{o_1,\ldots,o_k\}$ of $k$ distinct points, with optimal value $\diver{O } = R^*$.
Property~\ref{prop:ballsgreedy} implies that the $k$ balls $B_i = B(o_i,\frac{R^*}{2+\epsilon})$ for each $i=1,\ldots,k$ are pairwise disjoint. Indeed, assume two balls $B_i$ and $B_j$ share a common point $u$, then because $o_i \in B(u,\frac{R^*}{2+\epsilon})$ and $o_j \in B(u,\frac{R^*}{2+\epsilon})$, Property~\ref{prop:ballsgreedy} states that $\dmin(o_i,o_j)<R^*$, contradicting the optimality of $O$.

While $|S|<k$ in Algorithm~\ref{algo:greedydmin}, there exists some ball $B_i$ that does not contain any point from $S$. Therefore, $o_i$ is available for selection and $\dmin(o_i,S) \geq \frac{R^*}{2+\epsilon}$. As Algorithm~\ref{algo:greedydmin} picks the vertex with the maximum $\dmin$ distance towards the $S$, it holds that $\diver{S}\geq \frac{R^*}{2+\epsilon}$ in each iteration. The performance guarantee follows. 

To show that this ratio is tight, consider the following example of a pseudometric on three points $\{x,y,z\}$ with $d(x,y) = d(y,x) = 1$,
$d(x,z) = d(y,z) = \frac{1}{2+\epsilon}$ and 
$d(z,x) = d(z,y) = \frac{1+\epsilon}{2+\epsilon}$. This instance is $\epsilon$-symmetric and satisfies the directed triangle inequality. For $k=2$, the optimum is $O=\{x,y\}$ with $R^*=1$, but Algorithm~\ref{algo:greedydmin} returns a set with value $\frac{1}{2+\epsilon}$ if the first point chosen is $z$.

Algorithm~\ref{algo:greedydmin} can be implemented to run in $\bigO(kn)$ time by maintaining $d(u,S)$ for each $u \in U$ in each iteration. If $v$ is selected in an iteration, then the distances are updated as $d(u,S \cup \{v\}) = \min\{d(u,S),d(u,v)\}$. There are $k$ iterations and $n$ updates in each iteration, with each update requiring constant time.


\subsection{Proof of Theorem~\ref{thm:naivemaxanti}}
Recall that $R^*$ is the optimal value for an AMMD instance space $(U,d)$ with parameter $k$, and $O$ is a corresponding optimal set of $k$ points.
There are at most $n(n-1)$ unique pairwise distances in $d$, and $R^*$ equals one of them. Algorithm~\ref{algo:app_ma} will iterate over all unique distances $R$ and creates a digraph $G_R = (U,A)$ with $ij \in E$ if and only if $d(i,j)< \frac{R}{n-k+1}$. 

Now we argue that for any $R \leq R^*$, and for any two $x, y \in O, x \neq y$, it must hold that $y$ is unreachable from $x$ in $G_R$.
Suppose that $y$ is reachable from $x$. So there is a path $p$ in $G_R$ from $x$ to $y$.
We may assume that $p$ only has vertices that are not in $O$, except for the first vertex $x$ and the last vertex $y$. Otherwise, we continue the argument by shortening $p$ to start at $x$ until encountering the first point in $O$. Using the triangle inequality along the edges in $p$, it follows that
\[
	d(x,y) < (n-k+1) \frac{R}{n-k+1} = R,
\]
as there are at most $n-k+1$ edges in $p$, and the distance between consecutive points in $p$ is strictly smaller than $\frac{R}{n-k+1}$ by definition of $G_R$. This contradicts with $d(x,y) \geq R^*$, by optimality of $x, y \in O$.

As the points in an optimal solution are pairwise unreachable in $G_R$ for $R \leq R^*$, it follows that the maximum antichain $M$ in $G_R$ will have at least $k$ vertices, and we simply select $k$ arbitrary vertices from $M$ and return them as our final solution. As two distinct vertices $i, j \in M, i \neq j$ satisfy $d(i,j) \geq \frac{R}{n-k+1}$, the approximation guarantee follows by iterating over all unique distances so that for one of the iterations we have $R=R^*$.
The running time follows from Proposition~\ref{caceres}, by using the maximum antichain algorithm from \cite{caceres2022minimum}.

\subsection{Table of running times}

\begin{table}[H] \centering
\caption{Running times for our approximation algorithms (\textbf{\algbac{}}, \textbf{\algbacb{}}, and \textbf{\algbacf{}}), the greedy algorithm (\textbf{Greedy}), and the random baseline (\textbf{Rand.}) for fixed parameter value $k=10$ on the real-world datasets from Table~\ref{tab:datasets}.}
\tcbset{colframe=black!5, colback=black!5, size=fbox, on line}
\label{tab:running_times}
\begin{tabular}{@{}lrrrrr@{}}\toprule
Data & BAC & BCR & BCF & Greedy & Rand. \\
\midrule
\emph{ft70} & 1s & 4s & 0s & 0s & 0s \\
\emph{kro124p} & 1s & 5s & 0s & 0s & 0s \\
\emph{celegans} & 0s & 0s & 0s & 0s & 0s \\
\emph{rbg323} & 0s & 1s & 0s & 0s & 0s \\
\emph{wiki-vote} & 0s & 1s & 0s & 0s & 0s \\
\emph{US airports} & 1m 44s & 7m 25s & 0s & 0s & 0s \\
\emph{moreno} & 3s & 15s & 0s & 0s & 0s \\
\emph{openflights} & 2s & 2s & 1s & 0s & 0s \\
\emph{cora} & 9s & 41s & 1s & 0s & 0s \\
\emph{bitcoin} & 51s & 5m 13s & 2s & 0s & 0s \\
\emph{gnutella} & 11s & 23s & 2s & 0s & 0s \\
\bottomrule
\end{tabular}
\end{table}