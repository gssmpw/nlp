%%%%%%%%%%%%%%%%%%%%%%%%%%%%%%%%%
\section{Related Work}
\label{sec:related}
%%%%%%%%%%%%%%%%%%%%%%%%%%%%%%%%
\ptitle{MMD} The MMD problem was shown to be \NP -complete by \citet{wang1988study}.
Both \citet{tamir1991obnoxious} and \citet{ravi1994heuristic} showed that the greedy next-furthest point algorithms yield a $\frac{1}{2}$-approximation for MMD. \citet{moumoulidou2020diverse} noted that these greedy algorithms are essentially identical to the well-known Gonzalez-heuristic \cite{gonzalez1985clustering} which gives a 2-approximation for the $k$-center problem. By a reduction from max. clique, \citet{ravi1994heuristic} showed that for any $\epsilon>0$ MMD is \NP -hard to approximate within a factor of $\frac{1}{2}+\epsilon$, thereby showing the optimality of the Gonzalez-heuristic in terms of worst-case performance.
\citet{ravi1994heuristic} also showed that MMD without the triangle inequality constraints is \NP -hard to approximate within any multiplicative factor.

\citet{akagi2018exact} showed that MMD can be solved exactly by solving $\bigO(\log n)$ $k$-clique problems, resulting in an exact algorithm in $\bigO(n^{\omega k/3}\log n)$ time by using the $k$-clique algorithm from \cite{nevsetvril1985complexity}.


\ptitle{Fairness and matroid constraints} Fairness variants of the MMD problem have recently gained a lot of research attention \cite{Addanki0MM22,celis2018fair,wang2023max,Moumoulidou0M21,wang2022streaming,wang2023fair}.
A typical fair MMD problem statement assumes the universe is partitioned into $|C|$ disjoint groups. Solutions are now required to contain a certain amount of points from each group. It is still undetermined whether constant-factor approximation algorithms exist for arbitrary $|C|$ \cite{wang2023max,Addanki0MM22}.
The fairness constraints are a special case of diversity maximization under matroid constraints \cite{abbassi2013diversity,borodin2012max,ceccarello2020general,ceccarello2018fast,cevallos2017local}.

\ptitle{Other objective functions} \citet{CHANDRA2001438} considered a range of diversity problems with different objective functions besides MMD. Among them is the popular Max-Sum Diversification (MSD) problem, which aims to maximize the sum of pairwise distances instead of the minimum \cite{kuby1987programming,abbassi2013diversity,aghamolaei2015diversity,bauckhage2020adiabatic,
borassi2019better, indyk2014composable, hassin1997approximation,bhattacharya2011consideration}. 
\citet{cevallos2017local} provided a local search $(1-\bigO(1/k))$-approximation for MSD for distances of negative type (which includes several non-metric distances), subject to general matroid constraints. \citet{zhang2020maximizing} consider a variant of MSD adapted to a clustered data setting. Besides MSD there are numerous other objective functions for diversity maximization, depending on the topic area. We refer to~\cite{zheng2017survey} for a summary of results in query result diversification.

Nonetheless, all of these works consider symmetric distances between points. We also note that the asymmetric generalization of the MSD problem is trivially approximated by considering the symmetric distance $d'$ formed by $d'(u,v) = d(u,v)+d(v,u)$ and applying any MSD approximation algorithm on $d'$. This strategy does not lead to a guarantee for AMMD, as the distance in one direction might be very small compared to the other direction.
