%%%%%%%%%%%%%%%%%%%%%%%%%%%%%%%%%%%%%%%%%
%%%%%%%%%%%%%%%%%%%%%%%%%%%%%%%%%%%%%%%%%
\section{Notation and Preliminaries}
\label{sec:notation}
%%%%%%%%%%%%%%%%%%%%%%%%%%%%%%%%%%%%%%%%%%
%%%%%%%%%%%%%%%%%%%%%%%%%%%%%%%%%%%%%%%%%%
\ptitle{Graphs}
Given a directed graph (digraph) $G=(V, A)$, vertex $v$ is reachable from vertex $u$ if there exists a path from $u$ to $v$ in $G$. Otherwise, $v$ is called unreachable from $u$. The transitive closure $\TC(G)$ of a digraph $G$ is a digraph with the same nodes as $G$, and with edges $(u,v) \in \TC(G)$ if $v$ is reachable from $u$ in $G$.
We say a digraph $G$ is transitively closed if $G=\TC(G)$.
A set $X \subseteq V$ is called independent if there are no edges in the induced subgraph $G[X]$.
An edge $(i,j) \in A$ is often abbreviated as $ij$.

\ptitle{AMMD problem}
As mentioned in the introduction, we will assume a pseudometric space $(U,d)$ without symmetry constraints on the distance function $d$, thus satisfying $d(u,u)=0$, nonnegativity $d(u,v) \geq 0$ and the directed triangle inequality $d(u,v) \leq d(u,w)+d(w,v)$ for all $u,v,w \in U$. Instances of AMMD can be viewed as complete digraphs or distance matrices representing the distances between each pair of points. An example of an AMMD instance space is the
\emph{metric closure} of a nonnegatively weighted digraph, formed by defining distances as weighted shortest path lengths between all the pairs of nodes in the graph.

We are interested in the following problem.
\begin{prob}[AMMD]
\label{prob:ammd}
Given $(U, d)$ and integer $k$, find a set $O \subseteq U$ with $\abs{O} = k$ such that $\diver{O}$, as defined in Eq.~\ref{def:divscore}, is maximized.
\end{prob}
Throughout the paper, we let $R^* = \diver{O}$ be the optimal value for an AMMD instance space $(U,d)$ with parameter $k$, where $O$ is a corresponding optimal set of $k$ points. We assume $R^* > 0$, as when $R^*=0$ any set of $k$ points would be an optimal solution.

%and the diversity score $\text{div}$ is given by Eq.~(\ref{def:divscore}).

With $d$ we define two new functions $\dmin$ and $\dmax$ as $\dmin(u,v) = \min\{d(u,v), d(v,u)\}$ and $\dmax(u,v) = \max\{d(u,v), d(v,u)\}$ for all $u,v \in U$. Both functions are clearly symmetric, and $\dmax$ satisfies the triangle inequality while $\dmin$ does not. Finally, using the $\dmin$ and $\dmax $ functions we quantify the asymmetry of a given pseudometric space. A pseudometric space $(U, d)$ is called \emph{$\epsilon$-symmetric} for some $\epsilon \geq 0$ if for all $u,v \in U: (1+\epsilon)\dmin(u,v) \geq \dmax(u,v)$.




\ptitle{Maximum antichains}
\label{sec:maxanti}
We will see that there is a connection between AMMD and the maximum independent set (MIS) problem.
Unfortunately, finding a MIS is not feasible. Instead, we will look for a maximum antichain (MA).
An \emph{antichain} of a digraph $G=(V, E)$ is a set of vertices that are pairwise unreachable in $G$. In other words, an antichain is an independent set in $\TC(G)$,
and therefore also in $G$.
We have the following problem.
\begin{prob}[MA]\label{prob:maxa}
Given a digraph $G$ and integer $k$, find an antichain of size $k$ or decide no such set exists.
\end{prob}

Unlike finding a MIS, MA can be solved in polynomial time.
A common approach is to reduce the problem into a minimum flow (with edge demands) problem~\cite{ntafos1979path,marchal2018parallel,
caceres2023minimum,pijls2013another}. Such reduction has $2|V|+ 2$ vertices and $3|V|+|E|$ edges. The minimum flow problem is then solved by reducing it to a maximum flow problem, we refer to \cite{caceres2023minimum} for the explicit construction.

This reduction has been combined with the recent breakthrough paper on maximum flows by~\citet{chen2022maximum}
to obtain the following result.
\begin{prop}[Corollary 2.1 \cite{caceres2022minimum}\label{caceres}]
Given a digraph $G=(V,E)$ with $|V|=n$ and $|E|=m$, a maximum antichain of $G$ can be computed in $\bigO(m^{1 + o(1)} \log n)$ time with high probability.
\end{prop}

There is a subtle complication in using the solver by~\citet{chen2022maximum}.
The algorithm solves the maximum flow problem with high probability, that is, there is a small $o(1)$ chance of failure.\!\footnote{By definition, this probability goes to 0 as the instance size grows.}
The issue is that in certain cases we need to solve MA several times, say $s$ times, and \emph{all} of them need to succeed. 
Here, the asymptotic probability of failure depends on how quickly the failure rate goes to 0 for a single MA as the instance size grows.
However, if we make the failure rate of a single instance $o(s^{-1})$, then the union bound immediately shows that the probability of a single failure
in $s$ instances is then in $o(1)$. The standard technique for guaranteeing an $o(s^{-1})$ failure bound is to run the solver $\bigO(\log s)$ times
and select the best result.
To summarize, we have the following proposition.
\begin{prop}\label{prop:multima}
Given $s$ digraphs $G_i = (V_i,E_i)$ with $\abs{V_i} \leq n$ and $\abs{E_i} \leq m$, computing $s$ maximum antichains of each $G_i$ can be done in $\bigO(s m^{1 + o(1)} \log n \log s)$ time with high probability.
\end{prop}

We should point out that in practice we do not use the solver by~\citet{chen2022maximum} due to its high complexity and high constants.
Instead, we settle for a more practical algorithm by~\citet{goldberg2008partial} that solves maximum flow instances in $\bigO(n^2\sqrt{m})$ time. 

We conclude this section by discussing alternative techniques for solving maximum antichains.
The first approach
requires the computation of the transitive closure of the digraph. The idea is to reduce the MA problem to finding a maximum cardinality matching in a bipartite graph encoding the reachability relation between vertices~\cite{felsner2003recognition, dilworth1987decomposition, fulkerson1956note}.
The downside of this approach is that computing the transitive closure is expensive.

Moreover, there is a large body of work on width-parametrized\footnote{The size of the maximum antichain is often called the \emph{width} of the graph.} algorithms for the MA problems~\cite{caceres2023minimum,caceres2022minimum,caceressoda,kowaluk2008path,makinen2019sparse,felsner2003recognition}. These algorithms are practical in a small-width regime~\cite{caceres2023minimum}.
However, we opted to use the maximum flow approach as this was sufficiently fast for us, while also being the asymptotically fastest algorithm.