%%
%% This is file `sample-sigconf.tex',
%% generated with the docstrip utility.
%%
%% The original source files were:
%%
%% samples.dtx  (with options: `sigconf')
%% 
%% IMPORTANT NOTICE:
%% 
%% For the copyright see the source file.
%% 
%% Any modified versions of this file must be renamed
%% with new filenames distinct from sample-sigconf.tex.
%% 
%% For distribution of the original source see the terms
%% for copying and modification in the file samples.dtx.
%% 
%% This generated file may be distributed as long as the
%% original source files, as listed above, are part of the
%% same distribution. (The sources need not necessarily be
%% in the same archive or directory.)
%%
%%
%% Commands for TeXCount
%TC:macro \cite [option:text,text]
%TC:macro \citep [option:text,text]
%TC:macro \citet [option:text,text]
%TC:envir table 0 1
%TC:envir table* 0 1
%TC:envir tabular [ignore] word
%TC:envir displaymath 0 word
%TC:envir math 0 word
%TC:envir comment 0 0
%%
%%
%% The first command in your LaTeX source must be the \documentclass command.
%\pdfminorversion=4
%\RequirePackage{hyperref}
%\documentclass[sigconf, anonymous, review]{acmart}
\documentclass[sigconf]{acmart}


%% Rights management information.  This information is sent to you
%% when you complete the rights form.  These commands have SAMPLE
%% values in them; it is your responsibility as an author to replace
%% the commands and values with those provided to you when you
%% complete the rights form.
%\setcopyright{acmcopyright}
%\copyrightyear{2018}
%\acmYear{2018}
%\acmDOI{10.1145/1122445.1122456}
%
%%% These commands are for a PROCEEDINGS abstract or paper.
%\acmConference[WWW '23]{ACM International Web Conference}{April 30-- May 04, 2023}{Austin, Texas, USA}
%%\acmBooktitle{Woodstock '18: ACM Symposium on Neural Gaze Detection,
%%  June 03--05, 2018, Woodstock, NY}
%\acmPrice{15.00}
%\acmISBN{978-1-4503-XXXX-X/18/06}


\copyrightyear{2024}
\acmYear{2024}
\setcopyright{rightsretained}
\acmConference[KDD '24]{Proceedings of the 30th ACM SIGKDD Conference on
Knowledge Discovery and Data Mining}{August 25--29, 2024}{Barcelona, Spain}
\acmBooktitle{Proceedings of the 30th ACM SIGKDD Conference on Knowledge
Discovery and Data Mining (KDD '24), August 25--29, 2024, Barcelona,
Spain}
\acmDOI{10.1145/3637528.3671757}
\acmISBN{979-8-4007-0490-1/24/08}

% The following includes the CC license icon appropriate for your paper.
% Download the image from www.scomminc.com/pp/acmsig/4ACM-CC-by-88x31.eps
% and place within your figs or figures folder

\makeatletter
\gdef\@copyrightpermission{
  \begin{minipage}{0.3\columnwidth}
   \href{https://creativecommons.org/licenses/by/4.0/}{\includegraphics[width=0.90\textwidth]{4ACM-CC-by-88x31.eps}}
  \end{minipage}\hfill
  \begin{minipage}{0.7\columnwidth}
   \href{https://creativecommons.org/licenses/by/4.0/}{This work is licensed under a Creative Commons Attribution International 4.0 License.}
  \end{minipage}
  \vspace{5pt}
}
\makeatother

\settopmatter{printacmref=true}

%%
%% Submission ID.
%% Use this when submitting an article to a sponsored event. You'll
%% receive a unique submission ID from the organizers
%% of the event, and this ID should be used as the parameter to this command.
%%\acmSubmissionID{123-A56-BU3}

%%
%% The majority of ACM publications use numbered citations and
%% references.  The command \citestyle{authoryear} switches to the
%% "author year" style.
%%
%% If you are preparing content for an event
%% sponsored by ACM SIGGRAPH, you must use the "author year" style of
%% citations and references.
%% Uncommenting
%% the next command will enable that style.
%%\citestyle{acmauthoryear}

%%
%% end of the preamble, start of the body of the document source.
\usepackage{hyperref}
\usepackage{amsmath}
\usepackage{soul}
\usepackage{graphicx}
\usepackage{algorithm}
\usepackage[noend]{algpseudocode}
\usepackage{natbib}
\usepackage{xspace}
\usepackage{subfig}
\usepackage{bbm}
\usepackage{bm}
\usepackage{centernot}
\usepackage{tcolorbox}
\usepackage{balance}

\usepackage{tikz}
\usepackage{pgfplots}

\definecolor{yafcolor1}{rgb}{0.4, 0.165, 0.553}
\definecolor{yafcolor2}{rgb}{0.949, 0.482, 0.216}
\definecolor{yafcolor3}{rgb}{0.47, 0.549, 0.306}
\definecolor{yafcolor4}{rgb}{0.925, 0.165, 0.224}
\definecolor{yafcolor5}{rgb}{0.141, 0.345, 0.643}
\definecolor{yafcolor6}{rgb}{0.965, 0.933, 0.267}
\definecolor{yafcolor7}{rgb}{0.627, 0.118, 0.165}
\definecolor{yafcolor8}{rgb}{0.878, 0.475, 0.686}

\newcommand{\tO}{\widetilde{O}}
\newcommand{\dmax}{d_{\text{max}}}
\newcommand{\dmin}{d_{\text{min}}}
\newcommand{\TC}{\text{TC}}
\def\Exp{\mathbb{E}}
\def\Var{\mathrm{Var}}
\newcommand{\simon}[1]{\textcolor{blue}{Simon: #1}}
\usepackage{algorithm}
\usepackage[noend]{algpseudocode}
\DeclareMathOperator*{\argmax}{arg\,max}
\newcommand{\Pcal}{\mathcal{P}}
\newcommand{\bigO}{\ensuremath{\mathcal{O}}\xspace}
\newcommand{\NP}{\ensuremath{\mathbf{NP}}\xspace}
\newcommand{\poly}{\ensuremath{\mathbf{P}}\xspace}
\algrenewcommand\algorithmicrequire{\textbf{Input:}}
\algrenewcommand\algorithmicensure{\textbf{Output:}}

%dashed line
\usepackage{array}
\usepackage{arydshln}
\setlength\dashlinedash{0.2pt}
\setlength\dashlinegap{1.5pt}
\setlength\arrayrulewidth{0.3pt}


\newcommand{\ptitle}[1]{\smallskip\noindent{\bf #1.}}
\newcommand{\pttitle}[1]{\smallskip\noindent{\it #1.}}
\newcommand{\parfat}[1]{{\bf #1.}}
\newcommand{\ptitlenoskip}[1]{\noindent{\bf #1.}}

\newcommand{\set}[1]{\left\{#1\right\}}
\newcommand{\pr}[1]{\left(#1\right)}
\newcommand{\fpr}[1]{\mathopen{}\left(#1\right)}
\newcommand{\spr}[1]{\left[#1\right]}
\newcommand{\fspr}[1]{\mathopen{}\left[#1\right]}
\newcommand{\brak}[1]{\left<#1\right>}
\newcommand{\abs}[1]{{\left|#1\right|}}
\newcommand{\floor}[1]{{\left\lfloor#1\right\rfloor}}
\newcommand{\ceil}[1]{{\left\lceil#1\right\rceil}}
\newcommand{\norm}[1]{\left\|#1\right\|}
\newcommand{\enset}[2]{\left\{#1 ,\ldots , #2\right\}}
\newcommand{\enpr}[2]{\pr{#1 ,\ldots , #2}}
\newcommand{\enlst}[2]{{#1} ,\ldots , {#2}}
\newcommand{\vect}[1]{\spr{#1}}
\newcommand{\envec}[2]{\vect{#1 ,\ldots , #2}}
\newcommand{\define}{\leftarrow}

\DeclareRobustCommand{\dispfunc}[2]{%
    \ensuremath{%
        \ifthenelse{\equal{#2}{}}%
            {\mathit{#1}}%
            {\mathit{#1}\fpr{#2}}}}


\newcommand{\diver}[1]{\dispfunc{div}{#1}}

\newcommand{\algclust}[1]{\dispfunc{\textsc{Clust}}{#1}}
\newcommand{\algextract}[1]{\dispfunc{\textsc{Extract}}{#1}}
\newcommand{\algbac}[1]{\dispfunc{\textsc{BAC}}{#1}}
\newcommand{\algbacb}[1]{\dispfunc{\textsc{BCR}}{#1}}
\newcommand{\algbacf}[1]{\dispfunc{\textsc{BCF}}{#1}}



\newtheorem{theorem}{Theorem}
\newtheorem{lemma}{Lemma}
\newtheorem{corollary}{Corollary}
\newtheorem{claim}{Claim}
\newtheorem{prop}{Proposition}
\newtheorem{prob}{Problem}
\newtheorem{defi}{Definition}


% PGF stuff

\pgfdeclarelayer{background}
\pgfdeclarelayer{foreground}
\pgfsetlayers{background,main,foreground}


\definecolor{yafaxiscolor}{rgb}{0.3, 0.3, 0.3}

\newlength{\yafaxispad}
\setlength{\yafaxispad}{-2pt}
\newlength{\yaftlpad}
\setlength{\yaftlpad}{\yafaxispad}
\addtolength{\yaftlpad}{-0pt}
\newlength{\yaflabelpad}
\setlength{\yaflabelpad}{-2pt}
\newlength{\yafaxiswidth}
\setlength{\yafaxiswidth}{1.2pt}
\newlength{\yafticklen}
\setlength{\yafticklen}{2pt}

\makeatletter
\def\pgfplots@drawtickgridlines@INSTALLCLIP@onorientedsurf#1{}
\makeatother

\newcommand{\yafdrawaxis}[4]{
	\pgfplotstransformcoordinatex{#1}\let\xmincoord=\pgfmathresult 
	\pgfplotstransformcoordinatex{#2}\let\xmaxcoord=\pgfmathresult 
	\pgfplotstransformcoordinatey{#3}\let\ymincoord=\pgfmathresult 
	\pgfplotstransformcoordinatey{#4}\let\ymaxcoord=\pgfmathresult 
	\pgfsetlinewidth{\yafaxiswidth} 
	\pgfsetcolor{yafaxiscolor}
	\pgfpathmoveto{\pgfpointadd{\pgfpointadd{\pgfplotspointrelaxisxy{0}{0}}{\pgfqpointxy{\xmincoord}{0}}}{\pgfqpoint{-0.5\yafaxiswidth}{\yafaxispad}}}
	\pgfpathlineto{\pgfpointadd{\pgfpointadd{\pgfplotspointrelaxisxy{0}{0}}{\pgfqpointxy{\xmaxcoord}{0}}}{\pgfqpoint{0.5\yafaxiswidth}{\yafaxispad}}}
	\pgfpathmoveto{\pgfpointadd{\pgfpointadd{\pgfplotspointrelaxisxy{0}{0}}{\pgfqpointxy{0}{\ymincoord}}}{\pgfqpoint{\yafaxispad}{-0.5\yafaxiswidth}}}
	\pgfpathlineto{\pgfpointadd{\pgfpointadd{\pgfplotspointrelaxisxy{0}{0}}{\pgfqpointxy{0}{\ymaxcoord}}}{\pgfqpoint{\yafaxispad}{0.5\yafaxiswidth}}}
	\pgfusepath{stroke}
}

\pgfplotscreateplotcyclelist{yaf}{% 
{yafcolor5,mark options={scale=0.75},mark=o}, 
{yafcolor2,mark options={scale=0.75},mark=square},
{yafcolor3,mark options={scale=0.75},mark=triangle},
{yafcolor4,mark options={scale=0.75},mark=o},
{yafcolor1,mark options={scale=0.75},mark=o},
{yafcolor8,mark options={scale=0.75},mark=o},
{yafcolor6,mark options={scale=0.75},mark=o},
{yafcolor7,mark options={scale=0.75},mark=o}} 

\pgfkeys{/pgf/number format/.cd,1000 sep={\,}}
\pgfplotsset{axis y line=left, axis x line=bottom,
	tick align=outside,
	tickwidth=\yafticklen,
	clip = false,
    x axis line style= {-, line width = 0pt, color=black!0},
    y axis line style= {-, line width = 0pt, color=black!0},
    x tick style= {line width = \yafaxiswidth, color=yafaxiscolor, yshift = \yafaxispad},
    y tick style= {line width = \yafaxiswidth, color=yafaxiscolor, xshift = \yafaxispad},
    x tick label style = {font=\small, yshift = \yaftlpad, inner xsep = 0pt},
    y tick label style = {font=\small, xshift = \yaftlpad},
    every axis y label/.style = {at = {(ticklabel cs:0.5)}, rotate=90, anchor=center, font=\small, yshift = -\yaflabelpad, inner sep = 0pt},
    every axis x label/.style = {at = {(ticklabel cs:0.5)}, anchor=center, font=\small, yshift = \yaflabelpad},
    x tick label style = {font=\small, yshift = 1pt},
    grid = major,
    major grid style  = {dash pattern = on 1pt off 3 pt},
	every axis plot post/.append style= {line width=\yafaxiswidth} ,
	legend cell align = left,
	legend style = {inner sep = 1pt, cells = {font=\scriptsize}},
	legend image code/.code={%
		\draw[mark repeat=2,mark phase=2,#1] 
		plot coordinates { (0cm,0cm) (0.15cm,0cm) (0.3cm,0cm) };% 
	} 
}



\begin{document}

%%
%% The "title" command has an optional parameter,
%% allowing the author to define a "short title" to be used in page headers.
\title{Max-Min Diversification with Asymmetric Distances}

%%
%% The "author" command and its associated commands are used to define
%% the authors and their affiliations.
%% Of note is the shared affiliation of the first two authors, and the
%% "authornote" and "authornotemark" commands
%% used to denote shared contribution to the research.

\author{Iiro Kumpulainen}
%\authornotemark[1]
\email{iiro.kumpulainen@helsinki.fi}
\affiliation{%
  \institution{University of Helsinki}
  \city{Helsinki}
  \country{Finland}
}

\author{Florian Adriaens}
%\authornote{}
\email{florian.adriaens@helsinki.fi}
\affiliation{%
  \institution{University of Helsinki, HIIT}
  \city{Helsinki}
  \country{Finland}
}


\author{Nikolaj Tatti}
%\authornotemark[1]
\email{nikolaj.tatti@helsinki.fi}
\affiliation{%
  \institution{University of Helsinki, HIIT}
  \city{Helsinki}
  \country{Finland}
}


%%
%% By default, the full list of authors will be used in the page
%% headers. Often, this list is too long, and will overlap
%% other information printed in the page headers. This command allows
%% the author to define a more concise list
%% of authors' names for this purpose.
\renewcommand{\shortauthors}{Iiro Kumpulainen, Florian Adriaens, \& Nikolaj Tatti}


\begin{abstract}  
Test time scaling is currently one of the most active research areas that shows promise after training time scaling has reached its limits.
Deep-thinking (DT) models are a class of recurrent models that can perform easy-to-hard generalization by assigning more compute to harder test samples.
However, due to their inability to determine the complexity of a test sample, DT models have to use a large amount of computation for both easy and hard test samples.
Excessive test time computation is wasteful and can cause the ``overthinking'' problem where more test time computation leads to worse results.
In this paper, we introduce a test time training method for determining the optimal amount of computation needed for each sample during test time.
We also propose Conv-LiGRU, a novel recurrent architecture for efficient and robust visual reasoning. 
Extensive experiments demonstrate that Conv-LiGRU is more stable than DT, effectively mitigates the ``overthinking'' phenomenon, and achieves superior accuracy.
\end{abstract}  

\ccsdesc[500]{Theory of computation~Approximation algorithms analysis}
\ccsdesc[500]{Mathematics of computing~Graph theory}
\keywords{Max-Min Diversification, Asymmetry, Maximum Antichain}
\maketitle

\section{Introduction}


\begin{figure}[t]
\centering
\includegraphics[width=0.6\columnwidth]{figures/evaluation_desiderata_V5.pdf}
\vspace{-0.5cm}
\caption{\systemName is a platform for conducting realistic evaluations of code LLMs, collecting human preferences of coding models with real users, real tasks, and in realistic environments, aimed at addressing the limitations of existing evaluations.
}
\label{fig:motivation}
\end{figure}

\begin{figure*}[t]
\centering
\includegraphics[width=\textwidth]{figures/system_design_v2.png}
\caption{We introduce \systemName, a VSCode extension to collect human preferences of code directly in a developer's IDE. \systemName enables developers to use code completions from various models. The system comprises a) the interface in the user's IDE which presents paired completions to users (left), b) a sampling strategy that picks model pairs to reduce latency (right, top), and c) a prompting scheme that allows diverse LLMs to perform code completions with high fidelity.
Users can select between the top completion (green box) using \texttt{tab} or the bottom completion (blue box) using \texttt{shift+tab}.}
\label{fig:overview}
\end{figure*}

As model capabilities improve, large language models (LLMs) are increasingly integrated into user environments and workflows.
For example, software developers code with AI in integrated developer environments (IDEs)~\citep{peng2023impact}, doctors rely on notes generated through ambient listening~\citep{oberst2024science}, and lawyers consider case evidence identified by electronic discovery systems~\citep{yang2024beyond}.
Increasing deployment of models in productivity tools demands evaluation that more closely reflects real-world circumstances~\citep{hutchinson2022evaluation, saxon2024benchmarks, kapoor2024ai}.
While newer benchmarks and live platforms incorporate human feedback to capture real-world usage, they almost exclusively focus on evaluating LLMs in chat conversations~\citep{zheng2023judging,dubois2023alpacafarm,chiang2024chatbot, kirk2024the}.
Model evaluation must move beyond chat-based interactions and into specialized user environments.



 

In this work, we focus on evaluating LLM-based coding assistants. 
Despite the popularity of these tools---millions of developers use Github Copilot~\citep{Copilot}---existing
evaluations of the coding capabilities of new models exhibit multiple limitations (Figure~\ref{fig:motivation}, bottom).
Traditional ML benchmarks evaluate LLM capabilities by measuring how well a model can complete static, interview-style coding tasks~\citep{chen2021evaluating,austin2021program,jain2024livecodebench, white2024livebench} and lack \emph{real users}. 
User studies recruit real users to evaluate the effectiveness of LLMs as coding assistants, but are often limited to simple programming tasks as opposed to \emph{real tasks}~\citep{vaithilingam2022expectation,ross2023programmer, mozannar2024realhumaneval}.
Recent efforts to collect human feedback such as Chatbot Arena~\citep{chiang2024chatbot} are still removed from a \emph{realistic environment}, resulting in users and data that deviate from typical software development processes.
We introduce \systemName to address these limitations (Figure~\ref{fig:motivation}, top), and we describe our three main contributions below.


\textbf{We deploy \systemName in-the-wild to collect human preferences on code.} 
\systemName is a Visual Studio Code extension, collecting preferences directly in a developer's IDE within their actual workflow (Figure~\ref{fig:overview}).
\systemName provides developers with code completions, akin to the type of support provided by Github Copilot~\citep{Copilot}. 
Over the past 3 months, \systemName has served over~\completions suggestions from 10 state-of-the-art LLMs, 
gathering \sampleCount~votes from \userCount~users.
To collect user preferences,
\systemName presents a novel interface that shows users paired code completions from two different LLMs, which are determined based on a sampling strategy that aims to 
mitigate latency while preserving coverage across model comparisons.
Additionally, we devise a prompting scheme that allows a diverse set of models to perform code completions with high fidelity.
See Section~\ref{sec:system} and Section~\ref{sec:deployment} for details about system design and deployment respectively.



\textbf{We construct a leaderboard of user preferences and find notable differences from existing static benchmarks and human preference leaderboards.}
In general, we observe that smaller models seem to overperform in static benchmarks compared to our leaderboard, while performance among larger models is mixed (Section~\ref{sec:leaderboard_calculation}).
We attribute these differences to the fact that \systemName is exposed to users and tasks that differ drastically from code evaluations in the past. 
Our data spans 103 programming languages and 24 natural languages as well as a variety of real-world applications and code structures, while static benchmarks tend to focus on a specific programming and natural language and task (e.g. coding competition problems).
Additionally, while all of \systemName interactions contain code contexts and the majority involve infilling tasks, a much smaller fraction of Chatbot Arena's coding tasks contain code context, with infilling tasks appearing even more rarely. 
We analyze our data in depth in Section~\ref{subsec:comparison}.



\textbf{We derive new insights into user preferences of code by analyzing \systemName's diverse and distinct data distribution.}
We compare user preferences across different stratifications of input data (e.g., common versus rare languages) and observe which affect observed preferences most (Section~\ref{sec:analysis}).
For example, while user preferences stay relatively consistent across various programming languages, they differ drastically between different task categories (e.g. frontend/backend versus algorithm design).
We also observe variations in user preference due to different features related to code structure 
(e.g., context length and completion patterns).
We open-source \systemName and release a curated subset of code contexts.
Altogether, our results highlight the necessity of model evaluation in realistic and domain-specific settings.





\putsec{related}{Related Work}

\noindent \textbf{Efficient Radiance Field Rendering.}
%
The introduction of Neural Radiance Fields (NeRF)~\cite{mil:sri20} has
generated significant interest in efficient 3D scene representation and
rendering for radiance fields.
%
Over the past years, there has been a large amount of research aimed at
accelerating NeRFs through algorithmic or software
optimizations~\cite{mul:eva22,fri:yu22,che:fun23,sun:sun22}, and the
development of hardware
accelerators~\cite{lee:cho23,li:li23,son:wen23,mub:kan23,fen:liu24}.
%
The state-of-the-art method, 3D Gaussian splatting~\cite{ker:kop23}, has
further fueled interest in accelerating radiance field
rendering~\cite{rad:ste24,lee:lee24,nie:stu24,lee:rho24,ham:mel24} as it
employs rasterization primitives that can be rendered much faster than NeRFs.
%
However, previous research focused on software graphics rendering on
programmable cores or building dedicated hardware accelerators. In contrast,
\name{} investigates the potential of efficient radiance field rendering while
utilizing fixed-function units in graphics hardware.
%
To our knowledge, this is the first work that assesses the performance
implications of rendering Gaussian-based radiance fields on the hardware
graphics pipeline with software and hardware optimizations.

%%%%%%%%%%%%%%%%%%%%%%%%%%%%%%%%%%%%%%%%%%%%%%%%%%%%%%%%%%%%%%%%%%%%%%%%%%
\myparagraph{Enhancing Graphics Rendering Hardware.}
%
The performance advantage of executing graphics rendering on either
programmable shader cores or fixed-function units varies depending on the
rendering methods and hardware designs.
%
Previous studies have explored the performance implication of graphics hardware
design by developing simulation infrastructures for graphics
workloads~\cite{bar:gon06,gub:aam19,tin:sax23,arn:par13}.
%
Additionally, several studies have aimed to improve the performance of
special-purpose hardware such as ray tracing units in graphics
hardware~\cite{cho:now23,liu:cha21} and proposed hardware accelerators for
graphics applications~\cite{lu:hua17,ram:gri09}.
%
In contrast to these works, which primarily evaluate traditional graphics
workloads, our work focuses on improving the performance of volume rendering
workloads, such as Gaussian splatting, which require blending a huge number of
fragments per pixel.

%%%%%%%%%%%%%%%%%%%%%%%%%%%%%%%%%%%%%%%%%%%%%%%%%%%%%%%%%%%%%%%%%%%%%%%%%%
%
In the context of multi-sample anti-aliasing, prior work proposed reducing the
amount of redundant shading by merging fragments from adjacent triangles in a
mesh at the quad granularity~\cite{fat:bou10}.
%
While both our work and quad-fragment merging (QFM)~\cite{fat:bou10} aim to
reduce operations by merging quads, our proposed technique differs from QFM in
many aspects.
%
Our method aims to blend \emph{overlapping primitives} along the depth
direction and applies to quads from any primitive. In contrast, QFM merges quad
fragments from small (e.g., pixel-sized) triangles that \emph{share} an edge
(i.e., \emph{connected}, \emph{non-overlapping} triangles).
%
As such, QFM is not applicable to the scenes consisting of a number of
unconnected transparent triangles, such as those in 3D Gaussian splatting.
%
In addition, our method computes the \emph{exact} color for each pixel by
offloading blending operations from ROPs to shader units, whereas QFM
\emph{approximates} pixel colors by using the color from one triangle when
multiple triangles are merged into a single quad.


% !TeX root = main.tex 


\newcommand{\lnote}{\textcolor[rgb]{1,0,0}{Lydia: }\textcolor[rgb]{0,0,1}}
\newcommand{\todo}{\textcolor[rgb]{1,0,0.5}{To do: }\textcolor[rgb]{0.5,0,1}}


\newcommand{\state}{S}
\newcommand{\meas}{M}
\newcommand{\out}{\mathrm{out}}
\newcommand{\piv}{\mathrm{piv}}
\newcommand{\pivotal}{\mathrm{pivotal}}
\newcommand{\isnot}{\mathrm{not}}
\newcommand{\pred}{^\mathrm{predict}}
\newcommand{\act}{^\mathrm{act}}
\newcommand{\pre}{^\mathrm{pre}}
\newcommand{\post}{^\mathrm{post}}
\newcommand{\calM}{\mathcal{M}}

\newcommand{\game}{\mathbf{V}}
\newcommand{\strategyspace}{S}
\newcommand{\payoff}[1]{V^{#1}}
\newcommand{\eff}[1]{E^{#1}}
\newcommand{\p}{\vect{p}}
\newcommand{\simplex}[1]{\Delta^{#1}}

\newcommand{\recdec}[1]{\bar{D}(\hat{Y}_{#1})}





\newcommand{\sphereone}{\calS^1}
\newcommand{\samplen}{S^n}
\newcommand{\wA}{w}%{w_{\mathfrak{a}}}
\newcommand{\Awa}{A_{\wA}}
\newcommand{\Ytil}{\widetilde{Y}}
\newcommand{\Xtil}{\widetilde{X}}
\newcommand{\wst}{w_*}
\newcommand{\wls}{\widehat{w}_{\mathrm{LS}}}
\newcommand{\dec}{^\mathrm{dec}}
\newcommand{\sub}{^\mathrm{sub}}

\newcommand{\calP}{\mathcal{P}}
\newcommand{\totspace}{\calZ}
\newcommand{\clspace}{\calX}
\newcommand{\attspace}{\calA}

\newcommand{\Ftil}{\widetilde{\calF}}

\newcommand{\totx}{Z}
\newcommand{\classx}{X}
\newcommand{\attx}{A}
\newcommand{\calL}{\mathcal{L}}



\newcommand{\defeq}{\mathrel{\mathop:}=}
\newcommand{\vect}[1]{\ensuremath{\mathbf{#1}}}
\newcommand{\mat}[1]{\ensuremath{\mathbf{#1}}}
\newcommand{\dd}{\mathrm{d}}
\newcommand{\grad}{\nabla}
\newcommand{\hess}{\nabla^2}
\newcommand{\argmin}{\mathop{\rm argmin}}
\newcommand{\argmax}{\mathop{\rm argmax}}
\newcommand{\Ind}[1]{\mathbf{1}\{#1\}}

\newcommand{\norm}[1]{\left\|{#1}\right\|}
\newcommand{\fnorm}[1]{\|{#1}\|_{\text{F}}}
\newcommand{\spnorm}[2]{\left\| {#1} \right\|_{\text{S}({#2})}}
\newcommand{\sigmin}{\sigma_{\min}}
\newcommand{\tr}{\text{tr}}
\renewcommand{\det}{\text{det}}
\newcommand{\rank}{\text{rank}}
\newcommand{\logdet}{\text{logdet}}
\newcommand{\trans}{^{\top}}
\newcommand{\poly}{\text{poly}}
\newcommand{\polylog}{\text{polylog}}
\newcommand{\st}{\text{s.t.~}}
\newcommand{\proj}{\mathcal{P}}
\newcommand{\projII}{\mathcal{P}_{\parallel}}
\newcommand{\projT}{\mathcal{P}_{\perp}}
\newcommand{\projX}{\mathcal{P}_{\mathcal{X}^\star}}
\newcommand{\inner}[1]{\langle #1 \rangle}

\renewcommand{\Pr}{\mathbb{P}}
\newcommand{\Z}{\mathbb{Z}}
\newcommand{\N}{\mathbb{N}}
\newcommand{\R}{\mathbb{R}}
\newcommand{\E}{\mathbb{E}}
\newcommand{\F}{\mathcal{F}}
\newcommand{\var}{\mathrm{var}}
\newcommand{\cov}{\mathrm{cov}}


\newcommand{\calN}{\mathcal{N}}

\newcommand{\jccomment}{\textcolor[rgb]{1,0,0}{C: }\textcolor[rgb]{1,0,1}}
\newcommand{\fracpar}[2]{\frac{\partial #1}{\partial  #2}}

\newcommand{\A}{\mathcal{A}}
\newcommand{\B}{\mat{B}}
%\newcommand{\C}{\mat{C}}

\newcommand{\I}{\mat{I}}
\newcommand{\M}{\mat{M}}
\newcommand{\D}{\mat{D}}
%\newcommand{\U}{\mat{U}}
\newcommand{\V}{\mat{V}}
\newcommand{\W}{\mat{W}}
\newcommand{\X}{\mat{X}}
\newcommand{\Y}{\mat{Y}}
\newcommand{\mSigma}{\mat{\Sigma}}
\newcommand{\mLambda}{\mat{\Lambda}}
\newcommand{\e}{\vect{e}}
\newcommand{\g}{\vect{g}}
\renewcommand{\u}{\vect{u}}
\newcommand{\w}{\vect{w}}
\newcommand{\x}{\vect{x}}
\newcommand{\y}{\vect{y}}
\newcommand{\z}{\vect{z}}
\newcommand{\fI}{\mathfrak{I}}
\newcommand{\fS}{\mathfrak{S}}
\newcommand{\fE}{\mathfrak{E}}
\newcommand{\fF}{\mathfrak{F}}

\newcommand{\Risk}{\mathcal{R}}

\renewcommand{\L}{\mathcal{L}}
\renewcommand{\H}{\mathcal{H}}

\newcommand{\cn}{\kappa}
\newcommand{\nn}{\nonumber}


\newcommand{\Hess}{\nabla^2}
\newcommand{\tlO}{\tilde{O}}
\newcommand{\tlOmega}{\tilde{\Omega}}

\newcommand{\calF}{\mathcal{F}}
\newcommand{\fhat}{\widehat{f}}
\newcommand{\calS}{\mathcal{S}}

\newcommand{\calX}{\mathcal{X}}
\newcommand{\calY}{\mathcal{Y}}
\newcommand{\calD}{\mathcal{D}}
\newcommand{\calZ}{\mathcal{Z}}
\newcommand{\calA}{\mathcal{A}}
\newcommand{\fbayes}{f^B}
\newcommand{\func}{f^U}


\newcommand{\bayscore}{\text{calibrated Bayes score}}
\newcommand{\bayrisk}{\text{calibrated Bayes risk}}

\newtheorem{example}{Example}[section]
\newtheorem{exc}{Exercise}[section]
%\newtheorem{rem}{Remark}[section]

\newtheorem{theorem}{Theorem}[section]
\newtheorem{definition}{Definition}
\newtheorem{proposition}[theorem]{Proposition}
\newtheorem{corollary}[theorem]{Corollary}

\newtheorem{remark}{Remark}[section]
\newtheorem{lemma}[theorem]{Lemma}
\newtheorem{claim}[theorem]{Claim}
\newtheorem{fact}[theorem]{Fact}
\newtheorem{assumption}{Assumption}

\newcommand{\iidsim}{\overset{\mathrm{i.i.d.}}{\sim}}
\newcommand{\unifsim}{\overset{\mathrm{unif}}{\sim}}
\newcommand{\sign}{\mathrm{sign}}
\newcommand{\wbar}{\overline{w}}
\newcommand{\what}{\widehat{w}}
\newcommand{\KL}{\mathrm{KL}}
\newcommand{\Bern}{\mathrm{Bernoulli}}
\newcommand{\ihat}{\widehat{i}}
\newcommand{\Dwst}{\calD^{w_*}}
\newcommand{\fls}{\widehat{f}_{n}}


\newcommand{\brpi}{\pi^{br}}
\newcommand{\brtheta}{\theta^{br}}

% \newcommand{\M}{\mat{M}}
% \newcommand\Mmh{\mat{M}^{-1/2}}
% \newcommand{\A}{\mat{A}}
% \newcommand{\B}{\mat{B}}
% \newcommand{\C}{\mat{C}}
% \newcommand{\Et}[1][t]{\mat{E_{#1}}}
% \newcommand{\Etp}{\Et[t+1]}
% \newcommand{\Errt}[1][t]{\mat{\bigtriangleup_{#1}}}
% \newcommand\cnM{\kappa}
% \newcommand{\cn}[1]{\kappa\left(#1\right)}
% \newcommand\X{\mat{X}}
% \newcommand\fstar{f_*}
% \newcommand\Xt[1][t]{\mat{X_{#1}}}
% \newcommand\ut[1][t]{{u_{#1}}}
% \newcommand\Xtinv{\inv{\Xt}}
% \newcommand\Xtp{\mat{X_{t+1}}}
% \newcommand\Xtpinv{\inv{\left(\mat{X_{t+1}}\right)}}
% \newcommand\U{\mat{U}}
% \newcommand\UTr{\trans{\mat{U}}}
% \newcommand{\Ut}[1][t]{\mat{U_{#1}}}
% \newcommand{\Utinv}{\inv{\Ut}}
% \newcommand{\UtTr}[1][t]{\trans{\mat{U_{#1}}}}
% \newcommand\Utp{\mat{U_{t+1}}}
% \newcommand\UtpTr{\trans{\mat{U}_{t+1}}}
% \newcommand\Utptild{\mat{\widetilde{U}_{t+1}}}
% \newcommand\Us{\mat{U^*}}
% \newcommand\UsTr{\trans{\mat{U^*}}}
% \newcommand{\Sigs}{\mat{\Sigma}}
% \newcommand{\Sigsmh}{\Sigs^{-1/2}}
% \newcommand{\eye}{\mat{I}}
% \newcommand{\twonormbound}{\left(4+\DPhi{\M}{\Xt[0]}\right)\twonorm{\M}}
% \newcommand{\lamj}{\lambda_j}

% \renewcommand\u{\vect{u}}
% \newcommand\uTr{\trans{\vect{u}}}
% \renewcommand\v{\vect{v}}
% \newcommand\vTr{\trans{\vect{v}}}
% \newcommand\w{\vect{w}}
% \newcommand\wTr{\trans{\vect{w}}}
% \newcommand\wperp{\vect{w}_{\perp}}
% \newcommand\wperpTr{\trans{\vect{w}_{\perp}}}
% \newcommand\wj{\vect{w_j}}
% \newcommand\vj{\vect{v_j}}
% \newcommand\wjTr{\trans{\vect{w_j}}}
% \newcommand\vjTr{\trans{\vect{v_j}}}

% \newcommand{\DPhi}[2]{\ensuremath{D_{\Phi}\left(#1,#2\right)}}
% \newcommand\matmult{{\omega}}

%%%%%%%%%%%%%%%%%%%%%%%%%%%%%%
\section{Nearly symmetric instances}
\label{sec:algosnearlysym}
%%%%%%%%%%%%%%%%%%%%%%%%%%%%%
Greedily selecting the next-furthest points until $k$ points are selected is a $\frac{1}{2}$-approximation for symmetric MMD \cite{tamir1991obnoxious,ravi1994heuristic}, but can perform arbitrarily badly
on asymmetric instances as shown by the example in Figure~\ref{fig:toyexample_greedy}.

Theorem~\ref{thm:epsisymmgreedy} generalizes this result to asymmetric instances.
It states that on asymmetric instances that are $\epsilon$-symmetric (see Section~\ref{sec:notation}), the greedy approach applied on the $\dmin$ distances yields a $\frac{1}{2+\epsilon}$-approximation, and this ratio is tight.
The proof of Theorem~\ref{thm:epsisymmgreedy} can be found in the Appendix.

\begin{theorem}
\label{thm:epsisymmgreedy}
For any $\epsilon \geq 0$, Algorithm~\ref{algo:greedydmin} is a $\frac{1}{2+\epsilon}$-approximation on $\epsilon$-symmetric instances and can be implemented to run in $\bigO(kn)$ time.
Additionally, there exist $\epsilon$-symmetric instances for which Algorithm~\ref{algo:greedydmin} cannot achieve a performance ratio better than $\frac{1}{2+\epsilon}$.  
\end{theorem}

\begin{algorithm}[t]
\caption{Greedy with $\dmin$-distances.}
\label{algo:greedydmin}
\begin{algorithmic}[1]
%\Require graph $G$. 
\Require space $(U, d)$ and integer parameter $k\geq 2$.
\State $v \define \text{arbitrary vertex from } U$.
\State $S \define \{v\}$.
\While {$|S|<k$}
	\State $v \define \text{arg} \max_{u \in U} \dmin (u,S)$.
	\State $S \define S \cup \{v\}$.
\EndWhile
\Ensure the set $S$. 
\end{algorithmic}
\end{algorithm}



%%%%%%%%%%%%%%%%%%%%%%%%%%%%%%
\section{Ball-and-antichain method}
\label{sec:approx}
%%%%%%%%%%%%%%%%%%%%%%%%%%%%%
Section~\ref{sec:approxnk2} details a straightforward approximation algorithm for AMMD, exploiting the polynomial time complexity of the MA problem in digraphs, as discussed in the previous section.
This algorithm has an approximation guarantee of $\frac{1}{n-k+1}$, which is not very useful in a typical regime of small $k$, but it gives insight into the use of the MA problem for approximating AMMD.

In Section~\ref{sec:approx16k} we modify the algorithm from Section~\ref{sec:approxnk2} by first clustering the points based on the $\dmax$ distances between them. The subspace induced by the cluster centers has some very useful properties, which leads to an approximation algorithm with a multiplicative guarantee of $\frac{1}{6k}$ for AMMD.

Finally, in Section~\ref{sec:speeding}, we discuss improvements to look for better solutions and speed up the algorithm while maintaining the approximation guarantee.

\begin{algorithm}[t]
\caption{Naive Maximum Antichain method.}
\label{algo:app_ma}
\begin{algorithmic}[1]
%\Require graph $G$. 
\Require space $(U, d)$ and integer parameter $k\geq 2$.
\ForAll {$R \in \{d(i,j)>0 \mid i, j \in U,  i \neq j\}$}
	\State Create $G_R = (U,A)$, with $ij \in A \Leftrightarrow d(i,j)<\frac{R}{n-k+1}$. \label{line:naive_create_G}
	\State $M \define$ \text{Maximum antichain of} $G_R$.
	\If{$|M| \geq k$,} $S_R \leftarrow \text{any }k \text{ points from } M$.

	\EndIf
\EndFor
\Ensure the set $S_R$ with the largest $\diver{S_R}$ value. 
\end{algorithmic}
\end{algorithm}

%%%%%%%%%%%%%%%%%%%%%%%%%%%%%%
\subsection{Naive approach based on antichains}
\label{sec:approxnk2}
%%%%%%%%%%%%%%%%%%%%%%%%%%%%%

We begin by describing the naive approach for approximating AMMD.
Assume for the moment that we know $R = R^*$, the optimal value for an AMMD instance.
Consider a digraph $G = (U, A)$, where $ij \in A$ if and only if $d(i, j) < R$.
Then an independent set, say $O$, in $G$ of size $k$ will have $\diver{O} = R^*$.
Unfortunately, finding a maximum independent set in a graph is an \NP-hard problem with a weak approximation guarantee~\cite{hastad1996clique}.

Therefore, we lower the cutoff by setting it to $\frac{R}{n-k+1}$. This makes the underlying graph
so sparse that we can guarantee that the graph contains an antichain, say $S$, of size $k$.
Since an antichain is also an independent set, we know that $\diver{S} \geq \frac{R}{n-k+1}$.
We can find the antichain in polynomial time. Finally, we do not know $R^*$ but we know that it
is one of the distances. Therefore, we test every distance; there are at most $n(n - 1)$ of such distances.
The pseudo-code for the algorithm is given in Algorithm~\ref{algo:app_ma}.


\begin{theorem}
\label{thm:naivemaxanti}
Algorithm~\ref{algo:app_ma} is an $\frac{1}{n-k+1}$-approximation for AMMD in $\bigO(n^{4 + o(1)} \log n)$ time with high probability.
\end{theorem}

The proof is given in Appendix.

We finalize this section by observing that we could binary search for the largest $R$ value for which $G_R$ (defined in line~\ref{line:naive_create_G} in Algorithm~\ref{algo:app_ma}) still has an antichain of size $k$. First, sort the unique distances in time $\bigO(n^2 \log n)$, after which we need at most $\bigO(\log n)$ calls to find this $R$. Since for this $R$ we have $R \geq R^*$, we retain the same approximation guarantee. The binary search performs $\bigO(\log n)$ MA computations, and all of them need to succeed. Proposition~\ref{prop:multima} implies that the algorithm solves the problem in
$\bigO(n^{2 + o(1)}\log^2 n \log \log n)$ time with high probability.

%%%%%%%%%%%%%%%%%%%%%%%%%%%%%%
%%%%%%%%%%%%%%%%%%%%%%%%%%%%%%
\subsection{Refined approach: clustering and antichains}
\label{sec:approx16k}
%%%%%%%%%%%%%%%%%%%%%%%%%%%%%%
%%%%%%%%%%%%%%%%%%%%%%%%%%%%%%


The problem with Algorithm~\ref{algo:app_ma} is that it is using a very conservative cutoff of $\frac{R}{n-k+1}$, leading
to a weak guarantee. 
We show that we can relax this cutoff to $\frac{R}{6k}$, by first clustering the space $(U,d)$ according to the $\dmax$ distances.
The discovered cluster centers will have the property that two centers must have a large $\dmax$ distance between them.
A consequence of this property is that the resulting graph contains a large antichain, a large chordless cycle, or a large shortest path with no backward edges.
It turns out that we can search for all 3 subgraphs in polynomial time, and using those we can extract an independent set in polynomial time.

%%%%%%%%%%%%%%%%%%%%%%%%%%%%%%
%%%%%%%%%%%%%%%%%%%%%%%%%%%%%
\ptitle{Clustering step}
Next, we describe the clustering step, as given in
Algorithm~\ref{algo:ballcover}. Here the algorithm greedily covers $U$ with a family of pairwise disjoint sets $\set{A_t}$. 
Each $A_t$ is constructed by selecting an unmarked point as a center $c_t$, and adding all unmarked points $v$ with $\dmax(c_t, v) < R$. Since all the vertices in $A_t$ are marked at the end of step $t$ (line~5), they cannot be selected by any $A_{t'}$ for $t' > t$. It follows that for every $t \neq t'$ we have $A_t \cap A_{t'} = \emptyset$. 

Algorithm~\ref{algo:ballcover} terminates in at most $n$ steps, since every set $A_t$ contains at least one point, namely the center point $c_t$. Algorithm~\ref{algo:ballcover} runs in $\bigO(n^2)$ time.


Each $A_t$ is constructed by selecting vertices that have a small $\dmax$ distance to their center $c_t$. Line~4 in Algorithm~\ref{algo:ballcover} ensures that the $\dmax$ distances between two distinct centers are at least $R$. 

To analyze this further, let us define $R'$ as the smallest distance that is at least one third of the optimum $R^*$,
\begin{equation}
\label{def:R'}
R' = \min\{d(u,v) \mid d(u,v) \geq R^*/3, \ u,v \in U, \ u\neq v\}.
\end{equation}
If we then perform the clustering for any $R \leq R'$, there will be $k$ centers that also have a pairwise $\dmin$ distance of at least $R^*/3$ between them. This is captured by Proposition~\ref{prop:cluster} and Corollary~\ref{cor:clusterphase}.

\begin{algorithm}[t]
\caption{$\algclust{U, d, R}$, clusters $U$ according to $\dmax$.}\label{algo:ballcover} 
\begin{algorithmic}[1]
%\Require graph $G$. 
\Require space $(U, d)$ and parameter $R > 0$.
\State Label all $u \in U$ as unmarked, let ${U}' \define \emptyset$ and  $t \leftarrow 1$.
\While {there exists an unmarked point}
	\State $c_t \define $ any unmarked point.
	%\State $c_t \define $ unmarked point $\text{arg} \max_{c} \dmin (c,{U}')$
	\State $A_t \define \{\text{unmarked } v \in U \mid \dmax(c_t,v) < R \}.$ 
	\State Mark all $v \in A_t$. 
	\State ${U}' \define {U}' \cup \{c_t\}$ and $t \define t+1$.
\EndWhile
\Ensure ${U}'.$
\end{algorithmic}
\end{algorithm}

\begin{prop}
\label{prop:cluster}
Let $O$ be an optimal solution to AMMD with optimum $R^*$ and $R'$ as defined in Equation~\ref{def:R'}.
If $R \leq R'$, then the following two statements regarding Algorithm~\ref{algo:ballcover} are true.
\begin{itemize}
\item For all $t$ it holds that $|A_t \cap O| \leq 1$.
\item For any $t\neq t'$ for which $|A_t \cap O|=1$ and $|A_{t'} \cap O|=1$, it holds that $\dmin(c_t,c_{t'}) \geq R' \geq R^*/3$.
\end{itemize}
\end{prop}
\begin{proof}
For the first statement, suppose that $A_t$ contains two distinct $x, y \in O, x \neq y$.
Then it holds that $d(x,y) \leq d(x,c_t)+d(c_t,y)$. Since $d(x,c_t)$ and $d(c_t,y)$ are strictly less than $R\leq R'$, they must be less than $R^*/3$ because $R'$ is defined as the smallest distance greater or equal to $R^*/3$, so any distance strictly smaller than $R'$ must be less than $R^*/3$. Thus, $d(x,y) < 2R^*/3 < R^*$, a contradiction since $x, y \in O, x \neq y$ implies that $d(x,y) \geq R^*$.

For the second statement, assume that $A_t$ contains $x \in O$ and $A_{t'}$ contains $y \in O$. Since $A_t$ and $A_{t'}$ are disjoint, we have $x \neq y$. Note that $d(x,y) \leq d(x,c_t)+d(c_t,c_{t'})+d(c_{t'},y)$, where $d(x,c_t)<R \leq R'$ and $d(c_{t'},y)<R \leq R'$. This means $d(x,c_t)$ and $d(c_{t'},y)$ must be less than $R^*/3$ by the definition of $R'$. So if $d(c_t,c_{t'}) < R'$, we would have $d(x,y) < 3 R^*/3 = R^*$, a contradiction. Similarly, we cannot have $d(c_{t'},c_t) < R'$ either, which means $\dmin(c_t,c_{t'}) \geq R' \geq R^*/3$.
\end{proof}

\begin{corollary}
\label{cor:clusterphase}
Let ${U}' = \algclust{U, d, R}$.
For every $u, v \in {U}', u \neq v$ we have $\dmax(u,v) \geq R$. Additionally, if $R \leq R'$, then ${U}'$ contains a set $S$, for which $|S| = k$ and $\diver{S} \geq R' \geq R^*/3$.

\end{corollary}
 
Corollary~\ref{cor:clusterphase} states that as long as $R \leq R'$, from any instance space $(U, d)$ we can efficiently find a subset ${U}' \subseteq U$ 
such that ${U}'$ still contains $k$ points with a pairwise $\dmin$ distance of at least $R^*/3$ between them. This enables us to restrict ourselves to ${U}'$, at the expense of a decrease in the optimal value by a factor of three.


%%%%%%%%%%%%%%%%%%%%%%%%%%%%%%
\ptitle{The \algbac{} algorithm}
We are ready to describe our algorithm which we call \algbac{} (shortened for ball-and-antichain). The pseudocode is given in Algorithms~\ref{algo:bac}--\ref{algo:extract}.
Similar to the naive approach we iterate over all distances. For each candidate distance $R$,
we cluster the space to get the centers $U'$.
We then construct a graph $G$ with edges corresponding to distances shorter than $\frac{R}{2k}$.
We can guarantee that there is ($i$) a large chordless cycle, ($ii$) a long shortest path with no backward edges,
or ($iii$) a large antichain. In the first two cases, we can obtain an independent set by selecting $k$ vertices with odd indices.
In the last case, it is enough to select $k$ vertices from the found antichain.

\begin{algorithm}[t]
\caption{\algbac{U, d, k}, an $\frac{1}{6k}$-approx. algorithm for AMMD.}\label{algo:bac} 
\begin{algorithmic}[1]
\Require space $(U, d)$ and integer parameter $k\geq 2$.
\ForAll {$R \in \{d(i,j)>0 \mid i, j \in U,  i \neq j\}$} \label{line:loop}
	\State ${U}' \define \algclust{U, d, R}$. 
	\State $\algextract{U', d, R/(2k), k}$.
\EndFor 

\Ensure the set returned by $\algextract{}$ with the largest $\diver{}$ value.
\end{algorithmic}
\end{algorithm}

\begin{algorithm}[t]
\caption{\algextract{U, d, \sigma, k}, subroutine for extracting a candidate set.}\label{algo:extract} 
\begin{algorithmic}[1]
\Require space $(U, d)$, threshold $\sigma$, and integer parameter $k\geq 2$.
    \State Create $G = (U,A)$, with $ij \in A \Leftrightarrow d(i,j)< \sigma$.\label{line:create_G}
    \If{$G$ contains a cycle}
        \State $C \define$ chordless cycle in $G$. \label{line:cycle}
    \EndIf
    \State $G_c \define$ the condensation of $G$.
    \State $M \define$ maximum antichain of $G_c$.
    \State $L \define$ shortest path of length $2k-1$ in $G_c$ or longest found. \label{line:path}
    \If{$C$ exists and $|C| \geq 2|M|-1$ and $|C| \geq |L|$}
        \State $I \define$ points with odd indices from $C$.
    \ElsIf{$2|M|-1 \geq |L|$} 
        \State $I \define$ points in $G$ corresponding to points in $M$.
    \Else{} 
        \State $I \define$ points in $G$ corr. to points with odd indices in $L$.
    \EndIf
\Ensure greedily selected $k$ points from the set $I$, if found.
\end{algorithmic}
\end{algorithm}

Next, we will prove the approximation guarantee. 
First, we need Lemma~\ref{lem:cycles}, which states that there cannot exist small cycles in $G$.

\begin{lemma}
\label{lem:cycles}
For any $R > 0$, any cycle $C$ in the digraph $G$ constructed in $\algextract{U', d, \frac{R}{2k}, k}$ (see Alg.~\ref{algo:extract})
has at least $2k + 2$ distinct vertices.
\end{lemma}
\begin{proof}
Suppose $G$ contains a cycle $C = (v_1,\ldots,v_{\ell},v_{1})$ of length $\ell$. 
Since $C$ is a cycle in $G$, it holds that $d(v_{\ell},v_1) < \frac{R}{2k}$, by definition of $G$.
On the other hand, as ${U}'$ is the output of \algclust{}, Corollary~\ref{cor:clusterphase} states that $\dmax(v_1,v_{\ell}) \geq R$. This implies that $d(v_1,v_{\ell}) \geq R$. Now the triangle inequality for $d(v_1,v_{\ell})$ along the edges of cycle $C$ implies
\[
	R \leq d(v_1,v_{\ell}) \leq \sum_{i = 1}^{\ell - 1} d(v_i, v_{i + 1}) < (\ell - 1) \frac{R}{2k}. 
\]
Solving for $\ell$ leads to $\ell > 2k + 1$, which proves the claim.
\end{proof}

\begin{theorem}
\label{thm:approx}
\algbac{U, d, k} is an $\frac{1}{6k}$-approximation to AMMD. 
\end{theorem}

\begin{proof}
Line~\ref{line:loop} in Algorithm~\ref{algo:bac} iterates over all unique distances, and one of them is equal to $R'$ as defined in Eq.~\ref{def:R'}.
We will show that for $R \leq R'$, the digraph $G$, constructed in line~\ref{line:create_G} in Algorithm~\ref{algo:extract}, has an antichain $M$ of size $|M| \geq k$, or there exists either a shortest path
with no backward edges or a chordless cycle from which we can select $k$ independent vertices. An independent set $I$ of size $\abs{I} \geq k$ in graph $G$ then yields a solution with a diversity score of $\diver{I} \geq \frac{R}{2k}$, which for $R = R'$ is $\diver{I} \geq\frac{R'}{2k} \geq \frac{R^*}{6k}$ proving the theorem. 

Note that since the nodes in an antichain have no paths connecting them, it suffices to look for an antichain in the condensation $G_c$, whose vertices are the strongly connected components of $G$. If there exists an antichain $M$ of size $|M| \geq k$, we are done as the nodes in an antichain are independent. 

Consider the case where the maximum antichain $M$ has size $|M|<k$ and assume $G$ is a DAG. Note that when $G$ is a DAG the condensation $G_c$ is equivalent to $G$.

Corollary~\ref{cor:clusterphase} states that if $R \leq R'$ then ${U}'$ contains a subset
$S \subseteq {U}'$ for which $|S| = k$ and $\diver{S} \geq R'$. Then there must be a path in $G$ between some pair of distinct points in $S$. Otherwise, $S$ is an antichain of size $k$.

Let this pair of nodes be $x, y \in S, x \neq y$, with a path $(x = v_1, \ldots, v_\ell = y)$ from $x$ to $y$ in $G$.
Then the triangle inequality implies
\[
	R' \leq \diver{S} \leq d(x, y) \leq \sum_{i=1}^{\ell-1} d(v_i, v_{i + 1}) < (\ell - 1) \frac{R}{2k} \leq (\ell - 1) \frac{R'}{2k}.
\]
Therefore, $\ell$ must be at least $2k + 2$ meaning any path between $x$ and $y$ must have at least $2k + 2$ vertices. Hence, there is a shortest path of length $2k + 2$ while a shortest path $L$ of length $2k - 1$ is sufficient.
There cannot be any shortcut edges in $L$, since $L$ is a shortest path. Nor can there be any backward edges in $L$, since $G$ is a DAG.
Consequently, elements in $L$ with odd indices form an independent set $I$ of size $k$.

Finally, assume that $G$ is not a DAG. Then
there is a chordless cycle $C$. Lemma~\ref{lem:cycles} guarantees that $C$ has at least $2k + 2$ elements.
Then, elements in $C$ with odd indices form an independent set $I$ of size $\abs{I} \geq k+1 > k$.
\end{proof}


%%%%%%%%%%%%%%%%%%%%%%%%%%%%%%
%%%%%%%%%%%%%%%%%%%%%%%%%%%%%%
\ptitle{Time complexity of Algorithm~\ref{algo:bac}}
The iteration in line~\ref{line:loop} is over at most $\bigO(n^2)$ possible $R$
values.
%The digraph $G$ might be dense, so we assume its number of edges is $\bigO(n^2)$.
Both detecting a cycle in $G$ and extracting the chordless
cycle from it (line~\ref{line:cycle} of Algorithm~\ref{algo:extract}) take $\bigO(n^2)$ time. Computing the
maximum antichain can be done in $\bigO(n^{2 + o(1)} \log n)$ time
(Proposition~\ref{caceres}).

To compute the shortest path we can use the following approach:
Let $D = A + I$, where $A$ is the adjacency matrix of $G$,
and $I$ is the identity matrix. Then $D^\ell_{ij} > 0$ if and only if
there is a path of at most length $\ell$ from $i$ to $j$.
Consequently, there is a shortest path from $i$ to $j$ of length $2k - 1$ if and only if
$D^{2k-1}_{ij} > 0$ and $D^{2k-2}_{ij} = 0$. We can compute the necessary matrices
in $\bigO(n^{\omega}\log k)$ time, where $\omega < 2.373$ is the matrix multiplication exponent~\cite{alman2021refined}. Once $i$ is found, we use Dijkstra's algorithm to recover the path in $\bigO(n^2)$ time.

Overall we have a worst-case time complexity of $\bigO(n^{2+\omega} \log k)$.

Note that in practice, we do not use the matrix multiplication method. Instead,
we compute a shortest path tree from every node. This leads to a slower theoretical time but the algorithms
are still practical as demonstrated in the experiments.



%%%%%%%%%%%%%%%%%%%%%%%%%%%%%%%%%%%%%%%
\subsection{Practical improvements}
\label{sec:speeding}
%%%%%%%%%%%%%%%%%%%%%%%%%%%%%%%%%%%%%%%
We discuss several modifications, which speed up \algbac{}, and/or might improve the solution quality in practice. 

\ptitle{Algorithm \algbacb{}} The first modification to \algbac{}, given in Algorithm~\ref{algo:bacb}, is
aimed at improving solution quality, at the expense of a slightly larger
running time. We will call this modified algorithm \algbacb{}. Note that the $\frac{1}{6k}$-approximation guarantee comes from the fact
that we add an edge $ij$ to $G$ whenever $d(i, j) < \frac{R}{2k}$. If we can increase
this cutoff to, say $R \times \alpha$, \emph{and} still find a feasible set, then the found set $S$ 
is guaranteed to have $\diver{S} \geq R\alpha$, that is we will obtain an $\alpha$-approximation.


\begin{algorithm}[t]
\caption{\algbacb{U, d, k}, an $\frac{1}{6k}$-approx. algorithm for AMMD.}\label{algo:bacb} 
\begin{algorithmic}[1]
\Require space $(U, d)$ and integer parameter $k\geq 2$.
\State $R_1 < \ldots < R_{m} \define $ all unique positive distances sorted.
\For{every $i = 1, \ldots, m$} 
	\State ${U}' \define \algclust{U, d, R_i}$. 
	\If {$\algextract{U', d, R_i /(2k), k}$ exists}
		\State $a \define \min\set{s \mid R_i / (2k) < R_s}$, $b \define i$.
		\While {$a \leq b$}
			\State $t \define \floor{\frac{a+b}{2}}$.
			\If {$\algextract{U', d, R_t, k}$ exists}
				$a \define t + 1$
			\Else{}
				$b \define t - 1$
			\EndIf
		\EndWhile
	\EndIf
\EndFor 

\Ensure the set returned by $\algextract{}$ with the largest $\diver{}$ value.
\end{algorithmic}
\end{algorithm}


For every $R$ in the iteration of \algbacb{}
that gives a feasible solution for the threshold $\frac{R}{2k}$, we will try to
improve the solution value by searching for a cutoff value larger than
$\frac{R}{2k}$ when constructing the graph $G$ (line~\ref{line:create_G} of Algorithm~\ref{algo:extract}).
If \algbacb{} is unable to find a feasible solution for a
certain $R$, then we continue
iterating to the next $R$. 

To this end, if \algbacb{} has found a feasible set for some $R$, we use the binary search
to search for larger cutoffs in $[\frac{R}{2k},R]$.
Note that only when we use the cutoff $
\frac{R}{6k}$ are we theoretically guaranteed that we can extract $k$
independent points. Nonetheless, \algbacb{} will attempt to do this for larger
cutoffs as well. The binary search requires $\bigO(\log n)$ tests for
a single $R$ since we can assume that the cutoff is one of the distances. Hence,
the computational complexity of \algbacb{} is in $\bigO(n^{2+\omega} \log k \log
n)$.


\ptitle{Algorithm \algbacf{}} This algorithm speeds up \algbac{} while at the same time attempting to improve solution quality. Similarly to Section~\ref{sec:approxnk2}, we can replace the loop of Algorithm~\ref{algo:bac} (line~\ref{line:loop}) with a binary search in an attempt to find a maximal $R$ for which we find a feasible solution. This reduces the iterations from $\bigO(n^2)$ to $\bigO(\log n)$.

\begin{algorithm}[t]
\caption{\algbacf{U, d, k}, an $\frac{1}{6k}$-approx. algorithm for AMMD.}\label{algo:bacf} 
\begin{algorithmic}[1]
\Require space $(U, d)$ and integer parameter $k\geq 2$.
\State $R_1 < \ldots < R_{m} \define $ all unique positive distances sorted.
\State $a \define 1$, $b \define m$
\While {$a \leq b$}
    \State $t \define \floor{\frac{a+b}{2}}$.
	\State ${U}' \define \algclust{U, d, R_t}$. 
	\If {$\algextract{U', d, R_t / (2k), k}$ exists}
		$a \define t + 1$
	\Else{}
		$b \define t - 1$
	\EndIf
\EndWhile
\State $i \define a-1$.
\State ${U}' \define \algclust{U, d, R_i}$. 

\State $a \define \min\set{s \mid R_i / (2k) < R_s}$, $b \define i$.
\While {$a \leq b$}
    \State $t \define \floor{\frac{a+b}{2}}$.
	\If {$\algextract{U', d, R_t, k}$ exists}
		$a \define t + 1$
	\Else{}
		$b \define t - 1$
	\EndIf
\EndWhile

\Ensure the set returned by $\algextract{}$ with the largest $\diver{}$ value.
\end{algorithmic}
\end{algorithm}


Note that unlike in Section~\ref{sec:approxnk2}, this binary search might not find the globally largest $R$ value for which we can extract a feasible solution.
This is because there may be distance values $R_i > R_j > R'$ such that $R_i$ yields a feasible solution while $R_j$ does not.

However, we can still sort the
unique distances and use binary search to find
$R_{j}$ with a feasible solution, say $S_j$, such that the next value
$R_{j+1} > R_{j}$ does not yield a feasible solution.

Moreover, any $R \leq R'$ yields a feasible solution. This implies $R_{j} \geq R'$ and we get the same guarantee as \algbac{}, because
\[
	\diver{S_{j}} \geq \frac{R_j}{2k} \geq \frac{R'}{2k} \geq \frac{R^*}{6k}.
\]

Similarly to \algbacb{}, \algbacf{} then attempts to improve the cutoff
value for constructing $G$ by again binary searching for an improved cutoff
in the interval $[\frac{R_j}{2k},R_j]$ for which \algextract{} still finds a feasible
solution.
In the worst case, this adds
another $\bigO(\log n)$ iterations, which does not change the asymptotic
running time of the algorithm.
The running time for solving $\bigO(\log n)$ MA instances, as given by Proposition~\ref{prop:multima},
is dominated by the time needed to check for shortest paths of length $2k-1$.
In summary, the running time of \algbacf{} is
\[
	\bigO(n^2 \log n + 2 n^{\omega} \log n \log k) = \bigO(n^{\omega} \log n \log k).
\]

This is a considerable improvement over the running time of
\algbac{}, making \algbacf{} orders of magnitude faster while retaining the same theoretical guarantees.


\ptitle{Further improvements}
As the graph $G$ constructed in Algorithm~\ref{algo:extract} may be split into multiple disconnected components, we improve the search for independent sets by looking for the cycles, antichains, and long shortest paths in each weakly connected component of $G$ separately. We then take the union of the independent sets for each component, aiming to have $k$ points in total.

In addition, rather than choosing the centers arbitrarily in Algorithm~\ref{algo:ballcover}, we start by picking one of the vertices with the largest $\dmin$ distance. This heuristic is similar to the approach for solving MMD~\citep{ravi1994heuristic}. For the subsequent iterations, we choose the point furthest from the current set of chosen centers, as in Algorithm~\ref{algo:greedydmin}.

Finally, in practice many of the unique distances $R_1, \ldots, R_m$ may result in the same clustering ${U}'$ in Algorithms~\ref{algo:bac} and~\ref{algo:bacb}. We avoid these duplicate computations by grouping the $R$ values that yield the same clustering.

\section{Experiments}
\label{sec:experiments}
The experiments are designed to address two key research questions.
First, \textbf{RQ1} evaluates whether the average $L_2$-norm of the counterfactual perturbation vectors ($\overline{||\perturb||}$) decreases as the model overfits the data, thereby providing further empirical validation for our hypothesis.
Second, \textbf{RQ2} evaluates the ability of the proposed counterfactual regularized loss, as defined in (\ref{eq:regularized_loss2}), to mitigate overfitting when compared to existing regularization techniques.

% The experiments are designed to address three key research questions. First, \textbf{RQ1} investigates whether the mean perturbation vector norm decreases as the model overfits the data, aiming to further validate our intuition. Second, \textbf{RQ2} explores whether the mean perturbation vector norm can be effectively leveraged as a regularization term during training, offering insights into its potential role in mitigating overfitting. Finally, \textbf{RQ3} examines whether our counterfactual regularizer enables the model to achieve superior performance compared to existing regularization methods, thus highlighting its practical advantage.

\subsection{Experimental Setup}
\textbf{\textit{Datasets, Models, and Tasks.}}
The experiments are conducted on three datasets: \textit{Water Potability}~\cite{kadiwal2020waterpotability}, \textit{Phomene}~\cite{phomene}, and \textit{CIFAR-10}~\cite{krizhevsky2009learning}. For \textit{Water Potability} and \textit{Phomene}, we randomly select $80\%$ of the samples for the training set, and the remaining $20\%$ for the test set, \textit{CIFAR-10} comes already split. Furthermore, we consider the following models: Logistic Regression, Multi-Layer Perceptron (MLP) with 100 and 30 neurons on each hidden layer, and PreactResNet-18~\cite{he2016cvecvv} as a Convolutional Neural Network (CNN) architecture.
We focus on binary classification tasks and leave the extension to multiclass scenarios for future work. However, for datasets that are inherently multiclass, we transform the problem into a binary classification task by selecting two classes, aligning with our assumption.

\smallskip
\noindent\textbf{\textit{Evaluation Measures.}} To characterize the degree of overfitting, we use the test loss, as it serves as a reliable indicator of the model's generalization capability to unseen data. Additionally, we evaluate the predictive performance of each model using the test accuracy.

\smallskip
\noindent\textbf{\textit{Baselines.}} We compare CF-Reg with the following regularization techniques: L1 (``Lasso''), L2 (``Ridge''), and Dropout.

\smallskip
\noindent\textbf{\textit{Configurations.}}
For each model, we adopt specific configurations as follows.
\begin{itemize}
\item \textit{Logistic Regression:} To induce overfitting in the model, we artificially increase the dimensionality of the data beyond the number of training samples by applying a polynomial feature expansion. This approach ensures that the model has enough capacity to overfit the training data, allowing us to analyze the impact of our counterfactual regularizer. The degree of the polynomial is chosen as the smallest degree that makes the number of features greater than the number of data.
\item \textit{Neural Networks (MLP and CNN):} To take advantage of the closed-form solution for computing the optimal perturbation vector as defined in (\ref{eq:opt-delta}), we use a local linear approximation of the neural network models. Hence, given an instance $\inst_i$, we consider the (optimal) counterfactual not with respect to $\model$ but with respect to:
\begin{equation}
\label{eq:taylor}
    \model^{lin}(\inst) = \model(\inst_i) + \nabla_{\inst}\model(\inst_i)(\inst - \inst_i),
\end{equation}
where $\model^{lin}$ represents the first-order Taylor approximation of $\model$ at $\inst_i$.
Note that this step is unnecessary for Logistic Regression, as it is inherently a linear model.
\end{itemize}

\smallskip
\noindent \textbf{\textit{Implementation Details.}} We run all experiments on a machine equipped with an AMD Ryzen 9 7900 12-Core Processor and an NVIDIA GeForce RTX 4090 GPU. Our implementation is based on the PyTorch Lightning framework. We use stochastic gradient descent as the optimizer with a learning rate of $\eta = 0.001$ and no weight decay. We use a batch size of $128$. The training and test steps are conducted for $6000$ epochs on the \textit{Water Potability} and \textit{Phoneme} datasets, while for the \textit{CIFAR-10} dataset, they are performed for $200$ epochs.
Finally, the contribution $w_i^{\varepsilon}$ of each training point $\inst_i$ is uniformly set as $w_i^{\varepsilon} = 1~\forall i\in \{1,\ldots,m\}$.

The source code implementation for our experiments is available at the following GitHub repository: \url{https://anonymous.4open.science/r/COCE-80B4/README.md} 

\subsection{RQ1: Counterfactual Perturbation vs. Overfitting}
To address \textbf{RQ1}, we analyze the relationship between the test loss and the average $L_2$-norm of the counterfactual perturbation vectors ($\overline{||\perturb||}$) over training epochs.

In particular, Figure~\ref{fig:delta_loss_epochs} depicts the evolution of $\overline{||\perturb||}$ alongside the test loss for an MLP trained \textit{without} regularization on the \textit{Water Potability} dataset. 
\begin{figure}[ht]
    \centering
    \includegraphics[width=0.85\linewidth]{img/delta_loss_epochs.png}
    \caption{The average counterfactual perturbation vector $\overline{||\perturb||}$ (left $y$-axis) and the cross-entropy test loss (right $y$-axis) over training epochs ($x$-axis) for an MLP trained on the \textit{Water Potability} dataset \textit{without} regularization.}
    \label{fig:delta_loss_epochs}
\end{figure}

The plot shows a clear trend as the model starts to overfit the data (evidenced by an increase in test loss). 
Notably, $\overline{||\perturb||}$ begins to decrease, which aligns with the hypothesis that the average distance to the optimal counterfactual example gets smaller as the model's decision boundary becomes increasingly adherent to the training data.

It is worth noting that this trend is heavily influenced by the choice of the counterfactual generator model. In particular, the relationship between $\overline{||\perturb||}$ and the degree of overfitting may become even more pronounced when leveraging more accurate counterfactual generators. However, these models often come at the cost of higher computational complexity, and their exploration is left to future work.

Nonetheless, we expect that $\overline{||\perturb||}$ will eventually stabilize at a plateau, as the average $L_2$-norm of the optimal counterfactual perturbations cannot vanish to zero.

% Additionally, the choice of employing the score-based counterfactual explanation framework to generate counterfactuals was driven to promote computational efficiency.

% Future enhancements to the framework may involve adopting models capable of generating more precise counterfactuals. While such approaches may yield to performance improvements, they are likely to come at the cost of increased computational complexity.


\subsection{RQ2: Counterfactual Regularization Performance}
To answer \textbf{RQ2}, we evaluate the effectiveness of the proposed counterfactual regularization (CF-Reg) by comparing its performance against existing baselines: unregularized training loss (No-Reg), L1 regularization (L1-Reg), L2 regularization (L2-Reg), and Dropout.
Specifically, for each model and dataset combination, Table~\ref{tab:regularization_comparison} presents the mean value and standard deviation of test accuracy achieved by each method across 5 random initialization. 

The table illustrates that our regularization technique consistently delivers better results than existing methods across all evaluated scenarios, except for one case -- i.e., Logistic Regression on the \textit{Phomene} dataset. 
However, this setting exhibits an unusual pattern, as the highest model accuracy is achieved without any regularization. Even in this case, CF-Reg still surpasses other regularization baselines.

From the results above, we derive the following key insights. First, CF-Reg proves to be effective across various model types, ranging from simple linear models (Logistic Regression) to deep architectures like MLPs and CNNs, and across diverse datasets, including both tabular and image data. 
Second, CF-Reg's strong performance on the \textit{Water} dataset with Logistic Regression suggests that its benefits may be more pronounced when applied to simpler models. However, the unexpected outcome on the \textit{Phoneme} dataset calls for further investigation into this phenomenon.


\begin{table*}[h!]
    \centering
    \caption{Mean value and standard deviation of test accuracy across 5 random initializations for different model, dataset, and regularization method. The best results are highlighted in \textbf{bold}.}
    \label{tab:regularization_comparison}
    \begin{tabular}{|c|c|c|c|c|c|c|}
        \hline
        \textbf{Model} & \textbf{Dataset} & \textbf{No-Reg} & \textbf{L1-Reg} & \textbf{L2-Reg} & \textbf{Dropout} & \textbf{CF-Reg (ours)} \\ \hline
        Logistic Regression   & \textit{Water}   & $0.6595 \pm 0.0038$   & $0.6729 \pm 0.0056$   & $0.6756 \pm 0.0046$  & N/A    & $\mathbf{0.6918 \pm 0.0036}$                     \\ \hline
        MLP   & \textit{Water}   & $0.6756 \pm 0.0042$   & $0.6790 \pm 0.0058$   & $0.6790 \pm 0.0023$  & $0.6750 \pm 0.0036$    & $\mathbf{0.6802 \pm 0.0046}$                    \\ \hline
%        MLP   & \textit{Adult}   & $0.8404 \pm 0.0010$   & $\mathbf{0.8495 \pm 0.0007}$   & $0.8489 \pm 0.0014$  & $\mathbf{0.8495 \pm 0.0016}$     & $0.8449 \pm 0.0019$                    \\ \hline
        Logistic Regression   & \textit{Phomene}   & $\mathbf{0.8148 \pm 0.0020}$   & $0.8041 \pm 0.0028$   & $0.7835 \pm 0.0176$  & N/A    & $0.8098 \pm 0.0055$                     \\ \hline
        MLP   & \textit{Phomene}   & $0.8677 \pm 0.0033$   & $0.8374 \pm 0.0080$   & $0.8673 \pm 0.0045$  & $0.8672 \pm 0.0042$     & $\mathbf{0.8718 \pm 0.0040}$                    \\ \hline
        CNN   & \textit{CIFAR-10} & $0.6670 \pm 0.0233$   & $0.6229 \pm 0.0850$   & $0.7348 \pm 0.0365$   & N/A    & $\mathbf{0.7427 \pm 0.0571}$                     \\ \hline
    \end{tabular}
\end{table*}

\begin{table*}[htb!]
    \centering
    \caption{Hyperparameter configurations utilized for the generation of Table \ref{tab:regularization_comparison}. For our regularization the hyperparameters are reported as $\mathbf{\alpha/\beta}$.}
    \label{tab:performance_parameters}
    \begin{tabular}{|c|c|c|c|c|c|c|}
        \hline
        \textbf{Model} & \textbf{Dataset} & \textbf{No-Reg} & \textbf{L1-Reg} & \textbf{L2-Reg} & \textbf{Dropout} & \textbf{CF-Reg (ours)} \\ \hline
        Logistic Regression   & \textit{Water}   & N/A   & $0.0093$   & $0.6927$  & N/A    & $0.3791/1.0355$                     \\ \hline
        MLP   & \textit{Water}   & N/A   & $0.0007$   & $0.0022$  & $0.0002$    & $0.2567/1.9775$                    \\ \hline
        Logistic Regression   &
        \textit{Phomene}   & N/A   & $0.0097$   & $0.7979$  & N/A    & $0.0571/1.8516$                     \\ \hline
        MLP   & \textit{Phomene}   & N/A   & $0.0007$   & $4.24\cdot10^{-5}$  & $0.0015$    & $0.0516/2.2700$                    \\ \hline
       % MLP   & \textit{Adult}   & N/A   & $0.0018$   & $0.0018$  & $0.0601$     & $0.0764/2.2068$                    \\ \hline
        CNN   & \textit{CIFAR-10} & N/A   & $0.0050$   & $0.0864$ & N/A    & $0.3018/
        2.1502$                     \\ \hline
    \end{tabular}
\end{table*}

\begin{table*}[htb!]
    \centering
    \caption{Mean value and standard deviation of training time across 5 different runs. The reported time (in seconds) corresponds to the generation of each entry in Table \ref{tab:regularization_comparison}. Times are }
    \label{tab:times}
    \begin{tabular}{|c|c|c|c|c|c|c|}
        \hline
        \textbf{Model} & \textbf{Dataset} & \textbf{No-Reg} & \textbf{L1-Reg} & \textbf{L2-Reg} & \textbf{Dropout} & \textbf{CF-Reg (ours)} \\ \hline
        Logistic Regression   & \textit{Water}   & $222.98 \pm 1.07$   & $239.94 \pm 2.59$   & $241.60 \pm 1.88$  & N/A    & $251.50 \pm 1.93$                     \\ \hline
        MLP   & \textit{Water}   & $225.71 \pm 3.85$   & $250.13 \pm 4.44$   & $255.78 \pm 2.38$  & $237.83 \pm 3.45$    & $266.48 \pm 3.46$                    \\ \hline
        Logistic Regression   & \textit{Phomene}   & $266.39 \pm 0.82$ & $367.52 \pm 6.85$   & $361.69 \pm 4.04$  & N/A   & $310.48 \pm 0.76$                    \\ \hline
        MLP   &
        \textit{Phomene} & $335.62 \pm 1.77$   & $390.86 \pm 2.11$   & $393.96 \pm 1.95$ & $363.51 \pm 5.07$    & $403.14 \pm 1.92$                     \\ \hline
       % MLP   & \textit{Adult}   & N/A   & $0.0018$   & $0.0018$  & $0.0601$     & $0.0764/2.2068$                    \\ \hline
        CNN   & \textit{CIFAR-10} & $370.09 \pm 0.18$   & $395.71 \pm 0.55$   & $401.38 \pm 0.16$ & N/A    & $1287.8 \pm 0.26$                     \\ \hline
    \end{tabular}
\end{table*}

\subsection{Feasibility of our Method}
A crucial requirement for any regularization technique is that it should impose minimal impact on the overall training process.
In this respect, CF-Reg introduces an overhead that depends on the time required to find the optimal counterfactual example for each training instance. 
As such, the more sophisticated the counterfactual generator model probed during training the higher would be the time required. However, a more advanced counterfactual generator might provide a more effective regularization. We discuss this trade-off in more details in Section~\ref{sec:discussion}.

Table~\ref{tab:times} presents the average training time ($\pm$ standard deviation) for each model and dataset combination listed in Table~\ref{tab:regularization_comparison}.
We can observe that the higher accuracy achieved by CF-Reg using the score-based counterfactual generator comes with only minimal overhead. However, when applied to deep neural networks with many hidden layers, such as \textit{PreactResNet-18}, the forward derivative computation required for the linearization of the network introduces a more noticeable computational cost, explaining the longer training times in the table.

\subsection{Hyperparameter Sensitivity Analysis}
The proposed counterfactual regularization technique relies on two key hyperparameters: $\alpha$ and $\beta$. The former is intrinsic to the loss formulation defined in (\ref{eq:cf-train}), while the latter is closely tied to the choice of the score-based counterfactual explanation method used.

Figure~\ref{fig:test_alpha_beta} illustrates how the test accuracy of an MLP trained on the \textit{Water Potability} dataset changes for different combinations of $\alpha$ and $\beta$.

\begin{figure}[ht]
    \centering
    \includegraphics[width=0.85\linewidth]{img/test_acc_alpha_beta.png}
    \caption{The test accuracy of an MLP trained on the \textit{Water Potability} dataset, evaluated while varying the weight of our counterfactual regularizer ($\alpha$) for different values of $\beta$.}
    \label{fig:test_alpha_beta}
\end{figure}

We observe that, for a fixed $\beta$, increasing the weight of our counterfactual regularizer ($\alpha$) can slightly improve test accuracy until a sudden drop is noticed for $\alpha > 0.1$.
This behavior was expected, as the impact of our penalty, like any regularization term, can be disruptive if not properly controlled.

Moreover, this finding further demonstrates that our regularization method, CF-Reg, is inherently data-driven. Therefore, it requires specific fine-tuning based on the combination of the model and dataset at hand.
\vspace{-0.2cm}
\section{Impact: Why Free Scientific Knowledge?}
\vspace{-0.1cm}

Historically, making knowledge widely available has driven transformative progress. Gutenberg’s printing press broke medieval monopolies on information, increasing literacy and contributing to the Renaissance and Scientific Revolution. In today's world, open source projects such as GNU/Linux and Wikipedia show that freely accessible and modifiable knowledge fosters innovation while ensuring creators are credited through copyleft licenses. These examples highlight a key idea: \textit{access to essential knowledge supports overall advancement.} 

This aligns with the arguments made by Prabhakaran et al. \cite{humanrightsbasedapproachresponsible}, who specifically highlight the \textbf{ human right to participate in scientific advancement} as enshrined in the Universal Declaration of Human Rights. They emphasize that this right underscores the importance of \textit{ equal access to the benefits of scientific progress for all}, a principle directly supported by our proposal for Knowledge Units. The UN Special Rapporteur on Cultural Rights further reinforces this, advocating for the expansion of copyright exceptions to broaden access to scientific knowledge as a crucial component of the right to science and culture \cite{scienceright}. 

However, current intellectual property regimes often create ``patently unfair" barriers to this knowledge, preventing innovation and access, especially in areas critical to human rights, as Hale compellingly argues \cite{patentlyunfair}. Finding a solution requires carefully balancing the imperative of open access with the legitimate rights of authors. As Austin and Ginsburg remind us, authors' rights are also human rights, necessitating robust protection \cite{authorhumanrights}. Shareable knowledge entities like Knowledge Units offer a potential mechanism to achieve this delicate balance in the scientific domain, enabling wider dissemination of research findings while respecting authors' fundamental rights.

\vspace{-0.2cm}
\subsection{Impact Across Sectors}

\textbf{Researchers:} Collaboration across different fields becomes easier when knowledge is shared openly. For instance, combining machine learning with biology or applying quantum principles to cryptography can lead to important breakthroughs. Removing copyright restrictions allows researchers to freely use data and methods, speeding up discoveries while respecting original contributions.

\textbf{Practitioners:} Professionals, especially in healthcare, benefit from immediate access to the latest research. Quick access to newer insights on the effectiveness of drugs, and alternative treatments speeds up adoption and awareness, potentially saving lives. Additionally, open knowledge helps developing countries gain access to health innovations.

\textbf{Education:} Education becomes more accessible when teachers use the latest research to create up-to-date curricula without prohibitive costs. Students can access high-quality research materials and use LM assistance to better understand complex topics, enhancing their learning experience and making high-quality education more accessible.

\textbf{Public Trust:} When information is transparent and accessible, the public can better understand and trust decision-making processes. Open access to government policies and industry practices allows people to review and verify information, helping to reduce misinformation. This transparency encourages critical thinking and builds trust in scientific and governmental institutions.

Overall, making scientific knowledge accessible supports global fairness. By viewing knowledge as a common resource rather than a product to be sold, we can speed up innovation, encourage critical thinking, and empower communities to address important challenges.

\vspace{-0.2cm}
\section{Open Problems}
\vspace{-0.1cm}

Moving forward, we identify key research directions to further exploit the potential of converting original texts into shareable knowledge entities such as demonstrated by the conversion into Knowledge Units in this work:


\textbf{1. Enhancing Factual Accuracy and Reliability:}  Refining KUs through cross-referencing with source texts and incorporating community-driven correction mechanisms, similar to Wikipedia, can minimize hallucinations and ensure the long-term accuracy of knowledge-based datasets at scale.

\textbf{2. Developing Applications for Education and Research:}  Using KU-based conversion for datasets to be employed in practical tools, such as search interfaces and learning platforms, can ensure rapid dissemination of any new knowledge into shareable downstream resources, significantly improving the accessibility, spread, and impact of KUs.

\textbf{3. Establishing Standards for Knowledge Interoperability and Reuse:}  Future research should focus on defining standardized formats for entities like KU and knowledge graph layouts \citep{lenat1990cyc}. These standards are essential to unlock seamless interoperability, facilitate reuse across diverse platforms, and foster a vibrant ecosystem of open scientific knowledge. 

\textbf{4. Interconnecting Shareable Knowledge for Scientific Workflow Assistance and Automation:} There might be further potential in constructing a semantic web that interconnects publicly shared knowledge, together with mechanisms that continually update and validate all shareable knowledge units. This can be starting point for a platform that uses all collected knowledge to assist scientific workflows, for instance by feeding such a semantic web into recently developed reasoning models equipped with retrieval augmented generation. Such assistance could assemble knowledge across multiple scientific papers, guiding scientists more efficiently through vast research landscapes. Given further progress in model capabilities, validation, self-repair and evolving new knowledge from already existing vast collection in the semantic web can lead to automation of scientific discovery, assuming that knowledge data in the semantic web can be freely shared.

We open-source our code and encourage collaboration to improve extraction pipelines, enhance Knowledge Unit capabilities, and expand coverage to additional fields.

\vspace{-0.2cm}
\section{Conclusion}
\vspace{-0.1cm}

In this paper, we highlight the potential of systematically separating factual scientific knowledge from protected artistic or stylistic expression. By representing scientific insights as structured facts and relationships, prototypes like Knowledge Units (KUs) offer a pathway to broaden access to scientific knowledge without infringing copyright, aligning with legal principles like German \S 24(1) UrhG and U.S. fair use standards. Extensive testing across a range of domains and models shows evidence that Knowledge Units (KUs) can feasibly retain core information. These findings offer a promising way forward for openly disseminating scientific information while respecting copyright constraints.

\section*{Author Contributions}

Christoph conceived the project and led organization. Christoph and Gollam led all the experiments. Nick and Huu led the legal aspects. Tawsif led the data collection. Ameya and Andreas led the manuscript writing. Ludwig, Sören, Robert, Jenia and Matthias provided feedback. advice and scientific supervision throughout the project. 

\section*{Acknowledgements}

The authors would like to thank (in alphabetical order): Sebastian Dziadzio, Kristof Meding, Tea Mustać, Shantanu Prabhat for insightful feedback and suggestions. Special thanks to Andrej Radonjic for help in scaling up data collection. GR and SA acknowledge financial support by the German Research Foundation (DFG) for the NFDI4DataScience Initiative (project number 460234259). AP and MB acknowledge financial support by the Federal Ministry of Education and Research (BMBF), FKZ: 011524085B and Open Philanthropy Foundation funded by the Good Ventures Foundation. AH acknowledges financial support by the Federal Ministry of Education and Research (BMBF), FKZ: 01IS24079A and the Carl Zeiss Foundation through the project "Certification and Foundations of Safe ML Systems" as well as the support from the International Max Planck Research School for Intelligent Systems (IMPRS-IS). JJ acknowledges funding by the Federal Ministry of Education and Research of Germany (BMBF) under grant no. 01IS22094B (WestAI - AI Service Center West), under grant no. 01IS24085C (OPENHAFM) and under the grant DE002571 (MINERVA), as well as co-funding by EU from EuroHPC Joint Undertaking programm under grant no. 101182737 (MINERVA) and from Digital Europe Programme under grant no. 101195233 (openEuroLLM) 
%
\begin{acks}
This research is supported by the \grantsponsor{⟨malsome⟩}{Academy of Finland project MALSOME}{} (\grantnum[]{malsome}{343045}) and by the \grantsponsor{⟨hiit⟩}{Helsinki Institute for Information Technology (HIIT)}{}.
\end{acks}
%\clearpage

%%
%% The next two lines define the graphy style to be used, and
%% the bibliography file.
\bibliographystyle{ACM-Reference-Format}
\balance
\bibliography{maxmin_kdd}

%%
%% If your work has an appendix, this is the place to put it.

%\iffalse
\clearpage % this puts it on a new page
\appendix
\newtcolorbox{casebox}{
    colframe=black!50,    % Set border color
    colback=white,       % Background color inside the box
    coltitle=black,      % Title color (if you add a title)
    boxrule=0.4mm,       % Thicker border for a cleaner look
    arc=1mm,             % Slightly smaller rounded corners
    top=1mm,             % Add a little more space at the top
    bottom=1mm,          % Add a little more space at the bottom
    left=1.6mm,            % Adjusting side padding
    right=1.6mm,           % Adjusting side padding
    before skip=2mm,     % Add some vertical space before the box
    after skip=2mm,      % Add space after the box
    % fontupper=\small
}



\section{Appendix}

\subsection{Prompts for Baseline Models}
\label{sec:appendix_prompt_baseline}
We incorporate various prompt-based methods in our experiments to ensure a fair comparison. To maintain consistency, we keep most of the prompt content similar to that used for \themodel in Section~\ref{sec:ourprompt}. Below, we list the prompts used for the baseline models.

For the zero-shot and LoRA fine-tuning experiments, we use the following prompt:
\begin{mybox}
\textit{Based on the evidence, determine if the claim is supported by the evidence or refuted by it.\\
Claim: \texttt{[claim text $c$]}\\
Evidence: \texttt{(1)[evidence text $e_1$](2)...}}\\
\textit{Please respond with only whether the claim is ``Supported'' or ``Refuted.''}
\end{mybox}

For the zero-shot + CoT and STaR* experiments, the prompt is as follows:
\begin{mybox}
\textit{Based on the evidence, determine if the claim is supported by the evidence or refuted by it.\\
Claim: \texttt{[claim text $c$]}\\
Evidence: \texttt{(1)[evidence text $e_1$](2)...}}\\
\textit{Think step by step, output your response in the following format:\\
Chain: \texttt{[your reasoning chain]}\\
Answer:\texttt{[the claim is supported or the claim is refuted]}}
\end{mybox}

For the Few-shot experiment, the prompt is similar to the zero-shot prompt but includes examples:
\begin{mybox}
\textit{Based on the evidence, determine if the claim is supported by the evidence or refuted by it.\\
Please respond with only whether the claim is ``Supported'' or ``Refuted.'' Here are some examples:\\ 
Claim: Simon Grundel-Helmfelt is most ...\\
Evidence: (1) Baron Simon Grundel ... (2)...
Output: Refuted\\
\\(...more examples...)\\\\
Follow the above examples:\\
Claim: \texttt{[claim text $c$]}\\
Evidence: \texttt{(1)[evidence text $e_1$](2)...}\\
Output:
}
\end{mybox}

For the few-shot + structured CoT, we use the same prompt as \themodel, but with examples of structured reasoning:
\begin{mybox}
\textit{Based on the evidence, determine if the claim is supported by the evidence or refuted by it. Output the reasoning chain. Here are some examples:\\ 
Claim: Simon Grundel-Helmfelt is most ...\\
Evidence: (1) Baron Simon Grundel ... (2)...
Chain: C1: Simon Grundel ... \\
\\(...more examples...)\\\\
Follow the above examples:\\
Claim: \texttt{[claim text $c$]}\\
Evidence: \texttt{(1)[evidence text $e_1$](2)...}\\
Chain:
}
\end{mybox}

\subsection{Prompts with Hint}\label{sec:prompthint}
In \themodel, we add hint and regenerate reasoning chains for the ones that falsely predict the label of the claim. If the truth label is $p=\textit{Supported}$, we use prompt:
\begin{mybox}
\textit{Based on the evidence, determine if the claim is supported by the evidence or refuted by it. Output the reasoning chain.\\
Claim: \texttt{[claim text $c$]}\\
Evidence: \texttt{(1)[evidence text $e_1$](2)...}}\\
\textit{Hint: Every detail in this claim is supported.}
\end{mybox}
If the truth label is $p=\textit{Refuted}$, we use prompt:
\begin{mybox}
\textit{Based on the evidence, determine if the claim is supported by the evidence or refuted by it. Output the reasoning chain.\\
Claim: \texttt{[claim text $c$]}\\
Evidence: \texttt{(1)[evidence text $e_1$](2)...}}\\
\textit{Hint: The claim should be refuted, locate the error in the reasoning chain.}
\end{mybox}

\subsection{Dataset Statistics}\label{sec:datasetinfo}
The following table presents the information of the dataset (validation set) that we have tested on. ``HV'' represents HOVER, while ``FS'' stands for FEVEROUS.
\begin{table}[h]  
  \centering  
  \resizebox{\columnwidth}{!}{  
  \setlength{\tabcolsep}{3pt}
    \begin{tabular}{ccccc}  
    \toprule
    \textbf{Dataset} & \textbf{HV-2} & \textbf{HV-3} & \textbf{HV-4} & \textbf{FS-S} \\
    \midrule  
    Total  & 1,126 & 1,835 & 1,039 & 2,962 \\
    Supported Claims & 521 & 968 & 511 & 1411 \\
    Refuted Claims & 605 & 867 & 528 & 1551\\
    Avg. Words in Claim & 19.6 & 24.1 & 32.2 & 27.6 \\
    Avg. Evidence Pieces & 2.0 & 3.0 & 4.0 & 2.1 \\
    Avg. Words in Evidence & 137.3 & 211.1 & 278.2 & 50.3 \\
    \bottomrule 
    \end{tabular}%  
    }  
  \label{tab:datasets}  
\end{table}

\subsection{\themodel-Generated Reasoning Chains}\label{sec:cases}
In this section, we present several reasoning chains generated by \themodel. The evidence is omitted due to its length, which makes it unsuitable for display. Red coloring is used to highlight where the reasoning chains become incorrect.
\small{\begin{casebox}
\textbf{Claim}: The artist whose work was displayed in 1974 at Debel Gallery, was closely associated with the Viennese Actionism group, while Howard Zieff was an American television commercial director.\\
\textbf{Label}: Supported\\\\
\textbf{Model Output}:\\
C1: The artist whose work was displayed in 1974 at Debel Gallery was closely associated with the Viennese Actionism group.\\Entity Resolution: Artist -> Rudolf Schwarzkogler (from E2)\\Resolution Verification: E1 confirms Rudolf Schwarzkogler's work was displayed in 1974 at Debel Gallery.\\Verification: E2 supports that Rudolf Schwarzkogler was closely associated with the Viennese Actionism group.\\Status: Supported.\vspace{1mm}\\C2: Howard Zieff was an American television commercial director.\\Entity Resolution: Howard Zieff -> Howard B. Zieff (from E3)\\Resolution Verification: E3 confirms Howard B. Zieff was an American director, television commercial director, and advertising photographer.\\Verification: E3 supports that Howard Zieff was an American television commercial director.\\Status: Supported.
\end{casebox}}

\small{\begin{casebox}
\textbf{Claim}: Ben Karlin wrote the 2013 episode of the TV show, Netflix, that was directed by the actor who played Kevin Arnold in ``The Wonder Years''\\
\textbf{Label}: Refuted\\\\
\textbf{Model Output}:\\
C1: Ben Karlin wrote the 2013 episode of the TV show, Netflix.\\Entity Resolution: TV show -> Modern Family (from E1)\\Resolution Verification: E1 confirms 'ClosetCon '13' aired in 2013, and E1 also confirms Ben Karlin wrote it.\\Verification: E1 supports that Ben Karlin wrote the 2013 episode, but it's clear that the TV show is Modern Family, not Netflix.\\Status: Refuted.\vspace{1mm}\\C2: The actor who played Kevin Arnold in \"The Wonder Years\" directed the 2013 episode.\\Entity Resolution: Actor -> Fred Savage (from E2)\\Resolution Verification: E2 confirms Fred Savage played Kevin Arnold in ``The Wonder Years''.\\Verification: E2 supports that Fred Savage played Kevin Arnold, but it's clear that he directed the episode in 2013, not that he played Kevin Arnold in 2013.\\Status: Refuted.\vspace{1mm}\\C3: The TV show was Netflix.\\Entity Resolution: TV show -> Modern Family (from E1)\\Resolution Verification: E1 confirms 'ClosetCon '13' aired on Modern Family, not Netflix.\\Verification: E1 refutes that the TV show was Netflix.\\Status: Refuted.
\end{casebox}}

\small{\begin{casebox}
\textbf{Claim}: The builder of Shadow Creek Golf Course is an Jewish real estate developer. The builder also owns the Encore hotel and casino in Las Vegas.\\
\textbf{Label}: Refuted\\\\
\textbf{Model Output}:\\
C1: The builder of Shadow Creek Golf Course is an Jewish real estate developer.\\Entity Resolution: Builder of Shadow Creek Golf Course -> Steve Wynn (from E3)\\Resolution Verification: E3 confirms Steve Wynn built Shadow Creek Golf Course.\\Verification: E3 shows Steve Wynn built Shadow Creek Golf Course, but it doesn't provide any information about his religion. There's no evidence to support the claim that Steve Wynn is Jewish.\\Status: Refuted.\vspace{1mm}\\C2: The builder also owns the Encore hotel and casino in Las Vegas.\\Entity Resolution: Builder -> Steve Wynn (from E3)\\Resolution Verification: E3 confirms Steve Wynn built Shadow Creek Golf Course, and E2 confirms Wynn Resorts owns Encore.\\Verification: E3 shows Steve Wynn built Shadow Creek Golf Course, and E2 confirms Wynn Resorts owns Encore.\\Status: Supported.
\end{casebox}}

\small{\begin{casebox}
\textbf{Claim}: Carnegie Mellon University, not the institution which is home to the Lyme Academy of Fine Arts, is a university in Pennsylvania.\\
\textbf{Label}: Supported\\\\
\textbf{Model Output}:\\
C1: Carnegie Mellon University, not the institution which is home to the Lyme Academy of Fine Arts, is a university in Pennsylvania.\\Entity Resolution: institution which is home to the Lyme Academy of Fine Arts -> University of New Haven (from E1)\\Resolution Verification: E1 confirms Lyme Academy College of Fine Arts is part of University of New Haven, not CMU. E3 confirms CMU is in Pennsylvania.\\Verification: E1 refutes \textcolor{red}{the institution which is home to the Lyme Academy of Fine Arts being CMU}. E3 supports CMU being in Pennsylvania.\\Status: Refuted.
\end{casebox}}
(The model misunderstood the meaning of the claim.)

\small{\begin{casebox}
\textbf{Claim}: Orange is the main flavor of both \textcolor{red}{zero-calorie} soft drinks, Solo from Norway, and the drink advertised with Krupa (song).\\
\textbf{Label}: Refuted\\\\
\textbf{Model Output}:\\
C1: Orange is the main flavor of Solo from Norway.\\Verification: E1 confirms Solo is an orange-flavored soft drink from Norway.\\Status: Supported.\vspace{1mm}\\C2: Orange is the main flavor of the drink advertised with Krupa (song).\\Entity Resolution: Drink advertised with Krupa -> Sunkist\\Resolution Verification: E3 confirms Krupa is from an Sunkist advert.\\Verification: E3 confirms Krupa is from an Sunkist advert and E2 confirms Sunkist is primarily orange flavored.\\Status: Supported.
\end{casebox}}
(The model lost the information of "zero-calorie" in the claim during the Claim Decomposition process, leading to an error.)

\small{\begin{casebox}
\textbf{Claim}: The star of Forever Strong played a character that is based on the Hanna-Barbera show ``Birdman and the Galaxy Trio''.\\
\textbf{Label}: Supported\\\\
\textbf{Model Output}:\\
C1: The star of Forever Strong played a character that is based on the Hanna-Barbara show ``Birdman and the Galaxy Trio''.\\Entity Resolution: Star of Forever Strong -> Gary Cole (from E1)\\Resolution Verification: E1 confirms Gary Cole was a star of Forever Strong.\\Entity Resolution: \textcolor{red}{Character played by Gary Cole -> ? (from E2)}\\Resolution Verification: E2 confirms Gary Cole voiced Harvey Birdman, but it doesn't mention him playing \textcolor{red}{a character in Forever Strong}.\\Verification: E1 shows Gary Cole was a star of Forever Strong, but E2 shows \textcolor{red}{he voiced Harvey Birdman, not a character in Forever Strong}. Therefore, the claim is refuted.\\Status: Refuted.
\end{casebox}}
(The model incorrectly associated the character in the claim with ``Forever Strong,'' leading to an error in Entity Resolution.)
%\fi

\end{document}
\endinput

