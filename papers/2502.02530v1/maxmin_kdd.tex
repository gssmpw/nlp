%%
%% This is file `sample-sigconf.tex',
%% generated with the docstrip utility.
%%
%% The original source files were:
%%
%% samples.dtx  (with options: `sigconf')
%% 
%% IMPORTANT NOTICE:
%% 
%% For the copyright see the source file.
%% 
%% Any modified versions of this file must be renamed
%% with new filenames distinct from sample-sigconf.tex.
%% 
%% For distribution of the original source see the terms
%% for copying and modification in the file samples.dtx.
%% 
%% This generated file may be distributed as long as the
%% original source files, as listed above, are part of the
%% same distribution. (The sources need not necessarily be
%% in the same archive or directory.)
%%
%%
%% Commands for TeXCount
%TC:macro \cite [option:text,text]
%TC:macro \citep [option:text,text]
%TC:macro \citet [option:text,text]
%TC:envir table 0 1
%TC:envir table* 0 1
%TC:envir tabular [ignore] word
%TC:envir displaymath 0 word
%TC:envir math 0 word
%TC:envir comment 0 0
%%
%%
%% The first command in your LaTeX source must be the \documentclass command.
%\pdfminorversion=4
%\RequirePackage{hyperref}
%\documentclass[sigconf, anonymous, review]{acmart}
\documentclass[sigconf]{acmart}


%% Rights management information.  This information is sent to you
%% when you complete the rights form.  These commands have SAMPLE
%% values in them; it is your responsibility as an author to replace
%% the commands and values with those provided to you when you
%% complete the rights form.
%\setcopyright{acmcopyright}
%\copyrightyear{2018}
%\acmYear{2018}
%\acmDOI{10.1145/1122445.1122456}
%
%%% These commands are for a PROCEEDINGS abstract or paper.
%\acmConference[WWW '23]{ACM International Web Conference}{April 30-- May 04, 2023}{Austin, Texas, USA}
%%\acmBooktitle{Woodstock '18: ACM Symposium on Neural Gaze Detection,
%%  June 03--05, 2018, Woodstock, NY}
%\acmPrice{15.00}
%\acmISBN{978-1-4503-XXXX-X/18/06}


\copyrightyear{2024}
\acmYear{2024}
\setcopyright{rightsretained}
\acmConference[KDD '24]{Proceedings of the 30th ACM SIGKDD Conference on
Knowledge Discovery and Data Mining}{August 25--29, 2024}{Barcelona, Spain}
\acmBooktitle{Proceedings of the 30th ACM SIGKDD Conference on Knowledge
Discovery and Data Mining (KDD '24), August 25--29, 2024, Barcelona,
Spain}
\acmDOI{10.1145/3637528.3671757}
\acmISBN{979-8-4007-0490-1/24/08}

% The following includes the CC license icon appropriate for your paper.
% Download the image from www.scomminc.com/pp/acmsig/4ACM-CC-by-88x31.eps
% and place within your figs or figures folder

\makeatletter
\gdef\@copyrightpermission{
  \begin{minipage}{0.3\columnwidth}
   \href{https://creativecommons.org/licenses/by/4.0/}{\includegraphics[width=0.90\textwidth]{4ACM-CC-by-88x31.eps}}
  \end{minipage}\hfill
  \begin{minipage}{0.7\columnwidth}
   \href{https://creativecommons.org/licenses/by/4.0/}{This work is licensed under a Creative Commons Attribution International 4.0 License.}
  \end{minipage}
  \vspace{5pt}
}
\makeatother

\settopmatter{printacmref=true}

%%
%% Submission ID.
%% Use this when submitting an article to a sponsored event. You'll
%% receive a unique submission ID from the organizers
%% of the event, and this ID should be used as the parameter to this command.
%%\acmSubmissionID{123-A56-BU3}

%%
%% The majority of ACM publications use numbered citations and
%% references.  The command \citestyle{authoryear} switches to the
%% "author year" style.
%%
%% If you are preparing content for an event
%% sponsored by ACM SIGGRAPH, you must use the "author year" style of
%% citations and references.
%% Uncommenting
%% the next command will enable that style.
%%\citestyle{acmauthoryear}

%%
%% end of the preamble, start of the body of the document source.
\usepackage{hyperref}
\usepackage{amsmath}
\usepackage{soul}
\usepackage{graphicx}
\usepackage{algorithm}
\usepackage[noend]{algpseudocode}
\usepackage{natbib}
\usepackage{xspace}
\usepackage{subfig}
\usepackage{bbm}
\usepackage{bm}
\usepackage{centernot}
\usepackage{tcolorbox}
\usepackage{balance}

\usepackage{tikz}
\usepackage{pgfplots}

\definecolor{yafcolor1}{rgb}{0.4, 0.165, 0.553}
\definecolor{yafcolor2}{rgb}{0.949, 0.482, 0.216}
\definecolor{yafcolor3}{rgb}{0.47, 0.549, 0.306}
\definecolor{yafcolor4}{rgb}{0.925, 0.165, 0.224}
\definecolor{yafcolor5}{rgb}{0.141, 0.345, 0.643}
\definecolor{yafcolor6}{rgb}{0.965, 0.933, 0.267}
\definecolor{yafcolor7}{rgb}{0.627, 0.118, 0.165}
\definecolor{yafcolor8}{rgb}{0.878, 0.475, 0.686}

\newcommand{\tO}{\widetilde{O}}
\newcommand{\dmax}{d_{\text{max}}}
\newcommand{\dmin}{d_{\text{min}}}
\newcommand{\TC}{\text{TC}}
\def\Exp{\mathbb{E}}
\def\Var{\mathrm{Var}}
\newcommand{\simon}[1]{\textcolor{blue}{Simon: #1}}
\usepackage{algorithm}
\usepackage[noend]{algpseudocode}
\DeclareMathOperator*{\argmax}{arg\,max}
\newcommand{\Pcal}{\mathcal{P}}
\newcommand{\bigO}{\ensuremath{\mathcal{O}}\xspace}
\newcommand{\NP}{\ensuremath{\mathbf{NP}}\xspace}
\newcommand{\poly}{\ensuremath{\mathbf{P}}\xspace}
\algrenewcommand\algorithmicrequire{\textbf{Input:}}
\algrenewcommand\algorithmicensure{\textbf{Output:}}

%dashed line
\usepackage{array}
\usepackage{arydshln}
\setlength\dashlinedash{0.2pt}
\setlength\dashlinegap{1.5pt}
\setlength\arrayrulewidth{0.3pt}


\newcommand{\ptitle}[1]{\smallskip\noindent{\bf #1.}}
\newcommand{\pttitle}[1]{\smallskip\noindent{\it #1.}}
\newcommand{\parfat}[1]{{\bf #1.}}
\newcommand{\ptitlenoskip}[1]{\noindent{\bf #1.}}

\newcommand{\set}[1]{\left\{#1\right\}}
\newcommand{\pr}[1]{\left(#1\right)}
\newcommand{\fpr}[1]{\mathopen{}\left(#1\right)}
\newcommand{\spr}[1]{\left[#1\right]}
\newcommand{\fspr}[1]{\mathopen{}\left[#1\right]}
\newcommand{\brak}[1]{\left<#1\right>}
\newcommand{\abs}[1]{{\left|#1\right|}}
\newcommand{\floor}[1]{{\left\lfloor#1\right\rfloor}}
\newcommand{\ceil}[1]{{\left\lceil#1\right\rceil}}
\newcommand{\norm}[1]{\left\|#1\right\|}
\newcommand{\enset}[2]{\left\{#1 ,\ldots , #2\right\}}
\newcommand{\enpr}[2]{\pr{#1 ,\ldots , #2}}
\newcommand{\enlst}[2]{{#1} ,\ldots , {#2}}
\newcommand{\vect}[1]{\spr{#1}}
\newcommand{\envec}[2]{\vect{#1 ,\ldots , #2}}
\newcommand{\define}{\leftarrow}

\DeclareRobustCommand{\dispfunc}[2]{%
    \ensuremath{%
        \ifthenelse{\equal{#2}{}}%
            {\mathit{#1}}%
            {\mathit{#1}\fpr{#2}}}}


\newcommand{\diver}[1]{\dispfunc{div}{#1}}

\newcommand{\algclust}[1]{\dispfunc{\textsc{Clust}}{#1}}
\newcommand{\algextract}[1]{\dispfunc{\textsc{Extract}}{#1}}
\newcommand{\algbac}[1]{\dispfunc{\textsc{BAC}}{#1}}
\newcommand{\algbacb}[1]{\dispfunc{\textsc{BCR}}{#1}}
\newcommand{\algbacf}[1]{\dispfunc{\textsc{BCF}}{#1}}



\newtheorem{theorem}{Theorem}
\newtheorem{lemma}{Lemma}
\newtheorem{corollary}{Corollary}
\newtheorem{claim}{Claim}
\newtheorem{prop}{Proposition}
\newtheorem{prob}{Problem}
\newtheorem{defi}{Definition}


% PGF stuff

\pgfdeclarelayer{background}
\pgfdeclarelayer{foreground}
\pgfsetlayers{background,main,foreground}


\definecolor{yafaxiscolor}{rgb}{0.3, 0.3, 0.3}

\newlength{\yafaxispad}
\setlength{\yafaxispad}{-2pt}
\newlength{\yaftlpad}
\setlength{\yaftlpad}{\yafaxispad}
\addtolength{\yaftlpad}{-0pt}
\newlength{\yaflabelpad}
\setlength{\yaflabelpad}{-2pt}
\newlength{\yafaxiswidth}
\setlength{\yafaxiswidth}{1.2pt}
\newlength{\yafticklen}
\setlength{\yafticklen}{2pt}

\makeatletter
\def\pgfplots@drawtickgridlines@INSTALLCLIP@onorientedsurf#1{}
\makeatother

\newcommand{\yafdrawaxis}[4]{
	\pgfplotstransformcoordinatex{#1}\let\xmincoord=\pgfmathresult 
	\pgfplotstransformcoordinatex{#2}\let\xmaxcoord=\pgfmathresult 
	\pgfplotstransformcoordinatey{#3}\let\ymincoord=\pgfmathresult 
	\pgfplotstransformcoordinatey{#4}\let\ymaxcoord=\pgfmathresult 
	\pgfsetlinewidth{\yafaxiswidth} 
	\pgfsetcolor{yafaxiscolor}
	\pgfpathmoveto{\pgfpointadd{\pgfpointadd{\pgfplotspointrelaxisxy{0}{0}}{\pgfqpointxy{\xmincoord}{0}}}{\pgfqpoint{-0.5\yafaxiswidth}{\yafaxispad}}}
	\pgfpathlineto{\pgfpointadd{\pgfpointadd{\pgfplotspointrelaxisxy{0}{0}}{\pgfqpointxy{\xmaxcoord}{0}}}{\pgfqpoint{0.5\yafaxiswidth}{\yafaxispad}}}
	\pgfpathmoveto{\pgfpointadd{\pgfpointadd{\pgfplotspointrelaxisxy{0}{0}}{\pgfqpointxy{0}{\ymincoord}}}{\pgfqpoint{\yafaxispad}{-0.5\yafaxiswidth}}}
	\pgfpathlineto{\pgfpointadd{\pgfpointadd{\pgfplotspointrelaxisxy{0}{0}}{\pgfqpointxy{0}{\ymaxcoord}}}{\pgfqpoint{\yafaxispad}{0.5\yafaxiswidth}}}
	\pgfusepath{stroke}
}

\pgfplotscreateplotcyclelist{yaf}{% 
{yafcolor5,mark options={scale=0.75},mark=o}, 
{yafcolor2,mark options={scale=0.75},mark=square},
{yafcolor3,mark options={scale=0.75},mark=triangle},
{yafcolor4,mark options={scale=0.75},mark=o},
{yafcolor1,mark options={scale=0.75},mark=o},
{yafcolor8,mark options={scale=0.75},mark=o},
{yafcolor6,mark options={scale=0.75},mark=o},
{yafcolor7,mark options={scale=0.75},mark=o}} 

\pgfkeys{/pgf/number format/.cd,1000 sep={\,}}
\pgfplotsset{axis y line=left, axis x line=bottom,
	tick align=outside,
	tickwidth=\yafticklen,
	clip = false,
    x axis line style= {-, line width = 0pt, color=black!0},
    y axis line style= {-, line width = 0pt, color=black!0},
    x tick style= {line width = \yafaxiswidth, color=yafaxiscolor, yshift = \yafaxispad},
    y tick style= {line width = \yafaxiswidth, color=yafaxiscolor, xshift = \yafaxispad},
    x tick label style = {font=\small, yshift = \yaftlpad, inner xsep = 0pt},
    y tick label style = {font=\small, xshift = \yaftlpad},
    every axis y label/.style = {at = {(ticklabel cs:0.5)}, rotate=90, anchor=center, font=\small, yshift = -\yaflabelpad, inner sep = 0pt},
    every axis x label/.style = {at = {(ticklabel cs:0.5)}, anchor=center, font=\small, yshift = \yaflabelpad},
    x tick label style = {font=\small, yshift = 1pt},
    grid = major,
    major grid style  = {dash pattern = on 1pt off 3 pt},
	every axis plot post/.append style= {line width=\yafaxiswidth} ,
	legend cell align = left,
	legend style = {inner sep = 1pt, cells = {font=\scriptsize}},
	legend image code/.code={%
		\draw[mark repeat=2,mark phase=2,#1] 
		plot coordinates { (0cm,0cm) (0.15cm,0cm) (0.3cm,0cm) };% 
	} 
}



\begin{document}

%%
%% The "title" command has an optional parameter,
%% allowing the author to define a "short title" to be used in page headers.
\title{Max-Min Diversification with Asymmetric Distances}

%%
%% The "author" command and its associated commands are used to define
%% the authors and their affiliations.
%% Of note is the shared affiliation of the first two authors, and the
%% "authornote" and "authornotemark" commands
%% used to denote shared contribution to the research.

\author{Iiro Kumpulainen}
%\authornotemark[1]
\email{iiro.kumpulainen@helsinki.fi}
\affiliation{%
  \institution{University of Helsinki}
  \city{Helsinki}
  \country{Finland}
}

\author{Florian Adriaens}
%\authornote{}
\email{florian.adriaens@helsinki.fi}
\affiliation{%
  \institution{University of Helsinki, HIIT}
  \city{Helsinki}
  \country{Finland}
}


\author{Nikolaj Tatti}
%\authornotemark[1]
\email{nikolaj.tatti@helsinki.fi}
\affiliation{%
  \institution{University of Helsinki, HIIT}
  \city{Helsinki}
  \country{Finland}
}


%%
%% By default, the full list of authors will be used in the page
%% headers. Often, this list is too long, and will overlap
%% other information printed in the page headers. This command allows
%% the author to define a more concise list
%% of authors' names for this purpose.
\renewcommand{\shortauthors}{Iiro Kumpulainen, Florian Adriaens, \& Nikolaj Tatti}


\begin{abstract}
Retrieval-Augmented Generation (RAG) is often used with Large Language Models (LLMs) to infuse domain knowledge or user-specific information. In RAG, given a user query, a retriever extracts chunks of relevant text from a knowledge base. These chunks are sent to an LLM as part of the input prompt. Typically, any given chunk is repeatedly retrieved across user questions. However, currently, for every question, attention-layers in LLMs fully compute the key values (KVs) repeatedly for the input chunks, as state-of-the-art methods cannot reuse KV-caches when chunks appear at arbitrary locations with arbitrary contexts. Naive reuse leads to output quality degradation.  This leads to potentially redundant computations on expensive GPUs and increases latency. In this work, we propose \sys, a system for managing and reusing precomputed KVs corresponding to the text chunks (we call \textit{chunk-caches}) in RAG-based systems. We present how to identify \hl{\textit{chunk-caches} that are reusable}, how to efficiently perform a small fraction of recomputation to \textit{fix} the cache to maintain output quality, and how to efficiently store and evict \textit{chunk-caches} in the hardware for maximizing reuse while masking any overheads. With real production workloads as well as synthetic datasets, we show that \sys reduces redundant computation by \textbf{51\%} over SOTA prefix-caching and \textbf{75\%} over full recomputation.
\hl{Additionally, with continuous batching on a real production workload, we get a \textbf{1.6$\times$} speedup in throughput and a \textbf{2$\times$} reduction in end-to-end response latency over prefix-caching while maintaining quality, for both the \llama-3-8B and \llama-3-70B models. 
}
\end{abstract}






\ccsdesc[500]{Theory of computation~Approximation algorithms analysis}
\ccsdesc[500]{Mathematics of computing~Graph theory}
\keywords{Max-Min Diversification, Asymmetry, Maximum Antichain}
\maketitle

\section{Introduction}
\label{sec:intro}

\begin{figure*}[tb]
    \centering
    \includegraphics[width=0.848\linewidth]{figs/circuitnn.pdf} 
    \caption{Illustration of differentiable CircuitNN. CircuitNN is designed based on differentiable NAND gates. After DAS is guided by PI and PO pairs of the truth table, CircuitNN can get the precise circuit architecture logic equivalent to the truth table.}
    \label{fig:circuitnn}
\end{figure*}

% 1. Describe the importance of logic synthesis
% 2. Existing Problems
% (a) Neural Architecture Search: Unstable, Predefined Setting, etc.
% (b) Circuit Generation: Probabilistic Model, Logic Equivalence

With the rapid advancement of technology, the scale of integrated circuits (ICs) has expanded exponentially. 
This expansion has introduced significant challenges in chip manufacturing, particularly concerning power and area metrics.
A primary objective in IC design is achieving the same circuit function with fewer transistors, thereby reducing power usage and area occupancy.

Logic synthesis~\cite{hachtel2005logicsynth}, a critical step in electronic design automation (EDA), transforms behavioral-level circuit designs into optimized gate-level circuits, ultimately yielding the final IC layout. 
The primary goal of logic synthesis is to identify the physical implementation with the fewest gates for a given circuit function. 
This task constitutes a challenging NP-hard combinatorial optimization problem. 
Current logic synthesis tools~\cite{brayton2010abc, wolf2013yosys} rely on human-designed heuristics, often leading to sub-optimal outcomes.

Differentiable architecture search (DAS) techniques~\cite{liu2018darts, chu2020darts} offer novel perspectives on addressing challenges in this problem.
Circuit functions can be represented through truth tables, which map binary inputs to their corresponding outputs. 
Truth tables provide a precise representation of input-output relationships, ensuring the design of functionally equivalent circuits.
Inspired by this, researchers~\cite{deepmind2024ai4sys, wang2024tnet} have begun exploring the application of DAS to synthesize circuits directly from truth tables.
Specifically, \citet{deepmind2024ai4sys} proposed CircuitNN, a framework that learns differentiable connection structures with logic gates, enabling the automatic generation of logic circuits from truth tables.
This approach significantly reduces the complexity of traditional circuit generation. 
Building on this, \citet{wang2024tnet} introduced T-Net, a triangle-shaped variant of CircuitNN, incorporating regularization techniques to enhance the efficiency of DAS.

Despite these advancements, several challenges remain. 
The computational complexity of DAS grows quadratically with the number of gates, posing scalability issues.
Although triangle-shaped architecture~\cite{wang2024tnet} partially mitigates this problem, redundancy persists. 
%Additionally, DAS is susceptible to converging to local optima, limiting the ability to search architectures that satisfy the given truth tables~\cite{liu2018darts}. 
%Furthermore, hyperparameters (network depth and layer width) require extensive searches, introducing complexity and prolonging the synthesis process. 
Additionally, DAS is susceptible to converging to local optima~\cite{liu2018darts} and hyperparameters (network depth and layer width) require extensive searches. 
The challenges arise from the vast search space in DAS. 
% Even with predefined settings for CircuitNN, finding a configuration that meets the truth table requires extensive trial and error during the DAS process. 
Intuitively, limiting the search space through predefined parameters (network depth, gates per layer, and connection probabilities) can significantly reduce the complexity.

Recent advances~\cite{openai2023gpt4, abramson2024alphafold3, esser2024sd3, li2024mar} in conditional generative models have demonstrated remarkable performance across language, vision, and graph generation tasks. 
Motivated by these developments, we propose a novel approach to circuit generation that generates preliminary circuit structures to guide DAS in generating refined circuits matching specified truth tables. 
Firstly, we introduce CircuitVQ, a tokenizer with a discrete codebook for circuit tokenization. 
Built upon our Circuit AutoEncoder framework~\cite{hou2022graphmae,li2023maskgae,wu2025mgvga}, CircuitVQ is trained through a circuit reconstruction task. 
Specifically, the CircuitVQ encoder encodes input circuits into discrete tokens using a learnable codebook, while the decoder reconstructs the circuit adjacency matrix based on these tokens.
Subsequently, the CircuitVQ encoder serves as a circuit tokenizer for CircuitAR pretraining, which employs a masked autoregressive modeling paradigm~\cite{chang2022maskgit, li2023mage}. 
In this process, the discrete codes function as supervision signals. 
After training, CircuitAR can generate discrete tokens progressively, which can be decoded into initial circuit structures by the decoder of the CircuitVQ. 
These prior insights can guide DAS in producing refined circuits that match the target truth tables precisely.

Our key contributions can be summarized as follows:
\begin{itemize}
\item We introduce CircuitVQ, a circuit tokenizer that facilitates graph autoregressive modeling for circuit generation, based on our Circuit AutoEncoder framework;
\item Develop CircuitAR, a model trained using masked autoregressive modeling, which generates initial circuit structures conditioned on given truth tables;
\item Propose a refinement framework that integrates differentiable architecture search to produce functionally equivalent circuits guided by target truth tables;
\item Comprehensive experiments demonstrating the scalability and capability emergence of our CircuitAR and the superior performance of the proposed circuit generation approach.
\end{itemize}

% Motivation
% (a) Diffusion (Vision, Graph), Autoregressive (Language, Vision)
% (b) Circuit Generation for Predefined Setting
% (c) Neural Architecture Search for Strict Logic Equivalence

% Contribution
% (a) Circuit Tokenizer (new transformer arch, training strategy)
% (b) CircuitAR (train and gen strategies, post-ar strategy)
% (c) Extensive Evaluation including BitD (Bit Distance) for Scalability


\section{Related Work} \label{sec:related}

% \textbf{Adversarial Attack}
\textbf{Attacks on SLAM.} 
%With the rise of machine learning, 
The robustness of computer vision systems is being actively investigated. With the emergence of adversarial images in the digital domain by adding optimized noise directly to images~\cite{szegedy2013intriguing,carlini2017towards}, researchers find that such attacks also exist physically in the real world \cite{eykholt2018robust,song2018physical,zhao2019seeing}. To fill the gap between attacks in the digital and physical worlds, recent studies have demonstrated that attacks on real-world computer vision systems are practical \cite{eykholt2018robust,li2019adversarial,man2020ghostimage,sharif2016accessorize,zhao2019seeing,zhou2018invisible}. However, attacks on traditional computer vision methods such as SLAM are relatively less explored. \cite{yoshida2022adversarial} proposes an attack against the scan matching algorithm in LiDAR-based SLAM, while most SLAMs in AR/VR devices rely on different sensors like RGB/depth cameras and IMUs. \cite{ikram2022perceptual} and \cite{chen2024adversary} mislead visual SLAM by poisoning the images with special patterns, and \cite{wang2021can} causes the camera to fail using infrared light. In our work, we demonstrate attacks on Visual-Inertial SLAM (VI-SLAM) by perturbing the IMU readings, rather than cameras, and showing its impact on XR user experience. 

\textbf{Acoustic Injection Attacks.} Among various physical attacks, acoustic injection attacks are attractive due to their low cost. Son~\etal~\cite{son2015rocking} were the first to introduce acoustic attacks on MEMS gyroscopes, demonstrating how these attacks could lead to sensor denial-of-service and result in drone crashes. WALNUT~\cite{trippel2017walnut} expanded on this by developing output biasing and control attacks that enable precise manipulation of MEMS accelerometer outputs using modulated sound waves. Wang et al.~\cite{wang2017sonic} demonstrated a sonic gun, showcasing the vulnerability of various smart devices (\eg drones and self-balancing vehicles) to acoustic attacks. Tu et al. \cite{tu2018injected} designed side-swing and switching attacks to alter the outputs of MEMS gyroscopes and accelerometers. Furthermore, Ji et al. \cite{ji2021poltergeist} fool the object detectors by applying acoustic attack to the image stabilizers commonly used in modern cameras. However, none of the existing works study the relationship between the acoustic injections and SLAM outputs on recent XR devices. 

% \zijian{Do we need one session about security in AR/VR?}
% \yicheng{TODO}
%\jiasi{cite the AIVR paper (UMass Amherst?) paper is we have not already. They add IMU perturbation but w/o SLAM, iirc} \yicheng{Cited}

\textbf{XR Security and Privacy.} 
%Security and privacy concerns in XR systems have gained significant attention. 
For single-user XR systems, researchers have demonstrated various side-channel attacks to extract sensitive information (\eg keystrokes) through video feeds~\cite{ling2019know}, head movements~\cite{nair2023unique, slocum2023going}, architectural hints~\cite{zhang2023its,shang2020arspy}, power usage~\cite{li2024dangers}, and EM side-channel leakages~\cite{al2021vr}. In multi-user XR systems, Su et al.~\cite{su2024remote} use avatar motion data to infer keystrokes in shared VR environments. Slocum et al.~\cite{slocum2024doesn} reveal vulnerabilities in the shared state frameworks of multi-user AR. Similarly, Lebeck et al.~\cite{lebeck2017securing} highlight risks like deceptive virtual objects and emphasize access control for managing shared physical and virtual spaces. Ruth et al.~\cite{ruth2019secure} further propose a secure multi-user AR framework focusing on content sharing and permissions.
Chandio et al.~\cite{chandio2024stealthy} %introduced a multi-modal spatiotemporal attack that 
simultaneously manipulated visual and inertial sensors to disrupt XR pose estimation. However, their study evaluated the attack using offline datasets and assumed the attacker's capability to manipulate IMU data streams through acoustic means, without real experiments. Ours is the first to demonstrate acoustic injection attacks on recent XR devices, like the Hololens 2, in the real world.
 


\newcommand{\ours}{$\text{Q}$LASS}
%%%%%%%%%%%%%%%%%%%%%%%%%%%%%%
\section{Nearly symmetric instances}
\label{sec:algosnearlysym}
%%%%%%%%%%%%%%%%%%%%%%%%%%%%%
Greedily selecting the next-furthest points until $k$ points are selected is a $\frac{1}{2}$-approximation for symmetric MMD \cite{tamir1991obnoxious,ravi1994heuristic}, but can perform arbitrarily badly
on asymmetric instances as shown by the example in Figure~\ref{fig:toyexample_greedy}.

Theorem~\ref{thm:epsisymmgreedy} generalizes this result to asymmetric instances.
It states that on asymmetric instances that are $\epsilon$-symmetric (see Section~\ref{sec:notation}), the greedy approach applied on the $\dmin$ distances yields a $\frac{1}{2+\epsilon}$-approximation, and this ratio is tight.
The proof of Theorem~\ref{thm:epsisymmgreedy} can be found in the Appendix.

\begin{theorem}
\label{thm:epsisymmgreedy}
For any $\epsilon \geq 0$, Algorithm~\ref{algo:greedydmin} is a $\frac{1}{2+\epsilon}$-approximation on $\epsilon$-symmetric instances and can be implemented to run in $\bigO(kn)$ time.
Additionally, there exist $\epsilon$-symmetric instances for which Algorithm~\ref{algo:greedydmin} cannot achieve a performance ratio better than $\frac{1}{2+\epsilon}$.  
\end{theorem}

\begin{algorithm}[t]
\caption{Greedy with $\dmin$-distances.}
\label{algo:greedydmin}
\begin{algorithmic}[1]
%\Require graph $G$. 
\Require space $(U, d)$ and integer parameter $k\geq 2$.
\State $v \define \text{arbitrary vertex from } U$.
\State $S \define \{v\}$.
\While {$|S|<k$}
	\State $v \define \text{arg} \max_{u \in U} \dmin (u,S)$.
	\State $S \define S \cup \{v\}$.
\EndWhile
\Ensure the set $S$. 
\end{algorithmic}
\end{algorithm}



%%%%%%%%%%%%%%%%%%%%%%%%%%%%%%
\section{Ball-and-antichain method}
\label{sec:approx}
%%%%%%%%%%%%%%%%%%%%%%%%%%%%%
Section~\ref{sec:approxnk2} details a straightforward approximation algorithm for AMMD, exploiting the polynomial time complexity of the MA problem in digraphs, as discussed in the previous section.
This algorithm has an approximation guarantee of $\frac{1}{n-k+1}$, which is not very useful in a typical regime of small $k$, but it gives insight into the use of the MA problem for approximating AMMD.

In Section~\ref{sec:approx16k} we modify the algorithm from Section~\ref{sec:approxnk2} by first clustering the points based on the $\dmax$ distances between them. The subspace induced by the cluster centers has some very useful properties, which leads to an approximation algorithm with a multiplicative guarantee of $\frac{1}{6k}$ for AMMD.

Finally, in Section~\ref{sec:speeding}, we discuss improvements to look for better solutions and speed up the algorithm while maintaining the approximation guarantee.

\begin{algorithm}[t]
\caption{Naive Maximum Antichain method.}
\label{algo:app_ma}
\begin{algorithmic}[1]
%\Require graph $G$. 
\Require space $(U, d)$ and integer parameter $k\geq 2$.
\ForAll {$R \in \{d(i,j)>0 \mid i, j \in U,  i \neq j\}$}
	\State Create $G_R = (U,A)$, with $ij \in A \Leftrightarrow d(i,j)<\frac{R}{n-k+1}$. \label{line:naive_create_G}
	\State $M \define$ \text{Maximum antichain of} $G_R$.
	\If{$|M| \geq k$,} $S_R \leftarrow \text{any }k \text{ points from } M$.

	\EndIf
\EndFor
\Ensure the set $S_R$ with the largest $\diver{S_R}$ value. 
\end{algorithmic}
\end{algorithm}

%%%%%%%%%%%%%%%%%%%%%%%%%%%%%%
\subsection{Naive approach based on antichains}
\label{sec:approxnk2}
%%%%%%%%%%%%%%%%%%%%%%%%%%%%%

We begin by describing the naive approach for approximating AMMD.
Assume for the moment that we know $R = R^*$, the optimal value for an AMMD instance.
Consider a digraph $G = (U, A)$, where $ij \in A$ if and only if $d(i, j) < R$.
Then an independent set, say $O$, in $G$ of size $k$ will have $\diver{O} = R^*$.
Unfortunately, finding a maximum independent set in a graph is an \NP-hard problem with a weak approximation guarantee~\cite{hastad1996clique}.

Therefore, we lower the cutoff by setting it to $\frac{R}{n-k+1}$. This makes the underlying graph
so sparse that we can guarantee that the graph contains an antichain, say $S$, of size $k$.
Since an antichain is also an independent set, we know that $\diver{S} \geq \frac{R}{n-k+1}$.
We can find the antichain in polynomial time. Finally, we do not know $R^*$ but we know that it
is one of the distances. Therefore, we test every distance; there are at most $n(n - 1)$ of such distances.
The pseudo-code for the algorithm is given in Algorithm~\ref{algo:app_ma}.


\begin{theorem}
\label{thm:naivemaxanti}
Algorithm~\ref{algo:app_ma} is an $\frac{1}{n-k+1}$-approximation for AMMD in $\bigO(n^{4 + o(1)} \log n)$ time with high probability.
\end{theorem}

The proof is given in Appendix.

We finalize this section by observing that we could binary search for the largest $R$ value for which $G_R$ (defined in line~\ref{line:naive_create_G} in Algorithm~\ref{algo:app_ma}) still has an antichain of size $k$. First, sort the unique distances in time $\bigO(n^2 \log n)$, after which we need at most $\bigO(\log n)$ calls to find this $R$. Since for this $R$ we have $R \geq R^*$, we retain the same approximation guarantee. The binary search performs $\bigO(\log n)$ MA computations, and all of them need to succeed. Proposition~\ref{prop:multima} implies that the algorithm solves the problem in
$\bigO(n^{2 + o(1)}\log^2 n \log \log n)$ time with high probability.

%%%%%%%%%%%%%%%%%%%%%%%%%%%%%%
%%%%%%%%%%%%%%%%%%%%%%%%%%%%%%
\subsection{Refined approach: clustering and antichains}
\label{sec:approx16k}
%%%%%%%%%%%%%%%%%%%%%%%%%%%%%%
%%%%%%%%%%%%%%%%%%%%%%%%%%%%%%


The problem with Algorithm~\ref{algo:app_ma} is that it is using a very conservative cutoff of $\frac{R}{n-k+1}$, leading
to a weak guarantee. 
We show that we can relax this cutoff to $\frac{R}{6k}$, by first clustering the space $(U,d)$ according to the $\dmax$ distances.
The discovered cluster centers will have the property that two centers must have a large $\dmax$ distance between them.
A consequence of this property is that the resulting graph contains a large antichain, a large chordless cycle, or a large shortest path with no backward edges.
It turns out that we can search for all 3 subgraphs in polynomial time, and using those we can extract an independent set in polynomial time.

%%%%%%%%%%%%%%%%%%%%%%%%%%%%%%
%%%%%%%%%%%%%%%%%%%%%%%%%%%%%
\ptitle{Clustering step}
Next, we describe the clustering step, as given in
Algorithm~\ref{algo:ballcover}. Here the algorithm greedily covers $U$ with a family of pairwise disjoint sets $\set{A_t}$. 
Each $A_t$ is constructed by selecting an unmarked point as a center $c_t$, and adding all unmarked points $v$ with $\dmax(c_t, v) < R$. Since all the vertices in $A_t$ are marked at the end of step $t$ (line~5), they cannot be selected by any $A_{t'}$ for $t' > t$. It follows that for every $t \neq t'$ we have $A_t \cap A_{t'} = \emptyset$. 

Algorithm~\ref{algo:ballcover} terminates in at most $n$ steps, since every set $A_t$ contains at least one point, namely the center point $c_t$. Algorithm~\ref{algo:ballcover} runs in $\bigO(n^2)$ time.


Each $A_t$ is constructed by selecting vertices that have a small $\dmax$ distance to their center $c_t$. Line~4 in Algorithm~\ref{algo:ballcover} ensures that the $\dmax$ distances between two distinct centers are at least $R$. 

To analyze this further, let us define $R'$ as the smallest distance that is at least one third of the optimum $R^*$,
\begin{equation}
\label{def:R'}
R' = \min\{d(u,v) \mid d(u,v) \geq R^*/3, \ u,v \in U, \ u\neq v\}.
\end{equation}
If we then perform the clustering for any $R \leq R'$, there will be $k$ centers that also have a pairwise $\dmin$ distance of at least $R^*/3$ between them. This is captured by Proposition~\ref{prop:cluster} and Corollary~\ref{cor:clusterphase}.

\begin{algorithm}[t]
\caption{$\algclust{U, d, R}$, clusters $U$ according to $\dmax$.}\label{algo:ballcover} 
\begin{algorithmic}[1]
%\Require graph $G$. 
\Require space $(U, d)$ and parameter $R > 0$.
\State Label all $u \in U$ as unmarked, let ${U}' \define \emptyset$ and  $t \leftarrow 1$.
\While {there exists an unmarked point}
	\State $c_t \define $ any unmarked point.
	%\State $c_t \define $ unmarked point $\text{arg} \max_{c} \dmin (c,{U}')$
	\State $A_t \define \{\text{unmarked } v \in U \mid \dmax(c_t,v) < R \}.$ 
	\State Mark all $v \in A_t$. 
	\State ${U}' \define {U}' \cup \{c_t\}$ and $t \define t+1$.
\EndWhile
\Ensure ${U}'.$
\end{algorithmic}
\end{algorithm}

\begin{prop}
\label{prop:cluster}
Let $O$ be an optimal solution to AMMD with optimum $R^*$ and $R'$ as defined in Equation~\ref{def:R'}.
If $R \leq R'$, then the following two statements regarding Algorithm~\ref{algo:ballcover} are true.
\begin{itemize}
\item For all $t$ it holds that $|A_t \cap O| \leq 1$.
\item For any $t\neq t'$ for which $|A_t \cap O|=1$ and $|A_{t'} \cap O|=1$, it holds that $\dmin(c_t,c_{t'}) \geq R' \geq R^*/3$.
\end{itemize}
\end{prop}
\begin{proof}
For the first statement, suppose that $A_t$ contains two distinct $x, y \in O, x \neq y$.
Then it holds that $d(x,y) \leq d(x,c_t)+d(c_t,y)$. Since $d(x,c_t)$ and $d(c_t,y)$ are strictly less than $R\leq R'$, they must be less than $R^*/3$ because $R'$ is defined as the smallest distance greater or equal to $R^*/3$, so any distance strictly smaller than $R'$ must be less than $R^*/3$. Thus, $d(x,y) < 2R^*/3 < R^*$, a contradiction since $x, y \in O, x \neq y$ implies that $d(x,y) \geq R^*$.

For the second statement, assume that $A_t$ contains $x \in O$ and $A_{t'}$ contains $y \in O$. Since $A_t$ and $A_{t'}$ are disjoint, we have $x \neq y$. Note that $d(x,y) \leq d(x,c_t)+d(c_t,c_{t'})+d(c_{t'},y)$, where $d(x,c_t)<R \leq R'$ and $d(c_{t'},y)<R \leq R'$. This means $d(x,c_t)$ and $d(c_{t'},y)$ must be less than $R^*/3$ by the definition of $R'$. So if $d(c_t,c_{t'}) < R'$, we would have $d(x,y) < 3 R^*/3 = R^*$, a contradiction. Similarly, we cannot have $d(c_{t'},c_t) < R'$ either, which means $\dmin(c_t,c_{t'}) \geq R' \geq R^*/3$.
\end{proof}

\begin{corollary}
\label{cor:clusterphase}
Let ${U}' = \algclust{U, d, R}$.
For every $u, v \in {U}', u \neq v$ we have $\dmax(u,v) \geq R$. Additionally, if $R \leq R'$, then ${U}'$ contains a set $S$, for which $|S| = k$ and $\diver{S} \geq R' \geq R^*/3$.

\end{corollary}
 
Corollary~\ref{cor:clusterphase} states that as long as $R \leq R'$, from any instance space $(U, d)$ we can efficiently find a subset ${U}' \subseteq U$ 
such that ${U}'$ still contains $k$ points with a pairwise $\dmin$ distance of at least $R^*/3$ between them. This enables us to restrict ourselves to ${U}'$, at the expense of a decrease in the optimal value by a factor of three.


%%%%%%%%%%%%%%%%%%%%%%%%%%%%%%
\ptitle{The \algbac{} algorithm}
We are ready to describe our algorithm which we call \algbac{} (shortened for ball-and-antichain). The pseudocode is given in Algorithms~\ref{algo:bac}--\ref{algo:extract}.
Similar to the naive approach we iterate over all distances. For each candidate distance $R$,
we cluster the space to get the centers $U'$.
We then construct a graph $G$ with edges corresponding to distances shorter than $\frac{R}{2k}$.
We can guarantee that there is ($i$) a large chordless cycle, ($ii$) a long shortest path with no backward edges,
or ($iii$) a large antichain. In the first two cases, we can obtain an independent set by selecting $k$ vertices with odd indices.
In the last case, it is enough to select $k$ vertices from the found antichain.

\begin{algorithm}[t]
\caption{\algbac{U, d, k}, an $\frac{1}{6k}$-approx. algorithm for AMMD.}\label{algo:bac} 
\begin{algorithmic}[1]
\Require space $(U, d)$ and integer parameter $k\geq 2$.
\ForAll {$R \in \{d(i,j)>0 \mid i, j \in U,  i \neq j\}$} \label{line:loop}
	\State ${U}' \define \algclust{U, d, R}$. 
	\State $\algextract{U', d, R/(2k), k}$.
\EndFor 

\Ensure the set returned by $\algextract{}$ with the largest $\diver{}$ value.
\end{algorithmic}
\end{algorithm}

\begin{algorithm}[t]
\caption{\algextract{U, d, \sigma, k}, subroutine for extracting a candidate set.}\label{algo:extract} 
\begin{algorithmic}[1]
\Require space $(U, d)$, threshold $\sigma$, and integer parameter $k\geq 2$.
    \State Create $G = (U,A)$, with $ij \in A \Leftrightarrow d(i,j)< \sigma$.\label{line:create_G}
    \If{$G$ contains a cycle}
        \State $C \define$ chordless cycle in $G$. \label{line:cycle}
    \EndIf
    \State $G_c \define$ the condensation of $G$.
    \State $M \define$ maximum antichain of $G_c$.
    \State $L \define$ shortest path of length $2k-1$ in $G_c$ or longest found. \label{line:path}
    \If{$C$ exists and $|C| \geq 2|M|-1$ and $|C| \geq |L|$}
        \State $I \define$ points with odd indices from $C$.
    \ElsIf{$2|M|-1 \geq |L|$} 
        \State $I \define$ points in $G$ corresponding to points in $M$.
    \Else{} 
        \State $I \define$ points in $G$ corr. to points with odd indices in $L$.
    \EndIf
\Ensure greedily selected $k$ points from the set $I$, if found.
\end{algorithmic}
\end{algorithm}

Next, we will prove the approximation guarantee. 
First, we need Lemma~\ref{lem:cycles}, which states that there cannot exist small cycles in $G$.

\begin{lemma}
\label{lem:cycles}
For any $R > 0$, any cycle $C$ in the digraph $G$ constructed in $\algextract{U', d, \frac{R}{2k}, k}$ (see Alg.~\ref{algo:extract})
has at least $2k + 2$ distinct vertices.
\end{lemma}
\begin{proof}
Suppose $G$ contains a cycle $C = (v_1,\ldots,v_{\ell},v_{1})$ of length $\ell$. 
Since $C$ is a cycle in $G$, it holds that $d(v_{\ell},v_1) < \frac{R}{2k}$, by definition of $G$.
On the other hand, as ${U}'$ is the output of \algclust{}, Corollary~\ref{cor:clusterphase} states that $\dmax(v_1,v_{\ell}) \geq R$. This implies that $d(v_1,v_{\ell}) \geq R$. Now the triangle inequality for $d(v_1,v_{\ell})$ along the edges of cycle $C$ implies
\[
	R \leq d(v_1,v_{\ell}) \leq \sum_{i = 1}^{\ell - 1} d(v_i, v_{i + 1}) < (\ell - 1) \frac{R}{2k}. 
\]
Solving for $\ell$ leads to $\ell > 2k + 1$, which proves the claim.
\end{proof}

\begin{theorem}
\label{thm:approx}
\algbac{U, d, k} is an $\frac{1}{6k}$-approximation to AMMD. 
\end{theorem}

\begin{proof}
Line~\ref{line:loop} in Algorithm~\ref{algo:bac} iterates over all unique distances, and one of them is equal to $R'$ as defined in Eq.~\ref{def:R'}.
We will show that for $R \leq R'$, the digraph $G$, constructed in line~\ref{line:create_G} in Algorithm~\ref{algo:extract}, has an antichain $M$ of size $|M| \geq k$, or there exists either a shortest path
with no backward edges or a chordless cycle from which we can select $k$ independent vertices. An independent set $I$ of size $\abs{I} \geq k$ in graph $G$ then yields a solution with a diversity score of $\diver{I} \geq \frac{R}{2k}$, which for $R = R'$ is $\diver{I} \geq\frac{R'}{2k} \geq \frac{R^*}{6k}$ proving the theorem. 

Note that since the nodes in an antichain have no paths connecting them, it suffices to look for an antichain in the condensation $G_c$, whose vertices are the strongly connected components of $G$. If there exists an antichain $M$ of size $|M| \geq k$, we are done as the nodes in an antichain are independent. 

Consider the case where the maximum antichain $M$ has size $|M|<k$ and assume $G$ is a DAG. Note that when $G$ is a DAG the condensation $G_c$ is equivalent to $G$.

Corollary~\ref{cor:clusterphase} states that if $R \leq R'$ then ${U}'$ contains a subset
$S \subseteq {U}'$ for which $|S| = k$ and $\diver{S} \geq R'$. Then there must be a path in $G$ between some pair of distinct points in $S$. Otherwise, $S$ is an antichain of size $k$.

Let this pair of nodes be $x, y \in S, x \neq y$, with a path $(x = v_1, \ldots, v_\ell = y)$ from $x$ to $y$ in $G$.
Then the triangle inequality implies
\[
	R' \leq \diver{S} \leq d(x, y) \leq \sum_{i=1}^{\ell-1} d(v_i, v_{i + 1}) < (\ell - 1) \frac{R}{2k} \leq (\ell - 1) \frac{R'}{2k}.
\]
Therefore, $\ell$ must be at least $2k + 2$ meaning any path between $x$ and $y$ must have at least $2k + 2$ vertices. Hence, there is a shortest path of length $2k + 2$ while a shortest path $L$ of length $2k - 1$ is sufficient.
There cannot be any shortcut edges in $L$, since $L$ is a shortest path. Nor can there be any backward edges in $L$, since $G$ is a DAG.
Consequently, elements in $L$ with odd indices form an independent set $I$ of size $k$.

Finally, assume that $G$ is not a DAG. Then
there is a chordless cycle $C$. Lemma~\ref{lem:cycles} guarantees that $C$ has at least $2k + 2$ elements.
Then, elements in $C$ with odd indices form an independent set $I$ of size $\abs{I} \geq k+1 > k$.
\end{proof}


%%%%%%%%%%%%%%%%%%%%%%%%%%%%%%
%%%%%%%%%%%%%%%%%%%%%%%%%%%%%%
\ptitle{Time complexity of Algorithm~\ref{algo:bac}}
The iteration in line~\ref{line:loop} is over at most $\bigO(n^2)$ possible $R$
values.
%The digraph $G$ might be dense, so we assume its number of edges is $\bigO(n^2)$.
Both detecting a cycle in $G$ and extracting the chordless
cycle from it (line~\ref{line:cycle} of Algorithm~\ref{algo:extract}) take $\bigO(n^2)$ time. Computing the
maximum antichain can be done in $\bigO(n^{2 + o(1)} \log n)$ time
(Proposition~\ref{caceres}).

To compute the shortest path we can use the following approach:
Let $D = A + I$, where $A$ is the adjacency matrix of $G$,
and $I$ is the identity matrix. Then $D^\ell_{ij} > 0$ if and only if
there is a path of at most length $\ell$ from $i$ to $j$.
Consequently, there is a shortest path from $i$ to $j$ of length $2k - 1$ if and only if
$D^{2k-1}_{ij} > 0$ and $D^{2k-2}_{ij} = 0$. We can compute the necessary matrices
in $\bigO(n^{\omega}\log k)$ time, where $\omega < 2.373$ is the matrix multiplication exponent~\cite{alman2021refined}. Once $i$ is found, we use Dijkstra's algorithm to recover the path in $\bigO(n^2)$ time.

Overall we have a worst-case time complexity of $\bigO(n^{2+\omega} \log k)$.

Note that in practice, we do not use the matrix multiplication method. Instead,
we compute a shortest path tree from every node. This leads to a slower theoretical time but the algorithms
are still practical as demonstrated in the experiments.



%%%%%%%%%%%%%%%%%%%%%%%%%%%%%%%%%%%%%%%
\subsection{Practical improvements}
\label{sec:speeding}
%%%%%%%%%%%%%%%%%%%%%%%%%%%%%%%%%%%%%%%
We discuss several modifications, which speed up \algbac{}, and/or might improve the solution quality in practice. 

\ptitle{Algorithm \algbacb{}} The first modification to \algbac{}, given in Algorithm~\ref{algo:bacb}, is
aimed at improving solution quality, at the expense of a slightly larger
running time. We will call this modified algorithm \algbacb{}. Note that the $\frac{1}{6k}$-approximation guarantee comes from the fact
that we add an edge $ij$ to $G$ whenever $d(i, j) < \frac{R}{2k}$. If we can increase
this cutoff to, say $R \times \alpha$, \emph{and} still find a feasible set, then the found set $S$ 
is guaranteed to have $\diver{S} \geq R\alpha$, that is we will obtain an $\alpha$-approximation.


\begin{algorithm}[t]
\caption{\algbacb{U, d, k}, an $\frac{1}{6k}$-approx. algorithm for AMMD.}\label{algo:bacb} 
\begin{algorithmic}[1]
\Require space $(U, d)$ and integer parameter $k\geq 2$.
\State $R_1 < \ldots < R_{m} \define $ all unique positive distances sorted.
\For{every $i = 1, \ldots, m$} 
	\State ${U}' \define \algclust{U, d, R_i}$. 
	\If {$\algextract{U', d, R_i /(2k), k}$ exists}
		\State $a \define \min\set{s \mid R_i / (2k) < R_s}$, $b \define i$.
		\While {$a \leq b$}
			\State $t \define \floor{\frac{a+b}{2}}$.
			\If {$\algextract{U', d, R_t, k}$ exists}
				$a \define t + 1$
			\Else{}
				$b \define t - 1$
			\EndIf
		\EndWhile
	\EndIf
\EndFor 

\Ensure the set returned by $\algextract{}$ with the largest $\diver{}$ value.
\end{algorithmic}
\end{algorithm}


For every $R$ in the iteration of \algbacb{}
that gives a feasible solution for the threshold $\frac{R}{2k}$, we will try to
improve the solution value by searching for a cutoff value larger than
$\frac{R}{2k}$ when constructing the graph $G$ (line~\ref{line:create_G} of Algorithm~\ref{algo:extract}).
If \algbacb{} is unable to find a feasible solution for a
certain $R$, then we continue
iterating to the next $R$. 

To this end, if \algbacb{} has found a feasible set for some $R$, we use the binary search
to search for larger cutoffs in $[\frac{R}{2k},R]$.
Note that only when we use the cutoff $
\frac{R}{6k}$ are we theoretically guaranteed that we can extract $k$
independent points. Nonetheless, \algbacb{} will attempt to do this for larger
cutoffs as well. The binary search requires $\bigO(\log n)$ tests for
a single $R$ since we can assume that the cutoff is one of the distances. Hence,
the computational complexity of \algbacb{} is in $\bigO(n^{2+\omega} \log k \log
n)$.


\ptitle{Algorithm \algbacf{}} This algorithm speeds up \algbac{} while at the same time attempting to improve solution quality. Similarly to Section~\ref{sec:approxnk2}, we can replace the loop of Algorithm~\ref{algo:bac} (line~\ref{line:loop}) with a binary search in an attempt to find a maximal $R$ for which we find a feasible solution. This reduces the iterations from $\bigO(n^2)$ to $\bigO(\log n)$.

\begin{algorithm}[t]
\caption{\algbacf{U, d, k}, an $\frac{1}{6k}$-approx. algorithm for AMMD.}\label{algo:bacf} 
\begin{algorithmic}[1]
\Require space $(U, d)$ and integer parameter $k\geq 2$.
\State $R_1 < \ldots < R_{m} \define $ all unique positive distances sorted.
\State $a \define 1$, $b \define m$
\While {$a \leq b$}
    \State $t \define \floor{\frac{a+b}{2}}$.
	\State ${U}' \define \algclust{U, d, R_t}$. 
	\If {$\algextract{U', d, R_t / (2k), k}$ exists}
		$a \define t + 1$
	\Else{}
		$b \define t - 1$
	\EndIf
\EndWhile
\State $i \define a-1$.
\State ${U}' \define \algclust{U, d, R_i}$. 

\State $a \define \min\set{s \mid R_i / (2k) < R_s}$, $b \define i$.
\While {$a \leq b$}
    \State $t \define \floor{\frac{a+b}{2}}$.
	\If {$\algextract{U', d, R_t, k}$ exists}
		$a \define t + 1$
	\Else{}
		$b \define t - 1$
	\EndIf
\EndWhile

\Ensure the set returned by $\algextract{}$ with the largest $\diver{}$ value.
\end{algorithmic}
\end{algorithm}


Note that unlike in Section~\ref{sec:approxnk2}, this binary search might not find the globally largest $R$ value for which we can extract a feasible solution.
This is because there may be distance values $R_i > R_j > R'$ such that $R_i$ yields a feasible solution while $R_j$ does not.

However, we can still sort the
unique distances and use binary search to find
$R_{j}$ with a feasible solution, say $S_j$, such that the next value
$R_{j+1} > R_{j}$ does not yield a feasible solution.

Moreover, any $R \leq R'$ yields a feasible solution. This implies $R_{j} \geq R'$ and we get the same guarantee as \algbac{}, because
\[
	\diver{S_{j}} \geq \frac{R_j}{2k} \geq \frac{R'}{2k} \geq \frac{R^*}{6k}.
\]

Similarly to \algbacb{}, \algbacf{} then attempts to improve the cutoff
value for constructing $G$ by again binary searching for an improved cutoff
in the interval $[\frac{R_j}{2k},R_j]$ for which \algextract{} still finds a feasible
solution.
In the worst case, this adds
another $\bigO(\log n)$ iterations, which does not change the asymptotic
running time of the algorithm.
The running time for solving $\bigO(\log n)$ MA instances, as given by Proposition~\ref{prop:multima},
is dominated by the time needed to check for shortest paths of length $2k-1$.
In summary, the running time of \algbacf{} is
\[
	\bigO(n^2 \log n + 2 n^{\omega} \log n \log k) = \bigO(n^{\omega} \log n \log k).
\]

This is a considerable improvement over the running time of
\algbac{}, making \algbacf{} orders of magnitude faster while retaining the same theoretical guarantees.


\ptitle{Further improvements}
As the graph $G$ constructed in Algorithm~\ref{algo:extract} may be split into multiple disconnected components, we improve the search for independent sets by looking for the cycles, antichains, and long shortest paths in each weakly connected component of $G$ separately. We then take the union of the independent sets for each component, aiming to have $k$ points in total.

In addition, rather than choosing the centers arbitrarily in Algorithm~\ref{algo:ballcover}, we start by picking one of the vertices with the largest $\dmin$ distance. This heuristic is similar to the approach for solving MMD~\citep{ravi1994heuristic}. For the subsequent iterations, we choose the point furthest from the current set of chosen centers, as in Algorithm~\ref{algo:greedydmin}.

Finally, in practice many of the unique distances $R_1, \ldots, R_m$ may result in the same clustering ${U}'$ in Algorithms~\ref{algo:bac} and~\ref{algo:bacb}. We avoid these duplicate computations by grouping the $R$ values that yield the same clustering.

\section{Experiments: Planning outperforms Heuristics}
\label{sec:experiment}

We begin our empirical demonstrations by showcasing the effectiveness of our planning framework on both synthetic and real datasets. We focus on the simplest planning algorithm, 1-step lookaheads (Algorithm~\ref{alg:complete}), and show that even basic planning can hold great promise. 
We illustrate our framework using two uncertainty quantification modules---GPs and 
\ensembles/ \ensembleplus. 

Throughout this section, we focus on evaluating the mean squared error of 
a regression model $\model$,  and develop adaptive policies that minimize uncertainty on $g(f)$ defined in~\eqref{eqn:l2-g-f}.
When GPs provide a valid model of uncertainty, 
our experiments show that our planning framework significantly outperforms other baselines. 
We further demonstrate that our conceptual framework extends to deep learning-based uncertainty quantification methods such as  \ensembleplus while highlighting computational challenges that need to be resolved in order to scale our ideas. 
For simplicity, we assume a naive predictor, i.e., $\psi(\cdot) \equiv 0$. However, we emphasize that this problem is just as complex as if we were using a sophisticated model $\psi(.)$. The performance gap between the algorithms 
primarily depends
on the level  of uncertainty in our prior beliefs.

To evaluate the performance of our algorithm, we benchmark it against several baselines. 
%Active learning baselines use an acquisition function $\ac$ to select points that have the highest   function value: $X\opt_t \in \argmax_{X \in \xpoolj{t}} \ac({X})$ at every step $t$. These methods may also need an UQ module, which we simply use the same UQ module as in our algorithm, and it  outputs $V(X)$ that measures the the uncertainty of each point $X \in \xpoolj{t}$.
Our first set of baselines are from active learning~\citep{AggarwalKoGuHaPh14}:
\\ % \noindent\textbf{Active Learning Heuristics:} 
\textbf{(1)} 
\textsf{Uncertainty Sampling (Static):}  In this approach, we query the samples for which the model is least certain about. Specifically, we estimate the variance of the latent output $f(X)$ for each $X \in \xpool$ using the UQ module and select the top-$K$ points with the highest uncertainty. \\
\textbf{(2)} \textsf{Uncertainty Sampling (Sequential):} This is a greedy heuristic that sequentially selects the points with the highest uncertainty within a batch, while updating the posterior beliefs using pseudo labels from the current posterior state. Unlike \textsf{Uncertainty Sampling (Static)}, this method takes into account the information gained from each point within batch, and hence tries to diversify the selected points within a batch. 

 
We also compare our approach to the  \textbf{(3)} \textsf{Random Sampling}, which selects each batch uniformly at random from the pool. Additionally, we compare solving the planning problem using  \textsf{REINFORCE}-based policy gradients with   $\mathsf{Smoothed\text{-}Autodiff}$ policy gradients.\footnote{Our code repository is available at
  \url{https://github.com/namkoong-lab/adaptive-labeling}.}
%Detailed experimental setups are provided in Section \ref{sec:details-experiments}.

%We repeat all experiments with 10 random seeds.




\begin{figure}[t]
\centering
\begin{minipage}[b]{0.49\textwidth}
\centering
\includegraphics[width=\textwidth, height=5cm]{figures/original_scale/Var_of_l_2_loss.pdf}
\caption{(Synthetic data) Variance of mean squared loss evaluated through the posterior belief $\mu_t$ at each horizon $t$. This is the objective that policy gradient methods like \textsf{REINFORCE} and $\ouralgo$ optimizes. 1-step lookaheads are surprisingly effective even in long horizons.}
\label{fig:var-l2-sim}
\end{minipage}
\hfill
\begin{minipage}[b]{0.49\textwidth}
\centering \includegraphics[width=\textwidth, height=5cm]{figures/original_scale/Error_of_estimated_model_l_2_loss.pdf}
\caption{(Synthetic data) Error between MSE calculated based on collected data $\mc{D}^{0:T}$ vs. population oracle MSE over $\mc{D}_{\rm eval} \sim P_X$. Reducing uncertainty over posteriors directly leads to better OOD evaluations. 1-step lookaheads significantly outperform active learning heuristics in small horizons.}
\label{fig:mean-l2-sim}
\end{minipage}
%\caption{Simulated data for GPs}
%\label{fig:both_plots}
\end{figure}

\subsection{Planning with Gaussian processes}
\label{sec:experiment-plan-GP}
We now briefly describe the data generation process for the GP experiments,  deferring a more detailed discussion of the dataset generation to Section~\ref{sec:details-experiments}. 
We use both the synthetic data and the real data to test our methodology.
For the \emph{simulated data},  we construct a setting where the general population is distributed across \emph{51 non-overlapping clusters} while the initial labeled data $\dtrain$ just comes from one cluster. In contrast, both $\dpool \defeq (\xpool,\ypool),\deval \defeq (\xeval,\yeval)$ are generated   from all the clusters. 
We begin with a low-dimensional scenario, generating a one-dimensional regression setting using a GP. %Gaussian Process (GP).
Although the data-generating process is not known to the algorithms,  we assume that the GP hyperparameters are known to all the algorithms
to ensure fair comparisons. This can be viewed as a setting where our prior is well-specified, allowing us to isolate the effects
of different policy optimization approaches
 without any concerns about the misspecified priors. We select $10$ batches, each of size $K=5$ across $T = 10$ time horizons.

To examine the robustness of our method against the distributional assumptions made  in the simulated case, we then move to a real dataset where the correct prior is not known. We simulate selection bias from the eICU dataset~\citep{PollardJoRaCeMaBa18}, which contains real-world patient data with in-hospital mortality outcomes. 
We conduct a $k$-means clustering to generate 51 clusters and then select data from those clusters. We view this to be a credible replication of practice, as severe distribution shifts are common due to selection bias in clinical labels.  To convert the binary mortality labels into a regression setting, we train a  random forest classifier and fit a GP on predicted scores, which serves as the UQ module for all the algorithms. As before, the task is to select 10 batches, each consisting of 5 samples, across 10 time horizons.

 In Figures~\ref{fig:var-l2-sim} and~\ref{fig:mean-l2-sim}, we present results for the simulated data. 
Figure~\ref{fig:var-l2-sim} shows the variance of $\ell_2$ loss, and Figure~\ref{fig:mean-l2-sim} presents the error in the estimated $\ell_2$ loss using $\mu_t$ (relative to true $\ell_2$ loss, that is unknown to the algorithm). 
As we can see from these plots, our method one-step lookahead  gives substantial improvements  over active learning baselines and random sampling. In addition,
compared to the one-step lookahead planning approach using \textsf{REINFORCE}-based policy gradients, 
we observe that $\mathsf{Smoothed\text{-}Autodiff}$-based policy gradients provide significantly more robust performance over all horizons.

In Figures~\ref{fig:var-l2-real}~and~\ref{fig:mean-l2-real}, we observe similar findings on the eICU data. We see that planning policies (\textsf{REINFORCE} and $\mathsf{Smoothed\text{-}Autodiff}$) consistently outperform other heuristics by a large margin.  Active learning baselines perform poorly in these small-horizon batched problems and can sometimes be even worse than the random search baselines.  Overall, our results show the importance of careful planning in adaptive labeling for reliable model evaluation. 

We offer some intuition as to why one-step lookahead planning may outperform other heuristic algorithms. 
 First,  \textsf{Uncertainty sampling (Static)} while myopically selects the
 top-$K$ inputs with the highest uncertainty, it fails to consider 
the overlap in information content among the ``best” instances; see \citep{AggarwalKoGuHaPh14} for more details. 
In other words,  it might acquire points from the same region with high uncertainty while failing to induce diversity among the batch.
Although \textsf{Uncertainty Sampling (Sequential)} somewhat addresses the issue of information overlap, a significant drawback of 
this algorithm
is the disconnect between the objective we aim to optimize and the algorithm. For example, it might sample from a region with high uncertainty but very low density. 

\begin{figure}[t]
\centering
\begin{minipage}[b]{0.48\textwidth}
\centering
\includegraphics[width=\textwidth, height=5cm]{figures/original_scale/Var_of_l_2_loss_real.pdf}
\caption{(Real-world eICU data) Variance of mean squared loss evaluated through the posterior belief $\mu_t$ at each horizon $t$. Even 1-step lookaheads are extremely effective planners, and auto-differentiation-based pathwise policy gradients provide a reliable optimization algorithm based on low-variance gradient estimates.}
\label{fig:var-l2-real}
\end{minipage}
\hfill
\begin{minipage}[b]{0.48\textwidth}
\centering \includegraphics[width=\textwidth, height=5cm]{figures/original_scale/Error_of_estimated_model_l_2_loss_real.pdf}
\caption{(Real-world eICU data) Error between MSE calculated based on collected data $\mc{D}^{0:T}$ vs. population oracle MSE over $\mc{D}_{\rm eval} \sim P_X$. Reducing uncertainty over posteriors directly leads to better OOD evaluations. Our method significantly outperforms active learning-based heuristics, and random sampling.}
\label{fig:mean-l2-real}
\end{minipage}
%\caption{Real data for GPs}
\end{figure}
 
%\vspace{-1.5cm}
% \begin{wrapfigure}{r}{.32\columnwidth}
%   \vspace{-.5cm} 
%   \centering
% \includegraphics[scale=.29]{figures/Var of l2l_2 loss.pdf}
%   \vspace{-0.2cm}
%   \caption{Results of GP}
% \label{fig:var-l2-gp}
%   \vspace{-0.1cm}
% \end{wrapfigure}


% Attempts have been made  in the past to address these  drawbacks heuristically  (see \citep{AggarwalKoGuHaPh14}). We give a unified computational framework while approaching the problem in a more principled manner and solving it more optimally.




\subsection{Planning with  neural network-based uncertainty quantification methods ($\ensembleplus$)}


We now provide a proof-of-concept that shows the generalizability of our conceptual framework  to the deep learning-based UQ modules, specifically focusing on $\ensembleplus$ due to their previously observed superior performance~\citep{OsbandWenAsDwIbLuRo23}. Recall that implementing our framework with deep learning-based UQ modules  requires us to retrain the model across multiple possible random actions $\bm{a}(\theta)$ sampled from the current policy $\pi_\theta$.
This requires significant computational resources, in sharp contrast to the GPs where the posteriors are in closed form and can be readily updated and differentiated. 

Due to the computational constraints, we test $\ensembleplus$ on a toy setting to demonstrate the generalizability of our framework. We consider a setting where the general population consists of four clusters, while the initial labeled data only comes from one cluster. Again we generate data using GPs.  The task is to select a batch of 2 points in one horizon. We detail the $\ensembleplus$ architecture in Section \ref{sec:details-experiments}, and we assume prior uncertainty to be large (depends on the scaling of the prior generating functions). 
The results are summarized in the Table~\ref{tab:UQ_ensemble}.

% \begin{table}[H]
% \vspace{-10pt}
% \caption{Performance under \ensembleplus as UQ module}
%     \centering
%     \begin{tabular}{|m{3cm}|m{2.5cm}|m{2cm}|} 
%     \hline
%       Algorithm   & Variance of $\loss_2$ loss estimate & Error of $\loss_2$ loss estimate  \\ \hline Random Sampling 
%          & $1710.9 \pm 1352.1$ & $8.67\pm6.62$ 
%       \\ \hline \ouralgo & $1.30 \pm 0.68$ & $0.91\pm0.25$ \\ \hline
%     \end{tabular}
%     \label{tab:UQ_ensemble}
%     %\vspace{-10pt}
% \end{table}




\begin{table}[h]
\vspace{-10pt}
\caption{Performance under \ensembleplus as the UQ module}
\centering
\begin{tabular}{|l|l|l|}
\hline
Algorithm   & Variance of $\loss_2$ loss estimate & Error of $\loss_2$ loss estimate  \\
\hline
\textsf{Random sampling} & 7129.8 $\pm$ 1027.0 & 136.2 $\pm$ 8.28 \\ \hline
\textsf{Uncertainty sampling (Static)} & 10852 $\pm$ 0.0 & 162.156 $\pm$ 0.0 \\ \hline
\textsf{Uncertainty sampling (Sequential)} & 8585.5 $\pm$ 898.9 & 144 $\pm$ 6.93 \\ \hline
\textsf{REINFORCE} & 1697.1 $\pm$ 0.0 & 45.27 $\pm$ 0.0 \\ \hline
\ouralgo & 1697.1 $\pm$ 0.0 & 45.27 $\pm$ 0.0 \\ \hline
\end{tabular}
%\caption{Comparison of different algorithms based on variance   and   error in $\ell_2$ loss estimation with Ensemble $+$ as the UQ module. Our results demonstrate that {\ouralgo} and REINFORCE outperformthe other active learning based heuristics, confirming the benefits of our MDP formulation for the adaptive labeling problem, as also demonstrated in Section 4.\\
%\footnotesize{Experimental details: We use Gaussian Processes as our data generating process, GP parameters are the same as in Section D.3.  The task is to select a batch of 2 points along one horizon.The marginal distribution $p_X$ has 4 \textit{non-overlapping} clusters. Initial data comes from one cluster, while pool and evaluation points comes from all the clusters. We have $20$ initial labeled data points, $10$ pool points, and $252$ evaluation points.  Training procedures are similar to the one in Section D.3.} }
\label{tab:UQ_ensemble}
\end{table}



% We faced  issues in scaling up these experiments which will be our focus in the future. 





% \begin{itemize}
%     \item Posteriors should be consistent. Two dimensions: even with less training,  
%     \item the inference should be  fast enough
% \end{itemize}


% Potential research directions for uncertainty quantification

% In this section we consider a simple setting We consider a simpler setting and 


% For synthetic dataset generation, we use ...... For real datasets, we use ...... We compare our methodolgy to several baselines ()    This Section is structured as follows:
% \begin{itemize}
%     \item \textbf{GPs, square loss objective} (Section \ref{}): 
%     %the broad aim of the experiments  in this section is to isolate the performance of our methodology without any concerns for the inefficiencies induced due to a mis-specified prior or imperfect posterior inference. To accomplish this we generate synthetic datasets using GPs (detailed later). We use the well specified prior (GPs - with same hyperparameter setting) as our UQ module.   
%      As GPs provide differentaible posterior inference - any errors induced due to imperfect posterior updates are also isolated. We note that under this setting
%      \item In Section\ref{} we demonstrate why our methodology performs better than other baselines - by devising various synthetic experiments ()
%     \item  \textbf{UQ Benchmarking }(Section \ref{}): Before diving into the experiments using $\ensembleplus$ and ENNs,  we showcase our benchmarking experiments in Section \ref{}. We use real datasets We observe that ENNs perform better
%      \item \textbf{Ensemble $+$}, objective: recall, accuracy
%     \item \textbf{ENN}, objective: recall, accuracy
% \end{itemize}




% In Section {}, we test 
% \subsection{Experimental details}

% \begin{itemize}
%     \item UQ methodologies - GPs, ENNs
%     \item Objectives - Recall,  ATE
%     \item Datasets - ATE-synthetic datasets, Recall-synthetic, real datasets
%     \item Baselines - 
%     \begin{itemize}
%         \item Random sampling
%         \item Active learning - Uncertainty based sampling - In regression setting almost all of the 
%         \item Myopic greedy - Greedy Batch based sampling
%         \item Policy Gradient
%     \end{itemize}
    
% \end{itemize}

% \subsection{Experiments}
%     \begin{itemize}
%     \item GPs with square loss
%     \item Benchmarking ENN
%         \item ENNs with ATE
%         \item ENNs with Recall
%     \end{itemize}

% \subsection{Benefits over other algorithms - intuition and experiments}

%Active learning - Myopic greedy / Don't rely on the objective rather some entropy version.


%%% Local Variables:
%%% mode: latex
%%% TeX-master: "main"
%%% End:

\section{Conclusion Remarks}
This work proposes a RBG graph model for disease spreading via hubs. We study the joint effect of the agent density, hub density, and connection function. The existence of a critical hub density depends only on the boundedness of the support of the connection function, which relates to curbing the traveling distance of individuals. When it comes to dispersion, both the degree distribution and the percolation threshold suggest that increasing dispersion helps spread the disease. The percolation properties of RBG graphs relate to unipartite graphs with modified connection functions. 
An interesting question in this direction is if and when the properties of the RBG graphs can be well represented by unipartite graphs with some modified connection functions. Our conjecture is that for independent connections between different pairs of agents, such representation is unlikely due to the oblivion of the local dependence (present in the RBG models). 
 Another direction is to consider hybrid models where agents may get infected either through common hubs or direct interactions between agents. The former infection mechanism is more centralized than the latter. 
%
\begin{acks}
This research is supported by the \grantsponsor{⟨malsome⟩}{Academy of Finland project MALSOME}{} (\grantnum[]{malsome}{343045}) and by the \grantsponsor{⟨hiit⟩}{Helsinki Institute for Information Technology (HIIT)}{}.
\end{acks}
%\clearpage

%%
%% The next two lines define the graphy style to be used, and
%% the bibliography file.
\bibliographystyle{ACM-Reference-Format}
\balance
\bibliography{maxmin_kdd}

%%
%% If your work has an appendix, this is the place to put it.

%\iffalse
\clearpage % this puts it on a new page
\appendix
\section{Additional Experiments}

\subsection{Additional Experiment Setups}
%\fi

\end{document}
\endinput

