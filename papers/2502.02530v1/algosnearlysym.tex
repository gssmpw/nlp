%%%%%%%%%%%%%%%%%%%%%%%%%%%%%%
\section{Nearly symmetric instances}
\label{sec:algosnearlysym}
%%%%%%%%%%%%%%%%%%%%%%%%%%%%%
Greedily selecting the next-furthest points until $k$ points are selected is a $\frac{1}{2}$-approximation for symmetric MMD \cite{tamir1991obnoxious,ravi1994heuristic}, but can perform arbitrarily badly
on asymmetric instances as shown by the example in Figure~\ref{fig:toyexample_greedy}.

Theorem~\ref{thm:epsisymmgreedy} generalizes this result to asymmetric instances.
It states that on asymmetric instances that are $\epsilon$-symmetric (see Section~\ref{sec:notation}), the greedy approach applied on the $\dmin$ distances yields a $\frac{1}{2+\epsilon}$-approximation, and this ratio is tight.
The proof of Theorem~\ref{thm:epsisymmgreedy} can be found in the Appendix.

\begin{theorem}
\label{thm:epsisymmgreedy}
For any $\epsilon \geq 0$, Algorithm~\ref{algo:greedydmin} is a $\frac{1}{2+\epsilon}$-approximation on $\epsilon$-symmetric instances and can be implemented to run in $\bigO(kn)$ time.
Additionally, there exist $\epsilon$-symmetric instances for which Algorithm~\ref{algo:greedydmin} cannot achieve a performance ratio better than $\frac{1}{2+\epsilon}$.  
\end{theorem}

\begin{algorithm}[t]
\caption{Greedy with $\dmin$-distances.}
\label{algo:greedydmin}
\begin{algorithmic}[1]
%\Require graph $G$. 
\Require space $(U, d)$ and integer parameter $k\geq 2$.
\State $v \define \text{arbitrary vertex from } U$.
\State $S \define \{v\}$.
\While {$|S|<k$}
	\State $v \define \text{arg} \max_{u \in U} \dmin (u,S)$.
	\State $S \define S \cup \{v\}$.
\EndWhile
\Ensure the set $S$. 
\end{algorithmic}
\end{algorithm}


