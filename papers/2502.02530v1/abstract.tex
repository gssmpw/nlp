\begin{abstract}
One of the most well-known and simplest models for diversity maximization is the Max-Min Diversification (MMD) model, which has been extensively studied in the data mining and database literature.
In this paper, we initiate the study of the Asymmetric Max-Min Diversification (AMMD) problem. The input is a positive integer $k$ and a complete digraph over $n$ vertices, together with a nonnegative distance function over the edges obeying the directed triangle inequality. The objective is to select a set of $k$ vertices, which maximizes the smallest pairwise distance between them.
AMMD reduces to the well-studied MMD problem in case the distances are symmetric, and has natural applications to query result diversification, web search, and facility location problems.
Although the MMD problem admits a simple $\frac{1}{2}$-approximation by greedily selecting the next-furthest point, this strategy fails for AMMD and it remained unclear how to design good approximation algorithms for AMMD. 

We propose a combinatorial $\frac{1}{6k}$-approximation algorithm for AMMD by leveraging connections with the Maximum Antichain problem. We discuss several ways of speeding up the algorithm and compare its performance against heuristic baselines on real-life and synthetic datasets.
\end{abstract}