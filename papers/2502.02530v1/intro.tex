\section{Introduction}
\label{sec:intro}
Diversity maximization is a fundamental problem
with natural applications to query result diversification \cite{drosou2012disc,deng2013complexity}, recommender systems \cite{castells2021novelty,kaminskas2016diversity, abbassi2013diversity}, information exposure in social networks \cite{matakos2020tell}, web search \cite{xin2006extracting, radlinski2006improving,divtopk,bhattacharya2011consideration}, feature selection \cite{zadeh2017scalable} and data summarization \cite{celis2018fair,mahabadi2023improved,CHANDRA2001438,zheng2017survey}.
In a typical diverse data selection problem, one is interested in finding a \emph{diverse} group of items within the data, meaning that the selected items should be highly dissimilar to each other.
Typically, one is interested in finding sets of fixed and small sizes.

\begin{figure}
\centering
\includegraphics{flights_fig.pdf}

\caption{The result of our algorithm \algbacf{} for $k=5$ on the \emph{Flights Delay} dataset \cite{FlightsDelay}. It shows 5 airports located in U.S. territory which require a large flight time between any two of them, averaged over all flights in 2015. The airports are all spread out over different U.S. territories (Puerto Rico, Guam, American Samoa) and/or states (Alaska, California).
\label{fig:introflightex}}

\end{figure}

A simple and popular model for diversity maximization is the Max-Min Diversification (MMD) model. In this model, the data is represented as a universe of points $U$ of size $|U|=n$, and we are given a nonnegative distance function $d: U \times U \to \mathbb{R}_{\geq 0}$ measuring the dissimilarity between points. If $d(u,v)$ is large, then $u$ and $v$ are highly dissimilar and vice versa. 
The space $(U,d)$ is a pseudometric\footnote{Contrary to a metric, a pseudometric allows $d(u,v)=0$ for distinct $u \neq v$.}, implying that for all $u,v,w \in U: d(u,u) = 0$, $d(u,v) = d(v,u)$ (symmetry), and $d(u,v) \leq d(u,w)+ d(w,v)$ (triangle inequality).
The \emph{diversity score} $\diver{S}$ of a set of points $S \subseteq U$ is then defined as the smallest distance between any two distinct points in $S$. Formally,
\begin{align}
\label{def:divscore}
\diver{S} = \min \limits_{\substack{u,v \in S \\ u \neq v}} d(u,v).
\end{align}
The Max-Min Diversification (MMD) problem asks, given an integer $k$, to find a set $S \subseteq U$ of size $|S| = k$ which maximizes $\diver{S}$.
The MMD problem has originally been studied in the operations research literature as the max-min dispersion problem or $k$-dispersion problem \cite{kuby1987programming,erkut1990discrete,tamir1991obnoxious,ravi1994heuristic}.
The problem naturally describes a facility location problem where the proximity of selected facilities is undesirable.
Facilities that are close to each other might be undesirable due to economic or safety reasons.
Examples include the placement of store warehouses, nuclear power plants, or ammunition depot storages.

This paper initiates the study of the \emph{asymmetric generalization} of the MMD problem, which will be referred to as the Asymmetric Max-Min Diversification problem (AMMD). AMMD allows for asymmetric distances $d(u,v) \neq d(v,u)$ for $u \neq v \in U$, while retaining all other properties of the MMD model such as the triangle inequality.
A formal problem statement is given by Problem~\ref{prob:ammd}.

There are two good reasons for studying AMMD.
First, it can be argued that many real-world similarity measures and distances are in fact not symmetric. Straightforward examples are one-way traffic roads, uphill/downhill traveling, one-sided friendships and asymmetric flight times between airports, among others~\cite{kunegis2013konect, snapnets, tsplib}. Hence, one should consider studying models incorporating this property.

Secondly, for many optimization problems, the asymmetric generalization requires different, often more complicated, approximation algorithms than their symmetric restrictions. 
Additionally, they might also be significantly harder to approximate. An example is the $\Omega(\log^* n)$-hardness result \cite{chuzhoy2005asymmetric} for asymmetric $k$-center,\!\footnote{The iterated logarithm $\log^* n$ is defined as the number of times the $\log$ function can be iteratively applied to $n$ until the result is less than or equal to $1$.} and the corresponding $\bigO(\log^* n)$-approximation algorithms \cite{archer2001two,panigrahy1998ano}. Another example is the asymmetric traveling salesman problem, for which only recently a constant-factor approximation has been discovered \cite{svensson2020constant}.

Designing approximation algorithms for AMMD is seemingly also more challenging than for MMD.
While MMD admits a greedy $\frac{1}{2}$-approximation by iteratively selecting the next-furthest point until a set of size $k$ is obtained \cite{tamir1991obnoxious,ravi1994heuristic},
this greedy strategy can perform arbitrarily bad on asymmetric instances. Figure~\ref{fig:toyexample_greedy} shows such an example. 


\begin{figure}
  \centering
  \begin{tikzpicture}
  \node[inner sep=0pt, circle] (u2) at (90:1cm) {$u_2$};
  \node[inner sep=0pt, circle] (u1) at (162:1cm) {$u_1$};
  \node[inner sep=0pt, circle] (u4) at (234:1cm) {$u_4$};
  \node[inner sep=0pt, circle] (u5) at (306:1cm) {$u_5$};
  \node[inner sep=0pt, circle] (u3) at (378:1cm) {$u_3$};

  \draw[->, >=latex, yafcolor5] (u1) edge[bend left = 10] (u4);
  \draw[->, >=latex, yafcolor5] (u1) edge[bend right = 10] (u5);

  \draw[->, >=latex, yafcolor5] (u2) edge[bend left = 10] (u4);
  \draw[->, >=latex, yafcolor5] (u2) edge[bend right = 10] (u5);

  \draw[->, >=latex, yafcolor5] (u3) edge[bend left = 10] (u4);
  \draw[->, >=latex, yafcolor5] (u3) edge[bend right = 10] (u5);
  \end{tikzpicture}
  \caption{An input graph to AMMD on which the known $\frac{1}{2}$-approximations for symmetric MMD perform poorly. A blue directed edge $(u,v)$ indicates that the distance $d(u,v)$ is zero. All other distances (edges not drawn) are equal to some $R>0$. These distances satisfy the directed triangle inequality. For $k=3$ the optimal solution is $O =\{u_1,u_2,u_3\}$, with optimum $\diver{O}=R$. The greedy algorithm from \cite{tamir1991obnoxious} picks an arbitrary initial node. If $u_4$ or $u_5$ are selected we get a solution value of zero, regardless of how the remaining two nodes are selected. Similarly, the algorithm of \cite{ravi1994heuristic} initially selects a node pair with maximum distance. This could be $(u_4,u_5)$, since ties are broken arbitrarily, again resulting in a zero-valued solution.}
\label{fig:toyexample_greedy}
\end{figure} 

\ptitle{Results and techniques}
Throughout the paper we let $\omega < 2.373$ be the matrix multiplication exponent. We refer to \cite{alman2021refined} for the current best bound.

Our main result is the following theorem.
\begin{theorem}
Our algorithms \algbac{}, \algbacb{} and \algbacf{} from Section~\ref{sec:speeding} approximate AMMD within a factor of $\frac{1}{6k}$. Their worst-case time complexity is respectively $\bigO(n^{2+\omega} \log k)$, $\bigO(n^{2+\omega} \log k \log
n)$ and $\bigO(n^{\omega} \log k \log n )$.
\end{theorem} 

To the best of our knowledge, these are the first approximation algorithms for AMMD with a non-trivial guarantee. \algbac{} is the vanilla algorithm. \algbacb{} is a refined variant looking to improve solution quality, while \algbacf{} is a faster variant.

We also show that our practical implementation run fast on real-world and synthetic data and produces solutions with high diversity scores (see Section~\ref{sec:exps}). Figure~\ref{fig:introflightex} shows the output of \algbacf{} for $k=5$ on a real-world dataset consisting of U.S. domestic flight data by large air carriers in 2015 \cite{FlightsDelay}. Between the five airports, the shortest average flight time is from Adak Airport to Del Norte Country Airport. Flying from Mercedita Airport to Guam Int. Airport on average took 10\% longer than flying in the reverse direction. This is the largest ratio of bidirectional flying times among all routes between the five airports.

The high-level idea behind our algorithm is to create auxiliary graphs based on the distances in the AMMD instance. In these auxiliary graphs, an independent set of size $k$ will lead to good solutions. In order to efficiently extract the independent sets, we need to equip the auxiliary graphs with certain properties. We show that by first clustering the points in the AMMD instance, the graphs must contain a subgraph of a certain type (antichain, a large chordless cycle, or a long shortest path with no backward edges). We can search for all 3 subgraphs in polynomial time, and we can extract an independent set from them in polynomial time.

\ptitle{Roadmap} Section~\ref{sec:related} discusses related work on diversity maximization and the MMD problem in particular. Section~\ref{sec:notation} provides a formal problem statement and the necessary background literature on maximum antichains.
Section~\ref{sec:algosnearlysym} briefly discusses the performance of the greedy algorithm on instances that are nearly symmetric. Section~\ref{sec:approx} details our algorithms for AMMD. We present experimental evaluation in Section~\ref{sec:exps} and conclude the paper with remarks in Section~\ref{sec:conclusions}.
