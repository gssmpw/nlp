\section{Related Work}
Nowadays, Many IoT devices are constructed based on the low power lossy networks, e.g. Routing Protocol for Low Power and Lossy Networks (RPL), as it creates a great opportunity for many types of security attacks because these devices have constraints on their computational resources. Due to such a limitation of these networks, the traditional IDS approach could not adequately oversee and secure the IoT device traffic. Al Sawafi et al. \cite{al_sawafi2023hybrid} discussed in detail about the issue and suggested an IDS mechanism based on deep learning models, having both the supervised and semi-supervised classification algorithms to distinguish the category of network attacks regarding type of known and type of unseen attacks. They took sixteen researched attacks to save the data through packet sniffing; the results reveal that their deep learning-based model achieved 99\% accuracy and 98\% F1-score in identifying known attacks, which denotes deep learning plays a vital role in influencing IoT security positively.

Mahmud et al. \cite{mahmud2024optimized} utilized a large IoT dataset containing 820,834 data samples with 54 features to evaluate performance of various ML classifiers for network intrusion detection, which makes its contribution much more significant. The baseline dataset containing equal instances of normal and attack cases formed the backbone of their analyses. Similarly, their findings showed that the best accuracy of the Random Forest Classifier was 99.39\%. 

Zakariah et al. \cite{zakariah2023machine} introduced a novel intrusion detection model to secure IoT networks by addressing the limitations of current firewalls and encryption techniques to mitigate emerging threats in networks. They applied CNN and LSTM networks with attention mechanisms and complemented with adaptive synthetic sampling of imbalanced datasets. The hybrid model produced an accuracy of 89\%, and an F1 score of 90\%, compared with MLP baseline with an accuracy of 87\% and F1 score of 88\%.

Musleh et al. \cite{musleh2023intrusion} highlight that feature extraction crucially increases the efficiency of ML-based IDS for the IoT networks. In addition, the IEEE Dataport dataset is used in the paper to validate that by using the VGG-16 model combined with stacked ensemble classifiers, including a combining classifier based on KNN and sequential minimum optimization (SMO), can reach an accuracy of 98.3\%, while maintaining perfect precision.

Chaganti et al. \cite{chaganti2023deep} present an enhanced LSTM networks-based deep learning approach for intrusion detection to protect SDNL enabled IoT networks and this work boosts the detection and classification of attacks. It achieved a 97.7\% accuracy of classifying each multi class attack when applied to two datasets which incorporate network threat techniques that include SDNIoT, where the LSTM model is able to avoid DDoS, surveillance and other forms of network attacks.

Khan et al. \cite{khan2023hybrid} created a hybrid deep learning-based intrusion detection system that can detect intrusions of IoT systems through physical, network, and application levels using recurrent neural networks and gated recurrent units. An advanced deep learning model, optimized via the Adam and Adamax methods were used to evaluate the performance of the proposed model over ToN-IoT dataset.  The evaluation indicates high accuracy of 97\% and F1-score 96\%, for the Adamax optimizer.

Rose et al. \cite{rose2021intrusion} proposed a dataset and model to investigate the efficiency of network characterization and machine learning to safeguard IoT devices against cyber-attacks. Empirical data indicate that the proposed anomaly detection system obtains 98.35\% accuracy and 98.35\% false-positive alarms.

Mohy-Eddine et al. \cite{mohy-eddine2022ensemble} introduced a new IDS model to a more secure Industrial Internet of Things (IIoT); They studied some of the vulnerabilities that are standard in the IIoT environments. This approach uses Isolation Forest (IF) and Pearson Correlation Coefficient (PCC) in the feature engineering and employs Random Forest classifiers to improve the detection performance. As highlighted above, the SNCF-PCCIF was a promising classifier on the UNSW-NB15-v2 datasets equal to RF-PCCIF with 99.30\% accuracy.

% Mahmud et al. \cite{mahmud2024sdn} brought up a new machine-learning based method for the enhancement of network security using the UNSW-NB15 dataset. They experimented with multiple classifiers such as Gradient Boosting, Random Forest, Decision Tree, and AdaBoost, among others, where Gradient Boosting was found to be the best performing model. The deep learning model performed outstandingly well using the 3 metrics, achieving a 99.87\% accuracy, a 100\% recall, and an F1 score of 99.85\% which proves its effectiveness to classify normal and malicious traffic accurately. Random forest was another reliable model with 99.38\% accuracy.

With the purpose of addressing these problems, Potnurwar et al. \cite{potnurwar2023deep} emphasized a new model based on deep learning, hybrid feature selection, along with DFFNN and rule-based hybrid feature selection methods. The datasets used for the research include NSL-KDD and UNSW-NB15. For their best performing test using F1 scores the researchers were able to derive a true positive rate of very high 96\% in attack and 98\% in normal.