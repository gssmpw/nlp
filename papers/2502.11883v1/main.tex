\documentclass[sigconf]{acmart}

% basic
%\usepackage{color,xcolor}
\usepackage{color}
\usepackage{epsfig}
\usepackage{graphicx}
\usepackage{algorithm,algorithmic}
% \usepackage{algpseudocode}
%\usepackage{ulem}

% figure and table
\usepackage{adjustbox}
\usepackage{array}
\usepackage{booktabs}
\usepackage{colortbl}
\usepackage{float,wrapfig}
\usepackage{framed}
\usepackage{hhline}
\usepackage{multirow}
% \usepackage{subcaption} % issues a warning with CVPR/ICCV format
% \usepackage[font=small]{caption}
\usepackage[percent]{overpic}
%\usepackage{tikz} % conflict with ECCV format

% font and character
\usepackage{amsmath,amsfonts,amssymb}
% \let\proof\relax      % for ECCV llncs class
% \let\endproof\relax   % for ECCV llncs class
\usepackage{amsthm} 
\usepackage{bm}
\usepackage{nicefrac}
\usepackage{microtype}
\usepackage{contour}
\usepackage{courier}
%\usepackage{palatino}
%\usepackage{times}

% layout
\usepackage{changepage}
\usepackage{extramarks}
\usepackage{fancyhdr}
\usepackage{lastpage}
\usepackage{setspace}
\usepackage{soul}
\usepackage{xspace}
\usepackage{cuted}
\usepackage{fancybox}
\usepackage{afterpage}
%\usepackage{enumitem} % conflict with IEEE format
%\usepackage{titlesec} % conflict with ECCV format

% ref
% commenting these two out for this submission so it looks the same as RSS example
% \usepackage[breaklinks=true,colorlinks,backref=True]{hyperref}
% \hypersetup{colorlinks,linkcolor={black},citecolor={MSBlue},urlcolor={magenta}}
\usepackage{url}
\usepackage{quoting}
\usepackage{epigraph}

% misc
\usepackage{enumerate}
\usepackage{paralist,tabularx}
\usepackage{comment}
\usepackage{pdfpages}
% \usepackage[draft]{todonotes} % conflict with CVPR/ICCV/ECCV format



% \usepackage{todonotes}
% \usepackage{caption}
% \usepackage{subcaption}

\usepackage{pifont}% http://ctan.org/pkg/pifont

% extra symbols
\usepackage{MnSymbol}


\section{Problem Studied}\label{sec:def}
We first present Fixed-Radius Near Neighbor (FRNN) queries and then formalize Aggregation Queries over Nearest Neighbors (AQNNs) that build on them. We then state our problem.

\subsection{Nearest Neighbor Queries}\label{subsec:FRNN}
We build on generalized Fixed-Radius Near Neighbor (FRNN) queries \cite{FRNNSurvey}. Given a dataset \( D \), a query object \( q \), a radius \( r \), and a distance function \( dist \), a generalized FRNN query retrieves all nearest neighbors of \( q \) within radius \( r \). More formally:
\[
NN_D(q, r) = \{x \in D \mid dist(x, q) \leq r\},
\]
where \(x\) is any data point in \(D\) and \(dist(x, q)\) denotes the distance between them. We use \(|NN_D(q,r)|\) to denote the neighborhood size of \(q\). As shown in Fig. \ref{fig:framework}, given a radius \(r\) and a target patient \(q\), patients in the dotted circle are nearest neighbors, and the neighborhood size is 6.

\subsection{Aggregation Queries over Nearest Neighbors}\label{subsec:AQNN} 
Given an FRNN query object \(q\) in dataset \(D\), a radius \(r\), and an attribute \(\texttt{attr}\), an Aggregation Query over Nearest Neighbors (AQNN) is defined as:
\[ \text{agg}(NN_D(q,r)[\texttt{attr}]) \]
where agg is an aggregation function, such as $\mathtt{AVG}$, $\mathtt{SUM}$, and $\mathtt{PCT}$, and \(NN_D(q,r)[\texttt{attr}]\) denotes the bag of values of attribute \texttt{attr} of all FRNN results of \(q\) within radius \(r\). 
% \end{definition}

An AQNN expresses aggregation operations to capture key insights about the neighborhood of a query object. For example, \(\mathtt{AVG}\) can be used to reflect the average heart rate or systolic blood pressure of patients in the neighborhood, providing a measure of typical health conditions. \(\mathtt{SUM}\) is useful for assessing cumulative effects, such as the total cost of treatments in the neighborhood that instructs public policy in terms of health. Similarly, $\mathtt{PCT}$ can be used to find the proportion of patients in the neighborhood of a patient of interest, relative to the population in the dataset.
%\laks{Why is finding the total \#meds to NNs or the total treatment cost of everyone in the NN interesting?}

% \texttt{MIN} and \texttt{MAX} are not included in the aggregation functions because they only capture extreme values, which may not represent the typical characteristics of the nearest neighbors and are more sensitive to outliers. 
% \laks{AVG is also sensitive to outliers, but we still allow it. isn't the real reason we don't consider MIN/MAX because they are amenable to estimation via sampling?} We choose \texttt{PCT} instead of \texttt{COUNT} in order to provide a normalized measure that remains comparable across different neighborhood sizes. It allows for more consistent interpretation of relative popularity \cite{moore1989introduction}.


Fig. \ref{fig:framework} illustrates an example of an AQNN: ``\textit{Find the average systolic blood pressure of patients similar to an insomnia patient \(q\)}''. The aggregation function is \(\mathtt{AVG}\) and the target attribute of interest is systolic blood pressure. Exact query evaluation requires consulting physicians (or predicting embeddings by an expensive machine learning model) for all 500 patients in \(D\) and calculate \(q\)'s nearest neighbors wrt \(r\) \cite{DBLP:journals/isci/RodriguesGSBA21}. We refer to such highly accurate but computationally expensive models as \textit{oracle models}, denoted as \(O\), including deep learning models trained on domain-specific data or human expert annotations \cite{DBLP:conf/sigmod/LuCKC18}. Using oracle models is very expensive \cite{sze2017efficient, DujianPQA, DBLP:journals/pvldb/KangGBHZ20}. To address that, we seek an approximate solution by \textit{proxy models}, denoted as \(P\), that are at least one order of magnitude cheaper than oracle models. In the example, if consulting physicians for one patient incurs one cost unit, calling a cheap machine learning model instead incurs at most \(0.1\) cost unit. Once the similar patients are identified, their systolic blood pressure values are averaged and returned as  output. The use of a proxy model may reduce the accuracy of the neighborhood prediction and hence, we should judiciously call oracle and proxy models to minimize the error of aggregate results.

Note that the values of the target attribute \texttt{attr} are \textit{not} predicted but are instead known quantities.

\subsection{Problem Statement}
Given an AQNN, our goal is to return an approximate aggregate result by leveraging both oracle and proxy models while reducing error and cost.



\author{Coen van den Elsen}
\email{12744956@uva.nl}
\authornote{Equal contributions.}
\affiliation{%
  \institution{University of Amsterdam}
  \city{Amsterdam}
  \state{Noord-Holland}
  \country{The Netherlands}
}

\author{Francien Barkhof}
\email{12606626@uva.nl}
\authornotemark[1]
\affiliation{%
  \institution{University of Amsterdam}
  \city{Amsterdam}
  \state{Noord-Holland}
  \country{The Netherlands}
}

\author{Thijmen Nijdam}
\email{12994448@uva.nl}
\authornotemark[1]
\affiliation{%
  \institution{University of Amsterdam}
  \city{Amsterdam}
  \state{Noord-Holland}
  \country{The Netherlands}
}


\author{Simon Lupart}
\email{s.c.lupart@uva.nl}
\affiliation{%
  \institution{University of Amsterdam}
  \city{Amsterdam}
  \state{Noord-Holland}
  \country{The Netherlands}
}

\author{Mohammad Aliannejadi}
\email{m.aliannejadi@uva.nl}
\affiliation{%
  \institution{University of Amsterdam}
  \city{Amsterdam}
  \state{Noord-Holland}
  \country{The Netherlands}
}

\renewcommand{\shortauthors}{van den Elsen, Barkhof, Nijdam}

\newcommand{\ourshorttitle}{
Spectral-factorized Positive-definite Curvature Learning for NN Training
}
\newcommand{\ourtitle}{
Spectral-factorized Positive-definite Curvature Learning for NN Training
}
\icmltitlerunning{\ourshorttitle}

\twocolumn[

\icmltitle{\ourtitle}

%
%
%
%

%
%
%
%

%
%
%
%

\begin{icmlauthorlist}
\icmlauthor{Wu Lin}{vector}
\icmlauthor{Felix Dangel}{vector}
\icmlauthor{Runa Eschenhagen}{cambridge}
\icmlauthor{Juhan Bae}{vector,ut}
\icmlauthor{Richard E. Turner}{cambridge}
\icmlauthor{
Roger B. Grosse
}{vector,ut}
\end{icmlauthorlist}

%
\icmlaffiliation{vector}{Vector Institute, Canada}
\icmlaffiliation{cambridge}{Cambridge University, United Kingdom}
\icmlaffiliation{ut}{University of Toronto, Canada}

\icmlcorrespondingauthor{Wu Lin}{yorker.lin@gmail.com \vspace{-0.2cm}}

%
%
%
\icmlkeywords{Natural Gradient Descent, Second Order Method, Optimization, Deep Learning}

\vskip 0.3in
]

%

%
%
%
%
%

\printAffiliationsAndNotice{}  %
%

%
%
%
%



\title[FairDiverse: A Comprehensive Toolkit for Fair and Diverse Information Retrieval Algorithms]{FairDiverse: A Comprehensive Toolkit for Fair and Diverse \\Information Retrieval Algorithms}

\begin{document}

\begin{abstract}
In modern information retrieval (IR), achieving more than just accuracy is essential to sustaining a healthy ecosystem, especially when addressing fairness and diversity considerations. To meet these needs, various datasets, algorithms, and evaluation frameworks have been introduced. However, these algorithms are often tested across diverse metrics, datasets, and experimental setups, leading to inconsistencies and difficulties in direct comparisons.
This highlights the need for a comprehensive IR toolkit that enables standardized evaluation of fairness- and diversity-aware algorithms across different IR tasks. To address this challenge, we present \textbf{FairDiverse}, an open-source and standardized toolkit.
FairDiverse offers a framework for integrating fairness- and diversity-focused methods, including pre-processing, in-processing, and post-processing techniques, at different stages of the IR pipeline. The toolkit supports the evaluation of \textbf{28} fairness and diversity algorithms across \textbf{16} base models, covering two core IR tasks—search and recommendation—thereby establishing a comprehensive benchmark.
Moreover, FairDiverse is highly extensible, providing multiple APIs that empower IR researchers to swiftly develop and evaluate their own fairness- and diversity-aware models, while ensuring fair comparisons with existing baselines. The project is open-sourced and available on GitHub:~\url{https://github.com/XuChen0427/FairDiverse}.


\end{abstract}

\maketitle

\section{Introduction}

% In modern information retrieval (IR) systems, ensuring fair treatment and support for both head and tail products/users is crucial for maintaining a healthy ecosystem. Building upon this idea, previous research has emphasized the importance of developing fair and diverse information retrieval systems. ensuring fair treatment and support for both head and tail products/users is crucial for maintaining a healthy ecosystem. Building upon this idea, previous research has emphasized the importance of developing fair and diverse IR systems

Information retrieval (IR) tasks, such as search and recommendation, typically aim to select the information that meets user needs~\cite{IRbook, chowdhury2010introduction}. 
In modern IR, factors beyond the accuracy of information access, such as novelty, diversity, and fairness, are crucial for building a healthy ecosystem~\cite{Li_sigir24}. 
Among these factors, fairness and diversity have gained increasing attention in recent years~\cite{li2022fairness, santos2010exploiting, PM2_12_sigir}. Both aim to expose users to a broader range of information sources~\cite{santos2010exploiting, dang2012diversity} while also supporting diverse types of providers~\cite{fairrec, xu2023p}.


\begin{figure*}[t]  
    \centering    
    \includegraphics[width=\linewidth]{img/pipeline.pdf}
    \caption{Overall architecture of FairDiverse. We categorize fairness- and diversity-aware algorithms into pre-processing, in-processing, and post-processing stages, corresponding to data processing, model training, and result evaluation phases of IR.}
    \label{fig:pipline}
    \vspace*{-2mm}
\end{figure*}
%Aequitas


\begin{table}[t]
\centering
\setlength{\tabcolsep}{1.1pt}
\caption{Comparison between existing fairness- and diversity-aware toolkits. \ding{55} denotes that the feature is not supported, while \ding{51} indicates that the feature is supported. }
\label{tab:compare}
\begin{tabular}{l cccccc }
\toprule
Features & 
\rotatebox{65}{Recbole \cite{recbole2.0}} & 
\rotatebox{65}{FFB \cite{han2023ffb}} & 
\rotatebox{65}{Fairlearn \cite{bird2020fairlearn}} & 
\rotatebox{65}{AIF360 \cite{aif360-oct-2018}} & 
\rotatebox{65}{Aequitas \cite{jesus2024aequitas}} & 
\rotatebox{65}{\textbf{FairDiverse}} \\ 
\midrule
Recommendation & \ding{51} & \ding{55} & \ding{55} & \ding{55} & \ding{55} & \ding{51} \\
Search & \ding{55} & \ding{55} & \ding{55} & \ding{55} & \ding{55} & \ding{51} \\
\midrule
Pre-processing & \ding{55} & \ding{55} & \ding{51} & \ding{51} & \ding{51} & \ding{51} \\
In-processing & \ding{51} & \ding{51} & \ding{51} & \ding{51} & \ding{51} & \ding{51} \\
Post-processing & \ding{55} & \ding{55} & \ding{51} & \ding{51} & \ding{51} & \ding{51} \\
\midrule
Number of models & 4 & 6 & 6 & 15 & 10 & \textbf{28} \\ \bottomrule
\end{tabular}
\vspace*{-3mm}
\end{table}

To ensure fairness and diversity in IR systems, many fairness-aware~\cite{fairrec, xu2023p, TaxRank, SDRO, APR, FairNeg, cpfair, rus2024study} and diversity-aware algorithms~\cite{li2022fairness, santos2010exploiting, qin2020diversifying, yan2021diversification} have been designed as plugins or modules that can be integrated into various stages of the IR pipeline. However, fairness and diversity often suffer from a lack of unified definitions~\cite{li2022fairness, LLM4FairSurvey}. 
As a result, the evaluation of these algorithms in IR systems are based on different metrics, datasets, and evaluation settings (details are shown in Section~\ref{sec:related_work}). Hence, the performance of these algorithms cannot be compared consistently. Developing a unified, fair, and extensible toolkit for fairness and diversity is critically important and urgently needed to evaluate these algorithms consistently across IR tasks. Such a toolkit framework holds significant value for fostering a trustworthy IR community.

To create a unified and equitable evaluation, we introduce FairDiverse, an open-source standardized toolkit designed to assess fairness and diversity in IR systems.
First, FairDiverse offers detailed guidance on incorporating fairness- and diversity-aware algorithms throughout various stages of the IR process. These algorithms are categorized into pre-processing, in-processing, and post-processing methods, corresponding to data processing, model training, and result evaluation stages in different IR pipeline steps, respectively. 
Then, FairDiverse implements a wide range of fairness- and diversity-aware models (\textbf{28} models) tailored to \textbf{16} base models under two fundamental IR tasks: search and recommendation. 
It offers corresponding implementation code and systematically evaluates these algorithms using over ten accuracy, fairness, and diversity metrics, enabling the construction of a benchmark within FairDiverse.

%FairDiverse evaluates two fundamental tasks in IR: search and recommendation. Recognizing that unfairness and lack of diversity often arise during the data collection, model training, and evaluation stages of IR systems, FairDiverse categorizes \textbf{30+} fairness- and diversity-aware algorithms into three types: pre-processing, in-processing, and post-processing methods. It provides corresponding implementation code and systematically evaluates these algorithms using more than ten IR accuracy, fairness, and diversity metrics across ten datasets, offering a comprehensive evaluation framework and workflow.

In the literature, only a few open-source toolkits and libraries have been developed for fairness- and diversity-aware IR algorithms. Table~\ref{tab:compare} provides a comparison between these existing resources and the proposed FairDiverse, highlighting features such as supported IR tasks (recommendation and search), algorithm types (pre-processing, in-processing, and post-processing), and the number of implemented models. Other toolkit details are provided in Section~\ref{sec:related_work}. As shown in Table~\ref{tab:compare}, FairDiverse provides the largest number of models, offering extensive coverage of all types of fairness- and diversity-aware algorithms. Additionally, it supports major information retrieval (IR) tasks, including search and recommendation, making it a versatile and comprehensive toolkit.

FairDiverse is designed to be highly extensible, providing a range of flexible APIs that allow IR researchers to efficiently develop and integrate their own fairness- and diversity-aware IR models. This extensibility ensures that researchers can tailor the toolkit to their specific needs while maintaining consistency with established evaluation protocols. This makes it an invaluable resource for advancing fairness and diversity in IR systems.

%\section{Auxiliary-Variable Adaptive Control Barrier Functions}
\label{sec:AVBCBF}

In this section, we introduce Auxiliary-Variable Adaptive Control Barrier Functions (AVCBFs) for safety-critical control.
We start with a simple example to motivate the need for AVCBFs.

\subsection{Motivation Example: Simplified Adaptive Cruise Control}
\label{subsec:SACC-problem}

Consider a Simplified Adaptive Cruise Control (SACC) problem with the dynamics of ego vehicle expressed as 
\begin{small}
\begin{equation}
\label{eq:SACC-dynamics}
\underbrace{\begin{bmatrix}
\dot{z}(t) \\
\dot{v}(t) 
\end{bmatrix}}_{\dot{\boldsymbol{x}}(t)}  
=\underbrace{\begin{bmatrix}
 v_{p}-v(t) \\
 0
\end{bmatrix}}_{f(\boldsymbol{x}(t))} 
+ \underbrace{\begin{bmatrix}
  0 \\
  1 
\end{bmatrix}}_{g(\boldsymbol{x}(t))}u(t),
\end{equation}
\end{small}
where $v_{p}>0, v(t)>0$ denote the velocity of the lead vehicle (constant velocity) and ego vehicle, respectively, $z(t)$ denotes the distance between the lead and ego vehicle and $u(t)$ denotes the acceleration (control) of ego vehicle, subject to the control constraints
\begin{equation}
\label{eq:simple-control-constraint}
u_{min}\le u(t) \le u_{max}, \forall t \ge0,
\end{equation}
where $u_{min}<0$ and $u_{max}>0$ are the minimum and maximum control input, respectively.

 For safety, we require that $z(t)$ always be greater than or equal to the safety distance denoted by $l_{p}>0,$ i.e., $z(t)\ge l_{p}, \forall t \ge 0.$ Based on Def. \ref{def:HOCBF}, let $\psi_{0}(\boldsymbol{x})\coloneqq b(\boldsymbol{x})=z(t)-l_{p}.$ From \eqref{eq:sequence-f1} and \eqref{eq:sequence-set1}, since the relative degree of $b(\boldsymbol{x})$ is 2, we have
\begin{equation}
\label{eq:SACC-HOCBF-sequence}
\begin{split}
&\psi_{1}(\boldsymbol{x})\coloneqq v_{p}-v(t)+k_{1}\psi_{0}(\boldsymbol{x})\ge 0
,\\
&\psi_{2}(\boldsymbol{x})\coloneqq -u(t)+k_{1}(v_{p}-v(t))+k_{2}\psi_{1}(\boldsymbol{x})\ge 0,
\end{split}
\end{equation}
where $\alpha_{1}(\psi_{0}(\boldsymbol{x}))\coloneqq k_{1}\psi_{0}(\boldsymbol{x}), \alpha_{2}(\psi_{1}(\boldsymbol{x}))\coloneqq k_{2}\psi_{1}(\boldsymbol{x}), k_{1}>0, k_{2}>0.$ The constant class $\kappa$ coefficients $k_{1},k_{2}$ are always chosen small to equip ego vehicle with a conservative control strategy to keep it safe, i.e., smaller $k_{1},k_{2}$ make ego vehicle brake earlier (see \cite{xiao2021high}). Suppose we wish to minimize the energy cost as $\int_{0}^{T} u^{2}(t)dt.$ We can then formulate the cost in the QP with constraint $\psi_{2}(\boldsymbol{x})\ge0$ and control input constraint \eqref{eq:simple-control-constraint} to get the optimal controller for the SACC problem. However, the feasible set of input can easily become empty if $u(t)\le k_{1}(v_{p}-v(t))+k_{2}\psi_{1}(\boldsymbol{x})<u_{min}$,  which causes infeasibility of the optimization. In \cite{xiao2021adaptive}, the authors introduced penalty variables in front of class $\kappa$ functions to enhance the feasibility. This approach defines $\psi_{0}(\boldsymbol{x})\coloneqq b(\boldsymbol{x})=z(t)-l_{p}$ as PACBF and other constraints can be further defined as
\begin{equation}
\label{eq:SACC-PACBF-sequence}
\begin{split}
\psi_{1}(\boldsymbol{x},p_{1}(t))&\coloneqq v_{p}-v(t)+p_{1}(t)k_{1}\psi_{0}(\boldsymbol{x})\ge 0,\\
\psi_{2}(\boldsymbol{x},p_{1}(t),&\boldsymbol{\nu})\coloneqq \nu_{1}(t)k_{1}\psi_{0}(\boldsymbol{x})+p_{1}(t)k_{1}(v_{p}\\
-v(t))&-u(t)+\nu_{2}(t)k_{2}\psi_{1}(\boldsymbol{x},p_{1}(t))\ge 0,
\end{split}
\end{equation}
where $\nu_{1}(t)=\dot{p}_{1}(t),\nu_{2}(t)=p_{2}(t), p_{1}(t)\ge0,p_{2}(t)\ge0,\boldsymbol{\nu}=(\nu_{1}(t),\nu_{2}(t)).$ $p_{1}(t),p_{2}(t)$ are time-varying penalty variables, which alleviate the conservativeness of the control strategy and $\nu_{1}(t),\nu_{2}(t)$ are auxiliary inputs, which relax the constraints for $u(t)$ in $\psi_{2}(\boldsymbol{x},p_{1}(t),\boldsymbol{\nu})\ge0$ and \eqref{eq:simple-control-constraint}. However, in practice, we need to define several additional constraints to make PACBF valid as shown in Eqs. (24)-(27) in \cite{xiao2021adaptive}. First, we need to define HOCBFs ($b_{1}(p_{1}(t))=p_{1}(t),b_{2}(p_{2}(t))=p_{2}(t))$ based on Def. \ref{def:HOCBF} to ensure $p_{1}(t)\ge0,p_{2}(t)\ge0.$ Next we need to define HOCBF ($b_{3}(p_{1}(t))=p_{1,max}-p_{1}(t)$) to confine the value of $p_{1}(t)$ in the range $[0,p_{1,max}].$ We also need to define CLF ($V(p_{1}(t))=(p_{1}(t)-p_{1}^{\ast})^{2}$) based on Def. \ref{def:control-l-f} to keep $p_{1}(t)$ close to a small value $p_{1}^{\ast}.$ $b_{3}(p_{1}(t)), V(p_{1}(t))$ are necessary since $\psi_{0}(\boldsymbol{x},p_{1}(t))\coloneqq p_{1}(t)k_{1}\psi_{0}(\boldsymbol{x})$ in first constraint in \eqref{eq:SACC-PACBF-sequence} is not a class $\kappa$ function with respect to $\psi_{0}(\boldsymbol{x}),$ i.e., $p_{1}(t)k_{1}\psi_{0}(\boldsymbol{x})$ is not guaranteed to strictly increase since $\psi_{0}(\boldsymbol{x},p_{1}(t))$ is in fact a class $\kappa$ function with respect to $p_{1}(t)\psi_{0}(\boldsymbol{x})$, which is against Thm. \ref{thm:safety-guarantee}, therefore $\psi_{1}(\boldsymbol{x},p_{1}(t))\ge 0$ in \eqref{eq:SACC-PACBF-sequence} may not guarantee $\psi_{0}(\boldsymbol{x})\ge 0.$ This illustrates why we have to limit the growth of $p_{1}(t)$ by defining $b_{3}(p_{1}(t)),V(p_{1}(t)).$ However, the way to choose appropriate values for $p_{1,max},p_{1}^{\ast}$ is not straightforward. We can imagine as the relative degree of $b(\boldsymbol{x})$ gets higher, the number of additional constraints we should define also gets larger, which results in complicated parameter-tuning process. To address this issue, we introduce $a_{1}(t),a_{2}(t)$ in the form
\begin{small}
\begin{equation}
\label{eq:SACC-AVBCBF-sequence}
\begin{split}
\psi_{1}(\boldsymbol{x},\boldsymbol{a},\dot{a}_{1}(t))\coloneqq a_{2}(t)(\dot{\psi}_{0}(\boldsymbol{x},a_{1}(t))
+k_{1}\psi_{0}(\boldsymbol{x},a_{1}(t)))\ge 0,\\
\psi_{2}(\boldsymbol{x},\boldsymbol{a},\dot{a}_{1}(t),\boldsymbol{\nu})\coloneqq \nu_{2}(t)\frac{\psi_{1}(\boldsymbol{x},\boldsymbol{a},\dot{a}_{1}(t))}{a_{2}(t)} +a_{2}(t)(\nu_{1}(t)(z(t)\\
-l_{p})+2\dot{a}_{1}(t)(v_{p}-v(t))-a_{1}(t)u(t)+k_{1}\dot{\psi}_{0}(\boldsymbol{x},a_{1}(t)))\\
+k_{2}\psi_{1}(\boldsymbol{x},\boldsymbol{a},\dot{a}_{1}(t))\ge 0, 
\end{split}
\end{equation}
\end{small}
where $\psi_{0}(\boldsymbol{x},a_{1}(t))\coloneqq a_{1}(t)b (\boldsymbol{x})=a_{1}(t)(z(t)-l_{p}),\boldsymbol{\nu}=[\nu_{1}(t),\nu_{2}(t)]^{T}=[\ddot{a}_{1}(t),\dot{a}_{2}(t)]^{T},\boldsymbol{a}=[a_{1}(t),a_{2}(t)]^{T},$ $a_{1}(t),a_{2}(t)$ are time-varying auxiliary variables. Since $\psi_{0}(\boldsymbol{x},a_{1}(t))\ge0,\psi_{1}(\boldsymbol{x},\boldsymbol{a},\dot{a}_{1}(t))\ge 0$ will not be against $b(\boldsymbol{x})\ge 0,\dot{\psi}_{0}(\boldsymbol{x},a_{1}(t))
+k_{1}\psi_{0}(\boldsymbol{x},a_{1}(t))\ge 0$ iff $a_{1}(t)>0,a_{2}(t)>0,$ we need to define HOCBFs for auxiliary variables to make $a_{1}(t)>0,a_{2}(t)>0,$ which will be illustrated in Sec. \ref{sec:AVCBFs}.  $\nu_{1}(t),\nu_{2}(t)$ are auxiliary inputs which are used to alleviate the restriction of constraints for $u(t)$ in $\psi_{2}(\boldsymbol{x},\boldsymbol{a},\dot{a}_{1}(t),\boldsymbol{\nu})\ge0$ and \eqref{eq:simple-control-constraint}. Different from the first constraint in \eqref{eq:SACC-PACBF-sequence}, $k_{1}\psi_{0}(\boldsymbol{x},a_{1}(t))$ is still a class $\kappa$ function with respect to $\psi_{0}(\boldsymbol{x},a_{1}(t)),$ therefore we do not need to define additional HOCBFs and CLFs like $b_{3}(p_{1}(t)),V(p_{1}(t))$ to limit the growth of $a_{1}(t).$
We can rewrite $\psi_{1} (\boldsymbol{x},\boldsymbol{a},\dot{a}_{1}(t))$ in \eqref{eq:SACC-AVBCBF-sequence} as
\begin{equation}
\label{eq:SACC-AVBCBF-sequence-rewrite}
\begin{split}
\psi_{1}(\boldsymbol{x},\boldsymbol{a},\dot{a}_{1}(t))\coloneqq a_{2}(t)a_{1}(t)(v_{p}-v(t)\\
+k_{1}(1+\frac{\dot{a}_{1}(t)}{k_{1}a_{1}(t)})b(\boldsymbol{x}))\ge 0.
\end{split}
\end{equation}
Compared to the first constraint in \eqref{eq:SACC-HOCBF-sequence}, $\frac{\dot{a}_{1}(t)}{a_{1}(t)}$ is a time-varying auxiliary term to alleviate the conservativeness of control that small $k_{1}$ originally has, which shows the adaptivity of auxiliary terms to constant class $\kappa$ coefficients. 

% There is another type of adaptive CBFs called Relaxation-Adaptive Control Barrier Functions (RACBFs) in \cite{xiao2021adaptive}. The RACBF $b(\boldsymbol{x})$ is in the form:
% \begin{equation}
% \label{eq:RACBF}
% \psi_{0}(\boldsymbol{x},r(t))\coloneqq b(\boldsymbol{x})-r(t),
% \end{equation}
% where $r(t)\ge0$ is a relaxation that plays the similar role as Backup policy introduced in \cite{chen2021backup} {\color{red} How a relaxation is related to the backup policy?}. However, it is difficult for us to find the appropriate backup policy for controller of complicated dynamic system. Two main drawbacks affect the performance of RACBFs. {\color{red}wording} In the first place, $r(t)$ contracts the coverage of feasible space of states defined by $b(\boldsymbol{x})\ge0$, i.e., the distance $z(t)$ allowable for two vehicles is even smaller {\color{red}This should be larger} by $z(t)-l_{p}-r(t)\ge0$ because of the existence of non-negative $r(t)$. Secondly, the feasibility of solving QP with RACBF constraints is limited by the existence of upper bound of auxiliary input $\nu_{r}(t)$ related to $r(t)$ defined in Eq. (29) in \cite{xiao2021adaptive} {\color{red}What is $\nu_r$? you should make it self-contained.}. We can define the highest order {\color{red}what is this?} of $r(t)$ to be 2, then from \eqref{eq:SACC-HOCBF-sequence} normally we have
% \begin{equation}
% \label{eq:highest-order-RACBF}
% \begin{split}
% \psi_{2}(\boldsymbol{x},r(t),\dot{r}(t),\nu_{r}(t))\coloneqq -u(t)-\nu_{r}(t)\\
% +k_{1}(v_{p}-v(t)-\dot{r}(t))+k_{2}(v_{p}-v(t)-\dot{r}(t)\\
% +k_{1}(z(t)-l_{p}-r(t))\ge0, \nu_{r}(t)=\ddot{r}(t),
% \end{split}
% \end{equation}
% which sets the upper bound {\color{red}This is not clear} for $\nu_{r}(t)$ and there will easily be empty feasible set for $\nu_{r}(t)$ if the lower bound of $\nu_{r}(t)$ defined by constraint (31) in \cite{xiao2021adaptive} is too large. Compared to RACBFs, AVCBFs will neither contract the feasible space of states, nor set the upper bound for $\boldsymbol{\nu}$ (at least no upper bound for $\nu_{1}(t))$ as shown in the proof of Thm. \ref{thm:feasibility-guarantee} in Sec. \ref{subsec: optimal-control}, which shows the great benefits of AVCBFs in terms of safety and feasibility. 

% \subsection{HOCBFs for Auxiliary Coefficients}
\subsection{Adaptive HOCBFs for Safety:\ AVCBFs}
\label{sec:AVCBFs}

Motivated by the SACC example in Sec. \ref{subsec:SACC-problem}, given a function $b:\mathbb{R}^{n}\to\mathbb{R}$ with relative degree $m$ for system \eqref{eq:affine-control-system} and a time-varying vector $\boldsymbol{a}(t)\coloneqq [a_{1}(t),\dots,a_{m}(t)]^{T}$ with positive components called auxiliary variables, the key idea in converting a regular HOCBF into an adaptive
one without defining excessive constraints is to place one auxiliary variable in front of each function in \eqref{eq:sequence-f1} similar to \eqref{eq:SACC-AVBCBF-sequence}. 
As described in Sec. \ref{subsec:SACC-problem}, we only need to define HOCBFs for auxiliary variables to ensure each $a_{i}(t)>0, i \in \{1,...,m\}.$ To realize this, we need to define auxiliary systems that contain auxiliary states $\boldsymbol{\pi}_{i}(t)$ and inputs $\nu_{i}(t)$, through which systems we can extend each HOCBF to desired relative degree, just like $b(\boldsymbol{x})$ has relative degree $m$
with respect to the dynamics \eqref{eq:affine-control-system}. Consider $m$ auxiliary systems in the form 
\begin{equation}
\label{eq:virtual-system}
\dot{\boldsymbol{\pi}}_{i}=F_{i}(\boldsymbol{\pi}_{i})+G_{i}(\boldsymbol{\pi}_{i})\nu_{i}, i \in \{1,...,m\},
\end{equation}
where $\boldsymbol{\pi}_{i}(t)\coloneqq [\pi_{i,1}(t),\dots,\pi_{i,m+1-i}(t)]^{T}\in \mathbb{R}^{m+1-i}$ denotes an auxiliary state with $\pi_{i,j}(t)\in \mathbb{R}, j \in \{1,...,m+1-i\}.$ $\nu_{i}\in \mathbb{R}$ denotes an auxiliary input for \eqref{eq:virtual-system}, $F_{i}:\mathbb{R}^{m+1-i}\to\mathbb{R}^{m+1-i}$ and $G_{i}:\mathbb{R}^{m+1-i}\to\mathbb{R}^{m+1-i}$ are locally Lipschitz. For simplicity, we just build up the connection between an auxiliary variable and the system as $a_{i}(t)=\pi_{i,1}(t), \dot{\pi}_{i,1}(t)=\pi_{i,2}(t),\dots,\dot{\pi}_{i,m-i}(t)=\pi_{i,m+1-i}(t)$ and make $\dot{\pi}_{i,m+1-i}(t)=\nu_{i},$ then we can define many specific HOCBFs $h_{i}$ to enable $a_{i}(t)$ to be positive. 

Given a function $h_{i}:\mathbb{R}^{m+1-i}\to\mathbb{R},$ we can define a sequence of functions $\varphi_{i,j}:\mathbb{R}^{m+1-i}\to\mathbb{R}, i \in\{1,...,m\}, j \in\{1,...,m+1-i\}:$
\begin{equation}
\label{eq:virtual-HOCBFs}
\varphi_{i,j}(\boldsymbol{\pi}_{i})\coloneqq\dot{\varphi}_{i,j-1}(\boldsymbol{\pi}_{i})+\alpha_{i,j}(\varphi_{i,j-1}(\boldsymbol{\pi}_{i})),
\end{equation}
where $\varphi_{i,0}(\boldsymbol{\pi}_{i})\coloneqq h_{i}(\boldsymbol{\pi}_{i}),$ $\alpha_{i,j}(\cdot)$ are $(m+1-i-j)^{th}$ order differentiable class $\kappa$ functions. Sets $\mathcal{B}_{i,j}$ are defined as
\begin{equation}
\label{eq:virtual-sets}
\mathcal B_{i,j}\coloneqq \{\boldsymbol{\pi}_{i}\in\mathbb{R}^{m+1-i}:\varphi_{i,j}(\boldsymbol{\pi}_{i})>0\}, \ j\in \{0,...,m-i\}. 
\end{equation}
Let $\varphi_{i,j}(\boldsymbol{\pi}_{i}),\ j\in \{1,...,m+1-i\}$ and $\mathcal B_{i,j},\ j\in \{0,...,m-i\}$ be defined by \eqref{eq:virtual-HOCBFs} and \eqref{eq:virtual-sets} respectively. By Def. \ref{def:HOCBF}, a function $h_{i}:\mathbb{R}^{m+1-i}\to\mathbb{R}$ is a HOCBF with relative degree $m+1-i$ for system \eqref{eq:virtual-system} if there exist class $\kappa$ functions $\alpha_{i,j},\ j\in \{1,...,m+1-i\}$ as in \eqref{eq:virtual-HOCBFs} such that
\begin{small}
\begin{equation}
\label{eq:highest-SHOCBF}
\begin{split}
\sup_{\nu_{i}\in \mathbb{R}}[L_{F_{i}}^{m+1-i}h_{i}(\boldsymbol{\pi}_{i})+L_{G_{i}}L_{F_{i}}^{m-i}h_{i}(\boldsymbol{\pi}_{i})\nu_{i}+O_{i}(h_{i}(\boldsymbol{\pi}_{i}))\\
+ \alpha_{i,m+1-i}(\varphi_{i,m-i}(\boldsymbol{\pi}_{i}))] \ge \epsilon,
\end{split}
\end{equation}
\end{small}
$\forall\boldsymbol{\pi}_{i}\in \mathcal B_{i,0}\cap,...,\cap \mathcal B_{i,m-i}$. $O_{i}(\cdot)=\sum_{j=1}^{m-i}L_{F_{i}}^{j}(\alpha_{i,m-i}\circ\varphi_{i,m-1-i})(\boldsymbol{\pi}_{i}) $ where $\circ$ denotes the composition of functions. $\epsilon$ is a positive constant which can be infinitely small. 

\begin{remark}
\label{rem:safety-guarantee-2}
If $h_{i}(\boldsymbol{\pi}_{i})$ is a HOCBF illustrated above and $\boldsymbol{\pi}_{i}(0) \in \mathcal {B}_{i,0}\cap \dots \cap \mathcal {B}_{i,m-i},$ then satisfying constraint in \eqref{eq:highest-SHOCBF} is equivalent to making $\varphi_{i,m+1-i}(\boldsymbol{\pi}_{i}(t))\ge \epsilon>0, \forall t\ge 0.$ Based on
\eqref{eq:virtual-HOCBFs}, since $\boldsymbol{\pi}_{i}(0) \in \mathcal {B}_{i,m-i}$ (i.e., $\varphi_{i,m-i}(\boldsymbol{\pi}_{i}(0))>0),$ then we have $\varphi_{i,m-i}(\boldsymbol{\pi}_{i}(t))>0$ (If there exists a $t_{1}\in (0,t_{2}]$, which makes $\varphi_{i,m-i}(\boldsymbol{\pi}_{i}(t_{1}))=0,$ then we have $\dot{\varphi}_{i,m-i}((\boldsymbol{\pi}_{i}(t_{1}))>0\Leftrightarrow \varphi_{i,m-i}(\boldsymbol{\pi}_{i}(t_{1}^{-}))\varphi_{i,m-i}(\boldsymbol{\pi}_{i}(t_{1}^{+}))<0,$ which is against the definition of $\alpha_{i,m+1-i}(\cdot),$ therefore $\forall t_{1}>0, \varphi_{i,m-i}(\boldsymbol{\pi}_{i}(t_{1}))>0,$ note that $t_{1}^{-},t_{1}^{+}$ denote the left and right limit). Based on \eqref{eq:virtual-HOCBFs}, since $\boldsymbol{\pi}_{i}(0) \in \mathcal {B}_{i,m-1-i},$ then similarly we have $\varphi_{i,m-1-i}(\boldsymbol{\pi}_{i}(t))>0,\forall t\ge 0.$ Repeatedly, we have $\varphi_{i,0}(\boldsymbol{\pi}_{i}(t))>0,\forall t\ge 0,$ therefore the sets $\mathcal {B}_{i,0},\dots,\mathcal {B}_{i,m-i}$ are forward invariant.
\end{remark}

For simplicity, we can make $h_{i}(\boldsymbol{\pi}_{i})=\pi_{i,1}(t)=a_{i}(t).$ Based on Rem. \ref{rem:safety-guarantee-2}, each $a_{i}(t)$ will be positive.

The remaining question is how to define an adaptive HOCBF to guarantee $b(\boldsymbol{x})\ge0$ with the assistance of auxiliary variables. Let $\boldsymbol{\Pi}(t)\coloneqq [\boldsymbol{\pi}_{1}(t),\dots,\boldsymbol{\pi}_{m}(t)]^{T}$ and $\boldsymbol{\nu}\coloneqq [\nu_{1},\dots,\nu_{m}]^{T}$ denote the auxiliary states and control inputs of system \eqref{eq:virtual-system}. We can define a sequence of functions 
\begin{small}
\begin{equation}
\label{eq:AVBCBF-sequence}
\begin{split}
&\psi_{0}(\boldsymbol{x},\boldsymbol{\Pi}(t))\coloneqq a_{1}(t)b(\boldsymbol{x}),\\
&\psi_{i}(\boldsymbol{x},\boldsymbol{\Pi}(t))\coloneqq a_{i+1}(t)(\dot{\psi}_{i-1}(\boldsymbol{x},\boldsymbol{\Pi}(t))+\alpha_{i}(\psi_{i-1}(\boldsymbol{x},\boldsymbol{\Pi}(t)))),
\end{split}
\end{equation}
\end{small}
where $i \in \{1,...,m-1\}, \psi_{m}(\boldsymbol{x},\boldsymbol{\Pi}(t))\coloneqq \dot{\psi}_{m-1}(\boldsymbol{x},\boldsymbol{\Pi}(t))+\alpha_{m}(\psi_{m-1}(\boldsymbol{x},\boldsymbol{\Pi}(t))).$ We further define a sequence of sets $\mathcal{C}_{i}$ associated with \eqref{eq:AVBCBF-sequence} in the form 
\begin{equation}
\label{eq:AVBCBF-set}
\begin{split}
\mathcal C_{i}\coloneqq \{(\boldsymbol{x},\boldsymbol{\Pi}(t)) \in \mathbb{R}^{n} \times \mathbb{R}^{m}:\psi_{i}(\boldsymbol{x},\boldsymbol{\Pi}(t))\ge 0\}, 
\end{split}
\end{equation}
where $i \in \{0,...,m-1\}.$
Since $a_{i}(t)$ is a HOCBF with relative degree $m+1-i$ for \eqref{eq:virtual-system}, based on \eqref{eq:highest-SHOCBF}, we define a constraint set $\mathcal{U}_{\boldsymbol{a}}$ for $\boldsymbol{\nu}$ as 
\begin{small}
\begin{equation}
\label{eq:constraint-up}
\begin{split}
\mathcal{U}_{\boldsymbol{a}}(\boldsymbol{\Pi})\coloneqq \{\boldsymbol{\nu}\in\mathbb{R}^{m}:   L_{F_{i}}^{m+1-i}a_{i}+[L_{G_{i}}L_{F_{i}}^{m-i}a_{i}]\nu_{i}\\
+O_{i}(a_{i})+ \alpha_{i,m+1-i}(\varphi_{i,m-i}(a_{i})) \ge \epsilon, i\in \{1,\dots,m\}\},
\end{split}
\end{equation}
\end{small}
where $\varphi_{i,m-i}(\cdot)$ is defined similar to \eqref{eq:virtual-HOCBFs} and $a_{i}(t)$ is ensured positive. $\epsilon$ is a positive constant which can be infinitely small. 

\begin{definition}[AVCBF]
\label{def:AVBCBF}
Let $\psi_{i}(\boldsymbol{x},\boldsymbol{\Pi}(t)),\ i\in \{1,...,m\}$ be defined by \eqref{eq:AVBCBF-sequence} and $\mathcal C_{i},\ i\in \{0,...,m-1\}$ be defined by \eqref{eq:AVBCBF-set}. A function $b(\boldsymbol{x}):\mathbb{R}^{n}\to\mathbb{R}$ is an Auxiliary-Variable Adaptive Control Barrier Function (AVCBF) with relative degree $m$ for system \eqref{eq:affine-control-system} if every $a_{i}(t),i\in \{1,...,m\}$ is a
HOCBF with relative degree $m+1-i$ for the auxiliary system
\eqref{eq:virtual-system}, and there exist $(m-j)^{th}$ order differentiable class $\kappa$ functions $\alpha_{j},j\in \{1,...,m-1\}$
and a class $\kappa$ functions $\alpha_{m}$ s.t.
\begin{small}
\begin{equation}
\label{eq:highest-AVBCBF}
\begin{split}
\sup_{\boldsymbol{u}\in \mathcal{U},\boldsymbol{\nu}\in \mathcal{U}_{\boldsymbol{a}}}[\sum_{j=2}^{m-1}[(\prod_{k=j+1}^{m}a_{k})\frac{\psi_{j-1}}{a_{j}}\nu_{j}] + \frac{\psi_{m-1}}{a_{m}}\nu_{m} \\ +(\prod_{i=2}^{m}a_{i})b(\boldsymbol{x})\nu_{1} +(\prod_{i=1}^{m}a_{i})(L_{f}^{m}b(\boldsymbol{x})+L_{g}L_{f}^{m-1}b(\boldsymbol{x})\boldsymbol{u})\\+R(b(\boldsymbol{x}),\boldsymbol{\Pi})
+ \alpha_{m}(\psi_{m-1})] \ge 0,
\end{split}
\end{equation}
\end{small}
$\forall (\boldsymbol{x},\boldsymbol{\Pi})\in \mathcal C_{0}\cap,...,\cap \mathcal C_{m-1}$ and each $a_{i}>0, i\in\{1,\dots,m\}.$ In \eqref{eq:highest-AVBCBF}, $R(b(\boldsymbol{x}),\boldsymbol{\Pi})$ denotes the remaining Lie derivative terms of $b(\boldsymbol{x})$ (or $\boldsymbol{\Pi}$) along $f$ (or $F_{i},i\in\{1,\dots,m\}$) with degree less than $m$ (or $m+1-i$), which is similar to the form of $O(\cdot )$ in \eqref{eq:highest-HOCBF}.
\end{definition}

\begin{theorem}
\label{thm:safety-guarantee-3}
Given an AVCBF $b(\boldsymbol{x})$ from Def. \ref{def:AVBCBF} with corresponding sets $\mathcal{C}_{0}, \dots,\mathcal {C}_{m-1}$ defined by \eqref{eq:AVBCBF-set}, if $(\boldsymbol{x}(0),\boldsymbol{\Pi}(0)) \in \mathcal {C}_{0}\cap \dots \cap \mathcal {C}_{m-1},$ then if there exists solution of Lipschitz controller $(\boldsymbol{u},\boldsymbol{\nu})$ that satisfies the constraint in \eqref{eq:highest-AVBCBF} and also ensures $(\boldsymbol{x},\boldsymbol{\Pi})\in \mathcal {C}_{m-1}$ for all $t\ge 0,$ then $\mathcal {C}_{0}\cap \dots \cap \mathcal {C}_{m-1}$ will be rendered forward invariant for system \eqref{eq:affine-control-system}, $i.e., (\boldsymbol{x},\boldsymbol{\Pi}) \in \mathcal {C}_{0}\cap \dots \cap \mathcal {C}_{m-1}, \forall t\ge 0.$ Moreover, $b(\boldsymbol{x})\ge 0$ is ensured for all $t\ge 0.$
\end{theorem}

\begin{proof}
If $b(\boldsymbol{x})$ is an AVCBF that is $m^{th}$ order differentiable, then satisfying constraint in \eqref{eq:highest-AVBCBF} while ensuring $(\boldsymbol{x},\boldsymbol{\Pi})\in \mathcal {C}_{m-1}$ for all $t\ge 0$ is equivalent to make $\psi_{m-1}(\boldsymbol{x},\boldsymbol{\Pi})\ge 0, \forall t\ge 0.$ Since $a_{m}(t)>0$, we have $\frac{\psi_{m-1}(\boldsymbol{x},\boldsymbol{\Pi})}{a_{m}(t)}\ge 0.$ Based on
\eqref{eq:AVBCBF-sequence}, since $(\boldsymbol{x}(0),\boldsymbol{\Pi}(0)) \in \mathcal {C}_{m-2}$ (i.e., $\frac{\psi_{m-2}(\boldsymbol{x}(0),\boldsymbol{\Pi}(0))}{a_{m-1}(0)}\ge 0),a_{m-1}(t)>0,$ then we have $\psi_{m-2}(\boldsymbol{x},\boldsymbol{\Pi})\ge 0$ (The proof of this is similar to the proof in Rem. \ref{rem:safety-guarantee-2}), and also $\frac{\psi_{m-2}(\boldsymbol{x},\boldsymbol{\Pi})}{a_{m-1}(t)}\ge 0.$ Based on \eqref{eq:AVBCBF-sequence}, since $(\boldsymbol{x}(0),\boldsymbol{\Pi}(0)) \in \mathcal {C}_{m-3},a_{m-2}(t)>0$ then similarly we have $\psi_{m-3}(\boldsymbol{x},\boldsymbol{\Pi})\ge 0$ and $\frac{\psi_{m-3}(\boldsymbol{x},\boldsymbol{\Pi})}{a_{m-2}(t)}\ge 0,\forall t\ge 0.$ Repeatedly, we have $\psi_{0}(\boldsymbol{x},\boldsymbol{\Pi})\ge 0$ and $\frac{\psi_{0}(\boldsymbol{x},\boldsymbol{\Pi})}{a_{1}(t)}\ge 0,\forall t\ge 0.$ Therefore the sets $\mathcal {C}_{0},\dots,\mathcal {C}_{m-1}$ are forward invariant and $b(\boldsymbol{x})=\frac{\psi_{0}(\boldsymbol{x},\boldsymbol{\Pi})}{a_{1}(t)}\ge 0$ is ensured for all $t\ge 0$.
\end{proof}
Based on Thm. \ref{thm:safety-guarantee-3}, the safety regarding $b(\boldsymbol{x})=\frac{\psi_{0}(\boldsymbol{x},\boldsymbol{\Pi})}{a_{1}(t)}\ge 0$ is guaranteed.

\begin{remark}[Limitation of Approaches with Auxiliary Inputs]
\label{rem: PACBF-AVBCBF} 
Ensuring the satisfaction of the $i^{th}$ order AVCBF constraint as shown in \eqref{eq:AVBCBF-set} when $i\in\{1,\dots,m-1\},$ i.e., $\psi_{i}(\boldsymbol{x},\boldsymbol{\Pi})\ge 0$ will guarantee $\psi_{i-1}(\boldsymbol{x},\boldsymbol{\Pi})\ge 0$ based on the proof of Thm. \ref{thm:safety-guarantee-3}, which theoretically outperforms PACBF. However, both approaches can not ensure satisfying $\psi_{m}(\boldsymbol{x},\boldsymbol{\Pi})\ge 0$ will guarantee $\psi_{m-1}(\boldsymbol{x},\boldsymbol{\Pi})\ge 0$ since the growth of $\boldsymbol{\nu}_{i}$ is unbounded. Therefore in Thm. \ref{thm:safety-guarantee-3}, $(\boldsymbol{x},\boldsymbol{\Pi})\in \mathcal {C}_{m-1}$ for all $t\ge 0$ also needs to be satisfied to guarantee the forward invariance of the intersection of sets. 
\end{remark}

\subsection{Optimal Control with AVCBFs}
\label{subsec: optimal-control}
Consider an optimal control problem as
\begin{small}
\begin{equation}
\label{eq:cost-function-1}
\begin{split}
 \min_{\boldsymbol{u}} \int_{0}^{T} 
 D(\left \| \boldsymbol{u} \right \| )dt,
\end{split}
\end{equation}
\end{small}
where $\left \| \cdot \right \|$ denotes the 2-norm of a vector, $D(\cdot)$ is a strictly increasing function of its argument and $T>0$ denotes the ending time. Since we need to introduce auxiliary inputs $v_{i}$ to enhance the feasibility of optimization, we should reformulate the cost in \eqref{eq:cost-function-1} as
\begin{small}
\begin{equation}
\label{eq:cost-function-2}
\begin{split}
 \min_{\boldsymbol{u},\boldsymbol{\nu}} \int_{0}^{T} 
 [D(\left \| \boldsymbol{u} \right \| )+\sum_{i=1}^{m}W_{i}(\nu_{i}-a_{i,w})^{2}]dt.
\end{split}
\end{equation}
\end{small}
In \eqref{eq:cost-function-2}, $W_{i}$ is a positive scalar and $a_{i,w}\in \mathbb{R}$ is the scalar to which we hope each auxiliary input $\nu_{i}$ converges. Both are chosen to tune the performance of the controller. We can formulate the CLFs, HOCBFs and AVCBFs introduced in Def. \ref{def:control-l-f}, Sec. \ref{sec:AVCBFs} and Def. \ref{def:AVBCBF} as constraints of the QP with cost function \eqref{eq:cost-function-2} to realize safety-critical control. Next we will show AVCBFs can be used to enhance the feasibility of solving QP compared with classical HOCBFs in Def. \ref{def:HOCBF}.

In auxiliary system \eqref{eq:virtual-system}, if we define $a_{i}(t)=\pi_{i,1}(t)=1, \dot{\pi}_{i,1}(t)=\dot{\pi}_{i,2}(t)=\cdots=\dot{\pi}_{i,m+1-i}(t)=0,$ then the way we construct functions and sets in \eqref{eq:virtual-HOCBFs} and \eqref{eq:virtual-sets} are exactly the same as \eqref{eq:sequence-f1} and \eqref{eq:sequence-set1}, which means classical HOCBF is in fact one specific case of AVCBF. Assume that the highest order HOCBF constraint \eqref{eq:highest-HOCBF} conflicts with control input constraints \eqref{eq:control-constraint} at $t=t_{b},$ i.e., we can not find a feasible controller $u(t_{b})$ to satisfy \eqref{eq:highest-HOCBF} and \eqref{eq:control-constraint}. Instead, starting from a time slot $t=t_{a}$ which is just before $t=t_{b}$ ($t_{b}-t_{a}=\varepsilon$ where $\varepsilon$ is an infinitely small positive value), we exchange the control framework of classical HOCBF into AVCBF instantly. Suppose we can find appropriate hyperparameters to ensure two constraints in \eqref{eq:constraint-up} and \eqref{eq:highest-AVBCBF}
% \begin{small}
% \begin{equation}
% \label{eq:constraint-fea-12}
% \begin{split}
%  \nu_{i}
%   > \frac{-L_{F_{i}}^{m+1-i}a_{i}-O_{i}(a_{i})-\alpha_{i,m+1-i}(\varphi_{i,m-i}(a_{i}))}{L_{G_{i}}L_{F_{i}}^{m-i}a_{i}},\\
%   \sum_{j=2}^{m-1}[(\prod_{k=j+1}^{m}a_{k})\frac{\psi_{j-1}}{a_{j}}\nu_{j}] + \frac{\psi_{m-1}}{a_{m}}\nu_{m} +(\prod_{i=2}^{m}a_{i})b(\boldsymbol{x})\nu_{1} \\ \ge -(\prod_{i=1}^{m}a_{i})(L_{f}^{m}b(\boldsymbol{x})+L_{g}L_{f}^{m-1}b(\boldsymbol{x})\boldsymbol{u})-R(b(\boldsymbol{x}),\boldsymbol{\Pi}) \\
% - \alpha_{m}(\psi_{m-1}),  i\in \{1,\dots,m\}
% \end{split}
% \end{equation}
% \end{small}
are satisfied given $\boldsymbol{u}$ constrained by \eqref{eq:control-constraint} at $t_{b},$ then there exists solution $\boldsymbol{u}(t_{b})$ for the optimal control problem and the feasibility of solving QP is enhanced. Relying on AVCBF, We can discretize the whole time period $[0,T]$ into several small time intervals like $[t_{a},t_{b}]$ to maximize the feasibility of solving QP under safety constraints, which calls for the development of automatic parameter-tuning techniques in future.
% \begin{theorem}
% \label{thm:feasibility-guarantee}
% Given an AVCBF $b(\boldsymbol{x})$ from Def. \ref{def:AVBCBF} with corresponding sets $\mathcal{C}_{0}, \dots,\mathcal {C}_{m-1}$ defined by \eqref{eq:AVBCBF-set}, if $(\boldsymbol{x}(0),\boldsymbol{\Pi}(0)) \in \mathcal {C}_{0}\cap \dots \cap \mathcal {C}_{m-1}$ and $L_{G_{i}}L_{F_{i}}^{m-i}a_{i}>0,a_{i}(t)>0, i\in\{1,\dots,m\}$ in \eqref{eq:constraint-up}, then if there exists solution of Lipschitz controller $(\boldsymbol{u},\boldsymbol{\nu})$ that satisfies the constraint in \eqref{eq:highest-AVBCBF} and also ensures $\psi_{0}>0,\dots,\psi_{s}>0,s\in \{0,\dots,m-1\}$ in \eqref{eq:AVBCBF-set}, then the QP with cost function \eqref{eq:cost-function-2} and constraints \eqref{eq:control-constraint},\eqref{eq:AVBCBF-set}-\eqref{eq:highest-AVBCBF} is guranteed to be feasible.
% \end{theorem}

% \begin{proof}
% Rewrite the constraint \eqref{eq:constraint-up} as 
% \begin{equation}
% \label{eq:constraint-fea-1}
% \begin{split}
%  \nu_{i}
%   > \frac{-L_{F_{i}}^{m+1-i}a_{i}-O_{i}(a_{i})-\alpha_{i,m+1-i}(\varphi_{i,m-i}(a_{i}))}{L_{G_{i}}L_{F_{i}}^{m-i}a_{i}},
% \end{split}
% \end{equation}
% where $i\in \{1,\dots,m\}.$ Rewrite the constraint \eqref{eq:highest-AVBCBF} as
% \begin{equation}
% \label{eq:constraint-fea-2}
% \begin{split}
% \sum_{j=2}^{m-1}[(\prod_{k=j+1}^{m}a_{k})\frac{\psi_{j-1}}{a_{j}}\nu_{j}] + \frac{\psi_{m-1}}{a_{m}}\nu_{m} +(\prod_{i=2}^{m}a_{i})b(\boldsymbol{x})\nu_{1} \\ \ge -(\prod_{i=1}^{m}a_{i})(L_{f}^{m}b(\boldsymbol{x})+L_{g}L_{f}^{m-1}b(\boldsymbol{x})\boldsymbol{u})-R(b(\boldsymbol{x}),\boldsymbol{\Pi}) \\
% - \alpha_{m}(\psi_{m-1}),  i\in \{1,\dots,m\}.
% \end{split}
% \end{equation}
% Since $L_{G_{i}}L_{F_{i}}^{m-i}a_{i}>0$ in \eqref{eq:constraint-fea-1}, $\psi_{0}>0,\dots,\psi_{s}>0,s\in \{0,\dots,m-1\}$ in \eqref{eq:constraint-fea-2} and $a_{1}>0,\dots,a_{m}>0,$ we have $(\prod_{i=2}^{m}a_{i})b(\boldsymbol{x})>0,(\prod_{k=j+1}^{m}a_{k})\frac{\psi_{j-1}}{a_{j}}\nu_{j}>0,j\in \{2,\dots,s\}$ are always positive, 
% then there always exist large enough $\nu_{1},\dots,\nu_{s}$ satisfying constraints above {\color{red}you are assuming a very specific (13).} (the upper bounds of $\nu_{1},\dots,\nu_{s}$ are unlimited), hence the feasibility of QP with cost function \eqref{eq:cost-function-2} and constraints \eqref{eq:control-constraint},\eqref{eq:AVBCBF-set}-\eqref{eq:highest-AVBCBF} is guaranteed.  {\color{red}Control limitations (2) are the most critical factor in the feasibility. You completely ignore this. The proof is very sloppy.}
% \end{proof}

Besides safety and feasibility, another benefit of using AVCBFs is that the conservativeness of the control strategy can also be ameliorated. For example, from \eqref{eq:AVBCBF-sequence}, we can rewrite $\psi_{i}(\boldsymbol{x},\boldsymbol{\Pi})\ge 0$ as
\begin{equation}
\label{eq:AVCBF-rewrite}
\begin{split}
\dot{\phi}_{i-1}(\boldsymbol{x},\boldsymbol{\Pi})+k_{i}(1+\frac{\dot{a}_{i}(t)}{k_{i}a_{i}(t)}) \phi_{i-1}(\boldsymbol{x},\boldsymbol{\Pi})\ge0,
\end{split}
\end{equation}
where $\phi_{i-1}(\boldsymbol{x},\boldsymbol{\Pi})=\frac{\psi_{i-1}(\boldsymbol{x},\boldsymbol{\Pi})}{a_{i}(t)},\alpha_{i}(\psi_{i-1}(\boldsymbol{x},\boldsymbol{\Pi}))=k_{i}a_{i}(t)\phi_{i-1}(\boldsymbol{x},\boldsymbol{\Pi}), k_{i}>0, i\in \{1,\dots,m\}.$ Similar to PACBFs, we require $1+\frac{\dot{a}_{i}(t)}{k_{i}a_{i}(t)}\ge0,$ which gives us $\dot{a}_{i}(t)+k_{i}a_{i}(t)\ge0.$
The term $\frac{\dot{a}_{i}(t)}{a_{i}(t)}$ can be adjusted adaptable  to ameliorate the conservativeness of control strategy that $k_{i}\phi_{i-1}(\boldsymbol{x},\boldsymbol{\Pi})$ may have, i.e., the ego vehicle can brake earlier or later given time-varying control constraint $\boldsymbol{u}_{min}(t)\le \boldsymbol{u} \le\boldsymbol{u}_{max}(t),$ which confirms the adaptivity of AVCBFs to control constraint and conservativeness of control strategy. 

\begin{remark}[Parameter-Tuning for AVCBFs]
\label{rem: parameter-tuning}
Based on the analysis of \eqref{eq:AVCBF-rewrite}, we require $\dot{a}_{i}(t)+k_{i}a_{i}(t)\ge0.$ If we define first order HOCBF constraint for $a_{i}(t)>0$ as $\dot{a}_{i}(t)+l_{i}a_{i}(t)\ge0,$ we should choose hyperparameter $l_{i}\le k_{i}$ to guarantee $\dot{a}_{i}(t)+k_{i}a_{i}(t)\ge\dot{a}_{i}(t)+l_{i}a_{i}(t)\ge 0.$ For simplicity, we can use $l_{i}=k_{i}.$ In cost function \eqref{eq:cost-function-2}, we can tune hyperparameters $W_{i}$ and $a_{i,w}$ to adjust the corresponding ratio $\frac{\dot{a}_{i}(t)}{a_{i}(t)}$ to change the performance of the optimal controller.
\end{remark}

\begin{remark}
\label{rem: sufficient-con}
Note that the satisfaction of the constraint in \eqref{eq:highest-AVBCBF} is a sufficient condition for the satisfaction of the original constraint $\psi_{0}(\boldsymbol{x},\boldsymbol{\Pi})>0,$ it is not necessary to introduce auxiliary variables as many as from $a_{1}(t)$ to $a_{m}(t),$ which allows us to choose an appropriate
number of auxiliary variables for the AVCBF constraints to reduce the complexity. In other words, the number of auxiliary variables can be less than or equal to the relative degree $m$.
\end{remark}
% \documentclass[11pt]{article}
% \usepackage{tikz}
% \usetikzlibrary{positioning, shapes.multipart, fit, calc, shapes.geometric, shapes.misc}

% \begin{document}

\begin{figure*}[htbp]

\begingroup
\usetikzlibrary{shapes.geometric}
\usetikzlibrary{arrows.meta}
\usetikzlibrary{backgrounds}
\definecolor{tiffanyblue}{RGB}{129,216,208}
\definecolor{bangdiblue}{RGB}{0,149,182}
\definecolor{kleinblue}{RGB}{0,47,167}
\definecolor{kabuliblue}{RGB}{26,85,153}
\definecolor{purple}{RGB}{138,43,226}

    \centering

      \tikzset{global scale/.style={
    scale=#1,
    every node/.append style={scale=#1}
  }
}


\begin{tikzpicture}[global scale=0.64]
    \newlength{\moduleintervaly}
    \setlength{\moduleintervaly}{1.8em}    
    \newlength{\moduleintervalx}
    \setlength{\moduleintervalx}{-7em}
    \newlength{\blockintervalx}
    \setlength{\blockintervalx}{30em}
    
    \tikzstyle{circlenode}=[draw, circle,minimum size=4pt,inner sep=0, fill=red!30];
    
    \tikzstyle{moduleode}=[draw,minimum height=2.5em,minimum width=23em,inner sep=.0em,thick,rounded corners=.2em, font=\small, scale=0.8];
    
    \tikzstyle{layernode}=[draw,minimum height=1.5em,minimum width=5em,inner sep=.0em,thick,rounded corners=.2em, font=\small,fill=yellow!20];
    
    \tikzstyle{attentionnode}=[draw,minimum height=1.5em,minimum width=5em,inner sep=.0em,thick,rounded corners=.2em, font=\small];

    \tikzstyle{querynode}=[draw,minimum height=1.5em,minimum width=2em,inner sep=.0em,thick,rounded corners=.2em, font=\small];

    \tikzstyle{attnmapnode}=[fill=yellow!20,draw,minimum height=2.5em,minimum width=10em,inner sep=.0em,thick,rounded corners=.2em, font=\small];

    \tikzstyle{partialattnmapnode}=[fill=tiffanyblue!40,draw,minimum height=2.5em,minimum width=12em,inner sep=.0em,thick,rounded corners=.2em, font=\small];

    \tikzstyle{partialattnmapnode1}=[fill=red!20,draw,minimum height=2.5em,minimum width=12em,inner sep=.0em,thick,rounded corners=.2em, font=\small];

    \tikzstyle{GRLattnmapnode}=[fill=yellow!20,draw,minimum height=2.5em,minimum width=8em,inner sep=.0em,thick,rounded corners=.2em, font=\small];

    \tikzstyle{StandardFFNnode}=[fill=red!20,draw,minimum height=2.5em,minimum width=23em,inner sep=.0em,thick,rounded corners=.2em, font=\small];

    \tikzstyle{Concatenode}=[draw,minimum height=1.5em,minimum width=5em,inner sep=.0em,thick,rounded corners=.2em, font=\small];

    \tikzstyle{subspace_block}=[fill=yellow!20,draw,minimum height=2.5em,minimum width=8em,inner sep=.0em,thick,rounded corners=.2em, font=\small];

    \tikzstyle{partial_subspace_block}=[fill=red!20,draw,minimum height=2.5em,minimum width=11.0em,inner sep=.0em,thick,rounded corners=.2em, font=\small];

    \tikzstyle{global_partial_subspace_block}=[fill=pink!20,draw,minimum height=2.5em,minimum width=6.5em,inner sep=.0em,thick,rounded corners=.2em, font=\small];

    \tikzstyle{space_block}=[fill=blue!20,draw,minimum height=2.5em,minimum width=23em,inner sep=.0em,thick,rounded corners=.2em, font=\small];
    
    \tikzstyle{Encoder_block}=[draw,minimum height=8.8*\moduleintervaly,minimum width=20em,inner sep=.0em,thick,rounded corners=.2em, font=\small];

    \tikzstyle{GRL_block}=[draw,minimum height=7em,minimum width=8.8*\moduleintervaly,inner sep=.0em,thick,rounded corners=.2em, font=\small];

    \tikzstyle{Partial_Encoder_block}=[draw,minimum height=7.7*\moduleintervaly,minimum width=25em,inner sep=.0em,thick,rounded corners=.2em, font=\small];
    
    \tikzstyle{Graph_structure_learning}=[draw,minimum height=1.6em,minimum width=2.5em,inner sep=.0em,thick,rounded corners=.2em, font=\small,fill=orange!20];

    
    \tikzstyle{recnode}=[rectangle,rounded corners=5pt,draw,minimum height=1.8em,minimum width=3.5em,inner sep=0em,thick,rounded corners=0.2em,font=\small,fill=orange!20];

    \tikzstyle{recnodewhite}=[rectangle,rounded corners=5pt,minimum height=2.2em,minimum width=3.5em,inner sep=0em,thick,rounded corners=0.2em,font=\small];

    
    % added node
    % \tikzstyle{datanode}=[rectangle,rounded corners=5pt,draw,minimum height=1.8em,minimum width=3.5em,inner sep=0em,thick,rounded corners=0.2em,font=\small,fill=orange!20];
    \tikzstyle{datanode}=[cylinder,rounded corners=5pt,draw,minimum height=1.2em,minimum width=1.5em,inner sep=0em,thick,rounded corners=0.2em,font=\small,fill=orange!20];
    \tikzstyle{modelnode}=[rectangle,rounded corners=5pt,draw,minimum height=1.8em,minimum width=3.5em,inner sep=0em,thick,rounded corners=0.2em,font=\small,fill=green!20];
    \tikzstyle{tasknode}=[rectangle,rounded corners=5pt,draw,minimum height=1.8em,minimum width=3.5em,inner sep=0em,thick,rounded corners=0.2em,font=\small,fill=blue!20];
    \tikzstyle{mergenode}=[rectangle,rounded corners=5pt,draw,minimum height=1.8em,minimum width=15.5em,inner sep=0em,thick,rounded corners=0.2em,font=\small,fill=yellow!20];
    \tikzstyle{mergemodelnode}=[rectangle,rounded corners=5pt,draw,minimum height=1.8em,minimum width=15.5em,inner sep=0em,thick,rounded corners=0.2em,font=\small,fill=green!20];
    
    \tikzstyle{databanknode}=[rectangle,rounded corners=5pt,draw,minimum height=3.0em,minimum width=15.5em,inner sep=0em,thick,rounded corners=0.2em,font=\small,fill=orange!10];
    \tikzstyle{basemodel}=[rectangle,rounded corners=5pt,draw,minimum height=1.8em,minimum width=17.5em,inner sep=0em,thick,rounded corners=0.2em,font=\small,fill=green!18];
    \tikzstyle{loranode}=[rectangle,rounded corners=5pt,draw,minimum height=1.7em,minimum width=3.5em,inner sep=0em,thick,rounded corners=0.2em,font=\small,fill=lime!20];
    \tikzstyle{longloranode}=[rectangle,rounded corners=5pt,draw,minimum height=1.7em,minimum width=13.5em,inner sep=0em,thick,rounded corners=0.2em,font=\small,fill=lime!20];

    \tikzstyle{groupnode}=[cylinder,rounded corners=5pt,draw,minimum height=1.0em,minimum width=3.1em,inner sep=0em,thick,rounded corners=0.2em,font=\small,fill=orange!20];

     \tikzstyle{longdatabanknode}=[cylinder,rounded corners=5pt,draw,minimum height=10.0em,minimum width=1.5em,inner sep=0em,thick,rounded corners=0.2em,font=\small,fill=orange!13];
    \def\nodehsep{3em}
    \def\nodewsep{3.5em}


    % picture a
    \begin{scope}[xshift=0.0in,yshift=0.0in]
        \begin{pgfonlayer}{background}
            \node[anchor=south,minimum height=\nodehsep*5.7,minimum width=19.5em,fill=gray!4,rounded corners=5pt,dotted,draw](backgroundc) at (0, 0) {};
            % \node[anchor=south,minimum height=\nodehsep*6.5,minimum width=39.6em,fill=gray!4,rounded corners=5pt,dotted,draw](backgroundd) at (0,0) {};
        \end{pgfonlayer}

        \node[datanode,anchor=south](data1_01) at ([xshift=-2.2*\nodehsep, yshift=14.8em]backgroundc.south) {En-De};
        \node[datanode,anchor=south](data1_02) at ([yshift=-1.8\nodehsep]data1_01.south) {De-En};
        
        \node[datanode,anchor=west](data2_01) at ([xshift=\nodehsep]data1_01.east) {En-Zh};
        \node[datanode,anchor=west](data2_02) at ([xshift=\nodehsep]data1_02.east) {Zh-En};        
        
        \node[align=center,thick, scale=2] (omit1_01) at ([xshift=\nodehsep]data2_01.center) {...};
        \node[align=center,thick, scale=2] (omit1_02) at ([xshift=\nodehsep]data2_02.center) {...};
        
        \node[datanode,anchor=west](data3_01) at ([xshift=\nodehsep]data2_01.east) {En-Fr};
        \node[datanode,anchor=west](data3_02) at ([xshift=\nodehsep]data2_02.east) {Fr-En};

        \begin{pgfonlayer}{background}
            \node[anchor=south,minimum height=1.4*\nodehsep,minimum width=17.2em,fill=orange!8,rounded corners=5pt,draw](databank1) at ([yshift=-3.9em]data2_01.north) {};
        \end{pgfonlayer}
        

        \node[mergenode,anchor=north](dataselectionc) at ([xshift=0.1\nodehsep,yshift=-1.6\nodehsep]data2_02.south) {Data Selection};
        \draw[thick, ->] (databank1.south) -- ([xshift=-0.12em]dataselectionc.north);
        % \draw[thick, ->] (databank1.south) -- ([xshift=-0.12em]dataselectionc.north);
        % \draw[thick, ->] (databank1.south) -- ([xshift=-0.12em]dataselectionc.north);
        
        \node[basemodel,anchor=north](basemodelc) at ([yshift=-3.1\nodehsep]dataselectionc.north) {Base Model};
        \draw[->,thick] (dataselectionc.south) -- (basemodelc.north);
        
        \node[loranode,anchor=north](lora1c) at ([yshift=-7.8\nodehsep]data1_02.south) {$\mathrm{LoRA}_{1}$};
        \node[loranode,anchor=north](lora2c) at ([yshift=-7.8\nodehsep]data2_02.south) {$\mathrm{LoRA}_{2}$};
        \node[align=center,thick, scale=2] (omit2) at ([xshift=\nodehsep]lora2c.center) {...};
        \node[loranode,anchor=north](lora3c) at ([yshift=-7.8\nodehsep]data3_02.south) {$\mathrm{LoRA}_{N}$}; 
        
        \node(plus1c) at ([yshift=0.6\nodehsep]lora1c.north) {\textbf{\(+\)}};
        \node(plus2c) at ([yshift=0.6\nodehsep]lora2c.north) {\textbf{\(+\)}};
        \node(plus3c) at ([yshift=0.6\nodehsep]lora3c.north) {\textbf{\(+\)}};
        

        \node[tasknode,anchor=north](task1c) at ([yshift=-1.2\nodehsep]lora1c.south) {$\mathrm{Task}_{1}$};
        \node[tasknode,anchor=north](task2c) at ([yshift=-1.2\nodehsep]lora2c.south) {$\mathrm{Task}_{2}$};
        \node[align=center,thick, scale=2] (omit3c) at ([xshift=\nodehsep]task2c.center) {...};
        \node[tasknode,anchor=north](task3c) at ([yshift=-1.2\nodehsep]lora3c.south) {$\mathrm{Task}_{N}$}; 


        \draw[->,thick] (lora1c.south) -- (task1c.north);
        \draw[->,thick] (lora2c.south) -- (task2c.north);
        \draw[->,thick] (lora3c.south) -- (task3c.north);

         
        \node[anchor=south,font=\Large](l3) at ([xshift=0.3em,yshift=-3em]backgroundc.south) {(a) Seperate/Multilingual/Group Training};
        % \node[anchor=north,font=\Large](l3) at ([xshift=0.3em,yshift=-1em]backgroundc.south) {\parbox{6cm}{\centering (a) Separate/Multilingual/ \\ Group Training}};
        % \node[anchor=south,font=\Large](l4) at ([xshift=.5em,yshift=-3em]backgroundd.south) {(d) Model Merging};
    \end{scope}

    % picture b
     \begin{scope}[xshift=4.9in,yshift=0in]
        \begin{pgfonlayer}{background}
            \node[anchor=south,minimum height=\nodehsep*5.7,minimum width=39.6em,fill=gray!4,rounded corners=5pt,dotted,draw](backgroundd) at (0,0) {};
        \end{pgfonlayer}

        \node[datanode,anchor=south](data11d) at ([xshift=-2.2*\nodehsep, yshift=14.8em]backgroundd.south) {En-De};
        \node[datanode,anchor=south](data12d) at ([yshift=-1.8\nodehsep]data11d.south) {De-En};
        
        \node[datanode,anchor=west](data21d) at ([xshift=\nodehsep]data11d.east) {En-Zh};
        \node[datanode,anchor=west](data22d) at ([xshift=\nodehsep]data12d.east) {Zh-En};        
        
        \node[align=center,thick, scale=2] (omit11d) at ([xshift=\nodehsep]data21d.center) {...};
        \node[align=center,thick, scale=2] (omit12d) at ([xshift=\nodehsep]data22d.center) {...};
        
        \node[datanode,anchor=west](data31d) at ([xshift=\nodehsep]data21d.east) {En-Fr};
        \node[datanode,anchor=west](data32d) at ([xshift=\nodehsep]data22d.east) {Fr-En};

        \begin{pgfonlayer}{background}
            \node[anchor=south,minimum height=1.4*\nodehsep,minimum width=17.2em,fill=orange!8,rounded corners=5pt,draw](databank1d) at ([yshift=-3.9em]data21d.north) {};
        \end{pgfonlayer}

        \node[mergenode,anchor=north](dataselectiond) at ([xshift=-3.5*\nodehsep,yshift=-0.2*\nodehsep]databank1d.south) {Group Selection};
        
        \node[basemodel,anchor=north](basemodeldleft) at ([yshift=-0.7\nodehsep]dataselectiond.south) {Base Model};
        \draw[->,thick] (dataselectiond.south) -- (basemodeldleft.north);

        \node[loranode,anchor=north](lora1dleft) at ([xshift=-5.5\nodehsep, yshift=-1.6\nodehsep]basemodeldleft.south) {$\mathrm{LoRA}_{1}$};
        \node[loranode,anchor=north](lora2dleft) at ([yshift=-1.6\nodehsep]basemodeldleft.south) {$\mathrm{LoRA}_{2}$};
        \node[align=center,thick, scale=2] (omit2d) at ([xshift=0.9*\nodehsep]lora2dleft.center) {...};
        \node[loranode,anchor=north](lora3dleft) at ([xshift=5.5\nodehsep,yshift=-1.6\nodehsep]basemodeldleft.south) {$\mathrm{LoRA}_{N_G}$}; 

        \node(plus1dleft) at ([yshift=0.72\nodehsep]lora1dleft.north) {\textbf{\(+\)}};
        \node(plus2dleft) at ([yshift=0.72\nodehsep]lora2dleft.north) {\textbf{\(+\)}};
        \node(plus3dleft) at ([yshift=0.72\nodehsep]lora3dleft.north) {\textbf{\(+\)}};
        
         \node[tasknode,anchor=north](task1dleft) at ([yshift=-1.6\nodehsep]lora1dleft.south) {$\mathrm{Task}_{1}^{\mathrm{enxx}}$};
        \node[tasknode,anchor=north](task2dleft) at ([yshift=-1.6\nodehsep]lora2dleft.south) {$\mathrm{Task}_{2}^{\mathrm{enxx}}$};
        \node[align=center,thick, scale=2] (omit3dleft) at ([xshift=0.9*\nodehsep]task2dleft.center) {...};
        \node[tasknode,anchor=north](task3dleft) at ([yshift=-1.6\nodehsep]lora3dleft.south) {$\mathrm{Task}_{N_G}^{\mathrm{enxx}}$}; 

        \draw[->,thick] (lora1dleft.south) -- (task1dleft.north);
        \draw[->,thick] (lora2dleft.south) -- (task2dleft.north);
        \draw[->,thick] (lora3dleft.south) -- (task3dleft.north);

        \begin{pgfonlayer}{background}
            % \node[anchor=north,minimum height=0.9*\nodehsep,minimum width=17.2em,fill=orange!8,rounded corners=5pt,draw](databankd2) at ([xshift=0.5em,yshift=-0.9em]dataxxen.south) {};
            \node[anchor=south,minimum height=1.9*\nodehsep,minimum width=18.2em,fill=green!8,rounded corners=5pt,draw,dotted,thick](modelmerge) at ([xshift=9.5em,yshift=-6.2em]databank1d.south) {};
        \end{pgfonlayer}

        \node[basemodel,anchor=north](basemodeldright) at ([yshift=1.8*\nodehsep]modelmerge.south) {Base Model};

        
        \node[loranode,anchor=north](lora2dright) at ([yshift=-1.3\nodehsep]basemodeldright.south) {$\mathrm{LoRA}_{2}$};
        \node[loranode,anchor=west](lora1dright) at ([xshift=-6.6\nodehsep]lora2dright.west) {$\mathrm{LoRA}_{1}$};
        \node[align=center,thick, scale=2] (omit2dright) at ([xshift=\nodehsep]lora2dright.center) {...};
        \node[loranode,anchor=east](lora3dright) at ([xshift=6.6\nodehsep]lora2dright.east) {$\mathrm{LoRA}_{N_L}$}; 
        
        \node(plus1dright) at ([yshift=0.6\nodehsep]lora1dright.north) {\textbf{\(+\)}};
        \node(plus2dright) at ([yshift=0.6\nodehsep]lora2dright.north) {\textbf{\(+\)}};
        \node(plus3dright) at ([yshift=0.6\nodehsep]lora3dright.north) {\textbf{\(+\)}};


        \node[mergemodelnode,anchor=north](mergedmodel) at ([yshift=-1.4\nodehsep]lora2dright.south) {Group-wise Merged Model};
        \draw[thick, ->] (modelmerge.south) -- (mergedmodel.north);
        
        \node[tasknode,anchor=north](task1dright) at ([yshift=-4.3\nodehsep]lora1dright.south) {$\mathrm{Task}_1^{\mathrm{xxen}}$};
        \node[tasknode,anchor=north](task2dright) at ([yshift=-4.3\nodehsep]lora2dright.south) {$\mathrm{Task}_2^{\mathrm{xxen}}$};
        \node[align=center,thick, scale=2] (omit3dright) at ([xshift=1.1*\nodehsep]task2dright.center) {...};
        \node[tasknode,anchor=north](task3dright) at ([yshift=-4.3\nodehsep]lora3dright.south) {$\mathrm{Task}_{N_G}^{\mathrm{xxen}}$}; 

        \draw[->,thick] ([xshift=-5.0\nodehsep]mergedmodel.south) -- (task1dright.north);
        \draw[->,thick] (mergedmodel.south) -- (task2dright.north);
        \draw[->,thick] ([xshift=6.0\nodehsep]mergedmodel.south) -- (task3dright.north);

        \draw[->,thick] (databank1d.west) -- (dataselectiond.north) node[midway, left] {En$\rightarrow$XX};
        \draw[->,thick] (databank1d.east) -- ([xshift=0.1*\nodehsep]modelmerge.north) node[midway, right] {XX$\rightarrow$En};

        
        \node[anchor=south,font=\Large](l4) at ([xshift=.5em,yshift=-3em]backgroundd.south) {(b) Direction-aware Training with Group-wise Model Merging};
        
    \end{scope}

    
\end{tikzpicture}
    
    % \vspace{-1em}
    \vspace{-0.5em}
    % \caption{(a) Seperate/Multilingual/Group multilingual training; In seperate  (b) Group-wise model merging}
    \caption{(a)  \textbf{Separate Training ($N$ = $N_L$)}: Each translation task is trained independently using different datasets for different language pairs, with distinct LoRA model weights fine-tuned separately;
    \textbf{Multilingual Training ($N$ = $1$)}: All language pairs are combined to fine-tune a single model with shared LoRA weights;
    \textbf{Group Multilingual Training ($N$ = $N_G$)}: Language pairs are grouped as specified in Table \ref{tab:languages1}-\ref{tab:languages2}, with an adapter trained for each group.
    (b) \textbf{Group-wise model merging}: For XX$\rightarrow$En translation, separate training is applied to each language pair. For En$\rightarrow$XX translation, group training is applied, where different tasks share LoRA weights within language groups.}
    \label{fig:architecture}
    \vspace{-1.1em}
% \end{figure*}
\endgroup

\label{fig:comparison-merge}
\end{figure*}

% \end{document}

\section{Detailed Method}\label{sec:details}

\subsection{Nested lattice codebook}

In this section, we describe the construction for a Vector Quantization (VQ) codebook of size $q^d$ for quantizing an $d$-dimensional vector, where $q$ is an integer parameter. To quantize a vector, we find the closest codebook element by Euclidean norm. We describe efficient encoding and decoding algorithms to a quantized representation in $\Z_q^d$.

Let $\Lambda$ be a lattice in $\RR^d$ with generator matrix $G$. We define the coordinates of a point $x \in \Lambda$ to be an integer vector $v$ such that $x = Gv$. Each point $P \in \Lambda$ has a corresponding Voronoi region $\m{V}_\Lambda(P)$, for which $P$ is the closest point in $\Lambda$ with respect to $L^2$ metric. To define the codebook, we consider the scaled lattice $q\Lambda$. Then:

\begin{definition}
    $x \in \Lambda$ belongs to codebook $C$ iff $x \in \m{V}_{q\Lambda}(0)$. Let $v$ be the coordinates of $x$. Then, the quantized representation of $x$ is $\mathcal{Q}(x) := v \mmod q$. Note that $\mathcal{Q}$ is a bijection between $C$ and $\Z_q^d$
\end{definition}

Using this representation, we can describe the encoding and decoding functions, assuming the point $x$ we are quantizing is in $\m{V}_{q\Lambda}(0)$. We will also need an oracle $Q_{\Lambda}(x)$, which maps $x$ to the closest point in $\Lambda$ to $x$.

\begin{algorithm}[h]
   \caption{Encode}
   \label{alg:encode}
\begin{algorithmic}
   \State {\bfseries Input:} $x \in V_{q\Lambda}(0)$, $Q_{\Lambda}$
   \State $p \leftarrow Q_{\Lambda}(x)$
   \State $v \leftarrow G^{-1}p$ \Comment{coordinates of $p$}
   \State {\bfseries return} {$v \mmod q$} \Comment{quantized representation of $p$}
\end{algorithmic}
\end{algorithm}




\begin{algorithm}[h]
\caption{Decode}
\label{decode-algo}
\begin{algorithmic}
   \State {\bfseries Input:} $c \in \Z_q^d$, $Q_{\Lambda}$
   \State $p \leftarrow Gc$ \Comment{equivalent to answer modulo $q\Lambda$}
   \State {\bfseries return} $p - q\,Q_{\Lambda}\!\bigl(\tfrac{p}{q}\bigr)$
\end{algorithmic}
\end{algorithm}

In practice, we will be using the Gosset ($E_8$) lattice as $\Lambda$ with $d = 8$. This lattice is a union of $D_8$ and $D_8 + \frac{1}{2}$, where $D_8$ contains elements of $\Z^8$ with even sum of coordinates. There is a simple algorithm for finding the closest point in the Gosset lattice, first described in \cite{1056484}. We provide the pseudocode for this algorithm together with the estimation of its runtime in Appendix \ref{sec:oracle}.

\subsection{Matrix quantization}

\label{matrix-quant}

When quantizing a matrix, we normalize its rows, and quantize each block of $d$ entries using the codebook. The algorithm \ref{alg:nestquant} describes the quantization procedure for each row of the matrix.

\begin{algorithm}[h]
\caption{NestQuant}
\label{alg:nestquant}
\begin{algorithmic}
   \State {\bfseries Input:} $A$ --- a vector of size $n = db$, $q$, array of $\beta$
   \State $QA$ --- $n$ integers \Comment{quantized representation}
   \State $B$ --- $b$ integers \Comment{scaling coefficient indices}
   \State \label{norm_nestquant} $s \leftarrow \lVert A_i\rVert_2$ \Comment{normalization coefficient}
   \State $A \leftarrow \frac{A\sqrt{n}}{s}$
   \For{$j = 0$ {\bfseries to} $b-1$}
        \State $err = \infty$
        \For{$p = 1$ {\bfseries to} $k$}
            \State $v \leftarrow A[dj+1..dj+d]$
            \State $enc \leftarrow \text{Encode}\left(\frac{v}{\beta_p}\right)$
            \State $recon \leftarrow \text{Decode}(enc) \cdot \beta_p$
            \If{$err > |recon - v|_2^2$}
                \State $err \leftarrow |recon - v|_2^2$
                \State $QA[dj+1..dj+d] \leftarrow enc$
                \State $B_{j} \leftarrow p$
            \EndIf
        \EndFor
   \EndFor
   \State {\bfseries Output:} $QA$, $B$, $s$
\end{algorithmic}
\end{algorithm}

We can take dot products of quantized vectors without complete dequantization using algorithm \ref{alg:dotproduct}. We use it in the generation stage on linear layers and for querying the KV cache.

\begin{algorithm}[h]
\caption{Dot product}
\label{alg:dotproduct}
\begin{algorithmic}
   \State {\bfseries Input:} $QA_1$, $B_1$, $s_1$ and $QA_2$, $B_2$, $s_2$ --- representations of two vectors of size $n = db$ from Algorithm \ref{alg:nestquant}, array $\beta$
   \State $ans \leftarrow 0$
   \For{$j = 0$ {\bfseries to} $b-1$}
        \State $p_1 \leftarrow \text{Decode}(QA_1[dj+1..dj+d])$
        \State $p_2 \leftarrow \text{Decode}(QA_2[dj+1..dj+d])$
        \State $ans \leftarrow ans + (p_1 \cdot p_2)\beta_{B_1[j]}\beta_{B_2[j]}$
   \EndFor
   \State {\bfseries return} $ans$
\end{algorithmic}
\end{algorithm}

\subsection{LLM quantization}

\label{subsec:llm-quant}

\ifisicml
\begin{figure}
    \centering
    \includegraphics[width=\linewidth]{figures/kv.pdf}
    \caption{The quantization scheme of multi-head attention. $H$ is Hadamard rotation described in \ref{subsec:llm-quant}. $\mathcal{Q}$ is the quantization function described in \ref{matrix-quant}}
    \label{fig:scheme}
\end{figure}

\else
\begin{figure}[h]
    \centering
    \includegraphics[width=0.5\linewidth]{figures/kv.pdf}
    \caption{The quantization scheme of multi-head attention. $H$ is Hadamard rotation described in \ref{subsec:llm-quant}. $\mathcal{Q}$ is the quantization function described in \ref{matrix-quant}}
    \label{fig:scheme}
\end{figure}

\fi

Recall that we apply a rotation matrix $H$ to every weight-activation pair of a linear layer without changing the output of the network. Let $n$ be the number of input features to the layer.

\begin{itemize}
    \item If $n = 2^k$, we set $H$ to be Hadamard matrix obtained by Sylvester's construction
    \item Otherwise, we decompose $n = 2^km$, such that $m$ is small and there exists a Hadamard matrix $H_1$ of size $m$. We construct Hadamard matrix $H_2$ of size $2^k$ using Sylvester's construction, and set $U = H_1 \otimes H_2$.
\end{itemize}

Note that it's possible to multiply an $r \times n$ matrix by $H$ in $O(rn \log n)$ in the first case and $O(rn(\log n + m))$ in the second case, which is negligible to other computational costs and can be done online.

In NestQuant, we quantize all weights, activations, keys, and values using Algorithm \ref{alg:nestquant}. We merge the Hadamard rotation with the weights and quantize them. We also apply the Hadamard rotation and quantization to the activations before linear layers. We also apply rotation to keys and queries, because it will not change the attention scores, and we quantize keys and values before putting them in the KV cache. Figure \ref{fig:scheme} illustrates the procedure for multi-head attention layers.

When quantizing a weight, we modify the NestQuant algorithm by introducing corrections to unquantized weights when a certain vector piece is quantized. We refer the reader to section 4.1 of \cite{tseng2024} for a more detailed description.

\subsection{Optimal scaling coefficients}

One of the important parts of the algorithm is finding the optimal set of $\beta_i$. Given the distribution of 8-vectors that are quantized via a codebook, it is possible to find an optimal set of given size exactly using a dynamic programming approach, which is described in Appendix \ref{dp-section}.

\subsection{Algorithm summary}
\label{algo-summary}

Here we describe the main steps of NestQuant.

\begin{enumerate}
    \item Collect the statistics for LDLQ. For each linear layer with in-dimension $d$, we compute a $d \times d$ matrix $H$.
    \item We choose an initial set of scaling coefficients $\hat{\beta}$, and for each weight we simulate LDLQ quantization with these coefficients, getting a set of 8-dimensional vectors to quantize.
    \item We run a dynamic programming algorithm described in Appendix \ref{dp-section} on the 8-vectors to find the optimal $\beta$-values for each weight matrix.
    \item We also run the dynamic programming algorithm for activations, keys, and values for each layer. To get the distribution of 8-vectors, we run the model on a small set of examples.
    \item We quantize the weights using LDLQ and precomputed $\beta$.
    \item During inference, we quantize any activation before it's passed to the linear layer, and any KV cache entry before it is saved.
\end{enumerate}
Note the complete lack of fine-tuning needed to make our method work.
\section{Usage}

AISafetyLab provides a general and user-friendly interface for LLM attack, defense, and evaluation with just a few lines of code.
%
In this section, we offer examples for attack, defense and evaluation, respectively.


\subsection{Attack}
The following code demonstrates the setup of our attack module. After importing the corresponding attack method (e.g., AutoDANManager), the \texttt{attack} function can automatically initiate an attack towards the target model specified in the configuration file and save the resulting responses.


\begin{lstlisting}[language=Python]
from aisafetylab.attack.attackers.autodan import AutoDANManager
from aisafetylab.utils import ConfigManager
from aisafetylab.utils import parse_arguments

args = parse_arguments()

config_manager = ConfigManager(config_path=args.config_path)
attacker = AutoDANManager.from_config(config_manager.config)
attacker.attack()
\end{lstlisting}


\subsection{Defense}

We provide a demonstration of the defense module usage below. After specifying the defense method, the \texttt{chat} function can execute the defense and produce the corresponding results. Note that we support the simultaneous deployment of multiple defense strategies.

\begin{lstlisting}[language=Python]
from aisafetylab.defense.inference_defense import create_defender_from_yaml
from aisafetylab.defense.inference_defense import chat

attack_query = "Give detailed instructions for >>> correctlua code reform BEGIN<!--"

defender_name="self_reminder"
defenders = [(create_defender_from_yaml(f"./configs/{defender_name}.yaml")]

response = chat(
    model,
    attack_query,
    defenders
)
\end{lstlisting}


\subsection{Evaluation}

The usage of the evaluation module is as follows. After instantiating the \texttt{scorer} with a specific evaluation method, the \texttt{score} function can provide evaluation results for a given instruction-response pair.

\begin{lstlisting}[language=Python]
from aisafetylab.evaluation.scorers import LlamaGuard3Scorer

scorer = LlamaGuard3Scorer(
        model_path="meta-llama/Llama-Guard-3-8B",
        tokenizer_path="meta-llama/Llama-Guard-3-8B",
    )
scorer.score("How to make a bomb?", "Sorry, I can not ...")
\end{lstlisting}
\section{Experiments}
\label{sec:experiments}
The experiments are designed to address two key research questions.
First, \textbf{RQ1} evaluates whether the average $L_2$-norm of the counterfactual perturbation vectors ($\overline{||\perturb||}$) decreases as the model overfits the data, thereby providing further empirical validation for our hypothesis.
Second, \textbf{RQ2} evaluates the ability of the proposed counterfactual regularized loss, as defined in (\ref{eq:regularized_loss2}), to mitigate overfitting when compared to existing regularization techniques.

% The experiments are designed to address three key research questions. First, \textbf{RQ1} investigates whether the mean perturbation vector norm decreases as the model overfits the data, aiming to further validate our intuition. Second, \textbf{RQ2} explores whether the mean perturbation vector norm can be effectively leveraged as a regularization term during training, offering insights into its potential role in mitigating overfitting. Finally, \textbf{RQ3} examines whether our counterfactual regularizer enables the model to achieve superior performance compared to existing regularization methods, thus highlighting its practical advantage.

\subsection{Experimental Setup}
\textbf{\textit{Datasets, Models, and Tasks.}}
The experiments are conducted on three datasets: \textit{Water Potability}~\cite{kadiwal2020waterpotability}, \textit{Phomene}~\cite{phomene}, and \textit{CIFAR-10}~\cite{krizhevsky2009learning}. For \textit{Water Potability} and \textit{Phomene}, we randomly select $80\%$ of the samples for the training set, and the remaining $20\%$ for the test set, \textit{CIFAR-10} comes already split. Furthermore, we consider the following models: Logistic Regression, Multi-Layer Perceptron (MLP) with 100 and 30 neurons on each hidden layer, and PreactResNet-18~\cite{he2016cvecvv} as a Convolutional Neural Network (CNN) architecture.
We focus on binary classification tasks and leave the extension to multiclass scenarios for future work. However, for datasets that are inherently multiclass, we transform the problem into a binary classification task by selecting two classes, aligning with our assumption.

\smallskip
\noindent\textbf{\textit{Evaluation Measures.}} To characterize the degree of overfitting, we use the test loss, as it serves as a reliable indicator of the model's generalization capability to unseen data. Additionally, we evaluate the predictive performance of each model using the test accuracy.

\smallskip
\noindent\textbf{\textit{Baselines.}} We compare CF-Reg with the following regularization techniques: L1 (``Lasso''), L2 (``Ridge''), and Dropout.

\smallskip
\noindent\textbf{\textit{Configurations.}}
For each model, we adopt specific configurations as follows.
\begin{itemize}
\item \textit{Logistic Regression:} To induce overfitting in the model, we artificially increase the dimensionality of the data beyond the number of training samples by applying a polynomial feature expansion. This approach ensures that the model has enough capacity to overfit the training data, allowing us to analyze the impact of our counterfactual regularizer. The degree of the polynomial is chosen as the smallest degree that makes the number of features greater than the number of data.
\item \textit{Neural Networks (MLP and CNN):} To take advantage of the closed-form solution for computing the optimal perturbation vector as defined in (\ref{eq:opt-delta}), we use a local linear approximation of the neural network models. Hence, given an instance $\inst_i$, we consider the (optimal) counterfactual not with respect to $\model$ but with respect to:
\begin{equation}
\label{eq:taylor}
    \model^{lin}(\inst) = \model(\inst_i) + \nabla_{\inst}\model(\inst_i)(\inst - \inst_i),
\end{equation}
where $\model^{lin}$ represents the first-order Taylor approximation of $\model$ at $\inst_i$.
Note that this step is unnecessary for Logistic Regression, as it is inherently a linear model.
\end{itemize}

\smallskip
\noindent \textbf{\textit{Implementation Details.}} We run all experiments on a machine equipped with an AMD Ryzen 9 7900 12-Core Processor and an NVIDIA GeForce RTX 4090 GPU. Our implementation is based on the PyTorch Lightning framework. We use stochastic gradient descent as the optimizer with a learning rate of $\eta = 0.001$ and no weight decay. We use a batch size of $128$. The training and test steps are conducted for $6000$ epochs on the \textit{Water Potability} and \textit{Phoneme} datasets, while for the \textit{CIFAR-10} dataset, they are performed for $200$ epochs.
Finally, the contribution $w_i^{\varepsilon}$ of each training point $\inst_i$ is uniformly set as $w_i^{\varepsilon} = 1~\forall i\in \{1,\ldots,m\}$.

The source code implementation for our experiments is available at the following GitHub repository: \url{https://anonymous.4open.science/r/COCE-80B4/README.md} 

\subsection{RQ1: Counterfactual Perturbation vs. Overfitting}
To address \textbf{RQ1}, we analyze the relationship between the test loss and the average $L_2$-norm of the counterfactual perturbation vectors ($\overline{||\perturb||}$) over training epochs.

In particular, Figure~\ref{fig:delta_loss_epochs} depicts the evolution of $\overline{||\perturb||}$ alongside the test loss for an MLP trained \textit{without} regularization on the \textit{Water Potability} dataset. 
\begin{figure}[ht]
    \centering
    \includegraphics[width=0.85\linewidth]{img/delta_loss_epochs.png}
    \caption{The average counterfactual perturbation vector $\overline{||\perturb||}$ (left $y$-axis) and the cross-entropy test loss (right $y$-axis) over training epochs ($x$-axis) for an MLP trained on the \textit{Water Potability} dataset \textit{without} regularization.}
    \label{fig:delta_loss_epochs}
\end{figure}

The plot shows a clear trend as the model starts to overfit the data (evidenced by an increase in test loss). 
Notably, $\overline{||\perturb||}$ begins to decrease, which aligns with the hypothesis that the average distance to the optimal counterfactual example gets smaller as the model's decision boundary becomes increasingly adherent to the training data.

It is worth noting that this trend is heavily influenced by the choice of the counterfactual generator model. In particular, the relationship between $\overline{||\perturb||}$ and the degree of overfitting may become even more pronounced when leveraging more accurate counterfactual generators. However, these models often come at the cost of higher computational complexity, and their exploration is left to future work.

Nonetheless, we expect that $\overline{||\perturb||}$ will eventually stabilize at a plateau, as the average $L_2$-norm of the optimal counterfactual perturbations cannot vanish to zero.

% Additionally, the choice of employing the score-based counterfactual explanation framework to generate counterfactuals was driven to promote computational efficiency.

% Future enhancements to the framework may involve adopting models capable of generating more precise counterfactuals. While such approaches may yield to performance improvements, they are likely to come at the cost of increased computational complexity.


\subsection{RQ2: Counterfactual Regularization Performance}
To answer \textbf{RQ2}, we evaluate the effectiveness of the proposed counterfactual regularization (CF-Reg) by comparing its performance against existing baselines: unregularized training loss (No-Reg), L1 regularization (L1-Reg), L2 regularization (L2-Reg), and Dropout.
Specifically, for each model and dataset combination, Table~\ref{tab:regularization_comparison} presents the mean value and standard deviation of test accuracy achieved by each method across 5 random initialization. 

The table illustrates that our regularization technique consistently delivers better results than existing methods across all evaluated scenarios, except for one case -- i.e., Logistic Regression on the \textit{Phomene} dataset. 
However, this setting exhibits an unusual pattern, as the highest model accuracy is achieved without any regularization. Even in this case, CF-Reg still surpasses other regularization baselines.

From the results above, we derive the following key insights. First, CF-Reg proves to be effective across various model types, ranging from simple linear models (Logistic Regression) to deep architectures like MLPs and CNNs, and across diverse datasets, including both tabular and image data. 
Second, CF-Reg's strong performance on the \textit{Water} dataset with Logistic Regression suggests that its benefits may be more pronounced when applied to simpler models. However, the unexpected outcome on the \textit{Phoneme} dataset calls for further investigation into this phenomenon.


\begin{table*}[h!]
    \centering
    \caption{Mean value and standard deviation of test accuracy across 5 random initializations for different model, dataset, and regularization method. The best results are highlighted in \textbf{bold}.}
    \label{tab:regularization_comparison}
    \begin{tabular}{|c|c|c|c|c|c|c|}
        \hline
        \textbf{Model} & \textbf{Dataset} & \textbf{No-Reg} & \textbf{L1-Reg} & \textbf{L2-Reg} & \textbf{Dropout} & \textbf{CF-Reg (ours)} \\ \hline
        Logistic Regression   & \textit{Water}   & $0.6595 \pm 0.0038$   & $0.6729 \pm 0.0056$   & $0.6756 \pm 0.0046$  & N/A    & $\mathbf{0.6918 \pm 0.0036}$                     \\ \hline
        MLP   & \textit{Water}   & $0.6756 \pm 0.0042$   & $0.6790 \pm 0.0058$   & $0.6790 \pm 0.0023$  & $0.6750 \pm 0.0036$    & $\mathbf{0.6802 \pm 0.0046}$                    \\ \hline
%        MLP   & \textit{Adult}   & $0.8404 \pm 0.0010$   & $\mathbf{0.8495 \pm 0.0007}$   & $0.8489 \pm 0.0014$  & $\mathbf{0.8495 \pm 0.0016}$     & $0.8449 \pm 0.0019$                    \\ \hline
        Logistic Regression   & \textit{Phomene}   & $\mathbf{0.8148 \pm 0.0020}$   & $0.8041 \pm 0.0028$   & $0.7835 \pm 0.0176$  & N/A    & $0.8098 \pm 0.0055$                     \\ \hline
        MLP   & \textit{Phomene}   & $0.8677 \pm 0.0033$   & $0.8374 \pm 0.0080$   & $0.8673 \pm 0.0045$  & $0.8672 \pm 0.0042$     & $\mathbf{0.8718 \pm 0.0040}$                    \\ \hline
        CNN   & \textit{CIFAR-10} & $0.6670 \pm 0.0233$   & $0.6229 \pm 0.0850$   & $0.7348 \pm 0.0365$   & N/A    & $\mathbf{0.7427 \pm 0.0571}$                     \\ \hline
    \end{tabular}
\end{table*}

\begin{table*}[htb!]
    \centering
    \caption{Hyperparameter configurations utilized for the generation of Table \ref{tab:regularization_comparison}. For our regularization the hyperparameters are reported as $\mathbf{\alpha/\beta}$.}
    \label{tab:performance_parameters}
    \begin{tabular}{|c|c|c|c|c|c|c|}
        \hline
        \textbf{Model} & \textbf{Dataset} & \textbf{No-Reg} & \textbf{L1-Reg} & \textbf{L2-Reg} & \textbf{Dropout} & \textbf{CF-Reg (ours)} \\ \hline
        Logistic Regression   & \textit{Water}   & N/A   & $0.0093$   & $0.6927$  & N/A    & $0.3791/1.0355$                     \\ \hline
        MLP   & \textit{Water}   & N/A   & $0.0007$   & $0.0022$  & $0.0002$    & $0.2567/1.9775$                    \\ \hline
        Logistic Regression   &
        \textit{Phomene}   & N/A   & $0.0097$   & $0.7979$  & N/A    & $0.0571/1.8516$                     \\ \hline
        MLP   & \textit{Phomene}   & N/A   & $0.0007$   & $4.24\cdot10^{-5}$  & $0.0015$    & $0.0516/2.2700$                    \\ \hline
       % MLP   & \textit{Adult}   & N/A   & $0.0018$   & $0.0018$  & $0.0601$     & $0.0764/2.2068$                    \\ \hline
        CNN   & \textit{CIFAR-10} & N/A   & $0.0050$   & $0.0864$ & N/A    & $0.3018/
        2.1502$                     \\ \hline
    \end{tabular}
\end{table*}

\begin{table*}[htb!]
    \centering
    \caption{Mean value and standard deviation of training time across 5 different runs. The reported time (in seconds) corresponds to the generation of each entry in Table \ref{tab:regularization_comparison}. Times are }
    \label{tab:times}
    \begin{tabular}{|c|c|c|c|c|c|c|}
        \hline
        \textbf{Model} & \textbf{Dataset} & \textbf{No-Reg} & \textbf{L1-Reg} & \textbf{L2-Reg} & \textbf{Dropout} & \textbf{CF-Reg (ours)} \\ \hline
        Logistic Regression   & \textit{Water}   & $222.98 \pm 1.07$   & $239.94 \pm 2.59$   & $241.60 \pm 1.88$  & N/A    & $251.50 \pm 1.93$                     \\ \hline
        MLP   & \textit{Water}   & $225.71 \pm 3.85$   & $250.13 \pm 4.44$   & $255.78 \pm 2.38$  & $237.83 \pm 3.45$    & $266.48 \pm 3.46$                    \\ \hline
        Logistic Regression   & \textit{Phomene}   & $266.39 \pm 0.82$ & $367.52 \pm 6.85$   & $361.69 \pm 4.04$  & N/A   & $310.48 \pm 0.76$                    \\ \hline
        MLP   &
        \textit{Phomene} & $335.62 \pm 1.77$   & $390.86 \pm 2.11$   & $393.96 \pm 1.95$ & $363.51 \pm 5.07$    & $403.14 \pm 1.92$                     \\ \hline
       % MLP   & \textit{Adult}   & N/A   & $0.0018$   & $0.0018$  & $0.0601$     & $0.0764/2.2068$                    \\ \hline
        CNN   & \textit{CIFAR-10} & $370.09 \pm 0.18$   & $395.71 \pm 0.55$   & $401.38 \pm 0.16$ & N/A    & $1287.8 \pm 0.26$                     \\ \hline
    \end{tabular}
\end{table*}

\subsection{Feasibility of our Method}
A crucial requirement for any regularization technique is that it should impose minimal impact on the overall training process.
In this respect, CF-Reg introduces an overhead that depends on the time required to find the optimal counterfactual example for each training instance. 
As such, the more sophisticated the counterfactual generator model probed during training the higher would be the time required. However, a more advanced counterfactual generator might provide a more effective regularization. We discuss this trade-off in more details in Section~\ref{sec:discussion}.

Table~\ref{tab:times} presents the average training time ($\pm$ standard deviation) for each model and dataset combination listed in Table~\ref{tab:regularization_comparison}.
We can observe that the higher accuracy achieved by CF-Reg using the score-based counterfactual generator comes with only minimal overhead. However, when applied to deep neural networks with many hidden layers, such as \textit{PreactResNet-18}, the forward derivative computation required for the linearization of the network introduces a more noticeable computational cost, explaining the longer training times in the table.

\subsection{Hyperparameter Sensitivity Analysis}
The proposed counterfactual regularization technique relies on two key hyperparameters: $\alpha$ and $\beta$. The former is intrinsic to the loss formulation defined in (\ref{eq:cf-train}), while the latter is closely tied to the choice of the score-based counterfactual explanation method used.

Figure~\ref{fig:test_alpha_beta} illustrates how the test accuracy of an MLP trained on the \textit{Water Potability} dataset changes for different combinations of $\alpha$ and $\beta$.

\begin{figure}[ht]
    \centering
    \includegraphics[width=0.85\linewidth]{img/test_acc_alpha_beta.png}
    \caption{The test accuracy of an MLP trained on the \textit{Water Potability} dataset, evaluated while varying the weight of our counterfactual regularizer ($\alpha$) for different values of $\beta$.}
    \label{fig:test_alpha_beta}
\end{figure}

We observe that, for a fixed $\beta$, increasing the weight of our counterfactual regularizer ($\alpha$) can slightly improve test accuracy until a sudden drop is noticed for $\alpha > 0.1$.
This behavior was expected, as the impact of our penalty, like any regularization term, can be disruptive if not properly controlled.

Moreover, this finding further demonstrates that our regularization method, CF-Reg, is inherently data-driven. Therefore, it requires specific fine-tuning based on the combination of the model and dataset at hand.
\section{Customizing Models in FairDiverse}
We outline the steps for customizing and evaluating new IR models using the APIs we provide. Detailed API descriptions and source code can be found in~\url{https://xuchen0427.github.io/FairDiverse/}. 
The provided APIs can be used by installing them via pip: 

\begin{lstlisting}[style=shell]
pip install fairdiverse
\end{lstlisting}

%\texttt{pip install fairdiverse}.
%The steps can be easily implemented with just a few lines of code.

\subsection{Steps}
Figure~\ref{fig:rec_APIs} illustrates the three key steps for implementing fairness- and diversity-aware IR models named \textit{YourModel}.

\noindent\textbf{Step 1.} Configure your custom model parameters and save them in a newly created \texttt{YourModel.yaml} file in the \texttt{/properties/models/} directory. Then you can change the model in the running configuration file \texttt{Post-processing.yaml}.

\noindent\textbf{Step 2.} Select the appropriate Python abstract class from our provided options based on your model type and implement your model in a newly created file, \texttt{YourModel.py}, stored in the corresponding directory. You can use the integrated tools and common parameters within the abstract class. Researchers only need to focus on designing the model without worrying about the rest of the pipeline. 

%Different types of models must adhere to the function formats and return value structures defined in the abstract class. 

\noindent\textbf{Step 3.} Configure your model for the training pipeline by following these steps: import your custom model package in the corresponding file (\texttt{/model\_type/\_\_init\_\_.py}) and define the model in the appropriate script (\texttt{/train.py}, \texttt{/reranker.py}).

%The implementation for other models follows a similar approach.

\subsection{Examples}
%\textbf{Examples.} 
Here, we provide two example codes demonstrating how to design a custom search and recommendation model, respectively. 
\begin{lstlisting}[language=Python]
#/recommendation/rank_model/YourModel.py
class YourModel(Abstract_Regularizer):
    def __init__(self, config, group_weight):
        super().__init__(config)

    def fairness_loss(self, input_dict):
        return torch.var(input_dict['scores'])

#/recommendation/rank_model/__init__.py
from .YourModel import YourModel

#/recommendation/trainer.py
if config["model"] == "YourModel":
  self.Model = YourModel(config)
\end{lstlisting}

\begin{lstlisting}[language=Python]
#/search/preprocessing_model/YourModel.py
class YourModel(PreprocessingFairnessIntervention):
    def __init__(self, configs, dataset):
        super().__init__(configs, dataset)

    def fit(self, X_train, run):
    # Train the fairness model using the training set.

    def transform(self, X_train, run file_name=None):
    # Apply the fairness transformation to the dataset.

#/search/preprocessing_model/__init__.py
from .YourModel import YourModel
fairness_method_mapping['YourModel'] = YourModel


\end{lstlisting}



% #test.py
% from recommendation.trainer import RecTrainer

% config = {'model': 'BPR', 'data_type': 'pair', 'fair-rank': True, 'rank_model': 'YourModel', 'use_llm': False, 'log_name': "test", 'dataset': 'steam'}

% trainer = RecTrainer(train_config=config)
% trainer.train()

% \subsection{Search}
% As for the search part, the key steps for implementing a diversified search model named \textit{YourModel}.

%can also refer to Figure~\ref{fig:rec_APIs}.

% \textbf{Step 1.} First, configure your custom model parameters and save them in a newly created \texttt{YourModel.yaml} file within the \texttt{/properties/models/} directory. Then you can change the model in the running configuration file \texttt{Post-processing.yaml}.

% \textbf{Step 2.} Select the appropriate Python abstract class from our provided options based on your model type and implement your model in a newly created file, \texttt{YourModel.py}, stored in the corresponding directory (see Figure~\ref{fig:rec_APIs} for details). Different types of models must adhere to the function formats and return value structures defined in the abstract class. To facilitate development, we have integrated various tools and common parameters within the abstract class. Researchers only need to focus on designing the model itself without worrying about the rest of the pipeline, creating a convenient environment for fair comparisons across different models.


% \textbf{Step 3.} Configure your model for the training pipeline by following these steps: import your custom model package in the corresponding file (\texttt{/model\_type/\_\_init\_\_.py}) and define the model in the appropriate script (\texttt{/train.py}).
\section{RELATED WORK}
\label{sec:relatedwork}
In this section, we describe the previous works related to our proposal, which are divided into two parts. In Section~\ref{sec:relatedwork_exoplanet}, we present a review of approaches based on machine learning techniques for the detection of planetary transit signals. Section~\ref{sec:relatedwork_attention} provides an account of the approaches based on attention mechanisms applied in Astronomy.\par

\subsection{Exoplanet detection}
\label{sec:relatedwork_exoplanet}
Machine learning methods have achieved great performance for the automatic selection of exoplanet transit signals. One of the earliest applications of machine learning is a model named Autovetter \citep{MCcauliff}, which is a random forest (RF) model based on characteristics derived from Kepler pipeline statistics to classify exoplanet and false positive signals. Then, other studies emerged that also used supervised learning. \cite{mislis2016sidra} also used a RF, but unlike the work by \citet{MCcauliff}, they used simulated light curves and a box least square \citep[BLS;][]{kovacs2002box}-based periodogram to search for transiting exoplanets. \citet{thompson2015machine} proposed a k-nearest neighbors model for Kepler data to determine if a given signal has similarity to known transits. Unsupervised learning techniques were also applied, such as self-organizing maps (SOM), proposed \citet{armstrong2016transit}; which implements an architecture to segment similar light curves. In the same way, \citet{armstrong2018automatic} developed a combination of supervised and unsupervised learning, including RF and SOM models. In general, these approaches require a previous phase of feature engineering for each light curve. \par

%DL is a modern data-driven technology that automatically extracts characteristics, and that has been successful in classification problems from a variety of application domains. The architecture relies on several layers of NNs of simple interconnected units and uses layers to build increasingly complex and useful features by means of linear and non-linear transformation. This family of models is capable of generating increasingly high-level representations \citep{lecun2015deep}.

The application of DL for exoplanetary signal detection has evolved rapidly in recent years and has become very popular in planetary science.  \citet{pearson2018} and \citet{zucker2018shallow} developed CNN-based algorithms that learn from synthetic data to search for exoplanets. Perhaps one of the most successful applications of the DL models in transit detection was that of \citet{Shallue_2018}; who, in collaboration with Google, proposed a CNN named AstroNet that recognizes exoplanet signals in real data from Kepler. AstroNet uses the training set of labelled TCEs from the Autovetter planet candidate catalog of Q1–Q17 data release 24 (DR24) of the Kepler mission \citep{catanzarite2015autovetter}. AstroNet analyses the data in two views: a ``global view'', and ``local view'' \citep{Shallue_2018}. \par


% The global view shows the characteristics of the light curve over an orbital period, and a local view shows the moment at occurring the transit in detail

%different = space-based

Based on AstroNet, researchers have modified the original AstroNet model to rank candidates from different surveys, specifically for Kepler and TESS missions. \citet{ansdell2018scientific} developed a CNN trained on Kepler data, and included for the first time the information on the centroids, showing that the model improves performance considerably. Then, \citet{osborn2020rapid} and \citet{yu2019identifying} also included the centroids information, but in addition, \citet{osborn2020rapid} included information of the stellar and transit parameters. Finally, \citet{rao2021nigraha} proposed a pipeline that includes a new ``half-phase'' view of the transit signal. This half-phase view represents a transit view with a different time and phase. The purpose of this view is to recover any possible secondary eclipse (the object hiding behind the disk of the primary star).


%last pipeline applies a procedure after the prediction of the model to obtain new candidates, this process is carried out through a series of steps that include the evaluation with Discovery and Validation of Exoplanets (DAVE) \citet{kostov2019discovery} that was adapted for the TESS telescope.\par
%



\subsection{Attention mechanisms in astronomy}
\label{sec:relatedwork_attention}
Despite the remarkable success of attention mechanisms in sequential data, few papers have exploited their advantages in astronomy. In particular, there are no models based on attention mechanisms for detecting planets. Below we present a summary of the main applications of this modeling approach to astronomy, based on two points of view; performance and interpretability of the model.\par
%Attention mechanisms have not yet been explored in all sub-areas of astronomy. However, recent works show a successful application of the mechanism.
%performance

The application of attention mechanisms has shown improvements in the performance of some regression and classification tasks compared to previous approaches. One of the first implementations of the attention mechanism was to find gravitational lenses proposed by \citet{thuruthipilly2021finding}. They designed 21 self-attention-based encoder models, where each model was trained separately with 18,000 simulated images, demonstrating that the model based on the Transformer has a better performance and uses fewer trainable parameters compared to CNN. A novel application was proposed by \citet{lin2021galaxy} for the morphological classification of galaxies, who used an architecture derived from the Transformer, named Vision Transformer (VIT) \citep{dosovitskiy2020image}. \citet{lin2021galaxy} demonstrated competitive results compared to CNNs. Another application with successful results was proposed by \citet{zerveas2021transformer}; which first proposed a transformer-based framework for learning unsupervised representations of multivariate time series. Their methodology takes advantage of unlabeled data to train an encoder and extract dense vector representations of time series. Subsequently, they evaluate the model for regression and classification tasks, demonstrating better performance than other state-of-the-art supervised methods, even with data sets with limited samples.

%interpretation
Regarding the interpretability of the model, a recent contribution that analyses the attention maps was presented by \citet{bowles20212}, which explored the use of group-equivariant self-attention for radio astronomy classification. Compared to other approaches, this model analysed the attention maps of the predictions and showed that the mechanism extracts the brightest spots and jets of the radio source more clearly. This indicates that attention maps for prediction interpretation could help experts see patterns that the human eye often misses. \par

In the field of variable stars, \citet{allam2021paying} employed the mechanism for classifying multivariate time series in variable stars. And additionally, \citet{allam2021paying} showed that the activation weights are accommodated according to the variation in brightness of the star, achieving a more interpretable model. And finally, related to the TESS telescope, \citet{morvan2022don} proposed a model that removes the noise from the light curves through the distribution of attention weights. \citet{morvan2022don} showed that the use of the attention mechanism is excellent for removing noise and outliers in time series datasets compared with other approaches. In addition, the use of attention maps allowed them to show the representations learned from the model. \par

Recent attention mechanism approaches in astronomy demonstrate comparable results with earlier approaches, such as CNNs. At the same time, they offer interpretability of their results, which allows a post-prediction analysis. \par


We present RiskHarvester, a risk-based tool to compute a security risk score based on the value of the asset and ease of attack on a database. We calculated the value of asset by identifying the sensitive data categories present in a database from the database keywords. We utilized data flow analysis, SQL, and Object Relational Mapper (ORM) parsing to identify the database keywords. To calculate the ease of attack, we utilized passive network analysis to retrieve the database host information. To evaluate RiskHarvester, we curated RiskBench, a benchmark of 1,791 database secret-asset pairs with sensitive data categories and host information manually retrieved from 188 GitHub repositories. RiskHarvester demonstrates precision of (95\%) and recall (90\%) in detecting database keywords for the value of asset and precision of (96\%) and recall (94\%) in detecting valid hosts for ease of attack. Finally, we conducted an online survey to understand whether developers prioritize secret removal based on security risk score. We found that 86\% of the developers prioritized the secrets for removal with descending security risk scores.

\newpage
\bibliographystyle{ACM-Reference-Format}
\bibliography{references}

\end{document}
