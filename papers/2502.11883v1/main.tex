\documentclass[sigconf]{acmart}

%%%%%%%%%%%%%%%%%%%%%%%%%%%%%%%%%%%%%%%%%%%%%%%%%%%%%%%%%%%%%%%%%%%%%%%%%%%%%%

%% Beautiful mathematics
\usepackage{amsmath, amssymb, amsfonts} 
\usepackage{nicefrac}
\usepackage{mathtools}
\usepackage{bm, bbm}
\usepackage[scr=boondoxo,scrscaled=1.05]{mathalfa}

%% References in the correct format 
%\usepackage[square,numbers]{natbib}
%\def\bibfont{\footnotesize} % fix to have the same font size as without natbib

\usepackage[sort, compress, space]{cite}            


%% Enumerate nicely 
\usepackage{enumitem}

%% Different color comments and commenting large parts of the text
\usepackage{xcolor}
\usepackage{comment}
\usepackage{soul}

%% Hyper references
\usepackage{hyperref}
\usepackage{cleveref}
%\usepackage[numbers]{natbib}

\usepackage{tikz}
%\usepackage{thm-restate}
%% Appendix package
%\usepackage{appendix}

%% Random text to test spacing 
\usepackage{blindtext}

\usepackage{afterpage}

\usepackage{algorithm, algorithmic}    



\usepackage{dsfont}

\usepackage{tikz}
\usepackage{graphicx}
\usepackage{tikzscale}
\usepackage{pgfplots}
\pgfplotsset{compat=newest}
\usepackage{xfrac}

\usepackage{thm-restate}

%\usepackage{subcaption}

\usepackage{balance}

\usepackage{cite}
\usepackage{amsmath,amssymb,amsfonts}
\usepackage{balance}
\usepackage{algorithmic}
\usepackage{graphicx}
\usepackage{textcomp}
\usepackage{xcolor}
\usepackage{amsmath}
\usepackage{amssymb}
\usepackage[mathscr]{euscript}
\usepackage{comment}
\usepackage{xcolor}
\usepackage{enumitem} 
\usepackage{amsthm}


\section{Problem Studied}\label{sec:def}
We first present Fixed-Radius Near Neighbor (FRNN) queries and then formalize Aggregation Queries over Nearest Neighbors (AQNNs) that build on them. We then state our problem.

\subsection{Nearest Neighbor Queries}\label{subsec:FRNN}
We build on generalized Fixed-Radius Near Neighbor (FRNN) queries \cite{FRNNSurvey}. Given a dataset \( D \), a query object \( q \), a radius \( r \), and a distance function \( dist \), a generalized FRNN query retrieves all nearest neighbors of \( q \) within radius \( r \). More formally:
\[
NN_D(q, r) = \{x \in D \mid dist(x, q) \leq r\},
\]
where \(x\) is any data point in \(D\) and \(dist(x, q)\) denotes the distance between them. We use \(|NN_D(q,r)|\) to denote the neighborhood size of \(q\). As shown in Fig. \ref{fig:framework}, given a radius \(r\) and a target patient \(q\), patients in the dotted circle are nearest neighbors, and the neighborhood size is 6.

\subsection{Aggregation Queries over Nearest Neighbors}\label{subsec:AQNN} 
Given an FRNN query object \(q\) in dataset \(D\), a radius \(r\), and an attribute \(\texttt{attr}\), an Aggregation Query over Nearest Neighbors (AQNN) is defined as:
\[ \text{agg}(NN_D(q,r)[\texttt{attr}]) \]
where agg is an aggregation function, such as $\mathtt{AVG}$, $\mathtt{SUM}$, and $\mathtt{PCT}$, and \(NN_D(q,r)[\texttt{attr}]\) denotes the bag of values of attribute \texttt{attr} of all FRNN results of \(q\) within radius \(r\). 
% \end{definition}

An AQNN expresses aggregation operations to capture key insights about the neighborhood of a query object. For example, \(\mathtt{AVG}\) can be used to reflect the average heart rate or systolic blood pressure of patients in the neighborhood, providing a measure of typical health conditions. \(\mathtt{SUM}\) is useful for assessing cumulative effects, such as the total cost of treatments in the neighborhood that instructs public policy in terms of health. Similarly, $\mathtt{PCT}$ can be used to find the proportion of patients in the neighborhood of a patient of interest, relative to the population in the dataset.
%\laks{Why is finding the total \#meds to NNs or the total treatment cost of everyone in the NN interesting?}

% \texttt{MIN} and \texttt{MAX} are not included in the aggregation functions because they only capture extreme values, which may not represent the typical characteristics of the nearest neighbors and are more sensitive to outliers. 
% \laks{AVG is also sensitive to outliers, but we still allow it. isn't the real reason we don't consider MIN/MAX because they are amenable to estimation via sampling?} We choose \texttt{PCT} instead of \texttt{COUNT} in order to provide a normalized measure that remains comparable across different neighborhood sizes. It allows for more consistent interpretation of relative popularity \cite{moore1989introduction}.


Fig. \ref{fig:framework} illustrates an example of an AQNN: ``\textit{Find the average systolic blood pressure of patients similar to an insomnia patient \(q\)}''. The aggregation function is \(\mathtt{AVG}\) and the target attribute of interest is systolic blood pressure. Exact query evaluation requires consulting physicians (or predicting embeddings by an expensive machine learning model) for all 500 patients in \(D\) and calculate \(q\)'s nearest neighbors wrt \(r\) \cite{DBLP:journals/isci/RodriguesGSBA21}. We refer to such highly accurate but computationally expensive models as \textit{oracle models}, denoted as \(O\), including deep learning models trained on domain-specific data or human expert annotations \cite{DBLP:conf/sigmod/LuCKC18}. Using oracle models is very expensive \cite{sze2017efficient, DujianPQA, DBLP:journals/pvldb/KangGBHZ20}. To address that, we seek an approximate solution by \textit{proxy models}, denoted as \(P\), that are at least one order of magnitude cheaper than oracle models. In the example, if consulting physicians for one patient incurs one cost unit, calling a cheap machine learning model instead incurs at most \(0.1\) cost unit. Once the similar patients are identified, their systolic blood pressure values are averaged and returned as  output. The use of a proxy model may reduce the accuracy of the neighborhood prediction and hence, we should judiciously call oracle and proxy models to minimize the error of aggregate results.

Note that the values of the target attribute \texttt{attr} are \textit{not} predicted but are instead known quantities.

\subsection{Problem Statement}
Given an AQNN, our goal is to return an approximate aggregate result by leveraging both oracle and proxy models while reducing error and cost.



%%
%% The "author" command and its associated commands are used to define
%% the authors and their affiliations.
%% Of note is the shared affiliation of the first two authors, and the
%% "authornote" and "authornotemark" commands
%% used to denote shared contribution to the research.

\author{Zihan Wang}
\orcid{0000-0003-0493-2668}
\authornote{Both authors contributed equally to the paper.}
\affiliation{%
  \institution{University of Amsterdam}
  \city{Amsterdam}
  \country{The Netherlands}
}
\email{zhw.cypher@gmail.com}

\author{Ziqi Zhao}
\orcid{0009-0008-3011-5745}
\authornotemark[1]
\affiliation{
    \institution{Shandong University}
    \city{Qingdao}
    \country{China}
}
\email{ziqizhao.work@gmail.com}

\author{Yougang Lyu}
\orcid{0009-0000-1082-9267}
\affiliation{
    \institution{University of Amsterdam}
    \city{Amsterdam}
    \country{The Netherlands}
}
\email{youganglyu@gmail.com}

\author{Zhumin Chen}
\orcid{0000-0003-4592-4074}
\affiliation{
    \institution{Shandong University}
    \city{Jinan}
    \country{China}
}
\email{chenzhumin@sdu.edu.cn}

\author{Maarten de Rijke}
\orcid{0000-0002-1086-0202}
\affiliation{%
  \institution{University of Amsterdam}
  \city{Amsterdam}
  \country{The Netherlands}
}
\email{m.derijke@uva.nl}

\author{Zhaochun Ren}
\orcid{0000-0002-9076-6565}
%\authornote{Corresponding author.}
\affiliation{%
  \institution{Leiden University}
  \city{Leiden}
  \country{The Netherlands}
}
\email{z.ren@liacs.leidenuniv.nl}
\setcopyright{rightsretained}
\copyrightyear{2025}
\acmYear{2025}
\acmDOI{XXXXXXX.XXXXXXX}

%% These commands are for a PROCEEDINGS abstract or paper.
\acmConference[Conference acronym 'XX]{Make sure to enter the correct
  conference title from your rights confirmation emai}{June 03--05,
  2018}{Woodstock, NY}
%%
%%  Uncomment \acmBooktitle if the title of the proceedings is different
%%  from ``Proceedings of ...''!
%%
%%\acmBooktitle{Woodstock '18: ACM Symposium on Neural Gaze Detection,
%%  June 03--05, 2018, Woodstock, NY}
\acmISBN{978-1-4503-XXXX-X/18/06}



\title[FairDiverse: A Comprehensive Toolkit for Fair and Diverse Information Retrieval Algorithms]{FairDiverse: A Comprehensive Toolkit for Fair and Diverse \\Information Retrieval Algorithms}

\begin{document}

\begin{abstract}
In modern information retrieval (IR), achieving more than just accuracy is essential to sustaining a healthy ecosystem, especially when addressing fairness and diversity considerations. To meet these needs, various datasets, algorithms, and evaluation frameworks have been introduced. However, these algorithms are often tested across diverse metrics, datasets, and experimental setups, leading to inconsistencies and difficulties in direct comparisons.
This highlights the need for a comprehensive IR toolkit that enables standardized evaluation of fairness- and diversity-aware algorithms across different IR tasks. To address this challenge, we present \textbf{FairDiverse}, an open-source and standardized toolkit.
FairDiverse offers a framework for integrating fairness- and diversity-focused methods, including pre-processing, in-processing, and post-processing techniques, at different stages of the IR pipeline. The toolkit supports the evaluation of \textbf{28} fairness and diversity algorithms across \textbf{16} base models, covering two core IR tasks—search and recommendation—thereby establishing a comprehensive benchmark.
Moreover, FairDiverse is highly extensible, providing multiple APIs that empower IR researchers to swiftly develop and evaluate their own fairness- and diversity-aware models, while ensuring fair comparisons with existing baselines. The project is open-sourced and available on GitHub:~\url{https://github.com/XuChen0427/FairDiverse}.


\end{abstract}

\maketitle

\section{Introduction}

% In modern information retrieval (IR) systems, ensuring fair treatment and support for both head and tail products/users is crucial for maintaining a healthy ecosystem. Building upon this idea, previous research has emphasized the importance of developing fair and diverse information retrieval systems. ensuring fair treatment and support for both head and tail products/users is crucial for maintaining a healthy ecosystem. Building upon this idea, previous research has emphasized the importance of developing fair and diverse IR systems

Information retrieval (IR) tasks, such as search and recommendation, typically aim to select the information that meets user needs~\cite{IRbook, chowdhury2010introduction}. 
In modern IR, factors beyond the accuracy of information access, such as novelty, diversity, and fairness, are crucial for building a healthy ecosystem~\cite{Li_sigir24}. 
Among these factors, fairness and diversity have gained increasing attention in recent years~\cite{li2022fairness, santos2010exploiting, PM2_12_sigir}. Both aim to expose users to a broader range of information sources~\cite{santos2010exploiting, dang2012diversity} while also supporting diverse types of providers~\cite{fairrec, xu2023p}.


\begin{figure*}[t]  
    \centering    
    \includegraphics[width=\linewidth]{img/pipeline.pdf}
    \caption{Overall architecture of FairDiverse. We categorize fairness- and diversity-aware algorithms into pre-processing, in-processing, and post-processing stages, corresponding to data processing, model training, and result evaluation phases of IR.}
    \label{fig:pipline}
    \vspace*{-2mm}
\end{figure*}
%Aequitas


\begin{table}[t]
\centering
\setlength{\tabcolsep}{1.1pt}
\caption{Comparison between existing fairness- and diversity-aware toolkits. \ding{55} denotes that the feature is not supported, while \ding{51} indicates that the feature is supported. }
\label{tab:compare}
\begin{tabular}{l cccccc }
\toprule
Features & 
\rotatebox{65}{Recbole \cite{recbole2.0}} & 
\rotatebox{65}{FFB \cite{han2023ffb}} & 
\rotatebox{65}{Fairlearn \cite{bird2020fairlearn}} & 
\rotatebox{65}{AIF360 \cite{aif360-oct-2018}} & 
\rotatebox{65}{Aequitas \cite{jesus2024aequitas}} & 
\rotatebox{65}{\textbf{FairDiverse}} \\ 
\midrule
Recommendation & \ding{51} & \ding{55} & \ding{55} & \ding{55} & \ding{55} & \ding{51} \\
Search & \ding{55} & \ding{55} & \ding{55} & \ding{55} & \ding{55} & \ding{51} \\
\midrule
Pre-processing & \ding{55} & \ding{55} & \ding{51} & \ding{51} & \ding{51} & \ding{51} \\
In-processing & \ding{51} & \ding{51} & \ding{51} & \ding{51} & \ding{51} & \ding{51} \\
Post-processing & \ding{55} & \ding{55} & \ding{51} & \ding{51} & \ding{51} & \ding{51} \\
\midrule
Number of models & 4 & 6 & 6 & 15 & 10 & \textbf{28} \\ \bottomrule
\end{tabular}
\vspace*{-3mm}
\end{table}

To ensure fairness and diversity in IR systems, many fairness-aware~\cite{fairrec, xu2023p, TaxRank, SDRO, APR, FairNeg, cpfair, rus2024study} and diversity-aware algorithms~\cite{li2022fairness, santos2010exploiting, qin2020diversifying, yan2021diversification} have been designed as plugins or modules that can be integrated into various stages of the IR pipeline. However, fairness and diversity often suffer from a lack of unified definitions~\cite{li2022fairness, LLM4FairSurvey}. 
As a result, the evaluation of these algorithms in IR systems are based on different metrics, datasets, and evaluation settings (details are shown in Section~\ref{sec:related_work}). Hence, the performance of these algorithms cannot be compared consistently. Developing a unified, fair, and extensible toolkit for fairness and diversity is critically important and urgently needed to evaluate these algorithms consistently across IR tasks. Such a toolkit framework holds significant value for fostering a trustworthy IR community.

To create a unified and equitable evaluation, we introduce FairDiverse, an open-source standardized toolkit designed to assess fairness and diversity in IR systems.
First, FairDiverse offers detailed guidance on incorporating fairness- and diversity-aware algorithms throughout various stages of the IR process. These algorithms are categorized into pre-processing, in-processing, and post-processing methods, corresponding to data processing, model training, and result evaluation stages in different IR pipeline steps, respectively. 
Then, FairDiverse implements a wide range of fairness- and diversity-aware models (\textbf{28} models) tailored to \textbf{16} base models under two fundamental IR tasks: search and recommendation. 
It offers corresponding implementation code and systematically evaluates these algorithms using over ten accuracy, fairness, and diversity metrics, enabling the construction of a benchmark within FairDiverse.

%FairDiverse evaluates two fundamental tasks in IR: search and recommendation. Recognizing that unfairness and lack of diversity often arise during the data collection, model training, and evaluation stages of IR systems, FairDiverse categorizes \textbf{30+} fairness- and diversity-aware algorithms into three types: pre-processing, in-processing, and post-processing methods. It provides corresponding implementation code and systematically evaluates these algorithms using more than ten IR accuracy, fairness, and diversity metrics across ten datasets, offering a comprehensive evaluation framework and workflow.

In the literature, only a few open-source toolkits and libraries have been developed for fairness- and diversity-aware IR algorithms. Table~\ref{tab:compare} provides a comparison between these existing resources and the proposed FairDiverse, highlighting features such as supported IR tasks (recommendation and search), algorithm types (pre-processing, in-processing, and post-processing), and the number of implemented models. Other toolkit details are provided in Section~\ref{sec:related_work}. As shown in Table~\ref{tab:compare}, FairDiverse provides the largest number of models, offering extensive coverage of all types of fairness- and diversity-aware algorithms. Additionally, it supports major information retrieval (IR) tasks, including search and recommendation, making it a versatile and comprehensive toolkit.

FairDiverse is designed to be highly extensible, providing a range of flexible APIs that allow IR researchers to efficiently develop and integrate their own fairness- and diversity-aware IR models. This extensibility ensures that researchers can tailor the toolkit to their specific needs while maintaining consistency with established evaluation protocols. This makes it an invaluable resource for advancing fairness and diversity in IR systems.

%\section{Adaptive labeling as a Markov decision process} 
\label{sec:formulation}

We illustrate our formulation for model evaluation, and extend it to the ATE estimation setting at the end of the section. 
Our goal is to evaluate the performance of a prediction model $\model: \statdomain \to \mathcal \labeldomain$ over the input distribution $P_X$ that we expect to see during deployment.  Given inputs $X  \in \mc{X}$,   labels/outcomes are generated
 from some unknown function $f\opt$: $
      Y = f\opt(X) + \varepsilon$, where $\varepsilon$ is the noise.
      %~~~\mbox{where}~~\varepsilon \sim N(0, \sigma^2)
  % \begin{equation*}
  %     Y = f\opt(X) + \varepsilon~~~\mbox{where}~~\varepsilon \sim N(0, \sigma^2).
  % \end{equation*}
When ground truth outcomes are costly to obtain, previously collected labeled data $\mc{D}^0 := \{(X_i,Y_i)\}_{i \in \mc{I}}$ 
typically suffers selection bias and covers only a subset of the support of input distribution $P_X$ over which we aim to evaluate the model performance. 

Assuming we have a   pool of data $\xpool$, we design
 adaptive sampling algorithms that iteratively select
inputs in $\xpool$ to be labeled.
Since labeling inputs takes time in practice, we model
real-world instances by considering \emph{batched} settings. Our goal is to sequentially label batches of data to accurately estimate model performance over $P_X$ and therefore we assume we have access to a set of inputs $\xeval \sim P_X$. %We assume the modeler pays a fixed and equal cost for each label/outcome. 
%Our framework is general as we do not assume \xpool∼PX\xpool \sim P_X.
We use the squared loss to illustrate our framework,
where our goal is to evaluate $\E_{X \sim P_X}[ (Y - \model(X))^2]$. Under the ``likelihood" function $p(y | f, x) = p_{\varepsilon}(y - f(x))$,  let $g(f)$ be the performance of the AI model $\model(\cdot)$ under the  data generating function $f$, which we refer to as our estimand of interest.
When we consider the mean squared loss,  $g(f)$ is given by 
\begin{align}
    g(f) \defeq \E_{X \sim P_X}\left[ \E_{Y \sim p(\cdot|f,X) } \Big[ (Y - \model(X))^2 \Big] \mid f \right]. \label{eqn:l2-g-f}
\end{align}
Our framework is general and can be extended to other settings. For example, a clinically useful  metric is \texttt{Recall}, defined as the fraction of individuals that the model $\model(\cdot)$ correctly labels as positive  among all the individuals who actually have the positive label 
\begin{align*}
    g(f) \defeq  \E_{X \sim P_X}\left[ \E_{Y \sim p(\cdot|f,X) } \Big[\indic{\model(X)>0}|Y=1\Big] \mid f\right].
\end{align*}
 
  Since the true function $f\opt$ is unknown, we  model it from a Bayesian perspective by formulating a posterior given the available supervised data. We refer to uncertainty over the data generating function $f$ as \emph{epistemic} uncertainty---since we can resolve it with more data---and that over
 the measurement noise $\varepsilon$ as \emph{aleatoric} uncertainty. 
Assuming independence given features $X$, we model the  likelihood of the data via the product 
$p({Y}_{1:m}|f, {X}_{1:m}) = \prod_{i=1}^m p(Y_i|f,X_i)$.
 Our prior belief  $\mu$ over functions $f$   reflects our uncertainty about how
labels are generated given features. 
To adaptively label inputs from $\mc{X}_{\rm pool}$, we assume access to an uncertainty quantification (UQ) method that provides posterior beliefs $\mu(f \mid \mc{D})$ given
any supervised data $\mc{D}:= \{(X_i,Y_i)\}_{i \in \mc{I}}$. As we detail  in Section~\ref{sec:uq}, our framework can leverage both classical 
Bayesian models like Gaussian processes and recent advancements in deep learning-based UQ  methods.

As new batches are labeled, we update our posterior beliefs about $f$ over time, which we view as ``state transitions'' of a dynamical system.
Recalling the Markov decision process depicted in Figure~\ref{fig:overview}, we sequentially label a batch of inputs from $\mc{X}_{\rm pool}$ (actions), which lead to state transitions (posterior updates).
Specifically, our initial state is given by $\mu_0(\cdot) = \mu(\cdot \mid \mc{D}^0)$, where $\mc{D}^0$ represents the initial labeled dataset.
At each period $t$, we label a batch of $K_t$ inputs $\mc{X}^{t+1} \subset \mc{X}_{\rm pool}$ resulting in labeled data $ \mc{D}^{t+1} = (\mc{X}^{t+1}, \datay^{t+1})$. After acquiring the labels at each step $t$, we update the posterior state to $\mu_{t+1}(\cdot) = \mu_t(\cdot \mid \mc{D}^{t+1})$. Modeling practical instances, we consider a small horizon problem with limited adaptivity $T$. Formulating an MDP over posterior states has long conceptual roots, dating back to the Gittin's index for multi-armed bandits~\citep{Gittins79}.

 We denote by $\pi_t$ the adaptive labeling policy at period $t$. We account for randomized policies $\datax^{t+1} \sim \pi_t(\mu_t)$ with a flexible batch size $|\datax^{t+1}| = \batchsize_t$.   
We assume $\pi_t$ is $\mc{F}_t-$measurable for all $t < T$, where $\mc{F}_t$ is the filtration generated by the observations up to the end of step $t$.
 Observe that $\mu_{t+1}$ contains randomness in the policy $\pi_t$ as well as randomness in $\datay^{t+1} \mid (\datax^{t+1},\mu_t)$. Letting $\pi = \set{\pi_0,....,\pi_{T-1}}$,  we minimize the uncertainty over $g(f)$
 at the end of data collection
\begin{align}
H(\pi) \defeq \E_{\mc{D}^{1:T} \sim \pi} \left[G(\mu_{T}) \right]  \defeq  \E_{\mc{D}^{1:T} \sim \pi} \left[G(\mu(\cdot \mid \mc{D}^{0:T})) \right]
%\defeq \E_{\mc{D}^{1:T} \sim \pi} \left[ \V_{f \sim \mu_{T}}  g(f)  \right]
     = \E_{\mc{D}^{1:T} \sim \pi} \left[ \V_{f \sim \mu(\cdot \mid \mc{D}^{0:T})}  g(f)  \right],
     \label{eqn:general-obj}
\end{align}   
where $G(\mu_T) = \V_{f \sim \mu_T}  g(f)$.
In the above objective~\eqref{eqn:general-obj}, we assume
that the modeler pays a fixed and equal cost for each outcome. 
Our framework can also seamlessly accommodate variable labeling cost. Specifically, we can define a cost function $c(\cdot)$ 
 applied on the selected subsets 
 and update the objective~\eqref{eqn:general-obj} accordingly to include the term  $\lambda c(\mc{D}^{1:T})$,
 where $\lambda$
 is the penalty factor that controls the trade-off between minimizing variance and cost of acquiring samples.


Our framework can be easily extended to causal estimation problems.  Consider a feature vector ${X}$ and suppose we have two treatment arms $Z \in \{0,1\}$. Our objective is to evaluate the average treatment effect over the population distribution $P_X$.  Given feature vector $X$, and treatment $Z$,  outcomes are generated from an unknown function $f\opt$: 
$Y = f\opt(X,Z) + \varepsilon.$
%~~~\mbox{where}~~\varepsilon \sim N(0, \sigma^2)
  % \begin{equation*}
  %     Y = f\opt(X) + \varepsilon~~~\mbox{where}~~\varepsilon \sim N(0, \sigma^2).
  % \end{equation*}
The available data is denoted by $\mc{D}^0 := \{X_i,Y_i,Z_i\}_{i \in \mc{I}}$ and given a pool of candidates 
$\xpool$, we want an
 adaptive sampling algorithms that iteratively select
candidates in $\xpool$ to be assigned a random treatment so that we can estimate average treatment effect efficiently. Under the ``likelihood" function $p(y | f, x, z) = p_{\varepsilon}(y - f(x,z))$,  let $g(f)$ represent  the average treatment effect, which is our estimand of interest. Formally, this is expressed as:
\begin{align}
g(f) \defeq \E_{X \sim P_X} \left[\E_{Y_1 \sim p(\cdot|f,X,Z=1) , Y_0 \sim p(\cdot|f,X,Z=0)} \left[Y_1 -  Y_0 \right]\mid f \right]. \label{eqn:ate-g-f}
\end{align}




 %Again the true function $f\opt$ is unknown and we model it from a Bayesian perspective by formulating a posterior  given available data.
 Our prior belief  $\mu$ over functions $f$, now 
 reflects our uncertainty about how
outcomes are generated given features and treatments. 
  We sequentially observe outcome of a batch of inputs from $\mc{X}_{\rm pool}$ (actions), and treatments assigned to this batch. We assume that selected batch of inputs $\mc{X}^t$ is randomly assigned treatments $\mc{Z}^t$ with each $Z\sim p_Z$. We summarize our formulation in Figure~\ref{fig:MDP_framework_flowchart}.
%Specifically, our initial state is given by $\mu_0(\cdot) = \mu(\cdot \mid \mc{D}^0)$ and at each period $t$, we get outcomes for a batch of $K$ candidates $\mc{X}^{t+1} \subset \mc{X}_{\rm pool}$, with randomly assigned treatments $\mc{Z}^{t+1}$ (with each $Z\sim p_Z$) and get the data $ \mc{D}^{t+1} = (\mc{X}^{t+1} \times \datay^{t+1} \times \mc{Z}^{t+1} )$. After acquiring the data at each step $t$ we update posterior state to $\mu_{t+1}(\cdot) = \mu_t(\cdot \mid \mc{D}^{t+1})$. Modeling practical instances, we consider a small horizon problem with limited adaptivity $T$.  We denote by $\pi_t$ the adaptive labeling policy at period $t$. We account for randomized policies $\datax^{t+1} \sim \pi_t(\mu_t)$ with a flexible batch size $|\datax^{t+1}| = \batchsize_t$.   
Again, we assume $\mu_t$ is $\mc{F}_t-$measurable for all $t < T$, where $\mc{F}_t$ is the filtration generated by the observations up to the end of step $t$.
 Observe that $\mu_{t+1}$ contains randomness in the policy $\pi_t$, randomness in treatment assignment $\mc{Z}^{t+1}$ and randomness in $\datay^{t+1} \mid (\datax^{t+1}, \mc{Z}^{t+1},\mu_t)$. Letting $\pi = \set{\pi_0,....,\pi_{T-1}}$,  we minimize the uncertainty over $g(f)$
 at the end of data collection:
\begin{align}
\E_{\mc{D}^{1:T} \sim \pi} \left[G(\mu_{T}) \right] \defeq
\E_{\mc{D}^{1:T} \sim \pi} \left[ \V_{f \sim \mu_{T}}  g(f)  \right]
= \E_{\mc{D}^{1:T} \sim \pi} \left[ \V_{f \sim \mu(\cdot \mid \mc{D}^{0:T})}  g(f)  \right].
\label{eqn:general-ate-obj}
\end{align}  


 

\begin{figure}[ht]
\centering
\begin{tikzpicture}
[
roundnode/.style={circle, draw=black!60, very thick, minimum size=10mm},
squarednode/.style={rectangle, draw=black!60, very thick, minimum size=10mm, align =center,text width = 26mm},
]
%Nodes
\node[roundnode]      (maintopic)                              {$\mu$};
\node[roundnode]        (circle1)       [right=20mm of maintopic] {$\mu_0$};
\node[roundnode]      (circle2)       [right=5mm of circle1] {$\mu_t$};
\node[roundnode]        (circle3)       [right=20mm of circle2] {$\mu_{t+1}$};
\node[roundnode]        (circle4)       [right=5mm of circle3] {$\mu_T$};
\node[squarednode]        (circle5)       [right=5mm of circle4] {Reward/Cost $\E \left[ \V_{\mu_T} (g(f))\right]$};


%Lines
\draw[thick, ->, >=stealth] (maintopic.east) -- node[anchor=south] {$(\mathcal{X}^0,\mathcal{Y}^0,\mathcal{Z}^0)$} (circle1.west);
\draw[thick, ->, dashed] (circle1.east) --  (circle2.west);
\draw[thick, ->] (circle2.east)  -- node[above, align =center, text width = 18mm] { Query $(\mathcal{X}^t,\mathcal{Y}^t,\mathcal{Z}^t)$} (circle3.west);
\draw[thick, ->,dashed] (circle3.east) --  (circle4.west);
\draw[thick, ->] (circle4.east) --  (circle5.west);


\end{tikzpicture}
\caption{MDP framework for adaptive labeling to efficiently estimate the average treatment effect (ATE).}
\label{fig:MDP_framework_flowchart}
\end{figure}
 









\begin{comment}
\subsection{Broader applicability of the framework to other problem settings} \label{sec:broad-framework-accuracy}
 

Although  we describe our setting in a healthcare setting with the objective  to estimate the recall of a trained AI model $\model(\cdot)$, the framework caters to many other problem settings. The extension to the evaluation of model based on accuracy (in regression setting) is straightforward, we simply replace the definition of recall $g(f)$ in~\eqref{eqn:l2-g-f} with
\begin{align*}
    g(f) = \E_{\substack{ y \sim p(y|f,x) \\  \forall x \in \mathcal X}} \big( \E_{{\textbf x} \sim p_x} [y-\model(x)]^2 \big).
\end{align*}


\textcolor{red}{To discuss if we need to have it here}
We can also extend this setting to the efficient estimation of the ATE as well. We describe these in detail below:

\begin{itemize}
    
    \item Estimating accuracy:  \[g(f) = \E_{\substack{ y \sim p(y|f,x) \\  \forall x \in \mathcal X}} \big( \E_{{\textbf x} \sim p_x} [y-\model(x)]^2 \big)\]
%    \item Estimating ATE with known control arm: 
%\[g(f) = \E_{\substack{ y \sim p(y|f,x) \\  \forall x \in \mathcal X}} \big( \E_{{\textbf x} \sim p_x} [y-\model(x)] \big)\]
\item Estimating ATE  (with minor modifications - broad structure remains similar) : 


Consider feature vector ${\mathbf x} \in \mathcal X $  distributed as ${\mathbf x}  \sim p_{\mathbf x}$, treatment $z \in {\mathcal Z} = \{0,1\}$, and a class of random functions $f: {\mathcal X} \times {\mathcal Z} \to {\mathcal Y}$, which determines the likelihood $p(y_i|f,{\mathbf x_i},z_i)$. Note that $f$ is random and reflects our uncertainty about how
labels are generated given features and the treatment. Additionally, the joint likelihood is determined as follows,  

\[p(Y|f,X,Z) = \prod_{i} p(y_i|f,{\mathbf x_i}, z_i) \]

Assuming the prior over functions $f$ to be $\mu$, therefore we have 
\[p(Y|X,Z) = \int \prod_{i} p(y_i|f,{\mathbf x_i},z_i) d\mu(f) \]


Also, assuming that under the  true data generating function $f$ (if known precisely - which we don't), the estimand of interest is

\[ \E_{{\textbf x} \sim p_x}  \left( \E_{\substack{ y \sim p(y|x,f,z=1) }} y - \E_{\substack{ y \sim p(y|x,f,z=0) }} y \right) \]


Throughout the paper we assume the above data generating process.  Now, suppose we have some labeled  data $(\datax^0,\datay^0,Z^0) =({\mathbf x}_{1:m}^0,y_{1:m}^0, z_{1:m}^0)$. 
    We run a experiment, in which we want to query the labels (in batches), so as to minimize the uncertainty of the estimand of interest. Suppose, the horizon of the experiment is $T$. Now, given prior $\mu$ and labeled data $\datax^0,\datay^0,Z^0$, in the beginning of our experiment the posterior state is $\mu_0$.

 At each step $j$ ($j \geq 1$), we query labels for a batch (with size $k$) of unlabeled data $(\datax^j,Z^j) \subset \mathcal X \times \mathcal Z$  and get labels $\datay^j$. After acquiring the labels at each step $j$ we update posterior state to $\mu_{j+1}$, informed by $\mu_j$ and $(\datax^j,\datay^j,Z^j)$. 
 
 Let the policy at step $j$ be $\pi_j$ (potentially random), which gives $\datax^{j+1},Z^{j+1} \sim \pi_j(\mu_j)$.  Observe that $\mu_{j+1}$ is random because of the randomness of the policy $\pi_j$ and $\datay^{j+1}|\{\datax^{j+1},Z^{j+1},\mu_j\}$ (\textcolor{red}{can this be written in a better way?}). Let, $\pi = \{\pi_0,....,\pi_{T-1}\}$. Therefore, our objective is to

 
\[ \min_{\pi} \E \left[ {\mathbf {Var}}_{f \sim \mu_T} \left( \E_{{\textbf x} \sim p_x}  \left( \E_{\substack{ y \sim p(y|x,f,z=1) }} y - \E_{\substack{ y \sim p(y|x,f,z=0) }} y \right) \right) \right]\]

where, $\mu_T$ depends on $\{(\datax^i,\datay^i,Z^i)\}_{i=0}^T$ and outer expectation is over both $\pi$ and  $\datay^{j+1}|\{\datax^{j+1}, Z^{j+1},\mu_j\}$ for all $j \in [0,T-1]$.


%Constraining the action space is straightforward - by first choosing set of x's using k-subset and then assigning treatment with learnable probability parameters $w_1,...,w_n$.

\end{itemize}

 \[ g(f) = \E_{\substack{ y \sim p(y|f,x) \\  \forall x \in \mathcal X}}\E_{{\textbf x} \sim p_x} g(y,{\textbf x}) \approx \E_{\substack{ y \sim p(y|f,x) \\  \forall x \in  \datax^u}} \left( \frac{1}{n}\sum_{i=1}^n \tilde{g}(y,{\textbf x}_i^u) \right)\]




Notation borrowed from a combination of the following papers 
~\citep{LeeYuNaFoLe23, KatoOgKoIn24, FongHoWa24}

%  
\end{comment}


%%% Local Variables:
%%% mode: latex
%%% TeX-master: "main"
%%% End:

% \documentclass[11pt]{article}
% \usepackage{tikz}
% \usetikzlibrary{positioning, shapes.multipart, fit, calc, shapes.geometric, shapes.misc}

% \begin{document}

\begin{figure*}[htbp]

\begingroup
\usetikzlibrary{shapes.geometric}
\usetikzlibrary{arrows.meta}
\usetikzlibrary{backgrounds}
\definecolor{tiffanyblue}{RGB}{129,216,208}
\definecolor{bangdiblue}{RGB}{0,149,182}
\definecolor{kleinblue}{RGB}{0,47,167}
\definecolor{kabuliblue}{RGB}{26,85,153}
\definecolor{purple}{RGB}{138,43,226}

    \centering

      \tikzset{global scale/.style={
    scale=#1,
    every node/.append style={scale=#1}
  }
}


\begin{tikzpicture}[global scale=0.64]
    \newlength{\moduleintervaly}
    \setlength{\moduleintervaly}{1.8em}    
    \newlength{\moduleintervalx}
    \setlength{\moduleintervalx}{-7em}
    \newlength{\blockintervalx}
    \setlength{\blockintervalx}{30em}
    
    \tikzstyle{circlenode}=[draw, circle,minimum size=4pt,inner sep=0, fill=red!30];
    
    \tikzstyle{moduleode}=[draw,minimum height=2.5em,minimum width=23em,inner sep=.0em,thick,rounded corners=.2em, font=\small, scale=0.8];
    
    \tikzstyle{layernode}=[draw,minimum height=1.5em,minimum width=5em,inner sep=.0em,thick,rounded corners=.2em, font=\small,fill=yellow!20];
    
    \tikzstyle{attentionnode}=[draw,minimum height=1.5em,minimum width=5em,inner sep=.0em,thick,rounded corners=.2em, font=\small];

    \tikzstyle{querynode}=[draw,minimum height=1.5em,minimum width=2em,inner sep=.0em,thick,rounded corners=.2em, font=\small];

    \tikzstyle{attnmapnode}=[fill=yellow!20,draw,minimum height=2.5em,minimum width=10em,inner sep=.0em,thick,rounded corners=.2em, font=\small];

    \tikzstyle{partialattnmapnode}=[fill=tiffanyblue!40,draw,minimum height=2.5em,minimum width=12em,inner sep=.0em,thick,rounded corners=.2em, font=\small];

    \tikzstyle{partialattnmapnode1}=[fill=red!20,draw,minimum height=2.5em,minimum width=12em,inner sep=.0em,thick,rounded corners=.2em, font=\small];

    \tikzstyle{GRLattnmapnode}=[fill=yellow!20,draw,minimum height=2.5em,minimum width=8em,inner sep=.0em,thick,rounded corners=.2em, font=\small];

    \tikzstyle{StandardFFNnode}=[fill=red!20,draw,minimum height=2.5em,minimum width=23em,inner sep=.0em,thick,rounded corners=.2em, font=\small];

    \tikzstyle{Concatenode}=[draw,minimum height=1.5em,minimum width=5em,inner sep=.0em,thick,rounded corners=.2em, font=\small];

    \tikzstyle{subspace_block}=[fill=yellow!20,draw,minimum height=2.5em,minimum width=8em,inner sep=.0em,thick,rounded corners=.2em, font=\small];

    \tikzstyle{partial_subspace_block}=[fill=red!20,draw,minimum height=2.5em,minimum width=11.0em,inner sep=.0em,thick,rounded corners=.2em, font=\small];

    \tikzstyle{global_partial_subspace_block}=[fill=pink!20,draw,minimum height=2.5em,minimum width=6.5em,inner sep=.0em,thick,rounded corners=.2em, font=\small];

    \tikzstyle{space_block}=[fill=blue!20,draw,minimum height=2.5em,minimum width=23em,inner sep=.0em,thick,rounded corners=.2em, font=\small];
    
    \tikzstyle{Encoder_block}=[draw,minimum height=8.8*\moduleintervaly,minimum width=20em,inner sep=.0em,thick,rounded corners=.2em, font=\small];

    \tikzstyle{GRL_block}=[draw,minimum height=7em,minimum width=8.8*\moduleintervaly,inner sep=.0em,thick,rounded corners=.2em, font=\small];

    \tikzstyle{Partial_Encoder_block}=[draw,minimum height=7.7*\moduleintervaly,minimum width=25em,inner sep=.0em,thick,rounded corners=.2em, font=\small];
    
    \tikzstyle{Graph_structure_learning}=[draw,minimum height=1.6em,minimum width=2.5em,inner sep=.0em,thick,rounded corners=.2em, font=\small,fill=orange!20];

    
    \tikzstyle{recnode}=[rectangle,rounded corners=5pt,draw,minimum height=1.8em,minimum width=3.5em,inner sep=0em,thick,rounded corners=0.2em,font=\small,fill=orange!20];

    \tikzstyle{recnodewhite}=[rectangle,rounded corners=5pt,minimum height=2.2em,minimum width=3.5em,inner sep=0em,thick,rounded corners=0.2em,font=\small];

    
    % added node
    % \tikzstyle{datanode}=[rectangle,rounded corners=5pt,draw,minimum height=1.8em,minimum width=3.5em,inner sep=0em,thick,rounded corners=0.2em,font=\small,fill=orange!20];
    \tikzstyle{datanode}=[cylinder,rounded corners=5pt,draw,minimum height=1.2em,minimum width=1.5em,inner sep=0em,thick,rounded corners=0.2em,font=\small,fill=orange!20];
    \tikzstyle{modelnode}=[rectangle,rounded corners=5pt,draw,minimum height=1.8em,minimum width=3.5em,inner sep=0em,thick,rounded corners=0.2em,font=\small,fill=green!20];
    \tikzstyle{tasknode}=[rectangle,rounded corners=5pt,draw,minimum height=1.8em,minimum width=3.5em,inner sep=0em,thick,rounded corners=0.2em,font=\small,fill=blue!20];
    \tikzstyle{mergenode}=[rectangle,rounded corners=5pt,draw,minimum height=1.8em,minimum width=15.5em,inner sep=0em,thick,rounded corners=0.2em,font=\small,fill=yellow!20];
    \tikzstyle{mergemodelnode}=[rectangle,rounded corners=5pt,draw,minimum height=1.8em,minimum width=15.5em,inner sep=0em,thick,rounded corners=0.2em,font=\small,fill=green!20];
    
    \tikzstyle{databanknode}=[rectangle,rounded corners=5pt,draw,minimum height=3.0em,minimum width=15.5em,inner sep=0em,thick,rounded corners=0.2em,font=\small,fill=orange!10];
    \tikzstyle{basemodel}=[rectangle,rounded corners=5pt,draw,minimum height=1.8em,minimum width=17.5em,inner sep=0em,thick,rounded corners=0.2em,font=\small,fill=green!18];
    \tikzstyle{loranode}=[rectangle,rounded corners=5pt,draw,minimum height=1.7em,minimum width=3.5em,inner sep=0em,thick,rounded corners=0.2em,font=\small,fill=lime!20];
    \tikzstyle{longloranode}=[rectangle,rounded corners=5pt,draw,minimum height=1.7em,minimum width=13.5em,inner sep=0em,thick,rounded corners=0.2em,font=\small,fill=lime!20];

    \tikzstyle{groupnode}=[cylinder,rounded corners=5pt,draw,minimum height=1.0em,minimum width=3.1em,inner sep=0em,thick,rounded corners=0.2em,font=\small,fill=orange!20];

     \tikzstyle{longdatabanknode}=[cylinder,rounded corners=5pt,draw,minimum height=10.0em,minimum width=1.5em,inner sep=0em,thick,rounded corners=0.2em,font=\small,fill=orange!13];
    \def\nodehsep{3em}
    \def\nodewsep{3.5em}


    % picture a
    \begin{scope}[xshift=0.0in,yshift=0.0in]
        \begin{pgfonlayer}{background}
            \node[anchor=south,minimum height=\nodehsep*5.7,minimum width=19.5em,fill=gray!4,rounded corners=5pt,dotted,draw](backgroundc) at (0, 0) {};
            % \node[anchor=south,minimum height=\nodehsep*6.5,minimum width=39.6em,fill=gray!4,rounded corners=5pt,dotted,draw](backgroundd) at (0,0) {};
        \end{pgfonlayer}

        \node[datanode,anchor=south](data1_01) at ([xshift=-2.2*\nodehsep, yshift=14.8em]backgroundc.south) {En-De};
        \node[datanode,anchor=south](data1_02) at ([yshift=-1.8\nodehsep]data1_01.south) {De-En};
        
        \node[datanode,anchor=west](data2_01) at ([xshift=\nodehsep]data1_01.east) {En-Zh};
        \node[datanode,anchor=west](data2_02) at ([xshift=\nodehsep]data1_02.east) {Zh-En};        
        
        \node[align=center,thick, scale=2] (omit1_01) at ([xshift=\nodehsep]data2_01.center) {...};
        \node[align=center,thick, scale=2] (omit1_02) at ([xshift=\nodehsep]data2_02.center) {...};
        
        \node[datanode,anchor=west](data3_01) at ([xshift=\nodehsep]data2_01.east) {En-Fr};
        \node[datanode,anchor=west](data3_02) at ([xshift=\nodehsep]data2_02.east) {Fr-En};

        \begin{pgfonlayer}{background}
            \node[anchor=south,minimum height=1.4*\nodehsep,minimum width=17.2em,fill=orange!8,rounded corners=5pt,draw](databank1) at ([yshift=-3.9em]data2_01.north) {};
        \end{pgfonlayer}
        

        \node[mergenode,anchor=north](dataselectionc) at ([xshift=0.1\nodehsep,yshift=-1.6\nodehsep]data2_02.south) {Data Selection};
        \draw[thick, ->] (databank1.south) -- ([xshift=-0.12em]dataselectionc.north);
        % \draw[thick, ->] (databank1.south) -- ([xshift=-0.12em]dataselectionc.north);
        % \draw[thick, ->] (databank1.south) -- ([xshift=-0.12em]dataselectionc.north);
        
        \node[basemodel,anchor=north](basemodelc) at ([yshift=-3.1\nodehsep]dataselectionc.north) {Base Model};
        \draw[->,thick] (dataselectionc.south) -- (basemodelc.north);
        
        \node[loranode,anchor=north](lora1c) at ([yshift=-7.8\nodehsep]data1_02.south) {$\mathrm{LoRA}_{1}$};
        \node[loranode,anchor=north](lora2c) at ([yshift=-7.8\nodehsep]data2_02.south) {$\mathrm{LoRA}_{2}$};
        \node[align=center,thick, scale=2] (omit2) at ([xshift=\nodehsep]lora2c.center) {...};
        \node[loranode,anchor=north](lora3c) at ([yshift=-7.8\nodehsep]data3_02.south) {$\mathrm{LoRA}_{N}$}; 
        
        \node(plus1c) at ([yshift=0.6\nodehsep]lora1c.north) {\textbf{\(+\)}};
        \node(plus2c) at ([yshift=0.6\nodehsep]lora2c.north) {\textbf{\(+\)}};
        \node(plus3c) at ([yshift=0.6\nodehsep]lora3c.north) {\textbf{\(+\)}};
        

        \node[tasknode,anchor=north](task1c) at ([yshift=-1.2\nodehsep]lora1c.south) {$\mathrm{Task}_{1}$};
        \node[tasknode,anchor=north](task2c) at ([yshift=-1.2\nodehsep]lora2c.south) {$\mathrm{Task}_{2}$};
        \node[align=center,thick, scale=2] (omit3c) at ([xshift=\nodehsep]task2c.center) {...};
        \node[tasknode,anchor=north](task3c) at ([yshift=-1.2\nodehsep]lora3c.south) {$\mathrm{Task}_{N}$}; 


        \draw[->,thick] (lora1c.south) -- (task1c.north);
        \draw[->,thick] (lora2c.south) -- (task2c.north);
        \draw[->,thick] (lora3c.south) -- (task3c.north);

         
        \node[anchor=south,font=\Large](l3) at ([xshift=0.3em,yshift=-3em]backgroundc.south) {(a) Seperate/Multilingual/Group Training};
        % \node[anchor=north,font=\Large](l3) at ([xshift=0.3em,yshift=-1em]backgroundc.south) {\parbox{6cm}{\centering (a) Separate/Multilingual/ \\ Group Training}};
        % \node[anchor=south,font=\Large](l4) at ([xshift=.5em,yshift=-3em]backgroundd.south) {(d) Model Merging};
    \end{scope}

    % picture b
     \begin{scope}[xshift=4.9in,yshift=0in]
        \begin{pgfonlayer}{background}
            \node[anchor=south,minimum height=\nodehsep*5.7,minimum width=39.6em,fill=gray!4,rounded corners=5pt,dotted,draw](backgroundd) at (0,0) {};
        \end{pgfonlayer}

        \node[datanode,anchor=south](data11d) at ([xshift=-2.2*\nodehsep, yshift=14.8em]backgroundd.south) {En-De};
        \node[datanode,anchor=south](data12d) at ([yshift=-1.8\nodehsep]data11d.south) {De-En};
        
        \node[datanode,anchor=west](data21d) at ([xshift=\nodehsep]data11d.east) {En-Zh};
        \node[datanode,anchor=west](data22d) at ([xshift=\nodehsep]data12d.east) {Zh-En};        
        
        \node[align=center,thick, scale=2] (omit11d) at ([xshift=\nodehsep]data21d.center) {...};
        \node[align=center,thick, scale=2] (omit12d) at ([xshift=\nodehsep]data22d.center) {...};
        
        \node[datanode,anchor=west](data31d) at ([xshift=\nodehsep]data21d.east) {En-Fr};
        \node[datanode,anchor=west](data32d) at ([xshift=\nodehsep]data22d.east) {Fr-En};

        \begin{pgfonlayer}{background}
            \node[anchor=south,minimum height=1.4*\nodehsep,minimum width=17.2em,fill=orange!8,rounded corners=5pt,draw](databank1d) at ([yshift=-3.9em]data21d.north) {};
        \end{pgfonlayer}

        \node[mergenode,anchor=north](dataselectiond) at ([xshift=-3.5*\nodehsep,yshift=-0.2*\nodehsep]databank1d.south) {Group Selection};
        
        \node[basemodel,anchor=north](basemodeldleft) at ([yshift=-0.7\nodehsep]dataselectiond.south) {Base Model};
        \draw[->,thick] (dataselectiond.south) -- (basemodeldleft.north);

        \node[loranode,anchor=north](lora1dleft) at ([xshift=-5.5\nodehsep, yshift=-1.6\nodehsep]basemodeldleft.south) {$\mathrm{LoRA}_{1}$};
        \node[loranode,anchor=north](lora2dleft) at ([yshift=-1.6\nodehsep]basemodeldleft.south) {$\mathrm{LoRA}_{2}$};
        \node[align=center,thick, scale=2] (omit2d) at ([xshift=0.9*\nodehsep]lora2dleft.center) {...};
        \node[loranode,anchor=north](lora3dleft) at ([xshift=5.5\nodehsep,yshift=-1.6\nodehsep]basemodeldleft.south) {$\mathrm{LoRA}_{N_G}$}; 

        \node(plus1dleft) at ([yshift=0.72\nodehsep]lora1dleft.north) {\textbf{\(+\)}};
        \node(plus2dleft) at ([yshift=0.72\nodehsep]lora2dleft.north) {\textbf{\(+\)}};
        \node(plus3dleft) at ([yshift=0.72\nodehsep]lora3dleft.north) {\textbf{\(+\)}};
        
         \node[tasknode,anchor=north](task1dleft) at ([yshift=-1.6\nodehsep]lora1dleft.south) {$\mathrm{Task}_{1}^{\mathrm{enxx}}$};
        \node[tasknode,anchor=north](task2dleft) at ([yshift=-1.6\nodehsep]lora2dleft.south) {$\mathrm{Task}_{2}^{\mathrm{enxx}}$};
        \node[align=center,thick, scale=2] (omit3dleft) at ([xshift=0.9*\nodehsep]task2dleft.center) {...};
        \node[tasknode,anchor=north](task3dleft) at ([yshift=-1.6\nodehsep]lora3dleft.south) {$\mathrm{Task}_{N_G}^{\mathrm{enxx}}$}; 

        \draw[->,thick] (lora1dleft.south) -- (task1dleft.north);
        \draw[->,thick] (lora2dleft.south) -- (task2dleft.north);
        \draw[->,thick] (lora3dleft.south) -- (task3dleft.north);

        \begin{pgfonlayer}{background}
            % \node[anchor=north,minimum height=0.9*\nodehsep,minimum width=17.2em,fill=orange!8,rounded corners=5pt,draw](databankd2) at ([xshift=0.5em,yshift=-0.9em]dataxxen.south) {};
            \node[anchor=south,minimum height=1.9*\nodehsep,minimum width=18.2em,fill=green!8,rounded corners=5pt,draw,dotted,thick](modelmerge) at ([xshift=9.5em,yshift=-6.2em]databank1d.south) {};
        \end{pgfonlayer}

        \node[basemodel,anchor=north](basemodeldright) at ([yshift=1.8*\nodehsep]modelmerge.south) {Base Model};

        
        \node[loranode,anchor=north](lora2dright) at ([yshift=-1.3\nodehsep]basemodeldright.south) {$\mathrm{LoRA}_{2}$};
        \node[loranode,anchor=west](lora1dright) at ([xshift=-6.6\nodehsep]lora2dright.west) {$\mathrm{LoRA}_{1}$};
        \node[align=center,thick, scale=2] (omit2dright) at ([xshift=\nodehsep]lora2dright.center) {...};
        \node[loranode,anchor=east](lora3dright) at ([xshift=6.6\nodehsep]lora2dright.east) {$\mathrm{LoRA}_{N_L}$}; 
        
        \node(plus1dright) at ([yshift=0.6\nodehsep]lora1dright.north) {\textbf{\(+\)}};
        \node(plus2dright) at ([yshift=0.6\nodehsep]lora2dright.north) {\textbf{\(+\)}};
        \node(plus3dright) at ([yshift=0.6\nodehsep]lora3dright.north) {\textbf{\(+\)}};


        \node[mergemodelnode,anchor=north](mergedmodel) at ([yshift=-1.4\nodehsep]lora2dright.south) {Group-wise Merged Model};
        \draw[thick, ->] (modelmerge.south) -- (mergedmodel.north);
        
        \node[tasknode,anchor=north](task1dright) at ([yshift=-4.3\nodehsep]lora1dright.south) {$\mathrm{Task}_1^{\mathrm{xxen}}$};
        \node[tasknode,anchor=north](task2dright) at ([yshift=-4.3\nodehsep]lora2dright.south) {$\mathrm{Task}_2^{\mathrm{xxen}}$};
        \node[align=center,thick, scale=2] (omit3dright) at ([xshift=1.1*\nodehsep]task2dright.center) {...};
        \node[tasknode,anchor=north](task3dright) at ([yshift=-4.3\nodehsep]lora3dright.south) {$\mathrm{Task}_{N_G}^{\mathrm{xxen}}$}; 

        \draw[->,thick] ([xshift=-5.0\nodehsep]mergedmodel.south) -- (task1dright.north);
        \draw[->,thick] (mergedmodel.south) -- (task2dright.north);
        \draw[->,thick] ([xshift=6.0\nodehsep]mergedmodel.south) -- (task3dright.north);

        \draw[->,thick] (databank1d.west) -- (dataselectiond.north) node[midway, left] {En$\rightarrow$XX};
        \draw[->,thick] (databank1d.east) -- ([xshift=0.1*\nodehsep]modelmerge.north) node[midway, right] {XX$\rightarrow$En};

        
        \node[anchor=south,font=\Large](l4) at ([xshift=.5em,yshift=-3em]backgroundd.south) {(b) Direction-aware Training with Group-wise Model Merging};
        
    \end{scope}

    
\end{tikzpicture}
    
    % \vspace{-1em}
    \vspace{-0.5em}
    % \caption{(a) Seperate/Multilingual/Group multilingual training; In seperate  (b) Group-wise model merging}
    \caption{(a)  \textbf{Separate Training ($N$ = $N_L$)}: Each translation task is trained independently using different datasets for different language pairs, with distinct LoRA model weights fine-tuned separately;
    \textbf{Multilingual Training ($N$ = $1$)}: All language pairs are combined to fine-tune a single model with shared LoRA weights;
    \textbf{Group Multilingual Training ($N$ = $N_G$)}: Language pairs are grouped as specified in Table \ref{tab:languages1}-\ref{tab:languages2}, with an adapter trained for each group.
    (b) \textbf{Group-wise model merging}: For XX$\rightarrow$En translation, separate training is applied to each language pair. For En$\rightarrow$XX translation, group training is applied, where different tasks share LoRA weights within language groups.}
    \label{fig:architecture}
    \vspace{-1.1em}
% \end{figure*}
\endgroup

\label{fig:comparison-merge}
\end{figure*}

% \end{document}

\section{Further Experimental Details}
\label{apdx:details}
In this section, we append further experimental details and provide formal definitions of the baselines evaluated in the manuscript.

\subsection{Implementation Details}
For supervised fine-tuning, we utilize Low-rank Adaptation~\cite{hulora}.
In Table~\ref{tab:hyperparam}, we disclose detailed LoRA configurations and other training hyperparameters used for supervised fine-tuning.

\begin{figure*}[!ht]
    \centering
    \includegraphics[width=\linewidth]{rebuttal-figures-src/hyperparams.pdf}
    \vspace{-1.5em}
    \caption{Concept Sliders Comparison \& Hyperparameter analysis: (Left) Impact of PCA directions: SliderSpace with 10 directions matches the FID of 64 Concept Sliders. More directions, upto 40, leads to improved FID. (Right) Effect of LoRA rank: Given a fixed training budget rank-one sliders are efficient than higher rank versions and outperforms Concept Sliders}
    \vspace{-0.3em}
    \label{fig:reb-hyperparam}
\end{figure*}



\subsection{Baseline Definitions}
\label{app:baseline}
Here, we provide formal definitions for each baseline compared in Table~\ref{tab:req1}.


\noindent\textbf{Definition 1.} \textit{(\textbf{Zlib Score}) is the negated ratio of the log perplexity and the zlib compression size:}
\begin{equation}
    -\frac{1}{n}\sum_{i=1}^n \frac{ -\frac{1}{|\mathcal{T}_i|}\sum_{x_j \in \mathcal{T}_i} \log P_\theta(x_j | x_{<j})}{\text{Zlib}(\mathbf{x}_i).\text{size}},
\end{equation}
\textit{where $\mathcal{T}_i$ is the set of tokens from sample $i$.}~\cite{carlini2021extracting}

\noindent\textbf{Definition 2.} \textit{(\textbf{Perplexity Score}) is the negated average perplexity across samples:}
\begin{equation}
    -\frac{1}{n}\sum_{i=1}^n \text{exp}\bigg(-\frac{1}{|\mathcal{T}_i|}\sum_{x_j \in \mathcal{T}_i} \log P_\theta(x_j | x_{<j})\bigg),
\end{equation}
\textit{where $\mathcal{T}_i$ is the set of tokens from sample $i$.}~\cite{li2023estimating}


\noindent\textbf{Definition 3.} \textit{(\textbf{Min-K\% Score}) is the negated mean probability from bottom-$k\%$ tokens averaged across samples:}
\begin{equation}
    -\frac{1}{n \cdot |\mathcal{K}_i|}\sum_{i=1}^n \sum_{x_j \in \mathcal{K}_i} \log P_\theta(x_j | x_{<j}),
\end{equation}
\textit{where $\mathcal{K}_i$ is the set of bottom-$k\%$ tokens from sample $i$.}~\cite{shidetecting}

\noindent\textbf{Definition 4.} \textit{(\textbf{Min-K\%++ Score}) is the negated mean normalized probability from bottom-$k\%$ tokens averaged across samples:}
\begin{equation}
    -\frac{1}{n \cdot |\mathcal{K}_i|}\sum_{i=1}^n \sum_{x_j \in \mathcal{K}_i} \frac{\log P_\theta(x_j | x_{<j}) - \mu_{x_{<j}}}{\sigma_{x_{<j}}},
\end{equation}
\textit{where $\mathcal{K}_i$ is the set of bottom-$k\%$ tokens from sample $i$, $\mu_{x_{<j}} = \mathbb{E}_{z\sim p(\cdot | x_{<j})} [\log p(z | x_{<j})]$ is the expected log probability over the vocabulary of the model, and $\sigma_{x_{<j}} = \sqrt{\mathbb{E}_{z\sim p(\cdot | x_{<j})} [(\log p(z | x_{<j}) - \mu_{x_{<j}})^2]}$ is the standard deviation.}~\cite{zhang2024min}

Following the general guideline from \citet{shidetecting}, we take the bottom 20\% tokens for the Min-K\% Score and Min-K\%++ Score.

\noindent\textbf{Definition 5.} \textit{(\textbf{Fine-tuned Score Deviation}) is the difference of scores before and after supervised fine-tuning, averaged across samples:}
\begin{equation}
    \frac{1}{n}\sum_{i=1}^n S(\mathbf{x}_i; \theta) - S(\mathbf{x}_i; \theta'),
\end{equation}
\textit{where $\mathbf{x}_i$ is the $i$-th sample in the dataset, $S(\cdot;\cdot)$ is an existing scoring function~(e.g., Min-K\% or Perplexity Score), and $\theta, \theta'$ are models before and after fine-tuning, respectively.}~\cite{zhang2024fine}


\noindent\textbf{Definition 6.} \textit{(\textbf{Sharded Rank Comparison Test}) is the difference between the log likelihood of the canonical dataset sample ordering from the mean over shuffled sample orderings, averaged across dataset shards:}
\begin{equation}
    \frac{1}{r} \sum_{k=1}^r \bigg[ \log P([x_i^{(k)}]_{i=1}^n) - \frac{1}{|\frak{S}|} \sum_{\sigma \in \frak{S}} \log P([x_{\sigma(i)}^{(k)}]_{i=1}^n) \bigg],
\end{equation}
\textit{where $r$ is the number of shards, $\frak{S}$ is the set of sample permutations, and $[x_i^{(k)}]_{i=1}^n$ is the sequence of samples $x_1, x_2, \ldots, x_n$ in $k$-th shard of the dataset.}~\cite{orenproving}
\section{Toolkit Usage}




%\subsection{Usage steps}


Figure~\ref{fig:recommendation_yamls} provides an overview of the three main steps for utilizing our toolkit, FairDiverse. We will describe each step of usage in detail. The detailed configuration file parameters can be found in~\url{https://xuchen0427.github.io/FairDiverse/}.

\subsection{Usage Steps}
\textbf{Step 1.} First, download the dataset you wish to test, as described in Section~\ref{sec:datasets}, and store it in the \texttt{/dataset} directory. Then, specify the parameters in \texttt{/properties/dataset/\{data\_name\}.yaml.}
Next, you need to review all default parameters for data processing, model hyperparameters, and evaluation settings.

\noindent\textbf{Step 2.} Then, you need to create your own configuration file, specifying the selected models and the log directory path. If you want to modify the default pipeline parameters, you can specify them directly in your own configuration file, which will override the values in the default configuration files.

\noindent\textbf{Step 3.} Enter \texttt{/fairdiverse} dictionary and execute the command

\begin{lstlisting}[style=shell]
python main.py --task "recommendation" 
--stage "in-processing" --dataset "steam" 
--train_config_file "In-processing.yaml"
\end{lstlisting}

\noindent%
The args should specify the task (recommendation/search), stage (pre-processing, in-processing, post-processing), dataset, and your custom configuration file as defined in Step 2. Finally, the evaluation results and item/user utility allocations will then be recorded in your specified log file.

\subsection{Usage Example}
Besides our provided \texttt{main.py} file and shell command, you can also utilize the following test codes to run the toolkit.

\noindent\textbf{Recommendation.}
Our repository includes an example dataset, Steam.\footnote{\url{http://cseweb.ucsd.edu/~wckang/Steam_games.json.gz}} We provide the simple code snippets for running the in-processing and post-processing models, which are listed below. The user needs to specify the chooses ``model,'' ``dataset'' and ``log\_name'' for training and testing. 

\begin{lstlisting}[language=Python]
from recommendation.trainer import RecTrainer

config = {'model': 'BPR', 'data_type': 'pair', 'fair-rank': True, 'rank_model': 'APR', 'use_llm': False, 'log_name': "test", 'dataset': 'steam'}

trainer = RecTrainer(train_config=config)
trainer.train()
\end{lstlisting}

\begin{lstlisting}[language=Python]
from recommendation.reranker import RecReRanker

config = {'ranking_store_path': 'steam-base-mf', 'model': 'CPFair', 'fair-rank': True, 'log_name': 'test', 'fairness_metrics': ["GINI"], 'dataset': 'steam'}

reranker = RecReRanker(train_config=config)
reranker.rerank()
\end{lstlisting}

\noindent\textbf{Search.}
Our repository includes a running example of the pre-processing models on the COMPAS dataset. We provide the simple code snippet for running the pre-processing models, listed as follows. One can set the ``preprocessing\_model'' field to any of the supported models: CIFRank, LFR, gFair and iFair. Each pre-processing model has its own config file under \texttt{search/properties/models}  which is automatically loaded based on your choice. 

\begin{lstlisting}[language=Python]
from search.trainer_preprocessing_ranker import RankerTrainer
 
config={"train_ranker_config": {"preprocessing_model": "iFair", "name": "Ranklib", "ranker": "RankNet", "lr": 0.0001, "epochs": 10}}
 
reranker=RankerTrainer(train_config=config)
reranker.train()
\end{lstlisting}

For the post-processing models, our repository also provides a running example on the ClueWeb09 dataset. The simple code snippet for running these models is shown as follows. %The user needs to specify ``model'' for training and testing.

\begin{lstlisting}[language=Python]
from search.trainer import SRDTrainer
   
config={'model':'xquad', 'dataset':'clueweb09', 'log_name': 'test', 'model_save_dir': "model/", 'tmp_dir': "tmp/", 'mode': "train",}
 
trainer = SRDTrainer(train_config=config)
trainer.train()
\end{lstlisting}



%, for the recommendation task, along with template configuration files for testing the toolkit. The toy dataset provided for the search task includes the COMPAS dataset \cite{bias2016there}. You can directly execute the provided shell command to run the test.
\section{Experiments: Planning outperforms Heuristics}
\label{sec:experiment}

We begin our empirical demonstrations by showcasing the effectiveness of our planning framework on both synthetic and real datasets. We focus on the simplest planning algorithm, 1-step lookaheads (Algorithm~\ref{alg:complete}), and show that even basic planning can hold great promise. 
We illustrate our framework using two uncertainty quantification modules---GPs and 
\ensembles/ \ensembleplus. 

Throughout this section, we focus on evaluating the mean squared error of 
a regression model $\model$,  and develop adaptive policies that minimize uncertainty on $g(f)$ defined in~\eqref{eqn:l2-g-f}.
When GPs provide a valid model of uncertainty, 
our experiments show that our planning framework significantly outperforms other baselines. 
We further demonstrate that our conceptual framework extends to deep learning-based uncertainty quantification methods such as  \ensembleplus while highlighting computational challenges that need to be resolved in order to scale our ideas. 
For simplicity, we assume a naive predictor, i.e., $\psi(\cdot) \equiv 0$. However, we emphasize that this problem is just as complex as if we were using a sophisticated model $\psi(.)$. The performance gap between the algorithms 
primarily depends
on the level  of uncertainty in our prior beliefs.

To evaluate the performance of our algorithm, we benchmark it against several baselines. 
%Active learning baselines use an acquisition function $\ac$ to select points that have the highest   function value: $X\opt_t \in \argmax_{X \in \xpoolj{t}} \ac({X})$ at every step $t$. These methods may also need an UQ module, which we simply use the same UQ module as in our algorithm, and it  outputs $V(X)$ that measures the the uncertainty of each point $X \in \xpoolj{t}$.
Our first set of baselines are from active learning~\citep{AggarwalKoGuHaPh14}:
\\ % \noindent\textbf{Active Learning Heuristics:} 
\textbf{(1)} 
\textsf{Uncertainty Sampling (Static):}  In this approach, we query the samples for which the model is least certain about. Specifically, we estimate the variance of the latent output $f(X)$ for each $X \in \xpool$ using the UQ module and select the top-$K$ points with the highest uncertainty. \\
\textbf{(2)} \textsf{Uncertainty Sampling (Sequential):} This is a greedy heuristic that sequentially selects the points with the highest uncertainty within a batch, while updating the posterior beliefs using pseudo labels from the current posterior state. Unlike \textsf{Uncertainty Sampling (Static)}, this method takes into account the information gained from each point within batch, and hence tries to diversify the selected points within a batch. 

 
We also compare our approach to the  \textbf{(3)} \textsf{Random Sampling}, which selects each batch uniformly at random from the pool. Additionally, we compare solving the planning problem using  \textsf{REINFORCE}-based policy gradients with   $\mathsf{Smoothed\text{-}Autodiff}$ policy gradients.\footnote{Our code repository is available at
  \url{https://github.com/namkoong-lab/adaptive-labeling}.}
%Detailed experimental setups are provided in Section \ref{sec:details-experiments}.

%We repeat all experiments with 10 random seeds.




\begin{figure}[t]
\centering
\begin{minipage}[b]{0.49\textwidth}
\centering
\includegraphics[width=\textwidth, height=5cm]{figures/original_scale/Var_of_l_2_loss.pdf}
\caption{(Synthetic data) Variance of mean squared loss evaluated through the posterior belief $\mu_t$ at each horizon $t$. This is the objective that policy gradient methods like \textsf{REINFORCE} and $\ouralgo$ optimizes. 1-step lookaheads are surprisingly effective even in long horizons.}
\label{fig:var-l2-sim}
\end{minipage}
\hfill
\begin{minipage}[b]{0.49\textwidth}
\centering \includegraphics[width=\textwidth, height=5cm]{figures/original_scale/Error_of_estimated_model_l_2_loss.pdf}
\caption{(Synthetic data) Error between MSE calculated based on collected data $\mc{D}^{0:T}$ vs. population oracle MSE over $\mc{D}_{\rm eval} \sim P_X$. Reducing uncertainty over posteriors directly leads to better OOD evaluations. 1-step lookaheads significantly outperform active learning heuristics in small horizons.}
\label{fig:mean-l2-sim}
\end{minipage}
%\caption{Simulated data for GPs}
%\label{fig:both_plots}
\end{figure}

\subsection{Planning with Gaussian processes}
\label{sec:experiment-plan-GP}
We now briefly describe the data generation process for the GP experiments,  deferring a more detailed discussion of the dataset generation to Section~\ref{sec:details-experiments}. 
We use both the synthetic data and the real data to test our methodology.
For the \emph{simulated data},  we construct a setting where the general population is distributed across \emph{51 non-overlapping clusters} while the initial labeled data $\dtrain$ just comes from one cluster. In contrast, both $\dpool \defeq (\xpool,\ypool),\deval \defeq (\xeval,\yeval)$ are generated   from all the clusters. 
We begin with a low-dimensional scenario, generating a one-dimensional regression setting using a GP. %Gaussian Process (GP).
Although the data-generating process is not known to the algorithms,  we assume that the GP hyperparameters are known to all the algorithms
to ensure fair comparisons. This can be viewed as a setting where our prior is well-specified, allowing us to isolate the effects
of different policy optimization approaches
 without any concerns about the misspecified priors. We select $10$ batches, each of size $K=5$ across $T = 10$ time horizons.

To examine the robustness of our method against the distributional assumptions made  in the simulated case, we then move to a real dataset where the correct prior is not known. We simulate selection bias from the eICU dataset~\citep{PollardJoRaCeMaBa18}, which contains real-world patient data with in-hospital mortality outcomes. 
We conduct a $k$-means clustering to generate 51 clusters and then select data from those clusters. We view this to be a credible replication of practice, as severe distribution shifts are common due to selection bias in clinical labels.  To convert the binary mortality labels into a regression setting, we train a  random forest classifier and fit a GP on predicted scores, which serves as the UQ module for all the algorithms. As before, the task is to select 10 batches, each consisting of 5 samples, across 10 time horizons.

 In Figures~\ref{fig:var-l2-sim} and~\ref{fig:mean-l2-sim}, we present results for the simulated data. 
Figure~\ref{fig:var-l2-sim} shows the variance of $\ell_2$ loss, and Figure~\ref{fig:mean-l2-sim} presents the error in the estimated $\ell_2$ loss using $\mu_t$ (relative to true $\ell_2$ loss, that is unknown to the algorithm). 
As we can see from these plots, our method one-step lookahead  gives substantial improvements  over active learning baselines and random sampling. In addition,
compared to the one-step lookahead planning approach using \textsf{REINFORCE}-based policy gradients, 
we observe that $\mathsf{Smoothed\text{-}Autodiff}$-based policy gradients provide significantly more robust performance over all horizons.

In Figures~\ref{fig:var-l2-real}~and~\ref{fig:mean-l2-real}, we observe similar findings on the eICU data. We see that planning policies (\textsf{REINFORCE} and $\mathsf{Smoothed\text{-}Autodiff}$) consistently outperform other heuristics by a large margin.  Active learning baselines perform poorly in these small-horizon batched problems and can sometimes be even worse than the random search baselines.  Overall, our results show the importance of careful planning in adaptive labeling for reliable model evaluation. 

We offer some intuition as to why one-step lookahead planning may outperform other heuristic algorithms. 
 First,  \textsf{Uncertainty sampling (Static)} while myopically selects the
 top-$K$ inputs with the highest uncertainty, it fails to consider 
the overlap in information content among the ``best” instances; see \citep{AggarwalKoGuHaPh14} for more details. 
In other words,  it might acquire points from the same region with high uncertainty while failing to induce diversity among the batch.
Although \textsf{Uncertainty Sampling (Sequential)} somewhat addresses the issue of information overlap, a significant drawback of 
this algorithm
is the disconnect between the objective we aim to optimize and the algorithm. For example, it might sample from a region with high uncertainty but very low density. 

\begin{figure}[t]
\centering
\begin{minipage}[b]{0.48\textwidth}
\centering
\includegraphics[width=\textwidth, height=5cm]{figures/original_scale/Var_of_l_2_loss_real.pdf}
\caption{(Real-world eICU data) Variance of mean squared loss evaluated through the posterior belief $\mu_t$ at each horizon $t$. Even 1-step lookaheads are extremely effective planners, and auto-differentiation-based pathwise policy gradients provide a reliable optimization algorithm based on low-variance gradient estimates.}
\label{fig:var-l2-real}
\end{minipage}
\hfill
\begin{minipage}[b]{0.48\textwidth}
\centering \includegraphics[width=\textwidth, height=5cm]{figures/original_scale/Error_of_estimated_model_l_2_loss_real.pdf}
\caption{(Real-world eICU data) Error between MSE calculated based on collected data $\mc{D}^{0:T}$ vs. population oracle MSE over $\mc{D}_{\rm eval} \sim P_X$. Reducing uncertainty over posteriors directly leads to better OOD evaluations. Our method significantly outperforms active learning-based heuristics, and random sampling.}
\label{fig:mean-l2-real}
\end{minipage}
%\caption{Real data for GPs}
\end{figure}
 
%\vspace{-1.5cm}
% \begin{wrapfigure}{r}{.32\columnwidth}
%   \vspace{-.5cm} 
%   \centering
% \includegraphics[scale=.29]{figures/Var of l2l_2 loss.pdf}
%   \vspace{-0.2cm}
%   \caption{Results of GP}
% \label{fig:var-l2-gp}
%   \vspace{-0.1cm}
% \end{wrapfigure}


% Attempts have been made  in the past to address these  drawbacks heuristically  (see \citep{AggarwalKoGuHaPh14}). We give a unified computational framework while approaching the problem in a more principled manner and solving it more optimally.




\subsection{Planning with  neural network-based uncertainty quantification methods ($\ensembleplus$)}


We now provide a proof-of-concept that shows the generalizability of our conceptual framework  to the deep learning-based UQ modules, specifically focusing on $\ensembleplus$ due to their previously observed superior performance~\citep{OsbandWenAsDwIbLuRo23}. Recall that implementing our framework with deep learning-based UQ modules  requires us to retrain the model across multiple possible random actions $\bm{a}(\theta)$ sampled from the current policy $\pi_\theta$.
This requires significant computational resources, in sharp contrast to the GPs where the posteriors are in closed form and can be readily updated and differentiated. 

Due to the computational constraints, we test $\ensembleplus$ on a toy setting to demonstrate the generalizability of our framework. We consider a setting where the general population consists of four clusters, while the initial labeled data only comes from one cluster. Again we generate data using GPs.  The task is to select a batch of 2 points in one horizon. We detail the $\ensembleplus$ architecture in Section \ref{sec:details-experiments}, and we assume prior uncertainty to be large (depends on the scaling of the prior generating functions). 
The results are summarized in the Table~\ref{tab:UQ_ensemble}.

% \begin{table}[H]
% \vspace{-10pt}
% \caption{Performance under \ensembleplus as UQ module}
%     \centering
%     \begin{tabular}{|m{3cm}|m{2.5cm}|m{2cm}|} 
%     \hline
%       Algorithm   & Variance of $\loss_2$ loss estimate & Error of $\loss_2$ loss estimate  \\ \hline Random Sampling 
%          & $1710.9 \pm 1352.1$ & $8.67\pm6.62$ 
%       \\ \hline \ouralgo & $1.30 \pm 0.68$ & $0.91\pm0.25$ \\ \hline
%     \end{tabular}
%     \label{tab:UQ_ensemble}
%     %\vspace{-10pt}
% \end{table}




\begin{table}[h]
\vspace{-10pt}
\caption{Performance under \ensembleplus as the UQ module}
\centering
\begin{tabular}{|l|l|l|}
\hline
Algorithm   & Variance of $\loss_2$ loss estimate & Error of $\loss_2$ loss estimate  \\
\hline
\textsf{Random sampling} & 7129.8 $\pm$ 1027.0 & 136.2 $\pm$ 8.28 \\ \hline
\textsf{Uncertainty sampling (Static)} & 10852 $\pm$ 0.0 & 162.156 $\pm$ 0.0 \\ \hline
\textsf{Uncertainty sampling (Sequential)} & 8585.5 $\pm$ 898.9 & 144 $\pm$ 6.93 \\ \hline
\textsf{REINFORCE} & 1697.1 $\pm$ 0.0 & 45.27 $\pm$ 0.0 \\ \hline
\ouralgo & 1697.1 $\pm$ 0.0 & 45.27 $\pm$ 0.0 \\ \hline
\end{tabular}
%\caption{Comparison of different algorithms based on variance   and   error in $\ell_2$ loss estimation with Ensemble $+$ as the UQ module. Our results demonstrate that {\ouralgo} and REINFORCE outperformthe other active learning based heuristics, confirming the benefits of our MDP formulation for the adaptive labeling problem, as also demonstrated in Section 4.\\
%\footnotesize{Experimental details: We use Gaussian Processes as our data generating process, GP parameters are the same as in Section D.3.  The task is to select a batch of 2 points along one horizon.The marginal distribution $p_X$ has 4 \textit{non-overlapping} clusters. Initial data comes from one cluster, while pool and evaluation points comes from all the clusters. We have $20$ initial labeled data points, $10$ pool points, and $252$ evaluation points.  Training procedures are similar to the one in Section D.3.} }
\label{tab:UQ_ensemble}
\end{table}



% We faced  issues in scaling up these experiments which will be our focus in the future. 





% \begin{itemize}
%     \item Posteriors should be consistent. Two dimensions: even with less training,  
%     \item the inference should be  fast enough
% \end{itemize}


% Potential research directions for uncertainty quantification

% In this section we consider a simple setting We consider a simpler setting and 


% For synthetic dataset generation, we use ...... For real datasets, we use ...... We compare our methodolgy to several baselines ()    This Section is structured as follows:
% \begin{itemize}
%     \item \textbf{GPs, square loss objective} (Section \ref{}): 
%     %the broad aim of the experiments  in this section is to isolate the performance of our methodology without any concerns for the inefficiencies induced due to a mis-specified prior or imperfect posterior inference. To accomplish this we generate synthetic datasets using GPs (detailed later). We use the well specified prior (GPs - with same hyperparameter setting) as our UQ module.   
%      As GPs provide differentaible posterior inference - any errors induced due to imperfect posterior updates are also isolated. We note that under this setting
%      \item In Section\ref{} we demonstrate why our methodology performs better than other baselines - by devising various synthetic experiments ()
%     \item  \textbf{UQ Benchmarking }(Section \ref{}): Before diving into the experiments using $\ensembleplus$ and ENNs,  we showcase our benchmarking experiments in Section \ref{}. We use real datasets We observe that ENNs perform better
%      \item \textbf{Ensemble $+$}, objective: recall, accuracy
%     \item \textbf{ENN}, objective: recall, accuracy
% \end{itemize}




% In Section {}, we test 
% \subsection{Experimental details}

% \begin{itemize}
%     \item UQ methodologies - GPs, ENNs
%     \item Objectives - Recall,  ATE
%     \item Datasets - ATE-synthetic datasets, Recall-synthetic, real datasets
%     \item Baselines - 
%     \begin{itemize}
%         \item Random sampling
%         \item Active learning - Uncertainty based sampling - In regression setting almost all of the 
%         \item Myopic greedy - Greedy Batch based sampling
%         \item Policy Gradient
%     \end{itemize}
    
% \end{itemize}

% \subsection{Experiments}
%     \begin{itemize}
%     \item GPs with square loss
%     \item Benchmarking ENN
%         \item ENNs with ATE
%         \item ENNs with Recall
%     \end{itemize}

% \subsection{Benefits over other algorithms - intuition and experiments}

%Active learning - Myopic greedy / Don't rely on the objective rather some entropy version.


%%% Local Variables:
%%% mode: latex
%%% TeX-master: "main"
%%% End:

\section{Customizing Models in FairDiverse}
We outline the steps for customizing and evaluating new IR models using the APIs we provide. Detailed API descriptions and source code can be found in~\url{https://xuchen0427.github.io/FairDiverse/}. 
The provided APIs can be used by installing them via pip: 

\begin{lstlisting}[style=shell]
pip install fairdiverse
\end{lstlisting}

%\texttt{pip install fairdiverse}.
%The steps can be easily implemented with just a few lines of code.

\subsection{Steps}
Figure~\ref{fig:rec_APIs} illustrates the three key steps for implementing fairness- and diversity-aware IR models named \textit{YourModel}.

\noindent\textbf{Step 1.} Configure your custom model parameters and save them in a newly created \texttt{YourModel.yaml} file in the \texttt{/properties/models/} directory. Then you can change the model in the running configuration file \texttt{Post-processing.yaml}.

\noindent\textbf{Step 2.} Select the appropriate Python abstract class from our provided options based on your model type and implement your model in a newly created file, \texttt{YourModel.py}, stored in the corresponding directory. You can use the integrated tools and common parameters within the abstract class. Researchers only need to focus on designing the model without worrying about the rest of the pipeline. 

%Different types of models must adhere to the function formats and return value structures defined in the abstract class. 

\noindent\textbf{Step 3.} Configure your model for the training pipeline by following these steps: import your custom model package in the corresponding file (\texttt{/model\_type/\_\_init\_\_.py}) and define the model in the appropriate script (\texttt{/train.py}, \texttt{/reranker.py}).

%The implementation for other models follows a similar approach.

\subsection{Examples}
%\textbf{Examples.} 
Here, we provide two example codes demonstrating how to design a custom search and recommendation model, respectively. 
\begin{lstlisting}[language=Python]
#/recommendation/rank_model/YourModel.py
class YourModel(Abstract_Regularizer):
    def __init__(self, config, group_weight):
        super().__init__(config)

    def fairness_loss(self, input_dict):
        return torch.var(input_dict['scores'])

#/recommendation/rank_model/__init__.py
from .YourModel import YourModel

#/recommendation/trainer.py
if config["model"] == "YourModel":
  self.Model = YourModel(config)
\end{lstlisting}

\begin{lstlisting}[language=Python]
#/search/preprocessing_model/YourModel.py
class YourModel(PreprocessingFairnessIntervention):
    def __init__(self, configs, dataset):
        super().__init__(configs, dataset)

    def fit(self, X_train, run):
    # Train the fairness model using the training set.

    def transform(self, X_train, run file_name=None):
    # Apply the fairness transformation to the dataset.

#/search/preprocessing_model/__init__.py
from .YourModel import YourModel
fairness_method_mapping['YourModel'] = YourModel


\end{lstlisting}



% #test.py
% from recommendation.trainer import RecTrainer

% config = {'model': 'BPR', 'data_type': 'pair', 'fair-rank': True, 'rank_model': 'YourModel', 'use_llm': False, 'log_name': "test", 'dataset': 'steam'}

% trainer = RecTrainer(train_config=config)
% trainer.train()

% \subsection{Search}
% As for the search part, the key steps for implementing a diversified search model named \textit{YourModel}.

%can also refer to Figure~\ref{fig:rec_APIs}.

% \textbf{Step 1.} First, configure your custom model parameters and save them in a newly created \texttt{YourModel.yaml} file within the \texttt{/properties/models/} directory. Then you can change the model in the running configuration file \texttt{Post-processing.yaml}.

% \textbf{Step 2.} Select the appropriate Python abstract class from our provided options based on your model type and implement your model in a newly created file, \texttt{YourModel.py}, stored in the corresponding directory (see Figure~\ref{fig:rec_APIs} for details). Different types of models must adhere to the function formats and return value structures defined in the abstract class. To facilitate development, we have integrated various tools and common parameters within the abstract class. Researchers only need to focus on designing the model itself without worrying about the rest of the pipeline, creating a convenient environment for fair comparisons across different models.


% \textbf{Step 3.} Configure your model for the training pipeline by following these steps: import your custom model package in the corresponding file (\texttt{/model\_type/\_\_init\_\_.py}) and define the model in the appropriate script (\texttt{/train.py}).
\section{Related Work}
% \subsection{Vision Language Model}
% 시각장애인에서 상황을 설명할 DB가 없으니 만들었다. 그리고 이를 VLM에 튜닝했다.
\subsection{Technical approaches for assisting the visually-impaired}


\subsection{Datasets for visual instruction tuning}

Software development is increasingly conceived as a collaboration activity between developers and AIs. Indeed, IDEs already implement features to enable interactive development, with AI suggesting implementations that are reused by developers.

Although multiple studies show this interaction can be successful, there is still limited understanding of how the models must be configured and used in the context of code generation tasks. This study addresses this gap, systematically investigating the impact of several key parameters, including the repeated submission of a prompt to accommodate for the non-deterministic nature of the models.

Our study reveals several key findings about the usage of ChatGPT. In particular, we discovered how creativity, although up to a limited extent, is useful to increase the range of methods whose code can be generated correctly. A major role is played by parameter top-p, which is commonly underrated, and instead has a major impact on the correctness of the results, with lower values producing better results. Finally, prompts should be submitted multiple times, with $5$ repetitions combined with a temperature of $1.2$ resulting in an effective configuration in our experiments.  

Future work concerns two main research directions. One is about replicating this experiment with other AI assistants, to validate our findings in multiple contexts. The second research direction concerns finding strategies to deal with the need to submit the same prompt multiple times to obtain a useful result, and thus developing approaches able to select or merge multiple responses automatically. 

\newpage
\bibliographystyle{ACM-Reference-Format}
\bibliography{references}

\end{document}
