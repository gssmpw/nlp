\section{Formulation}
In this section, we aim to formulate the fairness and diversity-aware recommendation and search tasks. 
In general, both tasks can be considered ranking tasks; however, their inputs and utility definitions differ slightly.


\subsection{Recommendation}
In a recommender system (RS), let $\mathcal{U}$ denote the set of users, and $\mathcal{I}$ the set of items. Each user $u\in\mathcal{U}$ can have a different user profile $\mathcal{P}_u$ such as age, gender, occupation, etc.
Each item $i\in\mathcal{I}$ is associated with specific attributes, such as item categories, descriptions, and other related metadata. 

When the user $u$ arrives in the RS, the recommender model $f$ will 
firstly extract its user profile $\mathcal{P}_u$ and his/her browsing historical item list $H_u=[i_1, i_2, \cdots, i_n]$. Then RS will recommend a ranking list $L_K(u)\in\mathcal{I}^K$ to the user $u$:
\[
L_K(u) = f(\mathcal{P}_u, H_u),
\] 
where $n$ is the history length and $K$ is fixed ranking size.

In RS, fairness- and diversity-oriented algorithms typically focus on the group level, as groups can represent item categories, providers, or different user types~\cite{xu2023p, fairrec, li2021user}. Without loss of generality, we define the set of items or users associated with a specific group $g\in\mathcal{G}$. The concept of individual fairness~\cite{biega2018equity} or diversity can be extended to treat the group $g$ as an individual item $i$ or user $u$, allowing the algorithms to operate without any modifications to their underlying structure.

Then, the item and user utility of the group $g$ can be defined as $v_g^i$ and $v_g^u$, respectively:
\begin{equation}
     v_g^i = \sum_{u\in\mathcal{U}}\sum_{i\in L_K(u)} w_{u,i}\mathbb{I}(i\in\mathcal{G}),~v_g^u = \sum_{u\in\mathcal{G}}\sum_{i\in L_K(u)}w_{u,i},
\end{equation}
where $\mathbb{I}(\cdot)$ is the indicator function. Typically, $w_{u,i}$ can be defined in two ways: (1) as item exposure~\cite{xu2023p, fairrec}, meaning that an item gains one unit of utility if it is exposed to a user (\ie 
$w_{u,i}=1$); or (2) as item click~\cite{yang2019bid, liu2021neural}, meaning that an item gains one unit of utility only if it is clicked by a user. Since it is uncertain whether a user will click on an exposed item, we use the estimated probability of a user clicking on an item as the value of $w_{u,i}\in [0,1]$.

Finally, the goal of the fairness- and diversity algorithm $f$ is to make the less-advantaged item groups (\eg categories) get more utilities. 
That is, on the one hand, $f$ aims to maximize the sum of their utilities (\ie $\sum v_g^a,~\forall a=u,i$). On the other hand, $f$ aims to achieve as even utilities ($\bm{v}_g^a \approx \bm{v}_{g'}^a, \forall g,g'\in\mathcal{I}, a=u,i$) as possible. 





\subsection{Search}
xxx