\section{Threats to Validity}

\textbf{Data Source and Diversity}: While we have exclusively sourced our repositories from GitHub and all are open-sourced, this may limit the generalizability of our findings. However, our dataset includes a diverse array of app categories, enhancing its representativeness. For example, we have categorized apps into productivity tools like \textit{Orgzly}~\cite{orgzly} and \textit{nextcloud}~\cite{nextcloud-android}, social media apps such as \textit{Duckduckgo}~\cite{duckduckgo} and \textit{Facebook Android SDK}~\cite{facebook-android-sdk}, multimedia applications like \textit{Pocket Casts}~\cite{pocket-casts-android} and \textit{AntennaPod}~\cite{AntennaPod}, and security-focused apps including \textit{Password Store}~\cite{Password-Store} and \textit{Aegis}~\cite{Aegis}. According to Table~\ref{fig:data selection}, the selected repositories are highly popular, evidenced by high numbers of GitHub stars and extensive download numbers on Google Play. These repositories are also well-maintained, further attesting to the reliability of our data.

\textbf{Classification Subjectivity}: The categorization of apps was performed manually, which could introduce bias. To mitigate this, we employed an open coding approach, with discrepancies resolved through discussions among the research team. The inter-rater reliability, measured by Cohen’s kappa, was 0.81, indicating substantial agreement and reliability in our classification method.


\textbf{Platform Specificity}: Our study focuses solely on Android apps. While this provides in-depth insights into state management issues within this platform, the findings may also apply to other platforms due to the generic nature of login functionalities. This suggests that the problems identified, while analyzed in the context of Android, could potentially manifest in similar ways on other operating systems, thus providing a broader relevance to our conclusions.

% In conclusion, while our study is subject to certain limitations inherent in the scope of the data and methodology, the measures taken to ensure rigorous analysis and the diversity of the app categories examined support the robustness and applicability of our findings. Future research could expand this work by incorporating multi-platform studies to further validate and extend our insights.

