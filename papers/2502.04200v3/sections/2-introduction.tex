\section{Introduction}
% Among the various functionalities of mobile applications, login mechanisms are fundamental to user engagement and security~\cite{12tipreduce,13strategies,10099350}. 
% \lili{The previous citations seem not very appropriate. \cite{12tipreduce,13strategies} are just some random blogs. And, I didn't find the key word ``log in'' in \cite{10099350}}\zixu{We can remove them, blogs were shown android is popular among people}
Login plays a critical role in Android apps, typically representing the first point of interaction between the application and the users~\cite{Alsanousi2023}.
% \lili{What does it mean by ``setting the stage for the user's experience''?}\zixu{setting the stage for, this means build the fundament}
% \lili{Not understandable. Just removed it.}
% A seamless and efficient login process not only improves user satisfaction but also enhances the application's security ~\cite{13strategies,10679403}.
% \lili{We have login mechanisms and login processes. Do they have the same meaning? Or, are they referring to different things?}\zixu{In my perspective, login mechanisms include different login approaches, each login approach has a login process}\zixu{change login mechanisms to login}
% \lili{The previous sentence is very weird.[done]} 
% \zixu{Revise the whole sentence}
% \lili{I removed the previous sentence since it doesn't convey much information and is repetitive when reading together with the next one. In addition, our study didn't really reveal any security issues so I don't want to emphasize too much on this point.}\zixu{I agree, I remove all security from the paper}
Issues in login processes can significantly degrade the user experience~\cite{FTUE,5reasonsfailuser,Mobileappusage,9042272}.
% Issues such as login failures, unexpected logouts, and session management problems, can significantly degrade the user experience~\cite{FTUE,5reasonsfailuser,Mobileappusage,9042272}.
% \lili{The same citations are repeatedly cited several times. Please fix this issue.}
% \zixu{Reorg the citation}
% \lili{This sentence is repetitive with the last few sentences of the previous paragraph.[done]}
% \zixu{Remove this scentence}
Studies have shown that first-time login failures can lead to a direct and immediate drop in user engagement by as much as 25\%~\cite{FTUE,10679403}. Such statistics highlight the critical nature of robust login processes and demonstrate their potential economic impact on software developers, emphasizing the need for improved reliability in the login processes.




 Existing studies have extensively explored a wide spectrum of software defects and usability challenges in Android applications~\cite{Alsanousi2023,10679403,10.1145/3386685}. Studies were conducted to characterize diverse issues in the Android ecosystem ranging from compatibility issues~\cite{huang2023conffix,mahmud2023detecting,yang2023compatibility,chen2024demystifyingdevicespecificcompatibilityissues}, WebView problems~\cite{luo2011attacks,8999997,hu2023omegatest}, cross-platform usability challenges~\cite{steinbock2024comparing,chen2024your,yu2021layout} to specific security vulnerabilities~\cite{10.1145/3545948.3545955,10504267,7589802,10.1145/3301285,10.1145/3321705.3329831,10738442,7962350}.
 Some related studies analyze vulnerabilities related to Android app authentication, an important component in the login process~\cite{10.1145/3551349.3559524,10628996,10.1145/3321705.3329801,8952200}.
 % \lili{What are \cite{,10.1145/3551349.3559524}? Why the following discussed papers are not cited here but these 2 are not discussed later?}\zixu{Added the description to \cite{10.1145/3551349.3559524} later,and move \cite{10.1145/3545948.3545955} to above specific security vulnerabilities}  \lili{How is this one related to authentication? If you want to include this paper here, you need to make it clear.}\zixu{Removed this one}
 For example, Philippaerts et al.~\cite{10.1145/3551349.3559524} proposed Oauch, which can measure how well individual identity providers (IdPs) implement the security specifications defined in the OAuth standard and provide development suggestions to developers.
 Jannett et al.~\cite{10628996} conducted an empirical study on Single Sign-On (SSO) and developed a tool, Sok, focusing on testing SSO in Android by large-scale dual-window analysis. 
 % \lili{What does this tool do? ``Focusing on'' SSO is ambiguous.}\zixu{Add analysis name they offer in paper}
 Shi et al.~\cite{10.1145/3321705.3329801} developed MoSSOT, a backbox tester for the OAuth process in Android. Furthermore, Tamjid et al.~\cite{8952200} conducted an empirical study on the usage of OAuth APIs and their implications for mobile security, leading to the development of OAUTHLINT, a tool to identify vulnerabilities in OAuth implementations for Android apps. These studies mainly focused on vulnerabilities arising from the authentication process.
 However, the login process extends beyond authentication.

 
The login processes can be briefly described in three stages: Pre-login, login, and post-login. Initially, in the Pre-Login phase, the app checks for an existing valid token to either bypass or prompt for user credentials. During the Login phase, the app assesses the credentials and, if necessary, proceeds with multi-factor authentication before granting access. Once authentication is successful, the Post-Login phase commences, granting the user full access to the app's features.
 % For instance, a login process supporting 2FA requires the authors input the correct user credentials as well as a verification code.
 % In this process, the app needs to validate the correctness of the credentials, communicate with the login servers, as well as requesting for and validation the verification code.
 % \lili{I made up this example based on my understanding. If possible, please revise this example to include some descriptions of what can happen in the pre-login and post-login stages.}\zixu{Added some brief descriptions for each stage}
 % \lili{I mean, revise the example not revise the stages...}\zixu{Add a brief example for each stage }
 Developing reliable and robust login processes requires the app developers to properly manage complicated interactions between the app, the users and the servers.
 In addition, such interactions can interfere with the Android system events.
 Such complexities make it especially challenging to implement the login processes correctly.
 % Furthermore, if one wrong operation or status change occurs at any stage, the whole login process will be interrupted. Managing the entire login process requires developers to implement each stage of every login approach properly. For login processes, the complexity in it and it is easily interrupted limit developers' abilities to effectively tackle the problems in the login process.
 % \lili{Add more citations. In addition, you may need to discuss some more closely related papers in more detail. For example the OAuthLint paper. People will wonder: login seems to be related to authentication, what is the difference between our paper and some existing papers related to authentication (e.g., OAuthLint).}\zixu{Added two examples include OAuthLint.}
 % As a result, there remains a discernible gap in research specifically targeting the
 % \lili{Avoid using intricate or intricacies. It is obvious that this is generated by AIs[done]} 
 % \zixu{Remove all intricate and intricacies in the paper}
% login issues in Android applications.
% \lili{What are these references for?[done]}

No existing studies have systematically characterized the various issues in the login processes of Android apps. This paper presents an empirical analysis of the root causes, symptoms, and trigger conditions of login issues in Android apps to address this research gap. 
% \zixu{add why difficult. No states. intro states, complex transfer and easy to interrupt}
Our investigation is structured around the following research questions:
\begin{itemize}
    \item \textbf{RQ1: (Root Causes of Login Issues):}  \textit{What are the common root causes of login issues in Android apps?}
    % \lili{Will we answer the question in RQ1 - how can understanding these causes help developers prioritize their remediation efforts?[done]}\zixu{Remove this part}
    
    \item \textbf{RQ2: (Symptoms of Login Issues):} \textit{What symptoms are commonly observed when login issues occur, and how do they impact user experience?}
    % \lili{Can we answer ``how do they impact both user experience and overall app performance''?}\zixu{Does crash effects overall app performance}\yuki{Usually we don't refer the crash issue as "performance". When we are talking about the performance, we mostly focus on the latency and throughput of a system.}\zixu{only leave user experience}
    
    \item \textbf{RQ3: (Trigger Conditions of Login Issues):} \textit{Under what conditions are login issues likely triggered?}
    % \lili{Will we answer this question in RQ3 - how can this knowledge improve application design and testing?[done]}
    % \zixu{Remove this part}
\end{itemize}

Answering these research questions offers practical benefits to both developers and researchers.
RQ1 characterizes the common root causes of login issues in Android apps to understand how the issues are induced comprehensively. RQ2 and RQ3 investigate the symptoms and trigger conditions of login issues and can guide developers to generate targeted test cases and effective test oracles.
% \yuki{Nov 7: vulnerabilities seems inappropriate here. It's more likely about the security issues. Besides, do we really address some security problems instead of solely login failure? } \lili{I totally agree with Yuki on this point. I see several points we mention ``security'' in the paper but I don't think any of our issues are related to security.}\zixu{I changed to bugs}
% \lili{Now, the motivations of the RQs seem to be missing. What are the practical benefits? Why is it important to answer these RQs? Here, we still need to add some brief motivations for the RQ design. For example, in RQ1, we aim to xxxx. In RQ2, xxx. In RQ3, xxx.}\zixu{Added a simple motivation}
% \lili{The motivation still needs to be revised. I'll get back to this}\zixu{Add more concrete motivation}

We analyzed 361 login issues across 44 open-source Android apps to answer our research questions. 
% This thorough investigation led to several key insights that highlight critical areas for improvement in app stability and user experience.
This analysis leads to several key findings.
We proposed using state machines to model the complicated login processes. We disclosed that the majority (62\%) of the login issues can be attributed to errors in managing the states and their transitions.
% identified state management errors as the most prevalent root cause of login issues, accounting for 62\% of our studied issues.
% These errors typically occur when the app handles user interactions improperly or lacks the necessary design for some scenarios, leading to a variety of consequences.
For example, Home Assistant~\cite{home-assistant-android} requires users to input a URL for login to access remote services. The issue~\cite{HomeAssistantIssue} arose because developers inadequately managed the login process's state; they did not properly \lili{promptly or properly?}\zixu{Done}store and update the new URL provided by the user, leading to a blank screen even with the correct credentials entered. 
% \lili{Can you revise the previous sentence again? This sentence is very vague and not really related to states. Alternatively, we may just describe an example here.}\zixu{Add an  example}
This observation suggests that future research can focus on addressing such login flow errors.
Our proposed state machine can guide the test generation for exposing these errors.
In addition, our study reveals that login flow errors can induce various consequences, including failures in the authentication process, crashes, etc.
These observations can help define the test oracles for effective login flow error detection.

% \lili{Can we better present what state management errors are so that the readers can immediately understand the significance of this issue?}
% \lili{This explanation of state management errors needs to be revised. What is a user state?}\zixu{Revised to: when the app handles user interactions improperly or lacks the necessary design for some scenarios}
% Second, most of the issues lead to authentication failures where the users cannot successfully log into their accounts.
% Third, most of the 
% which often results from these state management discrepancies.
% \lili{Isn't ``failures during the login process" too general?}\zixu{This refers to our Authentication Failures, so I modify the login process to authentication}
% Lastly, the conditions that most frequently trigger these issues were specific login approaches which inspired developers to run test cases for all login approaches their apps supported to facilitate the testing of login issues.
% \lili{This finding also doesn't convey any  information. ``specific login process'' means nothing.}\zixu{Add one sentence to show how to facilitate testing login issues}
% \lili{No, I mean by the name itself ``specific login process'' means nothing. It's not a self-explainable name. For a new reader, what does ``specific login process'' mean? How are these issues triggered? Can you interpret it in a way that the readers can understand?}
% Furthermore, state management errors account for 62\% of overall root causes and can cause all kinds of symptoms. So implementing correct login processes requires developers to understand transactions in login. These findings can also help developers write test cases and test oracles to trigger and detect login issues. 
% \lili{We cannot simply copy the findings here. In the intro, without sufficient context, no one would understand what does it mean by state management errors. We need to introduce the findings in a way that the readers will understand the importance of these findings.}\zixu{Use a paragraph to linked three findings}

% \lili{This paragraph is still very abstract. I think you can move some sentences from here to the previous paragraph where I say we need to specify the motivations of the different RQs. You may remoe some sentences like ``This leads to a smoother user experience and quicker issue handling. We then synthesized the data on root causes, symptoms, and trigger conditions to derive implications. Understanding these implications following the answers to RQ1-3 is essential, as it extends the practical utility of our findings.'' - they don't convey any useful information at all. You can safely remove them. You only need to strengthen the concrete implications and their usefulness when introducing the findings.}\zixu{Remove all, and move some expression to the previous paragraph}


% \lili{This whole paragraph only motivates the RQs with some vague slogans, making it not convincing. While we can put down some motivations for the RQs here, we may need to remove some vague descriptions and include some concrete findings and implication taken from the RQ results. For example, we need to say: to answer the three research questions, we analyzed XX login issues in XX open-source Android apps. Our analysis discloses several interesting findings: (1) XXX, (2)XXX, ...}\zixu{Remove all vague expression and add key findings.}

 
To summarize, this paper makes the following major contributions:
% \lili{May not include such level of concrete contributions at the end of the intro. Just some brief summary or even remove this part.}\zixu{Use a brief version}
\begin{itemize}
\item \textbf{Login issues dataset:} We carried out the first empirical study on login issues within real-world Android apps. Our dataset includes 361 login issues sourced from 44 popular repositories. \textit{We have made this dataset available to support further research}~\cite{code}.


\item \textbf{Comprehensive Categorization:} We systematically identified and categorized the root causes, symptoms, and trigger conditions of login issues in Android apps.
\item \textbf{Bridging Theory and Practice:} By exploring the implications of our findings, this research not only can aid developers in modeling the login process but also help them to design targeted test cases and precise test oracles.
% \lili{I doubt whether the reviewers would agree with this contribution.}\zixu{More detailed usage of our findings, whether this will be better?}
\end{itemize}

All datasets and scripts used in this study are open-sourced and available, further enabling the community to replicate our findings and extend our work~\cite{code}.


% \yuki{As a reviewer, I would be interested in how RQ1 is relevant to RQ2, which is, what kinds of the root causes will lead to the corresponding issue symptom?. For example, apparently the "Null-check omission" will cause the application to crash. What about the "Improper Error State Handling" and "Improper State Saving or Restoration" categories? Which types of issue symptoms will these types of errors manifest?}\zixu{Add a new sentence to indicate the relevance}
% \yuki{The Implications section talks about the relevance, which is exactly I expect for this paper. It's cool!}

% \yuki{\textbf{About the Format:} We should not pay attention to the format or spaces of figures, tables, listings, answer boxes, etc. for now. Still a lot of essential text is required. And, if you want to include the "state machine-based detection tool" in this paper, watch out running out of pages (10 pages, two columns). I think we must have the "Tool Design", "Implementation", "Evaluation" parts for a proposed tool. }\zixu{We won't add tool in this version}