\subsection{RQ2: Symptoms}
Understanding the symptoms of login issues can help developers design effective test oracles.
%\rufeng{Consider adding one sentence of motivation here.}\zixu{Added}
In investigating login issues in Android apps, we carefully examined the descriptions of issues, developer responses, and the differences between the buggy and fixed code snippets. We identified seven major categories of symptoms manifested by login issues. The results are shown in Table~\ref{tab:symptoms}.

\begin{table}
  \centering
  \caption{Summary of Login Issue Symptoms}
  \label{tab:symptoms}
  \begin{tabular}{@{}c l r@{}} % c for center alignment of numbers, l for left, and r for right
    \toprule
    \textbf{No.} & \textbf{Symptoms} & \textbf{\#Issues} \\ 
    \midrule
    1 & Crash & 85 \\
    2 & Login Delays or Timeout & 57 \\
    3 & Incorrect Navigation Flow & 51 \\
    4 & Account Management Function Break & 48 \\
    5 & Credential Rejection & 41\\
    6 & User Interface Error & 40\\
    7 & Inaccurate Error Message Displayed & 39 \\
    \bottomrule
  \end{tabular}
\end{table}

\subsubsection{Crash}
85 of our studied issues are crashes. They frequently occur due to unhandled exceptions within the login process, significantly disrupting user experience and undermining the application's reliability. For example, issue \#109 of Home Assistant~\cite{home-assistant-android} identified that the app crashed when users tried to paste their username or password into the login screen. This issue was caused by conflicts with the Localise SDK, which interfered with clipboard functionality, leading to unhandled exceptions.

\subsubsection{Login Delay or Timeout}
We documented 57 instances where users experienced significant delays or timeouts during the login process. These issues predominantly involve excessive waiting time or complete failures to log in, which can significantly hinder user experience and service access. For instance, issue \#10043 from WooCommerce~\cite{woocommerce-android} mentions a scenario where users attempting to login Jetpack~\cite{Jetpack} using usernames and passwords encounter a premature dismissal of the WebView before they can approve the connection. This abrupt interruption prevents successful authentication, leading to timeouts and disrupting the overall login process.


\subsubsection{Incorrect Navigation Flow}\label{IncorrectNavigationFlow}
51 issues induced incorrect navigation flows, where users are not directed to the correct screen after successfully logging in. For example, The \textit{Open Event Attendee Android} application~\cite{open-event-attendee-android}, developed by FOSSASIA~\cite{fossasia}, is designed to facilitate event participation by allowing users to view event details, book tickets, and more, directly from their mobile devices. Issue \#1762 occurs when users cannot resume their previous activities, such as viewing tickets, after successfully logging in. This problem arises because the application fails to keep track of the user's progress. To remedy this, developers need to ensure that the app retains the user's last activity state through the login process, allowing them to return to their actions seamlessly after authentication.

\subsubsection{Account Management Function Break}
We identified 48 issues as account management function breaks. These issues are failures in account management functionalities such as adding or managing multiple accounts. An example is discussed in Section~\ref{Missingloginflow}, where a user is unable to add a new account.

\subsubsection{Credential Rejection}\label{CredentialRecognitionErrors}
41 issues were reported as credential rejections where valid login credentials were entered but mistakenly denied by the apps.
% \lili{Still needs to be revised}
For example, issue~\#10170 from NextCloud~\cite{nextcloud-android} demonstrates that the login process fails when users try to login with the credentials containing whitespaces.
This was induced by the wrong encoding method of the app. 
% Therefore, the credential would be rejected by could service. 


\subsubsection{User Interface Error}
% \lili{Please be reminded to revise the categories of the symptoms. I remember that some cases are just UI issues (e.g., stuck), but we don't really have this category.}\zixu{Added}
We identified 40 issues related to User Interface (UI) errors. UI errors include interface freezes that prevent user interaction and malfunctioning UI elements, collectively blocking the login process. For example, issue \#1920 of \textit{Open Event Attendee Android}~\cite{open-event-attendee-android} exhibits UI freezing, where the app becomes non-responsive to user input following Google login initiation. Similarly, issue \#1104 of PocketHub~\cite{PocketHub} demonstrates UI element malfunction through an inactive authorization control, where the authorization button remains disabled during credential submission.

\subsubsection{Inaccurate Error Message Displayed}\label{InaccurateErrorMessageDisplayed}
39 issues were identified where users receive inaccurate or misleading error messages. These incorrect messages often do not reflect the underlying problem accurately, inducing user confusion and frustration. For example, in LibreTube~\cite{LibreTube}, an open-source video streaming app, issue \#3104 documented three scenarios displaying incorrect error messages when handling authentication errors.
Scenario (A) involved a user incorrectly clicking the login button without an account, leading to an HTTP 401 displayed. Instead, a user-friendly warning should have been displayed to explain the problem. In scenario (B), errors occur when users attempt to re-access the already deleted accounts. However, the error message indicated that this was a network error. Scenario (C) described an HTTP 401 displayed when a user attempted to re-register with the same username and password.
The core issue of all these three scenarios was the absence of appropriate error messages corresponding to underlying problems.



\begin{figure}[ht!]
  \centering
\includegraphics[width=0.95\linewidth]{figures/HMRC_SY.pdf}
  \caption{Heatmap between Root Causes and Symptoms.  Abbreviations: AMFB (Account Management Function Break), Crash (Crash), IEMD (Inaccurate Error Message Displayed), INF (Incorrect Navigation Flow), UIE(User Interface Error), CR(Credential Rejection), LDT(Login Delays or Timeout).}
  \label{fig:btRCvsSymp}
\end{figure}

To summarize, despite the fact that crashes are the most prevalent symptoms, they only account for 23.54\% of the studied issues. In other words, around 80\% of the login issues cannot be detected by simply using crashes as the test oracle.
This indicates the need for effective oracles specific to the login issues.
Our symptom categories can serve as the guideline for designing such oracles.
Figure~\ref{fig:btRCvsSymp} depicts a heatmap between root causes and symptoms analysis.
% reveals significant insights into the symptoms and their associated root causes in Android apps.
Notably, in every category of symptoms, login flow errors are always the leading cause, demonstrating their significance in ensuring the reliability of the login processes in Android apps. 
%\rufeng{Symptoms are the manifestations of issues, while root causes are the underlying problems to be addressed. I would prefer "By thoroughly understanding these symptoms, developers can more effectively identify and address the root causes..."}\zixu{Add a heatmap and description}



\begin{tcolorbox}[left=3pt,right=3pt,top=1pt,bottom=1pt]
    \textbf{Answer to RQ2:} 
\textit{
% Our analysis revealed that the most significant symptom of login issues in Android apps is Crashes, with a total of 91 issues identified.
Our analysis results indicate that login issues can exhibit a variety of symptoms.
Around 80\% of the login issues do not exhibit crashes when triggered. This highlights the importance of designing effective test oracles to expose login issues. Our categories can serve as the guidelines for designing such test oracles.
}
\end{tcolorbox}