\section{Background}
\subsection{Android Authentication Methods}
Android apps utilize various authentication methods to manage user access and ensure security. Authentication is a critical component of the login process, serving as the method by which applications verify user identity to grant access to their features and ensure security.
\lili{One small gap here is: what are the relationships between authentication and login?}\zixu{Added a sentence to explain auth belongs to login}
The most common approaches include traditional username and password login, social media integration, and biometrics such as fingerprint scanning and facial recognition~\cite{9068715}. Each method offers distinct advantages and disadvantages. For instance, username and password combinations are widely used due to their simplicity and familiarity but are vulnerable to brute force attacks and phishing. Social media logins such as Google Lgoin~\cite{google-login-in} and Facebook Login~\cite{facebook-login-in} offer convenience by allowing users to sign in with existing accounts, reducing password fatigue and streamlining the user experience. However, they can raise concerns about privacy and data security. Biometric methods provide robust security and a quick authentication process but require specialized hardware and can be affected by environmental factors or changes in the user's physical condition. To further enhance security, Multi-Factor Authentication (MFA)~\cite{10348624} is increasingly being implemented, which requires users to provide multiple verification factors, significantly reducing the risk of unauthorized access.


\subsection{States in the Login Process}
The login process in Android apps can be divided into three distinct phases: pre-login, login, and post-login. 

\textbf{Pre-login Phase:} This initial stage involves checking whether the user has previously logged in and if there is a valid token present. If no valid token is found, the user is prompted to select a login approach and enter the corresponding credentials. This stage is crucial for determining the starting point of the authentication process, ensuring that returning users can proceed faster if they have valid sessions.

\textbf{Login Phase:} During this phase, the submitted credentials are validated to verify correctness. Additionally, if Multi-Factor Authentication (MFA) is enabled, the user is required to provide an MFA code. This code is then validated to ensure an extra layer of security. This phase is critical as it directly impacts the security of the user's access and the integrity of the application.

\textbf{Post-login Phase:} After successful verification, this phase grants the user access to the app's resources. It marks the transition from authentication to actual usage of the application, where the user interacts with the app's features based on their access level and permissions. This phase is essential for a seamless transition and ensuring the user's experience is smooth and secure.

