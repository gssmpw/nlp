\subsection{Implications}

The findings from our study emphasize the importance of login flow errors in the context of login issues within Android apps. Login flow errors are the primary root causes for our studied issues, accounting for 59.28\%. %. Their prevalence not only highlights their commonality but also their potential impact on application stability and user experience. %\lili{We have firstly here. Where is secondly?}\zixu{Added Secondly and Thirdly} 
% In addition, the significance of the analysis between root causes and symptoms, root causes, and trigger conditions is also evident.
In this section, we further investigate the impact of login flow errors by correlating them with the issue symptoms and trigger conditions.
\subsubsection{Root Causes and Symptoms}
As shown in Figure~\ref{fig:btRCvsSymp}, unlike other root causes that typically correspond to specific symptoms (e.g., null object check omission always induce crashes), login flow errors can trigger every type of symptom identified in our study. This broad impact suggests that issues in login flows can manifest in various detrimental ways.
% , affecting nearly every aspect of the login process.
Here we use examples to illustrate how login flow errors induced two types of symptoms as examples:

\blacknumber{1}\textbf{Incorrect Navigation Flow:} Login flow errors induced 41 incidents of incorrect navigation flow.  The issue mentioned in Section~\ref{IncorrectNavigationFlow} reports that users are unable to resume their previous activities after logging in.  The root cause of this issue is \textit{Interference with Other State Machines}. The app did not use \texttt{savedInstanceState} to save the activity before starting to login, resulting in information loss. To fix this issue, the developers implemented a solution where the user's current activity state is saved before authentication. Upon successful login, the application restores the previously saved state, allowing users to seamlessly continue from where they left off.
% This enhancement significantly improves user experience by maintaining continuity and reducing frustration associated with repeated navigation. Thus, mismanagement of states may direct users to incorrect application screens. 

\blacknumber{2}\textbf{Inaccurate Error Messages:} 35 inaccurate error message issues were induced by login flow errors.
% \lili{This example also needs to be re-written. We should first focus on how the login flow errors induced the symptom.}\zixu{Added}
In Section~\ref{InaccurateErrorMessageDisplayed}, we discussed an issue in LibreTube where error messages do not match the underlying root causes.
The root cause of this issue is \textit{Improper Error State Handling}. 
The login flow transits to incorrect error states and the app did not give precise error messages in the the error-handling logic.
% Faulty state logic can result in the display of misleading or incorrect error messages to users. 
  
\subsubsection{Root Causes and Trigger Conditions}
As shown in Figure~\ref{fig:btTCvsSymp}, login flow errors can be triggered by all kinds of conditions.
In addition, they often require a confluence of multiple trigger conditions to manifest, suggesting their complex nature in issue triggering. Here are two examples where multiple conditions are required simultaneously  to trigger login flow issues:

\blacknumber{1}Thunderbird~\cite{thunderbird-android} is an open-source email client developed by Mozilla that supports multiple email accounts, management, and protocols, making it popular for personal and professional communication.
In email systems, the port number is critical as it dictates the communication endpoint for a server. Different email services use specific ports to handle various email protocols.
For instance, IMAP typically uses port 143 or the encrypted 993.
The root cause of issue \#5088 
% \lili{Which issue number?}\zixu{Added}
in Thunderbird~\cite{ThunderbirdIssue} is identified as \textit{Missing Login Flow}. The developers missed the process in the login flow to update the port number according to the protocol set by the new account when users switch between email accounts. 
% \lili{Why this is a login flow error? Which subcategory does it refer to?}\zixu{Added}
This problem can be triggered when three conditions are satisfied: 1) the user adds a new account (\textit{Account Management Operation}), 2) initially sets up the account using the IMAP protocol (\textit{Specific Login Approach}), and 3) inputs a specific server address (\textit{Specific input}). These steps inadvertently lead to a situation where the port number from the previous account setup persists, causing connection issues or incorrect email data retrieval. 

\blacknumber{2}AntennaPod~\cite{AntennaPod} is an open-source podcast manager that allows users to sync their podcast data across devices using various login and synchronization methods, including NextCloud~\cite{nextcloud-android}. 
Issue \#5841 has been documented within AntennaPod's integration with NextCloud~\cite{AntennaPodissue}. 
This integration requires users to provide a server URL for syncing data.
In this issue, despite users entering the correct server URL, they encounter persistent login failures. The root cause is identified as \textit{Missing Login Flow} because the developers did not design the login flow for the input URLs with different certificates.  
% \lili{Why this is a login flow error? Which subcategory does it belong to? Needs to first explain why.}\zixu{Added}
This problem is triggered when three conditions are satisfied: 1)the device must have \texttt{Gpoddersync} installed (\textit{Specific Device}), 2) NextCloud selected as the sync and login approach (\textit{Specific Login Approach}), and 3) the entered URL must support HTTPS certificates (\textit{Specific Input}). The necessity for an HTTPS-supportive URL suggests that security protocols, particularly those related to SSL/TLS certificate verification, may be obstructing the login process. This issue can lead to unsuccessful authentication and synchronization, severely affecting the user experience and functionality of the app.

In conclusion, our study results reveal the pivotal role of login flow errors in compromising the functionality and security during the login process of Android applications. %\rufeng{What does this fig bring us? I guess most readers will be interested in the correlation between these three dimensions. For example, we can see that almost all "Incorrect Navigation Flow" is caused by "State Management Errors". So can we suggest that developers focus on checking the State Management code when they find such symptoms? Similarly, we can analyze the correlation between symptoms and triggering conditions. For some symptoms, can we reproduce them by using tests to simulate the triggering conditions? }\zixu{Add symptoms and triggering conditions at the end of the trigger condition section, this implication mainly focuses on state management error is important}\zixu{I moved the figure into the beginning main purpose is showing trigger conditions of state management error is complex}\zixu{Added Incorrect Navigation Flow example }
\subsubsection{Findings}
To summarize, the login flow errors can cause a wide range of symptoms, from login delays or timeouts to system crashes, and require the intersection of multiple conditions to trigger.
This indicates that further research should focus on login flow errors when testing for login issues in Android apps. First, to tackle these issues, it is essential to understand the states and transitions involved in the login flow and develop models that accurately represent these dynamics.
% The findings from our empirical study can inform the development of tools that automatically test for login issues in Android apps.
Our proposed state machine (Figure~\ref{fig:statemachine}) is a first attempt to achieve this goal.
Second, our results assist developers in identifying the causes of problems and designing targeted test cases. For instance, if users report an Incorrect Navigation Flow issue, developers should focus on login flow errors and verify the logical sequence of state transitions.
Additionally, developers can design tests that include special character inputs for all supported login methods and incorporate device orientation changes to assess the Incorrect Navigation Flow.
Third, our taxonomy of issue symptoms can help define effective test oracles. For instance, we identified 39 issues of “Inaccurate Error Message Displayed”. This indicates that developers can design test oracles to verify whether error messages accurately correspond to actual failures like invalid credentials or expired sessions.
Such test cases not only confirms issue fidelity for users but also ensures that the application's logic proactively manages these errors, thus improving both the reliability and user experience of the apps' login process. %\lili{Again, here are a lot of vague slogans to highlight the importance of the implications. They are not convincing. Can we just say some concrete suggestions to show that our results are really useful? We can say something like: This calls for future research on addressing state management errors in Android login processes. To address these issues, we need to derive models to precisely capture the complicated state transfers in the login processes. Our empirical study results can be used to guide the design of a tool to automatically test for login issues in Android applications. For example, xxx (some implications related to the triggering condition and how they can to used to guide the test generation). xxx (another example to use some implications or observations related to the symptom to derive test oracles for login issues.)}\zixu{Add two examples}