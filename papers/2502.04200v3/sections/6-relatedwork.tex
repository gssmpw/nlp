\section{Related Works}
\subsection{Empirical Study On Other Issues in Android}
Several studies have explored the challenges and bugs in Android apps, focusing on issues related to user accessibility~\cite{9525343}, GUI~\cite{7965405}, and configuration~\cite{jha2019empirical}. Additionally, numerous investigations into Android testing have been conducted. Fabiano et al.~\cite{pecorelli2022software}, Lin et al.~\cite{10.1145/3324884.3416623}, and Xiong et al.~\cite{10.1145/3597926.3598138} provide empirical studies on functional bugs in Android apps, though their focus remains predominantly on common bugs rather than specific login-related issues which critically impact user login. Ardito et al.~\cite{9217524} present an automated test selection framework for Android apps based on APK and activity classification, while Yan et al.~\cite{10.1145/3377811.3380347} propose a multiple-entry testing framework by constructing activity-launching contexts. Furthermore, Bose et al.~\cite{10172528} propose a systematic callback exploration framework for Android testing. Despite the advancements in testing techniques, these tools often do not fully address the complex features of login processes, such as authentication and session management, which are critical areas that our study specifically targets. 

These studies, while comprehensive in their scope, often overlook the nuanced complexities of login mechanisms and their repercussions on security and usability. Our research aims to fill this gap by focusing specifically on the systematic analysis of login issues, offering new insights into their causes, symptoms, and triggers that can inform more effective testing and development practices for login processes.

%\rufeng{Consider adding a conclusive sentence of these studies and a sentence mentioning ours. Same for the next subsection} \zixu{Added into two sections}

\subsection{Studies On Android Authentication Problems}
Several studies have also focused on the authentication processes in Android apps, yet none have conducted a comprehensive and systematic investigation of the entire login process.
Wan et al.~\cite{8919059} attempt to test user interfaces (UIs) within mobile applications, including those for login screens. However, their tools are unable to adequately handle UIs that require user input, such as entering usernames and passwords, posing significant challenges in fully automating the testing process for such critical components. Song et al.~\cite{9401988} developed VPDroid, a platform designed to test the automatic login features of Android apps. While their approach effectively evaluates the security of apps that rely on device consistency checks for auto-login functionalities, it does not extend to other login scenarios such as Google login or server address authentication, which are covered in our study. Several studies focus on specific login approaches, Ma et al.~\cite{10.1145/3359789.3359828} conducted an empirical study on one-time password authentication in Android.  Jannett et al.~\cite{10628996} developed a tool called SoK focused on Single Sign-on in Android. Shi et al.~\cite{10.1145/3321705.3329801} propose MoSSOT, a black box tester for OAuth. In addition, Tamjid et al.~\cite{8952200} conducted an empirical study on the usage of OAuth APIs and their implications for mobile security, leading to the development of OAUTHLINT, a tool that employs query-driven static analysis to identify vulnerabilities in OAuth implementations on the Google Play marketplace. While these studies offer significant insights into various aspects of login and authentication mechanisms within Android apps, they typically focus on specific components or security aspects. This narrow focus often bypasses the comprehensive challenges presented by the entire login process, particularly in how different authentication methods integrate and operate under diverse conditions. Our research seeks to bridge this gap by conducting a systematic and holistic analysis of the login process across authentication scenarios, not just limited to OAuth or single sign-on systems. By doing so, we aim to provide a more complete understanding of the functionality issues that can arise during user login in Android apps.  