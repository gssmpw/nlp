\subsection{RQ3:Trigger Conditions of Login Issues}

Understanding the trigger conditions of login issues is crucial for designing effective test cases to expose login issues. In RQ3, we aim to identify the trigger conditions of login issues by analyzing the bug reports, code revisions, and discussions between users and developers within our dataset.

Unlike root causes and symptoms,a single issue can require multiple conditions to trigger.
% To address this complexity, we deconstructed the trigger conditions into clearer combinations of factors.
As a result, the categories of trigger conditions for an issue are not mutual exclusive.
This approach resulted in nine common trigger conditions, with a summary shown in Table~\ref{tab:trigger-conditions}.

\begin{table}
  \centering
  \caption{Summary of Trigger Conditions}
  \label{tab:trigger-conditions}
  \begin{tabular}{@{}c l r@{}} % c for center alignment of numbers, l for left, and r for right
    \toprule
    \textbf{No.} & \textbf{Trigger Conditions} & \textbf{\#Issues} \\ 
    \midrule
    1 & Specific Login Approach & 122 \\
    2 & Account Management Operation & 93 \\
    3 & Specific User Input & 86 \\
    4 & No Specific Trigger Conditions & 38 \\
    5 & Communication Failure with the Cloud & 37 \\
    6 & Specific Device & 30 \\
    7 & Setting Change & 20 \\
    8 & Orientation Change & 18 \\
    9 & User Re-Login & 15 \\
    
    \bottomrule
  \end{tabular}
\end{table}


\subsubsection{Specific Login Approach}
The most frequent trigger condition was associated with specific login approaches, accounting for 122 issues.
Issues in this category can only be triggered under specific login approaches (e.g., using specific third-party login services or using MFA).
For example, issue \#4393 in Tusky~\cite{tusky} described how selecting the Akkoma account type for login led to an app crash. This example highlights the necessity of comprehensive test coverage for each supported login approach to prevent functional disruptions across different authentication scenarios.

\subsubsection{Account Management Operation}
Account management operations triggered 93 issues, where account management actions such as account creation, deletion, and account switch triggered problems. For example, issue \#6481 from NextCloud~\cite{nextcloud-android} is triggered when users try to add a new account to the app. This suggests the importance of covering the different account management operations in testing login issues.

\subsubsection{Specific User Input}
86 issues required specific inputs to trigger. These issues often arose from scenarios where special characters or unexpected data types in login forms. The issue we discussed in Section~\ref{CredentialRecognitionErrors} is an example. The developers may not anticipate all the variants of the user input, and such issues are triggered. Consequently, test cases including special  characters or data types are necessary for login issue testing.
% This is because users may not always adhere to the developers' guidelines. 

\subsubsection{No Specific Trigger Conditions}
Finally, there are 38 issues that do not require any specific trigger conditions.
These issues can be triggered by any user login attempts.

\subsubsection{Communication Failure with the Cloud}
Communication failures with cloud services triggered 37 issues.
These issues often occur during synchronization processes or when fetching authentication tokens. This trigger condition indicates the dependency on external cloud infrastructures and the need for effective handling of network-related uncertainties.
% For example, run the same test cases under different network qualities.

\subsubsection{Specific Device}
Device-specific conditions accounted for 30 issues involving various hardware configurations or operating system versions. These issues are essentially compatibility issues occurring in the login process~\cite{chen2024demystifying,7582761}. This highlights the challenges in developing robust applications across diverse hardware and software ecosystems. To avoid issues triggered by device-specific conditions, developers must execute all test cases on a range of devices with different operating system versions before releasing updates.

\subsubsection{Setting Changes}
20 issues were triggered by changes in the app settings, including privacy setting updates or security configurations that inadvertently impacted the login functionality.
% Understanding this trigger can help developers design more robust test cases that cover different settings.
This calls for testing the login functionalities under different settings.

\subsubsection{Orientation Change}
18 issues required device orientation changes to trigger, illustrating problems that occur when apps fail to handle changes in device orientation (i.e., changes in Activity lifecycles) during login processes. This condition highlights the need for responsive design strategies that ensure seamless user experiences regardless of device orientation.
It also calls for the developers to properly handle the interference between the login flow and the Activity lifecycles.
% \lili{I think this should also be related the activity lifecycles. I revised accordingly, please check if it is correct.}\zixu{Agree}

\subsubsection{User Re-Login}
15 issues were triggered only when users make a second attempt to log into the app.
This includes situations where users aim to change the account or add a new account.
% typically causing user frustration and potential risks due to repeated login attempts.
This trigger condition points to the necessity of designing test cases to re-login in a short period.


% This category, activated by any attempt to log in, indicates that all user login attempts will fail, significantly impacting the user experience. Therefore, this category is critical as it highlights fundamental vulnerabilities exploitable during routine login attempts. This trigger condition underscores the necessity for thorough code reviews and comprehensive testing of test cases before implementing any new login features.

\begin{figure}[ht!]
  \centering
  \includegraphics[width=0.9\linewidth]{figures/HMTC_SY.pdf}
  \caption{Heatmap between Trigger Conditions and Symptoms. Abbreviations: AMFB (Account Management Function Break),Crash (Crash), IEMD (Inaccurate Error Message Displayed), INF (Incorrect Navigation Flow), UIE(User Interface Error), CR(Credential Rejection), LDT(Login Delays or Timeout).}
  \label{fig:btTCvsSymp}
\end{figure}

% Each identified trigger condition not only highlights specific technical and user-experience flaws but also illustrates the complexity of login functionalities in Android apps. By addressing these conditions and crafting targeted test cases, developers can enhance the stability of the login process and safeguard the user experience.

Our results show that the majority (89.7\%) of the login issues require complicated conditions to trigger.
This highlights the complexity in testing login issues.
Furthermore, the heatmap presented in Figure~\ref{fig:btTCvsSymp} correlates trigger conditions with symptoms, guiding developers in designing effective test cases. For instance, the heatmap suggests that login delays or timeouts can be tested through a combination of four main conditions: specific login approaches, specific input, specific device and communicating with could failures. Developers are advised to create test cases that incorporate special character inputs across all designated login approaches and execute these under diverse network scenarios to simulate and address potential failures.
%\rufeng{One implication for me is "By understanding these trigger conditions, developers can design more effective tests which simulate the scenarios." Adding such concrete implications would make the analysis more actionable, showing how insights from trigger conditions can directly inform testing and improve login security and usability.}\zixu{Add last sentences for each category to say design test cases,add a heatmap and corresponding description in previous paragraph}
\begin{tcolorbox}[left=3pt,right=3pt,top=1pt,bottom=1pt]
    \textbf{Answer to RQ3:} \textit{The majority (89.7\%) of the studied login issues require complicated conditions to trigger, highlighting the difficulties in triggering and testing login issues in Android apps. Our analysis identified nine common trigger conditions. The predominant conditions include specific login approaches, account management operations and specific inputs. Our summarized trigger conditions can guide developers to better test their app login processes.
}
\end{tcolorbox}

