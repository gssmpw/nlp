\section{Personalized Story Generation}

Stage 2 of Figure~\ref{fig:method} illustrates our proposed method for personalized story generation using the Author Writing Sheet. 

To generate personalized stories, we prompt an LLM (\(\text{LLM\textsubscript{story}}\)) with the writing prompt, story length \citep{chakrabarty2024art}, source-specific metadata (e.g., fanfiction for AO3), story rules as actionable instructions in direct second-person form (categorized into four narrative categories), and persona description obtained from the Author Writing Sheet. 
%
We experiment with four types of methods for personalized story generation: 
1) A non-personalized baseline, \emph{Average Author}; 
2) Personalization baselines, \emph{RAG} and \emph{Delta}, that do not use the Author Writing Sheet; 
3) Our proposed methods, \emph{Writing Sheet} and \emph{Writing Summary}; 
and 
4) An \emph{Oracle} method that loosely resembles the upper bound performance. We detail each method type below.
See Appendix~\ref{app:story-gen} for the prompts. 


\paragraph{Non-Personalized - Average Author:}
%\label{sec:avg-author}
\hypertarget{sec:avg-author}{}
The \emph{Average Author} method serves as a non-personalized baseline that reflects the average behavior of authors, acquired during an LLM's pre-training process. Specifically, for each given source, we prompt an LLM to generate stories using an Average Author prompt that describes typical writing characteristics for the given story source obtained using audited GPT-4o prompting\footnote{We ask GPT-4o for the typical writing characteristics for the story source and manually verify it.} \citep{wang-etal-2024-rolellm}. For example, for AO3, the Average Author is defined as ``respecting fandom tone and style while experimenting with established tropes, unconventional pairings, or alternative universes.'' 

\paragraph{Personalization Baselines}

\paragraph{RAG:}
\hypertarget{sec:rag}{}
The Retrieval-Augmented Generation, \emph{RAG} baseline \citep{salemi-etal-2024-lamp} first retrieves the most similar writing prompt and author-written story from the profiling set using BM25 \citep{robertson2009probabilistic}. We then use the retrieved pair as a one-shot demonstration \citep{wang-etal-2024-rolellm} to elicit role-playing behavior from the LLM, mimicking the retrieved example's style.

\paragraph{Delta:}
\hypertarget{sec:delta}{}
The \emph{Delta} method generates personalized story rules by contrasting the \emph{Average Author} story (see \hyperlink{sec:avg-author}{Average Author}) with the corresponding author-written story for each writing prompt in the profiling set \citep{shashidhar-etal-2024-unsupervised}. We then use \emph{all} writing prompts in the profiling set, along with their corresponding generated story rules that are actionable instructions in direct second-person form, as few-shot demonstrations for the LLM, to generate personalized story rules for a new prompt in the generation set. 


\paragraph{Proposed Method - Writing Sheet and Summary}

\paragraph{Writing Sheet:}
\hypertarget{sec:writing-sheet}{}
Our \emph{Writing Sheet} method uses the Author Writing Sheet (Section~\ref{sec:author-writing-sheet}) to enable personalization (Stage 2 of Figure~\ref{fig:method}). First, we prompt an LLM (\(\text{LLM\textsubscript{persona}}\)) to generate a persona description that summarizes the author's story-writing style as a second-person narrative, included in the system prompt \citep{wang-etal-2024-rolellm, jiang2024evaluating}. Second, we prompt an LLM (\(\text{LLM\textsubscript{rule}}\)) to generate personalized story rules from the Author Writing Sheet tailored to the writing prompt in the generation set, included as constraints in the user prompt \citep{pham-etal-2024-suri}. Additionally, we include a one-shot demonstration, following the approach described in \hyperlink{sec:rag}{RAG} \citep{richardson2023integrating}. 

\paragraph{Writing Summary:}
\hypertarget{sec:writing-summary}{}
As an alternative to the Author Writing Sheet, the \emph{Writing Summary} method leverages the LLM's long-context capabilities \citep{ding2023longnet} by packing all past stories from the profiling set as input. Specifically, we include them in a single input prompt to generate an \emph{Author Writing Summary} in the same format as the Author Writing Sheet. Similar to the Writing Sheet method, persona descriptions, and story rules are derived from the Author Writing Summary and used as constraints for story generation, along with a one-shot demonstration. 
\paragraph{Oracle:}
\hypertarget{sec:oracle}{}

The \emph{Oracle} method establishes an upper bound on personalization performance, in terms of similarity to the ground truth, by using Oracle story rules derived directly from the ground-truth author-written story for each writing prompt in the generation set. These rules are obtained by contrasting the \hyperlink{sec:avg-author}{Average Author} with the ground-truth author story, following the same process as in \hyperlink{sec:delta}{Delta}. Additionally, we include a one-shot demonstration, following the approach described in \hyperlink{sec:rag}{RAG}.