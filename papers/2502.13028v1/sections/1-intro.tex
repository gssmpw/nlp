\section{Introduction}

\begin{figure*}[htbp]
\centering
\includegraphics[width=\linewidth]{figures/method.pdf}
\caption{Our two-stage pipeline for personalized story generation. Stage 1 constructs an Author Writing Sheet with Claim ($C$) and Evidence ($E$) pairs capturing the author’s story-writing characteristics across narrative categories. It is derived from the author’s history of writing prompts ($wp$), author-written stories ($s_a$), and LLM-generated Average Stories ($s_b$) representing a typical author's response to the same prompt. Stage 2 uses the Author Writing Sheet to role-play the author, incorporating tailored story rules and a persona description for personalized generation.} 
\label{fig:method}
\end{figure*}


Personalizing the outputs of large language models (LLMs) has promising applications for role-playing in chatbots and video games \citep{tseng-etal-2024-two, chen2024persona}, as well as providing personalized feedback in education \citep{stahl-etal-2024-exploring, li-etal-2024-using, makridis2024fairylandai}. However, story-writing personalization, a crucial writing system feature that can assist users in overcoming writer’s block or help second-language learners adapt to a new language \citep{yuan2022wordcraft, yeh2024ghostwriter}, remains largely unexplored. 

To address this gap, we systematically explore how to personalize LLM outputs based on individual authors' previous writings. We propose a novel two-stage pipeline for personalized story generation, as shown in Figure~\ref{fig:method}. In the first stage, we construct an \textit{Author Writing Sheet} to capture implicit writing characteristics from an author's past writings \citep{sun-etal-2025-persona, sun-etal-2024-revealing, ramos-etal-2024-transparent}. We categorize story-writing characteristics into four fine-grained narrative dimensions: Plot, Creativity, Development, and Language Use \citep{pavis1998dictionary, card1999characters, noble1994conflict, huot2024agents}. Each characteristic is represented as a Claim-Evidence pair, where general statements about the author's writing behavior are supported by specific story excerpts. This approach, inspired by Common Core Standards in English Language Arts \citep{national2010common} and recent work in writing education \citep{li2023teach}, provides a structured and grounded way to identify an author's distinctive writing patterns, as validated via our human evaluation (Section \ref{sec:author-writing-sheet}). In the second stage, we implement \textit{Personalized Story Generation} by having LLMs role-play authors based on their Writing Sheet. Our method uses persona prompts and story rules as user constraints to guide the generation. 

To enable and validate our approach, we construct \dataname, a dataset of 590 stories from 64 authors ($\approx$1500 tokens) across five sources with distinct writing settings: Reddit, AO3, Storium, Narrative Magazine, and New Yorker. The dataset splits each author's works chronologically into two sets: an earlier \emph{Profiling} set used to infer their writing characteristics, and a later \emph{Generation} set used to evaluate how well LLMs can emulate their distinctive writing behavior \citep{salemi-etal-2024-lamp}.

We evaluate personalized stories in two ways: First, we use automatic evaluation with LLM-as-a-judge, to assess faithfulness to the author's writing history and similarity to ground truth author writings, along with traditional metrics for lexical overlap, story diversity, and style similarity. Second, we conduct a human evaluation to assess the generated story's similarity to the author's ground-truth writings. 
We find that stories generated using Author Writing Sheets demonstrate clear advantages over a non-personalized baseline approach. When evaluated by LLMs, personalized stories have 61\% higher faithfulness to the author writing history. Human annotators prefer these stories by an 18\% absolute margin in terms of similarity to ground truth. Our personalization approach is especially effective for Reddit content and for capturing Creativity, and Language Use characteristics, with less pronounced impact on Plot \citep{tian-etal-2024-large-language, xu2024echoes}. In summary, our contributions are:
\begin{itemize}[noitemsep, topsep=0pt]
    \item \textbf{\dataname}: A novel dataset containing multiple stories written by the same authors, designed to support both author profiling and personalized story generation evaluation. 
    \item \textbf{Author Writing Sheet}: A structured approach to capturing an author’s writing behavior based on past works, grounded in narrative theory and educational practices and validated through human evaluation.
    \item \textbf{Personalized Story Generation}: A method for generating stories in an author's style using Author Writing Sheets, rigorously evaluated through automated and human assessments, demonstrating its superiority over baselines.
\end{itemize}

