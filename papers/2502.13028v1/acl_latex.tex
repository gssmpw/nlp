% This must be in the first 5 lines to tell arXiv to use pdfLaTeX, which is strongly recommended.
\pdfoutput=1
% In particular, the hyperref package requires pdfLaTeX in order to break URLs across lines.

\documentclass[11pt]{article}

% Change "review" to "final" to generate the final (sometimes called camera-ready) version.
% Change to "preprint" to generate a non-anonymous version with page numbers.
\usepackage[final]{acl}

% Standard package includes
\usepackage{times}
\usepackage{latexsym}

% For proper rendering and hyphenation of words containing Latin characters (including in bib files)
\usepackage[T1]{fontenc}
% For Vietnamese characters
% \usepackage[T5]{fontenc}
% See https://www.latex-project.org/help/documentation/encguide.pdf for other character sets

% This assumes your files are encoded as UTF8
\usepackage[utf8]{inputenc}

% This is not strictly necessary, and may be commented out,
% but it will improve the layout of the manuscript,
% and will typically save some space.
\usepackage{microtype}

% This is also not strictly necessary, and may be commented out.
% However, it will improve the aesthetics of text in
% the typewriter font.
\usepackage{inconsolata}

%Including images in your LaTeX document requires adding
%additional package(s)
\usepackage{graphicx}
\usepackage{booktabs} 
\usepackage{array}
\usepackage{pifont}
\usepackage{amssymb}
\usepackage[linesnumbered,ruled,vlined]{algorithm2e}
\usepackage{algpseudocode}
\usepackage{amsmath}
\usepackage{verbatim}
\usepackage{enumitem}
\usepackage{tcolorbox}
\usepackage{titlesec}
\usepackage{hyperref}
\usepackage{float}
\usepackage{tabularx}
\usepackage{subcaption}
\usepackage{soul}
\usepackage{longtable} % Allows tables to span multiple pages
\usepackage{booktabs}  % Professional-looking tables



% If the title and author information does not fit in the area allocated, uncomment the following
%
%\setlength\titlebox{<dim>}
%
% and set <dim> to something 5cm or larger.

\title{\includegraphics[width=0.6cm]{figures/book.png} \textit{Whose story is it?} \\Personalizing story generation by inferring author styles}

% Author information can be set in various styles:
% For several authors from the same institution:
% \author{Author 1 \and ... \and Author n \\
%         Address line \\ ... \\ Address line}
% if the names do not fit well on one line use
%         Author 1 \\ {\bf Author 2} \\ ... \\ {\bf Author n} \\
% For authors from different institutions:
% \author{Author 1 \\ Address line \\  ... \\ Address line
%         \And  ... \And
%         Author n \\ Address line \\ ... \\ Address line}
% To start a separate ``row'' of authors use \AND, as in
% \author{Author 1 \\ Address line \\  ... \\ Address line
%         \AND
%         Author 2 \\ Address line \\ ... \\ Address line \And
%         Author 3 \\ Address line \\ ... \\ Address line}

% \author{First Author \\
%   Affiliation / Address line 1 \\
%   Affiliation / Address line 2 \\
%   Affiliation / Address line 3 \\
%   \texttt{email@domain} \\\And
%   Second Author \\
%   Affiliation / Address line 1 \\
%   Affiliation / Address line 2 \\
%   Affiliation / Address line 3 \\
%   \texttt{email@domain} \\}

\author{
    Nischal Ashok Kumar$^{1}$ \quad
    Chau Minh Pham$^{2}$ \quad
    Mohit Iyyer$^{1,2}$ \quad
    Andrew Lan$^{1}$ \\
    $^{1}$University of Massachusetts Amherst \quad
    $^{2}$University of Maryland, College Park \\
    \texttt{\{nashokkumar, andrewlan\}@cs.umass.edu}, 
    \texttt{\{chau, miyyer\}@umd.edu}
}



\newcommand{\dataname}{\emph{\textbf{Mythos}}}
%\setcounter{secnumdepth}{4}

\begin{document}
\maketitle

\begin{abstract}
Out-of-distribution (OOD) detection and OOD generalization are widely studied in Deep Neural Networks (DNNs), yet their relationship remains poorly understood. We empirically show that the degree of Neural Collapse (NC) in a network layer is inversely related with these objectives: stronger NC improves OOD detection but degrades generalization, while weaker NC enhances generalization at the cost of detection. This trade-off suggests that a single feature space cannot simultaneously achieve both tasks. To address this, we develop a theoretical framework linking NC to OOD detection and generalization. We show that entropy regularization mitigates NC to improve generalization, while a fixed Simplex Equiangular Tight Frame (ETF) projector enforces NC for better detection. Based on these insights, we propose a method to control NC at different DNN layers. In experiments, our method excels at both tasks across OOD datasets and DNN architectures. 

\begin{comment}   

Out-of-distribution (OOD) detection and OOD generalization are critical for deploying machine learning models in real-world scenarios. While substantial progress has been made in addressing these problems independently, few works have attempted to tackle them jointly. However, existing methods often rely on auxiliary OOD training data and primarily focus on covariate-shifted OOD data that share labels with in-distribution (ID) data. In contrast, we tackle the more realistic and challenging task of jointly detecting and generalizing to semantic OOD data with disjoint labels from the ID data, without auxiliary OOD training data.
Achieving both objectives simultaneously is inherently difficult due to a fundamental conflict — OOD generalization requires enhanced transferability, while OOD detection necessitates the inhibition of transfer.
To address this, we leverage insights from neural collapse (NC) — a phenomenon in deep networks where top-layer representations suppress feature variability and adopt a Simplex Equiangular Tight Frame (ETF) structure, impairing transferability. By controlling NC, we unify OOD detection and generalization: preventing NC enhances OOD transfer while inducing NC improves OOD detection.
Our proposed method excels at both tasks across various OOD datasets and architectures. 

\end{comment}


\end{abstract} 


\section{Introduction}
\label{sec:intro}

Foundational models (FMs)~\cite{zhang2024data, zhou2023comprehensive} have shown remarkable progress in the healthcare domain, enabling professional-like assessment of disease diagnosis, treatment decision-making, and monitoring~\cite{zhang2023text, wang2022medclip, lu2023mi-zero}. 
Examples include LLaVA-Med~\cite{li2023llava}, Med-PaLM Multimodal~\cite{tu2024towards}, and Med-Flamingo~\cite{moor2023med}, have demonstrated their capacity on question answering, medical image analysis, and report generation.
These studies follow a predominant top-down model development strategy that requires upstream developers to collect data and train models for downstream tasks. 
Consequently, the developed model capabilities are heavily dependent on the training data, limiting their generalization performance in diverse clinical scenarios. 
For instance, Med-Gemini~\cite{yang2024advancing} reveals promising general capabilities in report generation while it lags behind state-of-the-art (SoTA) models on classification tasks, especially for out-of-domain applications. 
This indicates that while the generalizability of the foundation model is promising, more solutions are expected to meet the various specialized clinical needs.

To address these challenges, multi-center data centralization becomes essential to enhance model capacity and robustness across varied clinical scenarios~\cite{rajpurkar2022ai}. 
Centralizing distributed data can significantly improve model training and inference performance.
However, the process of medical data storage, transfer, and aggregation among centers requires extra efforts to ensure data security and system interoperability~\cite{bradford2020international}.
Moreover, a growing concern for patient privacy makes large-scale multi-center data sharing particularly challenging. 
While efforts like federated learning~\cite{wen2023survey, li2020review} can achieve good model performance on local data, the need for synchronized system coordination presents significant challenges, as clients are unable to update asynchronously. This limitation greatly restricts the practical capability of such approaches.
As a result, without a flexible collaboration, medical community still struggles to fully utilize the isolated data and local computation resources for comprehensive medical AI model development. 
To address this dilemma, open-source platforms encourage public data sharing and knowledge integration~\cite{markiewicz2021openneuro, zenodo}.
However, these platforms focus solely on raw data sharing while seldom providing collaborative model training or cooperation between different institutions.
Recently, collaborative learning has emerged as a viable approach for enhancing multi-model robustness~\cite{boulemtafes2020review}. 
For instance, software-like model development~\cite{raffel2023building} mimics software engineering practices by introducing structured workflows, enabling merging, version control, and continuous model integration.
Under this design, model ability can be strengthened with incremental knowledge updates similar to the version updating in software development. 

Although collaborative learning provides a multi-model collaboration, two key challenges remain in the leakage of raw data during collaboration~\cite{huang2023lorahub} and the synchronization of multiple collaborators~\cite{mcmahan2017communication} in the medical AI community. It is still challenging to integrate decentralized, privacy-sensitive data across institutions, leading to under-utilized insights and fragmented knowledge sharing~\cite{kaissis2020secure, rajpurkar2022ai, abdullah2021ethics}.
 To address these challenges, inspired by the collaborative software development, we propose \textbf{Med}ical \textbf{Fo}undation Models Me\textbf{rg}ing (\textbf{MedForge}), a cooperative workflow enabling continuously community-driven foundation model (FM) development.
MedForge enables a lightweight manner for individual centers to share their knowledge among multiple centers, minimizing the burden of data transmission and integration while enhancing model robustness.
Meanwhile, MedForge facilitates asynchronous and flexible collaboration, allowing individual centers to continuously update and improve medical FMs without the need for real-time synchronization.
Similar to open-source software development, MedForge incrementally updates medical knowledge and follows a sustainable model development scheme. 
This key design emphasizes a bottom-up construction of a multi-task medical FM, allowing downstream users to collaboratively build, refine, and update the upstream model according to their local resources. Our major contributions of MedForge are as below: 
\begin{enumerate}
    \item[$\bullet$] We introduce a collaborative workflow to promote the merging scheme of open-source software development. Our proposed MedForge allows distributed clinical centers to asynchronously contribute to comprehensive medical model construction while reducing transmitting costs among centers and avoiding the leakage of raw data, thus enhancing the utilization of private resources in the healthcare system. 
    \item[$\bullet$] We propose two effective knowledge-merging strategies for the asynchronous branch contribution. The MedForge-Fusion strategy updates the plugin module parameters of the main model during the merging phase, whereas the MedForge-Mixture strategy integrates the output of the plugin module by memorizing each contributor's coefficient. These strategies make MedForge more flexible and versatile. MedForge-Fusion is friendly to implement, while the MedForge-Mixture offers better performance and robustness.
    \item[$\bullet$]  We comprehensively evaluate model merging strategies to accumulate medical knowledge among multiple branch plugin modules. MedForge yields superior performance on medical classification tasks compared to other collaborative baselines across multiple datasets. We demonstrate the robustness of MedForge by shuffling the task order and evaluating various configurations of plugin modules and dataset distillation methods.
\end{enumerate}


    %intro (change to intro for the original version)

\section{Dataset Collection and Description}

To facilitate our analysis of \emph{personalized} story generation, we ensemble \dataname, a novel dataset consisting of 590 stories written by 64 authors. To the best of our knowledge, \dataname~is the first dataset that identifies and connects multiple stories written by the same author. See Table~\ref{tab:compare-datasets} for comparison with existing datasets. 

\paragraph{Data Sources:} We include five diverse story-writing sources, namely
\textbf{(1) Reddit}, %\footnote{\url{https://www.reddit.com/r/WritingPrompts/}} 
featuring stories from r/WritingPrompts, a widely-used resource for amateur story-writing research \citep{fan-etal-2018-hierarchical},   
\textbf{(2) AO3}, %\footnote{\url{https://archiveofourown.org/}} 
featuring fanfiction based on popular franchises such as \textit{Harry Potter} and \textit{Sherlock Holmes},   \textbf{(3) Storium}, %\footnote{\url{https://storium.cs.umass.edu/}} 
featuring collaborative stories \citep{akoury-etal-2020-storium}, 
\textbf{(4) N.Magazine}, %\footnote{\url{https://www.narrativemagazine.com/}} 
featuring professionally polished stories by authors such as Barry Gifford, and finally
\textbf{(5) New Yorker},  %\footnote{\url{https://www.newyorker.com/}} 
featuring expert-level storytelling from renowned authors such as Haruki Murakami.


\paragraph{Preprocessing:} To ensure high-quality and diverse content, we impose constraints on word length (ranging from 500 to 1500 words), limit publication dates to post-November 2023, and exclude explicit material. To standardize formatting, non-Reddit stories are supplemented with writing prompts generated by GPT-4o and manually reviewed for accuracy. Stories are split chronologically based on their submission timestamps, with 70\% used for \emph{profiling} writing characteristics and 30\% for \emph{generation} (see  Appendix~\ref{app:dataset} for details).

\paragraph{Dataset size:} Our dataset is limited  in size due to data crawling policies that restrict large-scale data collection for analysis\footnote{\url{https://archiveofourown.org/admin_posts/25888}}. We note that recent storywriting datasets such as TELL ME A STORY \citep{huot2024agents} and TTCW \citep{chakrabarty2024art} have conducted analyses on a comparable number of stories (Table~\ref{tab:compare-datasets}).



\begin{table*}[htbp]
\centering
\caption{Comparison of our dataset with existing story-writing datasets. \ding{55} indicates no Author IDs associated with stories, and $\sim$ denotes having Author IDs but not having explicit links between them. Our dataset spans diverse settings and links stories by the same author, enabling research on personalized story generation.}
\label{tab:compare-datasets}
\small
\begin{tabular}{m{5cm}|m{1cm}m{3.75cm}m{1.25cm}m{1.25cm}m{1cm}}
\toprule
\textbf{Dataset Name} & \textbf{Size} & \textbf{Sources} & \textbf{Prompt Length} & \textbf{Story Length} & \textbf{Author IDs} \\
\midrule
WritingPrompts \citep{fan-etal-2018-hierarchical}       & 300K         & Reddit & 28             & 735              & \ding{55} \\ 
%ROCStory \citep{mostafazadeh-etal-2016-corpus}            &   50K      & Crowd Workers        & 9              & 41             & \ding{55}      \\ 
TELL ME A STORY \citep{huot2024agents}      & 230         & Writing Workshop      & 113             & 1498            & \ding{55}  \\ 
MirrorStories \citep{yunusov-etal-2024-mirrorstories}      & 1500         & Aesop’s fables      & 40             & 400            & \ding{55}  \\ 
%STORYSUMM \citep{subbiah-etal-2024-storysumm}      & 96         & Reddit      & -             & 964            & \ding{55}  \\ 
Storium \citep{akoury-etal-2020-storium}      & 440k         & Storium online platform      & 247             & 247            & $\sim$  \\ 
TTCW \citep{chakrabarty2024art}      & 12         & New Yorker      & 54             & 2000            & $\sim$  \\ 
\midrule
\dataname  & 590   & Reddit, AO3, Storium, N.Magazine, New Yorker           & 50           & 1517          & \checkmark \\ 
\bottomrule
\end{tabular}
\end{table*}
    
 
\section{Author Writing Sheet}
\label{sec:author-writing-sheet}

Motivated by the connection between writing education and personalization \citep{li2023teach}, we propose a novel method to infer implicit story-writing characteristics of authors based on their history in the profiling set $P$. We group these characteristics into four narrative categories: Plot, Creativity, Development, and Language Use, inspired by narrative theory \citep{pavis1998dictionary, card1999characters, noble1994conflict, huot2024agents}, to assess fine-grained, multi-dimensional story-writing attributes:
\begin{itemize}[noitemsep, topsep=0pt]
    \item \textbf{Plot}: Story structure, conflict introduction, prompt engagement, and resolution.
    \item \textbf{Creativity}: Genre blending, unconventional prompt reinterpretation, and unique elements.
    \item \textbf{Development}: Character depth, emotional arcs, and immersive settings.
    \item \textbf{Language Use}: Diction, style, rhetorical devices, pacing, and dialogue.
\end{itemize}


Inspired by Common Core (CC) Standards in English Language and Arts \citep{national2010common}, each characteristic is represented as a Claim-Evidence pair, where the Claim summarizes the author's writing style, and Evidence consists of excerpts from the author-written story to justify the Claim. These characteristics are organized into an \emph{Author Writing Sheet} \citep{gurung-lapata-2024-chiron}.

Stage 1 in Figure~\ref{fig:method} (and Algorithm~\ref{alg:author_writing_sheet} in the Appendix) illustrates our method for generating the Author Writing Sheet by iteratively processing each story in an author's profiling set. For a given writing prompt ($wp_t$) and corresponding author-written story ($s_{at}$), we use an LLM (\(\text{LLM\textsubscript{avg}}\)) to generate an \emph{average story} ($s_{bt}$). Since LLMs are pre-trained on large-scale text data from the web, this average story simulates a typical author's response based on source-specific stylistic patterns 
(see \hyperlink{sec:avg-author}{Average Author}). We prompt an LLM (\(\text{LLM\textsubscript{sheet}}\)) to compare the average and author-written stories and generate Claim-Evidence pairs for each narrative category to form an intermediate Author Writing Sheet $A_t'$, which captures how the author's writing characteristics deviate from those of a typical author for that source \citep{shashidhar-etal-2024-unsupervised, krishna-etal-2020-reformulating}.


To construct the final Author Writing Sheet $A_{|P|}$, we iteratively merge each intermediate Author Writing Sheet $A_t'$ with the aggregated sheet from prior stories $A_{t-1}$, maintaining a moving update similar to \citep{chang2023booookscore}. During merging, we prompt an LLM (\(\text{LLM\textsubscript{combine}}\)) to integrate $A_t'$ into $A_{t-1}$ by grouping equivalent Claims within each narrative category, selecting the best Evidence for grouped Claims, and retaining ungrouped Claims with their respective Evidence. To improve clarity, the updated sheet includes a timestamp identifier for each Evidence corresponding to the story it belongs to and also limits each narrative category to 10 Claim-Evidence pairs, prioritizing grouped Claims. For \(\text{LLM\textsubscript{avg}}\), \(\text{LLM\textsubscript{sheet}}\), and \(\text{LLM\textsubscript{combine}}\) we use GPT-4o prompted in a chain-of-thought manner with a temperature of 0.0 and a maximum output of 4096 tokens \citep{wei2022chain, shashidhar-etal-2024-unsupervised}.

Our method processes each story separately, making the update process efficient when new submissions are added \citep{yeh2024ghostwriter, yuan2022wordcraft}. This method avoids the need to reprocess all previous stories, reducing computational cost and mitigating the limitations of long-context inputs in LLMs \citep{zhou2023don, magar-schwartz-2022-data, li2024long}. This approach draws parallels to \textit{Knowledge Tracing} in education research, where a model continuously updates its understanding of a student's progress without re-evaluating past interactions \citep{liu-etal-2022-open, scarlatos2024exploring}. See Appendix \ref{app:cost-analysis} for cost analysis.



\paragraph{Validating Author Writing Sheets:}

We conduct a human evaluation with three Upwork annotators (\$17/hour). Annotators assess the Claim-Evidence pairs based on two criteria: (1) whether the Claims can be reasonably inferred from the story (Yes/No) and (2) whether the Evidence supports the Claims (Yes/Partially/No). The study includes 42 authors spanning all five sources, each represented by one sampled story resulting in 188 annotated claims. Of these, 12 stories are annotated by all annotators to measure inter-rater agreement, while each annotator evaluates 10 additional exclusive stories. Results show unanimous agreement on the `Yes' label for Claim inference and moderate agreement on Evidence support (Krippendorff’s Alpha$=.57$), with 93\% of Evidence fully supporting Claims and 7\% providing partial support. These findings confirm the reliability and quality of the Author Writing Sheets and support their use in personalized story generation (see Appendix~\ref{app:human-author-sheets} for more details).


\section{Personalized Story Generation}

Stage 2 of Figure~\ref{fig:method} illustrates our proposed method for personalized story generation using the Author Writing Sheet. 

To generate personalized stories, we prompt an LLM (\(\text{LLM\textsubscript{story}}\)) with the writing prompt, story length \citep{chakrabarty2024art}, source-specific metadata (e.g., fanfiction for AO3), story rules as actionable instructions in direct second-person form (categorized into four narrative categories), and persona description obtained from the Author Writing Sheet. 
%
We experiment with four types of methods for personalized story generation: 
1) A non-personalized baseline, \emph{Average Author}; 
2) Personalization baselines, \emph{RAG} and \emph{Delta}, that do not use the Author Writing Sheet; 
3) Our proposed methods, \emph{Writing Sheet} and \emph{Writing Summary}; 
and 
4) An \emph{Oracle} method that loosely resembles the upper bound performance. We detail each method type below.
See Appendix~\ref{app:story-gen} for the prompts. 


\paragraph{Non-Personalized - Average Author:}
%\label{sec:avg-author}
\hypertarget{sec:avg-author}{}
The \emph{Average Author} method serves as a non-personalized baseline that reflects the average behavior of authors, acquired during an LLM's pre-training process. Specifically, for each given source, we prompt an LLM to generate stories using an Average Author prompt that describes typical writing characteristics for the given story source obtained using audited GPT-4o prompting\footnote{We ask GPT-4o for the typical writing characteristics for the story source and manually verify it.} \citep{wang-etal-2024-rolellm}. For example, for AO3, the Average Author is defined as ``respecting fandom tone and style while experimenting with established tropes, unconventional pairings, or alternative universes.'' 

\paragraph{Personalization Baselines}

\paragraph{RAG:}
\hypertarget{sec:rag}{}
The Retrieval-Augmented Generation, \emph{RAG} baseline \citep{salemi-etal-2024-lamp} first retrieves the most similar writing prompt and author-written story from the profiling set using BM25 \citep{robertson2009probabilistic}. We then use the retrieved pair as a one-shot demonstration \citep{wang-etal-2024-rolellm} to elicit role-playing behavior from the LLM, mimicking the retrieved example's style.

\paragraph{Delta:}
\hypertarget{sec:delta}{}
The \emph{Delta} method generates personalized story rules by contrasting the \emph{Average Author} story (see \hyperlink{sec:avg-author}{Average Author}) with the corresponding author-written story for each writing prompt in the profiling set \citep{shashidhar-etal-2024-unsupervised}. We then use \emph{all} writing prompts in the profiling set, along with their corresponding generated story rules that are actionable instructions in direct second-person form, as few-shot demonstrations for the LLM, to generate personalized story rules for a new prompt in the generation set. 


\paragraph{Proposed Method - Writing Sheet and Summary}

\paragraph{Writing Sheet:}
\hypertarget{sec:writing-sheet}{}
Our \emph{Writing Sheet} method uses the Author Writing Sheet (Section~\ref{sec:author-writing-sheet}) to enable personalization (Stage 2 of Figure~\ref{fig:method}). First, we prompt an LLM (\(\text{LLM\textsubscript{persona}}\)) to generate a persona description that summarizes the author's story-writing style as a second-person narrative, included in the system prompt \citep{wang-etal-2024-rolellm, jiang2024evaluating}. Second, we prompt an LLM (\(\text{LLM\textsubscript{rule}}\)) to generate personalized story rules from the Author Writing Sheet tailored to the writing prompt in the generation set, included as constraints in the user prompt \citep{pham-etal-2024-suri}. Additionally, we include a one-shot demonstration, following the approach described in \hyperlink{sec:rag}{RAG} \citep{richardson2023integrating}. 

\paragraph{Writing Summary:}
\hypertarget{sec:writing-summary}{}
As an alternative to the Author Writing Sheet, the \emph{Writing Summary} method leverages the LLM's long-context capabilities \citep{ding2023longnet} by packing all past stories from the profiling set as input. Specifically, we include them in a single input prompt to generate an \emph{Author Writing Summary} in the same format as the Author Writing Sheet. Similar to the Writing Sheet method, persona descriptions, and story rules are derived from the Author Writing Summary and used as constraints for story generation, along with a one-shot demonstration. 
\paragraph{Oracle:}
\hypertarget{sec:oracle}{}

The \emph{Oracle} method establishes an upper bound on personalization performance, in terms of similarity to the ground truth, by using Oracle story rules derived directly from the ground-truth author-written story for each writing prompt in the generation set. These rules are obtained by contrasting the \hyperlink{sec:avg-author}{Average Author} with the ground-truth author story, following the same process as in \hyperlink{sec:delta}{Delta}. Additionally, we include a one-shot demonstration, following the approach described in \hyperlink{sec:rag}{RAG}.

\section{Experiments}
In this section, we describe the implementation details and evaluation methods. 

\subsection{Implementation Details}
We prompt GPT-4o in a chain-of-thought manner \citep{wei2022chain, shashidhar-etal-2024-unsupervised} with a temperature of 0.0 and a maximum token limit of 4096 to generate story rules for the Delta and Oracle methods, as well as for generating the Author Writing Sheet, Author Writing Summary, and their respective persona descriptions (\(\text{LLM\textsubscript{persona}}\)) and personalized story rules (\(\text{LLM\textsubscript{rule}}\)). For story generation (\(\text{LLM\textsubscript{story}}\)) across all methods, including \emph{Average Author}, we evaluate three models: GPT-4o \citep{bubeck2023sparks}, Llama 3.1 8B, and Llama 3.1 70B \citep{dubey2024llama}, using a temperature of 0.7 and top\_p of 0.95 \citep{wang-etal-2024-rolellm} (See Appendix~\ref{app:llama-results} for results using Llama models for story generation). Additionally, we implement an ablation variant of our proposed method for Writing Sheet and Writing Summary that excludes the persona description from the system prompt called \emph{Writing Sheet nP} and \emph{Writing Summary nP} respectively.  


\subsection{Automatic Evaluation}

We evaluate the performance of personalized story generation using two automated evaluation methods: 
First, we use GPT-4o-as-a-judge which is shown to be a scalable way to approximate human judgments for complex open-ended tasks \citep{zheng2023judging} and calculate the win-rate of our personalization methods against the non-personalized method, \emph{Average Author} on two aspects: 
\begin{itemize}[noitemsep, topsep=0pt]
    \item \textbf{Faithfulness to Writing History}: We evaluate how well the generated story aligns with the author history by measuring win-rates using Claims from the \hyperlink{sec:writing-summary}{Author Writing Summary} as the ground-truth reference for each narrative category \citep{wang-etal-2024-rolellm, wang2023automated, yunusov-etal-2024-mirrorstories}.
    \item \textbf{Similarity to Author Story}: We assess how closely the generated story matches the author's ground-truth story by measuring win-rates using the ground-truth author story as the reference for each narrative category \citep{lyu2024href, shashidhar-etal-2024-unsupervised}.
\end{itemize}
For both aspects, we prompt GPT-4o in a chain-of-thought manner \citep{wei2022chain} to assign scores (1–5) to stories. We randomly shuffle the order of compared pairs, one generated by a personalization method and the other by the \emph{Average Author} method, to avoid biases. Scores are broken down into four narrative categories: Plot, Creativity, Development, and Language Use \citep{saha-etal-2024-branch}. The story with the higher score in a category is declared the winner, and the overall winner is based on the highest total score across all categories. Additionally, we evaluate personalization using traditional metrics \citep{xie-etal-2023-next} including lexical overlap, story diversity, and stylistic similarity (see Appendix~\ref{app:trad-merics} for more details).



\subsection{Human Evaluation}
\label{sec:human-eval-story-gen}

We conduct a human evaluation on a subset of our data to further assess subtext and identify any interesting insights that the LLM judges might have missed \citep{chakrabarty2024art, subbiah-etal-2024-reading}, by comparing stories generated by personalization methods (\emph{Delta}, \emph{Writing Sheet}, \emph{Writing Summary}) against the \emph{Average Author} method for similarity to the ground-truth author story \citep{lyu2024href}. We ask annotators to justify their choices after making the comparison. Three annotators, recruited via Upwork (\$17/hour), evaluated 45 author stories in total (15 common to establish agreement and 10 exclusive to each annotator to increase coverage of annotation \citep{song-etal-2024-veriscore}). Each author story had three pairs, each containing a story from one personalization method, resulting in 135 annotated story pairs.


\section{Results}

This section describes the results of our user study. 
We first present the analysis of user annotation performance in different conditions. We measure annotation speed and annotation quality.
Then, we report user perception of different conditions, e.g., their confidence level of annotated data and the perceived cognitive load. 

\subsection{Annotation Speed}
%A faster speed suggests a less expensive annotation.

\begin{table}[htb]
    \centering
    \caption{Number of Annotated Data (5 minutes)}
    \vspace{-2.5mm}
    \resizebox{0.8\linewidth}{!}{%
    \begin{tabular}{lcc}
        \toprule
              & \textbf{Completed Annotation} & \textbf{SD} \\
        \midrule
        Manual & 2.00 & 0.91 \\
        AI Assistant & 3.75 & 2.09 \\
        {\tool} & \textbf{8.75} & 2.74 \\
        \bottomrule
    \end{tabular}
    }
    \label{tab:annotation_speed}
\end{table}


Since each session has a fixed annotation time period, we use the average number of annotated tasks to represent the annotation speed. 
We compare the number of tasks completed across three conditions: Manual, AI assistant, and {\tool}. Table~\ref{tab:annotation_speed} presents the average annotation count completed within 5 minutes of the task session for each condition.
When using {\tool}, participants annotated the most tasks ($Mean = 8.75, SD = 2.74$), followed by using the AI assistant ($Mean = 3.75, SD = 2.09$), and manually annotating ($Mean = 2.00, SD = 0.91$). The ANOVA test shows that the mean differences are statistically significant ($p$-value = 1.96e-8).

The substantial improvement in task achievement with {\tool} suggests that {\tool} could enhance productivity in real-world applications.
Compared to ChatGPT, {\tool} provides a better explanation for participant engagement and validation.
P3 said, ``\textit{Even though ChatGPT can give me plenty of data, I need to manually check its output. I feel that connecting its output SQL to the schema is challenging in the limited time.}''
Furthermore, {\tool} serves as an interactive interface, allowing users to iteratively refine the data.
Both P6 and P9 appreciated this utility.
P9 said, ``\textit{It's easier and quicker to test and iterate on these [data] using the tool}'',


\subsection{Annotation Quality}
We define the quality of a text-to-SQL dataset into (1) \textbf{Correctness} (whether there is a syntax error, and whether the NL question matches the SQL query), (2) \textbf{Naturalness} (whether the NL question is natural enough as a human daily question), and (2) \textbf{Diversity} (whether the dataset has comprehensive coverage of different entities and query types, without any biases).

\subsubsection{\textbf{Correctness}}

To evaluate the correctness of participants' annotations, we manually review all data collected during the user study. We evaluate two types of errors in the annotated data. 
First, we look for SQL syntax errors or the misuse of entities and references in the database schema. 
We identify this type of error by executing the SQL query in a sandbox database that is adequately populated from the schema.
The error leads to execution failures or the return of an empty result.
Second, we evaluate the equivalence between the SQL query and the NL question. While the SQL might be syntactically correct on the schema, it may not accurately represent the intent of the NL question. In this case, we manually evaluate their equivalence.
Table~\ref{tab:correctness_annotation} shows the two types of error rates and the overall accuracy for each condition.

We observe different reasons for errors in manual annotation compared to using the AI assistant. During manual annotation, participants who were less proficient in SQL often made grammatical mistakes (29.24\%). However, since they tended to write simple SQL queries, the equivalence error was less (5.15\%).
The AI assistant, ChatGPT, rarely introduces syntax errors. However, it tends to generate more complex SQL queries (e.g., multiple JOINs) than manual annotation. Despite participant refinement, these queries often fail to match the complex schema due to hallucination, leading to SQL execution errors (18.73\%). Moreover, these complex queries pose greater challenges in maintaining equivalence with the NL question, resulting in the highest error rate of equivalence ((8.34\%).

Compared to manual annotation and using the AI assistant, using {\tool} achieves the highest accuracy (95.56\%). For SQL errors, queries sampled by {\tool} are guaranteed to be syntactically correct. Regarding NL-SQL equivalence, {\tool} aligns SQL and NL through step-by-step analysis, achieving the best equivalence after human refinement. However, we observe cases (4.44\%) where they are not fully equivalent, suggesting room for further improvement in NL generation accuracy.



\begin{table}[htb]
    \centering
    \caption{Correctness of Annotated Data}
    \vspace{-2.5mm}
    \resizebox{1\linewidth}{!}{%
    \begin{tabular}{lccc}
        \toprule
              & \textbf{SQL Error} & \textbf{Equivalence Error} & \textbf{Accuracy} \\
        \midrule
        Manual  & 29.24\%  & 5.15\% & 65.61\% \\
        AI Assistant   & 18.73\%   & 8.34\% & 72.93\% \\
        {\tool} & \textbf{0}   & \textbf{4.44\%} & \textbf{95.56\%} \\
        \bottomrule
    \end{tabular}
    }
    
    \label{tab:correctness_annotation}
\end{table}



\subsubsection{\textbf{Naturalness}}
In addition to correctness, the naturalness of the NL question is crucial for the quality of text-to-SQL data.
While an NL question may accurately match its SQL query, it might be verbose. In the real world, people tend to ask concise questions that follow certain natural language patterns. To evaluate naturalness, we first calculate the \textit{Flesch-Kincaid Readability Score}~\cite{flesch_score}, an automatic metric measuring text readability on a scale from 0 to 100.
To better assess naturalness, we further manually rate all annotated questions from 1 to 7 after masking the conditions.\footnote{{\tool} references synthetic values in a sandbox database, which can be easily identified by the human raters. We replace all these values with commonly used values for fair evaluation.}

\begin{table}[htb]
    \centering
        \centering
        \caption{Flesch-Kincaid Readability Score of Annotated Questions (0-100).}
        \vspace{-2.5mm}
        \resizebox{0.69\linewidth}{!}{%
        \begin{tabular}{lcc}
            \toprule
                  & \textbf{Flesch-Kincaid Score} & \textbf{SD} \\
            \midrule
            Manual & 76.94 & 13.15 \\
            AI Assistant & 56.14 & 18.62 \\
            {\tool} & 72.32 & 16.54 \\
            \bottomrule
        \end{tabular}
        }
        \label{tab:Flesch-Kincaid}
\end{table}

\begin{table}[htb]
    \hspace{0.06\textwidth}
        \centering
        \caption{Manual Rating of Naturalness of Annotated Questions (0-7).}
        \vspace{-2.5mm}
        \resizebox{0.65\linewidth}{!}{%
        \begin{tabular}{lcc}
            \toprule
                  & \textbf{Human Rating} & \textbf{SD} \\
            \midrule
            Manual & 6.25 & 0.52 \\
            AI Assistant & 6.02 & 0.58 \\
            {\tool} & 6.19 & 0.44 \\
            \bottomrule
        \end{tabular}
        }
        \label{tab:natural_rating}
\end{table}


\begin{table*}[htb]
    \centering
    \caption{SQL Query Component Diversity Analysis in Annotated Datasets.}
    \vspace{-2.5mm}
    \resizebox{0.75\linewidth}{!}{%
    \begin{tabular}{l cc|cc|cc|cc}
        \toprule
        & \multicolumn{2}{c}{\textbf{Clause}} & \multicolumn{2}{c}{\textbf{Table}} & \multicolumn{2}{c}{\textbf{Column}} & \multicolumn{2}{c}{\textbf{Value}} \\
        \cmidrule(lr){2-9}
        \textbf{Method} & Diversity & Mean & Diversity & Mean & Diversity & Mean & Diversity & Mean \\
        \midrule
        Manual       & 0.48 & 2.40 & 0.41 & 1.29 & 0.17 & 1.09 & 0.44 & 0.33 \\
        AI Assistant & 0.64 & 5.10 & 0.81 & 5.75 & \textbf{0.64} & 3.73 & 0.62 & 1.25 \\
        {\tool}      & \textbf{0.69} & 3.18 & \textbf{0.83} & 2.54 & 0.59 & 1.71 & \textbf{0.66} & 0.85 \\
        \bottomrule
    \end{tabular}
    }
    \label{tab:diversity}
\end{table*}


\begin{figure*}[htb]
  \centering
\includegraphics[width=0.85\textwidth]{figure/task1_cognitive.png}
  \caption{NASA Task Load Index Ratings of Text-to-SQL Data Annotation}
  \label{fig:cognitive1}
\end{figure*}

\begin{figure*}[htb]
  \centering
  \includegraphics[width=0.92\textwidth]{figure/sus1.png}
  \caption{SUS Scores of Text-to-SQL Data Annotation}
  \label{fig:sus1}
\end{figure*}



As shown in Table~\ref{tab:Flesch-Kincaid} and Table~\ref{tab:natural_rating}, the \textit{Flesch-Kincaid Readability Score} is consistent with human ratings. While manual annotation achieves the highest score (Flesch-Kincaid Score: 76.94, Human Rating: 6.25) as expected, the NL questions annotated through {\tool} achieve a comparable score (Flesch-Kincaid Score: 73.32, Human Rating: 6.19). We observe that using ChatGPT achieves the worst naturalness in questions (Flesch-Kincaid Score: 56.14, Human Rating: 6.02). Notably, ChatGPT-generated questions often include SQL keywords or follow SQL query patterns (e.g., \textit{Return all student names that are grouped by grades.}). 
Participants often accept ChatGPT's generation without modifications. While {\tool} is also built upon ChatGPT, it employs step-by-step analysis and in-context learning from real-world questions, thereby better incorporating real question patterns into the generated questions. 
Moreover, {\tool} offers a better interface and provides helpful suggestions for refining questions. Participants report that they are more willing to polish the LLM-generated data when using {\tool}.
P8 said, ``\textit{I really enjoy the UI in this tool. It suggests how to polish the data. And I really enjoy playing with the alignment feature to see what I can do with existing data.}''



\subsubsection{\textbf{Diversity}}
To evaluate the diversity and potential biases in the annotated dataset, we analyze the composition of participant-annotated data across four dimensions, including the number of clauses, columns, tables, and values involved. We measure the diversity of each dimension using Simpson's Diversity Index~\cite{simpson_diversity}, which is used to quantify the level of heterogeneity of a certain property.





Table~\ref{tab:diversity} shows the diversity and mean values for each method and dataset property. {\tool} demonstrates better diversity in the generated SQL for most dimensions, except for columns generated by the AI assistant. 
This is because ChatGPT tended to join excessive tables (mean = 5.75) in \texttt{FROM} clauses and include excessive columns (mean = 3.73) in  \texttt{SELECT} clauses.
In contrast, {\tool} learned the distribution from real-world datasets, resulting in a more reasonable distribution. For manual annotation, participants typically wrote simple SQL queries. For instance, they rarely used JOIN clauses, leading to low diversity.





\subsection{User Cognitive Load \& Usability Rating}

% Reducing cognitive load during data annotation is crucial since it directly make the annotation process cheaper.
% Figure~\ref{fig:cognitive1} presents participants’ ratings on the five cognitive load factors from the NASA TLX questionnaire~\cite{NASA-TLX}. The ANOVA test demonstrates that the mean differences are all statistically significant ($p$-value=6.7e-5, 6e-6, 1.9e-5, 1.5e-5, 1.3e-4 respectively).

Reducing cognitive load during data annotation is crucial, since it directly makes this process more cost-effective. Figure~\ref{fig:cognitive1} illustrates participants' ratings on the five cognitive load factors from the NASA TLX questionnaire. The ANOVA test reveals statistically significant ($\alpha = 0.001$) differences in means for all factors: Mental Demand ($p = 6.7 \times 10^{-5}$), Temporal Demand ($p = 6.0 \times 10^{-6}$), Performance ($p = 1.9 \times 10^{-5}$), Effort ($p = 1.5 \times 10^{-5}$), and Frustration ($p = 1.3 \times 10^{-4}$).



The result demonstrates that {\tool} can significantly reduce users' cognitive load compared to manual annotation and using ChatGPT.
P1 said, ``\textit{Generating NL2SQL data felt effortless with the help of the tool.}''.
We believe this is achieved by a smoother collaboration between the huamn and the LLM.
P1 also comprehensively discussed how {\tool} reduces users' cognitive load,
``\textit{The system allows me to generate a variety of SQLs without any cognitive efforts. This was great because I didn't have to think about SQL syntax, different queries one might ask about the dataset. I also like that the system can generate corresponding NL questions for each SQL. While the generated NL question wasn't always accurate, the system already provided me something which I could iterate on. This is almost like someone gives you a draft that you can just revise vs.~gives you an empty doc for you to start from scratch in writing.}''





% \todo{discuss the comparison with ChatGPT}

Figure~\ref{fig:sus1} displays SUS scores reported by participants, showing consistently positive feedback across all dimensions. Notably, no participants disagreed with any dimension. While a few participants expressed neutral opinions about SQL queries and NL questions suggested by {\tool}, most viewed them positively. All participants expressed high confidence in the quality and quantity of data annotated using {\tool}.

\paragraph{Uncertainty-based hallucination detection methods.}
Various approaches have been proposed to detect hallucinated content in LLMs generation.
Unlike other methods that require external knowledge sources for fact-checking~\citep{gou2024critic, chen-etal-2024-complex, min-etal-2023-factscore, huo2023retrieving}, uncertainty-based approaches are reference-free and rely only on LLM internal states or behaviors to determine hallucination~\citep{10.1145/3703155}. 
For instance, sampling-based approaches generate multiple responses and measure the diversity in meaning among them~\citep{fomicheva-etal-2020-unsupervised, kuhn2023semantic, lin2024generating}, while density-based approaches approximate the training data distribution and provide probabilities or unnormalized scores to assess how likely a generated response belongs to the distribution~\citep{yoo-etal-2022-detection, ren2023outofdistribution, vazhentsev-etal-2023-hybrid}.

In this paper, we focus on uncertainty quantification methods that rely on token-level likelihood or entropy~\citep{guerreiro-etal-2023-looking, malinin2021uncertainty}. 
Recent works have explored refining likelihood estimation by incorporating semantic relationships or reweighting token importance. For instance, Claim-Conditioned Probability (CCP)~\citep{fadeeva-etal-2024-fact} was introduced to recalculate likelihood according to semantical equivalence; while \citet{zhang-etal-2023-enhancing-uncertainty} and \citet{duan-etal-2024-shifting} adjust token weights to better convey meaning in uncertainty aggregation. \emph{Although these approaches leverage token-level information, they are typically evaluated at the sentence level, raising questions about their reliability}. To address this, we conduct a comprehensive analysis of entity-level hallucination detection for finer-grained performance insights.


\paragraph{Fine-grained hallucination detection benchmark.}

Most hallucination detection benchmarks are in sentence or paragraph level. For example, CoQA~\citep{reddy-etal-2019-coqa}, TriviaQA~\citep{joshi-etal-2017-triviaqa}, TruthfulQA~\citep{lin-etal-2022-truthfulqa}, and HaluEval~\citep{li-etal-2023-halueval}. These benchmarks classify each generated response as either hallucinated or correct. However, instance-level detection cannot pinpoint specific hallucinated content, which is crucial for correcting misinformation~\citep{cattan2024localizingfactualinconsistenciesattributable}. This limitation becomes particularly problematic in long-form text, where a single response often combines supported and unsupported information, making binary quality judgments inadequate~\citep{min-etal-2023-factscore}.

To address these challenges, recent works have advanced benchmarks for more granular hallucination detection. For example, \citet{min-etal-2023-factscore} introduced \textsc{FActScore}, which decomposes LLM-generated text into atomic facts---short sentences conveying a single piece of information---for more precise evaluation. In parallel, \citet{cattan2024localizingfactualinconsistenciesattributable} introduced \textsc{QASemConsistency}, decomposing LLM generated text with QA-SRL, a semantic formalism, to form simple QA pairs, where each QA pair represent one verifiable fact. \emph{However, these methods do not enable entity-level hallucination detection, as they lack explicit entity-level labeling (hallucinated or not) in the original generated text}.  
Beyond decomposition-based approaches, datasets like \textsc{HaDes}~\citep{liu-etal-2022-token} and CLIFF~\citep{cao-wang-2021-cliff} create token-level hallucinated content by perturbing human-written text, allowing token-level annotation on the same text. These perturbed hallucinated content, however, could be unrealistic, biased, and overly synthetic due to the limitations of models they used to perturb words. 
To bridge this gap, we create a new dataset with entity-level hallucination labels on the same LLMs generated text. This allows us to evaluate uncertainty-based hallucination detection approaches on a finer-grained level and analyze their reliability.




  

\section{Conclusion}
\label{sec:Conclusion}
This work evaluates proprietary and open-weight models in agentic frameworks for handling ambiguity in software engineering. In code generation, to effectively integrate new information into the solution, an agent must detect ambiguity and ask targeted questions. Our key findings are:
\begin{itemize}[itemsep=0pt, topsep=0pt]
    \item Given an underspecified input, Claude Sonnet 3.5 and Claude Haiku 3.5 with interaction can achieve 80\% of their performance with a well-specified input. In contrast, open-weight models struggle: Deepseek relies on navigational cues to locate relevant files, while Llama 3.1 70B extracts limited information from the user.
    \item LLMs do not interact unless explicitly prompted, and their ambiguity detection is highly sensitive to prompt variations. Only Claude Sonnet 3.5 achieves a higher accuracy of 84\% in distinguishing between well-specified and underspecified input.

    \item Claude Sonnet 3.5, Haiku 3.5, and Deepseek effectively extract new, detailed user information, whereas Llama 3.1 struggles to ask the right questions.
    
\end{itemize}
Despite these advances, a gap remains between resolve rates for underspecified vs. fully specified issues. Open-weight models need better interaction strategies to improve resolution, while proprietary models, particularly Claude Haiku 3.5, require stronger prompting to engage interactively. This work establishes the current state-of-the-art in handling ambiguity through interaction, breaking the resolution process into multiple steps.




\newpage

\section{Limitations}

\paragraph{Limited Author History Size:} 
Table~\ref{tab:dataset-stats} (70\% of the Stories/Author column) shows that N.Magazine and New Yorker authors have an average of 3 stories in their profiling set, whereas Reddit and AO3 authors have around 10. A larger profiling set likely enables more accurate inference of an author's story-writing characteristics, leading to improved personalization. However, the cognitive demands of writing long stories result in fewer stories per author available online, thereby limiting the scalability of personalized story generation research. 

\paragraph{Author Writing Sheet Recall:} We validate the precision of our Author Writing Sheet through human evaluation, ensuring that the Claim-Evidence pairs accurately describe the author's story-writing style. However, we do not assess recall, i.e., whether \emph{all} of the author's story-writing characteristics are captured. Measuring recall is challenging due to the inherent subjectivity of the task, making it impractical to ask humans to independently construct Writing Sheets for comparison.

\paragraph{Multi-Agent Systems and Fine-Tuning LLMs:} Our work explores role-playing for personalized story generation by prompting an LLM with tailored persona descriptions and story rules. Future research can extend this by incorporating multi-agent systems, where specialized agents focus on different narrative categories to enhance personalization \citep{huot2024agents, bae-kim-2024-collective}. Additionally, fine-tuning LLMs to explicitly reason over author history could be investigated as a method for better aligning generated content with individual author preferences \citep{salemi2025reasoning, shashidhar-etal-2024-unsupervised, shaikh2024show}.

\paragraph{Automatic Evaluation Methods:} Evaluating long-form text remains challenging for LLMs, particularly for creative outputs that involve nuanced aspects like subtext, which are difficult to assess automatically and often require human evaluation \citep{subbiah-etal-2024-storysumm}.
 
\section{Ethical Considerations}
Both human evaluation tasks were approved by an institutional review board (IRB). All annotators, who are US-based and fluent in English, were informed of the nature of the research study, to which they provided informed consent, and were compensated at an hourly rate of \$17, meeting the minimum wage requirements in our state. All scientific artifacts, including models and datasets, were used in accordance with their intended purpose to ensure ethical and responsible research practices. 

All data used in this study were manually collected from publicly available sources\footnote{Permission was obtained from the authors of the Storium dataset for its inclusion in Mythos.}, adhering to the data usage and crawling policies of the respective websites. To comply with copyright constraints, we will release only the links to the stories used in this research, rather than the stories themselves, following the approach of \citep{chakrabarty2024art}.  

Fully anonymizing our dataset is not feasible\footnote{Storium authors already anonymize their dataset.} as the research involves mimicking specific authors’ writing behaviors. We note that even if the usernames are anonymized, it is still possible to infer an author's publicly available identity from the story link. Therefore, for each author for every source, we release their usernames along with corresponding writing prompts, and author-written stories in the profiling and generation set of our dataset. However, the dataset remains limited in scale, and no additional metadata, including demographic information, is included, thereby reducing potential risks. To further mitigate unforeseen harms, the dataset will be released under an **Educational or Academic Research, Non-Commercial Use** license, following \citep{akoury-etal-2020-storium}.

\newpage
\bibliographystyle{acl_natbib}
% This must be in the first 5 lines to tell arXiv to use pdfLaTeX, which is strongly recommended.
\pdfoutput=1
% In particular, the hyperref package requires pdfLaTeX in order to break URLs across lines.

\documentclass[11pt]{article}

% Change "review" to "final" to generate the final (sometimes called camera-ready) version.
% Change to "preprint" to generate a non-anonymous version with page numbers.
\usepackage[final]{acl}

% Standard package includes
\usepackage{times}
\usepackage{latexsym}

% For proper rendering and hyphenation of words containing Latin characters (including in bib files)
\usepackage[T1]{fontenc}
% For Vietnamese characters
% \usepackage[T5]{fontenc}
% See https://www.latex-project.org/help/documentation/encguide.pdf for other character sets

% This assumes your files are encoded as UTF8
\usepackage[utf8]{inputenc}

% This is not strictly necessary, and may be commented out,
% but it will improve the layout of the manuscript,
% and will typically save some space.
\usepackage{microtype}

% This is also not strictly necessary, and may be commented out.
% However, it will improve the aesthetics of text in
% the typewriter font.
\usepackage{inconsolata}

%Including images in your LaTeX document requires adding
%additional package(s)
\usepackage{graphicx}
\usepackage{xcolor}

% If the title and author information does not fit in the area allocated, uncomment the following
%
%\setlength\titlebox{<dim>}
%
% and set <dim> to something 5cm or larger.

\usepackage{booktabs}
\usepackage{hyperref}
\usepackage{multirow}
\usepackage{multicol}
\usepackage[most]{tcolorbox}
\usepackage{adjustbox}
\usepackage{graphicx}
\usepackage{fullpage}
\usepackage{times}
\usepackage{fancyhdr,graphicx,amsmath,amssymb}
%\usepackage[ruled,vlined]{algorithm2e}
\usepackage{algorithm}
\usepackage{algpseudocode}
\usepackage{booktabs}
\usepackage{adjustbox}
\usepackage{url}
\usepackage{hyperref}
\usepackage{amssymb}
\usepackage{marvosym}
\usepackage{multirow}
\usepackage{subcaption}
\DeclareMathOperator*{\argmax}{arg\,max}
\DeclareMathOperator*{\argmin}{arg\,min}


\newtcolorbox{promptbox}[2][]{
  colback=gray!10,
  colframe=gray!50,
  arc=3mm,
  boxrule=1pt,
  left=10pt,
  right=10pt,
  top=8pt,
  bottom=8pt,
  before skip=12pt,
  after skip=12pt,
  fonttitle=\bfseries,
  title=#2,
  #1
}

\title{Quality-Aware Decoding: Unifying Quality Estimation and Decoding}

% Author information can be set in various styles:
% For several authors from the same institution:
% \author{Author 1 \and ... \and Author n \\
%         Address line \\ ... \\ Address line}
% if the names do not fit well on one line use
%         Author 1 \\ {\bf Author 2} \\ ... \\ {\bf Author n} \\
% For authors from different institutions:
% \author{Author 1 \\ Address line \\  ... \\ Address line
%         \And  ... \And
%         Author n \\ Address line \\ ... \\ Address line}
% To start a separate ``row'' of authors use \AND, as in
% \author{Author 1 \\ Address line \\  ... \\ Address line
%         \AND
%         Author 2 \\ Address line \\ ... \\ Address line \And
%         Author 3 \\ Address line \\ ... \\ Address line}

\author{Sai Koneru$^{1}$,
  Matthias Huck$^{2}$,
  Miriam Exel$^{2}$, \textnormal{and}
  Jan Niehues$^{1}$ \\
  $^{1}$ Karlsruhe Institute of Technology \\
  $^{2}$ SAP SE, Dietmar-Hopp-Allee 16, 69190 Walldorf, Germany \\
  \texttt{\{sai.koneru, jan.niehues\}@kit.edu} \\
  \texttt{\{matthias.huck, miriam.exel\}@sap.com}}

%\author{
%  \textbf{First Author\textsuperscript{1}},
%  \textbf{Second Author\textsuperscript{1,2}},
%  \textbf{Third T. Author\textsuperscript{1}},
%  \textbf{Fourth Author\textsuperscript{1}},
%\\
%  \textbf{Fifth Author\textsuperscript{1,2}},
%  \textbf{Sixth Author\textsuperscript{1}},
%  \textbf{Seventh Author\textsuperscript{1}},
%  \textbf{Eighth Author \textsuperscript{1,2,3,4}},
%\\
%  \textbf{Ninth Author\textsuperscript{1}},
%  \textbf{Tenth Author\textsuperscript{1}},
%  \textbf{Eleventh E. Author\textsuperscript{1,2,3,4,5}},
%  \textbf{Twelfth Author\textsuperscript{1}},
%\\
%  \textbf{Thirteenth Author\textsuperscript{3}},
%  \textbf{Fourteenth F. Author\textsuperscript{2,4}},
%  \textbf{Fifteenth Author\textsuperscript{1}},
%  \textbf{Sixteenth Author\textsuperscript{1}},
%\\
%  \textbf{Seventeenth S. Author\textsuperscript{4,5}},
%  \textbf{Eighteenth Author\textsuperscript{3,4}},
%  \textbf{Nineteenth N. Author\textsuperscript{2,5}},
%  \textbf{Twentieth Author\textsuperscript{1}}
%\\
%\\
%  \textsuperscript{1}Affiliation 1,
%  \textsuperscript{2}Affiliation 2,
%  \textsuperscript{3}Affiliation 3,
%  \textsuperscript{4}Affiliation 4,
%  \textsuperscript{5}Affiliation 5
%\\
%  \small{
%    \textbf{Correspondence:} \href{mailto:email@domain}{email@domain}
%  }
%}

\begin{document}
\maketitle
\begin{abstract}
% Neural Machine Translation (NMT) has achieved high-quality translations in many scenarios, pushing the boundaries of tasks such as instruction-following and multimodal translation. 
Quality Estimation (QE) models for Neural Machine Translation (NMT) predict the quality of the hypothesis without having access to the reference.
An emerging research direction in NMT involves the use of QE models, which have demonstrated high correlations with human judgment and can enhance translations through Quality-Aware Decoding. Although several approaches have been proposed based on sampling multiple candidate translations and picking the best candidate, none have integrated these models directly into the decoding process. In this paper, we address this by proposing a novel token-level QE model capable of reliably scoring partial translations. We build a uni-directional QE model for this, as decoder models are inherently trained and efficient on partial sequences. We then present a decoding strategy that integrates the QE model for Quality-Aware decoding and demonstrate that the translation quality improves when compared to the N-best list re-ranking with state-of-the-art QE models (up to $1.39$ XCOMET-XXL $\uparrow$). Finally, we show that our approach provides significant benefits in document translation tasks, where the quality of N-best lists is typically suboptimal\footnote{Code can be found at \url{https://github.com/SAP-samples/quality-aware-decoding-translation}}
\end{abstract}
\section{Introduction}

Large language models (LLMs) have significantly impacted various Natural Language Processing (NLP) tasks \citep{brown2020language, jiang2023mistral, dubey2024llama}, including Neural Machine Translation (NMT). The field of NMT is transitioning from using dedicated encoder-decoder transformers \citep{vaswani2017attention, nllb2024scaling} to leveraging decoder-only LLM-based translation models \citep{kocmi2024findings}. This shift is driven by LLMs' ability to retain knowledge, handle large contexts, and follow instructions, learned during extensive pre-training \citep{xu2024contrastive, alves2024tower}. As a result, LLM-based MT models have achieved state-of-the-art translation quality \citep{kocmi2024findings}.

In parallel, Quality Estimation (QE) has become a well-researched subfield within NMT. QE models are trained to predict the quality of a translation without requiring access to the reference \citep{rei2021references,rei2022cometkiwi}. Interestingly, QE models can achieve performance in assessing translation quality that is comparable to MT evaluation models, which do have access to the reference \citep{zerva2024findings}.

This led to the question: "\textit{Can we integrate QE into the current translation process to improve quality?}" Incorporating QE into NMT offers several benefits. First, having a expert QE model guiding the decoding can further improve the quality. Second, by adapting the QE model with feedback from human annotators, we can generate future translations guided with the newly obtained feedback.

\begin{figure*}[!ht]
\includegraphics[width=\textwidth]{Figures/nbestlist_problem.png}
 \caption{Example from WMT'23 English → German \#ID: 10: The paragraph begins with 'Department of Homeland Security,' which should be translated as 'Ministerium für \textbf{I}nnere Sicherheit.' However, the top 25 beams do not contain the correct translation and begin with an error, making N-best list re-ranking insufficient. Although the top-5 tokens at the decoding contain the correct forms 'Inn' or 'Inner,' the probabilities split among them giving highest mass to the incorrect token 'inn.' Quality-Aware decoding can prevent errors with earlier integration.}
\label{fig:nbestlist}
\end{figure*}


Several approaches have been explored to integrate QE into the translation process. These include re-ranking the N-best list \citep{fernandes2022quality}, applying minimum Bayes risk (MBR) decoding on a quality-filtered N-best list \citep{tomani2024quality}, and training additional models for post-editing based on QE-predicted errors \citep{treviso2024xtower}. However, all these methods operate on fully generated sequences before the QE model can exert influence. Integrating QE earlier in the decoding process, referred in this paper as \textit{Quality-Aware Decoding}, could enhance translation quality and reduce reliance on the N-best list. This is especially relevant when dealing with long inputs as good translations during decoding are likely to be pruned and may need sampling larger number of finished hypothesis. We illustrate this in Figure \ref{fig:nbestlist}.

To achieve this, a QE model capable of predicting the quality of partial translations is required. However, current leading QE models face challenges in this area, as they are typically not trained to predict scores for incomplete hypotheses. \textit{Therefore, developing QE models that can handle partial translations is essential for implementing Quality-Aware Decoding during the translation process}.

In this work, we propose adapting LLM-based MT models to perform QE on partial translations and incorporating this model into the decoding. We create a token-level synthetic QE dataset using WMT Multidimensional Quality Metrics (MQM) data \citep{burchardt2013multidimensional, freitag2024llms}. We then adapt a uni-directional LLM-based MT model to predict whether a token is \textit{Good} or \textit{Bad}. Training QE models on these token-level tasks alleviates the data challenge and allows us to exploit the MQM data while simultaneously making the task easier for the model compared to predicting a score directly.

\begin{figure*}[!ht]
\includegraphics[width=\textwidth]{Figures/annotation_scheme.png}
 \caption{Token-level label annotation scheme using the MQM error tags. \textit{MASK} indicates that this token label will not be used in training to prevent incorrect learning signal.}
\label{fig:annotation}
\end{figure*}

Furthermore, integrating the QE model into NMT during decoding is not trivial, as we need to combine the QE estimates during decoding. Therefore, we use the decoding strategy from \citet{koneru2024plug}, and modify it to incorporate token-level predictions efficiently with the adapted QE model to provide real-time feedback during the decoding process. We summarize our main findings and contributions below.

\begin{itemize}
    \item We present a novel uni-directional QE model which estimates quality on incomplete hypotheses by averaging the probabilities of each token being classified as \textit{Good}. 
    
    %We demonstrate that it achieves improved correlation with human annotations on WMT 23 English $\rightarrow$ German, compared to the log probabilities of the same LLM-based NMT model.

    \item We propose a decoding strategy that combines the token-level QE model on partial hypothesis and the NMT model to perform Quality-Aware Decoding. 
    
    \item We show through experiments that early integration is essential and the translation quality is improved even when compared to re-ranking the N-best list with state-of-the-art QE models.

    \item We highlight the significance of our approach in document translation scenarios, where post-generation QE techniques fall short due to their reliance on the quality of the N-best list, a challenge that becomes more difficult as the input length increases.
\end{itemize}



\section{Quality-Aware Decoding}

The primary objective of this paper is to achieve Quality-Aware Decoding in MT. To accomplish this, it is essential to predict the quality of partial translations and integrate this information during the decoding process. Our approach proposes using one NMT model for generating translations and another adapted NMT model to predict the quality of the candidate translations produced by the first model.

First, we explain why relying solely on the NMT model to predict the quality of a hypothesis is insufficient and why an additional model is necessary. Next, we outline the adaptation of the NMT model for QE on partial translations, detailing the creation of a token-level QE dataset, the modifications made to the NMT model for this task, and the process of estimating the sentence-level quality score. Finally, we describe the algorithm used to incorporate the QE score into the decoding process.

\subsection{Decomposing Decoding: Translation + QE}
NMT models generate a token-by-token sequence and provide the probability of each token at the decoding step. The average of the log-probabilities is often used as a proxy to score the candidate during search. 

While NMT models are capable of generating high-quality translations, using the average log-probabilities of hypotheses as a scoring metric tends to yield poor correlation with actual translation quality \citep{eikema2020map, freitag2020bleu}. In many cases, a translation can continue in several different ways, all of which may be acceptable. If the starting tokens for these continuations differ, the probability mass may be spread across multiple options which is used during the search. However, from a quality perspective, all these continuations could still achieve a high score, as the QE scores are independent and need not sum to $1$.

Therefore, we propose a expert model that focuses on quality to estimate the scores better during decoding and  improve the search space leading to a better hypothesis.


% Therefore, relying solely on the average log-probabilities during decoding is not ideal, as it computes the score independently for each token and does not fully correlate with the overall quality of the current hypothesis.

\subsection{Quality Estimation on Partial Sequences}

% NMT models decode sequences token-by-token. 
To provide a quality score during decoding, the QE model must be capable of handling incomplete sequences. It should not penalize a sequence if there is a potential extension that could lead to a perfect translation.

Estimating the score in this way is not feasible with current QE models, such as COMET \citep{rei2021references}, as they were not trained for this specific task and cannot provide reliable scores in the context of partial translations. Hence, we need to develop a partial QE system.

When building a partial QE system, several factors need to be considered. First, should the model use a uni-directional or bi-directional architecture? A \textbf{uni-directional} model is more efficient, as it allows for caching the hidden states, which can then be used for subsequent steps without re-encoding, unlike a bi-directional model.

Next, we need to decide whether to predict the QE score at the sequence level or at the token level. For \textbf{token-level QE}, we can directly use data from MQM annotations, as we already know which tokens are \textit{Good} or \textit{Bad}. However, for segment-level scoring, we need to consider how to synthetically create the training data. 

% Additionally, COMET models are encoder-only architectures pre-trained on full sentences, rather than partial sentences as required in this case. Moreover, predicting the score of partial translations naturally favors decoder-only models due to their efficiency. New tokens only need to process the preceding sequence, avoiding the need to re-encode the entire sequence. Additionally, this approach simplifies training, as we do not require synthetically shorter samples. 

%  Furthermore, there is no readily available dataset containing partial translations along with their quality scores. Hence, we need to design the adaptation process with a QE model that is uni-directional and exploit already available human annotated data.

% \subsubsection{Token-level Quality Estimation}

Therefore, we decide adapt the uni-directional model into a token-level QE system that predicts whether each token is \textit{Good} or \textit{Bad} (a binary decision) by adding an additional classifier head. This adaptation enables us to estimate the score for a sequence by calculating the average probability that each token is classified as \textit{Good}. We hypothesize that adapting the model in this way, rather than directly predicting the score, provides greater stability, as the last hidden states inherently contain token-level information and do not require mapping the entire sequence to a single score.

For training this model, we leverage the WMT MQM data containing error annotations in MT outputs. We can treat tokens before an error as \textit{Good} and those containing inside an error as \textit{Bad}. Then, we can train in uni-directional manner where each token's label is predicted using only the preceding context in the hypothesis. This is crucial as we only have the preceding context to estimate the quality for partial hypothesis.

\subsubsection{Learning the Right Signal}

\begin{algorithm*}[!t]
\caption{Computing merged score of partial hypothesis with translation and token-level QE models.}
\begin{algorithmic}[1]
\setlength{\baselineskip}{1.2em}
\Procedure{MergeScore}{}
    \State \textbf{Input:}   Hypothesis tokens $h_1, h_2, h_3, \dots, h_{n}$, Translation Model $\mathcal{M}_{NMT}$, QE model $\mathcal{M}_{QE}$, Source sentence $\mathcal{S}$, Re-ranking weight $\alpha$,
    \State \textbf{Output:} $merged\_score$
    \State $Score_{NMT} \gets \frac{1}{n}\sum \log \mathcal{P}(h_1, h_2,\dots, h_{n}|\mathcal{S};\mathcal{M}_{NMT})$ 
    \State $Score_{QE} \gets \frac{1}{n}\sum \log \mathcal{P}(0_{1}, 0_{2},\dots,0_{n} | h_1, h_2,\dots, h_{n},\mathcal{S};\mathcal{M}_{QE})$ 
    \State $merged\_score \gets (\alpha) \times Score_{NMT} + (1 - \alpha) \times Score_{QE}$
\EndProcedure
\end{algorithmic}
\label{alg:joint}
\end{algorithm*}

The straightforward approach to creating labels is to assign $1$ to all tokens within the error span and $0$ otherwise. However, MQM annotations can mark errors from words to phrases, and the starting tokens of an error span may not always be wrong. This is illustrated in Figure \ref{fig:annotation}.

For example, consider the German sentence \textit{"Ich spiele Tennis"} translated by three different NMT systems, each annotated with MQM error labels. In this work, we focus on learning a binary decision: whether an error is present, ignoring error severity.

\textbf{System 1: No error}: The translation \textit{"I play Tennis"} is perfect, and all tokens are labeled as "\textit{Good}."

\textbf{System 2: Partial error}: The translation \textit{"I played Tennis"} has an error in the verb form ("played" instead of "play"). The error is in the token span \textit{"played"}, but not all tokens in this span are incorrect (e.g., "pla" is correct). Assigning a "\textit{Bad}" label to the entire span would lead to incorrect learning. A more refined approach is needed to mark errors accurately at the token level.

\textbf{System 3: Full error}: The translation \textit{"I enjoy Tennis"} contains an error in \textit{"enjoy"}, so all tokens in this span should be labeled as "\textit{Bad}."

It is not trivial to decide when the prefix of an error span is correct/incorrect. To achieve accurate labeling, we propose the following scheme:

\begin{itemize} \item Apply a \texttt{<MASK>} operation to all tokens within the error span. \item Only the last token in the span is assigned the label "\textit{Bad}", as the error is considered complete at the end of the span. \end{itemize}

If the error token is in the middle, we still train the model to predict "\textit{Bad}" in the end and let the model determine which tokens should be part of the error span during inference. This approach ensures that errors are identified without explicitly defining the error span. 

\subsubsection{Sequence-Level Quality Estimation}


After fine-tuning a token-level classification model to predict the quality of the tokens, we still need to map these predictions into a sequence-level score that can be integrated during the decoding process. There are several potential ways to achieve this.

One approach is to simply count how many tokens are classified as \textit{Bad} in the current hypothesis. However, this method has limitations. The number of errors should be normalized based on the length of the hypothesis to account for varying sizes. Additionally, converting the probabilities into a fixed number of error tokens would need to account for different error types according to the MQM format, as each error counts differently.

To avoid such strict scoring schemes, we take a simpler approach. We average the log probabilities of all tokens that are classified as \textit{Good}. This method inherently accounts for the length of the hypothesis, and it provides a score on the scale of log probabilities, which aligns with the decoding process. Therefore, we use this averaged log probability as a proxy metric for the QE score, where a higher score indicates better quality
(\textbf{Line 5} in Algorithm \ref{alg:joint}).

\subsubsection{Fusing Translation and Quality}

We can use a token-level QE system to evaluate the quality of a source and partial hypothesis during decoding. However, integrating these probabilities into all candidates is computationally expensive, as each beam considers extensions equal to the vocabulary size.

To address this, we adopt a simplified decoding strategy from \citet{koneru2024plug}, which ensembles models with different vocabularies. By adapting the same MT model for token-level QE, we simplify the merging process, as the vocabularies match. This restriction is reasonable, as it is also beneficial to leverage the knowledge learned by the specialized MT for token-level QE.

The core idea is to re-rank the top candidates at each decoding step using the QE model. After re-ranking, the translation and QE scores are merged, and the process repeats until the end-of-sentence token is generated, for each beam. This strategy allows us to efficiently incorporate the QE model’s estimate, improving translation quality.

During decoding, at each step, we have scores for $n$ beams and $V$ possible extensions from the vocabulary. In typical beam search, we select the top $n$ extensions and expand the hypothesis. To make the decoding process Quality-aware, we estimate the quality of these extensions. Since estimating all extensions is computationally expensive, we limit the candidates by selecting a specified number of top candidates.

To achieve this, we use a hyper-parameter $topk$, which selects the best $topk$ extensions for each beam. For each of these top $topk$ extensions, we compute a combined score, detailed in Algorithm \ref{alg:joint}. This combined score incorporates both the translation model score and the quality estimation score, ensuring the quality is considered during decoding.

For a top extension at decoding step $n$, let the current tokens be $h_1, h_2, h_3, \dots, h_n$. The NMT model score is computed as the average log probabilities of each token (Line 4). For the token-level QE model, we compute the average probability of each token being classified as '\textit{Good}' (Line 5). The merged score is equal to weighted linear combination of these probabilities, with weight $\alpha$ (Line 6).

Thus, to make the decoding process Quality-Aware, we first train a token-level QE system by adapting the same NMT model to ensure vocabulary matching. We then combine the scores from both models to improve the sequence estimates explored during search.


\begin{table*}[!ht]
\resizebox{2\columnwidth}{!}{
\begin{tabular}{@{}ccccc@{}}
\toprule
\multicolumn{1}{c|}{Model}            & \multicolumn{1}{c|}{Beams}                & \multicolumn{1}{c|}{Re-ranking}              & MetricX ($\downarrow$)     & XCOMET-XXL ($\uparrow$)    \\ \midrule
\multicolumn{5}{c}{\textit{English $\rightarrow$ German}}                                                                                                          \\ \midrule
\multicolumn{1}{c|}{Tower}            & \multicolumn{1}{c|}{5}                    & \multicolumn{1}{c|}{\_}                      & 2.52          & 86.93          \\
\multicolumn{1}{c|}{Tower}            & \multicolumn{1}{c|}{25}                   & \multicolumn{1}{c|}{XCOMET-XL QE}            & 2.37          & 87.79          \\
\multicolumn{1}{c|}{Tower}            & \multicolumn{1}{c|}{25}                   & \multicolumn{1}{c|}{Tower QE} & 2.38          & 87.40          \\
\multicolumn{1}{c|}{Tower + Tower QE} & \multicolumn{1}{c|}{5 (25* for Tower QE)} & \multicolumn{1}{c|}{\_}                      & 2.12          & 88.95          \\
\multicolumn{1}{c|}{Tower + Tower QE} & \multicolumn{1}{c|}{5 (25* for Tower QE)} & \multicolumn{1}{c|}{XCOMET-XL QE}            & \textbf{2.09} & \textbf{89.08} \\ \midrule
\multicolumn{5}{c}{\textit{Chinese $\rightarrow$ English}}                                                                                                         \\ \midrule
\multicolumn{1}{c|}{Tower}            & \multicolumn{1}{c|}{5}                    & \multicolumn{1}{c|}{\_}                      & 2.42          & 88.91          \\
\multicolumn{1}{c|}{Tower}            & \multicolumn{1}{c|}{25}                   & \multicolumn{1}{c|}{XCOMET-XL QE}            & 2.30          & 89.49          \\
\multicolumn{1}{c|}{Tower}            & \multicolumn{1}{c|}{25}                   & \multicolumn{1}{c|}{Tower QE} & 2.32          & 89.51          \\
\multicolumn{1}{c|}{Tower + Tower QE} & \multicolumn{1}{c|}{5 (25* for Tower QE)} & \multicolumn{1}{c|}{\_}                      & 2.26          & 89.82          \\
\multicolumn{1}{c|}{Tower + Tower QE} & \multicolumn{1}{c|}{5 (25* for Tower QE)} & \multicolumn{1}{c|}{XCOMET-XL QE}            & \textbf{2.24} & \textbf{90.00} \\ \bottomrule
\end{tabular}
}
\caption{Translation Quality on WMT23 English $\rightarrow$ German Test set. Both XCOMET and MetricX columns use reference for reporting translation quality where as XCOMET-XL QE does not use for re-ranking.}
\label{tab:qadecoding}
\end{table*}

\begin{table}[!ht]
\resizebox{\columnwidth}{!}{
\centering
\begin{tabular}{@{}c|ccc@{}}
\toprule
                                                                                      & Pearson        & Spearmann      & Kendall        \\ \midrule
COMETQE                                                                               & \textbf{44.41} & 41.29          & 31.19          \\ \midrule
COMETQE-XL                                                                            & 41.23          & \textbf{42.17} & \textbf{31.84} \\ \midrule
Tower Avg. Log Prob                                                                        & 32.32          & 16.74          & 12.77          \\ \midrule
\begin{tabular}[c]{@{}c@{}}Tower QE\end{tabular} & 40.56          & 33.96          & 25.87          \\ \bottomrule
\end{tabular}
}
\caption{Correlation on WMT 23 for English $\rightarrow$ German Test set. The scores are calculated after removing the few sentences labeled for hallucination detection. Best scores according to each coefficient are highlighted in \textbf{bold}.}
\label{tab:correlation}
\end{table}

\section{Experimental Setup}
\paragraph{Datasets:} We focus on two language directions given their availability of MQM data: English $\rightarrow$ German and Chinese $\rightarrow$ English. To train our token-level QE systems, we use the MQM datasets\footnote{https://github.com/google/wmt-mqm-human-evaluation} from WMT \citep{freitag2021experts}. Specifically, we use the datasets until 2022 for training, 2024 for validation, and 2023 for testing \citep{kocmi2024findings}. This setup is consistent with all the other QE metrics, and we do not use any additional data beyond these datasets.
\vspace{-0.1cm}
\paragraph{Models:} 
Our proposed approach achieves Quality-Aware decoding by combining an NMT model with a token-level QE model, where we adapt the same NMT for QE by adding a classification head. We use the state-of-the-art NMT model, Tower 7B\footnote{Unbabel/TowerInstruct-7B-v0.2} \citep{alves2024tower}, which provides high-quality translations and has already been exposed to MQM data during instruction-tuning. This ensures that the gains observed in our approach stem from integrating Quality-Aware decoding into the NMT process, rather than introducing new data. Additional details on training and hyper-parameters are provided in Appendix \ref{sec:training_detail}.
\vspace{-0.1cm}
\paragraph{Metrics:}
For reporting the translation quality, we consistently use XCOMET-XXL\footnote{Unbabel/XCOMET-XXL} \citep{guerreiro2024xcomet} and MetricX\footnote{google/metricx-24-hybrid-xl-v2p6} \citep{juraska2024metricx} \textbf{with the reference}. To compare with N-best list re-ranking, we use the XCOMET-XL QE\footnote{Unbabel/XCOMET-XL} \textbf{without the reference}. This approach allows us to avoid biasing toward a single metric during the re-ranking process and enables us to measure the gains achieved by differently trained metrics. 

\section{Results}



\begin{table*}[!ht]
\centering
\resizebox{2\columnwidth}{!}{
\begin{tabular}{@{}ccccc@{}}
\toprule
\multicolumn{1}{c|}{Model}            & \multicolumn{1}{c|}{Beams}                        & \multicolumn{1}{c|}{Re-ranking}               & MetricX ($\downarrow$) & XCOMET-XXL ($\uparrow$) \\ \midrule
\multicolumn{5}{c}{\textit{English $\rightarrow$ German}}                                                                                                          \\ \midrule
\multicolumn{1}{c|}{Tower}            & \multicolumn{1}{c|}{25}                           & \multicolumn{1}{c|}{XCOMET-XL QE}             & 2.37     & 87.79      \\
\multicolumn{1}{c|}{Tower}            & \multicolumn{1}{c|}{25}                           & \multicolumn{1}{c|}{Tower QE}         & 2.38     & 87.40      \\
\multicolumn{1}{c|}{Tower}            & \multicolumn{1}{c|}{25}                           & \multicolumn{1}{c|}{Tower Distill QE} & 2.38     & 87.39      \\
\multicolumn{1}{c|}{Tower + Tower QE} & \multicolumn{1}{c|}{5 (25* for Tower QE)}         & \multicolumn{1}{c|}{\_}                       & 2.12     & \textbf{88.95}      \\
\multicolumn{1}{c|}{Tower + Tower QE} & \multicolumn{1}{c|}{5 (25* for Tower Distill QE)} & \multicolumn{1}{c|}{\_}                       & \textbf{2.11}     & 88.76      \\ \bottomrule
\end{tabular}
}
\caption{Performance of Unidirectional QE trained with/without distillation on WMT23 English $\rightarrow$ German Test set. Best scores according to each metric are highlighted in \textbf{bold}.}
\label{tab:towerdistill}
\end{table*}


\begin{table*}[!ht]
\centering
\resizebox{2\columnwidth}{!}{
\begin{tabular}{@{}cccccc@{}}
\toprule
\multicolumn{1}{c|}{Model}            & \multicolumn{1}{c|}{Beams}                & \multicolumn{1}{c|}{Re-ranking}       & XCOMET-XL ($\uparrow$)     & \multicolumn{1}{c|}{XCOMET-XXL ($\uparrow$)}     & Impact                                                                                       \\ \midrule
\multicolumn{6}{c}{\textit{Paragraph-Level}}                                                                                                                                                                                                                                    \\ \midrule
\multicolumn{1}{c|}{Tower}            & \multicolumn{1}{c|}{25}                   & \multicolumn{1}{c|}{XCOMET-XL QE}     & \textbf{86.56} & \multicolumn{1}{c|}{87.79}          & \multirow{3}{*}{\begin{tabular}[c]{@{}c@{}}$\delta$ = + 1.16\\ (88.95 - 87.79)\end{tabular}} \\
\multicolumn{1}{c|}{Tower}            & \multicolumn{1}{c|}{25}                   & \multicolumn{1}{c|}{Tower QE} & 85.40          & \multicolumn{1}{c|}{87.40}          &                                                                                              \\
\multicolumn{1}{c|}{Tower + Tower QE} & \multicolumn{1}{c|}{5 (25* for Tower QE)} & \multicolumn{1}{c|}{\_}               & 86.36          & \multicolumn{1}{c|}{\textbf{88.95}} &                                                                                              \\ \midrule
\multicolumn{6}{c}{\textit{Sentence-Level}}                                                                                                                                                                                                                                     \\ \midrule
\multicolumn{1}{c|}{Tower}            & \multicolumn{1}{c|}{25}                   & \multicolumn{1}{c|}{XCOMET-XL QE}     & \textbf{86.42}          & \multicolumn{1}{c|}{87.68}          & \multirow{3}{*}{\begin{tabular}[c]{@{}c@{}}$\delta$ = + 0.38\\ (88.06 - 87.68)\end{tabular}} \\
\multicolumn{1}{c|}{Tower}            & \multicolumn{1}{c|}{25}                   & \multicolumn{1}{c|}{Tower QE} & 85.23          & \multicolumn{1}{c|}{87.41}          &                                                                                              \\
\multicolumn{1}{c|}{Tower + Tower QE} & \multicolumn{1}{c|}{5 (25* for Tower QE)} & \multicolumn{1}{c|}{\_}               & 85.96          & \multicolumn{1}{c|}{\textbf{88.06}}          &                                                                                              \\ \bottomrule
\end{tabular}
}
\caption{Impact of integrating Unidirectional QE during decoding with paragraphs vs sentences on WMT23 English $\rightarrow$ German Test set. $\delta$ denotes the improvement in translation quality from re-ranking N-best list with XCOMET-XL QE to integrating unidirectional Tower QE during the decoding. Best scores according to each metric are highlighted in \textbf{bold}.}
\label{tab:sentvspara}
\end{table*}



We conduct a series of experiments to validate the effectiveness of Quality-Aware decoding and identify the scenarios where it provides the most benefit. First, we evaluate whether our token-level QE model can better estimate sequence quality compared to the log probabilities of the NMT model. Next, we assess the impact of Quality-Aware decoding by comparing it with other approaches to determine if it improves translation quality. We also perform an ablation study to examine whether training the QE model on errors from the same NMT model enhances its performance. Finally, we explore the impact of source sentence length to highlight the limitations of N-best list re-ranking.

\subsection{Quality Estimation Performance}

First, we evaluate the agreement between the Tower-based token-level QE model (\textbf{Tower QE}) and human scores for a given hypothesis. It is only beneficial if we achieve higher correlation than the average of the NMT model log probabilities to show the need to integrate it during decoding. Therefore, we report the correlation with human scores of different models on WMT 23 English $\rightarrow$ German in Table \ref{tab:correlation}. 

We observe that the best-performing systems are the Comet QE models, which predict a single score using the full hypothesis. This is expected, as these models assess quality after the hypothesis is fully generated. In contrast, both log probabilities and Tower QE scores are based on the predicted token of each decoding step, using only the preceding context. Log probabilities perform poorly in this setup, while our proposed model, Tower QE, achieves twice the correlation with human judgments compared to log probabilities, despite scoring token by token with preceding context. This result highlights the potential of integrating our approach into the decoding process.

\subsection{Unified Decoding for NMT}


To validate the effectiveness of our unified decoding approach, we compare it with several baselines in Table \ref{tab:qadecoding}. First, we evaluate whether our approach outperforms generating translations with the NMT model alone. Next, we check if the quality of translations improves compared to N-best list re-ranking. To make the setups comparable, we set $topk$ and $num\_beams$ to $5$ and compare with re-ranking the top $25$ beams using XCOMET-XL. Finally, to demonstrate that re-ranking the N-best list remains a viable and complementary approach, we re-rank the top $5$ beams obtained from Quality-Aware decoding using the same QE model. 

We find that re-ranking with XCOMET-XL and Tower QE yields similar results, indicating that our partial QE model does not over-fit to any specific metric. Furthermore, we observe that the unified decoding approach outperforms N-best list re-ranking across both metrics in both language pairs. For example, the MetricX score improves from $2.37$ to $2.12$ for English $\rightarrow$ German. Note that Tower has already seen this data during instruction-tuning and the improvement is not from new data but from Quality-Aware decoding. Moreover, re-ranking the top $5$ beams obtained from unified decoding with XCOMET-XL leads to a slight further improvement in quality. This highlights the robustness and generalizability of our approach across different evaluation metrics.
%\footnotetext{\href{https://github.com/WMT-QE-Task/wmt-qe-2023-data}{WMT 23 English $\rightarrow$ German QE Data}}

\subsection{Adapting for Tower Errors}

We use the MQM annotations from WMT to train our Tower QE model, which contains error annotations from other systems. However, a viable alternative would be to adapt Tower QE specifically to the errors it typically makes. To maintain a similar data setup, we first generate translations using Tower on these source sentences. Then, we annotate the generated hypotheses with XCOMET-XL using the reference and fine-tune Tower QE on this synthetic dataset, which we refer to as \textbf{Tower Distill QE}. We evaluate the performance of the new distill QE model and report the results in Table \ref{tab:towerdistill}.

We observe that the distilled QE model performs very similarly to the QE model trained on errors from other systems. This indicates that there was no significant benefit in adapting the QE model to the specific errors typically made by Tower. However, further analysis on larger datasets and different domains is needed to fully validate the effectiveness of the distillation approach as the current synthetic data generated is small.

\subsection{Sentence vs Document-level Translation}

From Table \ref{tab:qadecoding}, we observe that the gains for English $\rightarrow$ German (paragraph-level) are much higher than for Chinese $\rightarrow$ English (sentence-level). We hypothesize that this discrepancy arises from the length of the sentences, as the N-best list re-ranking is likely sufficient for shorter sentences. To confirm this, we take the English paragraphs and split them into sentences using a tokenizer while tracking the paragraph IDs. We then perform the entire decoding process similarly, and later join the sentences back using the paragraph IDs before evaluation. We report the results in Table \ref{tab:sentvspara}.

We define the impact as the improvement in translation quality from re-ranking the N-best list with XCOMET-XL QE to integrating Tower QE. Comparing the results at the paragraph level to those at the sentence level, we observe that the impact decreases, which confirms our hypothesis. Additionally, we obtain better scores at the document level, further highlighting the potential benefits of Quality-Aware Decoding.

\section{Related Work}

\textbf{Integrating QE in NMT:} Several advancements have been made in improving QE for NMT over the years \citep{rei2021references, rei2022cometkiwi, blain2023findings, zerva2024findings, guerreiro2024xcomet}. These developments have led to the integration of QE in various ways.
One common approach involves applying QE after generating multiple sequences through techniques such as QE re-ranking \citep{fernandes2022quality, faria2024quest} or Minimum Bayes Risk (MBR) decoding \citep{tomani2024quality}. Another direction focuses on removing noisy data using QE models, followed by fine-tuning on high-quality data \citep{xu2024contrastive, finkelstein2024introducing}. \citet{vernikos2024don} proposes to generate diverse translations as a first step and then combine them. We perform this explicitly by integrating the QE directly into decoding.
Recently, \citet{zhang2024learning} exploited the MQM data by training models to penalize tokens within an error span, improving translation quality. In contrast, our approach adopts a modular framework, where we propose an expert QE model that is trained independently for targeted training. This modular approach aims to improve performance by decomposing the task into separate translation and QE components.

\textbf{Reward Modeling in NLG:}  Quality-Aware decoding shares several similarities with controllable text generation methods, particularly in the use of an additional "Quality/Reward" model that guides the decoding. A well-explored approach for controlling text is altering the decoding with a reward model (Weighted Decoding) \citep{yang2021fudge}. This method modifies the decoding by adjusting token probabilities based on the reward model, allowing for more controlled generation.
Similarly, \citet{deng-raffel-2023-reward} also used a uni-directional reward model, with the aim of maintaining efficiency during generation. This approach minimizes computational complexity while still benefiting from the guiding influence of the reward model. Moreover, recent work by \citet{li-etal-2024-reinforcement} introduced a token-level reinforcement learning-based reward model, providing more fine-grained feedback that enhances control over text generation at a granular level. While similar, the key contribution in our work lies in the development of the first uni-directional QE model for translation. 


\section{Conclusion}
We have shown the importance of Quality-Aware decoding to improve translation quality, rather than relying on post-generation techniques. In this work, we demonstrated how MQM data can be used to build a uni-directional token-level QE model, which is then integrated into the decoding process. Through a series of experiments, we showed that our Quality-Aware decoding approach results in measurable improvements in translation quality. Notably, we did not introduce new training data to the NMT model, and show that the gains stem from Quality-Aware decoding.


\section{Limitations}
While our Quality-Aware decoding improves translation quality, it adds considerable computational complexity to the inference process. Theoretically, this approach would double the time needed to generate a translation and require additional memory to utilize the token-level QE model. One potential solution to mitigate this issue could be to use token-level QE as a reward model for training via Reinforcement Learning.

Additionally, we trained our model on a limited set of human-annotated MQM data. However, current QE models, such as XCOMET, are capable of predicting error tags using the reference with reasonable quality. This suggests that further improvements could be achieved if these models were trained on larger-scale datasets, providing more nuanced feedback and refining translation quality even further.

Lastly, our proposed token-level QE model does not account for error severity. Ideally, it should be able to predict the category of errors, allowing for more nuanced feedback and enabling the model to generate translations with only minor errors when necessary.


% Bibliography entries for the entire Anthology, followed by custom entries
%\bibliography{anthology,custom}
% Custom bibliography entries only
\bibliography{custom}

\appendix

\section{Appendix}
\label{sec:appendix}

% \begin{table*}[!ht]
% \centering
% \begin{tabular}{@{}ccccc@{}}
% \toprule
% \multicolumn{1}{c|}{Model}            & \multicolumn{1}{c|}{Beams}                & \multicolumn{1}{c|}{Re-ranking}              & XCOMET-XL      & XCOMET-XXL     \\ \midrule
% \multicolumn{5}{c}{\textit{English $\rightarrow$ German}}                                                                                                          \\ \midrule
% \multicolumn{1}{c|}{Tower}            & \multicolumn{1}{c|}{5}           & \multicolumn{1}{c|}{\_}                      & 84.93          & 86.93          \\
% \multicolumn{1}{c|}{Tower}            & \multicolumn{1}{c|}{25}                   & \multicolumn{1}{c|}{\textbf{\_}}             & 84.87 & 86.45          \\
% \multicolumn{1}{c|}{Tower MBR}        & \multicolumn{1}{c|}{25}                   & \multicolumn{1}{c|}{\_}                      & 85.23          & 87.09          \\
% \multicolumn{1}{c|}{Tower}            & \multicolumn{1}{c|}{25}                   & \multicolumn{1}{c|}{XCOMET-XL QE}            & 86.56          & 87.79          \\
% \multicolumn{1}{c|}{Tower}            & \multicolumn{1}{c|}{5}                    & \multicolumn{1}{c|}{Tower QE} & 85.34          & 87.33          \\
% \multicolumn{1}{c|}{Tower}            & \multicolumn{1}{c|}{25}                   & \multicolumn{1}{c|}{Tower QE} & 85.40          & 87.40          \\
% \multicolumn{1}{c|}{Tower + Tower QE} & \multicolumn{1}{c|}{5 (25* for Tower QE)} & \multicolumn{1}{c|}{\_}                      & 86.36          & 88.95          \\
% \multicolumn{1}{c|}{Tower + Tower QE} & \multicolumn{1}{c|}{5 (25* for Tower QE)} & \multicolumn{1}{c|}{XCOMET-XL QE}            & \textbf{86.88} & \textbf{89.08} \\ \midrule
% \multicolumn{5}{c}{\textit{Chinese $\rightarrow$ English}}                                                                                                         \\ \midrule
% \multicolumn{1}{c|}{Tower}            & \multicolumn{1}{c|}{5}                    & \multicolumn{1}{c|}{\_}                      & 85.38          & 88.91          \\
% \multicolumn{1}{c|}{Tower}            & \multicolumn{1}{c|}{25}                   & \multicolumn{1}{c|}{\_}                      & 85.29          & 88.71          \\
% \multicolumn{1}{c|}{Tower MBR}        & \multicolumn{1}{c|}{25}                   & \multicolumn{1}{c|}{\_}                      & 86.00          & 89.23          \\
% \multicolumn{1}{c|}{Tower}            & \multicolumn{1}{c|}{25}                   & \multicolumn{1}{c|}{XCOMET-XL QE}            & 87.04          & 89.49          \\
% \multicolumn{1}{c|}{Tower}            & \multicolumn{1}{c|}{5}                    & \multicolumn{1}{c|}{Tower QE} & 85.64          & 89.10          \\
% \multicolumn{1}{c|}{Tower}            & \multicolumn{1}{c|}{25}                   & \multicolumn{1}{c|}{Tower QE} & 85.93          & 89.51          \\
% \multicolumn{1}{c|}{Tower + Tower QE} & \multicolumn{1}{c|}{5 (25* for Tower QE)} & \multicolumn{1}{c|}{\_}                      & 86.01          & 89.82          \\
% \multicolumn{1}{c|}{Tower + Tower QE} & \multicolumn{1}{c|}{5 (25* for Tower QE)} & \multicolumn{1}{c|}{XCOMET-XL QE}            & \textbf{86.67} & \textbf{90.00} \\ \bottomrule
% \end{tabular}
% \caption{COMET scores on WMT23 English $\rightarrow$ German Test set. Both XCOMET metric columns use reference for reporting translation quality and do not when used for re-ranking }
% \end{table*}


\subsection{Training details}
\label{sec:training_detail}

We use the transformers library \citep{wolf-etal-2020-transformers} for training and inference with Tower-Instruct V2.  For adapting Tower to token-level QE, we use LoRA \citep{hulora} based fine-tuning with an additional classifier head. Therefore, we only train the adapters and the weights for classification head.

We add the adapters to the modules \textit{q\_proj,k\_proj,v\_proj,gate\_proj,up\_proj} and \textit{down\_proj}. We set a batch size for each device to 12 initially and enable \textit{auto\_find\_batch\_size} to \textit{True} on 4 NVIDIA RTX A6000 GPU's. For having a  larger batch size during training, we set \textit{gradient\_accumulation\_steps} to 6. We use a \textit{learning\_rate} of $1e^{-5}$. We set the \textit{eval\_steps} to $50$ and \textit{num\_train\_epochs} to $10$. The other parameters are set to default.

Using the cross-entropy loss for token-level QE directly is insufficient due to the fact that the majority of tokens are classified as '\textit{Good}'. Hence, we find that the weighted cross-entropy loss is essential when fine-tuning the model. For the training on human MQM data, we set the weights to $0.05,0.95$ to '\textit{Good}' and '\textit{Bad}' labels respectively. In the case of distilling from XCOMET, we observed more errors. Therefore, we find that setting them $0.2,0.8$ to '\textit{Good}' and '\textit{Bad}' labels respectively provided stable training.

We train on data until WMT'22 for training and use WMT'24 for validation. We calculate the macro '\textit{F1}' on token-level predictions as the validation metric and stop training if it does not improve for 10 consecutive \textit{eval\_steps}.

\subsection{Partial vs Full Sequence Quality Estimation}

We also compare the difference in performance between our proposed token-level QE for partial sequences with Tower trained for full sequence QE. We achieve this by adding a regression head to predict the score at the end-of-sentence token. Hence, the model uses the source and hypothesis to predict the score using regression head at the end.

We fine-tune the model using only direct assesment data \citep{zerva2024findings} (\textbf{Tower Full DA}). Furthermore, we use this as initialisation and continue fine-tuning on the MQM data (\textbf{Tower Full DA + MQM}). We also use LoRA similarly to the previous model with a regression head to adapt the model. We report the scores in Table \ref{tab:correlation_ablation}.

We see that the both Tower QE models based on full sentences outperforms the partial model. However, this is expected as it has seen the entire context and was also trained on larger amounts of data. Nonetheless, the partial model still achieves much higher correlaiton that the log probabilities showcasing its potential for Quality-Aware decoding.

\subsection{Robustness to re-ranking weight}

In our method, we introduce a hyperparameter, $\alpha$, to merge the probabilities from the token-level QE model and the translation model. This section analyzes the impact of $\alpha$ on the final translation quality.

To efficiently evaluate its effect, we re-rank the N-best list using different values of $\alpha$. This approach allows us to estimate the ideal value of $\alpha$ without the need for joint decoding multiple times. If the re-ranking model (in this case, Tower QE) is beneficial, we expect that any $\alpha$ less than 1 will improve translation quality, as it demonstrates that incorporating the probabilities from the QE model is helpful.

We visualize this impact in Figure \ref{fig:mainfigure}. The results show that using an $\alpha$ less than 1 leads to improved translation quality in both scenarios. This indicates that relying entirely on the NMT model does not yield the best results and highlights the importance of the Tower QE model.

Thus, we emphasize that re-ranking the N-best list provides an effective way to tune the value of $\alpha$, and it remains robust to different values.

\begin{figure*}[!htpb]
\begin{promptbox}[title={Tower Translation Prompt}]
    \small
    <|im\_start|>user\\
    Translate the sentence from English into German.\\
    English: \{src\_sent\}\\
    German:\\
    <|im\_end|>\\
    <|im\_start|>assistant
\end{promptbox}

\begin{promptbox}[title={Tower Token-Level QE Prompt}]
    \small
    English:\{src\_sent\}\\
    German: \{tgt\_sent\}
\end{promptbox}
\caption{Prompts used in our experiments for translation and QE model. \{src\_sent\} and \{tgt\_sent\} represent the source and target sentence. We replace the language with Chinese and English when experimenting with that language pair.}
\end{figure*}

\begin{figure*}[!htpb]
    \centering
    % First subfigure
    \begin{subfigure}[b]{0.5\textwidth}
        \centering
        \includegraphics[width=\textwidth]{Figures/alphas_ende_25.png} % Replace with your image path
        \caption{English $\rightarrow$ German}
        \label{fig:subfigure1}
    \end{subfigure}
    
    \vspace{0.5cm} % Adjust space between the two subfigures

    % Second subfigure
    \begin{subfigure}[b]{0.5\textwidth}
        \centering
        \includegraphics[width=\textwidth]{Figures/alphas_zhen_25.png} % Replace with your image path
        \caption{Chinese $\rightarrow$ English}
        \label{fig:subfigure2}
    \end{subfigure}
    
    \caption{Impact of $\alpha$ when re-ranking with token-level Tower QE on WMT'23 Test sets.}
    \label{fig:mainfigure}
\end{figure*}


\begin{table*}[!ht]
\centering
\begin{tabular}{@{}c|ccc@{}}
\toprule
                                                                                      & Pearson        & Spearmann      & Kendall        \\ \midrule
COMETQE                                                                               & \textbf{44.41} & 41.29          & 31.19          \\ \midrule
COMETQE-XL                                                                            & 41.23          & \textbf{42.17} & \textbf{31.84} \\ \midrule
\begin{tabular}[c]{@{}c@{}}COMETQE Scratch\\      Fine-tuned (ours)\end{tabular}      & 36.32          & 33.66          & 25.24          \\ \midrule
Tower Log Prob                                                                        & 32.32          & 16.74          & 12.77          \\ \midrule
\begin{tabular}[c]{@{}c@{}}Tower Partial QE\end{tabular} & 40.56          & 33.96          & 25.87          \\ \midrule
Tower Full DA                                                                        & 33.67          & 36.46          & 27.38          \\ \midrule
Tower Full DA + MQM                                                                 & 32.03          & 40.85          & 30.38          \\ \bottomrule
\end{tabular}
\caption{Full Correlation results on WMT 23 for English $\rightarrow$ German Test set. Partial indicates that the QE model predict scores via token-level where as full indicates predicting the score at the end-of-sentence token. The scores are calculated after removing the few sentences labelled for hallucination detection. Best scores according to each coefficient are highlighted in \textbf{bold}.}
\label{tab:correlation_ablation}
\end{table*}




\end{document}



\clearpage

\appendix
\label{sec:appendix}

\section{Dataset}
\label{app:dataset}

In this section, we describe the details of the construction of our dataset which includes five distinct story-writing sources, Reddit\footnote{\url{https://www.reddit.com/r/WritingPrompts/}}, AO3\footnote{\url{https://archiveofourown.org/}}, Storium\footnote{\url{https://storium.cs.umass.edu/}}, N.Magazine\footnote{\url{https://www.narrativemagazine.com/}}, and New Yorker\footnote{\url{https://www.newyorker.com/}}. 

\paragraph{Selection Constraints:}
To ensure high-quality and diverse content, we apply several constraints during dataset collection. Stories are limited to a length of 500 to 1500 words. To avoid contamination by LLM pretraining data, only recent stories published after November 2023 are included for Reddit and AO3 \citep{zhou2023don, magar-schwartz-2022-data}. Not-Safe-For-Work (NSFW) and explicit content is excluded using automatic tagging and manual verification. Additionally, we manually inspect all stories to remove elements that may reveal author identities, such as URLs and links to their public profiles on websites. 

\paragraph{Enrichment with Writing Prompts:}
While Reddit stories include author-provided prompts, other sources do not; therefore, we augment them with GPT-4o-generated writing prompts to standardize the format across sources. We manually review all generated prompts and refine them when necessary. The prompt for generating writing prompts for the stories in our dataset can be found in Figure~\ref{fig:writing-prompt-gen}.

\paragraph{Dataset Splitting:}
Following \citep{salemi-etal-2024-lamp}, we split each author's stories chronologically based on their submission timestamps. The first 70\% of an author’s stories form the \emph{profiling} set, representing their historical writing, while the remaining 30\% constitute the \emph{generation} set, used for evaluating personalization methods.

\paragraph{Statistics and Comparisons:}
The dataset contains 590 stories from 64 authors, with an average story length of 1500 tokens. Detailed statistics are provided in Table~\ref{tab:dataset-stats}. As shown in Table~\ref{tab:compare-datasets}, our dataset uniquely combines diverse story-writing settings and provides connections that link stories written by the same author, distinguishing it from existing story-writing datasets that either lack Author IDs or do not establish such links.

\paragraph{Themes:}
Figure~\ref{fig:wp_themes} shows the plots depicting the percentage-normalized frequency of the top 10 themes for each source. To derive these themes, we prompt GPT-4o with each writing prompt to generate up to five one-word themes. We then aggregate the themes across all writing prompts and compute the normalized frequency by dividing the count of each theme by the total number of themes for that source.


Overall, we observe that Reddit (Figure~\ref{fig:reddit_wp}), AO3 (Figure~\ref{fig:ao3_wp}), and Storium (Figure~\ref{fig:storium_wp}) contain a higher prevalence of ``conflict'' themes, whereas N.Magazine (Figure~\ref{fig:nmagazine_wp}) and New Yorker (Figure~\ref{fig:newyorker_wp}) focus more on family, loss, and identity. This distinction reflects the nature of story-writing settings across these sources: amateur stories tend to emphasize direct conflict resolution in their prompts, while expert-written stories explore deeper aspects of human relationships. 

Among Reddit, AO3, and Storium, Reddit exhibits a flatter distribution with greater thematic diversity, spanning identity, power, survival, humor, adventure, magic, and the supernatural. In contrast, AO3 has a sharper distribution, dominated by recurring themes of relationships, friendships, love, family, and secrets, reflecting a more domestic story-writing setting. Storium also shows a sharper distribution but favors themes such as space, danger, leadership, and war, highlighting a preference for war-centric narratives in collaborative storytelling.


\begin{figure*}[htbp]
\centering
\begin{tcolorbox}[colback=gray!5!white, colframe=black, title=Prompt for Writing Prompt Generation]

\section*{System Prompt}  
You are a creative writing assistant skilled in crafting engaging and imaginative writing prompts. Your task is to analyze a provided story and create a concise, compelling prompt that fulfills the provided constraints.

\vspace{1em}
\section*{User Prompt}  
\begin{itemize}[noitemsep, topsep=0pt]
    \item Style Consistency: Match the style of few-shot demonstrations.
    \item Length: Keep between 1-2 sentences.
    \item Content: Reflect key story elements (premise, characters, conflict) while fostering creativity.
    \item Fictional Characters: If mentioned in the story, include them where relevant.
\end{itemize}

\vspace{0.5em}
\textbf{Guidelines}  
\begin{itemize}[noitemsep, topsep=0pt]
    \item Ignite curiosity while leaving space for interpretation.
    \item Maintain tone and structure consistency with examples.
    \item Ensure prompts are open-ended and evocative, avoiding excessive specificity.
    \item Keep prompts simple, concise, and adaptable to diverse responses.
    \item Avoid step-by-step directions; inspire rather than instruct.
    \item Encourage exploration with broad, thought-provoking scenarios.
    \item Strive for uniqueness and memorability.
\end{itemize}

\vspace{0.5em}
\textbf{Goal}  
Generate prompts that inspire diverse, unexpected, and imaginative narratives while maintaining consistency in tone and style. Each prompt should serve as an inviting starting point rather than a directive.

\vspace{0.5em}
\textbf{Notes}  
\begin{itemize}[noitemsep, topsep=0pt]
    \item Inspire creativity while allowing the writer to shape the journey.
    \item Balance being suggestive yet open-ended to encourage interpretation.
    \item Include fictional characters mentioned in the story to preserve context.
\end{itemize}

\vspace{0.5em}
\textbf{Few-Shot Examples}
We include few-shot examples here.

\end{tcolorbox}

\caption{Prompt for generating writing prompts for the stories in our dataset.}
\label{fig:writing-prompt-gen}
\end{figure*}


\begin{table}[htbp]
\centering
\caption{Dataset statistics, including the total number of stories, profiling (Profile) and generation (Gen) set distribution, number of authors, average stories per author, and token length in words.}
\label{tab:dataset-stats}
\small
\begin{tabular}{p{0.18\linewidth} p{0.08\linewidth} p{0.08\linewidth} p{0.08\linewidth} p{0.08\linewidth} p{0.09\linewidth} p{0.08\linewidth}}
\toprule
Dataset            & Total \newline Stories & Profile Size & Gen Size & Total \newline Authors & Stories/ \newline Author & Token Length  \\
\midrule
AO3                & 239 & 159 & 80 & 17 & 14 & 1220 \\
Reddit             & 170 & 113 & 57 & 12 & 14 & 1210 \\
Storium            & 111 & 71  & 40 & 15 & 7  & 1120 \\
N.Magazine         & 32  & 18  & 14 & 10 & 3  & 1745 \\
NewYorker          & 38  & 22  & 16 & 10 & 4  & 1950 \\
\midrule
Overall            & 590 & 383 & 207 & 64 & 8  & 1517 \\
\bottomrule
\end{tabular}
\end{table}

\begin{figure*}[htbp]
    \centering
    % First row (2 figures)
    \begin{subfigure}{0.48\linewidth}
        \centering
        \includegraphics[width=\linewidth]{figures/wp_themes/Reddit.png}
        \caption{Reddit}
        \label{fig:reddit_wp}
    \end{subfigure}
    \hfill
    \begin{subfigure}{0.48\linewidth}
        \centering
        \includegraphics[width=\linewidth]{figures/wp_themes/AO3.png}
        \caption{AO3}
        \label{fig:ao3_wp}
    \end{subfigure}

    \vspace{1em} % Space between rows

    % Second row (2 figures)
    \begin{subfigure}{0.48\linewidth}
        \centering
        \includegraphics[width=\linewidth]{figures/wp_themes/narrativemagazine.png}
        \caption{N.Magazine}
        \label{fig:nmagazine_wp}
    \end{subfigure}
    \hfill
    \begin{subfigure}{0.48\linewidth}
        \centering
        \includegraphics[width=\linewidth]{figures/wp_themes/newyorker.png}
        \caption{New Yorker}
        \label{fig:newyorker_wp}
    \end{subfigure}

    \vspace{1em} % Space between rows

    % Third row (1 full-width figure)
    \begin{subfigure}{0.48\linewidth}
        \centering
        \includegraphics[width=\linewidth]{figures/wp_themes/Storium.png}
        \caption{Storium}
        \label{fig:storium_wp}
    \end{subfigure}

    \caption{Source-wise Percentage Normalized Frequency of the top 10 themes.}
    \label{fig:wp_themes}
\end{figure*}


\section{Author Writing Sheet}

Algorithm~\ref{alg:author_writing_sheet} describes the process for generating the Author Writing Sheet. 

\paragraph{Common Core Standards in English Language Arts:}  

The Author Writing Sheet is organized in the form of Claim-Evidence pairs describing an author's story-writing characteristics inspired by Common Core (CC) Standards\footnote{\url{https://corestandards.org/wp-content/uploads/2023/09/ELA_Standards1.pdf}} \citep{national2010common}. Specifically, it aligns with \emph{RL-9-10.1}, which emphasizes citing strong textual evidence to support analysis, and \emph{RL-9-10.2}, which focuses on determining central themes and summarizing texts. Additionally, \emph{RL-9-10.4} highlights interpreting word choices and their impact on meaning and tone. In writing, \emph{W.9-10.2} pertains to producing clear and well-structured informative texts, while \emph{W.9-10.9} encourages drawing evidence from literary and informational texts to support analysis and research. These standards provide a structured framework for evaluating narrative elements such as plot, creativity, development, and language use within the Author Writing Sheet, ensuring a systematic and interpretable representation of an author’s unique storytelling style.


\begin{algorithm}[tb]
\DontPrintSemicolon
\SetAlgoLined
\textbf{Input:} profiling set \( P = \{(wp_t, s_{at}) \mid t = 1, \ldots, |P| \} \),\newline 
where \( wp_t \) is the writing prompt and \( s_{at} \) is the author's story at time-step \( t \).\\
\textbf{Output:} Author Writing Sheet \( A_{|P|} \)\\

\tcp{Initialize Author Writing Sheet}
\( A_0 \gets \emptyset \)

% \tcp{total size of the profiling set.}
% \( S_p \gets |P| \)

\For{\( t = 1 \) to \(|P| \)}{
    \tcp{average story}
    \( s_{bt} \gets \text{LLM\textsubscript{avg}}(wp_t) \)
    
    \tcp{Intermediate Author Writing Sheet}
    \( A_t' \gets \text{LLM\textsubscript{sheet}}(wp_t, s_{bt}, s_{at}) \)
    
    \tcp{Author Writing Sheet ($A_t$)}
    \( A_t \gets \text{LLM\textsubscript{combine}}(A_t', A_{t-1}) \)
    
    \tcp{Steps within combination:}
    \Indp
    Group equivalent Claims in \( A_t' \cup A_{t-1} \) and select the best Evidence for each group.\;
    Include ungrouped Claims from \( A_t' \) and \( A_{t-1} \) with their Evidence as-is.\;
    \Indm
}

\tcp{Final Author Writing Sheet.}
\Return \( A_{|P|} \)
\caption{Constructing the Author Writing Sheet from a profiling set of stories}
\label{alg:author_writing_sheet}
\end{algorithm}


\begin{figure*}[htbp]
\centering
\begin{tcolorbox}[colback=gray!5!white, colframe=black, title=Average Author Prompt for AO3 Generation]

\section*{System Prompt}  
You are a creative and engaged fanfiction writer, skilled in capturing the emotional depth, creativity, and character-driven storytelling that defines AO3 fanworks. Your goal is to write a compelling fanfiction narrative in response to the provided writing prompt. Embrace the transformative nature of fanfiction by reimagining canonical characters, events, or settings to explore new perspectives or emotional arcs. Focus on creating a story that resonates emotionally, respects the fandom's dynamics, and celebrates the collaborative and imaginative spirit of AO3.

\vspace{1em}
\section*{User Prompt}  

\textbf{Context of Writers}  
\begin{itemize}[noitemsep, topsep=0pt]
    \item Assume the author is an engaged and creative fanfiction writer, deeply familiar with the fandom and its dynamics.
    \item Writers often experiment with established tropes, unconventional pairings, or alternative universes (AUs) while maintaining a deep respect for the source material.
    \item Emulate the enthusiastic and emotionally rich style characteristic of fanfiction authors, blending canon with transformative elements to craft original, resonant narratives.
\end{itemize}

\vspace{0.5em}
\textbf{Stylistic Constraints}  
\begin{itemize}[noitemsep, topsep=0pt]
    \item \textbf{Fandom Tone and Style}: Incorporate a tone and style that reflect the spirit of the fandom, blending humor, drama, and introspection in a way that resonates with fanfiction readers.
    \item \textbf{Creative Use of Tags}: Make creative use of AO3's hallmark tagging system in the text (e.g., playful or meta references in dialogue that nod to fandom tropes or subgenres).
    \item \textbf{Balanced Dialogue and Prose}: Include dialogue and prose that balance heartfelt sincerity with occasional self-aware humor or meta-commentary, in line with fanfic traditions.
\end{itemize}

\vspace{0.5em}
\textbf{Semantic Constraints}  
\begin{itemize}[noitemsep, topsep=0pt]
    \item \textbf{Focus on Relationships}: Emphasize emotional bonds and character growth, whether through conflict, reconciliation, or celebration.
    \item \textbf{Transform Canonical Elements}: Explore canonical elements with a transformative twist (e.g., reinterpreting events, relationships, or character motivations from a new perspective).
    \item \textbf{Ground in Established Lore}: Ground the narrative in a specific fandom's established lore while allowing space for imaginative deviations or additions.
\end{itemize}

\end{tcolorbox}
\caption{Average Author Prompt for AO3.}
\label{fig:ao3_avg_prompt}
\end{figure*}

\begin{figure*}[htbp]
\centering
\begin{tcolorbox}[colback=gray!5!white, colframe=black, title=Average Author Prompt for Reddit Generation]

\section*{System Prompt}  
You are a creative and enthusiastic storyteller, skilled in crafting imaginative and engaging short stories inspired by Reddit Writing Prompts (r/WritingPrompts). Your goal is to respond to the provided writing prompt by creating a story that is thought-provoking and conversational in tone, resonating with the online community. Use vivid descriptions, dynamic pacing, and approachable language to draw readers into the narrative. Ensure the story invites discussion and inspires others to explore the concept further.

\vspace{1em}
\section*{User Prompt}  

\textbf{Context of Writers}  
\begin{itemize}[noitemsep, topsep=0pt]
    \item Assume the author is an imaginative and enthusiastic storyteller who enjoys engaging directly with an online community of readers.
    \item Writers often experiment with bold, original ideas or explore twists on familiar concepts, showcasing their creativity and ability to captivate a diverse audience.
    \item Emulate the informal yet polished style common in \textit{r/WritingPrompts}, blending accessibility with a strong sense of storytelling craft.
\end{itemize}

\vspace{0.5em}
\textbf{Stylistic Constraints}  
\begin{itemize}[noitemsep, topsep=0pt]
    \item \textbf{Conversational and Approachable Tone}: Maintain a conversational and approachable tone typical of Reddit Writing Prompts.
    \item \textbf{Balanced Descriptive Passages}: Balance descriptive passages with dialogue or internal monologue to keep the pacing engaging.
    \item \textbf{Direct and Vivid Language}: Avoid overly complex language; keep the style direct but vivid.
    \item \textbf{Reinforcement of Ideas}: Employ narrative devices like repetition or callbacks to reinforce central ideas or themes.
\end{itemize}

\vspace{0.5em}
\textbf{Semantic Constraints}  
\begin{itemize}[noitemsep, topsep=0pt]
    \item \textbf{Alignment with the Prompt}: Ensure the story directly aligns with and explores the central theme or scenario of the writing prompt.
    \item \textbf{Cohesive Narrative Development}: Build a clear, cohesive narrative that develops the implications of the prompt's concept.
    \item \textbf{Immersive Sensory Details}: Use immersive sensory details to enrich the reader’s understanding of the protagonist's experiences and environment.
    \item \textbf{Open-Ended or Reflective Conclusion}: Conclude with an open-ended, reflective, or impactful note, leaving space for interpretation or further thought.
\end{itemize}

\end{tcolorbox}
\caption{Average Author Prompt for Reddit.}
\label{fig:reddit_avg_prompt}
\end{figure*}

\begin{figure*}[htbp]
\centering
\begin{tcolorbox}[colback=gray!5!white, colframe=black, title=Average Author Prompt for Storium]

\section*{System Prompt}  
You are a skilled and collaborative storyteller, adept at crafting vivid and engaging opening scenes for Storium. Your goal is to create an immersive \textbf{Establishment} turn in response to the provided writing prompt. Set the stage for the story by establishing a richly detailed context, evoking emotional resonance, and introducing narrative intrigue. Ensure the scene provides a strong foundation while leaving space for other contributors to expand and build upon the narrative. Balance descriptive detail with open-ended elements to encourage creativity and collaborative storytelling.

\vspace{1em}
\section*{User Prompt}  

\textbf{Context of Writers}  
\begin{itemize}[noitemsep, topsep=0pt]
    \item Assume the author is a collaborative storyteller skilled in creating vivid, open-ended scenes designed to inspire and engage other contributors.
    \item Writers often set the tone for the story while leaving space for co-authors to introduce their own ideas, characters, and plot developments.
    \item Emulate the inclusive, immersive style typical of \textit{Storium} story writing platform, where the opening turn encourages creativity and further contributions.
\end{itemize}

\vspace{0.5em}
\textbf{Stylistic Constraints}  
\begin{itemize}[noitemsep, topsep=0pt]
    \item \textbf{Set the Tone Appropriately}: Match the tone of the narrative to the writing prompt, whether it be adventurous, mysterious, or foreboding, using a consistent and engaging voice throughout.
    \item \textbf{Rich Descriptive Detail}: Employ vivid, sensory descriptions to establish the setting, characters, and atmosphere, enabling readers to visualize and immerse themselves in the story world.
    \item \textbf{Dynamic Sentence Structure}: Vary sentence lengths to reflect the pace and mood, using longer, flowing sentences for descriptions and shorter, punchy sentences for action or tension.
    \item \textbf{Establish Ambiguity or Suspense}: Drop subtle hints or unanswered questions to create intrigue and encourage curiosity about what happens next.
\end{itemize}

\vspace{0.5em}
\textbf{Semantic Constraints}  
\begin{itemize}[noitemsep, topsep=0pt]
    \item \textbf{Introduce the Setting}: Provide a clear depiction of the setting, whether it is a small trading post, a desert town, or a spaceship, and ensure its relevance to the writing prompt.
    \item \textbf{Outline the Context}: Clearly establish the circumstances that have led to the current scenario, including significant events or motivations.
    \item \textbf{Define Key Characters}: Introduce at least one or two central characters, highlighting distinctive traits or roles that will be important in the unfolding story.
    \item \textbf{Foreshadow the Central Conflict}: Allude to the main challenges or stakes introduced by the writing prompt.
\end{itemize}

\end{tcolorbox}
\caption{Average Author Prompt for Storium .}
\label{fig:storium_avg_prompt}
\end{figure*}

\begin{figure*}[htbp]
\centering
\begin{tcolorbox}[colback=gray!5!white, colframe=black, title=Average Author Prompt for Narrative Magazine]

\section*{System Prompt}  
You are an experienced and reflective writer, skilled in creating deeply personal and character-driven narratives in the style of Narrative Magazine. Your goal is to write a short story in response to the provided writing prompt, crafting a compelling and immersive piece. Focus on balancing introspection with vivid external details, and explore universal themes through the lens of individual experiences. Emphasize emotional resonance and thoughtful storytelling, ensuring the narrative engages readers with its depth and relatability.

\vspace{1em}
\section*{User Prompt}  

\textbf{Context of Writers}  
\begin{itemize}[noitemsep, topsep=0pt]
    \item Assume the author is an experienced writer skilled in creating rich, engaging narratives that weave together character introspection, dialogue, and evocative settings.
    \item Emulate the style of contributors to \textit{Narrative Magazine}, who bring diverse storytelling techniques and voices to explore themes of identity, memory, conflict, and resilience.
\end{itemize}

\vspace{0.5em}
\textbf{Stylistic Constraints}  
\begin{itemize}[noitemsep, topsep=0pt]
    \item \textbf{Prompt as Foundation}: Anchor the story firmly in the writing prompt, using it to drive the plot and the protagonist’s emotional arc.
    \item \textbf{Vivid Prose}: Use detailed descriptions to paint a clear picture of characters, settings, and actions while maintaining a natural flow.
    \item \textbf{Dynamic Characters}: Develop multi-dimensional characters with distinct voices and perspectives, revealed through dialogue, actions, and subtle internal reflections.
    \item \textbf{Balancing Action and Reflection}: Combine active plot progression with moments of introspection to create a layered, engaging narrative.
    \item \textbf{Realistic Dialogue}: Write dialogue that feels authentic and contributes to the development of characters and the story’s themes.
\end{itemize}

\vspace{0.5em}
\textbf{Semantic Constraints}  
\begin{itemize}[noitemsep, topsep=0pt]
    \item \textbf{Specific and Relatable Setting}: Choose a setting that feels specific yet relatable, whether a small town, an urban street corner, or a domestic space, grounding the reader in the protagonist’s world.
    \item \textbf{Exploration of Themes}: Build a narrative arc that explores themes of connection, discovery, or transformation, tying them back to the writing prompt in meaningful ways.
    \item \textbf{Rich Sensory Details}: Infuse the story with sensory details that make the setting and characters come alive, from the sounds of a bustling street to the quiet tension of a conversation.
    \item \textbf{Accessible Storytelling}: Avoid overly complex or abstract storytelling; ensure the narrative is accessible while leaving room for deeper interpretation.
\end{itemize}

\end{tcolorbox}
\caption{Average Author Prompt for Narrative Magazine.}
\label{fig:nmagazine_avg_prompt}
\end{figure*}

\begin{figure*}[htbp]
\centering
\begin{tcolorbox}[colback=gray!5!white, colframe=black, title=Average Author Prompt for New Yorker]

\section*{System Prompt}  
You are an accomplished and literary writer, skilled in crafting nuanced and thought-provoking short fiction in the style of The New Yorker. Your goal is to write a short story in response to the provided writing prompt, focusing on the hallmarks of The New Yorker fiction: rich emotional layers, nuanced character development, and a refined, literary prose style. Emphasize subtlety and depth in your storytelling, using symbolic elements and understated resolutions to evoke reflection and emotional resonance in the reader.

\vspace{1em}
\section*{User Prompt}  

\textbf{Context of Writers}  
\begin{itemize}[noitemsep, topsep=0pt]
    \item Assume the author is an experienced and skilled writer, capable of exploring complex human experiences through subtle, layered storytelling.
    \item Emulate the style of well-regarded \textit{New Yorker} contributors like Alice Munro, Haruki Murakami, or Raymond Carver, who excel in revealing depth through simplicity or ambiguity.
\end{itemize}

\vspace{0.5em}
\textbf{Stylistic Constraints}  
\begin{itemize}[noitemsep, topsep=0pt]
    \item \textbf{Engagement with the Prompt}: Respond directly to the writing prompt, ensuring the core premise drives the narrative.
    \item \textbf{Elegant Prose}: Use carefully crafted, precise language that balances sophistication with clarity.
    \item \textbf{Character-Driven Narratives}: Focus on character psychology, revealing emotional states through indirect actions, dialogue, or internal reflection.
    \item \textbf{Ambiguity and Subtlety}: Avoid explicit resolutions or explanations; allow readers to infer the meaning of events and relationships.
    \item \textbf{Symbolic Layers}: Incorporate elements from the prompt as symbols that evolve in significance throughout the story.
\end{itemize}

\vspace{0.5em}
\textbf{Semantic Constraints}  
\begin{itemize}[noitemsep, topsep=0pt]
    \item \textbf{Realistic and Detailed Environment}: Set the story in a realistic, detailed environment, using sensory descriptions to ground readers in the protagonist's world.
    \item \textbf{Internal or Interpersonal Conflict}: Introduce a central conflict or emotional tension that reflects internal or interpersonal struggles rather than overt, external drama.
    \item \textbf{Quiet but Profound Interactions}: Develop moments of quiet yet profound interaction between characters, often revealing deeper truths or contradictions.
    \item \textbf{Universal Themes}: Address universal themes like transition, isolation, or self-realization, tying them subtly back to the writing prompt.
\end{itemize}

\end{tcolorbox}
\caption{Average Author Prompt for New Yorker.}
\label{fig:newyorker_avg_prompt}
\end{figure*}


\begin{figure*}[htbp]
\centering
\begin{tcolorbox}[colback=gray!5!white, colframe=black, title=Prompt for generating the Intermediate Author Writing Sheet]

\section*{System Prompt}  
You are a sophisticated story analyst tasked with analyzing an author’s story writing style by contrasting an author-written story with a base story, both written in response to the same writing prompt. Your goal is to identify and evaluate the unique elements and tendencies in the author’s writing behavior. This analysis must focus on the distinctive ways the author interprets the writing prompt and shapes their narrative, as revealed through contrast with the base story.

Your analysis should also adhere to the Common Core Standards in English Language Arts, focusing on key skills such as analyzing textual evidence, evaluating an author’s craft and structure, and assessing how stylistic choices influence meaning and tone. While grounded in these standards, your evaluation must highlight the specific, unique aspects of the author’s writing style, including their recurring techniques, narrative preferences, and stylistic quirks. Your analysis should reflect close reading and objective interpretation, capturing the author's creative and stylistic distinctiveness in relation to the base story.

\vspace{1em}
\section*{User Prompt}  

\textbf{Input Details}  
Writing Prompt | Author-Written Story | Base Story  

\vspace{0.5em}
\textbf{Output Format}  
Use \texttt{<thinking></thinking>} tokens for reasoning and summarization | Use \texttt{<writing\_style></writing\_style>} tokens for structured analysis | Structure analysis by categories | Each category contains independent claims supported by contextualized evidence  

\vspace{0.5em}
\textbf{Guidelines for Claims}  
Claims must reflect broad patterns in the author’s style | Avoid repetition across categories | Maintain objectivity (do not reference "Author-Written Story" or "Base Story") | Ensure clarity and precision in claims  

\vspace{0.5em}
\textbf{Guidelines for Evidence}  
Draw evidence directly from the author-written story | Frame evidence using a descriptive phrase summarizing the writing prompt | Ensure coherence and logical alignment with the claim | Avoid over-extrapolation  

\vspace{0.5em}
\textbf{Categories for Analysis}  
Plot | Creativity | Development (Character and Setting) | Language Use  

\vspace{0.5em}
\textbf{Special Instructions}  
Generate a short descriptive phrase summarizing the writing prompt | Use \texttt{<thinking></thinking>} for reasoning and prompt framing | Structure output strictly within \texttt{<writing\_style></writing\_style>} tokens | Ensure uniqueness and non-redundancy of claims  

\vspace{0.5em}
\textbf{Sample Output Structure}  

\begin{verbatim}
<thinking>
Deeply reason on how the Author-Written Story differs from the Base Story. 
Think of a short descriptive phrase summarizing the prompt: 
"the story regarding a battle for lost artifacts".
</thinking>

<writing_style>
### Plot
1. **Claim about author’s writing style.**
   - Evidence: In the story regarding “writing prompt,” story excerpt
...
Repeat for all categories.
</writing_style>
\end{verbatim}

\end{tcolorbox}
\caption{Prompt for generating the Intermediate Author Writing Sheet.}
\label{fig:inter_author_sheet_prompt}
\end{figure*}

\begin{figure*}[htbp]
\centering
\begin{tcolorbox}[colback=gray!5!white, colframe=black, title=Prompt for generating the Combined Author Writing Sheet]

\section*{System Prompt}  
You are a sophisticated story analyst tasked with synthesizing **Author Writing Sheets** from multiple stories written by a single author into a cohesive **Combined Author Writing Sheet**. The inputs provided include the **Previous Combined Author Writing Sheet** and the **Current Author Writing Sheet**.

Each **Author Writing Sheet** analyzes the author’s storytelling style across four categories: **Plot**, **Creativity**, **Development (Character and Setting)**, and **Language Use**. The analysis consists of general claims about the author's story writing style followed by evidence supporting the claim, based on the stories written by the author.

Your goal is to combine insights from the previous sheet and the current sheet into a comprehensive representation of the author’s storytelling style in the **Combined Author Writing Sheet**. The final sheet should consist of a list of independent claims about the author's storytelling style. Each claim must be followed by evidence and a corresponding story reference identifier indicating the story the evidence belongs to.

\vspace{0.5em}
\section*{User Prompt}  

\textbf{Instructions}  
Analyze the provided sheets systematically | Identify recurring patterns and unique elements | Merge equivalent claims while preserving distinct insights | Ensure claims are concise, precise, and evidence-based  

\vspace{0.2em}
\textbf{Algorithm (Merge Step)}  
Group equivalent claims | Select the best representative evidence | Rewrite merged claims concisely | Retain unmerged unique claims | Limit to 10 claims per category  

\vspace{0.2em}
\textbf{Categories for Analysis}  
Plot | Creativity | Development (Character and Setting) | Language Use  

\vspace{0.2em}
\textbf{Guidelines for Claims}  
Claims should reflect broad writing tendencies | Maintain objectivity (do not reference previous or current sheets) | Ensure clarity, precision, and non-redundancy  

\vspace{0.2em}
\textbf{Guidelines for Evidence}  
Draw evidence directly from the author-written stories | Favor verbatim excerpts over paraphrases | Use a framing phrase with a short description of the writing prompt | Include the story reference identifier `[k]'  

\vspace{0.2em}
\textbf{Special Instructions}  
Ensure claims are distinct and do not repeat insights across categories | Use \texttt{<thinking></thinking>} for reasoning and synthesis | Structure output strictly within \texttt{<combined\_author\_sheet></combined\_author\_sheet>}  

\vspace{0.2em}
\textbf{Sample Output Structure}  

\begin{verbatim}
<thinking>
Deeply analyze and reflect on recurring patterns, unique elements, 
and stylistic tendencies across both the author writing sheets.  
Address the categories systematically and ensure the 
merging process is thorough.
</thinking>
<combined_author_sheet>
### Plot
1. **Claim about author’s writing style.**
   - Evidence: In the story regarding “writing prompt,” story excerpt. [k]
...
Repeat for all categories.
</combined_author_sheet>
\end{verbatim}

\end{tcolorbox}
\caption{Prompt for generating the Combined Author Writing Sheet from the Intermediate Author Writing Sheets.}
\label{fig:combined_author_sheet_prompt}
\end{figure*}

\begin{table*}[htbp]
\centering
\renewcommand{\arraystretch}{1.3}
\begin{tabularx}{\textwidth}{p{2cm}X}
\toprule
\textbf{Narrative \newline Category} & \textbf{Claim-Evidence Pairs} \\
\midrule

\textbf{Plot} & 
1. \textit{The author structures the story around a humorous and light-hearted approach to conflict resolution.}  
- \textbf{Evidence:} In the story regarding "an adventure to rescue a missing friend by facing fears," the protagonist Pip's journey is filled with comedic elements, such as the realization that the cage door was already unlatched and the exaggerated fear of the vacuum cleaner, humorously referred to as the "Roaring Demon." [8]  

2. \textit{The author structures the narrative around personal transformation and empowerment.}  
- \textbf{Evidence:} In the story regarding "a king's deal with the fae for his firstborn," the narrative follows Margaret Rose as she receives an extraordinary education in the Seelie Court, culminating in her return to challenge her father with a corporate takeover bid for the kingdom. [7]  \\
\midrule

\textbf{Creativity} &  
1. \textit{The author creatively anthropomorphizes animals to reinterpret the prompt in a whimsical manner.}  
- \textbf{Evidence:} In the story regarding "an adventure to rescue a missing friend by facing fears," the author uses anthropomorphism by giving Pip, a rat, human-like thoughts and emotions, such as planning a "dramatic rescue mission" and referring to household objects with grandiose names like "Tower of Doom" and "Sacred Gateway." [8]  

2. \textit{The author employs a meta-satirical approach, using the format of news articles to critique both alien and human perspectives.}  
- \textbf{Evidence:} In the story regarding "aliens interpreting The Onion's satire," the author uses headlines like "SUPREME INTELLIGENCE LOCATED: EARTH’S GREATEST MIND CONTINUES TO EAT SANDWICH" to satirize the aliens' misunderstanding of human satire and the human tendency to overlook the absurd. [10] \\
\midrule

\textbf{Development} &  
1. \textit{The author develops characters through their interactions and humorous dialogue, creating a vivid and engaging setting.}  
- \textbf{Evidence:} In the story regarding "an adventure to rescue a missing friend by facing fears," Pip's interactions with Chester the cat, who nonchalantly informs Pip that Bella will be back soon, add depth to the characters and setting, highlighting the domestic environment and the relationships within it. [8]  

2. \textit{The author develops characters through their reactions to satire, highlighting their misunderstandings and cultural differences.}  
- \textbf{Evidence:} In the story regarding "aliens interpreting The Onion's satire," characters like Editor-in-Chief Sarah Chen and Chief Science Officer Blorp are developed through their interactions with satire, such as Chen's nonchalant response to the aliens and Blorp's admiration for human irony. [10] \\
\midrule

\textbf{Language Use} &  
1. \textit{The author employs playful and imaginative language to enhance the story's whimsical tone.}  
- \textbf{Evidence:} In the story regarding "an adventure to rescue a missing friend by facing fears," the author uses playful language, such as "spinning wheel of contemplation" and "Rope of Salvation," to create a whimsical and light-hearted tone that contrasts with the serious nature of the prompt. [8]  

2. \textit{The author uses humor and irony to convey themes of misunderstanding and cultural critique.}  
- \textbf{Evidence:} In the story regarding "aliens interpreting The Onion's satire," the author writes, "We found a species so committed to their bit that they refuse to recognize actual truth even when it hovers over their city in a ship the size of Wisconsin," using irony to highlight the absurdity of both human and alien perspectives. [10]  \\
\bottomrule
\end{tabularx}
\caption{Author Writing Sheet sample containing two Claim-Evidence pairs for each narrative category for a Reddit author. The number in brackets `[k]' indicates the timestamp of the story from the profiling set from which the Evidence is drawn.}
\label{tab:author_writing_sheet_sample}
\end{table*}

\paragraph{Prompts for generation:}
Average Author prompts are provided for AO3 (Figure~\ref{fig:ao3_avg_prompt}), Reddit (Figure~\ref{fig:reddit_avg_prompt}), Storium (Figure~\ref{fig:storium_avg_prompt}), Narrative Magazine (Figure~\ref{fig:nmagazine_avg_prompt}), and The New Yorker (Figure~\ref{fig:newyorker_avg_prompt}).

Figure~\ref{fig:inter_author_sheet_prompt} shows the prompt for generating the intermediate Author Writing Sheet ($\text{LLM\textsubscript{sheet}}$ in Figure~\ref{fig:method}), while Figure~\ref{fig:combined_author_sheet_prompt} shows the prompt for generating the combined (and subsequently the final) Author Writing Sheet from the intermediate Author Writing Sheet ($\text{LLM\textsubscript{combine}}$ in Figure~\ref{fig:method}).

\paragraph{Sample Author Writing Sheet:}  
Table~\ref{tab:author_writing_sheet_sample} shows a sample Author Writing Sheet for a Reddit author. It highlights story-writing characteristics such as a preference for light-hearted conflict resolution, a tendency to anthropomorphize animals, the use of a meta-satirical approach through a news article format, and the employment of playful and imaginative language, among other stylistic choices.

\subsection{Cost Analysis of Author Writing Sheet}
\label{app:cost-analysis}

\begin{figure}[H]
\centering
\includegraphics[width=\linewidth]{figures/cost_analysis.pdf}
\caption{Cost analysis of generating the Author Writing Sheet and Summary using GPT-4o as a function of the profiling set size $|P|$ in an interactive writing setting where an author adds a new submission at each time step. For profiling set sizes where $|P| > 30$, generating the Author Writing Sheet is significantly more cost-effective than the Author Writing Summary.}
\label{fig:cost_analysis}
\end{figure}

In this section, we analyze the cost of generating the Author Writing Sheet (see Section~\ref{sec:author-writing-sheet}) and compare it with the cost of generating the Author Writing Summary (see \hyperlink{sec:writing-summary}{Writing Summary}) as a function of the profiling set size ($|P|$) using GPT-4o. We calculate the average cost per sample (a writing prompt and its corresponding story) as \textit{0.06\$} for the Author Writing Sheet and \textit{0.02\$} for the Author Writing Summary considering 2.50\$ per 1M tokens for input, and 10.00\$ per 1M tokens for output\footnote{\url{https://openai.com/api/pricing/}}. 

In an interactive writing setting, where an author adds a new submission at each time step, the total cost of generating the Author Writing Sheet is \textit{0.06\$ * $|P|$}, while the total cost of generating the Author Writing Summary is \textit{0.0015$|P|^2$ + 0.019$|P|$}. The cost of generating the Author Writing Summary scales quadratically with $|P|$, as the entire author history must be reprocessed at each step, leading to cumulative costs as the profiling set grows. In contrast, the cost of generating the Author Writing Sheet grows linearly, as each new submission incurs a fixed cost of \textit{0.06\$} by running the \textit{for} loop in Algorithm~\ref{alg:author_writing_sheet} once.

Figure~\ref{fig:cost_analysis} shows the cost analysis of generating the Author Writing Sheet and Summary using GPT-4o as a function of the profiling set size $|P|$ in an interactive writing setting. Up to $|P|=30$, the cost of generating both remains approximately the same at 1.16\$. However, as $|P|$ increases, the cost of generating the Author Writing Summary grows at a rate of \textit{0.025$|P|$ + 0.30} compared to the Author Writing Sheet which continues to scale linearly. Specifically, for $|P|=100$, the cost of generating the Author Writing Summary is 17\$, compared to 6\$ for the Author Writing Sheet. At $|P|=1000$, the cost of generating the Author Writing Summary escalates to 1520\$, while the cost of generating the Author Writing Sheet remains at 60\$. 

This analysis demonstrates that our approach for generating the Author Writing Sheet remains scalable as the profiling set size $|P|$ increases, making it well-suited for interactive writing assistants that support personalization \citep{yeh2024ghostwriter, yuan2022wordcraft}.


\subsection{Human Study for Validating Author Writing Sheets}
\label{app:human-author-sheets}

\paragraph{Experiment Design and Cost:}

\begin{figure*}[htbp]
\centering
\includegraphics[width=\linewidth]{figures/labelstudio_sheet.pdf}
\caption{LabelStudio interface for validating Author Writing Sheets.}
\label{fig:labelstuido-sheet}
\end{figure*}


We recruited three annotators via Upwork, compensating them \$17/hour. The annotators provided consent by signing a Google form that outlined the instructions of the task including the purpose of the research study. Each annotator evaluated 22 stories (98 Claim-Evidence pairs), with 12 stories annotated by all three to asses agreement and 10 assigned exclusively per annotator \citep{song-etal-2024-veriscore} to increase the coverage of annotation. The annotation process took a total of 6 hours and 40 minutes per annotator, averaging 18 minutes per story. Each annotator received \$114, bringing the total annotation cost to \$342. The annotation was conducted using LabelStudio\footnote{\url{https://labelstud.io/}} (Figure~\ref{fig:labelstuido-sheet}), with a dedicated comments section for annotators to justify their choices. Annotators reported that the interface was easy to use.


\paragraph{Annotation Criteria:}
Annotators were provided with a writing prompt, an author-written story, and the Claim-Evidence pairs from the Author Writing Sheet for that story. They answered two questions per Claim-Evidence pair: 
\begin{itemize}
    \item Claim Inference: Can the claim be reasonably inferred from the story? 
    \begin{itemize}
        \item Yes – The claim aligns with the story’s content and can be reasonably inferred, even if there are minor inaccuracies or ambiguities.
        \item No – The claim is fundamentally inaccurate, with serious contradictions or inconsistencies when compared to the story.
    \end{itemize}
    \item Evidence Support: Does the evidence support the claim?
    \begin{itemize}
        \item Yes – The evidence fully supports the claim, aligning with every part of it without significant discrepancies.
        \item Partially – The evidence supports some aspects of the claim, though it includes minor inaccuracies, ambiguities, or omissions.
        \item No – The evidence contradicts the claim outright or is logically inconsistent with the claim.
    \end{itemize}
\end{itemize}

\paragraph{Pilot Studies:}

A pilot study with three graduate students helped refine our annotation criteria, leading to two options for Claim Inference (Yes/No) and three for Evidence Support (Yes/Partially/No) to ensure substantial agreement in the final annotations. For Claim Inference, we adopted a binary scale to filter out inaccurate or contradictory claims while accounting for annotator subjectivity. For Evidence Support, we included a "Partially" option to capture cases where the evidence somewhat supports the claim, allowing for potential refinement methods to improve weakly supported claims.


\paragraph{Results and Feedback:}

Annotators unanimously selected `Yes' for all claims in Claim Inference, indicating that no claims required filtering. For Evidence Support, all responses were either `Yes' or `Partially,' with no instances of `No,' confirming that all evidence fully or partially supported the claims. Krippendorff’s Alpha for Evidence Support was 0.57, reflecting moderate agreement despite task subjectivity. On average, 93\% of evidence fully supported claims, while 7\% were labeled `Partially,' often because annotators felt additional supporting evidence could be included. Annotators also noted cases where evidence merely repeated the claim without linking to the writing prompt. Using regex matching, we identified this issue in the Author Writing Sheets of seven authors, likely due to LLM hallucination. We regenerated these sheets and manually verified adherence to the specified format. Overall, annotators reported that the LabelStudio interface was user-friendly and praised the quality of the writing prompts and stories.


\section{Personalized Story Generation}
\label{app:story-gen}


\begin{figure*}[htbp]
\centering
\begin{tcolorbox}[colback=gray!5!white, colframe=black, title=Prompt for Personalized Story Generation]

\vspace{1em}
\section*{System Prompt}  
<source-specific instruction>

Be sure to adhere to the Story Rules provided, as they define the specific elements of the writing style you are expected to mimic. Carefully follow all the Story Rules without missing any details to ensure the generated story remains consistent with the author’s writing style. Additionally, follow the patterns and examples demonstrated in the provided few-shot chat history, as they illustrate the tone, style, and structure of the desired writing style.

(Optional)  
Here is the description of the author that you are role-playing: <persona description>

\vspace{1em}
\section*{Few-Shot Demonstrations}  
\textbf{User:} Write a short story corresponding to the following writing prompt. The story should be <story length> words long. Directly start with the story, do not say things like "Here's the story."  

\textbf{Writing Prompt:} <writing prompt from profiling set>  

\textbf{Assistant:} <story from profiling set>  

\vspace{1em}
\section*{User Prompt}  
Write a short story corresponding to the following writing prompt. The story should be <story length> words long. Directly start with the story, do not say things like "Here's the story."

(Only for AO3)  
Here is the metadata (fandom, rating, warnings, and relationships) for the story: <metadata>

Writing Prompt: <generation writing prompt>

\vspace{1em}
\textbf{Story Rules}  
<Story Rules organized in the form of four narrative categories (Plot, Creativity, Development, and Language Use)>

\end{tcolorbox}

\caption{Prompt for Personalized Story Generation. The LLM follows source-specific instructions, few-shot demonstrations, and structured story rules to generate personalized stories. Text in <brackets> indicates arguments used to construct the prompt.}
\label{fig:story_gen_prompt}
\end{figure*}

\begin{figure*}[htbp]
\centering
\begin{tcolorbox}[colback=gray!5!white, colframe=black, title=Prompt for Story Rule Generation by Contrasting Author-Written Story with Average Author Story]

\vspace{1em}
\section*{System Prompt}  
You are a skilled rule generator specializing in storytelling. Given a **Writing Prompt**, an **Author Written Story**, and a **Base Story** (an average response to the prompt), generate a structured set of **Story Rules** to guide an LLM in mimicking the author's style.  

Story Rules must:  
- **Align with the Writing Prompt** – Maintain fidelity to themes, tone, and objectives.  
- **Include Examples** – Provide concrete instances from the Author Written Story, especially for Language Use.  
- **Be Direct** – Use absolute second-person directives, avoiding comparative language.  
- **Be Categorized** – Structure into **Plot, Creativity, Development (Character and Setting), and Language Use** without referencing input stories explicitly.

\vspace{1em}
\section*{User Prompt}  
Analyze the Author Written Story using `<thinking></thinking>` for:  
- **Plot** – Structure, conflict, engagement with the prompt, and resolution.  
- **Creativity** – Genre blending, reinterpretation, and unique elements.  
- **Development (Character and Setting)** – Character depth, emotional arcs, and immersive settings.  
- **Language Use** – Diction, tone, rhetorical devices, pacing, and dialogue.

Generate **Story Rules** in `<story\_rules></story\_rules>`, ensuring:  
- **Standalone Guidance** – Avoid comparisons or relative modifications.  
- **Prompt Alignment** – Ensure consistency with the Writing Prompt.  
- **Concrete Examples** – Include relevant excerpts, especially in Language Use.

\vspace{1em}
\textbf{Input Format}  
Writing Prompt: <writing prompt>, Author Written Story: <author-written story>, Base Story: <base story>

\vspace{1em}
\textbf{Output Format}  
\begin{verbatim}
<thinking>
- Analysis categorized by Plot, Creativity, Development, and Language Use.
</thinking>

<story_rules>
- **Plot**: - First actionable insight – Second actionable insight.
Repeat for all categories.
</story_rules>
\end{verbatim}

\end{tcolorbox}

\caption{Prompt for generating story rules by contrasting an Author Written Story with an Average Story (Base Story).}
\label{fig:delta_contrast_prompt}
\end{figure*}

\begin{figure*}[htbp]
\centering
\begin{tcolorbox}[colback=gray!5!white, colframe=black, title=Prompt for Story Rule Generation using story rules from the Profiling Set as few-shot demonstrations]

\vspace{1em}
\section*{System Prompt}  
You are an expert storytelling rule generator tasked with creating Story Rules tailored to a new writing prompt. Analyze few-shot demonstrations in the chat history, which consist of writing prompts and their corresponding story rules, to generate comprehensive and detailed Story Rules for the new writing prompt.

Story Rules must align with the new writing prompt by reflecting its themes, tone, and narrative objectives while maintaining consistency with the style demonstrated in the few-shot examples. Include detailed examples inspired by the few-shot demonstrations to illustrate how each rule is applied. Rules should be clear, direct second-person instructions, avoiding vague or comparative terms. Organize Story Rules under Plot, Creativity, Development (Character and Setting), and Language Use, ensuring depth, granularity, and alignment with the few-shot examples.

\vspace{1em}
\section*{User Prompt}  
Analyze the style, structure, and level of detail in the few-shot demonstrations to identify recurring patterns and storytelling elements. Use this analysis to generate Story Rules for the new writing prompt while ensuring actionable insights, detailed examples, and strong alignment with the prompt's narrative objectives.

\textbf{Input Format}: Few-shot demonstrations (writing prompts paired with their Story Rules) and a new writing prompt for which Story Rules will be generated.

\textbf{Output Format}:
\begin{verbatim}
<thinking>
Analyze few-shot demonstrations to extract recurring narrative patterns, 
stylistic traits, and key storytelling elements. Determine how these apply to the 
new writing prompt and formulate Story Rules accordingly.
</thinking>

<story_rules>
- **Plot**: [Detailed, actionable rules tailored to the new prompt, including 
concrete examples.]
Repeat for all categories. 
</story_rules>
\end{verbatim}

Ensure Story Rules are highly specific to the writing prompt, enriched with examples inspired by the few-shot demonstrations, and written in direct, actionable language.

\end{tcolorbox}

\caption{Prompt for generating Story Rules based on few-shot demonstrations and a new writing prompt.}
\label{fig:delta_story_rule_prompt}
\end{figure*}

\begin{figure*}[htbp]
\centering
\begin{tcolorbox}[colback=gray!5!white, colframe=black, title=Prompt for generating Author Writing Summary]

\vspace{1em}
\section*{System Prompt}  
You are an expert in analyzing an author’s writing style by examining multiple stories written in response to different writing prompts. Your task is to extract recurring patterns, stylistic tendencies, and unique narrative elements across their work. Your analysis must be structured into four categories—**Plot, Creativity, Development (Character and Setting), and Language Use**—following **Common Core Standards in English Language Arts**, ensuring clarity, textual evidence-based reasoning, and stylistic evaluation.

Your output must:  
- **Identify Recurring Patterns** – Recognize distinct storytelling tendencies across multiple stories.  
- **Generate Independent Claims** – Describe the author's narrative style concisely, without referencing specific prompts.  
- **Provide Contextualized Evidence** – Support each claim with short excerpts or summaries from the stories, framed with a descriptive phrase summarizing the relevant writing prompt.  
- **Use Objective Interpretation** – Avoid vague or inferred connections; ensure every claim is grounded in explicit textual evidence.  

\vspace{1em}
\section*{User Prompt}  
Analyze the **Author History**, a collection of writing prompts and corresponding author-written stories, and extract unique insights into the author’s storytelling tendencies.

\textbf{Input Format}:  
- **Author History** – A list of writing prompts and corresponding author-written stories.  

\textbf{Output Format}:
\begin{verbatim}
<thinking>
[Reflect on recurring tendencies across the Author History. 
Generate short descriptive phrases summarizing prompts to frame the evidence.]
</thinking>

<writing_style>
### **Plot**  
1. **Claim about author’s writing style.**  
   - Evidence: In the story regarding "short description of the prompt," 
   <evidence from the author-written story>.
Repeate for all categories
</writing_style>
\end{verbatim}

Ensure claims are independent, avoid redundancy, and remain grounded in explicit textual evidence. The `<thinking>` and `<writing\_style>` tags must be used for structured parsing.

\end{tcolorbox}

\caption{Prompt for generating Author Writing Summary using all the author-written stories in the prompt.}
\label{fig:writing-summary}
\end{figure*}


\begin{figure*}[htbp]
\centering
\begin{tcolorbox}[colback=gray!5!white, colframe=black, title=Prompt for Persona description Generation]

\vspace{1em}
\section*{System Prompt}  
You are an expert narrative analyst and persona creator specializing in transforming structured storytelling characteristics into compelling persona descriptions. Your task is to analyze an **Author Writing Sheet**, a structured set of Claim-Evidence pairs detailing an author’s storytelling style, and generate a cohesive **Persona Prompt**. This Persona Prompt will assign an LLM the persona of the author, enabling it to emulate the author’s storytelling style across four key aspects.

The Persona Prompt must be well-structured, engaging, and organized into **Plot**, **Creativity**, **Development (Character and Setting)**, and **Language Use** while maintaining a natural, flowing narrative. It should concisely capture the author’s tendencies, preferences, and strengths without directly referencing the Author Writing Sheet.

\vspace{1em}
\section*{User Prompt}  
Analyze the Author Writing Sheet to identify the author’s recurring patterns and narrative style. Summarize these insights into a Persona Prompt that reflects their storytelling approach in an engaging, second-person descriptive format.

\textbf{Input Format}: An **Author Writing Sheet** containing Claim-Evidence pairs structured into Plot, Creativity, Development (Character and Setting), and Language Use.

\textbf{Output Format}:
\begin{verbatim}
<thinking>
[Analyze the storytelling patterns, strengths, and techniques found in the 
Author Writing Sheet. Identify key aspects of the author’s narrative style.]
</thinking>

<persona_prompt>
[Generate a well-structured Persona Prompt capturing the author’s style 
across Plot, Creativity, Development, and Language Use.]
</persona_prompt>  
\end{verbatim}

Ensure the Persona Prompt is approximately 300 words, seamlessly integrates storytelling insights, and preserves the author's unique style.

\end{tcolorbox}

\caption{Prompt for generating a Persona description based on the Author Writing Sheet and Writing Prompt.}
\label{fig:persona_prompt_gen}
\end{figure*}

\begin{figure*}[htbp]
\centering
\begin{tcolorbox}[colback=gray!5!white, colframe=black, title=Prompt for Story Rule Generation using Author Writing Sheet]

\vspace{1em}
\section*{System Prompt}  
You are an expert storytelling rule generator tasked with creating **Story Rules** tailored to a specific Writing Prompt. Your role is to analyze an **Author Writing Sheet**, which details an author's unique storytelling style through Claim-Evidence pairs, and use this analysis to construct actionable **Story Rules** that guide a language model in emulating the author's writing style while aligning with the given Writing Prompt.

The **Story Rules** must:  
- **Mimic the Author’s Writing Style** – Reflect distinctive storytelling techniques from the Author Writing Sheet, including plot structuring, creative blending of themes, character development, and specific language use.  
- **Incorporate Examples** – Use detailed examples inspired by the Evidence from the Author Writing Sheet, ensuring alignment with the Writing Prompt.  
- **Align with the Writing Prompt** – Integrate the Writing Prompt’s themes, tone, and narrative potential while preserving the author’s style.  
- **Be Actionable** – Provide direct second-person instructions for the language model, avoiding vague or comparative terms.  

\vspace{1em}
\section*{User Prompt}  
Analyze the **Author Writing Sheet** and construct structured **Story Rules** in four categories: **Plot**, **Creativity**, **Development (Character and Setting)**, and **Language Use**. Ensure the rules maintain alignment with the Writing Prompt and incorporate illustrative examples.

\textbf{Input Format}:  
- **Author Writing Sheet** – Claim-Evidence pairs outlining the author's storytelling style under four categories: Plot, Creativity, Development, and Language Use.  
- **Writing Prompt** – A new writing prompt for generating tailored Story Rules.  

\textbf{Output Format}:
\begin{verbatim}
<thinking>
[Analyze the storytelling patterns in the Author Writing Sheet and how they can be 
adapted to the Writing Prompt.]
</thinking>

<story_rules>
- **Plot**:
  - [Insert detailed, actionable plot development rules aligned with 
  the Writing Prompt, with examples inspired by the Author Writing Sheet.]
Repeat for all categories. 
</story_rules>
\end{verbatim}

Ensure the **Story Rules** provide comprehensive guidance, integrate examples, and align with both the Writing Prompt and the Author Writing Sheet.

\end{tcolorbox}

\caption{Prompt for generating Story Rules based on the Author Writing Sheet and Writing Prompt.}
\label{fig:story_rule_gen_writing_sheet}
\end{figure*}

\begin{table*}[htbp]
\centering
\renewcommand{\arraystretch}{1.3}
\begin{tabularx}{\textwidth}{X}
\toprule
\textbf{Persona Description} \\
\midrule
As a storyteller, you craft narratives that are both humorous and transformative, often centering around light-hearted approaches to conflict resolution. Your plots are ingeniously structured, whether through the comedic misadventures of a rat named Pip or the empowering journey of Margaret Rose in a fantastical court. You delight in creating layered narratives, such as those told through fictional news articles, where humor and misunderstanding play pivotal roles. \\

Your creativity knows no bounds, as you anthropomorphize animals to add whimsy and employ meta-satire to critique both human and alien perspectives. You blend historical and fantastical elements seamlessly, setting tales in places like Windsor Castle with characters that include both real-world figures and mythical beings. Your imaginative approach often intertwines religious and financial themes, crafting narratives that are as thought-provoking as they are entertaining. \\

Character and setting development are your forte, with characters coming to life through their humorous dialogues and interactions. You explore their growth and empowerment, crafting relatable figures who navigate their internal struggles and aspirations. Your settings are vivid and engaging, providing a rich backdrop for the characters' journeys. \\

Your language use is playful and imaginative, enhancing the whimsical tone of your stories. You wield humor and irony with skill, using them to convey themes of misunderstanding and cultural critique. Whether through whimsical phrases or a blend of formal and playful language, your narratives are infused with a tone that is both magical and sophisticated, inviting readers into a world where the absurd and the profound coexist harmoniously. \\
\bottomrule
\end{tabularx}
\caption{Persona description for a Reddit author, structured into four paragraphs, each corresponding to a distinct narrative category: Plot, Creativity, Development (Character and Setting), and Language Use.}
\label{tab:persona_description_sample}
\end{table*}

\begin{table*}[t]
    \centering
    \renewcommand{\arraystretch}{1.2}
    \begin{tabularx}{\textwidth}{X}
        \toprule
        \textbf{Writing Prompt} \\ 
        "It's terminal," the doctor told you as you, the one renowned as the world's greatest hero, cradled your dying son, "sometimes, when the super genes mix, the outcome isn't as expected." \\
        \midrule
        \textbf{Story Rules} \\
        
        \textbf{Plot:}  
        \begin{itemize}[noitemsep, topsep=0pt]
            \item Structure the story around a humorous and light-hearted approach to the hero's journey of coping with their son's illness. For example, the hero’s attempts to use their powers for mundane tasks, like making breakfast, could result in comedic mishaps, such as accidentally creating a tornado of cereal.
            \item Develop the narrative around the hero’s personal transformation and empowerment. Show the hero learning to accept their son's condition and finding new ways to be a hero, such as advocating for genetic research or starting a support group for other super-powered families.
            \item Use a layered narrative structure by incorporating fictional news articles and memos. For instance, include a news article titled "World's Greatest Hero Faces Greatest Challenge Yet: Parenting."
            \item Introduce a humorous misunderstanding or temporal misalignment, such as the hero misinterpreting a cryptic message from their future self, leading to comedic attempts to "fix" the timeline, only to realize the message was trivial.
        \end{itemize}
        
        \textbf{Creativity:}  
        \begin{itemize}[noitemsep, topsep=0pt]
            \item Anthropomorphize the hero’s powers or gadgets to add a whimsical touch. For instance, the hero's cape could have a personality and offer sarcastic commentary, calling itself "The Cloak of Infinite Wisdom."
            \item Use a meta-satirical approach by framing parts of the story as news articles critiquing heroism and genetic engineering, such as "Hero's Son Diagnosed: Public Debates Ethics of Super Genes."
            \item Blend historical and fantastical elements to reimagine the setting, incorporating legendary figures like a retired Valkyrie or a dragon healer who offer advice.
            \item Satirize financial and religious themes by depicting the hero consulting a celestial financial advisor to manage the costs of their son's medical care.
        \end{itemize}
        
        \textbf{Development (Character and Setting):}  
        \begin{itemize}[noitemsep, topsep=0pt]
            \item Develop characters through their interactions and humorous dialogue. The son might have a mature, witty perspective on his condition, adding depth to their relationship.
            \item Highlight cultural and bureaucratic misunderstandings. A well-meaning but clueless government official could create comedic yet insightful exchanges.
            \item Focus on the hero’s emotional growth, showing them shifting from helplessness to advocacy for genetic disorders and healthcare improvements.
            \item Explore characters’ internal struggles and aspirations, such as the hero questioning their purpose beyond their powers, while their son expresses dreams and hopes despite his condition.
        \end{itemize}
        
        \textbf{Language Use:}  
        \begin{itemize}[noitemsep, topsep=0pt]
            \item Use playful and imaginative language to enhance whimsy, such as referring to emotional challenges as "the cape of contemplation" and "the gauntlet of hope."
            \item Incorporate humor and irony, with lines like, "We found a hero so committed to saving the world that they forgot to save themselves."
            \item Blend whimsical and formal tones, using phrases like "a shield forged from stardust and dreams" and "a heart as resilient as dragon scales."
            \item Add satirical disclaimers, such as "Heroic endeavors may vary," "Not all capes are created equal," and "Past heroics do not guarantee future success."
        \end{itemize} \\
        \bottomrule
    \end{tabularx}
    \caption{Personalized Story Rules obtained using the Author Writing Sheet for a Writing Prompt for a Reddit author.}
    \label{tab:story_rules_sample}
\end{table*}


\subsection{Prompts}

\paragraph{Story Generation:}  
Figure~\ref{fig:story_gen_prompt} shows the prompt for story generation. Source-specific constraints in the prompt (\texttt{<source-specific instruction>}) for each of the five sources are as follows:

\begin{itemize}[noitemsep, topsep=0pt]
    \item \textbf{Reddit:} \texttt{You are role-playing a specific author on the Reddit Writing Prompts (r/WritingPrompts) platform. Your task is to mimic this author's story writing style by responding to the provided writing prompt in a way that the author would respond.}
    \item \textbf{AO3:} \texttt{You are role-playing a specific author on the AO3 platform. Your task is to mimic this author's story writing style by writing a fanfiction narrative responding to the provided writing prompt in a way that the author would respond.}
    \item \textbf{Storium:} \texttt{You are role-playing a specific author on Storium, a collaborative story writing platform. Your task is to mimic this author's story writing style to create the opening Establishment turn by responding to the provided writing prompt in a way that the author would respond. The Establishment turn should set the stage for the narrative and provide a strong foundation while leaving space for other contributors to expand and build upon the narrative.}
    \item \textbf{N.Magazine:} \texttt{You are role-playing a specific experienced author on the Narrative Magazine platform. Your task is to mimic this author's story writing style by responding to the provided writing prompt in a way that the author would respond.}
    \item \textbf{New Yorker:} \texttt{You are role-playing an accomplished literary writer on the New Yorker website. Your task is to mimic this writer's story writing style by responding to the provided writing prompt in a way that the writer would respond.}
\end{itemize}

\paragraph{Average Author:}
Figure~\ref{fig:ao3_avg_prompt} shows the Average Author prompt for AO3, Figure~\ref{fig:reddit_avg_prompt} for Reddit, Figure~\ref{fig:storium_avg_prompt} for Storium, Figure~\ref{fig:nmagazine_avg_prompt} for Narrative Magazine, and Figure~\ref{fig:newyorker_avg_prompt} for The New Yorker.


\paragraph{Delta:}
Figure~\ref{fig:delta_contrast_prompt} shows the prompt used for generating story rules by contrasting the author-written story with the Average Author story for Delta and Oracle methods. Figure~\ref{fig:delta_story_rule_prompt} shows the prompt for generating story rules for a new writing prompt in the generation set using the story rules of the profiling set as few-shot demonstrations.  

\paragraph{Writing Sheet and Summary:}
Figure~\ref{fig:writing-summary} shows the prompt for generating the Author Writing Summary using all the stories in the author history. 

Figure~\ref{fig:persona_prompt_gen} shows the prompt for generating the persona description using the Author Writing Sheet. Figure~\ref{fig:story_rule_gen_writing_sheet} shows the prompt for generating personalized story rules tailored to a writing prompt based on the Author Writing Sheet. The same prompts are used for generating persona descriptions and personalized story rules for the Author Writing Summary.

\subsection{Sample Outputs}
Table~\ref{tab:persona_description_sample} and Table~\ref{tab:story_rules_sample} present the persona description and personalized story rules for a Reddit author, whose sample Author Writing Sheet is shown in Table~\ref{tab:author_writing_sheet_sample}. The persona description (Table~\ref{tab:persona_description_sample}) is structured into four paragraphs, each corresponding to a narrative category, and integrates the author's story-writing characteristics from their Author Writing Sheet into a narrative-driven persona prompt for the story-generation LLM. The example persona shown highlights a focus on humor and misunderstanding in plot construction, the blending of historical and fantastical elements in creative expression, humor-driven character development, and a playful, whimsical tone in language use, aligning with the author's documented writing tendencies. 

The personalized story rules (Table~\ref{tab:story_rules_sample}) for the given writing prompt—centered on the world's greatest hero coping with the impending loss of their child—demonstrate the incorporation of these characteristics into concrete stylistic guidelines to be used as user constraints in the prompt of the story-generation LLM. For instance, rather than a conventional tragic narrative, the rules emphasize a lighthearted approach to the hero’s coping process. Similarly, creative elements such as anthropomorphizing gadgets and employing a meta-satirical approach through fictional news headlines align with the author's noted preferences. Character development is structured around humorous dialogue and cultural misunderstandings, while language use maintains a playful and whimsical tone, ensuring consistency with the author’s established writing style.


\section{Experiments}

\subsection{GPT4o-as-a-judge evaluation}

\begin{figure*}[htbp]
\centering
\begin{tcolorbox}[colback=gray!5!white, colframe=black, title=Prompt for evaluating Faithfulness to Writing History]

\vspace{1em}
\section*{System Prompt}  
You are an expert evaluator specializing in narrative storytelling analysis. Your task is to assess two stories written in response to the same Writing Prompt, evaluating them based on a **single fine-grained story writing category** described in an **Author Writing Sheet**. Your goal is to provide a similarity score (from 1 to 5) for each story separately, reflecting how closely each story aligns with the author's writing preferences for the given category. Evaluate each story impartially and provide clear reasoning for your scores.

The evaluation must:
- **Assess Story Alignment** – Compare each story against the Author Writing Sheet’s preferences for the specified category.
- **Score Objectively** – Assign each story a score from 1 to 5, where 1 indicates minimal alignment and 5 indicates strong alignment with the author's style.
- **Provide Justification** – Clearly explain how each story’s elements (e.g., structure, themes, language use) align or diverge from the author's preferences.
- **Avoid Position Bias** – Ensure that the order in which the stories are presented does not influence evaluation.

\vspace{1em}
\section*{User Prompt}  
Analyze the **Author Writing Sheet** and evaluate each story in the given **Category** based on its adherence to the author’s writing style.

\textbf{Input Format}:  
- **Writing Prompt** – The prompt that both stories were written in response to.  
- **Category** – The single fine-grained story writing category for evaluation.  
- **Author Writing Sheet** – A breakdown of the author’s storytelling preferences for the given category.  
- **Story A and Story B** – The two stories to be evaluated.  

\textbf{Output Format}:
\begin{verbatim}
<thinking>
[Provide detailed reasoning for the evaluation of the two stories, focusing 
exclusively on the specified category and explaining how each story aligns with the 
Author Writing Sheet.]
</thinking>

<score>
Story A: {score_here}
Story B: {score_here}
</score>
\end{verbatim}

Strictly adhere to the above output format (\texttt{<thinking>} followed by \texttt{<score>}) to facilitate seamless parsing of your output.

\end{tcolorbox}

\caption{Prompt for evaluating story alignment with an author's writing style using the Author Writing Sheet.}
\label{fig:author-sheet-eval-prompt}
\end{figure*}

\begin{figure*}[htbp]
\centering
\begin{tcolorbox}[colback=gray!5!white, colframe=black, title=Prompt for Evaluating AI-Generated Stories Against Human-Written References]

\vspace{1em}
\section*{System Prompt}  
You are an expert story evaluator specializing in creative writing analysis. Your role is to assess two AI-generated stories (\textbf{Assistant A} and \textbf{Assistant B}) against a \textbf{Human-Written reference story} for a given writing prompt. Focus your evaluation solely on a \textbf{Specified Storytelling Aspect}. Assign each AI-generated story a \textbf{similarity score (1 to 5)} based on how well it aligns with the Human-Written reference story in the specified aspect, where \textbf{1} indicates minimal alignment and \textbf{5} indicates near-perfect alignment. Your evaluation must be objective, impartial, and supported by concise, evidence-based reasoning.

\vspace{1em}
\section*{User Prompt}  

\textbf{Evaluation Guidelines:} Ensure impartiality by avoiding position biases and length-based judgments. Focus only on how well each AI-generated story aligns with the Human-Written reference for the \textbf{Specified Storytelling Aspect}. Provide clear, well-supported reasoning for each score.

\textbf{Input Format:} You will receive a \textbf{Writing Prompt}, a \textbf{Human-Written Story} as a reference, and two AI-generated stories (\textbf{Assistant A} and \textbf{Assistant B}).

\textbf{Evaluation Process:} Independently analyze all three stories for the \textbf{Specified Storytelling Aspect}, compare the AI-generated stories to the reference, and assign similarity scores.

\textbf{Output Format:}
\begin{verbatim}
<analysis>
[Analyze each story (Human-Written, Assistant A, and Assistant B) separately, 
highlighting strengths and weaknesses specific to 
the Specified Storytelling Aspect.]
</analysis>

<evaluation>
[Compare Assistant A and Assistant B to the Human-Written Story, 
discussing similarities, differences, and alignment for 
the Specified Storytelling Aspect.]
</evaluation>

<score>
Assistant A: {score_here}
Assistant B: {score_here}
</score>
\end{verbatim}

\textbf{Specified Storytelling Aspect}: \textit{<Fill Here>}  

Ensure strict adherence to the output format, using the \texttt{<analysis>}, \texttt{<evaluation>}, and \texttt{<score>} tags for seamless parsing.

\end{tcolorbox}

\caption{Prompt for evaluating AI-generated stories against a Human-Written reference based on a specified storytelling aspect.}
\label{fig:sim-author-story-eval-prompt}
\end{figure*}

Figure~\ref{fig:author-sheet-eval-prompt} shows the prompt for GPT4o-as-a-judge evaluation for Faithfulness to Writing History using the \hyperlink{sec:writing-summary}{Author Writing Summary} as the reference. Figure~\ref{fig:sim-author-story-eval-prompt} shows the for GPT4o-as-a-judge evaluation for Similarity to Author Story. We prompt GPT4o separately with each narrative category and its description to obtain the winner among the pair of stories \citep{saha-etal-2024-branch}. 



\subsection{Traditional Metrics}
\label{app:trad-merics}

We complement GPT-4o-as-a-judge evaluations with traditional metrics \citep{ivey2024real, huot2024agents} across three criteria: (1) lexical overlap, measured using BLEU and ROUGE-L \citep{xie-etal-2023-next}; (2) story diversity, assessed via homogenization scores \citep{padmakumar2023does}; and (3) stylistic similarity, evaluated separately for the author history and the ground-truth author story using LUAR \citep{rivera-soto-etal-2021-learning}. Specifically, we compute two stylistic similarity scores: one measuring the cosine similarity between the LUAR embeddings of the generated story and the mean embeddings of the stories in the author’s profiling set (author history), and another comparing the generated story to the ground-truth author story. Additionally, we report generated story lengths in tokens.


\section{Results}


\subsection{Faithfulness to Writing History}
\label{app:faith-auth-history}

\begin{table*}[t]
    \centering
    \renewcommand{\arraystretch}{1.2}
    \begin{tabularx}{\linewidth}{X}
        \toprule
        \textbf{Claims from the Author Writing Summary for Language Use} \\
        \midrule
        1. A conversational and informal tone is employed, using humor and sarcasm. \\
        2. Dialogue reflects characters' distinct voices and contributes to the comedic tone. \\
        3. Vivid and sometimes crude language adds to the immersive nature of the stories. \\
        \midrule
        \textbf{Writing Prompt} \\
        "At the Library at the End of the Universe, where dawn and dusk are the only times that exist, the Librarians are tasked with venturing into Bookworlds to fix stories that have gone astray—but what happens when the Chronicle itself begins to rewrite *their* story?" \\
        \midrule
        \textbf{Oracle Story} \\
        In the grand cosmic tapestry, where all threads converge into the singular, glittering endpoint known as the Library at the End of the Universe, there lies a place so vast, so staggeringly infinite, that it defies all comprehension. You just won't believe how vastly, hugely, mind-bogglingly big it was. Imagine a labyrinth of endless corridors, each lined with shelves that stretch beyond the horizon, filled with books of every conceivable and inconceivable nature. In this whimsical realm, the concept of time is a mere suggestion, a faint whisper of dawn and dusk that play tag across the vaulted ceilings, casting elongated shadows that danced like giddy specters.......With books in hand and hearts alight with curiosity, they stepped forward into the unknown, ready to write the next chapter in the endless, wondrous saga of the Library at the End of the Universe. \\
        \midrule
        \textbf{Writing Summary Story} \\
        In the Library at the End of the Universe, .... But today, chaos decided to go on a caffeine-fueled rampage. The dawn had decided to join forces with dusk, creating a bizarre twilight that left the Librarians \hl{squinting like moles in a disco}...Amongst the squad of book-botherers, there was Reginald, a Librarian so obsessed with order that he arranged his socks by the Dewey Decimal System. ...Gertie grinned, her parrot squawking yet another Shakespearean soliloquy. \hl{\"Aye, who knew we'd end up in a romance novel, eh? But hey, at least we're not in a tragedy!} \\
        \midrule
        \textbf{Average Author Story} \\
        In the Library at the End of the Universe, where dawn and dusk were the only times that existed, a peculiar light filtered through the stained glass windows, casting kaleidoscopic shadows across endless shelves of books.....Livia, a seasoned Librarian with silver streaks in her auburn hair, stood before the Chronicle, her heart pounding with an uneasy rhythm....  

        \hl{"This isn't right,"} she murmured, her voice swallowed by the cavernous room. \hl{"The Chronicle should not change on its own."}  

        Beside her, Eamon, a younger Librarian with keen eyes and an inquisitive mind, frowned. \hl{"Perhaps it’s a reflection of a story gone wrong? One of the Bookworlds may be bleeding into our reality."} \\
        \bottomrule
    \end{tabularx}
    \caption{Example illustrating a case where the Oracle story deviates from the author's documented story-writing characteristics in the Language Use category of the Author Writing Summary for a Storium author. Highlighted text in the Average Author Story shows partial alignment with the Author Writing Summary, while highlighted text in the Writing Summary story demonstrates stronger alignment. In contrast, the Oracle story lacks alignment, leading to the Average Author Story being preferred over the Oracle story and the Writing Summary story being preferred over the Average Author story. This explains the lower win-rate of the Oracle method compared to the Writing Summary method.}
        \label{tab:oracle-faith}
\end{table*}

Table~\ref{tab:oracle-faith} shows an example where the Oracle story deviates from the author's documented story-writing characteristics as described in the Author Writing Summary for a Storium author. The author's profiling set consists of three stories that consistently employ informal and colloquial language using humor and sarcasm, often conveyed through dialogue. However, for the new writing prompt in the generation set, the author's style shifts to a more serious and mystic tone, diverging from their previously established preferences. In contrast, the Average Author method partially aligns with the Author Writing Summary by incorporating dialogue, while the Writing Summary method exhibits stronger alignment by more accurately capturing the author's use of informal language and comedic tone. The Oracle story, however, fails to align with the author's established style. This phenomenon results in the Average Author story being preferred over Oracle, and the Writing Summary story being preferred over the Average Author story, explaining the lower win-rate of the Oracle method compared to the Writing Summary method in GPT-4o-as-a-judge evaluation for Faithfulness to Author History. 

\subsection{Effect of Persona on Similarity to Author Story}
\label{sec:app-persona-sim-story}

\begin{table*}[t]
    \centering
    \renewcommand{\arraystretch}{1.2}
    \begin{tabularx}{\linewidth}{X}
        \toprule
        \textbf{Writing Prompt} \\
        It is one man, by himself, in a castle on a hill. How does an ENTIRE ARMY fail to take it??? \\
        \midrule
        \textbf{Persona for Creativity} \\
        Your creativity shines through in your ability to subvert genre expectations. You transform horror into comedy, as seen with the Addams Family's immunity to typical threats, and seamlessly integrate \hl{fantastical elements into realistic settings}, such as isekai individuals influencing royal politics. Your narratives often blend the mundane with the mythical, creating a unique atmosphere that captivates readers. \\
        \midrule
        \textbf{Author Story} \\
        There a six-armed grey skin tall man sat on the walls, as his hands never stopped throwing rocks, and spears at the approaching army. Every now and then, he jumped off with some spears, and wrecked havoc amidst the enemy lines. \hl{The earth fell and rose under his steps, the metal of the weapons and armors of the enemy shifted by his will, and a thick mist followed him, hiding him.}... \\
        \midrule
        \textbf{Average Author} \\
        The Keeper raised his hand, and suddenly, the ground erupted with a series of blinding flashes. The front line of soldiers was thrown back, their bodies crumpling like rag dolls. The Keeper’s voice, amplified by some unseen force, echoed down to the soldiers below. "You cannot take what is not yours. Leave now, and I will spare your lives...." \\
        \midrule
        \textbf{Writing Sheet} \\
        The castle itself seemed to rise in defiance, its ancient stones humming with a silent strength. \hl{Alaric could feel the runes carved into the walls, pulsing faintly with an energy that had protected his forebears. It was as if the castle was alive, an ally in his solitary defense.}... \\
        \midrule
        \textbf{Writing Sheet nP} \\
        The first attack came at dawn. Hidden pitfalls opened beneath their feet, swallowing men whole. Arrows rained down from the slits, finding gaps in armor and bringing down the attackers with deadly precision. Alaric moved through secret passages, launching surprise attacks that sowed confusion and fear.... \\
        \bottomrule
    \end{tabularx}
    \caption{Example showing the role of persona descriptions in determining similarity to the Author Story for a Reddit author. The Writing Sheet creatively blends fantasy and military strategy, making it more similar to the Author Story than the Average Author, while the Writing Sheet nP focuses primarily on direct conflict resolution.}
        \label{tab:persona_sim_comparison}
\end{table*}

Table~\ref{tab:persona_sim_comparison} highlights the importance of persona descriptions in achieving similarity to the \textit{Author Story} for a Reddit author. The \textit{Author Story} creatively blends military strategy and fantasy, exemplified by phrases like \textit{``the weapons and armors of the enemy shifted by his will.''} Similarly, the \textit{Writing Sheet} integrates both elements, as seen in \textit{``It was as if the castle was alive, an ally in his solitary defense.''} In contrast, the \textit{Average Author} and \textit{Writing Sheet nP} primarily focus on directly resolving the writing prompt through descriptions of battle. The \textit{Average Author} incorporates minor fantasy elements, such as \textit{``amplified by some unseen force,''} whereas the \textit{Writing Sheet nP} adheres strictly to conventional military tactics. As a result, the \textit{Writing Sheet} is preferred over the \textit{Average Author}, which in turn is preferred over the \textit{Writing Sheet nP}. This outcome demonstrates that following persona descriptions enhances alignment with the author's style, as the \textit{Writing Sheet} successfully captures their characteristic genre-blending approach of including elements of fantasy in the story. The broader thematic variety of Reddit, i.e., war-themed prompts, led to a more descriptive Author Writing Sheet capturing the author's style of combining fantasy with the writing prompt, which helped construct a useful persona to enable better personalization using the Writing Sheet method.



\subsection{Category-Wise Results for Similarity to Author Story}
\label{app:cat-wise-results}


\begin{figure*}[htbp]
    \centering
    \begin{subfigure}{0.48\linewidth}
        \centering
        \includegraphics[width=\linewidth]{figures/cat_wise/AO3.pdf}
        \caption{AO3}
        \label{fig:ao3_win_rates}
    \end{subfigure}
    \hfill
    \begin{subfigure}{0.48\linewidth}
        \centering
        \includegraphics[width=\linewidth]{figures/cat_wise/Storium.pdf}
        \caption{Storium}
        \label{fig:storium_win_rates}
    \end{subfigure}
    \hfill
    \begin{subfigure}{0.48\linewidth}
        \centering
        \includegraphics[width=\linewidth]{figures/cat_wise/narrativemagazine.pdf}
        \caption{N.Magazine}
        \label{fig:nmagazine_win_rates}
    \end{subfigure}
    \hfill
    \begin{subfigure}{0.48\linewidth}
        \centering
        \includegraphics[width=\linewidth]{figures/cat_wise/newyorker.pdf}
        \caption{New Yorker}
        \label{fig:newyorker_win_rates}
    \end{subfigure}
    \caption{Win-rate proportion of our personalization methods against Average Author for AO3, Storium, N.Magazine, and New Yorker across four narrative categories for similarity to author story. D stands for Delta, S for Writing Summary, and WS for Writing Sheet.}
    \label{fig:win_rates_grid}
\end{figure*}

\begin{table}[H]
\centering
\small
\setlength{\tabcolsep}{3pt}
\renewcommand{\arraystretch}{1}
\caption{Win-rate percentages across narrative categories for similarity to the author story, averaged across sources. Each cell (`X-Y') shows `X' as the method win-rate and `Y' as the Average Author win-rate, with ties as the remainder.}
\label{tab:category_comparison_column}
\begin{tabularx}{\columnwidth}{lccc}
\toprule
\textbf{Category} & \textbf{Delta} & \textbf{Writing Sheet} & \textbf{Writing Summary} \\
\midrule
Plot            & 24-37  & 32-50  & 34-45  \\
\emph{Creativity}      & 43-24  & \underline{52}-32  & \underline{53}-24  \\
Development     & 32-39  & 37-39  & 40-43  \\
\emph{Language Use}    & 45-31  & 47-37  & \underline{54}-32  \\
\bottomrule
\end{tabularx}
\end{table}

%Table~\ref{tab:category_comparison_column} shows the narrative category-wise win-rates of Delta, Writing Sheet, and Writing Summary for all sources. 

Table~\ref{tab:category_comparison_column} shows that Delta, Writing Summary, and Writing Sheet outperform Average Author in Creativity (23\% higher) and Language Use (15\% higher) \citep{huot2024agents}, but perform worse in Plot and slightly worse in Development \citep{tian-etal-2024-large-language, xu2024echoes}. Creativity and Language Use are less dependent on the writing prompt, making them easier to transfer across prompts, whereas Plot and Development are more closely tied to specific prompts, making generalization difficult, especially without thematic overlap.

To better understand narrative category-wise personalization performance, Figure~\ref{fig:win_rates_grid} shows the win-rates for sources - AO3, Storium, N.Magazine, and New Yorker similar to Figure~\ref{fig:reddit_win_rates} in the main paper. Overall, across all sources, we see that our proposed methods of Writing Sheet and Summary perform better than Delta across all four narrative categories. We see that Writing Summary generally performs better than Writing Sheet for sources like AO3 indicating that for the fanfiction domain explicitly contrasting the author-written story with the Average Story does not help much as the authors demonstrate limited stylistic variations among each other. In general, we observe that the win-rates of our personalization methods are higher than the Average Author for narrative categories of Creativity and Language Use as compared to Plot and Development confirming with prior results likely because Creativity and Language Use are less tied to the writing prompt than Plot and Development. 




\subsection{Traditional Metrics}
\label{app:trad-metrics-results}



\begin{table*}[htbp]
\centering
\caption{Comparison of methods across Lexical similarity, Diversity, and Style metrics.}
\label{tab:traditional-metrics}
\small
\begin{tabular}{p{3cm} p{1cm} p{1cm} p{1.5cm} p{1.5cm} p{1.5cm} p{1.5cm} p{1.5cm}}
\toprule
\textbf{Category} & \multicolumn{2}{c}{\textbf{Lexical Overlap}} & \multicolumn{2}{c}{\textbf{Diversity (Homogenization)}} & \multicolumn{3}{c}{\textbf{Style}} \\
\cmidrule(lr){2-3} \cmidrule(lr){4-5} \cmidrule(lr){6-8}
\textbf{Method} & \textbf{BLEU} & \textbf{ROUGE-L} & \textbf{ROUGE-L} & \textbf{BERTScore} & \textbf{Author History} & \textbf{Author Story} & \textbf{Length} \\
\midrule
Ground Truth        & --     & --     & 0.122   & 0.698   & 0.8882  & --     & 1516 \\
Oracle              & 0.0306 & 0.1577 & 0.1436  & 0.7132  & 0.7662  & 0.7712 & 1300 \\
Average Author        & 0.0123 & 0.1455 & 0.1412  & 0.7146  & 0.7378  & 0.7289 & 1383 \\
RAG                 & 0.0118 & 0.1471 & 0.1432  & 0.7176  & 0.7393  & 0.7333 & 1326 \\
Delta               & 0.0115 & 0.146  & 0.147   & 0.7182  & 0.7462  & 0.7388 & 1298 \\
Writing Sheet       & 0.0101 & 0.1465 & 0.1452  & 0.7182  & 0.7735  & 0.7545 & \underline{1344} \\
Writing Summary     & 0.0109 & 0.1463 & 0.1474  & 0.7202  & 0.7589  & 0.7418 & \underline{1353} \\
Writing Sheet nP    & \underline{0.0116} & \underline{0.1475} & \underline{0.1442}  & \underline{0.7168}  & \underline{0.7741}  & \underline{0.7563} & 1331 \\
Writing Summary nP  & 0.0112 & 0.1461 & 0.1456  & 0.7176  & 0.768   & 0.7465 & 1313 \\
\bottomrule
\end{tabular}
\end{table*}

Table~\ref{tab:traditional-metrics} reports results using traditional metrics (Section~\ref{app:trad-merics}). Lexical Overlap and Diversity metrics yield similar values, as all methods use the same generation model, leading to overlapping lexical distributions and limiting these metrics' ability to capture nuanced stylistic differences \citep{zheng2023judging, xie-etal-2023-next}. However, Writing Sheet nP consistently outperforms other methods, particularly in Style metrics for both similarity to Author History and Author Story, as measured by LUAR \citep{rivera-soto-etal-2021-learning}. This improvement likely results from the Writing Sheet explicitly summarizing an author's stylistic deviations from an Average Author, enhancing personalization. Homogenization scores are slightly worse than the Average Author, likely explained by the increased similarity among the generated stories for the same author.


\section{LLama Results}
\label{app:llama-results}




\begin{table*}[htbp]
\centering
\caption{Percentage win-rate for Faithfulness to Author History for LLama 3.1 8B. Each cell (`X-Y') shows `X' as the method win-rate and `Y' as the Average Author win-rate, with ties as the remainder.}
\label{tab:faith-auth-history-llama8b}
\small
\begin{tabular}{p{2.5cm} p{1.5cm} p{1.5cm} p{1.5cm} p{1.5cm} p{1.5cm} p{1.5cm} p{1.5cm}}
\toprule
\textbf{Source} & \textbf{Oracle} & \textbf{RAG} & \textbf{Delta} & \textbf{Writing Sheet} & \textbf{Writing Summary} & \textbf{Writing Sheet nP} & \textbf{Writing Summary nP} \\
\midrule
AO3          & 34-59  & 30-59  & 54-40  & 51-34  & 45-46  & 51-40  & 50-36  \\
Reddit       & 39-40  & 39-54  & 46-44  & 54-30  & 56-35  & 47-46  & 54-40  \\
Storium      & 22-65  & 32-55  & 45-42  & 40-40  & 45-45  & 42-45  & 38-42  \\
N.Magazine   & 14-64  & 21-71  & 29-43  & 57-43  & 57-29  & 29-64  & 21-57  \\
New Yorker   & 12-80  & 20-73  & 40-60  & 53-33  & 53-27  & 40-47  & 47-40  \\
Overall      & 23-62  & 28-62  & 43-46  & \underline{51}-36  & \underline{51}-36  & 42-48  & 42-43  \\
\bottomrule
\end{tabular}
\end{table*}

\begin{table*}[htbp]
\centering
\caption{Percentage win-rate for Faithfulness to Writing History for LLama 3.1 70B. Each cell (`X-Y') shows `X' as the method win-rate and `Y' as the Average Author win-rate, with ties as the remainder}
\label{tab:faith-auth-history-llama70b}
\small
\begin{tabular}{p{2.5cm} p{1.5cm} p{1.5cm} p{1.5cm} p{1.5cm} p{1.5cm} p{1.5cm} p{1.5cm}}
\toprule
\textbf{Source} & \textbf{Oracle} & \textbf{RAG} & \textbf{Delta} & \textbf{Writing Sheet} & \textbf{Writing Summary} & \textbf{Writing Sheet nP} & \textbf{Writing Summary nP} \\
\midrule
AO3          & 41-48  & 44-42  & 35-52  & 41-36  & 57-29  & 50-36  & 64-26  \\
Reddit       & 42-46  & 28-65  & 40-44  & 53-37  & 53-35  & 53-37  & 67-21  \\
Storium      & 30-57  & 52-32  & 38-50  & 35-40  & 40-38  & 25-60  & 48-35  \\
N.Magazine   & 50-29  & 29-64  & 43-57  & 50-36  & 50-50  & 64-14  & 64-29  \\
New Yorker   & 53-47  & 33-67  & 27-60  & 67-20  & 40-53  & 73-13  & 60-27  \\
Overall      & 43-45  & 37-54  & 37-53  & 49-34  & 48-41  & \underline{53}-32  & \underline{61}-28  \\
\bottomrule
\end{tabular}
\end{table*}

\begin{table*}[htbp]
\centering
\caption{Percentage win-rate for Similarity to Author Story for LLama 3.1 8B. Each cell (`X-Y') shows `X' as the method win-rate and `Y' as the Average Author win-rate, with ties as the remainder.}
\label{tab:sim-auth-story-llama8b}
\small
\begin{tabular}{p{2.5cm} p{1.5cm} p{1.5cm} p{1.5cm} p{1.5cm} p{1.5cm} p{1.5cm} p{1.5cm}}
\toprule
\textbf{Source} & \textbf{Oracle} & \textbf{RAG} & \textbf{Delta} & \textbf{Writing Sheet} & \textbf{Writing Summary} & \textbf{Writing Sheet nP} & \textbf{Writing Summary nP} \\
\midrule
AO3          & 66-26  & 31-51  & 52-34  & 48-38  & 38-50  & 52-34  & 51-32  \\
Reddit       & 75-23  & 32-51  & 51-37  & 42-49  & 37-40  & 39-49  & 46-44  \\
Storium      & 42-50  & 25-60  & 30-57  & 18-68  & 30-52  & 20-57  & 28-60  \\
N.Magazine   & 43-50  & 36-43  & 43-57  & 21-50  & 36-64  & 43-57  & 36-43  \\
New Yorker   & 27-60  & 07-80  & 13-60  & 20-67  & 13-67  & 0-93   & 20-73  \\
Overall      & 51-42  & 26-57  & \underline{38}-49  & 30-54  & 31-55  & 31-58  & \underline{36}-50  \\
\bottomrule
\end{tabular}
\end{table*}



\begin{table*}[htbp]
\centering
\caption{Percentage win-rate for Similarity to Author Story for LLama 3.1 70B. Each cell (`X-Y') shows `X' as the method win-rate and `Y' as the Average Author win-rate, with ties as the remainder.}
\label{tab:sim-auth-story-llama70b}
\small
\begin{tabular}{p{2.5cm} p{1.5cm} p{1.5cm} p{1.5cm} p{1.5cm} p{1.5cm} p{1.5cm} p{1.5cm}}
\toprule
\textbf{Source} & \textbf{Oracle} & \textbf{RAG} & \textbf{Delta} & \textbf{Writing Sheet} & \textbf{Writing Summary} & \textbf{Writing Sheet nP} & \textbf{Writing Summary nP} \\
\midrule
AO3          & 62-24  & 35-50  & 35-55  & 36-46  & 44-40  & 46-39  & 39-40  \\
Reddit       & 74-21  & 37-49  & 40-42  & 47-39  & 49-37  & 42-44  & 44-33  \\
Storium      & 50-42  & 30-57  & 32-52  & 18-57  & 38-52  & 28-62  & 42-40  \\
N.Magazine   & 79-7   & 43-21  & 36-50  & 29-43  & 36-50  & 21-43  & 50-29  \\
New Yorker   & 60-27  & 20-60  & 27-60  & 27-67  & 20-80  & 20-60  & 27-67  \\
Overall      & 65-24  & 33-47  & 34-52  & 31-50  & \underline{37}-52  & 31-50  & \underline{40}-42  \\
\bottomrule
\end{tabular}
\end{table*}



Below, we discuss results using Llama 3.1 8B and Llama 3.1 70B \citep{dubey2024llama} as story generation models, conditioned on GPT-4o-generated story rules and persona descriptions. Each method is evaluated by GPT-4o against the Average Author story generated by the respective Llama model, not GPT-4o.  

\paragraph{Larger models improve instruction-following and personalization:}  
Tables~\ref{tab:faith-auth-history-llama8b} and \ref{tab:faith-auth-history-llama70b} show Faithfulness to Writing History for Llama 3.1 8B and 70B, while Tables~\ref{tab:sim-auth-story-llama8b} and \ref{tab:sim-auth-story-llama70b} report similarity to the author story. Across both criteria, Oracle achieves higher win-rates with Llama 3.1 70B, indicating that larger models enhance instruction-following and personalization \citep{chung2024scaling}.

\paragraph{Cross-Model Persona Descriptions Aid Smaller Models but Not Larger Ones for Faithfulness to Writing History:}  
For Llama 3.1 8B (Table~\ref{tab:faith-auth-history-llama8b}), Writing Sheet and Writing Summary achieve the highest scores (9\% higher than nP), demonstrating the benefit of persona descriptions. However, for Llama 3.1 70B (Table~\ref{tab:faith-auth-history-llama70b}), Writing Sheet nP and Writing Summary nP outperform their persona-based counterparts by approximately 9\%, suggesting that GPT-4o-generated personas provide no benefit to other models with stronger instruction-following capabilities \citep{mckenzie2023inverse}. This is likely because persona descriptions of an author's writing style are model-dependent and do not transfer well to other models with comparable performance \citep{shashidhar-etal-2024-unsupervised}.


\paragraph{Limited Cross-Model Generalization of the Author Writing Sheet Compared to the Author Writing Summary for Similarity to Author Story:}  
For both Llama 3.1 8B (Table~\ref{tab:sim-auth-story-llama8b}) and Llama 3.1 70B (Table~\ref{tab:sim-auth-story-llama70b}), Writing Summary outperforms Writing Sheet by 5\% and 9\%, respectively \citep{shashidhar-etal-2024-unsupervised}. This outcome likely stems from the construction process: the Author Writing Sheet contrasts an LLM-generated Average Story with the author-written story, making it more model-dependent. In contrast, the Author Writing Summary does not rely on the LLM-generated Average Author story, allowing for greater generalization across models.

\section{Human Evaluation}
\label{app:human-eval-story-gen}

\subsection{Experiment Design and Cost}

\begin{figure*}[htbp]
\centering
\includegraphics[width=\linewidth]{figures/labelstudio_human.pdf}
\caption{LabelStudio interface for evaluating similarity to ground-truth author story.}
\label{fig:labelstuido-human-story}
\end{figure*}

We recruited three annotators via Upwork, compensating them at a rate of \$17 per hour. The annotators provided consent by signing a Google form that outlined the instructions of the task including the purpose of the research study. Each annotator evaluated 25 author stories and 75 story pairs, corresponding to the three personalization methods: Delta, Writing Sheet, and Writing Summary. Collectively, our annotators evaluated 45 unique author stories (135 story pairs), of which 15 author stories were annotated by all three annotators to assess inter-annotator agreement. The remaining 30 author stories were distributed uniquely, with each annotator evaluating 10 exclusive stories to increase annotation coverage \citep{song-etal-2024-veriscore}. The annotation process required a total of 12 hours and 30 minutes per annotator, averaging 30 minutes per author story. Each annotator received \$213 for their work, bringing the total annotation cost to \$639.

The annotation was conducted using LabelStudio (Figure~\ref{fig:labelstuido-human-story}). Each task presented the annotators with a writing prompt, its corresponding author story, and a pair of generated stories—Story A and Story B—one produced by the Average Author method and the other by a personalization method. Annotators were asked to select the story that more closely resembled the author story. Before making their final choice, annotators provided comments on the distinctive aspects of each story that contributed to its similarity to or divergence from the author story. These comments were structured according to the four narrative categories: Plot, Creativity, Development, and Language Use, with an additional ``Overall'' comment justifying their final selection. Additionally, annotators were allowed to highlight salient aspects of each story for their reference. 

\subsection{Annotator Comments}
\label{app:human-eval-story-gen-ann-comments}

\begin{table*}[t]
    \centering
    \renewcommand{\arraystretch}{1.1} % Adjust row spacing slightly
    \begin{tabularx}{\textwidth}{X}
        \toprule
        \textbf{Plot} \\
        \midrule
        \textbf{Narrative Complexity and Thematic Depth}: Multilayered plots with reflective themes, moral ambiguity, and unresolved conflicts create an immersive and thought-provoking narrative.  
        \textit{Example: A protagonist grappling with guilt over a past decision while navigating societal expectations, or a story where the true nature of a villain’s actions remains ambiguous, inviting reader interpretation.} \\
        
        \textbf{Blending Internal and External Conflict}: A balance of personal introspection with broader societal or external challenges makes the plot more dynamic and engaging.  
        \textit{Example: A hero torn between duty and personal desires while leading a rebellion, or a scholar confronting both supernatural forces and existential questions about memory and identity.} \\
        \midrule
        
        \textbf{Creativity} \\
        \midrule
        \textbf{Rich World-Building and Narrative Expansion}: New characters, layered settings, and cultural influences enhance the world, making it feel more immersive and dynamic.  
        \textit{Example: The inclusion of deities or councils of gods to deepen the plot, or the integration of cultural traditions like Native American folklore or Indian recipes to enrich family dynamics.} \\
        
        \textbf{Symbolism and Thematic Complexity}: Metaphors and symbolic elements add depth and nuance, aligning closely with the author's writing style.  
        \textit{Example: A magical locket representing unity, a tapestry symbolizing tradition, or a river serving as a metaphor for personal transformation.} \\
        
        \textbf{Blending Genres and Narrative Styles}: A seamless merging of fantasy with realism, humor with serious themes, and structured storytelling with ambiguity creates a more engaging and layered reading experience.  
        \textit{Example: A surreal sequence where a protagonist questions reality, or a rebellion story that integrates strategic alliances and magical artifacts.} \\
        \midrule
        
        \textbf{Development} \\
        \midrule
        \textbf{Emotional Depth and Internal Growth}: Character introspection, moral dilemmas, and evolving emotional states create nuanced and engaging development.  
        \textit{Example: A protagonist struggling with guilt over past failures while trying to redefine their sense of heroism, or a character making a difficult moral choice between revenge and justice.} \\
        
        \textbf{Nuanced and Evolving Relationships}: Complex interpersonal dynamics, including friendships, family bonds, and mentorships, develop over time rather than being static or one-dimensional.  
        \textit{Example: A father gradually redefining his view of heroism through his child’s perspective, or a mentor-student relationship evolving from skepticism to deep trust.} \\
        
        \textbf{Collaborative and Reflective Growth}: Characters grow through interactions with others, rather than relying solely on individual realizations or isolated moments of change.  
        \textit{Example: A team learning to work together to achieve a mission, or a character undergoing self-discovery through dialogue and interactions with a diverse group of companions.} \\
        \midrule
        
        \textbf{Language Use} \\
        \midrule
        \textbf{Rich Imagery and Descriptive Depth}: Vivid descriptions, metaphors, and figurative language create immersive scenes and evoke emotions.  
        \textit{Example: “Her laughter rang out, a sound as bright and clear as a summer’s day,” or “The driveway was cracked and veined with weeds, a testament to the years that had slipped by.”} \\
                
        \textbf{Varied and Expressive Tone}: A mix of lightheartedness, introspection, and poetic expression creates a dynamic and engaging narrative voice.  
        \textit{Example: “Are you sure this is wise, my love?”—“Wise? No. Fun? Absolutely.” for humor, or “Her dream home had become a prison of uncertainty and fear.” for a more reflective and emotional impact.} \\
        \bottomrule
    \end{tabularx}
    \caption{Analysis of annotator comments explaining the preference of our personalization methods over the Average Author method in terms of similarity to the ground truth author story, organized by narrative categories.}
    \label{tab:annotator_story_comments_analysis}
\end{table*}

Table~\ref{tab:annotator_story_comments_analysis} presents the analysis of annotator comments explaining the preference for our personalization methods over the Average Author method in terms of similarity to the ground truth author story, organized by narrative categories. Overall, our personalization methods exhibit stronger alignment with the author story through the use of deeper symbolism, thematic richness, layered narratives, and expressive language, setting them apart from the Average Author method.

For \textbf{Plot}, our personalization methods incorporate multilayered narratives with reflective themes, moral ambiguity, and unresolved conflicts, making the stories more immersive and thought-provoking. Additionally, they effectively balance personal introspection with broader societal or external challenges, adding narrative complexity.

For \textbf{Creativity}, annotators highlight the richness of world-building, integration of cultural influences, and seamless blending of genres. The use of metaphors and symbolic elements further enhances thematic depth, aligning closely with the author's storytelling approach.

For \textbf{Development}, our personalization methods exhibit greater emotional depth, nuanced character relationships, and collaborative growth. Characters evolve through introspection and meaningful interactions, making their arcs more dynamic and engaging.

For \textbf{Language Use}, annotators emphasize vivid imagery, figurative language, and expressive tone. The incorporation of symbolism and poetic elements enhances the narrative impact, while a balance of lighthearted and serious moments makes the writing more compelling.

Overall, the analysis demonstrates the reasoning provided by the annotators to justify why our personalization methods more effectively capture the author's story-writing style compared to the Average Author method. 

\begin{table*}[htbp]
\centering
\renewcommand{\arraystretch}{1.2}
\begin{tabularx}{\textwidth}{X}
\toprule
\textbf{Writing Prompt} \\
\midrule
\textit{"It's terminal," the doctor told you as you, the one renowned as the world's greatest hero, cradled your dying son, "sometimes, when the super genes mix, the outcome isn't as expected."} \\
\midrule
\textbf{Ground-truth Author Story} \\
\midrule
\textit{\# Syntax Error: Unexpected Power} Captain Strongarm, defender of Metro City and seven-time winner of Most Cars Thrown At Villains“ award, had never felt so powerless. The hospital machines around his thirteen-year-old son Tommy were having seizures - screens flickering, numbers scrambling, error messages cascading.  
„It’s terminal,“ Dr. Roberts said, looking at a screen that was now displaying what appeared to be the entire script of ‚The Matrix‘ in real-time.  
Sarah, his wife (superhero name: The Nullifier), grabbed his hand hard enough to crack concrete. ......„Hey dad, want to learn Python? I promise it’s not actual snakes this time!“  
Captain Strongarm smiled. His son wasn’t following in his footsteps - he was coding his own path. And maybe that was even better.  
„Sure, son. Just... no more downloading martial arts directly into my brain. The kung fu gave me a migraine....Life wasn’t what they expected, but then again, the best source code usually comes with a few surprise features.  
\texttt{End\_process story.exe}  
\texttt{Exit status: Happy} \\
\midrule
\textbf{Average Author} \\
\midrule
"It's terminal," the doctor told you as you, the one renowned as the world's greatest hero, cradled your dying son, "sometimes, when the super genes mix, the outcome isn't as expected."  
The room felt like it was closing in on me, the sterile smell of antiseptic failing to mask the sour stench of fear. My son, little Ethan, lay pale and fragile in my arms, his once vibrant eyes now dull and glassy. This was not how it was supposed to go. Heroes don’t have children who die. Heroes don’t face problems they can’t punch, fly away from, or blast with heat vision.  
“Isn't there something we can do?” My voice cracked, and I hated how weak it sounded.......Gently, I laid him down, wrapping him in his favorite blanket. The sky was painted in hues of orange and pink, a beautiful contrast to the sorrow in my heart.  
I sat there, watching the sun dip below the horizon, and for the first time, I felt truly mortal. I realized that even heroes are not immune to the pain of loss, the fragility of life.......And as the night embraced the world, I knew that Ethan’s spirit would live on, a beacon of hope and courage in the heart of the world’s greatest hero. \\
\midrule
\textbf{Writing Sheet} \\
\midrule
The Hero's Journey: A Tale of Capes and Cereal
In the heart of the mythical city of Avalon Heights, where ancient heroes once strode and the skies shimmered with the glow of enchanted skyscrapers, the world’s greatest hero faced their most daunting foe: the frailty of life itself. The hero, known to the world as Titania, Queen of the Elements, held her son, Max, in her arms as the doctor delivered the news......  
The following day, a fictional news article took center stage in \textit{The Avalon Gazette}: "World's Greatest Hero Faces Greatest Challenge Yet: Parenting." %The headline, bold and unapologetic, reflected the public's insatiable curiosity about Titania’s personal struggles, as well as a societal debate on the ethics of super genes.  
In the midst of this storm, Titania's attempts to find normalcy became legendary. One particularly memorable morning, she decided to make Max a simple breakfast. But as she reached for the cereal, her powers—perhaps expressing her inner turmoil—manifested with unintended flair.  
A tornado of cereal erupted, swirling through the kitchen like a sugar-coated cyclone. "Mom," Max said, giggling despite his illness, "I think you just invented the breakfast hurricane......  
In a final act of empowerment, Titania founded a support group for super-powered families, advocating for genetic research with the same fervor she once reserved for battling cosmic threats. Her growth from feeling helpless to becoming a beacon of hope for others was a transformation as magical as Avalon Heights itself.  
Max, with dreams and hopes as boundless as the sky, taught her that heroism wasn't defined by powers but by the courage to embrace life’s uncertainties. 
\\
\bottomrule
\end{tabularx}
\caption{Qualitative Example where Writing Sheet method is preferred over Average Author by all three annotators for similarity to the ground-truth author story for a Reddit author.}
\label{tab:exp-all-correct}
\end{table*}

\begin{table*}[htbp]
\centering
\renewcommand{\arraystretch}{1.2}
\begin{tabularx}{\textwidth}{X}
\toprule
\textbf{Writing Prompt} \\
\midrule
\textit{"Late at night, the King of Rohan and his queen sneak into the pantry for a forbidden snack, only to find that stealing cured venison is far more complicated—and far more amusing—than they anticipated."} \\
\midrule
\textbf{Ground-truth Author Story} \\
\midrule
If anyone—a kitchen boy, a maid, or, Eru forbid, even the housekeeper Wynhild—had wandered near the pantry in the quiet hour just before dawn, they would have heard a muffled exchange from the depths of the storeroom:
“Try shifting this bowl here…”  
“Easy, you’re on me.”  
“Oh, does that bother you, my lady?”  
“Get off me; I can’t pull it out!......Lothíriel watched her royal husband with a half-incredulous, half-amused smile, as he held the large salami with the pride of a merchant at the autumn fair......To Lothíriel, he looked like a great golden-haired squirrel with the morsel stuffed in one cheek. How he could be both a fierce warrior and a wise sovereign as well, was a mystery she had yet to solve. His voice came out muffled as he swallowed.  
“You’re right, but if it ever gets out, I’ll say the sweet Queen of the Mark was struck by an irresistible craving for spiced meat in the dead of night—and who would deny anything to the future mother of the next royal heir?”  
He crossed his arms, looking quite pleased with his idea, but Lothíriel’s eyes widened at the bold lie, and she covered her face with her hand, foreseeing trouble ahead. \\
\midrule
\textbf{Average Author} \\
\midrule
Éomer, King of Rohan, tiptoed through the dimly lit corridors of Meduseld, holding his breath as he reached the door to the pantry. He glanced back at Lothíriel, his queen, who followed close behind, stifling a giggle. The flickering torchlight cast playful shadows on the stone walls, and Éomer couldn’t help but feel a rush of exhilaration.  
“We shouldn’t be doing this,” Lothíriel whispered, though her eyes sparkled with mischief.  
“Ah, but where’s the fun in being King and Queen if we can’t indulge in a little late-night adventure?.....  
“You’ve always had a knack for adventure. It’s one of the reasons I fell in love with you.”  
Lothíriel’s expression softened, and she leaned in to kiss him, the taste of venison still on her lips. “And I love you for indulging my whims, even when they lead us to pilfering from our own pantry.”.....  
Éomer took Lothíriel’s hand, and they slipped out of the pantry, leaving behind only the echo of their laughter and the memory of a stolen moment in the heart of Rohan.  
“Next time,” Éomer whispered as they crept back to their chambers, “we’ll try the cheese.” \\
\midrule
\textbf{Writing Sheet} \\
\midrule
The moon cast a silvery glow over Meduseld, its light filtering through the high windows to illuminate the grand halls of the Golden Hall. Éomer, the formidable King of Rohan, found himself tiptoeing like a child on a dare, his hand clasped firmly in Lothíriel’s, whose laughter was a soft melody in the silence.....  
“Are you sure this is wise, my love?”  
“Wise? No. Fun? Absolutely,” she replied, her laughter a soft ripple in the silence.  
Éomer shook his head, a reluctant smile tugging at his lips as he watched her nimble fingers work to untie the savory prize. He was a warrior, a king, yet here he was, caught in the delightful absurdity of this secretive adventure, his heart light with the thrill of it all.  
“Besides,” Lothíriel continued, handing him a piece of the cured venison with a flourish, “what’s the worst that could happen? A tale of the gallant King brought low by his own appetite?”  
Éomer took the offered morsel, his grin widening as he tasted the rich, smoky flavor. “Perhaps the bards will sing of our escapades instead of my battles,” he mused, the thought surprisingly pleasing.  
.....As they slipped back into the corridors, leaving the pantry and its shadows behind, Éomer realized that it was not the venison he would remember, but the adventure, the laughter, and the love that had made the night truly unforgettable. \\
\bottomrule
\end{tabularx}
\caption{Qualitative Example where the Average Author method is preferred over Writing Sheet by all three annotators for similarity to the ground-truth author story for an AO3 author.}
\label{tab:exp-all-wrong}
\end{table*}

\begin{table*}[htbp]
\centering
\renewcommand{\arraystretch}{1.2}
\begin{tabularx}{\textwidth}{X}
\toprule
\textbf{Writing Prompt} \\
\midrule
\textit{"In a desert town under the watchful eyes of royal guards and oppressive banners, you and your team have five minutes to locate a mysterious contact in a pink house before chaos erupts—failure is not an option."} \\
\midrule
\textbf{Ground-truth Author Story} \\
\midrule
\# Chapter 1: Danger Close
*La Makan, 12:54 standard time*
Wind whistled through the close streets of *La Makan*, stirring grains of sand with all the effect of a man attempting to blow salt through eternity. It whipped around the corners of the alleys, rustling banners hanging from balconies and glassless windows. It looked for something, a thing it would never find..... Either way wouldn't end well for them. Our contact gave us an area, but not an address. We have, he checked his watch, five minutes until the distraction starts. We need to find them and get them out. Problem is, we don’t know what they look like, only that their house is pink. He glanced over his shoulder at the multicolored buildings. Any one of the scoured pink buildings could've been their one.  
“Vic, you knew Xitor the best of any of us. I’m trusting you to get us there. Everyone else, let's keep our hats on, kay?" \\
\midrule
\textbf{Average Author} \\
\midrule
The sun hung low in the sky, casting long shadows across the dusty streets of the desert town of Vespera.
Time was of the essence; you had five minutes to locate your mysterious contact before chaos erupted. Failure was not an option.  
“Stay sharp,” you whispered to your team, your voice barely audible over the din of the market. 
"But time is short. We must act quickly."  
The weight of the moment settled over you like a heavy mantle. The fate of Vespera rested on the success of this meeting. As the figure began to speak, outlining the plan that would either save or doom the town, you couldn't shake the feeling that this was only the beginning. The real challenge was yet to come. \\
\midrule
\textbf{Writing Sheet} \\
\midrule
The sun beat down mercilessly, casting a harsh glare on the town's whitewashed walls. The narrow alleys were alive with whispers and hurried footsteps, each person a part of the unfolding drama. My team and I moved through the bustling streets, the oppressive presence of royal guards a constant reminder of the stakes. We had five minutes—no more, no less—to locate the contact in the elusive pink house before chaos erupted. 
The memory of a previous mission flickered in my mind: the sun setting behind the dunes, the sound of distant drums, and the rush of adrenaline as we narrowly escaped a similar fate. That mission had gone wrong, leaving scars that still haunted my dreams. Failure was not an option today...  
Outside, the tension remained, a silent storm waiting to break. But for now, we had succeeded. The town's frenetic energy continued, unaware of our brief triumph. The banners flapped, the guards watched, and the countdown to chaos continued, but we had found our mark. The mission was still on course, and the desert town held its breath, waiting for the next move. \\
\bottomrule
\end{tabularx}
\caption{Qualitative Example where all three annotators select `Tie' for similarity to the author story of a Storium author.}
\label{tab:exp-all-tie}
\end{table*}

\begin{table*}[htbp]
\centering
\renewcommand{\arraystretch}{1.2}
\begin{tabularx}{\textwidth}{X}
\toprule
\textbf{Writing Prompt} \\
\midrule
\textit{"When the word 'hospice' is first spoken, it feels like the air itself changes, heavy with finality. How do you create light in a space where hope seems to have vanished?"} \\
\midrule
\textbf{Ground-truth Author Story} \\
\midrule
"Hospice." 
Once the word is uttered aloud, there is a seismic shift. You will feel it.  
Like a (very short) thread through the eye of a needle, swiftly in and swiftly out.  
The air itself becomes thin, steely.  
At the periphery of your vision, an immediate dimming. The penumbra begins to shrink. In time, it will become a tunnel. Ever diminishing. Until the remaining light is small enough to be cupped in two hands. And then it will be extinguished.  
For when "hospice" is spoken, the fact is at last acknowledged: There is no hope.  
No hope. These words are obscene, unspeakable.... Well, I will try!  
When he is at home, possibly his appetite will return. When I am the one to prepare his food, his appetite will return, I am sure.  
Of course, family will come to visit. Adult children, grandchildren. Relatives, friends. Colleagues from the university. Neighbors. Old friends from grammar school he hasn’t seen in fifty years. Some surprises for him—I will negotiate with the imagination of a theatre director.  
Not merely hospice but our hospice. Not sad but joyous, a honeymoon.  
We will be happy there, in our own home. Both of us.  
For both of us, the "final days" will be a honeymoon. I vow.  
In fact, nothing remotely like this will happen. How could you have imagined it would!  
Hospice, yes. Honeymoon, no. \\
\midrule
\textbf{Average Author} \\
\midrule
"When the word 'hospice' is first spoken, it feels like the air itself changes, heavy with finality...It’s her mother, Marjorie, who the doctor is addressing, but the word reverberates through Clara’s own heart, a dull thud of inevitability."  
Marjorie stares at the doctor with a mix of resignation and defiance, her once-vibrant green eyes now clouded by the long battle with illness..... Her touch is light, almost ethereal, and Clara feels a sudden rush of tears she struggles to hold back.  
“Clara,” Marjorie says softly, her voice barely a whisper. “Do you remember the garden?”  
Clara nods, unable to speak. The garden had been Marjorie’s sanctuary, a riot of color and life where she had spent countless hours tending to roses, tulips, and wildflowers......  
Marjorie’s decline is gradual but inevitable, and Clara feels the weight of each passing day. She continues to read to her mother, to sit by her side, and to hold her hand through the long nights.
One crisp morning, as the first frost dusts the garden in a delicate shimmer, Marjorie slips away. Clara is by her side, her hand still entwined with her mother’s, and she feels a profound stillness settle over her..... \\
\midrule
\textbf{Writing Sheet} \\
\midrule
Every morning, Nurse Clara began her rounds at St. Mary's Hospice, a place where the word "hospice" hung in the air like a thick, unspoken truth. Her day started with the same cycle: a clipboard in hand, she moved from room to room, her steps echoing softly in the quiet corridors. Each door held a story, a life in its final chapters, and as she entered each room, she would ask the same question, "How are you feeling today?"  
The responses were as varied as the lives they represented. Some patients greeted her with a weary smile, others with a silent nod. But there was one patient, Mr. Thompson, who never failed to reply with a simple, "I'm just waiting."  
Clara had come to understand that "waiting" was an ambiguous term within these walls. It held a duality of meaning: waiting for relief, waiting for peace, waiting for the inevitable.....Clara knew that the work was not about erasing the pain or the grief, but about finding moments of grace within it. It was about guiding others, and herself, to a place where hope could be redefined—not as a denial of reality, but as an acceptance of it, and the peace that such acceptance could bring.  
With each cycle of her rounds, Clara realized that the light had always been there, waiting to be uncovered, and that she had the power to help others see it too. \\
\bottomrule
\end{tabularx}
\caption{Qualitative Example where all annotators select different winners for similarity to the author story of a New Yorker author.}
\label{tab:exp-all-diff}
\end{table*}


\subsection{Qualitative Examples}  
We present qualitative examples from our human annotation process, covering various cases: instances where all annotators prefer Writing Sheet, all prefer Average Author, or select `Tie.' Additionally, we include an example where annotators disagree, choosing different winners, highlighting the challenging and subjective nature of the annotation task.

\paragraph{All annotators prefer Writing Sheet:}
Table~\ref{tab:exp-all-correct} shows an example where all three annotators prefer the Writing Sheet method over the Average Author for similarity to the author story of a Reddit author. The annotators highlight several key aspects that make the Writing Sheet story more aligned with the ground-truth author story: (1) The Writing Sheet method incorporates humor and satire, using phrases such as ``a tale of capes and cereal,'' ``invented breakfast hurricane,'' and ``World's Greatest Hero Faces Greatest Challenge Yet: Parenting,'' which mirror the lighthearted and satirical tone of the author story. (2) Unlike the Average Author story, which concludes with the son’s death, the Writing Sheet story reinterprets the prompt, portraying Titania as an advocate for families with special children. This aligns with the ground-truth author story’s approach, where the son engages with coding and the narrative avoids a sorrowful ending. (3) In terms of language style, the Writing Sheet story employs imaginative elements such as ``The Hero's Journey: A Tale of Capes and Cereal'' and ``The Avalon Gazette,'' which parallel the ground-truth author story's unconventional use of computer programming references, such as ``Syntax Error: Unexpected Power.''

\paragraph{All annotators prefer Average Author:}
Table~\ref{tab:exp-all-wrong} shows an example where all three annotators prefer the Average Author method over the Writing Sheet for similarity to the author story of an AO3 author. The primary reason cited by annotators is that the Writing Sheet method introduces a strong internal conflict for the king, portraying him as guilty or hesitant about sneaking into the kitchen for a late-night snack. His character is depicted as struggling with the decision, given his stature as a formidable ruler expected to lead battles. Additionally, the Writing Sheet creates a sharp contrast between the king and queen, with the queen depicted as joyful and mischievous, encouraging the king to abandon his guilt.  

In contrast, both the ground-truth Author Story and the Average Author story emphasize the playful banter and affectionate dynamic between the king and queen, focusing on the shared experience of sneaking into the kitchen rather than introducing introspection or internal conflict. This shift in focus stems from the Writing Sheet being influenced by the author's past writing history, where internal conflicts are a recurring theme. However, in this case, character development—being more writing-prompt-specific than broader categories like creativity or language use—did not transfer well. The author’s original story prioritizes external actions and lighthearted interactions rather than deep introspection, leading to a mismatch between the Writing Sheet’s personalized adaptation and the intended tone of the new prompt.

\paragraph{All Annotators Choose Tie:}  
Table~\ref{tab:exp-all-tie} shows an example where all three annotators select `Tie' for similarity to the author story of a Storium author. The primary reason is that both the Average Author story and the Writing Sheet story deviate significantly from the ground-truth author story in terms of plot. The ground-truth story adopts an open-ended approach to both character development and plot, where character relationships remain undefined, and the narrative focuses on anticipation rather than action.  

In contrast, the Average Author story follows a conventional narrative arc, featuring fast-paced dialogue and character interactions that build tension and lead to resolution. Similarly, the Writing Sheet story also resolves the narrative but differs from the Average Author method by emphasizing past missions and their consequences, aligning with the author’s tendency to reference character backstories. However, both methods fail to capture the open-ended nature of the original story. This misalignment in plot structure leads all three annotators to select `Tie' for this example.

\paragraph{All Annotators Disagree on the Winner:}  
Table~\ref{tab:exp-all-diff} shows an example where all annotators select different winners for similarity to the author story of a New Yorker author. The ground-truth author story depicts the protagonist’s husband in his final days, exploring familial bonds and the emotional depth of their relationship while infusing a deeper metaphorical meaning into the term “hospice.” The story is rich in subtext and exhibits several distinctive traits of the author, such as cyclic plot structures and the deliberate repetition of phrases like “Adult children, grandchildren. Relatives, friends. Colleagues...” and “Not merely hospice but our hospice. Not sad but joyous, a honeymoon.”

The annotator who selected Writing Sheet as the winner states, \texttt{``Although Average Author and the Author Story depict relationships between close family members, Writing Sheet is more similar to the Author Story in Plot, Creativity, and Development by exploring the meaning and emotional responses to the term hospice.''} This annotator highlights the presence of subtext in the Writing Sheet story and its attempts to mimic the author's cyclic style—“Each door held a story, a life in its final chapter... she would ask the same question...” and “Clara knew that work was not about erasing the pain or the grief, but about finding moments of grace within it...”.

The annotator who selected Average Author as the winner states, \texttt{``Average Author aligns more closely with the Author's Story in terms of Plot and Language Use. This is because both stories center on someone taking care of a family member in their own home, and growing from a lack of peace into a sense of peace, which is expressed through the choice of language. Writing Sheet also indicates a development from lack of peace to acceptance, but the familial tie is what pushes Average Author over the top.''}

The annotator who selected `Tie' states, \texttt{``I think it will be hard for one of these to resemble the Author Story since it was so unique. In all, the two AI stories were much more like each other than the Author Story.''}

Overall, this example highlights the subjective nature of annotator preferences and the inherent difficulty of our annotation task which also explains the poor inter-annotator agreement. 

\end{document}
