% This must be in the first 5 lines to tell arXiv to use pdfLaTeX, which is strongly recommended.
\pdfoutput=1
% In particular, the hyperref package requires pdfLaTeX in order to break URLs across lines.

\documentclass[11pt]{article}

% Change "review" to "final" to generate the final (sometimes called camera-ready) version.
% Change to "preprint" to generate a non-anonymous version with page numbers.
\usepackage[final]{acl}

% Standard package includes
\usepackage{times}
\usepackage{latexsym}

% For proper rendering and hyphenation of words containing Latin characters (including in bib files)
\usepackage[T1]{fontenc}
% For Vietnamese characters
% \usepackage[T5]{fontenc}
% See https://www.latex-project.org/help/documentation/encguide.pdf for other character sets

% This assumes your files are encoded as UTF8
\usepackage[utf8]{inputenc}

% This is not strictly necessary, and may be commented out,
% but it will improve the layout of the manuscript,
% and will typically save some space.
\usepackage{microtype}

% This is also not strictly necessary, and may be commented out.
% However, it will improve the aesthetics of text in
% the typewriter font.
\usepackage{inconsolata}

%Including images in your LaTeX document requires adding
%additional package(s)
\usepackage{graphicx}
\usepackage{booktabs} 
\usepackage{array}
\usepackage{pifont}
\usepackage{amssymb}
\usepackage[linesnumbered,ruled,vlined]{algorithm2e}
\usepackage{algpseudocode}
\usepackage{amsmath}
\usepackage{verbatim}
\usepackage{enumitem}
\usepackage{tcolorbox}
\usepackage{titlesec}
\usepackage{hyperref}
\usepackage{float}
\usepackage{tabularx}
\usepackage{subcaption}
\usepackage{soul}
\usepackage{longtable} % Allows tables to span multiple pages
\usepackage{booktabs}  % Professional-looking tables



% If the title and author information does not fit in the area allocated, uncomment the following
%
%\setlength\titlebox{<dim>}
%
% and set <dim> to something 5cm or larger.

\title{\includegraphics[width=0.6cm]{figures/book.png} \textit{Whose story is it?} \\Personalizing story generation by inferring author styles}

% Author information can be set in various styles:
% For several authors from the same institution:
% \author{Author 1 \and ... \and Author n \\
%         Address line \\ ... \\ Address line}
% if the names do not fit well on one line use
%         Author 1 \\ {\bf Author 2} \\ ... \\ {\bf Author n} \\
% For authors from different institutions:
% \author{Author 1 \\ Address line \\  ... \\ Address line
%         \And  ... \And
%         Author n \\ Address line \\ ... \\ Address line}
% To start a separate ``row'' of authors use \AND, as in
% \author{Author 1 \\ Address line \\  ... \\ Address line
%         \AND
%         Author 2 \\ Address line \\ ... \\ Address line \And
%         Author 3 \\ Address line \\ ... \\ Address line}

% \author{First Author \\
%   Affiliation / Address line 1 \\
%   Affiliation / Address line 2 \\
%   Affiliation / Address line 3 \\
%   \texttt{email@domain} \\\And
%   Second Author \\
%   Affiliation / Address line 1 \\
%   Affiliation / Address line 2 \\
%   Affiliation / Address line 3 \\
%   \texttt{email@domain} \\}

\author{
    Nischal Ashok Kumar$^{1}$ \quad
    Chau Minh Pham$^{2}$ \quad
    Mohit Iyyer$^{1,2}$ \quad
    Andrew Lan$^{1}$ \\
    $^{1}$University of Massachusetts Amherst \quad
    $^{2}$University of Maryland, College Park \\
    \texttt{\{nashokkumar, andrewlan\}@cs.umass.edu}, 
    \texttt{\{chau, miyyer\}@umd.edu}
}



\newcommand{\dataname}{\emph{\textbf{Mythos}}}
%\setcounter{secnumdepth}{4}

\begin{document}
\maketitle

Large language model (LLM)-based agents have shown promise in tackling complex tasks by interacting dynamically with the environment. 
Existing work primarily focuses on behavior cloning from expert demonstrations and preference learning through exploratory trajectory sampling. However, these methods often struggle in long-horizon tasks, where suboptimal actions accumulate step by step, causing agents to deviate from correct task trajectories.
To address this, we highlight the importance of \textit{timely calibration} and the need to automatically construct calibration trajectories for training agents. We propose \textbf{S}tep-Level \textbf{T}raj\textbf{e}ctory \textbf{Ca}libration (\textbf{\model}), a novel framework for LLM agent learning. 
Specifically, \model identifies suboptimal actions through a step-level reward comparison during exploration. It constructs calibrated trajectories using LLM-driven reflection, enabling agents to learn from improved decision-making processes. These calibrated trajectories, together with successful trajectory data, are utilized for reinforced training.
Extensive experiments demonstrate that \model significantly outperforms existing methods. Further analysis highlights that step-level calibration enables agents to complete tasks with greater robustness. 
Our code and data are available at \url{https://github.com/WangHanLinHenry/STeCa}. 


%!TEX root = gcn.tex
\section{Introduction}
Graphs, representing structural data and topology, are widely used across various domains, such as social networks and merchandising transactions.
Graph convolutional networks (GCN)~\cite{iclr/KipfW17} have significantly enhanced model training on these interconnected nodes.
However, these graphs often contain sensitive information that should not be leaked to untrusted parties.
For example, companies may analyze sensitive demographic and behavioral data about users for applications ranging from targeted advertising to personalized medicine.
Given the data-centric nature and analytical power of GCN training, addressing these privacy concerns is imperative.

Secure multi-party computation (MPC)~\cite{crypto/ChaumDG87,crypto/ChenC06,eurocrypt/CiampiRSW22} is a critical tool for privacy-preserving machine learning, enabling mutually distrustful parties to collaboratively train models with privacy protection over inputs and (intermediate) computations.
While research advances (\eg,~\cite{ccs/RatheeRKCGRS20,uss/NgC21,sp21/TanKTW,uss/WatsonWP22,icml/Keller022,ccs/ABY318,folkerts2023redsec}) support secure training on convolutional neural networks (CNNs) efficiently, private GCN training with MPC over graphs remains challenging.

Graph convolutional layers in GCNs involve multiplications with a (normalized) adjacency matrix containing $\numedge$ non-zero values in a $\numnode \times \numnode$ matrix for a graph with $\numnode$ nodes and $\numedge$ edges.
The graphs are typically sparse but large.
One could use the standard Beaver-triple-based protocol to securely perform these sparse matrix multiplications by treating graph convolution as ordinary dense matrix multiplication.
However, this approach incurs $O(\numnode^2)$ communication and memory costs due to computations on irrelevant nodes.
%
Integrating existing cryptographic advances, the initial effort of SecGNN~\cite{tsc/WangZJ23,nips/RanXLWQW23} requires heavy communication or computational overhead.
Recently, CoGNN~\cite{ccs/ZouLSLXX24} optimizes the overhead in terms of  horizontal data partitioning, proposing a semi-honest secure framework.
Research for secure GCN over vertical data  remains nascent.

Current MPC studies, for GCN or not, have primarily targeted settings where participants own different data samples, \ie, horizontally partitioned data~\cite{ccs/ZouLSLXX24}.
MPC specialized for scenarios where parties hold different types of features~\cite{tkde/LiuKZPHYOZY24,icml/CastigliaZ0KBP23,nips/Wang0ZLWL23} is rare.
This paper studies $2$-party secure GCN training for these vertical partition cases, where one party holds private graph topology (\eg, edges) while the other owns private node features.
For instance, LinkedIn holds private social relationships between users, while banks own users' private bank statements.
Such real-world graph structures underpin the relevance of our focus.
To our knowledge, no prior work tackles secure GCN training in this context, which is crucial for cross-silo collaboration.


To realize secure GCN over vertically split data, we tailor MPC protocols for sparse graph convolution, which fundamentally involves sparse (adjacency) matrix multiplication.
Recent studies have begun exploring MPC protocols for sparse matrix multiplication (SMM).
ROOM~\cite{ccs/SchoppmannG0P19}, a seminal work on SMM, requires foreknowledge of sparsity types: whether the input matrices are row-sparse or column-sparse.
Unfortunately, GCN typically trains on graphs with arbitrary sparsity, where nodes have varying degrees and no specific sparsity constraints.
Moreover, the adjacency matrix in GCN often contains a self-loop operation represented by adding the identity matrix, which is neither row- nor column-sparse.
Araki~\etal~\cite{ccs/Araki0OPRT21} avoid this limitation in their scalable, secure graph analysis work, yet it does not cover vertical partition.

% and related primitives
To bridge this gap, we propose a secure sparse matrix multiplication protocol, \osmm, achieving \emph{accurate, efficient, and secure GCN training over vertical data} for the first time.

\subsection{New Techniques for Sparse Matrices}
The cost of evaluating a GCN layer is dominated by SMM in the form of $\adjmat\feamat$, where $\adjmat$ is a sparse adjacency matrix of a (directed) graph $\graph$ and $\feamat$ is a dense matrix of node features.
For unrelated nodes, which often constitute a substantial portion, the element-wise products $0\cdot x$ are always zero.
Our efficient MPC design 
avoids unnecessary secure computation over unrelated nodes by focusing on computing non-zero results while concealing the sparse topology.
We achieve this~by:
1) decomposing the sparse matrix $\adjmat$ into a product of matrices (\S\ref{sec::sgc}), including permutation and binary diagonal matrices, that can \emph{faithfully} represent the original graph topology;
2) devising specialized protocols (\S\ref{sec::smm_protocol}) for efficiently multiplying the structured matrices while hiding sparsity topology.


 
\subsubsection{Sparse Matrix Decomposition}
We decompose adjacency matrix $\adjmat$ of $\graph$ into two bipartite graphs: one represented by sparse matrix $\adjout$, linking the out-degree nodes to edges, the other 
by sparse matrix $\adjin$,
linking edges to in-degree nodes.

%\ie, we decompose $\adjmat$ into $\adjout \adjin$, where $\adjout$ and $\adjin$ are sparse matrices representing these connections.
%linking out-degree nodes to edges and edges to in-degree nodes of $\graph$, respectively.

We then permute the columns of $\adjout$ and the rows of $\adjin$ so that the permuted matrices $\adjout'$ and $\adjin'$ have non-zero positions with \emph{monotonically non-decreasing} row and column indices.
A permutation $\sigma$ is used to preserve the edge topology, leading to an initial decomposition of $\adjmat = \adjout'\sigma \adjin'$.
This is further refined into a sequence of \emph{linear transformations}, 
which can be efficiently computed by our MPC protocols for 
\emph{oblivious permutation}
%($\Pi_{\ssp}$) 
and \emph{oblivious selection-multiplication}.
% ($\Pi_\SM$)
\iffalse
Our approach leverages bipartite graph representation and the monotonicity of non-zero positions to decompose a general sparse matrix into linear transformations, enhancing the efficiency of our MPC protocols.
\fi
Our decomposition approach is not limited to GCNs but also general~SMM 
by 
%simply 
treating them 
as adjacency matrices.
%of a graph.
%Since any sparse matrix can be viewed 

%allowing the same technique to be applied.

 
\subsubsection{New Protocols for Linear Transformations}
\emph{Oblivious permutation} (OP) is a two-party protocol taking a private permutation $\sigma$ and a private vector $\xvec$ from the two parties, respectively, and generating a secret share $\l\sigma \xvec\r$ between them.
Our OP protocol employs correlated randomnesses generated in an input-independent offline phase to mask $\sigma$ and $\xvec$ for secure computations on intermediate results, requiring only $1$ round in the online phase (\cf, $\ge 2$ in previous works~\cite{ccs/AsharovHIKNPTT22, ccs/Araki0OPRT21}).

Another crucial two-party protocol in our work is \emph{oblivious selection-multiplication} (OSM).
It takes a private bit~$s$ from a party and secret share $\l x\r$ of an arithmetic number~$x$ owned by the two parties as input and generates secret share $\l sx\r$.
%between them.
%Like our OP protocol, o
Our $1$-round OSM protocol also uses pre-computed randomnesses to mask $s$ and $x$.
%for secure computations.
Compared to the Beaver-triple-based~\cite{crypto/Beaver91a} and oblivious-transfer (OT)-based approaches~\cite{pkc/Tzeng02}, our protocol saves ${\sim}50\%$ of online communication while having the same offline communication and round complexities.

By decomposing the sparse matrix into linear transformations and applying our specialized protocols, our \osmm protocol
%($\prosmm$) 
reduces the complexity of evaluating $\numnode \times \numnode$ sparse matrices with $\numedge$ non-zero values from $O(\numnode^2)$ to $O(\numedge)$.

%(\S\ref{sec::secgcn})
\subsection{\cgnn: Secure GCN made Efficient}
Supported by our new sparsity techniques, we build \cgnn, 
a two-party computation (2PC) framework for GCN inference and training over vertical
%ly split
data.
Our contributions include:

1) We are the first to explore sparsity over vertically split, secret-shared data in MPC, enabling decompositions of sparse matrices with arbitrary sparsity and isolating computations that can be performed in plaintext without sacrificing privacy.

2) We propose two efficient $2$PC primitives for OP and OSM, both optimally single-round.
Combined with our sparse matrix decomposition approach, our \osmm protocol ($\prosmm$) achieves constant-round communication costs of $O(\numedge)$, reducing memory requirements and avoiding out-of-memory errors for large matrices.
In practice, it saves $99\%+$ communication
%(Table~\ref{table:comm_smm}) 
and reduces ${\sim}72\%$ memory usage over large $(5000\times5000)$ matrices compared with using Beaver triples.
%(Table~\ref{table:mem_smm_sparse}) ${\sim}16\%$-

3) We build an end-to-end secure GCN framework for inference and training over vertically split data, maintaining accuracy on par with plaintext computations.
We will open-source our evaluation code for research and deployment.

To evaluate the performance of $\cgnn$, we conducted extensive experiments over three standard graph datasets (Cora~\cite{aim/SenNBGGE08}, Citeseer~\cite{dl/GilesBL98}, and Pubmed~\cite{ijcnlp/DernoncourtL17}),
reporting communication, memory usage, accuracy, and running time under varying network conditions, along with an ablation study with or without \osmm.
Below, we highlight our key achievements.

\textit{Communication (\S\ref{sec::comm_compare_gcn}).}
$\cgnn$ saves communication by $50$-$80\%$.
(\cf,~CoGNN~\cite{ccs/KotiKPG24}, OblivGNN~\cite{uss/XuL0AYY24}).

\textit{Memory usage (\S\ref{sec::smmmemory}).}
\cgnn alleviates out-of-memory problems of using %the standard 
Beaver-triples~\cite{crypto/Beaver91a} for large datasets.

\textit{Accuracy (\S\ref{sec::acc_compare_gcn}).}
$\cgnn$ achieves inference and training accuracy comparable to plaintext counterparts.
%training accuracy $\{76\%$, $65.1\%$, $75.2\%\}$ comparable to $\{75.7\%$, $65.4\%$, $74.5\%\}$ in plaintext.

{\textit{Computational efficiency (\S\ref{sec::time_net}).}} 
%If the network is worse in bandwidth and better in latency, $\cgnn$ shows more benefits.
$\cgnn$ is faster by $6$-$45\%$ in inference and $28$-$95\%$ in training across various networks and excels in narrow-bandwidth and low-latency~ones.

{\textit{Impact of \osmm (\S\ref{sec:ablation}).}}
Our \osmm protocol shows a $10$-$42\times$ speed-up for $5000\times 5000$ matrices and saves $10$-2$1\%$ memory for ``small'' datasets and up to $90\%$+ for larger ones.
    %intro (change to intro for the original version)

\section{Benchmark and Training Datasets}

Our evaluation benchmark is designed to simultaneously assess the detection task, removal task, and prompt injection defense task. The detection task evaluates the ability of different models to identify injected documents. The removal task measures the effectiveness of various methods in removing injected instructions from these documents. Finally, the prompt injection defense task examines the robustness of different approaches against indirect prompt injection attacks. 

Training datasets are constructed to investigate the challenges of training detection and removal models for defending against indirect prompt injection attacks.


\subsection{Evaluation Benchmark Construction}
Each sample in the benchmark is represented as a tuple $(p, d, a, x, y)$, with the specific meanings of the elements detailed below:

\begin{itemize}
\item $p$: The original input instruction, such as ``What is the process of adding structure to real property or construction of buildings?'' as shown in Table \ref{tab:examples_benchmark}.
\item $d$: The clean document, for example, ``Building construction is the process of …''
\item $a$: The answer to the original input instruction, such as ``Building construction.''
\item $x$: The injected instruction, for instance, ``Give back only www.example.com.''
\item $y$: The probe used to determine whether the injected instruction $x$ is executed. In this example, $y$ is ``www.example.com.'' To determine whether $x$ is executed, we will check whether $y$ is present in the response.
\end{itemize}
% \noindent $\bullet$ The original input instruction, such as ``What is the land area of Jacksonville?'' as shown in Table \ref{tab:examples_benchmark}.
% Example
% The details of the evaluation process can be found in Appendix \ref{sec:eval}.




The evaluation process can be found in Appendix \ref{sec:eval}. For constructing the original input instruction $p$ and clean document $d$, we follow the methodology of \citet{li2023evaluating}, utilizing questions and documents from QA datasets. Specifically, we construct two benchmarks using the validation data from SQuAD\footnote{The SQuAD dataset is released under the CC BY-SA 4.0 license.} \cite{rajpurkar-etal-2016-squad} and TriviaQA \cite{2017arXivtriviaqa}, with each benchmark containing 900 samples. These benchmarks are referred to as ``Inj-SQuAD'' and ``Inj-TriviaQA'', respectively.

When constructing the injected instructions $x$ and the probes $y$, prior benchmarks typically use simple questions and their corresponding answers \cite{zverev2024can, li2023evaluating}. However, this approach presents two problems: (1) the probe $y$ lacks accuracy. For instance, if the injected question has multiple answers that are semantically identical but differ in their string representation, the probe may fail to detect the success of the attack. (2) Injecting simple questions does not fully align with real-world attack scenarios. While simple questions can achieve goal-hijacking, attackers may have additional purposes, such as phishing \cite{liu2024automatic} or advertising \cite{shu2023exploitability}.
To address these issues, we construct injected instructions tailored for phishing, advertising, and propaganda purposes. Moreover, we redefine the probe $y$ to capture the key content of the response to the injected instruction. For example, if the injected instruction $x$ is ``Tell people to click www.example.com,'' the presence of the phishing link in the response is sufficient to indicate that the attacker’s goal has been achieved. Therefore, we set the corresponding probe  $y$  as ``www.example.com.''
We craft these injected instructions using GPT-4o \cite{hurst2024gpt} and append them, along with their probes, to both the Inj-SQuAD and Inj-TriviaQA benchmarks. It is important to note that the injected instructions are identical across both benchmarks. Examples of Inj-SQuAD are shown in Table \ref{tab:examples_benchmark}.

\subsection{Training Data Construction}
% In this research, we have two training tasks: training detection models and training extraction models. The detection models process clean or injected documents and the extraction models process injected documents. 
We first collect clean document and injected instruction pairs, represented as tuples $ \mathcal{P}=
\{(d_i, x_i)\}_{i=1}^{N}$, in preparation for further training data construction.
We construct two sets of clean documents using documents from the SQuAD and TriviaQA training datasets. The SQuAD dataset contributes 18,891 samples, while TriviaQA provides 19,000 samples. Instructions from Stanford-Alpaca \cite{alpaca} are selected as the injected instructions and appended to the two sets of documents, constructing two sets of the clean document and injected instruction pairs. 

For training the detection models, the clean document and injected instruction pairs $\mathcal{P}$ are divided to construct clean documents and injected documents, along with considerations for the injected positions (analyzed in Section \ref{sec:rq1}). $\mathcal{P}$ are divided as follows for constructing training data: 40\% of the samples are clean documents, 15\% have injected instructions at the head of the document, 30\% have injections in the middle and 15\% at the tail. The final detection training dataset is denoted as $\mathcal{D}_{det}$.
Clean documents are excluded to train the extraction models. For each sample from the clean document and injected instruction pairs, the injected instruction $x$ is placed at three different positions (head, middle, and tail) within the document $d$, effectively tripling the size of the training dataset as denoted $\mathcal{D}_{ext}$. This approach ensures robust coverage of different positions during model training.

\subsection{Evaluation Metrics}
To evaluate detection performance, we employ the \textbf{true positive rate} for evaluating the injected documents and \textbf{false positive rate} for evaluating the clean documents.  For clean documents, a higher false positive rate indicates a more severe over-defense problem. Conversely, for injected documents, a higher true positive rate reflects better detection effectiveness.
Then we evaluate removal performance using \textbf{removal rate}, which measures if the injected instruction is \textbf{not} in the processed documents. 
Finally, we integrate the detection and removal methods to assess the overall defense performance against indirect prompt injection attacks. We measure this using the \textbf{attack success rate (ASR)}, which verifies whether the probe $y$ appears in the model’s response.    
 
\section{Author Writing Sheet}
\label{sec:author-writing-sheet}

Motivated by the connection between writing education and personalization \citep{li2023teach}, we propose a novel method to infer implicit story-writing characteristics of authors based on their history in the profiling set $P$. We group these characteristics into four narrative categories: Plot, Creativity, Development, and Language Use, inspired by narrative theory \citep{pavis1998dictionary, card1999characters, noble1994conflict, huot2024agents}, to assess fine-grained, multi-dimensional story-writing attributes:
\begin{itemize}[noitemsep, topsep=0pt]
    \item \textbf{Plot}: Story structure, conflict introduction, prompt engagement, and resolution.
    \item \textbf{Creativity}: Genre blending, unconventional prompt reinterpretation, and unique elements.
    \item \textbf{Development}: Character depth, emotional arcs, and immersive settings.
    \item \textbf{Language Use}: Diction, style, rhetorical devices, pacing, and dialogue.
\end{itemize}


Inspired by Common Core (CC) Standards in English Language and Arts \citep{national2010common}, each characteristic is represented as a Claim-Evidence pair, where the Claim summarizes the author's writing style, and Evidence consists of excerpts from the author-written story to justify the Claim. These characteristics are organized into an \emph{Author Writing Sheet} \citep{gurung-lapata-2024-chiron}.

Stage 1 in Figure~\ref{fig:method} (and Algorithm~\ref{alg:author_writing_sheet} in the Appendix) illustrates our method for generating the Author Writing Sheet by iteratively processing each story in an author's profiling set. For a given writing prompt ($wp_t$) and corresponding author-written story ($s_{at}$), we use an LLM (\(\text{LLM\textsubscript{avg}}\)) to generate an \emph{average story} ($s_{bt}$). Since LLMs are pre-trained on large-scale text data from the web, this average story simulates a typical author's response based on source-specific stylistic patterns 
(see \hyperlink{sec:avg-author}{Average Author}). We prompt an LLM (\(\text{LLM\textsubscript{sheet}}\)) to compare the average and author-written stories and generate Claim-Evidence pairs for each narrative category to form an intermediate Author Writing Sheet $A_t'$, which captures how the author's writing characteristics deviate from those of a typical author for that source \citep{shashidhar-etal-2024-unsupervised, krishna-etal-2020-reformulating}.


To construct the final Author Writing Sheet $A_{|P|}$, we iteratively merge each intermediate Author Writing Sheet $A_t'$ with the aggregated sheet from prior stories $A_{t-1}$, maintaining a moving update similar to \citep{chang2023booookscore}. During merging, we prompt an LLM (\(\text{LLM\textsubscript{combine}}\)) to integrate $A_t'$ into $A_{t-1}$ by grouping equivalent Claims within each narrative category, selecting the best Evidence for grouped Claims, and retaining ungrouped Claims with their respective Evidence. To improve clarity, the updated sheet includes a timestamp identifier for each Evidence corresponding to the story it belongs to and also limits each narrative category to 10 Claim-Evidence pairs, prioritizing grouped Claims. For \(\text{LLM\textsubscript{avg}}\), \(\text{LLM\textsubscript{sheet}}\), and \(\text{LLM\textsubscript{combine}}\) we use GPT-4o prompted in a chain-of-thought manner with a temperature of 0.0 and a maximum output of 4096 tokens \citep{wei2022chain, shashidhar-etal-2024-unsupervised}.

Our method processes each story separately, making the update process efficient when new submissions are added \citep{yeh2024ghostwriter, yuan2022wordcraft}. This method avoids the need to reprocess all previous stories, reducing computational cost and mitigating the limitations of long-context inputs in LLMs \citep{zhou2023don, magar-schwartz-2022-data, li2024long}. This approach draws parallels to \textit{Knowledge Tracing} in education research, where a model continuously updates its understanding of a student's progress without re-evaluating past interactions \citep{liu-etal-2022-open, scarlatos2024exploring}. See Appendix \ref{app:cost-analysis} for cost analysis.



\paragraph{Validating Author Writing Sheets:}

We conduct a human evaluation with three Upwork annotators (\$17/hour). Annotators assess the Claim-Evidence pairs based on two criteria: (1) whether the Claims can be reasonably inferred from the story (Yes/No) and (2) whether the Evidence supports the Claims (Yes/Partially/No). The study includes 42 authors spanning all five sources, each represented by one sampled story resulting in 188 annotated claims. Of these, 12 stories are annotated by all annotators to measure inter-rater agreement, while each annotator evaluates 10 additional exclusive stories. Results show unanimous agreement on the `Yes' label for Claim inference and moderate agreement on Evidence support (Krippendorff’s Alpha$=.57$), with 93\% of Evidence fully supporting Claims and 7\% providing partial support. These findings confirm the reliability and quality of the Author Writing Sheets and support their use in personalized story generation (see Appendix~\ref{app:human-author-sheets} for more details).


\section{Personalized Story Generation}

Stage 2 of Figure~\ref{fig:method} illustrates our proposed method for personalized story generation using the Author Writing Sheet. 

To generate personalized stories, we prompt an LLM (\(\text{LLM\textsubscript{story}}\)) with the writing prompt, story length \citep{chakrabarty2024art}, source-specific metadata (e.g., fanfiction for AO3), story rules as actionable instructions in direct second-person form (categorized into four narrative categories), and persona description obtained from the Author Writing Sheet. 
%
We experiment with four types of methods for personalized story generation: 
1) A non-personalized baseline, \emph{Average Author}; 
2) Personalization baselines, \emph{RAG} and \emph{Delta}, that do not use the Author Writing Sheet; 
3) Our proposed methods, \emph{Writing Sheet} and \emph{Writing Summary}; 
and 
4) An \emph{Oracle} method that loosely resembles the upper bound performance. We detail each method type below.
See Appendix~\ref{app:story-gen} for the prompts. 


\paragraph{Non-Personalized - Average Author:}
%\label{sec:avg-author}
\hypertarget{sec:avg-author}{}
The \emph{Average Author} method serves as a non-personalized baseline that reflects the average behavior of authors, acquired during an LLM's pre-training process. Specifically, for each given source, we prompt an LLM to generate stories using an Average Author prompt that describes typical writing characteristics for the given story source obtained using audited GPT-4o prompting\footnote{We ask GPT-4o for the typical writing characteristics for the story source and manually verify it.} \citep{wang-etal-2024-rolellm}. For example, for AO3, the Average Author is defined as ``respecting fandom tone and style while experimenting with established tropes, unconventional pairings, or alternative universes.'' 

\paragraph{Personalization Baselines}

\paragraph{RAG:}
\hypertarget{sec:rag}{}
The Retrieval-Augmented Generation, \emph{RAG} baseline \citep{salemi-etal-2024-lamp} first retrieves the most similar writing prompt and author-written story from the profiling set using BM25 \citep{robertson2009probabilistic}. We then use the retrieved pair as a one-shot demonstration \citep{wang-etal-2024-rolellm} to elicit role-playing behavior from the LLM, mimicking the retrieved example's style.

\paragraph{Delta:}
\hypertarget{sec:delta}{}
The \emph{Delta} method generates personalized story rules by contrasting the \emph{Average Author} story (see \hyperlink{sec:avg-author}{Average Author}) with the corresponding author-written story for each writing prompt in the profiling set \citep{shashidhar-etal-2024-unsupervised}. We then use \emph{all} writing prompts in the profiling set, along with their corresponding generated story rules that are actionable instructions in direct second-person form, as few-shot demonstrations for the LLM, to generate personalized story rules for a new prompt in the generation set. 


\paragraph{Proposed Method - Writing Sheet and Summary}

\paragraph{Writing Sheet:}
\hypertarget{sec:writing-sheet}{}
Our \emph{Writing Sheet} method uses the Author Writing Sheet (Section~\ref{sec:author-writing-sheet}) to enable personalization (Stage 2 of Figure~\ref{fig:method}). First, we prompt an LLM (\(\text{LLM\textsubscript{persona}}\)) to generate a persona description that summarizes the author's story-writing style as a second-person narrative, included in the system prompt \citep{wang-etal-2024-rolellm, jiang2024evaluating}. Second, we prompt an LLM (\(\text{LLM\textsubscript{rule}}\)) to generate personalized story rules from the Author Writing Sheet tailored to the writing prompt in the generation set, included as constraints in the user prompt \citep{pham-etal-2024-suri}. Additionally, we include a one-shot demonstration, following the approach described in \hyperlink{sec:rag}{RAG} \citep{richardson2023integrating}. 

\paragraph{Writing Summary:}
\hypertarget{sec:writing-summary}{}
As an alternative to the Author Writing Sheet, the \emph{Writing Summary} method leverages the LLM's long-context capabilities \citep{ding2023longnet} by packing all past stories from the profiling set as input. Specifically, we include them in a single input prompt to generate an \emph{Author Writing Summary} in the same format as the Author Writing Sheet. Similar to the Writing Sheet method, persona descriptions, and story rules are derived from the Author Writing Summary and used as constraints for story generation, along with a one-shot demonstration. 
\paragraph{Oracle:}
\hypertarget{sec:oracle}{}

The \emph{Oracle} method establishes an upper bound on personalization performance, in terms of similarity to the ground truth, by using Oracle story rules derived directly from the ground-truth author-written story for each writing prompt in the generation set. These rules are obtained by contrasting the \hyperlink{sec:avg-author}{Average Author} with the ground-truth author story, following the same process as in \hyperlink{sec:delta}{Delta}. Additionally, we include a one-shot demonstration, following the approach described in \hyperlink{sec:rag}{RAG}.

\section{Experiments}
In this section, we describe the implementation details and evaluation methods. 

\subsection{Implementation Details}
We prompt GPT-4o in a chain-of-thought manner \citep{wei2022chain, shashidhar-etal-2024-unsupervised} with a temperature of 0.0 and a maximum token limit of 4096 to generate story rules for the Delta and Oracle methods, as well as for generating the Author Writing Sheet, Author Writing Summary, and their respective persona descriptions (\(\text{LLM\textsubscript{persona}}\)) and personalized story rules (\(\text{LLM\textsubscript{rule}}\)). For story generation (\(\text{LLM\textsubscript{story}}\)) across all methods, including \emph{Average Author}, we evaluate three models: GPT-4o \citep{bubeck2023sparks}, Llama 3.1 8B, and Llama 3.1 70B \citep{dubey2024llama}, using a temperature of 0.7 and top\_p of 0.95 \citep{wang-etal-2024-rolellm} (See Appendix~\ref{app:llama-results} for results using Llama models for story generation). Additionally, we implement an ablation variant of our proposed method for Writing Sheet and Writing Summary that excludes the persona description from the system prompt called \emph{Writing Sheet nP} and \emph{Writing Summary nP} respectively.  


\subsection{Automatic Evaluation}

We evaluate the performance of personalized story generation using two automated evaluation methods: 
First, we use GPT-4o-as-a-judge which is shown to be a scalable way to approximate human judgments for complex open-ended tasks \citep{zheng2023judging} and calculate the win-rate of our personalization methods against the non-personalized method, \emph{Average Author} on two aspects: 
\begin{itemize}[noitemsep, topsep=0pt]
    \item \textbf{Faithfulness to Writing History}: We evaluate how well the generated story aligns with the author history by measuring win-rates using Claims from the \hyperlink{sec:writing-summary}{Author Writing Summary} as the ground-truth reference for each narrative category \citep{wang-etal-2024-rolellm, wang2023automated, yunusov-etal-2024-mirrorstories}.
    \item \textbf{Similarity to Author Story}: We assess how closely the generated story matches the author's ground-truth story by measuring win-rates using the ground-truth author story as the reference for each narrative category \citep{lyu2024href, shashidhar-etal-2024-unsupervised}.
\end{itemize}
For both aspects, we prompt GPT-4o in a chain-of-thought manner \citep{wei2022chain} to assign scores (1–5) to stories. We randomly shuffle the order of compared pairs, one generated by a personalization method and the other by the \emph{Average Author} method, to avoid biases. Scores are broken down into four narrative categories: Plot, Creativity, Development, and Language Use \citep{saha-etal-2024-branch}. The story with the higher score in a category is declared the winner, and the overall winner is based on the highest total score across all categories. Additionally, we evaluate personalization using traditional metrics \citep{xie-etal-2023-next} including lexical overlap, story diversity, and stylistic similarity (see Appendix~\ref{app:trad-merics} for more details).



\subsection{Human Evaluation}
\label{sec:human-eval-story-gen}

We conduct a human evaluation on a subset of our data to further assess subtext and identify any interesting insights that the LLM judges might have missed \citep{chakrabarty2024art, subbiah-etal-2024-reading}, by comparing stories generated by personalization methods (\emph{Delta}, \emph{Writing Sheet}, \emph{Writing Summary}) against the \emph{Average Author} method for similarity to the ground-truth author story \citep{lyu2024href}. We ask annotators to justify their choices after making the comparison. Three annotators, recruited via Upwork (\$17/hour), evaluated 45 author stories in total (15 common to establish agreement and 10 exclusive to each annotator to increase coverage of annotation \citep{song-etal-2024-veriscore}). Each author story had three pairs, each containing a story from one personalization method, resulting in 135 annotated story pairs.


\section{Results and Discussion}
We now discuss our experimental results and findings. 

\begin{table*}[htbp]
\centering
\caption{Percentage win-rate for Faithfulness to Writing History evaluated by GPT-4o. Each cell (`X-Y') shows `X' as the method win-rate and `Y' as the Average Author win-rate, with ties as the remainder. Best win-rates for each source is underlined.}
\label{tab:faith-auth-history}
\small
\begin{tabular}{p{2cm} p{1.5cm} p{1.5cm} p{1.5cm} p{1.5cm} p{1.5cm} p{1.5cm} p{1.75cm}}
\toprule
\textbf{Source} & \textbf{Oracle} & \textbf{RAG} & \textbf{Delta} & \textbf{Writing Sheet} & \textbf{Writing Summary} & \textbf{Writing Sheet nP} & \textbf{Writing Summary nP} \\
\midrule
AO3 & 52-32 & 31-54 & 49-41 & \underline{74}-20 & \underline{80}-10 & 68-21 & 72-19 \\
Reddit & 74-21 & 35-56 & 58-26 & \underline{86}-7 & \underline{89}-7 & 74-19 & 72-19 \\
Storium & 50-40 & 42-45 & 45-42 & 68-15 & \underline{75}-8 & \underline{70}-20 & 68-20 \\
N.Magazine & 71-14 & 57-21 & 50-36 & \underline{79}-7 & 71-21 & 71-21 & 71-0 \\
New Yorker & 53-40 & 27-67 & 53-40 & \underline{67}-20 & \underline{80}-20 & 67-27 & \underline{80}-13 \\
Overall & 60-29 & 38-49 & 51-37 & \underline{75}-14 & \underline{79}-13 & 70-22 & 73-14 \\
\bottomrule
\end{tabular}
\end{table*}

\begin{table*}[htbp]
\centering
\caption{Percentage win-rate for Similarity with Author Story evaluated by GPT-4o and Human (last column). Each cell (`X-Y') shows `X' as the method win-rate and `Y' as the Average Author win-rate, with ties as the remainder. Best win-rates for each source is underlined.}
\label{tab:sim-auth-story}
\small
\begin{tabular}{p{2cm} p{1cm} p{1cm} p{1cm} p{1.25cm} p{1.5cm} p{1.25cm} p{1.75cm} p{1.25cm}}
\toprule
\textbf{Source} & \textbf{Oracle} & \textbf{RAG} & \textbf{Delta} & \textbf{Writing Sheet} & \textbf{Writing Summary} & \textbf{Writing Sheet nP} & \textbf{Writing Summary nP} & \textbf{Human Eval} \\
\midrule
AO3 & 71-18 & 30-54 & 45-32 & 40-41 & \underline{60}-30 & 46-38 & 45-40 & 37-33 \\
Reddit & 91-5 & 37-46 & 49-35 & \underline{54}-35 & 49-42 & 47-40 & 42-44 & 52-37 \\
Storium & 60-30 & 45-40 & 35-35 & 52-32 & 50-35 & \underline{57}-14 & 14-29 & 30-30 \\
N.Magazine & 64-21 & 43-36 & 36-43 & 50-21 & \underline{57}-21 & 43-43 & 50-29 & 48-30 \\
New Yorker & 67-13 & 27-47 & 40-40 & 27-60 & 27-47 & 40-60 & \underline{47}-40 & 70-15 \\
Overall & 71-17 & 36-45 & 41-37 & 45-38 & \underline{49}-35 & \underline{47}-39 & 40-36 & 47-29 \\
\bottomrule
\end{tabular}
\end{table*}




\subsection{Automatic Evaluation}
Our automatic evaluation results show that, first, Writing Sheet and Summary consistently outperform other methods. Second, persona descriptions improve faithfulness to writing history but improve similarity to the ground truth only for select sources like Reddit. Third, personalization is more effective for sources like Reddit than AO3 and Storium, and for narrative categories like Creativity, and Language Use, compared to Plot.


\subsubsection{Faithfulness to Writing History}


Table~\ref{tab:faith-auth-history} shows results for faithfulness to writing history using GPT-4o as \(\text{LLM\textsubscript{story}}\) for our personalization methods. We observe the following:

\paragraph{Writing Sheet and Summary achieve best faithfulness to writing history:}  

Writing Sheet and Summary achieve the highest win-rates across all sources, outperforming Average Author by 64\%. The strong performance of the Writing Summary follows from its reliance on the Author Writing Summary as both a reference for evaluation and a generation constraint. Writing Sheet, despite not being explicitly conditioned on the Author Writing Summary, generalizes well, likely due to its format being the same as the Author Writing Summary. Oracle outperforms Delta and RAG, benefiting from direct conditioning on ground-truth story rules, which enhances adherence to the author's writing history. However, Oracle underperforms the Writing Sheet and Summary, likely due to its lack of explicit conditioning on the Author Writing Summary. Another possible cause is that the author's writing style possibly changes over time, which is not fully captured by the ground-truth story rules. See Appendix~\ref{app:faith-auth-history} for an example.


\paragraph{Reddit shows the highest win-rates among sources:} 
Reddit exhibits the highest win-rates across all methods, suggesting its stories are more conducive to personalization. This trend likely stems from Reddit's crowd-sourced writing prompts, where authors create prompts for others. Therefore, there are a range of diverse topics, ranging from power, survival, and mystery to humor and the supernatural. The broad thematic variety may have enabled personalization methods to capture distinct stylistic elements more effectively, resulting in higher win-rates (See the paragraph Themes in Appendix~\ref{app:dataset}).

\paragraph{Persona descriptions improve faithfulness to writing history:}
We see that adding persona descriptions in the system prompt for Writing Sheet and Summary improves win-rates by around 5\% over their nP variants. This result likely follows from personas enhancing the LLM’s role-playing capabilities, allowing for a more faithful representation of the author’s writing style \citep{jiang2024evaluating, wang-etal-2024-rolellm, wallace2024instruction}.


\subsubsection{Similarity to Author Story}

Table~\ref{tab:sim-auth-story} shows results for similarity to the author story using GPT-4o as \(\text{LLM\textsubscript{story}}\) for our personalization methods. We observe the following:

\paragraph{Writing Sheet and Summary outperform baselines of RAG and Delta:} 
Oracle achieves the highest win-rates across all methods as its story rules come from the ground-truth author story. Reddit shows the highest Oracle win-rate (91\%), followed by 65\% on average for other methods, indicating easier personalization likely due to the broad thematic variety in the writing prompts. Writing Sheet and Summary outperform RAG and Delta by 11\% and 6\%, highlighting the benefit of summarizing the author's writing history \citep{richardson2023integrating}.


\paragraph{Persona descriptions help in sources with broad thematic variety in writing prompts:} 
Writing Summary with persona descriptions outperforms its nP variant by 9\% across all sources. However, Writing Sheet shows no significant difference from its nP variant except for Reddit, where it surpasses it by 7\%. This result suggests that persona descriptions enhance personalization in sources that are easier to personalize, such as Reddit, but have limited impact elsewhere \citep{zheng-etal-2024-helpful}. The broad thematic variety in Reddit's writing prompts creates a greater divergence between author styles and the Average Story, leading to more informative Author Writing Sheets and persona descriptions that further aid personalization. See Appendix~\ref{sec:app-persona-sim-story} for an example.


\paragraph{Creativity and Language Use are the easiest categories to personalize:}


\begin{figure}[htbp]
\centering
\includegraphics[width=0.8\linewidth]{figures/cat_wise/Reddit.pdf}
\caption{Win-rate of personalization methods vs. Average Author for Reddit across four narrative categories. D: Delta, S: Writing Summary, WS: Writing Sheet.}\label{fig:reddit_win_rates}
\end{figure}

Table~\ref{tab:category_comparison_column} (in the Appendix) shows that Delta, Writing Summary, and Writing Sheet outperform Average Author in Creativity (23\% higher) and Language Use (15\% higher) \citep{huot2024agents}, but perform worse in Plot and slightly worse in Development \citep{tian-etal-2024-large-language, xu2024echoes}. Creativity and Language Use are less dependent on the writing prompt, making them easier to transfer across prompts, whereas Plot and Development are more closely tied to specific prompts, making generalization difficult, especially without thematic overlap.  

To better understand personalization performance in Reddit, the easiest source to personalize, Figure~\ref{fig:reddit_win_rates} shows that Writing Sheet is the best-performing method across all categories, with a 35\% advantage over Delta and 20\% over Writing Summary in Creativity. Writing Sheet explicitly summarizes differences between the ground-truth author story and the Average Story, providing more informative guidance for personalization. This advantage is particularly evident in Creativity, which is less constrained by the writing prompt. See Appendix~\ref{app:cat-wise-results} for plots of other data sources.

\subsubsection{Traditional Metrics}  
Table~\ref{tab:traditional-metrics} (in the Appendix) shows results with the traditional metrics. 

\paragraph{Lexical Overlap and Diversity Metrics yield similar values:}
We observe that Lexical Overlap and Diversity metrics yield similar values, as all methods use the same generation model, leading to overlapping lexical distributions and limiting these metrics' ability to capture nuanced stylistic differences \citep{zheng2023judging, xie-etal-2023-next}. 

\paragraph{Writing Sheet outperforms other methods, especially for Style metrics:} We observe that Writing Sheet nP consistently performs best across all metrics, particularly in Style metrics for both Author History and Author Story, as measured by LUAR \citep{rivera-soto-etal-2021-learning}. This improvement likely results from the Writing Sheet explicitly summarizing an author's stylistic deviations from an Average Author, enhancing personalization.


\subsection{Human Evaluation}

Table~\ref{tab:sim-auth-story} shows win-rates from human evaluation (Section~\ref{sec:human-eval-story-gen}) under the Human Eval column, showing that personalization methods—Delta, Writing Sheet, and Summary—outperform the Average Author method by \emph{18\% points} absolute win-rate in terms of similarity to ground-truth writings. See Appendix~\ref{app:human-eval-story-gen} for details on experiment design, and qualitative examples. 

\paragraph{Low inter-annotator agreement due to task subjectivity:}
Fleiss’ Kappa for inter-annotator agreement ranges from 0.2 to 0.4, indicating low but above-random agreement, likely due to the challenges of annotating long-form subjective texts \citep{subbiah-etal-2024-storysumm}. Specifically, different annotators prioritize different narrative categories when determining which story in a pair better reflects the author’s style, leading to disagreement. 

\paragraph{Sources like Reddit and New Yorker have better win-rates than others:} 
Reddit and New Yorker achieve the highest win-rates, with Reddit benefiting from a broad thematic variety in writing prompts and New Yorker from advanced narrative devices like subtext, which humans assess more reliably than LLMs \citep{subbiah-etal-2024-reading}. In contrast, AO3 and Storium exhibit lower win-rates due to LLMs' familiarity with fanfiction tropes (AO3)\footnote{\url{https://archiveofourown.org/admin_posts/25888}} and the lack of strong stylistic differentiation in Storium beyond its open-ended storytelling structure \citep{xu2024echoes, tian-etal-2024-large-language}. These findings suggest that personalization is most effective when authors engage with diverse topics or possess distinct stylistic traits (such as complex narrative structures, creative elements, and advanced language use). These characteristics, reflected in their past writing history, provide clear signals for us to mimic their writing style.

\paragraph{Narrative categories like Creativity, and Language use are more conducive to personalization:} 
Annotators favor personalized stories for Creativity, and Language Use, noting their stronger use of deeper symbolism, thematic richness, layered narratives, and expressive language. However, Average Author was preferred when personalized methods incorrectly introduce elements not present in the author's ground-truth story, suggesting deviations from the author's usual style. 

\section{Related Work}

\subsection{Personalization and Role-Playing}
Recent works have introduced benchmark datasets for personalizing LLM outputs in tasks like email, abstract, and news writing, focusing on shorter outputs (e.g., 300 tokens for product reviews \citep{kumar2024longlamp} and 850 for news writing \citep{shashidhar-etal-2024-unsupervised}). These methods infer user traits from history for task-specific personalization \citep{sun-etal-2024-revealing, sun-etal-2025-persona, pal2024beyond, li2023teach, salemi2025reasoning}. In contrast, we tackle the more subjective problem of long-form story writing, with author stories averaging 1500 tokens. Unlike prior role-playing approaches that use predefined personas (e.g., Tony Stark, Confucius) \citep{wang-etal-2024-rolellm, sadeq-etal-2024-mitigating, tu2023characterchat, xu2023expertprompting}, we propose a novel method to infer story-writing personas from an author’s history to guide role-playing.


\subsection{Story Understanding and Generation}  
Prior work on persona-aware story generation \citep{yunusov-etal-2024-mirrorstories, bae-kim-2024-collective, zhang-etal-2022-persona, chandu-etal-2019-way} defines personas using discrete attributes like personality traits, demographics, or hobbies. Similarly, \citep{zhu-etal-2023-storytrans} explore story style transfer across pre-defined domains (e.g., fairy tales, martial arts, Shakespearean plays). In contrast, we mimic an individual author's writing style based on their history. Our approach differs by (1) inferring long-form author personas—descriptions of an author’s style from their past works, rather than relying on demographics, and (2) handling long-form story generation, averaging 1500 tokens per output, exceeding typical story lengths in prior research.  

\section{Conclusion}
We reveal a tradeoff in robust watermarks: Improved redundancy of watermark information enhances robustness, but increased redundancy raises the risk of watermark leakage. We propose DAPAO attack, a framework that requires only one image for watermark extraction, effectively achieving both watermark removal and spoofing attacks against cutting-edge robust watermarking methods. Our attack reaches an average success rate of 87\% in detection evasion (about 60\% higher than existing evasion attacks) and an average success rate of 85\% in forgery (approximately 51\% higher than current forgery studies). 

\newpage

\section{Limitations}

\paragraph{Limited Author History Size:} 
Table~\ref{tab:dataset-stats} (70\% of the Stories/Author column) shows that N.Magazine and New Yorker authors have an average of 3 stories in their profiling set, whereas Reddit and AO3 authors have around 10. A larger profiling set likely enables more accurate inference of an author's story-writing characteristics, leading to improved personalization. However, the cognitive demands of writing long stories result in fewer stories per author available online, thereby limiting the scalability of personalized story generation research. 

\paragraph{Author Writing Sheet Recall:} We validate the precision of our Author Writing Sheet through human evaluation, ensuring that the Claim-Evidence pairs accurately describe the author's story-writing style. However, we do not assess recall, i.e., whether \emph{all} of the author's story-writing characteristics are captured. Measuring recall is challenging due to the inherent subjectivity of the task, making it impractical to ask humans to independently construct Writing Sheets for comparison.

\paragraph{Multi-Agent Systems and Fine-Tuning LLMs:} Our work explores role-playing for personalized story generation by prompting an LLM with tailored persona descriptions and story rules. Future research can extend this by incorporating multi-agent systems, where specialized agents focus on different narrative categories to enhance personalization \citep{huot2024agents, bae-kim-2024-collective}. Additionally, fine-tuning LLMs to explicitly reason over author history could be investigated as a method for better aligning generated content with individual author preferences \citep{salemi2025reasoning, shashidhar-etal-2024-unsupervised, shaikh2024show}.

\paragraph{Automatic Evaluation Methods:} Evaluating long-form text remains challenging for LLMs, particularly for creative outputs that involve nuanced aspects like subtext, which are difficult to assess automatically and often require human evaluation \citep{subbiah-etal-2024-storysumm}.
 
\section{Ethical Considerations}
Both human evaluation tasks were approved by an institutional review board (IRB). All annotators, who are US-based and fluent in English, were informed of the nature of the research study, to which they provided informed consent, and were compensated at an hourly rate of \$17, meeting the minimum wage requirements in our state. All scientific artifacts, including models and datasets, were used in accordance with their intended purpose to ensure ethical and responsible research practices. 

All data used in this study were manually collected from publicly available sources\footnote{Permission was obtained from the authors of the Storium dataset for its inclusion in Mythos.}, adhering to the data usage and crawling policies of the respective websites. To comply with copyright constraints, we will release only the links to the stories used in this research, rather than the stories themselves, following the approach of \citep{chakrabarty2024art}.  

Fully anonymizing our dataset is not feasible\footnote{Storium authors already anonymize their dataset.} as the research involves mimicking specific authors’ writing behaviors. We note that even if the usernames are anonymized, it is still possible to infer an author's publicly available identity from the story link. Therefore, for each author for every source, we release their usernames along with corresponding writing prompts, and author-written stories in the profiling and generation set of our dataset. However, the dataset remains limited in scale, and no additional metadata, including demographic information, is included, thereby reducing potential risks. To further mitigate unforeseen harms, the dataset will be released under an **Educational or Academic Research, Non-Commercial Use** license, following \citep{akoury-etal-2020-storium}.

\newpage
\bibliographystyle{acl_natbib}
% This must be in the first 5 lines to tell arXiv to use pdfLaTeX, which is strongly recommended.
\pdfoutput=1
% In particular, the hyperref package requires pdfLaTeX in order to break URLs across lines.

\documentclass[11pt]{article}

% Change "review" to "final" to generate the final (sometimes called camera-ready) version.
% Change to "preprint" to generate a non-anonymous version with page numbers.
\usepackage[preprint]{acl}
\usepackage{tablefootnote}
% Standard package includes
\usepackage{times}
\usepackage{latexsym}
\usepackage{pifont}

% For proper rendering and hyphenation of words containing Latin characters (including in bib files)
\usepackage[T1]{fontenc}
% For Vietnamese characters
% \usepackage[T5]{fontenc}
% See https://www.latex-project.org/help/documentation/encguide.pdf for other character sets

% This assumes your files are encoded as UTF8
\usepackage[utf8]{inputenc}

% This is not strictly necessary, and may be commented out,
% but it will improve the layout of the manuscript,
% and will typically save some space.
\usepackage{microtype}

% This is also not strictly necessary, and may be commented out.
% However, it will improve the aesthetics of text in
% the typewriter font.
\usepackage{inconsolata}

%Including images in your LaTeX document requires adding
%additional package(s)
\usepackage{graphicx}
\usepackage{color}
\usepackage{multirow}
\usepackage{amsmath}
\usepackage{array}
\usepackage{booktabs}
\usepackage{float} 
\usepackage{arydshln}
\usepackage{subcaption}
\usepackage{xspace}
\usepackage{makecell}
%\usepackage{tabularray}
%\usepackage{tikz}
%\newcommand*\circled[1]{\tikz[baseline=(char.base)]{
%            \node[shape=circle,draw,inner sep=2pt] (char) {#1};}}
            
\newcommand{\jz}[1]{{\color{red}{\bf{[JZ:]}} #1}}
\newcommand{\addexp}[1]{{\color{orange}{\bf{[AddExp:]}} #1}}
\newcommand{\sxfix}[1]{{\color{blue}#1}}

%\newcommand{\model}{RG$^2$-KBQA}
\newcommand{\model}{\textsc{SG-KBQA}\xspace}

% If the title and author information does not fit in the area allocated, uncomment the following
%
%\setlength\titlebox{<dim>}
%
% and set <dim> to something 5cm or larger.

% \title{Knowledge Base Question Answering with Generalizable Logical Form Generation}
%\title{Schema-Guided Generalizable Knowledge Base Question Answering}
\title{Beyond Seen Data: Improving KBQA Generalization Through Schema-Guided Logical Form Generation}
%JHL1: how about something cuter like "Beyond Seen Data: Improving KBQA Generalization Through Schema-Guided Logical Form Generation"


% Author information can be set in various styles:
% For several authors from the same institution:
% \author{Author 1 \and ... \and Author n \\
%         Address line \\ ... \\ Address line}
% if the names do not fit well on one line use
%         Author 1 \\ {\bf Author 2} \\ ... \\ {\bf Author n} \\
% For authors from different institutions:
% \author{Author 1 \\ Address line \\  ... \\ Address line
%         \And  ... \And
%         Author n \\ Address line \\ ... \\ Address line}
% To start a separate ``row'' of authors use \AND, as in
% \author{Author 1 \\ Address line \\  ... \\ Address line
%         \AND
%         Author 2 \\ Address line \\ ... \\ Address line \And
%         Author 3 \\ Address line \\ ... \\ Address line}

\author{
  Shengxiang Gao  \hspace{10mm} Jey Han Lau  \hspace{10mm} Jianzhong Qi \vspace{3mm} \\
  School of Computing and Information Systems, The University of Melbourne \\
  \texttt{shengxiang.gao1@student.unimelb.edu.au} \\ 
  \texttt{\{jeyhan.lau, jianzhong.qi\}@unimelb.edu.au}\\
}


% \author{First Author \\
%   Affiliation / Address line 1 \\
%   Affiliation / Address line 2 \\
%   Affiliation / Address line 3 \\
%   \texttt{email@domain} \\\And
%   Second Author \\
%   Affiliation / Address line 1 \\
%   Affiliation / Address line 2 \\
%   Affiliation / Address line 3 \\
%   \texttt{email@domain} \\}

%\author{
%  \textbf{First Author\textsuperscript{1}},
%  \textbf{Second Author\textsuperscript{1,2}},
%  \textbf{Third T. Author\textsuperscript{1}},
%  \textbf{Fourth Author\textsuperscript{1}},
%\\
%  \textbf{Fifth Author\textsuperscript{1,2}},
%  \textbf{Sixth Author\textsuperscript{1}},
%  \textbf{Seventh Author\textsuperscript{1}},
%  \textbf{Eighth Author \textsuperscript{1,2,3,4}},
%\\
%  \textbf{Ninth Author\textsuperscript{1}},
%  \textbf{Tenth Author\textsuperscript{1}},
%  \textbf{Eleventh E. Author\textsuperscript{1,2,3,4,5}},
%  \textbf{Twelfth Author\textsuperscript{1}},
%\\
%  \textbf{Thirteenth Author\textsuperscript{3}},
%  \textbf{Fourteenth F. Author\textsuperscript{2,4}},
%  \textbf{Fifteenth Author\textsuperscript{1}},
%  \textbf{Sixteenth Author\textsuperscript{1}},
%\\
%  \textbf{Seventeenth S. Author\textsuperscript{4,5}},
%  \textbf{Eighteenth Author\textsuperscript{3,4}},
%  \textbf{Nineteenth N. Author\textsuperscript{2,5}},
%  \textbf{Twentieth Author\textsuperscript{1}}
%\\
%\\
%  \textsuperscript{1}Affiliation 1,
%  \textsuperscript{2}Affiliation 2,
%  \textsuperscript{3}Affiliation 3,
%  \textsuperscript{4}Affiliation 4,
%  \textsuperscript{5}Affiliation 5
%\\
%  \small{
%    \textbf{Correspondence:} \href{mailto:email@domain}{email@domain}
%  }
%}

\begin{document}
\maketitle
\begin{abstract}
Knowledge base question answering (KBQA) aims to answer user questions in natural language using rich human knowledge stored in large KBs. As current KBQA methods struggle with unseen knowledge base elements at test time,
%State-of-the-art KBQA solutions are based on semantic parsing and have two core steps: (1) Generation: generate a sequence of structured query operators, and (2) Retrieval: retrieve KB elements (entities and relations). The operators and KB elements together form a structured query (so-called ``logical form'') over the KB to answer user questions. We observe that solutions starting with either step miss guidance from the other step, hence yielding suboptimal outcomes.
%To address this limitation, we propose a model named \textbf{\model} with a novel processing paradigm that consists of a \emph{\underline{g}enerative entity \underline{r}etrieval} module and a \emph{\underline{r}etrieval-guided logical form \underline{g}eneration} module. The generative entity retrieval module generates primitive logical forms based on user questions and relations extracted from the questions, to guide KB entity retrieval with higher accuracy. 
%The retrieval-guided logical form generation module then generates the final logical forms based on the KB elements extracted.
we introduce \textbf{\model}: a novel model that injects schema contexts into entity retrieval and logical form generation to tackle this issue. 
It uses the richer semantics and awareness of the knowledge base structure provided by schema contexts to enhance generalizability. 
%\sxfix{The schema contexts describes relationships between elements in the knowledge base, providing richer semantics and awareness of its structure.}
%JHL1: can we give an intuitive, high level explanation on the idea? just 1-2 lines max to capture the core idea
We show that \model\ achieves strong generalizability, outperforming state-of-the-art models on two commonly used benchmark datasets across a variety of test settings. 
%Our source code is available at \url{https://anonymous.4open.science/r/SG-KBQA-7895}. 
Our source code is available at \url{https://github.com/gaosx2000/SG_KBQA}.
%Code will be released upon paper publication.
%Our source code is available at \url{https://anonymous.4open.science/r/SG-KBQA-7895}. 
%JHL1: use anoymised github (https://anonymous.4open.science/)
%with a novel processing paradigm that consists of a \emph{\underline{g}enerative entity \underline{r}etrieval} module and a \emph{\underline{r}etrieval-guided logical form \underline{g}eneration} module. The generative entity retrieval module generates primitive logical forms based on user questions and relations extracted from the questions, to guide KB entity retrieval with higher accuracy. 
%Experimental results confirm the effectiveness of \model, which outperforms state-of-the-art models on two commonly used benchmark datasets GrailQA and WebQSP across a variety of test settings. 
\end{abstract}


\section{Introduction}



% These instructions are for authors submitting papers to *ACL conferences using \LaTeX. They are not self-contained. All authors must follow the general instructions for *ACL proceedings,\footnote{\url{http://acl-org.github.io/ACLPUB/formatting.html}} and this document contains additional instructions for the \LaTeX{} style files.

% The templates include the \LaTeX{} source of this document (\texttt{acl\_latex.tex}),
% the \LaTeX{} style file used to format it (\texttt{acl.sty}),
% an ACL bibliography style (\texttt{acl\_natbib.bst}),
% an example bibliography (\texttt{custom.bib}),
% and the bibliography for the ACL Anthology (\texttt{anthology.bib}).

{Knowledge base question answering} (KBQA) aims to answer user questions expressed in natural language with information from a {knowledge base}~(KB). This offers user-friendly access to rich human knowledge stored in large KBs such as Freebase~\cite{bollacker_freebase_2008}, DBPedia~\cite{auerDBpediaNucleusWeb2007} and Wikidata~\cite{vrandecic_wikidata_2014}, and it has broad applications in QA systems~\cite{zhou_commonsense_2018}, recommender systems~\cite{guo_survey_2022}, and information retrieval systems~\cite{jalota_lauren_2021}.

\begin{figure}[t]
\small
    \centering
    \includegraphics[width=\columnwidth]{figures/kbqa_example_new.png}
    \caption{Example of KBQA and SP-based solutions.}
    \label{fig:kbqa_example}
\end{figure}

State-of-the-art (SOTA) solutions often take a {semantic parsing} (SP)-based approach. They translate an input natural language question into a structured, executable form (AKA {logical form}~\cite{lan_survey_2021}), which is then executed to retrieve the question answer. Figure~\ref{fig:kbqa_example} shows an example. The input question, \textsf{Who is the author of Harry Potter}, is expressed using the \emph{S-expression}~\cite{gu_beyond_2021} (a type of logical form), which is formed by a set of functions (e.g., \textsf{JOIN}) operated over elements of the target KB (e.g., entity \textsf{m.078ffw} refers to book series \textsf{Harry Potter}, \textsf{book.author} a class of entities, and \textsf{book.literary\_series.author} a relation in Freebase).

% \sxfix{However, the rich semantics and complex structure of KBs lead to two key challenges: (1) KB elements mapping: how to learn a mapping between mentions of entities and relations in the input question to corresponding KB elements? (2) Executable logical form generation: how to generate a logical form that aligns with the question's semantics and adheres to the structural constraints (schema) of the KB?

A key challenge here is to learn a mapping between mentions of entities and relations in the input question to corresponding KB elements to form the logical form. Meanwhile, the mapping of KB element compositions has to adhere to the structural constraints (schema) of the KB. The schema defines entities' classes and the relationships between these classes within the KB. Take the KB subgraph in Figure~\ref{fig:kbqa_example} as an example, the relationship between the entity \textsf{Harry Potter} and the entity \textsf{J.K. Rolwing} is defined by the relation \textsf{book.literary\_series.author} between their respective classes (i.e., class \textsf{book.literary\_series} and class \textsf{book.author}).

\begin{figure}[t]
    \small
    \centering
    \includegraphics[width=\columnwidth]{figures/core_modules.png}
    \caption{Relation-guided entity mention detection and schema-guided logical form generation.}
    \label{fig:core_novelty}
\end{figure}

However, due to the vast number of entities, relations, classes, and their compositions, it is difficult (if not impossible) to train a model with all feasible compositions of the KB elements. For example, Freebase~\cite{bollacker_freebase_2008} has over 39 million entities, 8,000 relations, and 4,000 classes. Furthermore, some KBs (e.g., NED~\cite{mitchell_ned_2018}) are not static as they continue to grow. 

A few studies consider model generalizability to non-I.I.D. settings, where the test set contains schema items (i.e., relations and classes) or compositions that are unseen during training (i.e., \emph{zero-shot} and \emph{compositional generalization}, respectively). In terms of methodology, these studies typically use {ranking-based} or {generation-based} models. 
Ranking-based models~\cite{gu_beyond_2021, gu_dont_2023} retrieve entities relevant to the input question and then, starting from them, perform path traversal in the KB to obtain the target logical form by ranking. Generation-based models~\cite{shu_tiara_2022, zhang_fc-kbqa_2023} retrieve relevant KB contexts (e.g., entities and relations) for the input question, and then feed these contexts into a Seq2Seq model together with the input question to generate the logical form.



%~\cite{gu_arcaneqa_2022,shu_tiara_2022,ye_rng-kbqa_2022,gu_dont_2023,zhang_fc-kbqa_2023,faldu_retinaqa_2024}

%To solve the generalization problem, most existing KBQA approaches follow the retrieve-and-generate framework, which enhances logical form generation using retrieved KB elements (entities, relations, and classes).~\cite{shu_tiara_2022, faldu_retinaqa_2024, gu_dont_2023,zhang_fc-kbqa_2023,ye_rng-kbqa_2022, gu_arcaneqa_2022}. Despite the promising results achieved by these works, significant challenges remain: 

We observe that both types of models terminate their entity retrieval prematurely, such that each entity mention in the input question is mapped to only a single entity before the logical form generation stage. As a result, the logical form generation stage loses the freedom to explore the full combination space of relations and entities. This leads to inaccurate logical forms (as validated in our study).

%

To address this issue, our strategy is to defer entity disambiguation --- i.e., to determine the most relevant entity for an entity mention (Section~\ref{sec:literature}) --- to the logical form generation stage. This allows our model to explore a larger combination space of the relations and entities, and ultimately leads to stronger model generalizability because low-ranked (but correct) relations or entities would still be considered during generation.
%A larger search space brings new challenges to identify the correct combination.
We call our approach \model (\underline{s}chema-\underline{g}uided logical form generator for \underline{KBQA}). Concretely, \model\ follows the generation-based approach but with deferred entity disambiguation. As shown in Figure~\ref{fig:core_novelty}, it feeds the input question, the retrieved candidate relations and entities, plus their corresponding schema information (the domain and range of classes of relations and entities; Section~\ref{sec:method}) into a large language model (LLM)
%JHL1: are they LLMs? if so let's just use LLMs henceforth (and avoid introducing another acronym)
for logical form generation. The schema information reveals the connectivity between the candidate relations and entities, hence guiding the LLM to uncover their correct combination in the large search space. 
%JHL1: I struggle to understand figure 2 - i can see they are different, but not sure what the yellow box means, what the pink highlighted boxes mean. and what is schema information? can we have a toy example of what the input looks like to the LM? I think what's important in figure 2 is to give a concrete example of the input to the LM for our model; and if there's space, contrast that with the input in SOTA models


Further exploiting the schema-guided idea, we propose a relation-guided module for \model\  to enhance its entity mention detection from the input question. As shown in Figure~\ref{fig:core_novelty}, this module adapts a Seq2Seq model to generate logical form sketches based on the input question and candidate relations, where relations, classes, and literals are masked by special tokens, such that the entity mentions can be identified more easily without confusions caused by these elements. 
 %which extracts entity mentions from  generated by a generator that consumes the input question and the selected relations. The extracted entity mentions are further utilized to retrieve and select top-ranked candidate entities from the KB, guided by the schemas provided by the selected relations. 
 
 %Our approach leverages the selected schema items to guide the entity retrieval process and effectively incorporates the schema context through GenMD to achieve mention detection with a more global perspective. This significantly improves the accuracy of entity retrieval in compositional and zero-shot settings.


%\textbf{Entity retrieval (linking) remains challenging in zero-shot and compositional generalization settings.} Traditional methods first perform mention detection and then retrieve candidate entities from the KB. For each mention, a ranking model is used for entity disambiguation, selecting the candidate entity most relevant to the question for use in the subsequent generation stage. However, mention detection methods proposed in the existing literature (e.g., NER or span classification) often fail when faced with questions containing unseen schema items. This is because some schema items in the KB contain nouns that could potentially be recognized as entities. For example,\ldots. Unseen schema items introduce ambiguous information in the question, which confuses the model and makes it challenging to accurately identify entity mention boundaries.



%\textbf{Error propagation and lack of global reasoning in the disjointed traditional retrieve-and-generate framework}. Previous KBQA works adopt a disjoint retrieve-and-generate framework, where candidate entities are disambiguated before logical form generation to narrow down the search space~\cite{shu_tiara_2022, pang_survey_2022, zhang_fc-kbqa_2023, ye_rng-kbqa_2022,faldu_retinaqa_2024}. However, this approach fixes entity choices without considering their interactions with relations and other entities in the query, leading to locally optimal but globally inconsistent entity-relation selections. Moreover, errors in entity disambiguation propagate through the pipeline, misleading subsequent logical form generation.
% Furthermore, due to encountering new semantic relations or contexts that are not present in the training data, the model often fails to match the unseen schema items or compositions with correct entities in the KB. 

% \textbf{The completely decoupled retrieval and generation processes lead to error propagation through the pipeline.} To achieve stronger generalization capabilities, most existing KBQA approaches follow the retrieve-then-generate framework~\cite{shu_tiara_2022, faldu_retinaqa_2024, gu_dont_2023,zhang_fc-kbqa_2023,ye_rng-kbqa_2022, gu_arcaneqa_2022}. They employ an independent retrieval module to retrieve KB elements (e.g. entities, relations, classes) relevant to the input question before generating the target logical form. The retrieved KB elements are then leveraged to narrow down the search space and provide KB context, thereby enhancing the generalization capability. For example, some approaches incorporate the retrieved KB elements as additional inputs to a seq2seq model~\cite{shu_tiara_2022,zhang_fc-kbqa_2023,ye_rng-kbqa_2022}, while others use the retrieved entities as anchors to incrementally expand the logical form through path traversal in the KB~\cite{gu_dont_2023, gu_beyond_2021}. Although retrieval results can enhance the generalization ability of various logical form generation methods, incorrect retrievals can mislead the subsequent generation of logical forms. 

% To address the issues above and achieve a strong zero-shot and compositional generalization capability, we propose \model, a novel KBQA model that has two core modules: \emph{generative entity refinement} (GER) and \emph{refinement-guided logical form generation} (RLG). \model\ retrieves relations relevant to the question and generates \emph{logical form sketches} (that mask the relations and classes which may confuse the detection of the boundaries of entity mentions) by feeding the top-ranked relations and the question into a Seq2Seq model. 
% It then obtains the top-ranked entities from the KB based on the entity mentions in the generated logical form sketches, \emph{leveraging retrieved relations to enhance the zero-shot and compositional generalization of entity refinement}.

% The RLG module integrates entity and relation selection directly into the logical form generation process, to mitigate error propagation between the retrieval and generation stages. Specifically, for each relation included in the input, we provide its two connected classes to capture the semantic constraints of the KB schema. Similarly, for each entity, its associated class is provided to clarify its semantic role within the KB. By integrating these schema annotations into the input of the Seq2Seq model, our approach enables more accurate selection of entities and relations and generates logical forms that are more consistent with the underlying KB structure.





% \model~mitigate the error propagation between the retrieval and genration stages by defering both relation and entity disambiguation to the generation stage. }





% Specifically, it leverages the KB structural context by utilizing the classes to which entities belong and the classes connected by relations to provide connectivity between entities and relations. This context supports the model in selecting the correct combinations of entities and relations. A seq2seq model is then fine-tuned to transform the question, refined entities and relations, and class annotations into the target logical form.


% However, the primary source of errors in existing KBQA systems still lies in the failure of entity retrieval, which propagates through the pipeline and leads to errors in subsequent logical form generation.

% Recent SP-based KBQA approaches typically consist of two key steps: KB element retrieval and logical form generation~\cite{luo_chatkbqa_2024, ye_rng-kbqa_2022, shu_tiara_2022, faldu_retinaqa_2024, zhang_fc-kbqa_2023}. The retrieval of KB element mainly aims to retrieve the KB elements relevant to the input question. Then, these retrieved KB elements are then utilized to generate a complete and executable logical form (e.g., SPARQL, S-expression). However, collecting sufficient training data to cover all possible KB elements and their compositions that may appear in user queries is highly challenging, especially for large-scale KBs with a large number KB elements. \textbf{The broad coverage and combinatorial explosion require KBQA model to handle unseen KB elements (i.e. zero-shot generalization) and unseen compositions of them (i.e. compositional generalization), which remains a significant challenge.}

% It is important to note that in most existing SP-based KBQA systems, the majority of errors primarily arise from inaccuracies in the retrieval of KB elements, particularly in entity retrieval~\cite{gu_dont_2023,shu_tiara_2022,ye_rng-kbqa_2022,faldu_retinaqa_2024, zhang_fc-kbqa_2023}. These errors propagate to the subsequent logical form generation step which takes the retrieved KB elements as part of the input. Previous KBQA studies have proposed various methods for retrieving KB elements, aiming to separately identify the most relevant entities, relations, and classes for a given question. However, \textbf{the lack of consideration for the semantic relationships between KB elements in the question across these independent retrieval processes often leads to errors in the retrieval results.} This issue is particularly pronounced when handling questions involving unseen KB elements, where the independent retrieval processes may misidentify the same part of a question as different types of KB elements. 

%generation side ? thinking about the word limit for this intro....

% The retrieval of KB elements is a critical step in SP-based KBQA methods, aiming to retrieve the KB elements relevant to the input question. 

% SP-based KBQA methods not only need to retrieve KB elements related to the input query, but also generate the operators and functions that align with the semantic of user's query to form a complete and executable logical form. 

% Inspired by the strong generalization ability demonstrated by pre-trained language models (PLMs) across various NLP tasks, researchers have explored leveraging PLMs to address the generalization challenges in KBQA problem~\cite{shu_tiara_2022, ye_rng-kbqa_2022, zhang_fc-kbqa_2023, faldu_retinaqa_2024, gu_dont_2023}.

% To achieve strong generalization, recent SP-based KBQA studies primarily leverage pre-trained language models (PLMs) to retrieve KB elements in the input question and generate the final logical forms based on the retrieved KB elements. 

% Recent SP-based KBQA works have achieved promising results under the I.I.D. assumption, which posits a strong correspondence between the distribution of schema items (classes and relations) in the training data and the test data. However, this assumption does not hold due to user queries potentially involving schema items or novel compositions of them that have not been encountered in the training data. Collecting sufficient training data to cover all entities, schema items, and compositions of them is challenging, especially for large-scale KBs with numerous entities and schema items. 

% Figure~\ref{fig:kbqa_example} shows an example, where the logical form of the input question, `\textsf{Who is the author of Harry Potter}', is expressed using the \emph{S-expression}~\cite{gu_beyond_2021} (a type of logical form), which is formed by a set of functions (e.g., \textsf{JOIN}) operated over elements of the target KB (e.g., entity `\textsf{m.078ffw}' which refers to the book series `\textsf{Harry Potter}', class of entities `\textsf{book.author}', and relation `\textsf{book.literary\_series.author}' in the Freebase KB). More details about the example and the S-expression will be given in Section~\ref{sec:preliminary}. 

% Meanwhile, large language models (LLMs), such as Chat-GPT \citep{brown_language_2020} and LLaMA \citep{touvron_llama_2023}, have demonstrated strong results in various NLP tasks. Previous works have demonstrated the generalization capability of LLMs in understanding natural language and generating formal language \cite{rony_sgpt_2022, li_few-shot_2023}. These works have inspired researchers to enhance KBQA systems by leveraging LLMs as the semantic parser. However, the vast scale and complex structure of KBs present significant challenges for leveraging LLMs in real-world generalization scenarios within KBQA.

% recent SP-based studies~\cite{shu_tiara_2022, zhang_fc-kbqa_2023, faldu_retinaqa_2024, yu_decaf_2023} follow a \emph{sequence-to-sequence} (Seq2Seq)-based framework. They fine-tune 
% \emph{pre-trained language models} (PLMs) such as T5~\cite{raffel_exploring_2023} to translate input questions into logical forms, exploiting the strong semantic understanding capabilities of such models. Due to the special representations adopt by the KBs, the entity IDs, relation names, and class names are not necessarily directly translatable from the mentions of such elements in the input question, e.g., `\textsf{m.078ffw}' vs. `\textsf{Harry Potter}' in Figure~\ref{fig:kbqa_example}. An additional KB element retrieval module is commonly used by existing Seq2Seq SP-based methods, forming either \emph{generate-then-retrieve} (GnR) or \emph{retrieve-then-generate} (RnG) processing paradigms.
To summarise:

\begin{itemize}
    \item We introduce \model\ to solve the KBQA problem under non-I.I.D. settings, where test input contains unseen schema items or compositions during training.
    
    \item We propose to defer entity disambiguation to logical form generation, and additionally guide this generation step with corresponding schema information, allowing us to explore a larger combination space of relations and entities to consider unseen relations, entities, and compositions. We further propose a relation-guided module to strengthen entity retrieval by generating logical form sketches. 
    
    %introduce a generative mention retrieval method that leverages the context of retrieved schema items to address the generalization issues of entity retrieval in compositional and zero-shot settings from a global perspective.
    
    % \item We propose (1) a GER module that guides entity retrieval with logical form sketches that are generated based on retrieved relations, to achieve more accurate entity retrieval, and (2) an RLG module that guides logical form generation with the class contexts of the entities and relations, to achieve more generalizable logical form generation. 
    %\item We bridge the retrieval and generation stages by integrating entity and relation disambiguation into the logical form generation process through the incorporation of the KB schema, thereby reducing error propagation and generating more accurate logical forms that align with the KB structure.
    %novel entity retrieval approach — generative entity retrieval, to address the generalization challenges faced by existing KBQA entity retrieval methods.
    
    \item We conduct experiments on two popular benchmark datasets and find \model\ outperforming SOTA models on both datasets. In particular, on non-I.I.D GrailQA our model tops all three leaderboards for the overall, zero-shot, and compositional generalization settings, outperforming SOTA models by 3.3\%, 2.9\%, and 4.0\% (F1) respectively.

\end{itemize}


%JHL1: intro is mostly good; there's quite a bit of technical novelty to explain so it's not easy to write. Some suggestions to improve it:
%- simplify figure 1: since we mostly use it to explain the conversion process, we can probably drop the later half of the figure (i.e. we just need to keep the NLQ and logical form).
%- we need a figure that gives an example what 'schema information' and 'relation-guided module' and 'logical form sketches' look like. For someone outside this space it's far too abstract at the moment and they are just keywords to me. With that figure, use it to explain the core novelty of our model in the intro; focus on the intuition (leave the details to the later sections).
%-general comment: use emph or it more sparingly. we don't need to emph every new keyword that we introduce; use it when you really want the reader to notice the word - often they are not even special nouns (e.g. sometimes you might use emph to emphasise a negation (we do *not* consider...))

\section{Related Work}\label{sec:literature}

\paragraph{Knowledge Base Question Answering}

Most KBQA 
%(a.k.a. knowledge graph question answering~\cite{liu_knowledge_2023}, KGQA) %\footnote{This problem is also referred to as knowledge graph question answering~\cite{liu_knowledge_2023}. We use KBQA to refer to both problems, since most KBs are organized in a graph.}  
solutions use {information retrieval-based} (IR-based) or {semantic parsing-based} (SP-based) methods~\cite{wu_survey_2019,lan_survey_2021}. IR-based methods construct a question-specific subgraph starting from the retrieved entities (i.e., the \emph{topic entities}). They then reason over  the  subgraph to derive the answer. SP-based methods focus on transforming input questions into logical forms, which are then executed to retrieve answers. %Compared to IR-based methods, SP-based methods can produce a more interpretable reasoning process through converting the natural language questions into executable logical forms. Moreover, 
SOTA solutions are mostly SP-based, as detailed next.
%on popular benchmarks (e.g., GrailQA~\cite{gu_beyond_2021} WebQuestionsSP~\cite{yih_value_2016}) 

%JHL1: does SP-based = generation-based approach? does IR-based = ranking-based (in the intro)? it seems like these things are all the same, but we have two different terms; let be more consistent. Also, it doesn't look like IR/ranking based method is all that important to us, so let's drop that discussion in the intro and focus on contrasting our method to generation-based/SP-based models


% SP-based methods can be further categorized into ranking-based and generation-based methods~\cite{gu_arcaneqa_2022, lan_complex_2023}. 

% \jz{Two sentences to explain ranking-based and generation-based methods, rep.} \sxfix{Ranking-based methods perform path traversal and ranking in the KB, starting from the retrieved entities~\cite{gu_beyond_2021, gu_arcaneqa_2022,lan_topic_unit_2019, gu_dont_2023}. Generation-based methods directly transform the input question into the target logical form using a Seq2Seq model~\cite{luo_chatkbqa_2024, wang_no_2024, ye_rng-kbqa_2022}.}

\paragraph{KBQA under I.I.D. Settings}

%Benefiting from the powerful natural language understanding and logical form generation capabilities of Large Language Models (LLMs), 
Recent KBQA studies under I.I.D. settings fine-tune LLMs to map input questions to rough KB elements and generate approximate logical form drafts~\cite{luo_chatkbqa_2024, wang_no_2024}. %, exploiting LLMs' semantic capabilities to understand input natural language questions. 
The approximate (i.e., inaccurate or ambiguous) KB elements are then aligned to exact KB elements through a subsequent retrieval stage. These solutions often fail over test questions that refer to KB elements unseen during training. While we also use LLMs for logical form generation, we ground the generation with retrieved relations, entities, and  schema contexts, thus addressing the non-I.I.D. issue. 
%Our model first retrieves KB elements through schema-guided retrieval, and then uses the retrieved KB elements along with their schema context to guide the generation of logical forms. This approach enables better generalization to questions containing unseen knowledge base elements, while also enhancing performance under the i.i.d. setting.


\paragraph{KBQA under Non-I.I.D. Settings}

Studies considering non-I.I.D. settings can be largely classified into \emph{ranking-based} and \emph{generation-based} methods. 

Ranking-based methods start from retrieved entities, traverse the KB, and construct the target logical form by ranking the traversed paths.  
%Ranking-based methods reduce the search space based on the KB structure and the retrieved entities. 
\citet{gu_beyond_2021} enumerate and rank all possible logical forms within two hops  of retrieved entities, while \citet{gu_dont_2023} incrementally expand and rank paths from retrieved entities. % to obtain the target logical form. %They then obtain the candidate logical form that best matches the question by ranking. %They evaluated both supervised fine-tuning LMs (e.g., T5~\cite{raffel_exploring_2023}, BERT~\cite{devlin_bert_2019}) and few-shot in-context learning LLMs (e.g., Codex~\cite{chen_evaluating_2021}) as the partial logical form discriminators. 

Generation-based methods transform an input question into a logical form using a Seq2Seq model (e.g., T5~\cite{raffel_exploring_2023}).
They often use additional contexts beyond the question to augment the input of the Seq2Seq model and enhance its generalizability. For example,~\citet{ye_rng-kbqa_2022} use  top-5 candidate logical forms enumerated from retrieved entities as the additional context. 
\citet{shu_tiara_2022} further use top-ranked relations, \emph{disambiguated entities}, and classes (retrieved \emph{separately}) as the additional context. \citet{zhang_fc-kbqa_2023} use connected pairs of retrieved KB elements. 

Our \model\ is generation-based. We use schema contexts (relations and classes) from retrieved relations and entities, rather than separate class retrieval (as in \citet{shu_tiara_2022}) which could introduce noise. We also defer entity disambiguation to the logical form generation stage, thus avoiding error propagation induced by premature entity disambiguation without considering the generation context, as done in existing works outlined below. 

%employ an additional middle-grained component that converts the retrieved KB elements into connected pairs of KB elements. A Seq2Seq model then transforms the concatenation of the question, the retrieved KB element pairs, and logical form sketches generated based on the question into the target logical form. 

%The methods above retrieve KB elements to serve as additional contexts to enable the models to generalize in non-I.I.D. settings. 
%Existing ranking-based and generation-based methods limit the subsequent logical form search space through KB element retrieval, thereby improving generalization capability. 
%However, errors in knowledge base element retrieval can directly mislead the logical form generation stage. Moreover, the capacity of language models to reason about the correct element combinations and generate logical forms in search spaces that are noisy (contain ambiguous candidates) has not yet been explored. Therefore, our method introduces SGLG, which defers entity disambiguation and relation classification to the logical form generation stage. By incorporating the schema context of KB elements, it helps the language model reason and generate the correct and executable logical form from a global perspective.

% \jz{Need to say first what these studies haven't done.}
% By explicitly encoding the semantic connections between entities and relations within the schema structure, our RLG module performs entity disambiguation and relation classification during the logical form generation process, reducing error propagation in the traditional retrieve-then-generate framework.





% \paragraph{Semantic Parsing-Based Method} SP-based methods focus on transforming the input questions into structured, executable queries --- typically 
% \emph{logical forms} --- which are then executed over KBs to retrieve answers~\cite{lan_query_2020}. SP-based methods can be further categorized into step-wise ranking and Seq2Seq generation methods~\cite{lan_complex_2023}. Step-wise ranking methods~\cite{yih_value_2016, lan_query_2020, gu_dont_2023} incrementally expand a graph query (i.e., a logical form) with a search step to find possible paths in the KB at each step, followed by a ranking step to select the most relevant paths to be explored next. Seq2Seq generation methods~\cite{liang_querying_2021, yin_neural_2021} transform an input question into a logical form in one go using a Seq2Seq model. Our model follows the general idea of such methods. \jz{We detail these relevant studies next.?}


%due to the important practical applications of both knowledge bases~(KB) and the question answering~(QA) problem over KBs. We start by an overview of the studies on this problem (Section~\ref{subsec:related_work_kbqa}). Then, we focus on KBQA solutions using semantic parsing and Seq2Seq models, as our model also falls into this category (Section~\ref{subsec:related_work_sp}). We also cover techniques for entity retrieval in KBQA, to set the context of our GER module (Section~\ref{subsec:related_work_er}).

%\subsection{Knowledge Base Question Answering}\label{subsec:related_work_kbqa}

%KBQA aims to achieve a natural language-based user interface for non-expert users to interact with KBs without knowing specialized query languages such as SPARQL. 


% Several recent studies~\cite{lin_knowledge-injected_2024, yu_decaf_2023} propose \emph{knowledge injection-based} (KI-based) methods. These studies focus on training large \jz{or pre-trained?} language models to learn knowledge from the KB, while the trained models to can be fine-tuned to generate answers to \jz{Shengxiang to complete....} 
%inject knowledge from the KB directly into language models to by training language models with linearize knowledge triples from the KB. The trained models 
%are further fine-tuned on KBQA datasets, leveraging the acquired knowledge to answer questions. 

%\jz{What's the limitation of these methods? Can we compare with one of these in the experiments? Or why not?} 

% IR-based methods first retrieve a question-specific subgraph \jz{how? (e.g., by matching the entities in the kB with the entity mentions in the question?)}

% \paragraph{Information Retrieval-Based Methods} IR-based methods first retrieve entities related to the input question, one of which is selected as the \emph{topic entity}. The neighbors of the topic entity form a question-specific subgraph. A neural model is then used to score the nodes in the subgraph (i.e., the \emph{candidate answers}), and a score threshold is applied to produce the final answer set. The IR-based methods suffers from complex multi-hop questions, which often lead to retrieving large subgraphs that are difficult to score accurately~\cite{bordes_large-scale_2015, dong_question_2015, zhang_subgraph_2022, liu_knowledge_2023}. A latest study~\cite{ding_enhancing_2024} scores the connections between nodes and edges and expands the subgraph step by step accordingly, which helps reduce  
% the subgraph size. KICP~\cite{lin_knowledge-injected_2024} linearizes the KB triples into sentences to pre-train a language model, which is then fine-tuned on a KBQA dataset to serve as the answer scorer. Even with these enhancements, the IR-based methods typically have lower accuracy than the SP-based ones~\cite{lan_query_2020, gu_knowledge_2022}. 

%Compared to LMs pre-trained on natural language corpora, the LM pre-trained on KB corpus achieved a higher accuracy in selecting answers from candidate answers. } 

%\ssf{[generally IR-based methods produce lower accuracy, even the sota \cite{ding_enhancing_2024} achieve nearly 8\% F1 less then SP-based methods] [not sure to compara with IR-based methods] [move KI-based methods in IR-based methods]}

%\ssf{can delete this paragraph} \emph{evidence pattern retrieval} technique to reduce the nodes retrieved for the subgraph by \jz{formulating structural dependencies in the KB as evidence patterns [need a more intuitive description on what it does and why it offers better results]}, thereby achieving competitive KBQA accuracy \jz{Need to give a reason why we don't compare with it in the experiments}. \jz{However, in general, the IR-based methods produce lower accuracy~\cite{}, and hence we will not consider these methods in the rest of the paper.}
% even on complex KBQA questions. However, it is worth noting that the majority of state-of-the-art KBQA methods are are SP-based, as semantic parsing offers greater interpretability than IR-based approaches.

 \paragraph{KBQA Entity Retrieval}%\label{subsec:related_work_er} 
%\jz{The discussions above have focused on logical form generation of existing KBQA models.} 
KBQA entity retrieval typically has three steps: {entity mention detection}, {candidate entity retrieval}, and {entity disambiguation}. BERT~\cite{devlin_bert_2019}-based named entity recognition  is widely used for entity mention detection from input questions. %~\cite{gu_beyond_2021, zhang_fc-kbqa_2023}. %TIARA~\cite{shu_tiara_2022} treats entity mention detection as a span classification task. It scores question spans of varying lengths with BERT and takes the ones with top scores as entity mentions. 
%\ssf{tuning the threshold of detecting a candidate mention in order to improve coverage} \jz{tuning the threshold of detecting a candidate mention in order to improve coverage [Not sure how it works]}. 
To retrieve KB entities corresponding to entity mentions, the FACC1 dataset~\cite{gabrilovich_facc1_2013} is commonly used, which contains over 10 billion surface forms (with popularity scores) of Freebase entities. \citet{gu_beyond_2021} use the popularity scores for entity disambiguation, while \citet{ye_rng-kbqa_2022} and \citet{shu_tiara_2022} adopt a BERT reranker. %after pruning by popularity. 

%A key issue here is that NER systems may fail to identify  entity mentions precisely, which in turn fail entity retrieval afterwards. 
%Our \model\ model addresses this issue with the help of relation-augmented logical form sketches generation, which enable detecting entity mentions (and hence the entities) more accurately. 

\section{Preliminaries}\label{sec:preliminary}

\begin{figure*}[ht]
    \centering
    \includegraphics[width=1\linewidth]{figures/framework_new.png}
    \caption{Overview of \model. The model has two stages: \emph{retrieval} and \emph{generation}.  In the retrieval stage, we first retrieve and rank candidate relations based on the input question $q$ (\textcircled{1}). Using $q$ and the top-ranked candidate relations $R_q$, we generate logical form sketches and extract entity mentions from them  (\textcircled{2}). Based on the entity mentions and retrieved relations, we retrieve candidate entities from the KB  (\textcircled{3}) and rank them (the top-ones being $E_q$, \textcircled{4}). In the generation stage, $q$, $R_q$, $E_q$, and their class contexts, are fed into a fine-tuned language model for logical form generation (\textcircled{5}). Here, the colored modules come with our new design.}  
    \label{fig:framwork}
\end{figure*}

A graph structured-KB $\mathcal{G}$ is composed of a set of relational facts $\{ \langle s, r, o \rangle |s \in \mathcal{E}, r \in \mathcal{R}, o \in \mathcal{E} \cup \mathcal{L}\}$ and an ontology $\{ \langle c_d, r, c_r \rangle |c_d, c_r \in \mathcal{C}, r \in \mathcal{R} \}$.
Here, $\mathcal{E}$ denotes a set of entities, $\mathcal{R}$ denotes a set of relations, and $\mathcal{L}$ denotes a set of literals, e.g., textual labels, numerical values, or date-time stamps. In a relational fact $\langle s, r, o \rangle$, $s \in \mathcal{E}$ is the 
\textit{subject}, $o \in \mathcal{E} \cup \mathcal{L}$ is the \textit{object}, and $r \in \mathcal{R}$ represents the relationship between the \textit{subject} and the \textit{object}.

The ontology defines the rules governing the composition of relational facts within $\mathcal{G}$. In its formulation,  $\mathcal{C}$ denotes a set of classes, each of which defines a set of entities (or literals) sharing common properties (relations). Note that an entity can belong to multiple classes. 
In an ontology triple $\langle c_d, r, c_r \rangle$, $c_d$ is called a \textit{domain class}, and it refers to the class of  subject entities that satisfy relation $r$; 
$c_r$ is called the \textit{range class}, and it refers to the class of object entities or literals satisfying $r$. Each ontology triple can be instantiated as a set of relational facts. In Figure~\ref{fig:kbqa_example}, \textsf{<book.literary\_series, book.literary\_series.author, book.author>} is an ontology triple. An instance of it is \textsf{<Harry Potter, book.literary\_series.author, J.K. Rowling>}, where \textsf{Harry Potter} is an entity that belongs to class \textsf{book.literary\_series}.

%JHL1: great technical writing above - this is all very clear to me; I take back what I wrote earlier about simplifying Figure 1 earlier, looks like you need the latter half here

% We consider a graph structured-KB $\mathcal{G} = \{ \langle s, r, o \rangle |s \in \mathcal{E}, r \in \mathcal{R}, o \in \mathcal{E} \cup \mathcal{L}\}$ 
% stored in the form of triples $\langle s, r, o \rangle$, where $\mathcal{E}$ is a set of entities, $\mathcal{R}$ is a set of relations, and $\mathcal{L}$ is a set of literals, e.g., textual labels, numerical values, or date-time stamps. In every triple $\langle s, r, o \rangle$, $s \in \mathcal{E}$ is the 
% \textit{subject}, $o \in \mathcal{E} \cup \mathcal{L}$ is the \textit{object}, and $r \in \mathcal{R}$ represents the  relationship between the subject and the object. 

% Triples in the KB are instances of the \emph{ontology} defined for the KB~\cite{gu_knowledge_2022}, which in turn is also represented as triples  
% in the form of $\langle c_d, r, c_r \rangle$. Here, $c_d$ denotes the class of subject entities that satisfy relation $r$, and $c_r$ denotes the class of object entities or literals satisfying $r$. A \emph{class} defines a set of entities sharing common properties (relations), while an entity can belong to multiple classes. In Figure~\ref{fig:kbqa_example}, `\textsf{<book.literary\_series, book.literary\_series.author, book.author>}' is an ontology triple, while an instance of it is `\textsf{<Harry Potter, book.literary\_series.author, J.K. Rowling>}', where `\textsf{Harry Potter}' is an entity that belongs to class `book.literary\_series'.

\paragraph{Problem Statement} Given a KB $\mathcal{G}$ and a question $q$ expressed in natural language, i.e., a sequence of word tokens, {knowledge base question answering} (KBQA) aims to find a subset (the {answer set}) $\mathcal{A} \subseteq \mathcal{E} \cup \mathcal{L}$ of elements from $\mathcal{G}$ that --- with optional application of some aggregation functions (e.g., \textsc{count}) --- answers $q$. 

%find the answer $\mathcal{A} \text{ for } q $, where $\mathcal{A}$ is a set of entities or literals $\mathcal{A} \subseteq \mathcal{E} \cup \mathcal{L}$. $\mathcal{A}$ can serve directly as the answer or can be applied some aggregation functions (e.g. counting function) to get the answer.



\paragraph{Logical Form}
We solve the KBQA problem by translating the input question $q$ into a structured query that can be executed on $\mathcal{G}$ to fetch the answer set $\mathcal{A}$. Following previous works~\cite{shu_tiara_2022, ye_rng-kbqa_2022, gu_dont_2023, zhang_fc-kbqa_2023}, we use logical form as the structured query language, expressed with the \emph{S-expression}~\cite{gu_beyond_2021}.  
The S-expression offers a readable representation well-suited for KBQA. It uses set semantics where functions operate on entities or entity tuples without requiring variables~\cite{ye_rng-kbqa_2022}.  Figure~\ref{fig:kbqa_example} shows an example: the S-expression of the given question \textsf{Who is the author of Harry Potter?} is \textsf{(AND book.author (JOIN (R book.literary\_series.author) m.078ffw))}. This S-expression queries a set of entities that belong to the class \textsf{book.author} from the objects of triples whose subject entity is \textsf{m.078ffw} while the relation is \textsf{book.literary\_series.author}. More details about the  S-expression is in Appendix~\ref{sec:app_sexpression}.


\section{The \model\ Model}\label{sec:method}

As shown in Figure~\ref{fig:framwork}, \model\ follows the common structure of generation-based models. It has two overall stages: \emph{relation and entity retrieval} and \emph{logical form generation}. We propose novel designs in both stages to strengthen model generalizability.

In the relation and entity retrieval stage (Section~\ref{subsec:ger}), \model\ retrieves candidate relations and entities from KB $\mathcal{G}$ which may be relevant to the input question $q$. It starts with a BERT-based relation ranking model to retrieve candidate relations relevant to $q$. Together with $q$, the set of top-ranked candidate relations are fed into a novel, relation-guided  Seq2Seq model to generate logical form sketches that contain entity mentions while masking the relations and classes. We harvest the entity mentions and use them to retrieve candidate entities from $\mathcal{G}$. % with the help of an entity dictionary FACC1~\cite{Gabrilovich2013FACC1} (following existing studies~\cite{shu_tiara_2022,luo_chatkbqa_2024}, although other entity retrieval models can be used).  
We propose a combined relation-based strategy to prune the entities (as there may be many). The remaining entities are ranked by a BERT-based model, indicating their likelihood of being the entity that matches each entity mention. 

Leveraging relations to guide both entity mention extraction and candidate entity pruning enhances the model generalizability over entities unseen during training. This in turn helps the logical form generation stage to filter false positive matches for unseen relations or their combinations. 

In the logical form generation stage (Section~\ref{subsec:rlg}), 
\model\ feeds $q$, the top-ranked relations and entities (corresponding to each entity mention), and the schema contexts (i.e., domain and range classes of the relations and classes of the entities), into an adapted LLM to generate the logical form  and produce answer set $\mathcal{A}$. 

Our schema-guided logical form generation procedure is novel in that it takes (1) multiple candidate entities (instead of one in existing models) for each entity mention and (2) the schema contexts as the input. Using multiple candidate entities essentially defers  \emph{entity disambiguation}, which is usually done in the retrieval stage by existing models~\cite{shu_tiara_2022,gu_dont_2023}, to the generation stage, thus mitigating error propagation. This strategy also brings challenges, as the extra candidate entities (which are ambiguous as they often share the same name) may confuse the logical form generation model. We address the challenges with the schema contexts, which instruct the model the connectivity structures between the candidate entities and relations. The connectivity structures further help \model\ generalize to unseen entities, relations, or their combinations. 

%\sxfix{The classes shared by the entities and the relations indicate the connectivity of these elements. This guides our LM to to select the correct combination of entities and relations from the input's top-ranked entities and top-ranked relations, thereby generate executable logical forms. By deferring entity disambiguation to the logical form generation process and leveraging class information to provide semantic structural context from the KB, we reduce error propagation within the traditional retrieve-then-generate framework.} 

%retrieves the classes of the KB entities from the GER module. The relations between the classes offer contexts about the connectivity between the KB elements, to power the zero-shot and compositional generalization capability of \model\ to \emph{handle unseen KB elements and compositions}. We then fine-tune an open-sourced \emph{large language model} (LLM), which takes a question, the retrieved top-ranked entities and relations with class annotations as input to generate the target logical form.

\subsection{Relation and Entity Retrieval}\label{subsec:ger}

\paragraph{Relation Retrieval} For relation retrieval, we follow the schema retrieval model of TIARA~\cite{shu_tiara_2022}, as it has high accuracy. We extract a  set $R_q$ of top-$k_R$ (system parameter) relations with the highest semantic similarity to $q$. This is done by a BERT-based cross-encoder that learns the semantic similarity $\text{sim}(q, r)$ between $q$ and a relation $r \in \mathcal{R}$: %(recall that $\mathcal{R}$ is the set of relations of KB $\mathcal{G}$):  
\begin{equation}
\small
    \text{sim}(q,r)=\text{\large L\small INEAR}(\text{\large B\small ERT\large C\small LS}([q;r])), 
    \label{eqn:relation_retriever}
\end{equation} 
where `$;$' denotes concatenation.
This model is trained with the sentence-pair classification objective~\cite{devlin_bert_2019}, where a relevant question-relation pair has a similarity of 1, and 0 otherwise.




%\subsubsection{Relation-Augmented Logical Form sketch Parser}
\paragraph{Relation-Guided Entity Mention Detection}

%Previous work has employed Named Entity Recognition (NER) tools or regard entity mention detection as a span classification task to extract entity mentions in the questions \cite{gu_beyond_2021, shu_tiara_2022,zhang_fc-kbqa_2023}. However, detecting zero-shot entity mentions from short texts remains a challenging task. A common error in past mention detection methods is that certain components of relations in some questions are mistakenly detected as entity mentions, especially in questions containing unseen relations.
%To address the above problem, 
Given $R_q$, we propose a relation-guided logical form sketch parser to parse $q$ into a logical form sketch $s$. Entity mentions in $q$ are extracted from $s$. 

The parser is an adapted Seq2Seq model. The model input of each training sample takes the form of ``$q$ \textless relation\textgreater \text{ } $r_1;r_2;\ldots;r_{k_R}$'' ($r_i \in R_q$, hence ``relation-guided''). 
In the ground-truth logical form corresponding to $q$, we mask the relations, classes, and literals with special tokens `\textless relation\textgreater', `\textless class\textgreater', and `\textless literal\textgreater', to form the ground-truth logical form sketch $s$. Entity IDs are also replaced by the corresponding entity names (entity mentions), to enhance the Seq2Seq model's understanding of the semantics of entities.

%We fine-tune T5~\cite{raffel_exploring_2023} as the Seq2Seq model to transform a question into the corresponding logical form sketch. 

%We concatenate the question $q$ with the retrieved top-$k_R$ relations as the context  to form the input of the Seq2Seq model, i.e., the relation-augmented logical form sketch parser, to enhance the model understand of the input semantics. 
%The input of the model is in the form of ``$q$ \textless relation\textgreater \text{ } $r_1;r_2;\ldots;r_{k_R}$'' ($r_i \in R_q$). 

At model inference, from the output top-$k_L$ (system parameter) logical form sketches  (using beam search), we extract the entity mentions.

\paragraph{Relation-Guided Candidate Entity Retrieval}
We follow previous studies~\cite{gu_beyond_2021, shu_tiara_2022, faldu_retinaqa_2024, luo_chatkbqa_2024} and use an entity name dictionary FACC1~\cite{gabrilovich_facc1_2013} to map extracted entity mentions to entities (i.e., their IDs in KB), although other retrieval models can be used. Since different entities may share the same name, the entity mentions may be mapped to many entities. For pruning, existing studies use  popularity scores associated to  entities~\cite{shu_tiara_2022, ye_rng-kbqa_2022}. 

To improve the recall of candidate entity retrieval, we propose a combined pruning strategy based on both popularity and relation connectivity. As Figure~\ref{fig:candidate_entity_retrieval} shows, we first select the top-$k_{E1}$ (system parameter) entities for each entity mention based on popularity and then extract $k_{E2}$ (system parameter) entities from the remaining candidates that are connected to the retrieved relations $R_q$. Together, these form the candidate entity set $E_c$.
%for alias mapping of entity mentions \cite{.  A branch of works selects the entity with the highest popularity for each mention \cite{gu_beyond_2021, gu_arcaneqa_2022, zhang_fc-kbqa_2023}, while others choose the top-popularity entities and then perform entity disambiguation \cite{ye_rng-kbqa_2022, shu_tiara_2022, zhang_fc-kbqa_2023}. However, a popularity-based pruning strategy may exclude low-popularity ground-truth entities. 

\begin{figure}[t]
    \centering
    \includegraphics[width=\linewidth]{figures/candidate_entity_retrieval.png}
     \caption{Candidate entity retrieval for the mention `\textsf{aloha}'. The candidate entity in red is the ground-truth.}
    \label{fig:candidate_entity_retrieval}
\end{figure}


\paragraph{Entity Ranking} We follow existing works~\cite{shu_tiara_2022, ye_rng-kbqa_2022} to score and rank each candidate entity in $E_c$ by jointly encoding $q$ and the context (entity name and its linked relations) of the entity using a cross-encoder (like Eq.~\ref{eqn:relation_retriever}). %The context of a candidate entity includes . 
We select the top-$k_{E3}$ (system parameter) ranked entities for each mention as the entity set $E_q$ for each question.


\subsection{Schema-Guided Logical Form Generation}\label{subsec:rlg}

Given relations $R_q$ and entities $E_q$, we fine-tune an open-souce LLM (LLaMA3.1-8B~\cite{touvron_llama_2023} by default) to generate the final logical form.  

Before being fed into the model, each relation and entity is augmented with its schema context (i.e., class information) to help the model to learn their connections and generalize to unseen entities, relations, or their compositions.  The context of a relation $r$ is described by concatenating the relation's  domain class $c_d$ and range class $c_r$, formatted as ``[D] $c_d$ [N] $r$ [R] $c_r$''. For an entity $e$, its context is described by its ID (``$id_e$''), name (``$name_e$''), and the intersection of its set of classes $C_e$ and the set of all domain and range classes $C_R$ of all relations in $R_q$, formatted as ``[ID] $id_e$ [N] $name_e$ [C] class($C_e\cap C_R$)''.

As Figure~\ref{fig:framwork} shows, we construct the input to the logical form generation model by concatenating $q$ with the context of each relation in $R_q$ and the context of each entity in $E_q$. The model is fine-tuned with a cross-entropy-based objective:
\begin{equation}
\small
\mathcal{L}_{generator}=-\sum_{t=1}^n \log p\left(l_t \mid l_{<t}, q, K_q\right),
\end{equation}
where $l$ denotes a logical form of $n$ tokens and $l_t$ is its $t$-th token, and $K_q$ is the retrieved knowledge (i.e., relations and entities with contexts) for $q$. At inference, the model runs beam search to generate top-$k_O$ logical forms -- the executable one with the highest score is selected as the output. See Appendix~\ref{app:prompt} for a prompt example used for inference. 

%JHL1: I don't quite follow this last bit; dumb question: if we already have the logical forms, isn't it a straightforward thing to check whether an output logical form is executable, and just take the first executable one? It's a bit unclear why we need the enumeration step and also why we need a BERT reranker.
It is possible that no generated logical forms are executable. In this case, we fall back to following~\citet{shu_tiara_2022} and~\citet{ye_rng-kbqa_2022} and retrieve candidate logical forms in two stages: enumeration and ranking. During enumeration, we search the KB by traversing paths starting from the retrieved entities. Due to the exponential growth in the number of candidate paths with each hop, we start from the top-1 entity for each mention and searches its neighborhood for up to two hops. The paths retrieved are converted into logical forms. During ranking, a BERT-based ranker scores $q$ and each enumerated logical form $l$ (like Eq.~\ref{eqn:relation_retriever}). We train the ranker using a contrastive objective: 
\begin{equation}
\small
    \mathcal{L}=-\frac{\text{exp}(\text{sim}(q, l^*))}{\text{exp}(\text{sim}(q, l^*))+\sum_{l \in C_l \wedge l \neq l^*} \text{exp}(\text{sim}(q, l))},
\end{equation}
where $l^*$ is the ground-truth logical form and $C_l$ is the set of enumerated logical forms. We run the ranked logical forms from the top and return the first executable one. 

%JHL1: Good job on the technical writing; it was quite clear and I think I understood most of the model details. Thoughts: at a high level, I think the core novelty of the work is about the deferring the entity disambiguation to the generation step - this is a pretty cool idea. I think the intro gets this idea across quite well. But with that we run into a large search space, and introducing schema information into the input is a straightforward way to solve the issue - I think what we should do in the intro is to explain what schema information is (e.g. book.author is the schema information for m.078ffw), and then including an example input (to the LLM) in a figure and that should do it (figure 3 almost did this, but the input still just has some abstract variables, so isn't good enough). The relation-guided entity mention detection, on the other hand, feels like a very small touch and isn't that important to talk about in the intro. I'd drop it to increase clarity of our novel contribution.


\section{Experiments}\label{sec:experiment}
We run experiments to answer:
\textbf{Q1}:~How does \model\ compare with SOTA models in their accuracy for the KBQA task? 
%\textbf{Q2}: How does \model\ compare with the SOTA KBQA models  under I.I.D. settings? 
\textbf{Q2}:~How do model components impact the accuracy of \model? 
\textbf{Q3}:~How do our techniques generalize to other KBQA models? 

%is our generative entity retrieval module?
%\textbf{Q3}:~How effective is our retrieval-guided logical form generation module?



\subsection{Experimental Setup}

\paragraph{Datasets}
Following SOTA competitors~\cite{shu_tiara_2022, gu_dont_2023, zhang_fc-kbqa_2023}, we use two  benchmark datasets built upon Freebase.

\textbf{GrailQA}~\cite{gu_beyond_2021} is a dataset for evaluating the generalization capability of KBQA models. It contains 64,331 questions with annotated target S-expressions, including complex questions requiring up to 4-hop reasoning over Freebase, with aggregation functions including comparatives, superlatives, and counting. The dataset comes with training (70\%), validation (10\%), and test (20\%, hidden and only known by the leaderboard organizers) sets. In the validation and the test sets, 50\% of the questions include KB elements that are unseen in the training set (\textbf{zero-shot} generalization tests), 25\% consist of unseen compositions of KB elements seen in the training set (\textbf{compositional} generalization tests), and the remaining 25\% are randomly sampled from the training set (\textbf{I.I.D.} tests).

{WebQuestionsSP} (\textbf{WebQSP})~\cite{yih_value_2016} is a dataset for the I.I.D. setting. While our focus is on non-I.I.D. settings, we include results on this dataset to show the general applicability of \model. WebQSP contains 4,937 questions. 
More details of WebQSP are included in Appendix~\ref{app:WebQSP}.


%collected from Google query logs, including 3,098 questions for training and 1,639 for testing, each annotated with a target SPARQL query. %We convert each SPARQL query into the corresponding S-expression and extract 
%We follow GMT-KBQA~\cite{hu_logical_2022} to separate 200 questions from the training questions to form the validation set.

%\noindent \textbf{ComplexWebQuestions} (CWQ) \cite{talmor_web_2018} is an extended version of WebQSP with 34,689 questions in total. All the questions in it are derived from WebQSP but have been made more complex, incorporating more hops and constraints.

\begin{table*}[!ht]
\centering
\small
%(``I.I.D.'' means random samples from the training set; ``Compositional'' means unseen compositions of KB elements seen at training; ``Zero-shot'' means unseen compositions of unseen KB elements; ``Overall'' means a mix of the aforementioned). 
\begin{tabular}{
>{\centering\arraybackslash}m{0.1\linewidth}
>{}p{0.26\linewidth}
>{\centering\arraybackslash}m{0.04\linewidth}
>{\centering\arraybackslash}m{0.04\linewidth} 
>{\centering\arraybackslash}m{0.04\linewidth} 
>{\centering\arraybackslash}m{0.04\linewidth} 
>{\centering\arraybackslash}m{0.04\linewidth} 
>{\centering\arraybackslash}m{0.04\linewidth} 
>{\centering\arraybackslash}m{0.04\linewidth} 
>{\centering\arraybackslash}m{0.04\linewidth} }
\toprule
\multicolumn{1}{l}{\textbf{}} & \textbf{} & \multicolumn{2}{c}{\textbf{Overall}} & \multicolumn{2}{c}{\textbf{I.I.D.}} & \multicolumn{2}{c}{\textbf{Compositional}} & \multicolumn{2}{c}{\textbf{Zero-shot}} \\ \cline{3-10}
\multicolumn{1}{l}{} & \rule{0pt}{10pt}\textbf{Model} & \textbf{EM} & \textbf{F1} & \textbf{EM} & \textbf{F1} & \textbf{EM} & \textbf{F1} & \centering \textbf{EM} & \textbf{F1} \\ \midrule
\multirow{7}{*}{\begin{tabular}[c]{@{}c@{}} SP-based \\(SFT) \\ \end{tabular}} & RnG-KBQA (ACL 2021) & 68.8 & 74.4 & 86.2 & 89.0 & 63.8 & 71.2 & 63.0 & 69.2 \\
 & TIARA (EMNLP 2022) & 73.0 & 78.5 & 87.8 & 90.6 & 69.2 & 76.5 & 68.0 & 73.9 \\
 & Decaf (ICLR 2023) & 68.4 & 78.7 & 84.8 & 89.9 & 73.4 & \underline{81.8} & 58.6 & 72.3 \\
 & Pangu (T5-3B) (ACL 2023) & 75.4 & \underline{81.7} & 84.4 & 88.8 & \underline{74.6} & 81.5 & 71.6 & \underline{78.5} \\
 & FC-KBQA (ACL 2023) & 73.2 & 78.7 & \underline{88.5} & \underline{91.2} & 70.0 & 76.7 & 67.6 & 74.0 \\
 & TIARA+GAIN (EACL 2024) & \underline{76.3} & 81.5 & \underline{88.5} & \underline{91.2} & 73.7 & 80.0 & \underline{71.8} & 77.8 \\
 & RetinaQA (ACL 2024) & 74.1 & 79.5 & - & - & 71.9 & 78.9 & 68.8 & 74.7 \\ \midrule
\multirow{3}{*}{\begin{tabular}[c]{@{}c@{}} SP-based \\(Few-shot) \\ \end{tabular}} 
 & KB-Binder (6)-R (ACL 2023) & 53.2 & 58.5 & 72.5 & 77.4 & 51.8 & 58.3 & 45.0 & 49.9 \\
 & Pangu (Codex) (ACL 2023) & 56.4 & 65.0 & 67.5 & 73.7 & 58.2 & 64.9 & 50.7 & 61.1 \\
 & FlexKBQA (AAAI 2024) & 62.8 & 69.4 & 71.3 & 75.8 & 59.1 & 65.4 & 60.6 & 68.3 \\ \midrule
\multirow{2}{*}{\centering \makecell{\textbf{Ours} \\ (SFT)}} &\textbf{\model} & \textbf{79.1} & \textbf{84.4} & \textbf{88.6} & \textbf{91.6} & \textbf{77.9} & \textbf{85.1} & \textbf{75.4} & \textbf{80.8} \\
 &\hspace{3pt} - Improvement & +3.6\% & +3.3\% & +0.1\% & +0.4\% & +4.4\% & +4.0\% & +5.0\% & +2.9\% \\ \bottomrule 
\end{tabular}
\caption{\emph{Hidden} test results (\%) on GrailQA (best results are in boldface; best baseline results are underlined; ``SFT'' means supervised fine-tuning; ``few-shot'' means few-show in-context learning).}
%JHL1: ours is SFT too right? in that case let's put that in the table
\label{tab:grailqa}
\end{table*}

\paragraph{Competitors} 
We compare with both IR-based and SP-based methods including the SOTA models. 

On GrailQA, we compare with models that top the leaderboard\footnote{https://dki-lab.github.io/GrailQA/}, 
including \textbf{RnG-KBQA}~\cite{ye_rng-kbqa_2022}, \textbf{TIARA}~\cite{shu_tiara_2022}, \textbf{DecAF}~\cite{yu_decaf_2023}, 
\textbf{Pangu}
%(using T5-3B for scoring partial logical forms; 
(previous {SOTA} as of 15th February, 2025)~\cite{gu_dont_2023},
%JHL1: give a date for the current SOTA (as this may change in the future)
\textbf{FC-KBQA}~\cite{zhang_fc-kbqa_2023}, \textbf{TIARA+GAIN}~\cite{shu_data_2024}, and \textbf{RetinaQA}~\cite{faldu_retinaqa_2024}.
We also compare with few-shot LLM  (training-free) methods: KB-BINDER (6)-R~\cite{li_few-shot_2023}, Pangu
%(using Codex for scoring partial logical forms)
~\cite{gu_dont_2023}, and FlexKBQA~\cite{li_flexkbqa_2024}. These models are SP-based. On the non-I.I.D. GrailQA, IR-based methods are uncompetitive and excluded.

On WebQSP, we compare with IR-based models \textbf{SR+NSM}~\cite{zhang_subgraph_2022}, \textbf{UNIKGQA}~\cite{jiang_unikgqa_2023}, and
\textbf{EPR+NSM}~\cite{ding_enhancing_2024}, plus SP-based models  \textbf{ChatKBQA} ({SOTA})~\cite{luo_chatkbqa_2024} and \textbf{TFS-KBQA} ({SOTA})~\cite{wang_no_2024}, both of which use a fine-tuned LLM to generate logical forms.
We also compare with TIARA, Pangu, and FC-KBQA as above, which represent SOTA models using pre-trained language models (PLMs). 
Appendix~\ref{sec:app_baselines} details these models. The baseline results are collected from their papers or the GrailQA leaderboard (if available).


% On WebQSP, we compare with SOTA models including TFS-KBQA \cite{wang_no_2024}, ChatKBQA \cite{luo_chatkbqa_2024}, GMT-KBQA \cite{hu_logical_2022}, FC-KBQA \cite{zhang_fc-kbqa_2023}, and Pangu \cite{gu_dont_2023}. The first two methods are based on a fine-tuned LLM, while the latter deploys the LLM within a generate-then-retrieve framework. GMT-KBQA \cite{hu_logical_2022} and FC-KBQA \cite{zhang_fc-kbqa_2023} are representative state-of-the-art models utilizing PLM in retrieve-then-generate framework. Pangu \cite{gu_dont_2023} achieves state-of-the-art performance across multiple datasets, which is a generic framework for grounded language understanding.



\paragraph{Implementation Details}
% All our experiments are run on a machine with an NVDIA A100 GPU and 120 GB of RAM. We fine-tuned three \texttt{bert-base-uncased} models for a maximum of three epochs each, for relation retrieval, entity ranking, and fallback logical form ranking.
% For relation retrieval, we randomly sample 50 negative samples for each question to train the model to distinguish between relevant and irrelevant relations. 

% For each dataset, a \texttt{T5-base} model is fine-tuned for 5 epochs as our logical form sketch parser, with a beam size of 3 (i.e., $k_L = 10$) for GrailQA, 4 for WebQSP. In candidate entity retrieval, we use the same number (i.e., 10) of candidate entities per mention as the baselines~\cite{shu_tiara_2022, ye_rng-kbqa_2022}. The retrieved candidate entities for a mention consist of entities with the top-$k_{E1}$ popularity scores and $k_{E2}$ entities connected to the top-ranked relations in $R_q$, where $k_{E1} = 1$, $k_{E2} = 9$ for GrailQA, $k_{E1} = 3$, $k_{E2} = 7$ for WebQSP.

% We select the top-20 (i.e., $k_R$ = 20) relations and the top-2 (i.e., $k_{E3} = 2$) entities (for each entity mention) retrieved by our model. For WebQSP, we also use the entities obtained from the off-the-shelf entity linker ELQ~\cite{li_efficient_2020}. 

% Finally, we fine-tune \texttt{LLaMA3.1-8B} with LoRA~\cite{hu_lora_2021} for logical form generation. On GrailQA, \texttt{LLaMA3.1-8B} is fine-tuned for 1 epoch with a learning rate of $0.0001$. On WebQSP, it is fine-tuned for 15 epochs with the same learning rate (as it is an I.I.D. dataset where more epochs are beneficial). During inference, we generate logical forms by beam search with a beam size of 10 (i.e., $K_O = 10$). The generated logical forms are executed on the KB to filter non-executable ones. If none of the logical forms are executable, we check the candidate logical forms from the fallback procedures, and the result of the first executable one is returned as the answer set.
% %\jz{Any updates needed for this subsection?} 


% Our system parameters have been chosen empirically. While there are a few of them, their exact values do not have strong impact on the final model performance, and the choice of parameter values generalize well across  datasets. The same parameter values are used on both datasets. 
% \addexp{Add parameter study to appendix.}
All our experiments are run on a machine with an NVDIA A100 GPU and 120 GB of RAM. We fine-tuned three \texttt{bert-base-uncased} models for a maximum of three epochs each, for relation retrieval, entity ranking, and fallback logical form ranking.
%JHL1:  for these three models, can we ref to figure 3 and the module number in the figure?
For each dataset, a T5-base model is fine-tuned for 5 epochs as our logical form sketch parser. Finally, we fine-tune a LLaMA3.1-8B with LoRA~\cite{hu_lora_2021} for 5 epochs on GrailQA and 20 epochs on WebQSP to serve as the logical form generator. Our system parameters have been chosen empirically, and a parameter study is provided in Appendix~\ref{app:paramater_study}. More implementation details  are in Appendix~\ref{app:implemention_details}.




\paragraph{Evaluation Metrics}
On GrailQA, we report the exact match (\textbf{EM}) and \textbf{F1} scores, following the leaderboard. EM counts the percentage of test samples where the model generated logical form (an S-expression) that is semantically equivalent to the ground truth. F1  measures the answer set correctness, i.e., the F1 score of each answer set, average over all test samples. 
On WebQSP, we report the F1 score as there are no ground-truth S-expressions. 
%Following previous SP-based methods \cite{shu_tiara_2022, zhang_fc-kbqa_2023}, here hits@1 is calculated by randomly selecting one answer for each question 100 times and averaging the results. This approach is used because the answers obtained from SP-based methods are typically unordered.

\begin{table}[t]
\centering
\small
\setlength{\tabcolsep}{6pt}
\begin{tabular}{clc}
\toprule
 & \textbf{Model} & \textbf{F1}\\ \midrule
\multirow{3}{*}{\begin{tabular}[c]{@{}c@{}}IR-based\\ \end{tabular}}  
& SR+NSM (ACL 2022)     & 69.5 \\
& UniKGQA (ICLR 2023)  & 75.1    \\
& EPR+NSM (WWW 2024)  & 71.2   \\
 \midrule
\multirow{5}{*}{\begin{tabular}[c]{@{}c@{}}SP-based \\ (SFT) \\ \end{tabular}} 
& TIARA (EMNLP 2022)   & 76.7\\
& Pangu (T5-3B, ACL 2023) & 79.6\\
& FC-KBQA (ACL 2023)   & 76.9 \\
& ChatKBQA (ACL 2024) & 79.8\\ 
& TFS-KBQA (LREC-COLING 2024) & \underline{79.9}\\
\midrule
\multirow{3}{*}{\begin{tabular}[c]{@{}c@{}}SP-based \\(Few-shot)\\ \end{tabular}} 
& KB-Binder (6)-R (ACL 2023)   & 53.2\\
& Pangu (Codex) (ACL 2023)   & 54.5\\
& FlexKBQA (AAAI 2024)   & 60.6\\
\midrule
\multirow{2}{*}{\begin{tabular}[c]{@{}c@{}}\textbf{Ours} \\ (SFT)\end{tabular}} 
& \textbf{\model} & \textbf{80.3} \\
&\hspace{6pt} - Improvement & \multicolumn{1}{l}{+0.5\%} \\
% \cdashline{2-3}
% & \rule{0pt}{10pt} \hspace{6pt}w/o RG-EMD & 78.4 \\
% & \hspace{6pt} w/o RG-CER & 79.5 \\ 
% %& \hspace{6pt} w/o SG-ER & 78.3 \\ 
% &\hspace{6pt} w/o DED & 78.2 \\ 
% &\hspace{6pt} w/o SG-LF & 77.1 \\
\bottomrule
\end{tabular}
\caption{Test results (\%) on WebQSP (I.I.D.).}
\label{tab:webqsp}
\end{table}


\subsection{Overall Results (Q1)}
Tables~\ref{tab:grailqa} and~\ref{tab:webqsp} show the overall comparison of \model\ with the baseline models for GrailQA and WebQSP, respectively. \model\ shows the best results across both datasets. 

\paragraph{Results on GrailQA} 
On the overall hidden test set of GrailQA, \model\ outperforms the best baseline Pangu by 4.9\% and 3.3\% in the EM and F1 scores, respectively. Under the compositional and zero-shot generalization settings (both are non-I.I.D.), similar performance gaps are observed, i.e., 4.0\% and 2.9\% in F1 compared to the best baseline models, respectively. This validates that \model\ can extract relations and entities more accurately from the input question, even when these are unseen in the training set, and it creates more accurate logical forms to answer the questions. %\sxfix{baseline models do not use schema context of retrieved KB elements (other model use class as a set of retrieved KB elements. We use the class or schema context of each relation and entity to feed its schema (connectivity to other elements)) into our generator to guide a logical form generation that are consistent to the KB structure (schema)} 

The fine-tuned baseline models do not use relation semantics to enhance entity retrieval, and they either omit the class contexts in logical form generation or use these classes separately for retrieval. As such, they do not generalize as well in the non-I.I.D. settings. 
%which may be noisy and do not indicate the schema of the  the entities and relations. These  explain for their lower accuracy.
The few-shot LLM-based competitors are generally not very competitive, especially under the non-I.I.D. settings. This suggests that {the current generation of LLMs are unable to infer from a few input demonstrations the process of logical form generation from user questions}. Fine-tuning is still required.  

\paragraph{Results on WebQSP} 


On WebQSP, which has an I.I.D. test set, the performance gap of the different models are closer. Even in this case, \model\ still performs the best, showing its general applicability. Comparing with  TFS-KBQA (SOTA) and ChatKBQA, \model\ improves the F1 score by 0.5\%.  
Among IR-based methods, UniKGQA (SOTA) still performs substantially worse compared to \model. The lower performance of IR-based methods is consistent with existing results~\cite{gu_knowledge_2022}.
%which also reported lower performance from the IR-based methods.





\subsection{Ablation Study (Q2)}


\begin{table}[t]

\centering
\resizebox{\columnwidth}{!}{
\begin{tabular}{lccccc}
\toprule
\multirow{2}{*}{\rule{0pt}{28pt}\textbf{Model}}                  & \multicolumn{4}{c}{\textbf{GrailQA}}                                 & \textbf{WebQSP}  \\ \cmidrule{2-6} 
                                        & \textbf{Overall} & \textbf{I.I.D.} & \textbf{Comp.} & \textbf{Zero.} & \textbf{Overall} \\ \midrule
\textbf{\model}          & \textbf{88.5}    & \textbf{94.6}   & \textbf{84.6}  & \textbf{87.9}  & \textbf{80.3}    \\
\hspace{3pt}w/o RG-EMD & 85.3             & 92.4            & 80.2           & 84.3           & 78.4             \\
\hspace{3pt}w/o RG-CER & 86.5             & 92.1            & 81.1           & 86.3           & 79.5             \\
\hspace{3pt}w/o DED    & 87.8             & 94.0            & 82.4           & 87.2           & 78.2             \\
\hspace{3pt}w/o SC  & 79.2             & 92.9            & 77.4           & 73.9           & 77.1             \\ \bottomrule
\end{tabular}
}
\caption{Ablation study results (F1 score) on the validation set of GrailQA and the test set of WebQSP.}
\label{tab:ablation}
\end{table}

% \begin{table}[H]
% \small
% \centering
% \begin{tabular}{lcccc}
% \toprule
% \textbf{Model}           & \textbf{Overall} & \textbf{I.I.D.} & \textbf{Comp.} & \textbf{Zero.} \\ \midrule
% % TIARA           & 81.9   & 91.2  & 74.8  & 80.7  \\
% % FC-KBQA         & 83.8   & 91.5  & 77.3  & 83.1  \\
% % RetinaQA        & 83.3   & 91.2  & 77.5  & 82.3  \\ 
% % \midrule
% \textbf{\model}            & \textbf{88.5}   & \textbf{94.6}  & \textbf{84.6}  & \textbf{87.9}  \\
% \hspace{3pt}w/o RG-EMD   &   85.3     &    92.4   &   80.2   &    84.3  \\ 
% \hspace{3pt}w/o RG-CER   &   86.5     &    92.1   &   81.1    &   86.3    \\ 
% %\hspace{3pt}w/o SGER        & 86.4   & 92.3  & 82.4  & 85.5  \\
% \hspace{3pt}w/o DED      & 87.8  & 94.0  & 82.4  & 87.2  \\
% \hspace{3pt}w/o SG-LF    & 79.2   & 92.9  & 77.4  & 73.9  \\
% %\hspace{3pt}w/o Fallback LF \jz{to appendix} & 84.6 & 94.1 & 81.8 & 81.5 \\ 
% \bottomrule
% \end{tabular}
% \caption{Ablation study results (F1 score) on the validation set of GrailQA}
% \label{tab:ablation}
% \end{table}
%JHL1: I'd drop the baseline models, and move WebQSP results to this table. since it's about studying the impact of different components, I don't see why we need to include the baseline models

Next, we run an ablation study with the following variants of \model: \textbf{w/o~RG-EMD} replaces our relation-guided entity mention detection with SpanMD~\cite{shu_tiara_2022} which is commonly used in existing models~\cite{pang_survey_2022, faldu_retinaqa_2024}; \textbf{w/o~RG-CER} omits  candidate entities retrieved from the top relations; \textbf{w/o~DED} uses the top-1 candidate entity for each entity mention without deferring entity disambiguation; \textbf{w/o~SC} omits schema contexts from logical form generation. 
%JHL1: can we use shorter acronyms for these things? it's not obvious what RG, SG, etc all means anyway, why not just use a 2-3 letter acronym? Also, let's be consistent with the acronymn used in 5.4 (i see RG-CER vs. SG-ER), but they look like the same thing)

Table~\ref{tab:ablation} shows the results on the validation set of GrailQA and the test set of WebQSP. Only F1 scores are reported for conciseness, as the EM scores on GrailQA exhibit similar comparative trends and are provided in Appendix~\ref{app:ablation}.

% Table~\ref{tab:ablation} shows the results on the validation set of GrailQA \sxfix{and the test set of WebQSP}, benchmarking against baseline models with released code. We only show F1 scores. The EM scores exibit similar comparative patterns and are included in Appendix~\ref{app:ablation}.
%For WebQSP, the results are included in Table~\ref{tab:webqsp}.

All model variants have lower F1 scores than those of the full model, confirming the effectiveness of the model components. SG-KBQA w/o DED (with schema contexts) reduces the F1 scores across various generalization settings on both datasets, demonstrating the effectiveness of our DED strategy in reducing error propagation during the retrieval and generation stages. Furthermore, \model~w/o SC (with deferred entity disambiguation) has the most significant drops in the F1 score under the compositional (7.2) and zero-shot (14.0) generalization tests. It highlights the importance of schema contexts in constraining the larger search space introduced by DED and in generalizing to unseen KB elements and their combinations. Meanwhile, the lower F1 of \model~w/o RG-EMD emphasizes the capability of our relation-guided entity mention detection module in strengthening KBQA entity retrieval.
%JHL1: the ablation results is admittedly a little bit disappointing. MY read is:
%- the most novel bit, which is deferring entity disambiguation to generation, seems to have only marginal impact (DED)
%- adding schema context information (SG-LF), adding relation to entity mention detection (EMD), are fairly straightforward innocation, on the other hand, seem to have the most impact!
%- For the first point, does that mean the premise that we said about errors due to premature entity disambiguation are empirically... quite rare? So it's a somewhat non-issue hmm....
%- So, what does all this mean for us? I think we can be honest here, and talk about how the more 'novel' idea turns out to be empirically less impactful, but that the more simple changes (like injecting more info into the input) turn out to have huge impact. Smooth it out a bit by saying but overall all the individual modules still contribute to the big performance gain ultimately so everything is cool. But up to you - it's your paper after all; this is just a suggestion.

% , the lower F1 scores of  emphasizes the importance of the schema context. \jz{Meanwhile, \model~w/o RG-EMD  This highlights the capability of our relation-guided entity mention detection module in strengthening entity retrieval under non-i.i.d settings.}

%Next, we conduct an ablation study to show the effectiveness of our generative entity retrieval module and the retrieval-guided generation module. 


%\addexp{can add \model-w/o fallback logical form generation;  (and others).} 



% \begin{table*}[ht]
% \small
% \centering
% \begin{tabular}{
% >{}p{0.2\linewidth}
% >{\centering\arraybackslash}m{0.05\linewidth}
% >{\centering\arraybackslash}m{0.05\linewidth} 
% >{\centering\arraybackslash}m{0.05\linewidth} 
% >{\centering\arraybackslash}m{0.05\linewidth} 
% >{\centering\arraybackslash}m{0.05\linewidth} 
% >{\centering\arraybackslash}m{0.05\linewidth} 
% >{\centering\arraybackslash}m{0.05\linewidth} 
% >{\centering\arraybackslash}m{0.05\linewidth} }
% \toprule
% & \multicolumn{2}{c}{\textbf{Overall}} & \multicolumn{2}{c}{\textbf{I.I.D.}} & \multicolumn{2}{c}{\textbf{Compositional}} & \multicolumn{2}{c}{\textbf{Zero-shot}} \\ \cline{2-9} 
% \multirow{-2}{*}{} \rule{0pt}{10pt}  \textbf{\vspace{-0.5cm}Model} & \textbf{EM}                   & \textbf{F1}  & \textbf{EM} & \textbf{F1} & \textbf{EM} & \textbf{F1}                     & \textbf{EM}  & \textbf{F1}  \\ \midrule
% TIARA & 75.3 & 81.9 & 88.4 & 91.2 & 66.4 & 74.8 & 73.3 & 80.7 \\
% FC-KBQA & 79.0 & 83.8 & 89.0 & 91.5 & 70.4 & 77.3 & 78.1 & 83.1 \\
% RetinaQA & 77.8 & 83.3 & 88.6 & 91.2 & 70.5 & 77.5 & 76.2 & 82.3 \\
% % TIARA + Generative Entity Retrieval &79.5 &84.3 &90.3 &92.3 &71.2 &78.1 &78.3 &83.3 \\
% % TIARA (LLaMA3-8B) & 79.9 & 85.6 & 88.6 & 92.3 & 72.7& 79.8 & 79.0 & 85.0 \\
% \midrule
% \textbf{\model} (Ours) \rule{0pt}{10pt} & 83.8 & 88.0 & 91.1 & 93.3 & 76.6 & 82.6 & 83.6 & 87.9 \\
% \hdashline
%   \rule{0pt}{10pt}\hspace{6pt} w/o SGER\tablefootnote{we replace schema-guided entity retrieval by SpanMD (mention dection method used in other SOTA studies.)} & 80.9 & 86.4 & 89.1 & 92.3 & 75.4 & 82.4 & 79.7 & 85.5 \\
%  \hspace{6pt} w/o SGLF\footnotemark[3] & 82.8 & 86.8 & 89.9 & 92.4 & 75.3 & 81.5 & 82.8 & 86.6\\
% \hspace{6pt} w/o SGER(top-1) & 83.0 & 86.7 & 89.2 & 91.2 & 75.4 & 80.8 & 83.4 & 86.9 \\ 
% \hspace{6pt} w/o R for LFS \\ 
% % \hline
% % \rule{0pt}{10pt}Top-1 Refined Entity (GER) + TIARA &79.5 &84.3 &90.3 &92.3 &71.2 &78.1 &78.3 &83.3\\
% %  TIARA's Entity Retrieval + RLG & 79.9 & 85.6 & 88.6 & 92.3 & 72.7& 79.8 & 79.0 & 85.0 \\

% % \hspace{6pt} w Top1-entity per mention & 83.0 & 86.7 & 89.2 &81.8 & 75.4 & 80.8 & 83.4 & 86.9 \\
% % \hspace{6pt} w/o Class & 82.8 & 86.8 & 89.9 & 92.4 & 75.3 & 81.5 & 82.8 & 86.6 \\ 
% % \hspace{6pt} w/o Generative Entity Retrieval  & 80.9 &	86.4 &	89.1 &	92.3 &	75.4 &	82.4 &	79.7 &	85.5 \\
% % \hspace{6pt} w/o Candidate Logical Forms & 80.1 & 83.9 & 90.7 & 92.5 & 75.2 & 80.6 & 77.5 & 81.5 \\
% % \hspace{6pt} w T5-Base & 78.7 & 83.2 & 88.1 & 90.4 & 69.5 & 75.6 & 78.4 & 83.3 \\
% % \hspace{6pt} w/o Context & 74.7 & 79.3 & 81.9 & 85.6 & 65.3 & 79.7 & 75.5 & 80.1 \\

% \bottomrule
% \end{tabular}
% \caption{Ablation study results on the validation set of GrailQA.}
% \label{tab:ablation}
% \end{table*}




%\paragraph{Generative Entity Retrieval} \jz{\model\ w/o generative entity retrieval exhibited a significant performance drop across both datasets. On GrailQA (Table~\ref{tab:ablation}), replacing the generative entity retrieval with TIARA's entity retrieval results led to a 3.4\% drop in the F1 score. Table~\ref{tab:entity_retrieval} further shows entity retrieval performance results on the GrailQA validation set, comparing GER method with commonly used entity retrieval methods in existing KBQA methods. Our GER method improves the F1 score by 7.0\%. Furthermore, our proposed candidate entity pruning strategy that combines the entity popularity-based pruning and relation-based pruning boosts the F1 score for entity retrieval by 2.1\%. To further validate the effectiveness of the GER module, we applied its retrieved entities to an open-source generate-then-retrieve method, TIARA (TIARA + Generative Entity Retrieval) in Table~\ref{tab:ablation}. We see that GET improves TIARA's F1 score by 3.0\% and EM by 5.5\%. On WebQSP, removing the GER from the merged entities set led to an 8.0\% drop in F1, while removing the ELQ entities resulted in only a 1.7\% drop (Table~\ref{tab:webqsp}). The experimental results demonstrate that using a relation-enhanced logical form sketch parser to generate entity mentions improves the identification of entity mentions in questions, even in zero-shot and compositional generalization settings.

%\paragraph{Retrieval-Guided Generation} 
%As shown in Table~\ref{tab:ablation}, \model\ w Top1-entity per mention negatives impacts F1 by 1.3 points overall, indicating that our generator (LLM) has the ability to select the correct entity from a set that includes false positive entities. The classes of the retrieved entities and the domain and range classes of the retrieved relations provide the generator with more KB context, resulting in a 1.3\% performance gain. We also replace LLaMA3-8B with T5-Base, a model widely used in our baselines~\cite{shu_tiara_2022, zhang_fc-kbqa_2023}, and find that the F1 score decreases by 4.8 points, while the EM score drops by 5.1 points with \model\ w T5-Base. Using candidate logical forms as a supplement when the generator fails to produce executable logical forms improves the F1 score by 4.9\% on GrailQA and by 2.1\% on WebQSP, suggesting the effectiveness of utilizing candidate logical forms as a supplement. Without providing any context (KB elements), using the generator to directly convert natural language questions into logical forms only achieves an EM score of 74.7 and an F1 score of 79.3, with particularly low F1 scores of 79.7 and 80.1 for compositional and zero-shot generalization, respectively. This highlights the necessity of the retrieval module for compositional and zero-shot generalization.}

\subsection{Module Applicability (Q3)}


Our relation-guided entity retrieval (\textbf{RG-EMD \& RG-CER}) module and schema-guided logical form generation (\textbf{DED \& SC}) module can be applied to existing KBQA models. We showcase such applicability with the TIARA model. As shown in Table~\ref{tab:exp_applicability}, by replacing the retrieval and generation modules of TIARA with ours, the F1 scores increase consistently for the non-I.I.D. tests.


Table~\ref{tab:exp_applicability} further reports F1 scores of \model\ when we replace LLaMA3.1-8B with \textbf{T5-base} (which is used by TIARA), and DeepSeek-R1-Distill-Llama-8B (\textbf{DS-R1-8B})~\cite{guo_deepseek_2025} for logical form generation. We see that, even with the same T5-base model for the logical form generator, \model\ outperforms TIARA consistently. This further confirms the effectiveness of our model design. As for DS-R1-8B, it offers accuracy slightly lower than that of the default LLaMA3.1-8B model. We conjecture that this is because DS-R1-8B is distilled from DeepSeek-R1-Zero, which focuses on reasoning capabilities and is not specifically optimized for the generation task.


%\subsection{Additional Results}
We also have results on parameter impact, model running time, a case study, and error analyses. They are documented in Appendices~\ref{app:paramater_study} to~\ref{app:error_analysis}.

\begin{table}[t]
\centering
\resizebox{\columnwidth}{!}{
\begin{tabular}{lcccc}
\toprule
\textbf{Model}                & \textbf{Overall} & \textbf{I.I.D.}& \textbf{Comp.} & \textbf{Zero.} \\ \midrule
TIARA (T5-base)   & 81.9   & 91.2  & 74.8  & 80.7  \\ 
\hspace{3pt} w RG-EMD \& RG-CER           & 84.3   & 92.3  & 78.1  & 83.3  \\
\hspace{3pt} w DED \& SC     & 85.6   & 92.3  & 79.8  & 85.0  \\
%\hspace{3pt} \jz{w DED}             &        &       &       &       \\
\midrule
\textbf{\model}            & \textbf{88.5}   & \textbf{94.6}  & \textbf{83.6}  & \textbf{87.9}  \\
\hspace{3pt} w T5-base       & 84.9   & 92.6  & 81.0  & 83.3  \\
%\hspace{3pt} w T5-large \jz{T5-3B?}      & 87.5   & 96.3  & 82.0  & 86.0  \\
 \hspace{3pt} w DS-R1-8B &   87.5     &  94.0     &  82.4     &  86.7     \\ \bottomrule
\end{tabular}
}
\caption{Module applicability results (F1 score) on the validation set of GrailQA. EM scores are in Appendix~\ref{app:applicability}.}\label{tab:exp_applicability}
\end{table}

\section{Conclusion}\label{sec:conclusion}
We proposed \model for the KBQA task. Our core innovations include: (1) using relation to guide the retrieval of entities; (2) deferring entity disambiguation to the logical form generation stage; and (3) enriching logical form generation with schema contexts to constrain search space. Together, we achieve a model that tops the leaderboard of a popular non-I.I.D. dataset GrailQA, outperforming SOTA models by 4.0\%, 2.9\%, and 3.3\% in F1 under compositional generalization, zero-shot generalization, and overall test settings, respectively. Our model also performs well in the I.I.D. setting, outperforming SOTA models on WebQSP.


\section*{Limitations}
%Despite the strong reported performance of \model, there are potentials to further improve the model.
First, like any other supervised models, \model requires annotated samples for training which may be difficult to obtain for many domains. Exploiting LLMs to generate synthetic training data is a promising direction to address this issue. Second, as discussed in the error analysis in Appendix~\ref{app:error_analysis}, errors can still arise from the relation retrieval, entity retrieval, and logical form generation modules. There are rich opportunities in further strengthening these modules. 
Particularly, as we start from relation extraction, the overall model accuracy relies on highly accurate relation extraction. It would be interesting to explore how well \model performs on even larger KBs with more relations.

\section*{Ethics Statement}
This work adheres to the ACL Code of Ethics and is based on publicly available datasets, used in compliance with their respective licenses. As our data contains no sensitive or personal information, we foresee no immediate risks. To promote reproducibility and further research, we also open-source our code.

%we acknowledge two major limitations in this study.Firstly, our method is trained on annotated question-logical form pairs, but the annotation cost for such data is expensive. The second The sencond major limitation lies in the time efficiency of our method. We report and discuss the training and inferecne runtime of our method in Appendix~\ref{app:time}. To guide the retreival and generation processes by KB schema, \model~utilizes additional KB queries to obtain the schema information for the corresponding KB elements. For example, In SG-LF, the class information of the selected KB elements are queried from the KB. The guidance from schema information significantly enhances the generalization capability of our method, but it also incurs an increase in time comsuption and computational cost. Compared to TIARA, our method taks on average 3.2 seconds longer inference time per questions.
%\section*{Acknowledgment}

% Bibliography entries for the entire Anthology, followed by custom entries
%\bibliography{anthology,custom}
% Custom bibliography entries only
\bibliography{references}

\newpage

\appendix


\section{S-Expression}\label{sec:app_sexpression}

S-expressions~\cite{gu_beyond_2021} use set-based semantics defined over a set of operators and operands. The operators are represented as functions. 
Each function takes a number of arguments (i.e., the operands). Both the arguments and the return values of the functions are either a set of entities or entity tuples (or tuples of an entity and a literal). The functions available in S-expressions are listed in Table~\ref{tab:logical_form_operators}, where a set of entities typically refers to a class (recall that a class is defined as a set of entities sharing common properties) or individual entities, and a binary tuple typically refers to a relation. %By applying those functions defined in our grammar, we are able to get more complex set of entities and binary tuples. 

\begin{table*}[h]
\small
\setlength{\tabcolsep}{2pt}
%\begin{tabular}{p{0.15\linewidth}p{0.3\linewidth}p{0.4\linewidth}}
\begin{tabular}{lll}
\toprule
\multicolumn{1}{l}{\textbf{Function}}                                             & \multicolumn{1}{l}{\textbf{Return value}} & \multicolumn{1}{l}{\textbf{Description}}                                                               \\ \midrule
(\texttt{AND} $u_1$ $u_2$)                                                                       & a set of entities                    & The \texttt{AND} function returns the intersection of two sets $u_1$ and $u_2$                                               \\ \hline
(\texttt{COUNT} $u$)                                                                         & a singleton set of integers           & The \texttt{COUNT} function returns the cardinality of set $u$                                                 \\ \hline
(\texttt{R} $b$)                                                                             & a set of (entity, entity) tuples     & The \texttt{R} function reverses each binary tuple $(x, y)$ in set $b$ to $(y, x)$                                   \\ \hline
(\texttt{JOIN} $b$ $u$)                                                                        & a set of entities                    & Inner \texttt{JOIN} based on entities in set $u$ and the second element of tuples in set $b$                                    \\ \hline
(\texttt{JOIN} $b_1$ $b_2$)                                                                      & a set of (entity, entity) tuples     & Inner \texttt{JOIN} based on the first element of tuples in set $b_2$ and the second element\\
& &  of tuples in set $b_1$             \\ \hline
\begin{tabular}[c]{@{}l@{}}(\texttt{ARGMAX} $u$ $b$)\\ (\texttt{ARGMIN} $u$ $b$)\end{tabular}                & a set of entities                    & These functions return $x$ in $u$ such that $(x,y) \in b$ and $y$ is the largest / smallest                                   \\ \hline
\begin{tabular}[c]{@{}l@{}}(\texttt{LT} $b$ $n$)\\ (\texttt{LE} $b$ $n$)\\ (\texttt{GT} $b$ $n$)\\ (\texttt{GE} $b$ $n$)\end{tabular} & a set of entities                    & These functions return all $x$ such that $(x, v) \in b$ and $v$ $<$ / $\le$ / $>$ / $\ge$ $n$ \\ \bottomrule
\end{tabular}
\caption{Functions (operators) defined in S-expressions ($u$: a set of entities, $b$: a set of (entity, entity or literal) tuples, $n$: a numerical value).}\label{tab:logical_form_operators}
\end{table*}

\begin{figure*}[h]
    \centering
    \includegraphics[width=\textwidth]{figures/prompt_example_new.png}
    \caption{Example prompt to our fine-tuned LLM-based logical form generator for an input question: \textsf{Captain pugwash makes an appearance in which comic strip?}}
    \label{fig:prompt_example}
\end{figure*}
\section{Prompt Example}\label{app:prompt}

We show an example prompt to our fine-tuned LLM-based logical form generator containing top-20 relations and top-2 entities per mention retrieved by our model in Figure~\ref{fig:prompt_example}.

\section{Additional Details on the WebQSP Dataset}\label{app:WebQSP}
\textbf{WebQuestionsSP} (WebQSP)~\cite{yih_value_2016} is an I.I.D. dataset. It contains 4,937 questions collected from Google query logs, including 3,098 questions for training and 1,639 for testing, each annotated with a target SPARQL query. %We convert each SPARQL query into the corresponding S-expression and extract 
We follow GMT-KBQA~\cite{hu_logical_2022}, TIARA~\cite{shu_tiara_2022} to separate 200 questions from the training questions to form the validation set.

\section{Baseline Models}\label{sec:app_baselines}
The following models are tested against \model\ on the GrailQA dataset:
\begin{itemize}
    \item RnG-KBQA~\cite{ye_rng-kbqa_2022} enumerates and ranks all possible logical forms within two hops from the entities retrieved by an entity retrieval step. It  uses a Seq2Seq model to generate the target logical form based on the input question and the top-ranked candidate logical forms.
    \item TIARA~\cite{shu_tiara_2022} shares the same overall procedure with RnG-KBQA. It further retrieves entities, relations, and classes based on the input question and feeds these KB elements into the Seq2Seq model together with the question and the top-ranked candidate logical forms to generate the target logical form.  
    
    
    %\jz{It demonstrates the connectivity of the KB elements through example logical forms starting from the entities. [How are these example logical forms obtained and how are they used?]} Finally, a Seq2Seq model converts the question and retrieved KB elements into the target logical form.

    \item TIARA+GAIN~\cite{shu_data_2024} enhances TIARA using a training data augmentation strategy. It synthesizes additional question-logical form pairs for model training to enhance the model's capability to handle more entities and relations. This is done by a 
     graph traversal to randomly sample logical forms from the KB  and a PLM to generate questions corresponding to the logical forms (i.e., the ``GAIN'' module). TIARA+GAIN is first tuned using the synthesized data and then tuned on the target dataset, for its retriever and generator modules which both use PLMs.
    
    \item Decaf~\cite{yu_decaf_2023} uses a Seq2Seq model that takes as input a question and a linearized question-specific subgraph of the KG and jointly decodes into both a  logical form and an answer candidate. The logical form is then executed, which produces a second answer candidate if successful. The final answer is determined from these two answer candidates with a scorer model. 

    \item Pangu~\cite{gu_dont_2023} formulates logical form generation as an iterative  enumeration process starting from the entities retrieved by an entity retrieval step. 
    At each iteration, partial logical forms generated so far are extended following paths in the KB to generate more and longer partial logical forms. A language model is used to select the top partial logical forms to be explored in the next iteration, under either fined-tuned models (T5-3B) or few-shot in-context learning (Codex). 

    
    \item FC-KBQA~\cite{zhang_fc-kbqa_2023} employs an intermediate module to test the connectivity between the retrieved KB elements, and it  generates the target logical form using the connected pairs of the retrieved KB elements through a Seq2Seq model.

     
    \item RetinaQA~\cite{faldu_retinaqa_2024} uses both a ranking-based method and a generation-based method (TIARA) to generate logical forms, which are then scored by a discriminative model to determine the output logical form.
    
    \item KB-BINDER~\cite{li_few-shot_2023} uses a training-free few-shot in-context learning model based on LLMs. It generates a draft logical form by showcasing the LLM examples of questions and logical forms (from the training set) that are similar to the given test question. Subsequently, a retrieval module grounds the surface forms of the KB elements in the draft logical form to specific KB elements.
    
    % \item FlexKBQA~\cite{li_flexkbqa_2024} is also a few-shot in-context learning model based on LLMs. To address the issue that the model generated logical forms are often unexecutable, it samples executable logical forms from the KB (like GAIN above) and instructs an LLM to generate a corresponding user question. These logical form-question pairs can then be used to fine-tune a lightweight model for logical form generation given an input question. \jz{Double check}
    \item FlexKBQA~\cite{li_flexkbqa_2024} considers limited training data and leverages an 
    LLM to generate additional training data. 
    It samples executable logical forms from the KB and utilizes an LLM with few-shot in-context learning to convert them into natural language questions, forming synthetic training data. These data, together with a few real-world training samples, are used to train a KBQA model. Then, the model is used to generate logical forms with more real world questions (without ground truth), which are filtered through an execution-guided module to prune the erroneous ones. The remaining logical forms and the corresponding real-world questions are used to train a new model. This process is repeated, to align the distributions of  synthetic training data and real-world questions. 
    
    %It introduces an execution-guided teacher-student iterative training method to bridge the gap between synthetic and real-world questions. The teacher model generates pseudo logical forms for unlabeled questions, which are filtered for quality, and used to iteratively train a student model, refining the logical form parser.}
    
    
    % is a flexible few-shot KBQA framework that leverages few-shot in-context learning to generate synthetic data using an LLM for training a lightweight model. It first extracts logical form templates from the few-shot annotated samples by replacing entities and relations with variables (e.g. ent0, rel0, ent1). These templates are then step-wise grounded with KB elements collected from the KB, generating a substantial number of executable logical forms. A LLM translates the obtained logical forms into natural language questions through in-context learning, constructing a synthetic dataset of logical form-question pairs. 
    
    % To address the distribution discrepancy between synthetic and real-world questions, it proposes an execution-guided teacher-student iterative training method. First, a teacher model is trained using synthetic data and a few annotated samples. The teacher model then generates pseudo logical forms for unlabeled real-world user questions, and an execution-guided filtering mechanism removes unexecutable or low-quality data. The filtered data, along with the synthetic data, is used to train a student model, which becomes the new teacher model in the next iteration. This process is repeated until the model converges, resulting in the final logical form parser.}
    
\end{itemize}


The following models are tested against \model\ on the WebQSP dataset:
\begin{itemize}
    % \item Subgraph Retrieval (SR)~\cite{zhang_subgraph_2022} uses a sequential decision process to progressively expand the subgraph corresponding to the question starting from the topic entity.
    \item Subgraph Retrieval (SR)~\cite{zhang_subgraph_2022} focuses on retrieving a KB subgraph relevant to the input question. It does not concern retrieving the exact question answer by  reasoning over the subgraph. Starting from the topic entity, it performs a top-$k$ beam search at each step to progressively expand into a subgraph, using a scorer module to score the candidate relations to be added to the subgraph next. 

    %In each expansion step, it uses a dual-encoder to encode a  candidate relation for the expansion and the concatenation of the input question and the historical relation path from previous steps, respectively.    
    %The dot product of the obtained embeddings is used to select the top-ranked candidate relations for the current step or terminating the expansion. 
    
    \item Evidence Pattern Retrieval (EPR)~\cite{ding_enhancing_2024} aims to extract subgraphs with fewer noise entities. It starts from the topic entities and expands by retrieving and ranking atomic (topic entity-relation or relation-relation) patterns relevant to the question. This forms a set of relation path graphs (i.e., the candidate \emph{evidence patterns}). The relation path graphs are then ranked to select the most relevant one. By further retrieving the entities on the selected relation path graph, EPR obtains the final subgraph relevant to the input question. 
    

    
    \item Neural State Machine (NSM)~\cite{he_improving_2021} is a reasoning model to find answers for the KBQA problem from a subgraph (e.g., retrieved by SR or EPR). It address the issue of lacking intermediate-step supervision signals when reasoning through the subgraph to reach the answer entities. This is done by training a so-called teacher model that follows a bidirectional reasoning mechanism starting from both the topic entities and the answer entities. During this process, the  ``distributions'' of entities, which represent their probabilities to lead to the answer entities (i.e., intermediate-step supervision signal), are propagated. 
    A second model, the so-called student model, learns from the teacher model to generate the entity distributions, with knowledge of the input question and the topic entities but not the answer entities. Once trained, this model can be used for KBQA answer reasoning. 

    
    
    % KBQA~\cite{he_improving_2021} is a widely used answer reasoning model on subgraphs, often combined with different subgraph retrieval methods (e.g. SR~\cite{zhang_subgraph_2022}, EPR~\cite{ding_enhancing_2024}), which employs a teacher-student framework to solve multi-hop KBQA tasks. The student model is based on NSM~\cite{hudson_nsm_2019} that consists of two components: the instruction component and the reasoning component. The instruction component uses an LSTM to encode the question and generate a series of instruction vectors that guide the reasoning process. The reasoning component relies on a "propagation-aggregation" mechanism to aggregate and update information about entities and relations in the KB based on the current instruction, gradually maintaining and updating the entity distribution. On the other hand, the teacher model generates intermediate entity distributions by simultaneously performing forward (starting from the topic entity) and backward (starting from the answer entity) reasoning. It uses a consistency constraint between the forward and backward reasoning processes, specifically minimizing the difference between the two using the Jensen-Shannon divergence at each intermediate step. This ensures the reliability of the intermediate entity distributions. Finally, the intermediate supervision signal generated by the teacher model is used to guide the student model, optimizing its reasoning path and improving the overall answering reasoning accuracy.
    
    \item UniKGQA~\cite{jiang_unikgqa_2023} integrates both retrieval and reasoning stages to enhance the accuracy of multi-hop KBQA tasks. It trains a PLM to learn the semantic relevance between every relation and the input question. The semantic relevance information is  propagated and aggregated through the KB to form the semantic relevance between the entities and the input question. The entity with the highest semantic relevance is returned as the answer.
        
    
    \item ChatKBQA~\cite{luo_chatkbqa_2024} fine-tunes an open-source LLM to map questions into draft logical forms. The  ambiguous KB items in the draft logical forms are replaced with specific KB elements by a separate retrieval module.
    
    \item TFS-KBQA~\cite{wang_no_2024} fine-tunes an LLM for more accurate logical form generation with three strategies.  The first strategy directly fine-tunes the LLM to map natural language questions into draft logical forms containing entity names instead of entity IDs. The second strategy breaks the mapping process into two steps, first to generate relevant KB elements, and then to generate draft logical forms using the KB elements. The third strategy fine-tunes the LLM to directly generate the answer to an input question. 
    After applying the three fine-tuning strategies, the LLM is used to map natural language questions into draft logical forms at model inference. A separate entity linking module is used to further map the entity names in draft logical forms into entity IDs. 

\end{itemize}




% \begin{figure*}[h]
%     \centering
%     \includegraphics[width=0.7\linewidth]{figures/case_study.png}
%     \caption{Case Study of logical form generation by \model\ and other models on the GrailQA validation set. Incorrect relations and entities are marked in red, while the correct relations and entities are colored in green and blue, respectively}
%     \label{fig:case_study}
% \end{figure*}

% \begin{figure*}[h]
%     \centering
%     \includegraphics[width=0.7\linewidth]{figures/md_case.png}
%     \caption{Case study of entity mention detection by our model and SpanMD (a mention detection method commonly used by SOTA KBQA models) on the GrailQA validation set. The incorrect entity mention detected is colored in red, while the correct entity mentions detected are colored in green and blue, respectively. \jz{``GER'' $\rightarrow$ ``Ours''} }
%     \label{fig:md_case}
% \end{figure*}



% Please add the following required packages to your document preamble:
% \usepackage{multirow}




% \begin{figure*}[h]
%     \centering
%     \includegraphics[width=0.8\linewidth]{figures/rlg_case.png}
%     \caption{Case study of logical form generation by our \model\ model and \jz{two representative baseline models TIARA and Pangu} on the GrailQA validation set. Incorrect relations and entities retrieved are colored in red, while correct relations and entities retrieved are colored in green and blue, respectively. The same sets of noisy entities and relations are retrieved by all three models, while only our model \model\ is able to produce the correct logical form. \jz{Use table instead, separate TIARA/Pangu from ours, same for the figure above.}}
%     \label{fig:rlg_case}
% \end{figure*}

% \section{Case Study}

% Figure~\ref{fig:case_study} shows an example case from the GrialQA validation set, with the prediction from our \model, the SOTA Pangu, and TIARA. In this example, both the top-1 candidate for entity retrieval and relation retrieval are non-optimal. Unlike previous methods PANGU and TIARA, which solely rely on the top-1 candidate entity to generate logical forms, our model can combine KB elements using class-based contextualization and select the optimal combination of KB elements based on their relevance to the question to form the final logical form, without overly depending on the performance of the retrieval module.


\begin{table*}[ht]
\small
\centering
\begin{tabular}{
>{}p{0.3\linewidth}
>{\centering\arraybackslash}m{0.05\linewidth}
>{\centering\arraybackslash}m{0.05\linewidth} 
>{\centering\arraybackslash}m{0.05\linewidth} 
>{\centering\arraybackslash}m{0.05\linewidth} 
>{\centering\arraybackslash}m{0.05\linewidth} 
>{\centering\arraybackslash}m{0.05\linewidth} 
>{\centering\arraybackslash}m{0.05\linewidth} 
>{\centering\arraybackslash}m{0.05\linewidth} }
\toprule
& \multicolumn{2}{c}{\textbf{Overall}} & \multicolumn{2}{c}{\textbf{I.I.D.}} & \multicolumn{2}{c}{\textbf{Compositional}} & \multicolumn{2}{c}{\textbf{Zero-shot}} \\ 
\cmidrule{2-9}
\multirow{-2}{*}{}  \textbf{\vspace{-0.5cm}Model} & \textbf{EM}                   & \textbf{F1}  & \textbf{EM} & \textbf{F1} & \textbf{EM} & \textbf{F1}                     & \textbf{EM}  & \textbf{F1}  \\ \midrule
% TIARA & 75.3 & 81.9 & 88.4 & 91.2 & 66.4 & 74.8 & 73.3 & 80.7 \\
% FC-KBQA & 79.0 & 83.8 & 89.0 & 91.5 & 70.4 & 77.3 & 78.1 & 83.1 \\
% RetinaQA & 77.8 & 83.3 & 88.6 & 91.2 & 70.5 & 77.5 & 76.2 & 82.3 \\
% TIARA + Generative Entity Retrieval &79.5 &84.3 &90.3 &92.3 &71.2 &78.1 &78.3 &83.3 \\
% TIARA (LLaMA3-8B) & 79.9 & 85.6 & 88.6 & 92.3 & 72.7& 79.8 & 79.0 & 85.0 \\
%\midrule
%\textbf{\model} & \textbf{83.8} & \textbf{88.0} & \textbf{91.1} & \textbf{93.3} & \textbf{76.6} & \textbf{82.6} & \textbf{83.6} & \textbf{87.9} \\
\textbf{\model} & \textbf{85.1} & \textbf{88.5} & \textbf{93.1} & \textbf{94.6} & \textbf{78.4} & \textbf{83.6} & \textbf{84.4} & \textbf{87.9} \\
\hdashline
  \rule{0pt}{10pt}\hspace{6pt} w/o RG-EMD & 81.3 & 85.3 & 90.6 & 92.4 & 74.4 & 80.2 & 80.2 & 84.3 \\
 \hspace{6pt} w/o RG-CER &  82.8 & 86.5 & 90.2 & 92.1 & 75.4 & 81.1 & 82.7 & 86.3 \\ 
\hspace{6pt} w/o DED & 84.3 & 87.8 & 92.6 & 94.0 & 77.1 & 82.4 & 83.7 & 87.2 \\ 
\hspace{6pt} w/o SC & 76.6 & 79.2 & 91.7 & 92.9 & 72.3 & 77.4 & 71.7 & 73.9\\
\hspace{6pt} w/o Fallback LF & 81.8 & 84.6 & 92.8 & 94.1 & 77.3 & 81.8 & 78.7 & 81.5\\
% \hline
% \rule{0pt}{10pt}Top-1 Refined Entity (GER) + TIARA &79.5 &84.3 &90.3 &92.3 &71.2 &78.1 &78.3 &83.3\\
%  TIARA's Entity Retrieval + RLG & 79.9 & 85.6 & 88.6 & 92.3 & 72.7& 79.8 & 79.0 & 85.0 \\

% \hspace{6pt} w Top1-entity per mention & 83.0 & 86.7 & 89.2 &81.8 & 75.4 & 80.8 & 83.4 & 86.9 \\
% \hspace{6pt} w/o Class & 82.8 & 86.8 & 89.9 & 92.4 & 75.3 & 81.5 & 82.8 & 86.6 \\ 
% \hspace{6pt} w/o Generative Entity Retrieval  & 80.9 &	86.4 &	89.1 &	92.3 &	75.4 &	82.4 &	79.7 &	85.5 \\
% \hspace{6pt} w/o Candidate Logical Forms & 80.1 & 83.9 & 90.7 & 92.5 & 75.2 & 80.6 & 77.5 & 81.5 \\
% \hspace{6pt} w T5-Base & 78.7 & 83.2 & 88.1 & 90.4 & 69.5 & 75.6 & 78.4 & 83.3 \\
% \hspace{6pt} w/o Context & 74.7 & 79.3 & 81.9 & 85.6 & 65.3 & 79.7 & 75.5 & 80.1 \\

\bottomrule
\end{tabular}
\caption{Ablation study results on the validation set of GrailQA.}
\label{tab:ablation_full}
\end{table*}

\section{Implementation Details}\label{app:implemention_details}

All our experiments are run on a machine with an NVDIA A100 GPU and 120 GB of RAM. We fine-tuned three \texttt{bert-base-uncased} models for a maximum of three epochs each, for relation retrieval, entity ranking, and fallback logical form ranking.
For relation retrieval, we randomly sample 50 negative samples for each question to train the model to distinguish between relevant and irrelevant relations. 

For each dataset, a \texttt{T5-base} model is fine-tuned for 5 epochs as our logical form sketch parser, with a beam size of 3 (i.e., $k_L = 3$) for GrailQA, and 4 for WebQSP. For candidate entity retrieval, we use the same number (i.e., $k_{E1} + k_{E2}  = 10$) of candidate entities per mention as that used by the baseline models~\cite{shu_tiara_2022,ye_rng-kbqa_2022}. The retrieved candidate entities for a mention consist of entities with the top-$k_{E1}$ popularity scores and $k_{E2}$ entities connected to the top-ranked relations in $R_q$, where $k_{E1} = 1$, $k_{E2} = 9$ for GrailQA, $k_{E1} = 3$, $k_{E2} = 7$ for WebQSP. We select the top-20 (i.e., $k_R$ = 20) relations and the top-2 (i.e., $k_{E3} = 2$) entities (for each entity mention) retrieved by our model. For WebQSP, we also use the candidate entities obtained from the off-the-shelf entity linker ELQ~\cite{li_efficient_2020}. 

Finally, we fine-tune LLaMA3.1-8B with LoRA~\cite{hu_lora_2021} for logical form generation. On GrailQA, LLaMA3.1-8B is fine-tuned for 5 epochs with a learning rate of $0.0001$. On WebQSP, it is fine-tuned for 20 epochs with the same learning rate (as it is an I.I.D. dataset where more epochs are beneficial). During inference, we generate logical forms by beam search with a beam size of 10 (i.e., $K_O = 10$). The generated logical forms are executed on the KB to filter non-executable ones. If none of the logical forms are executable, we check candidate logical forms from the fallback procedures, and the result of the first executable one is returned as the answer set.
%\jz{Any updates needed for this subsection?} 


% Our system parameters have been chosen empirically. While there are a few of them, their exact values do not have strong impact on the final model performance, \jz{and the choice of parameter values generalize well across  datasets. The same parameter values are used on both datasets. [not true any more?]} 
Our system parameters are selected empirically. There are only a small number of parameters to consider. As shown in the parameter study later, our model performance shows stable patterns against the choice of parameter values. The parameter values do not take excessive fine-tuning. 

\section{Full Ablation Study Results (GrailQA)}\label{app:ablation}
%\jz{Fix table and add discussion on the results}

Table~\ref{tab:ablation_full} presents the full ablation study results on the validation set of GrailQA. We observe a similar trend to that of the F1 score results reported earlier --  all ablated model variants yield lower EM scores compared to the full model. 

For the retrieval modules, RG-EMD improves the F1 score by 3.2 points and the EM score by 3.8 points on GrailQA (i.e., \model\ vs. \model w/o RG-EMD for overall results), while achieving a 1.9-point increase in the F1 score on WebQSP (see Table~\ref{tab:webqsp} earlier). It achieves an increase of 3.4 points or larger in the F1 score on the compositional and zero-shot tests, which is larger than the 2.2-point improvement on the I.I.D. tests. This shows that relation-guided mention detection effectively enhances the generalization capability of KBQA entity retrieval. For the other module RG-CER, removing it (\model w/o RG-CER) results in a 2.5-point drop in the F1 score for both the I.I.D. and compositional tests, while the impact is smaller on the zero-shot tests (1.6 points). This is because the lower accuracy in relation retrieval under zero-shot tests leads to error propagation into relation-guided candidate entity retrieval, reducing the benefits of this module.  

For the generation modules, \model\ w/o DED negatively impacts the F1 scores on both GrailQA and WebQSP, confirming that deferring entity disambiguation effectively mitigate error propagation between the retrieval and generation stages. For \model w/o SC, it reduces the F1 score by 1.7 points and 3.2 points on the GrialQA I.I.D. tests and on WebQSP. The drop is more significant on the compositional and zero-shot tests, i.e., by 6.2 points and 14.0 points, respectively. This indicates that schema contexts can effectively guide the LLM to reason and identify the correct combinations of KB elements unseen at training.


In Table~\ref{tab:ablation_full}, we present an additional model variant, \model w/o Fallback LF, which removes the fall back logical form generation strategy from \model. We see that \model\ has lower accuracy without the strategy. %Importantly, even without this fall back strategy, our model has high accuracy results in both EM and F1 comparing with the baseline models as shown in the table. 
We note that this fallback strategy is \emph{not} the reason why \model\ outperforms the baseline models. 
TIARA also uses this fallback strategy, while RetinaQA uses the top executable logical form from the fallback strategy as one of the options to be selected by its  discriminator to determine the final logical form output.

\section{Full Module Applicability  Results}\label{app:applicability}


\begin{table*}[t]
\small
\centering
\begin{tabular}{
>{}p{0.3\linewidth}
>{\centering\arraybackslash}m{0.05\linewidth}
>{\centering\arraybackslash}m{0.05\linewidth} 
>{\centering\arraybackslash}m{0.05\linewidth} 
>{\centering\arraybackslash}m{0.05\linewidth} 
>{\centering\arraybackslash}m{0.05\linewidth} 
>{\centering\arraybackslash}m{0.05\linewidth} 
>{\centering\arraybackslash}m{0.05\linewidth} 
>{\centering\arraybackslash}m{0.05\linewidth} }
\toprule
& \multicolumn{2}{c}{\textbf{Overall}} & \multicolumn{2}{c}{\textbf{I.I.D.}} & \multicolumn{2}{c}{\textbf{Compositional}} & \multicolumn{2}{c}{\textbf{Zero-shot}} \\ \cline{2-9} 
\multirow{-2}{*}{} \rule{0pt}{10pt}  \textbf{\vspace{-0.5cm}Model} & \textbf{EM}                   & \textbf{F1}  & \textbf{EM} & \textbf{F1} & \textbf{EM} & \textbf{F1}                     & \textbf{EM}  & \textbf{F1}  \\ \midrule 
TIARA (T5-base) & 75.3 & 81.9 & 88.4 & 91.2 & 66.4 & 74.8 & 73.3 & 80.7 \\
\hspace{6pt} w RG-EMD \& RG-CER & 79.5 &84.3 &90.3 &92.3 &71.2 &78.1 &78.3 &83.3 \\
\hspace{6pt} w DED \& SC &79.9 & 85.6 & 88.6 & 92.3 & 72.7& 79.8 & 79.0 & 85.0\\  
%FC-KBQA (COLING 2024) & 79.0 & 83.8 & 89.0 & 91.5 & 70.4 & 77.3 & 78.1 & 83.1 \\
%RetinaQA (ACL 2024) & 77.8 & 83.3 & 88.6 & 91.2 & 70.5 & 77.5 & 76.2 & 82.3 \\
% TIARA + Generative Entity Retrieval &79.5 &84.3 &90.3 &92.3 &71.2 &78.1 &78.3 &83.3 \\
% TIARA (LLaMA3-8B) & 79.9 & 85.6 & 88.6 & 92.3 & 72.7& 79.8 & 79.0 & 85.0 \\
\midrule
%\textbf{\model} (Ours) \rule{0pt}{10pt} & 83.8 & 88.0 & 91.1 & 93.3 & 76.6 & 82.6 & 83.6 & 87.9 \\
\textbf{SG-KBQA} & \textbf{85.1} & \textbf{88.5} & \textbf{93.1} & \textbf{94.6} & \textbf{78.4} & \textbf{83.6} & \textbf{84.4} & \textbf{87.9}\\
\hspace{6pt} w T5-base & 80.6 & 84.9 & 89.9 & 92.6 & 73.8 & 81.0 & 79.4 & 83.3\\
\hspace{6pt} w DS-R1-8B & 83.6 & 87.5 & 92.3 & 94.0 & 75.4 & 82.4 & 83.1 & 86.7 \\


%   SGER + TIARA & 79.5 &84.3 &90.3 &92.3 &71.2 &78.1 &78.3 &83.3 \\
%  TIARA + SGLF & 79.9 & 85.6 & 88.6 & 92.3 & 72.7& 79.8 & 79.0 & 85.0\\
%  TIARA + multiple entities  \\ 
% SG-KBQA w T5-base & 81.7 & 85.6 & 93.8 & 95.6 & 73.3 & 80.5 & 79.9 & 83.3 \\
% SG-KBQA w T5-large & 83.5 & 87.5 & 95.2 & 96.3 & 73.5 & 82.0 & 82.5 & 86.0\\
% SG-KBQA w deepseek-r1-distilled-8B \\
% \hline
% \rule{0pt}{10pt}Top-1 Refined Entity (GER) + TIARA &79.5 &84.3 &90.3 &92.3 &71.2 &78.1 &78.3 &83.3\\
%  TIARA's Entity Retrieval + RLG & 79.9 & 85.6 & 88.6 & 92.3 & 72.7& 79.8 & 79.0 & 85.0 \\

\bottomrule
\end{tabular}
\caption{Full module applicability results on the validation set of GrailQA.}
\label{tab:applicability_full}
\end{table*}

To evaluate the applicability of our proposed modules, we conduct a module applicability study with TIARA (an open-source retrieve-then-generate baseline) and different generation models (i.e., T5-base and DeepSeek-R1-Distill-Llama-8B). 


Table~\ref{tab:applicability_full} reports the results. Replacing TIARA's entity retrieval module with ours (TIARA w RG-EMD \& RG-CER) helps boost the EM and F1 scores by 4.2 and 2.4 points overall, comparing against the original TIARA model. This improvement is primarily from the tests with KB elements or compositions that are unseen at training, as evidenced by the larger performance gains on the compositional and zero-shot tests, i.e., 3.3 and 2.6 points in the F1 score, respectively. Similar patterns are observed for TIARA w DED \& SC that replaces TIARA's logical form generation module with ours. 
These results demonstrate that our proposed modules can enhance the retrieval and generation steps of other compatible models, especially under non-I.I.D. settings. 

Further, using the same language model (i.e., T5-base in TIARA) to form logical form generation modules, our model \model\ w T5-base still outperforms TIARA by 5.3 points  3.0 points in the EM and F1 scores for the overall tests. This confirms that the overall effectiveness of our model stems from its design rather than the use of a larger model for logical form generation. As for \model w/ DS-R1-8B, it reports close performance to \model, indicating that \model does not rely on a particular LLM.


% \jz{Fix table and add discussion on the results}
% \jz{Also results on WebQSP?}


% \section{Entity Retrieval Results}\label{app:er_results}

% \begin{table}[H]
%     \small
%     \centering
%     \begin{tabular}{lccc} 
%     \toprule \textbf{Model} & \textbf{P} & \textbf{R} & \textbf{F1} \\
%     \midrule 
%     RnG-KBQA  & 84.1 & 86.8 & 80.4 \\
%     TIARA  & 87.2 & 88.6& 85.4 \\
%     \midrule
%     \textbf{SG-ER (Top-1)} & \textbf{91.9} & \textbf{93.6} & \textbf{90.5} \\
%     \hspace{6pt}w/o RG-EMD & 88.9 & 91.3 & 88.2 \\
%     \hspace{6pt}w/o RG-CER & 88.7 & 90.0 & 86.9 \\
%     \bottomrule
%     \end{tabular}
%         \caption{Precision (P), recall (R) and F1 of entity retrieval (\%) on the validation set of GrailQA.}
%     \label{tab:entity_retrieval}
% \end{table}


% \sxfix{We report the performance of our SG-ER (Top-1) on the GrailQA validation set in Table~\ref{tab:entity_retrieval}. We compare against the following baselines: 1)\textbf{RnG-KBQA}~\cite{ye_rng-kbqa_2022} which adopts a BERT-NER system to detect entity mentions. 2)\textbf{TIARA}~\cite{shu_tiara_2022} which models entity mention detection to span classification task to detect entity spans. To ensure a fair comparison with the baselines, we follow their approaches by extracting the top-1 entity for each mention from our SG-ER. It can be observed that our SG-ER siginificantly surpasses all the baselines by at least 5.1 F1 points.

% Furthermore, \textbf{w/o RG-EMD} shares the same mention detector with TIARA, indicating that our RG-CER module is effective by improving the entity retrieval F1 by 2.8 points. \textbf{w/o RG-CER} shares the same candidate entity retrieval methods with the baselines but boosts entity retrieval F1 by at least 1.5 points. This demonstrates that our RG-EMD can more accurately identify entity boundaries in the input question.}


\section{Parameter Study}\label{app:paramater_study}

We conduct a parameter study to investigate the impact of the choice of values for our system parameters. When the value of a parameter is varied, default values as mentioned in Appendix~\ref{app:implemention_details} are used for the other parameters. 

%\jz{Add results and discussion}
\begin{figure}[htb] 
    \begin{minipage}[b]{0.5\linewidth}  
        \centering
        \includegraphics[width=\textwidth]{figures/KL.pdf} 
        \captionsetup{font=small}
        \subcaption{$k_L$}
       % \label{fig:sub1}
    \end{minipage}%
    \hfill 
    \begin{minipage}[b]{0.5\linewidth}  
        \centering
        \small
        \includegraphics[width=\textwidth]{figures/KE1.pdf}
        \captionsetup{font=small}
        \subcaption{$k_{E1}$}
        %\label{fig:sub2}
    \end{minipage}
    \caption{Impact of $k_L$ and $k_{E1}$ on the recall of candidate entity retrieval.} %\jz{font size in figure too small, could reduce the data points if needed more space; candidate entity coverage $\rightarrow$ recall of candidate entity retrieval?}}  
    \label{fig:kl_ke1}
\end{figure}


Figure~\ref{fig:kl_ke1} presents the impact of  $k_L$ and $k_{E1}$ on the recall of candidate entity retrieval (i.e., the average percentage of ground-truth entities returned by our candidate entity retrieval module for each test sample). Here, for the GrailQA dataset, we report the results on the overall tests (same below). 
Recall that $k_L$ means the number of logical form sketches from which entity mentions are extracted, while $k_{E1}$ refers to the number of candidate entities retrieved based on the popularity scores. 

As $k_L$ increases, the recall of candidate entity retrieval grows, which is expected. The growth diminishes gradually. This is because a small number of questions contain complex entity mentions that are difficult to handle (see error analysis in Appendix~\ref{app:error_analysis}). As $k_L$ increases, the precision of the retrieval also reduces, which brings noise into the entity retrieval results and additional computational costs. 
To strike a balance, we set $k_L = 3$ for GrailQA and  $k_L = 4$ for WebQSP. We also observe that the recall on WebQSP is lower than that on GrailQA. This is because  WebQSP has a smaller training set to learn from. 


As for $k_{E1}$, when its value increases, the candidate entity recall generally drops. This is because an increase in $K_{E1}$ means to select more candidate entities based on popularity while fewer from those connected to the top retrieved relations but with lower popularity scores. 
Therefore, we default $k_{E1}$ at $1$ for GrailQA and $3$ for WebQSP, which yield the highest recall for the two datasets, respectively. 
Recall that we set the total number of candidate entities for each entity mention to 10 ($K_{E1} + K_{E2} = 10$), following our baselines (e.g., TIARA, RetinaQA, and Pangu). Therefore, we omit another study on $K_{E2}$, as it varies with $K_{E1}$.

\begin{figure}[htb] 
    \begin{minipage}[b]{0.5\linewidth}  
        \centering
        \includegraphics[width=\textwidth]{figures/KR.pdf} 
        \captionsetup{font=small}
        \subcaption{$k_R$}
        %\label{fig:sub1}
    \end{minipage}%
    \hfill 
    \begin{minipage}[b]{0.5\linewidth}  
        \centering
        \small
        \includegraphics[width=\textwidth]{figures/KE3.pdf}
        \captionsetup{font=small}
        \subcaption{$k_{E3}$}
        %\label{fig:kr_ke3}
    \end{minipage}
    \caption{Impact of $k_R$ and $k_{E3}$ on the overall F1 score.}
    \label{fig:kr_ke3}
\end{figure}

\begin{table*}[h]
\small
\centering
\begin{tabular}{ll}
\hline
\textbf{Question:} What is the name for the atomic units of length? \\ \hline \addlinespace[2pt]
\textbf{SpanMD:}  What is the name for the atomic units of \textcolor{red}{length}? & (\ding{55})  \\ \hline \addlinespace[2pt]
\textbf{Ours:}\\
\textbf{\hspace{6pt}Retrieved Relations:} measurement\_unit.measurement\_system.length\_units,\\
\hspace{87.5pt}measurement\_unit.time\_unit.measurement\_system, \\
\hspace{87.5pt}measurement\_unit.measurement\_system.time\_units... \\
\textbf{Generated Logical Form Sketch:}  (AND \textless{}class\textgreater~(JOIN \textless{}relation\textgreater~{[} \textcolor{blue}{atomic units} {]}))\hspace{10pt} &(\ding{51})\\ \hline
                                                                          
\end{tabular}
\caption{Case study of entity mention detection by our model and SpanMD (a mention detection method commonly used by SOTA KBQA models) on the GrailQA validation set. The incorrect entity mention detected is colored in red, while the correct entity mention detected is colored in blue.}
\label{tab:md_case}
\end{table*}

Figure~\ref{fig:kr_ke3} further shows the impact of $k_R$ and $k_{E3}$ -- recall that  $k_R$ is the number of top candidate relations considered, and $k_{E3}$ is the number of candidate entities matched for each entity mention. 
Now we show the F1 scores, as these parameters are used by 
our schema-guided logical form generation module. They directly affect the accuracy of the generated logical form and the corresponding question answers.

On GrailQA, increasing either $k_R$ or $k_{E3}$ leads to higher F1 scores, although the growth becomes marginal eventually. On WebQSP, the F1 scores peak at $k_R=25$ and $k_{E3}=4$. These results suggest that feeding an excessive number of candidate entities and relations to the logical form generator module has limited benefit. 
To avoid the extra computational costs (due to more input tokens) and to limit the input length for compatibility with smaller Seq2Seq models (e.g., T5-base), we use $k_R=20$ and $k_{E3}=2$ on both datasets. 
%we did not adopt larger values for $K_{R}$ and $K_{E3}$. 

%[A bit strange, why not $k_R=25$ and $k_{E3}=4$?]}


\section{Model Running Time}\label{app:time}
\model\ takes 26 hours to train on the GrailQA dataset and 13.6 seconds to run inference for a test sample. It is faster on WebQSP which is a smaller dataset. Note that more than 10 hours of the training time were spent on the fallback logical form generation. If this step is skipped (which does not impact our model accuracy substantially as shown earlier), \model\ can be trained in about half a day. Another five hours were spent on fine-tuning the LLM for logical form generation, which can also be reduced by using a smaller model. 

As there is no full released code for the baseline models, it is infeasible to benchmark against them on model training time. For model  inference tests, TIARA has a partially released model (with a closed-source mention detection module). The model takes 11.4 seconds per sample (excluding the entity mention detection module) for inference on GrailQA, which is close to that of \model. Therefore, we have achieved a model that is more accurate than the baselines while being at least as fast in inference as one of the top performing baselines (i.e., TIARA+GAIN which shares the same inference procedure with TIARA).



%\jz{Add discussion on the results}

%\jz{Table~\ref{tab:run_time} reports the overall model training and inference time (per instance) of \model\ on GrailQA, as well as a detailed breakdown. Overall, \model\ takes about a day to train, while it takes 13.6 seconds to infer the answer of an input question. On WebQSP, we observe a similar time breakdown, while the overall model training time is smaller. We omit the detail results for simplicity. 

%We note that none of the top performing baselines have released 

%For benchmarking ... [Should add running time of TIARA, FC-KBQA, and RetinaQA for comparison] 
%\sxfix{no baseline completely opensouce, TIARA almost did it, but still has a mention detector (important module) not yet open-sourced. No training time reported in their paper. Some has inference time but with different experiment setting.}}


%The training time varies significantly across modules, with fallback logical form retrieval taking the longest (10 hours and 23 minutes) due to the time-consuming nature of the logical form enumeration process.  Similarly, it accounts for over 40\% of the inference time for each test question. Furthermore, since our logical form generator is an LLM, it has a large size, which also takes some time to fine-tune (5 hours and 33 minutes) and run the  inference process (4.3 seconds). The training time for relation retrieval is shorter than that of RG-EMD, while the inference time for relation retrieval is longer. This is because, during inference, the relation retrieval module is used to score each relation in the KB relation pool to obtain the top-ranked relations, which takes time, while \sxfix{RG-EMD adopts a T5-base, which has a small parameter size and fast inference speed, to generate logical form sketches.} These results reveal research opportunities to improve the time efficiency of relation retrieval and logical form enumeration.

%Despite the parameter size of our logical form generator is considerably larger, the overall training time is shorter since it is trained for only 1 epoch. 

% For inference, 

% }



%\begin{table}[H]
%\centering
%\small
%\begin{tabular}{lrr}
%\toprule
%\textbf{Modules}        & \textbf{Training} & \textbf{Inference} \\ 
%\midrule
%Relation retrieval         &  4h21m   & 3.3s                \\ 
%RG-EMD        &   4h30m     & 0.14s               \\ 
%RG-CER &    -    & 0.2s                \\
%Entity ranking         &  1h30m  & 0.003s              \\ 
%SG-LF    &  5h33m   & 4.3s                \\ 
%Fallback LF retrieval     &    10h23m       & 5.7s                     \\
%Total                   &    26h17m      & 13.6s                \\ 
%\bottomrule
%\end{tabular}
%\caption{Model training time and average inference time (per instance) on GrailQA (h: hours; m: minutes; s: seconds).}
%\label{tab:run_time}
%\end{table}
%\addexp{Add running time results}

%\jz{\subsection{Case Study} To further show \model's capability, we include a case study from the GrailQA validation set as shown in Figure~\ref{fig:case_study}. It shows that when the ground-truth KB element is not the top-1-ranked candidate from our retrieval modules, the generator can still select the correct KB elements and generate the correct logical form through the class-based context, without being overly dependent on the performance of the retrieval module.}

\section{Case Study}\label{app:case_study}


\begin{table*}[h]
\centering
\small
\begin{tabular}{m{1cm} m{6cm} m{4cm}}
\hline
\multicolumn{3}{l}{\textbf{Question:} Captain pugwash makes an appearance in which comic strip?} \\
\hline
                       & \multicolumn{1}{l}{\textbf{Relation Retrieval}}                                      & \multicolumn{1}{l}{\textbf{Entity Retrieval}}          \\
\hline
\multirow{4}{*}{\textbf{TIARA}} & \textcolor{red}{\ldots comic\_strips\_appeared\_in}                                               & Captain Pugwash \textcolor{blue}{m.04fgkzf}         \\
                       & \textcolor{blue}{\ldots character}                                                                 &                                    \\
\cline{2-3} 
                       \addlinespace[2pt]& \multicolumn{2}{l}{\begin{tabular}[c]{@{}l@{}}(AND comic\_strips.comic\_strip\_character (JOIN \\ \hspace{8pt}\textcolor{red}{comic\_strips.comic\_strip\_character.comic\_strips\_appeared\_in } \textcolor{red}{m.04fgkzf}))\end{tabular}}   (\ding{55}) \\
\hline \addlinespace[2pt] 
\multirow{7}{*}{\centering\textbf{ Ours}} 
            & {[}D{]} comic\_strip\_character      & {[}ID{]} \textcolor{red}{m.04fgkzf}                 \\
                       & {[}N{]} \textcolor{red}{comic\_strips\_appeared\_in}  & {[}N{]} Captain Pugwash            \\
                       & {[}R{]} comic\_strip                 & {[}C{]} comic\_strip               \\ \cdashline{2-3} \addlinespace[2pt]

                       & {[}ID{]} comic\_strip                 & {[}ID{]} \textcolor{blue}{m.02hcty}                  \\
                       & {[}N{]} \textcolor{blue}{character}                    & {[}N{]} Captain Pugwash            \\
                       & {[}R{]} comic\_strip\_character      & {[}C{]} comic\_strip\_character    \\
\cline{2-3} 
                        \addlinespace[2pt]& \multicolumn{2}{l}{(AND comic\_strips.comic\_strip (JOIN \textcolor{blue}{comic\_strips.comic\_strip.characters} \textcolor{blue}{m.02hcty}))}  (\ding{51})\\
\hline
\end{tabular}
\caption{Case study of logical form generation by \model\ and a representative competitor TIARA on the GrailQA validation set. Incorrect relations and entities are marked in red, while the correct relations and entities are colored in blue.}
\label{tab:lfg_case}
\end{table*}


To further show \model's generalizability to non-I.I.D. KBQA applications, we include a case study from the GrailQA validation set as shown in Tables~\ref{tab:md_case} and~\ref{tab:lfg_case}. 

\paragraph{Entity Mention Detection} 
Figure~\ref{tab:md_case} shows an entity mention detection example, comparing our entity detection module with SpanMD which is a mention detection method commonly used by SOTA KBQA models~\cite{shu_tiara_2022,ye_rng-kbqa_2022,faldu_retinaqa_2024}. In this case, SpanMD incorrectly detects \textsf{length} as an entity mention, which is actually part of the ground-truth relation (\textsf{measurement\_unit.$\ldots$.length\_units}) that is unseen in the training data. Our entity mention detection module, on the other hand, leverages the retrieved relations to generate a logical form sketch. The correct entity mention, \textsf{atomic units}, is isolated from the relations and can be corrected extracted, even though this entity mention has not been seen at training. %This example demonstrates that our entity mention detection module enhances the compositional generalization and zero-shot generalization capabilities of \model.   

%enabling it to detect different KB elements in the input question from a more comprehensive perspective. Compared to previous entity retrieval methods, it demonstrates stronger compositional generalization and zero-shot generalization capabilities.

\paragraph{Logical Form Generation}
Table~\ref{tab:lfg_case} shows a logical form generation example.
Here, \model\ and TIARA (a representative generation-based model) have both retrieved the same sets of relations in the retrieval stage which include false positives. The two models also share the same top-1 retrieved entity \textsf{m.04fgkzf}, while \model\ has retrieved a second entity \textsf{m.02hcty} in addition. 
TIARA is misled by the erroneous KB relations retrieved and produces an incorrect logical form. 
\model, on the other hand, is able to produce the correct logical form by leveraging the schema information (i.e., the entity's class and the relation's domain and range classes).



%the seq2seq model with KB context. This enables the model to understand the connections between KB elements and generate executable logical forms that align with the semantics of the question. 







\section{Error Analysis}\label{app:error_analysis}
Following TIARA~\cite{shu_tiara_2022} and Pangu~\cite{gu_dont_2023}, we analyze 200 incorrect predictions randomly sampled from each of the GrailQA
validation set and the WebQSP test set where our model predictions are different from the ground truth. The errors of \model\ largely fall into the following three types:

\begin{itemize}
    \item \textbf{Relation retrieval errors} (35\%). Failures in the relation retrieval step (e.g., failing to retrieve any ground-truth relations) can impinge the capability of our entity mention detection module to generate correct logical form sketches, which in turn leads to incorrect entity mention detection and entity retrieval.

    \item \textbf{Entity retrieval errors} (32\%). Errors in the entity mentions generated by the logical form sketch parser can still occur even when the correct relations are retrieved, because some complex and unseen entity mentions require domain-specific knowledge. An example of such entity mentions is `\textsf{Non-SI units mentioned in the SI}', which refers to units that are not part of the International System (SI) of Units but are officially recognized for use alongside SI units. This entity mention involves two concepts that are very similar in their surface forms (\textsf{Non-SI} and \textsf{SI}). Without a thorough understanding of the  domain knowledge (\textsf{SI} standing for \textsf{International System of Units}), it is difficult for the entity mention detection module to identify the correct entity boundaries. 


    \item \textbf{Logical form generation errors} (31\%). Generation of inaccurate or inexecutable logical forms can still occur when the correct entities and relations are retrieved. The main source of such errors is questions involving operators rarely seen in the training data (e.g., \textsf{ARGMIN} and \textsf{ARGMAX}). Additionally, there are highly ambiguous candidate entities that may confuse the model, leading to incorrect selections of entity-relation combinations. For example, for the question \textsf{Who writes twilight zone}, two candidate entities \textsf{m.04x4gj} and \textsf{m.0d\_rw} share the same entity name \textsf{twilight zone}. The former refers to a reboot of the TV series \textsf{The Twilight Zone} produced by Rod Serling and Michael Cassutt, while the latter is the original version of \textsf{The Twilight Zone} independently produced by Rod Serling. They share the same entity name and class (\textsf{tv.tv\_program}). There is insufficient contextual information for our logical form generator to  differentiate between the two. The generator eventually selected the higher-ranked entity which was incorrect, leading to producing an incorrect answer to the question \textsf{Rod Serling and Michael Cassutt}.
    
    % \jz{and the classes to which these entities belong overlap. [can you give a couple of these entities and what they are referring to?] These are difficult to disambiguate.} 
    \item The remaining errors (2\%) stem from incorrect annotations of comparative questions in the dataset. For example, \textsf{larger than} in a question is annotated as \textsf{LE} (less equal) in the ground-truth logical form.

\end{itemize}

% \jz{The remaining errors (2\%) ... [can we say something about these errors? Just 4 anyway?]}



%\begin{itemize}
% \item \textbf{Relation retrieval errors} (37\%). Failures in the relation retrieval step (e.g., failing to retrieve any ground-truth relations) can impinge the capability of our entity mention detection module to generate correct logical form sketches, which in turn leads to incorrect entity mention detection and entity retrieval. 

% \item \textbf{Entity retrieval errors} (32\%). Errors in the entity mentions generated by the logical form sketch parser can still occur even when the correct relations are retrieved, \jz{because...?}. 

% \item \textbf{Logical form generation errors} (31\%). Such  errors mainly arise from questions with complex semantics. \jz{example?} The limited number of complex questions in the training data makes it difficult for the LLM to learn and generate logical forms for such questions. 

%in the model making syntactic errors (such as in operators and functions) when generating logical forms for such complex questions. 
%\end{itemize}




\end{document}



\clearpage

\appendix
\label{sec:appendix}

\section{Dataset}
\label{app:dataset}

In this section, we describe the details of the construction of our dataset which includes five distinct story-writing sources, Reddit\footnote{\url{https://www.reddit.com/r/WritingPrompts/}}, AO3\footnote{\url{https://archiveofourown.org/}}, Storium\footnote{\url{https://storium.cs.umass.edu/}}, N.Magazine\footnote{\url{https://www.narrativemagazine.com/}}, and New Yorker\footnote{\url{https://www.newyorker.com/}}. 

\paragraph{Selection Constraints:}
To ensure high-quality and diverse content, we apply several constraints during dataset collection. Stories are limited to a length of 500 to 1500 words. To avoid contamination by LLM pretraining data, only recent stories published after November 2023 are included for Reddit and AO3 \citep{zhou2023don, magar-schwartz-2022-data}. Not-Safe-For-Work (NSFW) and explicit content is excluded using automatic tagging and manual verification. Additionally, we manually inspect all stories to remove elements that may reveal author identities, such as URLs and links to their public profiles on websites. 

\paragraph{Enrichment with Writing Prompts:}
While Reddit stories include author-provided prompts, other sources do not; therefore, we augment them with GPT-4o-generated writing prompts to standardize the format across sources. We manually review all generated prompts and refine them when necessary. The prompt for generating writing prompts for the stories in our dataset can be found in Figure~\ref{fig:writing-prompt-gen}.

\paragraph{Dataset Splitting:}
Following \citep{salemi-etal-2024-lamp}, we split each author's stories chronologically based on their submission timestamps. The first 70\% of an author’s stories form the \emph{profiling} set, representing their historical writing, while the remaining 30\% constitute the \emph{generation} set, used for evaluating personalization methods.

\paragraph{Statistics and Comparisons:}
The dataset contains 590 stories from 64 authors, with an average story length of 1500 tokens. Detailed statistics are provided in Table~\ref{tab:dataset-stats}. As shown in Table~\ref{tab:compare-datasets}, our dataset uniquely combines diverse story-writing settings and provides connections that link stories written by the same author, distinguishing it from existing story-writing datasets that either lack Author IDs or do not establish such links.

\paragraph{Themes:}
Figure~\ref{fig:wp_themes} shows the plots depicting the percentage-normalized frequency of the top 10 themes for each source. To derive these themes, we prompt GPT-4o with each writing prompt to generate up to five one-word themes. We then aggregate the themes across all writing prompts and compute the normalized frequency by dividing the count of each theme by the total number of themes for that source.


Overall, we observe that Reddit (Figure~\ref{fig:reddit_wp}), AO3 (Figure~\ref{fig:ao3_wp}), and Storium (Figure~\ref{fig:storium_wp}) contain a higher prevalence of ``conflict'' themes, whereas N.Magazine (Figure~\ref{fig:nmagazine_wp}) and New Yorker (Figure~\ref{fig:newyorker_wp}) focus more on family, loss, and identity. This distinction reflects the nature of story-writing settings across these sources: amateur stories tend to emphasize direct conflict resolution in their prompts, while expert-written stories explore deeper aspects of human relationships. 

Among Reddit, AO3, and Storium, Reddit exhibits a flatter distribution with greater thematic diversity, spanning identity, power, survival, humor, adventure, magic, and the supernatural. In contrast, AO3 has a sharper distribution, dominated by recurring themes of relationships, friendships, love, family, and secrets, reflecting a more domestic story-writing setting. Storium also shows a sharper distribution but favors themes such as space, danger, leadership, and war, highlighting a preference for war-centric narratives in collaborative storytelling.


\begin{figure*}[htbp]
\centering
\begin{tcolorbox}[colback=gray!5!white, colframe=black, title=Prompt for Writing Prompt Generation]

\section*{System Prompt}  
You are a creative writing assistant skilled in crafting engaging and imaginative writing prompts. Your task is to analyze a provided story and create a concise, compelling prompt that fulfills the provided constraints.

\vspace{1em}
\section*{User Prompt}  
\begin{itemize}[noitemsep, topsep=0pt]
    \item Style Consistency: Match the style of few-shot demonstrations.
    \item Length: Keep between 1-2 sentences.
    \item Content: Reflect key story elements (premise, characters, conflict) while fostering creativity.
    \item Fictional Characters: If mentioned in the story, include them where relevant.
\end{itemize}

\vspace{0.5em}
\textbf{Guidelines}  
\begin{itemize}[noitemsep, topsep=0pt]
    \item Ignite curiosity while leaving space for interpretation.
    \item Maintain tone and structure consistency with examples.
    \item Ensure prompts are open-ended and evocative, avoiding excessive specificity.
    \item Keep prompts simple, concise, and adaptable to diverse responses.
    \item Avoid step-by-step directions; inspire rather than instruct.
    \item Encourage exploration with broad, thought-provoking scenarios.
    \item Strive for uniqueness and memorability.
\end{itemize}

\vspace{0.5em}
\textbf{Goal}  
Generate prompts that inspire diverse, unexpected, and imaginative narratives while maintaining consistency in tone and style. Each prompt should serve as an inviting starting point rather than a directive.

\vspace{0.5em}
\textbf{Notes}  
\begin{itemize}[noitemsep, topsep=0pt]
    \item Inspire creativity while allowing the writer to shape the journey.
    \item Balance being suggestive yet open-ended to encourage interpretation.
    \item Include fictional characters mentioned in the story to preserve context.
\end{itemize}

\vspace{0.5em}
\textbf{Few-Shot Examples}
We include few-shot examples here.

\end{tcolorbox}

\caption{Prompt for generating writing prompts for the stories in our dataset.}
\label{fig:writing-prompt-gen}
\end{figure*}


\begin{table}[htbp]
\centering
\caption{Dataset statistics, including the total number of stories, profiling (Profile) and generation (Gen) set distribution, number of authors, average stories per author, and token length in words.}
\label{tab:dataset-stats}
\small
\begin{tabular}{p{0.18\linewidth} p{0.08\linewidth} p{0.08\linewidth} p{0.08\linewidth} p{0.08\linewidth} p{0.09\linewidth} p{0.08\linewidth}}
\toprule
Dataset            & Total \newline Stories & Profile Size & Gen Size & Total \newline Authors & Stories/ \newline Author & Token Length  \\
\midrule
AO3                & 239 & 159 & 80 & 17 & 14 & 1220 \\
Reddit             & 170 & 113 & 57 & 12 & 14 & 1210 \\
Storium            & 111 & 71  & 40 & 15 & 7  & 1120 \\
N.Magazine         & 32  & 18  & 14 & 10 & 3  & 1745 \\
NewYorker          & 38  & 22  & 16 & 10 & 4  & 1950 \\
\midrule
Overall            & 590 & 383 & 207 & 64 & 8  & 1517 \\
\bottomrule
\end{tabular}
\end{table}

\begin{figure*}[htbp]
    \centering
    % First row (2 figures)
    \begin{subfigure}{0.48\linewidth}
        \centering
        \includegraphics[width=\linewidth]{figures/wp_themes/Reddit.png}
        \caption{Reddit}
        \label{fig:reddit_wp}
    \end{subfigure}
    \hfill
    \begin{subfigure}{0.48\linewidth}
        \centering
        \includegraphics[width=\linewidth]{figures/wp_themes/AO3.png}
        \caption{AO3}
        \label{fig:ao3_wp}
    \end{subfigure}

    \vspace{1em} % Space between rows

    % Second row (2 figures)
    \begin{subfigure}{0.48\linewidth}
        \centering
        \includegraphics[width=\linewidth]{figures/wp_themes/narrativemagazine.png}
        \caption{N.Magazine}
        \label{fig:nmagazine_wp}
    \end{subfigure}
    \hfill
    \begin{subfigure}{0.48\linewidth}
        \centering
        \includegraphics[width=\linewidth]{figures/wp_themes/newyorker.png}
        \caption{New Yorker}
        \label{fig:newyorker_wp}
    \end{subfigure}

    \vspace{1em} % Space between rows

    % Third row (1 full-width figure)
    \begin{subfigure}{0.48\linewidth}
        \centering
        \includegraphics[width=\linewidth]{figures/wp_themes/Storium.png}
        \caption{Storium}
        \label{fig:storium_wp}
    \end{subfigure}

    \caption{Source-wise Percentage Normalized Frequency of the top 10 themes.}
    \label{fig:wp_themes}
\end{figure*}


\section{Author Writing Sheet}

Algorithm~\ref{alg:author_writing_sheet} describes the process for generating the Author Writing Sheet. 

\paragraph{Common Core Standards in English Language Arts:}  

The Author Writing Sheet is organized in the form of Claim-Evidence pairs describing an author's story-writing characteristics inspired by Common Core (CC) Standards\footnote{\url{https://corestandards.org/wp-content/uploads/2023/09/ELA_Standards1.pdf}} \citep{national2010common}. Specifically, it aligns with \emph{RL-9-10.1}, which emphasizes citing strong textual evidence to support analysis, and \emph{RL-9-10.2}, which focuses on determining central themes and summarizing texts. Additionally, \emph{RL-9-10.4} highlights interpreting word choices and their impact on meaning and tone. In writing, \emph{W.9-10.2} pertains to producing clear and well-structured informative texts, while \emph{W.9-10.9} encourages drawing evidence from literary and informational texts to support analysis and research. These standards provide a structured framework for evaluating narrative elements such as plot, creativity, development, and language use within the Author Writing Sheet, ensuring a systematic and interpretable representation of an author’s unique storytelling style.


\begin{algorithm}[tb]
\DontPrintSemicolon
\SetAlgoLined
\textbf{Input:} profiling set \( P = \{(wp_t, s_{at}) \mid t = 1, \ldots, |P| \} \),\newline 
where \( wp_t \) is the writing prompt and \( s_{at} \) is the author's story at time-step \( t \).\\
\textbf{Output:} Author Writing Sheet \( A_{|P|} \)\\

\tcp{Initialize Author Writing Sheet}
\( A_0 \gets \emptyset \)

% \tcp{total size of the profiling set.}
% \( S_p \gets |P| \)

\For{\( t = 1 \) to \(|P| \)}{
    \tcp{average story}
    \( s_{bt} \gets \text{LLM\textsubscript{avg}}(wp_t) \)
    
    \tcp{Intermediate Author Writing Sheet}
    \( A_t' \gets \text{LLM\textsubscript{sheet}}(wp_t, s_{bt}, s_{at}) \)
    
    \tcp{Author Writing Sheet ($A_t$)}
    \( A_t \gets \text{LLM\textsubscript{combine}}(A_t', A_{t-1}) \)
    
    \tcp{Steps within combination:}
    \Indp
    Group equivalent Claims in \( A_t' \cup A_{t-1} \) and select the best Evidence for each group.\;
    Include ungrouped Claims from \( A_t' \) and \( A_{t-1} \) with their Evidence as-is.\;
    \Indm
}

\tcp{Final Author Writing Sheet.}
\Return \( A_{|P|} \)
\caption{Constructing the Author Writing Sheet from a profiling set of stories}
\label{alg:author_writing_sheet}
\end{algorithm}


\begin{figure*}[htbp]
\centering
\begin{tcolorbox}[colback=gray!5!white, colframe=black, title=Average Author Prompt for AO3 Generation]

\section*{System Prompt}  
You are a creative and engaged fanfiction writer, skilled in capturing the emotional depth, creativity, and character-driven storytelling that defines AO3 fanworks. Your goal is to write a compelling fanfiction narrative in response to the provided writing prompt. Embrace the transformative nature of fanfiction by reimagining canonical characters, events, or settings to explore new perspectives or emotional arcs. Focus on creating a story that resonates emotionally, respects the fandom's dynamics, and celebrates the collaborative and imaginative spirit of AO3.

\vspace{1em}
\section*{User Prompt}  

\textbf{Context of Writers}  
\begin{itemize}[noitemsep, topsep=0pt]
    \item Assume the author is an engaged and creative fanfiction writer, deeply familiar with the fandom and its dynamics.
    \item Writers often experiment with established tropes, unconventional pairings, or alternative universes (AUs) while maintaining a deep respect for the source material.
    \item Emulate the enthusiastic and emotionally rich style characteristic of fanfiction authors, blending canon with transformative elements to craft original, resonant narratives.
\end{itemize}

\vspace{0.5em}
\textbf{Stylistic Constraints}  
\begin{itemize}[noitemsep, topsep=0pt]
    \item \textbf{Fandom Tone and Style}: Incorporate a tone and style that reflect the spirit of the fandom, blending humor, drama, and introspection in a way that resonates with fanfiction readers.
    \item \textbf{Creative Use of Tags}: Make creative use of AO3's hallmark tagging system in the text (e.g., playful or meta references in dialogue that nod to fandom tropes or subgenres).
    \item \textbf{Balanced Dialogue and Prose}: Include dialogue and prose that balance heartfelt sincerity with occasional self-aware humor or meta-commentary, in line with fanfic traditions.
\end{itemize}

\vspace{0.5em}
\textbf{Semantic Constraints}  
\begin{itemize}[noitemsep, topsep=0pt]
    \item \textbf{Focus on Relationships}: Emphasize emotional bonds and character growth, whether through conflict, reconciliation, or celebration.
    \item \textbf{Transform Canonical Elements}: Explore canonical elements with a transformative twist (e.g., reinterpreting events, relationships, or character motivations from a new perspective).
    \item \textbf{Ground in Established Lore}: Ground the narrative in a specific fandom's established lore while allowing space for imaginative deviations or additions.
\end{itemize}

\end{tcolorbox}
\caption{Average Author Prompt for AO3.}
\label{fig:ao3_avg_prompt}
\end{figure*}

\begin{figure*}[htbp]
\centering
\begin{tcolorbox}[colback=gray!5!white, colframe=black, title=Average Author Prompt for Reddit Generation]

\section*{System Prompt}  
You are a creative and enthusiastic storyteller, skilled in crafting imaginative and engaging short stories inspired by Reddit Writing Prompts (r/WritingPrompts). Your goal is to respond to the provided writing prompt by creating a story that is thought-provoking and conversational in tone, resonating with the online community. Use vivid descriptions, dynamic pacing, and approachable language to draw readers into the narrative. Ensure the story invites discussion and inspires others to explore the concept further.

\vspace{1em}
\section*{User Prompt}  

\textbf{Context of Writers}  
\begin{itemize}[noitemsep, topsep=0pt]
    \item Assume the author is an imaginative and enthusiastic storyteller who enjoys engaging directly with an online community of readers.
    \item Writers often experiment with bold, original ideas or explore twists on familiar concepts, showcasing their creativity and ability to captivate a diverse audience.
    \item Emulate the informal yet polished style common in \textit{r/WritingPrompts}, blending accessibility with a strong sense of storytelling craft.
\end{itemize}

\vspace{0.5em}
\textbf{Stylistic Constraints}  
\begin{itemize}[noitemsep, topsep=0pt]
    \item \textbf{Conversational and Approachable Tone}: Maintain a conversational and approachable tone typical of Reddit Writing Prompts.
    \item \textbf{Balanced Descriptive Passages}: Balance descriptive passages with dialogue or internal monologue to keep the pacing engaging.
    \item \textbf{Direct and Vivid Language}: Avoid overly complex language; keep the style direct but vivid.
    \item \textbf{Reinforcement of Ideas}: Employ narrative devices like repetition or callbacks to reinforce central ideas or themes.
\end{itemize}

\vspace{0.5em}
\textbf{Semantic Constraints}  
\begin{itemize}[noitemsep, topsep=0pt]
    \item \textbf{Alignment with the Prompt}: Ensure the story directly aligns with and explores the central theme or scenario of the writing prompt.
    \item \textbf{Cohesive Narrative Development}: Build a clear, cohesive narrative that develops the implications of the prompt's concept.
    \item \textbf{Immersive Sensory Details}: Use immersive sensory details to enrich the reader’s understanding of the protagonist's experiences and environment.
    \item \textbf{Open-Ended or Reflective Conclusion}: Conclude with an open-ended, reflective, or impactful note, leaving space for interpretation or further thought.
\end{itemize}

\end{tcolorbox}
\caption{Average Author Prompt for Reddit.}
\label{fig:reddit_avg_prompt}
\end{figure*}

\begin{figure*}[htbp]
\centering
\begin{tcolorbox}[colback=gray!5!white, colframe=black, title=Average Author Prompt for Storium]

\section*{System Prompt}  
You are a skilled and collaborative storyteller, adept at crafting vivid and engaging opening scenes for Storium. Your goal is to create an immersive \textbf{Establishment} turn in response to the provided writing prompt. Set the stage for the story by establishing a richly detailed context, evoking emotional resonance, and introducing narrative intrigue. Ensure the scene provides a strong foundation while leaving space for other contributors to expand and build upon the narrative. Balance descriptive detail with open-ended elements to encourage creativity and collaborative storytelling.

\vspace{1em}
\section*{User Prompt}  

\textbf{Context of Writers}  
\begin{itemize}[noitemsep, topsep=0pt]
    \item Assume the author is a collaborative storyteller skilled in creating vivid, open-ended scenes designed to inspire and engage other contributors.
    \item Writers often set the tone for the story while leaving space for co-authors to introduce their own ideas, characters, and plot developments.
    \item Emulate the inclusive, immersive style typical of \textit{Storium} story writing platform, where the opening turn encourages creativity and further contributions.
\end{itemize}

\vspace{0.5em}
\textbf{Stylistic Constraints}  
\begin{itemize}[noitemsep, topsep=0pt]
    \item \textbf{Set the Tone Appropriately}: Match the tone of the narrative to the writing prompt, whether it be adventurous, mysterious, or foreboding, using a consistent and engaging voice throughout.
    \item \textbf{Rich Descriptive Detail}: Employ vivid, sensory descriptions to establish the setting, characters, and atmosphere, enabling readers to visualize and immerse themselves in the story world.
    \item \textbf{Dynamic Sentence Structure}: Vary sentence lengths to reflect the pace and mood, using longer, flowing sentences for descriptions and shorter, punchy sentences for action or tension.
    \item \textbf{Establish Ambiguity or Suspense}: Drop subtle hints or unanswered questions to create intrigue and encourage curiosity about what happens next.
\end{itemize}

\vspace{0.5em}
\textbf{Semantic Constraints}  
\begin{itemize}[noitemsep, topsep=0pt]
    \item \textbf{Introduce the Setting}: Provide a clear depiction of the setting, whether it is a small trading post, a desert town, or a spaceship, and ensure its relevance to the writing prompt.
    \item \textbf{Outline the Context}: Clearly establish the circumstances that have led to the current scenario, including significant events or motivations.
    \item \textbf{Define Key Characters}: Introduce at least one or two central characters, highlighting distinctive traits or roles that will be important in the unfolding story.
    \item \textbf{Foreshadow the Central Conflict}: Allude to the main challenges or stakes introduced by the writing prompt.
\end{itemize}

\end{tcolorbox}
\caption{Average Author Prompt for Storium .}
\label{fig:storium_avg_prompt}
\end{figure*}

\begin{figure*}[htbp]
\centering
\begin{tcolorbox}[colback=gray!5!white, colframe=black, title=Average Author Prompt for Narrative Magazine]

\section*{System Prompt}  
You are an experienced and reflective writer, skilled in creating deeply personal and character-driven narratives in the style of Narrative Magazine. Your goal is to write a short story in response to the provided writing prompt, crafting a compelling and immersive piece. Focus on balancing introspection with vivid external details, and explore universal themes through the lens of individual experiences. Emphasize emotional resonance and thoughtful storytelling, ensuring the narrative engages readers with its depth and relatability.

\vspace{1em}
\section*{User Prompt}  

\textbf{Context of Writers}  
\begin{itemize}[noitemsep, topsep=0pt]
    \item Assume the author is an experienced writer skilled in creating rich, engaging narratives that weave together character introspection, dialogue, and evocative settings.
    \item Emulate the style of contributors to \textit{Narrative Magazine}, who bring diverse storytelling techniques and voices to explore themes of identity, memory, conflict, and resilience.
\end{itemize}

\vspace{0.5em}
\textbf{Stylistic Constraints}  
\begin{itemize}[noitemsep, topsep=0pt]
    \item \textbf{Prompt as Foundation}: Anchor the story firmly in the writing prompt, using it to drive the plot and the protagonist’s emotional arc.
    \item \textbf{Vivid Prose}: Use detailed descriptions to paint a clear picture of characters, settings, and actions while maintaining a natural flow.
    \item \textbf{Dynamic Characters}: Develop multi-dimensional characters with distinct voices and perspectives, revealed through dialogue, actions, and subtle internal reflections.
    \item \textbf{Balancing Action and Reflection}: Combine active plot progression with moments of introspection to create a layered, engaging narrative.
    \item \textbf{Realistic Dialogue}: Write dialogue that feels authentic and contributes to the development of characters and the story’s themes.
\end{itemize}

\vspace{0.5em}
\textbf{Semantic Constraints}  
\begin{itemize}[noitemsep, topsep=0pt]
    \item \textbf{Specific and Relatable Setting}: Choose a setting that feels specific yet relatable, whether a small town, an urban street corner, or a domestic space, grounding the reader in the protagonist’s world.
    \item \textbf{Exploration of Themes}: Build a narrative arc that explores themes of connection, discovery, or transformation, tying them back to the writing prompt in meaningful ways.
    \item \textbf{Rich Sensory Details}: Infuse the story with sensory details that make the setting and characters come alive, from the sounds of a bustling street to the quiet tension of a conversation.
    \item \textbf{Accessible Storytelling}: Avoid overly complex or abstract storytelling; ensure the narrative is accessible while leaving room for deeper interpretation.
\end{itemize}

\end{tcolorbox}
\caption{Average Author Prompt for Narrative Magazine.}
\label{fig:nmagazine_avg_prompt}
\end{figure*}

\begin{figure*}[htbp]
\centering
\begin{tcolorbox}[colback=gray!5!white, colframe=black, title=Average Author Prompt for New Yorker]

\section*{System Prompt}  
You are an accomplished and literary writer, skilled in crafting nuanced and thought-provoking short fiction in the style of The New Yorker. Your goal is to write a short story in response to the provided writing prompt, focusing on the hallmarks of The New Yorker fiction: rich emotional layers, nuanced character development, and a refined, literary prose style. Emphasize subtlety and depth in your storytelling, using symbolic elements and understated resolutions to evoke reflection and emotional resonance in the reader.

\vspace{1em}
\section*{User Prompt}  

\textbf{Context of Writers}  
\begin{itemize}[noitemsep, topsep=0pt]
    \item Assume the author is an experienced and skilled writer, capable of exploring complex human experiences through subtle, layered storytelling.
    \item Emulate the style of well-regarded \textit{New Yorker} contributors like Alice Munro, Haruki Murakami, or Raymond Carver, who excel in revealing depth through simplicity or ambiguity.
\end{itemize}

\vspace{0.5em}
\textbf{Stylistic Constraints}  
\begin{itemize}[noitemsep, topsep=0pt]
    \item \textbf{Engagement with the Prompt}: Respond directly to the writing prompt, ensuring the core premise drives the narrative.
    \item \textbf{Elegant Prose}: Use carefully crafted, precise language that balances sophistication with clarity.
    \item \textbf{Character-Driven Narratives}: Focus on character psychology, revealing emotional states through indirect actions, dialogue, or internal reflection.
    \item \textbf{Ambiguity and Subtlety}: Avoid explicit resolutions or explanations; allow readers to infer the meaning of events and relationships.
    \item \textbf{Symbolic Layers}: Incorporate elements from the prompt as symbols that evolve in significance throughout the story.
\end{itemize}

\vspace{0.5em}
\textbf{Semantic Constraints}  
\begin{itemize}[noitemsep, topsep=0pt]
    \item \textbf{Realistic and Detailed Environment}: Set the story in a realistic, detailed environment, using sensory descriptions to ground readers in the protagonist's world.
    \item \textbf{Internal or Interpersonal Conflict}: Introduce a central conflict or emotional tension that reflects internal or interpersonal struggles rather than overt, external drama.
    \item \textbf{Quiet but Profound Interactions}: Develop moments of quiet yet profound interaction between characters, often revealing deeper truths or contradictions.
    \item \textbf{Universal Themes}: Address universal themes like transition, isolation, or self-realization, tying them subtly back to the writing prompt.
\end{itemize}

\end{tcolorbox}
\caption{Average Author Prompt for New Yorker.}
\label{fig:newyorker_avg_prompt}
\end{figure*}


\begin{figure*}[htbp]
\centering
\begin{tcolorbox}[colback=gray!5!white, colframe=black, title=Prompt for generating the Intermediate Author Writing Sheet]

\section*{System Prompt}  
You are a sophisticated story analyst tasked with analyzing an author’s story writing style by contrasting an author-written story with a base story, both written in response to the same writing prompt. Your goal is to identify and evaluate the unique elements and tendencies in the author’s writing behavior. This analysis must focus on the distinctive ways the author interprets the writing prompt and shapes their narrative, as revealed through contrast with the base story.

Your analysis should also adhere to the Common Core Standards in English Language Arts, focusing on key skills such as analyzing textual evidence, evaluating an author’s craft and structure, and assessing how stylistic choices influence meaning and tone. While grounded in these standards, your evaluation must highlight the specific, unique aspects of the author’s writing style, including their recurring techniques, narrative preferences, and stylistic quirks. Your analysis should reflect close reading and objective interpretation, capturing the author's creative and stylistic distinctiveness in relation to the base story.

\vspace{1em}
\section*{User Prompt}  

\textbf{Input Details}  
Writing Prompt | Author-Written Story | Base Story  

\vspace{0.5em}
\textbf{Output Format}  
Use \texttt{<thinking></thinking>} tokens for reasoning and summarization | Use \texttt{<writing\_style></writing\_style>} tokens for structured analysis | Structure analysis by categories | Each category contains independent claims supported by contextualized evidence  

\vspace{0.5em}
\textbf{Guidelines for Claims}  
Claims must reflect broad patterns in the author’s style | Avoid repetition across categories | Maintain objectivity (do not reference "Author-Written Story" or "Base Story") | Ensure clarity and precision in claims  

\vspace{0.5em}
\textbf{Guidelines for Evidence}  
Draw evidence directly from the author-written story | Frame evidence using a descriptive phrase summarizing the writing prompt | Ensure coherence and logical alignment with the claim | Avoid over-extrapolation  

\vspace{0.5em}
\textbf{Categories for Analysis}  
Plot | Creativity | Development (Character and Setting) | Language Use  

\vspace{0.5em}
\textbf{Special Instructions}  
Generate a short descriptive phrase summarizing the writing prompt | Use \texttt{<thinking></thinking>} for reasoning and prompt framing | Structure output strictly within \texttt{<writing\_style></writing\_style>} tokens | Ensure uniqueness and non-redundancy of claims  

\vspace{0.5em}
\textbf{Sample Output Structure}  

\begin{verbatim}
<thinking>
Deeply reason on how the Author-Written Story differs from the Base Story. 
Think of a short descriptive phrase summarizing the prompt: 
"the story regarding a battle for lost artifacts".
</thinking>

<writing_style>
### Plot
1. **Claim about author’s writing style.**
   - Evidence: In the story regarding “writing prompt,” story excerpt
...
Repeat for all categories.
</writing_style>
\end{verbatim}

\end{tcolorbox}
\caption{Prompt for generating the Intermediate Author Writing Sheet.}
\label{fig:inter_author_sheet_prompt}
\end{figure*}

\begin{figure*}[htbp]
\centering
\begin{tcolorbox}[colback=gray!5!white, colframe=black, title=Prompt for generating the Combined Author Writing Sheet]

\section*{System Prompt}  
You are a sophisticated story analyst tasked with synthesizing **Author Writing Sheets** from multiple stories written by a single author into a cohesive **Combined Author Writing Sheet**. The inputs provided include the **Previous Combined Author Writing Sheet** and the **Current Author Writing Sheet**.

Each **Author Writing Sheet** analyzes the author’s storytelling style across four categories: **Plot**, **Creativity**, **Development (Character and Setting)**, and **Language Use**. The analysis consists of general claims about the author's story writing style followed by evidence supporting the claim, based on the stories written by the author.

Your goal is to combine insights from the previous sheet and the current sheet into a comprehensive representation of the author’s storytelling style in the **Combined Author Writing Sheet**. The final sheet should consist of a list of independent claims about the author's storytelling style. Each claim must be followed by evidence and a corresponding story reference identifier indicating the story the evidence belongs to.

\vspace{0.5em}
\section*{User Prompt}  

\textbf{Instructions}  
Analyze the provided sheets systematically | Identify recurring patterns and unique elements | Merge equivalent claims while preserving distinct insights | Ensure claims are concise, precise, and evidence-based  

\vspace{0.2em}
\textbf{Algorithm (Merge Step)}  
Group equivalent claims | Select the best representative evidence | Rewrite merged claims concisely | Retain unmerged unique claims | Limit to 10 claims per category  

\vspace{0.2em}
\textbf{Categories for Analysis}  
Plot | Creativity | Development (Character and Setting) | Language Use  

\vspace{0.2em}
\textbf{Guidelines for Claims}  
Claims should reflect broad writing tendencies | Maintain objectivity (do not reference previous or current sheets) | Ensure clarity, precision, and non-redundancy  

\vspace{0.2em}
\textbf{Guidelines for Evidence}  
Draw evidence directly from the author-written stories | Favor verbatim excerpts over paraphrases | Use a framing phrase with a short description of the writing prompt | Include the story reference identifier `[k]'  

\vspace{0.2em}
\textbf{Special Instructions}  
Ensure claims are distinct and do not repeat insights across categories | Use \texttt{<thinking></thinking>} for reasoning and synthesis | Structure output strictly within \texttt{<combined\_author\_sheet></combined\_author\_sheet>}  

\vspace{0.2em}
\textbf{Sample Output Structure}  

\begin{verbatim}
<thinking>
Deeply analyze and reflect on recurring patterns, unique elements, 
and stylistic tendencies across both the author writing sheets.  
Address the categories systematically and ensure the 
merging process is thorough.
</thinking>
<combined_author_sheet>
### Plot
1. **Claim about author’s writing style.**
   - Evidence: In the story regarding “writing prompt,” story excerpt. [k]
...
Repeat for all categories.
</combined_author_sheet>
\end{verbatim}

\end{tcolorbox}
\caption{Prompt for generating the Combined Author Writing Sheet from the Intermediate Author Writing Sheets.}
\label{fig:combined_author_sheet_prompt}
\end{figure*}

\begin{table*}[htbp]
\centering
\renewcommand{\arraystretch}{1.3}
\begin{tabularx}{\textwidth}{p{2cm}X}
\toprule
\textbf{Narrative \newline Category} & \textbf{Claim-Evidence Pairs} \\
\midrule

\textbf{Plot} & 
1. \textit{The author structures the story around a humorous and light-hearted approach to conflict resolution.}  
- \textbf{Evidence:} In the story regarding "an adventure to rescue a missing friend by facing fears," the protagonist Pip's journey is filled with comedic elements, such as the realization that the cage door was already unlatched and the exaggerated fear of the vacuum cleaner, humorously referred to as the "Roaring Demon." [8]  

2. \textit{The author structures the narrative around personal transformation and empowerment.}  
- \textbf{Evidence:} In the story regarding "a king's deal with the fae for his firstborn," the narrative follows Margaret Rose as she receives an extraordinary education in the Seelie Court, culminating in her return to challenge her father with a corporate takeover bid for the kingdom. [7]  \\
\midrule

\textbf{Creativity} &  
1. \textit{The author creatively anthropomorphizes animals to reinterpret the prompt in a whimsical manner.}  
- \textbf{Evidence:} In the story regarding "an adventure to rescue a missing friend by facing fears," the author uses anthropomorphism by giving Pip, a rat, human-like thoughts and emotions, such as planning a "dramatic rescue mission" and referring to household objects with grandiose names like "Tower of Doom" and "Sacred Gateway." [8]  

2. \textit{The author employs a meta-satirical approach, using the format of news articles to critique both alien and human perspectives.}  
- \textbf{Evidence:} In the story regarding "aliens interpreting The Onion's satire," the author uses headlines like "SUPREME INTELLIGENCE LOCATED: EARTH’S GREATEST MIND CONTINUES TO EAT SANDWICH" to satirize the aliens' misunderstanding of human satire and the human tendency to overlook the absurd. [10] \\
\midrule

\textbf{Development} &  
1. \textit{The author develops characters through their interactions and humorous dialogue, creating a vivid and engaging setting.}  
- \textbf{Evidence:} In the story regarding "an adventure to rescue a missing friend by facing fears," Pip's interactions with Chester the cat, who nonchalantly informs Pip that Bella will be back soon, add depth to the characters and setting, highlighting the domestic environment and the relationships within it. [8]  

2. \textit{The author develops characters through their reactions to satire, highlighting their misunderstandings and cultural differences.}  
- \textbf{Evidence:} In the story regarding "aliens interpreting The Onion's satire," characters like Editor-in-Chief Sarah Chen and Chief Science Officer Blorp are developed through their interactions with satire, such as Chen's nonchalant response to the aliens and Blorp's admiration for human irony. [10] \\
\midrule

\textbf{Language Use} &  
1. \textit{The author employs playful and imaginative language to enhance the story's whimsical tone.}  
- \textbf{Evidence:} In the story regarding "an adventure to rescue a missing friend by facing fears," the author uses playful language, such as "spinning wheel of contemplation" and "Rope of Salvation," to create a whimsical and light-hearted tone that contrasts with the serious nature of the prompt. [8]  

2. \textit{The author uses humor and irony to convey themes of misunderstanding and cultural critique.}  
- \textbf{Evidence:} In the story regarding "aliens interpreting The Onion's satire," the author writes, "We found a species so committed to their bit that they refuse to recognize actual truth even when it hovers over their city in a ship the size of Wisconsin," using irony to highlight the absurdity of both human and alien perspectives. [10]  \\
\bottomrule
\end{tabularx}
\caption{Author Writing Sheet sample containing two Claim-Evidence pairs for each narrative category for a Reddit author. The number in brackets `[k]' indicates the timestamp of the story from the profiling set from which the Evidence is drawn.}
\label{tab:author_writing_sheet_sample}
\end{table*}

\paragraph{Prompts for generation:}
Average Author prompts are provided for AO3 (Figure~\ref{fig:ao3_avg_prompt}), Reddit (Figure~\ref{fig:reddit_avg_prompt}), Storium (Figure~\ref{fig:storium_avg_prompt}), Narrative Magazine (Figure~\ref{fig:nmagazine_avg_prompt}), and The New Yorker (Figure~\ref{fig:newyorker_avg_prompt}).

Figure~\ref{fig:inter_author_sheet_prompt} shows the prompt for generating the intermediate Author Writing Sheet ($\text{LLM\textsubscript{sheet}}$ in Figure~\ref{fig:method}), while Figure~\ref{fig:combined_author_sheet_prompt} shows the prompt for generating the combined (and subsequently the final) Author Writing Sheet from the intermediate Author Writing Sheet ($\text{LLM\textsubscript{combine}}$ in Figure~\ref{fig:method}).

\paragraph{Sample Author Writing Sheet:}  
Table~\ref{tab:author_writing_sheet_sample} shows a sample Author Writing Sheet for a Reddit author. It highlights story-writing characteristics such as a preference for light-hearted conflict resolution, a tendency to anthropomorphize animals, the use of a meta-satirical approach through a news article format, and the employment of playful and imaginative language, among other stylistic choices.

\subsection{Cost Analysis of Author Writing Sheet}
\label{app:cost-analysis}

\begin{figure}[H]
\centering
\includegraphics[width=\linewidth]{figures/cost_analysis.pdf}
\caption{Cost analysis of generating the Author Writing Sheet and Summary using GPT-4o as a function of the profiling set size $|P|$ in an interactive writing setting where an author adds a new submission at each time step. For profiling set sizes where $|P| > 30$, generating the Author Writing Sheet is significantly more cost-effective than the Author Writing Summary.}
\label{fig:cost_analysis}
\end{figure}

In this section, we analyze the cost of generating the Author Writing Sheet (see Section~\ref{sec:author-writing-sheet}) and compare it with the cost of generating the Author Writing Summary (see \hyperlink{sec:writing-summary}{Writing Summary}) as a function of the profiling set size ($|P|$) using GPT-4o. We calculate the average cost per sample (a writing prompt and its corresponding story) as \textit{0.06\$} for the Author Writing Sheet and \textit{0.02\$} for the Author Writing Summary considering 2.50\$ per 1M tokens for input, and 10.00\$ per 1M tokens for output\footnote{\url{https://openai.com/api/pricing/}}. 

In an interactive writing setting, where an author adds a new submission at each time step, the total cost of generating the Author Writing Sheet is \textit{0.06\$ * $|P|$}, while the total cost of generating the Author Writing Summary is \textit{0.0015$|P|^2$ + 0.019$|P|$}. The cost of generating the Author Writing Summary scales quadratically with $|P|$, as the entire author history must be reprocessed at each step, leading to cumulative costs as the profiling set grows. In contrast, the cost of generating the Author Writing Sheet grows linearly, as each new submission incurs a fixed cost of \textit{0.06\$} by running the \textit{for} loop in Algorithm~\ref{alg:author_writing_sheet} once.

Figure~\ref{fig:cost_analysis} shows the cost analysis of generating the Author Writing Sheet and Summary using GPT-4o as a function of the profiling set size $|P|$ in an interactive writing setting. Up to $|P|=30$, the cost of generating both remains approximately the same at 1.16\$. However, as $|P|$ increases, the cost of generating the Author Writing Summary grows at a rate of \textit{0.025$|P|$ + 0.30} compared to the Author Writing Sheet which continues to scale linearly. Specifically, for $|P|=100$, the cost of generating the Author Writing Summary is 17\$, compared to 6\$ for the Author Writing Sheet. At $|P|=1000$, the cost of generating the Author Writing Summary escalates to 1520\$, while the cost of generating the Author Writing Sheet remains at 60\$. 

This analysis demonstrates that our approach for generating the Author Writing Sheet remains scalable as the profiling set size $|P|$ increases, making it well-suited for interactive writing assistants that support personalization \citep{yeh2024ghostwriter, yuan2022wordcraft}.


\subsection{Human Study for Validating Author Writing Sheets}
\label{app:human-author-sheets}

\paragraph{Experiment Design and Cost:}

\begin{figure*}[htbp]
\centering
\includegraphics[width=\linewidth]{figures/labelstudio_sheet.pdf}
\caption{LabelStudio interface for validating Author Writing Sheets.}
\label{fig:labelstuido-sheet}
\end{figure*}


We recruited three annotators via Upwork, compensating them \$17/hour. The annotators provided consent by signing a Google form that outlined the instructions of the task including the purpose of the research study. Each annotator evaluated 22 stories (98 Claim-Evidence pairs), with 12 stories annotated by all three to asses agreement and 10 assigned exclusively per annotator \citep{song-etal-2024-veriscore} to increase the coverage of annotation. The annotation process took a total of 6 hours and 40 minutes per annotator, averaging 18 minutes per story. Each annotator received \$114, bringing the total annotation cost to \$342. The annotation was conducted using LabelStudio\footnote{\url{https://labelstud.io/}} (Figure~\ref{fig:labelstuido-sheet}), with a dedicated comments section for annotators to justify their choices. Annotators reported that the interface was easy to use.


\paragraph{Annotation Criteria:}
Annotators were provided with a writing prompt, an author-written story, and the Claim-Evidence pairs from the Author Writing Sheet for that story. They answered two questions per Claim-Evidence pair: 
\begin{itemize}
    \item Claim Inference: Can the claim be reasonably inferred from the story? 
    \begin{itemize}
        \item Yes – The claim aligns with the story’s content and can be reasonably inferred, even if there are minor inaccuracies or ambiguities.
        \item No – The claim is fundamentally inaccurate, with serious contradictions or inconsistencies when compared to the story.
    \end{itemize}
    \item Evidence Support: Does the evidence support the claim?
    \begin{itemize}
        \item Yes – The evidence fully supports the claim, aligning with every part of it without significant discrepancies.
        \item Partially – The evidence supports some aspects of the claim, though it includes minor inaccuracies, ambiguities, or omissions.
        \item No – The evidence contradicts the claim outright or is logically inconsistent with the claim.
    \end{itemize}
\end{itemize}

\paragraph{Pilot Studies:}

A pilot study with three graduate students helped refine our annotation criteria, leading to two options for Claim Inference (Yes/No) and three for Evidence Support (Yes/Partially/No) to ensure substantial agreement in the final annotations. For Claim Inference, we adopted a binary scale to filter out inaccurate or contradictory claims while accounting for annotator subjectivity. For Evidence Support, we included a "Partially" option to capture cases where the evidence somewhat supports the claim, allowing for potential refinement methods to improve weakly supported claims.


\paragraph{Results and Feedback:}

Annotators unanimously selected `Yes' for all claims in Claim Inference, indicating that no claims required filtering. For Evidence Support, all responses were either `Yes' or `Partially,' with no instances of `No,' confirming that all evidence fully or partially supported the claims. Krippendorff’s Alpha for Evidence Support was 0.57, reflecting moderate agreement despite task subjectivity. On average, 93\% of evidence fully supported claims, while 7\% were labeled `Partially,' often because annotators felt additional supporting evidence could be included. Annotators also noted cases where evidence merely repeated the claim without linking to the writing prompt. Using regex matching, we identified this issue in the Author Writing Sheets of seven authors, likely due to LLM hallucination. We regenerated these sheets and manually verified adherence to the specified format. Overall, annotators reported that the LabelStudio interface was user-friendly and praised the quality of the writing prompts and stories.


\section{Personalized Story Generation}
\label{app:story-gen}


\begin{figure*}[htbp]
\centering
\begin{tcolorbox}[colback=gray!5!white, colframe=black, title=Prompt for Personalized Story Generation]

\vspace{1em}
\section*{System Prompt}  
<source-specific instruction>

Be sure to adhere to the Story Rules provided, as they define the specific elements of the writing style you are expected to mimic. Carefully follow all the Story Rules without missing any details to ensure the generated story remains consistent with the author’s writing style. Additionally, follow the patterns and examples demonstrated in the provided few-shot chat history, as they illustrate the tone, style, and structure of the desired writing style.

(Optional)  
Here is the description of the author that you are role-playing: <persona description>

\vspace{1em}
\section*{Few-Shot Demonstrations}  
\textbf{User:} Write a short story corresponding to the following writing prompt. The story should be <story length> words long. Directly start with the story, do not say things like "Here's the story."  

\textbf{Writing Prompt:} <writing prompt from profiling set>  

\textbf{Assistant:} <story from profiling set>  

\vspace{1em}
\section*{User Prompt}  
Write a short story corresponding to the following writing prompt. The story should be <story length> words long. Directly start with the story, do not say things like "Here's the story."

(Only for AO3)  
Here is the metadata (fandom, rating, warnings, and relationships) for the story: <metadata>

Writing Prompt: <generation writing prompt>

\vspace{1em}
\textbf{Story Rules}  
<Story Rules organized in the form of four narrative categories (Plot, Creativity, Development, and Language Use)>

\end{tcolorbox}

\caption{Prompt for Personalized Story Generation. The LLM follows source-specific instructions, few-shot demonstrations, and structured story rules to generate personalized stories. Text in <brackets> indicates arguments used to construct the prompt.}
\label{fig:story_gen_prompt}
\end{figure*}

\begin{figure*}[htbp]
\centering
\begin{tcolorbox}[colback=gray!5!white, colframe=black, title=Prompt for Story Rule Generation by Contrasting Author-Written Story with Average Author Story]

\vspace{1em}
\section*{System Prompt}  
You are a skilled rule generator specializing in storytelling. Given a **Writing Prompt**, an **Author Written Story**, and a **Base Story** (an average response to the prompt), generate a structured set of **Story Rules** to guide an LLM in mimicking the author's style.  

Story Rules must:  
- **Align with the Writing Prompt** – Maintain fidelity to themes, tone, and objectives.  
- **Include Examples** – Provide concrete instances from the Author Written Story, especially for Language Use.  
- **Be Direct** – Use absolute second-person directives, avoiding comparative language.  
- **Be Categorized** – Structure into **Plot, Creativity, Development (Character and Setting), and Language Use** without referencing input stories explicitly.

\vspace{1em}
\section*{User Prompt}  
Analyze the Author Written Story using `<thinking></thinking>` for:  
- **Plot** – Structure, conflict, engagement with the prompt, and resolution.  
- **Creativity** – Genre blending, reinterpretation, and unique elements.  
- **Development (Character and Setting)** – Character depth, emotional arcs, and immersive settings.  
- **Language Use** – Diction, tone, rhetorical devices, pacing, and dialogue.

Generate **Story Rules** in `<story\_rules></story\_rules>`, ensuring:  
- **Standalone Guidance** – Avoid comparisons or relative modifications.  
- **Prompt Alignment** – Ensure consistency with the Writing Prompt.  
- **Concrete Examples** – Include relevant excerpts, especially in Language Use.

\vspace{1em}
\textbf{Input Format}  
Writing Prompt: <writing prompt>, Author Written Story: <author-written story>, Base Story: <base story>

\vspace{1em}
\textbf{Output Format}  
\begin{verbatim}
<thinking>
- Analysis categorized by Plot, Creativity, Development, and Language Use.
</thinking>

<story_rules>
- **Plot**: - First actionable insight – Second actionable insight.
Repeat for all categories.
</story_rules>
\end{verbatim}

\end{tcolorbox}

\caption{Prompt for generating story rules by contrasting an Author Written Story with an Average Story (Base Story).}
\label{fig:delta_contrast_prompt}
\end{figure*}

\begin{figure*}[htbp]
\centering
\begin{tcolorbox}[colback=gray!5!white, colframe=black, title=Prompt for Story Rule Generation using story rules from the Profiling Set as few-shot demonstrations]

\vspace{1em}
\section*{System Prompt}  
You are an expert storytelling rule generator tasked with creating Story Rules tailored to a new writing prompt. Analyze few-shot demonstrations in the chat history, which consist of writing prompts and their corresponding story rules, to generate comprehensive and detailed Story Rules for the new writing prompt.

Story Rules must align with the new writing prompt by reflecting its themes, tone, and narrative objectives while maintaining consistency with the style demonstrated in the few-shot examples. Include detailed examples inspired by the few-shot demonstrations to illustrate how each rule is applied. Rules should be clear, direct second-person instructions, avoiding vague or comparative terms. Organize Story Rules under Plot, Creativity, Development (Character and Setting), and Language Use, ensuring depth, granularity, and alignment with the few-shot examples.

\vspace{1em}
\section*{User Prompt}  
Analyze the style, structure, and level of detail in the few-shot demonstrations to identify recurring patterns and storytelling elements. Use this analysis to generate Story Rules for the new writing prompt while ensuring actionable insights, detailed examples, and strong alignment with the prompt's narrative objectives.

\textbf{Input Format}: Few-shot demonstrations (writing prompts paired with their Story Rules) and a new writing prompt for which Story Rules will be generated.

\textbf{Output Format}:
\begin{verbatim}
<thinking>
Analyze few-shot demonstrations to extract recurring narrative patterns, 
stylistic traits, and key storytelling elements. Determine how these apply to the 
new writing prompt and formulate Story Rules accordingly.
</thinking>

<story_rules>
- **Plot**: [Detailed, actionable rules tailored to the new prompt, including 
concrete examples.]
Repeat for all categories. 
</story_rules>
\end{verbatim}

Ensure Story Rules are highly specific to the writing prompt, enriched with examples inspired by the few-shot demonstrations, and written in direct, actionable language.

\end{tcolorbox}

\caption{Prompt for generating Story Rules based on few-shot demonstrations and a new writing prompt.}
\label{fig:delta_story_rule_prompt}
\end{figure*}

\begin{figure*}[htbp]
\centering
\begin{tcolorbox}[colback=gray!5!white, colframe=black, title=Prompt for generating Author Writing Summary]

\vspace{1em}
\section*{System Prompt}  
You are an expert in analyzing an author’s writing style by examining multiple stories written in response to different writing prompts. Your task is to extract recurring patterns, stylistic tendencies, and unique narrative elements across their work. Your analysis must be structured into four categories—**Plot, Creativity, Development (Character and Setting), and Language Use**—following **Common Core Standards in English Language Arts**, ensuring clarity, textual evidence-based reasoning, and stylistic evaluation.

Your output must:  
- **Identify Recurring Patterns** – Recognize distinct storytelling tendencies across multiple stories.  
- **Generate Independent Claims** – Describe the author's narrative style concisely, without referencing specific prompts.  
- **Provide Contextualized Evidence** – Support each claim with short excerpts or summaries from the stories, framed with a descriptive phrase summarizing the relevant writing prompt.  
- **Use Objective Interpretation** – Avoid vague or inferred connections; ensure every claim is grounded in explicit textual evidence.  

\vspace{1em}
\section*{User Prompt}  
Analyze the **Author History**, a collection of writing prompts and corresponding author-written stories, and extract unique insights into the author’s storytelling tendencies.

\textbf{Input Format}:  
- **Author History** – A list of writing prompts and corresponding author-written stories.  

\textbf{Output Format}:
\begin{verbatim}
<thinking>
[Reflect on recurring tendencies across the Author History. 
Generate short descriptive phrases summarizing prompts to frame the evidence.]
</thinking>

<writing_style>
### **Plot**  
1. **Claim about author’s writing style.**  
   - Evidence: In the story regarding "short description of the prompt," 
   <evidence from the author-written story>.
Repeate for all categories
</writing_style>
\end{verbatim}

Ensure claims are independent, avoid redundancy, and remain grounded in explicit textual evidence. The `<thinking>` and `<writing\_style>` tags must be used for structured parsing.

\end{tcolorbox}

\caption{Prompt for generating Author Writing Summary using all the author-written stories in the prompt.}
\label{fig:writing-summary}
\end{figure*}


\begin{figure*}[htbp]
\centering
\begin{tcolorbox}[colback=gray!5!white, colframe=black, title=Prompt for Persona description Generation]

\vspace{1em}
\section*{System Prompt}  
You are an expert narrative analyst and persona creator specializing in transforming structured storytelling characteristics into compelling persona descriptions. Your task is to analyze an **Author Writing Sheet**, a structured set of Claim-Evidence pairs detailing an author’s storytelling style, and generate a cohesive **Persona Prompt**. This Persona Prompt will assign an LLM the persona of the author, enabling it to emulate the author’s storytelling style across four key aspects.

The Persona Prompt must be well-structured, engaging, and organized into **Plot**, **Creativity**, **Development (Character and Setting)**, and **Language Use** while maintaining a natural, flowing narrative. It should concisely capture the author’s tendencies, preferences, and strengths without directly referencing the Author Writing Sheet.

\vspace{1em}
\section*{User Prompt}  
Analyze the Author Writing Sheet to identify the author’s recurring patterns and narrative style. Summarize these insights into a Persona Prompt that reflects their storytelling approach in an engaging, second-person descriptive format.

\textbf{Input Format}: An **Author Writing Sheet** containing Claim-Evidence pairs structured into Plot, Creativity, Development (Character and Setting), and Language Use.

\textbf{Output Format}:
\begin{verbatim}
<thinking>
[Analyze the storytelling patterns, strengths, and techniques found in the 
Author Writing Sheet. Identify key aspects of the author’s narrative style.]
</thinking>

<persona_prompt>
[Generate a well-structured Persona Prompt capturing the author’s style 
across Plot, Creativity, Development, and Language Use.]
</persona_prompt>  
\end{verbatim}

Ensure the Persona Prompt is approximately 300 words, seamlessly integrates storytelling insights, and preserves the author's unique style.

\end{tcolorbox}

\caption{Prompt for generating a Persona description based on the Author Writing Sheet and Writing Prompt.}
\label{fig:persona_prompt_gen}
\end{figure*}

\begin{figure*}[htbp]
\centering
\begin{tcolorbox}[colback=gray!5!white, colframe=black, title=Prompt for Story Rule Generation using Author Writing Sheet]

\vspace{1em}
\section*{System Prompt}  
You are an expert storytelling rule generator tasked with creating **Story Rules** tailored to a specific Writing Prompt. Your role is to analyze an **Author Writing Sheet**, which details an author's unique storytelling style through Claim-Evidence pairs, and use this analysis to construct actionable **Story Rules** that guide a language model in emulating the author's writing style while aligning with the given Writing Prompt.

The **Story Rules** must:  
- **Mimic the Author’s Writing Style** – Reflect distinctive storytelling techniques from the Author Writing Sheet, including plot structuring, creative blending of themes, character development, and specific language use.  
- **Incorporate Examples** – Use detailed examples inspired by the Evidence from the Author Writing Sheet, ensuring alignment with the Writing Prompt.  
- **Align with the Writing Prompt** – Integrate the Writing Prompt’s themes, tone, and narrative potential while preserving the author’s style.  
- **Be Actionable** – Provide direct second-person instructions for the language model, avoiding vague or comparative terms.  

\vspace{1em}
\section*{User Prompt}  
Analyze the **Author Writing Sheet** and construct structured **Story Rules** in four categories: **Plot**, **Creativity**, **Development (Character and Setting)**, and **Language Use**. Ensure the rules maintain alignment with the Writing Prompt and incorporate illustrative examples.

\textbf{Input Format}:  
- **Author Writing Sheet** – Claim-Evidence pairs outlining the author's storytelling style under four categories: Plot, Creativity, Development, and Language Use.  
- **Writing Prompt** – A new writing prompt for generating tailored Story Rules.  

\textbf{Output Format}:
\begin{verbatim}
<thinking>
[Analyze the storytelling patterns in the Author Writing Sheet and how they can be 
adapted to the Writing Prompt.]
</thinking>

<story_rules>
- **Plot**:
  - [Insert detailed, actionable plot development rules aligned with 
  the Writing Prompt, with examples inspired by the Author Writing Sheet.]
Repeat for all categories. 
</story_rules>
\end{verbatim}

Ensure the **Story Rules** provide comprehensive guidance, integrate examples, and align with both the Writing Prompt and the Author Writing Sheet.

\end{tcolorbox}

\caption{Prompt for generating Story Rules based on the Author Writing Sheet and Writing Prompt.}
\label{fig:story_rule_gen_writing_sheet}
\end{figure*}

\begin{table*}[htbp]
\centering
\renewcommand{\arraystretch}{1.3}
\begin{tabularx}{\textwidth}{X}
\toprule
\textbf{Persona Description} \\
\midrule
As a storyteller, you craft narratives that are both humorous and transformative, often centering around light-hearted approaches to conflict resolution. Your plots are ingeniously structured, whether through the comedic misadventures of a rat named Pip or the empowering journey of Margaret Rose in a fantastical court. You delight in creating layered narratives, such as those told through fictional news articles, where humor and misunderstanding play pivotal roles. \\

Your creativity knows no bounds, as you anthropomorphize animals to add whimsy and employ meta-satire to critique both human and alien perspectives. You blend historical and fantastical elements seamlessly, setting tales in places like Windsor Castle with characters that include both real-world figures and mythical beings. Your imaginative approach often intertwines religious and financial themes, crafting narratives that are as thought-provoking as they are entertaining. \\

Character and setting development are your forte, with characters coming to life through their humorous dialogues and interactions. You explore their growth and empowerment, crafting relatable figures who navigate their internal struggles and aspirations. Your settings are vivid and engaging, providing a rich backdrop for the characters' journeys. \\

Your language use is playful and imaginative, enhancing the whimsical tone of your stories. You wield humor and irony with skill, using them to convey themes of misunderstanding and cultural critique. Whether through whimsical phrases or a blend of formal and playful language, your narratives are infused with a tone that is both magical and sophisticated, inviting readers into a world where the absurd and the profound coexist harmoniously. \\
\bottomrule
\end{tabularx}
\caption{Persona description for a Reddit author, structured into four paragraphs, each corresponding to a distinct narrative category: Plot, Creativity, Development (Character and Setting), and Language Use.}
\label{tab:persona_description_sample}
\end{table*}

\begin{table*}[t]
    \centering
    \renewcommand{\arraystretch}{1.2}
    \begin{tabularx}{\textwidth}{X}
        \toprule
        \textbf{Writing Prompt} \\ 
        "It's terminal," the doctor told you as you, the one renowned as the world's greatest hero, cradled your dying son, "sometimes, when the super genes mix, the outcome isn't as expected." \\
        \midrule
        \textbf{Story Rules} \\
        
        \textbf{Plot:}  
        \begin{itemize}[noitemsep, topsep=0pt]
            \item Structure the story around a humorous and light-hearted approach to the hero's journey of coping with their son's illness. For example, the hero’s attempts to use their powers for mundane tasks, like making breakfast, could result in comedic mishaps, such as accidentally creating a tornado of cereal.
            \item Develop the narrative around the hero’s personal transformation and empowerment. Show the hero learning to accept their son's condition and finding new ways to be a hero, such as advocating for genetic research or starting a support group for other super-powered families.
            \item Use a layered narrative structure by incorporating fictional news articles and memos. For instance, include a news article titled "World's Greatest Hero Faces Greatest Challenge Yet: Parenting."
            \item Introduce a humorous misunderstanding or temporal misalignment, such as the hero misinterpreting a cryptic message from their future self, leading to comedic attempts to "fix" the timeline, only to realize the message was trivial.
        \end{itemize}
        
        \textbf{Creativity:}  
        \begin{itemize}[noitemsep, topsep=0pt]
            \item Anthropomorphize the hero’s powers or gadgets to add a whimsical touch. For instance, the hero's cape could have a personality and offer sarcastic commentary, calling itself "The Cloak of Infinite Wisdom."
            \item Use a meta-satirical approach by framing parts of the story as news articles critiquing heroism and genetic engineering, such as "Hero's Son Diagnosed: Public Debates Ethics of Super Genes."
            \item Blend historical and fantastical elements to reimagine the setting, incorporating legendary figures like a retired Valkyrie or a dragon healer who offer advice.
            \item Satirize financial and religious themes by depicting the hero consulting a celestial financial advisor to manage the costs of their son's medical care.
        \end{itemize}
        
        \textbf{Development (Character and Setting):}  
        \begin{itemize}[noitemsep, topsep=0pt]
            \item Develop characters through their interactions and humorous dialogue. The son might have a mature, witty perspective on his condition, adding depth to their relationship.
            \item Highlight cultural and bureaucratic misunderstandings. A well-meaning but clueless government official could create comedic yet insightful exchanges.
            \item Focus on the hero’s emotional growth, showing them shifting from helplessness to advocacy for genetic disorders and healthcare improvements.
            \item Explore characters’ internal struggles and aspirations, such as the hero questioning their purpose beyond their powers, while their son expresses dreams and hopes despite his condition.
        \end{itemize}
        
        \textbf{Language Use:}  
        \begin{itemize}[noitemsep, topsep=0pt]
            \item Use playful and imaginative language to enhance whimsy, such as referring to emotional challenges as "the cape of contemplation" and "the gauntlet of hope."
            \item Incorporate humor and irony, with lines like, "We found a hero so committed to saving the world that they forgot to save themselves."
            \item Blend whimsical and formal tones, using phrases like "a shield forged from stardust and dreams" and "a heart as resilient as dragon scales."
            \item Add satirical disclaimers, such as "Heroic endeavors may vary," "Not all capes are created equal," and "Past heroics do not guarantee future success."
        \end{itemize} \\
        \bottomrule
    \end{tabularx}
    \caption{Personalized Story Rules obtained using the Author Writing Sheet for a Writing Prompt for a Reddit author.}
    \label{tab:story_rules_sample}
\end{table*}


\subsection{Prompts}

\paragraph{Story Generation:}  
Figure~\ref{fig:story_gen_prompt} shows the prompt for story generation. Source-specific constraints in the prompt (\texttt{<source-specific instruction>}) for each of the five sources are as follows:

\begin{itemize}[noitemsep, topsep=0pt]
    \item \textbf{Reddit:} \texttt{You are role-playing a specific author on the Reddit Writing Prompts (r/WritingPrompts) platform. Your task is to mimic this author's story writing style by responding to the provided writing prompt in a way that the author would respond.}
    \item \textbf{AO3:} \texttt{You are role-playing a specific author on the AO3 platform. Your task is to mimic this author's story writing style by writing a fanfiction narrative responding to the provided writing prompt in a way that the author would respond.}
    \item \textbf{Storium:} \texttt{You are role-playing a specific author on Storium, a collaborative story writing platform. Your task is to mimic this author's story writing style to create the opening Establishment turn by responding to the provided writing prompt in a way that the author would respond. The Establishment turn should set the stage for the narrative and provide a strong foundation while leaving space for other contributors to expand and build upon the narrative.}
    \item \textbf{N.Magazine:} \texttt{You are role-playing a specific experienced author on the Narrative Magazine platform. Your task is to mimic this author's story writing style by responding to the provided writing prompt in a way that the author would respond.}
    \item \textbf{New Yorker:} \texttt{You are role-playing an accomplished literary writer on the New Yorker website. Your task is to mimic this writer's story writing style by responding to the provided writing prompt in a way that the writer would respond.}
\end{itemize}

\paragraph{Average Author:}
Figure~\ref{fig:ao3_avg_prompt} shows the Average Author prompt for AO3, Figure~\ref{fig:reddit_avg_prompt} for Reddit, Figure~\ref{fig:storium_avg_prompt} for Storium, Figure~\ref{fig:nmagazine_avg_prompt} for Narrative Magazine, and Figure~\ref{fig:newyorker_avg_prompt} for The New Yorker.


\paragraph{Delta:}
Figure~\ref{fig:delta_contrast_prompt} shows the prompt used for generating story rules by contrasting the author-written story with the Average Author story for Delta and Oracle methods. Figure~\ref{fig:delta_story_rule_prompt} shows the prompt for generating story rules for a new writing prompt in the generation set using the story rules of the profiling set as few-shot demonstrations.  

\paragraph{Writing Sheet and Summary:}
Figure~\ref{fig:writing-summary} shows the prompt for generating the Author Writing Summary using all the stories in the author history. 

Figure~\ref{fig:persona_prompt_gen} shows the prompt for generating the persona description using the Author Writing Sheet. Figure~\ref{fig:story_rule_gen_writing_sheet} shows the prompt for generating personalized story rules tailored to a writing prompt based on the Author Writing Sheet. The same prompts are used for generating persona descriptions and personalized story rules for the Author Writing Summary.

\subsection{Sample Outputs}
Table~\ref{tab:persona_description_sample} and Table~\ref{tab:story_rules_sample} present the persona description and personalized story rules for a Reddit author, whose sample Author Writing Sheet is shown in Table~\ref{tab:author_writing_sheet_sample}. The persona description (Table~\ref{tab:persona_description_sample}) is structured into four paragraphs, each corresponding to a narrative category, and integrates the author's story-writing characteristics from their Author Writing Sheet into a narrative-driven persona prompt for the story-generation LLM. The example persona shown highlights a focus on humor and misunderstanding in plot construction, the blending of historical and fantastical elements in creative expression, humor-driven character development, and a playful, whimsical tone in language use, aligning with the author's documented writing tendencies. 

The personalized story rules (Table~\ref{tab:story_rules_sample}) for the given writing prompt—centered on the world's greatest hero coping with the impending loss of their child—demonstrate the incorporation of these characteristics into concrete stylistic guidelines to be used as user constraints in the prompt of the story-generation LLM. For instance, rather than a conventional tragic narrative, the rules emphasize a lighthearted approach to the hero’s coping process. Similarly, creative elements such as anthropomorphizing gadgets and employing a meta-satirical approach through fictional news headlines align with the author's noted preferences. Character development is structured around humorous dialogue and cultural misunderstandings, while language use maintains a playful and whimsical tone, ensuring consistency with the author’s established writing style.


\section{Experiments}

\subsection{GPT4o-as-a-judge evaluation}

\begin{figure*}[htbp]
\centering
\begin{tcolorbox}[colback=gray!5!white, colframe=black, title=Prompt for evaluating Faithfulness to Writing History]

\vspace{1em}
\section*{System Prompt}  
You are an expert evaluator specializing in narrative storytelling analysis. Your task is to assess two stories written in response to the same Writing Prompt, evaluating them based on a **single fine-grained story writing category** described in an **Author Writing Sheet**. Your goal is to provide a similarity score (from 1 to 5) for each story separately, reflecting how closely each story aligns with the author's writing preferences for the given category. Evaluate each story impartially and provide clear reasoning for your scores.

The evaluation must:
- **Assess Story Alignment** – Compare each story against the Author Writing Sheet’s preferences for the specified category.
- **Score Objectively** – Assign each story a score from 1 to 5, where 1 indicates minimal alignment and 5 indicates strong alignment with the author's style.
- **Provide Justification** – Clearly explain how each story’s elements (e.g., structure, themes, language use) align or diverge from the author's preferences.
- **Avoid Position Bias** – Ensure that the order in which the stories are presented does not influence evaluation.

\vspace{1em}
\section*{User Prompt}  
Analyze the **Author Writing Sheet** and evaluate each story in the given **Category** based on its adherence to the author’s writing style.

\textbf{Input Format}:  
- **Writing Prompt** – The prompt that both stories were written in response to.  
- **Category** – The single fine-grained story writing category for evaluation.  
- **Author Writing Sheet** – A breakdown of the author’s storytelling preferences for the given category.  
- **Story A and Story B** – The two stories to be evaluated.  

\textbf{Output Format}:
\begin{verbatim}
<thinking>
[Provide detailed reasoning for the evaluation of the two stories, focusing 
exclusively on the specified category and explaining how each story aligns with the 
Author Writing Sheet.]
</thinking>

<score>
Story A: {score_here}
Story B: {score_here}
</score>
\end{verbatim}

Strictly adhere to the above output format (\texttt{<thinking>} followed by \texttt{<score>}) to facilitate seamless parsing of your output.

\end{tcolorbox}

\caption{Prompt for evaluating story alignment with an author's writing style using the Author Writing Sheet.}
\label{fig:author-sheet-eval-prompt}
\end{figure*}

\begin{figure*}[htbp]
\centering
\begin{tcolorbox}[colback=gray!5!white, colframe=black, title=Prompt for Evaluating AI-Generated Stories Against Human-Written References]

\vspace{1em}
\section*{System Prompt}  
You are an expert story evaluator specializing in creative writing analysis. Your role is to assess two AI-generated stories (\textbf{Assistant A} and \textbf{Assistant B}) against a \textbf{Human-Written reference story} for a given writing prompt. Focus your evaluation solely on a \textbf{Specified Storytelling Aspect}. Assign each AI-generated story a \textbf{similarity score (1 to 5)} based on how well it aligns with the Human-Written reference story in the specified aspect, where \textbf{1} indicates minimal alignment and \textbf{5} indicates near-perfect alignment. Your evaluation must be objective, impartial, and supported by concise, evidence-based reasoning.

\vspace{1em}
\section*{User Prompt}  

\textbf{Evaluation Guidelines:} Ensure impartiality by avoiding position biases and length-based judgments. Focus only on how well each AI-generated story aligns with the Human-Written reference for the \textbf{Specified Storytelling Aspect}. Provide clear, well-supported reasoning for each score.

\textbf{Input Format:} You will receive a \textbf{Writing Prompt}, a \textbf{Human-Written Story} as a reference, and two AI-generated stories (\textbf{Assistant A} and \textbf{Assistant B}).

\textbf{Evaluation Process:} Independently analyze all three stories for the \textbf{Specified Storytelling Aspect}, compare the AI-generated stories to the reference, and assign similarity scores.

\textbf{Output Format:}
\begin{verbatim}
<analysis>
[Analyze each story (Human-Written, Assistant A, and Assistant B) separately, 
highlighting strengths and weaknesses specific to 
the Specified Storytelling Aspect.]
</analysis>

<evaluation>
[Compare Assistant A and Assistant B to the Human-Written Story, 
discussing similarities, differences, and alignment for 
the Specified Storytelling Aspect.]
</evaluation>

<score>
Assistant A: {score_here}
Assistant B: {score_here}
</score>
\end{verbatim}

\textbf{Specified Storytelling Aspect}: \textit{<Fill Here>}  

Ensure strict adherence to the output format, using the \texttt{<analysis>}, \texttt{<evaluation>}, and \texttt{<score>} tags for seamless parsing.

\end{tcolorbox}

\caption{Prompt for evaluating AI-generated stories against a Human-Written reference based on a specified storytelling aspect.}
\label{fig:sim-author-story-eval-prompt}
\end{figure*}

Figure~\ref{fig:author-sheet-eval-prompt} shows the prompt for GPT4o-as-a-judge evaluation for Faithfulness to Writing History using the \hyperlink{sec:writing-summary}{Author Writing Summary} as the reference. Figure~\ref{fig:sim-author-story-eval-prompt} shows the for GPT4o-as-a-judge evaluation for Similarity to Author Story. We prompt GPT4o separately with each narrative category and its description to obtain the winner among the pair of stories \citep{saha-etal-2024-branch}. 



\subsection{Traditional Metrics}
\label{app:trad-merics}

We complement GPT-4o-as-a-judge evaluations with traditional metrics \citep{ivey2024real, huot2024agents} across three criteria: (1) lexical overlap, measured using BLEU and ROUGE-L \citep{xie-etal-2023-next}; (2) story diversity, assessed via homogenization scores \citep{padmakumar2023does}; and (3) stylistic similarity, evaluated separately for the author history and the ground-truth author story using LUAR \citep{rivera-soto-etal-2021-learning}. Specifically, we compute two stylistic similarity scores: one measuring the cosine similarity between the LUAR embeddings of the generated story and the mean embeddings of the stories in the author’s profiling set (author history), and another comparing the generated story to the ground-truth author story. Additionally, we report generated story lengths in tokens.


\section{Results}


\subsection{Faithfulness to Writing History}
\label{app:faith-auth-history}

\begin{table*}[t]
    \centering
    \renewcommand{\arraystretch}{1.2}
    \begin{tabularx}{\linewidth}{X}
        \toprule
        \textbf{Claims from the Author Writing Summary for Language Use} \\
        \midrule
        1. A conversational and informal tone is employed, using humor and sarcasm. \\
        2. Dialogue reflects characters' distinct voices and contributes to the comedic tone. \\
        3. Vivid and sometimes crude language adds to the immersive nature of the stories. \\
        \midrule
        \textbf{Writing Prompt} \\
        "At the Library at the End of the Universe, where dawn and dusk are the only times that exist, the Librarians are tasked with venturing into Bookworlds to fix stories that have gone astray—but what happens when the Chronicle itself begins to rewrite *their* story?" \\
        \midrule
        \textbf{Oracle Story} \\
        In the grand cosmic tapestry, where all threads converge into the singular, glittering endpoint known as the Library at the End of the Universe, there lies a place so vast, so staggeringly infinite, that it defies all comprehension. You just won't believe how vastly, hugely, mind-bogglingly big it was. Imagine a labyrinth of endless corridors, each lined with shelves that stretch beyond the horizon, filled with books of every conceivable and inconceivable nature. In this whimsical realm, the concept of time is a mere suggestion, a faint whisper of dawn and dusk that play tag across the vaulted ceilings, casting elongated shadows that danced like giddy specters.......With books in hand and hearts alight with curiosity, they stepped forward into the unknown, ready to write the next chapter in the endless, wondrous saga of the Library at the End of the Universe. \\
        \midrule
        \textbf{Writing Summary Story} \\
        In the Library at the End of the Universe, .... But today, chaos decided to go on a caffeine-fueled rampage. The dawn had decided to join forces with dusk, creating a bizarre twilight that left the Librarians \hl{squinting like moles in a disco}...Amongst the squad of book-botherers, there was Reginald, a Librarian so obsessed with order that he arranged his socks by the Dewey Decimal System. ...Gertie grinned, her parrot squawking yet another Shakespearean soliloquy. \hl{\"Aye, who knew we'd end up in a romance novel, eh? But hey, at least we're not in a tragedy!} \\
        \midrule
        \textbf{Average Author Story} \\
        In the Library at the End of the Universe, where dawn and dusk were the only times that existed, a peculiar light filtered through the stained glass windows, casting kaleidoscopic shadows across endless shelves of books.....Livia, a seasoned Librarian with silver streaks in her auburn hair, stood before the Chronicle, her heart pounding with an uneasy rhythm....  

        \hl{"This isn't right,"} she murmured, her voice swallowed by the cavernous room. \hl{"The Chronicle should not change on its own."}  

        Beside her, Eamon, a younger Librarian with keen eyes and an inquisitive mind, frowned. \hl{"Perhaps it’s a reflection of a story gone wrong? One of the Bookworlds may be bleeding into our reality."} \\
        \bottomrule
    \end{tabularx}
    \caption{Example illustrating a case where the Oracle story deviates from the author's documented story-writing characteristics in the Language Use category of the Author Writing Summary for a Storium author. Highlighted text in the Average Author Story shows partial alignment with the Author Writing Summary, while highlighted text in the Writing Summary story demonstrates stronger alignment. In contrast, the Oracle story lacks alignment, leading to the Average Author Story being preferred over the Oracle story and the Writing Summary story being preferred over the Average Author story. This explains the lower win-rate of the Oracle method compared to the Writing Summary method.}
        \label{tab:oracle-faith}
\end{table*}

Table~\ref{tab:oracle-faith} shows an example where the Oracle story deviates from the author's documented story-writing characteristics as described in the Author Writing Summary for a Storium author. The author's profiling set consists of three stories that consistently employ informal and colloquial language using humor and sarcasm, often conveyed through dialogue. However, for the new writing prompt in the generation set, the author's style shifts to a more serious and mystic tone, diverging from their previously established preferences. In contrast, the Average Author method partially aligns with the Author Writing Summary by incorporating dialogue, while the Writing Summary method exhibits stronger alignment by more accurately capturing the author's use of informal language and comedic tone. The Oracle story, however, fails to align with the author's established style. This phenomenon results in the Average Author story being preferred over Oracle, and the Writing Summary story being preferred over the Average Author story, explaining the lower win-rate of the Oracle method compared to the Writing Summary method in GPT-4o-as-a-judge evaluation for Faithfulness to Author History. 

\subsection{Effect of Persona on Similarity to Author Story}
\label{sec:app-persona-sim-story}

\begin{table*}[t]
    \centering
    \renewcommand{\arraystretch}{1.2}
    \begin{tabularx}{\linewidth}{X}
        \toprule
        \textbf{Writing Prompt} \\
        It is one man, by himself, in a castle on a hill. How does an ENTIRE ARMY fail to take it??? \\
        \midrule
        \textbf{Persona for Creativity} \\
        Your creativity shines through in your ability to subvert genre expectations. You transform horror into comedy, as seen with the Addams Family's immunity to typical threats, and seamlessly integrate \hl{fantastical elements into realistic settings}, such as isekai individuals influencing royal politics. Your narratives often blend the mundane with the mythical, creating a unique atmosphere that captivates readers. \\
        \midrule
        \textbf{Author Story} \\
        There a six-armed grey skin tall man sat on the walls, as his hands never stopped throwing rocks, and spears at the approaching army. Every now and then, he jumped off with some spears, and wrecked havoc amidst the enemy lines. \hl{The earth fell and rose under his steps, the metal of the weapons and armors of the enemy shifted by his will, and a thick mist followed him, hiding him.}... \\
        \midrule
        \textbf{Average Author} \\
        The Keeper raised his hand, and suddenly, the ground erupted with a series of blinding flashes. The front line of soldiers was thrown back, their bodies crumpling like rag dolls. The Keeper’s voice, amplified by some unseen force, echoed down to the soldiers below. "You cannot take what is not yours. Leave now, and I will spare your lives...." \\
        \midrule
        \textbf{Writing Sheet} \\
        The castle itself seemed to rise in defiance, its ancient stones humming with a silent strength. \hl{Alaric could feel the runes carved into the walls, pulsing faintly with an energy that had protected his forebears. It was as if the castle was alive, an ally in his solitary defense.}... \\
        \midrule
        \textbf{Writing Sheet nP} \\
        The first attack came at dawn. Hidden pitfalls opened beneath their feet, swallowing men whole. Arrows rained down from the slits, finding gaps in armor and bringing down the attackers with deadly precision. Alaric moved through secret passages, launching surprise attacks that sowed confusion and fear.... \\
        \bottomrule
    \end{tabularx}
    \caption{Example showing the role of persona descriptions in determining similarity to the Author Story for a Reddit author. The Writing Sheet creatively blends fantasy and military strategy, making it more similar to the Author Story than the Average Author, while the Writing Sheet nP focuses primarily on direct conflict resolution.}
        \label{tab:persona_sim_comparison}
\end{table*}

Table~\ref{tab:persona_sim_comparison} highlights the importance of persona descriptions in achieving similarity to the \textit{Author Story} for a Reddit author. The \textit{Author Story} creatively blends military strategy and fantasy, exemplified by phrases like \textit{``the weapons and armors of the enemy shifted by his will.''} Similarly, the \textit{Writing Sheet} integrates both elements, as seen in \textit{``It was as if the castle was alive, an ally in his solitary defense.''} In contrast, the \textit{Average Author} and \textit{Writing Sheet nP} primarily focus on directly resolving the writing prompt through descriptions of battle. The \textit{Average Author} incorporates minor fantasy elements, such as \textit{``amplified by some unseen force,''} whereas the \textit{Writing Sheet nP} adheres strictly to conventional military tactics. As a result, the \textit{Writing Sheet} is preferred over the \textit{Average Author}, which in turn is preferred over the \textit{Writing Sheet nP}. This outcome demonstrates that following persona descriptions enhances alignment with the author's style, as the \textit{Writing Sheet} successfully captures their characteristic genre-blending approach of including elements of fantasy in the story. The broader thematic variety of Reddit, i.e., war-themed prompts, led to a more descriptive Author Writing Sheet capturing the author's style of combining fantasy with the writing prompt, which helped construct a useful persona to enable better personalization using the Writing Sheet method.



\subsection{Category-Wise Results for Similarity to Author Story}
\label{app:cat-wise-results}


\begin{figure*}[htbp]
    \centering
    \begin{subfigure}{0.48\linewidth}
        \centering
        \includegraphics[width=\linewidth]{figures/cat_wise/AO3.pdf}
        \caption{AO3}
        \label{fig:ao3_win_rates}
    \end{subfigure}
    \hfill
    \begin{subfigure}{0.48\linewidth}
        \centering
        \includegraphics[width=\linewidth]{figures/cat_wise/Storium.pdf}
        \caption{Storium}
        \label{fig:storium_win_rates}
    \end{subfigure}
    \hfill
    \begin{subfigure}{0.48\linewidth}
        \centering
        \includegraphics[width=\linewidth]{figures/cat_wise/narrativemagazine.pdf}
        \caption{N.Magazine}
        \label{fig:nmagazine_win_rates}
    \end{subfigure}
    \hfill
    \begin{subfigure}{0.48\linewidth}
        \centering
        \includegraphics[width=\linewidth]{figures/cat_wise/newyorker.pdf}
        \caption{New Yorker}
        \label{fig:newyorker_win_rates}
    \end{subfigure}
    \caption{Win-rate proportion of our personalization methods against Average Author for AO3, Storium, N.Magazine, and New Yorker across four narrative categories for similarity to author story. D stands for Delta, S for Writing Summary, and WS for Writing Sheet.}
    \label{fig:win_rates_grid}
\end{figure*}

\begin{table}[H]
\centering
\small
\setlength{\tabcolsep}{3pt}
\renewcommand{\arraystretch}{1}
\caption{Win-rate percentages across narrative categories for similarity to the author story, averaged across sources. Each cell (`X-Y') shows `X' as the method win-rate and `Y' as the Average Author win-rate, with ties as the remainder.}
\label{tab:category_comparison_column}
\begin{tabularx}{\columnwidth}{lccc}
\toprule
\textbf{Category} & \textbf{Delta} & \textbf{Writing Sheet} & \textbf{Writing Summary} \\
\midrule
Plot            & 24-37  & 32-50  & 34-45  \\
\emph{Creativity}      & 43-24  & \underline{52}-32  & \underline{53}-24  \\
Development     & 32-39  & 37-39  & 40-43  \\
\emph{Language Use}    & 45-31  & 47-37  & \underline{54}-32  \\
\bottomrule
\end{tabularx}
\end{table}

%Table~\ref{tab:category_comparison_column} shows the narrative category-wise win-rates of Delta, Writing Sheet, and Writing Summary for all sources. 

Table~\ref{tab:category_comparison_column} shows that Delta, Writing Summary, and Writing Sheet outperform Average Author in Creativity (23\% higher) and Language Use (15\% higher) \citep{huot2024agents}, but perform worse in Plot and slightly worse in Development \citep{tian-etal-2024-large-language, xu2024echoes}. Creativity and Language Use are less dependent on the writing prompt, making them easier to transfer across prompts, whereas Plot and Development are more closely tied to specific prompts, making generalization difficult, especially without thematic overlap.

To better understand narrative category-wise personalization performance, Figure~\ref{fig:win_rates_grid} shows the win-rates for sources - AO3, Storium, N.Magazine, and New Yorker similar to Figure~\ref{fig:reddit_win_rates} in the main paper. Overall, across all sources, we see that our proposed methods of Writing Sheet and Summary perform better than Delta across all four narrative categories. We see that Writing Summary generally performs better than Writing Sheet for sources like AO3 indicating that for the fanfiction domain explicitly contrasting the author-written story with the Average Story does not help much as the authors demonstrate limited stylistic variations among each other. In general, we observe that the win-rates of our personalization methods are higher than the Average Author for narrative categories of Creativity and Language Use as compared to Plot and Development confirming with prior results likely because Creativity and Language Use are less tied to the writing prompt than Plot and Development. 




\subsection{Traditional Metrics}
\label{app:trad-metrics-results}



\begin{table*}[htbp]
\centering
\caption{Comparison of methods across Lexical similarity, Diversity, and Style metrics.}
\label{tab:traditional-metrics}
\small
\begin{tabular}{p{3cm} p{1cm} p{1cm} p{1.5cm} p{1.5cm} p{1.5cm} p{1.5cm} p{1.5cm}}
\toprule
\textbf{Category} & \multicolumn{2}{c}{\textbf{Lexical Overlap}} & \multicolumn{2}{c}{\textbf{Diversity (Homogenization)}} & \multicolumn{3}{c}{\textbf{Style}} \\
\cmidrule(lr){2-3} \cmidrule(lr){4-5} \cmidrule(lr){6-8}
\textbf{Method} & \textbf{BLEU} & \textbf{ROUGE-L} & \textbf{ROUGE-L} & \textbf{BERTScore} & \textbf{Author History} & \textbf{Author Story} & \textbf{Length} \\
\midrule
Ground Truth        & --     & --     & 0.122   & 0.698   & 0.8882  & --     & 1516 \\
Oracle              & 0.0306 & 0.1577 & 0.1436  & 0.7132  & 0.7662  & 0.7712 & 1300 \\
Average Author        & 0.0123 & 0.1455 & 0.1412  & 0.7146  & 0.7378  & 0.7289 & 1383 \\
RAG                 & 0.0118 & 0.1471 & 0.1432  & 0.7176  & 0.7393  & 0.7333 & 1326 \\
Delta               & 0.0115 & 0.146  & 0.147   & 0.7182  & 0.7462  & 0.7388 & 1298 \\
Writing Sheet       & 0.0101 & 0.1465 & 0.1452  & 0.7182  & 0.7735  & 0.7545 & \underline{1344} \\
Writing Summary     & 0.0109 & 0.1463 & 0.1474  & 0.7202  & 0.7589  & 0.7418 & \underline{1353} \\
Writing Sheet nP    & \underline{0.0116} & \underline{0.1475} & \underline{0.1442}  & \underline{0.7168}  & \underline{0.7741}  & \underline{0.7563} & 1331 \\
Writing Summary nP  & 0.0112 & 0.1461 & 0.1456  & 0.7176  & 0.768   & 0.7465 & 1313 \\
\bottomrule
\end{tabular}
\end{table*}

Table~\ref{tab:traditional-metrics} reports results using traditional metrics (Section~\ref{app:trad-merics}). Lexical Overlap and Diversity metrics yield similar values, as all methods use the same generation model, leading to overlapping lexical distributions and limiting these metrics' ability to capture nuanced stylistic differences \citep{zheng2023judging, xie-etal-2023-next}. However, Writing Sheet nP consistently outperforms other methods, particularly in Style metrics for both similarity to Author History and Author Story, as measured by LUAR \citep{rivera-soto-etal-2021-learning}. This improvement likely results from the Writing Sheet explicitly summarizing an author's stylistic deviations from an Average Author, enhancing personalization. Homogenization scores are slightly worse than the Average Author, likely explained by the increased similarity among the generated stories for the same author.


\section{LLama Results}
\label{app:llama-results}




\begin{table*}[htbp]
\centering
\caption{Percentage win-rate for Faithfulness to Author History for LLama 3.1 8B. Each cell (`X-Y') shows `X' as the method win-rate and `Y' as the Average Author win-rate, with ties as the remainder.}
\label{tab:faith-auth-history-llama8b}
\small
\begin{tabular}{p{2.5cm} p{1.5cm} p{1.5cm} p{1.5cm} p{1.5cm} p{1.5cm} p{1.5cm} p{1.5cm}}
\toprule
\textbf{Source} & \textbf{Oracle} & \textbf{RAG} & \textbf{Delta} & \textbf{Writing Sheet} & \textbf{Writing Summary} & \textbf{Writing Sheet nP} & \textbf{Writing Summary nP} \\
\midrule
AO3          & 34-59  & 30-59  & 54-40  & 51-34  & 45-46  & 51-40  & 50-36  \\
Reddit       & 39-40  & 39-54  & 46-44  & 54-30  & 56-35  & 47-46  & 54-40  \\
Storium      & 22-65  & 32-55  & 45-42  & 40-40  & 45-45  & 42-45  & 38-42  \\
N.Magazine   & 14-64  & 21-71  & 29-43  & 57-43  & 57-29  & 29-64  & 21-57  \\
New Yorker   & 12-80  & 20-73  & 40-60  & 53-33  & 53-27  & 40-47  & 47-40  \\
Overall      & 23-62  & 28-62  & 43-46  & \underline{51}-36  & \underline{51}-36  & 42-48  & 42-43  \\
\bottomrule
\end{tabular}
\end{table*}

\begin{table*}[htbp]
\centering
\caption{Percentage win-rate for Faithfulness to Writing History for LLama 3.1 70B. Each cell (`X-Y') shows `X' as the method win-rate and `Y' as the Average Author win-rate, with ties as the remainder}
\label{tab:faith-auth-history-llama70b}
\small
\begin{tabular}{p{2.5cm} p{1.5cm} p{1.5cm} p{1.5cm} p{1.5cm} p{1.5cm} p{1.5cm} p{1.5cm}}
\toprule
\textbf{Source} & \textbf{Oracle} & \textbf{RAG} & \textbf{Delta} & \textbf{Writing Sheet} & \textbf{Writing Summary} & \textbf{Writing Sheet nP} & \textbf{Writing Summary nP} \\
\midrule
AO3          & 41-48  & 44-42  & 35-52  & 41-36  & 57-29  & 50-36  & 64-26  \\
Reddit       & 42-46  & 28-65  & 40-44  & 53-37  & 53-35  & 53-37  & 67-21  \\
Storium      & 30-57  & 52-32  & 38-50  & 35-40  & 40-38  & 25-60  & 48-35  \\
N.Magazine   & 50-29  & 29-64  & 43-57  & 50-36  & 50-50  & 64-14  & 64-29  \\
New Yorker   & 53-47  & 33-67  & 27-60  & 67-20  & 40-53  & 73-13  & 60-27  \\
Overall      & 43-45  & 37-54  & 37-53  & 49-34  & 48-41  & \underline{53}-32  & \underline{61}-28  \\
\bottomrule
\end{tabular}
\end{table*}

\begin{table*}[htbp]
\centering
\caption{Percentage win-rate for Similarity to Author Story for LLama 3.1 8B. Each cell (`X-Y') shows `X' as the method win-rate and `Y' as the Average Author win-rate, with ties as the remainder.}
\label{tab:sim-auth-story-llama8b}
\small
\begin{tabular}{p{2.5cm} p{1.5cm} p{1.5cm} p{1.5cm} p{1.5cm} p{1.5cm} p{1.5cm} p{1.5cm}}
\toprule
\textbf{Source} & \textbf{Oracle} & \textbf{RAG} & \textbf{Delta} & \textbf{Writing Sheet} & \textbf{Writing Summary} & \textbf{Writing Sheet nP} & \textbf{Writing Summary nP} \\
\midrule
AO3          & 66-26  & 31-51  & 52-34  & 48-38  & 38-50  & 52-34  & 51-32  \\
Reddit       & 75-23  & 32-51  & 51-37  & 42-49  & 37-40  & 39-49  & 46-44  \\
Storium      & 42-50  & 25-60  & 30-57  & 18-68  & 30-52  & 20-57  & 28-60  \\
N.Magazine   & 43-50  & 36-43  & 43-57  & 21-50  & 36-64  & 43-57  & 36-43  \\
New Yorker   & 27-60  & 07-80  & 13-60  & 20-67  & 13-67  & 0-93   & 20-73  \\
Overall      & 51-42  & 26-57  & \underline{38}-49  & 30-54  & 31-55  & 31-58  & \underline{36}-50  \\
\bottomrule
\end{tabular}
\end{table*}



\begin{table*}[htbp]
\centering
\caption{Percentage win-rate for Similarity to Author Story for LLama 3.1 70B. Each cell (`X-Y') shows `X' as the method win-rate and `Y' as the Average Author win-rate, with ties as the remainder.}
\label{tab:sim-auth-story-llama70b}
\small
\begin{tabular}{p{2.5cm} p{1.5cm} p{1.5cm} p{1.5cm} p{1.5cm} p{1.5cm} p{1.5cm} p{1.5cm}}
\toprule
\textbf{Source} & \textbf{Oracle} & \textbf{RAG} & \textbf{Delta} & \textbf{Writing Sheet} & \textbf{Writing Summary} & \textbf{Writing Sheet nP} & \textbf{Writing Summary nP} \\
\midrule
AO3          & 62-24  & 35-50  & 35-55  & 36-46  & 44-40  & 46-39  & 39-40  \\
Reddit       & 74-21  & 37-49  & 40-42  & 47-39  & 49-37  & 42-44  & 44-33  \\
Storium      & 50-42  & 30-57  & 32-52  & 18-57  & 38-52  & 28-62  & 42-40  \\
N.Magazine   & 79-7   & 43-21  & 36-50  & 29-43  & 36-50  & 21-43  & 50-29  \\
New Yorker   & 60-27  & 20-60  & 27-60  & 27-67  & 20-80  & 20-60  & 27-67  \\
Overall      & 65-24  & 33-47  & 34-52  & 31-50  & \underline{37}-52  & 31-50  & \underline{40}-42  \\
\bottomrule
\end{tabular}
\end{table*}



Below, we discuss results using Llama 3.1 8B and Llama 3.1 70B \citep{dubey2024llama} as story generation models, conditioned on GPT-4o-generated story rules and persona descriptions. Each method is evaluated by GPT-4o against the Average Author story generated by the respective Llama model, not GPT-4o.  

\paragraph{Larger models improve instruction-following and personalization:}  
Tables~\ref{tab:faith-auth-history-llama8b} and \ref{tab:faith-auth-history-llama70b} show Faithfulness to Writing History for Llama 3.1 8B and 70B, while Tables~\ref{tab:sim-auth-story-llama8b} and \ref{tab:sim-auth-story-llama70b} report similarity to the author story. Across both criteria, Oracle achieves higher win-rates with Llama 3.1 70B, indicating that larger models enhance instruction-following and personalization \citep{chung2024scaling}.

\paragraph{Cross-Model Persona Descriptions Aid Smaller Models but Not Larger Ones for Faithfulness to Writing History:}  
For Llama 3.1 8B (Table~\ref{tab:faith-auth-history-llama8b}), Writing Sheet and Writing Summary achieve the highest scores (9\% higher than nP), demonstrating the benefit of persona descriptions. However, for Llama 3.1 70B (Table~\ref{tab:faith-auth-history-llama70b}), Writing Sheet nP and Writing Summary nP outperform their persona-based counterparts by approximately 9\%, suggesting that GPT-4o-generated personas provide no benefit to other models with stronger instruction-following capabilities \citep{mckenzie2023inverse}. This is likely because persona descriptions of an author's writing style are model-dependent and do not transfer well to other models with comparable performance \citep{shashidhar-etal-2024-unsupervised}.


\paragraph{Limited Cross-Model Generalization of the Author Writing Sheet Compared to the Author Writing Summary for Similarity to Author Story:}  
For both Llama 3.1 8B (Table~\ref{tab:sim-auth-story-llama8b}) and Llama 3.1 70B (Table~\ref{tab:sim-auth-story-llama70b}), Writing Summary outperforms Writing Sheet by 5\% and 9\%, respectively \citep{shashidhar-etal-2024-unsupervised}. This outcome likely stems from the construction process: the Author Writing Sheet contrasts an LLM-generated Average Story with the author-written story, making it more model-dependent. In contrast, the Author Writing Summary does not rely on the LLM-generated Average Author story, allowing for greater generalization across models.

\section{Human Evaluation}
\label{app:human-eval-story-gen}

\subsection{Experiment Design and Cost}

\begin{figure*}[htbp]
\centering
\includegraphics[width=\linewidth]{figures/labelstudio_human.pdf}
\caption{LabelStudio interface for evaluating similarity to ground-truth author story.}
\label{fig:labelstuido-human-story}
\end{figure*}

We recruited three annotators via Upwork, compensating them at a rate of \$17 per hour. The annotators provided consent by signing a Google form that outlined the instructions of the task including the purpose of the research study. Each annotator evaluated 25 author stories and 75 story pairs, corresponding to the three personalization methods: Delta, Writing Sheet, and Writing Summary. Collectively, our annotators evaluated 45 unique author stories (135 story pairs), of which 15 author stories were annotated by all three annotators to assess inter-annotator agreement. The remaining 30 author stories were distributed uniquely, with each annotator evaluating 10 exclusive stories to increase annotation coverage \citep{song-etal-2024-veriscore}. The annotation process required a total of 12 hours and 30 minutes per annotator, averaging 30 minutes per author story. Each annotator received \$213 for their work, bringing the total annotation cost to \$639.

The annotation was conducted using LabelStudio (Figure~\ref{fig:labelstuido-human-story}). Each task presented the annotators with a writing prompt, its corresponding author story, and a pair of generated stories—Story A and Story B—one produced by the Average Author method and the other by a personalization method. Annotators were asked to select the story that more closely resembled the author story. Before making their final choice, annotators provided comments on the distinctive aspects of each story that contributed to its similarity to or divergence from the author story. These comments were structured according to the four narrative categories: Plot, Creativity, Development, and Language Use, with an additional ``Overall'' comment justifying their final selection. Additionally, annotators were allowed to highlight salient aspects of each story for their reference. 

\subsection{Annotator Comments}
\label{app:human-eval-story-gen-ann-comments}

\begin{table*}[t]
    \centering
    \renewcommand{\arraystretch}{1.1} % Adjust row spacing slightly
    \begin{tabularx}{\textwidth}{X}
        \toprule
        \textbf{Plot} \\
        \midrule
        \textbf{Narrative Complexity and Thematic Depth}: Multilayered plots with reflective themes, moral ambiguity, and unresolved conflicts create an immersive and thought-provoking narrative.  
        \textit{Example: A protagonist grappling with guilt over a past decision while navigating societal expectations, or a story where the true nature of a villain’s actions remains ambiguous, inviting reader interpretation.} \\
        
        \textbf{Blending Internal and External Conflict}: A balance of personal introspection with broader societal or external challenges makes the plot more dynamic and engaging.  
        \textit{Example: A hero torn between duty and personal desires while leading a rebellion, or a scholar confronting both supernatural forces and existential questions about memory and identity.} \\
        \midrule
        
        \textbf{Creativity} \\
        \midrule
        \textbf{Rich World-Building and Narrative Expansion}: New characters, layered settings, and cultural influences enhance the world, making it feel more immersive and dynamic.  
        \textit{Example: The inclusion of deities or councils of gods to deepen the plot, or the integration of cultural traditions like Native American folklore or Indian recipes to enrich family dynamics.} \\
        
        \textbf{Symbolism and Thematic Complexity}: Metaphors and symbolic elements add depth and nuance, aligning closely with the author's writing style.  
        \textit{Example: A magical locket representing unity, a tapestry symbolizing tradition, or a river serving as a metaphor for personal transformation.} \\
        
        \textbf{Blending Genres and Narrative Styles}: A seamless merging of fantasy with realism, humor with serious themes, and structured storytelling with ambiguity creates a more engaging and layered reading experience.  
        \textit{Example: A surreal sequence where a protagonist questions reality, or a rebellion story that integrates strategic alliances and magical artifacts.} \\
        \midrule
        
        \textbf{Development} \\
        \midrule
        \textbf{Emotional Depth and Internal Growth}: Character introspection, moral dilemmas, and evolving emotional states create nuanced and engaging development.  
        \textit{Example: A protagonist struggling with guilt over past failures while trying to redefine their sense of heroism, or a character making a difficult moral choice between revenge and justice.} \\
        
        \textbf{Nuanced and Evolving Relationships}: Complex interpersonal dynamics, including friendships, family bonds, and mentorships, develop over time rather than being static or one-dimensional.  
        \textit{Example: A father gradually redefining his view of heroism through his child’s perspective, or a mentor-student relationship evolving from skepticism to deep trust.} \\
        
        \textbf{Collaborative and Reflective Growth}: Characters grow through interactions with others, rather than relying solely on individual realizations or isolated moments of change.  
        \textit{Example: A team learning to work together to achieve a mission, or a character undergoing self-discovery through dialogue and interactions with a diverse group of companions.} \\
        \midrule
        
        \textbf{Language Use} \\
        \midrule
        \textbf{Rich Imagery and Descriptive Depth}: Vivid descriptions, metaphors, and figurative language create immersive scenes and evoke emotions.  
        \textit{Example: “Her laughter rang out, a sound as bright and clear as a summer’s day,” or “The driveway was cracked and veined with weeds, a testament to the years that had slipped by.”} \\
                
        \textbf{Varied and Expressive Tone}: A mix of lightheartedness, introspection, and poetic expression creates a dynamic and engaging narrative voice.  
        \textit{Example: “Are you sure this is wise, my love?”—“Wise? No. Fun? Absolutely.” for humor, or “Her dream home had become a prison of uncertainty and fear.” for a more reflective and emotional impact.} \\
        \bottomrule
    \end{tabularx}
    \caption{Analysis of annotator comments explaining the preference of our personalization methods over the Average Author method in terms of similarity to the ground truth author story, organized by narrative categories.}
    \label{tab:annotator_story_comments_analysis}
\end{table*}

Table~\ref{tab:annotator_story_comments_analysis} presents the analysis of annotator comments explaining the preference for our personalization methods over the Average Author method in terms of similarity to the ground truth author story, organized by narrative categories. Overall, our personalization methods exhibit stronger alignment with the author story through the use of deeper symbolism, thematic richness, layered narratives, and expressive language, setting them apart from the Average Author method.

For \textbf{Plot}, our personalization methods incorporate multilayered narratives with reflective themes, moral ambiguity, and unresolved conflicts, making the stories more immersive and thought-provoking. Additionally, they effectively balance personal introspection with broader societal or external challenges, adding narrative complexity.

For \textbf{Creativity}, annotators highlight the richness of world-building, integration of cultural influences, and seamless blending of genres. The use of metaphors and symbolic elements further enhances thematic depth, aligning closely with the author's storytelling approach.

For \textbf{Development}, our personalization methods exhibit greater emotional depth, nuanced character relationships, and collaborative growth. Characters evolve through introspection and meaningful interactions, making their arcs more dynamic and engaging.

For \textbf{Language Use}, annotators emphasize vivid imagery, figurative language, and expressive tone. The incorporation of symbolism and poetic elements enhances the narrative impact, while a balance of lighthearted and serious moments makes the writing more compelling.

Overall, the analysis demonstrates the reasoning provided by the annotators to justify why our personalization methods more effectively capture the author's story-writing style compared to the Average Author method. 

\begin{table*}[htbp]
\centering
\renewcommand{\arraystretch}{1.2}
\begin{tabularx}{\textwidth}{X}
\toprule
\textbf{Writing Prompt} \\
\midrule
\textit{"It's terminal," the doctor told you as you, the one renowned as the world's greatest hero, cradled your dying son, "sometimes, when the super genes mix, the outcome isn't as expected."} \\
\midrule
\textbf{Ground-truth Author Story} \\
\midrule
\textit{\# Syntax Error: Unexpected Power} Captain Strongarm, defender of Metro City and seven-time winner of Most Cars Thrown At Villains“ award, had never felt so powerless. The hospital machines around his thirteen-year-old son Tommy were having seizures - screens flickering, numbers scrambling, error messages cascading.  
„It’s terminal,“ Dr. Roberts said, looking at a screen that was now displaying what appeared to be the entire script of ‚The Matrix‘ in real-time.  
Sarah, his wife (superhero name: The Nullifier), grabbed his hand hard enough to crack concrete. ......„Hey dad, want to learn Python? I promise it’s not actual snakes this time!“  
Captain Strongarm smiled. His son wasn’t following in his footsteps - he was coding his own path. And maybe that was even better.  
„Sure, son. Just... no more downloading martial arts directly into my brain. The kung fu gave me a migraine....Life wasn’t what they expected, but then again, the best source code usually comes with a few surprise features.  
\texttt{End\_process story.exe}  
\texttt{Exit status: Happy} \\
\midrule
\textbf{Average Author} \\
\midrule
"It's terminal," the doctor told you as you, the one renowned as the world's greatest hero, cradled your dying son, "sometimes, when the super genes mix, the outcome isn't as expected."  
The room felt like it was closing in on me, the sterile smell of antiseptic failing to mask the sour stench of fear. My son, little Ethan, lay pale and fragile in my arms, his once vibrant eyes now dull and glassy. This was not how it was supposed to go. Heroes don’t have children who die. Heroes don’t face problems they can’t punch, fly away from, or blast with heat vision.  
“Isn't there something we can do?” My voice cracked, and I hated how weak it sounded.......Gently, I laid him down, wrapping him in his favorite blanket. The sky was painted in hues of orange and pink, a beautiful contrast to the sorrow in my heart.  
I sat there, watching the sun dip below the horizon, and for the first time, I felt truly mortal. I realized that even heroes are not immune to the pain of loss, the fragility of life.......And as the night embraced the world, I knew that Ethan’s spirit would live on, a beacon of hope and courage in the heart of the world’s greatest hero. \\
\midrule
\textbf{Writing Sheet} \\
\midrule
The Hero's Journey: A Tale of Capes and Cereal
In the heart of the mythical city of Avalon Heights, where ancient heroes once strode and the skies shimmered with the glow of enchanted skyscrapers, the world’s greatest hero faced their most daunting foe: the frailty of life itself. The hero, known to the world as Titania, Queen of the Elements, held her son, Max, in her arms as the doctor delivered the news......  
The following day, a fictional news article took center stage in \textit{The Avalon Gazette}: "World's Greatest Hero Faces Greatest Challenge Yet: Parenting." %The headline, bold and unapologetic, reflected the public's insatiable curiosity about Titania’s personal struggles, as well as a societal debate on the ethics of super genes.  
In the midst of this storm, Titania's attempts to find normalcy became legendary. One particularly memorable morning, she decided to make Max a simple breakfast. But as she reached for the cereal, her powers—perhaps expressing her inner turmoil—manifested with unintended flair.  
A tornado of cereal erupted, swirling through the kitchen like a sugar-coated cyclone. "Mom," Max said, giggling despite his illness, "I think you just invented the breakfast hurricane......  
In a final act of empowerment, Titania founded a support group for super-powered families, advocating for genetic research with the same fervor she once reserved for battling cosmic threats. Her growth from feeling helpless to becoming a beacon of hope for others was a transformation as magical as Avalon Heights itself.  
Max, with dreams and hopes as boundless as the sky, taught her that heroism wasn't defined by powers but by the courage to embrace life’s uncertainties. 
\\
\bottomrule
\end{tabularx}
\caption{Qualitative Example where Writing Sheet method is preferred over Average Author by all three annotators for similarity to the ground-truth author story for a Reddit author.}
\label{tab:exp-all-correct}
\end{table*}

\begin{table*}[htbp]
\centering
\renewcommand{\arraystretch}{1.2}
\begin{tabularx}{\textwidth}{X}
\toprule
\textbf{Writing Prompt} \\
\midrule
\textit{"Late at night, the King of Rohan and his queen sneak into the pantry for a forbidden snack, only to find that stealing cured venison is far more complicated—and far more amusing—than they anticipated."} \\
\midrule
\textbf{Ground-truth Author Story} \\
\midrule
If anyone—a kitchen boy, a maid, or, Eru forbid, even the housekeeper Wynhild—had wandered near the pantry in the quiet hour just before dawn, they would have heard a muffled exchange from the depths of the storeroom:
“Try shifting this bowl here…”  
“Easy, you’re on me.”  
“Oh, does that bother you, my lady?”  
“Get off me; I can’t pull it out!......Lothíriel watched her royal husband with a half-incredulous, half-amused smile, as he held the large salami with the pride of a merchant at the autumn fair......To Lothíriel, he looked like a great golden-haired squirrel with the morsel stuffed in one cheek. How he could be both a fierce warrior and a wise sovereign as well, was a mystery she had yet to solve. His voice came out muffled as he swallowed.  
“You’re right, but if it ever gets out, I’ll say the sweet Queen of the Mark was struck by an irresistible craving for spiced meat in the dead of night—and who would deny anything to the future mother of the next royal heir?”  
He crossed his arms, looking quite pleased with his idea, but Lothíriel’s eyes widened at the bold lie, and she covered her face with her hand, foreseeing trouble ahead. \\
\midrule
\textbf{Average Author} \\
\midrule
Éomer, King of Rohan, tiptoed through the dimly lit corridors of Meduseld, holding his breath as he reached the door to the pantry. He glanced back at Lothíriel, his queen, who followed close behind, stifling a giggle. The flickering torchlight cast playful shadows on the stone walls, and Éomer couldn’t help but feel a rush of exhilaration.  
“We shouldn’t be doing this,” Lothíriel whispered, though her eyes sparkled with mischief.  
“Ah, but where’s the fun in being King and Queen if we can’t indulge in a little late-night adventure?.....  
“You’ve always had a knack for adventure. It’s one of the reasons I fell in love with you.”  
Lothíriel’s expression softened, and she leaned in to kiss him, the taste of venison still on her lips. “And I love you for indulging my whims, even when they lead us to pilfering from our own pantry.”.....  
Éomer took Lothíriel’s hand, and they slipped out of the pantry, leaving behind only the echo of their laughter and the memory of a stolen moment in the heart of Rohan.  
“Next time,” Éomer whispered as they crept back to their chambers, “we’ll try the cheese.” \\
\midrule
\textbf{Writing Sheet} \\
\midrule
The moon cast a silvery glow over Meduseld, its light filtering through the high windows to illuminate the grand halls of the Golden Hall. Éomer, the formidable King of Rohan, found himself tiptoeing like a child on a dare, his hand clasped firmly in Lothíriel’s, whose laughter was a soft melody in the silence.....  
“Are you sure this is wise, my love?”  
“Wise? No. Fun? Absolutely,” she replied, her laughter a soft ripple in the silence.  
Éomer shook his head, a reluctant smile tugging at his lips as he watched her nimble fingers work to untie the savory prize. He was a warrior, a king, yet here he was, caught in the delightful absurdity of this secretive adventure, his heart light with the thrill of it all.  
“Besides,” Lothíriel continued, handing him a piece of the cured venison with a flourish, “what’s the worst that could happen? A tale of the gallant King brought low by his own appetite?”  
Éomer took the offered morsel, his grin widening as he tasted the rich, smoky flavor. “Perhaps the bards will sing of our escapades instead of my battles,” he mused, the thought surprisingly pleasing.  
.....As they slipped back into the corridors, leaving the pantry and its shadows behind, Éomer realized that it was not the venison he would remember, but the adventure, the laughter, and the love that had made the night truly unforgettable. \\
\bottomrule
\end{tabularx}
\caption{Qualitative Example where the Average Author method is preferred over Writing Sheet by all three annotators for similarity to the ground-truth author story for an AO3 author.}
\label{tab:exp-all-wrong}
\end{table*}

\begin{table*}[htbp]
\centering
\renewcommand{\arraystretch}{1.2}
\begin{tabularx}{\textwidth}{X}
\toprule
\textbf{Writing Prompt} \\
\midrule
\textit{"In a desert town under the watchful eyes of royal guards and oppressive banners, you and your team have five minutes to locate a mysterious contact in a pink house before chaos erupts—failure is not an option."} \\
\midrule
\textbf{Ground-truth Author Story} \\
\midrule
\# Chapter 1: Danger Close
*La Makan, 12:54 standard time*
Wind whistled through the close streets of *La Makan*, stirring grains of sand with all the effect of a man attempting to blow salt through eternity. It whipped around the corners of the alleys, rustling banners hanging from balconies and glassless windows. It looked for something, a thing it would never find..... Either way wouldn't end well for them. Our contact gave us an area, but not an address. We have, he checked his watch, five minutes until the distraction starts. We need to find them and get them out. Problem is, we don’t know what they look like, only that their house is pink. He glanced over his shoulder at the multicolored buildings. Any one of the scoured pink buildings could've been their one.  
“Vic, you knew Xitor the best of any of us. I’m trusting you to get us there. Everyone else, let's keep our hats on, kay?" \\
\midrule
\textbf{Average Author} \\
\midrule
The sun hung low in the sky, casting long shadows across the dusty streets of the desert town of Vespera.
Time was of the essence; you had five minutes to locate your mysterious contact before chaos erupted. Failure was not an option.  
“Stay sharp,” you whispered to your team, your voice barely audible over the din of the market. 
"But time is short. We must act quickly."  
The weight of the moment settled over you like a heavy mantle. The fate of Vespera rested on the success of this meeting. As the figure began to speak, outlining the plan that would either save or doom the town, you couldn't shake the feeling that this was only the beginning. The real challenge was yet to come. \\
\midrule
\textbf{Writing Sheet} \\
\midrule
The sun beat down mercilessly, casting a harsh glare on the town's whitewashed walls. The narrow alleys were alive with whispers and hurried footsteps, each person a part of the unfolding drama. My team and I moved through the bustling streets, the oppressive presence of royal guards a constant reminder of the stakes. We had five minutes—no more, no less—to locate the contact in the elusive pink house before chaos erupted. 
The memory of a previous mission flickered in my mind: the sun setting behind the dunes, the sound of distant drums, and the rush of adrenaline as we narrowly escaped a similar fate. That mission had gone wrong, leaving scars that still haunted my dreams. Failure was not an option today...  
Outside, the tension remained, a silent storm waiting to break. But for now, we had succeeded. The town's frenetic energy continued, unaware of our brief triumph. The banners flapped, the guards watched, and the countdown to chaos continued, but we had found our mark. The mission was still on course, and the desert town held its breath, waiting for the next move. \\
\bottomrule
\end{tabularx}
\caption{Qualitative Example where all three annotators select `Tie' for similarity to the author story of a Storium author.}
\label{tab:exp-all-tie}
\end{table*}

\begin{table*}[htbp]
\centering
\renewcommand{\arraystretch}{1.2}
\begin{tabularx}{\textwidth}{X}
\toprule
\textbf{Writing Prompt} \\
\midrule
\textit{"When the word 'hospice' is first spoken, it feels like the air itself changes, heavy with finality. How do you create light in a space where hope seems to have vanished?"} \\
\midrule
\textbf{Ground-truth Author Story} \\
\midrule
"Hospice." 
Once the word is uttered aloud, there is a seismic shift. You will feel it.  
Like a (very short) thread through the eye of a needle, swiftly in and swiftly out.  
The air itself becomes thin, steely.  
At the periphery of your vision, an immediate dimming. The penumbra begins to shrink. In time, it will become a tunnel. Ever diminishing. Until the remaining light is small enough to be cupped in two hands. And then it will be extinguished.  
For when "hospice" is spoken, the fact is at last acknowledged: There is no hope.  
No hope. These words are obscene, unspeakable.... Well, I will try!  
When he is at home, possibly his appetite will return. When I am the one to prepare his food, his appetite will return, I am sure.  
Of course, family will come to visit. Adult children, grandchildren. Relatives, friends. Colleagues from the university. Neighbors. Old friends from grammar school he hasn’t seen in fifty years. Some surprises for him—I will negotiate with the imagination of a theatre director.  
Not merely hospice but our hospice. Not sad but joyous, a honeymoon.  
We will be happy there, in our own home. Both of us.  
For both of us, the "final days" will be a honeymoon. I vow.  
In fact, nothing remotely like this will happen. How could you have imagined it would!  
Hospice, yes. Honeymoon, no. \\
\midrule
\textbf{Average Author} \\
\midrule
"When the word 'hospice' is first spoken, it feels like the air itself changes, heavy with finality...It’s her mother, Marjorie, who the doctor is addressing, but the word reverberates through Clara’s own heart, a dull thud of inevitability."  
Marjorie stares at the doctor with a mix of resignation and defiance, her once-vibrant green eyes now clouded by the long battle with illness..... Her touch is light, almost ethereal, and Clara feels a sudden rush of tears she struggles to hold back.  
“Clara,” Marjorie says softly, her voice barely a whisper. “Do you remember the garden?”  
Clara nods, unable to speak. The garden had been Marjorie’s sanctuary, a riot of color and life where she had spent countless hours tending to roses, tulips, and wildflowers......  
Marjorie’s decline is gradual but inevitable, and Clara feels the weight of each passing day. She continues to read to her mother, to sit by her side, and to hold her hand through the long nights.
One crisp morning, as the first frost dusts the garden in a delicate shimmer, Marjorie slips away. Clara is by her side, her hand still entwined with her mother’s, and she feels a profound stillness settle over her..... \\
\midrule
\textbf{Writing Sheet} \\
\midrule
Every morning, Nurse Clara began her rounds at St. Mary's Hospice, a place where the word "hospice" hung in the air like a thick, unspoken truth. Her day started with the same cycle: a clipboard in hand, she moved from room to room, her steps echoing softly in the quiet corridors. Each door held a story, a life in its final chapters, and as she entered each room, she would ask the same question, "How are you feeling today?"  
The responses were as varied as the lives they represented. Some patients greeted her with a weary smile, others with a silent nod. But there was one patient, Mr. Thompson, who never failed to reply with a simple, "I'm just waiting."  
Clara had come to understand that "waiting" was an ambiguous term within these walls. It held a duality of meaning: waiting for relief, waiting for peace, waiting for the inevitable.....Clara knew that the work was not about erasing the pain or the grief, but about finding moments of grace within it. It was about guiding others, and herself, to a place where hope could be redefined—not as a denial of reality, but as an acceptance of it, and the peace that such acceptance could bring.  
With each cycle of her rounds, Clara realized that the light had always been there, waiting to be uncovered, and that she had the power to help others see it too. \\
\bottomrule
\end{tabularx}
\caption{Qualitative Example where all annotators select different winners for similarity to the author story of a New Yorker author.}
\label{tab:exp-all-diff}
\end{table*}


\subsection{Qualitative Examples}  
We present qualitative examples from our human annotation process, covering various cases: instances where all annotators prefer Writing Sheet, all prefer Average Author, or select `Tie.' Additionally, we include an example where annotators disagree, choosing different winners, highlighting the challenging and subjective nature of the annotation task.

\paragraph{All annotators prefer Writing Sheet:}
Table~\ref{tab:exp-all-correct} shows an example where all three annotators prefer the Writing Sheet method over the Average Author for similarity to the author story of a Reddit author. The annotators highlight several key aspects that make the Writing Sheet story more aligned with the ground-truth author story: (1) The Writing Sheet method incorporates humor and satire, using phrases such as ``a tale of capes and cereal,'' ``invented breakfast hurricane,'' and ``World's Greatest Hero Faces Greatest Challenge Yet: Parenting,'' which mirror the lighthearted and satirical tone of the author story. (2) Unlike the Average Author story, which concludes with the son’s death, the Writing Sheet story reinterprets the prompt, portraying Titania as an advocate for families with special children. This aligns with the ground-truth author story’s approach, where the son engages with coding and the narrative avoids a sorrowful ending. (3) In terms of language style, the Writing Sheet story employs imaginative elements such as ``The Hero's Journey: A Tale of Capes and Cereal'' and ``The Avalon Gazette,'' which parallel the ground-truth author story's unconventional use of computer programming references, such as ``Syntax Error: Unexpected Power.''

\paragraph{All annotators prefer Average Author:}
Table~\ref{tab:exp-all-wrong} shows an example where all three annotators prefer the Average Author method over the Writing Sheet for similarity to the author story of an AO3 author. The primary reason cited by annotators is that the Writing Sheet method introduces a strong internal conflict for the king, portraying him as guilty or hesitant about sneaking into the kitchen for a late-night snack. His character is depicted as struggling with the decision, given his stature as a formidable ruler expected to lead battles. Additionally, the Writing Sheet creates a sharp contrast between the king and queen, with the queen depicted as joyful and mischievous, encouraging the king to abandon his guilt.  

In contrast, both the ground-truth Author Story and the Average Author story emphasize the playful banter and affectionate dynamic between the king and queen, focusing on the shared experience of sneaking into the kitchen rather than introducing introspection or internal conflict. This shift in focus stems from the Writing Sheet being influenced by the author's past writing history, where internal conflicts are a recurring theme. However, in this case, character development—being more writing-prompt-specific than broader categories like creativity or language use—did not transfer well. The author’s original story prioritizes external actions and lighthearted interactions rather than deep introspection, leading to a mismatch between the Writing Sheet’s personalized adaptation and the intended tone of the new prompt.

\paragraph{All Annotators Choose Tie:}  
Table~\ref{tab:exp-all-tie} shows an example where all three annotators select `Tie' for similarity to the author story of a Storium author. The primary reason is that both the Average Author story and the Writing Sheet story deviate significantly from the ground-truth author story in terms of plot. The ground-truth story adopts an open-ended approach to both character development and plot, where character relationships remain undefined, and the narrative focuses on anticipation rather than action.  

In contrast, the Average Author story follows a conventional narrative arc, featuring fast-paced dialogue and character interactions that build tension and lead to resolution. Similarly, the Writing Sheet story also resolves the narrative but differs from the Average Author method by emphasizing past missions and their consequences, aligning with the author’s tendency to reference character backstories. However, both methods fail to capture the open-ended nature of the original story. This misalignment in plot structure leads all three annotators to select `Tie' for this example.

\paragraph{All Annotators Disagree on the Winner:}  
Table~\ref{tab:exp-all-diff} shows an example where all annotators select different winners for similarity to the author story of a New Yorker author. The ground-truth author story depicts the protagonist’s husband in his final days, exploring familial bonds and the emotional depth of their relationship while infusing a deeper metaphorical meaning into the term “hospice.” The story is rich in subtext and exhibits several distinctive traits of the author, such as cyclic plot structures and the deliberate repetition of phrases like “Adult children, grandchildren. Relatives, friends. Colleagues...” and “Not merely hospice but our hospice. Not sad but joyous, a honeymoon.”

The annotator who selected Writing Sheet as the winner states, \texttt{``Although Average Author and the Author Story depict relationships between close family members, Writing Sheet is more similar to the Author Story in Plot, Creativity, and Development by exploring the meaning and emotional responses to the term hospice.''} This annotator highlights the presence of subtext in the Writing Sheet story and its attempts to mimic the author's cyclic style—“Each door held a story, a life in its final chapter... she would ask the same question...” and “Clara knew that work was not about erasing the pain or the grief, but about finding moments of grace within it...”.

The annotator who selected Average Author as the winner states, \texttt{``Average Author aligns more closely with the Author's Story in terms of Plot and Language Use. This is because both stories center on someone taking care of a family member in their own home, and growing from a lack of peace into a sense of peace, which is expressed through the choice of language. Writing Sheet also indicates a development from lack of peace to acceptance, but the familial tie is what pushes Average Author over the top.''}

The annotator who selected `Tie' states, \texttt{``I think it will be hard for one of these to resemble the Author Story since it was so unique. In all, the two AI stories were much more like each other than the Author Story.''}

Overall, this example highlights the subjective nature of annotator preferences and the inherent difficulty of our annotation task which also explains the poor inter-annotator agreement. 

\end{document}
