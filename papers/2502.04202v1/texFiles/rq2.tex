\subsection{Reporting Performance Bugs Based on \tool's Results}
%\peter{needs to be changed}\wei{Done in this section} 
We applied the \tool to screencasts recorded during real-world mobile app testing, successfully identifying many instances of severe GUI lags with high precision and recall. 
%Most of these lags were confirmed by engineers, achieving a precision of over 0.9 across all lag types. 
These detected lag instances are particularly valuable for reporting performance bugs, as they indicate potential issues and provide developers with actionable insights for debugging and locating performance problems. Below, we outline how we prioritize the severe lags identified by \tool and generate corresponding performance bug reports. %, and share feedback from developers regarding their usefulness.

%These reports highlight the most severe performance bottlenecks that negatively affect the user experience, enabling development teams to allocate resources to the most critical areas.

We prioritize the identified severe lags based on their frequency, duration, user interaction, and visual context to ensure effective ranking, as higher-ranked lags are more likely to be perceived as performance issues. 1) \textbf{Frequency:} Each severe lag is assigned a rank level, with more frequent occurrences receiving a higher rank. For example, janky frames that occur repeatedly are flagged as a high-priority issue due to their potential to disrupt user interactions consistently. 2) \textbf{Duration:} The longer the duration of the severe lags, the higher their rank. For example, load frames that persist for extended periods are prioritized over those with shorter duration. 3) \textbf{User Interaction:} Lags during high-interaction events receive higher priority. Users are less likely to tolerate delays in response to direct interactions, such as scrolling. 
%For instance, users are less likely to tolerate delays in response to direct interactions, such as button taps, than they are during passive actions like background image loading. 
4) \textbf{Visual Context:} We also consider the visual prominence of each lagging element. Loading frames affecting a major visual component on the screen, such as the main content area, are prioritized over those impacting minor or secondary elements. 

%By accounting for both user interactions and visual prominence, \tool’s reports are tailored to the actual user experience, ensuring that the most perceptible issues are marked as critical.  
% This context-sensitive approach allows \tool to capture user expectations accurately, prioritizing lags that are more likely to be perceived as performance issues.
% By distinguishing between these types and severity levels, \tool ensures that the reports it generates highlight the most disruptive and perceptible lags, helping developers quickly identify the areas in need of immediate attention. Lower-severity issues that are unlikely to significantly impact user experience are deprioritized but still included in the report for completeness and future optimization opportunities.

% 5. Prioritizing High-Impact Bugs
% To help teams focus on the most user-centric issues, \tool’s reports emphasize high-impact lags at the top of each report. Lags that exceed thresholds for duration and frequency, particularly those occurring during critical user interactions, are marked as top-priority issues. This prioritization ensures that resources are directed towards performance optimizations that will have the greatest effect on user satisfaction and app responsiveness.

After prioritizing the identified severe lags, we generate comprehensive bug reports that include the following information:
\begin{itemize}
    \item Test Scenario: The specific test case and user interaction that trigger the lag.
    \item Type and Ranking: The lag type (janky, long loading, or frozen frames) along with its ranking.
    \item Frequency and Duration: Metrics on how often the lag occurred and its duration, helping developers to understand the impact and consistency of the issue.
    \item Context: Information on where the lag occurred, including the start and end frames, the user interaction involved, and the affected element.
    \item System and Resource Data: Performance metrics such as CPU and memory usage during the lag to assist in diagnosing underlying causes.
    \item Screencast Segment: A video snippet of the affected frames, enabling developers to visually assess the lag's impact on user experience.
\end{itemize}


%These comprehensive reports ensure that development teams have a clear understanding of the most pressing performance issues, complete with contextual insights and actionable data. By enabling developers to see the specific user interactions and UI elements affected by each lag, \tool’s reports facilitate efficient debugging and targeted performance improvements. 

%To improve efficiency, we also aggregates recurring issues within the same testing session or across multiple CI test runs. For example, if a specific load frame is repeatedly flagged on a particular screen, \tool will group these instances together and report them as a single recurring issue. This helps developers understand whether certain performance problems are isolated incidents or part of a broader pattern that requires systemic changes. Aggregated data also provides insights into trends over time, enabling development teams to track improvements and regressions across versions.

%In summary, \tool's prioritization process is designed to identify the most disruptive GUI lags, aggregating, categorizing, and contextualizing these issues to enable targeted and effective bug fixing. By focusing on high-impact performance issues, \tool helps development teams ensure that their applications meet user expectations for smooth, responsive interactions, enhancing overall app quality and user satisfaction.

We received positive feedback from the developers on the performance bug reports based on the lags. They confirmed that the top-ranked bug reports have a larger impact on user satisfaction. Thus, ranked bug reports help developers focus on the most user-centric issues. The comprehensive information provided, such as the specific test scenario, user interactions, and the frequency of the lags, greatly assists developers in reproducing bugs and identifying their root cause.

%\rqbox{\tool demonstrates high precision in identifying severe GUI lags, which are valuable for reporting performance bugs in downstream tasks. Positive feedback from practitioners confirms that the comprehensive bug reports generated from these lags help developers focus on the most critical issues and debug them efficiently.}
\rqbox{Positive practitioner feedback confirms that the comprehensive bug reports generated from the lags detected by \tool help developers focus on the more critical issues and debug them efficiently.}


\begin{comment}
=============================
In the previous subsection, we demonstrated how \tool effectively detects different types of GUI lags. However, the impact of these lags on user experience can differ significantly. For example, some janky frames that only slightly exceed 100ms threshold may not be noticeable, especially when the app is used in scenarios that do not involve heavy media consumption, such as browsing without video playback. Conversely, prolonged loading times, where frames take as long as five seconds to load, can lead to considerable users frustration. In such cases, users are forced to wait excessively for the content to appear on their screen, disrupting the overall experience and making the issue far more noticeable. Such severe lags are more likely to be perceived as serious performance bugs that directly affects usability. Once \tool identifies these severe GUI lags, we report them to developers, together with detailed information such as the specific test scenarios and user interactions. This enables developers to more effectively reproduce and diagnose the issue.
\end{comment}

\begin{comment}
\begin{table}
	\centering
	\caption{Results of identifying lag alerts by \tool.}
	\label{tab:resutls_detecting_lags}
	\scalebox{1}{
	\begin{tabular}{lrrr}
		\toprule
		lag type&\tabincell{r}{Precision}&\tabincell{r}{Recall}&\tabincell{r}{F1}\\
		\midrule
			Janky frames &	    &       &\\
			Frozen frames&0.4		&0.53	    &\\
			Load frames& 		&		&\\
		\midrule
		\tabincell{l}{Avg. across } 	&		& 		&\\
		\bottomrule
	\end{tabular}}
\end{table}
\end{comment}

%\rqbox{We find that \tool }