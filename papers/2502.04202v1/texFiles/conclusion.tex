\section{Conclusion}
\label{sec:conclusion}

%In this paper, we present \tool, a framework designed to detect and analyze GUI lags in mobile applications. It leverages computer vision techniques to analyze screencasts to detect three types of lag-inducing frames—janky, frozen, and load frames. 
%A threshold-based filter is then used to prioritize the most severe lags that could significantly impact user experience. Our evaluation on top-rated mobile applications demonstrated \tool's high precision and recall in detecting GUI lags. Additionally, the framework has been running stably in a real-world production environment for three months, continuously monitoring GUI performance and successfully identifying critical GUI performance issues.

In this paper, we presented \tool, a novel framework designed to detect GUI lags in mobile applications. \tool identifies user-perceived performance issues, such as janky frames, long loading frames, and frozen frames, by analyzing screencasts directly from the end user perspective. Unlike traditional performance analysis tools that depend on system-level metrics, \tool offers a user-centered approach that aligns more closely with real-world user experiences. 
Through our collaboration with Company A, we successfully integrated \tool into a production environment, enabling continuous monitoring of GUI performance as part of their CI pipeline. The evaluation demonstrated \tool's high precision and recall. We also discuss how \tool helps developers prioritize fixes by providing detailed information on each lag's type, duration, and context.  
We hope this work can inspire practitioners and researchers to adopt more user-centered approaches in their performance testing processes. 

%Furthermore, GUIWatcher’s platform-agnostic design ensures adaptability across different mobile platforms, making it a versatile tool for multi-platform application teams. This adaptability, combined with the user-centered approach to performance monitoring, highlights GUIWatcher as a valuable addition to the toolkit of developers focused on optimizing GUI performance.

%Future work includes exploring enhancements such as machine learning-driven lag detection and integrating user feedback mechanisms to further refine its alignment with user expectations. We hope that GUIWatcher inspires further advancements in user-centric performance analysis, ultimately contributing to smoother, more responsive applications that elevate user satisfaction and engagement.
\section{Acknowledgement}
\label{sec:ack}
We want to thank Company A for providing access to the enterprise systems we used in our case study. The findings and opinions expressed in this paper are those of the authors and do not necessarily represent or reflect those of Company A and/or its subsidiaries and affiliation. Our results do not in any way reflect the quality of Company A's products.

