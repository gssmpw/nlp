\section{Background}
\label{sec:background}
In this section, we first provide background information on mobile screencasts that capture the rendered GUI as videos. 
Next, we discuss the details of three types of GUI lags and how we determine if a lag affects the user-perceived quality.  % in UI rendering that may cause end-users to perceive lag, referred to as lag-inducing frames. Finally, we present the definition of GUI lags as used in this study.

\begin{figure*}
	\centering
    \includegraphics[width=0.9\textwidth]{figures/normal_frame.pdf}
	\caption{A sequence of frames in a recorded screencast, with each frame annotated by index (e.g, $f_{1}$) and PTS.}
	\label{fig_normal_frame}
\end{figure*}

\subsection{Mobile Testing and Screencasts} 
Users expect mobile applications to be smooth and responsive, as performance issues, such as slow loading times or an unresponsive GUI, can significantly degrade the user experience, leading to low app ratings or even uninstallation~\cite{2022_JSS_What_factors_affect_the_UX_in_mobile_apps,2022_TSE_Survey_of_Performance_Optimization}. Consequently, GUI-based mobile testing is essential for ensuring app quality as it reflects actual app usage. Developers typically write test scripts or utilize automated testing tools, such as Monkey~\cite{Monkey}, to simulate user interactions and mimic app usage. After test execution, developers analyze the performance metrics (e.g., CPU usage and memory consumption) and running logs collected during testing to identify potential performance issues. However, these metrics often fail to capture the user’s real experience. For instance, high CPU utilization does not always correlate with perceivable lags, and logs may overlook visual issues. In contrast, screencasts provide a visual recording of the app’s behavior, allowing for a more comprehensive user experience analysis. By analyzing screencasts recorded during automated tests, we can better uncover user-perceived performance issues. 

%In this study, we utilize computer vision techniques to analyze mobile screencasts to detect performance issues, as these recordings offer the most direct insight into how applications behave from the end user’s perspective. 
A mobile screencast is a recording that captures the behavior of applications on a mobile device’s screen in response to a user's actions (e.g., tapping). It consists of a sequence of individual frames, each representing a static UI image at a specific time point, marked by a timestamp. 
Screencasts are often recorded at the same frame rate as the device~\cite{screen_record_frame_rate}. Each time the screen updates, a new frame is captured. For instance, mobile devices typically operate at a standard frame rate of 60 Frames Per Second (FPS)~\cite{Android_frame_rate}, which means that the screencast is also recorded at 60 FPS, with approximately 16.7 milliseconds~\cite{Android_Performance_Patterns} between consecutive frames. 
Therefore, screencast playbacks reflect what users see on the mobile device, and analyzing the screencast can help identify performance issues (e.g., lags) that occur during app usage. 


%meaning 60 frames are rendered per second to ensure smooth visual output. As a result, screencasts are also often recorded at 60 FPS, with an interval of 16.67 milliseconds between consecutive frames (i.e., 60 frames in 1000ms).
%Therefore, playback of the screencast accurately reflects what users see on the mobile device. This screencast can be used to assess the GUI performance of mobile applications.
Figure~\ref{fig_normal_frame} illustrates a screencast composed of 9 frames, each annotated with a timestamp indicating its Presentation Time Stamp (PTS). For example, frame 1 ($f_1$) is shown at timestamp 0 ms, followed by frame 2 ($f_2$) at timestamp 16 ms, and so on, creating a seamless transition that users perceive as smooth application performance. However, if these transitions are not rendered smoothly, they can result in noticeable GUI lags. 
%\peter{we need to discuss the performance overhead here. Find some references, and say, we cannot say how we got these screencasts due to NDA, but common solutions (cite) show that such recording has low overhead.}





\subsection{Three Types of GUI Lags that Indicate Possible Performance Issues} 
In our collaboration with Company A, we uncovered three distinct types of GUI lags that significantly impacted the user experience: janky frames, long loading frames, and frozen frames. 
These performance issues reflect GUI lags and negatively impact user-perceived app quality. 
Every GUI lag is annotated as an interval $[f_{start}, f_{end}]$, where $f_{start}$ and $f_{end}$ denote the start and end frame indices of the lag within the screencast (i.e., a sequence of frames). 
For the lag to be perceptible to end-users, the frames before and after the interval need to exhibit continuous transition, which allows users to perceive the lag in the transition frames. Below, we discuss the three types of GUI lags that we studied. 
%Additionally, the frames before and after the interval need to exhibit continuous transformation. Then, end-users can perceive that the frames lag.

\begin{figure}
	\centering
	\begin{subfigure}[b]{0.45\textwidth}
		\centering
		\includegraphics[width=\textwidth]{figures/jank_frame.pdf}	
		\caption{\texttt{Janky frames between $[f_{2}, f_{3}]$}}
        \label{fig:GUI_perf_jank}
	\end{subfigure}

    \begin{subfigure}[b]{0.45\textwidth}
		\includegraphics[width=\textwidth]{figures/load_frame.pdf}
		\caption{\texttt{Long loading frames between $[f_{2}, f_{4}]$}}
        \label{fig:GUI_perf_load}
	\end{subfigure}

     \begin{subfigure}[b]{0.45\textwidth}
		\includegraphics[width=\textwidth]{figures/frozen_frame.pdf}
		\caption{\texttt{Frozen frames between $[f_{2}, f_{4}]$}}
        \label{fig:GUI_perf_frozen}
	\end{subfigure}

    \caption{Three types of GUI lags.}
    \vspace{-3mm}
	\label{fig:GUI_perf_lags}
\end{figure}

\subsubsection{Janky Frames} 
Mobile OSes like Android render the UI by generating and displaying frames on the screen. Ideally, a mobile app should run at a consistent frame rate of 60 Frames Per Second (FPS). However, if the app experiences slow UI rendering, the system may be forced to skip some frames, causing a recurring flicker on the screen known as jank~\cite{Developers_UI_jank_detection}. This jank can lead to an unstable frame rate and increased latency~\cite{Perfetto_Android_Jank_detection_with_FrameTimeline}, resulting in a bad user experience. 
For instance, as illustrated in Figure~\ref{fig:GUI_perf_jank}, the Android system renders a sequence of frames. After rendering frame 2, the system skips or delays the rendering of subsequent frames. Consequently, at the 33ms and 50ms timestamps, the content of frame 2 remains displayed on the screen, as no new frames are rendered (indicated by the images surrounded by dotted lines). Frame 3 is only rendered and displayed on the screen until the timestamp reaches 66ms. As a result, users may perceive the screen as stuck between 16ms and 66ms timestamps, as the screen display does not change, leading to a noticeable lag. Thus, the frames $[f_{2}, f_{3}]$ are identified as janky frames.


\subsubsection{Long Loading Frames}
Mobile apps need to load essential resources, such as images, to display various content and enhance the user experience. However, if loading and displaying take too long, end-users may experience more challenges as page resources load gradually~\cite{speed_matter}. 
For example, inefficient or improper image loading and display operations can severely impact app performance~\cite{2019_SANER_Characterizing_and_Detecting_Inefficient_Image_Displaying_Issues, 2021_ICSE_IMGDroid_Detecting_Image_Loading_Defects}, potentially leading to long loading frames that affect the user-perceived quality of the app. As shown in Figure~\ref{fig:GUI_perf_load}, an image placeholder begins to appear in frame $f_{2}$ and is not fully filled with an image until frame $f_{5}$, exhibiting three consecutive loading frames. If the number of consecutive loading frames is large, the user may feel the app is unresponsive. %Therefore, the interval $[f_{2}, f_{4}]$ represents the duration of load frames.  


\subsubsection{Frozen Frames}
Responsiveness is a fundamental metric that significantly impacts user experience~\cite{2022_TSE_Survey_of_Performance_Optimization}. However, certain performance anti-patterns, such as accessing data on the UI thread, can cause screen frames to appear sluggish or frozen, adversely affecting users' perception of the app’s performance~\cite{2019_EMSE_iPerfDetector}. These data access operations can be computation-intensive and time-consuming, making the UI unresponsive until the data is retrieved, causing frozen frames. For example, in Figure~\ref{fig:GUI_perf_frozen}, the page scrolls accordingly when the user scrolls up on the screen. However, the frames $[f_{2}, f_{4}]$ remain frozen during this transition, which negatively impacts the user experience and reflects potential performance issues. %Unlike load frames, there are no placeholders on frozen frames.


% \subsection{Abnormal frames}
% Figure~\ref{fig_GUI_perf_abnormal} shows an example of abnormal frames (i.e., f\_{2}).
% \wei{@Peter Abnormal frames are bugs (maybe caused by poor design) but not performance issues. Also, Abnormal frames are not common issues. We only detected about 5 instances among 2K videos. Calculating Precision Recall for this type seems inappropriate. We will not discuss this type in the paper.}

% \begin{figure}
% 	\centering
%     \includegraphics[width=0.6\linewidth]{figures/abnormal_frame.pdf}
% 	\caption{Abnormal frames}
% 	\label{fig_GUI_perf_abnormal}
% \end{figure}

\subsection{Determining if GUI Lags Affect the User-perceived Quality} 
Assessing whether GUI lags impact the user-perceived quality of an app is crucial for maintaining a smooth, responsive experience that satisfies users. However, not all GUI lags are equally noticeable or problematic. Short and minor lags often go unnoticed and do not affect how users perceive the quality of an app~\cite{2015_High_Performance_Android}. Therefore, determining the threshold at which lags become perceptible and affect user satisfaction is key to prioritizing performance improvements.



Research in Human-Computer Interaction (HCI) and user experience design shows that UI feedback should occur within 100ms (approximately six frames at a 60Hz refresh rate) to ensure a smooth experience~\cite{1968_AFIPS_Response_time_in_man_computer, 1994_Usability_Engineering, zippy_Android_apps, 2015_MOBILESoft_performance_parameters}. Therefore, we follow the guideline and set 100ms as the threshold for identifying GUI lags caused by \textit{janky frames} or \textit{frozen frames}. On the other hand, the image loading time in mobile applications can vary significantly depending on factors such as image size, image format, network situation, and other related variables. Hence, we set the threshold for \textit{long loading frames} to one second. %Developers generally aim for image load times under 1 second to maintain a good user experience. In this paper, we define a \textbf{GUI lag} occurs when janky or frozen frames persist for over 100ms, or when load frames exceed 1s, as they can significantly degrade the user experience and create the perception of poor app performance.

