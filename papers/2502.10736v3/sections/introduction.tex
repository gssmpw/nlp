\section{Introduction}
With the increasing availability of consumer-level virtual reality head-mounted displays (HMDs) ~\cite{anthes2016state, li2021social}, Social Virtual Reality (VR) emerges as a promising platform that brings immersive, interactive, and engaging experiences to users for a wide range of collaborative activities. 
Social VR now supports remote conferencing ~\cite{abdullah2021videoconference}, immersive learning ~\cite{thanyadit2022xr, peng2021exploring}, healthcare ~\cite{li2020designing, udapola2022social}, collaborative design ~\cite{mei2021cakevr}, music composing ~\cite{men2022supporting} and even daily activities such as sleeping ~\cite{maloney2020falling} and drinking ~\cite{chen2024drink}.
Among these diverse application areas, social VR helps users overcome physical dispersion, enabling them to communicate and collaborate in the same virtual environment ~\cite{li2021social, palmer1995interpersonal} by offering channels for both verbal and non-verbal communication ~\cite{dzardanova2022virtual, mcveigh2021case, wei2022communication}.
Specifically, text messaging and voice chat are two channels commonly used for verbal communication. 
And pictures ~\cite{li2019measuring}, avatars ~\cite{fu2023mirror, baker2021avatar, kolesnichenko2019understanding, freeman2021body}, and embodied interactions ~\cite{maloney2020talking, wieland2022non} are channels that can convey non-verbal information in social VR.

Yet, these existing communication channels in social VR are preliminary, as they are either borrowed from traditional media (e.g., texts, voices, and pictures) lacking adaptations to the nature of social VR ~\cite{montoya2023wordsphere, rzayev2021reading, dzardanova2022virtual, hsieh2020bridging}, or difficult to be reproduced very well by current VR technologies (e.g., gestures, facial expressions, embodied actions) ~\cite{sykownik2023vr, tanenbaum2020make, wei2022communication}.
Specifically, texts as a communication medium in VR still suffer from difficulties in user inputting ~\cite{montoya2023wordsphere} and readability issues in presentations ~\cite{rzayev2021reading, hsieh2020bridging}. 
Voice-based conversations in VR can easily be disturbed by sound intrusions from the environment ~\cite{akselrad2023body}. 
Furthermore, non-verbal communication mediums, such as gestures and facial expressions, cannot be well-simulated with the avatars in current VR applications ~\cite{baker2021avatar, aseeri2021influence, freeman2021body}, and users may find it hard to achieve specific actions or movements using controllers with joysticks and buttons ~\cite{sykownik2023vr, tanenbaum2020make, li2019measuring}.

Except for social VR, previous research in HCI and CSCW has explored ways to enhance interpersonal communication in different scenarios. Those scenarios include video conferencing ~\cite{liu2023visual, xia2023crosstalk}, augmented reality (AR) ~\cite{lee2023exploring, liao2022realitytalk, leong2022wemoji}, and text messaging interfaces ~\cite{aoki2022emoballoon}. 
These different scenarios demonstrate the power of the integration of both verbal and non-verbal cues in the mediums to support interpersonal communication. 
On the other hand, researchers have also envisioned the potential of social VR in providing situational awareness and enabling new social interactions with novel communication mediums that are unattainable in traditional computer-mediated communication (CMC) platforms ~\cite{mcveigh2021case, mcveigh2022beyond}. 

On top of the prior work, we proposed a new approach to support interpersonal communication in social VR. Our approach is inspired by \textit{Impact Captions}, a type of expressive typographic 2D visual effect with engaging visual elements and animated motions. Impact captions are traditionally used in entertaining television shows and online videos ~\cite{o2010japanese, sasamoto2014impact} to attract viewers' attention and create engaging watching experiences ~\cite{sasamoto2021hookability}. 
Particularly, we introduced impact captions as a communication medium to facilitate interpersonal communication and interactions in social VR. In our design, impact captions encode both verbal and non-verbal information extracted from real-time speech; they also offer interactivity that allows users to intentionally play with the captions for communicative purposes.

To implement and evaluate our impact-caption-inspired approach, we first explored the visual and interaction design space of impact captions through a content analysis on TV show videos.
Based on the design space, we held a co-design session with 4 experts (i.e., HCI researchers) to further derive specific design goals for building a system to demonstrate our idea.
Following the findings so far, we developed \system{}, a proof-of-concept system that takes real-time speech as input to generate interactive impact captions with customized visual appearances in a multi-user VR environment.
Using \system{}, we conducted an in-lab study with 14 participants to assess impact captions as a medium for supporting interpersonal communication in social VR. 

The user study results highlighted the integration of verbal and non-verbal information offered by impact captions, indicating the design can make social VR communication experiences not only clear but also engaging, while the risks of miscommunication caused by ambiguity inherent from the complex visual design were noticed.
Moreover, we found that impact captions can also facilitate interactions between users beyond enhancing speech conversations in social VR.
Furthermore, we illustrated three application scenarios to demonstrate the generalizability of our impact-caption-inspired approach. 
Finally, this research ends up with discussions on the findings from the user study, in which we provided insights and implications for future work and explained the limitations and possible improvements on the scalability of the current \system{} system.

In summary, this research contributes:
\altcolor{
(i) A design space of impact captions for supporting interpersonal communication in social VR;
}
(ii) A proof-of-concept system, \system{}, that enables users to have real-time voice-based conversations with the support of interactive impact captions in social VR;
(iii) An evaluation study of \system{} that demonstrates the effectiveness of our impact-caption-inspired approach for communication, highlighting the value of employing creative mediums for supporting interpersonal communication in social VR.


