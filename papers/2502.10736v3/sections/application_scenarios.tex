\section{Application Scenarios}
\label{sec_apps}
Besides the scenarios and potential creative use cases committed in the user study, \system{} further supports a variety of communicative needs under different contexts. We briefly introduce the following additional application scenarios to illustrate the generalizability of \system{}.

\subsection{Making Social VR Accessible to Deaf and Hard-of-Hearing (DHH) People}

Recent research revealed that current VR applications fail to provide sufficient accessibility support for deaf and hard-of-hearing (DHH) people to experience immersive digital content or engage in remote communication and socialization ~\cite{jain2021towards, borna2024applications}.
Captioning systems, widely used to enhance accessibility in other digital media such as videos, remain under-explored in VR contexts ~\cite{kim2023visible, de2023visualization, de2024caption}.
\system{} not only has similar functionality to previous systems that convert speech into visible captions for DHH users ~\cite{kim2023visible, de2023visualization, li2022soundvizvr}, but also extends this concept by integrating visual cues and interactivity to convey rich non-verbal information (e.g., valence, excitement).

For example, in a conversation about a vacation plan in social VR with \system{}, a DHH user can discern the speaker’s positive mood through bright-colored captions accompanied by a laughing emoji. A shivering motion further indicates the speaker’s excitement about the trip. 
By dragging and placing impact captions of the names of the must-visit spots mentioned in speech, the speaker can demonstrate the plan with clarity. 
Additionally, when a speaker picks up a caption labeled ``volcano'' while discussing hiking and exploration, the corresponding icon reinforces the speech context.
Overall, informative captions, engaging visuals, and interactive features in \system{} expand opportunities for DHH individuals to connect with others and engage with the virtual world in social VR


\subsection{Enhancing Interactions for Teaching and Learning in Social VR}
VR has shown great potential in education by enhancing social presence in remote learning and providing access to learning contexts otherwise unavailable in reality ~\cite{thanyadit2022xr, peng2021exploring, jensen2018review}. However, it remains far from mainstream adoption due to challenges in creating easily reviewable educational content, ensuring inclusivity, and supporting collaborative learning ~\cite{jin2022will}.   
In virtual lectures, \system{} enables instructors to create instructional cues by placing keyword-based impact captions in the virtual space. Shared interactive captions facilitate interaction between instructors and students, supporting activities like Q\&A and hands-on demonstrations.

For example, when teaching the structures of plants in a botany class, the instructor would usually introduce the roots, stems, leaves, and flowers in sequence. With \system{}, once these words were mentioned, the relevant impact captions would be generated with texts accompanied by symbolic icons.
Then, the instructor can grab and stretch to enlarge the caption for emphasizing, and further captures the students' attention by shaking the caption to trigger the blinking effect.
If a student has questions about a specific concept mentioned before, he could also make an impact caption of words for that concept and shoot the words, creating an explosion of the words so that the instructor can easily know which part the student is confused about. 
In addition, further use cases can be explored to adapt to different contexts in teaching and learning. \system{} in the current stage provides a starting point to a way to achieve engaging and playful interpersonal interactions for social VR users in the future.


\subsection{Facilitating Engaging Live Streaming Experiences in Social VR}
Recent research suggests that VR streamers face challenges in building emotional connections, as VR headsets obscure facial expressions, making it harder for viewers to perceive their emotions directly ~\cite{wu2023interactions}, while emotional connection is a key factor in live streaming ~\cite{lu2018you}.
\system{} visualizes emotions for streamers by detecting moods in speech and adding appropriate emojis and motion effects to captions, and it also enhances communication by rendering viewers’ reactions as interactive captions in the streamer’s virtual space, fostering engagement.
These features of \system{} address the need for visible and spatialized objects, creating more engaging live streaming experiences ~\cite{wu2023interactions, lu2018you}.

For example, in a typical VR live streaming scenario, a streamer was self-evaluating his playing performance on Beat Saber, a popular VR rhythm game. He didn't do well in the game but would like to tell the audience that he had a lot of fun playing it and felt like a champion. This strong emotion can be communicated and augmented by \system{} with colored captions of words like ``Fun'' and ``Exciting'' accompanied by laughing emojis. 
He then made a ``Champion'' caption with a trophy icon and attached it to his avatar's head like a funny crown to show self-mockery.
By shooting the caption ``Exciting'' outwards, the explosion motion effect can be triggered once the caption crashes with an object in the VR scene, making the atmosphere more interesting and engaging. From the viewers' perspective, they can also send their feedback and comments as impact captions, using the visible and interactive captions to make humorous responses to the streamer's poor performance.


% \subsection{Enhancing the Feeling of Intimacy between existing close-ties}
% Social VR offers a new way to mediate and support interpersonal relationships – it leverages both the anonymity and fexibility of virtual spaces and the physical and bodily experiences in offine face-to-face interactions, making the development of such relationships more 
% immersive and realistic.\cite{freeman2021hugging}. However, 
% \subsection{Relieving the Uneasiness of Interacting with Strangers in social VR}


