\section{Co-designing an Impact-Caption-Inspired Communication Tool in Social VR}
To operationalize our impact-caption-inspired approach for interpersonal communication in social VR, we organized a co-design activity to obtain and refine the design goals for build a proof-of-concept system.

\subsection{Procedure}
The procedure consists of two sub-sessions. The first focused on brainstorming and open-ended discussion, and the second aimed at paper prototyping for the proof-of-concept system.

\subsubsection{Participants}
Besides the four authors of this paper, we recruited four extra HCI researchers from our university to this session. During the procedure, the first author played the role of instructor to host the sessions and also participated in the activities to contribute ideas. 
The eight contributors are either developers or designers of digital products. They all have rich experiences in VR gaming and social VR (e.g., VRChat\footnote{https://hello.vrchat.com/}).

\subsubsection{Sub-session 1: Brainstorming and Discussion}
In the first sub-session, the objective was to understand the scenarios, needs, and purposes of users' interpersonal communication behaviors in social VR. 
This session was followed by an initial brainstorming and discussion activity in which the participants shared their own experiences of interpersonal communication in VR to identify the potential pain points and possible improvements. 
The instructor then presented several examples of impact captions (sourced from the sample videos for investigating the design space) to illustrate the concept and potential usages. 
With the inspiration of impact captions, next the participants presented their views on how the captions can be applied to improve the communication experience in social VR.
The results of this session were recorded and summarized by the instructor.

\subsubsection{Sub-session 2: Prototyping and Presentation}
After the brainstorming and discussion, the goal of the second sub-session is to create low-fidelity prototypes so that the previous ideas can be visualized and demonstrated in more concrete forms.
At the beginning of this sub-session, the instructor led a review of the results of the previous session and then informed the participants that they had 30 minutes to create their own prototype using white paper by drawing or handcrafting. Participants were encouraged to discuss their ideas with others and to think aloud during the session, while they still needed to create prototypes individually. After the creation was completed, the participants presented their prototypes to the group and developed open-ended discussions on the prototypes.

% \begin{figure}[htb]
%     \includegraphics[width=\linewidth]{img/co_design.png}
%     \caption{}
%     \label{fig:co_design}
% \end{figure}


\subsubsection{Data Analysis} 
We recorded the entire procedure of the co-design activity with the permissions of the participants, transcribed oral discussions into texts, and collected the prototypes they designed. 
Using the method proposed by Chen and Zhang ~\cite{chen2015remote}, three authors worked as coders to conduct open coding and categorization of the design prototypes. 
In particular, components involved in the design of each prototype (such as expected input and output), information flow, context, and other visual elements (such as present style, text content, and colors) were coded by referring to related work about augmenting presentation and communications ~\cite{liu2023visual, liao2022realitytalk, de2024caption}.
Finally, we revealed qualitative findings regarding how impact captions can support interpersonal communication and derived design goals for a prototype system. 

\subsection{Findings}
Through the co-design activity, we identified three aspects of interpersonal communication in which impact captions can help.

\subsubsection{Enhancing Emotional Expression}
Emotions are important in communication but sometimes it could be implicit and inconspicuous. Impact captions can be a way to materialize human emotions into visible and tangible shapes in social VR. With impact captions, information senders can share their emotional feelings in visible and touchable forms so that the receivers can perceive the sender's moods and intentions with multi-modal perceptions beyond verbal signals and body movements.
% Based on the design space and prototypes, the color and typeface of textual elements can be designed differently for words that indicate different valence. And the non-textual emojis and the shapes of speech bubbles integrated in impact captions can also convey emotional reactions and excitement, similar to the situations of social media and instant messaging interfaces ~\cite{aoki2022emoballoon, lo2008nonverbal}.

\subsubsection{Highlighting Key Information in Speech}
Impact captions can be used to highlight key information in a conversation. Unlike normal captioning system that aims to display all the words in a speech, impact captions should be made for words that conveys key information.
Additionally, typeface, color, and text size are useful dimensions to enhance the visual attractiveness of impact captions to make the key information visible and outstanding in a lengthy speech in VR.

% Additionally, simpler and more regular fonts are more readable compared to overly complex artistic typefaces. In practical scenarios, multiple visual dimensions can be used individually or simultaneously to emphasize key information while aligning with the artistic style of the context.

\subsubsection{Identifying Speakers}
Although spatial audio allows users of social VR to trace the source of sounds and identify speakers, it still lacks accuracy and explicit signals, especially when multiple people speak simultaneously ~\cite{yan2023conespeech}.
Impact captions, with their positioning or speech bubble identifiers, can help users to easily identify the source of speech or sound through visual cues. This can be a complement to the audio cues to enrich the communication experience in social VR.

% Furthermore, when impact captions generated during conversations are allowed to persist in the environment for a prolonged period, newcomers can also see previous conversation content and gauge the level of activity of other participants based on the number of impact captions.



\subsection{Design Goals}
\label{section_design_goals}
Based on the qualitative findings from the co-design activity, we propose the following design goals to implement a proof-of-concept tool to support interpersonal communication in social VR through impact captions.

\subsubsection{G1: Generating Impact Captions in Time}
It was agreed by the participants that the impact captions should be generated when users were speaking, as voice-based ``face-to-face'' conversation is the most common way to interpersonal communication in social VR. Ideally, impact captions should appear simultaneously with the speech.

\subsubsection{G2: Maintaining Readability and Clarity}
Once impact captions were generated, a key aspect in making them useful for communication is to maintain readability and clarity so that other users in the VR space can clearly perceive the information carried out by impact captions. This requires each impact caption to appear in a proper location with a correct facing direction. In addition, multiple impact captions should not overlap each other.

\subsubsection{G3: Enabling Interpersonal Interactions with Playful Impact Captions}
Interactive impact captions could enhance or introduce new forms of interaction between users, making these captions a type of new medium that facilitates connections among multiple users in social VR. 
Taking advantage of the interactivity capabilities provided by VR techniques, users should be able to intuitively handle and play with the impact captions generated by themselves. 

