

%%%%%%%%%%%%%
\section{Learning Causal Representations for CATE}

\label{sec:theory}
Our task involves learning four functions: $\RInvhat_t$ that extracts the causal representations from $X(t)$ and $\realmuhat_t$ that estimates the outcomes $Y(t)$ for $t \in \set{0,1}$. With access to \textit{counterfactual simulated} data $\synD$ and \textit{observational real} data $\trnD$, one can come up with the following approaches for estimating CATE: 1) \simonly, which only uses $\synD$, and
2) \realonly\, which only uses $\trnD$ to estimate $\realmu_t$,
\textit{(3)} \muonly, which uses $\synD$ to estimate $\RInv_t$ and subsequently, $\trnD$ to estimate $\realmu_t$. We now discuss the training approach for each of these methods, delve into their shortcomings, and then present our proposed method \our. 


To illustrate the shortcomings, we consider a test instance $\xb^\star$ generated under treatment $T=1$ (without loss of generality) and derive the CATE error expression for it in the population setting, as $|\trnD| \rightarrow \infty$ and $|\synD| \rightarrow \infty$.

% , \textit{(4)} \our, our proposed estimator, which uses both $\synD$ and $\trnD$.  


\subsection{The \simonly\ Estimator}
\simonly\ solely uses $\synD$. It leverages the counterfactual supervision provided by the simulator and identifies the simulator's DGP as follows:

(Step 1) Estimate the synthetic causal representation extractor $\SInv$ from covariate pairs $\{\xb_i^S(0), \xb_i^S(1)\}$ using contrastive learning \cite{von2021self}:
\begin{align}
    \{\SInvTilde_0, \SInvTilde_1\}  = \argmin_{\{\SInvhat_0, \SInvhat_1\}}\; \sum_{i=1}^{|\synD|}\left[- \log{\frac{\exp({\textrm{sim}(\hat{z}_i(1), \hat{z}_i(0)})}{\sum_{j \neq i}\sum_{t, t'} \exp({\textrm{sim}(\hat{z}_i(t), \hat{z}_j(t'))})}}\right]~~\text{where}~~\hat{z}_i(t) = \SInvhat_t(\xb^S_i(t))
    \label{eq:cont_loss}
\end{align}


where $\textrm{sim}(\bullet, \bullet)$ is cosine similarity, $(\xb^S_i(t), \xb^S_i(1-t))$ denotes a positive pair with the same underlying latent $z_i$. A negative pair $(\xb^S_i(t), \xb^S_j(t'))$ has different $(z_i, z_j)$. Contrastive learning increases similarity of representations of positive pairs $(\hat{z}_i(0), \hat{z}_i(1))$ while pushing apart the negative pairs $(\hat{z}_i(t), \hat{z}_{j \neq i}(t'))$.


\begin{lemma}
    As $|\synD| \rightarrow \infty$, contrastive training with paired counterfactual covariates as shown in Eq. \ref{eq:cont_loss} recovers $\SInvTilde_t = h \circ \SInv_t$ where $h$ is a diffeomorphic transformation. Moreover, when the latent space $\Zspace \subset \mathbb{S}^{(n_z - 1)}$ (unit-norm hypersphere in $\RR^{n_z}$), $h$ is a rotation transform by Extended Mazur-Ulam Theorem as shown in~\citep{zimmermann2021cl} (Proposition 2).
    \label{app:lemma:cl_rotation}
\end{lemma}
\textit{[Refer Appendix \ref{app:sec:mazurulam} for more details.]} 
% \todo[inline]{L: We are not replicating the proof. But we are trying to explain verbose how that proof fits in our CATE context.}

The main insight from the above lemma is that, given counterfactual supervision, it is possible to recover causal representations $Z$ from post-treatment covariates $X$ up to a rotation $h$, making CATE identifiable in the simulated distribution, as we demonstrate below. 

(Step 2) Estimate $\syntauTilde(z) = \synmuTilde_1(z) - \synmuTilde_0(z)$ with supervision on difference of outcomes $\syntau(\SInv_t(\xb_i^S(t))) = \y_i^S(1) - y_i^S(0)$ as 
\begin{align}
    \syntauTilde = \argmin_{\widehat{\syntau}} \sum_{\xb^S \in \synD} \left[\syntau(\SInv_t(\xb^S(t))) -  \widehat{\syntau}(\SInvTilde_t(\xb^S(t)))\right]^2 \label{eq:syntautilde}
\end{align}



 % Thus, $\tauhat = \syntauTilde, \RInvhat = \SInvTilde$, resulting in the following CATE estimate:
%
%
The \simonly\ method uses these estimates as-is on real data, i.e. $\tauhat = \syntauTilde$ and $\RInvhat_t = \SInvTilde_t, ~~\forall t \in \Tspace$. We analyze below the error incurred with such estimates on real data. 

\textbf{CATE error:} In the population setting, since $\SInvTilde_t = h \circ \SInv_t$, we see that the optimization problem in Eq.~\ref{eq:syntautilde} yields $\syntauTilde = \syntau \circ h^{-1}$ as its solution. 
% we observe that $\syntauTilde$ in Eq.~\ref{eq:syntautilde} equals $\syntau \circ h^{-1}$ in the limit of simulator samples. 
Thus, for an instance $\xb^\star$ from the real distribution under treatment 1, the true CATE is $\tau(\RInv_1(\xb^\star))$. The CATE error using \simonly\ becomes:

$$
\left[ \tau(\RInv_1(\xb^\star)) - \syntau \circ h^{-1}(h \circ \SInv_1(\xb^\star)) \right]^2 = \left[ \tau(\RInv_1(\xb^\star)) - \syntau(\SInv_1(\xb^\star)) \right]^2.
$$

This shows that for \simonly\ to provide accurate CATE estimates, the simulator must perfectly align with the real world; i.e., $\syntau = \tau$ and $\RInv_t = \SInv_t$ for all $t$. However, designing such simulators is highly challenging in practice, making this method unsuitable for CATE.




%



\subsection{The \realonly\ Estimator}
\realonly\ solely uses real observational data $\trnD$. Since $\trnD$ lacks counterfactual covariates, this model cannot apply contrastive training and therefore cannot explicitly supervise the recovery of causal representations. Instead, it focuses on regressing the factual outcomes $y_i(t_i)$ from post-treatment covariates $\xb_i(t_i)$. In terms of the four learning parameters, its learning objective is:
$$
    \argmin_{\{\realmuhat_0,\realmuhat_1 , \RInvhat_0, \RInvhat_1\}} 
    \sum_{i=1}^{|\trnD|}(y_i - \realmuhat_{t_i}(\RInvhat_{t_i}(\xb_i)))^2
$$
However, since $\realmuhat_t, \RInvhat_t$ are not individually supervised, we might as well collapse them into a composition $\realmu_t^F= \realmu_t \circ \RInv_t$; yielding $\realmu_{t_i}^F(\xb_i) = y_i$, and thereby CATE as $\ITExhat(\xb, t) = \realmuhat^F_1(\xb) - \realmuhat^F_0(\xb)$. 

\realonly\ is consistent in estimating the factual outcomes, because as $|\trnD| \to \infty$, we have
$
     \realmuhat_t^F = \argmin_{\realmuhat^F_t} \EE_{\xb \sim P(\xb|t)}\left[\left(\realmuhat^F_t(\xb) - \realmu^F_t(\xb)\right)^2\right] = \realmu_t^F \text{ and therefore, the factual error } \ErrF^t = 0
$.
% \end{align*}
However, \realonly\ incurs a significant error when estimating the counterfactual outcome, which in turn contributes to the CATE error, as shown below.



\textbf{CATE error:} The true CATE for $\xb^\star$ obtained using treatment $1$ can be written as $\tau(\RInv_1(\xb^\star)) = \realmu_1(\RInv_1(\xb^\star)) - \realmu_0(\RInv_1(\xb^\star))$. Then, the CATE error for \realonly\ is computed as:
$$
\left[\big(\realmu_1(\RInv_1(\xb^\star)) - \realmu_0(\RInv_1(\xb^\star))\big) - \big( \realmuhat_1^F(\xb^\star) - \realmuhat_0^F(\xb^\star) \big)\right]^2 = \left[\big(\realmu_1(\RInv_1(\xb^\star)) - \realmuhat_1^F(\xb^\star) \big) - \big( \realmu_0(\RInv_1(\xb^\star)) - \realmuhat_0^F(\xb^\star) \big)\right]^2 
$$

In the population setting, using the consistency of factual estimates, the CATE error reduces to $\left[ \realmu_0(\RInv_1(\xb^\star)) - \realmu_0^F(\xb^\star)\right]^2$. This error is zero when $\RInv_1(\xb^\star) = \RInv_0(\xb^\star)$. Thus, for \realonly\ to provide accurate CATE estimates, the treatment must not affect the post-treatment covariates, i.e., $g_0(z) = g_1(z)~ \forall z$ in which case their inverse are equal $f_0 = f_1$. However, this assumption is often unrealistic. For instance, in pharmacology, different drugs typically induce distinct effects on patient covariates, limiting the applicability of this model in such settings.

\textbf{Remark:} The post-treatment covariates $X$ can be viewed as a special case of the pre-treatment covariates $Z$ when the covariate-generating functions $\R_0 = \R_1$. In such cases, our proposed \our\ algorithm offers no distinct advantage, and existing CATE methods designed for pre-treatment covariates suffice and should be used instead.


\subsection{The \muonly\ Estimator}
\vspace{-0.2cm}
% We posit that the $(\RInv_0, \RInv_1)$ gap would be much larger than the $(\RInv_t,\SInv_t)$ gap for a good simulator.
Unlike \simonly, which uses $\synD$ to learn both $\RInvhat_t$ and $\realmuhat_t$, this approach leverages $\synD$ solely to learn the representation extractor $\RInvhat_t$. Specifically, it assumes that $\RInvhat_t = \SInvTilde_t$, as obtained from Eq. \ref{eq:cont_loss}.
% , \forall t \in \Tspace$. which is derived from $\synD$, to estimate pre-treatment covariates  $\hat{z}_i=\SInvCont_{t_i}(\xb_i)$. 
Thereafter, it learns the $\realmuhat_t$ parameters by applying a factual loss on $\trnD$ to estimate 
$$
\realmuhat_0, \realmuhat_1 = \argmin_{\{\realmuhat_0, \realmuhat_1\}} \sum_{\trnD}(y_i - \realmuhat_{t_i}(\SInvCont_{t_i}(\xb_i)))^2
$$
We call this method \muonly\ since it learns the outcome parameters $\realmu$ from real samples while learning representation extractor $\RInv_t$ from the simulator.

\textbf{CATE error:} One condition under which the  \muonly\ model achieves zero CATE error is when $\SInvTilde_t = \RInv_t$ for each treatment $t$. This requires that the simulator aligns with real-world covariates, specifically $\xb_t = g_t(z) = g_t^S(z) = \xb_t^S$. This limitation arises because the model learns the representation extractor solely from $\synD$, without making adjustments for real covariates. 


In summary, we described three possible CATE estimators and showed that each method would provide accurate CATE estimates under certain strong assumptions about the real and simulator DGPs. Given that none of these assumptions would hold in practice, we now turn to exploring a joint training framework that learns simultaneously from both real and simulated samples. 






\subsection{The \our\ Estimator}
\label{sec:simponet}

We first conduct a theoretical analysis to derive a generalization bound that characterizes the CATE error as a function of the mismatch between the real and simulator distributions. This analysis forms the basis for our proposed method, \our, whose loss function is inspired by the bound.

\begin{filecontents*}{lemmasimponetbound.tex}
Assume $\tau$ is $K_{\tau}$-Lipschitz, and $\SInvTilde$ and $\syntauTilde$ are estimates from the simulator DGP obtained from the optimization in Eq. \ref{eq:cont_loss}, \ref{eq:syntautilde}. Then, the CATE error on the estimates $\RInvhat_t$ and $\tauhat$ admits the following bound: 
    \begin{align*}
      \ErrITE^t(\RInvhat_t, \tauhat)  \leq [8\ErrF^t + 12d_{h}(\tauhat, \syntauTilde) + 12 K_{\tau}^2 \,d_{\xb|t}(\RInvhat_t,\SInvTilde_t)] + \textcolor{blue}{[12 d_z(\tau, \syntau) + 12 K_{\tau}^2 \,d_{\xb|t}(\RInv_t,\SInv_t)]}
\end{align*}
 where $d_{\xb|t}, d_z, d_{h(z)}$ are distance functions in Sec.~\ref{sec:problem_formulation} and $\ErrF^t$ is the factual loss.\end{filecontents*}
\begin{lemma}
\label{lemma:simponetbound}
\input{lemmasimponetbound}
 [Proof in Appendix~\ref{app:lemma:simponetbound}.]
\end{lemma}
The expressions in \textcolor{blue}{blue} are constants that capture the discrepancy between real and simulated distributions and cannot be minimized. In contrast, the remaining terms can be minimized by training on $\trnD$ and $\synD$. As $|\trnD|$ approaches infinity, the factual error $\ErrF^t$ can be made to approach zero, while the other minimizable distance terms act as regularizers. The term $d_{h}(\tauhat, \syntauTilde)$ can assist in regularizing the outcome parameters $\realmuhat_t$, whereas $d_{\xb|t}(\RInvhat_t, \SInvTilde_t)$ can aid in regularizing the parameters of the causal representation extractor functions $\RInvhat_t$. This analysis leads to our proposed approach \our\ whose overall loss is as follows:



\begin{equation}
\label{eq:overall_objective}
\resizebox{\textwidth}{!}{%
  $\begin{aligned}
    \min_{\{\realmuhat_t, \RInvhat_t\}} 
     & {\underbrace{%
        \sum_{\trnD}\left(y_i - \realmuhat_{t_i}(\RInvhat_{t_i}(\xb_i))\right)^2
        }_{\text{Factual Loss on $\trnD$}}}  + 
     {\underbrace{%
          \Lphi \sum_{\trnD} \Vert \SInvCont_{t_i}(\xb_i) - \RInvhat_{t_i}(\xb_i) \Vert_2^2
        }_{d(\SInvCont_t, \RInvhat_t) \text{ regularizer}}}  +
     {\underbrace{%
        \Ltau \sum_{\synD} \sum_{t \in \{0,1\}}
        \left(\tau^S_i -  \widehat{\tau}(\SInvCont_{t}(\xb_i^S(t))) \right)^2
        }_{
        \tau^S \text{ regularizer on }\synD
        }} 
    \end{aligned}$ 
    }
\end{equation}
where $\tau^S_i = y_i^S(1) - y_i^S(0)$ and $\Ltau, \Lphi > 0$ are loss weights. $\widehat{\tau}(\bullet) = \realmuhat_1(\bullet) - \realmuhat_0(\bullet)$ denotes the estimated CATE.

\our\ relaxes the strict equality $\RInvhat_t = \SInvTilde_t$ used by \muonly, and instead uses $\SInvTilde_t$ as a regularizer, while ensuring that $\realmuhat_t$ accurately predicts the factual outcomes for instances in $\trnD$. It also imposes the $\syntau$ loss on simulated instances to leverage any potential similarity between the true treatment effect, $\tau$, and the simulated treatment effect, $\syntau$. Furthermore, the $\tau^S$ loss is essential to prevent degenerate solutions that would cause \our\ to collapse to the \muonly\ estimator. 
% Note that $\ITEx = \tau \circ \RInv_t$ has two degrees of freedom in $\tau$ and $\RInv_t$. 
This is because applying regularization solely on $\RInvhat_t$ can drive the regularizer $||\RInvhat_t(\xb) - \SInvTilde_t(\xb)||_2^2$ to zero, leading to $\RInvhat_t = \SInvTilde_t$, while still minimizing the factual error $\ErrF^t$ by updating $\realmuhat_t$ accordingly. Consequently, \our\ would collapse into the \muonly\ estimator, making the $\tau^S$ loss critical in avoiding such degeneracies.


\xhdr{Adjusting Loss Weights} \our\ adjusts the loss weight $\Lphi$ for learning $\RInvhat_t$ by comparing the \textit{factual errors} of the \realonly\ model, which trains on $X$, with those of the \muonly\ model, trained using simulated causal representations $\SInvCont_t(\xb)$. If \realonly\ consistently outperforms \muonly\ in factual error, we infer that the simulated representations may not generalize well to the real distribution, prompting \our\ to reduce $\Lphi$. By default, $\Lphi$ is set to $1$; however, if \muonly\ exhibits a notably higher factual error, \our\ lowers $\Lphi$ to $10^{-4}$.


In contrast, tuning $\Ltau$ requires $\tau$ supervision on real data, which is unavailable. Prior work~\citep{inducbias, pairnet, xlearner} argue that while outcome functions $\realmu_t$ can be complex, the difference function $\tau = \realmu_1 - \realmu_0$ is often simpler. For instance, if we consider $\realmu^S_t = \realmu_t + c$ (with $c > 0$), we can make the factual outcomes to diverge arbitrarily while their corresponding $\tau$ and $\tau^S$ remain equal. Therefore, while comparing the factual errors between \simonly\ and \realonly\ models to set $\Ltau$ is appealing, it maybe a poor choice in practice. So, \our\ always sets $\Ltau$ to its default $1$.
% We can view \our\ as a multi-task model where it predicts $\tau, \tau^S$ and an adequately parameterized neural network can handle both tasks. \todo{What if there is -ve interference?}


We present the \our's pseudocode in Appendix \ref{app:pcode}.
