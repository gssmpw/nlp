\section{Related Work}
\label{sec:related}

Recognizing rare words in Automatic Speech Recognition (ASR) systems remains a significant challenge due to their infrequent occurrence in training data. Various strategies have been proposed to address this issue, including post-processing methods, meta-learning approaches, and the integration of Large Language Models (LLMs).

Post-processing techniques focus on error detection and context-aware correction to enhance rare word recognition. For example, He et al. introduced an ASR post-processing method that targets predicted error positions and leverages a rare word list to provide additional contextual knowledge, resulting in improved recognition of rare words~\cite{he2023edcec}.

Meta-learning approaches have also been explored to enable few-shot adaptation for rare word recognition in end-to-end ASR systems. Lux and Vu proposed a method that generates meaningful embeddings for speech and adapts meta-learning algorithms to perform keyword spotting in continuous signals, thus improving the recognition of rare words~\cite{lux2021metalearning}.

The integration of LLMs into ASR systems has gained attention as a means to enhance transcription accuracy, particularly for rare words. For instance, Pu et al. \cite{pu2023multistage} proposed an approach that utilizes the reasoning capabilities of LLMs to improve transcription accuracy. Yang et al. proposed a CTC-assisted LLM-based contextual ASR model that uses coarse CTC decoding results to filter potential relevant hotwords and incorporate them into LLM prompt input, demonstrating significant improvements in recognizing rare long-tail words~\cite{yang2024ctcassisted}. Additionally, Min and Wang explored the potential of using LLMs' in-context learning capabilities to enhance ASR performance, though their findings indicated challenges in leveraging LLMs for error correction in speech recognition transcriptions~\cite{min2023exploring}.

Despite these advancements, challenges persist in effectively integrating LLMs with ASR systems to improve rare word recognition. This study aims to build upon existing research by investigating the roles of speech encoders and LLMs in determining overall Word Error Rate (WER) performance and rare word recognition, utilizing a large-scale dataset to provide empirical evidence for the efficacy of LLM-ASR integration.