%\documentclass[gray]{jmlr} % test grayscale version
%\documentclass[tablecaption=bottom]{jmlr}% journal article
\documentclass[pmlr,twocolumn,10pt]{jmlr} % W&CP article

% The following packages will be automatically loaded:
% amsmath, amssymb, natbib, graphicx, url, algorithm2e

%\usepackage{rotating}% for sideways figures and tables
%\usepackage{longtable}% for long tables

% The booktabs package is used by this sample document
% (it provides \toprule, \midrule and \bottomrule).
% Remove the next line if you don't require it.

\usepackage{booktabs}
% The siunitx package is used by this sample document
% to align numbers in a column by their decimal point.
% Remove the next line if you don't require it.
%\usepackage[load-configurations=version-1]{siunitx} % newer version 
\usepackage{siunitx}

% The lineno package is required for denoting line
% numbers for paper review.
\usepackage[switch]{lineno}

\usepackage{bm}
\usepackage{multirow}
\usepackage{multicol}
\usepackage{makecell}
\usepackage{lipsum}

% The following command is just for this sample document:
\newcommand{\cs}[1]{\texttt{\char`\\#1}}% remove this in your real article

% The following is to recognise equal contribution for authorship
\newcommand{\equal}[1]{{\hypersetup{linkcolor=black}\thanks{#1}}}

\newcommand{\ours}{LabTOP}
\newcommand{\oursfull}{Lab Test Outcome Predictor}

\newcommand{\todo}[1]{\textcolor{red}{TODO:~#1}}

% Define an unnumbered theorem just for this sample document for
% illustrative purposes:
\theorembodyfont{\upshape}
\theoremheaderfont{\scshape}
\theorempostheader{:}
\theoremsep{\newline}
\newtheorem*{note}{Note}

% change the arguments, as appropriate, in the following:
% \jmlrvolume{LEAVE UNSET}
% \jmlryear{2025}
% \jmlrsubmitted{LEAVE UNSET}
% \jmlrpublished{LEAVE UNSET}
\jmlrworkshop{Submitted to Conference on Health, Inference, and Learning (CHIL) 2025} % W&CP title

% The optional argument of \title is used in the header
% \title[Short Title]{Full Title of Article\titlebreak This Title Has A Line Break}
\title[\ours{}]{LabTOP: A Unified Model for Lab Test Outcome Prediction\titlebreak
on Electronic Health Records}
% title에서 잘 드러나야 할 것들
% EHR 사용한다는 거
% language modeling (generative) 방식으로 접근했다는 거

% Anything in the title that should appear in the main title but 
% not in the article's header or the volume's table of
% contents should be placed inside \titletag{}

%\title{Title of the Article\titletag{\thanks{Some footnote}}}


% Use \Name{Author Name} to specify the name.
% If the surname contains spaces, enclose the surname
% in braces, e.g. \Name{John {Smith Jones}} similarly
% if the name has a "von" part, e.g \Name{Jane {de Winter}}.
% If the first letter in the forenames is a diacritic
% enclose the diacritic in braces, e.g. \Name{{\'E}louise Smith}

% \thanks must come after \Name{...} not inside the argument for
% example \Name{John Smith}\nametag{\thanks{A note}} NOT \Name{John
% Smith\thanks{A note}}

% Anything in the name that should appear in the title but not in the 
% article's header or footer or in the volume's
% table of contents should be placed inside \nametag{}

% Two authors with the same address
% \author{%
%  \Name{Author Name1\nametag{\thanks{A note}}} \Email{abc@sample.com}\and
%  \Name{Author Name2} \Email{xyz@sample.com}\\
%  \addr Address
% }

% Three or more authors with the same address:
\author{%
 \Name{Sujeong Im}\equal{These authors contributed equally} \Email{sujeongim@kaist.ac.kr}\\
 \Name{Jungwoo Oh}\footnotemark[1] \Email{ojw0123@kaist.ac.kr}\\
 \Name{Edward Choi} \Email{edwardchoi@kaist.ac.kr}\\
 % \Name{Author Name4} \Email{an4@sample.com}\\
 % \Name{Author Name5} \Email{an5@sample.com}\\
 % \Name{Author Name6} \Email{an6@sample.com}\\
 % \Name{Author Name7} \Email{an7@sample.com}\\
 % \Name{Author Name8} \Email{an8@sample.com}\\
 % \Name{Author Name9} \Email{an9@sample.com}\\
 % \Name{Author Name10} \Email{an10@sample.com}\\
 % \Name{Author Name11} \Email{an11@sample.com}\\
 % \Name{Author Name12} \Email{an12@sample.com}\\
 % \Name{Author Name13} \Email{an13@sample.com}\\
 % \Name{Author Name14} \Email{an14@sample.com}\\
 \addr KAIST, Republic of Korea
}

% Authors with different addresses and equal first authors:
% \author{%
% \Name{Anonymous Author(s)}
% \equal{These authors contributed equally}
% \Email{Anonymous@sample.com}\\
% \addr University X, Country 1
% \AND
% % footnotemark[1] is to refer to the \equal footnote
% \Name{Anonymous First Author 2}\footnotemark[1] \Email{def@sample.com}\\
% \addr University Y, Country 2
% \AND
% \Name{Anonymous Last Author} \Email{ghi@sample.com}\\
% \addr University Z, Country 3
% }

%%%%%%%%%%%%%%%%%%%%%%%%%%%%%%%%%%%%%%%%%%%%%%%%%%%%%%%%%%%%%%%%%%%%%%%%
%%%%%%%%%%%%% Remove the \linenumbers in the final version %%%%%%%%%%%%%
%%%%%%%%%%%%%%%%%%%%%%%%%%%%%%%%%%%%%%%%%%%%%%%%%%%%%%%%%%%%%%%%%%%%%%%%
% \linenumbers % Activate line numbering

\begin{document}

\maketitle

\begin{abstract}
Lab tests are fundamental for diagnosing diseases and monitoring patient conditions.
However, frequent testing can be burdensome for patients, and test results may not always be immediately available.
To address these challenges, we propose \textbf{\oursfull{} (\ours)}, a unified model that predicts lab test outcomes by leveraging a language modeling approach on EHR data.
Unlike conventional methods that estimate only a subset of lab tests or classify discrete value ranges, LabTOP performs continuous numerical predictions for a diverse range of lab items.
We evaluate LabTOP on three publicly available EHR datasets and demonstrate that it outperforms existing methods, including traditional machine learning models and state-of-the-art large language models.
We also conduct extensive ablation studies to confirm the effectiveness of our design choices.
We believe that LabTOP will serve as an accurate and generalizable framework for lab test outcome prediction, with potential applications in clinical decision support and early detection of critical conditions.

\end{abstract}
\paragraph*{Data and Code Availability}
This paper uses the three EHR datasets, MIMIC-IV~\citep{johnson2023mimic}, eICU~\citep{pollard2018eicu}, and HiRID~\citep{hyland2020early}, which are publicly available on the PhysioNet repository~\citep{johnson2024physionet, pollard2019physionet, yeche2021physionet}.
More details about datasets can be found at Section~\ref{sec:data}.
Our implementation code can be accessed at this repository.\footnote{\url{https://anonymous.4open.science/r/LabTOP-DE7B}}

\paragraph*{Institutional Review Board (IRB)}
This research does not require IRB approval.

\section{Introduction}
\label{sec:intro}

Electronic Health Records (EHR) are essential to modern healthcare systems, serving as comprehensive databases of patient data, including treatments, clinical interventions, and lab test results~\citep{gunter2005emergence}.
These records provide a longitudinal view of a patient's medical history, allowing for the tracking of individual health trends~\citep{kruse2017impact}.
Among them, lab test results play a particularly important role by capturing numerical changes in key biomarkers, such as blood glucose and creatinine.
These results reflect a patient’s physiological and pathological state, supporting the management of disease and the assessment of treatment efficacy~\citep{sikaris2017enhancing, cabalar2024role}.

Despite their clinical importance, conducting lab tests often faces challenges in real-world settings.
For instance, the need for frequent lab tests in unstable patients conflicts with the increased distress they experience from repeated invasive procedures, creating a trade-off between clinical necessity and patient burden~\citep{ambasta2019expert,zhi2024hgbnet}.
Additionally, some tests often take significant time to obtain results, making it difficult to assess the patient's condition promptly~\citep{shiferaw2019magnitude}.
As a result, there are limitations in comprehensively assessing a patient's condition from multiple perspectives through various lab tests, forcing healthcare providers to rely solely on the available test results to make clinical judgments.
This challenge highlights the need for methods to estimate lab test results without conducting the actual tests, allowing healthcare providers to understand a patient's condition comprehensively through the predicted lab test results.

Meanwhile, the advancement of machine learning has greatly contributed to healthcare for several clinical prediction tasks such as readmission, mortality risk, and length of hospital stay~\citep{song2018attend, xiao2018readmission,hur2022unifying,hur2023genhpf,kim2023general,wornow2023ehrshot, renc2024zero}.
Although these works have demonstrated the value of lab test data in supporting clinical decisions, there has been no attempt to directly estimate various lab test results within a single unified model.
Instead, most prior studies approach lab test outcome prediction by classifying discrete levels of a small subset of selected lab items~\citep{hur2023genhpf,kim2023general,wornow2023ehrshot}.
Even among studies that attempt to estimate continuous values, they are specifically designed for only a few selected lab tests rather than whole set of lab items~\citep{zhi2024hgbnet, jiang2024probabilistic, duan2020ngboost, liu2023machine, fu2023implementation, langarica2024deep}.
This narrow focus limits their practical applicability in clinical settings, where precise numerical predictions for a broader range of lab test results can facilitate early intervention and support more informed clinical decision-making across diverse medical conditions.

To address these limitations, we propose \textbf{\oursfull{} (\ours{})}, a novel method for predicting a diverse range of lab test outcomes based on a patient's medical history within a single unified model.
Inspired by autoregressive models widely used in Natural Language Processing~\citep{radford2018improving, radford2019language}, we leverage language modeling paradigm to predict numeric values in an autoregressive manner, instead of regressing each lab test value separately.
This approach enables us to build a unified model capable of predicting outcomes for multiple lab items within a single framework, without requiring us to train a separate model for each lab item.
Furthermore, unlike prior approaches, \ours{} is designed to estimate continuous lab test measurements across a wide spectrum of biomarkers, offering greater granularity and broader applicability in clinical practice.
The main contributions of this work can be summarized as follows:
\begin{itemize}
    \item We propose \ours{}, a new model that estimates continuous values for a wide range of lab test measurements given the patient's previous medical history.
    To the best of our knowledge, this is the first comprehensive approach to predict various lab test results within a single unified model trained on EHR data.

    \item \ours{} outperforms existing approaches, including a recent large language model (LLM), across three publicly available datasets.
    This demonstrates the generalizability of our method to diverse patient populations and clinical contexts.

    \item Considering the time-series characteristics of EHR data where each patient can be represented as a sequence of medical events, we conduct ablation studies to explore the best way to represent the time and numeric features of medical events for the model.
    We believe these experiments will provide valuable insights into effectively modeling EHR data.

\end{itemize}

\section{Related Works}

\paragraph{Predicting clinical outcomes using EHR data}
Previous machine learning models leveraging EHR data have primarily focused on predicting clinical outcomes, such as readmission, mortality risk, and length of stay~\citep{song2018attend, xiao2018readmission,hur2022unifying,hur2023genhpf,kim2023general,wornow2023ehrshot, renc2024zero}.
Among them, \citet{song2018attend} introduced the \textit{SAnD}, employing Transformer architecture~\citep{vaswani2017attention} to predict in-hospital mortality and length of hospital stay.
They constructed the input embeddings based on structured clinical codes (e.g., ICD, LOINC, RxNorm) to process EHR data, which makes it difficult to process multiple EHRs with different schemas within a single model.
To address these challenges, \citet{hur2022unifying} proposed a text-based embedding approach (DescEmb) to represent EHR events using descriptive text rather than codes, showing superior performances compared to the code-based approach.
Based on this text-based approach, GenHPF~\citep{hur2023genhpf} extended it by incorporating multi-task learning to handle various predictive tasks simultaneously.
Furthermore, REMed~\citep{kim2023general} integrated an event-retriever module into this framework to selectively extract relevant events with target tasks, ensuring that only the important events contribute to the final predictions.

\paragraph{Generative models for EHR data}
In addition to the studies employing discriminative models for predictive tasks, generative approaches have also been explored for modeling EHR data.
Specifically, MedGPT~\citep{kraljevic2021medgpt} adopted the GPT-2 architecture~\citep{radford2019language} to predict next SNOMED-CT disorder concept given the patient's previous history of disorders.
Although this work utilized only the disorder concepts instead of a comprehensive set of clinical events in EHR, they demonstrated the potential of generative methods for simulating patient trajectories and predicting future clinical events.
In addition, ETHOS~\citep{renc2024zero} employed a generative approach based on language modeling to simulate clinical scenarios using a structured EHR data.
Similar to the code-based approach, they converted each medical event into 1 to 7 tokens to construct the input sequence for the Transformer~\citep{vaswani2017attention}, and trained the model to predict every next token given the prior sequences to simulate the clinical outcomes of the patient, such as whether the patient will die or not (\textit{i.e.,} mortality prediction).
% In this way, the model simulates the clinical outcomes of the patient, such as whether the patient will die or not (\textit{i.e.,} mortality prediction), and whether the patient will readmit or not (\textit{i.e.,} readmission prediction).

\paragraph{Lab test outcome prediction}
Although previous studies have significantly advanced the field of machine learning for healthcare, none of them have addressed the direct estimation of continuous values for various lab tests within a single unified model.
Instead, most of previous studies have explored the prediction of lab test results by designing a specific model for only a few selected lab items or classifying discrete levels of selected lab items, such as normal or abnormal ranges.
For example, HgbNet~\citep{zhi2024hgbnet} employed an attention-based LSTM model to estimate hemoglobin levels, while \citet{jiang2024probabilistic} presented a probabilistic model using NGBoost~\citep{duan2020ngboost} to predict expected values of several lab items, such as wbc and hemoglobin.
Similarly, several works~\citep{liu2023machine,fu2023implementation,langarica2024deep} attempted to estimate glucose values using machine learning or deep learning approach.
However, these approaches are specifically designed for a small subset of lab tests, making it challenging to generalize across diverse clinical measurements.

Meanwhile, GenHPF~\citep{hur2023genhpf} and REMed~\citep{kim2023general} formulated this task as classifying the discrete level of values for several selected lab items.
Although these works have shown high performances on the defined task, their predictions remain limited in terms of extensibility and granularity.
First, they rely on predefined level thresholds for each lab item, determined through domain knowledge, which makes it difficult to extend the approach to encompass all lab items recorded in the EHR database.
Second, as the task is defined within a fixed prediction window (\textit{e.g.,} predicting average levels of lab items in the next 12 hours given the first 12 hours), these models lack granularity in providing immediate estimations of lab test results, which is critical for clinical scenarios where timely and dynamic predictions are necessary to inform urgent decisions.

\begin{figure*}[ht] % Use figure* for spanning both columns
\floatconts
    {fig:overview}
    {\vspace{-8mm}
    \caption{\textbf{Training and Inference of \ours{}.} During training, demographic information $D$ and a sequence of medical events $P=[\mathcal{M}_1,\dotsc,\mathcal{M}_N]$ are tokenized, and fed into the Transformer to generate next token at each position. The training loss is then computed only for lab test value tokens along with their units and the corresponding $\texttt{[EOE]}$ token. During inference, given the preceding sequence of medical events and the target lab event (\textit{i.e.,} $(t_k,e_k,d_k)$), the model autoregressively generates numeric value tokens for its outcome until the [EOE] token is encountered.}}
    {
        \includegraphics[width=1.0\textwidth]{figures/training_inference.pdf}
    }
    \vspace{-6mm}
\end{figure*}

\section{Methods}

\subsection{Preliminaries}
In EHR data, patient medical events that occur in a hospital, such as lab tests and prescriptions, are recorded in a structured format.
Thus, we can represent a patient $P$ as a sequence of medical events $[\mathcal{M}_1, \dotsc, \mathcal{M}_N]$, where $N$ is the total number of medical events, and $\mathcal{M}_i$ is the $i$-th medical event for the patient.  
Each medical event typically includes a timestamp, medical event codes (\textit{e.g.,} ICD, LOINC, RxNorm), a numerical value, and its corresponding unit of measurement.
Accordingly, the $i$-th medical event $\mathcal{M}_i$ can be represented as a 5-tuple of event features $(t_i, e_i, c_i, v_i, u_i)$, where $t_i$ represents the timestamp of the event, $e_i$ stands for the type of the event (\textit{e.g.,} ``labevent'', ``prescription''), $c_i$ is the corresponding medical code (\textit{e.g.,``2160-0"}), $v_i$ denotes the measured values (\textit{e.g.,} 1.2), and $u_i$ is the associated unit of measurement (\textit{e.g., mg/dL}).
The detailed definitions of these event features are described in the following section, as well as depicted in Figure~\ref{fig:overview}.

\subsection{Preprocessing}

\paragraph{Textual representation of medical events}
Unlike conventional methods that directly embed medical code $c_i$ into a fixed-dimensional vector space to represent a medical event $\mathcal{M}_i$~\citep{miotto2016deep, choi2016doctor, choi2020learning, song2018attend, shang2019pre, renc2024zero}, we start by converting each medical code $c_i$ (\textit{e.g.,``2160-0"}) into its corresponding textual description $d_i$ (\textit{e.g.,``Creatinine [Mass/volume] in Serum or Plasma"}).
Additionally, we also treat its numerical value $v_i$ as plain text and split it into each digit, so that each digit is processed as a separate token in the input sequence~\citep{hur2022unifying, hur2023genhpf, kim2023general}.
In this way, we reconstruct a 4-subtuple of $(e_i, c_i, v_i, u_i)$ included in the medical event $\mathcal{M}_i$ into a textual sequence, and denote as $m_i$. %denoted as $m_i$.
For example, a lab test event standing for a creatinine value of 1.23 mg/dL measured at the time of $t_i$ is transformed into $(t_i, m_i)$ where $m_i$ is a plain text of ``labevent creatinine 1~ .~ 2~ 3 mg/dL''. 

\paragraph{Absolute time encoding}
\label{sec:abs_time_encoding}
Temporal information also plays a crucial role in modeling the progression of a patient's health status accurately, as the timing and frequency of the events provide essential insights into clinical trajectories.
While previous studies have primarily employed a relative time encoding strategy to represent temporal intervals between consecutive events~\citep{hur2023genhpf, renc2024zero}, this approach may not always be optimal due to the following reasons.
Medical events for inpatients, such as prescriptions and lab tests, usually follow regular patterns dictated by hospital routines, clinical protocols, or daily schedules.
Additionally, a patient's physiological status can also depend on time-of-day patterns, making absolute time representation more effective in capturing meaningful trends.
For example, blood glucose measurements are heavily affected by a patient's fasting state~\citep{moebus2011impact}.
By incorporating absolute time encoding, the model can infer contextual factors such as feeding schedules or overnight fasting periods, leading to improved predictive performances for lab test outcomes.

To achieve this, we represent each timestamp $t_i$ as a combination of four types of special tokens, capturing essential temporal attributes that influence patient health status.
Specifically, these tokens include (1) the elapsed days since the ICU admission, (2) the day of the week the event occurred, (3) the absolute hour of the event, and (4) the aggregated minute value.
The minute value is aggregated in 10-minute intervals to balance granularity and computational efficiency, ensuring that time representations remain meaningful without introducing excessive sparsity.
For example, for the event recorded on the first day since the ICU admission, at 13:38 on a Tuesday, the timestamp $t_i$ is defined as $[\texttt{[DAY1]},\texttt{[TUE]},\texttt{[13h]}, \texttt{[30m]}]$.
In addition, we append a special $\texttt{[EOE]}$ token indicating the end of the event to every textual medical event $m_i$ and concatenate a sequence of medical events, so that a patient $P$ is now represented as
\begin{equation}
    P = \Big\|_{i=1}^{N}[t_i, m_i, \texttt{[EOE]}]
\end{equation}
where $\|$ is a symbol denoting the sequence concatenation operator.

\paragraph{Demographic encoding}
Physiological ranges for lab tests can vary significantly based on demographic attributes such as age, gender, and race.
For example, creatinine levels tend to be higher in males due to greater muscle mass~\citep{culleton1999prevalence}, while hemoglobin levels may vary depending on both gender and age~\citep{murphy2014sex}.
To incorporate this information into our model, we convert each patient's demographic attributes into a textual representation, denoted as $D$.
For example, for a 65-year-old Asian female patient, $D$ is defined as ``gender F age 65 race Asian''.
We prepend this demographic information to the patient's sequence of medical events $P$ and tokenize the sequence using the Bio-Clinical BERT tokenizer~\citep{alsentzer2019publicly} to obtain the final input sequence for the model, which is defined as
\begin{equation}
    x = f(D \| P) = f(\Big[D, \Big\|_{i=1}^{N}[t_i, m_i, \texttt{[EOE]}] \Big])
\end{equation}
 where $f(\cdot)$ denotes the tokenization function that converts the text sequence into a sequence of token indices.

\subsection{LabTOP Training}
\paragraph{Model architecture}
We design our model following the GPT-2 architecture~\citep{radford2019language}, which is composed of Transformer decoder layers~\citep{vaswani2017attention}.
Specifically, the input sequence $x$ is embedded into a sequence of latent vectors $\bm{X}\in\mathbb{R}^{M \times h}$ by the learnable embedding layer, where $M$ is the length of the input token sequence and $h$ is the hidden dimension of the model.
They are then fed into the Transformer decoder to model the conditional probability of the next token given the previous sequence, following an autoregressive language modeling approach.
% :
% \begin{equation}
    % p(x) = \prod_{i=1}^{M}p(x_i|x_{<i};\theta)
% \end{equation}
% where $\theta$ represents the model parameters, and $x_{<i}$ denotes all preceding tokens.

\paragraph{Objective}
Unlike traditional language modeling where the objective is to predict the next token for every position in the sequence, we modify the loss computation to focus exclusively on lab test outcome prediction.
That is, instead of computing the loss for all tokens, we selectively compute the loss only on the tokens corresponding to lab test values, encouraging the model to focus on accurately predicting lab test outcomes while still benefiting from a broader contextual understanding of the patient's medical history.
Accordingly, the loss is defined as
\begin{equation}
    \mathcal{L}=- \sum_{i\in{\mathcal{I}_{lab}}}\log{p(x_i|x_{<i};\theta)}
\end{equation}
where $\mathcal{I}_{lab}$ represents the set of token positions corresponding to lab test values along with their unit of measurements, as well as their special $\texttt{[EOE]}$ token.

\paragraph{Random permutation}
As mentioned in the previous section, timestamps are discretized into 10-minute interval tokens (\textit{e.g.,} 13:38 $\rightarrow$ $[\texttt{[13h]},\texttt{[30m]}]$), which can result in multiple medical events having the same timestamp tokens.
Accordingly, in order to make the model invariant to their ordering during training, we apply random permutation to medical events that share the same timestamp.
Specifically, given a set of events having the same timestamp, we randomly shuffle the order of the events at each training iteration to prevent the model from overfitting to any arbitrary order within the same timestamp.


\subsection{LabTOP Inference}
We formulate the lab test outcome prediction task as a language modeling task to predict numeric values token-by-token in an autoregressive manner.
That is, the model generates the corresponding numeric value of a target lab item given the sequence of prior medical events and the target lab item's name.
Accordingly, to construct inference samples, we extract every lab item appearing in the input sequence and build a corresponding inference prompt.
Specifically, for each target lab test item occurring at timestamp $t_k$, we concatenate all prior medical events up to $t_k$ along with the target lab test's textual representation, $e_k$ and $d_k$, to create an inference sequence.
Thus, the $k$-th inference sample of a patient is defined as
\begin{equation}
    x_{test}^{(k)} = f(\Big[D, \Big\|_{t_i<t_k}[t_i, m_i, \texttt{[EOE]}],[t_k,e_k,d_k]\Big])
\end{equation}
where $e_k$ and $d_k$ denote the event type (\textit{i.e.,} ``labevent'') and textual description of the target lab test (\textit{e.g.,} ``bicarbonate''), as aforementioned.

Once the inference sequence is constructed, it is fed into the trained model, which generates tokens autoregressively until the $\texttt{[EOE]}$ token is encountered.
In this process, our primary objective is to accurately predict the outcome value of a specific lab test at a given timestamp rather than probabilistically generating potential future lab tests.
Thus, to ensure precise predictions, we do not sample from the model's output distribution but instead select the most probable token (\textit{i.e.,} top-1 token) at each decoding step.
The generated sequence is then decoded back into a numerical value (\textit{e.g.,} 2 4 . 0 meq / l $\rightarrow$ 24.0 meq/l).

\section{Experiments}
\subsection{Data}
\label{sec:data}
To demonstrate the generalizability of our approach, we conduct experiments on three publicly available EHR datasets: MIMIC-IV~\citep{johnson2023mimic}, eICU~\citep{pollard2018eicu}, and HiRID~\citep{yeche2021hirid}.
Specifically, MIMIC-IV consists of 94,458 ICU records of 364,627 patients from the Beth Israel Deaconess Medical Center, including structured time-series data such as lab test results, and medication administrations.
eICU is a large-scale, multi-center dataset that includes 200,859 ICU records of 139,367 patients from hospitals across the United States, consisting of a broader range of patient demographics and treatment variations compared to single-center datasets.
HiRID is a high-resolution ICU dataset collected from the Bern University Hospital in Switzerland, composed of 33,905 ICU patients.
% This dataset provides continuously monitored data, making it well-suited for fine-grained temporal modeling of patient medical status.
For all these datasets, we take the ICU records whose length of stay is at least 6 hours.
We then randomly split them into training, validation, and test sets in an 8:1:1 based on their patient IDs, and treat each ICU record as an individual sample for our model.
% TODO appendix ref
Detailed dataset statistics are described in Appendix~\ref{app:data_statistics}.

To determine which types of EHR events should be involved for lab test outcome prediction, we categorized EHR events into two broad types: observation-type events, which represent the patient's physiological state (\textit{e.g.,} lab tests, microbiology tests), and intervention-type events, which correspond to medical treatments, prescriptions, and procedures administered during the ICU stay.
We then prioritized these event types based on their expected influence on lab test results:
(1) Firstly, medication events that directly reflect the patient's physiological status by indicating administered drugs and dosages.
(2) Secondly, other intervention-related events, including input events (\textit{e.g.,} fluid intake) and procedure events, which affect the patient's overall condition.
(3) Lastly, observation-type events that do not directly alter a patient's state but provide useful contextual information about their physiological trends.
Based on these criteria, the following tables for each dataset are used:
\begin{itemize}
    \item MIMIC-IV: \textit{labevents}, \textit{emar}, \textit{emar\_detail}, \textit{inputevents}, \textit{procedureevents}, \textit{outputevents}, \textit{microbiologyevents}

    \item eICU: \textit{lab}, \textit{medication}, \textit{infusiondrug}, \textit{treatment}, \textit{intakeoutput}, \textit{microlab}

    \item HiRID: \textit{observation\_tables}, \textit{pharma\_records}
\end{itemize}
Note that HiRID collected every observation-type event into one table (\textit{i.e.,} \textit{observation\_tables}), which leads to multiple types of events, such as lab tests and microbiology tests, being contained in this single table.
However, there is only an indicator specifying whether an event corresponds to a lab test or not, so we cannot differentiate other observation-type events within this table.
Accordingly, we select only those events explicitly identified as lab tests, excluding all other events from this table.

Additionally, to ensure a focus on the most clinically relevant medical events, we select only the medical items that account for the top 90\% of occurrences within their respective tables for each dataset, including the lab test tables as well as other event types such as medication, input events, and procedure events.
As a result, we select 44 unique lab tests from MIMIC-IV, 41 from eICU, and 27 from HiRID.
% TODO appendix ref
The included lab items from each dataset are detailed in Appendix~\ref{app:data_statistics}.

\subsection{Evaluation Metrics}
\paragraph{NMAE}
Normalized Mean Absolute Error (NMAE) is a normalized version of Mean Absolute Error (MAE) that accounts for differences in scale across multiple variables with different units and ranges.
Unlike standard MAE, which directly measures the average absolute difference between the predicted and ground-truth values, NMAE adjusts for variations in measurement scales by normalizing each error term based on the range of the corresponding lab item.
Specifically, for each unique lab item $l$, we compute its scale using the difference between the 99th percentile and the 1st percentile values (\textit{i.e.,} $v_{99\%}^{(l)} - v_{1\%}^{(l)}$) in the test set to mitigate the influence of extreme outliers.
The NMAE for a given lab test $l$ is then computed by dividing its MAE by this scale, which is defined as
\begin{equation}
    \text{NMAE}_{l}=\frac{1}{v_{99\%}^{(l)}-v_{1\%}^{(l)}}\sum_{i\in\mathcal{I}_l}\frac{|y_i-\hat{y}_i|}{N_l}
\end{equation}
where $\mathcal{I}_l$ represents the set of sample indices corresponding to the target lab test, $y_i$ and $\hat{y}_i$ denote the ground-truth and predicted value for $i$-th sample, and $N_l$ is the number of target lab samples in the test set.

\paragraph{SMAPE}
Symmetric Mean Absolute Percentage Error (SMAPE) is a widely used metric to evaluate the accuracy of prediction for continuous values, providing an intuitive measure of how much predicted values deviate from ground-truth values in percentage terms.
Unlike traditional Mean Absolute Percentage Error (MAPE), SMAPE ensures a symmetric scaling by normalizing the error terms based on the sum of the absolute values of the ground-truth and predicted values, preventing over-penalization when the ground-truth values are small.
This property makes SMAPE especially suitable for evaluating lab test outcome predictions, where the ranges of different lab items can vary significantly.
The SMAPE for a given lab test $l$ is defined as
\begin{equation}
    \text{SMAPE}_l = \frac{1}{N_l} \sum_{i \in \mathcal{I}_l} \frac{|y_i - \hat{y}_i|}{(|y_i| + |\hat{y}_i|) / 2} \times 100
\end{equation}
% Note that SMAPE basically can have a range from 0\% to 200\%, but typically falls within the 0\% to 100\% range in most practical cases.

\subsection{Baselines}
To demonstrate the effectiveness of \ours, we compare its performance with the following baseline methods, including both traditional machine learning approaches and state-of-the-art LLMs.
\begin{itemize}
    \item \textbf{Naive}: A simple method that predicts the lab test value at a given timestamp by using the most recent measured value of the same lab test, commonly referred to as the naive forecasting method.
    \item \textbf{Naive($\mu$)}: A variant of the naive method that predicts the lab test value by averaging all previously recorded values of the same lab test before the prediction timestamp.
    \item \textbf{GenHPF}~\citep{hur2023genhpf}: A multi-task learning model that incorporates hierarchical patient features for clinical prediction tasks. We adapt GenHPF to estimate lab test values by formulating it as a regression task.
    \item \textbf{XGBoost}~\citep{chen2016xgboost}: A widely used gradient boosting method designed for structured data. For this baseline, we use several statistics of past measurements, such as count and mean.
    \item \textbf{GPT-4o} and \textbf{GPT-4o-mini}: LLMs designed for general-purpose tasks. We evaluate these models to assess the capability of general-purpose models in predicting lab test outcomes. Since these models process long patient histories as input, running inference across the entire test set would require substantial token usage. To balance computational feasibility with representative evaluation, we sample 10\% of the test set\footnote{To be more specific, running inference in a batch on the sampled 10\% of the test set incurred a total cost of approximately \$500.}. All experiments employing GPTs were executed using the HIPAA-compliant GPT model available on Azure.\footnote{\url{https://learn.microsoft.com/en-us/azure/compliance/offerings/offering-hipaa-us}}
\end{itemize}
More details about each model's implementation, such as training hyperparameters and feature engineering for XGBoost, are provided in Appendix~\ref{app:implementation_details}.

\subsection{Results}

\documentclass{MITstyle}

%\usepackage[table]{xcolor}
\usepackage{chngcntr}
\usepackage{hyperref}
\usepackage{microtype}

\title{A Lightweight and Extensible Cell Segmentation and Classification Model for Whole Slide Images}

\author{Nikita Shvetsov~$^{1, }$\footnote{Correspondence e-mail: nikita.shvetsov@uit.no}, Thomas K. Kilvaer~$^{2, 3}$, Masoud Tafavvoghi~$^{4}$, Anders Sildnes~$^{1}$, \\ Kajsa Møllersen~$^{4}$, Lill-Tove Rasmussen Busund~$^{5, 6}$, Lars Ailo Bongo~$^{1}$ \\
%
\vspace{1em} % Space between authors and afilliations
%
\normalfont{\small $^{1}$Department of Computer Science, UiT The Arctic University of Norway}\\
\normalfont{\small $^{2}$Department of Oncology, University Hospital of North Norway}\\
\normalfont{\small $^{3}$Department of Clinical Medicine, UiT The Arctic University of Norway}\\
\normalfont{\small $^{4}$Department of Community Medicine, UiT The Arctic University of Norway}\\
\normalfont{\small $^{5}$Department of Medical Biology, UiT The Arctic University of Norway} \\
\normalfont{\small $^{6}$Department of Clinical Pathology, University Hospital of North Norway} %\vspace{2em}
}

\begin{document}
\maketitle

\section*{Abstract}

% \begin{abstract}
% Developing clinically useful cell-level analysis tools in digital pathology remains challenging due to limitations in dataset granularity, inconsistent annotations, computational demands of advanced models, and difficulties in integrating new technologies into clinical workflows. To address these challenges, we propose a multi-faceted solution that enhances data quality, model performance, and usability to create a lightweight and extensible cell segmentation and classification model.

% First, we update data labels by employing a cross-relabeling process that refines the labels of two existing datasets, PanNuke and MoNuSAC, to create a new unified dataset with enhanced granularity, encompassing seven distinct cell types. Second, we leverage the H-Optimus foundation model as a fixed encoder to improve feature representation for simultaneous cell segmentation and classification tasks. Third, to address the computational demands of foundation models, we employ knowledge distillation to reduce model size and complexity while maintaining comparable performance. Finally, to facilitate integration into clinical workflows, we integrate the distilled model into the QuPath software, a widely used open-source platform in digital pathology.

% Our results demonstrate improvements in cell segmentation and classification performance using the H‑Optimus-based model compared to a CNN-based model. Specifically, the average $R^2$ improved from 0.575 to 0.871, and the average $PQ$ score improved from 0.450 to 0.492, indicating better alignment with actual cell counts and enhanced segmentation and classification quality. Furthermore, the distilled student model maintains performance comparable to the larger foundation model while reducing the parameter count by a factor of 48.
% Overall, by reducing computational complexity and integrating it into existing workflows, the proposed approach may significantly impact diagnostic processes, reduce the workload of pathologists, and contribute to improved patient outcomes. Though our approach shows potential enhancements in efficiency and usability of cell segmentation and classification models in digital pathology, extensive validation is needed to deploy these models in clinical practice.
% \end{abstract}

%%% shortened abstract
\begin{abstract}
Developing clinically useful cell-level analysis tools in digital pathology remains challenging due to limitations in dataset granularity, inconsistent annotations, high computational demands, and difficulties integrating new technologies into workflows. To address these issues, we propose a solution that enhances data quality, model performance, and usability by creating a lightweight, extensible cell segmentation and classification model. 

First, we update data labels through cross-relabeling to refine annotations of PanNuke and MoNuSAC, producing a unified dataset with seven distinct cell types. Second, we leverage the H-Optimus foundation model as a fixed encoder to improve feature representation for simultaneous segmentation and classification tasks. Third, to address foundation models' computational demands, we distill knowledge to reduce model size and complexity while maintaining comparable performance. Finally, we integrate the distilled model into QuPath, a widely used open-source digital pathology platform. 

Results demonstrate improved segmentation and classification performance using the H-Optimus-based model compared to a CNN-based model. Specifically, average $R^2$ improved from 0.575 to 0.871, and average $PQ$ score improved from 0.450 to 0.492, indicating better alignment with actual cell counts and enhanced segmentation quality. The distilled model maintains comparable performance while reducing parameter count by a factor of 48. By reducing computational complexity and integrating into workflows, this approach may significantly impact diagnostics, reduce pathologist workload, and improve outcomes. Although the method shows promise, extensive validation is necessary prior to clinical deployment.
\end{abstract}
\clearpage

\section{Introduction}
In digital pathology, accurate segmentation and classification of cells are crucial for many diagnostic, prognostic, and predictive analyses \cite{Jaber_Beziaeva_etal._2019,Lin_Pan_etal._2022,Park_Ock_etal._2022,Shen_Choi_etal._2024}. Nowadays, developments in computational pathology offer multiple solutions \cite{H._Qu_P._Wu_etal._2020,Javed_Mahmood_etal._2020} to utilize cell-level datasets to train machine learning models that solve these problems. The quality and specificity of training datasets are critical for robust and accurate models. Adhering to the principle of "garbage in, garbage out", it is essential to ensure that these datasets are extensively and accurately labeled with distinct classes that reflect the diverse biological characteristics of different cell types. Unfortunately, the number of open-source datasets comprising such high-quality annotations is limited. Existing cell segmentation datasets \cite{Gamper_Koohbanani_etal._2019,Graham_Vu_etal._2019,Verma_Kumar_etal._2021} may offer extensive annotations for certain cell types while providing more general labels for others. For example, in PanNuke, which is one of the largest open-source datasets comprising labeled cells, various types of morphologically and functionally different inflammatory cells like macrophages and lymphocytes are clustered in a broad "inflammatory" class. Consequently, these classes are frequently omitted from analyses or aggregated into broader meta-classes \cite{Gamper_Koohbanani_etal._2020} and likely interfere with other cell classes included in the dataset. This and similar inconsistencies in annotation granularity limit the ability of machine learning models to learn the comprehensive and nuanced features necessary for accurate cell segmentation and classification. To address these challenges, methods for refining and standardizing dataset annotations are essential to enhance the quality of training data.

A complementary approach to mitigate the absence of high-quality training data is the use of foundation models. Foundation models as encoders are defined as large-scale, versatile networks pre-trained on vast, diverse datasets using self-supervised learning, contrasting with convolutional neural network (CNN) pre-trained encoders that rely on supervised learning with labeled data. In practice, foundation models leverage enormous amounts of weakly or unlabeled data from millions of whole slide images (WSIs) and employ self-attention mechanisms to capture long-range dependencies and global context \cite{Chen_Ding_etal._2024,Saillard_Jenatton_etal._2024,Vorontsov_Bozkurt_etal._2024,Xu_Usuyama_etal._2024}. As a consequence, foundation models are able to produce transferable feature representations across different cell types and tissue environments. The feature representations can be leveraged by decoder networks to produce segmentation masks and pixel-level classifications. Because foundation models have comprehensive feature representations, they can be effectively fine-tuned using much smaller amounts of cell-level data compared to the large datasets needed to train models from scratch. Furthermore, foundation models incorporate adversarial training elements or contrastive learning \cite{Chen_Ding_etal._2024,Xu_Usuyama_etal._2024}, enhancing their resilience and adaptability by exposing them to challenging and varied scenarios during training. This may result in more generalizable models, often making them well-suited for diverse and complex tasks in digital pathology.

Despite the inherent advantages of foundation models, their deployment for practical use faces its own obstacles. In particular, they require substantial computational power, financial investments and rigorous testing to ensure reliability and efficacy for a given task \cite{Akkus_Dangott_etal._2022,Dragomir_Cocuz_etal._2022,Go_2022,Jafri_Farooqui_etal._2024}. Moreover, while foundation models enhance feature representation and performance, they depend on the quality of available annotations for decoder fine-tuning and, like any other model, cannot resolve existing inconsistencies or ambiguities in data labels. Therefore, there remains a critical need for solutions that address both data quality and practical deployment considerations.
Further, integrating new technologies into existing clinical workflows often encounters resistance, as it necessitates adjustments to established diagnostic processes. So, there is a need to develop solutions that could be integrated into current practices, minimizing the burden on medical professionals to adopt new tools \cite{King_Williams_etal._2023}.

Existing solutions \cite{Goldsborough_Philps_etal._2024,Hörst_Rempe_etal._2024}, while addressing some aspects of these challenges, fall short in providing a comprehensive approach. To address the data quality and clinical deployment issues, we propose a multi-faceted solution that encompasses data refinement, model optimization, and integration with existing pathology tools (\hyperref[fig:fig1]{Figure 1}). The outcome is a lightweight cell segmentation and classification model that can be integrated into digital pathology workflows for practical clinical use.

\begin{figure}[h!]
    \centering
    \includegraphics[width=\textwidth, height=0.82\textheight, keepaspectratio]{images/Figure_1.pdf}
    \caption{Overview of the proposed solution, including 1) Data refinement using cross-relabeling, 2) Teacher model development and fine tuning, 3) Student model optimization with knowledge distillation and 4) Student model and QuPath integration}
    \label{fig:fig1}
\end{figure}
\clearpage

Our approach begins with preparing the data for the fine-tuning and training of the machine learning models. We create a refined dataset, acquired via cross-relabeling two cell-level datasets, enhancing annotation specificity and consistency of the labeled data. Subsequently, we create a cell segmentation and classification model based on the foundation model. We leverage the foundation model as a fixed encoder and fine-tune a decoder using the refined dataset to improve generalization across diverse tissue- and cell types.
To ensure that the model remains lightweight and deployable in a possibly resource-constrained environment, we employ knowledge distillation to approximate the functionality of the foundation model. Finally, to facilitate the practical application of our model in digital pathology workflows, we integrate it with the QuPath \cite{Bankhead_Loughrey_etal._2017} application. Each methodological component contributes to the overarching goal of enhancing model performance, generalizability, and usability in clinical settings.

The primary contributions of this paper are:
\begin{enumerate}
    \item \textit{Data labels refinement through cross-relabeling:}
    
    We propose a new method for refining labels of cell-level datasets through cross-relabeling. This method employs classification models to re-label broad and ambiguous instances, resulting in a more diverse dataset. Our evaluation demonstrates that these classification models achieve high accuracy on test subsets, indicating the reliability of the method for label refinement.

    \item \textit{Enhanced model performance via foundation models:}
    
    We employ a foundation model as a feature extractor for the cell segmentation and classification task. In comparison with training a CNN model from scratch, the foundation model backbone only needs fine-tuning, which significantly reduces training time, computational resources and data requirements. We show that using a foundation model encoder leads to better performance in cell segmentation and classification networks than using a CNN-based encoder. This improvement may enable the model to generalize more effectively across various tissue types and imaging methods.
    
    \item \textit{Model optimization through knowledge distillation:}
    
    We show that a smaller student model trained using knowledge distillation on the refined dataset obtained via our cross-relabeling approach from a foundation model achieves comparable performance in cell segmentation and quantification tasks. As a result, this model is more suitable for deployment in environments without high-performance computing resources.
    
    \item \textit{Integration with QuPath:}
    
    We integrate the distilled cell segmentation and classification model into QuPath, a widely used open-source digital pathology platform, to accelerate clinical adaptation by enabling pathologists to more easily incorporate advanced computational tools into their existing workflows.
\end{enumerate}

Through these methodological steps, we aim to bridge the gap between advanced machine learning techniques and practical clinical applications, making accurate and efficient digital pathology accessible in a broader range of healthcare settings.

\section{Refining Existing Datasets Using Cross-Relabeling}
To address the limitations of sparse and ambiguous labeling of cell-level datasets, we propose a generalizable cross-relabeling strategy that can be applied to any dataset containing broadly categorized or imprecisely labeled cell types. This approach involves training and subsequently leveraging classification models to refine broad categories into more specific or biologically relevant classes.
When applied to cell-level data, the methodology includes extracting individual cell images from the dataset patches, preprocessing these images to standardize the size and accommodate partial cells, and then training deep learning classifiers capable of distinguishing between the finer cell subtypes within the coarser categories. 
To illustrate our approach, we focus on the PanNuke \cite{Gamper_Koohbanani_etal._2020, Gamper_Koohbanani_etal._2019} and MoNuSAC \cite{Verma_Kumar_etal._2021} datasets that we have used to train models for cell quantification in our previous works \cite{Shvetsov_Grønnesby_etal._2022,Shvetsov_Sildnes_etal._2024}. We find that for better cell differentiation we have to introduce more granular labels. PanNuke includes a broad classification of "inflammatory" cells, encompassing lymphocytes, macrophages, and neutrophils. Each cell type differs significantly in structure, function, and clinical relevance. Conversely, MoNuSAC uses the label "epithelial" for a class that comprises both benign epithelial cells and malignant neoplastic cells. This practice makes it challenging to differentiate between benign and malignant epithelial cells in the dataset, which is a critical distinction when identifying tumor areas within tissue samples. To address these issues, we implement a cross-relabeling strategy as shown in \hyperref[fig:fig2]{Figure 2}. The key components are two classification models: one is trained on singular cell images from PanNuke data to classify the epithelial meta-class into epithelial and neoplastic classes. The other is trained on MoNuSAC to refine the inflammatory class into lymphocytes, neutrophils, and macrophages.

\begin{figure}[h!]
    \centering
    \includegraphics[width=\textwidth]{images/Figure_2.pdf}
    \caption{Refined dataset generation via cross relabeling}
    \label{fig:fig2}
\end{figure}

The refining approach consists of three consecutive steps. The first is the preprocessing step, in which we extract individual cells from both datasets (\hyperref[fig:fig3]{Figure 3}). The specifics of PanNuke and MoNuSAC patch preparation before cell preprocessing are provided in \hyperref[chap:S1]{Appendix S1}.

\begin{figure}[h!]
    \centering
    \includegraphics[width=\textwidth]{images/Figure_3.pdf}
    \caption{Cell instances preprocessing including (1) cell map extraction, (2) bounding box delineation, (3) adjusting cell boxes and (4) cropping and resizing of cell images}
    \label{fig:fig3}
\end{figure}

During preprocessing, we extract cell type maps from the ground truth label mask and calculate bounding boxes around each cell instance. To accommodate partial cells at patch borders, a common issue in cropped patch images, we employ mirror padding and extend the field of view of the cell label by 15 pixels to capture adjacent cells. We then crop and resize the identified regions to $64 \times 64$ pixels using bicubic interpolation.

The preprocessed PanNuke dataset comprises 68,031 neoplastic and 23,207 epithelial cell images, while MoNuSAC comprises  33,104 lymphocytes, 1,252 neutrophils, and 1,695 macrophages, which we subsequently use in training cell classification models and classifying the cell image data \hyperref[fig:S2]{Appendix Figure S2 (1)}. 

The next step is to train two distinct ResNet50-based classifiers tailored to address the specific labeling challenges inherent in each dataset. We use ResNet50 for classification models due to its proven effectiveness for image classification tasks in histopathology \cite{pan2022reviewmachinelearningapproaches}, and its compatibility with small images. For the PanNuke dataset, we design the classifier, trained on MoNuSAC data, to disaggregate the heterogeneous "inflammatory" cell category into distinct subtypes: lymphocytes, macrophages, and neutrophils. Similarly, for the MoNuSAC dataset, the classifier is trained on PanNuke data and distinguishes between benign and malignant epithelial cells within the overarching "epithelial" label. By applying these targeted classifiers to their respective datasets, we assign more specific labels to individual cell instances, thus enabling us to create a unified dataset.
To ensure a balanced representation of classes, we train both models on datasets that had been equalized to match the size of the least represented class. Thus, we obtain datasets comprising 23,207 samples per class for PanNuke and 1,252 samples per class for MoNuSAC data. Next, we partition both of them into training (70\%), validation (20\%), and testing (10\%) subsets. To mitigate the risk of overfitting, we use a single dropout layer with a rate of p=0.5 in both models and data augmentation using randomized color perturbations, rotation, and horizontal and vertical flipping. We employ AdamW optimizer and the cross-entropy loss function for the training criterion.

To evaluate the two trained models, we measure the classification accuracy on the respective test subsets. The accuracies on the test subset for both classifiers are presented in \hyperref[tab:1]{Table 1}. The PanNuke model achieves an average accuracy of 93.57\%, with higher accuracy for neoplastic cells (96.06\%) compared to epithelial cells (86.26\%). The confusion matrix in Figure A3.1 shows that the model predominantly distinguishes accurately between epithelial and neoplastic tissues, with a substantial number of correct classifications and relatively few misclassifications. The MoNuSAC model demonstrates an average accuracy of 98.92\%, excelling in classifying lymphocytes (99.67\%) and macrophages (94.12\%), with lower performance for neutrophils (85.71\%). The confusion matrix in Figure A3.2 shows that the model identifies lymphocytes and performs reasonably well with macrophages and neutrophils.

\begin{table}[h!]
\renewcommand{\arraystretch}{1.5}
  \centering
  \caption{Cell classification results for PanNuke and MoNuSAC trained models (CI 95\%).}
  \label{tab:1}
  \begin{tabular}{|l|c|c|}
   \hline
   %\rowcolor{gray!30}
    Accuracy               & PanNuke model              & MoNuSAC model              \\
    \hline
    Average      & 0.936 (0.931--0.941)         & 0.989 (0.986--0.993)        \\
    \hline
    Neoplastic   & 0.961 (0.956--0.965)         & -                          \\
    \hline
    Epithelial   & 0.863 (0.849--0.877)         & -                          \\
    \hline
    Lymphocytes  & -                          & 0.997 (0.995--0.999)        \\
    \hline
    Neutrophils  & -                          & 0.857 (0.796--0.918)        \\
    \hline
    Macrophages  & -                          & 0.941 (0.906--0.976)        \\
    \hline
  \end{tabular}
\end{table}

Finally, during the last step, we use the model trained on PanNuke data for epithelial cells in MoNuSAC and the model trained on MoNuSAC for the inflammatory cells class in PanNuke. Specifically, we use classifier models to relabel epithelial cells in MoNuSAC and inflammatory cells in PanNuke data. Then we combine cells with refined labels and the rest of the cells in both datasets to create a refined dataset (\hyperref[fig:S2]{Appendix Figure S2 (2)}). The process of relabeling cells and visualizing them on a patch is shown in \hyperref[fig:fig4]{Figure 4}. The cell counts in the refined dataset are provided in \hyperref[tab:S4]{Appendix Table S4}.

\begin{figure}[h!]
    \centering
    \includegraphics[width=\textwidth, height=0.42\textheight, keepaspectratio]{images/Figure_4.pdf}
    \caption{Cell relabeling procedure for epithelial and inflammatory cell classes}
    \label{fig:fig4}
\end{figure}

%\hfill

Relabeling and combining datasets have been explored in a prior study \cite{Parulekar_Kanwat_etal._2023}, where consecutive fine-tuning on multiple datasets was employed to account for hierarchical class label structures. While the method presented in \cite{Parulekar_Kanwat_etal._2023} is intuitive, it often lacks consistency and requires multiple fine-tuning runs, which can be cumbersome and time-consuming. 
In contrast, cross-relabeling simplifies this process by using specialized classification models tailored to each dataset's specific labeling challenges. This approach provides better transparency and produces a unified dataset encompassing seven distinct cell types across multiple tissue samples, enhancing data diversity for further model training or fine-tuning.

Despite these improvements, cross-relabeling does not entirely resolve issues related to poor labeling quality or the amount of labeled data. Specifically, our results show lower accuracies persist for underrepresented classes, such as macrophages, which may stem from a limited sample availability and intrinsic challenges in distinguishing these cells based solely on H\&E staining. Furthermore, while our method enhances label specificity, it relies on the initial quality of the broad labels; thus, any fundamental inaccuracies in the original annotations can propagate through the relabeling process. Addressing the overall problem of limited data labels may require integrating additional data sources or utilizing complementary immunohistochemical staining methods.
Although the reported performance metrics are obtained from evaluations on the native test sets of each dataset, it is important to note that the primary application of these classifiers is to perform cross-relabeling, where a model trained on one dataset (e.g., PanNuke) is applied to another (e.g., MoNuSAC) and vice versa. We acknowledge that a more systematic evaluation of cross-dataset generalization is needed and could be performed in future work.

Overall, the refined dataset produced by our approach can enhance the supervised training or fine-tuning of cell segmentation and classification models, especially those that utilize pre-trained foundation models to improve feature extraction robustness. In addition, these models can detect nuanced classes that enable researchers to conduct more detailed analyses of biological processes in computational pathology.

\section{Foundation models for robust cell segmentation and classification}

Accurate cell segmentation and classification in digital pathology are hindered by limited labeled data and the fact that conventional CNNs are unable to capture global contextual information due to their local receptive field constraints \cite{Gheflati_Rivaz_2022,Yang_Marcus_etal.}. Traditional approaches in cell quantification have predominantly relied on CNN encoders, such as ResNet50, given their proven effectiveness in semantic segmentation tasks \cite{Deshmane_2023,Graham_Vu_etal._2019,Mukasheva_Koishiyeva_etal._2024,Stringer_Wang_etal._2021}. However, approaches that include fine-tuning of pretrained CNNs, data augmentation, and stain normalization to partially increase data variability and address staining differences often fail to achieve the necessary generalization and robustness across diverse tissue types and staining conditions \cite{G._Wang_W._Li_etal._2018,Gao_Bagci_etal._2018,Karim_El_Khoury_Martin_Fockedey_etal._2021}.

To overcome these challenges, we leverage an encoder-decoder network that uses a foundation model as the encoder and a CNN upsampling decoder (\hyperref[fig:fig5]{Figure 5}) for simultaneous cell segmentation and classification in 2D patches extracted from WSIs. Foundation models with transformer-based architectures are viable alternatives to CNN-based encoders \cite{Shamshad_Khan_etal._2023,Sourget_2023}. They enable the creation of more advanced architectures that can decode or transform learned features more effectively \cite{Chen_Duan_etal._2023,Cheng_Misra_etal._2022,Xie_Wang_etal._2021}.

\begin{figure}[h!]
    \centering
    \includegraphics[width=\textwidth]{images/Figure_5.pdf}
    \caption{UNETR-like model with foundational model as backbone}
    \label{fig:fig5}
\end{figure}

By utilizing a transformer-based encoder, we incorporate global contextual information into the feature extraction process, which is a key advantage of such architectures \cite{Chen_Lu_etal._2021}. This foundation model integration facilitates accurate pixel-wise segmentation and classification without the need for extensive encoder training, thereby potentially improving generalization across varied cellular structures and tissue types.
In our implementation, we employ a modified UNETR \cite{Hatamizadeh_Tang_etal._2021} architecture that combines a vision transformer (ViT) \cite{Dosovitskiy_Beyer_etal._2021} encoder with a CNN-based decoder. The encoder utilizes the pretrained H-Optimus foundation model, which contains 1.1 billion parameters and is trained on over 500,000 H\&E stained WSIs \cite{Saillard_Jenatton_etal._2024}. We extract outputs from four evenly spaced transformer blocks $Z_i$, where $i \in [1, 14, 26, 38]$, to serve as residual connections for the CNN decoder. We select these blocks based on our observation that features from non-adjacent levels of the encoder lead to better overall performance on the test subset.

The CNN decoder upsamples the feature representations, acquired from the transformer blocks, to generate an intermediate vector that is handled by two task-specific layers that generate cell segmentation and classification masks. The first task-specific layer is the ‘Cellpose head’,  which is used to delineate cell instances. The layer generates horizontal and vertical gradient maps to form vector fields that are refined through gradient tracking in a post-processing step using the Cellpose algorithm \cite{Stringer_Wang_etal._2021}, known for its efficacy in cell segmentation tasks and generalizability across multiple domains \cite{Pachitariu_Stringer_2022,Stringer_Pachitariu_2024}. The second task-specific layer is the "Cell type head", which assigns labels to individual pixels. In the post-processing step, we determine the output classification label of each segmented cell instance by majority voting over the labeled pixels that comprise the cell in the segmentation map.

To evaluate model performance and measure the impact of adding a foundation model as backbone, we compare it to a ResNet50-based model. ResNet50 is a widely used solution for encoders in segmentation architectures in the medical domain \cite{Deshmane_2023,Graham_Vu_etal._2019,Mukasheva_Koishiyeva_etal._2024,Stringer_Wang_etal._2021}. For the H-Optimus-based model, we utilize frozen weights for the encoder and only fine-tune the decoder to take advantage of the extensive pre-training of the foundation model. For the ResNet50-based model we start with ImageNet \cite{Deng_Dong_etal.} weights and train both encoder and decoder parts. Hyperparameters for the training step are set to be identical, where possible, for comparable evaluation. 
For this evaluation, we deliberately use the PanNuke dataset to provide a standardized and controlled comparison between the H‑Optimus and ResNet50-based models (\hyperref[fig:S2]{Appendix Figure S2 (3)}). Specifically, we use two of the default PanNuke dataset splits (66\%) for training and validation, and reserve the third split (33\%) for testing.

To address the challenge of cell class imbalance in the PanNuke dataset, which is a common characteristic in most cell-level H\&E patch datasets, both models’ training processes employ a weighted loss function comprising cross-entropy and focal loss \cite{Lin_Goyal_etal._2018}. The focal loss component is adjusted with coefficients derived from each cell class' instance frequency, emphasizing learning from underrepresented classes and enhancing the model's sensitivity to rare but significant cellular patterns. The cross-entropy loss is augmented with spectral decoupling regularization \cite{Pezeshki_Kaba_etal._2021,Pohjonen_Stürenberg_etal._2022} and spatially varying label smoothing \cite{Islam_Glocker_2021}, which potentially stabilizes training and improves generalization in case of complex tissue morphologies. For optimization, we employ the \textit{AdamW} \cite{Loshchilov_Hutter_2019} to counter unbalanced class scenarios, with cosine annealing learning rate scheduler.

We utilize the scikit-learn library \cite{Van_der_Walt_Schönberger_etal._2014} and HoVer-Net \cite{Graham_Vu_etal._2019} implementations of $R^2$ (the coefficient of determination) and $PQ$ (panoptic quality) to evaluate our experiments. Complete mathematical formulations and detailed explanations of these metrics are provided in \hyperref[chap:S5]{Appendix S5}. To compute confidence intervals, we use nonparametric bootstrapping, where after calculating the metric on the full sample, we generated 1000 bootstrap replicates by resampling with replacement and then determined the 95\% confidence intervals as the 2.5th and 97.5th percentiles of the resulting empirical distribution.

%\hfill

The model comparisons are summarized in \hyperref[tab:2]{Table 2}. The H‑Optimus-based model achieves higher $R^2$ across all cell classes compared to the ResNet50-based model, which means that its predictions are more closely aligned with the PanNuke cell counts, indicating a stronger correlation with the observed data. Notably, the improvement of $R^2_{dead}$ may be an indicator of better global contextual representations provided by the foundation model backbone. In terms of segmentation and classification quality combined, measured by the PQ score, the H‑Optimus-based model demonstrates notable improvements across most cell classes. Overall, the average $R^2$ improved from 0.575 to 0.871, while the average $PQ$ score improved from 0.450 to 0.492, demonstrating better performance of the H-Optimus-based model.

\begin{table}[h!]
\renewcommand{\arraystretch}{1.5}
  \centering
  \caption{Cell quantification metrics for baseline and proposed models (CI 95\%).}
  \label{tab:2}
  \begin{tabular}{|l|c|c|}
    \hline
    %\rowcolor{gray!30}
    Metric             & Resnet50-based            & H-optimus-based              \\
    \hline
    $R^2_{neoplastic}$    & 0.681 (0.576--0.769)       & \textbf{0.941 (0.917--0.960)} \\
    \hline
    $R^2_{inflammatory}$  & 0.863 (0.778--0.903)       & \textbf{0.949 (0.918--0.966)} \\
    \hline
    $R^2_{connective}$    & 0.600 (0.488--0.698)       & 0.609 (0.436--0.772)          \\
    \hline
    $R^2_{dead}$          & 0.097 (-11.389--0.669)     & 0.925 (0.404--0.982)          \\
    \hline
    $R^2_{epithelial}$    & 0.635 (0.490--0.747)       & \textbf{0.930 (0.886--0.964)} \\
    \hline
    $PQ_{neoplastic}$       & 0.517 (0.499--0.535)       & \textbf{0.589 (0.575--0.604)} \\
    \hline
    $PQ_{inflammatory}$     & 0.455 (0.429--0.482)       & \textbf{0.528 (0.507--0.549)} \\
    \hline
    $PQ_{connective}$       & 0.416 (0.400--0.431)       & \textbf{0.451 (0.436--0.465)} \\
    \hline
    $PQ_{dead}$             & 0.374 (0.342--0.408)       & 0.292 (0.209--0.365)          \\
    \hline
    $PQ_{epithelial}$       & 0.488 (0.460--0.519)       & \textbf{0.599 (0.579--0.618)} \\
    \hline
  \end{tabular}
\end{table}

Our results  show that integrating the H‑Optimus foundation model within the UNETR architecture enhances the model's ability to segment and classify cells across diverse tissues from PanNuke data. The pretrained transformer encoder provides robust feature representations, resulting in higher average $R^2$ and $PQ$ scores compared to the CNN-based model. This leads to more reliable cell quantification and more accurate downstream analysis. Additionally, the streamlined fine-tuning process reduces computational overhead and training time, making the model more adaptable for new data.

Despite these advancements, the foundation model-based approach does not fully resolve all challenges related to cell segmentation and classification. We observe lower metric scores for underrepresented classes in the training data. Furthermore, foundation models typically encompass billions of parameters, resulting in substantial computational and memory requirements. It therefore poses challenges for deployment in resource-constrained environments, limiting their practical applicability in certain clinical settings.

\section{Model optimization via Knowledge Distillation}

To address the limitations posed by the extensive size of foundation models, we implement knowledge distillation — a model compression technique that leverages the teacher-student paradigm \cite{Hinton_Vinyals_etal._2015}. By training a smaller, more efficient student model to replicate the output of a larger, pre-trained teacher model, we retain performance while significantly reducing the model's complexity and resource requirements (\hyperref[fig:fig6]{Figure 6}).

\begin{figure}[h!]
    \centering
    \includegraphics[width=\textwidth, height=0.45\textheight, keepaspectratio]{images/Figure_6.pdf}
    \caption{Knowledge distillation framework for training a student model using a pre-trained teacher}
    \label{fig:fig6}
\end{figure}

We employ knowledge distillation to compress the H‑Optimus-based teacher model into a more efficient student model. The teacher model is the modified UNETR architecture with the H‑Optimus foundation model described in the previous chapter. The student model is based on a UNet architecture augmented with residual connections and incorporates a smaller ViT encoder with 9 million parameters \cite{Steiner_Kolesnikov_etal._2022,Wightman_2019}. 

First, we fine-tune the teacher model using the refined dataset from the cross-relabeling procedure (Section 2). Initially we train the decoder of the teacher model while keeping the encoder weights frozen. We split the refined dataset into train (70\%), validation (20\%) and test (10\%) subsets (\hyperref[fig:S2]{Appendix Figure S2 (4)}). During fine-tuning, we use the train and validation subsets, while leaving the test subset for model evaluation. We set the training procedure and model hyperparameters to be identical to those that were used to demonstrate the utility of foundation models for the simultaneous cell segmentation and classification task.

Next, we perform knowledge distillation from teacher to student using the refined dataset used to fine-tune the teacher model. The student model is trained to replicate the teacher model's outputs. We utilize a specialized loss function that aligns the student's predicted probability distribution with the teacher's, incorporating the teacher's class probability distribution derived from the output. Following the methodology of Hinton et al. \cite{Hinton_Vinyals_etal._2015}, we experiment with various hyperparameter settings for the temperature ($T$) and the balancing coefficients ($\alpha$ and $\beta$) in the loss function. We vary $T$ from 1 to 20 and adjust $\alpha$ and $\beta$ to balance the distillation and student losses. Through iterative tuning and evaluation, we identify that setting $T=14$, $\alpha=0.3$, and $\beta=0.7$ yields a configuration that converges and closely approximates the teacher model's performance during training.

Finally, we assess the performance of both models using the $R^2$ and $PQ$ (defined in \hyperref[chap:S5]{Appendix S5}) on the test set of the refined dataset (\hyperref[tab:3]{Table 3}). We observe that the 95\% confidence intervals overlap for most cell types, so we cannot claim statistically significant performance differences between the teacher and student models. One exception appears in the neoplastic class. The teacher model produces an $R^2$ of 0.919, while the student model shows an $R^2$ of 0.852. In addition, the student model achieves higher $PQ$ values for the neoplastic and connective classes, though the confidence intervals show overlap.

\begin{table}[h!]
\renewcommand{\arraystretch}{1.5}
  \centering
  \caption{Cell quantification metrics for teacher and distilled student models (CI 95\%).}
  \label{tab:3}
  \begin{tabular}{|l|c|c|}
    \hline
    %\rowcolor{gray!30}
    Metric & Teacher & Student \\
    \hline
    $R^2_{neoplastic}$    & \textbf{0.919} (0.898--0.939) & 0.852 (0.800--0.891) \\
    \hline
    $R^2_{lymphocyte}$    & 0.969 (0.956--0.977)         & 0.969 (0.956--0.978) \\
    \hline
    $R^2_{connective}$    & 0.694 (0.548--0.809)         & 0.618 (0.469--0.741) \\
    \hline
    $R^2_{dead}$          & 0.755 (0.400--0.908)         & 0.424 (0.100--0.731) \\
    \hline
    $R^2_{epithelial}$    & 0.922 (0.870--0.958)         & 0.843 (0.738--0.917) \\
    \hline
    $R^2_{macrophage}$    & 0.384 (-0.369--0.724)        & 0.704 (0.352--0.859) \\
    \hline
    $R^2_{neutrofil}$     & 0.854 (0.578--0.929)         & 0.833 (0.502--0.925) \\
    \hline
    $PQ_{neoplastic}$       & 0.581 (0.569--0.593)         & 0.601 (0.588--0.613) \\
    \hline
    $PQ_{lymphocyte}$       & 0.536 (0.520--0.553)         & 0.563 (0.544--0.579) \\
    \hline
    $PQ_{connective}$       & 0.436 (0.421--0.451)         & 0.457 (0.441--0.474) \\
    \hline
    $PQ_{dead}$             & 0.272 (0.235--0.315)         & 0.279 (0.201--0.369) \\
    \hline
    $PQ_{epithelial}$       & 0.522 (0.500--0.545)         & 0.530 (0.506--0.555) \\
    \hline
    $PQ_{macrophage}$       & 0.524 (0.459--0.588)         & 0.474 (0.405--0.543) \\
    \hline
    $PQ_{neutrofil}$        & 0.541 (0.490--0.592)         & 0.565 (0.522--0.607) \\
    \hline
  \end{tabular}
\end{table}


We further decompose the $PQ$ metric into its $SQ$ and $DQ$ components (\hyperref[tab:S6]{Appendix Table S6}). Both models produce nearly identical $SQ$ values, which indicates that they predict instance boundaries with similar precision. Although the student model shows some improvement in $DQ$ scores for certain classes, the confidence intervals overlap and do not confirm a statistically significant difference.

We observe that the student and teacher models yield comparable detection performance despite the student model using a much smaller and simpler architecture. A model with fewer parameters reduces the risk of overfitting when training data are scarce relative to the model’s complexity \cite{Farias_Ludermir_etal._2022}. The knowledge distillation process also encourages the student model to focus on the most generalizable detection features learned from the teacher. These factors enable the student model to achieve similar detection performance across different cell types.

Additionally, considering the model sizes reported in \hyperref[tab:4]{Table 4}, the distilled model achieves a significant reduction compared to the teacher model, with a 48-fold decrease in parameter count and a 5.5-fold reduction in on-disk size. In inference mode, the teacher model requires 16 GB of VRAM for a batch size of 32, while the distilled model only needs 3 GB of VRAM for the same batch size. These reductions make the distilled model significantly more practical for fine-tuning and deployment in resource-constrained environments.

\begin{table}[h!]
\renewcommand{\arraystretch}{1.5}
  \centering
  \caption{Parameter counts and size of teacher and distilled model}
  \label{tab:4}
  \adjustbox{max width=\textwidth}{%
  \begin{tabular}{|l|c|c|c|}
    \hline
    %\rowcolor{gray!30}
    Metric & H-optimus-based (Teacher) & mobileViT-based (Student) & Magnitude of difference \\
    \hline
    Parameters count       & 1,158,917,906   & \textbf{24,093,393}   & \textbf{48x}  \\
    \hline
    Estimated Total Size (MB) & 87,912       & \textbf{15,935}    & \textbf{5.5x} \\
    \hline
  \end{tabular}%
}
\end{table}

%\hfill

With recent advancements in complex network architectures and the use of pretrained encoders to achieve state-of-the-art performance \cite{Baumann_Dislich_etal._2024,Hörst_Rempe_etal._2024} in cell segmentation and classification tasks, model size, computational complexity, and processing times have increased. This limits the scalability and accessibility of these models. As we demonstrate, this may be mitigated using knowledge distillation. Studies in the field of natural language processing have demonstrated the efficacy of knowledge distillation in retaining the capabilities of the teacher model while achieving significant reductions in size and complexity \cite{Huangpu_Gao_2024,Sun_Yu_etal.}. 

We demonstrate the feasibility of knowledge distillation in digital pathology, specifically for cell segmentation and classification tasks. Moreover, we achieve this performance while also significantly reducing the parameter count. In addressing the challenge of knowledge transfer, we found that distillation from a transformer-based model to a smaller transformer is more straightforward than attempting to map transformer features to CNN blocks. In our experiments, using a CNN-based network as a student results in worse cell quantification performance due to the structural constraints of CNN feature space dimensions. 

Although our primary approach relies on a transformer-based student model that performs well, it can be further optimized to incorporate advantages from CNN architectures. For example, employing alternative techniques such as using ViT adapters \cite{Chen_Duan_etal._2023} or $1 \times 1$ convolutions to adjust feature map sizes may be beneficial for harnessing CNN advantages like enhanced local feature extraction. Moreover, if additional performance improvements are desired, the process can be further enhanced by applying supplementary knowledge distillation techniques, such as self-distillation \cite{Zhang_Song_etal._2019} or online distillation \cite{Houyon_Cioppa_etal._2023}.

Despite these promising results, further validation on independent datasets is necessary to fully understand the model's limitations. Underrepresented classes may pose challenges when addressing complex cases. Pathologists need to validate these models to adopt them in clinical settings. While the distilled models are smaller and more deployable, a technological gap persists because pathologists traditionally rely on established methods for inspecting WSIs and diagnosing diseases. Addressing the complexities involved in deploying models for inference and supporting pathologists in adopting new tools is essential for integrating these models into clinical workflows.

\section{Model integration with QuPath}
Digital pathology tools with graphical user interfaces are essential for visualizing and analyzing WSIs. To make our student model useful in clinical pathology workflows, it needs to be integrated into a tool that enables inspecting regions, creating annotations, and providing quantitative analyses of biomarkers. Therefore, we integrate the trained student model from the previous chapter into the QuPath open‑source platform \cite{Bankhead_Loughrey_etal._2017}. QuPath provides the required annotation, visualization, and analysis tools to interpret complex histological data, including workflows for cell segmentation, classification, and quantification (\hyperref[fig:fig7]{Figure 7}). 

\begin{figure}[h!]
    \centering
    \includegraphics[width=\textwidth]{images/Figure_7.pdf}
    \caption{Visualization of model-generated cell quantification annotations (left) and the corresponding unannotated slide (right) in QuPath}
    \label{fig:fig7}
\end{figure}

To identify the regions in a WSI critical for prognosticating tumor development, such as specific tumor areas or border regions without overlapping healthy tissue, the pathologist uses QuPath to outline these regions. Then, the pathologist initiates a cell segmentation and classification script through the QuPath interface for the selected regions. The resulting annotations and quantified cell information are then directly overlaid onto the WSI in the QuPath interface. Additional design and implementation details are in \hyperref[chap:S7]{Appendix S7}. 

Two common approaches for integrating deep learning models into QuPath are Java‑based native QuPath extensions \cite{Goldsborough_Philps_etal._2024} and the execution of RESTful API requests to a model server coupled with handling the response via an extension, as demonstrated in the application of cell segmentation models applied to immunofluorescence images \cite{Sugawara_2023}. While the community is actively working on these integration strategies, there is currently no universal solution that fully addresses all integration and performance requirements.

Extensions may offer better integration with QuPath, allowing slightly improved performance and more widespread usage of the built-in QuPath models, but they lack the flexibility to customize models and modify their behavior. For example, the newest version of QuPath includes models such as StarDist \cite{Weigert_Schmidt} and InstanSeg \cite{Goldsborough_Philps_etal._2024} that can perform cell segmentation. Both models pose limitations when applied to simultaneous cell segmentation and classification. StarDist performs well only on convex, round shapes by design, whereas some neoplastic, inflammatory, and connective cells exhibit complex and non-convex shapes. InstanSeg provides only semantic segmentation without assigning classes to the segmented cells.

%\hfill

In contrast, our approach offers an alternative integration strategy. It utilizes the paquo library to directly interact with QuPath’s internal application programming interface from within Python. This enables data exchange and processing without the need for intermediate conversion steps and provides greater control over model customization, retraining, and the incorporation of custom processing steps.

The integration of our custom model with QuPath underscores its potential to significantly enhance the diagnostic process by reducing the time burden on pathologists and enabling them to focus on more complex interpretative tasks using familiar software. Leveraging a tool that is already well-established among pathologists increases the likelihood of its adoption into daily clinical workflows. The quantitative data generated through the automated workflow is critical for both clinical decision-making and research, facilitating more accurate biomarker analysis, enabling robust statistical evaluations, and supporting hypothesis generation and testing. Additionally, by streamlining cell segmentation and classification, the tool enhances the scalability and reproducibility of pathological assessments, ultimately contributing to improved diagnostic accuracy and patient outcomes.

\section{Conclusion and future work}

In this study, we address critical challenges in digital pathology and tackle the usability and deployment issues of the developed models in standard computing environments without the need for high-performance computing systems. Our multi-faceted approach encompasses data refinement through cross-relabeling, leveraging foundation models for robust cell segmentation and classification, optimizing model performance via knowledge distillation, and integrating the optimized model into the QuPath software for practical application. This approach is used to construct a capable, versatile, and adjustable model for cell segmentation and classification, with enhanced performance and usability.

\begin{sloppypar}
While our approach shows potential in the field of computational pathology, certain limitations persist. 
For example, our implementation currently exhibits lower performance in detecting macrophages. 
This serves as an instance of the broader challenge of accurately identifying complex cell types. In order to address this issue, extending our approach to incorporate additional data sources, exploring alternative modeling approaches, and integrating other imaging modalities such as immunohistochemical staining may help improve detection accuracy. Moreover, although the distilled model reduces computational demands, integrating advanced deep learning models into clinical practice requires addressing technological gaps and potential resistance to adopting new tools within established diagnostic processes.
\end{sloppypar}

Future work could focus on several key areas to refine the proposed approach and facilitate its adoption in clinical environments. Enhancing the cell-relabeling process with additional datasets \cite{Graham_Jahanifar_etal._2021} could improve the representation of underrepresented cell types and enhance overall model performance. Also, incorporating additional data sources, such as multi-modal imaging or complementary staining methods, may address limitations related to cell type differentiation and class imbalance. Exploring other foundation models \cite{Vorontsov_Bozkurt_etal._2024,Zimmermann_Vorontsov_etal._2024} or introducing additional modalities \cite{Ding_Wagner_etal._2024,Vaidya_Zhang_etal._2025} may provide alternative architectures better suited to specific tasks or offer improved efficiency. Implementing more complex knowledge distillation techniques \cite{Houyon_Cioppa_etal._2023,Zhang_Song_etal._2019} could further optimize the model's performance and adaptability. Additionally, deeper integration with QuPath or other digital pathology software could provide pathologists more control over cell quantification analysis directly within the QuPath interface, thereby increasing accessibility and usability. Such enhancements would not only refine model performance but also ensure greater adaptability and scalability within various clinical environments. Finally, extensive validation of the model by pathologists and benchmarking against independent datasets are essential steps toward establishing the model's reliability and fostering confidence in its clinical utility.

\section*{Acknowledgments} 
This work was funded in part by the Research Council of Norway grant no. 309439 SFI Visual Intelligence, and the North Norwegian Health Authority grant no. HNF1521-20.

\bibliographystyle{IEEEtran}
\begin{sloppypar}
\begin{thebibliography}{99}

\bibitem{chaplot2020neural} Chaplot, Devendra Singh, et al. "Neural topological slam for visual navigation." Proceedings of the IEEE/CVF conference on computer vision and pattern recognition. 2020.

\bibitem{maksymets2021thda} Maksymets, Oleksandr, et al. "Thda: Treasure hunt data augmentation for semantic navigation." Proceedings of the IEEE/CVF International Conference on Computer Vision. 2021.

\bibitem{mezghan2022memory} Mezghan, Lina, et al. "Memory-augmented reinforcement learning for image-goal navigation." 2022 IEEE/RSJ International Conference on Intelligent Robots and Systems (IROS). IEEE, 2022.

\bibitem{al2022zero} Al-Halah, Ziad, Santhosh Kumar Ramakrishnan, and Kristen Grauman. "Zero experience required: Plug \& play modular transfer learning for semantic visual navigation." Proceedings of the IEEE/CVF Conference on Computer Vision and Pattern Recognition. 2022.

\bibitem{ye2021auxiliary} Ye, Joel, et al. "Auxiliary tasks and exploration enable objectgoal navigation." Proceedings of the IEEE/CVF international conference on computer vision. 2021.

\bibitem{chaplot2020object} Chaplot, Devendra Singh, et al. "Object goal navigation using goal-oriented semantic exploration." Advances in Neural Information Processing Systems 33 (2020)

\bibitem{ramakrishnan2022poni} Ramakrishnan, Santhosh Kumar, et al. "Poni: Potential functions for objectgoal navigation with interaction-free learning." Proceedings of the IEEE/CVF Conference on Computer Vision and Pattern Recognition. 2022.

\bibitem{ramrakhya2022habitat} Ramrakhya, Ram, et al. "Habitat-web: Learning embodied object-search strategies from human demonstrations at scale." Proceedings of the IEEE/CVF Conference on Computer Vision and Pattern Recognition. 2022.

\bibitem{mousavian2019visual} Mousavian, Arsalan, et al. "Visual representations for semantic target driven navigation." 2019 International Conference on Robotics and Automation (ICRA). IEEE, 2019.

\bibitem{dhariwal2021diffusion} Dhariwal, Prafulla, and Alexander Nichol. "Diffusion models beat gans on image synthesis." Advances in neural information processing systems 34 (2021)

\bibitem{ho2022classifier} Ho, Jonathan, and Tim Salimans. "Classifier-free diffusion guidance." arXiv preprint arXiv:2207.12598 (2022).

\bibitem{nichol2021glide} Nichol, Alex, et al. "Glide: Towards photorealistic image generation and editing with text-guided diffusion models." arXiv preprint arXiv:2112.10741 (2021)

\bibitem{brooks2023instructpix2pix} Brooks, Tim, Aleksander Holynski, and Alexei A. Efros. "Instructpix2pix: Learning to follow image editing instructions." Proceedings of the IEEE/CVF Conference on Computer Vision and Pattern Recognition. 2023.

\bibitem{fu2023guiding} Fu, Tsu-Jui, et al. "Guiding instruction-based image editing via multimodal large language models." arXiv preprint arXiv:2309.17102 (2023).

\bibitem{geng2024instructdiffusion} Geng, Zigang, et al. "Instructdiffusion: A generalist modeling interface for vision tasks." Proceedings of the IEEE/CVF Conference on Computer Vision and Pattern Recognition. 2024.

\bibitem{zhou2024minedreamer} Zhou, Enshen, et al. "Minedreamer: Learning to follow instructions via chain-of-imagination for simulated-world control." arXiv preprint arXiv:2403.12037 (2024).

\bibitem{zhou2023esc} Zhou, Kaiwen, et al. "Esc: Exploration with soft commonsense constraints for zero-shot object navigation." International Conference on Machine Learning. PMLR, 2023.

\bibitem{yu2023l3mvn} Yu, Bangguo, Hamidreza Kasaei, and Ming Cao. "L3mvn: Leveraging large language models for visual target navigation." 2023 IEEE/RSJ International Conference on Intelligent Robots and Systems (IROS). IEEE, 2023.

\bibitem{gadre2023cows} Gadre, Samir Yitzhak, et al. "Cows on pasture: Baselines and benchmarks for language-driven zero-shot object navigation." Proceedings of the IEEE/CVF Conference on Computer Vision and Pattern Recognition. 2023.

\bibitem{shah2023navigation} Shah, Dhruv, et al. "Navigation with large language models: Semantic guesswork as a heuristic for planning." Conference on Robot Learning. PMLR, 2023.

\bibitem{cai2024bridging} Cai, Wenzhe, et al. "Bridging zero-shot object navigation and foundation models through pixel-guided navigation skill." 2024 IEEE International Conference on Robotics and Automation (ICRA). IEEE, 2024.

\bibitem{yu2023co} Yu, Bangguo, Hamidreza Kasaei, and Ming Cao. "Co-NavGPT: Multi-robot cooperative visual semantic navigation using large language models." arXiv preprint arXiv:2310.07937 (2023).

\bibitem{wu2024voronav} Wu, Pengying, et al. "Voronav: Voronoi-based zero-shot object navigation with large language model." arXiv preprint arXiv:2401.02695 (2024).

\bibitem{qin2023mp5} Qin, Yiran, et al. "Mp5: A multi-modal open-ended embodied system in minecraft via active perception." arXiv preprint arXiv:2312.07472 (2023).

\bibitem{du2024learning} Du, Yilun, et al. "Learning universal policies via text-guided video generation." Advances in Neural Information Processing Systems 36 (2024).

\bibitem{ajay2024compositional} Ajay, Anurag, et al. "Compositional foundation models for hierarchical planning." Advances in Neural Information Processing Systems 36 (2024).

\bibitem{liang2024skilldiffuser} Liang, Zhixuan, et al. "Skilldiffuser: Interpretable hierarchical planning via skill abstractions in diffusion-based task execution." Proceedings of the IEEE/CVF Conference on Computer Vision and Pattern Recognition. 2024.

\bibitem{heusel2017gans} Heusel, Martin, et al. "Gans trained by a two time-scale update rule converge to a local nash equilibrium." Advances in neural information processing systems 30 (2017).

\bibitem{zhang2018unreasonable} Zhang, Richard, et al. "The unreasonable effectiveness of deep features as a perceptual metric." Proceedings of the IEEE conference on computer vision and pattern recognition. 2018.

\bibitem{brown2020language} Brown, Tom B. "Language models are few-shot learners." arXiv preprint arXiv:2005.14165 (2020).

\bibitem{podell2023sdxl} Podell, Dustin, et al. "Sdxl: Improving latent diffusion models for high-resolution image synthesis." arXiv preprint arXiv:2307.01952 (2023).

\bibitem{brohan2022rt} Brohan, Anthony, et al. "Rt-1: Robotics transformer for real-world control at scale." arXiv preprint arXiv:2212.06817 (2022).

\bibitem{brohan2023rt} Brohan, Anthony, et al. "Rt-2: Vision-language-action models transfer web knowledge to robotic control." arXiv preprint arXiv:2307.15818 (2023).

\bibitem{li2024manipllm} Li, Xiaoqi, et al. "Manipllm: Embodied multimodal large language model for object-centric robotic manipulation." Proceedings of the IEEE/CVF Conference on Computer Vision and Pattern Recognition. 2024.

\bibitem{shah2023vint} Shah, Dhruv, et al. "ViNT: A foundation model for visual navigation." arXiv preprint arXiv:2306.14846 (2023).

\bibitem{liu2024visual} Liu, Haotian, et al. "Visual instruction tuning." Advances in neural information processing systems 36 (2024).

\bibitem{hu2021lora} Hu, Edward J., et al. "Lora: Low-rank adaptation of large language models." arXiv preprint arXiv:2106.09685 (2021).

\bibitem{qin2023supfusion} Qin, Yiran, et al. "SupFusion: Supervised LiDAR-camera fusion for 3D object detection." Proceedings of the IEEE/CVF International Conference on Computer Vision. 2023.

\bibitem{qin2024worldsimbench} Qin, Yiran, et al. "Worldsimbench: Towards video generation models as world simulators." arXiv preprint arXiv:2410.18072 (2024).

\bibitem{yu2025gamefactory} Yu, Jiwen, et al. "GameFactory: Creating New Games with Generative Interactive Videos." arXiv preprint arXiv:2501.08325 (2025).

\bibitem{zhou2024code} Zhou, Enshen, et al. "Code-as-Monitor: Constraint-aware Visual Programming for Reactive and Proactive Robotic Failure Detection." arXiv preprint arXiv:2412.04455 (2024).

\bibitem{zhang2024ad} Zhang, Zaibin, et al. "AD-H: Autonomous Driving with Hierarchical Agents." arXiv preprint arXiv:2406.03474 (2024).

\bibitem{wang2024toward} Wang, Chaoqun, et al. "Toward Accurate Camera-based 3D Object Detection via Cascade Depth Estimation and Calibration." arXiv preprint arXiv:2402.04883 (2024).

\bibitem{huang2024story3d} Huang, Yuzhou, et al. "Story3d-agent: Exploring 3d storytelling visualization with large language models." arXiv preprint arXiv:2408.11801 (2024).

\bibitem{savinov2018semi} Savinov, Nikolay, Alexey Dosovitskiy, and Vladlen Koltun. "Semi-parametric topological memory for navigation." arXiv preprint arXiv:1803.00653 (2018).

\bibitem{majumdar2022zson} Majumdar, Arjun, et al. "Zson: Zero-shot object-goal navigation using multimodal goal embeddings." Advances in Neural Information Processing Systems 35 (2022): 32340-32352.

\bibitem{yadav2023offline} Yadav, Karmesh, et al. "Offline visual representation learning for embodied navigation." Workshop on Reincarnating Reinforcement Learning at ICLR 2023. 2023.

\bibitem{yadav2023ovrl} Yadav, Karmesh, et al. "Ovrl-v2: A simple state-of-art baseline for imagenav and objectnav." arXiv preprint arXiv:2303.07798 (2023).

\bibitem{sun2024fgprompt} Sun, Xinyu, et al. "FGPrompt: fine-grained goal prompting for image-goal navigation." Advances in Neural Information Processing Systems 36 (2024).

\bibitem{zhu2017target} Zhu, Yuke, et al. "Target-driven visual navigation in indoor scenes using deep reinforcement learning." 2017 IEEE international conference on robotics and automation (ICRA). IEEE, 2017.

\bibitem{koh2024generating} Koh, Jing Yu, Daniel Fried, and Russ R. Salakhutdinov. "Generating images with multimodal language models." Advances in Neural Information Processing Systems 36 (2024).

\bibitem{krantz2022instance} Krantz, Jacob, et al. "Instance-specific image goal navigation: Training embodied agents to find object instances." arXiv preprint arXiv:2211.15876 (2022).

\bibitem{schulman2017proximal} Schulman, John, et al. "Proximal policy optimization algorithms." arXiv preprint arXiv:1707.06347 (2017).

\bibitem{anderson2018evaluation} Anderson, Peter, et al. "On evaluation of embodied navigation agents." arXiv preprint arXiv:1807.06757 (2018).

\bibitem{lin2024navcot} Lin, Bingqian, et al. "NavCoT: Boosting LLM-Based Vision-and-Language Navigation via Learning Disentangled Reasoning." arXiv preprint arXiv:2403.07376 (2024).

\bibitem{NavGPT} Zhou, Gengze, Yicong Hong, and Qi Wu. "Navgpt: Explicit reasoning in vision-and-language navigation with large language models." Proceedings of the AAAI Conference on Artificial Intelligence.

\bibitem{hahn2021no} Hahn, Meera, et al. "No rl, no simulation: Learning to navigate without navigating." Advances in Neural Information Processing Systems 34 (2021): 26661-26673.

\bibitem{li2025t2isafety} Li, Lijun, et al. "T2ISafety: Benchmark for Assessing Fairness, Toxicity, and Privacy in Image Generation." arXiv preprint arXiv:2501.12612 (2025).

\bibitem{an2024agfsync} An, Jingkun, et al. "AGFSync: Leveraging AI-Generated Feedback for Preference Optimization in Text-to-Image Generation." arXiv preprint arXiv:2403.13352 (2024).


\end{thebibliography}
\end{sloppypar}

\clearpage
\beginsupplement
\section*{Appendix}
\renewcommand{\thesubsection}{S\arabic{subsection}}

\subsection{\label{chap:S1}PanNuke and MoNuSAC preprocessing}
The PanNuke dataset comprises a set of 7,901 RGB patches, each with dimensions of $256 \times 256$ pixels, which we set as the standard patch size for our analysis. In contrast, the MoNuSAC dataset encompasses 294 images of heterogeneous dimensions. To standardize the MoNuSAC images with our experiments, we implement a standardization protocol. Specifically, for images exceeding the dimensions of $256 \times 256$ pixels, we segment them into equal-sized patches and apply mirror padding to the remaining portions to avoid information loss at the peripherals. Patches with dimensions less than $128 \times 128$ pixels are excluded from the dataset due to the insufficient resolution to capture relevant cellular details. For patches where either dimension falls between 128 and 256 pixels, we employ upsampling to achieve the standard patch size. As a result, we obtain a total of 2,823 RGB patches derived from the MoNuSAC dataset for subsequent analysis. For additional details on the MoNuSAC data preparation process, refer to the source code \cite{Shvetsov_2025a}.
\clearpage

\subsection{\label{chap:S2}Data usage for the methodology}

\counterwithin{figure}{subsection}
\renewcommand{\thefigure}{S\arabic{subsection}}

\begin{figure}[h!]
    \centering
    \includegraphics[width=\textwidth, height=0.85\textheight, keepaspectratio]{images/A2.pdf}
    \caption{Overview of the methodology for cross-labeling, dataset refinement, and model comparison. (1) Cross-relabeling - training and testing cell classification models, (2) Cross-relabeling - using cell classification models to create refined dataset, (3) Fine-tuning and training models for comparison, (4) Student knowledge distillation with refined dataset}
    \label{fig:S2}
\end{figure}
\clearpage

\subsection{\label{chap:S3}Confusion matrices for classification models}
\counterwithin{figure}{subsection}
\renewcommand{\thefigure}{S\arabic{subsection}.\arabic{figure}}

\begin{figure}[h!]
    \centering
    \includegraphics[width=\textwidth, height=0.4\textheight, keepaspectratio]{images/A3_1.pdf}
    \caption{Confusion matrix for PanNuke trained model}
    \label{fig:S3.1}
\end{figure}

\begin{figure}[h!]
    \centering
    \includegraphics[width=\textwidth, height=0.4\textheight, keepaspectratio]{images/A3_2.pdf}
    \caption{Confusion matrix for MoNuSAC trained model}
    \label{fig:S3.2}
\end{figure}

\clearpage

\subsection{\label{chap:S4}Datasets cell counts}

\counterwithin{table}{subsection}
\renewcommand{\thetable}{S\arabic{subsection}}

\begin{table}[h!]
\renewcommand{\arraystretch}{2.0}
\centering
\caption{\label{tab:S4}Cell counts for PanNuke, MoNuSAC and refined datasets. Numbers in parentheses indicate preprocessed cell counts for cell classifier models training and testing.}
%\adjustbox{max width=\textwidth}{%
\begin{tabular}{|l|c|c|c|}
\hline
%\rowcolor{gray!30}
Cell type & PanNuke & MoNuSAC & Refined \\
\hline
Neoplastic & 77,403 (68,031) & - & 105,451 \\
\hline
Epithelial & 26,572 (23,207) & - & 29,926 \\
\hline
Epithelial (benign and malignant) & - & 31,402 & - \\
\hline
Inflammatory & 32,276 & - & - \\
\hline
Lymphocytes & - & 37,045 (33,104) & 65,275 \\
\hline
Neutrophils & - & 1,355 (1,252) & 3,833 \\
\hline
Macrophage & - & 1,842 (1,695) & 3,410 \\
\hline
Dead & 2,908 & - & 2,908 \\
\hline
Connective & 50,585 & - & 50,585 \\
\hline
\end{tabular}
%
%}
\end{table}



\clearpage

\subsection{\label{chap:S5}Definition of validation metrics}
\counterwithin{equation}{subsection}
\renewcommand{\theequation}{\arabic{equation}}

\subsubsection{\label{chap:S5.1}R\textsuperscript{2}}
The coefficient of determination, denoted as $R^2$, is a statistical measure that represents the proportion of variance in the dependent variable that is predictable from the independent variables. In the context of cell quantification in pathology, $R^2$ is used to assess how well the predicted quantities of different cell types in a patch align with the actual quantities observed in the ground truth data, with higher values representing more accurate quantification. $R^2$ is defined as
\begin{equation*}
R^2 = 1 - \frac{\sum_{i=1}^n (y_i - \hat{y}_i)^2}{\sum_{i=1}^n (y_i - \bar{y})^2},
\end{equation*}
where $y_i$ represents the actual number of cells of a specific type in the $i$-th image, $\hat{y}_i$ represents the predicted number of cells of that type in the $i$-th image, $\bar{y}$ is the mean of the actual numbers across all images, and $n$ is the total number of images in the dataset.

The $R^2$ metric has a range of $(-\infty, 1]$. An $R^2$ of 1 indicates perfect prediction, where all predicted values exactly match the actual values. An $R^2$ of 0 suggests that the model explains none of the variability of the response data around its mean. If $R^2$ is negative, it indicates that the model performs worse than a model that simply predicts the mean of the actual values for all observations.

\subsubsection{\label{chap:S5.2}PQ}
Panoptic Quality ($PQ$) is a comprehensive metric used to evaluate the performance of segmentation models in tasks that require both instance segmentation and classification. $PQ$ provides a single score that encapsulates both the detection accuracy (i.e., how many objects were correctly identified) and the segmentation quality (i.e., how accurately the objects' boundaries were delineated). This metric is particularly useful in multiclass scenarios where each pixel is classified into distinct categories, such as different cell types in pathology images.

$PQ$ is calculated as the product of two terms: Detection Quality ($DQ$) and Segmentation Quality ($SQ$). It can be expressed as
\begin{equation*}
PQ = DQ \cdot SQ,
\end{equation*}
where
\begin{equation*}
DQ = \frac{TP}{TP + 0.5\, FP + 0.5\, FN},
\end{equation*}
\begin{equation*}
SQ = \frac{\sum_{(p, g) \in \mathcal{M}} IoU(p, g)}{TP}.
\end{equation*}
In these formulas, $TP$ denotes the number of correctly matched instances between ground truth and prediction, $FP$ denotes the predicted instances that have no corresponding ground truth, $FN$ denotes the ground truth instances that were not detected, $IoU(p, g)$ is the Intersection over Union for a pair of matched instances $p$ (prediction) and $g$ (ground truth), and $\mathcal{M}$ is the set of matched pairs.

The $PQ$ metric is calculated for each class and is averaged across classes to provide a global performance measure.

The $PQ$ score has a range of $[0, 1.0]$, where a higher score indicates better performance in both detecting and segmenting the instances correctly. A $PQ$ of 1 signifies perfect identification and segmentation of all instances, whereas a $PQ$ of 0 indicates that no instances were correctly identified and segmented.

\clearpage

\subsection{\label{chap:S6}Segmentation and Detection quality metrics for teacher and student models}

\begin{table}[h!]
\renewcommand{\arraystretch}{2.0}
\centering
\caption{Segmentation and detection quality for student and teacher models (CI 95\%)}
\label{tab:S6}
%\adjustbox{max width=\textwidth}{%
\begin{tabular}{|l|c|c|}
\hline
%\rowcolor{gray!30}
Metric & Teacher & Student \\
\hline
$SQ_{neoplastic}$ & 0.819 (0.815--0.823) & 0.824 (0.819--0.828) \\
\hline
$SQ_{lymphocyte}$ & 0.795 (0.788--0.802) & 0.790 (0.783--0.796) \\
\hline
$SQ_{connective}$ & 0.770 (0.762--0.776) & 0.780 (0.772--0.786) \\
\hline
$SQ_{dead}$ & 0.659 (0.623--0.688) & 0.657 (0.624--0.695) \\
\hline
$SQ_{epithelial}$ & 0.780 (0.770--0.790) & 0.788 (0.779--0.797) \\
\hline
$SQ_{macrophage}$ & 0.788 (0.760--0.810) & 0.757 (0.730--0.783) \\
\hline
$SQ_{neutrofil}$ & 0.782 (0.761--0.801) & 0.775 (0.759--0.792) \\
\hline
$DQ_{neoplastic}$ & 0.706 (0.692--0.719) & 0.727 (0.712--0.741) \\
\hline
$DQ_{lymphocyte}$ & 0.675 (0.656--0.698) & 0.713 (0.691--0.734) \\
\hline
$DQ_{connective}$ & 0.566 (0.546--0.584) & 0.583 (0.565--0.602) \\
\hline
$DQ_{dead}$ & 0.410 (0.361--0.465) & 0.435 (0.306--0.561) \\
\hline
$DQ_{epithelial}$ & 0.668 (0.639--0.694) & 0.673 (0.644--0.702) \\
\hline
$DQ_{macrophage}$ & 0.657 (0.583--0.727) & 0.615 (0.531--0.703) \\
\hline
$DQ_{neutrofil}$ & 0.691 (0.625--0.753) & 0.729 (0.679--0.778) \\
\hline
\end{tabular}
%
%}
\end{table}

\clearpage

\subsection{\label{chap:S7}QuPath integration method}
We adopt an integration strategy leveraging the paquo \cite{Bayer_AG} library, a Python package that enables direct interaction with QuPath’s internal API, thereby facilitating seamless data exchange without intermediate conversion steps. The data processing pipeline (\hyperref[fig:S7]{Appendix Figure S7}) begins with the acquisition of WSIs and their associated annotations from QuPath, which are represented as Shapely \cite{Gillies_Wel_etal._2024} polygons. Utilizing paquo, we directly read, create, and modify these annotations and detections within a QuPath project in the Python environment. Images are then cropped using these polygons and processed by cell segmentation and classification models employing standard vision processing toolkits such as OpenCV, pyvips, and PyTorch. Additionally, QuPath employs Groovy scripts to initiate a Python process that starts the entire pipeline from QuPath graphical interface: fetching polygons, extracting images from them, and running deep learning model inference on the cropped images. 
The results are returned to QuPath, leveraging paquo's Python bindings to manipulate QuPath data while minimizing the computational overhead typically associated with cross-environment communication.

\counterwithin{figure}{subsection}
\renewcommand{\thefigure}{S\arabic{subsection}}

\begin{figure}[h!]
    \centering
    \includegraphics[width=\textwidth]{images/A7.pdf}
    \caption{QuPath integration workflow using Python environment}
    \label{fig:S7}
\end{figure}

Compared to traditional workflows that involve exporting annotations as GeoJSON, classifying them in Python, and reimporting them into QuPath, our approach offers several advantages. We eliminate the need to switch between programming languages, providing a cohesive and streamlined development process entirely within QuPath software and removing the necessity to use other tools. Meanwhile, we avoid storing annotations as intermediate JSON files unless required for external use or archiving. By conducting the entire inference and post-processing workflow within the Python environment, we leverage the power and flexibility of Python libraries for image processing and machine learning. This approach also enables adjustments to any set of labels and models, thereby improving its applicability.

%\hfill

The distilled model and QuPath integration code are packaged into a Docker container, enabling streamlined execution with the Docker engine. Detailed integration code and deployment instructions can be found in the GitHub repository \cite{Shvetsov_2025b}.

Despite these benefits, we acknowledge that the paquo library is a proof‑of‑concept project in its early development stage and has not been tested across all versions of QuPath.

\clearpage

\subsection{\label{chap:S8}Data and code availability statement}
All datasets, models, and code used in this study are publicly available and can be obtained from the repositories listed below. 
The PanNuke \cite{Gamper_Koohbanani_etal._2019} and MoNuSAC \cite{Verma_Kumar_etal._2021} datasets are publicly accessible, and download information along with detailed descriptions can be found in their respective articles. Preprocessing scripts for PanNuke and MoNuSAC data, as well as individual cell extraction scripts, are available on GitHub \cite{Shvetsov_2025a}. The H-Optimus foundation model used in our experiments can be downloaded from the HuggingFace repository \cite{hoptimus2024}, and model information is available on GitHub \cite{Saillard_Jenatton_etal._2024}. In addition, the integration code for QuPath and the distilled model packaged in a Docker container are provided in the repository \cite{Shvetsov_2025b}, and paquo Python library is available from the authors GitHub repository \cite{Bayer_AG}.
\clearpage

\end{document}

The average results of lab test outcome prediction for three datasets are presented in Table~\ref{tab:main}.

\ours{} achieves the best performance in both NMAE and SMAPE across all datasets, except for NMAE in HiRID, where it still performs competitively.
Notably, naive approaches (Naive and Naive($\mu$)) show relatively higher SMAPE in HiRID compared to MIMIC-IV and eICU.
We speculate that this is due to the characteristics of HiRID, where lab tests are conducted more frequently than in the other two datasets.
Specifically, we found that the average interval between lab measurements is about 15.8 hours in MIMIC-IV, 19.8 hours in eICU, and only 12.9 hours in HiRID.
This shorter interval likely benefits methods that rely on simple statistical heuristics, such as carrying forward the last measured value (Naive) or computing historical averages (Naive($\mu$)), making them more effective in this setting.
However, our primary motivation is to predict lab test outcomes in environments where frequent testing is not always feasible.
Given this perspective, the strong performance of \ours{} on MIMIC-IV and eICU highlights its effectiveness in scenarios where repeated testing may not be feasible.

Interestingly, \ours{} also outperforms GPT-4o and GPT-4o-mini, confirming that general-purpose LLMs struggle with lab test outcome prediction task.
This underscores the importance of explicitly learning EHR data rather than relying on general LLM capabilities.
Furthermore, GPT-4o-mini exhibits significantly lower performance, which we attribute to its tendency to generate unrealistic lab test values.
Upon manual inspection, we found that GPT-4o-mini often outputs numerically implausible values that are far outside the expected range (\textit{e.g.,} predicting \textit{ph of arterial blood} as 135 whereas the ground-truth is 7.2).
This result demonstrates that our data processing and training strategy play a crucial role in achieving robust performance by effectively leading the model to produce clinically meaningful predictions.

\subsection{Ablation Studies}
To systematically evaluate the contributions of different components in our model, we conduct ablation studies along the following perspectives.
All experiments in this section are conducted using the MIMIC-IV dataset to ensure consistency.

\begin{table}[t]
\floatconts
    {tab:abl_unified}
    {\caption{Ablation study results from different perspectives on MIMIC-IV dataset.} \vspace{-5mm}}
    {\resizebox{1.0\linewidth}{!}{
        \begin{tabular}{lccc}
            \toprule
            Strategy & & NMAE & SMAPE (\%) \\
            \midrule
            \multicolumn{2}{l}{\textit{Embedding \& training strategy}} & & \\
            % \cmidrule{1-2}
            \multicolumn{2}{l}{code-based \& Full AR} & $0.102$ & $20.64$ \\
            \multicolumn{2}{l}{code-based \& Only Lab AR} & $0.095$ & $19.62$ \\
            \multicolumn{2}{l}{\multirow{3}{*}{\makecell[l]{text-based \& Only Lab AR\\\textbf{(\ours)}}}} & \multirow{3}{*}{$\bm{0.064}$} & \multirow{3}{*}{$\bm{14.91}$} \\
            \\
            \\
            \hline
            \noalign{\vskip 0.4ex}
            \multicolumn{2}{l}{\textit{Numeric value representation}} & & \\
            % \cmidrule{1-2}
            \multirow{3}{*}{\makecell[l]{quantile-based\\}} & 5-quantiles & $0.077$ & $18.26$ \\
            & 10-quantiles & $0.072$ & $16.81$ \\
            & 20-quantiles & $0.067$ & $15.65$ \\
            \multicolumn{2}{l}{\multirow{3}{*}{\makecell[l]{digit-wise tokenization\\\textbf{(\ours)}}}} & \multirow{3}{*}{$\bm{0.064}$} & \multirow{3}{*}{$\bm{14.91}$} \\
            \\
            \\
            \hline
            \noalign{\vskip 0.4ex}
            \multicolumn{2}{l}{\textit{Timestamp representation}} \\
            \multicolumn{2}{l}{relative time encoding} & $0.136$ & $29.07$ \\
            \multicolumn{2}{l}{\multirow{3}{*}{\makecell[l]{absolute time encoding\\\textbf{(\ours)}}}} & \multirow{3}{*}{$\bm{0.064}$} & \multirow{3}{*}{$\bm{14.91}$} \\
            \\
            \\
            \hline
        \end{tabular}
    }}
\vspace{-2mm}
\end{table}
\subsubsection{Embedding and Training Strategy}
EHR events can be processed using different embedding strategies, which we categorize into \textbf{code-based} and \textbf{text-based} representations.
The code-based approach directly embeds structured medical codes (\textit{e.g.,} LOINC, ICD), while the text-based approach converts medical codes into their textual descriptions before embedding.
Additionally, unlike conventional methods that apply autoregressive loss to all tokens (\textit{i.e.,} \textbf{Full AR}), we specifically designed our model to focus on lab test values by applying autoregressive loss selectively (\textit{i.e.,} \textbf{Only LAB AR}).
To examine the impact of these design choices, we compare our model trained with different combinations of embeddings and loss strategies.

In the code-based embedding strategy, a new vocabulary is defined to accomodate structured medical tokens, following the embedding approach introduced in ETHOS~\citep{renc2024zero}.
Specifically, one or two tokens are defined per event: one token for the total event features that encodes the event type, item name and unit of measurement (if applicable) as a single entity (\textit{e.g.,} [LAB\_inr(pt)] or [LAB\_glucose\_mg/dL]), and the other token for the corresponding value mapped to one of 10 quantile tokens if the event has a numeric value.
Additionally, for medications, we map drug names to their corresponding ATC (Anatomical Therapeutic Chemical) codes whenever available.

As shown in Table~\ref{tab:abl_unified}, the incremental application of our embedding and training strategy components consistently improves performance.
Specifically, switching from training all tokens in an autoregressive manner to focusing solely on lab test events has a positive impact on the model's predictive accuracy.
This suggests that selective loss computation allows the model to effectively develop a deeper understanding of patterns and relationships specific to lab test outcomes.
In addition, incorporating a text-based embedding approach further enhances performance.
This improvement again highlights the advantage of providing the model with rich and interpretable representations of medical events, enabling it to better capture relationships within EHR data.
% These findings underscore the effectiveness of our strategy in optimizing lab test outcome prediction.

\subsubsection{Numeric Value Representation}
An alternative approach to handling numeric values in language models is discretization into quantiles, where continuous values are mapped to a set of predefined bins and treated as categorical tokens (\textit{i.e.,} \textbf{quantile-based tokenization})~\citep{renc2024zero}.
In contrast, our model treats numeric values as text and tokenizes them at the digit level (\textit{i.e.,} \textbf{digit-wise tokenization}).
To evaluate the effectiveness of this approach against the quantile-based method, we conduct experiments using different levels of quantization (5, 10, and 20 quantiles) and compare the results with the digit-wise tokenization strategy.

For the quantile-based method, the predicted value $\hat{y}$ is computed as the expected value of the quantile probabilities for each lab item.
Specifically, given a set of quantile bins $\mathcal{Q}$, each bin $q\in\mathcal{Q}$ is associated with a probability $p(q)$.
The predicted value is defined as the weighted sum of the expected values of the quantile intervals:
\begin{equation}
    \hat{y}=\sum_{q\in\mathcal{Q}}\mu_q \cdot p(q)
\end{equation}
where $\mu_q$ is the average value of the target lab item samples belonging to the quantile $q$ in the training set.
% where $\mu_q$ is the expected value of the quantile $q$, which is computed by the average value of the total samples belonged to the quantile $q$ in the training set.

As shown in Table~\ref{tab:abl_unified}, performance improves consistently as the number of quantile bins increases, suggesting that finer-grained numeric representations lead to better predictions.
Following this trend, digit-wise tokenization, as used in \ours{}, achieves the best performance, likely because it provides the highest resolution representation by preserving the exact numerical value rather than grouping them into fixed bins.
We believe this approach allows the model to capture subtle variations in lab test values that might otherwise be lost in a coarser representation.
Thus, these results highlight the importance of maintaining fine-grained detail when encoding numerical values to process EHR data effectively.

\subsubsection{Timestamp Representation}
As discussed in Section~\ref{sec:abs_time_encoding}, a patient's medical state can be influenced by daily routines and hospital schedules, making absolute time encoding more suitable than relative time encoding.
To validate the effectiveness of our approach, we compare our model trained with \textbf{relative time encoding} against the one trained with \textbf{absolute time encoding}, as presented in Table~\ref{tab:abl_unified}.
For relative time encoding, we calculate temporal intervals in minutes between consecutive events and append each interval to the corresponding medical event $\mathcal{M}_i$ after converting it to a text format, instead of prepending absolute timestamp tokens which we defined.

The result demonstrates that absolute time encoding leads to superior performance compared to relative time encoding, suggesting that absolute time encoding implicitly provides a more meaningful temporal context than merely representing time as intervals between events.
% One possible explanation for this improvement is that
In other words, absolute time encoding allows the model to capture daily patterns and periodic trends that influence lab test results, such as fasting-related fluctuations in glucose levels or routine morning lab draws in ICU settings.
In contrast, relative time encoding only provides information about the time elapsed since the previous event, making it more challenging for the model to infer broader temporal structures.
These findings emphasize the importance of designing time representations that align with the structured nature of clinical workflows.

\begin{figure}[t] % Use figure* for spanning both columns
    \centering
    \includegraphics[width=1\columnwidth]{figures/nmae_smape_trends.pdf} % Adjust width as needed
    \vspace{-4mm}
    \caption{
    Performance trends across different configurations of event types.
    }
    \label{fig:abl_tables}
    \vspace{-8mm}
\end{figure}

\subsubsection{Combination of Different Event Types}
As described in Section~\ref{sec:data}, we prioritized event types into three levels based on their expected influence on lab test results.
To assess how including a broader range of clinical events affects performances, we train models with different configurations of event types according to the criterion described in Section~\ref{sec:data}, and analyze their performance trends.
Specifically, each configuration incrementally includes (1) medication events (\textit{i.e., emar} and \textit{emar\_detail}), (2) input and procedure events (\textit{i.e., inputevents} and \textit{procedureevents}), and (3) output and microbiology events (\textit{i.e., outputevents} and \textit{microbiologyevents}).
Here, while utilizing more event types can provide richer contextual information, it can also lead the model to see a shorter historical window within the fixed sequence length.
For example, when using only two types of events (lab test and medication), the average covered time span per sample is 28.8 hours with a sequence length of 2048 tokens and 47.2 hours with 4096 tokens.
In contrast, when including six types of events, the average covered time span drops to 18.5 hours for 2048 tokens and 33.3 hours for 4096 tokens.
Hence, to systematically examine this trade-off, we also conduct these experiments by adjusting the sequence length to 2048 tokens from 4096 tokens, and compare the trends between them.
% TODO appendix ref 
Detailed statistics on the average covered time span for each table combination are provided in Appendix~\ref{app:data_statistics}.

Figure~\ref{fig:abl_tables} illustrates the impact of progressively incorporating more event types on the lab test outcome prediction performances.
The results show that when the sequence length is limited to 2048 tokens, increasing the number of event types leads to a degradation in performance for both sMAE and SMAPE.
This suggests that the reduced temporal coverage outweighs the benefits of incorporating more event types in this constrained setting.
In contrast, when using a sequence length of 4096 tokens, expanding the range of event types improves performance in general, indicating that longer sequences allow the model to leverage additional event information effectively.
These findings suggest that while our current experiments are still constrained to a sequence length of 4096 tokens, further increasing the sequence length may unlock additional benefits from incorporating an even broader range of EHR event types.

\section{Conclusion}
In this work, we proposed \textbf{\ours{}}, a unified model for lab test outcome prediction that leverages an autoregressive generative modeling approach on EHR data.
By integrating effective data processing and training strategy, \ours{} has shown the best performance across three public EHR datasets.
Specifically, through evaluations on MIMIC-IV, eICU, and HiRID, LabTOP consistently outperformed both traditional machine learning models and state-of-the-art LLMs, demonstrating its potential for clinical decision support and early detection of critical conditions.
Additionally, our ablation studies highlighted the efficacy of the proposed data processing and training strategy in modeling EHR data for the lab test outcome prediction task.
We believe that \ours{} will serve as an accurate and generalizable framework for lab test outcome prediction, especially in constrained environments where repeated testing is not feasible.

\paragraph{Limitations and future work}
Our experiments have some limitations as follows.
First, \ours{} requires relatively longer sequence length, leading to increased computational costs during training and inference, especially when we involve more types of EHR events.
Future work could explore more efficient tokenization strategies to balance precision and computational efficiency.
Second, our experiments were focused on retrospective EHR data, and its real-world applicability in prospective clinical settings remains to be validated.
Further research could assess \ours{}'s performance in real-time hospital environments and evaluate its impact on clinical decision-making.

% \acks{Acknowledgments go here \emph{but should only appear in the
% camera-ready version of the paper if it is accepted}.
% Acknowledgments do not count toward the paper page limit.}

\clearpage
\bibliography{chil-sample}

\clearpage

\appendix
\section{Dataset Statistics}
\label{app:data_statistics}
In this appendix, we describe dataset statistics used throughout our experiments. 

\begin{table}[h]
\floatconts
    {tab:app_datastat1}
    {\caption{Number of ICU stays in the training, validation, and test splits for each dataset (MIMIC-IV, eICU, and HiRID).}}
    {
        \begin{tabular}{lccc}
            \toprule
             & MIMIC-IV & eICU & HiRID \\
            \midrule
            Train & 73,236 & 141,712 & 26,597 \\
            Validation & 9,151 & 17,716 & 3,325 \\
            Test & 9,152 & 17,710 & 3,323 \\
            \bottomrule        
        \end{tabular}
    }
\end{table}

\paragraph{Number of ICU records}
Table~\ref{tab:app_datastat1} presents the number of ICU records in the training, validation, and test splits for each dataset (MIMIC-IV, eICU, and HiRID).
We split ICU records based on a patient level, ensuring that records from the same patient do not appear in multiple subsets.
The datasets follow an 8:1:1 split ratio, where 80\% of ICU records are assigned to training, 10\% to validation, and 10\% to test splits.

\begin{table}[h]
\floatconts
    {tab:app_datastat2}
    {\caption{Number of lab events in the training, validation, and test splits for each dataset.}}
    {
        \begin{tabular}{p{1.5cm}ccc}
            \toprule
            & MIMIC-IV & eICU & HiRID \\
            \midrule
            Train & 14,716,753 & 15,225,218 & 3,768,827 \\
            Validation & 1,739,599 & 1,914,090 & 454,994 \\
            Test & 1,799,223 & 1,848,832 & 456,609 \\
            \bottomrule        
        \end{tabular}
    }
\end{table}

\paragraph{Number of lab test events}
Table~\ref{tab:app_datastat2} presents the number of lab test events that appeared in the training, validation, and test splits for each dataset.
Given the substantial volume of data, the model was exposed to a diverse range of lab test events, enabling a thorough evaluation of its ability to predict lab test outcomes.

\begin{table}[t]
\floatconts
    {tab:time_span_mimiciv}
    {\caption{Average time span per sample in hours on MIMIC-IV.}}
    {\resizebox{1.0\linewidth}{!}{
    \begin{tabular}{lcccc}
        \toprule
        Sequence & \multirow{2}{*}{Lab-only} & Lab w/ & Lab w/ & Lab w/ \\
        Length & & T1 & T2 & T3 \\
        \midrule
        2048 & 32.8 & 20.0 & 20.5 & 18.5 \\
        4096 & 43.6 & 34.8 & 23.1 & 20.3 \\
        \midrule
        \multicolumn{5}{l}{\footnotesize T1: \{medication event\}} \\
        \multicolumn{5}{l}{\footnotesize T2: T1 $\cup$ \{input and procedure events\}} \\
        \multicolumn{5}{l}{\footnotesize T3: T2 $\cup$ \{output and microbiology events\}}
    \end{tabular}}
    }
\end{table}


\paragraph{Selected lab items}
To focus on the most clinically relevant medical events, we selected lab tests and other event types based on their frequency, retaining only those that accounted for the top 90\% occurrences within their respective tables.
As a result, we included 44 lab tests from MIMIC-IV, 41 from eICU, and 27 from HiRID, as shown in Table~\ref{tab:lab_items}.
While core biomarkers such as glucose, sodium, potassium, chloride, creatinine, bicarbonate, hemoglobin, platelets, and white blood cell count are present across all datasets, notable differences exist: (1) MIMIC-IV includes a broader set of liver function tests (AST, ALT, bilirubin), coagulation markers (PT, INR), and red blood cell indices (MCV, MCH, RDW); (2) eICU incorporates hematological parameters (MPV, lymphocytes, monocytes) and oxygenation measures (FiO2, O2 saturation, base excess); and (3) HiRID focuses more on arterial blood gas (ABG) parameters (pH, PaO2, PaCO2, lactate, methemoglobin, carboxyhemoglobin), reflecting its emphasis on real-time ICU monitoring.

\paragraph{Time span coverage per sample}
Table~\ref{tab:time_span_mimiciv} presents the average time span per sample (in hours) for different combinations of event types in MIMIC-IV, according to a sequence length of 2048 and 4096 tokens.
The time span represents the total duration covered within each sample's sequence.
As expected, increasing the sequence length from 2048 to 4096 results in a longer time span coverage across all event combinations.
Specifically, for lab-only sequences, the time span increases from 32.8 hours to 43.6 hours.
However, adding more event types reduces the time span per sample, as additional events make each sample more dense within the same sequence length.
For example, when we use a sequence length of 2048 tokens, incorporating one event type (Lab w/ T1) reduces the time span to 20.0 hours, while including two and four more event types (Lab w/ T2 and Lab w/ T3) results in 20.5 hours and 18.5 hours respectively.
Similarly, for a sequence length of 4096 tokens, the time span decreases from 43.6 hours (Lab-only) to 34.8 hours (Lab w/ T1), 23.1 hours (Lab w/ T2), and 20.3 hours (Lab w/ T3).
% These results illustrate the trade-off between time span coverage and event diversity, where longer sequences allow for broader temporal coverage, but incorporating multiple event types captures more detailed patient history within a shorter time window.

\begin{table*}[t]
\floatconts
    {tab:lab_items}
    {\caption{Lab items included in each dataset (MIMIC-IV, eICU, and HiRID)}}
    {
        \begin{tabular}{p{4cm}p{10cm}}
            \toprule
            \textbf{Dataset} & \textbf{Lab Items} \\
            \midrule
            MIMIC-IV (44 items) & glucose, potassium, sodium, chloride, ph, hemoglobin, hematocrit, bicarbonate, creatinine, anion gap, urea nitrogen, magnesium, phosphate, calcium, total, platelet count, white blood cells, mchc, red blood cells, mcv, mch, rdw, po2, base excess, calculated total co2, pco2, ptt, pt, inr(pt), h, l, i, lactate, rdw-sd, free calcium, oxygen saturation, potassium, whole blood, asparate aminotransferase (ast), bilirubin, total, alanine aminotransferase (alt), alkaline phosphatase, temperature, lactate dehydrogenase (ld), albumin, lymphocytes \\
            \midrule
            eICU (41 items)& bedside glucose, potassium, sodium, glucose, hgb, chloride, creatinine, bun, hct, calcium, bicarbonate, platelets x 1000, wbc x 1000, rbc, mcv, mchc, mch, rdw, anion gap, magnesium, mpv, -lymphs, -monos, pao2, paco2, ph, hco3, fio2, -eos, phosphate, -polys, -basos, albumin, o2 sat (\%), base excess, ast (sgot), total protein, alt (sgpt), alkaline phos., total bilirubin, pt - inr \\
            \midrule
            HiRID (27 items) & glucose [moles/volume] in serum or plasma, sodium [moles/volume] in blood, potassium [moles/volume] in blood, bicarbonate [moles/volume] in arterial blood, base excess in arterial blood by calculation, oxygen saturation in arterial blood, methemoglobin/hemoglobin.total in arterial blood, carboxyhemoglobin/hemoglobin.total in arterial blood, calcium.ionized [moles/volume] in blood, lactate [mass/volume] in arterial blood, chloride [moles/volume] in blood, ph of arterial blood, hemoglobin [mass/volume] in arterial blood, oxygen [partial pressure] in arterial blood, carbon dioxide [partial pressure] in arterial blood, hemoglobin [mass/volume] in blood, leukocytes [\#/volume] in blood, mcv [entitic volume], mch [entitic mass], mchc [mass/volume] in cord blood, platelets [\#/volume] in blood, inr in blood by coagulation assay, creatinine [moles/volume] in blood, c reactive protein [mass/volume] in serum or plasma, urea [moles/volume] in venous blood, phosphate [moles/volume] in blood, magnesium [moles/volume] in blood
            \\
            \bottomrule        
        \end{tabular}
    }
\end{table*}


\section{Implementation Details}
\label{app:implementation_details}

This appendix provides implementation details, including data processing methods, model configurations, training hyperparameters, data processing steps, and hardware specifications used in the experiments.

\paragraph{\ours{}}
The model architecture of LabTOP follows GPT-2, specifically implemented using the architecture of HuggingFace GPT2LMHeadModel\footnote{\url{https://huggingface.co/docs/transformers/model_doc/gpt2##transformers.GPT2LMHeadModel}}.
We use 12 attention decoder layers, each with 8 heads and a hidden dimension of 512.
All the experiments in this study are conducted using a maximum sequence length of 4096 tokens by default.
If a sequence of an ICU record exceeds the maximum sequence length, we repeatedly crop the sequence by this maximum length based on the event level.
Additionally, we again prepend demographic information $D$ to all the cropped samples.
For training, we applied early stopping with the patience of 5 epochs based on the validation loss, which yielded about $120$k, $110$k, and $34$k training steps with the learning rate of 1e-4 for MIMIC-IV, eICU, and HiRID, respectively.
As a result, we trained the model for approximately 96, 92, and 72 hours with 2 A6000 48GB GPUs for each dataset, respectively.

\paragraph{Naive}
If no prior measurement exists for the target lab test within the corresponding ICU record, the prediction is set to the mean of that lab test's values computed across the training set.

\paragraph{Naive($\mu$)}
If no prior measurements exist, the prediction defaults to the mean of that lab test's values across the training set, similar to the Naive method.

\paragraph{GenHPF}
GenHPF has a hierarchical architecture consisting of an event encoder and an event aggregator, designed to perform multiple predictive tasks in a single framework.
To adapt it for regressing multiple lab test outcomes, we construct samples by grouping the original samples that have the same prior medical history (\textit{i.e.,} lab test samples that occurred at the same timestamp).
Then, we manipulate the GenHPF model to regress every single lab test outcome given a sequence of medical events but compute the Mean Squared Error (MSE) loss only for the observed lab items for which we can define the ground-truth labels.
Here, since lab items with large magnitudes may disproportionately influence the model training, we standardize the outcome values for each lab item using the mean and standard deviation computed from the training set.
We maintain GenHPF's core architecture, with key hyperparameters including a prediction dimension of 128, an embedding dimension of 128, 4 attention heads, and 2 transformer layers.
Similar to LabTOP, we applied early stopping with the patience of 5 epochs based on the validation loss, which resulted in about $83$k, $210$k, and $71$k training steps with the learning rate of 3e-4 for each dataset, respectively.
As a result, we trained the model for approximately $39$, $67$, and $30$ hours with a single 80GB A100 GPU for each respective dataset.

% aggregate lab items measured at the same time to create labels while using the list of prior events as input.
% Each label of sample is represented as an $l$-dimensional array, where $l$ denotes the number of lab item types, containing the actual values of lab tests recorded at the target time.
% If a lab item is not measured at the target time, its value is set to -1e5 to indicate missing data.
% The input sequence format follows the original GenHPF approach, incorporating all column information, column names, and time gaps to maintain consistency with its event representation.
% Unlike the original GenHPF, which performs classification, our adaptation applies regression, producing an $(N, l)$ matrix, where $N$ is the batch size, and the loss is computed only for observed values, with missing values masked out.
% Since lab items with large magnitudes may disproportionately influence learning, we apply standardization using the mean and standard deviation computed from the training dataset.
% The model retains GenHPF’s core architecture, with key hyperparameters including a prediction dimension of 128, an embedding dimension of 128, 4 attention heads, and 2 transformer layers.
% GenHPF was trained for $83k$, $210k$, $71k$ iterations over approximately 39, 67, 30 hours using a single 80GB A100 GPU for each dataset, respectively. Early stopping was employed based on the validation loss, with patience set to 5 epochs.

\paragraph{XGBoost}
For XGBoost, the input features include the count, mean, min, and max values for all medical items used in the dataset, which are treated as separate columns.
As a result, the input consists of an array of the statistical summaries (count, mean, min, and max) of the unique events that precede the prediction time for the target lab item.
We trained this model for each lab item existed in the dataset with a learning rate of 0.1, a maximum depth of 5, and 100 estimators.

% The target lab item’s value serves as the label, while the input consists of an array containing the statistical summaries (count, mean, min, and max) of events recorded prior to the target lab item.
% This results in an input of shape $(N, c)$, where $N$ is the sample size and $c$ is the number of the derived features (i.e., 4 $\times$ the number of unique medical items).
% The model outputs a prediction of shape $(N, 1)$, corresponding to the estimated value of the target lab item.
% We used a learning rate of 0.1, a maximum depth of 5, and 100 estimators for training.

\paragraph{GPT-4o and GPT-4o-mini}
To create samples for GPT-4o and GPT-4o-mini\footnote{Specifically, at the time of the experiments, GPT-4o referred to \texttt{gpt-4o-2024-11-20} and GPT-4o-mini corresponded to \texttt{gpt-4o-mini-2024-07-18}.}, we followed the same strategy with GenHPF.
In other words, we grouped test samples that have the same prior medical history (\textit{i.e.,} the lab items that occurred at the same time), and the LLM was instructed to answer values of the target lab items based on their prior sequence of medical events as prompt.
The prompt we used for lab test outcome prediction can be found in Figure~\ref{fig:prompt}.
\begin{figure*}[b] % Use figure* for spanning both columns
    \centering
    \includegraphics[width=1.0\textwidth]{figures/prompt.pdf} % Adjust width as needed
    \vspace{-4mm}
    \caption{
    Template prompt used for GPT-4o and GPT-4o-mini.
    Note that [medical history] refers to the textual sequence of the test sample.
    }
    \label{fig:prompt}
    \vspace{-8mm}
\end{figure*}


% \item \textbf{Naive}: 
% \item \textbf{Naive($\mu$)}: 
% \item \textbf{GenHPF}~\citep{hur2023genhpf}:
% \item \textbf{XGBoost}~\citep{chen2016xgboost}: 
% \item \textbf{GPT-4o} and \textbf{GPT-4o-mini}: 

% GenHPF: model architecture 간단히 언급. predictive task를 위해 design된 모델인데 우리는 얘를 lab test value를 regression하는 task를 하도록 adapt했음 (이걸 어떻게 했는지). 여러 lab test 종류를 multi-task로 regression 시키기 위해서 각 lab item들을 mean, std로 standardize했다는 거 언급.

% XGBoost: 어떤 feature들 썼는지, configuration

% GPT-4o, GPT-4o-mini: 어떻게 프로세싱 했는지 prompt 알려주기

%\section{Detailed Experimental Results}
%\label{app:detailed_results}
% mimiciv, eicu, hirid에서 각 개별 lab item 별 성능
%% \begin{table}[t]
% \floatconts
%     {tab:main}
%     {\caption{
%     Lab test outcome prediction performances on different EHR datasets.
%     The best performances for each dataset are highlighted with \textbf{boldface}.}
%     \vspace{-5mm}
%     }
%     {
%         \begin{tabular}{lccc}
%             \toprule
%              & MIMIC-IV & eICU & HiRID \\
%             \midrule
%             \multicolumn{4}{l}{\textit{\textbf{NMAE}}} \\
%             % \midrule
%             Naive & $0.090$ & $0.102$ & $0.101$ \\
%             Naive($\mu$) & $0.108$ & $0.116$ & $0.119$ \\
%             GenHPF & $0.121$ & $0.094$ & $0.102$\\
%             XGBoost & $0.083$ & & $0.100$ \\
%             GPT-4o & $0.195$ & $0.230$ & $0.165$ \\
%             GPT-4o-mini & $0.792$ & $1.027$ & $2.969$ \\
%             \noalign{\vskip 0.4ex}
%             \hline
%             \noalign{\vskip 0.4ex}
%             \textbf{\ours{}} & $\bm{0.064}$ & $\bm{0.075}$ & $\bm{0.086}$ \\
%             \midrule
%             \multicolumn{4}{l}{\textit{\textbf{SMAPE (\%)}}} \\
%             Naive & $17.64$ & $18.29$ & $\bm{15.98}$ \\
%             Naive($\mu$) & $21.78$ & $20.83$ & $18.99$ \\
%             GenHPF & $26.22$ & $22.61$ & $17.26$\\
%             XGBoost & $20.42$ & & $17.18$ \\
%             GPT-4o & $19.67$ & $21.18$ & $21.17$ \\
%             GPT-4o-mini & $40.86$ & $45.67$ & $43.88$ \\
%             \midrule
%             \textbf{\ours{}} & $\bm{14.91}$ & $\bm{15.73}$ & $16.60$ \\
%             \bottomrule
%         \end{tabular}
%     }
% \vspace{-4mm}
% \end{table}


\begin{table*}[t]
\floatconts
    {tab:app_res_lab_mimiciv}
    {\caption{Lab test outcome prediction performances on MIMIC-IV}}
    {\begin{tabular}{llrr}
        \toprule
        \multicolumn{1}{c}{\multirow{3}{*}{Lab Item}} &       & \multicolumn{2}{c}{LabTOP} \\
        \cmidrule{3-4}
                                                    & count &  NMAE &  SMAPE \\
        \midrule
        \multicolumn{1}{c}{\textit{Micro average}}  & \multicolumn{1}{c}{-} & 0.064 &  14.91 \\
        \midrule
        Alanine Aminotransferase (ALT) & 14755 & 0.017 & 44.98 \\
        Albumin & 7602 & 0.103 & 10.72 \\
        Alkaline Phosphatase & 14724 & 0.044 & 23.59 \\
        Anion Gap & 57071 & 0.085 & 13.85 \\
        Aspartate Aminotransferase (AST) & 15051 & 0.016 & 43.33 \\
        Base Excess & 50678 & 0.042 & 66.19 \\
        Bicarbonate & 57184 & 0.059 & 7.07 \\
        Bilirubin, Total & 14879 & 0.020 & 33.21 \\
        Calcium, Total & 52138 & 0.071 & 3.60 \\
        Calculated Total CO2 & 50595 & 0.042 & 5.81 \\
        Chloride & 59383 & 0.064 & 2.11 \\
        Creatinine & 56984 & 0.028 & 13.91 \\
        Free Calcium & 28816 & 0.078 & 4.25 \\
        Glucose & 72967 & 0.163 & 20.66 \\
        H & 33429 & 0.054 & 98.52 \\
        Hematocrit & 56561 & 0.074 & 6.40 \\
        Hemoglobin & 57956 & 0.076 & 6.80 \\
        I & 33435 & 0.008 & 26.37 \\
        INR (PT) & 35053 & 0.043 & 10.40 \\
        L & 33189 & 0.048 & 38.16 \\
        Lactate & 31656 & 0.043 & 23.81 \\
        Lactate Dehydrogenase (LD) & 8795 & 0.025 & 28.94 \\
        Lymphocytes & 7646 & 0.092 & 72.90 \\
        Magnesium & 55424 & 0.072 & 6.52 \\
        MCH & 50182 & 0.049 & 2.17 \\
        MCHC & 50225 & 0.079 & 1.98 \\
        MCV & 50291 & 0.049 & 1.89 \\
        Oxygen Saturation & 19036 & 0.151 & 12.13 \\
        PCO2 & 50580 & 0.067 & 9.91 \\
        pH & 60130 & 0.074 & 2.89 \\
        Phosphate & 52663 & 0.071 & 15.03 \\
        Platelet Count & 51261 & 0.048 & 16.66 \\
        PO2 & 50562 & 0.080 & 26.87 \\
        Potassium & 60689 & 0.097 & 7.29 \\
        Potassium, Whole Blood & 16387 & 0.085 & 7.35 \\
        PT & 34739 & 0.044 & 9.97 \\
        PTT & 38596 & 0.080 & 16.61 \\
        RDW & 50257 & 0.039 & 3.23 \\
        RDW - SD & 29462 & 0.040 & 3.60 \\
        Red Blood Cells & 50196 & 0.072 & 6.76 \\
        Sodium & 60577 & 0.067 & 1.44 \\
        Temperature & 10385 & 0.093 & 1.31 \\
        Urea Nitrogen & 56789 & 0.036 & 16.65 \\
        White Blood Cells & 50245 & 0.057 & 19.64 \\
        \bottomrule
    \end{tabular}}
\end{table*}

% \section{Second Appendix}\label{apd:second}

% This is the second appendix.

\end{document}
