\section{Related Works}
\paragraph{Predicting clinical outcomes using EHR data}
Previous machine learning models leveraging EHR data have primarily focused on predicting clinical outcomes, such as readmission, mortality risk, and length of stay____.
Among them, ____ introduced the \textit{SAnD}, employing Transformer architecture____ to predict in-hospital mortality and length of hospital stay.
They constructed the input embeddings based on structured clinical codes (e.g., ICD, LOINC, RxNorm) to process EHR data, which makes it difficult to process multiple EHRs with different schemas within a single model.
To address these challenges, ____ proposed a text-based embedding approach (DescEmb) to represent EHR events using descriptive text rather than codes, showing superior performances compared to the code-based approach.
Based on this text-based approach, GenHPF____ extended it by incorporating multi-task learning to handle various predictive tasks simultaneously.
Furthermore, REMed____ integrated an event-retriever module into this framework to selectively extract relevant events with target tasks, ensuring that only the important events contribute to the final predictions.

\paragraph{Generative models for EHR data}
In addition to the studies employing discriminative models for predictive tasks, generative approaches have also been explored for modeling EHR data.
Specifically, MedGPT____ adopted the GPT-2 architecture____ to predict next SNOMED-CT disorder concept given the patient's previous history of disorders.
Although this work utilized only the disorder concepts instead of a comprehensive set of clinical events in EHR, they demonstrated the potential of generative methods for simulating patient trajectories and predicting future clinical events.
In addition, ETHOS____ employed a generative approach based on language modeling to simulate clinical scenarios using a structured EHR data.
Similar to the code-based approach, they converted each medical event into 1 to 7 tokens to construct the input sequence for the Transformer____, and trained the model to predict every next token given the prior sequences to simulate the clinical outcomes of the patient, such as whether the patient will die or not (\textit{i.e.,} mortality prediction).
% In this way, the model simulates the clinical outcomes of the patient, such as whether the patient will die or not (\textit{i.e.,} mortality prediction), and whether the patient will readmit or not (\textit{i.e.,} readmission prediction).

\paragraph{Lab test outcome prediction}
Although previous studies have significantly advanced the field of machine learning for healthcare, none of them have addressed the direct estimation of continuous values for various lab tests within a single unified model.
Instead, most of previous studies have explored the prediction of lab test results by designing a specific model for only a few selected lab items or classifying discrete levels of selected lab items, such as normal or abnormal ranges.
For example, HgbNet____ employed an attention-based LSTM model to estimate hemoglobin levels, while ____ presented a probabilistic model using NGBoost____ to predict expected values of several lab items, such as wbc and hemoglobin.
Similarly, several works____ attempted to estimate glucose values using machine learning or deep learning approach.
However, these approaches are specifically designed for a small subset of lab tests, making it challenging to generalize across diverse clinical measurements.

Meanwhile, GenHPF____ and REMed____ formulated this task as classifying the discrete level of values for several selected lab items.
Although these works have shown high performances on the defined task, their predictions remain limited in terms of extensibility and granularity.
First, they rely on predefined level thresholds for each lab item, determined through domain knowledge, which makes it difficult to extend the approach to encompass all lab items recorded in the EHR database.
Second, as the task is defined within a fixed prediction window (\textit{e.g.,} predicting average levels of lab items in the next 12 hours given the first 12 hours), these models lack granularity in providing immediate estimations of lab test results, which is critical for clinical scenarios where timely and dynamic predictions are necessary to inform urgent decisions.

\begin{figure*}[ht] % Use figure* for spanning both columns
\floatconts
    {fig:overview}
    {\vspace{-8mm}
    \caption{\textbf{Training and Inference of \ours{}.} During training, demographic information $D$ and a sequence of medical events $P=[\mathcal{M}_1,\dotsc,\mathcal{M}_N]$ are tokenized, and fed into the Transformer to generate next token at each position. The training loss is then computed only for lab test value tokens along with their units and the corresponding $\texttt{[EOE]}$ token. During inference, given the preceding sequence of medical events and the target lab event (\textit{i.e.,} $(t_k,e_k,d_k)$), the model autoregressively generates numeric value tokens for its outcome until the [EOE] token is encountered.}}
    {
        \includegraphics[width=1.0\textwidth]{figures/training_inference.pdf}
    }
    \vspace{-6mm}
\end{figure*}