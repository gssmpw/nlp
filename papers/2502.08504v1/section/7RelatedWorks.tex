\section{Related Works}\label{sec: RelatedWorks}



\noindent  \textbf{Root Cause Analysis for AI Systems}
Inspired by testing approaches in other AI systems\cite{shi2024finding, xie2023mosaic, wang2022exploratory, yu2024survey, bothe2020neuromorphic,kim2020control, hossen2023care}, recent years have seen attempts to introduce root cause analysis into the testing of ADS and related robotic systems.
Swarmbug\cite{jung2021swarmbug} treats the AI system as a black box, and proposes \textit{Degree of Causal Contribution} to measure how configurations affect the behaviour of the swarm robotics. In contrast, the \oracle in \tool focuses more on the outputs of individual modules within the system, allowing us to identify the root causes of issues at a finer granularity.
RVPLAYER\cite{choi2022rvplayer} and ROCAS\cite{feng2024rocas} propose an algorithm that replays the accident scenarios and checks if the accident can be avoided by changing some parameters to locate the root cause. %ROCAS\cite{feng2024rocas}, under the replay-based localization approach, further proposed the \textit{Message Difference Ratio}, which conducts a differential analysis of the log information from various components during accident scenarios and normal operations. Thus, ROCAS allows for the localization of potential anomalous components.
Compared to the methods in these two works, our \oracle establishes analysis metrics for each module and develops a module-to-system mapper to address the challenge of module-level error measurement. This allows for the analysis to be conducted with just a single run of the accident scenario, making the process more efficient and accurate in identifying the \mccs.

\noindent \textbf{Search-based Scenario Generation for ADS Testing}
Search-based methods have become one of the most popular algorithms in scenario-based testing due to their ability to efficiently explore complex scenario spaces\cite{ding2023survey,zhong2021survey}. From an algorithmic framework perspective, it can be categorized into evolutionary algorithm\cite{han2021preliminary,tang2021systematic,zhou2023specification,calo2020generating,humeniuk2022search}, model-based searching\cite{haq2022efficient,haq2023many,zhong2022neural,feng2023dense,li2023generative}, and fuzzing methods\cite{pang2022mdpfuzz,cheng2023behavexplor,fu2024icsfuzz,li2020av,cheng2024evaluating,cheng2024drivetester} which we use in \tool. 
Unlike these methods, \tool goes beyond efficiently generating collision or failure scenarios. It establishes a relationship between module errors and system failures, generating \mccs for specific modules within the ADS. This approach is designed to better help ADS developers in enhancing their system components.