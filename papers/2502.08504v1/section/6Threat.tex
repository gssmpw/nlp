\section{Discussion}
This paper is the first to investigate ADS failures from module-level root causes. Although \tool can effectively and efficiently detect {\mccs}s, there are several potential areas for improvement that warrant further discussion.

\noindent \textbf{Non-\mccs Failures}
In \tool, we employ a strict oracle to determine \mccs by requiring that only one module fails before a collision occurs. Consequently, many failure scenarios are not classified as being caused by a specific module, rendering them non-\mccs failures. However, in some non-\mccs scenarios, if the ground truth for a specific module is provided, as demonstrated in RQ2, normal operation might resume. These potential \mccs may warrant further investigation.

\noindent \textbf{Future Works} There are two potential directions for future work. (1) Currently, \tool only considers single-module analysis. However, our \tool can be easily extended to support the detection of {\mccs}s induced by multiple modules, which we plan to explore in future work. (2) In our individual module metrics, we focus solely on safety as the metric. Nevertheless, evaluating the ADS from non-safety-critical aspects is also important. In future work, we will incorporate additional metrics to broaden the evaluation scope.

\section{Threats to Validity}\label{sec: threats}

\noindent\textbf{Internal Validity.}
The accuracy of Root Cause Analysis is critical for \abb{}, as it affects the evaluation of the generated test scenarios and serves as the foundation for the subsequent fuzzing process. To achieve the most precise detecting for \mccs, we employ the strictest criterion: within the collision's effect window, there must be one and only one module that experiences an error for us to attribute the collision to that module's fault. Although this approach may not always result in a relatively high proportion of qualifying scenarios(see Section~\ref{sec:exp_rq1}), it ensures that the collisions in the generated test scenarios are indeed associated with specific modules(see Section~\ref{sec:exp_rq2}).

\noindent \textbf{External Validity.}
Since different ADSs employ varying combinations of modules and each module implementation has its own strengths and weaknesses, the experimental results can only be fully guaranteed to be directly related to the specific ADS and its corresponding model under test. To address this limitation, we will subsequently conduct tests on different ADSs and replace the existing module implementations within the ADS to further explore and validate our approach.
