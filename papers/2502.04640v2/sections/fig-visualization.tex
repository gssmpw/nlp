\begin{figure*}[!t]
    \includegraphics[width=\linewidth]{figures/visualization_BAL.pdf}
    \caption{Visualization of BAL datasets. \textbf{Top:} Our \nameshort solver. \textbf{Middle:} \textsc{Ceres-GT-0.01}. \textbf{Bottom:} \textsc{Ceres-GT-0.1}. Both our \nameshort solver and \textsc{Ceres-GT-0.01} accurately recover the ground truth camera poses and landmarks, whereas \textsc{Ceres-GT-0.1} fails.}
    \label{fig:visualization-BAL}
    \vspace{-2mm}
\end{figure*}

\begin{figure*}[!t]
    \includegraphics[width=\linewidth]{figures/visualization_replica.pdf}
    \caption{Visualization of Replica datasets. \textbf{Top:} Our \nameshort solver. \textbf{Middle:} \glomap. \textbf{Bottom:} \colmap. All methods achieve high accuracy, producing nearly identical reconstruction results. \glomap sometime produce outliers (see column 2 and 3).
    }
    \label{fig:visualization-replica}
    \vspace{-5mm}
\end{figure*}

\begin{figure*}[!t]
    \includegraphics[width=\linewidth]{figures/visualization_mipnerf.pdf}
    \caption{Visualization of Mip-Nerf datasets. \textbf{Top:} \colmap. \textbf{Bottom:} Our \nameshort solver. 3D-gaussian renderings are the same.}
    \label{fig:visualization-mipnerf}
    \vspace{-0mm}
\end{figure*}

\begin{figure*}[!t]
    \includegraphics[width=\linewidth]{figures/visualization_IMC.pdf}
    \caption{Visualization of IMC2023 datasets. \textbf{Top:} Our \nameshort solver. \textbf{Bottom:} \glomap.}
    \label{fig:visualization-imc}
    \vspace{-0mm}
\end{figure*}

\begin{figure*}[!t]
    \includegraphics[width=\linewidth]{figures/visualization_TUM.pdf}
    \caption{Visualization of TUM datasets using our \nameshort solver.}
    \label{fig:visualization-tum}
    \vspace{-0mm}
\end{figure*}

\begin{figure*}[!t]
    \includegraphics[width=\linewidth]{figures/visualization_c3vd.pdf}
    \caption{Visualization of C3VD medical datasets. \textbf{Top:}  With ground truth depth. \textbf{Bottom:} With learned depth.}
    \label{fig:visualization-c3vd}
    \vspace{-0mm}
\end{figure*}