%!TEX root = ../main.tex
 \vspace{-2mm}
\begin{abstract}
   Global bundle adjustment is made easy by depth prediction and convex optimization. We (\emph{i}) propose a scaled bundle adjustment (SBA) formulation that lifts 2D keypoint measurements to 3D with learned depth, (\emph{ii}) design an empirically tight convex semidefinite program (SDP) relaxation that solves SBA to certifiable global optimality, (\emph{iii}) solve the SDP relaxations at extreme scale with Burer-Monteiro factorization and a CUDA-based trust-region Riemannian optimizer (dubbed \nameshort), (\emph{iv}) build a structure from motion (SfM) pipeline with \nameshort as the optimization engine and show that \xmsfm dominates or compares favorably with existing SfM pipelines in terms of reconstruction quality while being faster, more scalable, and initialization-free. 
%    All code and visualizations are available at \url{https://computationalrobotics.seas.harvard.edu/XM/}.
    % We propose \nameshort (Conve\underline{X} Structure from \underline{M}otion), a global structure-from-motion pipeline that integrates the rich data-driven priors of vision foundation models with the robust and precise 3D reasoning of convex optimization. Given a set of monocular 2D images, XM reconstructs 3D structures by (a) building a lifted view graph (LVG), where nodes represent unknown camera poses and 3D points, and edges correspond to 3D measurements derived from traditional 2D feature matching via foundational depth prediction models, and (b) solving a scaled bundle adjustment problem to certify global optimality using convex semidefinite programming (SDP) relaxation. XM is global (requires no initialization and avoids local minima), certifiable (providing sub-optimality gaps), and fast (implemented in CUDA with advanced optimization algorithms). XM outperforms state-of-the-art methods in speed without sacrificing accuracy, reconstructing thousands of frames in under a minute.
\end{abstract}
