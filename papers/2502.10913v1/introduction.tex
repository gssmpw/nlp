\section{Introduction}

Businesses have increasingly leveraged messaging applications to enhance various aspects of their operations \cite{anderson2016getting, lo2022mobile, doi:10.1177/20501579241246721}. Specifically, messaging applications like WeChat \cite{yang2016role}, WhatsApp \cite{10.1145/3613905.3651034}, Facebook Messenger and Telegram \cite{LoPresti2021}  have proven invaluable for business communication, aiding in customer acquisition, retention, and overall marketing efforts. Within HCI (Human-Computer Interaction) literature, scholars have investigated the various ways in which communication technologies have been appropriated by communities to empower themselves (see for eg., \cite{10.1145/2660398.2660427, 10.1145/3328020.3353927, 10.1145/3318140, 10.1145/2145204.2145220}). 

Before Meta introduced WhatsApp Business as a professionalized platform, WhatsApp was already widely used for business purposes. In India, particularly in smaller, rural areas \cite{doi:10.1177/20501579241246721} and financially constrained communities \cite{10.1145/3014362.3014367} with limited technical resources, WhatsApp became a crucial tool for conducting business and supporting local economies \cite{modak2017dancing}. Additionally, the platform's broad use across different age groups, including older adults, also spoke for its important role and impact \cite{10.1145/3449212}. However, as its appropriation for business became widespread, Meta introduced a formalized version of WhatsApp Business. This update included changes to the interface and additional costs. Alongside these changes, Meta began regularly altering WhatsApp’s features and functionality \cite{10.1145/3613905.3651034}. For example, features like account verification, message-based pricing, and business verification for accessing full features became necessary to appear as legitimate \cite{meta_verified_business_2024}.




While change is inevitable and often driven by policies, design updates, and business strategies, it remains understudied within the HCI community to understand its impact on communities who have successfully appropriated technologies for empowerment \cite{10.1145/3443686, 10.1145/3613905.3651034}. It is important to investigate how platforms, initially shaped and \textit{infrastructured} through user contributions \cite{10.1145/3313831.3376201}, are later co-opted by larger corporations \cite{doi:10.1177/1461444816629474}. These corporations often modify the platforms, shifting control and benefits away from the original users, which exemplifies digital colonialism \cite{Thorat2020}. Thus, I am interested in understanding how the consequent rapid changes in technology impact traditionally decentered users' ability to understand and adapt. This focus is crucial for designing and implementing changes inclusively, which are key topics in HCI \cite{10.1145/3544548.3581040}. Thus, in this study, I examine how WhatsApp users—whether they use personal accounts or WhatsApp Business accounts—adapt to changes on the platform. I start by exploring how these users initially adopted WhatsApp for business purposes and then analyze their responses and experiences with subsequent changes to the platform. Thus, I ask:

\textbf{RQ1: How did users initially transition from using WhatsApp as a personal communication tool to adopting it for business purposes?}
    
\textbf{RQ2: What impact did the frequent changes and updates in the WhatsApp Business platform have on users' business operations and how did they adapt to these changes?}

I conducted interviews with 14 users who employed WhatsApp for home-based small businesses in India and subsequently applied thematic analysis to address the research questions. The findings showed that users leveraged the \textit{personal nature of WhatsApp} to rely on their networks and personal experiences to adapt WhatsApp for business purposes. Despite experiencing confusion and challenges in adapting to changes, participants often devised innovative solutions to maintain their familiarity with the platform. However, they also often trivialized the changes, feeling \textit{undeserving} of such tools due to their perceived marginalized status and their belief that they were not the platform’s primary target users. They perceived these updates as more beneficial for larger businesses, pointing to various new features released by Meta as part of WhatsApp Business that were aimed at enhancing business professionalism. 


I contribute to HCI literature in several ways. First, I discuss how personal communication tools become important spaces where innovation and community-based appropriations occur through collaborative processes. These informal settings provide a comfortable environment for many users to experiment and adapt tools in creative ways, as exemplified by WhatsApp Business. Consequently, I explain WhatsApp Business as an infrastructuralized platform \cite{doi:10.1177/1461444816661553} and use the appropriation matrix \cite{10.1145/3613904.3642590} to document how its business features, initially a local innovation, have been \textit{reverse-appropriated} by Meta \cite{10.1145/3613904.3642590}. I then introduce \textit{Coercive Professionalization}, which explains how large tech companies, such as Meta with WhatsApp, monetize and formalize informal practices originally developed by local users. Drawing on Lampland and Star's \cite{lampland2009standards} work on \textit{standards}, I explore how they impose normative categories on messy real-world situations, and how participants present their feelings and understandings of changes they see as attempts to appease larger businesses, often with a sense of \textit{resigned acceptance}. Finally, drawing from De's \cite{10.1145/3613905.3651034} work on \textit{situated infrastructuring}, I analyze how, combined with coercive professionalization, it reinforces colonial practices, exacerbating power imbalances and marginalization in ways that may not be apparent to those affected.











 

















































 











