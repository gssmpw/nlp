\section{Discussion}



The findings reveal that WhatsApp’s rapid adoption for business purposes marks a significant shift, as personal communication tools are appropriated for professional use within complex sociotechnical contexts. This aligns with prior research showing how users extend the platform’s functionality to support their professional activities \cite{rupvcic2020emergence, dominguez2020distributed, 10.1145/3025453.3025643}. As WhatsApp Business professionalized, Meta engaged in reverse appropriation by monetizing the platform and altering its infrastructure \cite{10.1145/3613905.3651034, owen2024monetization}. This led to continual infrastructuring, where users had to frequently adapt their practices. Despite their efforts, users often undervalued this labor and invisibilized their own legitimate practices \cite{crain2016invisible, doi:10.1177/1050651920979999}. Understanding users' practices in why they perceive their own efforts as unworthy is crucial for understanding the broader implications of reverse appropriation and infrastructuring on their experiences \cite{munro2012social, 10.1145/3555584, 10.1145/3415170}.


\subsection{Appropriation through Personal Use}

\label{appper}

The \textit{personal} use and ritualization of WhatsApp \cite{doi:10.1177/01968599221095177} allow individuals to interact with and adapt the platform in informal ways. This flexibility helps them achieve more than they might with more rigid, formal tools. For example, users like P8 and P3 use WhatsApp’s personal communication features to indulge in \textit{informal rumors} \cite{10.1145/3290605.3300563} to collaboratively address business challenges and adapt to new updates. This community-driven approach helps them manage their businesses effectively, even with limited digital knowledge and experience \cite{10.1145/3637429}. Through this adaptation, users have appropriated WhatsApp to meet their needs, benefiting both themselves and Meta \cite{10.1145/3313831.3376201}. The platform’s simplicity and personal nature are crucial for its success in these informal business contexts. A more complex and professional tool might not fit well with the users’ sociotechnical environments, potentially overwhelming or deterring them from pursuing entrepreneurial opportunities.

%\begin{figure*}[t]
 %   \centering
  %  \includegraphics[width=0.8\linewidth]{images/appro.png}
   % \caption{Initially, WhatsApp's personal nature allows users, particularly those in marginalized or less technologically savvy roles, to adapt the platform informally for business purposes. This personal use supports the exchange of informal rumors and collaborative problem-solving, enabling effective management despite limited digital knowledge. However, as WhatsApp is increasingly professionalized—introducing more rigid features and formal structures—the original personal touch diminishes. This shift leads to the loss of the platform’s flexibility, making it less accessible and usable for its initial user base. Consequently, these users can no longer appropriate WhatsApp for their personal and business needs, resulting in decreased confidence and effectiveness in their entrepreneurial activities.}
   % \label{fig:appro}
%\end{figure*}


When users are required to adapt to more formal structures, the situation changes significantly. As WhatsApp evolves and potentially introduces more complex features, users who are accustomed to its informal, personal use struggle with the demands of the professional setting \cite{10.1145/3613905.3651034}. This creates a tension: while the personal nature of tools like WhatsApp promotes innovation and accessibility, their evolution to meet professional needs must be carefully managed to avoid alienating users who value their simplicity and familiarity. Thus, I advocate for further research on artifacts that are inherently personal, appropriated, and ritualized to be used for a wider range of tasks. Examples of this can be shared points of internet connectivity like cyber cafes, telehealth infrastructures, and entrepreneurial applications, among others. Some of these have already revealed design \cite{furuholt2018role} and sociotechnical implications \cite{Rangaswamy2011CuttingChai}, which could be leveraged and co-opted to \textit{center} marginalized populations to legitimize and solidify their appropriation practices \cite{berger2020doing}. In the next section, I explore how users have adapted to the changes.





\subsection{Coercive Professionalization and its Consequences}


In this section, I examine the experiences of small business owners adapting to WhatsApp's changes through the concepts of appropriation \cite{10.1145/3613904.3642590}, colonization \cite{doi:10.1177/1527476419831640}, and infrastructuring \cite{doi:10.1177/00027649921955326}, with a focus on their sense of undeservedness. This study and previous research show that WhatsApp has reached a state of \textit{infrastructuralization} through unique processes of \textit{appropriation} \cite{10.1145/3613905.3651034} and \textit{ritualization} \cite{doi:10.1177/01968599221095177}. Users have extended the app’s personal use into broader, non-personal contexts. However, WhatsApp’s evolving features and monetization strategies impose a standardized, professional framework that is perceived as mainly benefiting larger businesses and marginalizing small business owners. For example, P1 expressed concern that WhatsApp’s push towards more \textit{legitimizing} features might compel her business into an unwanted level of visibility-- a direct consequence of the ways in which WhatsApp was wanting its users to professionalize. Likewise, P9 and P10 were concerned that if the app continued to evolve with more features, it might expand beyond its primary messaging function, potentially forcing them to use more professional tools. Others speculated about the complications this could cause and were uncertain about how to manage such changes, fearing it might push the platform into a realm they were uncomfortable with. This is what I note as perceived \textit{coercive professionalization}, where users are pressured to \textbf{adhere to global, often Eurocentric, professional standards, often ignoring their specific local contexts}. Similar to a colonial-style reverse-appropriation \cite{10.1145/3613904.3642590}, this process forces users to conform to norms that do not reflect their unique needs or cultural situations-- impacting how users perceive and respond to the platform’s changes. Thorat \cite{Thorat2020} had similarly critiqued Facebook's \textit{Free Basics} initiative, arguing that

\begin{quote}
    \textit{Free Basics evokes paradigms of national development, modernization, and progressivism that are rooted in technoutopian narratives. At this nexus of corporate and state interest in digital infrastructure lies a technoutopian belief that social, political, and economic problems in the Global South can be resolved by technological advancement, and nationwide issues of inequality and disparity will be ameliorated if poor citizens have access to the Internet} 
\end{quote}

This illustrates how technologies initially adapted for local needs are later taken over by big tech companies, revealing uneven power dynamics in design \cite{10.1145/3630106.3658934, 10.1145/3637316, 10.1145/1753326.1753522}. Additionally, these systems impose and claim to empower people who may no longer be able to understand or utilize them due to said standardization. It exemplifies digital colonialism, as it ignores decolonial perspectives and promotes a \textit{foreign is great} ideology, leading to feelings of unworthiness \cite{oji2020digital} and, consequently, a sense of \textit{resigned acceptance}. 


%Regarding changes in the monetization model, P3 mentioned that she cannot avoid extra fees due to the lack of alternative platforms. 

%However, she also struggles financially if such fees become the norm. 


Additionally, as discussed in section \ref{appper}, many participants discussed their dependence on the inherently personal nature of WhatsApp, as they rely on their trusted circles for practicing infrastructuring. Through coercive professionalization, Meta neglects the value of immaterial labor, such as building trust and empathy, which are essential in the Indian business context \cite{berger2020doing}. This aligns with feminist critiques of labor, which argue that immaterial labor—often feminized and undervalued—is frequently dismissed as "non-real" labor, despite its significant contribution to success \cite{oksala2016affective, terranova2012free}. Societal ideals and Meta’s exploitation of users lead to a feeling that their immaterial labor is undervalued, despite its role in generating financial gains. This undermines the legitimacy of users' practices and exposes the coloniality inherent in legitimization through coercive professionalization \cite{wyrtzen2017}. In this context, coercive professionalization is evident in how Meta imposes a standardized, profit-driven framework that disregards the local, relational, and affective forms of labor essential to the users' success. By devaluing these personal forms of labor, Meta's actions exemplify the coercive nature of professionalization, forcing users into a model that erases the subjective and context-- dependent labor that underpins their practices-- and this is another characteristic of coercive professionalization.





Consequently, users' marginalization within this system is evident in their acceptance of disruptions as normal, reflecting the ongoing impact of coercive professionalization similar to the effects of digital colonialism \cite{Sturmer2021}. Many participants resigned themselves to the platform’s changes, viewing them as inherent challenges of operating in a digital space not designed for their needs. A severe consequence was the emergence of feelings of \textit{inadequacy and doubt}—as P11 noted, despite supporting her family through her business, she questioned, \textit{"Am I really a businesswoman?"} Similarly, P4 and P7 described their business activities as secondary or hobby-like, managing significant responsibilities within a framework that does not support their scale or needs. Despite this, they did not express dissatisfaction, believing they were not engaged in \textit{real business}. While the perceived lack of agency may stem from insufficient technological expertise and diverse cultural factors, it also reflects a rationalization of diminished self-worth due to the repossession \cite{doi:10.1177/1461444816629474} of infrastructure through coercive professionalization. This process again echoes colonial dynamics, where the imposition of external standards undermines local practices and reduces the professional agency of those involved \cite{10.1145/3274340}. 

Finally, in India and many other Global South cultures, self-promotion is often approached with considerable restraint, and individuals are typically less inclined to overtly display their achievements \cite{merkin2018individualism}. This understated approach is valued as a means of achieving more with less and is held in high regard. However, Meta's coercive professionalization is such that by imposing a professional framework that is not contextually aligned with those who originally built the business and whose needs it purports to address, it reinforces ideals of digital colonialism that marginalize and devalue local practices \cite{10.1145/3274340}.







\subsection{Implications for Design}

The findings emphasize the need to include decolonial perspectives when analyzing how platforms are infrastructuralized through appropriation. This aligns with prior research that ask to \textit{make visible the power dynamics and influence of Western European standardization processes on knowledge-making and communication practices} \cite{10.1145/3328020.3353927, 10.1145/3283458.3283497}. Additionally, for cultures that are not normative, and undocumented within \textit{western rigor}, designing systems that show \textit{forms of user-centred design that strive to fit technologies to a stabilized notion of the user} should be questioned \cite{doi:10.1177/0162243910389594, 10.1145/2662155.2662195}. 


%Thus, while addressing design changes, my suggestions also critique the rapid influence of big tech in the Global South and contribute to policy discussions on infrastructure and its implications. 

Decolonizing infrastructure requires systems to adapt to technological changes in ways that respect and accommodate diverse cultural needs \cite{schaefer2021understanding}. While efficiency and automation are important, this study emphasizes that platforms should focus on stabilizing systems that truly support local communities, rather than transforming them into something unfamiliar, like WhatsApp Business. Developments should enhance, not obscure, the tools communities appropriate and rely on. This approach aligns with participatory design, involving stakeholders in the process rather than imposing generic, one-size-fits-all solutions \cite{herbst2023developing}. Drawing from the findings in this study, I outline methods to support and encourage successful appropriation. 

%For successful appropriation, WhatsApp Business should support users who benefit from its personal communication features, thereby rethinking and decolonizing its approach to changing and standardizing infrastructure (see Figure \ref{fig:appro}). 

Firstly, WhatsApp could maintain and enhance core personal features that small business owners rely on. For instance, users like P8 effectively utilize personal communication tools, such as status updates and group chats, for business purposes. Therefore, WhatsApp (and other platforms) could consider implementing a \textit{"Legacy Mode"} that allows users to retain and access familiar features while integrating new ones, thus supporting a culture of appropriation and tailoring \cite{DiSalvo2013, 10.1145/97243.97271}. Users could have the option to keep using their personalized WhatsApp setup, with that setup being recognized as legitimate. This could be made possible if WhatsApp allowed users to choose and verify the version they want to keep, giving them official approval for their customized setup.



Additionally, simplifying user interface updates is crucial. As P4 and P7 noted, frequent updates can be confusing and time-consuming. WhatsApp could introduce a \textit{"What’s New"} section that provides concise summaries and visual guides in local languages. This feature could reduce the learning curve and associated frustration, while also fostering professional agency \cite{10.1145/3274340, 10.1145/97243.97271, wulf2008component}. By presenting updates in digestible formats with localized visual cues, WhatsApp could lower the cognitive burden on users, particularly those with limited technical expertise \cite{oviatt2006human}, thereby facilitating smoother transitions with each update.


Furthermore, fostering collaborative learning could significantly benefit small business users \cite{doi:10.1177/1461444816629474}. When it comes to communities based on communal interests, values, or goals, collaborative learning can be particularly powerful, as these communities often have strong shared ties that can foster deeper, more meaningful interactions \cite{panitz1998encouraging}. P3 and P8 highlighted how they relied on personal networks to manage changes. WhatsApp could formalize this by creating support forums or user communities where small business owners can share experiences and solutions \cite{10.1145/97243.97271, wulf2008component}. This would facilitate peer-to-peer learning and provide a platform for collective problem-solving. 

Lastly, WhatsApp could reconsider its approach to monetization, particularly for small businesses, as rising costs and complexity can be significant barriers, as seen with users like P3 and P13. Implementing tiered pricing for small businesses could help reduce financial burdens and support long-term growth—a practice common in developing countries within pharmaceutical industries \cite{moon2011winwin}. This approach could enhance adaptability by allowing users at different business levels to pay for a version that they need, as opposed to each feature, as scalability needs of different users may vary \cite{4085532}. Instead of charging per conversation or per message, a tiered system could offer plans based on the capabilities utilized, providing a more holistic pricing model that aligns with the specific needs of each business.




























































































