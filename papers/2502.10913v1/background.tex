\section{Background}

\subsection{WhatsApp in India: A Brief Review}


\begin{table*}[t]
\footnotesize
  \caption{ \label{tab: Changes} Examples of WhatsApp Business' platform changes since its deployment \cite{whatsapp_business_pricing_updates_2024}; these changes are ongoing and being rolled out in different locations and for individual users at various times. }
  \label{tab:bus}
\resizebox{\textwidth}{!}{%
\begin{tabular}{p{6cm} p{6cm}} 
   \toprule
   \textbf{Original Plan} & \textbf{Changed State} \\
   \midrule
   
   \textbf{Pricing Model (Per Conversation)}: Businesses were charged per conversation, regardless of the number of messages. & \textbf{Pricing Model (Per Message)}: Businesses are now charged per message, with different rates for marketing, utility, and service messages. \\   \textbf{Utility Template Charges}: Utility templates outside the customer service window—defined as the designated hours during which customer support is available—were previously charged per conversation.& \textbf{Utility Template Charges}: Utility templates are now charged per message, rather than per conversation, when sent outside the service window.\\   
   \textbf{Entry Point Conversations (Free Service Window)}: Entry point conversations allowed for a free service window, but were counted as a conversation. & \textbf{Entry Point Conversations (Free Service Window)}: Entry point conversations are free for 24 hours, and then open a customer service window lasting 72 hours. \\
   
  \textbf{Authentication Rate Eligibility (verifying the identity of a business account)}: Eligibility for international authentication rates was previously based on the number of conversations opened. &\textbf{Authentication Rate Eligibility}: Eligibility is now based on the number of authentication messages sent, rather than the number of conversations opened.\\ 

   \textbf{Business Verification (Earlier Context)}: Meta business verification was required, but businesses could proceed with limited features without it. & \textbf{Business Verification (Current State)}: Business verification is a norm now, with several subscription plans that businesses must choose from to appear reliable and use all functionalities of the application. \\
   
   \bottomrule
\end{tabular}%
}
\end{table*}


Founded in 2009 and now owned by Meta, WhatsApp is the world's most used messaging app \cite{dixon2024}, with India being its largest user base \cite{nair2024}. What makes WhatsApp so popular is both its lack of complexity, as well as the relatively low barrier of participation that it expects \cite{10.1145/3014362.3014367, Pang2020, Garimella2024}. This helps \textit{emergent users} overcome complexity barriers, allowing them to engage with the platform more effectively than with more complicated alternatives \cite{10.1145/3014362.3014367, 10.1145/2493190.2493225}. WhatsApp appeals to the masses by offering flexible, user-preference-based communication without algorithmic mediation, providing a convenient and personal way to connect with distant contacts \cite{10.1145/3512964, 10.1145/2531602.2531679}. 

Researchers have highlighted WhatsApp's ritualized role in India's communication landscape, referring to Couldry's \cite{rothenbuhler2005media} concept of ritualization, which encourages researchers to examine the connections between ritual actions and broader social contexts, including the practices, beliefs, and categories that make these rituals possible. This helps explain why people prefer it over other tools for maintaining their social lives \cite{doi:10.1177/01968599221095177}. The following sections highlight WhatsApp's unique content, its evolution and adaptation as a business tool, and how localized practices in the Global South enhance the effectiveness of WhatsApp Business. I emphasize the need to study how marginalized users adapt WhatsApp Business in culturally specific ways. This research aims to address gaps in the literature by examining how marginalized groups adapt to changes, proposing improvements to better support them, and contributing to strategies that prevent their decentering in design processes. Here, decentered users refer to individuals or communities who, due to broader systemic forces and technologies, are systematically displaced from the center of design considerations, often leading to their needs being overlooked or inadequately addressed \cite{belfer_center_design_from_the_margins}. This research emphasizes the importance of repositioning these users at the core of design processes to ensure more inclusive, equitable, and effective outcomes \cite{belfer_center_design_from_the_margins_2}.


\subsubsection{WhatsApp's Unique Content Sharing Practices}
Unlike Twitter and Facebook, where original posts remain intact when shared, WhatsApp often fragments and disperses content without retaining the original sources \cite{10.1145/3512964}. This informality and ease of information sharing also make it highly susceptible to misinformation and disinformation \cite{paris2024hidden, 10.1145/3641010, 10.1145/3432948}. Additionally, it poses challenges related to spam, information overload, and other concerns prevalent in digital applications \cite{shahid2024one}. While tools to combat misinformation are widely studied, scholars also note that community-based code-switching helps assess trustworthiness and information credibility \cite{10.1145/3637429}. In India, the ritualized use of WhatsApp frequently hinges on such community-based, trust-oriented interactions \cite{10.1145/3613905.3651034, 10.1145/3491102.3517575}, where personal connections play a central role even in professional settings \cite{Durgungoz2022, 10.1145/3411764.3445221}. Business on WhatsApp, where personal and professional uses are intertwined, represents a unique techno-cultural phenomenon that warrants further investigation.

\subsubsection{WhatsApp for Business: An Evolution}

Using WhatsApp for business represents a culturally unique yet natural adaptation of the platform, showcasing how it empowers users in resource-constrained settings and warrants further investigation \cite{10.1145/3555584}. Users have adopted fragmented and informal methods for business operations, including creating groups and communities to engage customers, facilitate transactions, and leverage word-of-mouth for promotion \cite{10.1145/3613905.3651034, Sugiyantoro2022, modak2017dancing, kottani2021study, bagdare2021whatsapp}. Particularly, during the pandemic, online business use surged as in-person stores, often trusted by customers, had to close temporarily or shut down \cite{sumarni2022utilization, Sugiyantoro2022}. Many businesses transitioned to digital modes maintain connections with their loyal customers and continue operations virtually \cite{doi:10.1177/20501579241246721}. However, a critical gap remains in understanding how the appropriation of a personal messaging tool into a business tool occurs. This study aims to address this gap using WhatsApp as a case example.


WhatsApp for Business differs from the personal version by offering features tailored for business use. While WhatsApp (personal) is designed for private communication, WhatsApp for Business includes tools like customizable business profiles, product catalogs, automated messaging, and quick replies to streamline customer interactions. Additionally, businesses can run ads that direct users to WhatsApp chats, a feature not available in the personal app. Overall, WhatsApp for Business is designed to support professional communication and scalability, unlike the personal app \cite{WhatsAppPricing2024}.


WhatsApp Business, originally designed to help businesses of all sizes communicate with customers through features like business profiles, catalogs, and automated responses, has gradually evolved into a more professionalized platform \cite{whatsappforbusiness} (Also see table \ref{tab: Changes} that details selected changes). Meta's shift to per-message pricing for WhatsApp Business, following earlier per-conversation pricing changes, significantly increased costs for small businesses, prompting some to explore alternatives \cite{10.1145/3613905.3651034}. This shift meant that businesses were charged for each individual message sent, including marketing, authentication, and utility templates outside of the customer service window, which increased costs for those with large-scale messaging operations \cite{cornish2023whatsapp}. Other recent updates, including pricing changes, verification process, and new messaging features, also signal a shift toward a more enterprise-level platform \cite{WhatsAppPricing2024}. For small businesses, this leads to higher costs, especially for customer acquisition. It also introduces other complexities like enhanced payment options which can be activated by linking a bank account and using UPI (Unified Payments Interface), a real-time payment system that enables instant fund transfers, with payments completed after reviewing the order and entering a UPI code for authentication \cite{whatsapp_payment_guide_2024}. However, these changes have made the platform more expensive and are perceived as less accessible for smaller businesses \cite{10.1145/3613905.3651034}.



Drawing from prior research on the rhetoric of WhatsApp Business and its formal introduction in 2018—following users' prior appropriation of personal WhatsApp accounts for local, community-centric purposes—this study examines users' adaptive strategies and responses to platform updates \cite{10.1145/3613905.3651034}. It empirically explores how marginalized business owners, who may face challenges due to limited digital skills, use WhatsApp Business to maintain and grow their operations. Studying this phenomenon is crucial for the HCI community because it sheds light on how digital tools impact marginalized groups and highlights the need for design practices that accommodate diverse user contexts-- so as to not decenter them through design.  By understanding the adaptive strategies of these users, this research can contribute to the development of more inclusive and supportive technologies that enhance successful appropriation and empowerment for all users, particularly those with limited digital resources.

Thus, I aim to bridge the gap in existing literature by examining how marginalized users—particularly those from resource-constrained settings—adapt WhatsApp Business to meet their specific needs. These users, often overlooked in the design and development of digital tools, face unique challenges that hinder their ability to fully utilize evolving technologies. By exploring how these users navigate frequent platform updates and integrate WhatsApp Business into their operations, the study seeks to offer insights into more inclusive design practices. Centering them can contribute to the development of more accessible, equitable, and user-centered digital tools that empower marginalized communities and ensure their needs are addressed in future technology design.













\subsection{Appropriating Infrastructuralized Platforms}



Platforms are digital systems that enable users to interact, create content, or exchange services, often with some level of reprogrammability or customization (e.g., social media platforms, app stores) \cite{doi:10.1177/2056305115603080, 10.1145/3313831.3376201, doi:10.1177/1461444816661553}. In contrast, infrastructures are large-scale foundational systems that support other services or systems, typically serving a broad range of users and becoming embedded in everyday practices (e.g., the internet, electricity grids) \cite{star1996steps}. As has been discussed in prior work, Meta’s global network of interconnected platforms, through its control over communication channels and data infrastructure, functions as a critical communication infrastructure, supporting various business operations, user interactions, and content delivery worldwide \cite{lunden2024meta}. Early views of infrastructure were constrained to physical, technical, and material systems, such as railways and power grids, which were seen as foundational \cite{article2006}. Contemporary research \cite{star1996steps} views infrastructure as a sociotechnical and relational phenomenon rather than merely a supportive framework, which is the perspective I adopt in this work. My focus is on internet infrastructures, which also develop gradually and are influenced by a range of social, political, and economic factors \cite{doi:10.1177/14614448231152546}. 

Platforms evolve into infrastructures when they extend beyond their initial purpose, become deeply embedded in everyday life, and function as foundational elements for other systems. This transformation often occurs through user appropriation and ongoing processes of infrastructuring-- defined as \textit{"the intentional production of infrastructure as a means of achieving a particular goal or the desire to solve a particular problem"} \cite{10.1145/3359175, 10.1145/2818048.2820015, doi:10.1177/0162243913516012}. Thus, platforms evolve from providing specific services to becoming essential, ubiquitous components of the broader technological and social landscape \cite{doi:10.1177/1461444816661553}-- and users' appropriation is often a contributor to the creation of infrastructuralized platforms-- as explicated in \cite{10.1145/3313831.3376201} about WeChat. I draw from Stevens \cite{stevens2009appropriation}, who conceptualizes appropriation as an "\textit{ongoing process}" in which users adapt and transform a technology (in my case a platform) to suit their context, simultaneously shaping its functionality and meaning. This shift allow platforms to become default choices, enabling their owners—such as corporations or service providers—to maintain their original reputations and collect more data from users \cite{10.1145/3313831.3376201}. 

In this study, I use infrastructuring as a dynamic lens where users continuously adapt \cite{karasti2018studying} and refine their digital business practices using WhatsApp, tailoring the platform to meet their specific needs and changing circumstances. Once platforms become infrastructure, their owners gain substantial power over users due to limited alternatives, causing users to continually adjust their \textit{folk theories} to adapt to ongoing changes \cite{10.1145/3476080, 10.1145/3392847, 10.1145/3533700, 10.1145/3359146}. When platforms like WhatsApp are vital to livelihoods, such as for small businesses, users engage in what can be described as \textit{routine infrastructuring} \cite{10.1145/3359175}. This involves being resilient and adapting to frequent disruptions within the platform, which requires ongoing effort to maintain and adjust the infrastructure.


%Amidst the platform's continual changes, users consistently reconstruct the environment to serve their needs, not only to market their products but also to create resilient systems that adapt to and work around the limitations of \textit{formal infrastructures} \cite{10.1145/3359175, gordon2018grassroots}. 

Mwesigwa and Csíkszentmihályi \cite{10.1145/3613904.3642590} developed the appropriation matrix to explain \textit{appropriation} (initial introduction of technologies by companies and governments), \textit{re-appropriation} (secondary use and appropriation of technologies by local communities and development actors), and \textit{reverse appropriation} (formal integration of these uses into more professional and often unaligned systems by companies). This can be explained through WhatsApp's lifecycle. Initially, WhatsApp was introduced as a communication tool for personal messaging (appropriation). However, local businesses, especially small enterprises, began using WhatsApp not just for messaging, but also for customer service, order processing, and even marketing (re-appropriation)—which is the focus of my first research question. This use was not part of the original design, but it became so widespread that WhatsApp eventually introduced business-specific features, such as WhatsApp Business accounts, which allow businesses to manage customer interactions more professionally (reverse appropriation). The adaptation to these changes is the focus of my second research question. 

Users of infrastructural platforms for purposes such as care \cite{10.1145/3544548.3581040}, networking \cite{Crabu2018}, protection \cite{10.1145/3313831.3376339}, and well-being \cite{wilson2016infrastructure} may encounter alterations in these systems that create vulnerabilities, often unnoticed and overlooked for profit-driven reasons due to reverse appropriation \cite{Hague2019, 10.1145/3476080}. This is evident in how Meta, for example, capitalizes on users' appropriated systems, such as their use of WhatsApp for business, by \textit{reverse-appropriating} these practices through the introduction of the new and formalized platform \cite{10.1145/3613905.3651034}. The co-opting of local infrastructure as a form of reverse appropriation is often framed within the discourse of \textit{technosolutionism}, which has been specifically identified as a mode of both data and digital colonialism \cite{mahmoudi2021race}. With the widespread adoption of platforms like WhatsApp and their pervasive presence in daily life, examining these platforms as infrastructures can provide valuable insights \cite{doi:10.1177/1461444816661553, 10.1145/3613905.3651034}. The study of infrastructures as \textit{sociotechnical geometries of power} \cite{graham2001splintering} within HCI has highlighted the significant inequalities and the persistent maximization of monopolistic behaviors \cite{10.1145/2702123.2702573, 10.1145/3491101.3505649, 10.1145/3557890}. 

I contribute to HCI research that highlights the importance of understanding how users who appropriate platforms (and infrastructure them) adapt to and identify challenges when the platforms change \cite{10.1145/3613904.3642590, 10.1145/3359175, doi:10.1177/1461444816629474}. By using the concept of \textit{infrastructuring}, I emphasize the ongoing effort where users, as key actors, dynamically and continually \textit{appropriate an infrastructuralized platform}. 

%This section explores how platforms like WhatsApp evolve into infrastructures through user appropriation and infrastructuring. I highlights gaps in understanding how users, especially small businesses, adapt to disruptions within these platforms. The study addresses these gaps by examining how users engage in routine infrastructuring to maintain resilience. This research contributes to HCI by focusing on user adaptation and the power dynamics involved in infrastructuralized platforms, with research questions centered on how WhatsApp is appropriated by users and its impact on business operations.


\subsection{Digital Colonialism and the Impact on Community-Based Informal Practices}

Digital colonialism refers to the imposition of digital norms and practices by powerful entities that reinforce existing power imbalances. It can be manifested through the control of digital infrastructures, data flows, and platforms that prioritize the interests of dominant actors \cite{mohamed2020decolonial}. This relates to informality and standards, as local, informal practices can be overshadowed or controlled by dominant standards, perpetuating these imbalances \cite{adamu2023no}. Colonial power structures often influence technology design by either providing universal solutions that overlook local specifics or assuming that people (especially from formerly colonized regions) cannot address their own issues \cite{10.1145/3544548.3581538, Couldry2021}-- thus often aiming for a global standard, emphasizing quantification and formalization \cite{lampland2009standards}, which frequently dominate reasoning frameworks \cite{pohawpatchoko2018cultural, cutajar2008knowledge}. HCI scholars have been notably critical of this trend in computing and design research. For example, Mignolo \cite{Mason2020} critiques global computerization driven by colonial expansion and calls on HCI scholars to challenge Eurocentric knowledge and value practices.



Within HCI, scholars have investigated trade practices in the Global South, focusing on how \textit{informality} facilitates and encourages unique business practices \cite{10.1145/3025453.3025643}. Additionally, it was noted that despite the prevalence of \textit{corporatized spaces}, local business places maintain distinct, localized characteristics that cater to the specific needs of communities \cite{10.1145/3025453.3025970}. As previously mentioned, researchers highlight that ritualized systems like WhatsApp are often preferred in informal practices over newer, more sophisticated platforms \cite{Cyr2008, 10.1145/3476057, 10.1145/3361116}. This preference is further influenced by the role of \textit{informal rumors}, which help individuals learn about new technologies and engage in collective sensemaking \cite{10.1145/3290605.3300563}. In this context, those who perceive themselves as marginalized often develop strategies to protect their interests while using services designed for more affluent users \cite{10.1145/3460112.3471961}. WhatsApp exemplifies this adaptation, showing how marginalized individuals appropriate it for their business \cite{10.1145/3613905.3651034}. 

Researchers have highlighted a common pattern where technologies initially adapted for local empowerment through context-specific modifications are later co-opted by large tech companies, leading to uneven power relations in design practices \cite{10.1145/3630106.3658934, 10.1145/3637316, 10.1145/1753326.1753522}- also termed repossession \cite{doi:10.1177/1461444816629474}. Additionally, \textit{material infrastructures and symbolic constructions}, such as the application itself and the ability of developers to design and redesign it in ways that maximize their profits, reinforce power imbalances and reflect colonial undertones \cite{doi:10.1177/1527476419831640}. In the past, Facebook's \textit{Free Basics} has been criticized as a form of digital colonialism, primarily because it was seen as a data collection tactic disguised as free internet access, while promoting large-scale technosolutionism that lacked substantial value \cite{Thorat2020}. Similarly, WhatsApp Business, Meta’s formal application, aims to enhance user engagement and sales but imposes costs on users in the Global South, who lack alternative platforms \cite{whatsapp_business, 10.1145/3613905.3651034}. Imposing on users with non-contextual designs pushes them toward Western business practices that emphasize formal competence, standardized procedures, impersonal task roles, and strong organizational commitment \cite{berger2020doing, Sinha1990}, which often contrast with traits like trust and empathy that are more prevalent in non-western contexts \cite{berger2020doing}. Researchers emphasize incorporating decolonial perspectives into appropriation to challenge Western standardization of knowledge and communication practices \cite{10.1145/3328020.3353927, wagner2020decolonizing}, while advocating for a reexamination of user-centered design approaches in non-normative cultures often overlooked by Western frameworks \cite{doi:10.1177/0162243910389594, 10.1145/2662155.2662195}.


 In conclusion, digital colonialism reinforces power imbalances by imposing standardized systems that overlook local needs, highlighting how small businesses in the Global South adapt informal practices to cope with disruptions and pushing users toward Westernized models. It emphasizes the need to recognize and respect local practices in design, challenging uniform digital norms. In this study, I address a critical gap in HCI by challenging \textit{top-down, paternalistic} views of technology in business contexts \cite{10.1145/3025453.3025643, cruz2021decolonizing}. My empirical work aims to demonstrate how users extend informal methods to manage business disruptions, especially when these disruptions push them toward standardized approaches that may not align with their sociotechnical and cultural contexts. By highlighting these adaptation practices, I contribute to HCI research on appropriation through informal practices \cite{10.1145/3328020.3353927, 10.1145/3571811}, underscoring the limitations of imposing uniform solutions and advocating for approaches that better respect and integrate the diverse needs of users.

















