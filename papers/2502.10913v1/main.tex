

%%
%% This is file `sample-sigconf.tex',
%% generated with the docstrip utility.
%%
%% The original source files were:
%%
%% samples.dtx  (with options: `sigconf')
%% 
%% IMPORTANT NOTICE:
%% 
%% For the copyright see the source file.
%% 
%% Any modified versions of this file must be renamed
%% with new filenames distinct from sample-sigconf.tex.
%% 
%% For distribution of the original source see the terms
%% for copying and modification in the file samples.dtx.
%% 
%% This generated file may be distributed as long as the
%% original source files, as listed above, are part of the
%% same distribution. (The sources need not necessarily be
%% in the same archive or directory.)
%%
%% Commands for TeXCount
%TC:macro \cite [option:text,text]
%TC:macro \citep [option:text,text]
%TC:macro \citet [option:text,text]
%TC:envir table 0 1
%TC:envir table* 0 1
%TC:envir tabular [ignore] word
%TC:envir displaymath 0 word
%TC:envir math 0 word
%TC:envir comment 0 0
%%
%%
%% The first command in your LaTeX source must be the \documentclass command.
\documentclass[sigconf]{acmart}
%\documentclass[manuscript,review, anonymous]{acmart}

%% NOTE that a single column version may be required for 
%% submission and peer review. This can be done by changing
%% the \doucmentclass[...]{acmart} in this template to 
%% \documentclass[manuscript,screen]{acmart}
%% 
%% To ensure 100% compatibility, please check the white list of
%% approved LaTeX packages to be used with the Master Article Template at
%% https://www.acm.org/publications/taps/whitelist-of-latex-packages 
%% before creating your document. The white list page provides 
%% information on how to submit additional LaTeX packages for 
%% review and adoption.
%% Fonts used in the template cannot be substituted; margin 
%% adjustments are not allowed.
%%
%%
%% \BibTeX command to typeset BibTeX logo in the docs
\AtBeginDocument{%
  \providecommand\BibTeX{{%
    \normalfont B\kern-0.5em{\scshape i\kern-0.25em b}\kern-0.8em\TeX}}}

%% Rights management information.  This information is sent to you
%% when you complete the rights form.  These commands have SAMPLE
%% values in them; it is your responsibility as an author to replace
%% the commands and values with those provided to you when you
%% complete the rights form.
%\setcopyright{acmcopyright}
%\copyrightyear{2024}
%\acmYear{2024}
%\acmDOI{XXXXXXX.XXXXXXX}

%% These commands are for a PROCEEDINGS abstract or paper.
%\acmConference[CHI' 25]{Make sure to enter the correct
 % conference title from your rights confirmation emai}{April 26-- May 01,
  %2025}{Woodstock, NY}
%
%  Uncomment \acmBooktitle if th title of the proceedings is different
%  from ``Proceedings of ...''!
%
%\acmBooktitle{Woodstock '18: ACM Symposium on Neural Gaze Detection,
%  June 03--05, 2018, Woodstock, NY} 
%\acmPrice{15.00}
%\acmISBN{978-1-4503-XXXX-X/18/06}


\copyrightyear{2025}
\acmYear{2025}
\setcopyright{acmlicensed}\acmConference[CHI '25]{CHI Conference on Human Factors in Computing Systems}{April 26-May 1, 2025}{Yokohama, Japan}
\acmBooktitle{CHI Conference on Human Factors in Computing Systems (CHI '25), April 26-May 1, 2025, Yokohama, Japan}
\acmDOI{10.1145/3706598.3713988}
\acmISBN{979-8-4007-1394-1/25/04}


%%
%% Submission ID.
%% Use this when submitting an article to a sponsored event. You'll
%% receive a unique submission ID from the organizers
%% of the event, and this ID should be used as the parameter to this command.
%%\acmSubmissionID{123-A56-BU3}

%%
%% For managing citations, it is recommended to use bibliography
%% files in BibTeX format.
%%
%% You can then either use BibTeX with the ACM-Reference-Format style,
%% or BibLaTeX with the acmnumeric or acmauthoryear sytles, that include
%% support for advanced citation of software artefact from the
%% biblatex-software package, also separately available on CTAN.
%%
%% Look at the sample-*-biblatex.tex files for templates showcasing
%% the biblatex styles.
%%

%%
%% The majority of ACM publications use numbered citations and
%% references.  The command \citestyle{authoryear} switches to the
%% "author year" style.
%%
%% If you are preparing content for an event
%% sponsored by ACM SIGGRAPH, you must use the "author year" style of
%% citations and references.
%% Uncommenting
%% the next command will enable that style.
%%\citestyle{acmauthoryear}

%%
%% end of the preamble, start of the body of the document source.
\begin{document}



%%
%% The "title" command has an optional parameter,
%% allowing the author to define a "short title" to be used in page headers.
\title[WhatsApp for Business]{"Business on WhatsApp is tough now—but am I really a businesswoman?" Exploring Challenges with Adapting to Changes in WhatsApp Business}

%%
%% The "author" command and its associated commands are used to define
%% the authors and their affiliations.
%% Of note is the shared affiliation of the first two authors, and the
%% "authornote" and "authornotemark" commands
%% used to denote shared contribution to the research.



\author{Ankolika De}
\email{apd5873@psu.edu}
%\orcid{}
\affiliation{%
  \institution{College of Information Sciences and Technology, Pennsylvania State University}
  \country{USA}
}
%%
%% By default, the full list of authors will be used in the page
%% headers. Often, this list is too long, and will overlap
%% other information printed in the page headers. This command allows
%% the author to define a more concise list
%% of authors' names for this purpose.
\renewcommand{\shortauthors}{De}

%%
%% The abstract is a short summary of the work to be presented in the
%% article.
\begin{abstract}

This study examines how WhatsApp has evolved from a personal communication tool to a professional platform, focusing on its use by small business owners in India. Initially embraced in smaller, rural communities for its ease of use and familiarity, WhatsApp played a crucial role in local economies. However, as Meta introduced WhatsApp Business with new, formalized features, users encountered challenges in adapting to the more complex and costly platform. Interviews with 14 small business owners revealed that while they adapted creatively, they felt marginalized by the advanced tools. This research contributes to HCI literature by exploring the transition from personal to professional use and introduces the concept of \textit{Coercive Professionalization}. It highlights how standardization by large tech companies affects marginalized users, exacerbating power imbalances and reinforcing digital colonialism, concluding with design implications for supporting community-based appropriations.





\end{abstract}

%%
%% The code below is generated by the tool at http://dl.acm.org/ccs.cfm.
%% Please copy and paste the code instead of the example below.
%%
\begin{CCSXML}
<ccs2012>
<concept>
<concept_id>10003120.10003121</concept_id>
<concept_desc>Human-centered computing~Human computer interaction (HCI)</concept_desc>
<concept_significance>500</concept_significance>
</concept>
<concept>
<concept_id>10003120.10003121.10003122.10003334</concept_id>
<concept_desc>Human-centered computing~User studies</concept_desc>
<concept_significance>100</concept_significance>
</concept>
</ccs2012>
\end{CCSXML}

\ccsdesc[500]{Human-centered computing~Human computer interaction (HCI)}
\ccsdesc[100]{Human-centered computing~User studies}

%%
%% Keywords. The author(s) should pick words that accurately describe
%% the work being presented. Separate the keywords with commas.
\keywords{Infrastructure, Appropriation, Global South, Mobile Phones, Decoliniality, WhatsApp}

%%
%% This command processes the author and affiliation and title
%% information and builds the first part of the formatted document.
\maketitle

\section{Introduction}
\label{sec:introduction}
The business processes of organizations are experiencing ever-increasing complexity due to the large amount of data, high number of users, and high-tech devices involved \cite{martin2021pmopportunitieschallenges, beerepoot2023biggestbpmproblems}. This complexity may cause business processes to deviate from normal control flow due to unforeseen and disruptive anomalies \cite{adams2023proceddsriftdetection}. These control-flow anomalies manifest as unknown, skipped, and wrongly-ordered activities in the traces of event logs monitored from the execution of business processes \cite{ko2023adsystematicreview}. For the sake of clarity, let us consider an illustrative example of such anomalies. Figure \ref{FP_ANOMALIES} shows a so-called event log footprint, which captures the control flow relations of four activities of a hypothetical event log. In particular, this footprint captures the control-flow relations between activities \texttt{a}, \texttt{b}, \texttt{c} and \texttt{d}. These are the causal ($\rightarrow$) relation, concurrent ($\parallel$) relation, and other ($\#$) relations such as exclusivity or non-local dependency \cite{aalst2022pmhandbook}. In addition, on the right are six traces, of which five exhibit skipped, wrongly-ordered and unknown control-flow anomalies. For example, $\langle$\texttt{a b d}$\rangle$ has a skipped activity, which is \texttt{c}. Because of this skipped activity, the control-flow relation \texttt{b}$\,\#\,$\texttt{d} is violated, since \texttt{d} directly follows \texttt{b} in the anomalous trace.
\begin{figure}[!t]
\centering
\includegraphics[width=0.9\columnwidth]{images/FP_ANOMALIES.png}
\caption{An example event log footprint with six traces, of which five exhibit control-flow anomalies.}
\label{FP_ANOMALIES}
\end{figure}

\subsection{Control-flow anomaly detection}
Control-flow anomaly detection techniques aim to characterize the normal control flow from event logs and verify whether these deviations occur in new event logs \cite{ko2023adsystematicreview}. To develop control-flow anomaly detection techniques, \revision{process mining} has seen widespread adoption owing to process discovery and \revision{conformance checking}. On the one hand, process discovery is a set of algorithms that encode control-flow relations as a set of model elements and constraints according to a given modeling formalism \cite{aalst2022pmhandbook}; hereafter, we refer to the Petri net, a widespread modeling formalism. On the other hand, \revision{conformance checking} is an explainable set of algorithms that allows linking any deviations with the reference Petri net and providing the fitness measure, namely a measure of how much the Petri net fits the new event log \cite{aalst2022pmhandbook}. Many control-flow anomaly detection techniques based on \revision{conformance checking} (hereafter, \revision{conformance checking}-based techniques) use the fitness measure to determine whether an event log is anomalous \cite{bezerra2009pmad, bezerra2013adlogspais, myers2018icsadpm, pecchia2020applicationfailuresanalysispm}. 

The scientific literature also includes many \revision{conformance checking}-independent techniques for control-flow anomaly detection that combine specific types of trace encodings with machine/deep learning \cite{ko2023adsystematicreview, tavares2023pmtraceencoding}. Whereas these techniques are very effective, their explainability is challenging due to both the type of trace encoding employed and the machine/deep learning model used \cite{rawal2022trustworthyaiadvances,li2023explainablead}. Hence, in the following, we focus on the shortcomings of \revision{conformance checking}-based techniques to investigate whether it is possible to support the development of competitive control-flow anomaly detection techniques while maintaining the explainable nature of \revision{conformance checking}.
\begin{figure}[!t]
\centering
\includegraphics[width=\columnwidth]{images/HIGH_LEVEL_VIEW.png}
\caption{A high-level view of the proposed framework for combining \revision{process mining}-based feature extraction with dimensionality reduction for control-flow anomaly detection.}
\label{HIGH_LEVEL_VIEW}
\end{figure}

\subsection{Shortcomings of \revision{conformance checking}-based techniques}
Unfortunately, the detection effectiveness of \revision{conformance checking}-based techniques is affected by noisy data and low-quality Petri nets, which may be due to human errors in the modeling process or representational bias of process discovery algorithms \cite{bezerra2013adlogspais, pecchia2020applicationfailuresanalysispm, aalst2016pm}. Specifically, on the one hand, noisy data may introduce infrequent and deceptive control-flow relations that may result in inconsistent fitness measures, whereas, on the other hand, checking event logs against a low-quality Petri net could lead to an unreliable distribution of fitness measures. Nonetheless, such Petri nets can still be used as references to obtain insightful information for \revision{process mining}-based feature extraction, supporting the development of competitive and explainable \revision{conformance checking}-based techniques for control-flow anomaly detection despite the problems above. For example, a few works outline that token-based \revision{conformance checking} can be used for \revision{process mining}-based feature extraction to build tabular data and develop effective \revision{conformance checking}-based techniques for control-flow anomaly detection \cite{singh2022lapmsh, debenedictis2023dtadiiot}. However, to the best of our knowledge, the scientific literature lacks a structured proposal for \revision{process mining}-based feature extraction using the state-of-the-art \revision{conformance checking} variant, namely alignment-based \revision{conformance checking}.

\subsection{Contributions}
We propose a novel \revision{process mining}-based feature extraction approach with alignment-based \revision{conformance checking}. This variant aligns the deviating control flow with a reference Petri net; the resulting alignment can be inspected to extract additional statistics such as the number of times a given activity caused mismatches \cite{aalst2022pmhandbook}. We integrate this approach into a flexible and explainable framework for developing techniques for control-flow anomaly detection. The framework combines \revision{process mining}-based feature extraction and dimensionality reduction to handle high-dimensional feature sets, achieve detection effectiveness, and support explainability. Notably, in addition to our proposed \revision{process mining}-based feature extraction approach, the framework allows employing other approaches, enabling a fair comparison of multiple \revision{conformance checking}-based and \revision{conformance checking}-independent techniques for control-flow anomaly detection. Figure \ref{HIGH_LEVEL_VIEW} shows a high-level view of the framework. Business processes are monitored, and event logs obtained from the database of information systems. Subsequently, \revision{process mining}-based feature extraction is applied to these event logs and tabular data input to dimensionality reduction to identify control-flow anomalies. We apply several \revision{conformance checking}-based and \revision{conformance checking}-independent framework techniques to publicly available datasets, simulated data of a case study from railways, and real-world data of a case study from healthcare. We show that the framework techniques implementing our approach outperform the baseline \revision{conformance checking}-based techniques while maintaining the explainable nature of \revision{conformance checking}.

In summary, the contributions of this paper are as follows.
\begin{itemize}
    \item{
        A novel \revision{process mining}-based feature extraction approach to support the development of competitive and explainable \revision{conformance checking}-based techniques for control-flow anomaly detection.
    }
    \item{
        A flexible and explainable framework for developing techniques for control-flow anomaly detection using \revision{process mining}-based feature extraction and dimensionality reduction.
    }
    \item{
        Application to synthetic and real-world datasets of several \revision{conformance checking}-based and \revision{conformance checking}-independent framework techniques, evaluating their detection effectiveness and explainability.
    }
\end{itemize}

The rest of the paper is organized as follows.
\begin{itemize}
    \item Section \ref{sec:related_work} reviews the existing techniques for control-flow anomaly detection, categorizing them into \revision{conformance checking}-based and \revision{conformance checking}-independent techniques.
    \item Section \ref{sec:abccfe} provides the preliminaries of \revision{process mining} to establish the notation used throughout the paper, and delves into the details of the proposed \revision{process mining}-based feature extraction approach with alignment-based \revision{conformance checking}.
    \item Section \ref{sec:framework} describes the framework for developing \revision{conformance checking}-based and \revision{conformance checking}-independent techniques for control-flow anomaly detection that combine \revision{process mining}-based feature extraction and dimensionality reduction.
    \item Section \ref{sec:evaluation} presents the experiments conducted with multiple framework and baseline techniques using data from publicly available datasets and case studies.
    \item Section \ref{sec:conclusions} draws the conclusions and presents future work.
\end{itemize}

\section{Background}\label{sec:backgrnd}

\subsection{Cold Start Latency and Mitigation Techniques}

Traditional FaaS platforms mitigate cold starts through snapshotting, lightweight virtualization, and warm-state management. Snapshot-based methods like \textbf{REAP} and \textbf{Catalyzer} reduce initialization time by preloading or restoring container states but require significant memory and I/O resources, limiting scalability~\cite{dong_catalyzer_2020, ustiugov_benchmarking_2021}. Lightweight virtualization solutions, such as \textbf{Firecracker} microVMs, achieve fast startup times with strong isolation but depend on robust infrastructure, making them less adaptable to fluctuating workloads~\cite{agache_firecracker_2020}. Warm-state management techniques like \textbf{Faa\$T}~\cite{romero_faa_2021} and \textbf{Kraken}~\cite{vivek_kraken_2021} keep frequently invoked containers ready, balancing readiness and cost efficiency under predictable workloads but incurring overhead when demand is erratic~\cite{romero_faa_2021, vivek_kraken_2021}. While these methods perform well in resource-rich cloud environments, their resource intensity challenges applicability in edge settings.

\subsubsection{Edge FaaS Perspective}

In edge environments, cold start mitigation emphasizes lightweight designs, resource sharing, and hybrid task distribution. Lightweight execution environments like unikernels~\cite{edward_sock_2018} and \textbf{Firecracker}~\cite{agache_firecracker_2020}, as used by \textbf{TinyFaaS}~\cite{pfandzelter_tinyfaas_2020}, minimize resource usage and initialization delays but require careful orchestration to avoid resource contention. Function co-location, demonstrated by \textbf{Photons}~\cite{v_dukic_photons_2020}, reduces redundant initializations by sharing runtime resources among related functions, though this complicates isolation in multi-tenant setups~\cite{v_dukic_photons_2020}. Hybrid offloading frameworks like \textbf{GeoFaaS}~\cite{malekabbasi_geofaas_2024} balance edge-cloud workloads by offloading latency-tolerant tasks to the cloud and reserving edge resources for real-time operations, requiring reliable connectivity and efficient task management. These edge-specific strategies address cold starts effectively but introduce challenges in scalability and orchestration.

\subsection{Predictive Scaling and Caching Techniques}

Efficient resource allocation is vital for maintaining low latency and high availability in serverless platforms. Predictive scaling and caching techniques dynamically provision resources and reduce cold start latency by leveraging workload prediction and state retention.
Traditional FaaS platforms use predictive scaling and caching to optimize resources, employing techniques (OFC, FaasCache) to reduce cold starts. However, these methods rely on centralized orchestration and workload predictability, limiting their effectiveness in dynamic, resource-constrained edge environments.



\subsubsection{Edge FaaS Perspective}

Edge FaaS platforms adapt predictive scaling and caching techniques to constrain resources and heterogeneous environments. \textbf{EDGE-Cache}~\cite{kim_delay-aware_2022} uses traffic profiling to selectively retain high-priority functions, reducing memory overhead while maintaining readiness for frequent requests. Hybrid frameworks like \textbf{GeoFaaS}~\cite{malekabbasi_geofaas_2024} implement distributed caching to balance resources between edge and cloud nodes, enabling low-latency processing for critical tasks while offloading less critical workloads. Machine learning methods, such as clustering-based workload predictors~\cite{gao_machine_2020} and GRU-based models~\cite{guo_applying_2018}, enhance resource provisioning in edge systems by efficiently forecasting workload spikes. These innovations effectively address cold start challenges in edge environments, though their dependency on accurate predictions and robust orchestration poses scalability challenges.

\subsection{Decentralized Orchestration, Function Placement, and Scheduling}

Efficient orchestration in serverless platforms involves workload distribution, resource optimization, and performance assurance. While traditional FaaS platforms rely on centralized control, edge environments require decentralized and adaptive strategies to address unique challenges such as resource constraints and heterogeneous hardware.



\subsubsection{Edge FaaS Perspective}

Edge FaaS platforms adopt decentralized and adaptive orchestration frameworks to meet the demands of resource-constrained environments. Systems like \textbf{Wukong} distribute scheduling across edge nodes, enhancing data locality and scalability while reducing network latency. Lightweight frameworks such as \textbf{OpenWhisk Lite}~\cite{kravchenko_kpavelopenwhisk-light_2024} optimize resource allocation by decentralizing scheduling policies, minimizing cold starts and latency in edge setups~\cite{benjamin_wukong_2020}. Hybrid solutions like \textbf{OpenFaaS}~\cite{noauthor_openfaasfaas_2024} and \textbf{EdgeMatrix}~\cite{shen_edgematrix_2023} combine edge-cloud orchestration to balance resource utilization, retaining latency-sensitive functions at the edge while offloading non-critical workloads to the cloud. While these approaches improve flexibility, they face challenges in maintaining coordination and ensuring consistent performance across distributed nodes.



\subsection{Greedies}
We have two greedy methods that we're using and testing, but they both have one thing in common: They try every node and possible resistances, and choose the one that results in the largest change in the objective function.

The differences between the two methods, are the function. The first one uses the median (since we want the median to be >0.5, we just set this as our objective function.)

Second one uses a function to try to capture more nuances about the fact that we want the median to be over 0.5. The function is:

\[
\text{score}(\text{opinion}) =
\begin{cases} 
\text{maxScore}, & \text{if } \text{opinion} \geq 0.5 \\
\min\left(\frac{50}{0.5 - \text{opinion}}, \frac{\text{maxScore}}{2}\right), & \text{if } \text{opinion} < 0.5 
\end{cases}
\] 

Where we set maxScore to be $10000$.

\subsection{find-c}
Then for the projected methods where we use Huber-Loss, we also have a $find-c$ version (temporary name). This method initially finds the c for the rest of the run. 

The way it does it it randomly perturbs the resistances and opinions of every node, then finds the c value that most closely approximates the median for all of the perturbed scenarios (after finding the stable opinions). 


\section{Analysis} \label{sec:analysis}
In this section, we provide a comprehensive analysis of Satori. First, we demonstrate that Satori effectively leverages self-reflection to seek better solutions and enhance its overall reasoning performance. Next, we observe that Satori exhibits test-scaling behavior through RL training, where it progressively acquires more tokens to improve its reasoning capabilities. Finally, we conduct ablation studies on various components of Satori's training framework. Additional results are provided in Appendix~\ref{app:results}.



\paragraph{COAT Reasoning v.s. CoT Reasoning.}
\begin{table}[h]
  \begin{center}
  \scriptsize
  \captionsetup{font=small}
  \caption{\textbf{COAT Training v.s. CoT Training.} Qwen-2.5-Math-7B trained with COAT reasoning format (Satori-Qwen-7B) outperforms the same base model but trained with classical CoT reasoning format (Qwen-7B-CoT)}
  \setlength{\tabcolsep}{1.3pt}
  \begin{tabular}{cccccccccc}
    \toprule
    \textbf{Model} & \textbf{GSM8K} & \textbf{MATH500}  &  \textbf{Olym.} & \textbf{AMC2023} & \textbf{AIME2024} \\
    \midrule
    Qwen-2.5-Math-7B-Instruct & 95.2 & 83.6 &41.6& 62.5 &16.7 \\
    Qwen-7B-CoT (SFT+RL) & 93.1 & 84.4  &	42.7 &	60.0 & 10.0 \\
    \midrule
    \textbf{Satori-Qwen-7B}  & 93.2 & 85.6  & 46.6  & 67.5  & 20.0 \\
    \bottomrule
  \end{tabular}
  \label{table:ablation-coat}
  \end{center}
\vspace{-1em}
\end{table}
We begin by conducting an ablation study to demonstrate the benefits of COAT reasoning compared to the classical CoT reasoning. Specifically, starting from the synthesis of demonstration trajectories in the format tuning stage, we ablate the ``reflect'' and  ``explore'' actions, retaining only the ``continue'' actions. Next, we maintain all other training settings, including the same amount of SFT and RL data and consistent hyper-parameters. This results in a typical CoT LLM (Qwen-7B-CoT) without self-reflection or self-exploration capabilities. As shown in Table~\ref{table:ablation-coat}, the performance of Qwen-7B-CoT is suboptimal compared to Satori-Qwen-7B and fails to surpass Qwen-2.5-Math-7B-Instruct, suggesting the advantages of COAT reasoning over CoT reasoning.



\paragraph{Satori Exhibits Self-correction Capability.}
% Please add the following required packages to your document preamble:
% \usepackage{multirow}
\begin{table}[h]
\scriptsize
\captionsetup{font=small}
\caption{\textbf{Satori's Self-correction Capability.} T$\rightarrow$F: negative self-correction; F$\rightarrow$T: positive self-correction.}
\setlength{\tabcolsep}{5pt}
\begin{tabular}{lcccccc}
\toprule
\multirow{3}{*}{\textbf{Model}} & \multicolumn{4}{c}{\textbf{In-Domain}}                                                                            & \multicolumn{2}{c}{\textbf{Out-of-Domain}}              \\ \cmidrule[0.2pt]{2-7} 
                                & \multicolumn{2}{c}{\textbf{MATH500}}                    & \multicolumn{2}{c}{\textbf{OlympiadBench}}              & \multicolumn{2}{c}{\textbf{MMLUProSTEM}}         \\
                                & \textbf{T$\rightarrow$F} & \textbf{F$\rightarrow$T} & \textbf{T$\rightarrow$F} & \textbf{F$\rightarrow$T} & \textbf{T$\rightarrow$F} & \textbf{F$\rightarrow$T} \\ \midrule[0.5pt]
Satori-Qwen-7B-FT                  & 79.4\%                    & 20.6\%                    & 65.6\%                    & 34.4\%                    & 59.2\%                    & 40.8\%                    \\
\textbf{Satori-Qwen-7B}                     & 39.0\%                       & 61.0\%                       & 42.1\%                    & 57.9\%                    & 46.5\%                    & 53.5\%                    \\ \bottomrule
\end{tabular}
\label{table:finegrain-reflect}
\end{table}
We observe that Satori frequently engages in self-reflection during the reasoning process (see demos in Section~\ref{sec:demo}), which occurs in two scenarios: (1) it triggers self-reflection at intermediate reasoning steps, and (2) after completing a problem, it initiates a second attempt through self-reflection. We focus on quantitatively evaluating Satori's self-correction capability in the second scenario. Specifically, we extract responses where the final answer before self-reflection differs from the answer after self-reflection. We then quantify the percentage of responses in which Satori's self-correction is positive (i.e., the solution is corrected from incorrect to correct) or negative (i.e., the solution changes from correct to incorrect). The evaluation results on in-domain datasets (MATH500 and Olympiad) and out-of-domain datasets (MMLUPro) are presented in Table~\ref{table:finegrain-reflect}. First, compared to Satori-Qwen-FT which lacks the RL training stage, Satori-Qwen demonstrates a significantly stronger self-correction capability. Second, we observe that this self-correction capability extends to out-of-domain tasks (MMLUProSTEM). These results suggest that RL plays a crucial role in enhancing the model's true reasoning capabilities.


\paragraph{RL Enables Satori with Test-time Scaling Behavior.}
\begin{figure}[h]
    \centering
    \includegraphics[width=0.5\textwidth]{Figures/rm_shaping_tot_len.pdf}
    \vspace{-2em}
\caption{\textbf{Policy Training Acc. \& Response length v.s. RL Train-time Compute.} Through RL training, Satori learns to improve its reasoning performance through longer thinking.}
\label{fig:test_time_scaling}
\end{figure}
\begin{figure}[h]
    \centering
    \includegraphics[width=0.45\textwidth]{Figures/length_across_levels.pdf}
    \vspace{-1.5em}
\caption{\textbf{Above: Test-time Response Length v.s. Problem Difficulty Level; Below: Test-time Accuracy v.s. Problem Difficulty Level.} Compared to FT model (Satori-Qwen-FT), Satori-Qwen uses more test-time compute to tackle more challenging problems.}
\label{fig:difficulty_level}
\vspace{-1em}
\end{figure}

Next, we aim to explain how reinforcement learning (RL) incentivizes Satori's autoregressive search capability. First, as shown in Figure~\ref{fig:test_time_scaling}, we observe that Satori consistently improves policy accuracy and increases the average length of generated tokens with more RL training-time compute. This suggests that Satori learns to allocate more time to reasoning, thereby solving problems more accurately. One interesting observation is that the response length first decreases from 0 to 200 steps and then increases. Upon a closer investigation of the model response, we observe that in the early stage, our model has not yet learned self-reflection capabilities. During this stage, RL optimization may prioritize the model to find a shot-cut solution without redundant reflection, leading to a temporary reduction in response length. However, in later stage, the model becomes increasingly good at using reflection to self-correct and find a better solution, leading to a longer response length.
 
Additionally, in Figure~\ref{fig:difficulty_level}, we evaluate Satori's test accuracy and response length on MATH datasets across different difficulty levels. Interestingly, through RL training, Satori naturally allocates more test-time compute to tackle more challenging problems, which leads to consistent performance improvements compared to the format-tuned (FT) model.



\paragraph{Large-scale FT v.s. Large-scale RL.}
\begin{table}[h]
  \begin{center}
  \scriptsize
  \captionsetup{font=small}
  \caption{\textbf{Large-scale FT V.S. Large-scale RL} Satori-Qwen (10K FT data + 300K RL data) outperforms same base model Qwen-2.5-Math-7B trained with 300K FT data (w/o RL) across all math and out-of-domain benchmarks.}
  \setlength{\tabcolsep}{1.15pt}
  \vspace{-0.5em}
\begin{tabular}{lccccc}
\toprule
\textbf{(In-domain)}   & \textbf{GSM8K}   & \textbf{MATH500} & \textbf{Olym.} & \textbf{AMC2023} & \textbf{AIME2024} \\ \midrule
Qwen-2.5-Math-7B-Instruct & 95.2 & 83.6                     & 41.6                  & 62.5             & 16.7                 \\
Satori-Qwen-7B-FT (300K)     & 92.3 & 78.2                       & 40.9           & 65.0               & 16.7              \\
\textbf{Satori-Qwen-7B}         & 93.2        & 85.6                     & 46.6           & 67.5             & 20.0                \\ \midrule
\textbf{(Out-of-domain)}  & \textbf{BGQA}    & \textbf{CRUX}  & \textbf{STGQA} & \textbf{TableBench}   & \textbf{STEM}     \\ \midrule
Qwen-2.5-Math-7B-Instruct & 51.3             & 28.0             & 85.3           & 36.3             & 45.2              \\
Satori-Qwen-7B-FT (300K)     & 50.5             & 29.5           & 74.0             & 35.0               & 47.8              \\
\textbf{Satori-Qwen-7B}               & 61.8             & 42.5           & 86.3           & 43.4             & 56.7              \\ \bottomrule
\end{tabular}
  \label{table:ablation-ft-rl}
  \end{center}
\end{table}
We investigate whether scaling up format tuning (FT) can achieve performance gains comparable to RL training. We conduct an ablation study using Qwen-2.5-Math-7B, trained with an equivalent amount of FT data (300K). As shown in Table~\ref{table:ablation-ft-rl}, on the math domain benchmarks, the model trained with large-scale FT (300K) fails to match the performance of the model trained with small-scale FT (10K) and large-scale RL (300K). Additionally, the large-scale FT model performs significantly worse on out-of-domain tasks, demonstrates RL’s advantage in generalization.


\paragraph{Distillation Enables Weak-to-Strong Generalization.} 
\begin{figure}[!t]
    \centering
     \includegraphics[width=0.4\textwidth]
     {Figures/distillation.pdf}
     \vspace{-1.5em}
\caption{\textbf{Format Tuning v.s. Distillation.} Distilling from a Stronger model (Satori-Qwen-7B) to weaker base models (Llama-8B and Granite-8B) are more effective than directly applying format tuning on weaker base models.}
\label{fig:distill}
\vspace{-1em}
\end{figure}
Finally, we investigate whether distilling a stronger reasoning model can enhance the reasoning performance of weaker base models. Specifically, we use Satori-Qwen-7B to generate 240K synthetic data to train weaker base models, Llama-3.1-8B and Granite-3.1-8B. For comparison, we also synthesize 240K FT data (following Section~\ref{subsec:format}) to train the same models. We evaluate the average test accuracy of these models across all math benchmark datasets, with the results presented in Figure~\ref{fig:distill}. The results show that the distilled models outperform the format-tuned models. 

This suggests a new, efficient approach to improve the reasoning capabilities of weaker base models: (1) train a strong reasoning model through small-scale
FT and large-scale RL (our Satori-Qwen-7B) and (2) distill the strong reasoning capabilities of the model into weaker base models. Since RL only requires answer labels as supervision, this approach introduces minimal costs for data synthesis, i.e., the costs induced by a multi-agent data synthesis framework or even more expensive human annotation.




Our findings challenge the conjecture that code-comment coherence, as measured by SIDE \cite{mastropaolo2024evaluating}, is a critical quality attribute for filtering instances of code summarization datasets. By selecting $\langle code, summary \rangle$ pairs with high-coherence for training allow to achieve the same results that would be achieved by randomly selecting such a number of instances. At the same time, we observed that reducing the datasets size up to 50\% of the training instances does not significantly affect the effectiveness of the models, even when the instances are randomly selected. These results have several implications.

First, code-comment consistency might not be a problem in state-of-the-art datasets in the first place, as also suggested in the results of RQ$_0$. Also, the DL models we adopted (and, probably, bigger models as well) are not affected by inconsistent code-comment pairs, even when these inconsistencies are present in the training set.
Despite the theoretical benefits of filtering by SIDE \cite{mastropaolo2024evaluating}, that is the state-of-the-art metric for measuring code-comment alignment, our results indicate its limitations in improving the \textit{overall} quality of the training sets for code summarization task.
Nevertheless, other quality aspects of code and comments that have not been explored yet (such as readability) may be important for smartly selecting the training instances.
Future work should explore such quality aspects further.

Our results clearly indicate that state-of-the-art datasets contain instances that do not contribute to improving the models' effectiveness. This finding is related to a general phenomenon observed in Machine Learning and Deep Learning. Models reach convergence when they are trained for a certain amount of time (epochs). Additional training provides smaller improvements and increases the risk of overfitting. We show that the same is true for data. In terms of effectiveness, model convergence is achieved with fewer training instances than previously assumed. Limiting the number of epochs may make it possible to reach model convergence with a subset of training data, maintaining model effectiveness, reducing resource demands and minimizing the risk of overfitting.
Future work could explore different criteria for data selection that identify the most informative subsets for training.
Conversely, this insight suggests that currently available datasets suffer from poor diversity (thus causing the previously discussed phenomenon).
This latter insight constitutes a clear warning for researchers interested in building code summarization datasets, which should include instances that add relevant information instead of adding more data, which might turn out to be useless.

Finally, it is worth pointing out that another benefit of the reduction we performed is the environmental impact. Reducing the number of training instances implies a reduced training time, which, in turn, lowers the resources necessary to perform training and, thus, energy consumption and CO$_2$ emissions.
We performed a rough estimation of the training time across different selections of \textit{TL-CodeSum} and \textit{Funcom} datasets and estimated a proxy of the CO$_2$ emissions for each model training phase by relying on the ML CO$_2$ impact calculator\footnote{\url{https://mlco2.github.io/impact/\#compute}} \cite{lacoste2019quantifying}. Such a calculator considers factors such as the total training time, the infrastructure used, the carbon efficiency, and the amount of carbon offset purchased. The estimation of CO$_{2}$ emissions needed to train the model with the \textit{Full} selection of \textit{Funcom} ($\sim$ 200 hours) is equal to 26.05 Kg, while with the optimized training set, \ie $SIDE_{0.9}$ ($\sim$ 90 hours), the estimation is 11.69 Kg of $CO_2$ (-55\% emissions).
While we recognize that this method provides an estimation rather than a precise measurement, it offers a glimpse into the environmental impact of applying data reduction.


\section{Limitation}
The use of 3D-printed PLA for structural components improves improving ease of assembly and reduces weight and cost, yet it causes deformation under heavy load, which can diminish end-effector precision. Using metal, such as aluminum, would remedy this problem. Additionally, \robot relies on integrated joint relative encoders, requiring manual initialization in a fixed joint configuration each time the system is powered on. Using absolute joint encoders could significantly improve accuracy and ease of use, although it would increase the overall cost. 

%Reliance on commercially available actuators simplifies integration but imposes constraints on control frequency and customization, further limiting the potential for tailored performance improvements.

% The 6 DoF configuration provides sufficient mobility for most tasks; however, certain bimanual operations could benefit from an additional degree of freedom to handle complex joint constraints more effectively. Furthermore, the limited torque density of commercially available proprioceptive actuators restricts the payload and torque output, making the system less suitability for handling heavier loads or high-torque applications. 

The 6 DoF configuration of the arm provides sufficient mobility for single-arm manipulation tasks, yet it shows a limitation in certain bimanual manipulation problems. Specifically, when \robot holds onto a rigid object with both hands, each arm loses 1 DoF because the hands are fixed to the object during grasping. This leads to an underactuated kinematic chain which has a limited mobility in 3D space. We can achieve more mobility by letting the object slip inside the grippers, yet this renders the grasp less robust and simulation difficult. Therefore, we anticipate that designing a lightweight 3 DoF wrist in place of the current 2 DoF wrist allows a more diverse repertoire of manipulation in bimanual tasks.

Finally, the limited torque density of commercially available proprioceptive actuators restricts the performance. Currently, all of our actuators feature a 1:10 gear ratio, so \robot can handle up to 2.5 kg of payload. To handle a heavier object and manipulate it with higher torque, we expect the actuator to have 1:20$\sim$30 gear ratio, but it is difficult to find an off-the-shelf product that meets our requirements. Customizing the actuator to increase the torque density while minimizing the weight will enable \robot to move faster and handle more diverse objects.

%These constraints highlight opportunities for improvement in future iterations, including alternative materials for enhanced rigidity, custom actuator designs for higher control precision and torque density, the adoption of absolute joint encoders, and optimized configurations to balance dexterity and weight.



\section{Conclusion}

I interviewed 14 WhatsApp users from India who used WhatsApp to manage their home-based small businesses. My research explored how they 1) appropriated WhatsApp for business purposes, 2) adapted to changes in WhatsApp Business, and 3) perceived these changes. I found that users frequently leveraged their personal contacts and connections for business, translating their personal use of WhatsApp into professional contexts. Additionally, they struggled to keep up with rapid changes and often felt that such platforms were not designed with their needs in mind. Many participants developed workarounds for changes that did not suit them. Lastly, I observed a diminished sense of self-esteem related to the ongoing processes of appropriation and infrastructuring. I contributed to HCI research in several ways. Firstly, I introduced the concept of \textit{coercive professionalization}, which describes how Meta has sought to co-opt infrastructural platforms initially developed through local community appropriation. This process represents a form of reverse appropriation, where these platforms are reclaimed by big tech companies. I also explored the features of coercive professionalization in how they relate to coloniality and marginalization. Finally, I discussed designs that can support the legitimization of local customization and appropriation of systems.























%%
%% The acknowledgments section is defined using the "acks" environment
%% (and NOT an unnumbered section). This ensures the proper
%% identification of the section in the article metadata, and the
%% consistent spelling of the heading.
%\begin{acks}
%Acknowledgments
%\end{acks}

%%
%% The next two lines define the bibliography style to be used, and
%% the bibliography file.
\bibliographystyle{ACM-Reference-Format}
\bibliography{references.bib}


%%
%% If your work has an appendix, this is the place to put it.
%\appendix

%\section{Appendix header}

\end{document}
\endinput
%%
%% End of file `sample-sigconf.tex'.
