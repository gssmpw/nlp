\section{Findings}
%provide summary and correct pronouns based on translated transcripts

The findings, divided into three sections, address the corresponding research questions. The first section explains how users adapt business practices on WhatsApp, while the next two sections explore the implications of ongoing infrastructuring on users, answering the second question.

\subsection{Translating and Enacting the Personal to the Professional}

In this section, I discuss the interviewees' experiences of using WhatsApp for business, focusing on how they came to adopt the platform. While WhatsApp’s informal and personal nature played a key role in its use, it also blended with professional tasks, creating additional labor that was often overlooked. This led business owners to have dynamic, evolving views on the role WhatsApp played in their work.


\subsubsection{Personal Familiarity and Ease of Transition}

Interviewees explained that their adoption of WhatsApp for Business was a natural extension of personal use, often driven by job losses, family responsibilities, the pandemic, or changes in family members' professional circumstances. For example, P5 appreciated the convenience of WhatsApp as a system for managing her business, reflecting on how her professional work integrated seamlessly with her personal use: "\textit{WhatsApp is convenient. It’s easier to use than other platforms, and it’s where most of my customers are.}" Here, P5 described how using WhatsApp as a natural platform was not only convenient for her but also for her customers, who were already on WhatsApp and did not have to make an extra effort to engage with her business. Additionally, P1 emphasized that, given her existing familiarity with the app, it was a "\textit{smart move to leverage it for a wider range of purposes}". P9 shared a similar perspective, noting that WhatsApp felt like a natural progression due to its familiarity. She explained that using WhatsApp casually for personal connections made it \textit{"the most straightforward option"} when everything shifted to virtual platforms. According to P9, \textit{"It was like a lifeline—I could instantly reach my customers, share updates, and manage orders all from an app I was already comfortable with."} She also emphasized that since everyone was already on WhatsApp, it \textit{"avoided the need to transition to a new platform."} P12 confirmed this choice of WhatsApp for her business, explaining that she felt familiar with the platform and its nuances, or at least with the aspects necessary to run her business effectively-

\begin{quote}
     \textit{I’d already been using WhatsApp to make groups, video call my friends and share messages and interesting information, so it was very obvious to use it for my business. I started using it to chat with customers, update them on new stuff, and even handle orders.}
\end{quote}


In particular, P1 added that her WhatsApp business was a direct response to the challenges posed by the pandemic: "\textit{I started using WhatsApp for business about 5 or 6 years ago, but the real shift happened during the pandemic when I had to move everything online due to shop closures.}" This sentiment was echoed by P3, who said,\textit{“I used to run my boutique offline. But when everything closed, I had to find a way to keep the business going. WhatsApp allowed me to continue reaching my customers and even expand my network.}” Given that the customers and staff of the boutique were already familiar and somewhat close, WhatsApp felt like a natural extension of their existing communication, even though people were not accustomed to \textit{app-based businesses} (P11). It simply served as an intuitive tool to stay connected and continue business as usual.




\subsubsection{Influence of Personal Connections and Informal Engagement}

Some interviewees mentioned that a personal connection had introduced or encouraged them to use WhatsApp for business. As P2 explained: \textit{"My sister started using WhatsApp for business during lockdown. I saw that she could do it so easily, so why not try it? Since I was already familiar with the app, it felt pretty easy to shift to!"} This guidance from someone within their personal circle, also provided confidence. 

Many individuals also aspired to use WhatsApp to extend their personal use into a side project, and referred to their business as a \textit{hobby} that even their "\textit{ close family, like [their] my parents, don't know about}" (P7). P7 shared, 

\begin{quote}
    \textit{I don’t really consider myself a business person— and even my close family like my parents don’t know much about it. WhatsApp is just really convenient for me to handle these little orders and keep in touch with everyone without making a big fuss about it. Now I have more than 300 daily customers but its just my everyday thing, I would cook even if I did not sell food so its just like doing my regular thing.} 
\end{quote}

Some interviewees framed that their engagement with WhatsApp was informal. For example, P7, was adamant about labeling her activities as informal engagements rather than a business, often referring to them as hobbies or something "\textit{on the side}". Additionally, P12 explicitly stated that these pursuits were outside the construct of any type of "\textit{serious endeavour}" and P11 added, they did not require "\textit{professional expertise}" (P11). P4 exemplified this by saying that her WhatsApp business activities were personal ventures, explicitly denying its existence as a formal business. She explained, how her customers are "\textit{friends or friends of friends, and because they know [her] me personally, it almost feels like [she] I just cook(s) for [her] my friends}" while at the same time admitting that she "\textit{benefits financially, and helps support [her] my home}". This reflected how participants often described their WhatsApp business activities as informal engagements, where the personal nature of their interactions with customers made it difficult to view their work as a formal business. However, this view was dynamic, nuanced and also debated by some, as explained in the next section. 


\subsubsection{Labor, Legitimacy, and Recognition of Professionalism}


The prevailing narrative about WhatsApp had largely depicted it as a tool for personal communication, overshadowing its potential and significance in business contexts, as illustrated by the participants. However, P14, who owned a salon during the pandemic but had to shut down her business, was open about how her now WhatsApp business was valid and required significant planning, infrastructuring, and management. P14 detailed the effort involved in tasks such as "\textit{scheduling messages, sharing rates, and ensuring the right products or content is sent to the right people}." She explained, 

\begin{quote}
   \textit{You know, this business takes up so much of my time. I’m not sure if I want to keep doing it long-term because I almost make as much as my husband does [as a middle school teacher]. Sometimes I feel like I’m neglecting cooking and taking care of the house. People think because it’s a WhatsApp business that it’s just chatting and videos, but they don’t see the effort I put into building relationships with my customers. It’s only when the money comes in, like during wedding season, that people realize I’m actually running a serious business.}
\end{quote}

Here, the participant was concerned with the frustration of having a business that others dismissed as insignificant. She felt disheartened that its value was only recognized when it benefited the family. Additionally, she was unhappy that the affective and relational labor she invested in building connections and trust within the business was not acknowledged, making the business seem less legitimate than she felt it was.
 

However, another participant, P2, echoed a different view, suggesting that WhatsApp Business was relatively informal and this informality was what made her comfortable. She explained, \textit{“[WhatsApp] just has all these contacts, and I can add them to my business group to sell things I would usually sell”} and \textit{"It's nothing extra I do, but I can make some extra money, and it's helpful like that"}-- specifying how she herself did not see any of this as extranous labor, aligned with feminist interpretations of labor \cite{lokot2020unequal}.

These perspectives illustrate how both P2 and P14 defined and developed their understandings of \textit{labor}—in deciding on the types of labor that were considered legitimate and worthy of recognition in a business context. Despite the \textit{immaterial labors} P2 invested in, she felt that these efforts were not \textit{“serious enough”}, even though they contributed to material gains. In this way, there was a range of perspectives that defied simple binaries in how participants viewed their own businesses and their use of WhatsApp. P8 added complexity to it, as she navigated a middle ground between running a full-fledged business and pursuing a hobby project, reflecting a variety of personal approaches and comfort levels sharing- 

\begin{quote}
    \textit{I offer part-time home salon services, especially in high demand during wedding season. At other times, my earnings are minimal, so it’s not very formal. I just understand what people want, and since I used to do my friends' makeup, I thought, why not turn this into a service and charge a bit for it?}
\end{quote}


In conclusion, WhatsApp produced diverse experiences around labor, legitimacy, and professionalism, shaped by users' own statuses, situated ideas of labor, and their capabilities. These varying experiences also reflected how WhatsApp allowed users to navigate and negotiate labor and professionalism, informed by their personal circumstances and unique perspectives on what counted as legitimate and professional-- and this flexibility was useful to them. 

Next, I highlight how WhatsApp's evolution required users to adapt without explicit compensation or recognition as Meta engaged in coercive professionalization, during which this flexibility was lost. Users often navigated through workarounds; however, as further explained, these adjustments were frequently unacknowledged by the users themselves.



\subsection{Continual Infrastructuring as a way of Use}
 
In this section, I examine how users adapted to platform changes by learning through experience, often with the support of personal networks, and the time investment required in the process.

\subsubsection{Learning by Experience}

Despite the obstacles presented by frequent interface updates, users developed methods to adapt and integrate new features into their routines, treating each change as an opportunity to refine their use of the app. For instance, P10 exemplified,

\begin{quote}
\textit{Whenever I was chatting with my sister and noticed a new button or feature, I would try to use it right away. This way, when I needed to use it for my business, I already knew how it worked.}
\end{quote}

This underscored how users’ extensive personal use of WhatsApp allowed them to learn and adapt to its features beyond the context of business. Even those who might not have considered themselves technologically adept assumed WhatsApp to be "\textit{easier, with little need for external manuals}" (P5). P9 additionally explained- 

\begin{quote}
    \textit{You know I don't understand Facebook and Instagram when they change. But I think WhatsApp is just to message-- and even if there is more now, I try to always only stick to messaging. That way, if I know how to send one message I know how to use it well- and sometimes that may take me hours and hours of trying to understand the new change, but once I do, I am back at it.}
\end{quote}


Through this process, users leveraged their routine interactions with the app to develop and refine their skills. By experimenting with new features in personal messaging, users effectively navigated the learning curve in a low-risk environment before applying these skills in business settings. This method of experiential learning demonstrated that frequent engagement with the app, as a routine infrastructure, facilitated a deeper understanding of its features, despite ongoing updates and changes.


However, this notion of continual change was also approached with caution. Both P10 and P9 expressed concerns that if the app continued to evolve with \textit{too many features} (P9), it might expand beyond its primary function of messaging. Participants speculated about the potential challenges this \textit{complication} (P15) could pose and expressed uncertainty about how to manage such changes. For instance, P10 shared,

\begin{quote}
    \textit{I feel like WhatsApp is turning into something like Instagram, which my daughter only knows how to use. I need WhatsApp to stay simple so I can keep using it. If it adds too many features and becomes too complicated, I won’t see the point, and neither will my friends or customers.}
\end{quote}

In order to continue, as P4 shared- "\textit{figuring out WhatsApp, because I [she] will anyways use it for chatting but then when it becomes too complex and I [she] am [is] no longer understanding}", explaining if the complexity of the tool were to increase to a point that it becomes "\textit{unrecognizable and not just a text message app anymore}", then "\textit{[she] would not have the independence to continue doing this kind of business or even continue using it easily enough}". While learning through experience made WhatsApp a preferred tool, participants also noted its limitations and discussed its boundaries. They explained that relying on routine infrastructuring was effective only as long as those experiences did not become overly cumbersome or obstructive to new ways of adapting to the platform.




\subsubsection{Personalized Networked Learning}
%\subsubsection{Personalized Social Collaborative Learning}

Participants explained how they used their personal connections on WhatsApp to manage changes, even when they did not fully understand all of them. The core functionality of the tool inherently supported this form of learning by enabling users to apply the informal, personal communication aspects of the tool to their professional contexts, thereby facilitating their adaptation and learning.

Using her personal network, P3 shared her experience with infrastructuring only after confirming with her social connections. She explained how she had friends in her community with whom she was accustomed to \textit{discussing} issues, and how this process continued iteratively as new changes emerged - involving the same group of people to discuss and develop solutions.

%\begin{quote}
%\textit{When I started, I made all my groups after discussing with my building friends. And now, when new updates come, I again discuss with my building friends and then decide what to do.}
%\end{quote}

Consequently, others had other set people in their personal lives who also used WhatsApp similarly, working collaboratively through a trial and error method. P8 shared,

\begin{quote}
    \textit{I started using WhatsApp with help from my sister. We learned together how to post updates on my status and in groups. [...] Whenever either of us finds something useful, we share it with each other.}
\end{quote}

Finally, some participants reflected on their personal use of WhatsApp to explain how collaborative learning unfolded. As \textit{P13} shared, \textit{"It feels like there are always changes happening online. I usually reach out to my friends on WhatsApp, and we end up discussing things and figuring stuff out together, even if it’s by mistake."}

Here, the personal nature of WhatsApp played a pivotal role in enhancing the collaborative learning process. The platform's accessibility and informal communication style enabled participants to engage in spontaneous discussions, which, even if unintentional, led to shared problem-solving and learning. This ease of connection allowed for a flexible and dynamic approach to learning, where adjustments were made naturally in response to ongoing interactions. Thus, WhatsApp facilitated not only communication but also collaborative adjustment and discovery in an organic way.

However, not everyone had the ability to immediately access networks that could provide intellectual or practical support within their immediate surroundings and social circles. For instance, P6 who was frustrated with the interface changes, shared- "\textit{I know the inner features are the same [..] but sometimes the status icon moves from the top to the bottom and back again. It’s tough to keep up.}" 

Similarly, P1’s difficulty with disappearing interface elements underscored how even simple tasks become daunting when users can’t rely on a stable, predictable design:

\begin{quote} \textit{Sometimes, I’ll see the three dots at the top of the screen, and then they just disappear. A few days later, they’re back again. It’s frustrating because it makes simple tasks, like choosing who to message, much more complicated.} \end{quote}

Her reliance on her daughter for guidance further highlighted her difficulty accessing external support—\textit{‘My daughter understands how it works, but she's very busy and living abroad’}—pointing to the limitations of informal help and the isolation that can result when support isn't immediately available.

Overall, for P1, the workflow and changes were not intuitive. Although she tried to use her network, she was less fortunate than others because her contacts were not in close proximity. As a result, while personalized learning from social contacts was available, it was not as convenient or straightforward for her.


Overall, these accounts suggested that frustration stemmed not only from navigating changing interfaces but also from the ongoing reliance on external help and the limited availability of immediate, local support, highlighting the challenges users faced in adapting to both the evolving technology and the scarcity of accessible, reliable resources. However, among the available workarounds, infrastructuring through personal networks and informal communication was the most useful and sought-after method for learning. Participants used WhatsApp’s core messaging features to navigate time-consuming changes, with some benefiting from collaborative efforts within their social circles, while others faced challenges due to limited access to supportive networks. These experiences highlight how infrastructuring, shaped by personal and social contexts, was crucial in managing technological changes. The next section discusses how this process often became laborious and time-consuming.

\subsubsection{Time-Consuming Adaptation}

The process of adapting to WhatsApp’s frequent updates was both time-consuming and overwhelming. Infrastructuring, characterized by its continuous nature and constant need for learning, further compounded the time investment required at each step of the adaptation process. P7 shared:

\begin{quote}
\textit{It takes a lot of time to learn how to deal with WhatsApp’s changes- and I already tell my husband and son that this is something on the side and when I spend this much time, and I still end up not really doing as much business because I am still understanding the changes it becomes pointless!}
\end{quote}

This quote highlighted the considerable effort required to stay abreast of changes while managing the demands of  other responsibilities. P4 additionally operationalized a change and how it impacted her, 

\begin{quote}
    \textit{I keep wasting time trying to figure out these new changes. One day, I could send my product to all my groups, and then suddenly I could only send it to five. It took me a whole week to understand the new update, and by then, my clients had moved on to other stores or they wanted other things. A lot of them were confused about why I couldn’t get products to them properly or reply on time.}
\end{quote}

Both P7 and P4 interpret and perceive their work within their unique sociocultural contexts, and the ongoing process of infrastructuring consumed more time than they had available. Similarly, the ability to allocate time, or the lack thereof, further influenced what it meant to successfully appropriate the platform through infrastructuring. As another participant, P3, explained, she didn't mind \textit{discussing with her friend group} how the ways in which WhatsApp was changing could be "\textit{a hit or a miss for [her] business}". She elaborated that the changes were manageable for her, as she was a "\textit{recent retiree from a bank job}" and understood why companies needed to implement these updates but felt that they missed the point. She explained-


\begin{quote}
    \textit{I get that WhatsApp is trying to cut down on spam, but it doesn't really help small businesses like ours. We still get spam calls and video calls. We just share these numbers in our friends' group and block them ourselves.}
\end{quote}



Others, like P13, found the changes disruptive and demotivating. P13 explained that he \textit{randomly received video calls from unknown numbers}, and found it difficult to block them. He said, "\textit{All these new symbols and buttons are overwhelming, and now they're asking me if I want some automated process. No, I just want to block the spam!}", adding that "\textit{I don't have so much time to accost WhatsApp's changes and also keep up with all the nonsense}".

Likewise, both P2 and P4 were extremely dissatisfied as due to their personal contexts, their learning curves were quite steep. P4 mentioned, \textit{"You know, now that it’s so overwhelming, I can’t continue working as domestic help. I have to choose between quitting everything and focusing on this business or finding time to work out. It’s always about having to choose—whether to stay at home and do this or to work outside."} P2 shared a similar experience, explaining that "\textit{her ability to keep up with changes was interfering with her household responsibilities}" and her husband might "\textit{soon ask her to stop.}"


In conclusion, adapting to WhatsApp’s frequent updates highlighted the significant role of infrastructuring in users’ experiences. The continual need to manage these updates—a core aspect of infrastructuring—was both time-consuming and overwhelming for many. Participants such as P7 and P4 noted that keeping up with these changes often consumed more time than they had available, impacting their ability to manage their businesses and other responsibilities effectively. This situation revealed that the ability to cope with updates was influenced by life factors, with those having stable personal lives and strong social support handling updates better, while others facing personal challenges or lacking support experienced more difficulties. It is also of note, that while WhatsApp's informal nature initially facilitated a seamless transition from personal to professional use, it has evolved into a more structured and demanding platform. This evolution has led to a form of coercive professionalization, where users, particularly marginalized entrepreneurs, are compelled to adapt to constant and \textit{standard} platform changes which do not support their local needs. I discuss more about coercive professionalization in the discussions.

\subsection{Normalizing Invisibilized labors of Continual Infrastructuring}

In this section, I share how participants demonstrated a resigned acceptance of WhatsApp's continual changes, normalizing the process of learning and adapting as an inherent part of their daily lives. This acceptance extended beyond their businesses, reflecting a broader pattern where adjusting to challenges and navigating constant updates with technologies became a routine aspect of their labor, shaped by their broader experiences of marginalization and the necessity of making do with available resources.


\subsubsection{Continual Appropriation to Infrastructure through Changes}


As WhatsApp continually changed and redesigned its user interface and functionalities, the participants I interviewed frequently encountered issues and had to figure out workarounds. However, when asked about this ongoing challenge, they often maintained that they did not view it as extra work. They accepted these difficulties as normal, given their status as \textit{small players} who \textit{always had to find workarounds}, since the platform was not designed with their needs in mind. P10 explicated this-

\begin{quote}
    \textit{WhatsApp keeps changing so much—now you need verification, your number can get blocked, and there are communities instead of just groups-- its all so professional [..] I’m not very tech-savvy, so all these updates are really tough for me. I don’t want to complain too much because I’m just glad to still have WhatsApp. [..] I can’t switch to other platforms; I wouldn’t understand them, and neither would my customers.}
\end{quote}

The concern extended beyond personal inconvenience, presenting the participant with issues related to customer retention. If WhatsApp became unusable or if users were required to switch platforms, they faced the dual challenge of adapting to a new, unfamiliar system while also managing the risk of losing their existing customer base. Users felt \textit{trapped in this} (P14) situation where they had to endure ongoing changes to maintain their business connections, as any switch could disrupt their operations and alienate their customers. Likewise, others such as P3 and P8 shared their sentiment of being extremely careful in terms of their expectations, with P8 sharing- "\textit{I try not to anticipate too much because the platform is so unpredictable now. It feels like every time I get used to something, it’s replaced with something new}". P3 shared this opinion, saying- \textit{I don't try anything new, I only use what I need}- particularly due to prior experiences of attempting to set up an infrastructure on something new, only to have that made paid and \textit{too expensive to use}. However, both P3 and P8 also agreed with the others, in finishing \textit{But you know what? At least WhatsApp is still here. I can adjust my strategy as needed, but at the end of the day, it is WhatsApp. Even if they charge me a bit, I know I won't find anything else like it.} (P8). P3 added to this-

\begin{quote}
    \textit{My husband and daughter don’t have time to help me. It’s sad if WhatsApp increases its price or asks me to make websites or something, but I’ll have to pay it. If I can’t, I’ll have to stop my business because I can’t do it anywhere else.}
\end{quote}


As an example of the challenges posed by platform complexity, P1 noted that WhatsApp's increasing sophistication could make businesses appear more legitimate, but due to her sociocultural constraints, it might force her to shut down her business.

\begin{quote}
    \textit{If the platform became more complex and made my business look more legitimate with verification and everything, it might mean letting everyone in my extended family know about it. This visibility could lead to me having to close the business, as my family might not support it.}
\end{quote}

To manage these updates, P1 tried to keep her business operations unchanged unless WhatsApp made automatic adjustments. However, she often found that any action she took seemed to trigger further changes, as she described: \textit{"It’s like anything I touch, some change takes place."} To cope with this, she often shut down her business for long periods of time. 

\subsubsection{Resigned Acceptance of Normalized Labor}


More often than not, unnoticed effort highlighted the constant work needed to incorporate new features into daily business practices-- these were often dismissed by the participants. Instead, they normalized the extraneous labor of constant infrastructuring as something implicit and expected of them, as P9 shared- "\textit{Why will companies or anyone make anything for us? We are poor people}". Like P9, many participants, who faced marginalization through various personal identities, viewed WhatsApp’s availability as a significant benefit that they were undeserving of. P11, who described her WhatsApp business use as a way to \textit{make a little more money so [they] can eat outside once in a while}, further shared "\textit{But now; Business on WhatsApp is tough now—but am I really a businesswoman?}".

This perspective revealed that business owners were generally satisfied with having access to WhatsApp, accepting its evolving features without scrutinizing how their use was monetized or exploited. For example, P7 explained,


\begin{quote}
\textit{I've been using WhatsApp for my business for many years now. Even though WhatsApp keeps changing and updating, I'm just glad to have it. I manage with whatever new features come along and make it work for me. For a small home business, I don't need all the fancy-professional tools that big companies might use. I take things as they come and handle any changes because, honestly, having WhatsApp is a huge help. I realize that some of the new tools might be designed more for bigger companies, that need formal presence but I make do with what I have.}
\end{quote}


Here, the participant thought that she lacked agency to assert that the platform was designed with her needs in mind. Instead, she referred to \textit{big companies}, effectively downplaying her own needs as a businesswoman and questioning her own entitlement to the platform's features-- another description of the professionalized platform that she was coerced to adapt to. Inspite of the experienced marginalization, they remained appreciative. P12, whose son resided in the United States, shared her thoughts on this situation, stating:

\begin{quote}
    \textit{My son in the US tells me that WhatsApp Business is going to be a big deal, but I’m not so sure. I’ve been using WhatsApp for Business for the past nine years and never thought much of it. Now that it’s more popular, suddenly my account gets blocked, my numbers are reaching too many people, and many of the new features seem to come with a cost. I don’t have the money for that, but I guess I’m just older now, so I’ll just accept whatever comes my way}
\end{quote}

Here, the normalized acceptance of \textit{whatever comes} reflects a sentiment where participants felt they were undeserving of WhatsApp amid its changes and developments. This idea was intertwined with their identities, where they were used to experiencing marginalization, as P5 explained, 

\begin{quote}
    \textit{I feel like I’m not really cut out for doing business, and thats what my son and husband keep saying [..] All this complex technology confuses me—I only understand and use WhatsApp. No matter what happens, I’m just so happy it’s here [..] I’ll be grateful for whatever I’ve had}
\end{quote}

They incorporated a sense of undeservedness into their understanding of the platform's continual modifications, which diminished self-confidence, while rationalizing that enduring these challenges was preferable to seeking alternatives due to gratitude for the service's availability. This prevented them from critically evaluating or negotiating the platform's conditions, especially since Meta obscured how it benefited from the arrangement and appropriated the infrastructure users built. As a result, users faced the pressures of adapting to constant updates and managing potential cost increases, while understanding that without WhatsApp, their businesses would face significant operational difficulties. 














