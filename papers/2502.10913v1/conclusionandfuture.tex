\section{Conclusion}

I interviewed 14 WhatsApp users from India who used WhatsApp to manage their home-based small businesses. My research explored how they 1) appropriated WhatsApp for business purposes, 2) adapted to changes in WhatsApp Business, and 3) perceived these changes. I found that users frequently leveraged their personal contacts and connections for business, translating their personal use of WhatsApp into professional contexts. Additionally, they struggled to keep up with rapid changes and often felt that such platforms were not designed with their needs in mind. Many participants developed workarounds for changes that did not suit them. Lastly, I observed a diminished sense of self-esteem related to the ongoing processes of appropriation and infrastructuring. I contributed to HCI research in several ways. Firstly, I introduced the concept of \textit{coercive professionalization}, which describes how Meta has sought to co-opt infrastructural platforms initially developed through local community appropriation. This process represents a form of reverse appropriation, where these platforms are reclaimed by big tech companies. I also explored the features of coercive professionalization in how they relate to coloniality and marginalization. Finally, I discussed designs that can support the legitimization of local customization and appropriation of systems.




















