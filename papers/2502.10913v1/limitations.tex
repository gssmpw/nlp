\section{Limitations and Future Work}

%This study is not without limitations. As with other qualitative research, it has a small sample size but does not aim to generalize findings \cite{eisenhart2009generalization} but rather to elucidate the distinctive practices of platform users and how these practices contribute to the infrastructural development of profit-oriented companies \cite{au2019thinking}.

This study has limitations, including a small sample size, which is common in qualitative HCI research, especially in critical cultural contexts \cite{eisenhart2009generalization}. Consequently, it does not aim to generalize findings but rather to elucidate the unique practices of platform users and their role in the infrastructural development of profit-oriented companies \cite{au2019thinking}. Moreover, there is a potential skew towards women in the demographics because, in many Global South cultures, women are more likely to use messaging apps extensively, while men often engage in traditional jobs. This pattern can affect the data, as it reflects broader cultural and economic roles rather than indicating a limitation of the study itself \cite{kantor2002sectoral}. Additionally, the interviews were conducted while I was situated in a different country from the participants \cite{Tuyisenge2023, Tarrant2013}. Despite sharing similar cultural backgrounds and experiences, this geographical separation may have affected the establishment of trust and ongoing communication, potentially influencing the participants' willingness to fully disclose information \cite{mullings1999insider}. However, the recruitment process involved multiple rounds of purposeful sampling, with many participants being recruited through personal connections and referrals. This method is noted to bring a sense of \textit{word-of-mouth} trust, which likely encouraged more open and honest discussions \cite{friedman2015qualitative}. 

Future work should focus on two key studies: First, a digital ethnographic study within WhatsApp groups of marginalized sellers to explore how they appropriate the platform for business communication, identifying infrastructural needs and challenges from within-- that could supplement results in this work to explicate actual features that may be useful for the community. Second, a feasibility analysis of WhatsApp and similar messaging applications to assess their accessibility, usability, and potential for adaptation, with the goal of refining these platforms to be user-friendly and effective for individuals with limited technical literacy. Each of these studies could also formulate strategies to support the appropriation of digital tools for innovative local uses, potentially exploring ways in which this appropriation can be legitimized.












%Through usability testing, including task-based assessments and surveys, it could also evaluate how well these platforms meet the needs of less tech-savvy users, offering recommendations to improve their design and ensure they are both user-friendly and functional for marginalized groups.





















