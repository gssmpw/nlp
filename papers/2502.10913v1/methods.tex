\section{Methods}

I used a combination of purposeful and snowball sampling methods to recruit and interview 14 Indian WhatsApp users who use the platform to run their businesses.Consequently, I performed an inductive-- iterative thematic analysis of the data to address the research questions. The study and methods were approved by the review board. The specifics of this process are outlined below.


\subsection{Data Collection}

I used several different channels to recruit participants for my study, as they were part of a very informal network of workers, and accessing them through more formal communication methods was challenging \cite{agarwala2009economic, 10.1145/3025453.3025643, 10.1145/3290605.3300563}. Therefore, I employed purposeful \cite{suri2011purposeful} and snowball sampling \cite{parker2019snowball} across four platforms: WhatsApp, Facebook, Instagram, and Quora. The recruitment messages varied depending on the context of each platform. On WhatsApp, I started with personal contacts (n=12) who were involved in business and whom I had previously interacted with as a customer. I asked them to refer individuals who might be interested in my study and to contact me via my personal WhatsApp, ensuring a degree of separation between myself and the participants. A total of 16 people were referred, among which 7 responded when I messaged them and agreed to do the interview. For Facebook, I joined three public groups related to WhatsApp Business in India, identified through keywords such as "WhatsApp Business in India," "Indian Businesses using WhatsApp," and "WhatsApp for Business in India." I engaged with posts advertising WhatsApp businesses and used links provided to contact WhatsApp group admins (n=42) among which 3 responded, and agreed to interview for the study. On Instagram, I targeted accounts that promoted local Indian businesses and reached out via direct message to the admins (n=17) of businesses that had a WhatsApp option listed on their profiles, and 2 of the admins responded, and interviewed with me using this method. Lastly, on Quora, I posted a recruitment message in three channels related to WhatsApp Business in India and used similar keywords to those used on Facebook to invite interested individuals to contact me via email, and 2 participants were interviewed via this method. As potential participants contacted me via WhatsApp or email, I screened them to ensure their home-based businesses met the legal turnover range and local criteria for small businesses in India (non--public company with share capital below Rs. 40 million.) \cite{govindia2023}. Recruitment ceased when data saturation was reached, meaning no new insights emerged from interviews \cite{Charmaz2012}. This was confirmed through corresponding preliminary analysis to ensure all research questions were thoroughly addressed. 


 \begin{table*}[t]
\footnotesize
  \caption{Self-Reported Participant Demographics, where an asterisk (*) indicates that the business was operated on a part-time basis, while the absence of an asterisk signifies full-time involvement. Despite WhatsApp's payment options, all participants used Unified Payments Interface (UPI), Google Pay, or cash.}
  \label{tab:freq}
\resizebox{\textwidth}{!}{%
\begin{tabular}{p{1cm} p{1cm} p{1cm} p{1cm} p{2cm} p{1.7cm} p{2.7cm}}  
%{cccccccl}
   \toprule    ID&Age&Gender&Education&Monthly Income& Approximate number of Customers&Type of Business\\
    \midrule
    
    P1 & 32& Female& High School & Rs. 5000-8000 & 500&Local handmade sarees and kurtas (Reselling) \\
    P2 & 28& Female& College diploma & Rs. 1500- 1200 & 230&Food Business* \\
    P3 & 31& Female& High School & Rs. 15000- 20000 & 2000& Handmade dress materials*\\
    P4 & 43& Female& High School & Rs. 8000-10000 & 330& Food Business*\\
    P5 & 48& Female& High School & Rs. 9000-12000 & 5000& Beauty Services*\\ 
    P6 & 51& Male& College Diploma & Rs. 9000-12000 & 8000&Reselling dress materials and sarees \\
    P7 & 33& Female& College Diploma & Rs. 800-1200 & 700&Food Business*\\
    P8 & 31& Female& College Diploma & Rs. 1000-3000 & 120&Salon services* \\
    P9 & 37& Female& College Diploma & Rs. 2000-4000 & 100& Handmade Artwork\\
    P10 & 35& Female& College Diploma & Rs. 3000-5000 & 170& Handmade Artwork*\\
    P11 & 39& Female& High School & Rs. 500-1000 & 40&Food Business*\\
    P12 & 48& Female& High School & Rs. 1000-1500 & 300& Reselling locally manufactured sarees\\
    P13 & 52& Male& High School & Rs. 8000-10000 & 200&Reselling \\
    P14 & 54& Female& High School & Rs. 10000-15000 & 6000&Makeup and Salon \\
\bottomrule
\end{tabular}%
}
\end{table*}



\subsection{Participants and Interviews}

I conducted semi-structured interviews with 14 participants in India who had spent between two and nine years running home-based small businesses on WhatsApp. Five participants were involved in the clothing business, purchasing garments from artisans or manufacturers to sell via WhatsApp. Four participants sold food through WhatsApp for pickup or delivery. Three worked in the beauty industry, offering home services or selling courses and products. Finally, two ran creative businesses, making and selling mud, clay, or painted artifacts. Participants had varied amounts of digital skills, and in particular, this manifested throughout the interviews, with many coming from cultures where they did not receive much support or access to digital resources. For some, the lack of formal training or mentorship in digital technologies meant they had to rely heavily on self-learning, often navigating challenges alone. Others highlighted the impact of cultural norms that did not prioritize or encourage digital literacy, which further shaped their engagement with technology. These differences in background and support systems contributed to distinct approaches and levels of comfort when using digital tools, with some participants feeling more confident and others struggling to keep up in an increasingly digital world. Most (n=13) started by using WhatsApp’s personal app and either continued using it for business or used both the personal and business versions of the app. Their reported customer numbers ranged from 40 to 8,000. Additional participant demographics were collected during the interviews and are listed in table \ref{tab:freq}.

The interview protocol was semi-structured and participant-centered, designed to foster open dialogue and make participants feel comfortable. It began with general questions about their work, businesses, and use of WhatsApp, allowing for a broad understanding of their context. To address the first research question, the protocol then explored the transition from WhatsApp as a personal communication tool to its use in business, focusing on how participants integrate the platform into their professional lives. Questions probed the technical aspects of using WhatsApp in business, including its benefits, challenges, and role in supporting business activities, aiming to understand its evolving use in both personal and professional spheres. For instance, participants were asked: "How did you first start using WhatsApp for business purposes?" and "What features of WhatsApp do you find most useful in your business?" The interview also focused on distinguishing between WhatsApp (the personal app) and WhatsApp Business, with questions like "What made you choose to use WhatsApp Business instead of the regular WhatsApp app?" or "What are the main differences you’ve noticed between WhatsApp and WhatsApp Business in your day-to-day operations?" This allowed for insights into participants' preferences, and challenges in using either application. Following this, the interview shifted to understanding the process of adaptation, with questions aimed at uncovering how recent and anticipated changes in WhatsApp usage were perceived. For example, interviewees were asked, "Have you noticed any significant changes in how you use WhatsApp for business in the past year?" and "What changes do you foresee in the future, and how do you feel about them?" In many cases, participants discussed these shifts spontaneously, providing rich, unprompted insights into their experiences and expectations. As the sole author, I also memoed my experiences during and immediately  after each interview, which served as a way to guide the analysis, and motivate emerging themes \cite{10.1145/3359174, 10.1145/3491101.3516392}. The semi-structured interview guide is provided as an artifact, though it is brief and broad, as is typical in interviews where questions evolve and depart from the protocol as the conversation unfolds \cite{roulston2018qualitative}.




All interviews were conducted via WhatsApp calls, and Zoom was utilized to record them using a loudspeaker. Additionally, each interview began by informing the participant of their rights and obtaining their consent to be recorded. The interviews, conducted in Hindi, English, or Bengali, lasted between 15 and 120 minutes. One interview lasted only 15 minutes due to the participant's limited access to Wi-Fi and inability to reschedule because of personal issues. The average interview duration was 37 minutes, with a standard deviation of 27.47 minutes and a median duration of 40 minutes. I translated and transcribed the interviews directly into English for analysis. 

\subsection{Data Analysis and Positionality}

As data were being collected, I simultaneously analyzed the existing transcripts \cite{smith2024qualitative}. This approach, inspired by grounded theory \cite{charmaz2015grounded}, allowed me to refine and adjust the interview protocol based on preliminary findings and to enhance the ongoing analysis. Although inspired by grounded theory, I did not strictly adhere to this methodology. Instead, I incorporated select elements of grounded theory to guide the development of codes and facilitate the thematic analysis \cite{braun2012thematic}.

Analysis began with a thorough review of the transcripts, during which open codes \cite{strauss2004open} were assigned at a descriptive level (for eg., communication block, technical difficulty, business tactics) \cite{terry2017thematic}. After approximately three rounds of open coding, a more focused coding \cite{thornberg2014grounded} approach was employed to identify relevant themes (for eg., infrastructuring as ongoing labor ) that addressed the research questions more directly. The reported findings show an even higher level of analysis, as more codes were conflated and corresponding themes for the research questions emerged. This research adopts an interpretivist approach \cite{10.1145/3633200}, wherein themes and codes are subject to interpretation \cite{10.1145/3359174}. As such, there is no singular, fixed understanding of these themes and codes. This perspective challenges the notion of objective meaning, emphasizing that my subjectivity, positionality, and interpretive stance contribute to deriving meaning from the data \cite{10.1145/3359174}. To motivate this interpretive analysis, memos were triangulated with the transcripts to produce the insights and findings presented in the following sections \cite{10.1145/3359174}. The final codebook is shared as an artifact. 



As a researcher born and raised in India with personal connections to individuals who use WhatsApp for business, my understanding of their socioeconomic backgrounds, views, beliefs, and motivations is influenced by my own experiences and relationships. In this research, I made a deliberate effort to identify and address my own subjectivity during the analysis of data by being explicit in allowing the participant to guide the conversations, through semi-structured interviews \cite{berger2015now, doi:10.1177/1468794112439005}. I also took into account the power dynamics when interviewing individuals who might be wary of my current affiliation with a university in the United States. To address these concerns, I established at the outset that the interviews would be participant-led and semi-structured, drawing on feminist research methodologies that emphasize the empowerment and agency of participants. This approach has been used to engage traditionally marginalized users \cite{doi:10.1177/1468794112439005}. While I had key questions aligned with the research objectives, I allowed participants to guide the conversation and raise emerging concerns. This not only gave them more autonomy but also facilitated the collection of more authentic information, encouraging participants to express their perspectives and experiences more freely \cite{doi:10.1177/1468794112439005}. Acknowledging the complex nature of the \textit{"at home"} versus \textit{"not at home"} discourse, I focused on identifying commonalities and embraced the absence of such binaries as a way to foster honesty and genuine respect \cite{doi:10.1177/1468794114550440}. This approach was crucial, given that there were many aspects I did not know or experience because of my different affiliation. Thus, while my relative positionality influenced the analysis, I employed rigorous inductive coding to ensure that the data itself guided the findings.



































