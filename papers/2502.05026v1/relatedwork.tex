\section{Related Work brainstorming}
\label{sec:relwork}
%%%%%%%%%%%%%%%%%%%%%%%%%%%%%%%%%%%%%%%%%%%%%%%%%%%%%%%%%%%%%%%%%%%%%%


% Nature Climate Change papers:

% Opinion: "How scientists are making the most of Reddit" https://www.nature.com/articles/d41586-024-00906-y

% "Growing polarization around climate change on social media"
% https://www.nature.com/articles/s41558-022-01527-x

% Find more papers: https://www.nature.com/search?q=social+media&journal=nclimate&article_type=research&order=relevance




\red{ADD RELATED  WORKS ABOUT:}

\begin{itemize}



\item \textbf{Discussions about climate change on social media, and news}

- in news media, datasets, and event-related studies, about disasters\\

SOCIAL MEDIA IN GENERAL:\\

• \cite{mavrodieva2019role} title: "Role of Social Media as a Soft Power Tool in Raising Public Awareness and Engagement in Addressing Climate Change". It studies the bottom-up initiatives around the world demonstrating how they, especially when they use social media (in general) as a “Soft Power Tool”, can affect policy and decision-making. "The present paper aims to (1) define the forms and characteristics of social media as a soft power method, (2) analyze its influence on the awareness of societies, and (3) assess if increased public awareness could influence the official political and policy processes. [...] The study finds that even though it is difficult to assess the effects of social media as a soft power tool with certainty, there are visible links between social media and changing public perceptions, with the possibility of public opinion influencing political decision-making."\\

TWITTER:\\


• \cite{upadhyaya2023multi} They use stance detection + sentiment analysis to study the Twitter debate around climate change, which is polarized into denials and believers. "In this paper, we propose a framework that helps identify denier statements on Twitter and thus classifies the stance of the tweet into one of the two attitudes towards climate change (denier/believer) [...] Therefore, our work focuses on learning two closely related tasks: Stance Detection and Sentiment Analysis of climate change tweets." \\

• \cite{williams2015network} title "Network analysis reveals open forums and echo chambers in social media discussions of climate change". They use Twitter and find "homophily and segregation into polarised “sceptic” and “activist” groups. Most users interact only with like-minded others, in communities dominated by a single view. However, we also find mixed-attitude communities in which sceptics and activists frequently interact."\\

• \cite{effrosynidis2022climate} title: The climate change Twitter dataset\\

• \cite{karimiziarani2023toward} title "Toward reduction of detrimental effects of hurricanes using a social media data analytic Approach: How climate change is perceived?". "This research highlights a thorough examination of the textual content of users’ posts shared on Twitter across the 48 contiguous U.S. states (CONUS) during hurricanes Harvey (2017) and Dorian (2019). " they use Sentiment analysis, topic modeling, and topic classification to study the social response to hurricanes. \\

• \cite{walter2019scientific} title: Scientific networks on Twitter: Analyzing scientists’ interactions in the climate change debate


- Contribution: we study this topic on one of the largest social media: Reddit:

\item \textbf{Discussions about climate change on Reddit}

- list the papers:\\
REDDIT IN GENERAL:\\
• Some recent works have focused on Reddit as a useful data source for computer social science studies, especially during prominent events, as shown in \cite{proferes2021studying}. For example, \cite{desiderio2023recurring} analyzes this kind of interaction by focusing on the temporal and semantic dimensions of the conversations in four subreddits about politics and sports.\\ 


• \cite{de2021no} title: No echo in the chambers of political interactions on Reddit

REDDIT \& CLIMATGE CHANGE:\\
 %less relevant maybe:
• \cite{du2020public} title: "Public opinion matters: Mining social media text for environmental management" Collect different papers about:
sentiment analysis, air/water quality vs social media, events vs Social media, Urban policy/local laws vs social media, Traffic and Mobility Issues, Energy and Resource Conservation, and Disaster and Resilience.  


%these
• \cite{treen2022discussion} title: "Discussion of climate change on Reddit: Polarized discourse or deliberative debate?". examines climate change discourse (selected via a keywords approach) from April to June 2017, searching for polarization. It studies the reply network in which nodes are the users and edges represent comments of one user to another. Then, it focuses on the six bigger communities inside the GCC of this graph, analyzing their dominant topics, with their main sources of information and ideological bias. It does not find traces of polarization in the discourse. \\

• \cite{gadanidis2020discourse} title: "The discourse of climate change on Reddit". applies LDA to study how five selected subreddits (/r/climate, /r/climateskeptics, /r/collapse, /r/climate\_science, and /r/ClimateOffensive) discuss climate change, finding peculiar frames in the discussion that may differ from the mass media and institutional ones. The RQs are "How is climate change constructed and framed by commenters on public forums? Which frames are dominant in each community, and how can we explain differences in which frames are most commonly used in different communities? Are these findings consistent with mass media discourse on climate change, and if not, how do they differ, and how can these differences be explained?" and they perform a critical discourse analysis (CDA) on the text\\

• \cite{oswald2022climate} title: "How do climate change skeptics engage with opposing views online? Evidence from a major climate change skeptic forum on Reddit".  focus on the skeptical side of the debate, studying a sample of posts and comments from the subreddit "r/climateskeptics" and trying to understand how skeptic users interact with opposing views (i.e. with climate believers). They manually label 3000 posts as “consonant” or “dissonant” with respect to the "default leaning" of this subreddit.  \\

• \cite{villanueva2021climate} title: "Climate Change Frames and Emotional Responses on Reddit". presents a temporal and subreddit study about the different frames and emotions in a period surrounding the 2019 UN Climate Action Summit. It focuses on six subreddits:  2 believers, 2 neutral, and 2 skeptics. “The /r/climateskeptics and /r/The\_Donald communities were chosen to represent the climate skeptic groups that are in direct opposition to the climate change believer communities.”\\

• \cite{parsa2022analyzing}  title: "Analyzing climate change discussions on reddit". has a sophisticated keyword approach, it focuses mostly on 5 subreddits (r/climate, r/environment, r/climatechange, r/climateskeptics, and r/climateOffensive). After a manual inspection of the top engaging posts, it makes a topic modeling and a URL study.\\

• \cite{shah2021climate} title:"Climate denial fuels climate change discussions more than local climate-related disasters". examines the occurrence of Reddit discussions around three types of climate-related events: natural disasters, political events, and policy events. They focus on six USA cities subreddits and select posts/comments matching with a detailed list of keywords. "The findings suggest that collective discussions on climate change are driven more by political leaders’ controversial positions than life-threatening local natural disasters themselves."\\

• \cite{cagle2019shades} title: "Shades of denialism: Discovering possibilities for a more nuanced deliberation about climate change in online discussion forums".  It studies the different shadows of the debate about climate change (with its rhetorics) in the subreddit “r/Changemyview”. It taxonomizes the data according to the rhetorical style of the text.\\




- Contribution: we consider the largest span of time and also of subreddits for this topic.\\
Generally, all these works focus on a few subreddits in a small period: to the best of our knowledge, this is the first study to cover 14 full years, from 01-2009 to 12-2022, considering, throughout Reddit, all the posts and comments involved in the climate change discussion. \\



\item \textbf{Citation of scientific literature on social media}\\

• \cite{gruppi2023scilander} There is a lot of misinformation about science: to address this issue the authors focus on 1M news articles about COVID-19 from 500 sources and introduce "SciLander, a method for learning representations of news sources reporting on science-based topics". This tool combines four features in a unified framework to train unsupervised news source embeddings with a triplet margin loss objective. The features are Copying of news stories between sources, Semantic shifts of terms, Jargon, Stance toward the citation.


- citation anywhere on social media

• \cite{hwang2023information} considers the most-mentioned papers on Blogs, Facebook, News, Twitter, and Wikipedia. It defines a basic “information retention score” from the TextRank algorithm (i.e. variation of the PageRank for sentences) to quantify the quality of the mention considers only the bursts of attention related to each paper, of all the platforms, and merges them together to obtain a temporal sequence. \\

• \cite{cao2023scientific} title "How do scientific papers from different journal tiers gain attention on social media?". It explores the diffusion patterns of ∼170,000 papers on Twitter using network analysis and time series analysis to characterize papers’ dynamic diffusion process. "Results show that though elite journal papers spread significantly faster and more broadly than non-elite, some non-elite journal papers reached a sizable audience."\\

\textcolor{blue}{scientific communication}

• \cite{peters2013gap} title: "Gap between science and media revisited: Scientists as public communicators". 
"Most scientists assume a two-arena model with a gap between the arenas of internal scientific and public communication. They want to meet the public in the public arena, not in the arena of internal scientific communication."

• \cite{lewandowsky2019science} title: "Science by social media: Attitudes towards climate change are mediated by perceived social consensus". "This article examines how blog posts and blog comments interact to affect readers’ attitudes and beliefs concerning" climate change. It studies the consensus of the public, in particular, how “the interaction between the content of blog posts and blog comments affects readers’ attitudes [...] The authors “find that beliefs are partially shaped by readers’ perception of how widely an opinion expressed in a blog post appears to be shared by other readers. The perceived social consensus among readers, in turn, is determined by whether blog comments endorse or reject the contents of a post. When comments reject the content, perceived reader consensus is lower than when comments endorse the content.” 

• \cite{walter2019scientific} title: "Scientific networks on Twitter: Analyzing scientists' interactions in the climate change debate". It shows “that scientists interact most intensively with their peers, but also communication beyond the scientific community is important. The findings suggest that scientists adjust their communication style to their audience: They use more neutral language when communicating with other scientists, and more words expressing negative emotions when communicating with journalists, civil society, and politicians.”
results: scientists are at the margin of the network, and the low in-degree shows that they receive little attention from others. Politicians are the opposite, and journalists are located somewhere between the two groups.

• \cite{li2021social} title: "Social media communication of the scientific and technological literature in emergency under COVID-19" They use AltMetrics which “reflects the social influence and academic influence of academic papers through online social network platforms such as mentioning, sharing and collecting academic papers.” They do several analyses and statistics from these data, comparing different media.

• \cite{nane2021role} They combine scientific information from journal articles and preprints with their appearance in popular media, including social media. They focus on Spain, South Africa, and the Netherlands and study the social discussion around the scientific topic of COVID-19 and the responses of the policy-makers regarding the schools’ questions.






- on Reddit specifically 
focus  on a few particular subreddits: 

• \cite{moriarty2024reddit} title: "Reddit and Engaged Science Communication Online: An Examination of Reddit's R/Science Ask-Me-Anythings and Science Discussion Series" I CAN'T OPEN THE PAPER from the abstract "To understand how scientists can successfully engage with audiences in dynamic online environments, we examine Reddit’s science subreddit, attending to the acclaimed Ask-Me-Anything (AMA) series, and subsequent Science Discussion Series (SDS)."

• \cite{hara2019emerging} title "An emerging form of public engagement with science: Ask Me Anything (AMA) sessions on Reddit r/science"  "The current study examines scientists’ perspectives on the interactions between laypeople and scientists by asking questions and sharing information on social media platforms, specifically, through Ask Me Anything (AMA) sessions on Reddit’s “Science” subreddit (r/science). We had three main questions: (1) who are the participant scientists hosting r/science AMAs, (2) what are their experiences like as hosts, and (3) what type of discussions do they have on this platform? Survey results suggested that these scientists recognize the promising interactive nature of Reddit and are interested in continuing to use this platform as a tool for public engagement. Survey respondents generally had positive experiences as AMA hosts, but further research is needed to examine negative experiences. Overall, this study has significant implications for how scientists can engage public audiences online and more effectively communicate scientific findings to the general populace." 



- Differences in scientific citations by political parties or opinion stances (vaccination?)

- Contribution: we are the first to systematically study the citation of scientific literature on Reddit (about climate change). The tools that we develop can be applied to the study of any topical domain.\\

\item \textbf{misinformation on social media}\

• \cite{west2021misinformation} title: "Misinformation in and about science". IN THE INTRO THEY CITE A LOT OF PAPERS ABOUT MISINFORMATION ON SOCIAL MEDIA (then they cite also exaggerations of some papers' abstracts or wrongly reporting of preprints or preliminary studies, “quotation errors” (i.e., citations that are used to justify claims that are not factually supported by the cited documents), Fake News and Predatory Publishing, Filter Bubbles and Echo Chambers, Data and Science Distortion, ...),  "Nearly 40\% of Americans viewed content from untrustworthy websites during the 2016 US election, but these articles only represented about 6\% of all news articles consumed". 


\item \textbf{Polarization / Siloing}

- start with general

• \cite{antonio2011unbearable} title: "The unbearable lightness of politics: Climate change denial and political polarization"

it contains a general discourse about democrats, republicans, and neo-liberals (who are mostly republicans). The latter "promote free-market ideology. They built a network of think tanks that called for deregulation, privatization, welfare cuts, and reduced taxation to revive high corporate profits and economic growth [...]Thus, anti-environmentalism has been, from the start, a keystone of neoliberal antiregulatory politics. But the perceived threats posed by climate change discourse intensified this opposition, mobilizing energy companies and other related industries and broader free-market forces. Discrediting global climate change claims began in earnest in 1989 when the Marshall Institute issued their first report disputing climate
science. Although climate change denial is a latecomer to neoliberal anti-environmentalism, it has now become the countermovement’s pivotal issue in battles against environmental regulations. [...] Pew Research Center Surveys indicate that the partisan divide over warming accelerated between 2007 and 2009". 

• \cite{fisher2013does} title: "Where does political polarization come from? Locating polarization within the US climate change debate" 
it mentions the conservative think tanks that attack climate change scientific studies. "this paper analyzes the content of
Congressional debates around climate change [from 2005 to 2009] to understand the polarization of this issue more fully, specifically looking at where the polarization is coming from: the science of the issue or the threat of legislation." 
"In the U.S., this polarization is focused around, not the science of the issue, but perspectives on the policy instrument and the economic implications of the implementation of such an instrument."



• \cite{de2021no} title: No echo in the chambers of political interactions on Reddit



- specifically polarization around climate change

• \cite{falkenberg2022growing} title "Growing polarization around climate change on social media" 
• \cite{farrell2016corporate} title: "Corporate funding and ideological polarization about climate change"
• \cite{smith2024polarisation} from Nature. title: "Polarisation of Climate and Environmental Attitudes in the United States, 1973-2022" They found "two distinct historical patterns of asymmetric polarisation within environmental and climate change attitudes: first, with Republicans becoming less pro-environmental, beginning in the early 1990s, and second, a more recent greening of Democratic environmental attitudes since the mid-2010s. "




MAYBE WE CAN CONSIDER NEWER PEW'S STUDIES:
• AUGUST 9, 2023. Title: "What the data says about Americans’ views of climate change" \url{https://www.pewresearch.org/short-reads/2023/08/09/what-the-data-says-about-americans-views-of-climate-change/} 


- Contribution: we study the siloing of scientific discussion and citation

• silos on Reddit? About climate change or politics?

\end{itemize}









\subsection{Reddit}
%   HOW IT WORKS. THE INITIAL DATASET WE HAVE
Reddit has a quite different architecture from the other popular social media platforms. Users do not have a \emph{profile} with their personal information in which they can share or write posts: they interact in topic-based communities called \emph{subreddits}, which create a huge ecosystem where almost every possible topic can be discussed. This is the reason Reddit is also known as "the front page of the Internet". After choosing a subreddit, a user can write a new  \emph{post}, also called \emph{submission}, or add an answer to another post, with a new \emph{comment}. \\

\textcolor{blue}{from \cite{villanueva2021climate} pag 8 
"With over 430 million active users and billions of page-views a month, Reddit is a fruitful source of data for researchers (Amaya et al., 2019). Around 22\% of American adults aged 18-29 and 14\% of adults aged 30-49 use Reddit (Tankovska, 2021), and 70\% of those users use it as their primary source of news (Balther et al., 2016)." etc}\\


We start the study from two sets of monthly datasets, the first containing the submissions and the second the comments, written on Reddit. \textcolor{blue}{collected with the  \emph{pushshift.io} Reddit API\footnote{\url{https://github.com/pushshift/api}} \cite{baumgartner2020pushshift}}  


%%%%%%%%%%%%%%%%%%%%%%%%%%%%%%%%%%%%%%%%%%%%%%%%%