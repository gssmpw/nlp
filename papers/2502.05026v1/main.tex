\documentclass[manuscript,nonacm,screen]{acmart} % 
%\documentclass[sigconf,anonymous]{acmart}

%
\setlength\unitlength{1mm}
\newcommand{\twodots}{\mathinner {\ldotp \ldotp}}
% bb font symbols
\newcommand{\Rho}{\mathrm{P}}
\newcommand{\Tau}{\mathrm{T}}

\newfont{\bbb}{msbm10 scaled 700}
\newcommand{\CCC}{\mbox{\bbb C}}

\newfont{\bb}{msbm10 scaled 1100}
\newcommand{\CC}{\mbox{\bb C}}
\newcommand{\PP}{\mbox{\bb P}}
\newcommand{\RR}{\mbox{\bb R}}
\newcommand{\QQ}{\mbox{\bb Q}}
\newcommand{\ZZ}{\mbox{\bb Z}}
\newcommand{\FF}{\mbox{\bb F}}
\newcommand{\GG}{\mbox{\bb G}}
\newcommand{\EE}{\mbox{\bb E}}
\newcommand{\NN}{\mbox{\bb N}}
\newcommand{\KK}{\mbox{\bb K}}
\newcommand{\HH}{\mbox{\bb H}}
\newcommand{\SSS}{\mbox{\bb S}}
\newcommand{\UU}{\mbox{\bb U}}
\newcommand{\VV}{\mbox{\bb V}}


\newcommand{\yy}{\mathbbm{y}}
\newcommand{\xx}{\mathbbm{x}}
\newcommand{\zz}{\mathbbm{z}}
\newcommand{\sss}{\mathbbm{s}}
\newcommand{\rr}{\mathbbm{r}}
\newcommand{\pp}{\mathbbm{p}}
\newcommand{\qq}{\mathbbm{q}}
\newcommand{\ww}{\mathbbm{w}}
\newcommand{\hh}{\mathbbm{h}}
\newcommand{\vvv}{\mathbbm{v}}

% Vectors

\newcommand{\av}{{\bf a}}
\newcommand{\bv}{{\bf b}}
\newcommand{\cv}{{\bf c}}
\newcommand{\dv}{{\bf d}}
\newcommand{\ev}{{\bf e}}
\newcommand{\fv}{{\bf f}}
\newcommand{\gv}{{\bf g}}
\newcommand{\hv}{{\bf h}}
\newcommand{\iv}{{\bf i}}
\newcommand{\jv}{{\bf j}}
\newcommand{\kv}{{\bf k}}
\newcommand{\lv}{{\bf l}}
\newcommand{\mv}{{\bf m}}
\newcommand{\nv}{{\bf n}}
\newcommand{\ov}{{\bf o}}
\newcommand{\pv}{{\bf p}}
\newcommand{\qv}{{\bf q}}
\newcommand{\rv}{{\bf r}}
\newcommand{\sv}{{\bf s}}
\newcommand{\tv}{{\bf t}}
\newcommand{\uv}{{\bf u}}
\newcommand{\wv}{{\bf w}}
\newcommand{\vv}{{\bf v}}
\newcommand{\xv}{{\bf x}}
\newcommand{\yv}{{\bf y}}
\newcommand{\zv}{{\bf z}}
\newcommand{\zerov}{{\bf 0}}
\newcommand{\onev}{{\bf 1}}

% Matrices

\newcommand{\Am}{{\bf A}}
\newcommand{\Bm}{{\bf B}}
\newcommand{\Cm}{{\bf C}}
\newcommand{\Dm}{{\bf D}}
\newcommand{\Em}{{\bf E}}
\newcommand{\Fm}{{\bf F}}
\newcommand{\Gm}{{\bf G}}
\newcommand{\Hm}{{\bf H}}
\newcommand{\Id}{{\bf I}}
\newcommand{\Jm}{{\bf J}}
\newcommand{\Km}{{\bf K}}
\newcommand{\Lm}{{\bf L}}
\newcommand{\Mm}{{\bf M}}
\newcommand{\Nm}{{\bf N}}
\newcommand{\Om}{{\bf O}}
\newcommand{\Pm}{{\bf P}}
\newcommand{\Qm}{{\bf Q}}
\newcommand{\Rm}{{\bf R}}
\newcommand{\Sm}{{\bf S}}
\newcommand{\Tm}{{\bf T}}
\newcommand{\Um}{{\bf U}}
\newcommand{\Wm}{{\bf W}}
\newcommand{\Vm}{{\bf V}}
\newcommand{\Xm}{{\bf X}}
\newcommand{\Ym}{{\bf Y}}
\newcommand{\Zm}{{\bf Z}}

% Calligraphic

\newcommand{\Ac}{{\cal A}}
\newcommand{\Bc}{{\cal B}}
\newcommand{\Cc}{{\cal C}}
\newcommand{\Dc}{{\cal D}}
\newcommand{\Ec}{{\cal E}}
\newcommand{\Fc}{{\cal F}}
\newcommand{\Gc}{{\cal G}}
\newcommand{\Hc}{{\cal H}}
\newcommand{\Ic}{{\cal I}}
\newcommand{\Jc}{{\cal J}}
\newcommand{\Kc}{{\cal K}}
\newcommand{\Lc}{{\cal L}}
\newcommand{\Mc}{{\cal M}}
\newcommand{\Nc}{{\cal N}}
\newcommand{\nc}{{\cal n}}
\newcommand{\Oc}{{\cal O}}
\newcommand{\Pc}{{\cal P}}
\newcommand{\Qc}{{\cal Q}}
\newcommand{\Rc}{{\cal R}}
\newcommand{\Sc}{{\cal S}}
\newcommand{\Tc}{{\cal T}}
\newcommand{\Uc}{{\cal U}}
\newcommand{\Wc}{{\cal W}}
\newcommand{\Vc}{{\cal V}}
\newcommand{\Xc}{{\cal X}}
\newcommand{\Yc}{{\cal Y}}
\newcommand{\Zc}{{\cal Z}}

% Bold greek letters

\newcommand{\alphav}{\hbox{\boldmath$\alpha$}}
\newcommand{\betav}{\hbox{\boldmath$\beta$}}
\newcommand{\gammav}{\hbox{\boldmath$\gamma$}}
\newcommand{\deltav}{\hbox{\boldmath$\delta$}}
\newcommand{\etav}{\hbox{\boldmath$\eta$}}
\newcommand{\lambdav}{\hbox{\boldmath$\lambda$}}
\newcommand{\epsilonv}{\hbox{\boldmath$\epsilon$}}
\newcommand{\nuv}{\hbox{\boldmath$\nu$}}
\newcommand{\muv}{\hbox{\boldmath$\mu$}}
\newcommand{\zetav}{\hbox{\boldmath$\zeta$}}
\newcommand{\phiv}{\hbox{\boldmath$\phi$}}
\newcommand{\psiv}{\hbox{\boldmath$\psi$}}
\newcommand{\thetav}{\hbox{\boldmath$\theta$}}
\newcommand{\tauv}{\hbox{\boldmath$\tau$}}
\newcommand{\omegav}{\hbox{\boldmath$\omega$}}
\newcommand{\xiv}{\hbox{\boldmath$\xi$}}
\newcommand{\sigmav}{\hbox{\boldmath$\sigma$}}
\newcommand{\piv}{\hbox{\boldmath$\pi$}}
\newcommand{\rhov}{\hbox{\boldmath$\rho$}}
\newcommand{\upsilonv}{\hbox{\boldmath$\upsilon$}}

\newcommand{\Gammam}{\hbox{\boldmath$\Gamma$}}
\newcommand{\Lambdam}{\hbox{\boldmath$\Lambda$}}
\newcommand{\Deltam}{\hbox{\boldmath$\Delta$}}
\newcommand{\Sigmam}{\hbox{\boldmath$\Sigma$}}
\newcommand{\Phim}{\hbox{\boldmath$\Phi$}}
\newcommand{\Pim}{\hbox{\boldmath$\Pi$}}
\newcommand{\Psim}{\hbox{\boldmath$\Psi$}}
\newcommand{\Thetam}{\hbox{\boldmath$\Theta$}}
\newcommand{\Omegam}{\hbox{\boldmath$\Omega$}}
\newcommand{\Xim}{\hbox{\boldmath$\Xi$}}


% Sans Serif small case

\newcommand{\Gsf}{{\sf G}}

\newcommand{\asf}{{\sf a}}
\newcommand{\bsf}{{\sf b}}
\newcommand{\csf}{{\sf c}}
\newcommand{\dsf}{{\sf d}}
\newcommand{\esf}{{\sf e}}
\newcommand{\fsf}{{\sf f}}
\newcommand{\gsf}{{\sf g}}
\newcommand{\hsf}{{\sf h}}
\newcommand{\isf}{{\sf i}}
\newcommand{\jsf}{{\sf j}}
\newcommand{\ksf}{{\sf k}}
\newcommand{\lsf}{{\sf l}}
\newcommand{\msf}{{\sf m}}
\newcommand{\nsf}{{\sf n}}
\newcommand{\osf}{{\sf o}}
\newcommand{\psf}{{\sf p}}
\newcommand{\qsf}{{\sf q}}
\newcommand{\rsf}{{\sf r}}
\newcommand{\ssf}{{\sf s}}
\newcommand{\tsf}{{\sf t}}
\newcommand{\usf}{{\sf u}}
\newcommand{\wsf}{{\sf w}}
\newcommand{\vsf}{{\sf v}}
\newcommand{\xsf}{{\sf x}}
\newcommand{\ysf}{{\sf y}}
\newcommand{\zsf}{{\sf z}}


% mixed symbols

\newcommand{\sinc}{{\hbox{sinc}}}
\newcommand{\diag}{{\hbox{diag}}}
\renewcommand{\det}{{\hbox{det}}}
\newcommand{\trace}{{\hbox{tr}}}
\newcommand{\sign}{{\hbox{sign}}}
\renewcommand{\arg}{{\hbox{arg}}}
\newcommand{\var}{{\hbox{var}}}
\newcommand{\cov}{{\hbox{cov}}}
\newcommand{\Ei}{{\rm E}_{\rm i}}
\renewcommand{\Re}{{\rm Re}}
\renewcommand{\Im}{{\rm Im}}
\newcommand{\eqdef}{\stackrel{\Delta}{=}}
\newcommand{\defines}{{\,\,\stackrel{\scriptscriptstyle \bigtriangleup}{=}\,\,}}
\newcommand{\<}{\left\langle}
\renewcommand{\>}{\right\rangle}
\newcommand{\herm}{{\sf H}}
\newcommand{\trasp}{{\sf T}}
\newcommand{\transp}{{\sf T}}
\renewcommand{\vec}{{\rm vec}}
\newcommand{\Psf}{{\sf P}}
\newcommand{\SINR}{{\sf SINR}}
\newcommand{\SNR}{{\sf SNR}}
\newcommand{\MMSE}{{\sf MMSE}}
\newcommand{\REF}{{\RED [REF]}}

% Markov chain
\usepackage{stmaryrd} % for \mkv 
\newcommand{\mkv}{-\!\!\!\!\minuso\!\!\!\!-}

% Colors

\newcommand{\RED}{\color[rgb]{1.00,0.10,0.10}}
\newcommand{\BLUE}{\color[rgb]{0,0,0.90}}
\newcommand{\GREEN}{\color[rgb]{0,0.80,0.20}}

%%%%%%%%%%%%%%%%%%%%%%%%%%%%%%%%%%%%%%%%%%
\usepackage{hyperref}
\hypersetup{
    bookmarks=true,         % show bookmarks bar?
    unicode=false,          % non-Latin characters in AcrobatÕs bookmarks
    pdftoolbar=true,        % show AcrobatÕs toolbar?
    pdfmenubar=true,        % show AcrobatÕs menu?
    pdffitwindow=false,     % window fit to page when opened
    pdfstartview={FitH},    % fits the width of the page to the window
%    pdftitle={My title},    % title
%    pdfauthor={Author},     % author
%    pdfsubject={Subject},   % subject of the document
%    pdfcreator={Creator},   % creator of the document
%    pdfproducer={Producer}, % producer of the document
%    pdfkeywords={keyword1} {key2} {key3}, % list of keywords
    pdfnewwindow=true,      % links in new window
    colorlinks=true,       % false: boxed links; true: colored links
    linkcolor=red,          % color of internal links (change box color with linkbordercolor)
    citecolor=green,        % color of links to bibliography
    filecolor=blue,      % color of file links
    urlcolor=blue           % color of external links
}
%%%%%%%%%%%%%%%%%%%%%%%%%%%%%%%%%%%%%%%%%%%


\newcommand{\KK}[1]{\textcolor{red}{KK: \textbf{#1}}}
\newcommand{\red}[1]{\textcolor{red}{\textbf{#1}}}
\newcommand{\blue}[1]{\textcolor{blue}{\textbf{#1}}}
\newcommand{\MT}[1]{\textcolor{black}{MT: \textbf{#1}}}
\usepackage{graphicx} % Required for inserting images
\usepackage{longtable}
\usepackage{tabularx,ragged2e,booktabs,caption,adjustbox}
%\usepackage{subfigure}
\usepackage{multicol,capt-of}
\usepackage{hhline}
\usepackage{multirow}
\usepackage{array}
\usepackage{nicematrix}\graphicspath{{img/}}
\usepackage{xcolor}
\usepackage{adjustbox}\usepackage{mathtools}
\usepackage{hhline}
\usepackage{array}\usepackage{makecell}
%\newcolumntype{P}[1]{>{\centering\arraybackslash}p{#1}}

% Nature Climate Change
% https://www.nature.com/nclimate/
% https://www.nature.com/articles/s41558-022-01527-x
% https://www.nature.com/documents/NRJs-guide-to-preparing-final-artwork.pdf
% Word limit: Main text – up to 3,000 words, excluding abstract, online Methods, references and figure legends.
% Submission system: https://mts-nclim.nature.com/cgi-bin/main.plex

%%%%%% WORD COUNT ON Sep 23, 2024: around 1800

% PLOS Climate
% https://journals.plos.org/climate/article?id=10.1371/journal.pclm.0000277

% PNAS Nexus
% https://academic.oup.com/pnasnexus


\newcommand{\tabhead}[1]{\textbf{#1}}
\title[]{The Role of Science in the Climate Change Discussions on Reddit}

% \author[]{Paolo Cornale,$^1$ Michele Tizzani,$^1$ Fabio Ciulla,$^1$ Kyriaki Kalimeri,$^1$ \\Elisa Omodei,$^2$ Daniela Paolotti,$^1$ Yelena Mejova$^1$\\}

% \affiliation{%
%  \institution{$^1$ ISI Foundation, Turin\country{Italy}}
%  \institution{\\$^2$ Central European University, Vienna\country{Austria}}
% }

\author{Paolo Cornale}
\affiliation{%
  \institution{ISI Foundation}
  \city{Turin}
  \country{Italy}
}
\author{Michele Tizzani}
\affiliation{%
  \institution{ISI Foundation}
  \city{Turin}
  \country{Italy}
}
\author{Fabio Ciulla}
\affiliation{%
  \institution{ISI Foundation}
  \city{Turin}
  \country{Italy}
}
\author{Kyriaki Kalimeri}
\affiliation{%
  \institution{ISI Foundation}
  \city{Turin}
  \country{Italy}
}
\author{Elisa Omodei}
\affiliation{%
  \institution{Central European University}
  \city{Vienna}
  \country{Austria}
}
\author{Daniela Paolotti}
\affiliation{%
  \institution{ISI Foundation}
  \city{Turin}
  \country{Italy}
}
\author{Yelena Mejova}
\affiliation{%
  \institution{ISI Foundation}
  \city{Turin}
  \country{Italy}
}
\email{yelenamejova@acm.org}




% up to 150 words
\begin{abstract}

Collective and individual action necessary to address climate change hinges on the public's understanding of the relevant scientific findings. 
In this study, we examine the use of scientific sources in the course of 14 years of public deliberation around climate change on one of the largest social media platforms, Reddit.
We find that only 4.0\% of the links in the Reddit posts, and 6.5\% in the comments, point to domains of scientific sources, although these rates have been increasing in the past decades. 
These links are dwarfed, however, by the citations of mass media, newspapers, and social media, the latter of which peaked especially during 2019-2020. 
Further, scientific sources are more likely to be posted by users who also post links to sources having central-left political leaning, and less so by those posting more polarized sources.
Unfortunately, scientific sources are not often used in response to links to unreliable sources.

\end{abstract}


\begin{document}

\maketitle

\pagestyle{plain}

%%%%%%%%%%%%%%%%%%%%%%%%%%%%%%%%%%%%%%%%%%%%%%%%%
\section{Introduction}
\label{sec:intro}
%%%%%%%%%%%%%%%%%%%%%%%%%%%%%%%%%%%%%%%%%%%%%%%%%


% Climate change is bad
Climate change poses a critical threat that requires urgent global action. 
Despite a broad scientific agreement around a strong anthropogenic component of climate change \cite{abbass_review_2022}, %individual and collective actions play a critical role in shaping future outcomes. 
as of 2023, only 56\% of US respondents to the Yale Climate Opinion survey thought that ``most scientists think global warming is happening''~\cite{marlon2022change,yale2023data}. 
Given the importance of public understanding of the latest scientific findings necessary for informed decision-making, in this study, we examine to what degree scientific resources are used to drive or substantiate the online discussions around climate change, in comparison to other sources, such as news and social media, including sources known to be unreliable.

%Human activities, primarily the burning of fossil fuels, have significantly increased greenhouse gas emissions, leading to global warming and disruptions in weather patterns \cite{intergovernmental_panel_on_climate_change_ipcc_carbon_2014}. 
%The consequences are far-reaching, including rising sea levels \cite{mengel_committed_2018, nerem_climate-changedriven_2018}, more frequent extreme weather events \cite{katz_extreme_1992, ebi_extreme_2021}, biodiversity loss \cite{parmesan_globally_2003}, and health threats \cite{tsui2024impacts}. 
%Scientific evidence clearly shows that climate change is underway, with potentially catastrophic impacts if left unchecked. 
%Despite the advances in alternative energy technologies and other scientific breakthroughs, the greatest obstacle to implementing a truly global, impactful solution to climate change may hinge on achieving a popular consensus around the basic scientific facts.

% All the other complicated aspects of taking action
% However, the drivers of environmental behavior are multifaceted, and influenced by cultural variability, moral values, and political leanings. 
% Cultural differences can significantly affect how individuals perceive climate risks and support mitigation policies, with studies indicating that those with egalitarian values are more likely to engage in pro-environmental behaviors \cite{eom_cultural_2016, kahan_polarizing_2012}. 
% Additionally, moral values shape environmental actions, where individuals prioritizing the well-being of the planet are more inclined to adopt sustainable practices \cite{steg_values_2016}. 


% Aspects that WE study

% RQ1: how much sci communication is there, how good is it
%To achieve this, effective communication of scientific findings is crucial. 
Despite its privileged status in academia and industry, scientific communication competes for clicks in a cutthroat attention economy of the Web, contending with the fickle proprietary recommendation systems and shortening attention spans of their users \cite{HYLAND2023101253}.
%Science, and data specifically, can ``appear objective, precise, and replicable''~\cite{west2021misinformation}, but cited out of context, cherry-picked, or even purposefully misrepresented, and can be used to lend support for interest groups. 
%Throughout the 20th century, the tobacco industry has engaged for decades in the discrediting of the increasing evidence linking tobacco to cancer, by employing agnotogenesis---the sowing of doubt---about the scientific consensus~\cite{oreskes2011merchants}.
%Similarly, the science around climate change has been challenged by industry-supported organizations such as the Heartland Institute (which has also attempted to discredit the health risks of secondhand smoke) the Global Climate Coalition, and American Petroleum Institute~\cite{dunlap2013climate,oreskes2011merchants}.
%However, the battle over climate change gained a political charge and contributed to its polarization, starting from the efforts of neo-liberal market deregulation in the 1970s~\cite{antonio2011unbearable,smith2024polarisation}, and the subsequent ideologically-driven communication campaigns~\cite{cook_understanding_2022,farrell2016corporate}.
Further, the perception, understanding, and citation of scientific literature by non-experts depend on a myriad of factors, including numerical literacy~\cite{roozenbeek2020susceptibility}, religious beliefs~\cite{schwarz2016blinded} and spirituality~\cite{browne2015going}, as well as the social context~\cite{lewandowsky2019science}.  
%The direct citation of science on the Web and social media may then suffer from decreased information retention and interest over time~\cite{hwang2023information}, while also may be replaced with mainstream news media reporting as a mediator in the access to scientific news~\cite{takahashi2016media}.
%As social media becomes an increasingly important source of information and discussion for the majority of people, it becomes imperative to understand the extent to which these social platforms rely on scientific sources in their deliberation on important issues, such as climate change. 
% Some points we can mention:
% It is difficult to communicate science to a non-science audience
% Social media is not helping because the format is short and decontextualized
% Perception of science can be biased by preexisting views 
% Misinformation
% West is a great paper: https://www.pnas.org/doi/full/10.1073/pnas.1912444117
%
%
% RQ2: this communication is often on social media
Social media is becoming an increasingly important source of information, with Pew Research Center concluding in 2023 that half of U.S.~adults get news at least sometimes from social media~\cite{pew2023social}.
%Moreover, people with higher news media knowledge are more likely to include social media in their news repertoire~\cite{schulz2024role}.
%As such, social media is a source of new information on current events, and a space for discussion, augmentation, and synthesis of this information into one's worldview.
Climate change debate has been extensively studied on Twitter~\cite{effrosynidis2022climate,karimiziarani2023toward}, yielding observations of homophilous segregation of users into like-minded camps of ``skeptics'' and ``activists'', % or ``believers'', 
which can be detected via the posted content~\cite{upadhyaya2023multi} or network analysis~\cite{williams2015network}, and which intensify during events such as the COP Climate Change conference~\cite{falkenberg2022growing}. % williams2015network use mention, follower, retweet networks
However, Reddit---the fifth most visited website in the US~\cite{similarweb2024top},  which is much less studied---has been shown to display much less polarization than Twitter~\cite{de2021no} and may foster more deliberative interactions. 
%The conversations on climate change specifically have been found to generally not contain echo chambers, although some individual communities were dominated by particular ideological viewpoints~\cite{treen2022discussion}.
%Whether this situation extends to a broad use of scientific evidence is unclear.
%On Twitter, an analysis of the debate around climate change has shown that the scientists were on the periphery of the conversation and that they interacted most intensively with their peers~\cite{walter2019scientific}. 
%\KK{citing a couple here. youtube: \cite{allgaier2019science,shapiro2015more}, wiki: \cite{meier2024using}}
The literature is lacking in the broad, longitudinal examination of scientific discourse on this platform, focusing on particular subreddits such as r/science~\cite{moriarty2024reddit,hara2019emerging}, or those relevant to the climate change debate, i.e.~r/climate or r/climateskeptics~\cite{gadanidis2020discourse,oswald2022climate,villanueva2021climate,parsa2022analyzing}. % , r/climatechange, r/climate\_science, etc.
%Understanding the extent to which scientific information is siloed in the broader climate change debate on Reddit is an important step in gauging the breadth of the public's scientific engagement, which is necessary to reach an agreement on the basic scientific facts.

Further, in the US, the debate is complicated by the politicization of stances around climate change: in 2023, 23\% of Republicans considered climate change a major threat, compared to 78\% of Democrats~\cite{pew_partisan_opinion_climate_change}. 
In fact, the attitudes towards science, in general, are different among the two parties:  47\% of Republicans view science as benefiting society, compared to 69\% of Democrats~\cite{viswanathan2023americans}. 
In this study, we use 14 years of climate change-related Reddit posts and comments spanning thousands of subreddits to gauge the use of scientific resources in this deliberative space, including in the context of political interest.
Its results point to a scant, but increasing, utilization of scientific sourcing, more frequently used by the users having center-left political interests.
Alas, we find little evidence of it cited in response to information from unreliable sources.
%Thus, it is important to understand whether political leaning of users and communities on social media may be associated with a limited exposure to scientific literature in the argumentation around climate change.

% RQ3: this communication is often politicized
% Politics affects how people perceive (scientific) facts (RQ3)
% CITATION: political polarization - believing in science
% Political ideology also influences climate attitudes, as individuals on the political right often express skepticism towards climate change, contrasting with the more proactive stance of those on the left   
%In the US, the debate is complicated by the politicization of climate change stances: in 2023, 23\% of the Republicans considered climate change a major threat, compared to 78\% of the Democrats~\cite{pew_partisan_opinion_climate_change}. 
%In fact, the attitudes towards science, in general, are different among the two parties:  47\% of Republicans view science as benefiting society, compared to 69\% of Democrats~\cite{viswanathan2023americans}. 
%Indeed, the stated policies of the two parties concerning climate change differ substantially: whereas the Democratic party elites have been consistently supportive of the climate consensus~\cite{merkley2021party}, the Republican party, and especially its neoliberal champions, argue that environmentalists in the government ``intrude'' on society by curtailing consumer choice and property rights~\cite{antonio2011unbearable,smith2024polarisation}.
%Concerning science and academia, in the past decade, there has been a sharp decline among Republicans of those who ``believe that colleges and universities have a ``positive effect'' on the country''~\cite{atkeson2019partisan}.
%Thus, it is important to understand whether political leaning of users and communities on social media may be limiting the access to and discussion of scientific literature in the argumentation around climate change.

%\cite{dunlap_climate_2015}. from Michele bib


% Most relevant social media studies
%To cite: Build a tool for tracking science citations in news articles, find two large clusters of unreliable sources to which different types of conspiracy news sources flock. One of them concentrates on alternative health misinformation, and the other promotes hyper-partisan political conspiracies. \cite{gruppi2023scilander}


% Impact statement -- move to Discussion
% To address these pressing challenges, in this study, we examine the extent to which scientific literature, as well as other resources, has been used in the debate around climate change on Reddit in the past 14 years.
% We find that domains of scientific journals, magazines, aggregators, and preprints are not nearly as much cited as mass media, newspapers, or social media. \red{should we include percentages?}
% But, along with governmental and Wikimedia links, science is much more used in the comments than in the posts, pointing to their importance in substantiating debate. 
% All domain categories we examine display substantial inequality in terms of subreddits in which they are posted, and the users that post them, but we do not find that the scientific domains are disproportionally more siloed than, say, social media or newspapers. 
% Within the conversations, we do find that science begets science: posts with scientific links are much more likely to have other scientific links in their comments.
% Considering political affiliation, we find that the users who post center-left and center-right URLs are also more likely to link to scientific sources (former more than latter), much more than those favoring more extreme right or left. 
% Worryingly, unreliable URLs were shared in posts more (5.4\%) than science (4.0\%), but fortunately, they were not used in the further conversation in comments (2.0\%) as much as science ones (6.5\%).
% Furthermore, we do not find science disproportionately used in the response to posts with links to unreliable URLs, potentially missing an opportunity for substantiated debunking.
% Thus, in the past 14 years, whereas the drivers of the Reddit conversation on climate change were news and mass media, scientific sources have been used to substantiate the ongoing debate, especially by those in the political center-left. 
% However, more effort is necessary to respond to information from unreliable sources by making scientific communication more accessible.





%%%%%%%%%%%%%%%%%%%%%%%%%%%%%%%%%%%%%%%%%%%%%%%%%
\section{Results}
%%%%%%%%%%%%%%%%%%%%%%%%%%%%%%%%%%%%%%%%%%%%%%%%%

%RQ 1. How much are scientific sources used in this conversation and how much response do they provoke? 
\subsection{Domain Citation and Engagement}


% Figures 1 a - d
\begin{figure}[t]  %ht
    \centering
    \includegraphics[width=0.90\textwidth,keepaspectratio]{Figures/summary_domains_ALL_num.pdf}
    \includegraphics[width=0.8\textwidth,keepaspectratio]{Figures/yearly_stackplot_num.pdf}
    \caption{Statistics of URL usage in the dataset: (a) proportion of URLs in a particular category, separately for posts (preceded with ``P'') and comments (preceded with ``C'' and dashed), (b) engagement with the posts containing a URL of a particular category in terms of the percentage of posts having at least one comment, the average number of comments for posts having at least one comment, and the average length of the comment in terms of words, (c-d) proportion of URLs in a particular category in posts and comments, over time.}
    \label{fig:domain_categories_and_engagement}
\end{figure}

% descriptive stats on posting

Using a dataset of 1.3M posts and 20.3M comments on Reddit around the topic of Climate Change, we examine the categories of URLs used in this discussion. 
%Figure \red{1a} shows the percentage of all URLs in the posts and comments that belong to each of the domain categories, as well as those labeled as unreliable, and those having right or left political leaning (see Methodology for the sources of each category). 
We were able to categorize 69.5\% of URLs in posts and 55.2\% in comments into 6 source types (see the legend of Figure \ref{fig:domain_categories_and_engagement}a).
The largest category of URLs cited in the past 14 years is mass media (30.2\% in posts and 15.9\% in comments), followed by newspapers and social media. 
The latter (social media) has especially peaked during 2019 and 2020 (see Figures \ref{fig:domain_categories_and_engagement}c-d).
Science-related URLs appear in only 4.0\% of URLs in posts and 6.5\% in comments, though the proportion of science-related URLs has been increasing in the last decade.
These URLs to scientific domains, along with Wikimedia projects (0.1\% in posts,~8.8\% in comments) and governmental domains (0.5\%,~5.7\%) appear mostly in comments, rather than posts, pointing to their importance in the substantiating of discussion.
%, whereas the use of mass media and newspapers decreases in comments compared to posts.
%A nested plot within Figure \red{1a} shows the breakdown of scientific domains present in the dataset.
Among the scientific URLs, journal domains are more likely to be cited in comments, in comparison to scientific magazines and aggregators (which may have less rigorous inclusion criteria).
%, we find that the journals are cited less in the posts, but are favored by the commenters (0.9\% vs.~3.7\%, respectively).
Interestingly, preprints, which may contain cutting-edge reporting that has not passed peer review, are the least cited scientific domain category, at 0.1\% and 0.2\% in posts and comments, respectively.
The last three statistics shown in Figure \ref{fig:domain_categories_and_engagement}a are characteristics of URLs that may overlap with those above: domains known to publish unreliable content, and those having a right or left political leaning.
Science-related domains are cited at a similar rate to domains known to be unreliable, though these are less likely to be used in the comments (5.4\% of URLs in posts and 2.0\% in comments).
Furthermore, when the political leaning of a URL is known, it is more likely to be left-leaning than right-leaning, pointing to an unequal coverage of the topic.
The quality of these URLs is also different between the two sides: 55\% of right-leaning URLs in posts and 33\% in comments are listed as unreliable, compared to only 0.01\% of left-leaning URLs in posts and 0.09\% in comments (echoing findings in \cite{robertson2023users}).

%out of 42,229 right-political-leaning URLs in the posts, 23,261 (55.08\%) are from domains listed in the unreliable set (see Methodology), and, for the comments, out of 47,818 right-leaning URLs, 15,652 (32.73\%) are also unreliable. 
%In comparison, left-leaning URLs in posts and comments intersect with unreliable domains at 0.01\% and 0.09\%, respectively; similarly, mass media URLs at 10.06\% and 4.46\% and newspaper at 0.38\% and 4.46\%.

% engagement

%Figure \red{1b} shows the extent to which users engage with posts that have certain kinds of links, specifically, whether such posts receive comments (a proxy for user engagement), how many comments, and how long these comments are (a proxy of the amount of engagement per comment) (see Methods for details).
%We find that for all categories of domains, more than half of the posts receive no comments.
In terms of engagement, the categories of domains that receive at least one comment are those from Wikimedia, scientific journals, and magazines, followed by the governmental ones (see Figure \ref{fig:domain_categories_and_engagement}b). 
Interestingly, despite being a popular domain category, social media receives comparatively fewer comments.
%In terms of the number of comments received, newspapers and mass media receive more (on average 37 and 32 comments when at least one comment is present, respectively) than any of the science categories (no more than 27 comments), suggesting that, although scientific magazines and journals may provoke some response, the comment discussions are not as long as those provoked by newspapers and mass media.
However, the average length of the comments, which provides another way to measure engagement, remains remarkably stable across the domain categories, ranging around 47 - 56 words per comment. % (note that the URLs, punctuation and emoticons were removed). 
%Longer comments are made in response to posts with scientific and governmental URLs, and shorter for mass media and newspapers. 
Whereas left-leaning domains tend to receive more comments compared to right-leaning ones, it is the domains listed as unreliable that receive the longer comments.
% Considering the dimensions of reliability and political leaning, we find the number of comments to be the greatest for posts with left-leaning domains.
% On the other hand, the unreliable domains receive the longest responses in terms of characters (interestingly, not in the number of comments).
% In summary, although social media, newspapers, and mass media are more likely to be cited in posts, scientific and governmental domains provoke more engagement. 

% top posting users and subreddits

%\red{Move most of it to supplementary} 
%Next, we examine the subreddits and the users that contribute the most scientific citations in the Climate Change discussion.
The distribution of the scientific links among the subreddits is extremely concentrated, with the top 10 subreddits (by the number of scientific links) accounting for 40.2\% of all science-related URLs in our dataset.
These include r/worldnews (contributing \num{20451} scientific URLs, which make up 7.2\% of all URLs in that community in our data), r/science (\num{11235}, 15.4\%), r/environment (\num{9665}, 7.2\%), r/politics (\num{8995}, 3.5\%), r/collapse (\num{7646}, 9.0\%), r/climate (\num{7140}, 9.8\%), r/climateskeptics (\num{6764}, 8.0\%), r/Futurology (\num{6422}, 8.3\%), r/climatechange (\num{5741}, 14.4\%), and r/AskReddit (\num{4600}, 8.4\%). 
Among several communities around Climate Change specifically, we also find more general ones, such as r/worldnews and r/politics, as well as r/AskReddit, attesting to the mainstream interest in the topic and the prevalence of scientific referencing even in non-specialized circles. 
Interestingly, r/climateskeptics, a community dedicated to ``Questioning climate related environmentalism'' (in the description of the subreddit), contributes over 6K links to science-related domains (which are 8.0\% of all domains on that subreddit in our dataset, a similar rate to r/worldnews and r/climate).
%
% Turning to users who have contributed the most number of scientific links, at the top we find \emph{ILikeNeurons} (\num{15685}), 
% \emph{worldnews} (\num{20451}), 
% \emph{BurnerAcc2020} (\num{3254}), 
% \emph{AutoModerator} (\num{2962}), 
% \emph{ZephirAWT} (\num{2926}), 
% \emph{Jaagsiekte} (\num{2412}), 
% \emph{EcoInternetNewsfeed} (\num{3382}), 
% \emph{MmmBaconBot} (\num{3382}), 
% \emph{kamjaxx} (\num{1093}), and
% \emph{avogadros\_number} (\num{1684}).
% These top 10 users contribute 16.6\% of all scientific URLs in our data.
Examining the top users posting scientific links (see Supplementary section \ref{appendix_topusers}), we find several accounts (3/10) that are explicitly bots: \emph{AutoModerator}, \emph{EcoInternetNewsfeed}, and \emph{MmmBaconBot}, echoing previous findings on the prevalence of bots on the platform~\cite{ferrara2016rise}.
%However, from the behavior of the other accounts, it is likely that they are also at least partially automated, such as the \emph{ILikeNeurons}. 
%This finding suggests that there is a high number of bots on Reddit, something that was already noted in previous literature \cite{ferrara2016rise}. %in 2016 illustrates the rising prevalence of automated users across social media platforms}, 
These bots, however, are not necessarily malicious, as they are explicit about their nature, and follow the ``bottiquette''\footnote{\url{https://www.reddit.com/wiki/bottiquette/}} \cite{hurtado2019bot,ma2020empirical}.
%Despite possible automation, these accounts may still provide relevant content and support engagement around the topic.
%For instance, the most active user in terms of posting scientific links, ILikeNeurons, has shared links in 106 subreddits in our dataset, contributes both posts and comments and is still active at the time of writing. 
%The account even manages 7 subreddits, including r/ClimateOffensive and r/climatesolutions. 
Given the importance of (semi-)automated accounts, we do not attempt to remove them from our dataset.



%%%%%%%%%%%%%%%%%%%%%%%%%%%%%%%%%%%%%%%%%%%%%%%%%




% RQ3. Do users of different political leanings post and reply with scientific sources at different rates? 
\subsection{Political Leaning and Scientific Citation}

% 3.1 Do users of particular political leaning post differently? 

%%% Figure 2 
\begin{figure}[t]  %ht
    \centering
    \includegraphics[width=\textwidth,keepaspectratio]{Figures/domain_dist_per_poli_means.pdf} % without stats: violin_plots_biased_users_cropped.pdf
    \caption{Distributions of the proportion of URLs posted by a user that are from a particular category, grouped by users whose URLs have a particular political leaning. Under each distribution, the mean proportion is shown, and a * is shown between two consecutive groups if their distributions differ using the Kolmogorov-Smirnov two-sample test at $p<0.001$.
    }
    \label{fig:violin_plots_biased_users}
\end{figure}


Next, we consider users whose link posting activity puts them into the left, center-left, center-right, or right political leaning (as defined by the media bias ratings website AllSides.com).
We find that those in the center (especially center-left) are more likely to cite science than those in the extremes (see leftmost panel of Figure \ref{fig:violin_plots_biased_users}, and Supplementary Table at \ref{appendix_sharing_behavior_users}). % \KK{yes, i don't see it either. perhaps we can give this as a table? as a graph is questionable.} \red{Y: We'll insert a table with statistics in Supplementary.}
In particular, out of all URLs posted by the center-left users, on average, 7.4\% are science-related, whereas 3.3\% of the ones posted by center-right users (compared to 2.2\% and 0.8\% for left and right, respectively).
%For comparison, we compute the URL category distributions for other categories.
Due to the large dataset sizes, we find all of the within-category differences between the consecutive groups statistically significant at $p<0.001$, despite the small effect size, except in the case of social media (which all groups of users use at roughly indistinguishable rates). 
Among the other categories, we find right-leaning users to favor mass media, whereas left, and center-left users -- newspapers. 
The most striking difference between the groups, however, is the posting of unreliable sources, which are more likely to be on the right political spectrum.


% 3.2 What other posting interests determine scientific posting?

%%% Figure 3
\begin{figure}[t]  %ht
    \centering
    \includegraphics[width=0.66\textwidth,keepaspectratio]{Figures/shap.pdf}
    \caption{SHapley Additive exPlanations (SHAP), a game theoretical approach for explaining the contribution of each feature to the final output of a ML model~\citep{NIPS2017_7062}. The Random Forest model predicts how many (log-normalized) science-related URLs a user has posted in our dataset (here, ``high'' means more URLs were posted), using behavioral features including the categories of the other URLs they shared, as well as the political leaning of those URLs. Top 100 most popular subreddits are used as features, and all others are summed in ``other subreddits''. }
    \label{fig:shap}
\end{figure}


To understand the importance of political leaning in comparison with other climate change-related interests of the users, we train a model to predict how many (log-normalized) scientific URLs a user posts in our dataset (Figure \ref{fig:shap}). %Random Forest 
%Its variables include not only the political leaning of the user's posts and comments, but also the statistics of the sharing rates for each category of the URLs and posting/commenting activity in the top 100 most popular subreddits (note that during the computation of these statistics, the scientific URLs are removed). 
%Figure \red{3b} shows the SHapley Additive exPlanations (SHAP) graph, a game theoretical approach developed to explain the contribution of each feature to the final output of a ML model~\citep{NIPS2017_7062}.  -- TO THE CAPTION
We find that posting such URLs is associated with posting on popular scientific subreddits such as r/science, r/collapse, and r/environment, and, echoing previous results, with posting links with a center-left political leaning. 
Further, posting scientific URLs is more likely by users who post many kinds of URLs in comments, % though not necessarily by those who do so in posts. 
this relationship is reversed for social media -- those posting social media links in their posts are \emph{less} likely to also post science. 

% 3.3 Is scientific sourcing used to respond to polarized (left vs right) posting or posting of unreliable information? 

%%% Figure 4
\begin{figure}[t]  %ht
    \centering
    \includegraphics[width=0.6\textwidth,keepaspectratio]{Figures/conditional_probs.pdf}
    \caption{Conditional probability of a first-level comment with a URL containing a certain URL category (columns), given it is in response to a post having a scientific, unreliable, right- or left-leaning URL (rows).}
    \label{fig:conditional_probs}
\end{figure}


%As we often find a distinct behavior in posts and comments, we next check whether there is a difference in the way the scientific URLs (as well as other categories of URLs) are posted in response to politically charged messages.
%Figure \red{3c} shows, given a post with a URL of a particular category, the conditional probability that a URL of another category is used in a first-level comment to that post. 
Finally, we explore the relationship between citations in posts and comments (see Figure \ref{fig:conditional_probs}). 
We find that posts with scientific URLs are much more likely to be responded to with other scientific URLs (26\% of the time). 
Unfortunately, this is not the case for the posts sharing a link to an unreliable source: only 5\% of URLs in reply to these are to science sources, instead, links from social media, mass media, and ---more likely--- other unreliable sources are posted in response.
Similarly, posts having right or left-leaning URLs have about the same chance, about 5\%, of having a scientific URL in response.
%Thus, we do not find a politically siloed effect in the way science is used in the response to posts of different political leanings. 
%The fact that science is also not used in response to URLs from unreliable sources suggests that the potential ``debunking'' using scientific literature is not common (instead, links from social media, mass media, and other unreliable sources are more likely to be posted in response).
%Considering cross-party responses, we find the left-leaning URLs to be posted in response to posts both with left and right-leaning URLs, but much more to the left-leaning ones, pointing to a more active involvement of users likely to cite left-leaning sources.\KK{here we a bit disconnected from science.}\blue{Y: If we are pressed for space, we should definitely cut non-science results, CAN MOVE IT INTO CAPTION LOL}
%On the other side, right-leaning sources are almost never cited in response to posts with left-leaning or scientific URLs. 

In summary, when considering individual users and the political leaning of other sources they cite, we do not find a strong polarization in terms of using scientific sources, with those on center-left being more likely to post science than on center-right. 
However, the defining feature of Reddit is the community, many of which can revolve around a topic or political theme, and which may have their own cultures beyond individual users. 
We explore these next.

%%%%%%%%%%%%%%%%%%%%%%%%%%%%%%%%%%%%%%%%%%%%%%%


\subsection{Case Study}

% case study of a selection of subreddits

To delve further into the nature of the Reddit communities (subreddits), we select several subreddits out of those contributing the most posts to our dataset such that they span a variety of points of view: r/climate and r/climateskeptics (largest climate-related ones); r/science, r/worldnews, and r/politics (more general ones); and r/The\_Donald (dedicated to the Republican politician Donald Trump), and r/SandersForPresident (supporting the Independent/Democrat politician Bernie Sanders). 
Note that, besides having the largest share of posts in our data out of politically-oriented subreddits, Donald Trump and Bernie Sanders represent political extremes of the right and left, respectively.
See Supplementary table \ref{appendix_casestudy} for statistics on the sizes of these subreddits.
% The first two subreddits are central to the debate
% and have been used in previous studies of discourse around Climate Change on Reddit \cite{villanueva2021climate,parsa2022analyzing,gadanidis2020discourse,oswald2022climate}.
% These are followed by two general ones around science and world news (both studied in \cite{villanueva2021climate}).
% Finally, as the topic is highly politicized, we include a general political subreddit, as well as communities around two prominent politicians from the right-wing (Donald Trump -whose subreddit, \emph{The\_Donald}, is also studied in \cite{villanueva2021climate}-) and left-wing (Bernie Sanders) political orientations.
% Figure \ref{fig:casestudy_perc_url} shows (a) the percentage of URLs having a particular domain category and (b) the percentage of URLs having a particular political leaning.
% Statistics are shown separately for posts (P) and comments (C), as we find a marked difference in the kinds of URLs shared in each.
%We find a greater variety of URLs in the comments than in posts, such that we are able to identify the category of fewer URLs in comments (see Figure \red{2a}).
Again, we find science, Wikipedia, and governmental links to often appear in the comments more than in posts (see Figure \ref{fig:case_study}a).
%Also, the posts more often contain newspapers, mass, and social media links than the comments.
%Instead, the comments more often cite Wikipedia and governmental links and more often science ones, suggesting these are important sources of argumentation.
Unsurprisingly, science links are mentioned the most in the r/science subreddit, but also in the r/climate and r/worldnews.
Both of the political subreddits do not share references to science, suggesting that the political discussion of the subject is not explicitly supported by direct scientific references.
In the case of r/climateskeptics, few science URLs are included in posts, but many more are cited in the comments.
Unsurprisingly, the subreddit with the most right-leaning URLs is r/The\_Donald, with the posts having many more right-leaning URLs than comments (see Figure \ref{fig:case_study}b).
We see a similar behavior in r/climateskeptics. 
The URLs in the rest of the selected subreddits are leaning to the left, with r/SandersForPresident and r/climate ones being the most left-leaning, which is surprising, since r/climate is ostensibly not a political subreddit.
%It also has the smallest share of right-leaning URLs.
In summary, our case study suggests that, despite the Climate Change debate being highly politicized, the use of scientific evidence is lacking in the communities centered around politics, and instead is more prevalent in scientific and even in science-skeptic communities (especially in their comments).


% Figure 2
\begin{figure}[t]  %ht
    \centering
    \includegraphics[width=1\textwidth,keepaspectratio]{Figures/case_study_ALL_num.pdf}
    \caption{Case study of select subreddits, (a) the percentage of URLs having a particular domain category and (b) the percentage of URLs having a particular political leaning. Statistics are shown separately for posts (P) and comments (C).}
    \label{fig:case_study}
\end{figure}




%%%%%%%%%%%%%%%%%%%%%%%%%%%%%%%%%%%%%%%%%%%%%%%%%
\section{Discussion}
%%%%%%%%%%%%%%%%%%%%%%%%%%%%%%%%%%%%%%%%%%%%%%%%%

Thomas Jefferson is often attributed the (likely apocryphal) quote ``An educated citizenry is a vital requisite for our survival as a free people" \cite{bowman2019thomas}.
Since then, the connection between democratic deliberation and scientific education has been promoted by educators and reformers, such as John Dewey in \emph{How We Think} \cite{dewey1933how}, and more recently by the U.S.~National Research Council, positing that ``knowledge of science and engineering is required to
engage with the major public policy issues of today'' \cite{nrc2012framework}. 
Our findings show that, in the Reddit discussions of climate change, scientific sources have been dwarfed by links to news and social media, although the share of links to scientific resources has increased in the past decade. 
When they do appear, they are more likely to be in the comments (along with the links to governmental sources and Wikipedia), pointing to the importance of these resources to the deliberative process around this topic.
Unfortunately, we find that scientific links are much more likely to be posted in response to posts with other scientific links, whereas posts having links to unreliable sources do not often receive scientific links in their replies.
Instead, other unreliable sources are more likely to be cited.

Meanwhile, surveys show that between 2009 and 2019 (roughly in the duration of the examined data), the share of US respondents who acknowledge an increase in average global temperature rose by 8 percentage points, and the share who believe that humans have contributed to this rose by 11 percentage points \cite{carlsson2021climate}.
Whether the use of scientific resources contributed to this change of opinion is questionable. 
Experimental results suggest articles linking to scientific articles promote greater trust, however linking to any mainstream media may have the same effect \cite{verma2017human}. 
We find that scientific links appear not only in communities asserting the existence of anthropogenic climate change, but also in those ``skeptical'' of it.
According to the 2024 report by the Center for Countering Digital Health, climate change denialism has been morphing in the past few years, as global temperatures rose dramatically \cite{ccdg2024new}. 
Instead of opposing the concept of climate change itself, the ``New Denial'' themes include ``the impacts of global warming are beneficial or harmless'', ``climate solutions won't work'' and ``climate science and the climate movement are unreliable''.
The report points specifically to social media companies (including Instagram, Facebook, TikTok, and X (previously Twitter)) as potentially benefiting from the popularity of such content, and allowing for the monetization and direct profit for its creators.
What role the latest scientific evidence may play in tackling this New Denial themes is an open question.

However, the mere presence of scientific evidence may not necessarily correspond to a cross-partisan conversation.
Psychology literature suggests that individuals more knowledgeable on an issue are more susceptible to selection bias and motivated reasoning \cite{bushey2012everyday,hannon2022knowledgeable,kahan2013ideology}.
Our discovery that 9.4\% of the URLs in r/climateskeptics community's comments are scientific links points to the active use of science in the community. 
Previous studies found that, for instance, vaccine skeptics often express respect for the scientific method and are interested in the rigorous scientific examination of matters affecting them personally \cite{koltai2017questioning}. 
Further, the level of one's education may affect the trust in climate science via the perception of having a lower or higher social status \cite{hoekstra2024educational}. 
% Excellent resource: https://www.sciencedirect.com/science/article/pii/S0959378024000554#b0200
To some, attitudes towards climate science may be ``not just an opinion on an issue, but [an] aspect of self that defines who they are, what they stand for, and who they stand with (and against)'' \cite{bliuc2015public}. 
Thus, the citation of science may be a part of the construction of self-evaluation as ``eco-habitus'', a concept favoring environmental actions and engagement in a ``green'' lifestyle \cite{kennedy2019eco}. 
\citet{hoekstra2024educational} found that such subjective view of one's social status may contribute to the distrust in climate science.
What role the citation of scientific literature plays in the individuals' formation of a self-image is an interesting future research direction.

In the U.S., a major aspect of such self-image may be one's political affiliation.
The stated policies of the two parties concerning climate change differ substantially: whereas the Democratic party elites have been consistently supportive of the climate consensus~\cite{merkley2021party}, the Republican party, and especially its neoliberal champions, argue that environmentalists in the government ``intrude'' on society by curtailing consumer choice and property rights~\cite{antonio2011unbearable,smith2024polarisation}.
Furthermore, concerning science and academia, in the past decade, there has been a sharp decline among Republicans of those who ``believe that colleges and universities have a ``positive effect'' on the country''~\cite{atkeson2019partisan}.
However, in our case study, we find that both the Republican (r/The\_Donald) and Democrat/Independent (r/SandersForPresident) community had a negligible number of links to scientific sources. 
The little scientific citation that does circulate in politically-oriented discussions may be influenced by the communiqu\'{e}s of NGOs, think tanks, and government reports (papers cited by such reports are more likely to be highly cited \cite{bornmann2022relevant}), each bringing its own agenda.
Conversely, the perception of a scientific source may be affected by the political stances of its editors \cite{zhang2023political}.
Instead, social media dominates these communities' links (21\% in posts on r/The\_Donald and 40\% on r/SandersForPresident).
As the major social media platforms have been thoroughly documented for spreading scientific misinformation \cite{osman2022youtube,do2022infodemics}, and some smaller ones boasting even more permissive policies \cite{bar2023new}, the extensive use of these as a resource for policy and science discussions is highly concerning and should be further investigated.

%\KK{here we close a bit roughly. do we want to add a phrase on the responsibility of scientists/journalists to communicate more or in a more accessible way? or something as an overall takeaway?}


% social media in r/The_Donald:
% POSTS: 21.07892%
% COMMENTS: 15.4368632%
% social media in r/SandersForPresident:
% POSTS: 39.9053627%
% COMMENTS: 20.52517866%

%At the user scale, we also find the center, center-left users to cite science more than those favoring sources from the political periphery. 

% In fact, some predict a more desirable``dysfunctional objectivity'' by individuals who are uninformed and politically apathetic \cite{taber2006motivated}.

% AI? Tools to help people understand science? Other platforms like video?



%%%%%%%%%%%%%%%%%%%%%%% NOTES %%%%%%%%%%%%%%%%%%%%%%%
%%%%%%%%%%%%%%%%%%%%%%% NOTES %%%%%%%%%%%%%%%%%%%%%%%

%%%%%%%%%%%%%%%%%%%%%%%%%%%%
%%%% Citing science is important for people taking action %%%%

% Message: knowing science is related with action? Find papers that link scientific knowledge with... political engagement? environmentalism? vegetarian? other actions?  Finding: pro-social behaviors that could be linked to knowing science: vaccination, recycling, 

% Do companies participate in this debate?

% \cite{zajas2021religiosity}:  "The highest correlation could be observed between religiosity, political conservatism, anti-evolutionism, and Tradition value preference [...] Conservatism was a positive predictor of climate change denialism" "political conservatism was a significant predictor of climate change skepticism "


%%%%%%%%%%%%%%%%%%%%%%%%%%%%
%%%% Can we use science to change people's minds? -- Answer probably no %%%% 

% \cite{verma2017human} -- People perceive science differently depending on whether there is a link to a scientific article (and whether they click on it) "Participants who clicked on hyperlinks to scientific articles placed higher trust in the claims than those who did not " 
% https://asistdl.onlinelibrary.wiley.com/doi/pdfdirect/10.1002/pra2.2017.14505401046

% "Users choose to engage with more partisan news than they are exposed to on Google Search" -- conclude: "exposure to and engagement with partisan or unreliable news on Google Search are driven not primarily by algorithmic curation but by users’ own choices"
% https://www.nature.com/articles/s41586-023-06078-5



%%%% Some notes that could be interesting for Discussion / Conclusion

%Maybe: US is a big contributor to climate change. US is politically influential. (We cannot distinguish users from US) Focus on English-speaking developed world? 
%\KK{i wouldn't frame it as a US issue.}

%It is well-known that the production of content, especially on social media, follows the Pareto principle, wherein a small number of contributors are responsible for the majority of content \cite{mustafaraj2011vocal}. Is the scientific citation more or less susceptible to this phenomenon? 

% For Conclusion? \textcolor{blue}{from \cite{cagle2019shades}: -see conclusion- "Climate change is one of the most pressing environmental challenges we face in the Anthropocene era; it alters the functioning of the Earth system and threatens human and nonhuman inhabitants of the planet. It is also one of the most charged political and cultural debates in America today, and presents a daunting challenge to environmental rhetoric."}

% r/science is huge: https://www.dailydot.com/debug/reddit-r-science-ama/


% IF WE HAD A LIMITATION SECTION
% Discuss that some of the posts and comments may not be relevant, especially those comments that did not match the keywords.


% IDEA: for the limitations see the limitation part of: https://arxiv.org/pdf/2401.13656 






%%%%%%%%%%%%%%%%%%%%%%%%%%%%%%%%%%%%%%%%%%%%%%%%%
\section{Methodology}
%%%%%%%%%%%%%%%%%%%%%%%%%%%%%%%%%%%%%%%%%%%%%%%%%

\subsection{Reddit Dataset Collection}

We choose a high-precision keyword-based approach to collect a dataset related to Climate Change. 
This method has been used extensively in the literature \cite{treen2022discussion,shah2021climate,parsa2022analyzing}, as it has been shown that conversation relevant to climate change happens in many subreddits, most of them not devoted exclusively to this topic. 
%\citep{shah2021climate} and \citep{villanueva2021climate} for example, show that this discussion is driven by political opinions (especially controversial ones) rather than by natural disasters, scientific, or health analysis.
We use the data collected by Pushshift via the Reddit API \cite{baumgartner2020pushshift} in the 168 months between January 2009 and December 2022. 
%Upon an initial examination of a data sample using the keywords "climate", "global warming", and  "carbon tax", we find that the word "climate" produces many false positives. 
Using manual examination of the dataset, we compose a set of 64 bigrams that maximize the coverage and minimize false positives (see Supplementary \ref{appendix_keywords}). 
%Initially, we built a low-precision submission dataset with only three basic keywords: "climate", "global warming", and  "carbon tax" by keeping the posts where these keywords have a match in the \emph{title} or in the \emph{content} attributes. After a manual evaluation of a random sample of 388 submissions (two for each month, from 2007-01 to 2023-02), we find that a part of this data is not relevant, i.e. \emph{false positive}, because it contains posts about other topics. For example, the word "climate" can be used when talking about travel, botany, and zoology, or also with a metaphoric meaning when talking about sentiments, politics, religion, or society. For this reason, it is necessary to filter this general dataset, selecting only the relevant bigrams in posts about climate change. Therefore, we use the pertinent submissions to produce a set of 64 specific keywords (see Appendix \ref{appendix_1}), and use it to build a high-precised dataset. \\
We first collect the posts and comments that contain at least one of these bigrams in the title (for posts) or the content (posts and comments). 
We then add to our dataset all comments to the selected posts regardless of their match to the keywords list.
The matching resulted in $1,301,970$ posts and $6,428,051$ comments, and an additional $15,273,754$ comments in response to the posts.
After removing duplicates, a total of $20,279,912$ comments remain in our dataset.
The volume of posting increases over time, and peaks around 2019-2020 (see Supplementary graph in \ref{appendix_volume}).

%This definitive dataset is, actually, the merging of three different datasets: first, we scrape the submission datasets as described before, then we collect two comments datasets. The first one contains all the comments posted as a response to the selected submissions (regardless of whether they contain the keywords, as done in \cite{treen2022discussion}). The second one is collected by using the same keywords approach as the submissions dataset, looking at the matches in the \emph{body} attribute. The two comments datasets are partially overlapping: in the analysis, we remove the duplicates.
%Therefore, we have two kinds of comments depending on their affiliation. We call \emph{comments-from-IDs} the ones collected by searching the responses to the posts (in fact, the metadata of each comment has the attribute "link\_id” whose characters, from the fourth one, match the ID of the submission the comment is a reply to). By looking at this attribute we can rebuild the "conversation tree" beneath each post. We call \emph{comments-from-keywords} the ones collected by looking at the matches with the keywords set. \\

To assess the relevance of the resulting dataset, all six authors, who are fluent in English, manually labeled a random sample of the submissions and comments that matched keywords by using three labels: "relevant", "partially relevant" and "non-relevant". 
After labeling 324 posts, we find that 84.9 \% were labeled as relevant, 10.5\% as partially relevant, and 4.6\% as non-relevant. 
The Cohen's kappa, computed on a sample of 60 posts, is $\kappa=0.55$.
% https://docs.google.com/spreadsheets/d/1KDC3MzrJVR6lnj7ROcUOWMwKH8RB9-JgKbIuf4M6u_Y/edit#gid=2045761730
% About only tweets with overlap: $75.0\%$ of the labelers agree on their vote: $68.3\%$ of the posts are considered relevant ($73.3\%$ if we consider the post with a "maybe" label as partially relevant). 
After labeling 384 comments that matched keywords, 79\% were judged to be relevant, 16\% as partially relevant, and 5\% as non-relevant.
The Cohen's kappa, computed on a sample of 66 comments, is $\kappa=0.50$.
%the agreement rate is $74.2\%$, and $69.7\%$ of the comments are considered relevant ($72.7\%$  if we add the maybes).
% https://docs.google.com/spreadsheets/d/186y_4X30La4MFLU6gI18GOZpGEJAr2E3kr8q-Le9uAg/edit#gid=2113514614
The comments in response to relevant posts were often too short and uninformative to be accurately labeled. We assume they are relevant in the discussion, because they are answers to posts mostly considered pertinent. 


% \textcolor{blue}{\emph{SHALL I ADD A DESCRIPTION? LIKE WHY DO WE HAVE THESE PEAKS IN THE RATES, MEANING CC IS MORE POPULAR ON REDDIT? MAYBE READ A SAMPLE OF POSTS/COMMENTS AND SEE WHAT THEY TALK ABOUT? OR ADDING THE RELEVANT EVENTS THAT HAPPENED IN THE WORLD, LIKE COP15 OR DONALD TRUMP ANNOUNCING USA WITHDREW FROM PARIS AGREEMENTS}}\\


%The number of posts and comments collected referring to the first years of the dataset, until the end of 2008, is very low. Reddit was not very popular in its first years: there are months having less than 10 climate change posts. For this reason, we decide to consider only the period of 168 months between 01-2009 and 12-2022. During this time, $1,924,739,791$ submissions have been written on Reddit, and $1,301,970$ are the selected ones about climate change ($0.0676\%$); $14,303,482,027$ is the whole number of comments, and, in particular, $15,273,754$ from IDs ($0.1068\%$) and $6,428,051$ from keywords ($0.0449\%$). The whole comment dataset, without duplicates, contains $20,279,912$ comments.\\








\subsection{Sources of Information}
\label{sec:infosources}

As the focus of this study is the citation of different kinds of information, and specifically science, we consider the URLs shared in the posts and comments that talk about climate change. 
We disregard URLs pointing to Reddit itself and resolve URLs to web.archive.org or archive.is --the two most popular archive services for web pages in our dataset.  
In our study, we consider only the subreddits that have shared at least 10 URLs in the 14 years of the dataset. 
The remaining 7,837 subreddits (12.19\% of the total) allow us to keep 778,728 (96.83\%) URLs shared in the posts, and 2,929,061 (99.00\%) URLs of the comments.

We then focus on the domains of the extracted URLs and define six categories characterizing them as a source of information: social media, newspapers, mass media, WikiMedia, governmental sources, and scientific sources (see Supplementary \ref{appendix_categories} for a summary of the categories).
We use both outside sources to create the lists of domains of interest, as well as examine the top 100 domains used in our dataset that are not a part of any list. 
By social media, we consider the six most popular ones in our dataset: Twitter, YouTube, Facebook, Instagram, LinkedIn, and Discord\footnote{`twitter.com', `youtube.com', `facebook.com', `instagram.com', `linkedin.com', `discord.gg', `discord.com'}.
We obtain the list of 4,898 newspapers (with their domains) from the media portal \emph{Scimago}\footnote{\url{https://www.scimagomedia.com/rankings.php}}, with the addition of Financial Times "ft.com" from the manual domain examination. 
The list of mass media is taken from the media ranking website AllSides\footnote{https://www.allsides.com/media-bias/ratings}, in particular by looking at four types of sources: News Media, Reference, Fact Check, and Think Tank/Policy Groups. We do not consider the individual authors. After removing domains already in other categories, the mass media list has 1,623 domains. We cleaned this list by removing the sources added to other categories, and added 9 domains from manual examination\footnote{`huffingtonpost.com', `businessinsider.com', `abc.net.au', `pbs.twimg.com',  `msn.com', `news.gallup.com', `nationalobserver.com', `ctvnews.ca', `oann.com'}. 
The WikiMedia category contains all the domains from the official webpage of the Wikimedia Foundation Project\footnote{https://wikimediafoundation.org/our-work/wikimedia-projects/\#a2-collections https://foundation.wikimedia.org/wiki/Home}, adding also ``upload.wikimedia.org", used for access to the media files. In total, there is an amount of 28 domains in this category. 
For the governmental sources, we consider the domains ending with ``.gov'', dropping the 4 that appear in the scientific journal list (`cdc.gov', `ehp.niehs.nih.gov', `nist.gov', `wwwnc.cdc.gov') and `eric.ed.gov' that is a preprint domain. In total, this category has 3,194 .gov domains. To this ones, we added 39 domains collected from the official UN website\footnote{https://www.un.org/en/about-us/un-system}. 

For the scientific sources, we make a distinction between four subcategories: magazines, journals, scientific news aggregators, and preprints. 

\begin{itemize}

    \item Magazines -- the mainstream scientific source of information written for the general non-expert public. We obtain the list of the most popular English-language magazines, with their websites, from Wikipedia\footnote{\url{https://en.wikipedia.org/wiki/Category:Science_and_technology_magazines_by_country}} and manual research on the Web. After removing the few peer-reviewed ones (because they are considered journals), we collected 185 magazine URLs.

    \item Journals -- a peer-reviewed publication, written by and for experts. We obtain a list of journals by scraping the platform \emph{Web of Science}\footnote{\url{https://wosjournal.com/}}. After removing the journals without a URL (generally, they have only the link to the publisher, which could be misleading), we manually added different variations of domains, resulting in 1943 different domains.

    \item Scientific news aggregators -- web applications that aggregate scientific or technological content from different sources, which are not necessarily peer-reviewed. After manual research on the Web, we collect 5: ``sciencedaily.com", ``phys.org", ``eurekalert.org", ``esciencenews.com" and ``researchgate.net". 

    \item Preprints -- scientific papers published before the peer-review process. We scrape their directories from the \emph{Directory of Open Access Preprint Repositories} webpage\footnote{\url{https://doapr.coar-repositories.org/repositories/}}, collecting 83 preprint domains.
\end{itemize}

We make the full list of domains and their categories available to the research community\footnote{\url{https://anonymous.4open.science/r/domains_types-31D5/README.md}}. \\

To supplement our understanding of the quality of the domains, we use previous literature and reputable sources to create a list of unreliable domains.
For this purpose, we use the Wikipedia Lists of fake news websites\footnote{\url{https://en.wikipedia.org/wiki/List\_of\_fake\_news\_websites\#Lists}\\ \url{https://en.wikipedia.org/wiki/List\_of\_miscellaneous\_fake\_news\_websites}}, Media Bias Fact-Check (MBFC) lists of conspiracy and fake news webistes\footnote{\url{https://mediabiasfactcheck.com/conspiracy/}\\  \url{https://mediabiasfactcheck.com/fake-news/}}
and previous literature on climate change on Reddit \cite{gadanidis2020discourse}. 
Additionally, we use the media ranking website All Sides political bias domain labels, merging ``right'' and ``center right'' into ``right'' labels and similarly for ``left".

Finally, we enrich the domain list with (US-centric) political leaning information from Allsides that provides five labels: left, left-center, center, right-center, and right. 
For computing statistics, we merge ``right'' and ``center right'' into ``right'' labels and similarly for ``left".
On the other hand, when we compute a political bias score for each subreddit based on the URLs appearing in its posts and comments, we assign numerical values to these labels from -2 to 2 (from ``left'' to ``right'') and average these scores for each subreddit. 
We perform the same computation for the users to summarize the political leaning of the URLs they have shared in our dataset. 
To avoid noise due to sparsity (wherein not enough URLs were posted by each user), we examine the distribution of these scores, compute the Jansen-Shannon distance between the consecutive distributions, and determine the cutoff of 5 URLs with bias labels, such that the distribution is stable (see Supplementary \ref{appendix_userbias}).

\subsection{Modeling Scientific URL Use}

We compute the political bias score for the users in a similar way we have for the subreddits and we keep only the users that shared at least 5 biased links (see appendix \ref{appendix_userbias} for details). Therefore, we consider 26,620 users. 
In order to find the most relevant attributes related to the sharing of scientific links, we decided to build an explanatory model by focusing on the different number of categories of domains and on the top 100 subreddits (by the number of links) in our dataset. The remaining subreddits are placed in the "other subreddits" variable. We remove every scientific reference in the design matrix, both in the politically biased links (some scientific sources have a "central" bias) and in the number of URLs shared on the subreddits.
After shuffling the data, we take the logarithm of each of these numeric values (to which 1 was previously added~\cite{pither2023introduction}, to compute the logarithm of the zeros) because the data is highly skewed/asymmetrical. and run a Random Forest model having the number of scientific links as the dependent variable. We use 3-fold cross-validation to find the best values of the hyperparameters (number of trees and their depth) obtaining an average score (i.e. mean accuracy) of 0.63. Finally, we explain the model with SHAP\footnote{\url{https://shap.readthedocs.io/en/latest/index.html}} (SHapley Additive exPlanations) a method that uses the Shapley values from cooperative game theory to explain how the coefficients of the model interact with the output. 
%Similar results were found using the Poisson GLM, which we provide in the Supplementary section \ref{appendix_poissonmodel}.


\subsection{Conditional Probability of URL in Response}

To better understand how the different URL categories are used in response to potentially politically biased content, we compute the conditional probabilities as follows.
Given a post with a URL of a particular category, we compute the conditional probability that a URL of another category is used in a first-level comment to that post. 
Note that, for this computation, we consider all posts that have at least one URL, and all first-level comments to them that have at least one URL. 



%%%%%%%%%%%%%%%%%%%%%%%%%%%%%%%%%%%%%%%%%%%%%%%%%
\bibliographystyle{ACM-Reference-Format}
\bibliography{referencesM,references}
%%%%%%%%%%%%%%%%%%%%%%%%%%%%%%%%%%%%%%%%%%%%%%%%%




\appendix



\pagebreak
\section{Supplementary Material}
\label{supplementary}
% \section{Appendix}
% \label{appendix}




\subsection{Keywords Regarding Climate Change}
\label{appendix_keywords}

"climate change", "global warming",  "climate crisis", "climate action", "climate strike", "carbon tax", "climate accord", "climate activist", "climate agenda", "climate agreement", "climate alarmism", "climategate", "climate anxiety", "changing climate", "climate camp", "climate catastrophe", "climate chaos", "climate commitment", "climate concern", "climate conference", "climate consensus", "climate cooperation",  "climate council", "climate crises", "climate deal", "climate debate", "climate denial", "climate denier", "climate disaster", "climate dissent", "climate emergency", "climate fight", "climate goal", "climate impact", "climate issue", "climate justice", "climate march", "climate meeting", "climate migrant", "climate model", "climate movement", "climate myth", "climate pact", "climate panel", "climate plan", "climate policies", "climate policy", "climate-change", "climate protection", "climate protest", "climate rebellion", "global heating", "climate report", "climate research", "climate risk", "climate scam", "climate sceptic", "climate science", "climate scientist", "climate skeptic", "climate summit", "climate talk", "climate threat", "climatechange"



\subsection{URL domain categories and their sources}
\label{appendix_categories}

\begin{table}[H]
  \centering
  \renewcommand{\arraystretch}{1.2}
  \begin{NiceTabular}{|p{5cm}|c|c|}
    \hline
\textbf{Domain category}  & \textbf{No. domains} & \textbf{Sources}\\ \hline\hline   
Social Media & 7 & Extracted from data
\\ \hline 
Wikimedia Foundation Project & 28 & Wikimedia Project 
\\ \hline 
Newspapers & 4898 & Scimago Media Rankings 
\\ \hline 
Mass Media & 1621 & AllSides and Extracted from data 
\\ \hline 
Governmental: \emph{.gov} domains & 3194 & Extracted from data
\\ \hline 
Governmental: United Nations System & 39 &  United Nations Webpage
\\ \hline 
Scientific Aggregators & 5 & Wikipedia and extracted from data
\\ \hline 
Scientific Magazines & 185 & Wikipedia
\\ \hline 
Scientific Journals & 1943 & Web of Science 
\\ \hline 
Scientific Preprints & 83 &  Directory of Open Access Preprint Repositories 
\\ \hline\hline
Unreliable & 2469  & Wikipedia, Media Bias/Fact Check
\\ \hline 
Right political leaning &  159 & AllSides
\\ \hline 
Left political leaning & 322  & AllSides
\\ \hline 

\end{NiceTabular}
\caption{Summary of the domain categories and their sources. Those above double-line are mutually exclusive, whereas those below (unreliable and political leanings) are not.} 
\label{tab:summary_labels}
\end{table}



\subsection{Volume of Reddit Posts and Comments around Climate Change}
\label{appendix_volume}

Figures \ref{fig:plots_dataset} (a, b) show the temporal statistics of the dataset (both posts and comments), aggregated by month. 
Figure \ref{fig:plots_dataset} (a) shows the raw number of posts and comments, and (b) shows the proportion of these posts and comments of the whole Reddit posting volume.
Although we find that the volume of posts and comments increases over time (peaking especially in 2019-2020), as a share of the overall Reddit posting volume, the years 2009, 2017, and 2019 show an especially higher focus on the topic of the overall Reddit activity.

\begin{figure}[t]  %ht
    \centering
    \includegraphics[width=0.9\textwidth,keepaspectratio]{Figures/plots_dataset.pdf}
    \caption{Temporal statistics of the dataset (both posts and comments), aggregated by month.}
    \label{fig:plots_dataset}
\end{figure}



\subsection{Top Users Posting Scientific Links}
\label{appendix_topusers}

Turning to users who have contributed the most number of scientific links, at the top we find \emph{ILikeNeurons} (\num{15685}), 
\emph{worldnews} (\num{20451}), 
\emph{BurnerAcc2020} (\num{3254}), 
\emph{AutoModerator} (\num{2962}), 
\emph{ZephirAWT} (\num{2926}), 
\emph{Jaagsiekte} (\num{2412}), 
\emph{EcoInternetNewsfeed} (\num{3382}), 
\emph{MmmBaconBot} (\num{3382}), 
\emph{kamjaxx} (\num{1093}), and
\emph{avogadros\_number} (\num{1684}).
These top 10 users contribute 16.6\% of all scientific URLs in our data.
%We find many accounts that are explicitly bots, such as \emph{AutoModerator}, \emph{EcoInternetNewsfeed}, and \emph{MmmBaconBot}. However, from the behavior of the other accounts, it is likely that they are also at least partially automated, such as the \emph{ILikeNeurons}. 



\subsection{Link-based Political Leaning of Users}
\label{appendix_userbias}

We use information about the domains to which each user has linked to in their posts and comments as indicators of the sources of information they favor to compute a user-specific political leaning score.
To each link we assign a numerical political bias score  mapped to AllSides from -2 to 2 (from ``left'' to ``right'').
We then average the scores of all links a user has posted.
To gauge the sparsity of the dataset, we plot the distribution of the user scores in Figure \ref{fig:threshold_biased_users} and compute the Jensen-Shannon Distance between couples of consecutive distributions. 
We choose as the threshold 5, as the first time the Jensen-Shannon Distance between the distributions is lower than 0.2 is between thresholds 5 and 6. 
Thus, for further analysis we consider only the users that shared at least 5 URLs with a known bias score, resulting in \num{24453}, which is 14.6\% of the users in the whole dataset.

\begin{figure}[H]  %ht
    \centering
    \includegraphics[width=0.86\textwidth,keepaspectratio]{Figures/threshold_biased_users.pdf}
    \caption{\textbf{Distributions of the users' political score with different thresholds of the number of biased links shared}}
    \label{fig:threshold_biased_users}
\end{figure}


% Jensen-Shannon Distance between 1 and 2:    0.27815437870992576
% Jensen-Shannon Distance between 2 and 3:    0.23373420588875704
% Jensen-Shannon Distance between 3 and 4:    0.21335258258940948
% Jensen-Shannon Distance between 4 and 5:    0.20371059775094005
% Jensen-Shannon Distance between 5 and 6:    0.19921039380853595
% Jensen-Shannon Distance between 6 and 7:    0.19503736308942815
% Jensen-Shannon Distance between 7 and 8:    0.19389433733581637
% Jensen-Shannon Distance between 8 and 9:    0.19393540319942026
% Jensen-Shannon Distance between 9 and 10:    0.19082127444607533

%threshold 5 = 14.596453109051078\% = 24453 users


% \subsection{Poisson model of the users}
% \label{appendix_poissonmodel}

% In addition to the Random Forest model presented above, we run a Poisson GLM that predicts the number of scientific links a user posts in our dataset. 

% \begin{figure}[b]  %ht
%     \centering
%     \includegraphics[width=0.75\textwidth,keepaspectratio]{Figures/poisson_model.pdf}
%     \caption{\textbf{Poisson model}}
%     \label{fig:poisson_model}
% \end{figure}

% run with 
% "import statsmodels.api as sm
% poisson\_training\_results = sm.GLM(y, x, family=sm.families.Poisson()).fit()"
% keep only features with a p-value < 0.001/len(x.columns). Bonferroni correction\\

\subsection{URL Sharing Behavior of Users with Political Leaning}
\label{appendix_sharing_behavior_users}

\begin{table}[h]
\caption{Mean, median, and standard deviation of the proportion of URLs of a certain category posted by users having a specific political leaning (identified via other URLs they have shared).}
\label{tab:sharing_behavior_users}
\centering
\begin{NiceTabular}{|c|ccc|ccc|ccc|ccc|}
\hline
%\toprule

\tabhead{} & \multicolumn{3}{c|}{\tabhead{Left}} & \multicolumn{3}{c|}{\tabhead{Center-left}} & \multicolumn{3}{c|}{\tabhead{Center-right}} & \multicolumn{3}{c|}{\tabhead{Right}} \\
\cmidrule(lr){2-4} \cmidrule(lr){5-7} \cmidrule(lr){8-10} \cmidrule(lr){11-13}
\tabhead{} & \textbf{Mean} & \textbf{Med.} & \textbf{Std} & \textbf{Mean} & \textbf{Med.} & \textbf{Std} & \textbf{Mean} & \textbf{Med.} & \textbf{Std} & \textbf{Mean} & \textbf{Med.} & \textbf{Std} \\
\hline\hline 
Science &  2.24 & 0 & 11.31 &
7.45 & 1 & 133.36 &
3.35 & 0 & 12.82 & 
0.80 & 0 & 5.02 \\
Social media &  4.59 & 1 & 30.38 &
7.72 & 1 & 192.53  &
4.76 & 0 & 35.17 &
9.34 & 0 & 103.54 \\
Mass media &  16.90 & 6 & 74.50 & 
22.63 & 6 & 316.76 &
12.72 & 6 & 39.23 &
30.04 & 7 & 145.37\\
Newspapers & 7.06 & 3 & 35.08 & 
15.06 & 3 & 265.84 &
4.61 & 2 & 13.72 & 
3.79 & 1 & 14.85   \\
Governmental &  1.45 & 0 & 6.00 & 
4.35 & 1 &  65.97 &
2.07 & 0 & 8.03 &
0.53 & 0 & 1.95 \\
Wikimedia & 2.57 & 0 & 9.42 & 
5.10 & 1 & 69.13 &
2.38 & 0 & 8.30 &
0.92 & 0 & 4.95 \\
Unreliable &  0.43 & 0 & 4.84 & 
1.56 & 0 & 33.50 &
7.12 & 1 & 51.35 &
27.41 & 5 & 132.50\\
\hline
\end{NiceTabular}
\end{table}







\subsection{Subreddits in Case Study}
\label{appendix_casestudy}

We chose a selection of subreddits to explore in more depth out of those which have contributed the most posts to our dataset. 
For each subreddit, the table below shows its description (from Reddit), its membership in terms of users, number of posts and comments in our dataset, and the number of URLs present in these.
All statistics were collected in April 2024, except those for The\_Donald, which was banned at the time, for which its Wikipedia page was used\footnote{https://en.wikipedia.org/wiki/R/The\_Donald}.

\begin{table}[H]
\caption{Statistics on subreddits considered in the case study.}
\label{tab:special_subs}
\centering
\begin{NiceTabular}{|p{7cm}|c|c|c|}
\hline 
%\toprule
\tabhead{Subreddit, Description} & \tabhead{Members} & \tabhead{No. Posts \& Comments} & \tabhead{No. URLs} \\
\midrule \hline 

\textbf{r/climate}: Information about the world's climate. Truthful and accurate information about the world's climate, as well as related activism and politics & \num{187919} & \num{175833} & \num{73047}  \\ \hline 
\textbf{r/climateskeptics}: Climate Skeptics: Trying to see through the alarmism. Questioning climate related environmentalism & \num{43440} & \num{281120} & \num{84897}  \\ \hline 
\textbf{r/science}: This community is a place to share and discuss new scientific research. Read about the latest advances in astronomy, biology, medicine, physics, social science, and more. Find and submit new publications and popular science coverage of current research & 32M & \num{591697} & \num{72873}  \\ \hline 
\textbf{r/worldnews}: place for major news from around the world, excluding US-internal news & 36M & \num{2123981} & \num{282328}  \\ \hline 
\textbf{r/politics}: is for news and discussion about U.S. politics & 8.5M & \num{633062} & \num{256341}  \\ \hline 
\textbf{r/The\_Donald}: is a never-ending rally dedicated to the 45th President of the United States, Donald J. Trump  & max \num{790000}  & \num{212769} & \num{38160}  \\ \hline 
\textbf{r/SandersForPresident}: Bernie Sanders 2024 & \num{506918} & \num{281120} & \num{84897} \\ \hline 
%\bottomrule\\
\end{NiceTabular}
\end{table}



\end{document}
