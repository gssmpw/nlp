\section{Understanding Parental Homework Involvement in China}

To design technologies for improving parental practices, it's crucial to understand parental involvement in homework scenarios. In this research, we conducted a 4-week real-world data collection, recording parent-child conversations during homework sessions. 
%This approach provides detailed insights necessary for designing technologies for improving parenting strategies. %Specifically, we aim to answer the following questions: \textit{1. What does parental homework involvement in China entail, and do parents experience emotional changes post-homework?}
%\textit{2. What types of parental behaviours and parent-child conflicts arise during homework involvement in China?}
%\textit{3. How are parents' emotions, behaviours, and conflicts interconnected during homework involvement?} 
The data collection was approved by the Human Research Ethics Committee at our university.

\begin{table}

%\begin{minipage}[]{6cm}
\caption{Demographics of parent participants}
\label{tab:par_parent}
\begin{tabular}{@{}llll@{}}
\toprule
\textbf{Item}               & \textbf{Option    }                             & \textbf{Count} & \textbf{Percentage} \\ \midrule
\multirow{2}{*}{\textit{Gender}}    & Female (mother)                        & 68    & 87.18\%    \\
                           & Male (father)                          & 10    & 12.82\%    \\\hline
\multirow{4}{*}{\textit{Age}}       & $\leq$ 30                                    & 5     & 6.41\%     \\
                           & 31-40                                  & 48    & 61.54\%    \\
                           & 41-50                                  & 25    & 32.05\%    \\
                           & $\geq$50                                    & 0     & 0.00\%     \\\hline
\multirow{5}{*}{\begin{tabular}[c]{@{}l@{}}\textit{Education} \\ \textit{Level}\end{tabular}} & Junior high school and below           & 0     & 0          \\
                           & High school/Secondary school & 4     & 5.13\%     \\
                           & Associate degree (Junior college)      & 7     & 8.97\%     \\
                           & Bachelor's degree                      & 45    & 57.69\%    \\
                           & Postgraduate degree and above          & 22    & 28.21\%    \\ \bottomrule
\end{tabular}
%\end{minipage}

\iffalse
\hspace{2cm}
\begin{minipage}[]{6cm}
\caption{Basic information of the children of parent participants}
\label{tab:par_child}
\begin{tabular}{@{}llll@{}}
\toprule
\textbf{Item}               & \textbf{Option}            & \textbf{Ct.} & \textbf{Percent.} \\ \midrule
\multirow{2}{*}{\textit{Gender}}& Female (daughter) & 38    & 48.72   \\
& Male (son)        & 40    & 51.28\% \\\hline
\multirow{3}{*}{\textit{Grade}} & Year 1            & 31    & 39.74\% \\
& Year 2            & 27    & 34.62\% \\
& Year 3            & 20    & 25.64\% \\\hline
\multirow{5}{*}{\begin{tabular}[c]{@{}l@{}}\textit{Academic} \\ \textit{Rank in} \\ \textit{Class}\end{tabular}} & Bottom            & 1     & 1.28\%  \\
& Below Average     & 9     & 11.54\% \\
& Average           & 13    & 16.67\% \\
& Above Average     & 35    & 44.87\% \\
& Top               & 20    & 25.64\% \\ \bottomrule 
\end{tabular}

\end{minipage}
\fi
\end{table}

\subsection{Participants}
\label{subsec: participants}
Participants were recruited based on specific criteria: they had to be parents of primary school students in Years 1-3, who were regularly assisted with homework and communicated in Mandarin at home. We distributed advertising leaflets and posters through various channels, including word of mouth, social platforms like WeChat, and school principals who facilitated further distribution through teachers. 

Within a week of registration, a total of 121 individuals completed the registration questionnaire and provided consent for data collection. Among them, 106 met the eligibility criteria, with 15 participants excluded due to reasons such as their children being in Year 4, the involvement of home-based teachers rather than parents, or infrequent homework involvement (defined as one or fewer instances per week). Out of the 106 eligible participants, 85 were successfully contacted; however, seven of them withdrew without contributing data.

The final sample consisted of 78 Chinese parents, with only one parent from each family included.  Detailed demographic information is provided in Table \ref{tab:par_parent}. Notably, the majority of participants (87.18\%) were mothers. Additionally, the education level of the participants was generally higher than the national average in China \cite{gps}. This may be attributed to two primary factors: 1) more educated parents might be more motivated to participate to obtain a family education report as part of the compensation; 2) these parents likely had better access to our advertisements through social media and other channels. %Regarding the children, their academic rankings, as reported by parents, were not evenly distributed, with most being described as \textit{``above-average''}. This imbalance suggests a sampling bias, where parents of higher-achieving students might be more invested in education and thus more inclined to participate. 
We acknowledge this limitation and address it further in Section \ref{sec:limit}.

\subsection{Procedure}

Before formal data collection began, participants completed a registration questionnaire that gathered basic demographic information, details on their habits of parental home involvement, and their satisfaction and emotions related to these activities. The insights gained from this preliminary stage informed the design of daily surveys. 

During the data collection, participants first completed an online background survey\footnote{The first background survey data was not used in this research.}. They were then instructed to record audio during homework sessions with their children based on their usual routines (not necessarily daily). To minimize the \textit{Hawthorne Effect} \cite{adair1984hawthorne} — where individuals alter their behaviour due to awareness of being observed - participants were encouraged to act naturally and were assured that their privacy would be strictly protected, with data used solely for research purposes.


After recording, participants could upload their audio via a private link or send it to our assistant, who stored it on secured hardware. On days when audio was recorded, participants were asked to complete a daily online questionnaire immediately after the homework session. Each parent could submit up to 10 sets of audio recordings and corresponding daily questionnaires over the four weeks. This flexibility accommodated variations in the frequency of homework involvement in different families, with some parents participating daily and others two or three times per week. Additionally, participants were asked to complete a second background survey\footnote{The second background survey data was not used in this research.} at the end of the first week. To minimize the burden, parents used their own smartphones or preferred devices to record the entire homework process. Although some parents may check in only at the end of the session, we still require recordings from start to finish to fully understand when and how parents intervene during homework sessions. 

%Participants could upload audio via a private link or send it to our assistant, who stored it on secured hardware. On days when audio was recorded, participants completed a daily online questionnaire immediately after the session. Each parent could submit up to 10 sets of audio recordings and daily questionnaires over the four-week period, allowing for variation in homework involvement frequency, with some parents participating daily and others two to three times per week. Additionally, participants completed a second background survey\footnote{The second background survey data was not used in this research.} at the end of the first week. To minimize burden, parents used their own devices to record the entire homework process, even if they usually only check in at the end, allowing us to capture how and when parents intervene during homework sessions.

\begin{figure}
    \centering
    \begin{minipage}[t][6cm]{0.45\textwidth}
        \includegraphics[width=0.95\textwidth]{figure/dis_length.pdf}
        \caption{The distribution of audio recording durations}
        \label{fig:dis_length}
    \end{minipage}
    \hspace{0.2cm}
    \begin{minipage}[t][6cm]{0.45\textwidth}
    \centering
        \includegraphics[width=0.94\textwidth]{figure/dis_count.pdf}
        \caption{Number of audio recordings for different participants}
        \label{fig:dis_count}
    \end{minipage}
    
\end{figure}
Participants were compensated up to RMB 240 for their involvement (RMB 40 for the background survey and RMB 20 for each audio recording and its corresponding daily survey, up to a maximum of ten sets). Additionally, those who provided complete data would receive a family education report with tailored suggestions from our team of family education specialists. This incentive was designed, in part, to encourage parents to behave as naturally as they would at home. It is worth noting that the participation in this research project is entirely voluntary. Participants were free to withdraw from the study at any time. To ensure privacy privacy, all participants were anonymized and assigned a unique ID during the data collection.

%It is worth noting that participation in this research project isvoluntary. Participants are free to withdraw from the project at any stage if they change their minds. Besides, we anonymized all the participants to protect their privacy





\subsection{Collected Data}

We gathered data from several sources: registration questionnaires, background surveys, audio recordings of parental homework involvement, and daily surveys. We collected 121 registration questionnaires and 78 background surveys, but only used the data for participants who completed both. We also collected 602 valid audio recordings and 620 daily surveys. 

For the record data, while we only require one complete audio file, some parents accidentally produce multiple recordings due to phone calls or pressing the wrong button. These recordings, which constitute about 10.43\% (66 out of 633 homework sessions) of our data, are merged into a single file before processing, leaving out 603 audio files. After removing one corrupted file, we collected 602 valid audio files. The total duration of these recordings is 474.89 hours, with an average length of 47.33 minutes per audio. Figure \ref{fig:dis_length} shows the distribution of audio recording durations, while Figure \ref{fig:dis_count} illustrates the number of recordings contributed by different participants. On average, 24 recordings were submitted daily, with 66 parents successfully contributing at least one recording. 

For the daily survey\footnote{Additional data, such as child behaviour evaluations, parental behaviours, and daily stress, were collected but not utilized in this research.}, it assessed the parents' affective reactions before and after homework involvement, focusing on pleasure, arousal, and dominance, using the \textit{Self-Assessment Manikin} (SAM) \cite{bradley1994measuring} and \textit{PAD} emotion state model \cite{mehrabian1974approach}. Figure \ref{fig:self-report dis} presents the parents' perceived emotions before and after homework, with pleasure (1-5) ranging from extremely unhappy to extremely happy, arousal (1-5) from completely calm to highly aroused, and dominance (1-5) from highly submissive to highly dominant.

\begin{figure}
    \subfigure[ {Pleasure} \label{subfig: age}]{\includegraphics[width=0.285\textwidth]{figure/dis_valence_no.pdf}}
	\hspace{0cm}
    \subfigure[ {Arousal}\label{subfig: venn}]{\includegraphics[width=0.285\textwidth]{figure/dis_arousal_no.pdf}}
    \subfigure[ {Dominance}\label{subfig: venn}]{\includegraphics[width=0.405\textwidth]{figure/dis_dominance.pdf}}
    \caption{Distribution of parental emotions before and after homework involvement}
    \label{fig:self-report dis}
\end{figure}

%To capture the overall trend and the diversity of emotional responses among parents, we conducted both group-level and individual-level analyse to thoroughly understand the emotional experiences of parents before and after homework involvement, based on self-reported data. First, we calculated the mean emotional values for each parent, both before and after homework. Then, we conducted a paired sample t-test on these group mean values, comparing the pre-homework means to the post-homework means. We observed statistically significant differences in the mean values across all dimensions, with pleasure (p < 0.001), arousal (p < 0.001), and dominance (p < 0.001) all showing significant changes. It revealed whether parents, as a group, experienced a statistically significant emotional shift due to homework involvement. Next, we examined whether individual parents showed significant changes in their emotions before and after homework. This allowed us to identify those parents who experienced notable emotional shifts, providing insight into the variability of responses.

\subsection{Data Processing}
\subsubsection{Audio Preprocessing}
The audio preprocessing followed three key steps:
(1) \textit{Conversion to WAV Format}. 
%All audio files were converted to WAV, a lossless format that preserves the full quality of the recordings. This was essential to ensure no audio information was lost, as maintaining high fidelity was crucial for accurate analysis. 
All audio files were converted to lossless WAV to preserve full recording quality, ensuring no information was lost for accurate analysis.
(2) \textit{Resampling}. 
%The audio was resampled to a uniform rate of 16 kHz, a standard in speech processing. This rate strikes a balance between maintaining sufficient detail for accurate speech recognition and managing computational efficiency.
Audio was resampled to 16 kHz, balancing sufficient detail for speech recognition with computational efficiency.
(3) \textit{Normalization}.  Audio levels were normalized to ensure consistent volume across recordings, preventing bias during feature extraction. Noise removal was deliberately avoided after preliminary tests showed it could distort or remove crucial elements, particularly the child’s voice.



\subsubsection{Transcription}

Transcription was performed on the normalized audio files, ensuring pauses and silence were maintained to capture the natural flow of conversation. We employed the Xunfei API \cite{xunfei} for automatic transcription, with a "2-second rule" to accurately segment utterances during pauses. Given the informal nature of parent-child conversations, automatic transcription systems often produce errors such as homophone confusion, misinterpretations, and segmentation issues. To address this, we used a custom-designed prompt (see Appendix \ref{app:prompt_transcription}) for the GPT-4o model to assist in error correction. This prompt guided the model in identifying and fixing errors while maintaining the natural conversational tone. The key principles were minimal intervention, preserving the dialogue's original flow, and ensuring clarity and accuracy in the final transcription.
%Transcription was conducted on the normalized audio files, keeping periods of silence intact to accurately capture the start and end of speech segments. We used the Xunfei API for automatic transcription, applying a "2-second rule" to distinguish between different utterances, ensuring that pauses did not incorrectly segment the conversation. 
%Accurate transcription of parent-child conversations is crucial for the reliability of our study's analysis. Automatic Speech Recognition (ASR) systems, such as the one used, often introduce errors in informal speech contexts, including homophones, phonetic misinterpretations, omissions, additions, and incorrect word boundaries. To mitigate these issues, we employed a custom-designed prompt for the GPT-4o model to assist in error correction. The prompt was carefully crafted to guide the model in correcting transcription errors while preserving the natural conversational style of the dialogue. It provided a detailed structure for input data, specified the desired output format, and included step-by-step instructions for identifying and correcting errors. Key principles emphasized in the prompt were minimal intervention, maintaining the original conversational tone, and ensuring clarity and accuracy in the corrected text.

\subsubsection{Role Recognition} \label{subsec: roles}
The Xunfei API could identify multiple speakers but labeled them generically (e.g., "Speaker 1"). This posed challenges in identifying specific roles, such as distinguishing between parent and child or between different speakers assigned the same label. To resolve this, we designed a model-specific prompt (see Appendix \ref{app: prompt_role recognition}) to clarify speaker roles. The prompt used contextual cues and conversational patterns to assign accurate roles to each speaker. It also addressed cases where speakers with the same label might actually represent different individuals, ensuring accurate identification of all participants.

%The Xunfei API used in transcription could identify multiple speakers but only labeled them generically as "Speaker 1," "Speaker 2," etc., without specifying their identities or roles. This presented two challenges: (1) determining which speaker corresponded to which role (e.g., parent or child) and (2) identifying whether multiple speakers of the same label belonged to the same role or different ones. To address these challenges, we developed a specific prompt to clarify speaker roles post-transcription. The prompt was designed to instruct the model to distinguish between different speakers based on contextual cues and conversational patterns, ensuring accurate role identification. The prompt also guided the model in determining whether speakers labeled identically by the API actually represented different roles.

%\tobe{Expert Labeling: We selected 30 audio recordings for a thorough evaluation. Two experts independently labeled the roles based on the transcription. If their labels matched, the roles were confirmed. In cases of disagreement, the experts listened to the audio to resolve the discrepancies and accurately define the roles. These expert-labeled roles were considered the ground truth. Model Testing: For each of the 30 audio recordings, the role recognition prompt was run 10 times to assess consistency and accuracy. The results were compared against the expert-labeled ground truth to evaluate the effectiveness of the role recognition process.}
