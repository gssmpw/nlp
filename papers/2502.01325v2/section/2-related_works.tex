\section{Related Works}

\subsection{Parental Homework Involvement in Education}

Parental homework involvement is commonly defined as \textit{parents' monitoring, supervision, and participation in their children's schoolwork and academic performance} \cite{pomerantz2007whom}. Research has shown that such involvement plays a crucial role in children's academic success, motivation, and well-being \cite{patall2008parent, dettmers2019antecedents, cooper1989synthesis}. While much of the literature focuses on Western contexts, studies in China highlight unique dynamics. Chinese parents often engage more directly, employing motivational strategies such as reasoning, monitoring, and even criticism to ensure academic success \cite{kim2013parents}. These behaviours align with broader cultural expectations in China, where academic achievement is highly valued, and parents feel a strong responsibility to support their children's education.

Some studies have also examined how Chinese parents adapted their involvement during the COVID-19 pandemic, where concerns over online education led to increased parental engagement \cite{wang2021parental}. Research has identified various types of parental involvement in Chinese families, ranging from supportive to disengaged, with the former most closely linked to academic success \cite{gan2019parental}. %These findings underscore the importance of understanding homework involvement in non-Western contexts.
%Further research has explored how Chinese parents adapted their involvement to specific contexts. For instance, Wang et al. \cite{wang2021parental} examined parental homework involvement during the COVID-19 pandemic, finding that Chinese parents, particularly mothers, became even more engaged in their children's learning due to concerns over the efficacy of online education. Similarly, Gan et al. \cite{gan2019parental} identified different types of parental involvement in Chinese families, ranging from supportive to disengaged, with supportive involvement being the most closely linked to academic success. 
These findings highlight the importance of understanding homework involvement in non-Western contexts.% how Chinese parents navigate the complexities of homework involvement in a rapidly changing educational landscape.

%Additionally, Lau et al. \cite{lau2011parental} and Liu et al. \cite{liu2019parental} examined parental involvement in younger children and its long-term impacts on academic readiness and emotional regulation. Their findings emphasize that early parental support in homework-related activities significantly influences children's long-term academic self-efficacy and emotional well-being, laying the groundwork for continued parental engagement during later school years.

The study of parental homework involvement typically relied on self-report surveys  (e.g., EMBU \cite{arrindell1999development}, QPH \cite{dumont2014quality}), interviews, and observations. While these methods provide valuable insights, they often fail to capture the nuanced emotions and behaviours that occur during real-world homework involvement, where conflicts may arise that are not evident in the presence of a human observer (due to the \textit{Hawthorne Effect} \cite{adair1984hawthorne}). Additionally, reliance on self-reported data introduces bias and may not fully reflect the subtleties of everyday involvement. For example, high-level categorizations of involvement, such as Gan et al.'s four types of involvement \cite{gan2019parental}, may miss the nuanced ways these interactions occur in day-to-day life.


\subsection{Emotional Experiences and Parent-child Conflicts During Homework Involvement}


While parental homework involvement is generally associated with positive educational outcomes, it can also lead to emotional strain and conflicts within families. Nnamani et al. \cite{nnamani2020impact} found that although parental involvement positively impacts students' emotional adjustment and academic performance, the emotional burden on parents often goes unnoticed. This emotional toll is particularly evident in cultures where academic success is strongly emphasized, as is the case in China. Kim et al. \cite{kim2020dyadic} developing Dyadic Mirror, a wearable smart mirror that provides parents with a second-person live-view of their own expressions as seen by their child during face-to-face interactions. Studies have shown that parents experience tension when balancing the desire to foster autonomy with the need to control their children's learning \cite{cunha2015parents}. The emotional stress parents feel during homework sessions can negatively affect children, creating a feedback loop of stress and conflict \cite{moe2018brief}. This dynamic is particularly significant in cultures where academic achievement is heavily emphasized, as in China \cite{nnamani2020impact}.


Homework involvement can exacerbate family conflicts, especially during adolescence, as parents strive to ensure academic success. Solomon et al. \cite{solomon2002helping} explored how homework involvement can become a source of conflict and found that the pressure parents feel to ensure their children's academic success can exacerbate tensions, often turning homework sessions into battlegrounds where unresolved issues about control and expectations surface. This finding is echoed by Suarez et al. \cite{suarez2022parental}, who reported high levels of family conflict and stress during the COVID-19 pandemic, a time when parental involvement in homework increased dramatically due to school closures and remote learning. 

Above findings underscore the complexity of parental homework involvement, where well-intentioned efforts to support academic achievement can inadvertently result in emotional strain and conflict. Our research aims to further unpack these emotional dynamics and explore how they are intertwined with parental behaviours and conflicts during homework involvement in Chinese families.


\subsection{Technology Supported Parent-Child Interaction}

Technological interventions in HCI have demonstrated the potential to enhance parent-child interactions in educational settings. Liu et al. \cite{liu2024he} explored the use of image-based generative AI in family expressive arts therapy. Fan et al. \cite{fan2019character} developed a tangible system for improving literacy in children with dyslexia, while Zhang et al. \cite{zhang2022storybuddy} introduced an AI-driven storytelling tool to balance parental involvement in learning activities. Although such technologies support collaborative learning, they have largely overlooked the specific challenges of homework involvement. Kalanadhabhatta et al. \cite{kalanadhabhatta2024playlogue} developed a dataset for analyzing adult-child conversations during play, demonstrating the potential of systematic conversation analysis in understanding parent-child interactions. 


In one of the few studies addressing this gap, Kerawalla et al. \cite{kerawalla2007exploring} examined a tablet-based platform designed to enhance parental understanding of classroom methods. This study is one of the few that addresses the role of technology in supporting homework parental involvement, showing the potential for digital tools to improve educational outcomes. Similarly, recent innovations like EduChat \cite{dan2023educhat}, an LLM-based educational chatbot, highlight the potential of AI in offering personalized support for both parents and children. Yu et al. \cite{yu2021parental} proposed a framework for parental mediation in children's use of creation-oriented educational media, and outlined three dimensions of mediation—creative, preparative, and administrative—offering insights for designing media that fosters creative learning while involving parents in the process. 



While technology-supported applications have made progress in facilitating parent-child interactions through storytelling, literacy development, and specific learning activities, a significant gap remains in addressing homework involvement, a crucial but underexplored aspect of parent-child interaction. Most research, focused on Western contexts \cite{pomerantz2007whom, patall2008parent, dettmers2019antecedents}, prioritizes academic outcomes, often overlook the unique emotional and behavioural dynamics that arise during homework involvement in non-Western cultures, particularly in China \cite{kim2013parents, gan2019parental, wang2021parental}. Additionally, reliance on self-report methods \cite{gan2019parental, patall2008parent} introduces bias and fails to capture real-time interactions.




Our study distinguishes itself in several ways: (1) we focus on the Chinese cultural context, shaped by unique parental expectations and pressures \cite{kim2013parents, suarez2022parental}; (2) unlike prior work that largely depends on subjective data, we utilize audio recordings of real-world homework sessions for a richer and more objective analysis; (3) by exploring the interplay of parental behaviours, emotions, and conflicts, we aim to deepen understanding of the complexities in homework involvement, contributing valuable insights to the family education research and designing technologies to improving parenting practices in China.
