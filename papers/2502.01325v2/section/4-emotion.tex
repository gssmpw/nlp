
\section{Understanding Parents' Emotion Dynamics During Homework Invovlment}




\subsection{Perceived Emotion Shifts after Homework Involvement} 
To comprehensively understand parents' emotional experiences before and after homework involvement, we conducted both group-level and individual-level analyses using self-reported data. First, we calculated the mean emotional values for each parent before and after homework and performed a paired sample t-test on these group means. This revealed significant differences across all emotional dimensions—valence (p < 0.001), arousal (p < 0.001), and dominance (p < 0.001)—indicating a statistically significant emotional shift for parents as a group due to homework involvement. Specifically, parents' pleasure tends to decrease, arousal tends to decrease, and their sense of control (dominance) over their emotions tends to diminish after homework involvement. 

\begin{figure}
    \centering
    \includegraphics[width=0.99\linewidth]{figure/significance_heatmap.pdf}
    \caption{Significance heatmap of pleasure, arousal, and dominance. Red, blue, and green colors indicate participants with statistically significant differences (p < 0.05) in pleasure, arousal, and dominance, respectively, before and after homework involvement.}
    \label{fig:heatmap}
\end{figure}

\begin{figure}
    \subfigure[ {Pleasure} \label{subfig: age}]{\includegraphics[width=0.7\textwidth]{figure/dis_valence_change_par.pdf}}
	\hspace{0cm}
    \subfigure[ {Arousal}\label{subfig: venn}]{\includegraphics[width=0.7\textwidth]{figure/dis_arousal_change_par.pdf}}
    \subfigure[ {Dominance}\label{subfig: venn}]{\includegraphics[width=0.7\textwidth]{figure/dis_dominance_change_par.pdf}}
    \caption{Emotion shifts after the homework involvement. Red indicates the participants with statistically significant shifts (p<0.05). }
    \label{fig: daily_survey_emotion}
\end{figure}


Next, we analyzed the emotional changes of individual parents, calculated as the difference between their post-homework and pre-homework values, with a focus on those who showed statistically significant shifts, as depicted in Figure \ref{fig:heatmap}. Some parents, such as P0, P53, and P57, experienced significant changes in one or more emotional dimensions, while others did not exhibit statistically significant shifts. Detailed information on these emotional changes for each participant is shown in Figure \ref{fig: daily_survey_emotion}, where red bars represent participants with statistically significant shifts (p < 0.05). Interestingly, although some parents, such as P31 and P102, displayed noticeable emotional shifts, these changes did not reach statistical significance, likely due to the limited number of observations (e.g., only one sample). These findings underscore the variability in emotional responses to homework involvement among parents. While some parents experienced marked decreases in pleasure or dominance and an increase in arousal, others demonstrated resilience or maintained stability in their emotional states.

on{Emotion Fluctuations During Homework Involvement} 
\subsubsection{Extracting Emotion Annotations}

%To gain a nuanced understanding of the emotional fluctuations of both parents and children during homework interactions, we employed the recently developed emotion fine-tuning large language model, EmoLLMs \cite{liu2024emollms}. This suite of models and annotation tools excels at comprehensive affective analysis, showing outstanding performance in emotion regression and classification tasks, particularly using the three dimensions—Pleasure, Arousal, and Dominance—from the EmoBank dataset \cite{buechel2022emobank}. Given these strengths, we utilized EmoLLMs to annotate our experiment's transcribed data.

To understand the emotional fluctuations during homework interactions, we employed the EmoLLMs model suite \cite{liu2024emollms}, which excels in affective analysis, particularly in emotion regression and classification tasks using the three dimensions—Pleasure, Arousal, and Dominance—based on the EmoBank dataset \cite{buechel2022emobank}. Given its high performance, especially in \textit{Pleasure} analysis (Pearson Correlation Coefficient = 0.728), we used the EmoLLaMA-chat-7B model \cite{liu2024emollms} to annotate our transcribed data.

We first filtered the transcriptions to remove sentences lacking semantic clarity, such as short or misrecognized sentences, which could distort the results. We then applied EmoLLMs to infer emotional dimensions on a sentence-by-sentence basis, focusing on the \textit{Pleasure} dimension. This is because pleasure directly reflects the emotional polarity (positive or negative) critical for understanding parent-child interactions during homework \cite{pekrun2002academic}. Although future analyses may include Arousal and Dominance, current model limitations make it practical to focus on pleasure alone for accurate analysis.

%In our experiment, we first filtered the transcriptions, excluding text that met any of the following criteria: \textit{1) Sentences with a high frequency of interjections, such as ``um um um'' or ``oh oh ah''.2) Sentences containing fewer than five words, such as ``Okay'' (1 word) or ``I'm sorry'' (3 words).3) Sentences with content errors or logical inconsistencies caused by misrecognition from the speech recognition software.} These sentences lacked sufficient or accurate semantic information, often leading to poor recognition and unexpected outcomes by the model. By excluding them, we improved the overall recognition accuracy and ensured the integrity of our dataset. We then applied the pre-trained \textit{EmoLLaMA-chat-7B model}  \cite{liu2024emollms} to infer emotional dimensions on a sentence-by-sentence basis, assigning values between 1.00 and 5.00 across three dimensions (\textit{Pleasure}, \textit{Arousal}, \textit{Dominance}). For this study, we focused exclusively on the \textit{Pleasure} dimension for two key reasons: (1) Valence directly measures emotional polarity, capturing the continuum of positive and negative emotions—critical for understanding emotional dynamics during parent-child homework interactions \cite{pekrun2002academic}; and (2)  \textit{EmoLLaMA-chat-7B model} was reported the highest performance in \textit{Valence}-related emotion analysis, with a \textit{Pearson Correlation Coefficient} (PCC) \cite{cohen2009pearson} of 0.728. While adding the other two dimensions could enhance future analyses, it requires further improvements in model robustness and accuracy. 


\subsubsection{Emotional Variation Analysis}


\begin{figure}
    \centering
    \includegraphics[width=0.99\linewidth]{figure/dis_all_first10.pdf}
    \caption{Average pleasure for the first 10 minutes of sessions. The dark grey band indicates the standard error.}
    \label{fig:emotion_all}
\end{figure}

We collected emotional data from multiple homework sessions, totalling 40,356 pleasure measurements from 66 parents.
To capture the dynamics, we calculated the mean pleasure for each participant at 15-second intervals, a balance that retained key details without being overwhelmed by data limitations.  
To account for varied session lengths, we standardized our analysis by focusing on the first 10 minutes of each session, which aligns with similar studies \cite{tag2022emotion}. This allowed us to analyze emotional fluctuations across participants consistently. Although some sessions lasted longer than 10 minutes and others shorter, we used all available data to calculate the mean pleasure for the first 10 minutes of each session.



%For each parent, we collected recording data from multiple homework sessions. In total, we have 40,356 valence values from 66 parents, that is to say, 611 valence values on average for each parent from their multiple homework sessions. We first calcluate the mean level of pleasure for each participants with a granularity of 15 seconds. We try several times, and find finer granular is not suitable due to our limited data and larger grannular may lose the charactersitics of their valence changes.  As such, for each individual homework session, we are able to estimate the level of valence of each parent for every 15 seconds of that session. However, we note taht session have different duration, and therefore aggreating the data for each participant needs further consideration, for this reason, we choose to aggregate data in the first 10 minutes, similar to the studies \cite{tag2022emotion}. 

Figure \ref{fig:emotion_all} shows the average pleasure levels during the first 10 minutes, with LOESS smoothing (frac = 0.1) applied to capture the overall trend. The dark grey bands represent the standard error and are based on smoothed mean values rather than raw data to highlight clearer patterns amid the variability. Due to space constraints, we present data for 18 participants with the most pleasure measurements. We found that some parents exhibit an early decline in pleasure within the first 2-3 minutes (e.g., P32, P76, P82), while others show fluctuating pleasure levels throughout the first 10 minutes (e.g., P18, P95). This provides insights into the evolving emotional engagement of parents during the early phases of homework involvement.

%Figure \ref{fig:emotion_all} shows the average pleasure levels during the first 10 minutes, with LOESS smoothing applied to capture overall trends. The dark grey band represents the standard error. We highlight the data for 18 participants with the most valence measurements to showcase key patterns. Some parents exhibited an early decline in pleasure within the first 2-3 minutes (e.g., P32, P76, P82), while others showed fluctuating pleasure levels throughout the session (e.g., P18, P95). This provides a snapshot of how emotional engagement shifts early in homework involvement.

%It is worth note that, Future analyses could explore how these early emotional responses relate to longer-term trends or outcomes in homework effectiveness and parental involvement. Additionally, while the first 10 minutes provide useful insights, extending the analysis to cover entire sessions may uncover more nuanced patterns over time.

%First, we visualize the mean pleasure for each patient of the first 10 minutes of the homework session as shown in Figure \ref{fig:emotion_all}. This is calculated using means and loess smoothing (frac=0.1), while the standard error is shown as dark grey band around the mean. We visualize the error bars based on the average pleasure values, not based on the raw data, to more clearly identify the signal in the noise. Some sessions are shorter than 10 minutes, others are longer, but regardless we calculate the mean pleasure joy for each of the first 10 minutes using the available data. The rationale is that this visualisation provides an assessment of how pleasure evolves as parent begin homework ionvlment, but a downside is that data beyond the first 10 minutes is discarded. Due to the limit of space, we only show the top 18 participants with highliest number of valence measurements. We observe that some parents experience reduced levels of pleasure within the first 2-3 minutes (e.g., P32, P76, P82), and some participants exhibit extensive flucntions during first 10 minutes (e.g., P18, P95)

%Second, we provide an aggreatoin that over the limitations of the first aggregation strategy (first 10 minutes) as shown in Figure \ref{fig:emotion_all_normalised}. Here, we normalise the duration of each homeowkr session to be 1, and any measurement of pleausare during a session is indexed to a normalised timestamp between 0 and 1. In this manner, all sessions start at 0, end at 1, and all pleasure measurment are timestamped with a value between 0 and 1. This allows us to retain all our data when calculating theaverage pleasure. However, it does not capture the true magnitude (in seconds) of each session duration. \ref{fig:emotion_all_normalised} provides a visual overview of different patterns of changes in pleasure during homework involvement. We found some parents experience intense fluctuations during  the homework session (P28, P76, P82). A few participants (e.g., P18, P28) display a gradual increase in pleasure at the end of the session. P32, P expericen the stable e 

\iffalse
\begin{figure}
    \centering
    \includegraphics[width=0.99\linewidth]{figure/dis_all.pdf}
    \caption{Average pleasure for sessions, with lengths normalised to [0,1]. The dark grey band indicates the standard error}
    \label{fig:emotion_all_normalised}
\end{figure}
\fi