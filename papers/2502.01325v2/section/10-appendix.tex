\section*{Appendix}
\section{Coding Usage Guidelines for Homework Assistance\label{sec:guide}}
\begin{table}[h]
\centering
\footnotesize
\caption{Coding usage guidelines for parent behaviour during homework involvement}
\label{appen:tab:behaviour}
\begin{tabular}{p{0.22\textwidth} p{0.7\textwidth}}
\toprule
\textbf{Code Name} & \textbf{Usage Guidelines} \\ \midrule
\textit{Encouragement (ENC)} & Use this code when parents offer positive reinforcement and encouragement, especially when the child faces difficulties, regardless of the outcome. \\ \hline
\textit{Labelled Praise (LP)} & Use this code when parents clearly point out a specific behaviour, achievement, or performance and give positive feedback. \\ \hline
\textit{Unlabelled Praise (UP)} & Use this code when parents give general, unspecific praise without focusing on particular details or behaviours. \\ \hline
\textit{Guided Inquiry (GI)} & Use this code when parents help the child find a solution by asking questions or offering hints, rather than directly giving the answer. \\ \hline
\textit{Setting Rules (SR)} & Use this code when parents establish clear rules and expect the child to follow them. These rules are often related to homework order, time, or completion standards. \\ \hline
\textit{Sensitive Response (SRS)} & Use this code when parents recognize the child's emotions and provide comfort or emotional support. \\ \hline
\textit{Direct Instruction (DI)} & Use this code when parents provide direct answers or solutions without encouraging the child to think through the problem. \\ \hline
\textit{Information Teaching (IT)} & Use this code when parents provide systematic explanations to help the child understand new knowledge or concepts. \\ \hline
\textit{Error Correction (EC)} & Use this code when parents identify errors in the child's homework and guide them to make corrections. \\ \hline
\textit{Monitoring (MON)} & Use this code when parents monitor the child's homework progress or check the quality of their work. \\ \hline
\textit{Direct Command (DC)} & Use this code when parents give a firm command in a direct and authoritative manner. \\ \hline
\textit{Indirect Command (IC)} & Use this code when parents use a more subtle or indirect approach to encourage the child to complete tasks. \\ \hline
\textit{Criticism \& Blame (CB)} & Use this code when parents criticize or blame the child for not meeting expectations. \\ \hline
\textit{Forcing \& Threatening (FT)} & Use this code when parents use threats or force to make the child complete a task. \\ \hline
\textit{Neglect \& Indifference (NI)} & Use this code when parents ignore or display indifference toward the child's needs, emotions, or requests for attention. \\ \hline
\textit{Belittling \& Doubting (BD)} & Use this code when parents use belittling or doubting language to undermine the child's abilities directly. \\ \hline
\textit{Frustration \& Disappointment (FD)} & Use this code when parents express disappointment due to the child's academic performance or lack of progress. \\ \hline
\textit{Impatience \& Irritation (II)} & Use this code when parents become frustrated or impatient with the child's learning pace or performance. \\ \bottomrule
\end{tabular}
\end{table}

\begin{table}[h]
\centering
\footnotesize
\caption{Coding usage guidelines for parent-child conflict during homework involvement}
\label{appen:tab:conflict}
\begin{tabular}
{p{0.12\textwidth} p{0.83\textwidth}}

\toprule
\textbf{Code Name} & \textbf{Usage Guidelines} \\ 
\midrule

\textit{Expectation \newline Conflict} & Apply this code when there is a significant discrepancy between parents' expectations and the child’s self-perception or goals. This may also include cases where parents compare the child to peers or siblings, adding pressure. \\

\hline
\textit{Communication Conflict} & Use this code when communication breakdowns occur due to negative communication styles such as parental criticism or questioning, leading to frustration. Pay attention to whether the parent blames or belittles the child and the child’s emotional response to it. \\

\hline
\textit{Learning Method Conflict} & Use this code when parents attempt to enforce a change in the child’s learning method or directly intervene, particularly when this disagreement leads to conflict. \\

\hline
\textit{Rule Conflict} & Use this code when the conflict revolves around parental rules, boundaries, or control, such as how and when the child should complete their homework. Pay special attention to cases where the child expresses dissatisfaction with the rules or seeks more autonomy. \\

\hline
\textit{Time \newline Management Conflict} & Use this code when the conflict centers on how study time is managed, the frequency of study sessions, or the child’s energy allocation. This includes situations where the child expresses dissatisfaction with the time constraints imposed by parents. \\

\hline
\textit{Knowledge \newline Conflict} & Apply this code when the conflict stems from a difference in knowledge mastery, a lack of understanding of the child’s learning difficulties, or when the child questions the parent’s knowledge. Pay special attention to instances where the parent underestimates the difficulty of the material for the child. \\

\hline
\textit{Focus Conflict} & Use this code when the conflict is driven by the parent’s dissatisfaction with the child’s focus or attention during study, particularly when repeated parental intervention causes tension. \\

\bottomrule
\end{tabular}
\end{table}



This section outlines the specific coding usage guidelines for analysing parent-child conflicts and parental behaviours during homework involvement (see Table \ref{appen:tab:behaviour} and Table \ref{appen:tab:conflict}). The coding manual has been designed to ensure consistency and clarity in the application of codes, providing clear definitions, usage examples, and contextual insights. These guidelines serve as a reference for researchers or coders, helping to systematically identify and categorize the key conflict and behaviour patterns observed in homework sessions. Each code is accompanied by detailed instructions on when and how to apply it, ensuring the accuracy and reliability of the analysis.

\section{Analysis of Category Confusion in GPT-4o Coding Using Confusion Matrix\label{sec:Confusion}}



To further analyze the specific patterns of agreement and disagreement between GPT-4o and human experts, we employed confusion matrices to visualize which categories GPT-4o tended to confuse during the coding process. These confusion matrices allow us to assess the distribution of errors across different coding categories, highlighting which categories are coded correctly and which tend to be misclassified. The matrices also provide insights into systematic biases in GPT-4o’s performance, with certain categories being more prone to misclassification.
\begin{figure}
    \centering
    \includegraphics[width=0.65\linewidth]{figure/behavior_confusion_matrix.pdf}
    \caption{The behaviour confusion matrix illustrates the discrepancies between GPT-4o and expert annotations in coding parental behaviours during homework assistance. The diagonal shows correctly classified behaviours, while off-diagonal elements indicate where GPT-4o misclassified one behaviour as another. }
    \label{fig:behaviour_confusion}
\end{figure}

\begin{figure}
    \centering
    \includegraphics[width=0.6\linewidth]{figure/conflict_confusion_matrix.pdf}
    \caption{The conflict confusion matrix shows GPT-4o’s performance in classifying parent-child conflict types during homework involvement. Correct classifications appear on the diagonal, while misclassifications appear off-diagonal.}
    \label{fig:conflict_confusion}
\end{figure}
From the parent-child conflict confusion matrix, we can observe that GPT-4o performed well in certain categories, such as \textit{Information Teaching} (IT) (28 matches), \textit{Error Correction} (EC) (14 matches), and \textit{Direct Command} (DC) (19 matches), demonstrating strong consistency with the expert consensus. However, significant confusion was observed in categories like \textit{Indirect Command} (IC), which was frequently misclassified as \textit{Direct Command} (DC) and \textit{Guided Inquiry} (GI), suggesting difficulties in distinguishing between direct and indirect commands. Additionally, \textit{Sensitive Response} (SRS) and \textit{Impatience and Irritation} (II) were often confused with \textit{Encouragement} (ENC) and \textit{Monitoring} (MON), indicating challenges in recognizing emotional support behaviours. GPT-4o also struggled with emotional categories such as \textit{Neglect and Indifference} (NI) and \textit{Frustration and Disappointment} (FD), as well as forceful behaviours like \textit{Forcing and Threatening} (FT) and \textit{Criticism and Blame} (CB), where misclassifications were frequent, as shown in Figure \ref{fig:behaviour_confusion}.

Similarly, from the parental behaviour confusion matrix, GPT-4o showed strong performance in structured categories such as \textit{Knowledge Conflict} (KC) (26 matches), \textit{Time Management Conflict} (TMC) (19 matches), and \textit{Focus Conflict} (FC) (16 matches). However, frequent confusion occurred between \textit{Expectation Conflict} (EC) and \textit{Knowledge Conflict} (KC) (11 misclassifications), as well as between \textit{Learning Method Conflict} (LMC) and \textit{Time Management Conflict} (TMC), suggesting overlap in parental interventions. Emotional and communication-based categories, such as \textit{Emotional Conflict} (EMC) and \textit{Communication Conflict} (CC), were also often misclassified, reflecting difficulties in distinguishing between these nuanced conflicts. \textit{Rule Conflict} (RC) was another challenging category, with misclassifications occurring in relation to \textit{Time Management Conflict} (TMC) and \textit{Learning Method Conflict} (LMC), as illustrated in Figure \ref{fig:conflict_confusion}.

In conclusion, the confusion matrices reveal that GPT-4o demonstrates strong performance in clear and structured categories, but faces challenges in emotional and communication-based conflicts. Improving the AI’s ability to differentiate subtle emotional and contextual behaviours could enhance its performance in these complex areas.







\section{Prompt for Data Processing}

\subsection{Prompt for Fixing Transcription Errors}
\label{app:prompt_transcription}

\begingroup
\footnotesize
\begin{spverbatim}
# Correcting Transcription Errors in Parental Homework Involvement (Python Dictionary Format)

## Objective
Correct transcription errors in the text of parents helping their kids with homework while preserving the original expression style.

## Input
- A list of Python dictionaries, each containing:
  - Conversation ID (`id`)
  - Speaker (`speaker`)
  - Transcribed content (`content`)

## Processing Steps
1. **Parse the Dictionary List**: Read through each transcription entry.
2. **Identify Transcription Errors**: Carefully examine and identify homophones or typographical errors in the `content` section.
3. **Correct Errors**: Correct the errors without altering the original sentence style to improve the clarity and accuracy of the text.
4. **Maintain Original Style**: When updating the `content` field, ensure that the original expression style is preserved, striving to maintain the natural flow of the original text.
5. **Structure Validation**: After correction, ensure that the structure of each dictionary remains identical to the corresponding original dictionary, including but not limited to field order and data types.

## Output Format
Please return the output in the following strict format, retaining only the `id` and `content` fields and correcting only the `content` field:
```python
[
    {'id': 1, 'content': 'Corrected content 1'}, 
    {'id': 2, 'content': 'Corrected content 2'}, 
    {'id': 3, 'content': 'Corrected content 3'}, 
    ...
]

## Considerations
- **Retain Conversational Style**: Pay special attention to preserving the conversational characteristics of the transcribed text when correcting errors.
- **Minimize Intervention**: Adjustments should minimally impact the original meaning and expression, correcting only obvious typographical errors and homophone transcription errors while avoiding over-editing.
- **Clarity and Accuracy**: While retaining the conversational style, ensure that the clarity and accuracy of the text are enhanced.
- **Structural Consistency**: Ensure the output dictionary structure is fully consistent with the input data, with the same number of entries, field order, and data types, and only the `content` field is corrected as needed.
- **Length Consistency**: Ensure that each dictionary's length after correction matches the original data, avoiding any length changes due to content addition or omission.
- **Preserve All Entries**: Ensure all original entries are retained in the output data, with no omissions or deletions.
- **No Deletion of Original Content**: Ensure all original content is preserved, with corrections made without deleting any parts, especially numbers and sentences.

\end{spverbatim}
\endgroup

\subsection{Prompt for Role Recognition from Transcripts}
\label{app: prompt_role recognition}
\begingroup
\footnotesize
\begin{spverbatim}
# Speaker Role Identification Task

## Task Description
Use transcribed audio text to identify the roles of speakers in the context of parents helping their children with homework. This API can automatically identify the number of speakers and assign each segment of speech to a specific speaker (e.g., 'speaker 1', 'speaker 2', etc.). When two speakers are identified, it is usually assumed that one is the parent and the other is the child. If three or more speakers are identified, there might be inaccuracies in the number of speakers detected by the API. In such cases, two or more identified speakers might actually be the same person (e.g., both the parent or both the child) or might include others (e.g., someone playing a recording of a lesson).

## Input Format
The input is a list of dictionaries, each corresponding to a segment of transcribed text. Each dictionary contains the following keys:
- "id": The conversation ID.
- "speaker": A string representing the label assigned to the speaker (e.g., "speaker 1").
- "content": A string containing the transcribed text for that segment.

## Output
The output should be a dictionary mapping speaker labels to their inferred roles. The roles can be "parent", "child", or "others". The "others" role should only be used when there are three or more speakers.

### Role Inference Guidelines
- **parent**: Typically asks questions, assigns tasks, gives instructions, commands, explains concepts, criticizes, encourages, or provides help.
- **child**: Typically requests help/guidance/praise, expresses confusion, or asks questions.
- **others**: Recordings that appear in the conversation can be labeled as "others". "others" do not participate in the conversation and typically do not respond to "parent" or "child".

### Output Format
Please return the result in the following format, strictly as a dictionary without explanatory text:
```python
{
    "Speaker 1": "",
    "Speaker 2": "",
    ...
}

### Considerations
1. Role inference should be based on the content of the conversation, not just the sequence of speakers.
2. For three or more speakers, carefully distinguish between roles. Only label as "others" when it is very clear; otherwise, do not infer "others".
3. For each speaker, analyze their speech content in detail to avoid mistakenly labeling an actual participant in the conversation as "others". Consider marking "others" only in very rare cases.

\end{spverbatim}
\endgroup
\normalsize


\section{Prompt for Understanding Behavioural Dynamics During Homework Involvement}

\normalsize
\subsection{Prompt for Parent Behaviour Identification}

\begingroup
\footnotesize
\begin{spverbatim}
# Task Description
You are a family education expert, and your task is to analyze the specific behaviors of parents during homework assistance. You need to carefully review the provided dialogue content, identify and capture dialogue segments that reflect parental behavior, and classify these behaviors using predefined behavior categories.

## Behavior Categories

### Positive Behaviors

#### 1. **Encouragement (ENC)**
- **Code Definition**: The parent actively supports the child's efforts and progress through words or actions, boosting the child's confidence and motivation to learn, helping them overcome difficulties.
- **Usage Guidelines**: Use when the parent provides positive support and encouragement, especially when the child encounters difficulties, and the parent maintains a positive attitude regardless of the outcome.
- **Example**:  
  - Parent: "You've worked really hard, keep it up! I believe in you!"
  - Parent: "Don’t worry, we’ll take it step by step, you’ll definitely get it."

#### 2. **Labelled Praise (LP)**
- **Code Definition**: The parent explicitly points out specific behaviors or achievements of the child and offers praise, helping the child recognize their concrete progress and strengths.
- **Usage Guidelines**: Use when the parent clearly identifies a specific behavior, achievement, or performance of the child and provides positive feedback.
- **Example**:  
  - Parent: "You did this addition problem perfectly, no mistakes at all!"
  - Parent: "Your handwriting is really neat this time, keep it up!"

#### 3. **Unlabelled Praise (UP)**
- **Code Definition**: The parent offers general praise to the child but does not point out any specific behaviour or achievement.
- **Usage Guidelines**: Use when the parent gives unlabelled praise without mentioning specific details or actions.
- **Example**:  
  - Parent: "You're doing great, keep going!"
  - Parent: "Wow, amazing!"

#### 4. **Guided Inquiry (GI)**
- **Code Definition**: The parent guides the child to think independently and solve problems through questions or hints, rather than directly providing the answer, encouraging the child to explore and think critically.
- **Usage Guidelines**: Use when the parent helps the child find solutions through questions or hints rather than directly giving the answer.
- **Example**:  
  - Parent: "Where do you think this letter should go?"
  - Parent: "What strategy can we use to solve this problem? Think about a few ways."

#### 5. **Setting Rules (SR)**
- **Code Definition**: The parent sets clear rules or requirements for the child to complete their homework, helping the child establish good study habits and time management skills.
- **Usage Guidelines**: Use when the parent sets specific rules and requires the child to follow them, typically related to the order of tasks, time, or completion standards.
- **Example**:  
  - Parent: "You need to finish your language homework first before you can watch cartoons."
  - Parent: "You have to finish all your homework before dinner, then you can go out to play."

#### 6. **Sensitive Response (SRS)**
- **Code Definition**: The parent responds to the child's emotions, needs, and behaviors in a timely, appropriate, and empathetic manner. The parent perceives the child's feelings and provides emotional support.
- **Usage Guidelines**: Use when the parent recognizes the child’s emotions and offers understanding and emotional comfort or support.
- **Example**:  
  - Parent: "I know you're feeling tired, let's take a break and continue later, okay?"
  - Parent: "Are you finding this problem difficult? It’s okay, we’ll go over it together."

### Neutral Behaviors

#### 7. **Direct Instruction (DI)**
- **Code Definition**: The parent directly tells the child how to complete a task or solve a problem without using an exploratory or guided approach.
- **Usage Guidelines**: Use when the parent provides the answer or solution directly without guiding the child to think through questions or hints.
- **Example**:  
  - Parent: "You should do it like this, add 4 to 6, and it equals 10."
  - Parent: "Just copy the answer directly, don’t overthink it."

#### 8. **Information Teaching (IT)**
- **Code Definition**: The parent imparts new knowledge or skills to the child through explanations, such as explaining concepts or lessons, to help the child understand new study material.
- **Usage Guidelines**: Use when the parent provides systematic explanations to help the child learn new knowledge or skills.
- **Example**:  
  - Parent: "The word ‘tree’ is written with a wood radical on the left and ‘inch’ on the right, let’s write it together."
  - Parent: "The multiplication table goes like this, two times two equals four, two times three equals six. Let’s memorize these first."

#### 9. **Error Correction (EC)**
- **Code Definition**: The parent points out mistakes in the child’s homework and guides them to make corrections.
- **Usage Guidelines**: Use when the parent identifies mistakes in the child’s homework and guides the child to correct them.
- **Example**:  
  - Parent: "You missed the ‘wood’ radical here, write it again."
  - Parent: "This addition is wrong, redo it, and make sure to align the columns."

#### 10. **Monitoring (MON)**
- **Code Definition**: The parent periodically checks the child’s homework progress or completion to ensure tasks are done on time.
- **Usage Guidelines**: Use when the parent checks the child’s homework progress or reviews the quality of the work.
- **Example**:  
  - Parent: "How many pages have you written? Let me check for mistakes."
  - Parent: "I’ll check your pinyin homework today to see if there are any issues."

#### 11. **Direct Command (DC)**
- **Code Definition**: The parent uses clear, direct language to command the child to perform a task or behavior, often with a strong, imperative tone.
- **Usage Guidelines**: Use when the parent gives a command in a forceful manner, requiring the child to complete a task.
- **Example**:  
  - Parent: "Go do your math homework right now, no more delays!"
  - Parent: "Stop playing with your toys and finish copying your pinyin."

#### 12. **Indirect Command (IC)**
- **Code Definition**: The parent uses a more indirect approach to convey a request to the child, such as a suggestion or implication, rather than giving a direct order.
- **Usage Guidelines**: Use when the parent suggests or implies that the child should complete a task without directly commanding them.
- **Example**:  
  - Parent: "Have you finished your homework? Maybe it's time to get it done?"
  - Parent: "Let’s finish the homework first, so we don’t have to worry about it later."

### Negative Behaviors

#### 13. **Criticism & Blame (CB)**
- **Code Definition**: The parent expresses negative evaluations of the child’s mistakes or behavior, directly blaming the child for their shortcomings, potentially using belittling language.
- **Usage Guidelines**: Use when the parent criticizes or blames the child for not meeting expectations.
- **Example**:  
  - Parent: "How could you get such an easy character wrong?"
  - Parent: "I’ve told you a thousand times, why can't you remember?"

#### 14. **Forcing & Threatening (FT)**
- **Code Definition**: The parent applies pressure or threatens consequences to force the child to comply with their demands, aiming to achieve the desired behavior.
- **Usage Guidelines**: Use when the parent uses threats or force to demand task completion.
- **Example**:  
  - Parent: "If you don’t finish your homework, you won’t get to play with your blocks!"
  - Parent: "If you don’t finish it, I’ll take away your toys!"

#### 15. **Neglect & Indifference (NI)**
- **Code Definition**: The parent shows disregard or indifference to the child’s needs or emotions, providing no attention or response.
- **Usage Guidelines**: Use when the parent ignores the child’s requests, emotions, or behavior, showing no care or reaction.
- **Example**:  
  - Child: "Mom, can you help me with this problem?"
  - Parent (ignores the child and continues using the phone).

#### 16. **Belittling & Doubting (BD)**
- **Code Definition**: The parent belittles or doubts the child’s abilities, directly undermining the child’s confidence and motivation. This type of language or behavior often conveys distrust or dissatisfaction with the child’s abilities.
- **Usage Guidelines**: Use when the parent uses derogatory language or expresses doubt about the child’s abilities, typically showing a lack of confidence in the child’s learning capabilities or expressing strong dissatisfaction with their performance.
- **Example**:  
  - Parent: "How can you be so dumb, you can’t even do simple addition?"
  - Parent: "With grades like yours, there’s no way you’ll get into a good school."

#### 17. **Frustration & Disappointment (FD)**
- **Code Definition**: The parent expresses feelings of frustration or disappointment due to the child’s performance not meeting expectations, often accompanied by negative emotional expressions.
- **Usage Guidelines**: Use when the parent expresses disappointment due to the child’s academic performance or progress not meeting expectations.
- **Example**:  
  - Parent: "I didn’t expect you to do so poorly, you really let me down."
  - Parent: "I thought you would do better, I guess I was wrong."

#### 18. **Impatience & Irritation (II)**
- **Code Definition**: The parent shows impatience or irritation due to the child’s performance not meeting expectations, often accompanied by negative language or behavior.
- **Usage Guidelines**: Use when the parent expresses impatience or frustration with the child’s learning pace or performance.
- **Example**:  
  - Parent: "Why are you so slow? I’ve been waiting forever!"
  - Parent: "How come you still haven’t finished? You’re always dragging your feet!"

# Input Format
The input is a list of dictionaries representing audio segments, each dictionary contains:
- 'id': Identifier for the segment.
- 'speaker': The role of the speaker (parent, child, or others).
- 'content': The dialogue content of the segment.

# Output Format
The output is a JSON object, segmenting the dialogue record into different behaviors labeled with a "behaviour_id" (each behavior may consist of one or more sentences). The corresponding start and end IDs of the dialogue segment are marked for each behavior. Provide a description of the behavior in the "Description of behavior" and then map this behavior to the parent behavior code abbreviation "code," ensuring accurate classification.

The output should be strictly returned in the following format, enclosed in ```json```:

```json
[
    {
        "behaviour_id": 1,
        "Start ID": The start ID of the dialogue segment corresponding to this behavior,
        "End ID": The end ID of the dialogue segment corresponding to this behavior,
        "Description of behavior": "Description of the behavior",
        "Parent Behavior": "Specific words of the parent",
        "code": "Behavior abbreviation",
    }
    ...
]
```

# Example Output
```json
[
    {
        "behaviour_id": 1,
        "Start ID": The start ID of the dialogue segment corresponding to this behavior,
        "End ID": The end ID of the dialogue segment corresponding to this behavior,
        "Description of behavior": "The parent gave clear positive feedback on the child’s math performance.",
        "Parent Behavior": "You did a great job on your math today!",
        "code": "Labelled Praise (LP)"
    },
    {
        "behaviour_id": 2,
        "Start ID": The start ID of the dialogue segment corresponding to this behavior,
        "End ID": The end ID of the dialogue segment corresponding to this behavior,
        "Description of behavior": "The parent encouraged the child to keep trying and set a goal to reduce errors, expressing hope for future improvement.",
        "Parent Behavior": "Keep it up, let’s aim to make fewer mistakes next time!",
        "code": "Encouragement (ENC)"
    },
    {
        "behaviour_id": 3,
        "Start ID": The start ID of the dialogue segment corresponding to this behavior,
        "End ID": The end ID of the dialogue segment corresponding to this behavior,
        "Description of behavior": "The parent gave the answer directly without encouraging independent thinking.",
        "Parent Behavior": "The answer to this problem is 3, just write it down.",
        "code": "Direct Instruction (DI)"
    },
......
]
```
\end{spverbatim}
\endgroup
\normalsize








































































































































\subsection{Prompt for Parent-Child Conflict Identification}


\begingroup
\footnotesize
\begin{spverbatim}
# Parent-Child Conflict Identification Task

## Below are common classifications of parent-child conflicts during homework assistance:

### 1. **Expectation Conflict (EC)**
- **Code Definition**: Parents have high expectations for their child's academic performance, progress, or future, while the child's actual abilities, goals, or interests do not align with these expectations, leading to conflict. Parents may compare the child's performance with others, intensifying the conflict.
- **Usage Guidelines**: Use when parents have excessive demands on the child's performance or when there is a clear discrepancy between the parent's expectations and the child's self-perception. This includes situations where the parent compares the child to others (e.g., classmates, siblings), causing pressure.
- **Example**:  
  - Parent: "You should score full marks like your classmates. How can you make mistakes on such easy questions?"
  - Child: "I've done my best! Why do you always think I'm worse than others?"

### 2. **Communication Conflict (CC)**
- **Code Definition**: Conflict arising from different communication styles between parents and children during homework assistance. Parents may criticize, question, or belittle the child, leading the child to feel misunderstood or oppressed, worsening communication barriers.
- **Usage Guidelines**: Use when negative communication, such as emotional outbursts, criticism, or questioning, leads to conflict. Pay attention to whether the parent blames or belittles the child's performance and the child's resulting resistance.
- **Example**:  
  - Parent: "What's wrong with you? I've explained this so many times, and you still don't get it!"
  - Child: "I don't want to listen to you anymore! You always yell at me like this!"

### 3. **Learning Method Conflict (LMC)**
- **Code Definition**: Disagreements between parents and children on how to complete homework. Parents may feel the child's methods are inefficient and try to enforce their own, while the child insists on using their own approach and resists intervention.
- **Usage Guidelines**: Use when the parent tries to force the child to change learning methods or directly intervene in the learning process, especially when disagreements over methods lead to conflict.
- **Example**:  
  - Parent: "You can't study like this; you should finish all the questions first and then check them!"
  - Child: "This is how I like to do it! Why should I follow your way?"

### 4. **Rule Conflict (RC)**
- **Code Definition**: Conflict between parental control and the child's autonomy regarding rules set for studying. Parents may try to control the child's study schedule and methods through strict rules, while the child seeks more autonomy and flexibility.
- **Usage Guidelines**: Use when conflict involves rules, boundaries, or control set by parents, such as study time or homework completion methods. Pay attention to whether the child expresses dissatisfaction with the rules or seeks more autonomy.
- **Example**:  
  - Parent: "You must do your homework right after dinner; no more procrastinating!"
  - Child: "I want to play a little longer! You always control everything!"

### 5. **Time Management Conflict (TMC)**
- **Code Definition**: Disagreement between parents and children on how to allocate time and energy for studying. Parents may expect the child to follow a fixed schedule, while the child has a different time arrangement, leading to conflict.
- **Usage Guidelines**: Use when conflict revolves around study time, energy allocation, or the pace of study. Especially focus on whether the child expresses dissatisfaction with the parent's time requirements.
- **Example**:  
  - Parent: "You always leave your homework until late at night. Your efficiency is too low!"
  - Child: "I prefer studying later. I can't focus in the morning!"

### 6. **Knowledge Conflict (KC)**
- **Code Definition**: Conflict caused by differences in knowledge level or understanding between parents and children. Parents, having mastered certain knowledge, fail to understand why the child struggles or cannot explain things from the child’s perspective, leading the child to feel misunderstood or criticized. Additionally, parents' unfamiliarity with certain study content may cause the child to doubt the correctness of their guidance, further sparking conflict.
- **Usage Guidelines**: Use when conflict involves differences in knowledge mastery, the parent's lack of understanding of the child's learning difficulties, or the child's doubt in the parent's knowledge. Pay attention to whether parents underestimate the difficulty of the child's learning.
- **Example**:  
  - Parent: "This problem is so simple; how can you not get it?"
  - Child: "What you’re explaining is different from what the teacher said. I don’t understand."

### 7. **Focus Conflict (FC)**
- **Code Definition**: Parents are dissatisfied with the child's focus during the study, believing the child is distracted or inefficient. They try to remind or correct the child to stay focused, while the child may feel pressured by excessive interference, resulting in conflict.
- **Usage Guidelines**: Use when conflict arises because the parent feels the child is not paying attention or focusing on their studies, especially when the parent constantly interferes with the child's learning state.
- **Example**:  
  - Parent: "What are you daydreaming about? Focus and do your homework!"
  - Child: "I'm not daydreaming; I'm thinking about how to solve the problem!"

## Conflict Intensity Judgment Criteria

### Intensity Levels:
1. High 2. Medium 3. Low

### Judgment Criteria:

#### High
- **Tone**: The tone of the conversation is extremely intense, often involving shouting, criticism, or scolding.
- **Severity of Language**: The language includes insulting, belittling, or blaming remarks.
- **Length**: The conversation is lengthy, with escalating tension.
- **Body Language (if described)**: Involves intense gestures like slamming tables or throwing objects.

**Example**:
- Parent yells: "How can you be so useless? You can't even do this right!"
- Child shouts: "I've tried my best! Why do you always scold me?"

#### Medium
- **Tone**: The tone is relatively emotional but does not reach shouting, possibly involving strong dissatisfaction and dispute.
- **Severity of Language**: The language includes criticism or blame, but no serious insults.
- **Length**: The conversation is of medium length, with some dispute but not escalating further.

**Example**:
- Parent says: "Why did you do poorly again this time? It's so disappointing."
- Child replies: "I tried really hard; I'll do better next time."

#### Low
- **Tone**: The tone is calm or slightly dissatisfied, but generally controlled, possibly including mild criticism or suggestions.
- **Severity of Language**: The language does not contain severe negative words, focusing more on expressing disappointment or offering suggestions.
- **Length**: The conversation is short, with relatively mild conflict.

**Example**:
- Parent says: "This time's result isn't ideal; try harder next time."
- Child replies: "Okay, I'll work on it."

## Task Description
Using audio transcription text, identify **parent-child conflict** scenarios in conversations during homework assistance. Each scenario should be described with its trigger, process, specific behaviors of the parent and child, conflict type, and intensity level.

## Note: The number of conflicts in a single conversation is not fixed, and multiple conflicts may occur.

## Input Format
The input is a list of dictionaries, each representing a segment of transcribed text. Each dictionary contains the following keys:
- "id": the dialogue ID.
- 'speaker': the role of the speaker (parent, child, or others).
- "content": the string containing the transcribed text.

## Example Output
Below is an example of the output format, using JSON to describe **parent-child conflict** scenarios identified in homework assistance conversations. Each scenario includes the trigger, process, specific behaviors of the parent and child, conflict type, and intensity level. Additionally, the dialogue segment's start and end IDs corresponding to the conflict are marked.

```json
[
    {
        "scene_id": 1,
        "Start ID": The start ID of the dialogue segment corresponding to this conflict,
        "End ID": The end ID of the dialogue segment corresponding to this conflict,
        "trigger": "The child repeatedly made mistakes in math homework, and the parent became dissatisfied.",
        "process": "The parent noticed the child made a mistake in a simple addition problem and became increasingly impatient. The child showed dissatisfaction and began to resist.",
        "parent_behavior": "Parent criticizes: 'How could you get such an easy question wrong? I’ve already taught you!'",
        "child_behavior": "Child retorts: 'I’ve already done it several times! Why do you keep criticizing me?'",
        "conflict_type": "Expectation Conflict (EC)",
        "severity": "Medium"
    },
    {
        "scene_id": 2,
        "Start ID": The start ID of the dialogue segment corresponding to this conflict,
        "End ID": The end

 ID of the dialogue segment corresponding to this conflict,
        "trigger": "The child was distracted while writing pinyin, and after multiple reminders, the parent lost patience.",
        "process": "The parent noticed the child daydreaming during homework and, after several reminders, became strict. The child began to resist and refused to continue doing homework.",
        "parent_behavior": "Parent yells: 'What are you daydreaming about? Focus, or you won’t be allowed to watch TV after finishing your homework!'",
        "child_behavior": "Child mutters: 'I wasn’t daydreaming, I was thinking about how to write the problem!'",
        "conflict_type": "Focus Conflict (FC)",
        "severity": "High"
    },
    {
        "scene_id": 3,
        "Start ID": The start ID of the dialogue segment corresponding to this conflict,
        "End ID": The end ID of the dialogue segment corresponding to this conflict,
        "trigger": "The child insisted on using their own method during the learning process, and the parent tried to intervene.",
        "process": "The parent believed the child’s learning method was incorrect and tried to get the child to follow their instructions, but the child insisted on their own method, refusing to accept intervention.",
        "parent_behavior": "Parent says: 'You can’t do your homework like this. Look at the question carefully before you start!'",
        "child_behavior": "Child argues: 'I just want to do it this way. Why do I have to follow your instructions?'",
        "conflict_type": "Learning Method Conflict (LMC)",
        "severity": "Medium"
    },
    ...
]
```

## Output Format
Each scene should be described concisely with the trigger, process, specific behaviors of the parent and child, and the conflict type and intensity level. Additionally, mark the start and end IDs of the dialogue segment where the conflict occurred. Please strictly return the output in the following format:
```json
[
    {
        "scene_id": 1,
        "Start ID": The start ID of the dialogue segment corresponding to this conflict,
        "End ID": The end ID of the dialogue segment corresponding to this conflict,
        "trigger": "Trigger",
        "process": "Process",
        "parent_behavior": "Parent's Specific Behavior",
        "child_behavior": "Child's Specific Behavior",
        "conflict_type": "Conflict Type",
        "severity": "Intensity Level"
    }
    ...
]
```
\end{spverbatim}
\endgroup

