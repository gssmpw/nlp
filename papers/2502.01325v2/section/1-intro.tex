\section{Introduction}

\begin{quote}
    \centering
   \textit{``Every night, millions of parents and kids shed blood, sweat and tears 
   \\over the kitchen table.''} - Sharon Begley \cite{begley1998homework}
   %\textit{``The homework ate my family.''} - Romesh Ratnesar \cite{ratnesar1999homework}
\end{quote}

Family education, the cornerstone of individual development and societal progress, plays a pivotal role in shaping values, fostering emotional intelligence, and cultivating lifelong learning habits. A critical component of family education is parental homework involvement, which significantly influences students' academic achievement, motivation, and well-being \cite{patall2008parent, dettmers2019antecedents}. Schools commonly actively encourage this involvement, viewing it as both a means to facilitate student success \cite{cooper1989synthesis}, and an opportunity for parents to engage meaningfully in their children’s school life and academic progress \cite{heimgartner2012more}. In Asian countries like China, this engagement has intensified significantly: a national survey revealed that parents' weekly time helping primary school children with homework increased from 3.67 hours in 2010 to 5.88 hours in 2018  \cite{iccs}, reflecting the growing emphasis on parental homework involvement within family education.


While such involvement can be highly rewarding and keep parents connected with their children's education, it can also be a source of conflict \cite{solomon2002helping, patall2008parent}. Parents often struggle with improper family education strategies, including insufficient positive feedback \cite{pomerantz2007whom}, excessive assistance \cite{patall2008parent}, and unrealistic expectations \cite{pomerantz2007whom, cunha2015parents}. A 2019 survey of over 20,000 Chinese parents \cite{iccs} found that nine out of ten experienced emotional breakdowns during homework assistance, and four out of ten exhibited out-of-control behaviours such as scolding or physical punishment. More alarmingly, cases of heart attacks and strokes among parents have been reported due to the intense stress and anger associated with homework involvement \cite{news_Mailonline_2020,news_Nov_Nov_2020}. These challenges extend to children, potentially diminishing their learning interests, self-confidence, parent-child relationships, and long-term psychological health. Children may experience negative emotions while doing homework, sometimes to the point of feeling so frustrated with themselves, their parents, or their teachers that they stop working entirely for the night \cite{xu2005homework}, and these repeated negative experiences can turn children off from homework or even lead to premature burnout \cite{corno2004homework}.  

Improving parental education strategies is challenging despite their best intentions. Parents often find themselves in a predicament, striving to control their emotions yet struggling to avoid negative outbursts \cite{mcewan2004deal}. Waters et al. \cite{waters2020keep} found that emotion suppression by parents does not improve interactions with their children and may even have the opposite effect. Traditional interventions for improving family education strategies include counselling, reading books, attending workshops, taking online courses, joining support groups, and consulting with educators. Group-based methods like workshops and support groups, while more affordable, often provide general advice that may not be personalized to each family's unique situation. Individual-based methods, such as one-on-one counselling, offer specific suggestions but are typically more expensive and rely heavily on subjective self-reports from parents. Moreover, these methods primarily serve parents who actively seek help, potentially missing those unaware of their need for assistance. 


To develop effective family parenting strategies, it is essential to understand the dynamics of parents' emotions, behaviours, and parent-child conflicts during homework involvement. We conducted a 4-week study involving 78 Chinese families. We gathered audio recordings of entire parental homework involvement sessions. In total, we collected 602 valid audio recordings totalling 475 hours from 66 families, along with background and daily surveys. Leveraging large language models (LLMs), we analyzed parent-child transcripts to examine:
%of these parent-child conversations were analyzed to examine the dynamics of emotional shifts, common parental behaviours, and the nature of conflicts during homework involvement. Specifically, we aim to answer the following questions: 
\textit{1. What does parental homework involvement in China entail, and do parents experience emotional changes post-homework?}
\textit{2. What types of parental behaviours and parent-child conflicts arise during homework involvement in China?}
\textit{3. How are parents' emotions, behaviours, and conflicts interconnected during homework involvement?}

%Our analysis revealed several significant findings. We observed statistically significant emotional shifts in pleasure, arousal, and dominance across the parent population (p<0.001 for all dimensions based on the PAD emotion state model \cite{bradley1994measuring}). We identified and categorized 18 common parental behaviors (six each of positive, neutral, and negative) and seven types of parent-child conflicts. Notably, \textit{Knowledge Conflict} emerged as the most prevalent conflict type, and we found that even positive and neutral behaviors could significantly correlate with specific conflicts.

Through the analysis of parent-child conversations, we discovered statistically significant emotional shifts in pleasure, arousal, and dominance (based on the PAD emotion state model \cite{bradley1994measuring}) across the overall parent population before and after homework involvement (all dimensions p<0.001). However, individual parents exhibited variations and different patterns in these emotional shifts. By utilising the large language model and adopting suggestions from education experts, we categorized common parental behaviours during Chinese parental homework involvement, identifying six positive behaviours, six neutral behaviours, and six negative behaviours. Additionally, we developed a list of sevenht common parent-child conflicts. We then analyzed the distribution of these conflicts and found that \textit{Knowledge Conflict} is the most prevalent during homework involvement. Interestingly, our investigation into the relationship between emotions, behaviours, and conflicts revealed that even positive and neutral behaviours could be significantly associated with parent-child conflicts. For instance, \textit{Unlabelled Praise} is statistically significantly positively related to \textit{Expectation Conflict}. In sum, our paper makes the following contributions:


\begin{itemize}
    \item We conducted a 4-week in situ study with 78 Chinese families, collecting 602 audio recordings (totalling 475 hours) and daily surveys related to parental homework involvement. Using this data, we identified 18 common parental behaviours (6 positive, 6 neutral, and 6 negative) and 7 types of parent-child conflicts specific to Chinese parental homework involvement.

    \item We extracted parental behaviours and conflicts from extensive parent-child conversations by leveraging the large language model (i.e., GPT-4o). We then conducted two evaluation experiments, comparing GPT-4o's outputs with manual annotations made by four human experts, to validate the efficacy and reliability of these extractions. 

    \item We analyzed the dynamics of emotional, behavioural, and conflict, revealing several significant findings. For instance, we observed statistically significant emotional shifts in parents before and after homework involvement. We also identified ``knowledge conflict'' as the most common type of conflict in Chinese families. Notably, we found even positive and neutral parental behaviours were significantly positively correlated with specific conflicts. These unique insights inform the design of future intervention technologies aimed at improving parenting strategies in Chinese family environments.

\end{itemize}

