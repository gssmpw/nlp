\section{Implications and Limitations}
\label{sec:limit}
In this section, we briefly examine how our findings could inform potential design implications to improve parenting strategies in the future. It is important to note that the design space can be interpreted in various ways, depending on subjective perspectives.

%\subsubsection{Personalized Strategy for Different Families} We recommend developing individualized parent-child profiles based on specific family situations and refining them through historical interaction data. Our find that emotional responses during homework involvement varied widely among parents, with some showing significant decreases in pleasure and increases in arousal. Demographic data also showed diversity in education levels and children’s academic performance, suggesting varying family needs. Personalizing the system’s guidance based on these differences can improve intervention effectiveness by tailoring support to each family’s unique dynamics.


\textit{Adaptive Involvement Balancing}.
We suggest adjusting the level of parental involvement based on real-time emotional and behavioural cues to maintain a balance between support and autonomy. 
We found that even positive behaviours often led to parent-child conflicts except for \textit{Neglect and Indifference}. 
Future systems could use emotion and behaviour analysis to suggest optimal involvement levels, advising parents when to step back and allow the child more independence, especially during moments of tension. This approach aligns with \textit{Authoritative Parenting} \cite{gray1999unpacking}, characterized by high responsiveness and appropriate demands, which has been associated with positive child development outcomes. By dynamically adjusting involvement, parents can foster resilience and self-regulation in their children, promoting healthier emotional and social development.

%Parental involvement should be adjusted based on real-time emotional and behavioral cues to balance support and autonomy. Our study found that even positive behaviors often led to parent-child conflicts, except for "Neglect and Indifference" (NI). Future systems could use emotion and behavior analysis to suggest optimal involvement levels, advising parents when to step back and allow the child more independence, especially during moments of tension.

\textit{Behaviour-Specific Intervention Strategies}.
We suggest offering tailored interventions for specific parental behaviours that are strongly associated with particular types of conflicts, aiming to mitigate conflicts associated with each behaviour type. 
Behaviours like \textit{Setting Rules} and \textit{Error Correction} were associated with conflicts, while \textit{Labelled Praise} had fewer correlations.
It indicates each behaviour impacts conflict dynamics differently, and a one-size-fits-all approach may not be effective.
Tools offering real-time feedback could help parents replace conflict-inducing actions with more constructive alternatives, such as guiding parents who frequently engage in \textit{Error Correction} to reduce friction through supportive strategies.

%Tailored interventions for specific parental behaviours linked to conflicts can reduce tensions. Behaviors like "Setting Rules" (SR) and "Error Correction" (EC) were associated with conflicts, while "Labelled Praise" (LP) had fewer correlations. Tools offering real-time feedback could help parents replace conflict-inducing actions with more constructive alternatives, such as guiding parents who frequently engage in "Error Correction" to reduce friction through supportive strategies.

\textit{Emotional State-Aware Interaction Design}.
We suggest the system should adapt its interaction style and content based on the parent's emotional state before and during homework sessions.
Our analysis shows that parents with higher pleasure and dominance before a session are more likely to engage in positive behaviours and experience fewer conflicts, and vice versa.
Therefore, we suggest adapting the interaction style and content based on the pre-session emotional state. 
Systems could assess emotional states via self-reporting or subtle cues and adjust guidance accordingly, offering calming exercises or suggesting a delay if stress levels are high.

%The system should adapt its interaction style and content based on the parent's emotional state before and during homework sessions. Our analysis shows that parents with higher pleasure and dominance before a session are more likely to engage in positive behaviors and experience fewer conflicts. Systems could assess emotional states via self-reporting or subtle cues and adjust guidance accordingly, offering calming exercises or suggesting a delay if stress levels are high.


%\subsection{Limitations}
%\label{sec:limit}
This study is the first, to the best of our knowledge, to comprehensively investigate the emotional and behavioural dynamics of parental homework involvement through parent-child conversations. We had to make compromises that may limit its outreach: 
%While our study provides valuable insights into the emotional and behavioural dynamics of Chinese families during homework involvement, several limitations should be acknowledged:

\begin{figure}
    \centering
    \includegraphics[width=0.9
    \textwidth]{figure/hawthorn_dis1.pdf}
    \caption{Different impacts of recording on educational behaviours}
    \label{fig:hawthorn}
\end{figure}


\textit{Sampling Bias}. As outlined in Section \ref{subsec: participants}, the education levels of the parent participants were higher than the national average in China, 
%and most of the children were reported by parents to be performing 'above average' academically. 
This creates a sampling bias, as the data may not represent the broader spectrum of Chinese families. Future studies should aim to include a more diverse sample to better reflect the population as a whole.

\textit{Use of Transcripts Over Acoustic Data}. Our analysis relied solely on transcripts, excluding non-verbal cues such as tone and pitch. While this approach was driven to protect privacy and simplify data processing, it may limit the accuracy of emotion detection, especially since emotional nuances are often more precisely conveyed through audio. Additionally, parental behaviours or conflicts could have been more accurately identified through audio analysis, where variations in tone might reveal different levels of conflict or emotional states even when the verbal content remains the same.

%The analysis relied solely on transcripts, excluding non-verbal cues like tone and pitch. This approach, driven by privacy concerns, may have limited the accuracy of emotion detection and conflict identification, as audio data could better capture emotional nuances.



%Privacy was a top priority, with participants fully informed and data securely stored. However, using transcripts instead of more detailed audio-visual data due to privacy concerns may have constrained the depth of our analysis.

\textit{Hawthorne Effect}. The presence of audio recordings may have influenced parents' behaviour, a phenomenon known as the \textit{Hawthorne Effect}. To mitigate this, we surveyed participants daily, asking them to rate the extent to which the recordings affected their behaviour using a 5-point Likert scale\footnote{The question is \textit{``Due to the fact that you knew the homework involvement behaviour today was recorded, did this affect your true performance?''}}, using a 5-point Likert scale where 1 to 5 indicates \textit{``Completely unaffected''}, \textit{``Basically unaffected'}, \textit{``Slightly affected''}, \textit{``Significantly affected''} and \textit{`Extremely affected''}. As shown in Figure \ref{fig:hawthorn}, most participants (78.04\%) reported being either \textit{'Completely unaffected'} or \textit{'Basically unaffected'}, with only 7.26\% indicating a significant impact on their behaviour. While this suggests minimal influence, the possibility remains that the recordings altered parental behaviour.

\textit{Privacy Concerns}. Privacy is a critical consideration in our research. We took extensive measures to ensure that participants were fully informed about the data collection and future usage, and we strictly limited the scope of our study to homework-related interactions. Data was stored on secure hardware to protect the participants’ privacy. However, using transcripts instead of more detailed audio-visual data due to privacy concerns may have constrained the depth of our analysis.

%The presence of recordings may have influenced parental behavior. Daily surveys revealed that most participants (78.04%) felt their behavior was unaffected or minimally affected by the recordings, but there remains the possibility that behavior was altered due to awareness of being recorded.

%Howthorn effect. In our study, the audio recording may have led parents to become aware of this monitoring, subsequently influencing their educational behaviours, a phenomenon known as the \textit{hawthorne effect}. To address it, participants were surveyed daily to report the impact of the recording on their behaviour\footnote{The question is \textit{`Due to the fact that you knew the homework involvement behaviour today was recorded, did this affect your true performance?'}}, using a 5-point Likert scale where 1 to 5 indicates \textit{`Completely unaffected'}, \textit{`Basically unaffected'}, \textit{`Slightly affected'}, \textit{`Significantly affected'} and \textit{`Extremely affected'}. Figure \ref{fig:hawthorn} illustrates the distribution of responses across participants. Notably, most participants (78.04\%) reported being either \textit{`Completely unaffected'} or \textit{`Basically unaffected'}, with only 7.26\% indicating significant influence (either \textit{`Significantly affected'} or \textit{`Extremely affected'}). This suggests that the \textit{hawthorne effect} was minimal among our participants. Nonetheless, future research should explore audio recording's influence on parental involvement behaviours further.



