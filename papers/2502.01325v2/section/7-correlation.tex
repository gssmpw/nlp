\section{Relationships Between Emotions, Behaviours and Conflict in Chinese Families}

This section explores the intricate relationships among emotions, behaviours, and conflicts during homework involvement in Chinese families. We calculate the total occurrences of each behaviour and conflict for every homework session. Data from 511 homework sessions across 65 families (excluding one due to insufficient data) were analyzed, with emotional shifts being measured as the difference in perceived pleasure, arousal, and dominance levels before and after homework.


\begin{figure}
    \centering
    \includegraphics[width=1\linewidth]{figure/heat_conflict_nums.pdf}
    \caption{Correlation between parental behaviours and parent-child conflicts (* p<0.05, ** p<0.01, *** p<0.001)}
    \label{fig:heatmap:conflict_behaviours}
\end{figure}


Figure \ref{fig:heatmap:conflict_behaviours} displays the correlation matrix between parental behaviours and conflicts, with behaviours categorized as positive, neutral, or negative (indicated by the bold lines). Surprisingly, the results reveal numerous statistically significant positive correlations between behaviours and conflicts, regardless of whether the behaviour is classified as positive, neutral, or negative. Even positive behaviours like praise are often linked with increased conflicts.

Among all behaviours, \textit{Neglect and Indifference} stands out as having no significant correlation with any conflict. This raises key questions: Could a certain level of detachment or more passive involvement help reduce conflict? How do we strike the right balance between involvement and creating a harmonious homework environment?  %The observation that \textit{Neglect and Indifference} (NI) is not significantly correlated with any type of conflict prompts further consideration of the role of active versus passive involvement. 
If conflict tends to arise during active involvement, it may be worth exploring whether less direct involvement could sometimes be more beneficial. The challenge for parents, then, is not merely to be involved but to carefully navigate how they are involved, considering which behaviours might exacerbate conflicts and which might help minimise them.

%Among all behaviors, Neglect and Indifference (NI) notably shows no significant correlation with any form of conflict. This raises key questions: Could a certain level of detachment or passive involvement actually reduce conflict? How do we strike the right balance between engagement and creating a harmonious homework environment? The absence of a link between NI and conflict invites further exploration into the impact of active versus passive involvement. If conflict tends to arise from more direct involvement, perhaps reducing engagement at certain times could be beneficial. These findings highlight that while parental involvement, even with good intentions, often correlates with conflict during homework sessions. The challenge for parents is not just to be involved but to thoughtfully manage their involvement, understanding which behaviors might escalate conflicts and which could help reduce them.

%Figure \ref{fig:heatmap:conflict_behaviours} reveals numerous significant positive correlations between parental behaviours and conflicts, regardless of behaviour type (positive, neutral, or negative). Surprisingly, even positive behaviours like praise are often linked with increased conflicts. Only \textit{Neglect and Indifference} (NI) showed no significant correlation with conflict, raising questions about whether less involvement might reduce conflict. Parental behaviours such as \textit{Setting Rules}, \textit{Error Correction}, and \textit{Criticism} were linked to all types of conflicts, while behaviours like \textit{Praise} and \textit{Direct Instruction} were correlated with fewer conflicts but did not entirely eliminate them.

It also reveals that different parental behaviours are linked to specific types of conflicts. 
%For instance, \textit{Setting Rules (SR)}, \textit{Error Correction} (EC), \textit{Direct Command} (DC), \textit{Criticism and Blame} (CB), \textit{Belittling and Doubting} (BD), and \textit{Impatience and Irritation} (II) are statistically significantly positively correlated with all types of conflicts. 
Parental behaviours such as \textit{Setting Rules}, \textit{Error Correction}, and \textit{Criticism and Blame} were linked to all types of conflicts. This suggests that while these behaviours may be intended to guide or correct the child, they often lead to be associated with increased friction during homework sessions. On the other hand, behaviours like \textit{Labelled Praise}, \textit{Unlabelled Praise}, and \textit{Direct Instruction} show fewer correlations with conflict, although they are not entirely conflict-free. %This indicates that while these behaviours may help mitigate conflict to some extent, but do not completely prevent it.

%The result of "Neglect and Indifference" (NI) stands out as the only behaviour with no significant correlations to conflict, prompting us to reconsider the role of active versus passive involvement. If conflict cannot be entirely avoided during active involvement, it may be worth investigating whether less direct involvement could sometimes be more beneficial. These findings suggest that parental involvement, even when intended to be supportive, often leads to conflict during homework sessions. The challenge for parents, then, is not merely to be involved but to carefully navigate how they are involved, considering which behaviours might exacerbate conflicts and which might help minimize them.


%the key challenge for parents is to carefully consider how they are involved. Further investigation is needed to understand which behaviours are associated with reducing conflicts and promoting a more harmonious homework environment.

%Figure \ref{fig:heatmap:conflict_behaviours} shows the correlation between parental behaviours and parent-child conflicts, where bold lines divides the behaviours into three types: positive, neutral and negative. Surprsingly, we found lots of statistically significant positive correlation between parental beahivours and different types of conclits, no matter behaviours type. Even for the 'postive' behaviours, they are high positively related with the parent-child conflicts. Especailly, only for the 'negalect and indifference ' behaviour, there is no obvious conflict correlation, which reveals a very interesting phynomonon to be discussed: if invovment always bring in conflict, to what extent  should we  encourage involvement. does conflict cannot avoid? is neglect and difference should be encouraged? how to find a balance between...In addition, we also find that, different behaviours may positively correalted with different kind of conflicts. Especially, the Setting Rules (SR), Error Correction (EC), Direct Command (DC), and Critisim and Belame (CB), Belitting and Douting (BD) and Impatience and irritation (II) are positively correalted to all kinds of conflicts. Also, besides the Neglect and Indifference (NI), the other behaviours that are  corrleated to least conflicts are: Specific Prase (SP), General Praise (GP) and Direct Instruction (DI), It shows that...



Figure \ref{fig:heatmap:emotion_conflict_behaviours} illustrates the correlation between perceived emotions, parental behaviours, and parent-child conflicts. The bold lines divide behaviours into positive, neutral, and negative categories, while the red line separates behaviours from conflicts.  While correlation does not imply causation, it is valuable to examine how emotions before and after homework relate to behaviours and conflicts. For instance, parents reporting greater pleasure before homework are significantly positively correlated with positive behaviours like \textit{Encouragement}, and negatively correlated with negative behaviours, such as \textit{Criticism and Blame}, \textit{Forcing and Threatening}, and \textit{Belittling and Doubting}. This indicates that happier parents are more likely to engage in supportive behaviours and less likely to be critical or aggressive during the homework session.

%shows the correlation between perceived emotions, parental behaviors, and parent-child conflicts. The bold lines divide behaviors into positive, neutral, and negative categories, while the red line separates behaviors from conflicts. While correlation does not imply causation, it is valuable to examine how emotions before and after homework relate to behaviors and conflicts. For example, parents reporting greater pleasure before homework are positively correlated with positive behaviors, like Encouragement, and negatively correlated with negative behaviors, such as Criticism, Forcing, and Belittling. This indicates that happier parents are more likely to engage in supportive behaviors and less likely to be critical or aggressive during homework sessions.

\begin{figure}
    \centering
    \includegraphics[width=1\linewidth]{figure/heat_emotion_conflict_behavior.pdf}
    \caption{Correlation between emotions, behaviours and parent-child conflicts (* p<0.05, ** p<0.01, *** p<0.001)}
    \label{fig:heatmap:emotion_conflict_behaviours}
\end{figure}

A similar pattern is observed with dominance before homework. Parents who feel more in control (i.e., higher dominance) are positively correlated with positive behaviours like \textit{Encouragement} and \textit{Labelled/Unlabelled Praise}, while negatively correlated with negative behaviours such as \textit{Impatience and Irritation} and \textit{Frustration and Disappointment}. These parents also experience fewer conflicts, such as \textit{Expectation Conflict}, \textit{Communication Conflict}, and \textit{Time Management Conflict}. This indicates that parents who start homework sessions in a positive emotional state tend to display more constructive behaviours and experience fewer conflicts.

%A similar pattern is observed with dominance before homework. Parents who feel more in control (i.e., higher dominance) are more likely to engage in positive behaviours like \textit{Encouragement} (ENC) and \textit{Labelled/Unlabelled Praise} (LP/UP) and are less prone to negative behaviours like \textit{Impatience and Irritation} (II) and \textit{Frustration and Disappointment} (FD). These parents also experience fewer conflicts, such as \textit{Expectation Conflict} (EC), \textit{Communication Conflict} (CC), and \textit{Time Management Conflict} (TMC). This indicates that a positive emotional state before homework tends to result in more constructive behaviours and fewer conflicts.

Post-homework emotions show a similar trend. Positive behaviours such as \textit{Encouragement} and \textit{Praise} are associated with higher pleasure and dominance, and lower arousal after homework sessions.  This pattern reflects everyday experiences: parents who engage in supportive behaviours often feel more satisfied and calm after homework sessions. In contrast, five out of six negative behaviours are significantly negatively correlated with post-homework pleasure and dominance, and positively correlated with arousal, indicating that negative behaviours tend to leave parents feeling more emotionally agitated and less in control after the session. \textit{Neglect and Indifference}, however, shows no significant correlation with post-homework emotions, suggesting that it may reflect emotional disengagement or detachment during the session.

%Post-homework emotions show a similar trend. Positive behaviours like \textit{Encouragement} (ENC) and \textit{Praise} (LP, UP) are associated with higher pleasure and dominance, and lower arousal, reflecting satisfaction and calmness after homework sessions. On the contrary, five of six negative behaviours are linked to lower post-homework pleasure and dominance, and higher arousal, indicating these behaviours lead to emotional agitation and loss of control. \textit{Neglect and Indifference} (NI), however, shows no significant correlation with post-homework emotions, suggesting emotional disengagement or detachment during the session.

Regarding conflicts, all conflict types are significantly negatively correlated with post-homework pleasure. This suggests that when conflicts occur during homework, parents tend to feel less satisfied or happy afterwards. Interestingly, \textit{Rule Conflict} shows the weakest correlation with post-homework emotions, indicating that it may be perceived as more neutral or procedural, potentially evoking fewer emotional reactions compared to other types of conflicts. This could mean that while disagreements over rules happen, they may not be as emotionally taxing as other conflict types.
On the other hand, \textit{Communication Conflict} and \textit{Knowledge Conflict} show significant correlations with all post-homework emotions as well as with emotional shifts. This suggests that these two types of conflicts are more emotionally engaging for parents and might be more likely to destabilize their emotional states during and after the homework process. Communication conflicts likely involve frustration and misunderstandings that are deeply personal, while knowledge conflicts could highlight gaps in understanding between parents and children, triggering stress or self-doubt in parents. Both conflict types are associated with notable shifts in arousal (indicative of stress) and dominance (control), pointing to their potential to cause significant emotional upheaval.


The above findings underscore the complex role of parental emotions, behaviours and conflicts. While parents' pleasure and dominance before homework are associated with more positive behaviours and fewer conflicts, even parents who begin homework sessions in a good emotional state may encounter emotionally charged situations, particularly in the case of knowledge or communication conflicts. The correlation between post-homework emotions and specific conflicts also suggests that conflicts during homework can have lingering emotional effects on parents.
