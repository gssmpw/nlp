\section{Conclusion}

%While parental homework involvement is widely encouraged in education, the nuanced dynamics of these interactions remain poorly understood due to the limitations of traditional observation methods.
%Our research demonstrates how ubiquitous sensing technologies can reveal previously hidden patterns in family educational practices. Through a novel system combining continuous audio sensing and LLM-powered analysis deployed across 78 families in China, we captured and analyzed authentic parent-child interactions at an unprecedented scale (475 hours of natural conversations).

%While parental homework involvement is often seen as beneficial for students, it can also be a source of significant stress and conflict. Schools typically assume parents will naturally support homework without addressing the complex challenges they face. Our research challenges this assumption through an innovative methodology that combines large-scale audio data capture (475 hours of natural conversations from 78 families) with advanced LLM analysis, revealing previously hidden patterns in family educational practices.

While parental homework involvement is often seen as beneficial for students, it can also be a source of stress and conflict. Despite this, schools typically assume parents will naturally support homework, without fully addressing the complex challenges they face. Our research demonstrates how continuous rich audio data capture (475 hours of natural conversations from 78 families) could be analysed through LLMs-powered conversation analysis, enabling detailed examination of parent-child interactions that would be impractical through traditional manual annotation.

Our findings provide novel insights into the dynamics of homework interactions, demonstrating how parental emotions shift during involvement and how these changes correlate with specific behaviours and conflicts. Notably, we challenge conventional wisdom by showing that even "positive" parental behaviours can inadvertently trigger conflicts, suggesting the need for more nuanced approaches to family education support. These insights can inform the development of future technologies that help parents navigate the emotional complexities of homework involvement while fostering healthier parent-child relationships in educational contexts.

%we captured and analyzed authentic parent-child interactions at an unprecedented scale (475 hours of natural conversations), and exploring how emotions, behaviours, and conflicts emerge during homework interactions. Through an extensive field study of 78 Chinese families, we provide a detailed analysis of how parental emotions shift during homework involvement and how these shifts are linked to specific behaviours and conflicts. Our findings challenge the assumption that ``positive'' parental behaviours are beneficial, revealing the potential for well-meaning actions to inadvertently cause conflict.
%This research offers new perspectives for educators, policymakers, and parents in developing more effective strategies for supporting children's learning. Importantly, our insights have direct implications for the design of technologies in HCI that can help parents navigate these emotional complexities. By informing the development of personalized tools and interventions, we aim to foster healthier parent-child relationships and more effective educational outcomes through thoughtful, technology-driven solutions.

%Parental homework involvement is usually considered a highly rewarding experience for students, but it is also a source of conflict. Despite this, schools assume parents will support homework, with guidelines emphasizing the importance of parental involvement. However, these guidelines don’t fully address the challenges parents face, and homework involvement can often cause conflict and anxiety within families. This study makes a contribution to our understanding of parental homework involvement, particularly within the context of Chinese families. By exploring the dynamics of emotion, behaviours and conflicts during these critical interactions, our work highlights the complex and often challenging role that parents play in supporting their children's education.

%We conducted an extensive field study involving 78 Chinese families and provided a detailed analysis of how parental emotions shift during homework involvement and how these shifts are linked to specific behaviours and conflicts. This research not only deepens our knowledge of family education practices but also challenges the assumption that `positive' parental behaviours are inherently beneficial. By identifying the nuanced ways in which even well-intentioned behaviours can lead to conflict, our study offers new perspectives for educators, policymakers, and parents themselves to consider more effective strategies for supporting children's learning. Looking forward, these findings pave the way for more personalized interventions and tools in HCI fields that can help parents navigate the emotional complexities of homework involvement, ultimately fostering healthier parent-child relationships and more effective educational outcomes.

%In a broader sense, this work opens up new avenues for research and practice in family education. It encourages a rethinking of traditional approaches to parental involvement, advocating for more personalized and context-aware strategies that recognize the emotional complexities of parenting. As we move forward, these insights can inform the development of tools and interventions that better support parents in fostering positive educational environments, ultimately contributing to the well-being and success of future generations.

