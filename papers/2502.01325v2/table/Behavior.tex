\begin{table}
\centering
\scriptsize
\caption{Code definitions and examples of positive, neutral, and negative behaviours}
\label{tab:behaviours} 
\begin{tabular}{p{0.12\textwidth} p{0.40\textwidth} p{0.47\textwidth}}
\toprule
\textbf{Code Name} & \textbf{Code Definition} & \textbf{Example} \\ \midrule
\multicolumn{3}{c}{\textbf{\textit{Positive Behaviours}}}   \\\midrule
\textit{Encouragement (ENC)} & Parents provide verbal or behavioural support to encourage the child's effort and progress, boosting their confidence and motivation to overcome challenges. & \textit{"You've worked really hard, keep it up! I believe in you!"} \newline \textit{"Don't worry, let's take it step by step, you'll definitely get it."} \\ \hline
\textit{Labelled Praise (LP)} & Parents specifically highlight the child's particular action or achievement and offer praise, helping the child recognize their specific progress and strengths. & \textit{"You did great on this addition problem, no mistakes at all!"} \newline \textit{"Your handwriting is especially neat this time, keep it up!"} \\ \hline
\textit{Unlabelled \newline Praise (UP)} & Parents give general praise to the child without pointing out specific actions or achievements. & \textit{"You're amazing, keep going!"} \newline \textit{"Wow, that's awesome!"} \\ \hline
\textit{Guided \newline Inquiry (GI)} & Parents ask questions or provide clues to guide the child toward independent thinking and problem-solving. & \textit{"Where do you think this letter should go?"} \newline \textit{"What methods could we use to solve this problem? Think about a few ways."} \\ \hline
\textit{Setting Rules (SR)} & Parents set clear rules or requirements for completing homework, helping the child establish good study habits and time management skills. & \textit{"You need to finish your Chinese homework before watching cartoons."} \newline \textit{"All homework needs to be done before dinner if you want to go out and play."} \\ \hline
\textit{Sensitive \newline Response (SRS)} & Parents respond to the child's emotions, needs, and behaviours in a timely, appropriate, and caring manner. & \textit{"I can see you're a bit tired now, how about we take a break and continue later?"} \newline \textit{"Do you find this question difficult? Don't worry, let's take another look together."} \\ \midrule
\multicolumn{3}{c}{\textbf{\textit{Neutral Behaviours}}}   \\\midrule

\textit{Direct \newline Instruction (DI)} & Parents tell the child how to complete a task or solve a problem without using guided or inquiry-based methods. & \textit{"For this problem, you should do it like this: add 4 to 6 to get 10."} \newline \textit{"Just copy this answer down, don't overthink it."} \\ \hline
\textit{Information \newline Teaching (IT)} & Parents teach new knowledge or skills by explaining concepts, reading texts, or offering detailed instructions. & \textit{"The character 'tree' is written with a wood radical on the left and 'inch' on the right, let's write it together."} \newline \textit{"You need to memorize multiplication tables like this: two times two equals four, two times three equals six. Let's start with those."} \\ \hline
\textit{Error Correction (EC)} & Parents point out mistakes in the child's homework and guide them to correct or revise their work. & \textit{"You missed the 'wood' radical here, write it again."} \newline \textit{"The addition is wrong here, let's calculate it again. Remember to line up the numbers correctly."} \\ \hline
\textit{Monitoring (MON)} & Parents regularly check the child's homework progress or completion to ensure they stay on track. & \textit{"How many pages have you written? Let me check for mistakes."} \newline \textit{"Let me look over your pinyin homework today to see if everything is correct."} \\ \hline
\textit{Direct Command (DC)} & Parents use clear and direct language to request or command the child to perform a specific action or task. & \textit{"Go do your math homework right now, no more delays!"} \newline \textit{"Stop playing with your toys and go finish your pinyin practice."} \\ \hline
\textit{Indirect Command (IC)} & Parents indirectly convey their requests, often through suggestions or hints, rather than giving direct orders. & \textit{"Have you finished your homework? Maybe it's time to get it done."} \newline \textit{"How about we finish homework first and then go play? That way you won't have to worry about running out of time later."} \\ \midrule

\multicolumn{3}{c}{\textbf{\textit{Negative Behaviours}}}   \\\midrule
\textit{Criticism and \newline Blame (CB)} & Parents express negative evaluations of the child's mistakes or behaviours, often directly blaming the child. & \textit{"How could you mess up such a simple word?"} \newline \textit{"I've told you this a thousand times, why haven't you remembered it yet?"} \\ \hline
\textit{Forcing and \newline Threatening (FT)} & Parents use pressure or threats of consequences to force the child to comply with their demands. & \textit{"If you don't do your homework, you won't be allowed to play with your toys today!"} \newline \textit{"If you don't finish, I'll take away your toys!"} \\ \hline
\textit{Neglect and \newline Indifference (NI)} & Parents show a lack of attention or emotional response to the child's needs or feelings. & \textit{\textbf{Child}: "Mom, I don't understand this question, can you help me?"} \newline \textit{\textbf{Parent}: (no response, continues using phone)} \\ \hline
\textit{Belittling and \newline Doubting (BD)} & Parents belittle the child's abilities or question their performance, undermining the child's confidence and motivation. & \textit{"How could you be so stupid? You can't even solve simple addition."} \newline \textit{"With grades like these, you'll never get into a good school."} \\ \hline
\textit{Frustration and \newline Disappointment (FD)} & Parents express frustration or disappointment in the child's performance when it fails to meet their expectations. & \textit{"I can't believe you did so poorly on this test, I'm really disappointed."} \newline \textit{"I thought you'd do better, but I guess I was wrong."} \\ \hline
\textit{Impatience and \newline Irritation (II)} & Parents exhibit impatience or irritation when the child's performance falls short of expectations. & \textit{"Why are you so slow? I've been waiting forever!"} \newline \textit{"Why isn't this finished yet? You always take so long!"} \\ \bottomrule
\end{tabular}
\end{table}

\iffalse
\begin{table}[h]
\centering
\footnotesize
\caption{Neutral behaviours: Code Definitions and Examples}
\begin{tabular}{p{0.10\textwidth} p{0.40\textwidth} p{0.45\textwidth}}
\hline
\textbf{Code Name} & \textbf{Code Definition} & \textbf{Example} \\ \hline
\textbf{Direct Instruction (DI)} & Parents tell the child how to complete a task or solve a problem without using guided or inquiry-based methods. & "For this problem, you should do it like this: add 4 to 6 to get 10." \newline "Just copy this answer down, don't overthink it." \\ \hline
\textbf{Information \newline Teaching (IT)} & Parents teach new knowledge or skills by explaining concepts, reading texts, or offering detailed instructions. & "The character 'tree' is written with a wood radical on the left and 'inch' on the right, let's write it together." \newline "You need to memorize multiplication tables like this: two times two equals four, two times three equals six. Let's start with those." \\ \hline
\textbf{Error Correction (EC)} & Parents point out mistakes in the child's homework and guide them to correct or revise their work. & "You missed the 'wood' radical here, write it again." \newline "The addition is wrong here, let's calculate it again. Remember to line up the numbers correctly." \\ \hline
\textbf{Monitoring (MON)} & Parents regularly check the child's homework progress or completion to ensure they stay on track. & "How many pages have you written? Let me check for mistakes." \newline "Let me look over your pinyin homework today to see if everything is correct." \\ \hline
\textbf{Direct Command (DC)} & Parents use clear and direct language to request or command the child to perform a specific action or task. & "Go do your math homework right now, no more delays!" \newline "Stop playing with your toys and go finish your pinyin practice." \\ \hline
\textbf{Indirect Command (IC)} & Parents indirectly convey their requests, often through suggestions or hints, rather than giving direct orders. & "Have you finished your homework? Maybe it's time to get it done." \newline "How about we finish homework first and then go play? That way you won't have to worry about running out of time later." \\ \hline


\end{tabular}
\end{table}

\begin{table}[h]
\centering
\footnotesize
\caption{Negative behaviours: Code Definitions and Examples}
\begin{tabular}{p{0.10\textwidth} p{0.40\textwidth} p{0.45\textwidth}}
\hline
\textbf{Code Name} & \textbf{Code Definition} & \textbf{Example} \\ \hline
\textbf{Criticism and Blame (CB)} & Parents express negative evaluations of the child's mistakes or behaviours, often directly blaming the child. & "How could you mess up such a simple word?" \newline "I've told you this a thousand times, why haven't you remembered it yet?" \\ \hline
\textbf{Forcing and Threatening (FT)} & Parents use pressure or threats of consequences to force the child to comply with their demands. & "If you don't do your homework, you won't be allowed to play with your toys today!" \newline "If you don't finish, I'll take away your toys!" \\ \hline
\textbf{Neglect and Indifference (NI)} & Parents show a lack of attention or emotional response to the child's needs or feelings. & \textit{\textbf{Child}: "Mom, I don't understand this question, can you help me?"} \newline \textit{\textbf{Parent}: (no response, continues using phone).} \\ \hline
\textbf{Belittling and Doubting (BD)} & Parents belittle the child's abilities or question their performance, undermining the child's confidence and motivation. & "How could you be so stupid? You can't even solve simple addition." \newline "With grades like these, you'll never get into a good school." \\ \hline
\textbf{Frustration and Disappointment (FD)} & Parents express frustration or disappointment in the child's performance when it fails to meet their expectations. & "I can't believe you did so poorly on this test, I'm really disappointed." \newline "I thought you'd do better, but I guess I was wrong." \\ \hline
\textbf{Impatience and Irritation (II)} & Parents exhibit impatience or irritation when the child's performance falls short of expectations. & "Why are you so slow? I've been waiting forever!" \newline "Why isn't this finished yet? You always take so long!" \\ \hline
\end{tabular}
\end{table}
\fi


% \begin{table}[h]
% \centering
% \small
% \caption{Parent-Child Interaction Codes in Homework Tutoring Scenarios, synthesized by ChatGPT. As there was no comprehensive coding manual specifically tailored for positive, neutral, and negative behaviours in Homework Tutoring, we conducted literature reviews, interviewed educational experts, and leveraged GPT to identify and summarize common parent-child behaviours. After validation and refinement by educational experts, we developed specific definitions and examples for each behaviour type. These codes were then applied to categorize conversation transcripts.}
% \begin{tabular}{p{0.15\textwidth} p{0.40\textwidth} p{0.40\textwidth}}
% \toprule
% \textbf{Code Name} & \textbf{Code Definition} & \textbf{Example} \\ 
% \midrule
% Encouragement (ENC) & Parents provide verbal or behavioural support to encourage the child's effort and progress, boosting their confidence and motivation to overcome challenges. & "You've worked really hard, keep it up! I believe in you!" \\
% \midrule
% Specific Praise (SP) & Parents specifically highlight the child's particular action or achievement and offer praise, helping the child recognize their specific progress and strengths. & "You did great on this addition problem, no mistakes at all!" \\
% \midrule
% General Praise (GP) & Parents give general praise to the child without pointing out specific actions or achievements. & "You're amazing, keep going!" \\
% \midrule
% Guided Inquiry (GI) & Parents ask questions or provide clues to guide the child toward independent thinking and problem-solving. & "Where do you think this letter should go?" \\
% \midrule
% Setting Rules (SR) & Parents set clear rules or requirements for completing homework, helping the child establish good study habits and time management skills. & "You need to finish your Chinese homework before watching cartoons." \\
% \midrule
% Sensitive Response (SRS) & Parents respond to the child's emotions, needs, and behaviours in a timely, appropriate, and caring manner. & "I can see you're a bit tired now, how about we take a break and continue later?" \\
% \midrule
% Direct Instruction (DI) & Parents tell the child how to complete a task or solve a problem without using inquiry-based methods. & "For this problem, you should do it like this: add 4 to 6 to get 10." \\
% \midrule
% Information Teaching (IT) & Parents teach new knowledge or skills by explaining concepts or offering detailed instructions. & "The character 'tree' is written with a wood radical on the left and 'inch' on the right, let's write it together." \\
% \midrule
% Error Correction (EC) & Parents point out mistakes in the child's homework and guide them to correct or revise their work. & "You missed the 'wood' radical here, write it again." \\
% \midrule
% Monitoring (MON) & Parents regularly check the child's homework progress or completion to ensure the child is staying on track. & "How many pages have you written? Let me check for mistakes." \\
% \midrule
% Direct Command (DC) & Parents use clear and direct language to request or command the child to perform a specific action or task. & "Go do your math homework right now, no more delays!" \\
% \midrule
% Indirect Command (IC) & Parents indirectly convey their requests, often through suggestions or hints, rather than giving direct orders. & "Have you finished your homework? Maybe it's time to get it done." \\
% \midrule
% Criticism and Blame (CB) & Parents express negative evaluations of the child's mistakes or behaviours, often directly blaming the child. & "How could you mess up such a simple word?" \\
% \midrule
% Forcing and Threatening (FT) & Parents use pressure or threats of consequences to force the child to comply with their demands. & "If you don't do your homework, you won't be allowed to play with your toys today!" \\
% \midrule
% Neglect and Indifference (NI) & Parents show a lack of attention or emotional response to the child's needs or feelings. & Child: "Mom, I don't understand this question, can you help me?" (No response) \\
% \midrule
% Belittling and Doubting (BD) & Parents belittle the child's abilities or question their performance, undermining the child's confidence and motivation. & "How could you be so stupid? You can't even solve simple addition." \\
% \midrule
% Frustration and Disappointment (FD) & Parents express frustration or disappointment in the child's performance when it fails to meet their expectations. & "I can't believe you did so poorly on this test, I'm really disappointed." \\
% \midrule
% Impatience and Irritation (II) & Parents exhibit impatience or irritation when the child's performance falls short of expectations. & "Why are you so slow? I've been waiting forever!" \\
% \bottomrule
% \end{tabular}
% \end{table}
