\begin{table}[h]
\centering
\footnotesize
\caption{Coding usage guidelines for parent behaviour during homework involvement}
\label{appen:tab:behaviour}
\begin{tabular}{p{0.22\textwidth} p{0.7\textwidth}}
\toprule
\textbf{Code Name} & \textbf{Usage Guidelines} \\ \midrule
\textit{Encouragement (ENC)} & Use this code when parents offer positive reinforcement and encouragement, especially when the child faces difficulties, regardless of the outcome. \\ \hline
\textit{Labelled Praise (LP)} & Use this code when parents clearly point out a specific behaviour, achievement, or performance and give positive feedback. \\ \hline
\textit{Unlabelled Praise (UP)} & Use this code when parents give general, unspecific praise without focusing on particular details or behaviours. \\ \hline
\textit{Guided Inquiry (GI)} & Use this code when parents help the child find a solution by asking questions or offering hints, rather than directly giving the answer. \\ \hline
\textit{Setting Rules (SR)} & Use this code when parents establish clear rules and expect the child to follow them. These rules are often related to homework order, time, or completion standards. \\ \hline
\textit{Sensitive Response (SRS)} & Use this code when parents recognize the child's emotions and provide comfort or emotional support. \\ \hline
\textit{Direct Instruction (DI)} & Use this code when parents provide direct answers or solutions without encouraging the child to think through the problem. \\ \hline
\textit{Information Teaching (IT)} & Use this code when parents provide systematic explanations to help the child understand new knowledge or concepts. \\ \hline
\textit{Error Correction (EC)} & Use this code when parents identify errors in the child's homework and guide them to make corrections. \\ \hline
\textit{Monitoring (MON)} & Use this code when parents monitor the child's homework progress or check the quality of their work. \\ \hline
\textit{Direct Command (DC)} & Use this code when parents give a firm command in a direct and authoritative manner. \\ \hline
\textit{Indirect Command (IC)} & Use this code when parents use a more subtle or indirect approach to encourage the child to complete tasks. \\ \hline
\textit{Criticism \& Blame (CB)} & Use this code when parents criticize or blame the child for not meeting expectations. \\ \hline
\textit{Forcing \& Threatening (FT)} & Use this code when parents use threats or force to make the child complete a task. \\ \hline
\textit{Neglect \& Indifference (NI)} & Use this code when parents ignore or display indifference toward the child's needs, emotions, or requests for attention. \\ \hline
\textit{Belittling \& Doubting (BD)} & Use this code when parents use belittling or doubting language to undermine the child's abilities directly. \\ \hline
\textit{Frustration \& Disappointment (FD)} & Use this code when parents express disappointment due to the child's academic performance or lack of progress. \\ \hline
\textit{Impatience \& Irritation (II)} & Use this code when parents become frustrated or impatient with the child's learning pace or performance. \\ \bottomrule
\end{tabular}
\end{table}

\begin{table}[h]
\centering
\footnotesize
\caption{Coding usage guidelines for parent-child conflict during homework involvement}
\label{appen:tab:conflict}
\begin{tabular}
{p{0.12\textwidth} p{0.83\textwidth}}

\toprule
\textbf{Code Name} & \textbf{Usage Guidelines} \\ 
\midrule

\textit{Expectation \newline Conflict} & Apply this code when there is a significant discrepancy between parents' expectations and the child’s self-perception or goals. This may also include cases where parents compare the child to peers or siblings, adding pressure. \\

\hline
\textit{Communication Conflict} & Use this code when communication breakdowns occur due to negative communication styles such as parental criticism or questioning, leading to frustration. Pay attention to whether the parent blames or belittles the child and the child’s emotional response to it. \\

\hline
\textit{Learning Method Conflict} & Use this code when parents attempt to enforce a change in the child’s learning method or directly intervene, particularly when this disagreement leads to conflict. \\

\hline
\textit{Rule Conflict} & Use this code when the conflict revolves around parental rules, boundaries, or control, such as how and when the child should complete their homework. Pay special attention to cases where the child expresses dissatisfaction with the rules or seeks more autonomy. \\

\hline
\textit{Time \newline Management Conflict} & Use this code when the conflict centers on how study time is managed, the frequency of study sessions, or the child’s energy allocation. This includes situations where the child expresses dissatisfaction with the time constraints imposed by parents. \\

\hline
\textit{Knowledge \newline Conflict} & Apply this code when the conflict stems from a difference in knowledge mastery, a lack of understanding of the child’s learning difficulties, or when the child questions the parent’s knowledge. Pay special attention to instances where the parent underestimates the difficulty of the material for the child. \\

\hline
\textit{Focus Conflict} & Use this code when the conflict is driven by the parent’s dissatisfaction with the child’s focus or attention during study, particularly when repeated parental intervention causes tension. \\

\bottomrule
\end{tabular}
\end{table}

