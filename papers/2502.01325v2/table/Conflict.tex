\begin{table}
\centering
\footnotesize
\caption{Conversations of parent-child conflicts in homework involvement, as synthesized by ChatGPT. Due to the lack of a comprehensive conflict coding manual specifically tailored for Homework Tutoring Scenarios, we conducted extensive literature reviews, interviewed educational experts, and utilized GPT to analyze and summarize common types of Parent-Child Conflicts in these contexts. After validation and refinement by educational experts, we developed specific definitions and examples for each conflict type. We then employed GPT to code our conversation transcripts accordingly.}
\label{tab:conflict}
\begin{tabular}{p{0.10\textwidth} p{0.53\textwidth} p{0.33\textwidth}}
\toprule
\textbf{Code Name} & \textbf{Code Definition} & \textbf{Example} \\ 
\midrule
\textit{Expectation Conflict (EC)} & This conflict arises when parents have high expectations for their child's performance, progress, or future, but the child’s actual abilities, goals, or interests do not align with these expectations. Parents may also compare their child to others, intensifying the conflict. & \textit{\textbf{Parent}: “You should be like your classmate and get full marks! How could you get such an easy question wrong?”} \newline \textit{\textbf{Child}: “I’ve done my best. Why do you always think I’m worse than others?”} \\

\midrule
\textit{Communication Conflict (CC)} & This conflict arises when parents and children have different communication styles during homework sessions, leading to misunderstandings and emotional tension. Parents may criticize, question, or belittle the child, making the child feel misunderstood or oppressed, thus escalating the communication barrier. & \textit{\textbf{Parent}: “What’s wrong with you? I’ve explained this so many times and you still don’t get it!”} \newline \textit{\textbf{Child}: “I just don’t want to listen to you anymore. You always yell at me!”} \\

\midrule
\textit{Learning Method \newline Conflict (LMC)} & This conflict occurs when parents and children disagree on how to approach and complete homework. Parents may feel the child’s method is inefficient and try to impose their own approach, while the child insists on using their own method and resists parental intervention. & \textit{\textbf{Parent}: “You shouldn’t study like this. Finish all the questions first, then check your answers!”} \newline \textit{\textbf{Child}: “I’m used to doing it my way. Why should I follow what you say?”} \\

\midrule
\textit{Rule Conflict (RC)} & This conflict occurs when parents set strict rules for learning, and the child seeks more autonomy. Parents may try to control the pace or structure of the child’s study sessions, while the child resists these restrictions and pushes for greater flexibility and freedom. & \textit{\textbf{Parent}: “You must start your homework right after dinner. No more delays!”} \newline \textit{\textbf{Child}: “I want to play for a little longer. You’re always controlling everything!”} \\

\midrule
\textit{Time \newline Management Conflict (TMC)} & This conflict arises from disagreements about how time and energy should be allocated for studying. Parents may want the child to follow a fixed study schedule, while the child may prefer a different routine, leading to conflict. & \textit{\textbf{Parent}: “You always leave your homework until so late at night. You’re so inefficient!”} \newline \textit{\textbf{Child}: “I prefer studying later. I just can’t focus in the morning!”} \\

\midrule
\textit{Knowledge Conflict (KC)} & This conflict occurs when there is a mismatch in knowledge levels or understanding between parents and children. Parents may have already mastered certain knowledge and find it difficult to empathize with the child’s struggles, or they may explain concepts from a perspective the child cannot yet grasp. Additionally, parents might be unfamiliar with certain subjects, leading the child to question their guidance. & \textit{\textbf{Parent}: “This problem is so simple. How can you still not understand it?”} \newline \textit{\textbf{Child}: “You don’t understand what I’m struggling with! My teacher explained it differently from you.”} \\

\midrule
\textit{Focus Conflict (FC)} & This conflict arises when parents are dissatisfied with the child’s attention or focus during study time, believing the child is distracted or not concentrating sufficiently. Parents may attempt to intervene or remind the child to focus, while the child may feel pressured or overwhelmed by the interference, leading to emotional conflict. & \textit{\textbf{Parent}: “What are you daydreaming about? Focus on your homework!”} \newline \textit{\textbf{Child}: “I wasn’t distracted, I was just thinking about how to solve the problem.”} \\

\bottomrule
\end{tabular}
\end{table}