%%
%% This is file `sample-manuscript.tex',
%% generated with the docstrip utility.
%%
%% The original source files were:
%%
%% samples.dtx  (with options: `all,proceedings,bibtex,manuscript')
%% 
%% IMPORTANT NOTICE:
%% 
%% For the copyright see the source file.
%% 
%% Any modified versions of this file must be renamed
%% with new filenames distinct from sample-manuscript.tex.
%% 
%% For distribution of the original source see the terms
%% for copying and modification in the file samples.dtx.
%% 
%% This generated file may be distributed as long as the
%% original source files, as listed above, are part of the
%% same distribution. (The sources need not necessarily be
%% in the same archive or directory.)
%%
%%
%% Commands for TeXCount
%TC:macro \cite [option:text,text]
%TC:macro \citep [option:text,text]
%TC:macro \citet [option:text,text]
%TC:envir table 0 1
%TC:envir table* 0 1
%TC:envir tabular [ignore] word
%TC:envir displaymath 0 word
%TC:envir math 0 word
%TC:envir comment 0 0
%%
%%
%% The first command in your LaTeX source must be the \documentclass
%% command.
%%
%% For submission and review of your manuscript please change the
%% command to \documentclass[manuscript, screen, review]{acmart}.
%%
%% When submitting camera ready or to TAPS, please change the command
%% to \documentclass[sigconf]{acmart} or whichever template is required
%% for your publication.
%%
%%
\documentclass[acmlarge]{acmart}
%\documentclass[acmlarge,review,anonymous]{acmart}
%\documentclass[manuscript]{acmart}
%%
%% \BibTeX command to typeset BibTeX logo in the docs
\AtBeginDocument{%
  \providecommand\BibTeX{{%
    Bib\TeX}}}

%% Rights management information.  This information is sent to you
%% when you complete the rights form.  These commands have SAMPLE
%% values in them; it is your responsibility as an author to replace
%% the commands and values with those provided to you when you
%% complete the rights form.
\setcopyright{acmlicensed}
\acmJournal{IMWUT}
\acmYear{2025}
\acmVolume{9} \acmNumber{2} 
\acmDOI{XXXXXXX.XXXXXXX}
\acmArticle{108} \acmMonth{9}\acmDOI{XXXXX/XXXXXX}

%\acmYear{2024} \acmVolume{8} \acmNumber{3} \acmArticle{108} \acmMonth{9}\acmDOI{10.1145/3678525}


%% These commands are for a PROCEEDINGS abstract or paper.
%\acmConference[CHI '25]{ACM Conference on Human Factors in Computing Systems}{26 April - 1 May 2025}{Yokohama, Japan}
%%  Uncomment \acmBooktitle if the title of the proceedings is different
%%  from ``Proceedings of ...''!
%%
%%\acmBooktitle{Woodstock '18: ACM Symposium on Neural Gaze Detection,
%%  June 03--05, 2018, Woodstock, NY}
\acmISBN{978-1-4503-XXXX-X/18/06}
\usepackage{booktabs}
\usepackage{multirow}
\usepackage[normalem]{ulem}
\usepackage[utf8]{inputenc}
\usepackage{subfigure}
\usepackage{spverbatim}
\usepackage{booktabs}
\usepackage{array}
\usepackage{makecell}
\usepackage{longtable}
\usepackage{ragged2e}
\usepackage{tabularx} % 引入 tabularx 宏包
\usepackage{geometry}
\usepackage{enumitem}
\usepackage{listings}
\usepackage{xcolor}
\lstset{
    basicstyle=\ttfamily\footnotesize,
    keywordstyle=\color{blue},
    commentstyle=\color{gray},
    stringstyle=\color{orange},
    showstringspaces=false,
    columns=fullflexible,
    frame=single,
    breaklines=true,
}

%%
%% Submission ID.
%% Use this when submitting an article to a sponsored event. You'll
%% receive a unique submission ID from the organizers
%% of the event, and this ID should be used as the parameter to this command.
%%\acmSubmissionID{123-A56-BU3}

%%
%% For managing citations, it is recommended to use bibliography
%% files in BibTeX format.
%%
%% You can then either use BibTeX with the ACM-Reference-Format style,
%% or BibLaTeX with the acmnumeric or acmauthoryear sytles, that include
%% support for advanced citation of software artefact from the
%% biblatex-software package, also separately available on CTAN.
%%
%% Look at the sample-*-biblatex.tex files for templates showcasing
%% the biblatex styles.
%%

%%
%% The majority of ACM publications use numbered citations and
%% references.  The command \citestyle{authoryear} switches to the
%% "author year" style.
%%
%% If you are preparing content for an event
%% sponsored by ACM SIGGRAPH, you must use the "author year" style of
%% citations and references.
%% Uncommenting
%% the next command will enable that style.
%%\citestyle{acmauthoryear}


%%
%% end of the preamble, start of the body of the document source.

\newcommand{\yun}[1]{{\small\textcolor{blue}{\bf [#1]}}}
\newcommand{\tobe}[1]{{\small\textcolor{red}{\bf [#1]}}}

\begin{document}

%%
%% The "title" command has an optional parameter,
%% allowing the author to define a "short title" to be used in page headers.
%\title{From Conflict to Harmony: Developing Family Education Strategies through Parent-Child Conversations in Homework Involvement}
\title %[Exploring Emotions, Behaviours and  Conflicts Through Parent-Child Conversations in China]
{The Homework Wars: Exploring Emotions, Behaviours, and Conflicts in Parent-Child Homework Interactions
%Exploring Emotions, Behaviors, and Conflicts in Parental Homework Involvement through Parent-Child Conversations
}
%{Unpacking Parental Homework Involvement: Exploring Emotions, Behaviours and  Conflicts Through Parent-Child Conversations in China}

%% The "author" command and its associated commands are used to define
%% the authors and their affiliations.
%% Of note is the shared affiliation of the first two authors, and the
%% "authornote" and "authornotemark" commands
%% used to denote shared contribution to the research.
\author{Nan Gao}
%\authornote{Both authors contributed equally to this research.}
\email{nangao@tsinghua.edu.cn}
%\orcid{1234-5678-9012}
\affiliation{%
  \institution{Department of Computer Science and Technology, Tsinghua University}
  \city{Beijing}
  \country{China}
}
\author{Yibin Liu}
%\authornotemark[1]
\email{liuyibin@stumail.neu.edu.cn}
\affiliation{%
  \institution{ Information Science and Engineering College, Northeastern University}
  \city{Shenyang}
  \country{China}
}

\author{Xin Tang}
\email{tangxin@bupt.edu.cn}
\affiliation{%
  \institution{ Beijing University of Post and Telecommunication}
  \city{Beijing}
  \country{China}
}


\author{Yanyan Liu}
\email{yy-liu22@mails.tsinghua.edu.cn}
\affiliation{%
  \institution{Institute of Education, Tsinghua University}
  \city{Beijing}
  \country{China}
}

\author{Chun Yu}
\email{chunyu@tsinghua.edu.cn}
\affiliation{%
  \institution{Department of Computer Science and Technology, Tsinghua University}
  \city{Beijing}
  \country{China}
}

\author{Yun Huang}
\email{yunhuang@illinois.edu}
\affiliation{%
  \institution{School of Information Sciences, University of Illinois at Urbana-Champaign}
  \city{Champaign}
  \country{USA}
}

\author{Yuntao Wang}
\email{yuntaowang@tsinghua.edu.cn}
\affiliation{
  \institution{Department of Computer Science and Technology, Tsinghua University}
  \city{Beijing}
  \country{China}
  %\postcode{1466}
}

\author{Flora D. Salim}
\email{flora.salim@unsw.edu.au}
%\orcid{0000-0002-1237-1664}
\affiliation{
  \institution{University of New South Wales (UNSW)}
  \city{Sydney}
  \country{Australia}
  \postcode{1466}
}


\author{Xuhai "Orson" Xu}
\email{xx2489@columbia.edu}
\affiliation{%
  \institution{Google}
  \city{New York}
  \country{USA}
}

\author{Jun Wei}
\email{weijun8781@tsinghua.edu.cn}
\affiliation{%
  \institution{Institute of Education, Tsinghua University}
 \city{Beijing}
  \country{China}
}

\author{Yuanchun Shi}
\email{ shiyc@tsinghua.edu.cn}
\affiliation{%
  \institution{Department of Computer Science and Technology, Tsinghua University}
  \city{Beijing}
  \country{China}
}
\renewcommand{\shortauthors}{Gao et al.}

\begin{abstract}
%Parental involvement in homework is a crucial aspect of family education, yet it often leads to emotional strain and conflicts. Traditional methods of studying this involvement, such as one-time surveys and interviews, are susceptible to biases and fail to capture the nuanced dynamics at play. In this research, we conducted a 4-week study with 78 Chinese families, collecting 602 valid audio recordings (totalling 475 hours) alongside corresponding daily surveys that capture real-world parental homework involvement. We investigate Chinese parents' emotional shifts before and after homework involvement, identifying the common positive, natural and negative Chinese parent behaviours, and catalogued ten prevalent parent-child conflicts in Chinese families. We found that the communication style conflicts being the most frequent conflicts amount Chiense families. Notably, we found that even positive and neutral parental behaviours could be linked to conflicts, underscoring the complexity of Chiense parental homework involvement. The intersection of cultural differences and family education is a neglected area within HCI and we highlighted the in Chinese families

Parental involvement in homework is a crucial aspect of family education, but it often leads to emotional strain and conflicts that can severely impact family well-being. This paper presents findings from a 4-week in situ study involving 78 families in China, where we collected and analyzed 602 valid audio recordings (totalling 475 hours) and daily surveys. Leveraging large language models (LLMs) to analyze parent-child conversations, we gained a nuanced understanding of emotional and behavioural dynamics that overcomes the limitations of traditional one-time surveys and interviews. Our findings reveal significant emotional shifts in parents before and after homework involvement and summarise a range of positive, neutral and negative parental behaviours. We also catalogue seven common conflicts, with \textit{Knowledge Conflict} being the most frequent. Notably, we found that even well-intentioned parental behaviours, such as \textit{Unlabelled Praise}, were significantly positively correlated with specific conflict types. This work advances ubiquitous computing's research to sense and understand complex family dynamics, while offering evidence-based insights for designing future ambient intelligent systems to support healthy family education environments.
%This study advances ubiquitous computing research by demonstrating how continuous sensing and automated analysis can reveal nuanced patterns in family education practices, while offering evidence-based insights for designing future ambient intelligent systems to support healthy parent-child interactions.

%Parental involvement in homework is a crucial aspect of family education, but it often leads to emotional strain and conflicts, especially within the cultural context of China. This paper presents findings from a 4-week ubiquitous sensing study involving 78 Chinese families, where we continuously collected and analyzed 602 valid audio recordings (totalling 475 hours) and daily surveys. Leveraging an automated pipeline combining continuous audio sensing with large language models (LLMs), we gained a nuanced understanding of emotional and behavioural dynamics that overcomes the limitations of traditional one-time surveys and interviews. Our findings reveal significant emotional shifts in parents before and after homework involvement and summarise a range of positive, neutral and negative parental behaviours. We also catalogue seven common conflicts, with \textit{Knowledge Conflict} being the most frequent. Interestingly, even well-intentioned or neutral parental behaviours were found to be positively associated with conflicts, underscoring the complexity of these interactions. This study contributes to ubiquitous computing research by demonstrating how continuous sensing and automated analysis can reveal nuanced patterns in family education practices, while offering insights for designing future ambient intelligent systems.

%PCS version: Parental involvement in homework is important for family education but often leads to emotional strain and conflicts, especially within China's cultural context. This paper presents findings from a 4-week in-situ study involving 78 Chinese families, collecting 602 audio recordings (totalling 475 hours) and daily surveys. Utilising LLMs to analyze parent-child conversations, we gained a nuanced understanding of emotional and behavioural dynamics that traditional surveys and interviews miss. Our findings reveal significant emotional shifts in parents before and after homework and identify positive, neutral and negative parental behaviours. We also catalogue seven common conflicts, with knowledge conflict being the most frequent. Interestingly, even well-intentioned or neutral parental behaviours were positively associated with conflicts, underscoring the complexity of these interactions. This study contributes to a neglected area in HCI by exploring the intersection of family education and Chinese cultural norms, offering insights for designing technologies to improve parenting practices in China.  

\end{abstract}

%%
%% The code below is generated by the tool at http://dl.acm.org/ccs.cfm.
%% Please copy and paste the code instead of the example below.
%%
\begin{CCSXML}

%• Human-centered computing → Empirical studies in ubiq-uitous and mobile computing; • Applied computing 
<ccs2012>
 <concept>
  <concept_id>10010520.10010553.10010562</concept_id>
  <concept_desc>Human-centered computing~Empirical studies in ubiquitous and mobile computing</concept_desc>
  <concept_significance>500</concept_significance>
 </concept>
 <concept>
  <concept_id>10010520.10010575.10010755</concept_id>
  <concept_desc>Computer systems organization~Redundancy</concept_desc>
  <concept_significance>300</concept_significance>
 </concept>
 <concept>
  <concept_id>10010520.10010553.10010554</concept_id>
  <concept_desc>Computer systems organization~Robotics</concept_desc>
  <concept_significance>100</concept_significance>
 </concept>
 <concept>
  <concept_id>10003033.10003083.10003095</concept_id>
  <concept_desc>Networks~Network reliability</concept_desc>
  <concept_significance>100</concept_significance>
 </concept>
</ccs2012>
\end{CCSXML}
\ccsdesc[500]{Human-centered computing~Empirical studies in ubiquitous and mobile computing}
\ccsdesc[300]{Applied computing}
\ccsdesc{User studies}

%%
%% Keywords. The author(s) should pick words that accurately describe
%% the work being presented. Separate the keywords with commas.
\keywords{Family Education, Human Behavioural Modelling, Parent-Child Interaction, Large Language Models, Parental Homework Involvement}


%\received{20 February 2007}
%\received[revised]{12 March 2009}
%\received[accepted]{5 June 2009}

%%
%% This command processes the author and affiliation and title
%% information and builds the first part of the formatted document.
\maketitle

%!TEX root = gcn.tex
\section{Introduction}
Graphs, representing structural data and topology, are widely used across various domains, such as social networks and merchandising transactions.
Graph convolutional networks (GCN)~\cite{iclr/KipfW17} have significantly enhanced model training on these interconnected nodes.
However, these graphs often contain sensitive information that should not be leaked to untrusted parties.
For example, companies may analyze sensitive demographic and behavioral data about users for applications ranging from targeted advertising to personalized medicine.
Given the data-centric nature and analytical power of GCN training, addressing these privacy concerns is imperative.

Secure multi-party computation (MPC)~\cite{crypto/ChaumDG87,crypto/ChenC06,eurocrypt/CiampiRSW22} is a critical tool for privacy-preserving machine learning, enabling mutually distrustful parties to collaboratively train models with privacy protection over inputs and (intermediate) computations.
While research advances (\eg,~\cite{ccs/RatheeRKCGRS20,uss/NgC21,sp21/TanKTW,uss/WatsonWP22,icml/Keller022,ccs/ABY318,folkerts2023redsec}) support secure training on convolutional neural networks (CNNs) efficiently, private GCN training with MPC over graphs remains challenging.

Graph convolutional layers in GCNs involve multiplications with a (normalized) adjacency matrix containing $\numedge$ non-zero values in a $\numnode \times \numnode$ matrix for a graph with $\numnode$ nodes and $\numedge$ edges.
The graphs are typically sparse but large.
One could use the standard Beaver-triple-based protocol to securely perform these sparse matrix multiplications by treating graph convolution as ordinary dense matrix multiplication.
However, this approach incurs $O(\numnode^2)$ communication and memory costs due to computations on irrelevant nodes.
%
Integrating existing cryptographic advances, the initial effort of SecGNN~\cite{tsc/WangZJ23,nips/RanXLWQW23} requires heavy communication or computational overhead.
Recently, CoGNN~\cite{ccs/ZouLSLXX24} optimizes the overhead in terms of  horizontal data partitioning, proposing a semi-honest secure framework.
Research for secure GCN over vertical data  remains nascent.

Current MPC studies, for GCN or not, have primarily targeted settings where participants own different data samples, \ie, horizontally partitioned data~\cite{ccs/ZouLSLXX24}.
MPC specialized for scenarios where parties hold different types of features~\cite{tkde/LiuKZPHYOZY24,icml/CastigliaZ0KBP23,nips/Wang0ZLWL23} is rare.
This paper studies $2$-party secure GCN training for these vertical partition cases, where one party holds private graph topology (\eg, edges) while the other owns private node features.
For instance, LinkedIn holds private social relationships between users, while banks own users' private bank statements.
Such real-world graph structures underpin the relevance of our focus.
To our knowledge, no prior work tackles secure GCN training in this context, which is crucial for cross-silo collaboration.


To realize secure GCN over vertically split data, we tailor MPC protocols for sparse graph convolution, which fundamentally involves sparse (adjacency) matrix multiplication.
Recent studies have begun exploring MPC protocols for sparse matrix multiplication (SMM).
ROOM~\cite{ccs/SchoppmannG0P19}, a seminal work on SMM, requires foreknowledge of sparsity types: whether the input matrices are row-sparse or column-sparse.
Unfortunately, GCN typically trains on graphs with arbitrary sparsity, where nodes have varying degrees and no specific sparsity constraints.
Moreover, the adjacency matrix in GCN often contains a self-loop operation represented by adding the identity matrix, which is neither row- nor column-sparse.
Araki~\etal~\cite{ccs/Araki0OPRT21} avoid this limitation in their scalable, secure graph analysis work, yet it does not cover vertical partition.

% and related primitives
To bridge this gap, we propose a secure sparse matrix multiplication protocol, \osmm, achieving \emph{accurate, efficient, and secure GCN training over vertical data} for the first time.

\subsection{New Techniques for Sparse Matrices}
The cost of evaluating a GCN layer is dominated by SMM in the form of $\adjmat\feamat$, where $\adjmat$ is a sparse adjacency matrix of a (directed) graph $\graph$ and $\feamat$ is a dense matrix of node features.
For unrelated nodes, which often constitute a substantial portion, the element-wise products $0\cdot x$ are always zero.
Our efficient MPC design 
avoids unnecessary secure computation over unrelated nodes by focusing on computing non-zero results while concealing the sparse topology.
We achieve this~by:
1) decomposing the sparse matrix $\adjmat$ into a product of matrices (\S\ref{sec::sgc}), including permutation and binary diagonal matrices, that can \emph{faithfully} represent the original graph topology;
2) devising specialized protocols (\S\ref{sec::smm_protocol}) for efficiently multiplying the structured matrices while hiding sparsity topology.


 
\subsubsection{Sparse Matrix Decomposition}
We decompose adjacency matrix $\adjmat$ of $\graph$ into two bipartite graphs: one represented by sparse matrix $\adjout$, linking the out-degree nodes to edges, the other 
by sparse matrix $\adjin$,
linking edges to in-degree nodes.

%\ie, we decompose $\adjmat$ into $\adjout \adjin$, where $\adjout$ and $\adjin$ are sparse matrices representing these connections.
%linking out-degree nodes to edges and edges to in-degree nodes of $\graph$, respectively.

We then permute the columns of $\adjout$ and the rows of $\adjin$ so that the permuted matrices $\adjout'$ and $\adjin'$ have non-zero positions with \emph{monotonically non-decreasing} row and column indices.
A permutation $\sigma$ is used to preserve the edge topology, leading to an initial decomposition of $\adjmat = \adjout'\sigma \adjin'$.
This is further refined into a sequence of \emph{linear transformations}, 
which can be efficiently computed by our MPC protocols for 
\emph{oblivious permutation}
%($\Pi_{\ssp}$) 
and \emph{oblivious selection-multiplication}.
% ($\Pi_\SM$)
\iffalse
Our approach leverages bipartite graph representation and the monotonicity of non-zero positions to decompose a general sparse matrix into linear transformations, enhancing the efficiency of our MPC protocols.
\fi
Our decomposition approach is not limited to GCNs but also general~SMM 
by 
%simply 
treating them 
as adjacency matrices.
%of a graph.
%Since any sparse matrix can be viewed 

%allowing the same technique to be applied.

 
\subsubsection{New Protocols for Linear Transformations}
\emph{Oblivious permutation} (OP) is a two-party protocol taking a private permutation $\sigma$ and a private vector $\xvec$ from the two parties, respectively, and generating a secret share $\l\sigma \xvec\r$ between them.
Our OP protocol employs correlated randomnesses generated in an input-independent offline phase to mask $\sigma$ and $\xvec$ for secure computations on intermediate results, requiring only $1$ round in the online phase (\cf, $\ge 2$ in previous works~\cite{ccs/AsharovHIKNPTT22, ccs/Araki0OPRT21}).

Another crucial two-party protocol in our work is \emph{oblivious selection-multiplication} (OSM).
It takes a private bit~$s$ from a party and secret share $\l x\r$ of an arithmetic number~$x$ owned by the two parties as input and generates secret share $\l sx\r$.
%between them.
%Like our OP protocol, o
Our $1$-round OSM protocol also uses pre-computed randomnesses to mask $s$ and $x$.
%for secure computations.
Compared to the Beaver-triple-based~\cite{crypto/Beaver91a} and oblivious-transfer (OT)-based approaches~\cite{pkc/Tzeng02}, our protocol saves ${\sim}50\%$ of online communication while having the same offline communication and round complexities.

By decomposing the sparse matrix into linear transformations and applying our specialized protocols, our \osmm protocol
%($\prosmm$) 
reduces the complexity of evaluating $\numnode \times \numnode$ sparse matrices with $\numedge$ non-zero values from $O(\numnode^2)$ to $O(\numedge)$.

%(\S\ref{sec::secgcn})
\subsection{\cgnn: Secure GCN made Efficient}
Supported by our new sparsity techniques, we build \cgnn, 
a two-party computation (2PC) framework for GCN inference and training over vertical
%ly split
data.
Our contributions include:

1) We are the first to explore sparsity over vertically split, secret-shared data in MPC, enabling decompositions of sparse matrices with arbitrary sparsity and isolating computations that can be performed in plaintext without sacrificing privacy.

2) We propose two efficient $2$PC primitives for OP and OSM, both optimally single-round.
Combined with our sparse matrix decomposition approach, our \osmm protocol ($\prosmm$) achieves constant-round communication costs of $O(\numedge)$, reducing memory requirements and avoiding out-of-memory errors for large matrices.
In practice, it saves $99\%+$ communication
%(Table~\ref{table:comm_smm}) 
and reduces ${\sim}72\%$ memory usage over large $(5000\times5000)$ matrices compared with using Beaver triples.
%(Table~\ref{table:mem_smm_sparse}) ${\sim}16\%$-

3) We build an end-to-end secure GCN framework for inference and training over vertically split data, maintaining accuracy on par with plaintext computations.
We will open-source our evaluation code for research and deployment.

To evaluate the performance of $\cgnn$, we conducted extensive experiments over three standard graph datasets (Cora~\cite{aim/SenNBGGE08}, Citeseer~\cite{dl/GilesBL98}, and Pubmed~\cite{ijcnlp/DernoncourtL17}),
reporting communication, memory usage, accuracy, and running time under varying network conditions, along with an ablation study with or without \osmm.
Below, we highlight our key achievements.

\textit{Communication (\S\ref{sec::comm_compare_gcn}).}
$\cgnn$ saves communication by $50$-$80\%$.
(\cf,~CoGNN~\cite{ccs/KotiKPG24}, OblivGNN~\cite{uss/XuL0AYY24}).

\textit{Memory usage (\S\ref{sec::smmmemory}).}
\cgnn alleviates out-of-memory problems of using %the standard 
Beaver-triples~\cite{crypto/Beaver91a} for large datasets.

\textit{Accuracy (\S\ref{sec::acc_compare_gcn}).}
$\cgnn$ achieves inference and training accuracy comparable to plaintext counterparts.
%training accuracy $\{76\%$, $65.1\%$, $75.2\%\}$ comparable to $\{75.7\%$, $65.4\%$, $74.5\%\}$ in plaintext.

{\textit{Computational efficiency (\S\ref{sec::time_net}).}} 
%If the network is worse in bandwidth and better in latency, $\cgnn$ shows more benefits.
$\cgnn$ is faster by $6$-$45\%$ in inference and $28$-$95\%$ in training across various networks and excels in narrow-bandwidth and low-latency~ones.

{\textit{Impact of \osmm (\S\ref{sec:ablation}).}}
Our \osmm protocol shows a $10$-$42\times$ speed-up for $5000\times 5000$ matrices and saves $10$-2$1\%$ memory for ``small'' datasets and up to $90\%$+ for larger ones.


\section{Related Work}
\label{related}
\textbf{Copyright infringement in text-to-image models.}
Recent research \cite{carlini2023extracting, somepalli2023diffusion, somepalli2023understanding, gu2023memorization, wang2024replication, wen2024detecting, chiba2024probabilistic} highlights the potential for copyright infringement in text-to-image models. These models are trained on vast datasets that often include copyrighted material, which could inadvertently be memorized by the model during training \cite{vyas2023provable, ren2024copyright}. Additionally, several studies have pointed out that synthetic images generated by these models might violate IP rights due to the inclusion of elements or styles that resemble copyrighted works \cite{poland2023generative, wang2024stronger}. Specifically, models like stable diffusion \cite{Rombach_2022_CVPR} may generate images that bear close resemblances to copyrighted artworks, thus raising concerns about IP infringement \cite{shi2024rlcp}. 

\textbf{Image infringement detection and mitigation.}
The current mainstream infringing image detection methods primarily measure the distance or invariance in pixel or embedding space \cite{carlini2023extracting, somepalli2023diffusion, shi2024rlcp, wang2021bag, wang2024image}. For example, \citeauthor{carlini2023extracting} uses the $L_2$ norm to retrieve memorized images. \citeauthor{somepalli2023diffusion} use SSCD \cite{pizzi2022self}, which learns the invariance of image transformations to identify memorized prompts by comparing the generated images with the original training ones. \citeauthor{zhang2018unreasonable} compare image similarity using the LPIPS distance, which aligns with human perception but has limitations in capturing certain nuances. \citeauthor{wang2024image} transform the replication level of each image replica pair into a probability density function. Studies \cite{wen2024detecting, wang2024evaluating} have shown that these methods have lower generalization capabilities and lack interpretability because they do not fully align with human judgment standards. For copyright infringement mitigation, the current approaches mainly involve machine unlearning to remove the model's memory of copyright information \cite{bourtoule2021machine, nguyen2022survey, kumari2023ablating, zhang2024forget} or deleting duplicated samples from the training data \cite{webster2023duplication, somepalli2023understanding}. These methods often require additional model training. On the other hand, \citeauthor{wen2024detecting} have revealed the overfitting of memorized samples to specific input texts and attempt to modify prompts to mitigate the generation of memorized data. Similarly, \citeauthor{wang2024evaluating} leverage LVLMs to detect copyright information and use this information as negative prompts to suppress the generation of infringing images.















\section{Understanding Parental Homework Involvement in China}

To design technologies for improving parental practices, it's crucial to understand parental involvement in homework scenarios. In this research, we conducted a 4-week real-world data collection, recording parent-child conversations during homework sessions. 
%This approach provides detailed insights necessary for designing technologies for improving parenting strategies. %Specifically, we aim to answer the following questions: \textit{1. What does parental homework involvement in China entail, and do parents experience emotional changes post-homework?}
%\textit{2. What types of parental behaviours and parent-child conflicts arise during homework involvement in China?}
%\textit{3. How are parents' emotions, behaviours, and conflicts interconnected during homework involvement?} 
The data collection was approved by the Human Research Ethics Committee at our university.

\begin{table}

%\begin{minipage}[]{6cm}
\caption{Demographics of parent participants}
\label{tab:par_parent}
\begin{tabular}{@{}llll@{}}
\toprule
\textbf{Item}               & \textbf{Option    }                             & \textbf{Count} & \textbf{Percentage} \\ \midrule
\multirow{2}{*}{\textit{Gender}}    & Female (mother)                        & 68    & 87.18\%    \\
                           & Male (father)                          & 10    & 12.82\%    \\\hline
\multirow{4}{*}{\textit{Age}}       & $\leq$ 30                                    & 5     & 6.41\%     \\
                           & 31-40                                  & 48    & 61.54\%    \\
                           & 41-50                                  & 25    & 32.05\%    \\
                           & $\geq$50                                    & 0     & 0.00\%     \\\hline
\multirow{5}{*}{\begin{tabular}[c]{@{}l@{}}\textit{Education} \\ \textit{Level}\end{tabular}} & Junior high school and below           & 0     & 0          \\
                           & High school/Secondary school & 4     & 5.13\%     \\
                           & Associate degree (Junior college)      & 7     & 8.97\%     \\
                           & Bachelor's degree                      & 45    & 57.69\%    \\
                           & Postgraduate degree and above          & 22    & 28.21\%    \\ \bottomrule
\end{tabular}
%\end{minipage}

\iffalse
\hspace{2cm}
\begin{minipage}[]{6cm}
\caption{Basic information of the children of parent participants}
\label{tab:par_child}
\begin{tabular}{@{}llll@{}}
\toprule
\textbf{Item}               & \textbf{Option}            & \textbf{Ct.} & \textbf{Percent.} \\ \midrule
\multirow{2}{*}{\textit{Gender}}& Female (daughter) & 38    & 48.72   \\
& Male (son)        & 40    & 51.28\% \\\hline
\multirow{3}{*}{\textit{Grade}} & Year 1            & 31    & 39.74\% \\
& Year 2            & 27    & 34.62\% \\
& Year 3            & 20    & 25.64\% \\\hline
\multirow{5}{*}{\begin{tabular}[c]{@{}l@{}}\textit{Academic} \\ \textit{Rank in} \\ \textit{Class}\end{tabular}} & Bottom            & 1     & 1.28\%  \\
& Below Average     & 9     & 11.54\% \\
& Average           & 13    & 16.67\% \\
& Above Average     & 35    & 44.87\% \\
& Top               & 20    & 25.64\% \\ \bottomrule 
\end{tabular}

\end{minipage}
\fi
\end{table}

\subsection{Participants}
\label{subsec: participants}
Participants were recruited based on specific criteria: they had to be parents of primary school students in Years 1-3, who were regularly assisted with homework and communicated in Mandarin at home. We distributed advertising leaflets and posters through various channels, including word of mouth, social platforms like WeChat, and school principals who facilitated further distribution through teachers. 

Within a week of registration, a total of 121 individuals completed the registration questionnaire and provided consent for data collection. Among them, 106 met the eligibility criteria, with 15 participants excluded due to reasons such as their children being in Year 4, the involvement of home-based teachers rather than parents, or infrequent homework involvement (defined as one or fewer instances per week). Out of the 106 eligible participants, 85 were successfully contacted; however, seven of them withdrew without contributing data.

The final sample consisted of 78 Chinese parents, with only one parent from each family included.  Detailed demographic information is provided in Table \ref{tab:par_parent}. Notably, the majority of participants (87.18\%) were mothers. Additionally, the education level of the participants was generally higher than the national average in China \cite{gps}. This may be attributed to two primary factors: 1) more educated parents might be more motivated to participate to obtain a family education report as part of the compensation; 2) these parents likely had better access to our advertisements through social media and other channels. %Regarding the children, their academic rankings, as reported by parents, were not evenly distributed, with most being described as \textit{``above-average''}. This imbalance suggests a sampling bias, where parents of higher-achieving students might be more invested in education and thus more inclined to participate. 
We acknowledge this limitation and address it further in Section \ref{sec:limit}.

\subsection{Procedure}

Before formal data collection began, participants completed a registration questionnaire that gathered basic demographic information, details on their habits of parental home involvement, and their satisfaction and emotions related to these activities. The insights gained from this preliminary stage informed the design of daily surveys. 

During the data collection, participants first completed an online background survey\footnote{The first background survey data was not used in this research.}. They were then instructed to record audio during homework sessions with their children based on their usual routines (not necessarily daily). To minimize the \textit{Hawthorne Effect} \cite{adair1984hawthorne} — where individuals alter their behaviour due to awareness of being observed - participants were encouraged to act naturally and were assured that their privacy would be strictly protected, with data used solely for research purposes.


After recording, participants could upload their audio via a private link or send it to our assistant, who stored it on secured hardware. On days when audio was recorded, participants were asked to complete a daily online questionnaire immediately after the homework session. Each parent could submit up to 10 sets of audio recordings and corresponding daily questionnaires over the four weeks. This flexibility accommodated variations in the frequency of homework involvement in different families, with some parents participating daily and others two or three times per week. Additionally, participants were asked to complete a second background survey\footnote{The second background survey data was not used in this research.} at the end of the first week. To minimize the burden, parents used their own smartphones or preferred devices to record the entire homework process. Although some parents may check in only at the end of the session, we still require recordings from start to finish to fully understand when and how parents intervene during homework sessions. 

%Participants could upload audio via a private link or send it to our assistant, who stored it on secured hardware. On days when audio was recorded, participants completed a daily online questionnaire immediately after the session. Each parent could submit up to 10 sets of audio recordings and daily questionnaires over the four-week period, allowing for variation in homework involvement frequency, with some parents participating daily and others two to three times per week. Additionally, participants completed a second background survey\footnote{The second background survey data was not used in this research.} at the end of the first week. To minimize burden, parents used their own devices to record the entire homework process, even if they usually only check in at the end, allowing us to capture how and when parents intervene during homework sessions.

\begin{figure}
    \centering
    \begin{minipage}[t][6cm]{0.45\textwidth}
        \includegraphics[width=0.95\textwidth]{figure/dis_length.pdf}
        \caption{The distribution of audio recording durations}
        \label{fig:dis_length}
    \end{minipage}
    \hspace{0.2cm}
    \begin{minipage}[t][6cm]{0.45\textwidth}
    \centering
        \includegraphics[width=0.94\textwidth]{figure/dis_count.pdf}
        \caption{Number of audio recordings for different participants}
        \label{fig:dis_count}
    \end{minipage}
    
\end{figure}
Participants were compensated up to RMB 240 for their involvement (RMB 40 for the background survey and RMB 20 for each audio recording and its corresponding daily survey, up to a maximum of ten sets). Additionally, those who provided complete data would receive a family education report with tailored suggestions from our team of family education specialists. This incentive was designed, in part, to encourage parents to behave as naturally as they would at home. It is worth noting that the participation in this research project is entirely voluntary. Participants were free to withdraw from the study at any time. To ensure privacy privacy, all participants were anonymized and assigned a unique ID during the data collection.

%It is worth noting that participation in this research project isvoluntary. Participants are free to withdraw from the project at any stage if they change their minds. Besides, we anonymized all the participants to protect their privacy





\subsection{Collected Data}

We gathered data from several sources: registration questionnaires, background surveys, audio recordings of parental homework involvement, and daily surveys. We collected 121 registration questionnaires and 78 background surveys, but only used the data for participants who completed both. We also collected 602 valid audio recordings and 620 daily surveys. 

For the record data, while we only require one complete audio file, some parents accidentally produce multiple recordings due to phone calls or pressing the wrong button. These recordings, which constitute about 10.43\% (66 out of 633 homework sessions) of our data, are merged into a single file before processing, leaving out 603 audio files. After removing one corrupted file, we collected 602 valid audio files. The total duration of these recordings is 474.89 hours, with an average length of 47.33 minutes per audio. Figure \ref{fig:dis_length} shows the distribution of audio recording durations, while Figure \ref{fig:dis_count} illustrates the number of recordings contributed by different participants. On average, 24 recordings were submitted daily, with 66 parents successfully contributing at least one recording. 

For the daily survey\footnote{Additional data, such as child behaviour evaluations, parental behaviours, and daily stress, were collected but not utilized in this research.}, it assessed the parents' affective reactions before and after homework involvement, focusing on pleasure, arousal, and dominance, using the \textit{Self-Assessment Manikin} (SAM) \cite{bradley1994measuring} and \textit{PAD} emotion state model \cite{mehrabian1974approach}. Figure \ref{fig:self-report dis} presents the parents' perceived emotions before and after homework, with pleasure (1-5) ranging from extremely unhappy to extremely happy, arousal (1-5) from completely calm to highly aroused, and dominance (1-5) from highly submissive to highly dominant.

\begin{figure}
    \subfigure[ {Pleasure} \label{subfig: age}]{\includegraphics[width=0.285\textwidth]{figure/dis_valence_no.pdf}}
	\hspace{0cm}
    \subfigure[ {Arousal}\label{subfig: venn}]{\includegraphics[width=0.285\textwidth]{figure/dis_arousal_no.pdf}}
    \subfigure[ {Dominance}\label{subfig: venn}]{\includegraphics[width=0.405\textwidth]{figure/dis_dominance.pdf}}
    \caption{Distribution of parental emotions before and after homework involvement}
    \label{fig:self-report dis}
\end{figure}

%To capture the overall trend and the diversity of emotional responses among parents, we conducted both group-level and individual-level analyse to thoroughly understand the emotional experiences of parents before and after homework involvement, based on self-reported data. First, we calculated the mean emotional values for each parent, both before and after homework. Then, we conducted a paired sample t-test on these group mean values, comparing the pre-homework means to the post-homework means. We observed statistically significant differences in the mean values across all dimensions, with pleasure (p < 0.001), arousal (p < 0.001), and dominance (p < 0.001) all showing significant changes. It revealed whether parents, as a group, experienced a statistically significant emotional shift due to homework involvement. Next, we examined whether individual parents showed significant changes in their emotions before and after homework. This allowed us to identify those parents who experienced notable emotional shifts, providing insight into the variability of responses.

\subsection{Data Processing}
\subsubsection{Audio Preprocessing}
The audio preprocessing followed three key steps:
(1) \textit{Conversion to WAV Format}. 
%All audio files were converted to WAV, a lossless format that preserves the full quality of the recordings. This was essential to ensure no audio information was lost, as maintaining high fidelity was crucial for accurate analysis. 
All audio files were converted to lossless WAV to preserve full recording quality, ensuring no information was lost for accurate analysis.
(2) \textit{Resampling}. 
%The audio was resampled to a uniform rate of 16 kHz, a standard in speech processing. This rate strikes a balance between maintaining sufficient detail for accurate speech recognition and managing computational efficiency.
Audio was resampled to 16 kHz, balancing sufficient detail for speech recognition with computational efficiency.
(3) \textit{Normalization}.  Audio levels were normalized to ensure consistent volume across recordings, preventing bias during feature extraction. Noise removal was deliberately avoided after preliminary tests showed it could distort or remove crucial elements, particularly the child’s voice.



\subsubsection{Transcription}

Transcription was performed on the normalized audio files, ensuring pauses and silence were maintained to capture the natural flow of conversation. We employed the Xunfei API \cite{xunfei} for automatic transcription, with a "2-second rule" to accurately segment utterances during pauses. Given the informal nature of parent-child conversations, automatic transcription systems often produce errors such as homophone confusion, misinterpretations, and segmentation issues. To address this, we used a custom-designed prompt (see Appendix \ref{app:prompt_transcription}) for the GPT-4o model to assist in error correction. This prompt guided the model in identifying and fixing errors while maintaining the natural conversational tone. The key principles were minimal intervention, preserving the dialogue's original flow, and ensuring clarity and accuracy in the final transcription.
%Transcription was conducted on the normalized audio files, keeping periods of silence intact to accurately capture the start and end of speech segments. We used the Xunfei API for automatic transcription, applying a "2-second rule" to distinguish between different utterances, ensuring that pauses did not incorrectly segment the conversation. 
%Accurate transcription of parent-child conversations is crucial for the reliability of our study's analysis. Automatic Speech Recognition (ASR) systems, such as the one used, often introduce errors in informal speech contexts, including homophones, phonetic misinterpretations, omissions, additions, and incorrect word boundaries. To mitigate these issues, we employed a custom-designed prompt for the GPT-4o model to assist in error correction. The prompt was carefully crafted to guide the model in correcting transcription errors while preserving the natural conversational style of the dialogue. It provided a detailed structure for input data, specified the desired output format, and included step-by-step instructions for identifying and correcting errors. Key principles emphasized in the prompt were minimal intervention, maintaining the original conversational tone, and ensuring clarity and accuracy in the corrected text.

\subsubsection{Role Recognition} \label{subsec: roles}
The Xunfei API could identify multiple speakers but labeled them generically (e.g., "Speaker 1"). This posed challenges in identifying specific roles, such as distinguishing between parent and child or between different speakers assigned the same label. To resolve this, we designed a model-specific prompt (see Appendix \ref{app: prompt_role recognition}) to clarify speaker roles. The prompt used contextual cues and conversational patterns to assign accurate roles to each speaker. It also addressed cases where speakers with the same label might actually represent different individuals, ensuring accurate identification of all participants.

%The Xunfei API used in transcription could identify multiple speakers but only labeled them generically as "Speaker 1," "Speaker 2," etc., without specifying their identities or roles. This presented two challenges: (1) determining which speaker corresponded to which role (e.g., parent or child) and (2) identifying whether multiple speakers of the same label belonged to the same role or different ones. To address these challenges, we developed a specific prompt to clarify speaker roles post-transcription. The prompt was designed to instruct the model to distinguish between different speakers based on contextual cues and conversational patterns, ensuring accurate role identification. The prompt also guided the model in determining whether speakers labeled identically by the API actually represented different roles.

%\tobe{Expert Labeling: We selected 30 audio recordings for a thorough evaluation. Two experts independently labeled the roles based on the transcription. If their labels matched, the roles were confirmed. In cases of disagreement, the experts listened to the audio to resolve the discrepancies and accurately define the roles. These expert-labeled roles were considered the ground truth. Model Testing: For each of the 30 audio recordings, the role recognition prompt was run 10 times to assess consistency and accuracy. The results were compared against the expert-labeled ground truth to evaluate the effectiveness of the role recognition process.}



\section{Understanding Parents' Emotion Dynamics During Homework Invovlment}




\subsection{Perceived Emotion Shifts after Homework Involvement} 
To comprehensively understand parents' emotional experiences before and after homework involvement, we conducted both group-level and individual-level analyses using self-reported data. First, we calculated the mean emotional values for each parent before and after homework and performed a paired sample t-test on these group means. This revealed significant differences across all emotional dimensions—valence (p < 0.001), arousal (p < 0.001), and dominance (p < 0.001)—indicating a statistically significant emotional shift for parents as a group due to homework involvement. Specifically, parents' pleasure tends to decrease, arousal tends to decrease, and their sense of control (dominance) over their emotions tends to diminish after homework involvement. 

\begin{figure}
    \centering
    \includegraphics[width=0.99\linewidth]{figure/significance_heatmap.pdf}
    \caption{Significance heatmap of pleasure, arousal, and dominance. Red, blue, and green colors indicate participants with statistically significant differences (p < 0.05) in pleasure, arousal, and dominance, respectively, before and after homework involvement.}
    \label{fig:heatmap}
\end{figure}

\begin{figure}
    \subfigure[ {Pleasure} \label{subfig: age}]{\includegraphics[width=0.7\textwidth]{figure/dis_valence_change_par.pdf}}
	\hspace{0cm}
    \subfigure[ {Arousal}\label{subfig: venn}]{\includegraphics[width=0.7\textwidth]{figure/dis_arousal_change_par.pdf}}
    \subfigure[ {Dominance}\label{subfig: venn}]{\includegraphics[width=0.7\textwidth]{figure/dis_dominance_change_par.pdf}}
    \caption{Emotion shifts after the homework involvement. Red indicates the participants with statistically significant shifts (p<0.05). }
    \label{fig: daily_survey_emotion}
\end{figure}


Next, we analyzed the emotional changes of individual parents, calculated as the difference between their post-homework and pre-homework values, with a focus on those who showed statistically significant shifts, as depicted in Figure \ref{fig:heatmap}. Some parents, such as P0, P53, and P57, experienced significant changes in one or more emotional dimensions, while others did not exhibit statistically significant shifts. Detailed information on these emotional changes for each participant is shown in Figure \ref{fig: daily_survey_emotion}, where red bars represent participants with statistically significant shifts (p < 0.05). Interestingly, although some parents, such as P31 and P102, displayed noticeable emotional shifts, these changes did not reach statistical significance, likely due to the limited number of observations (e.g., only one sample). These findings underscore the variability in emotional responses to homework involvement among parents. While some parents experienced marked decreases in pleasure or dominance and an increase in arousal, others demonstrated resilience or maintained stability in their emotional states.

on{Emotion Fluctuations During Homework Involvement} 
\subsubsection{Extracting Emotion Annotations}

%To gain a nuanced understanding of the emotional fluctuations of both parents and children during homework interactions, we employed the recently developed emotion fine-tuning large language model, EmoLLMs \cite{liu2024emollms}. This suite of models and annotation tools excels at comprehensive affective analysis, showing outstanding performance in emotion regression and classification tasks, particularly using the three dimensions—Pleasure, Arousal, and Dominance—from the EmoBank dataset \cite{buechel2022emobank}. Given these strengths, we utilized EmoLLMs to annotate our experiment's transcribed data.

To understand the emotional fluctuations during homework interactions, we employed the EmoLLMs model suite \cite{liu2024emollms}, which excels in affective analysis, particularly in emotion regression and classification tasks using the three dimensions—Pleasure, Arousal, and Dominance—based on the EmoBank dataset \cite{buechel2022emobank}. Given its high performance, especially in \textit{Pleasure} analysis (Pearson Correlation Coefficient = 0.728), we used the EmoLLaMA-chat-7B model \cite{liu2024emollms} to annotate our transcribed data.

We first filtered the transcriptions to remove sentences lacking semantic clarity, such as short or misrecognized sentences, which could distort the results. We then applied EmoLLMs to infer emotional dimensions on a sentence-by-sentence basis, focusing on the \textit{Pleasure} dimension. This is because pleasure directly reflects the emotional polarity (positive or negative) critical for understanding parent-child interactions during homework \cite{pekrun2002academic}. Although future analyses may include Arousal and Dominance, current model limitations make it practical to focus on pleasure alone for accurate analysis.

%In our experiment, we first filtered the transcriptions, excluding text that met any of the following criteria: \textit{1) Sentences with a high frequency of interjections, such as ``um um um'' or ``oh oh ah''.2) Sentences containing fewer than five words, such as ``Okay'' (1 word) or ``I'm sorry'' (3 words).3) Sentences with content errors or logical inconsistencies caused by misrecognition from the speech recognition software.} These sentences lacked sufficient or accurate semantic information, often leading to poor recognition and unexpected outcomes by the model. By excluding them, we improved the overall recognition accuracy and ensured the integrity of our dataset. We then applied the pre-trained \textit{EmoLLaMA-chat-7B model}  \cite{liu2024emollms} to infer emotional dimensions on a sentence-by-sentence basis, assigning values between 1.00 and 5.00 across three dimensions (\textit{Pleasure}, \textit{Arousal}, \textit{Dominance}). For this study, we focused exclusively on the \textit{Pleasure} dimension for two key reasons: (1) Valence directly measures emotional polarity, capturing the continuum of positive and negative emotions—critical for understanding emotional dynamics during parent-child homework interactions \cite{pekrun2002academic}; and (2)  \textit{EmoLLaMA-chat-7B model} was reported the highest performance in \textit{Valence}-related emotion analysis, with a \textit{Pearson Correlation Coefficient} (PCC) \cite{cohen2009pearson} of 0.728. While adding the other two dimensions could enhance future analyses, it requires further improvements in model robustness and accuracy. 


\subsubsection{Emotional Variation Analysis}


\begin{figure}
    \centering
    \includegraphics[width=0.99\linewidth]{figure/dis_all_first10.pdf}
    \caption{Average pleasure for the first 10 minutes of sessions. The dark grey band indicates the standard error.}
    \label{fig:emotion_all}
\end{figure}

We collected emotional data from multiple homework sessions, totalling 40,356 pleasure measurements from 66 parents.
To capture the dynamics, we calculated the mean pleasure for each participant at 15-second intervals, a balance that retained key details without being overwhelmed by data limitations.  
To account for varied session lengths, we standardized our analysis by focusing on the first 10 minutes of each session, which aligns with similar studies \cite{tag2022emotion}. This allowed us to analyze emotional fluctuations across participants consistently. Although some sessions lasted longer than 10 minutes and others shorter, we used all available data to calculate the mean pleasure for the first 10 minutes of each session.



%For each parent, we collected recording data from multiple homework sessions. In total, we have 40,356 valence values from 66 parents, that is to say, 611 valence values on average for each parent from their multiple homework sessions. We first calcluate the mean level of pleasure for each participants with a granularity of 15 seconds. We try several times, and find finer granular is not suitable due to our limited data and larger grannular may lose the charactersitics of their valence changes.  As such, for each individual homework session, we are able to estimate the level of valence of each parent for every 15 seconds of that session. However, we note taht session have different duration, and therefore aggreating the data for each participant needs further consideration, for this reason, we choose to aggregate data in the first 10 minutes, similar to the studies \cite{tag2022emotion}. 

Figure \ref{fig:emotion_all} shows the average pleasure levels during the first 10 minutes, with LOESS smoothing (frac = 0.1) applied to capture the overall trend. The dark grey bands represent the standard error and are based on smoothed mean values rather than raw data to highlight clearer patterns amid the variability. Due to space constraints, we present data for 18 participants with the most pleasure measurements. We found that some parents exhibit an early decline in pleasure within the first 2-3 minutes (e.g., P32, P76, P82), while others show fluctuating pleasure levels throughout the first 10 minutes (e.g., P18, P95). This provides insights into the evolving emotional engagement of parents during the early phases of homework involvement.

%Figure \ref{fig:emotion_all} shows the average pleasure levels during the first 10 minutes, with LOESS smoothing applied to capture overall trends. The dark grey band represents the standard error. We highlight the data for 18 participants with the most valence measurements to showcase key patterns. Some parents exhibited an early decline in pleasure within the first 2-3 minutes (e.g., P32, P76, P82), while others showed fluctuating pleasure levels throughout the session (e.g., P18, P95). This provides a snapshot of how emotional engagement shifts early in homework involvement.

%It is worth note that, Future analyses could explore how these early emotional responses relate to longer-term trends or outcomes in homework effectiveness and parental involvement. Additionally, while the first 10 minutes provide useful insights, extending the analysis to cover entire sessions may uncover more nuanced patterns over time.

%First, we visualize the mean pleasure for each patient of the first 10 minutes of the homework session as shown in Figure \ref{fig:emotion_all}. This is calculated using means and loess smoothing (frac=0.1), while the standard error is shown as dark grey band around the mean. We visualize the error bars based on the average pleasure values, not based on the raw data, to more clearly identify the signal in the noise. Some sessions are shorter than 10 minutes, others are longer, but regardless we calculate the mean pleasure joy for each of the first 10 minutes using the available data. The rationale is that this visualisation provides an assessment of how pleasure evolves as parent begin homework ionvlment, but a downside is that data beyond the first 10 minutes is discarded. Due to the limit of space, we only show the top 18 participants with highliest number of valence measurements. We observe that some parents experience reduced levels of pleasure within the first 2-3 minutes (e.g., P32, P76, P82), and some participants exhibit extensive flucntions during first 10 minutes (e.g., P18, P95)

%Second, we provide an aggreatoin that over the limitations of the first aggregation strategy (first 10 minutes) as shown in Figure \ref{fig:emotion_all_normalised}. Here, we normalise the duration of each homeowkr session to be 1, and any measurement of pleausare during a session is indexed to a normalised timestamp between 0 and 1. In this manner, all sessions start at 0, end at 1, and all pleasure measurment are timestamped with a value between 0 and 1. This allows us to retain all our data when calculating theaverage pleasure. However, it does not capture the true magnitude (in seconds) of each session duration. \ref{fig:emotion_all_normalised} provides a visual overview of different patterns of changes in pleasure during homework involvement. We found some parents experience intense fluctuations during  the homework session (P28, P76, P82). A few participants (e.g., P18, P28) display a gradual increase in pleasure at the end of the session. P32, P expericen the stable e 

\iffalse
\begin{figure}
    \centering
    \includegraphics[width=0.99\linewidth]{figure/dis_all.pdf}
    \caption{Average pleasure for sessions, with lengths normalised to [0,1]. The dark grey band indicates the standard error}
    \label{fig:emotion_all_normalised}
\end{figure}
\fi

\section{Understanding Parents Behavioural Dynamics During Homework Involvement}

Parental behavioural dynamics play a crucial role in shaping the quality and effectiveness of homework assistance. Understanding these dynamics not only allows us to evaluate how parental actions influence a child’s academic development but also helps in identifying patterns that may lead to tension or conflict. By studying these dynamics, we gain insights into how different types of parental involvement can either support or hinder the child’s homework experience \cite{eccles1993parent,eccles2013family}.

In this section, we explore the parental behaviours and parent-child conflicts that arise during homework sessions. Using GPT-4o, we systematically analyze and categorize these behaviours and conflicts from transcribed conversations with the assistance of educational experts. We then validate the extracted behaviours and conflicts through comparisons with human annotations. Additionally, we examine the distribution of these behaviours and conflicts both across the overall population and on a per-user basis.


\subsection{Parental Behaviours and Parent-Child Conflicts}

%To establish a coding manual for parental behaviours and conflicts observed during homework assistance, we begin by defining these two key elements. \textbf{Parental behaviours} encompass the actions or responses of parents as they assist their children with homework. These behaviours can be positive, neutral, or negative depending on their nature and effect on the child. \textbf{Parent-child conflicts} refer to disagreements or tension that arise during homework sessions, which can be triggered by a range of factors, such as differing perspectives on learning methods or time management.

Although prior research has extensively explored parental behaviours and conflicts in general family and educational settings \cite{moe2018brief,solomon2002helping,nnamani2020impact}, to the best of our knowledge, there are no established definitions or a comprehensive codebook specifically addressing parental behaviours and parent-child conflicts during homework involvement. Therefore, we adopted a bottom-up coding process to inductively derive patterns from the raw data. This process is divided into three main steps: open coding, axial coding, and selective coding, ultimately resulting in the creation of a codebook for understanding parental behaviours and parent-child conflicts during homework sessions \cite{corbin1994grounded,charmaz2006constructing,glaser2017discovery}.

%Therefore, we begin by grounding our definitions in related literature on parenting and education but adapt them to the specific context of homework involvement. In cases where existing definitions are inadequate, we propose new ones based on our analysis of the interactions observed in this study.
 %coding manual

In this context, parental behaviour refers to the specific actions, responses, or strategies parents employ while assisting their children with homework. These behaviours include verbal and non-verbal interactions, guidance, feedback, or any form of involvement that influences the child's approach to homework. Parental behaviours may support, hinder, or remain neutral in the homework process, and they reflect the dynamic interaction between parent and child during educational activities \cite{cunha2015parents,eccles2013family}.
Similarly, parent-child conflicts are defined as moments of disagreement, tension, or friction that arise during homework involvement. These conflicts may stem from misunderstandings, differences in expectations, frustration, or emotional reactions from either the parent or the child, ranging from minor verbal disagreements to more disruptive disputes \cite{grolnick2009issues,benckwitz2023reciprocal,hanh2023current}.

%Similarly, parent-child conflicts during homework are defined as moments of disagreement, tension, or friction. These conflicts may arise from misunderstandings, differing expectations, frustration, or emotional reactions from either the parent or child, ranging from minor disagreements to more disruptive disputes.


%We randomly selected 50 transcripts from a larger dataset of 602 valid conversations for initial analysis. Using GPT-4o, we conducted two open coding tasks: one to identify parental behaviours and another to capture specific parent-child conflicts. This stage of open coding generated a total of 932 conflict scenarios and 2,161 instances of parental behaviour, resulting in 330 conflict codes and 950 behaviour codes. These codes formed the basis for further analysis and refinement.

\begin{table}
\centering
\scriptsize
\caption{Code definitions and examples of positive, neutral, and negative behaviours}
\label{tab:behaviours} 
\begin{tabular}{p{0.12\textwidth} p{0.40\textwidth} p{0.47\textwidth}}
\toprule
\textbf{Code Name} & \textbf{Code Definition} & \textbf{Example} \\ \midrule
\multicolumn{3}{c}{\textbf{\textit{Positive Behaviours}}}   \\\midrule
\textit{Encouragement (ENC)} & Parents provide verbal or behavioural support to encourage the child's effort and progress, boosting their confidence and motivation to overcome challenges. & \textit{"You've worked really hard, keep it up! I believe in you!"} \newline \textit{"Don't worry, let's take it step by step, you'll definitely get it."} \\ \hline
\textit{Labelled Praise (LP)} & Parents specifically highlight the child's particular action or achievement and offer praise, helping the child recognize their specific progress and strengths. & \textit{"You did great on this addition problem, no mistakes at all!"} \newline \textit{"Your handwriting is especially neat this time, keep it up!"} \\ \hline
\textit{Unlabelled \newline Praise (UP)} & Parents give general praise to the child without pointing out specific actions or achievements. & \textit{"You're amazing, keep going!"} \newline \textit{"Wow, that's awesome!"} \\ \hline
\textit{Guided \newline Inquiry (GI)} & Parents ask questions or provide clues to guide the child toward independent thinking and problem-solving. & \textit{"Where do you think this letter should go?"} \newline \textit{"What methods could we use to solve this problem? Think about a few ways."} \\ \hline
\textit{Setting Rules (SR)} & Parents set clear rules or requirements for completing homework, helping the child establish good study habits and time management skills. & \textit{"You need to finish your Chinese homework before watching cartoons."} \newline \textit{"All homework needs to be done before dinner if you want to go out and play."} \\ \hline
\textit{Sensitive \newline Response (SRS)} & Parents respond to the child's emotions, needs, and behaviours in a timely, appropriate, and caring manner. & \textit{"I can see you're a bit tired now, how about we take a break and continue later?"} \newline \textit{"Do you find this question difficult? Don't worry, let's take another look together."} \\ \midrule
\multicolumn{3}{c}{\textbf{\textit{Neutral Behaviours}}}   \\\midrule

\textit{Direct \newline Instruction (DI)} & Parents tell the child how to complete a task or solve a problem without using guided or inquiry-based methods. & \textit{"For this problem, you should do it like this: add 4 to 6 to get 10."} \newline \textit{"Just copy this answer down, don't overthink it."} \\ \hline
\textit{Information \newline Teaching (IT)} & Parents teach new knowledge or skills by explaining concepts, reading texts, or offering detailed instructions. & \textit{"The character 'tree' is written with a wood radical on the left and 'inch' on the right, let's write it together."} \newline \textit{"You need to memorize multiplication tables like this: two times two equals four, two times three equals six. Let's start with those."} \\ \hline
\textit{Error Correction (EC)} & Parents point out mistakes in the child's homework and guide them to correct or revise their work. & \textit{"You missed the 'wood' radical here, write it again."} \newline \textit{"The addition is wrong here, let's calculate it again. Remember to line up the numbers correctly."} \\ \hline
\textit{Monitoring (MON)} & Parents regularly check the child's homework progress or completion to ensure they stay on track. & \textit{"How many pages have you written? Let me check for mistakes."} \newline \textit{"Let me look over your pinyin homework today to see if everything is correct."} \\ \hline
\textit{Direct Command (DC)} & Parents use clear and direct language to request or command the child to perform a specific action or task. & \textit{"Go do your math homework right now, no more delays!"} \newline \textit{"Stop playing with your toys and go finish your pinyin practice."} \\ \hline
\textit{Indirect Command (IC)} & Parents indirectly convey their requests, often through suggestions or hints, rather than giving direct orders. & \textit{"Have you finished your homework? Maybe it's time to get it done."} \newline \textit{"How about we finish homework first and then go play? That way you won't have to worry about running out of time later."} \\ \midrule

\multicolumn{3}{c}{\textbf{\textit{Negative Behaviours}}}   \\\midrule
\textit{Criticism and \newline Blame (CB)} & Parents express negative evaluations of the child's mistakes or behaviours, often directly blaming the child. & \textit{"How could you mess up such a simple word?"} \newline \textit{"I've told you this a thousand times, why haven't you remembered it yet?"} \\ \hline
\textit{Forcing and \newline Threatening (FT)} & Parents use pressure or threats of consequences to force the child to comply with their demands. & \textit{"If you don't do your homework, you won't be allowed to play with your toys today!"} \newline \textit{"If you don't finish, I'll take away your toys!"} \\ \hline
\textit{Neglect and \newline Indifference (NI)} & Parents show a lack of attention or emotional response to the child's needs or feelings. & \textit{\textbf{Child}: "Mom, I don't understand this question, can you help me?"} \newline \textit{\textbf{Parent}: (no response, continues using phone)} \\ \hline
\textit{Belittling and \newline Doubting (BD)} & Parents belittle the child's abilities or question their performance, undermining the child's confidence and motivation. & \textit{"How could you be so stupid? You can't even solve simple addition."} \newline \textit{"With grades like these, you'll never get into a good school."} \\ \hline
\textit{Frustration and \newline Disappointment (FD)} & Parents express frustration or disappointment in the child's performance when it fails to meet their expectations. & \textit{"I can't believe you did so poorly on this test, I'm really disappointed."} \newline \textit{"I thought you'd do better, but I guess I was wrong."} \\ \hline
\textit{Impatience and \newline Irritation (II)} & Parents exhibit impatience or irritation when the child's performance falls short of expectations. & \textit{"Why are you so slow? I've been waiting forever!"} \newline \textit{"Why isn't this finished yet? You always take so long!"} \\ \bottomrule
\end{tabular}
\end{table}

\iffalse
\begin{table}[h]
\centering
\footnotesize
\caption{Neutral behaviours: Code Definitions and Examples}
\begin{tabular}{p{0.10\textwidth} p{0.40\textwidth} p{0.45\textwidth}}
\hline
\textbf{Code Name} & \textbf{Code Definition} & \textbf{Example} \\ \hline
\textbf{Direct Instruction (DI)} & Parents tell the child how to complete a task or solve a problem without using guided or inquiry-based methods. & "For this problem, you should do it like this: add 4 to 6 to get 10." \newline "Just copy this answer down, don't overthink it." \\ \hline
\textbf{Information \newline Teaching (IT)} & Parents teach new knowledge or skills by explaining concepts, reading texts, or offering detailed instructions. & "The character 'tree' is written with a wood radical on the left and 'inch' on the right, let's write it together." \newline "You need to memorize multiplication tables like this: two times two equals four, two times three equals six. Let's start with those." \\ \hline
\textbf{Error Correction (EC)} & Parents point out mistakes in the child's homework and guide them to correct or revise their work. & "You missed the 'wood' radical here, write it again." \newline "The addition is wrong here, let's calculate it again. Remember to line up the numbers correctly." \\ \hline
\textbf{Monitoring (MON)} & Parents regularly check the child's homework progress or completion to ensure they stay on track. & "How many pages have you written? Let me check for mistakes." \newline "Let me look over your pinyin homework today to see if everything is correct." \\ \hline
\textbf{Direct Command (DC)} & Parents use clear and direct language to request or command the child to perform a specific action or task. & "Go do your math homework right now, no more delays!" \newline "Stop playing with your toys and go finish your pinyin practice." \\ \hline
\textbf{Indirect Command (IC)} & Parents indirectly convey their requests, often through suggestions or hints, rather than giving direct orders. & "Have you finished your homework? Maybe it's time to get it done." \newline "How about we finish homework first and then go play? That way you won't have to worry about running out of time later." \\ \hline


\end{tabular}
\end{table}

\begin{table}[h]
\centering
\footnotesize
\caption{Negative behaviours: Code Definitions and Examples}
\begin{tabular}{p{0.10\textwidth} p{0.40\textwidth} p{0.45\textwidth}}
\hline
\textbf{Code Name} & \textbf{Code Definition} & \textbf{Example} \\ \hline
\textbf{Criticism and Blame (CB)} & Parents express negative evaluations of the child's mistakes or behaviours, often directly blaming the child. & "How could you mess up such a simple word?" \newline "I've told you this a thousand times, why haven't you remembered it yet?" \\ \hline
\textbf{Forcing and Threatening (FT)} & Parents use pressure or threats of consequences to force the child to comply with their demands. & "If you don't do your homework, you won't be allowed to play with your toys today!" \newline "If you don't finish, I'll take away your toys!" \\ \hline
\textbf{Neglect and Indifference (NI)} & Parents show a lack of attention or emotional response to the child's needs or feelings. & \textit{\textbf{Child}: "Mom, I don't understand this question, can you help me?"} \newline \textit{\textbf{Parent}: (no response, continues using phone).} \\ \hline
\textbf{Belittling and Doubting (BD)} & Parents belittle the child's abilities or question their performance, undermining the child's confidence and motivation. & "How could you be so stupid? You can't even solve simple addition." \newline "With grades like these, you'll never get into a good school." \\ \hline
\textbf{Frustration and Disappointment (FD)} & Parents express frustration or disappointment in the child's performance when it fails to meet their expectations. & "I can't believe you did so poorly on this test, I'm really disappointed." \newline "I thought you'd do better, but I guess I was wrong." \\ \hline
\textbf{Impatience and Irritation (II)} & Parents exhibit impatience or irritation when the child's performance falls short of expectations. & "Why are you so slow? I've been waiting forever!" \newline "Why isn't this finished yet? You always take so long!" \\ \hline
\end{tabular}
\end{table}
\fi


% \begin{table}[h]
% \centering
% \small
% \caption{Parent-Child Interaction Codes in Homework Tutoring Scenarios, synthesized by ChatGPT. As there was no comprehensive coding manual specifically tailored for positive, neutral, and negative behaviours in Homework Tutoring, we conducted literature reviews, interviewed educational experts, and leveraged GPT to identify and summarize common parent-child behaviours. After validation and refinement by educational experts, we developed specific definitions and examples for each behaviour type. These codes were then applied to categorize conversation transcripts.}
% \begin{tabular}{p{0.15\textwidth} p{0.40\textwidth} p{0.40\textwidth}}
% \toprule
% \textbf{Code Name} & \textbf{Code Definition} & \textbf{Example} \\ 
% \midrule
% Encouragement (ENC) & Parents provide verbal or behavioural support to encourage the child's effort and progress, boosting their confidence and motivation to overcome challenges. & "You've worked really hard, keep it up! I believe in you!" \\
% \midrule
% Specific Praise (SP) & Parents specifically highlight the child's particular action or achievement and offer praise, helping the child recognize their specific progress and strengths. & "You did great on this addition problem, no mistakes at all!" \\
% \midrule
% General Praise (GP) & Parents give general praise to the child without pointing out specific actions or achievements. & "You're amazing, keep going!" \\
% \midrule
% Guided Inquiry (GI) & Parents ask questions or provide clues to guide the child toward independent thinking and problem-solving. & "Where do you think this letter should go?" \\
% \midrule
% Setting Rules (SR) & Parents set clear rules or requirements for completing homework, helping the child establish good study habits and time management skills. & "You need to finish your Chinese homework before watching cartoons." \\
% \midrule
% Sensitive Response (SRS) & Parents respond to the child's emotions, needs, and behaviours in a timely, appropriate, and caring manner. & "I can see you're a bit tired now, how about we take a break and continue later?" \\
% \midrule
% Direct Instruction (DI) & Parents tell the child how to complete a task or solve a problem without using inquiry-based methods. & "For this problem, you should do it like this: add 4 to 6 to get 10." \\
% \midrule
% Information Teaching (IT) & Parents teach new knowledge or skills by explaining concepts or offering detailed instructions. & "The character 'tree' is written with a wood radical on the left and 'inch' on the right, let's write it together." \\
% \midrule
% Error Correction (EC) & Parents point out mistakes in the child's homework and guide them to correct or revise their work. & "You missed the 'wood' radical here, write it again." \\
% \midrule
% Monitoring (MON) & Parents regularly check the child's homework progress or completion to ensure the child is staying on track. & "How many pages have you written? Let me check for mistakes." \\
% \midrule
% Direct Command (DC) & Parents use clear and direct language to request or command the child to perform a specific action or task. & "Go do your math homework right now, no more delays!" \\
% \midrule
% Indirect Command (IC) & Parents indirectly convey their requests, often through suggestions or hints, rather than giving direct orders. & "Have you finished your homework? Maybe it's time to get it done." \\
% \midrule
% Criticism and Blame (CB) & Parents express negative evaluations of the child's mistakes or behaviours, often directly blaming the child. & "How could you mess up such a simple word?" \\
% \midrule
% Forcing and Threatening (FT) & Parents use pressure or threats of consequences to force the child to comply with their demands. & "If you don't do your homework, you won't be allowed to play with your toys today!" \\
% \midrule
% Neglect and Indifference (NI) & Parents show a lack of attention or emotional response to the child's needs or feelings. & Child: "Mom, I don't understand this question, can you help me?" (No response) \\
% \midrule
% Belittling and Doubting (BD) & Parents belittle the child's abilities or question their performance, undermining the child's confidence and motivation. & "How could you be so stupid? You can't even solve simple addition." \\
% \midrule
% Frustration and Disappointment (FD) & Parents express frustration or disappointment in the child's performance when it fails to meet their expectations. & "I can't believe you did so poorly on this test, I'm really disappointed." \\
% \midrule
% Impatience and Irritation (II) & Parents exhibit impatience or irritation when the child's performance falls short of expectations. & "Why are you so slow? I've been waiting forever!" \\
% \bottomrule
% \end{tabular}
% \end{table}


%To develop a coding manual for both parent-child conflicts and parental behaviours observed during homework assistance in families with young children, we first randomly sampled 50 transcripts from a larger set of 602 valid dialogue transcripts. Using GPT-4o, we conducted two open coding tasks: one focused on identifying instances of parent-child conflicts, and the other on specific parental behaviours during the homework assistance process. In this phase, GPT-4o was instructed to carefully review the dialogue content, capturing segments that reflected these two phenomena—conflicts and behaviours—and applying open coding to each. Through this open coding process, we identified \textit{932 conflict scenarios} and developed \textit{330 initial conflict codes}, as well as \textit{2,161 instances of parental behaviour} and \textit{950 behaviour codes}. These initial codes provided a rich foundation for further analysis and refinement in the subsequent stages.

\subsubsection{Coding and Categorisation}


To conduct this study, we randomly sampled 50 transcripts from 602 valid dialogues. Using GPT-4o, we performed two open coding tasks: one to identify parental behaviours and another to capture specific parent-child conflicts. Through this process, we identified 932 conflict scenarios and 2,161 instances of parental behaviour, which initially resulted in 330 conflict codes and 950 behaviour codes. These preliminary codes provided a rich foundation for further analysis.

In the \textit{axial coding} phase, we refined the initial codes by removing those that did not reflect genuine conflicts and merging similar codes to reduce redundancy. This refinement process resulted in \textit{166 conflict codes} and \textit{606 behaviour codes}. Using GPT-4o, we then categorized these codes based on content similarity and emerging patterns, grouping the conflict codes into \textit{12 categories} and the behaviour codes into \textit{34 categories}. These categories represent typical conflict and behavioural patterns, offering a structured framework for analyzing parent-child interactions during homework.

%In the \textit{axial coding} phase, we cleaned and consolidated the codes generated during open coding. This involved removing any codes that did not reflect genuine conflict characteristics, ensuring that the remaining codes accurately represented parent-child conflicts. We then merged similar codes to reduce redundancy, ultimately refining the dataset to \textit{166 conflict codes} and \textit{606 behaviour codes}. To clarify relationships between these codes, we once again employed GPT-4o to categorize them based on content similarity and pattern recognition. As a result, the conflict codes were grouped into \textit{12 categories}, while the behaviour codes were organized into \textit{34 categories}. Each group represented a typical conflict or behaviour pattern, providing a clearer structure for understanding the interactions and laying the groundwork for further refinement.

%In the \textit{selective coding} phase, we brought in a human expert to further refine the conflict and behaviour codes. The expert closely examined ten distinct parent-child dialogue samples, combining this detailed review with the classification outcomes from the axial coding stage. Special attention was given to core conflict types, and the expert considered the specific context of the data, including the educational and cultural background of Chinese parents. Through this process, redundant or ambiguous codes were further consolidated. Ultimately, the expert distilled the \textit{12 conflict groups} into \textit{8 core conflict types}, while the \textit{34 behaviour groups} were reduced to \textit{18 key behaviour categories}. Each conflict and behaviour type was given a precise definition and clear coding guidelines to ensure consistent application in future research. To further aid understanding, the expert selected representative dialogue examples for each code, demonstrating its practical use.

In the \textit{selective coding} phase, a human education expert reviewed ten parent-child dialogue samples and integrated the results from the axial coding stage. The expert focused on identifying core conflict types while considering the educational and cultural context of Chinese parents. Redundant or ambiguous codes were consolidated, reducing the \textit{12 conflict categories} to \textit{8 core conflict types} and the \textit{34 behaviour categories} to \textit{18 key behaviour categories}. Each conflict type and behaviour category was clearly defined, with representative dialogue examples chosen to illustrate their practical use.

Following this, three educational experts were invited to review the coding manual, evaluating its accuracy, relevance, and applicability. Based on their feedback, further refinements were made to ensure clarity and consistency in the code definitions and guidelines. After multiple revisions, the coding manual was finalized as a validated tool for future qualitative research, providing clear guidance for analyzing parental behaviours and parent-child conflict during homework involvement.

%After developing the initial version of the coding manual, we sought the evaluation of \textit{three educational experts}, who reviewed the coding scheme in depth. These experts assessed the accuracy, relevance, and applicability of the codes within the educational context, providing constructive feedback for improvement. Based on their recommendations, we made further refinements to the code definitions and coding guidelines, optimizing the manual for clarity and consistency. Through multiple rounds of revision and expert input, we finalized the coding manual, which now serves as a validated tool for future qualitative research. This manual offers clear guidance for the coding process, ensuring consistency across different researchers while providing a robust foundation for analyzing the underlying causes and patterns of parent-child conflict during homework assistance.

%We applied GPT-4o to the entire dataset to automate the coding of parental behaviours and conflicts. In the case of \textbf{parental behaviour coding}, GPT-4o segmented each dialogue into distinct behaviour units, each mapped to a corresponding code based on predefined categories. Behaviours were classified as positive, neutral, or negative, with descriptions provided for each identified instance. Similarly, for \textbf{parent-child conflict coding}, GPT-4o identified and labelled conflict scenarios, categorizing them by the conflict’s trigger, development, intensity, and the reactions of both parent and child.
\begin{table}
\centering
\footnotesize
\caption{Conversations of parent-child conflicts in homework involvement, as synthesized by ChatGPT. Due to the lack of a comprehensive conflict coding manual specifically tailored for Homework Tutoring Scenarios, we conducted extensive literature reviews, interviewed educational experts, and utilized GPT to analyze and summarize common types of Parent-Child Conflicts in these contexts. After validation and refinement by educational experts, we developed specific definitions and examples for each conflict type. We then employed GPT to code our conversation transcripts accordingly.}
\label{tab:conflict}
\begin{tabular}{p{0.10\textwidth} p{0.53\textwidth} p{0.33\textwidth}}
\toprule
\textbf{Code Name} & \textbf{Code Definition} & \textbf{Example} \\ 
\midrule
\textit{Expectation Conflict (EC)} & This conflict arises when parents have high expectations for their child's performance, progress, or future, but the child’s actual abilities, goals, or interests do not align with these expectations. Parents may also compare their child to others, intensifying the conflict. & \textit{\textbf{Parent}: “You should be like your classmate and get full marks! How could you get such an easy question wrong?”} \newline \textit{\textbf{Child}: “I’ve done my best. Why do you always think I’m worse than others?”} \\

\midrule
\textit{Communication Conflict (CC)} & This conflict arises when parents and children have different communication styles during homework sessions, leading to misunderstandings and emotional tension. Parents may criticize, question, or belittle the child, making the child feel misunderstood or oppressed, thus escalating the communication barrier. & \textit{\textbf{Parent}: “What’s wrong with you? I’ve explained this so many times and you still don’t get it!”} \newline \textit{\textbf{Child}: “I just don’t want to listen to you anymore. You always yell at me!”} \\

\midrule
\textit{Learning Method \newline Conflict (LMC)} & This conflict occurs when parents and children disagree on how to approach and complete homework. Parents may feel the child’s method is inefficient and try to impose their own approach, while the child insists on using their own method and resists parental intervention. & \textit{\textbf{Parent}: “You shouldn’t study like this. Finish all the questions first, then check your answers!”} \newline \textit{\textbf{Child}: “I’m used to doing it my way. Why should I follow what you say?”} \\

\midrule
\textit{Rule Conflict (RC)} & This conflict occurs when parents set strict rules for learning, and the child seeks more autonomy. Parents may try to control the pace or structure of the child’s study sessions, while the child resists these restrictions and pushes for greater flexibility and freedom. & \textit{\textbf{Parent}: “You must start your homework right after dinner. No more delays!”} \newline \textit{\textbf{Child}: “I want to play for a little longer. You’re always controlling everything!”} \\

\midrule
\textit{Time \newline Management Conflict (TMC)} & This conflict arises from disagreements about how time and energy should be allocated for studying. Parents may want the child to follow a fixed study schedule, while the child may prefer a different routine, leading to conflict. & \textit{\textbf{Parent}: “You always leave your homework until so late at night. You’re so inefficient!”} \newline \textit{\textbf{Child}: “I prefer studying later. I just can’t focus in the morning!”} \\

\midrule
\textit{Knowledge Conflict (KC)} & This conflict occurs when there is a mismatch in knowledge levels or understanding between parents and children. Parents may have already mastered certain knowledge and find it difficult to empathize with the child’s struggles, or they may explain concepts from a perspective the child cannot yet grasp. Additionally, parents might be unfamiliar with certain subjects, leading the child to question their guidance. & \textit{\textbf{Parent}: “This problem is so simple. How can you still not understand it?”} \newline \textit{\textbf{Child}: “You don’t understand what I’m struggling with! My teacher explained it differently from you.”} \\

\midrule
\textit{Focus Conflict (FC)} & This conflict arises when parents are dissatisfied with the child’s attention or focus during study time, believing the child is distracted or not concentrating sufficiently. Parents may attempt to intervene or remind the child to focus, while the child may feel pressured or overwhelmed by the interference, leading to emotional conflict. & \textit{\textbf{Parent}: “What are you daydreaming about? Focus on your homework!”} \newline \textit{\textbf{Child}: “I wasn’t distracted, I was just thinking about how to solve the problem.”} \\

\bottomrule
\end{tabular}
\end{table}
\subsubsection{Automated Coding Using GPT-4o}
We applied GPT-4o to the entire dataset to automate the coding of parental behaviours and conflicts. Regarding the parental behaviour coding, we utilise GPT-4o to analyze specific actions of the parent during the homework involvement process. Each segment identified distinct behaviours, segmenting the dialogue into behaviour units (denoted as \texttt{behaviour\_id}), where each behaviour could consist of one or more sentences. It then provided a brief description for each behaviour and mapped it to a relevant \textit{code} based on predefined categories. These behaviours were classified as \textit{positive}, \textit{neutral}, or \textit{negative}, ensuring accurate classification of each parental action. The specific coding manual is presented in Table \ref{tab:behaviours}, and the coding guidelines are detailed in Table \ref{appen:tab:behaviour}.


For the parent-child conflict coding, GPT-4o was prompted to identify conflict scenarios from the transcribed dialogues. Each conflict was described according to the following dimensions: \textit{trigger of the conflict}  (what initiated it), \textit{development} (how it unfolded), \textit{parent’s behaviour} (verbal or non-verbal reactions), \textit{child’s response}, \textit{type of conflict}, and \textit{its intensity} (categorized as high, medium, or low). GPT-4o assigned a short label, or code, to each dialogue segment representing a conflict. The specific coding manual for these conflicts is presented in Table \ref{tab:conflict}, and the coding guidelines are provided in Table \ref{appen:tab:conflict}.






\subsection{Evaluation of Coding Consistency}

To assess the consistency of GPT-4o's coding, we conducted an evaluation experiment comparing the AI-generated codes with those created by human experts. We randomly selected 200 instances from each coding task for detailed analysis and asked four human experts to code the same instances independently. Their coding results were compared with GPT-4o's outputs using \textit{Cohen's Kappa Coefficient} to measure the agreement. A \textit{Consensus Coding} was established via majority voting among experts, serving as the gold standard for comparison with GPT-4o. The experiment evaluated consistency across three dimensions: between human experts, between GPT-4o and individual experts, and between GPT-4o and the expert consensus.

%This section presents the evaluation experiment designed to assess the consistency of AI-based coding, specifically GPT-4o, in comparison to human experts. The primary objective of this experiment was to determine the reliability of the AI system in coding two key categories: \textit{parental behaviours} and \textit{parent-child conflicts}. These categories were derived from observations during homework interactions between parents and young children, capturing the nuances of behaviour and conflict in this context.

%To carry out the evaluation, a systematic process was adopted. The LLM GPT-4o was applied to the entire dataset, and a random sample of 200 instances from each coding task was selected for in-depth analysis. In parallel, \textit{four} human experts, each with similar levels of expertise in the domain, independently coded the same set of \textit{400} instances. Their coding results were then compared with those of GPT-4o using \textit{Cohen's Kappa Coefficient} to measure agreement. Additionally, a \textit{Consensus Coding} was established through majority voting among the experts, which served as the gold standard for comparison with GPT-4o. This experiment aimed to evaluate consistency across three dimensions: between human experts, between GPT-4o and each individual expert, and between GPT-4o and the expert consensus.

%In the behaviour coding task, as shown in Figure \ref{tab:consistency_analysis}, GPT-4o demonstrated \textbf{moderate agreement} with individual experts, with Kappa values ranging from 0.288 to 0.544. The model's agreement with the expert consensus was stronger, with a Kappa value of 0.560, indicating \textbf{substantial agreement}. In contrast, the conflict coding task showed \textbf{fair agreement} between GPT-4o and individual experts, with Kappa values between 0.256 and 0.344, and moderate agreement with the consensus (Kappa = 0.419). These results suggest that while GPT-4o performed reliably in behaviour coding, there is room for improvement in conflict coding.
%GPT-4o showed \textit{moderate agreement} with individual experts, with Kappa values ranging from \textbf{0.288} (b3) to \textbf{0.544} (b1). According to Landis and Koch \cite{landis1977measurement}, these values indicate \textit{moderate} agreement. GPT-4o's agreement with the expert consensus was higher, with a Kappa value of \textbf{0.560}, reflecting \textit{substantial agreement}. In the conflict coding task, Kappa values with individual experts ranged from \textbf{0.256} to \textbf{0.344}, showing \textit{fair agreement}, while the Kappa value with the consensus was \textbf{0.419}, indicating \textit{moderate agreement}. Overall, GPT-4o’s consistency with the expert consensus is comparable to that of human experts across both tasks, according to the Landis and Koch scale.


\begin{table}
    \footnotesize
    \centering
    \caption{Cohen's Kappa consistency analysis between GPT-4o and human experts. This table shows the inter-rater agreement (Cohen’s Kappa) between human experts (Expert 1 to Expert 4) and GPT-4o. Each value represents the level of agreement between expert pairs or between GPT-4o and individual experts. The final column provides Kappa values comparing GPT-4o to the consensus coding derived from majority voting among experts. In case of a tie, an additional education expert was consulted for arbitration. This consensus coding serves as the benchmark for consistency. Higher Kappa values indicate stronger agreement.}
    \label{tab:consistency_analysis} 
    \begin{tabular}{llccccc}
        \toprule
        \multirow{2}{*}{\textbf{Coding Task}} & \multirow{2}{*}{\textbf{Expert}} & \multicolumn{4}{c}{\textbf{Cohen's Kappa Value}} & \multirow{2}{*}{\textbf{Kappa Value w/ Consensus}} \\
        \cmidrule(lr){3-6}
        &  & Expert 1 & Expert 2 & Expert 3 & Expert 4 & \\
        \midrule
        \multirow{6}{*}{\textit{Conflict Coding}} 
        & Expert 1 & -- & 0.330 & 0.348 & 0.409 & \textbf{0.519} \\
        & Expert 2 & 0.330 & -- & 0.405 & 0.392 & \textbf{0.510} \\
        & Expert 3 & 0.348 & 0.405 & -- & 0.738 & \textbf{0.768} \\
        & Expert 4 & 0.409 & 0.392 & 0.738 & -- & \textbf{0.804} \\
        & \textbf{GPT-4o} & \textbf{0.410} & \textbf{0.444} & \textbf{0.458} & \textbf{0.500} & \textbf{0.517} \\
        \midrule
        \multirow{6}{*}{\textit{Behaviour Coding}} 
        & Expert 1 & -- & 0.473 & 0.457 & 0.467 & \textbf{0.587} \\
        & Expert 2 & 0.473 & -- & 0.524 & 0.627 & \textbf{0.697} \\
        & Expert 3 & 0.457 & 0.524 & -- & 0.802 & \textbf{0.797} \\
        & Expert 4 & 0.467 & 0.627 & 0.802 & -- & \textbf{0.852} \\
        & \textbf{GPT-4o} & \textbf{0.574} & \textbf{0.708} & \textbf{0.562} & \textbf{0.653} & \textbf{0.724} \\
        \bottomrule
    \end{tabular}
\end{table}




In the behavior coding task, as shown in Figure \ref{tab:consistency_analysis}, GPT-4o exhibited \textit{moderate agreement} with individual experts, with Kappa values ranging from \textbf{0.562} (Expert 3) to \textbf{0.708} (Expert 2). According to Landis and Koch \cite{landis1977measurement}, these values indicate \textit{substantial agreement} with Expert 2 and \textit{moderate agreement} with Expert 3. The agreement of GPT-4o with the expert consensus was higher, with a Kappa value of \textbf{0.724}, indicating \textit{substantial agreement}. In the conflict coding task, Kappa values with individual experts ranged from \textbf{0.410} (Expert 1) to \textbf{0.500} (Expert 4), showing \textit{moderate agreement}, while the Kappa value with the consensus was \textbf{0.517}, indicating \textit{moderate agreement}. Overall, GPT-4o’s consistency with the expert consensus is comparable to that of human experts across both tasks, according to the Landis and Koch scale.

%GPT-4o's performance was more varied, with Kappa values with individual experts ranging from \textbf{0.256} to \textbf{0.344}, demonstrating \textit{fair agreement}. The Kappa value between GPT-4o and the consensus coding for conflict instances was \textbf{0.419}, reflecting \textit{moderate agreement}. These results suggest that GPT-4o's consistency with the expert consensus is comparable to that of human experts across both tasks, as categorized by the Landis and Koch scale.
%To better understand where GPT-4o aligned with expert coding and where discrepancies occurred, we used confusion matrices to analyze specific categories of errors and misclassifications. These matrices, detailed in Appendix \ref{sec}, reveal patterns of agreement and confusion across coding categories, highlighting strengths and areas needing improvement.

The moderate agreement observed in the conflict coding task can be considered reasonable, given that dialogue act annotation and sentiment analysis tasks often involve subjective interpretations, as noted in previous research. Studies have highlighted that these tasks are inherently complex and prone to ambiguity, especially when dealing with conflicting or emotionally nuanced data. For instance, Stolcke et al. \cite{stolcke2000dialogue} discuss the challenges in annotating dialogue acts, pointing out that annotators may struggle with the subtlety and context-dependency of dialogues, leading to moderate or low Kappa values. Similarly, Latif et al.\cite{latif2023can}, and Litman et al. \cite{litman2003recognizing} argue that disagreement among annotators is common, even with proper training, especially in tasks involving emotion or sentiment, where subtle cues may be interpreted differently. These studies suggest that a Kappa range of 0.41 to 0.60 is typical in such contexts, underlining the subjective nature of the task.



 To better understand where GPT-4o aligned well with expert coding and where discrepancies occurred, we employed confusion matrices to analyse the specific categories of errors and misclassifications. These matrices, detailed in Appendix \ref{sec:Confusion}, provide patterns of agreement and confusion across different coding categories, highlighting strengths and areas needing improvement.


In addition, a \textit{Chi-Squared} test was conducted to assess the statistical significance of GPT-4o's consistency with the expert consensus. For behavior coding, the Chi-Square value $\chi^2$ was \text{1756.134} with a p-value of \text{1.19e-221}, indicating extremely significant agreement. Similarly, for conflict coding, the $\chi^2$ value was \text{333.131} with a p-value of \text{8.39e-50}, confirming statistical significance. These tests confirm that GPT-4o's agreement with the consensus is far beyond what could be attributed to random chance, further validating the model's reliability.

%In addition to evaluating agreement through Kappa values, we conducted a Chi-square test to assess the statistical significance of GPT-4o's consistency with the expert consensus. For behaviour coding, the Chi-square test yielded a value of \textbf{1415.547} with a p-value of \textbf{6.14e-160}, demonstrating that the observed agreement was extremely significant. Similarly, in conflict coding, the Chi-square value was \textbf{332.151} with a p-value of \textbf{1.04e-43}, again indicating a statistically significant result. These tests confirm that GPT-4o's agreement with the consensus is far beyond what could be attributed to random chance, further validating the model's reliability.

Overall, the evaluation experiments show that the LLM-driven coding showed \textit{substantial agreement} in behaviour coding and \textit{moderate agreement} in conflict coding by human expert annotators. While it aligns well with the expert consensus, particularly in behaviour coding, areas for improvement remain, especially with Experts 1. Refinement in conflict coding could enhance its performance in more complex cases.



%Overall, the evaluation demonstrates that GPT-4o performs reliably, achieving \textit{substantial agreement} in behaviour coding and \textit{moderate agreement} in conflict coding. While GPT-4o aligns well with the expert consensus, particularly in behaviour coding, there are opportunities for improvement, especially in addressing discrepancies with certain experts such as b3 and c3. These findings suggest that GPT-4o has the potential to be an effective coding tool, particularly in tasks where achieving perfect human agreement is challenging. However, further refinement of the model, especially in the areas of conflict coding, could enhance its performance and make it more robust in handling complex or subjective cases.

\subsection{Analysis of Behavioural Dynamics}

\subsubsection{Overall Trend}

For each homework involvement session, we extracted parental behaviours and parent-child conflicts. Figure \ref{fig:dis_behaviour}
illustrates the average number of positive, neutral, and negative behaviours per session across all participants. Among these categories, positive behaviours were the most prevalent, accounting for 47.21\% of total behaviours, while negative behaviours were the least frequent at 9.07\%. The most commonly exhibited behaviour was \textit{Guided Inquiry}, with parents engaging in this behaviour an average of 9.16 times per session. This indicates that parents often asked questions or provided clues to guide their children towards independent thinking and problem-solving. In contrast, negative behaviours occurred less frequently, with parents displaying an average of 3.02 negative behaviours per session. Among these, \textit{Criticism and Blame} was the most common negative behaviour.

\begin{figure}
    \centering
    \begin{minipage}[t]{0.525\textwidth}
        \includegraphics[width=1\textwidth]{figure/dis_behvaiour.pdf}
        \caption{Average number of positive, neutral, and negative behaviours for all participants, with positive behaviours (e.g., Encouragement
(ENC)), neutral behaviours (e.g., Direct
Instruction (DI)), and negative behaviours (e.g., Criticism and
Blame (CB)) defined in Table \ref{tab:behaviours}. Error bar indicates 0.95 confidence level.}
        \label{fig:dis_behaviour}
    \end{minipage}
    \hspace{0.2cm}
    \begin{minipage}[t]{0.445\textwidth}
    \centering
        \includegraphics[width=1\textwidth]{figure/dis_conflict.pdf}
        \caption{Average numbers of each conflict type per user, including types of conflicts defined in Table \ref{tab:conflict} (e.g., Knowledge Conflict (KC), Communication Conflict (CC), and Focus Conflict
(FC)). Error bar indicates 0.95 confidence level.}
        \label{fig:dis_conflict}
    \end{minipage}
    
\end{figure}


Despite the predominance of positive and neutral behaviours, the occurrence of parent-child conflicts was frequent, with an average of 8.63 per session. Figure \ref{fig:dis_conflict} highlights \textit{Knowledge Conflict} as the most common conflict (2.71 times per session), possibly due to the gap between parents’ higher education levels and their children’s earlier stages of learning. This aligns with the \textit{Curse of Knowledge} bias \cite{birch2007curse}, where experts struggle to understand novice perspectives. The second most common conflict type was \textit{Learning Method Conflict}, averaging 1.50 times per session, likely due to the disagreements between parents and children over the approach or strategies used for learning and completing homework tasks. 

\subsubsection{Individual Analysis}
To further understand negative parental behaviours, we analyzed their distribution across individual participants. Figure \ref{fig:per_behaviour_dis} presents the average number of behaviours per participant. While some participants exhibited a very small proportion of negative behaviours, others, such as P4, P28, and P104, displayed a significantly higher percentage of negative behaviours compared to their neutral and positive behaviours. This suggests that these individuals may benefit from improving their behaviours during homework involvement.

\begin{figure}
    \centering
    \includegraphics[width=.95\linewidth]{figure/dis_behvaiour_per.pdf}
    \caption{Average frequency of positive, neutral, and negative behaviours per homework session for each parent}
    \label{fig:per_behaviour_dis}
\end{figure}

We then examined the specific types of negative behaviours exhibited by each participant, as illustrated in Figure \ref{fig:per_behaviour_stacked}. We found that some participants demonstrated only limited types of negative behaviours. For example, P4 displayed just two types of negative behaviours, with \textit{Forcing and Threatening} accounting for 93.75\% of all negative behaviours. Similarly, P33 exhibited two types of negative behaviours, with \textit{Criticism and Blame} and \textit{Forcing and Threatening} each contributing 50\%. In contrast, participants such as P0, P6, P32, and P104 displayed a broader range of negative behaviours, suggesting a more varied pattern of interaction. These variations highlight the individual differences in how parents manage homework involvement and may point to areas for targeted behavioural improvement.
 
\begin{figure}
    \centering
    \includegraphics[width=1\linewidth]{figure/stacked_behvaiour.pdf}
    \caption{Proportion of various negative behaviours per homework session for each parent}
    \label{fig:per_behaviour_stacked}
\end{figure}

Finally, we investigated the average number of different conflict types per homework involvement session for each parent, as shown in Figure \ref{fig:dis_stacked_conflict}. Our findings reveal significant variability in the number of conflicts experienced by different families. For example, participants like P28, P32, and P104 averaged more than 15 conflicts per session, while others, such as P3 and P35, experienced fewer than 3 conflicts on average. Moreover, the variety of conflict types also differed among participants. Some, like P3, encountered only 3 types of conflict, while others such as P4, P35, and P87 experienced 4 types. In contrast, participants like P0, P16, and P28 encountered all types of conflicts. This suggests that the complexity of conflict during homework sessions can vary greatly among families. Although \textit{Knowledge Conflict} was the most common type of conflict overall, some participants experienced other types more frequently. For instance, P0 primarily dealt with \textit{Time Management Conflict}, while P96 had \textit{Rule Conflict} as their dominant issue. These findings highlight that not only the number of conflicts but also the types of conflicts vary widely among families. 

%Finally, we invesitage the average number of different conflict types per homework involvement session for each parent, as shown in Figure \ref{fig:dis_stacked_conflict}. Overall, we found that different parents experience very different number of conflicts. some participants such as P28, P32 and P104 have more than 15 conflicts on average, while some participants such as P3 and P35 have less than 3 conflicts on average. In addition, some participants have only limited types of conflicts, such as P3 have 3 types of conflict, P4, P35, P87 have 4 types of conflict. However, for some particoants such as P0, P16, P28 have all kinds of conflicts, this reveal... In addition, Though \textit{Knowledge Conflict} (KC) is the top conflicts that occurs overall, some participants have \textit{Time Management Conflict} (TMC) most such as P0, some have \textit{Rule Conflict} (RC) such as 96. This reveals different families have very different conflicts types..

\begin{figure}
    \centering
    \includegraphics[width=.95\linewidth]{figure/dis_stacked_conflict.pdf}
    \caption{Average frequency of various conflict types per homework session for each parent}
    \label{fig:dis_stacked_conflict}
\end{figure}



%\input{section/6-conflict}

\section{Relationships Between Emotions, Behaviours and Conflict in Chinese Families}

This section explores the intricate relationships among emotions, behaviours, and conflicts during homework involvement in Chinese families. We calculate the total occurrences of each behaviour and conflict for every homework session. Data from 511 homework sessions across 65 families (excluding one due to insufficient data) were analyzed, with emotional shifts being measured as the difference in perceived pleasure, arousal, and dominance levels before and after homework.


\begin{figure}
    \centering
    \includegraphics[width=1\linewidth]{figure/heat_conflict_nums.pdf}
    \caption{Correlation between parental behaviours and parent-child conflicts (* p<0.05, ** p<0.01, *** p<0.001)}
    \label{fig:heatmap:conflict_behaviours}
\end{figure}


Figure \ref{fig:heatmap:conflict_behaviours} displays the correlation matrix between parental behaviours and conflicts, with behaviours categorized as positive, neutral, or negative (indicated by the bold lines). Surprisingly, the results reveal numerous statistically significant positive correlations between behaviours and conflicts, regardless of whether the behaviour is classified as positive, neutral, or negative. Even positive behaviours like praise are often linked with increased conflicts.

Among all behaviours, \textit{Neglect and Indifference} stands out as having no significant correlation with any conflict. This raises key questions: Could a certain level of detachment or more passive involvement help reduce conflict? How do we strike the right balance between involvement and creating a harmonious homework environment?  %The observation that \textit{Neglect and Indifference} (NI) is not significantly correlated with any type of conflict prompts further consideration of the role of active versus passive involvement. 
If conflict tends to arise during active involvement, it may be worth exploring whether less direct involvement could sometimes be more beneficial. The challenge for parents, then, is not merely to be involved but to carefully navigate how they are involved, considering which behaviours might exacerbate conflicts and which might help minimise them.

%Among all behaviors, Neglect and Indifference (NI) notably shows no significant correlation with any form of conflict. This raises key questions: Could a certain level of detachment or passive involvement actually reduce conflict? How do we strike the right balance between engagement and creating a harmonious homework environment? The absence of a link between NI and conflict invites further exploration into the impact of active versus passive involvement. If conflict tends to arise from more direct involvement, perhaps reducing engagement at certain times could be beneficial. These findings highlight that while parental involvement, even with good intentions, often correlates with conflict during homework sessions. The challenge for parents is not just to be involved but to thoughtfully manage their involvement, understanding which behaviors might escalate conflicts and which could help reduce them.

%Figure \ref{fig:heatmap:conflict_behaviours} reveals numerous significant positive correlations between parental behaviours and conflicts, regardless of behaviour type (positive, neutral, or negative). Surprisingly, even positive behaviours like praise are often linked with increased conflicts. Only \textit{Neglect and Indifference} (NI) showed no significant correlation with conflict, raising questions about whether less involvement might reduce conflict. Parental behaviours such as \textit{Setting Rules}, \textit{Error Correction}, and \textit{Criticism} were linked to all types of conflicts, while behaviours like \textit{Praise} and \textit{Direct Instruction} were correlated with fewer conflicts but did not entirely eliminate them.

It also reveals that different parental behaviours are linked to specific types of conflicts. 
%For instance, \textit{Setting Rules (SR)}, \textit{Error Correction} (EC), \textit{Direct Command} (DC), \textit{Criticism and Blame} (CB), \textit{Belittling and Doubting} (BD), and \textit{Impatience and Irritation} (II) are statistically significantly positively correlated with all types of conflicts. 
Parental behaviours such as \textit{Setting Rules}, \textit{Error Correction}, and \textit{Criticism and Blame} were linked to all types of conflicts. This suggests that while these behaviours may be intended to guide or correct the child, they often lead to be associated with increased friction during homework sessions. On the other hand, behaviours like \textit{Labelled Praise}, \textit{Unlabelled Praise}, and \textit{Direct Instruction} show fewer correlations with conflict, although they are not entirely conflict-free. %This indicates that while these behaviours may help mitigate conflict to some extent, but do not completely prevent it.

%The result of "Neglect and Indifference" (NI) stands out as the only behaviour with no significant correlations to conflict, prompting us to reconsider the role of active versus passive involvement. If conflict cannot be entirely avoided during active involvement, it may be worth investigating whether less direct involvement could sometimes be more beneficial. These findings suggest that parental involvement, even when intended to be supportive, often leads to conflict during homework sessions. The challenge for parents, then, is not merely to be involved but to carefully navigate how they are involved, considering which behaviours might exacerbate conflicts and which might help minimize them.


%the key challenge for parents is to carefully consider how they are involved. Further investigation is needed to understand which behaviours are associated with reducing conflicts and promoting a more harmonious homework environment.

%Figure \ref{fig:heatmap:conflict_behaviours} shows the correlation between parental behaviours and parent-child conflicts, where bold lines divides the behaviours into three types: positive, neutral and negative. Surprsingly, we found lots of statistically significant positive correlation between parental beahivours and different types of conclits, no matter behaviours type. Even for the 'postive' behaviours, they are high positively related with the parent-child conflicts. Especailly, only for the 'negalect and indifference ' behaviour, there is no obvious conflict correlation, which reveals a very interesting phynomonon to be discussed: if invovment always bring in conflict, to what extent  should we  encourage involvement. does conflict cannot avoid? is neglect and difference should be encouraged? how to find a balance between...In addition, we also find that, different behaviours may positively correalted with different kind of conflicts. Especially, the Setting Rules (SR), Error Correction (EC), Direct Command (DC), and Critisim and Belame (CB), Belitting and Douting (BD) and Impatience and irritation (II) are positively correalted to all kinds of conflicts. Also, besides the Neglect and Indifference (NI), the other behaviours that are  corrleated to least conflicts are: Specific Prase (SP), General Praise (GP) and Direct Instruction (DI), It shows that...



Figure \ref{fig:heatmap:emotion_conflict_behaviours} illustrates the correlation between perceived emotions, parental behaviours, and parent-child conflicts. The bold lines divide behaviours into positive, neutral, and negative categories, while the red line separates behaviours from conflicts.  While correlation does not imply causation, it is valuable to examine how emotions before and after homework relate to behaviours and conflicts. For instance, parents reporting greater pleasure before homework are significantly positively correlated with positive behaviours like \textit{Encouragement}, and negatively correlated with negative behaviours, such as \textit{Criticism and Blame}, \textit{Forcing and Threatening}, and \textit{Belittling and Doubting}. This indicates that happier parents are more likely to engage in supportive behaviours and less likely to be critical or aggressive during the homework session.

%shows the correlation between perceived emotions, parental behaviors, and parent-child conflicts. The bold lines divide behaviors into positive, neutral, and negative categories, while the red line separates behaviors from conflicts. While correlation does not imply causation, it is valuable to examine how emotions before and after homework relate to behaviors and conflicts. For example, parents reporting greater pleasure before homework are positively correlated with positive behaviors, like Encouragement, and negatively correlated with negative behaviors, such as Criticism, Forcing, and Belittling. This indicates that happier parents are more likely to engage in supportive behaviors and less likely to be critical or aggressive during homework sessions.

\begin{figure}
    \centering
    \includegraphics[width=1\linewidth]{figure/heat_emotion_conflict_behavior.pdf}
    \caption{Correlation between emotions, behaviours and parent-child conflicts (* p<0.05, ** p<0.01, *** p<0.001)}
    \label{fig:heatmap:emotion_conflict_behaviours}
\end{figure}

A similar pattern is observed with dominance before homework. Parents who feel more in control (i.e., higher dominance) are positively correlated with positive behaviours like \textit{Encouragement} and \textit{Labelled/Unlabelled Praise}, while negatively correlated with negative behaviours such as \textit{Impatience and Irritation} and \textit{Frustration and Disappointment}. These parents also experience fewer conflicts, such as \textit{Expectation Conflict}, \textit{Communication Conflict}, and \textit{Time Management Conflict}. This indicates that parents who start homework sessions in a positive emotional state tend to display more constructive behaviours and experience fewer conflicts.

%A similar pattern is observed with dominance before homework. Parents who feel more in control (i.e., higher dominance) are more likely to engage in positive behaviours like \textit{Encouragement} (ENC) and \textit{Labelled/Unlabelled Praise} (LP/UP) and are less prone to negative behaviours like \textit{Impatience and Irritation} (II) and \textit{Frustration and Disappointment} (FD). These parents also experience fewer conflicts, such as \textit{Expectation Conflict} (EC), \textit{Communication Conflict} (CC), and \textit{Time Management Conflict} (TMC). This indicates that a positive emotional state before homework tends to result in more constructive behaviours and fewer conflicts.

Post-homework emotions show a similar trend. Positive behaviours such as \textit{Encouragement} and \textit{Praise} are associated with higher pleasure and dominance, and lower arousal after homework sessions.  This pattern reflects everyday experiences: parents who engage in supportive behaviours often feel more satisfied and calm after homework sessions. In contrast, five out of six negative behaviours are significantly negatively correlated with post-homework pleasure and dominance, and positively correlated with arousal, indicating that negative behaviours tend to leave parents feeling more emotionally agitated and less in control after the session. \textit{Neglect and Indifference}, however, shows no significant correlation with post-homework emotions, suggesting that it may reflect emotional disengagement or detachment during the session.

%Post-homework emotions show a similar trend. Positive behaviours like \textit{Encouragement} (ENC) and \textit{Praise} (LP, UP) are associated with higher pleasure and dominance, and lower arousal, reflecting satisfaction and calmness after homework sessions. On the contrary, five of six negative behaviours are linked to lower post-homework pleasure and dominance, and higher arousal, indicating these behaviours lead to emotional agitation and loss of control. \textit{Neglect and Indifference} (NI), however, shows no significant correlation with post-homework emotions, suggesting emotional disengagement or detachment during the session.

Regarding conflicts, all conflict types are significantly negatively correlated with post-homework pleasure. This suggests that when conflicts occur during homework, parents tend to feel less satisfied or happy afterwards. Interestingly, \textit{Rule Conflict} shows the weakest correlation with post-homework emotions, indicating that it may be perceived as more neutral or procedural, potentially evoking fewer emotional reactions compared to other types of conflicts. This could mean that while disagreements over rules happen, they may not be as emotionally taxing as other conflict types.
On the other hand, \textit{Communication Conflict} and \textit{Knowledge Conflict} show significant correlations with all post-homework emotions as well as with emotional shifts. This suggests that these two types of conflicts are more emotionally engaging for parents and might be more likely to destabilize their emotional states during and after the homework process. Communication conflicts likely involve frustration and misunderstandings that are deeply personal, while knowledge conflicts could highlight gaps in understanding between parents and children, triggering stress or self-doubt in parents. Both conflict types are associated with notable shifts in arousal (indicative of stress) and dominance (control), pointing to their potential to cause significant emotional upheaval.


The above findings underscore the complex role of parental emotions, behaviours and conflicts. While parents' pleasure and dominance before homework are associated with more positive behaviours and fewer conflicts, even parents who begin homework sessions in a good emotional state may encounter emotionally charged situations, particularly in the case of knowledge or communication conflicts. The correlation between post-homework emotions and specific conflicts also suggests that conflicts during homework can have lingering emotional effects on parents.


\section{Implications and Limitations}
\label{sec:limit}
In this section, we briefly examine how our findings could inform potential design implications to improve parenting strategies in the future. It is important to note that the design space can be interpreted in various ways, depending on subjective perspectives.

%\subsubsection{Personalized Strategy for Different Families} We recommend developing individualized parent-child profiles based on specific family situations and refining them through historical interaction data. Our find that emotional responses during homework involvement varied widely among parents, with some showing significant decreases in pleasure and increases in arousal. Demographic data also showed diversity in education levels and children’s academic performance, suggesting varying family needs. Personalizing the system’s guidance based on these differences can improve intervention effectiveness by tailoring support to each family’s unique dynamics.


\textit{Adaptive Involvement Balancing}.
We suggest adjusting the level of parental involvement based on real-time emotional and behavioural cues to maintain a balance between support and autonomy. 
We found that even positive behaviours often led to parent-child conflicts except for \textit{Neglect and Indifference}. 
Future systems could use emotion and behaviour analysis to suggest optimal involvement levels, advising parents when to step back and allow the child more independence, especially during moments of tension. This approach aligns with \textit{Authoritative Parenting} \cite{gray1999unpacking}, characterized by high responsiveness and appropriate demands, which has been associated with positive child development outcomes. By dynamically adjusting involvement, parents can foster resilience and self-regulation in their children, promoting healthier emotional and social development.

%Parental involvement should be adjusted based on real-time emotional and behavioral cues to balance support and autonomy. Our study found that even positive behaviors often led to parent-child conflicts, except for "Neglect and Indifference" (NI). Future systems could use emotion and behavior analysis to suggest optimal involvement levels, advising parents when to step back and allow the child more independence, especially during moments of tension.

\textit{Behaviour-Specific Intervention Strategies}.
We suggest offering tailored interventions for specific parental behaviours that are strongly associated with particular types of conflicts, aiming to mitigate conflicts associated with each behaviour type. 
Behaviours like \textit{Setting Rules} and \textit{Error Correction} were associated with conflicts, while \textit{Labelled Praise} had fewer correlations.
It indicates each behaviour impacts conflict dynamics differently, and a one-size-fits-all approach may not be effective.
Tools offering real-time feedback could help parents replace conflict-inducing actions with more constructive alternatives, such as guiding parents who frequently engage in \textit{Error Correction} to reduce friction through supportive strategies.

%Tailored interventions for specific parental behaviours linked to conflicts can reduce tensions. Behaviors like "Setting Rules" (SR) and "Error Correction" (EC) were associated with conflicts, while "Labelled Praise" (LP) had fewer correlations. Tools offering real-time feedback could help parents replace conflict-inducing actions with more constructive alternatives, such as guiding parents who frequently engage in "Error Correction" to reduce friction through supportive strategies.

\textit{Emotional State-Aware Interaction Design}.
We suggest the system should adapt its interaction style and content based on the parent's emotional state before and during homework sessions.
Our analysis shows that parents with higher pleasure and dominance before a session are more likely to engage in positive behaviours and experience fewer conflicts, and vice versa.
Therefore, we suggest adapting the interaction style and content based on the pre-session emotional state. 
Systems could assess emotional states via self-reporting or subtle cues and adjust guidance accordingly, offering calming exercises or suggesting a delay if stress levels are high.

%The system should adapt its interaction style and content based on the parent's emotional state before and during homework sessions. Our analysis shows that parents with higher pleasure and dominance before a session are more likely to engage in positive behaviors and experience fewer conflicts. Systems could assess emotional states via self-reporting or subtle cues and adjust guidance accordingly, offering calming exercises or suggesting a delay if stress levels are high.


%\subsection{Limitations}
%\label{sec:limit}
This study is the first, to the best of our knowledge, to comprehensively investigate the emotional and behavioural dynamics of parental homework involvement through parent-child conversations. We had to make compromises that may limit its outreach: 
%While our study provides valuable insights into the emotional and behavioural dynamics of Chinese families during homework involvement, several limitations should be acknowledged:

\begin{figure}
    \centering
    \includegraphics[width=0.9
    \textwidth]{figure/hawthorn_dis1.pdf}
    \caption{Different impacts of recording on educational behaviours}
    \label{fig:hawthorn}
\end{figure}


\textit{Sampling Bias}. As outlined in Section \ref{subsec: participants}, the education levels of the parent participants were higher than the national average in China, 
%and most of the children were reported by parents to be performing 'above average' academically. 
This creates a sampling bias, as the data may not represent the broader spectrum of Chinese families. Future studies should aim to include a more diverse sample to better reflect the population as a whole.

\textit{Use of Transcripts Over Acoustic Data}. Our analysis relied solely on transcripts, excluding non-verbal cues such as tone and pitch. While this approach was driven to protect privacy and simplify data processing, it may limit the accuracy of emotion detection, especially since emotional nuances are often more precisely conveyed through audio. Additionally, parental behaviours or conflicts could have been more accurately identified through audio analysis, where variations in tone might reveal different levels of conflict or emotional states even when the verbal content remains the same.

%The analysis relied solely on transcripts, excluding non-verbal cues like tone and pitch. This approach, driven by privacy concerns, may have limited the accuracy of emotion detection and conflict identification, as audio data could better capture emotional nuances.



%Privacy was a top priority, with participants fully informed and data securely stored. However, using transcripts instead of more detailed audio-visual data due to privacy concerns may have constrained the depth of our analysis.

\textit{Hawthorne Effect}. The presence of audio recordings may have influenced parents' behaviour, a phenomenon known as the \textit{Hawthorne Effect}. To mitigate this, we surveyed participants daily, asking them to rate the extent to which the recordings affected their behaviour using a 5-point Likert scale\footnote{The question is \textit{``Due to the fact that you knew the homework involvement behaviour today was recorded, did this affect your true performance?''}}, using a 5-point Likert scale where 1 to 5 indicates \textit{``Completely unaffected''}, \textit{``Basically unaffected'}, \textit{``Slightly affected''}, \textit{``Significantly affected''} and \textit{`Extremely affected''}. As shown in Figure \ref{fig:hawthorn}, most participants (78.04\%) reported being either \textit{'Completely unaffected'} or \textit{'Basically unaffected'}, with only 7.26\% indicating a significant impact on their behaviour. While this suggests minimal influence, the possibility remains that the recordings altered parental behaviour.

\textit{Privacy Concerns}. Privacy is a critical consideration in our research. We took extensive measures to ensure that participants were fully informed about the data collection and future usage, and we strictly limited the scope of our study to homework-related interactions. Data was stored on secure hardware to protect the participants’ privacy. However, using transcripts instead of more detailed audio-visual data due to privacy concerns may have constrained the depth of our analysis.

%The presence of recordings may have influenced parental behavior. Daily surveys revealed that most participants (78.04%) felt their behavior was unaffected or minimally affected by the recordings, but there remains the possibility that behavior was altered due to awareness of being recorded.

%Howthorn effect. In our study, the audio recording may have led parents to become aware of this monitoring, subsequently influencing their educational behaviours, a phenomenon known as the \textit{hawthorne effect}. To address it, participants were surveyed daily to report the impact of the recording on their behaviour\footnote{The question is \textit{`Due to the fact that you knew the homework involvement behaviour today was recorded, did this affect your true performance?'}}, using a 5-point Likert scale where 1 to 5 indicates \textit{`Completely unaffected'}, \textit{`Basically unaffected'}, \textit{`Slightly affected'}, \textit{`Significantly affected'} and \textit{`Extremely affected'}. Figure \ref{fig:hawthorn} illustrates the distribution of responses across participants. Notably, most participants (78.04\%) reported being either \textit{`Completely unaffected'} or \textit{`Basically unaffected'}, with only 7.26\% indicating significant influence (either \textit{`Significantly affected'} or \textit{`Extremely affected'}). This suggests that the \textit{hawthorne effect} was minimal among our participants. Nonetheless, future research should explore audio recording's influence on parental involvement behaviours further.





\section{Conclusion}
This paper presents {\tool}, a novel interactive text-to-SQL annotation system that enables users to create high-quality, schema-specific text-to-SQL datasets.
{\tool} integrates multiple functionalities, including schema customization, database synthesis, query alignment, dataset analysis, and additional features such as confidence scoring.
A user study with 12 participants demonstrates that by combining these features, {\tool} significantly reduces annotation time while improving the quality of the annotated data.
{\tool} effectively bridges the gap resulting from insufficient training and evaluation datasets for new or unexplored schemas.


%%
%% The acknowledgments section is defined using the "acks" environment
%% (and NOT an unnumbered section). This ensures the proper
%% identification of the section in the article metadata, and the
%% consistent spelling of the heading.
\begin{acks}
We sincerely thank the anonymous reviewers for their suggestions to improve this manuscript. 
\end{acks}

%%
%% The next two lines define the bibliography style to be used, and
%% the bibliography file.
\bibliographystyle{ACM-Reference-Format}
\bibliography{refs}

\appendix
%TC:ignore
\section{Dataset Examples}
\label{app:dataset-eg}
Figure \ref{fig:dataset-eg} illustrates example data instances from MemeCap, NewYorker, and YesBut.

\begin{figure*}[t]
  \includegraphics[width=\linewidth]{figures/dataset-eg.pdf} \hfill
  \caption {Dataset Examples on MemeCap, NewYorker, and YesBut.}
  \label{fig:dataset-eg}
\end{figure*}


\section{SentenceSHAP}
\label{app:sentence-shap}
In this section, we introduce SentenceSHAP, an adaptation of TokenSHAP \cite{horovicz-goldshmidt-2024-tokenshap}. While TokenSHAP calculates the importance of individual tokens, SentenceSHAP estimates the importance of individual sentences in the input prompt. The importance score is calculated using Monte Carlo Shapley Estimation, following the same principles as TokenSHAP.

Given an input prompt \( X = \{x_1, x_2, \dots, x_n\} \), where \( x_i \) represents a sentence, we generate all possible combinations of \( X \) by excluding each sentence \( x_i \) (i.e., \( X - \{x_i\} \)). Let \( Z \) represent the set of all combinations where each \( x_i \) is removed. To estimate Shapley values efficiently, we randomly sample from \( Z \) with a specified sampling ratio, resulting in a subset \( Z_s = \{X_1, X_2, \dots, X_s\} \), where each \( X_i = X - \{x_i\} \).

Next, we generate a base response \( r_0 \) using a VLM (or LLM) with the original prompt \( X \), and a set of responses \( R_s = \{r_1, r_2, \dots, r_s\} \), each generated by a prompt from one of the sampled combinations in \( Z_s \).

We then compute the cosine similarity between the base response \( r_0 \) and each response in \( R_s \) using Sentence Transformer (\texttt{BAAI/bge-large-en-v1.5}). The average similarity between combinations with and without \( x_i \) is computed, and the difference between these averages gives the Shapley value for sentence \( x_i \). This is expressed as:
\begin{align}
\notag
\phi(x_i) = \\ \notag
&\frac{1}{s} \sum_{j=1}^{s} \left( \text{cos}(r_0, r_j \mid x_i) - \text{cos}(r_0, r_j \mid \neg x_i) \right)
\end{align}
where \( \phi(x_i) \) represents the Shapley value for sentence \( x_i \), $\text{cos}(r_0, r_j \mid x_i)$ is the cosine similarity between the base response and the response that includes sentence $x_i$, $\text{cos}(r_0, r_j \mid \neg x_i)$ is the cosine similarity between the base response and the response that excludes sentence $x_i$, and $s$ is the number of sampled combinations in $Z_s$.

\section{Error Analysis Based on SentenceSHAP}
Figure \ref{fig:error-analysis} presents two examples of negative impacts from implications: dilution of focus and the introduction of irrelevant information.
\label{app:error-analysis-shap}
\begin{figure*}[t]
  \includegraphics[width=\linewidth]{figures/error-analysis.pdf} \hfill
  \caption {Examples of negative impact from implications from Phi (top) and GPT4o (bottom).}
  \label{fig:error-analysis}
\end{figure*}

\section{Details on human anntations}
\label{app:cloudresearch}
We present the annotation interface on CloudResearch used for human evaluation to validate our evaluation metric in Figure \ref{fig:cloud-research}. Refer to Sec.~\ref{sec:ethics} for details on annotator selection criteria and compensation.

\begin{figure*}[t]
  \includegraphics[width=\linewidth]{figures/cloud-research.pdf} \hfill
  \caption {Annotation interface on CloudResearch used for human evaluation to validate our evaluation metric.}
  \label{fig:cloud-research}
\end{figure*}



\section{Generation Prompts for Selection and Refinement}
\label{app:gen-prompts}
Figures \ref{fig:desc-prompt}, \ref{fig:seed-imp-prompt}, and \ref{fig:nonseed-imp-prompt} show the prompts used for generating image descriptions, seed implications (1st hop), and non-seed implications (2nd hop onward). Figure \ref{fig:cand-prompt} displays the prompt used to generate candidate and final explanations. Image descriptions are used for candidate explanations when existing data is insufficient but are not used for final explanations. For calculating Cross Entropy values (used as a relevance term), we use the prompt in Figure \ref{fig:cand-prompt}, substituting the image with image descriptions, as LLM is used to calculate the cross entropies.

\begin{figure*}[h]
\small
\begin{tcolorbox}[
    title=Prompt for Image Descriptions,
    colback=white,
    colframe=CadetBlue,
    arc=0pt,        % Remove rounded corners
    outer arc=0pt   % Remove outer rounded corners (important for some styles)    
]

Describe the image by focusing on the noun phrases that highlight the actions, expressions, and interactions of the main visible objects, facial expressions, and people.\\
\\
Here are some guidelines when generating image descriptions:\\
* Provide specific and detailed references to the objects, their actions, and expressions. Avoid using pronouns in the description.\\
* Do not include trivial details such as artist signatures, autographs, copyright marks, or any unrelated background information.\\
* Focus only on elements that directly contribute to the meaning, context, or main action of the scene.\\
* If you are unsure about any object, action, or expression, do not make guesses or generate made-up elements.\\
* Write each sentence on a new line.\\
* Limit the description to a maximum of 5 sentences, with each focusing on a distinct and relevant aspect that directly contribute to the meaning, context, or main action of the scene.\\
\\
Here are some examples of desired output:
---\\
\text{[Description]} (example of newyorker cartoon image):\\
Through a window, two women with surprised expressions gaze at a snowman with human arms.\\
---\\
\text{[Description]} (example of newyorker cartoon image):\\
A man and a woman are in a room with a regular looking bookshelf and regular sized books on the wall.\\
In the middle of the room the man is pointing to text written on a giant open book which covers the entire floor.\\
He is talking while the woman with worried expression watches from the doorway.\\
---\\
\text{[Description]} (example of meme):\\
The left side shows a woman angrily pointing with a distressed expression, yelling ``You said memes would work!''.\\
The right side shows a white cat sitting at a table with a plate of food in front of it, looking indifferent or smug with the text above the cat reads, ``I said good memes would work''.\\
---\\
\text{[Description]} (example of yesbut image):\\
The left side shows a hand holding a blue plane ticket marked with a price of ``\$50'', featuring an airplane icon and a barcode, indicating it's a flight ticket.\\
The right side shows a hand holding a smartphone displaying a taxi app, showing a route map labeled ``Airport'' and a price of ``\$65''.\\
---\\

Proceed to generate the description.\\
\text{[Description]}:

\end{tcolorbox}
\caption{A prompt used to generate image descriptions.} % Add a caption to the figure
\label{fig:desc-prompt}
\end{figure*}


%%%%%%%%%%%%%%%%%%%%%%%%%%% Prompt for implications %%%%%%%%%%%%%%%%%%%%%%%%%%%
\begin{figure*}[t]
\small
\begin{tcolorbox}[
    title=Prompt for Seed Implications,
    colback=white,
    colframe=Green,
    arc=0pt,        % Remove rounded corners
    outer arc=0pt,  % Remove outer rounded corners (important for some styles)    
    % breakable,
]

You are provided with the following inputs:\\
- \text{[}Image\text{]}: An image (e.g. meme, new yorker cartoon, yes-but image)\\
- \text{[}Caption\text{]}: A caption written by a human.\\
- \text{[}Descriptions\text{]}: Literal descriptions that detail the image.\\
\\
\#\#\# Your Task:\\
\texttt{[ One-sentence description of the ultimate goal of your task. Customize based on the task. ]}\\
Infer implicit meanings, cultural references, commonsense knowledge, social norms, or contrasts that connect the caption to the described objects, concepts, situations, or facial expressions.\\
\\
\#\#\# Guidelines:\\
- If you are unsure about any details in the caption, description, or implication, refer to the original image for clarification.\\
- Identify connections between the objects, actions, or concepts described in the inputs.\\
- Explore possible interpretations, contrasts, or relationships that arise naturally from the scene, while staying grounded in the provided details.\\
- Avoid repeating or rephrasing existing implications. Ensure each new implication introduces fresh insights or perspectives.\\
- Each implication should be concise (one sentence) and avoid being overly generic or vague.\\
- Be specific in making connections, ensuring they align with the details provided in the caption and descriptions.\\
- Generate up to 3 meaningful implications.\\
\\
\#\#\# Example Outputs:\\
\#\#\#\# Example 1 (example of newyorker cartoon image):\\
\text{[}Caption\text{]}: ``This is the most advanced case of Surrealism I've seen.''\\
\text{[}Descriptions\text{]}: A body in three parts is on an exam table in a doctor's office with the body's arms crossed as though annoyed.\\
\text{[}Connections\text{]}:\\
1. The dismembered body is illogical and impossible, much like Surrealist art, which often explores the absurd.\\
2. The body’s angry posture adds a human emotion to an otherwise bizarre scenario, highlighting the strange contrast.\\
\\
\#\#\#\# Example 2 (example of newyorker cartoon image):\\
\text{[}Caption\text{]}: ``He has a summer job as a scarecrow.''\\
\text{[}Descriptions\text{]}: A snowman with human arms stands in a field.\\
\text{[}Connections\text{]}:\\
1. The snowman, an emblem of winter, represents something out of place in a summer setting, much like a scarecrow's seasonal function.\\
2. The human arms on the snowman suggest that the role of a scarecrow is being played by something unexpected and seasonal.\\
\\
\#\#\#\# Example 3 (example of yesbut image):\\
\text{[}Caption\text{]}: ``The left side shows a hand holding a blue plane ticket marked with a price of `\$50'.''\\
\text{[}Descriptions\text{]}: The screen on the right side shows a route map labeled ``Airport'' and a price of `\$65'.\\
\text{[}Connections\text{]}:\\
1. The discrepancy between the ticket price and the taxi fare highlights the often-overlooked costs of travel beyond just booking a flight.\\
2. The image shows the hidden costs of air travel, with the extra fare representing the added complexity of budgeting for transportation.\\
\\
\#\#\#\# Example 4 (example of meme):\\
\text{[}Caption\text{]}: ``You said memes would work!''\\
\text{[}Descriptions\text{]}: A cat smirks with the text ``I said good memes would work.''\\
\text{[}Connections\text{]}:\\
1. The woman's frustration reflects a common tendency to blame concepts (memes) instead of the quality of execution, as implied by the cat’s response.\\
2. The contrast between the angry human and the smug cat highlights how people often misinterpret success as simple, rather than a matter of quality.\\
\\
\#\#\# Now, proceed to generate output:\\
\text{[}Caption\text{]}: \texttt{[ Caption ]}\\
\\
\text{[}Descriptions\text{]}:\\
\texttt{[ Descriptions ]}\\
\\
\text{[}Connections\text{]}:

\end{tcolorbox}
\caption{A prompt used to generate seed implications.} % Add a caption to the figure
\label{fig:seed-imp-prompt}
\end{figure*}


%%%%%%%%%%%%%%%%%%%%%%%%%%% Prompt for nonseed implications %%%%%%%%%%%%%%%%%%%%%%%%%%%
\begin{figure*}[t]
\small
%  \begin{tcolorbox}[
%  width=\textwidth,
%  colback={white},
%  title={Title},
%  colbacktitle={DarkGreen},
%  coltitle=white,
%  colframe={DarkGreen},
%  breakable
% ]
 % \parskip=5pt

\begin{tcolorbox}[
    % breakable,
    title=Prompt for Non-Seed Implications (2nd hop onward),
    colback=white,
    colframe=Green,
    arc=0pt,        % Remove rounded corners
    outer arc=0pt,  % Remove outer rounded corners (important for some styles)    
    % breakable,
]

You are provided with the following inputs:\\
- \text{[}Image\text{]}: An image (e.g. meme, new yorker cartoon, yes-but image)\\
- \text{[}Caption\text{]}: A caption written by a human.\\
- \text{[}Descriptions\text{]}: Literal descriptions that detail the image.\\
- \text{[}Implication\text{]}: A previously generated implication that suggests a possible connection between the objects or concepts in the caption and description.\\
\\
\#\#\# Your Task:\\
\texttt{[ One-sentence description of the ultimate goal of your task. Customize based on the task. ]}\\
Infer implicit meanings across the objects, concepts, situations, or facial expressions found in the caption, description, and implication. Focus on identifying relevant commonsense knowledge, social norms, or underlying connections.\\
\\
\#\#\# Guidelines:\\
- If you are unsure about any details in the caption, description, or implication, refer to the original image for clarification.\\
- Identify potential connections between the objects, actions, or concepts described in the inputs.\\
- Explore interpretations, contrasts, or relationships that naturally arise from the scene while remaining grounded in the inputs.\\
- Avoid repeating or rephrasing existing implications. Ensure each new implication provides fresh insights or perspectives.\\
- Each implication should be concise (one sentence) and avoid overly generic or vague statements.\\
- Be specific in the connections you make, ensuring they align closely with the details provided.\\
- Generate up to 3 meaningful implications that expand on the implicit meaning of the scene.\\
\\
\#\#\# Example Outputs:\\
\#\#\#\# Example 1 (example of newyorker cartoon image):\\
\text{[}Caption\text{]}: "This is the most advanced case of Surrealism I've seen."\\
\text{[}Descriptions\text{]}: A body in three parts is on an exam table in a doctor's office with the body's arms crossed as though annoyed.\\
\text{[}Implication\text{]}: Surrealism is an art style that emphasizes strange, impossible, or unsettling scenes.\\
\text{[}Connections\text{]}:\\
1. A body in three parts creates an unsettling juxtaposition with the clinical setting, which aligns with Surrealist themes.\\
2. The body’s crossed arms add humor by assigning human emotion to an impossible scenario, reflecting Surrealist absurdity.\\
... \\
\texttt{[ We used sample examples from the prompt for generating seed implications (see Figure \ref{fig:seed-imp-prompt}), following the above format, which includes [Implication]:. ]}
\\
---\\
\\
\#\#\# Proceed to Generate Output:\\
\text{[}Caption\text{]}: \texttt{[ Caption ]}\\
\\
\text{[}Descriptions\text{]}:\\
\texttt{[ Descriptions ]}\\
\\
\text{[}Implication\text{]}:\\
\texttt{[ Implication ]}\\
\\
\text{[}Connections\text{]}:
\end{tcolorbox}
\caption{A prompt used to generate non-seed implications.} % Add a caption to the figure
\label{fig:nonseed-imp-prompt}
\end{figure*}


%%%%%%%%%%%%%%%%%%%%%%%%%%% Prompt for nonseed implications %%%%%%%%%%%%%%%%%%%%%%%%%%%
\begin{figure*}[t]
\small
%  \begin{tcolorbox}[
%  width=\textwidth,
%  colback={white},
%  title={Title},
%  colbacktitle={DarkGreen},
%  coltitle=white,
%  colframe={DarkGreen},
%  breakable
% ]
 % \parskip=5pt

\begin{tcolorbox}[
    % breakable,
    title=Prompt for Candidate and Final Explanations,
    colback=white,
    colframe=RedViolet,
    arc=0pt,        % Remove rounded corners
    outer arc=0pt,  % Remove outer rounded corners (important for some styles)    
    % breakable,
]

You are provided with the following inputs:\\
- **\text{[}Image\text{]}:** A New Yorker cartoon image.\\
- **\text{[}Caption\text{]}:** A caption written by a human to accompany the image.\\
- **\text{[}Image Descriptions\text{]}:** Literal descriptions of the visual elements in the image.\\
- **\text{[}Implications\text{]}:** Possible connections or relationships between objects, concepts, or the caption and the image.\\
- **\text{[}Candidate Answers\text{]}:** Example answers generated in a previous step to provide guidance and context.\\
\\
\#\#\# Your Task:\\
Generate **one concise, specific explanation** that clearly captures why the caption is funny in the context of the image. Your explanation must provide detailed justification and address how the humor arises from the interplay of the caption, image, and associated norms or expectations.\\
\\
\#\#\# Guidelines for Generating Your Explanation:\\
1. **Clarity and Specificity:**  \\
   - Avoid generic or ambiguous phrases.  \\
   - Provide specific details that connect the roles, contexts, or expectations associated with the elements in the image and its caption.  \\
\\
2. **Explain the Humor:**  \\
- Clearly connect the humor to the caption, image, and any cultural, social, or situational norms being subverted or referenced.  \\
- Highlight why the combination of these elements creates an unexpected or amusing contrast.\\
\\
3. **Prioritize Clarity Over Brevity:**  \\
- Justify the humor by explaining all important components clearly and in detail.  \\
- Aim to keep your response concise and under 150 words while ensuring no critical details are omitted.  \\
\\
4. **Use Additional Inputs Effectively:**\\
- **\text{[}Image Descriptions\text{]}:** Provide a foundation for understanding the visual elements."   \\
- **\text{[}Implications\text{]}:** Assist in understanding relationships and connections but do not allow them to dominate or significantly alter the central idea.\\
- **\text{[}Candidate Answers\text{]}:** Adapt your reasoning by leveraging strengths or improving upon weaknesses in the candidate answers.\\
\\
Now, proceed to generate your response based on the provided inputs.\\
\\
\#\#\# Inputs:\\
\text{[}Caption\text{]}: \texttt{\text{[} Caption \text{]}}\\
\\
\text{[}Descriptions\text{]}:\\
\texttt{\text{[} Top-K Implications \text{]}}\\
\\
\text{[}Implications\text{]}:\\
\texttt{\text{[} Top-K Implications \text{]}}\\
\\
\text{[}Candidate Anwers\text{]}:\\
\texttt{\text{[} Top-K Candidate Explanations \text{]}}\\
\\
\text{[}Output\text{]}:\\

\end{tcolorbox}
\caption{A prompt used to generate candidate and final explanations.} % Add a caption to the figure
\label{fig:cand-prompt}
\end{figure*}


\section{Evaluation Prompts}
\label{app:eval-prompts}
Figures \ref{fig:recall-prompt} and \ref{fig:precision-prompt} present the prompts used to calculate recall and precision scores in our LLM-based evaluation, respectively.

%%%%%%%%%%%%%%%%%%%%%%%%%%% Prompt for nonseed implications %%%%%%%%%%%%%%%%%%%%%%%%%%%
\begin{figure*}[t]
\small
\begin{tcolorbox}[
    % breakable,
    title=Prompt for Evaluating Recall Score,
    colback=white,
    colframe=MidnightBlue,
    arc=0pt,        % Remove rounded corners
    outer arc=0pt,  % Remove outer rounded corners (important for some styles)    
    % breakable,
]

Your task is to assess whether \text{[}Sentence1\text{]} is conveyed in \text{[}Sentence2\text{]}. \text{[}Sentence2\text{]} may consist of multiple sentences.\\
\\
Here are the evaluation guidelines:\\
1. Mark 'Yes' if \text{[}Sentence1\text{]} is conveyed in \text{[}Sentence2\text{]}.\\
2. Mark 'No' if \text{[}Sentence2\text{]} does not convey the information in \text{[}Sentence1\text{]}.\\
\\
Proceed to evaluate. \\
\\
\text{[}Sentence1\text{]}: \texttt{[ One Atomic Sentence from Decomposed Reference Explanation ]} \\
\\
\text{[}Sentence2\text{]}: \texttt{[ Predicted Explanation ]}\\
\\
\text{[}Output\text{]}:

\end{tcolorbox}
\caption{Prompt for evaluating recall score.} % Add a caption to the figure
\label{fig:recall-prompt}
\end{figure*}


\begin{figure*}[t]
\small
\begin{tcolorbox}[
    % breakable,
    title=Prompt for Evaluating Precision Score,
    colback=white,
    colframe=MidnightBlue,
    arc=0pt,        % Remove rounded corners
    outer arc=0pt,  % Remove outer rounded corners (important for some styles)    
    % breakable,
]

Your task is to assess whether \text{[}Sentence1\text{]} is inferable from \text{[}Sentence2\text{]}. \text{[}Sentence2\text{]} may consist of multiple sentences.\\
\\
Here are the evaluation guidelines:\\
1. Mark "Yes" if \text{[}Sentence1\text{]} can be inferred from \text{[}Sentence2\text{]} — whether explicitly stated, implicitly conveyed, reworded, or serving as supporting information.\\
2. Mark 'No' if \text{[}Sentence1\text{]} is absent from \text{[}Sentence2\text{]}, cannot be inferred, or contradicts it.\\
\\
Proceed to evaluate. \\
\\
\text{[}Sentence1\text{]}: \texttt{[ One Atomic Sentence from Decomposed Predicted Explanation ]}\\
\\
\text{[}Sentence2\text{]}: \texttt{[ Reference Explanation ]}\\
\\
\text{[}Output\text{]}:


\end{tcolorbox}
\caption{Prompt for evaluating precision score.} % Add a caption to the figure
\label{fig:precision-prompt}
\end{figure*}

\section{Prompts for Baselines}
\label{app:base-prompts}

Figure \ref{fig:base-prompt} presents the prompt used for the ZS, CoT, and SR Generator methods. While the format remains largely the same, we adjust it based on the baseline being tested (e.g., CoT requires generating intermediate reasoning, so we add extra instructions for that).
Figure \ref{fig:critic-prompt} shows the prompt used in the SR critic model. The critic's criteria include: (1) \textit{correctness}, measuring whether the explanation directly addresses why the caption is humorous in relation to the image and its caption; (2) \textit{soundness}, evaluating whether the explanation provides a well-reasoned interpretation of the humor; (3) \textit{completeness}, ensuring all important aspects in the caption and image contributing to the humor are considered; (4) \textit{faithfulness}, verifying that the explanation is factually consistency with the image and caption; and (5) \textit{clarity}, ensuring the explanation is clear, concise, and free from unnecessary ambiguity.
\begin{figure*}
\small
\begin{tcolorbox}[
    % breakable,
    title=Prompt for Baselines,
    colback=white,
    colframe=Black,
    arc=0pt,        % Remove rounded corners
    outer arc=0pt,  % Remove outer rounded corners (important for some styles)    
    % breakable,
]

You are provided with the following inputs:\\
- **\text{[}Image\text{]}:** A New Yorker cartoon image.\\
- **\text{[}Caption\text{]}:** A caption written by a human to accompany the image.\\
\texttt{[ if Self-Refine with Critic is True: ]} \\
- **\text{[}Feedback for Candidate Answer\text{]}:** Feedback that points out some weakness in the current candidate responses.\\
\texttt{[ if Self-Refine is True: ]} \\
- **\text{[}Candidate Answers\text{]}:** Example answers generated in a previous step to provide guidance and context.\\
\\
\#\#\# Your Task:\\
Generate **one concise, specific explanation** that clearly captures why the caption is funny in the context of the image. Your explanation must provide detailed justification and address how the humor arises from the interplay of the caption, image, and associated norms or expectations.\\
\\
\#\#\# Guidelines for Generating Your Explanation:\\
1. **Clarity and Specificity:**  \\
   - Avoid generic or ambiguous phrases.  \\
   - Provide specific details that connect the roles, contexts, or expectations associated with the elements in the image and its caption.  \\
\\
2. **Explain the Humor:**  \\
- Clearly connect the humor to the caption, image, and any cultural, social, or situational norms being subverted or referenced.  \\
- Highlight why the combination of these elements creates an unexpected or amusing contrast.\\
\\
3. **Prioritize Clarity Over Brevity:**  \\
- Justify the humor by explaining all important components clearly and in detail.  \\
- Aim to keep your response concise and under 150 words while ensuring no critical details are omitted.  \\
\\
\texttt{[ if Self-Refine is True: ]}\\
4. **Use Additional Inputs Effectively:**\\
- **[Candidate Answers]:** Adapt your reasoning by leveraging strengths or improving upon weaknesses in candidate answers. \\
\texttt{[ if Self-Refine with Critic is True: ]}\\
- **[Feedback for Candidate Answer]:** Feedback that points out some weaknesses in the current candidate responses.\\
\\
\texttt{ [ if CoT is True: ]} \\
Begin by analyzing the image and the given context, and explain your reasoning briefly before generating your final response. \\
\\
Here is an example format of the output: \\
\{\{ \\
    "Reasoning": "...", \\
    "Explanation": "..."   \\
\}\} \\

Now, proceed to generate your response based on the provided inputs.\\
\\
\#\#\# Inputs:\\
\text{[}Caption\text{]}: \texttt{\text{[} Caption \text{]}}\\
\\
\text{[}Candidate Answers\text{]}: \texttt{\text{[} Candidate Explanations \text{]}}\\
\\
\text{[}[Feedback for Candidate Answer]:\text{]}: \texttt{\text{[} Feedback for Candidate Explanations \text{]}}\\
\\
\text{[}Output\text{]}:\\

\end{tcolorbox}
\caption{A prompt used for baseline methods, with conditions added based on the specific baseline being experimented with.} % Add a caption to the figure
\label{fig:base-prompt}
\end{figure*}


\begin{figure*}
\small
\begin{tcolorbox}[
    % breakable,
    title=Prompt for Self-Refine Critic,
    colback=white,
    colframe=Black,
    arc=0pt,        % Remove rounded corners
    outer arc=0pt,  % Remove outer rounded corners (important for some styles)    
    % breakable,
]
\texttt{[ Customize goal text here: ]} \\
\texttt{MemeCap:} You will be given a meme along with its caption, and a candidate response that describes what meme poster is trying to convey. \\
\texttt{NewYorker:} You will be given an image along with its caption, and a candidate response that explains why the caption is funny for the given image. \\
\texttt{YesBut:} You will be given an image and a candidate response that describes why the image is funny or satirical. \\
\\
Your task is to criticize the candidate response based on the following evaluation criteria: \\
- Correctness: Does the explanation directly address why the caption is funny, considering both the image and its caption? \\
- Soundness: Does the explanation provide a meaningful and well-reasoned interpretation of the humor? \\
- Completeness: Does the explanation address all relevant aspects of the caption and image (e.g., visual details, text) that contribute to the humor? \\
- Faithfulness: Is the explanation factually consistent with the details in the image and caption? \\
- Clarity: Is the explanation clear, concise, and free from unnecessary ambiguity? \\
 \\
Proceed to criticize the candidate response ideally using less than 5 sentences:\\
\\
\text{[}Caption\text{]}: \texttt{[ caption ]}\\
\\
\text{[}Candidate Response\text{]}: \\
 \texttt{\text{[} Candidate Response \text{]}}\\
\\
\text{[}Output\text{]}: \\
\end{tcolorbox}
\caption{A prompt used in SR critic model.} % Add a caption to the figure
\label{fig:critic-prompt}
\end{figure*}

% \begin{figure*}[t]
%   \includegraphics[width=\linewidth]{figures/error-analysis.pdf} \hfill
%   \vspace{-20pt}
%   \caption {Examples of negative impact from implications from Phi (top) and GPT4o (bottom).}
%   \label{fig:error-analysis}
% \end{figure*}
%TC:endignore
\end{document}
\endinput
%%
%% End of file `sample-manuscript.tex'.
