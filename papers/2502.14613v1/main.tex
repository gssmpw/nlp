% This must be in the first 5 lines to tell arXiv to use pdfLaTeX, which is strongly recommended.
\pdfoutput=1
% In particular, the hyperref package requires pdfLaTeX in order to break URLs across lines.

\documentclass[11pt]{article}
\usepackage{acl}

% standard packages
\usepackage{times}
\usepackage{latexsym}
\usepackage[T1]{fontenc}
\usepackage[utf8]{inputenc}
\usepackage{microtype}
\usepackage{inconsolata}

% custom packages
\usepackage{amsmath}
\usepackage{amssymb}
\usepackage{graphicx}
\usepackage{booktabs}
\usepackage{tabularx}
% \usepackage[table]{xcolor}
\usepackage{multirow}
\usepackage[capitalize,nameinlink,noabbrev]{cleveref}
\crefname{section}{\S}{\S\S}  % Like: § 4
\crefname{subsection}{\S}{\S\S}  % Like: § 4.1
\crefname{subsubsection}{\S}{\S\S} % Like: § 4.1.1
\newcounter{codeboxinput}
\crefname{codeboxinput}{Listing}{Listings}
\usepackage[shortcuts]{extdash}  % non-breaking dashes
\usepackage{subcaption}
\usepackage{fontawesome5}
\usepackage{algorithm}
\usepackage{algpseudocodex}

\usepackage{enumitem}
\newlist{todolist}{itemize}{2}
\setlist[todolist]{label=$\square$}

% Text highlights (uses soul package)
% Colors
\usepackage{soul}
\definecolor{pastelblue}{HTML}{A1C9F4}
\definecolor{pastelorange}{HTML}{FFB482}
\definecolor{pastelgreen}{HTML}{8DE5A1}
\definecolor{pastelred}{HTML}{FF9F9B}
\definecolor{pastelpurple}{HTML}{D0BBFF}
\newcommand{\hlred}[1]{\sethlcolor{pastelred}\hl{#1}}
\newcommand{\hlorange}[1]{\sethlcolor{pastelorange}\hl{#1}}
\newcommand{\hlblue}[1]{\sethlcolor{pastelblue}\hl{#1}}
\newcommand{\hlgreen}[1]{\sethlcolor{pastelgreen}\hl{#1}}

% This must be in the first 5 lines to tell arXiv to use pdfLaTeX, which is strongly recommended.
\pdfoutput=1
% In particular, the hyperref package requires pdfLaTeX in order to break URLs across lines.

\documentclass[11pt]{article}

% Change "review" to "final" to generate the final (sometimes called camera-ready) version.
% Change to "preprint" to generate a non-anonymous version with page numbers.
\usepackage[preprint]{acl}

% Standard package includes
\usepackage{times}
\usepackage{latexsym}
\usepackage{booktabs}
\usepackage{multirow} 
\usepackage{amsmath}
\usepackage{amsfonts}
\usepackage{colortbl}
\usepackage{graphicx} % 用于插入图片
\usepackage{subcaption} % 用于分割小图和添加子标题
% For proper rendering and hyphenation of words containing Latin characters (including in bib files)
\usepackage[T1]{fontenc}
% For Vietnamese characters
% \usepackage[T5]{fontenc}
% See https://www.latex-project.org/help/documentation/encguide.pdf for other character sets

% This assumes your files are encoded as UTF8
\usepackage[utf8]{inputenc}

% This is not strictly necessary, and may be commented out,
% but it will improve the layout of the manuscript,
% and will typically save some space.
\usepackage{microtype}

% This is also not strictly necessary, and may be commented out.
% However, it will improve the aesthetics of text in
% the typewriter font.
\usepackage{inconsolata}

%Including images in your LaTeX document requires adding
%additional package(s)
\usepackage{graphicx}

% If the title and author information does not fit in the area allocated, uncomment the following
%
%\setlength\titlebox{<dim>}
%
% and set <dim> to something 5cm or larger.

\title{RGAR: Recurrence Generation-augmented Retrieval for Factual-aware Medical Question Answering}

% Author information can be set in various styles:
% For several authors from the same institution:
% \author{Author 1 \and ... \and Author n \\
%         Address line \\ ... \\ Address line}
% if the names do not fit well on one line use
%         Author 1 \\ {\bf Author 2} \\ ... \\ {\bf Author n} \\
% For authors from different institutions:
% \author{Author 1 \\ Address line \\  ... \\ Address line
%         \And  ... \And
%         Author n \\ Address line \\ ... \\ Address line}
% To start a separate ``row'' of authors use \AND, as in
% \author{Author 1 \\ Address line \\  ... \\ Address line
%         \AND
%         Author 2 \\ Address line \\ ... \\ Address line \And
%         Author 3 \\ Address line \\ ... \\ Address line}

% \author{First Author \\
%   Affiliation / Address line 1 \\
%   Affiliation / Address line 2 \\
%   Affiliation / Address line 3 \\
%   \texttt{email@domain} \\\And
%   Second Author \\
%   Affiliation / Address line 1 \\
%   Affiliation / Address line 2 \\
%   Affiliation / Address line 3 \\
%   \texttt{email@domain} \\}

\author{
 \textbf{Sichu Liang \textsuperscript{1}}\thanks{Equal Contribution},
 \textbf{Linhai Zhang \textsuperscript{2}}\footnotemark[1],
 \textbf{Hongyu Zhu \textsuperscript{3}}\footnotemark[1],
 \textbf{Wenwen Wang\textsuperscript{4}},
 \textbf{Yulan He\textsuperscript{2, 5}},
 \textbf{Deyu Zhou \textsuperscript{1}}\thanks{Corresponding author} 
%  \textbf{Seventh Author\textsuperscript{1}},
%  \textbf{Eighth Author \textsuperscript{1,2,3,4}},
% \\
%  \textbf{Ninth Author\textsuperscript{1}},
%  \textbf{Tenth Author\textsuperscript{1}},
%  \textbf{Eleventh E. Author\textsuperscript{1,2,3,4,5}},
%  \textbf{Twelfth Author\textsuperscript{1}},
% \\
%  \textbf{Thirteenth Author\textsuperscript{3}},
%  \textbf{Fourteenth F. Author\textsuperscript{2,4}},
%  \textbf{Fifteenth Author\textsuperscript{1}},
%  \textbf{Sixteenth Author\textsuperscript{1}},
% \\
%  \textbf{Seventeenth S. Author\textsuperscript{4,5}},
%  \textbf{Eighteenth Author\textsuperscript{3,4}},
%  \textbf{Nineteenth N. Author\textsuperscript{2,5}},
%  \textbf{Twentieth Author\textsuperscript{1}}
% \\
\\
 \textsuperscript{1}School of Computer Science and Engineering, Key Laboratory of New Generation Artificial Intelligence \\Technology and Its Interdisciplinary Applications, Southeast University, Ministry of Education, China
 \\
\textsuperscript{2}Department of Informatics, King's College London, UK\\
 \textsuperscript{3}	School of Electronic Information and Electrical Engineering, Shanghai Jiao Tong University, China\\
 \textsuperscript{4}School of Electrical and Computer Engineering, Carnegie Mellon University, USA\\
 \textsuperscript{5}The Alan Turing Insitute, UK\\
}

\begin{document}
\maketitle
\begin{abstract}
%Medical question answering demands substantial access to specialized conceptual knowledge. The current paradigm, Retrieval-Augmented Generation (RAG), acquires medical knowledge through large-scale corpus retrieval and transfers this knowledge to a general-purpose large language model (LLM) for generating answers. 
Medical question answering requires extensive access to specialized \textit{conceptual knowledge}. The current paradigm, Retrieval-Augmented Generation (RAG), acquires expertise medical knowledge through large-scale corpus retrieval and uses this knowledge to guide a general-purpose large language model (LLM) for generating answers. 
%However, existing retrieval approaches lack dedicated attention to and consideration of factual knowledge, limiting the relevance and effectiveness of conceptual knowledge retrieval and hindering applications in real-world scenarios such as clinical decision-making based on Electronic Health Records (EHRs).
% However, existing retrieval approaches lack dedicated consideration of \textit{factual knowledge}, limiting the relevance of retrieved conceptual knowledge and hindering applications in real-world scenarios such as clinical decision-making based on Electronic Health Records (EHRs).
However, existing retrieval approaches often overlook the importance of \textit{factual knowledge}, which limits the relevance of retrieved conceptual knowledge and restricts its applicability in real-world scenarios, such as clinical decision-making based on Electronic Health Records (EHRs).
%This paper presents RGAR, a recurrence generation-augmented retrieval framework that retrieve relevant factual knowledge and conceptual knowledge from dual ends, allowing them to interact and update one another.
This paper introduces RGAR, a recurrence generation-augmented retrieval framework that retrieves both relevant \textit{factual} and \textit{conceptual} knowledge from dual sources (i.e., EHRs and the corpus), allowing them to interact and refine each another.
% This paper presents RGAR, a recurrence generation-augmented retrieval framework that leverages retrieved medical knowledge to continuously extract question-relevant factual knowledge from queries and transform it into retrieval-optimized representations, ultimately facilitating more relevant medical knowledge retrieval. 
% Through extensive evaluation across three factual-aware medical question answering benchmarks, RGAR sets a new state-of-the-art performance among medical RAG systems.
Through extensive evaluation across three factual-aware medical question answering benchmarks, RGAR establishes a new state-of-the-art performance among medical RAG systems.
Notably, the Llama-3.1-8B-Instruct model with RGAR surpasses the considerably larger, RAG-enhanced GPT-3.5. 
%Our findings reveal that extracting factual knowledge significantly enhances system performance, consistently yielding improved retrieval accuracy.
Our findings demonstrate the benefit of extracting factual knowledge for retrieval, which consistently yields improved generation quality.
\end{abstract}

\section{Introduction}
Large Language Models (LLMs) have demonstrated remarkable capabilities in general question answering (QA) tasks, achieving impressive performance across diverse scenarios \cite{achiam2023gpt}. However, when facing domain-specific questions that require specialized expertise, from medical diagnosis \cite{jin2021disease} to legal charge prediction \cite{wei-etal-2024-mud}, these models face significant challenges, often generating unreliable conclusions due to both hallucinations \cite{ji2023survey} and potentially stale knowledge embedded in their parameters \cite{wang2024knowledge}. % These issues can be especially dangerous in high-stakes domains such as healthcare, where incorrect information could lead to serious consequences \cite{tian2024opportunities}.

% Addressing the task of question answering (QA) presents significant challenges, as it demands intricate reasoning involving both the explicit constraints articulated in the questions and the implicit domain knowledge \cite{frisoni-etal-2024-generate}. Such difficult tasks effectively reflect the complexities of real-life scenarios and are prevalent in fields requiring specialized knowledge, ranging from medical diagnostics \cite{jin2021disease} to predictions of criminal charges \cite{wei-etal-2024-mud}. 

% Open-domain question answering (OpenQA) \cite{chen-etal-2017-reading} aims to deal with real-world queries without relying on expert knowledge from any predefined domain. This approach effectively reflects the complexities of real-life scenarios, where each potential question lacks a pre-labeled body of text containing the answer. As one of the most challenging forms of question answering, OpenQA has been widely applied in professional knowledge reasoning tasks, ranging from medical diagnostics \cite{jin2021disease} to criminal charge predictions \cite{wei-etal-2024-mud}.

\begin{figure}
    \centering
    \includegraphics[width=\linewidth]{small.pdf}
    \caption{a) Medical AI Systems from the Perspective of Bloom's Taxonomy. b) Two Types of Medical Question Answering Tasks.}
    \label{fig:enter-label}
\end{figure}

\textbf{Retrieval-Augmented Generation (RAG)} \cite{lewis2020retrieval} has emerged as a promising approach to address these challenges by leveraging extensive, trustworthy knowledge bases to support LLM reasoning. The effectiveness of this approach, however, heavily depends on the relevance of retrieved documents. Recent advances, such as \textbf{Generation-Augmented Retrieval (GAR)} \cite{mao-etal-2021-generation}, focus on enhancing retrieval performance by generating relevant context for query expansion.

% By retrieving relevant document chunks from extensive, trustworthy knowledge bases to assist LLMs, \textbf{Retrieval-Augmented Generation (RAG)} \cite{lewis2020retrieval} has shown promise in tackling the above challenges. However, the relevance of the retrieved documents plays a crucial role in the model's reasoning capability. Recent approaches, such as \textbf{Generation-Augmented Retrieval (GAR)} \cite{mao-etal-2021-generation}, 
% focus on query formulation and propose query expansion methods to enhance generation performance.
% focus on how query formulation influences document relevance, proposing methods expending queries into multiple variations to enhance generation performance.

In the medical domain, current RAG approaches concatenate all available contextual information from a given example into a single basic query for retrieval, aiming to provide comprehensive context for model reasoning \cite{xiong-etal-2024-benchmarking}. While this method has demonstrated substantial improvements on early \textit{knowledge-intensive} medical QA datasets such as PubMedQA \cite{jin-etal-2019-pubmedqa}, its limitations have become increasingly apparent with the emergence of EHR-integrated datasets that better reflect real-world clinical practices \cite{kweon2024ehrnoteqa}. Electronic Health Records (EHRs) typically contain extensive patient data, including comprehensive diagnostic test results and medical histories \cite{pang2021cehr}. However, for any specific medical query, only a small subset of this information is typically relevant, and retrieval performance can be significantly degraded when queries are diluted with extraneous EHR content \cite{johnson2023mimic, lovon-melgarejo-etal-2024-revisiting}.

% The current RAG approach to solving medical problems concatenates all contextual information from a given example into a basic query for retrieval, aiming to capture the most comprehensive content for model reasoning \cite{xiong-etal-2024-benchmarking}. This method has achieved significant improvements on early knowledge-intensive medical QA datasets like PubMedQA \cite{jin-etal-2019-pubmedqa}.
% However, the emergence of EHR-integrated datasets, which better align with real-world clinical practices \cite{kweon2024ehrnoteqa}, reveals critical limitations of this paradigm. Electronic Health Records (EHRs) typically contain extensive patient data, including all diagnostic test results and medical histories \cite{pang2021cehr}, yet only a small fraction of this data is relevant to a specific question. Retrieval performance can be impaired when queries contain lengthy, irrelevant texts from EHRs \cite{johnson2023mimic, lovon-melgarejo-etal-2024-revisiting}.

% We highlight that current \textit{retrieval methods} often fail to adequately consider factual information. Real-world medical scenarios are inherently \textbf{factual-aware}, emphasizing the importance of factual information, such as EHRs, which are crucial for providing personalized and accurate medical advice for a specific query.

We highlight that current \textit{retrieval methods} often fail to adequately consider \textit{factual information} in real-world medical scenarios. Crucially, even when applying query expansion with GAR, the persistent oversight of factual information fundamentally limits their ability to retrieve real relevant documents.

% In \textbf{real-world} medical QA scenarios, it is crucial to consider not only \textit{professionally knowledge} but also \textit{factual-aware information}. \textbf{Factual-aware information} consists of essential factual content, such as electronic health records (EHRs), which are essential for delivering personalized and accurate medical advice. Early medical QA datasets primarily focused on professional knowledge \cite{jin-etal-2019-pubmedqa}, while more recent ones have recognized the significance of factual information, designing resources that better align with real-world clinical practice \cite{kweon2024ehrnoteqa}.


% However, current \textit{retrieval methods} often fail to adequately account for factual information. Existing studies simply concatenate EHR data with the query, assuming that retrieving relevant professional knowledge is sufficient for LLMs to solve the corresponding problem \cite{xiong-etal-2024-benchmarking}. This approach relies on the oversimplified assumption that all EHR content is relevant to the specific question, which is rarely the case in practice. EHRs typically contain comprehensive patient data, including all diagnostic test results and medical histories \cite{pang2021cehr}, of which only a small fraction is relevant to any specific question. Furthermore, the verbosity of EHRs can hinder retrieval performance when irrelevant, lengthy texts are included \cite{johnson2023mimic, lovon-melgarejo-etal-2024-revisiting}.

Inspired by \textbf{Bloom's taxonomy} \cite{forehand2010bloom,markus2001toward}, we categorize the knowledge required to address real-world medical QA problems into four types: \textit{Factual Knowledge}, \textit{Conceptual Knowledge}, \textit{Procedural Knowledge}, and \textit{Metacognitive Knowledge}.
The latter two represent higher-order knowledge typically embedded within advanced RAG systems. Specifically, \textit{Procedural Knowledge} refers to the processes and strategies required to solve problems, such as problem decomposition and retrieval \cite{wei2022chain, zhou2023leasttomost}, while \textit{Metacognitive Knowledge} pertains to an LLM's ability to assess whether it has sufficient knowledge or evidence to perform effective reasoning \cite{kim-etal-2023-tree, wang-etal-2023-self-knowledge}.

\textit{Factual Knowledge} and \textit{Conceptual Knowledge} require retrieval from large databases containing substantial amounts of irrelevant content, corresponding to the EHRs of patients and medical corpora in answering medical questions. Unfortunately, current RAG systems do not differentiate between these types of \textit{retrieval targets}, overlooking the necessity of retrieval from EHRs.

% \textit{Factual Knowledge} and \textit{Conceptual Knowledge} involve processing information from extensive databases that contain significant amounts of irrelevant content, corresponding to the EHRs of patients and the corpora of medical knowledge, to answer medical questions. Unfortunately, current RAG systems do not distinguish between these types of \textit{retrieval targets}, overlooking the necessity of retrieving information specifically from EHRs.

To overcome this limitation, we propose \textbf{RGAR}, a system designed to simultaneously retrieves \textit{Factual Knowledge} and \textit{Conceptual Knowledge} through a recurrent query generation and interaction mechanism. This approach iteratively refines queries to enhance the relevance of retrieved professional and factual knowledge, thereby improving performance on \textit{knowledge-intensive} and \textit{factual-aware} medical QA tasks.

Our key contributions are listed as follows:
\begin{itemize}
\item We are the first to analyze RAG systems through the lens of Bloom's taxonomy, addressing the current underrepresentation of \textit{Factual Knowledge} in existing frameworks.
\item We introduce RGAR, a dual-end retrieval system that facilitates recurrent interactions between \textit{Factual} and \textit{Conceptual} Knowledge, bridging the gap between LLMs and real-world clinical applications.
\item Through extensive experiments on three medical QA datasets involving \textit{Factual Knowledge}, we demonstrate that RGAR achieves superior average performance compared to state-of-the-art (SOTA) methods, enabling Llama-3.1-8B-Instruct model to outperform the considerably larger RAG-enhanced GPT-3.5-turbo.
\end{itemize}

\section{Related Work}
\textbf{RAG Systems. } RAG systems are characterized as a "Retrieve-then-Read" framework \cite{gao2023retrieval}. The development of Naive RAG has primarily focused on retriever optimization, evolving from discrete retrievers such as BM25 \cite{friedman1977algorithm} to more sophisticated and domain-specific dense retrievers, including DPR \cite{karpukhin-etal-2020-dense} and MedCPT \cite{jin2023medcpt}, which demonstrate superior performance.

In recent years, numerous advanced RAG systems have emerged. Advanced RAG systems focus on designing multi-round retrieval structures, including iterative retrieval \cite{sun2019pullnet}, recursive retrieval \cite{sarthi2024raptor}, and adaptive retrieval \cite{jeong-etal-2024-adaptive}. A notable work in medical QA is MedRAG \cite{xiong-etal-2024-benchmarking}, which analyzes retrievers, corpora, and LLMs, offering practical guidelines. Follow-up work, $i$-MedRAG \cite{xiong2024improving}, improved performance through multi-round decomposition and iteration, albeit with significant computational costs.

These approaches focus solely on optimizing the retrieval process, overlooking the retrievability of \textit{factual knowledge}. In contrast, RGAR introduces a recurrent structure, enabling continuous query optimization through dual-end retrieval and extraction from EHRs and professional knowledge corpora, thereby enhancing access to both knowledge types.

\textbf{Query Optimization. } As the core interface in human-AI interaction, query optimization (also known as prompt optimization) is the key to improving AI system performance. It is widely applied in tasks such as text-to-image generation \cite{liu2022compositional, wu-etal-2024-universal} and code generation \cite{nazzal2024promsec}.

In the era of large language models, query optimization for retrieval tasks has gained increasing attention. Representative work includes GAR \cite{mao-etal-2021-generation}, which improves retrieval performance through query expansion using fine-tuned BERT models \cite{devlin-etal-2019-bert}. GENREAD \cite{yu2023generate} further explored whether LLM-generated contexts could replace retrieved professional documents as reasoning evidence. MedGENIE \cite{frisoni-etal-2024-generate} extended this approach to medical QA.

Another line of work focuses on query transformation and decomposition, breaking down original queries into multiple sub-queries tailored to specific tasks, enhancing retrieval alignment with model needs \cite{dhuliawala2023chain}. Subsequent work has reinforced the effectiveness of query decomposition through fine-tuning \cite{ma2023query}.

Using expanded queries directly as reasoning evidence lacks the transparency of RAG, as RAG relies on retrievable documents that provide traceable and trustworthy reasoning, which is crucial in the medical field.
Besides, the effectiveness of query expansion and query decomposition approaches is heavily dependent on fine-tuning LLMs, which limits scalability.

%Additionally, domain-specific LLMs that generate reasoning evidence face challenges in knowledge updating \cite{wang2024knowledge}, making RAG a more robust solution.

In contrast, our work focuses on query optimization without fine-tuning LLMs. Specifically, retrieval from EHRs can be seen as query filtering that eliminates irrelevant information, thereby obtaining pertinent \textit{factual knowledge}. Extracting factual knowledge enhances the effectiveness of retrieval from the corpus.

%\subsection{Medical Question Answering}

%Recent medical QA datasets such as MMLU-Med (Measuring Massive Multitask Language Understanding), PubMedQA (PubMedQA: A Dataset for Biomedical Research Question Answering), and BioASQ-Y/N (An Overview of the BIOASQ Large-Scale Biomedical Semantic Indexing and Question Answering Competition) require models to master vast amounts of medical knowledge not provided within the question context, exemplifying the challenges of open-domain question answering. The MIRAGE benchmark adopts a Question-Only Retrieval (QOR) paradigm, aligning with real-world cases of medical QA, where answer options should not be presented as input during retrieval.

%To better approximate clinical diagnosis scenarios, some datasets, such as MedQA-US (What Disease Does This Patient Have? A Large-Scale Open Domain Question Answering Dataset from Medical Exams) and MedMCQA (MedMCQA: A Large-Scale Multi-Subject Multi-Choice Dataset for Medical Domain Question Answering), incorporate specific patient cases within their questions, demanding that models apply medical knowledge to resolve practical issues. This represents a simplified form of factual-aware medical question answering. The latest dataset, EHRNoteQA, utilizes original EHR data from MIMIC-IV, necessitating that models accurately identify which factual information within the EHR aligns with the posed question and leverage specialized knowledge to formulate answers.

%Our approach adopts the MIRAGE benchmark's framework, focusing on enhancing models' capabilities in factual-aware medical question answering.

\begin{figure*}
    \centering
    \includegraphics[width=\linewidth]{pipeline.pdf}
    \caption{The Overall Framework of RGAR. a) The Recurrence Pipeline in § \ref{sec:pipeline}; b) Conceptual Knowledge Retrieval in § \ref{sec:Train-free}; c) Factual Knowledge Extraction in § \ref{sec:Extraction}; d) Response Template in § \ref{sec:pipeline}.}
    \label{fig:pipeline}
\end{figure*}

\section{Methodology}
% 开头整段都要改
% 已经改过了
In this section, we introduce RGAR framework, as illustrated in Figure \ref{fig:pipeline}. It begins by prompting a general-purpose LLM to generate multiple queries from an initial basic query. These multiple queries are then used to \textbf{retrieve conceptual knowledge} from the corpus (§ \ref{sec:Train-free}). Then retrieved conceptual knowledge is subsequently used to \textbf{extract factual knowledge} from the electronic health records (EHRs) and transform it into retrieval-optimized representations (§ \ref{sec:Extraction}). The \textbf{recurrence pipeline} continuously updates the basic query and iteratively executes the two aforementioned components. This process optimizes the retrieved results, ultimately improving the quality of responses.(§ \ref{sec:pipeline}).
\subsection{Task Formulation}
In \textit{factual-aware} medical QA, each data sample comprises the following elements: a patient's natural language query $\mathcal{Q}$, the electronic health record (EHR) as factual knowledge $\mathcal{F}$, and a set of candidate answer $\mathcal{A} = \{a_1, ..., a_{|\mathcal{A}|}\}$. The overall goal is to identify the correct answer $\hat{a}$ from $\mathcal{A}$.

A \textit{non-retrieval} approach directly prompts an LLM to act as a \textbf{reader}, processing the entire context and generating an answer, formulated as:

\begin{equation}
\hat{a}=\textbf{LLM}(\mathcal{F},\mathcal{Q},\mathcal{A}|\mathcal{T}_r)
\end{equation}

where $\mathcal{T}_r$ is the prompts. However, this approach relies exclusively on the conceptual knowledge encoded within LLM, without leveraging external, trustworthy medical knowledge sources.

To overcome this limitation, recent studies have explored \textit{retrieval-based} approaches, which enhance the model’s knowledge by retrieving a specified number $N$ of chunks, denoted as $\mathcal{C} = \{c_1, ..., c_N\}$, from a chunked corpus (knowledge base) $\mathcal{K}$. This answering process is expressed as:

\begin{equation}
\hat{a}=\textbf{LLM}(\mathcal{F},\mathcal{Q},\mathcal{A},\mathcal{C}|\mathcal{T}_r).
\label{eq:retrieval-augmented}
\end{equation}

\subsection{Conceptual Knowledge Retrieval (CKR)}
\label{sec:Train-free}
To maintain consistency with the \textit{option-free retrieval approach} proposed by \cite{xiong-etal-2024-benchmarking}, we do not incorporate the answer options $\mathcal{A}$ during retrieval. This design is in line with real-world medical quality assurance scenarios, where answer choices are typically not available in advance.

Following their method, we construct the \textbf{basic query} by concatenating the EHR and the patient's query, formally defined as $q_b = \mathcal{Q} \oplus \mathcal{F}$, where $\oplus$ denotes text concatenation.

Traditional dense retrievers, such as Dense Passage Retrieval (DPR) \cite{karpukhin-etal-2020-dense}, identify the top-$N$ relevant chunks $C$ from the knowledge base $\mathcal{K}$ by computing similarity scores using an encoder $E$:

\begin{equation}
\begin{split}
    &\text{sim}(q_b, c_i) = E(q_b)^\top E(c_i), \\
    &\mathcal{C} = \text{top-}N(\{\text{sim}(q_b, c_i)\}).
\end{split}
\end{equation}


Vanilla GAR \cite{mao-etal-2021-generation} expands $q_b$ using a fine-tuned BERT \cite{devlin-etal-2019-bert} to produce three types of content that enhance retrieval: potential answers $q_e^a$, contexts $q_e^c$, and titles $q_e^t$.
With the growing zero-shot generation capabilities of LLMs \cite{kojima2022large}, a common practice is to prompt LLMs to serve as train-free query \textbf{generators}, producing expanded content $\tilde{q}_e$ using prompt templates $\mathcal{T}_g$ \cite{frisoni-etal-2024-generate}. The three types of content generation process can be formulated as:

\begin{equation}
\label{eq:query-generation}
\begin{array}{l}
\tilde{q}_e^a = \textbf{LLM}(q_b |\mathcal{T}^a_g), \\[1ex]
\tilde{q}_e^c = \textbf{LLM}(q_b |\mathcal{T}^c_g), \\[1ex]
\tilde{q}_e^t = \textbf{LLM}(q_b |\mathcal{T}^t_g).
\end{array}
\end{equation}

% 这样改了一下,不知道合不合适
%非常合适
%The final score $Sc$ to get retrieved $\mathcal{C}$ is then obtained by normalizing and averaging the similarities of these expanded queries:
The final score $Sc$ for retrieving $\mathcal{C}$ is then computed by normalizing and averaging the similarities of these expanded queries:
\begin{equation}
\label{eq:normalized-retrieval-score}
\text{Sc}(c_i) = \sum_{\tilde{q}_e \in \{\tilde{q}_e^a, \tilde{q}_e^c, \tilde{q}_e^t\}} \frac{\exp(\text{sim}(\tilde{q}_e, c_i))}{\sum_{c_j} \exp(\text{sim}(\tilde{q}_e, c_j))}.
\end{equation}


\subsection{Factual Knowledge Extraction (FKE)}
\label{sec:Extraction}

In EHR, only a small portion of necessary information constitutes problem-relevant factual knowledge \cite{d2004evaluation}. Direct input of lengthy EHR content containing substantial irrelevant information into dense retrievers can degrade retrieval performance \cite{ren-etal-2023-thorough}. While a straightforward approach would be to retrieve EHR content based on question $\mathcal{Q}$ \cite{factual_aware}, this fails to fully utilize conceptual knowledge obtained from previous Conceptual Knowledge Retrieval Stage. Furthermore, the necessary chunking of EHR for retrieval introduces content discontinuity \cite{luo-etal-2024-landmark}.

Given that EHRs more closely resemble long passages from the Needle in a Haystack task \cite{kamradt2024needle} rather than necessarily chunked corpus, and inspired by large language models' capability to precisely locate answer spans in reading comprehension tasks \cite{cheng2024adapting}, we propose leveraging LLMs for text span tasks \cite{rajpurkar-etal-2016-squad} on EHR to filter relevant factual knowledge efficiently and effectively using conceptual knowledge. We define this filtered factual knowledge as $\mathcal{F}_s$, with prompts $\mathcal{T}_s$, expressed as:
\begin{equation}
    \mathcal{F}_s=\textbf{LLM}(\mathcal{F},\mathcal{Q},\mathcal{C}|\mathcal{T}_s).  
\end{equation}


In addition, EHRs often contain numerical report results \cite{lovon-melgarejo-etal-2024-revisiting} that require conceptual knowledge to interpret their significance. Furthermore, medical QA involves multi-hop questions \cite{pal2022medmcqa}, where retrieved conceptual knowledge can generate explainable new factual knowledge conducive to reasoning. Drawing from LLM zero-shot summarization prompting strategies \cite{wu2025towards}, we analyze and summarize the filtered EHR $\mathcal{F}_s$ with prompts $\mathcal{T}_e$, yielding an enriched representation $\mathcal{F}_e$:
\begin{equation}
    \mathcal{F}_e=\textbf{LLM}(\mathcal{F}_s,\mathcal{Q},\mathcal{C}|\mathcal{T}_e).  
\end{equation}



This process, which we refer to as the LLM \textbf{Extractor}, completes the extraction of original EHR information. In practice, RGAR implements these two phases using single-stage prompting to reduce time overhead. 

% This new reliable factual knowledge enables deeper reasoning and analysis in the GAR, generating multi-queries for multi-hop knowledge retrieval.
% The length and complexity of EHR documents often pose significant challenges when it comes to efficiently extracting relevant information \cite{d2004evaluation}. Feeding lengthy, question-irrelevant EHR content directly into dense retrievers can degrade retrieval performance. A straightforward solution is to segment the content into chunks and use $\mathcal{Q}$ to retrieve necessary information from these chunks. However, dense retrievers primarily measure textual similarity, and the query often lacks the direct semantic links needed to connect $\mathcal{Q}$ with the underlying medical concepts in the original EHR content $\mathcal{F}$. This highlights the importance of the retrieved conceptual knowledge $C = \{c_1, c_2, \dots, c_N\}$, which serves as a vital bridge between $\mathcal{Q}$ and $\mathcal{F}$.

% Consider two straightforward query construction strategies:

% 1. Single unified query:  
%    \begin{equation}
%    \label{eq:single-query}
%    q_{\text{all}} = \mathcal{Q} \oplus c_1 \oplus c_2 \oplus \cdots \oplus c_N
%    \end{equation}  
%    In this approach, all conceptual knowledge is concatenated directly to the query. However, since each \( c_i \) corresponds to a distinct medical concept, the resulting embeddings blend multiple types of information, making it difficult for the retriever to focus on the single relevant signal.

% 2. Separate queries for each concept \( c_i \):  
%    \begin{equation}
%    \label{eq:separate-query}
%    q_{c_i} = \mathcal{Q} \oplus c_i
%    \end{equation}  
%    Here, each \( c_i \) is used to create a separate query. Retrieval is performed for each query independently, and the final result is obtained by averaging the normalized similarity scores of all \( q_{c_i} \). While this approach more precisely captures the relationship between each concept and the corresponding EHR segments, it introduces substantial computational overhead, requiring \( N \) independent retrieval operations. As such, it is impractical for real-world deployment scenarios.

% Inspired by the ability of large language models (LLMs) to locate answers within lengthy text passages \cite{cheng2024adapting} and their role in relevance assessment in RAG systems \cite{es-etal-2024-ragas}, we propose leveraging LLMs’ conditional generation capabilities to \textit{approximate} the retrieval task. Specifically, we design a set of retrieval prompts $\mathcal{T}^r$ that guide the LLM to produce an output distribution approximating the results of traditional dense retrievers:  
% \begin{equation}
% \label{eq:conceptual-retrieval}
% p_\theta(\mathcal{F}_r \mid \mathcal{F}, \mathcal{Q}, \mathcal{T}^r) \approx p_r(\mathcal{F}_r \mid \mathcal{Q}, \mathcal{F}),
% \end{equation}
% where $\mathcal{F}_r \subseteq \mathcal{F}$ represents EHR fragments relevant to $\mathcal{Q}$, and $\mathcal{T}^r$ represents prompts specifically designed for the retrieval task.

% Since large language models possess strong contextual retrieval and understanding capabilities, they can mitigate the information loss that may occur when embedding all $C$ into a single query. Thus, we approximate:  
% \begin{equation}
% \begin{split}
% p_\theta(\mathcal{F}_r \mid \mathcal{F}, Q, C, \mathcal{T}^r) &\approx \frac{1}{N}\sum_{i=1}^{N} \\
% &p_\theta\Big(\mathcal{F}_r \,\big|\, \mathcal{F}, Q \oplus c_i, \mathcal{T}^r\Big).
% \end{split}
% \end{equation}

% In other words, the performance of a single concatenated query approximately matches the normalized average results of individual queries $q_{c_i}=Q\oplus c_i$. The accuracy of this approximation depends on $p_\theta$.

% EHR documents often contain numerous numerical test results, which can be difficult for retrievers to match conceptually. For example, “Platelet count 14,200/mm³” might correspond to “low platelet count” in medical literature. To address this, we use the concept knowledge $C$ as a supplementary condition and employ specialized rewriting prompts $\mathcal{T}^{\text{rew}}$ to guide the LLM in rewriting retrieved EHR fragments. The process is formalized as:
% \begin{equation}
% p_\theta(\mathcal{F}_{re} \mid \mathcal{F}_r, \mathcal{Q}, \mathcal{T}^{\text{rew}})
% \end{equation}
% The model is expected to achieve the following objectives under the guidance of the prompt template and the provided concept knowledge $C$:  

% \textbf{First}, normalization of numerical information: Transform numerical expressions into standardized text descriptions. For instance, rewriting “Platelet count 14,200/mm³” into “low platelet count” facilitates retrieving truly relevant articles.  
% \textbf{Second}, information fusion: When certain indicators are scattered across multiple document fragments, the model can integrate them into a more comprehensive interpretation. For example, if one fragment mentions “elevated white blood cell count” and another mentions “decreased platelet count,” the model might generate “the patient shows signs of an inflammatory response accompanied by thrombocytopenia.” This provides a more complete context for generating question-relevant contexts or answers in the GAR process.


\subsection{The Recurrence Pipeline and Response}
\label{sec:pipeline}
% The extracted EHR information serves as question-relevant Factual knowledge $\mathcal{F}_e$, updating the Basic query $q_b$ through $\mathcal{Q} \oplus \mathcal{F}_e$.
% Building on the $\mathcal{F}_r$ obtained in the previous stage, we update the basic query $q_b = \mathcal{Q} \oplus \mathcal{F}_r$. \textit{Training-free Generation-augmented Retrieval} and \textit{Conditional Generating Retrieved and Rewritten EHR} stages are then iteratively performed until a predetermined number of iterations is reached. Ultimately, this iterative optimization yields the final retrieved conceptual knowledge $C^*$. 

Building on the \(\mathcal{F}_e\), we \textbf{update} the basic query for Conceptual Knowledge Retrieval as \(q_b = \mathcal{Q} \oplus \mathcal{F}_e\). This establishes a \textbf{recurrence interaction} between factual and conceptual knowledge, guiding next retrieval toward more relevant content. Iterative execution enhances the stability of both retrieval and extraction. The entire pipeline recurs for a predefined number of iterations, ultimately yielding the final retrieved conceptual knowledge $\mathcal{C}^*$.

% 这里改了一下
% During the response phase, we adhere to the approach outlined in Equation \ref{eq:retrieval-augmented} to produce answers. In particular, the $\mathcal{F}_e$ are confined to the retrieval phase and are not utilized in the response phase. The only difference lies in the retrieved chunks, which allows us to clearly demonstrate the impact of retrieval quality on the response phase.
During the response phase, we follow the approach in Equation \ref{eq:retrieval-augmented} to generate answers. Notably, the $\mathcal{F}_e$ are restricted to the retrieval phase and are not used in the response phase. The sole difference lies in the retrieved chunks, highlighting the impact of retrieval quality on the responses.

\section{Experiments}
\subsection{Experimental Setup}
\subsubsection{Benchmark Datasets}

We evaluated RGAR on three \textit{factual-aware} medical QA benchmarks featuring multiple-choice questions that require human-level reading comprehension and expert reasoning to analyze patients' clinical conditions.

% We evaluated RGAR on three \textit{factual-aware} medical QA benchmarks featuring multiple-choice questions that require multi-hop reasoning and human-level reading comprehension.  

% We evaluated RGAR on three \textit{factual-aware} medical QA benchmarks. Table \ref{tab:qa_benchmarks} illustrates the statistical differences between these factual-aware questions and traditional question types. All three datasets are multiple-choice OpenQA benchmarks that require multi-hop reasoning and human-level reading comprehension capabilities.


\textbf{MedQA-USMLE} \cite{jin2021disease} and \textbf{MedMCQA} \cite{pal2022medmcqa} consist of questions derived from professional medical exams, evaluating specialized expertise such as disease symptom diagnosis and medication dosage requirements. The problems frequently involve patient histories, vital signs (e.g., blood pressure, temperature), and final diagnostic evaluations (e.g., CT scans), making it necessary to retrieve relevant medical knowledge tailored to the patient’s specific circumstances. However, due to their exam-oriented format, the provided information has already been filtered, reducing the difficulty of extracting factual knowledge from EHR.

\textbf{EHRNoteQA} \cite{kweon2024ehrnoteqa} is a recently introduced benchmark that provides authentic, complex EHR data derived from MIMIC-IV \cite{johnson2023mimic}. This dataset encompasses a wide range of topics and demands that models emulate genuine clinical consultations, ultimately generating accurate discharge recommendations. Consequently, EHRNoteQA challenges models to identify which \textit{factual details} within the EHR are relevant to the questions at hand and apply domain-specific knowledge to address them.

\begin{table}[htbp]
  \centering
  \caption{Medical QA Benchmark Statistics.}
  \resizebox{\linewidth}{!}{ % 让表格适应页面宽度
    \begin{tabular}{lccc}
      \toprule
      Benchmarks & Max. Len & Avg. Len & Min. Len \\
            \midrule
      \rowcolor{gray!20} \multicolumn{4}{c}{Non-EHR QA Benchmarks} \\
      \midrule
        BioASQ-Y/N & 52 & 17 & 9  \\
      PubMedQA & 57 & 23 & 10  \\
      \midrule
      \rowcolor{gray!20} \multicolumn{4}{c}{EHR QA Benchmarks} \\
      \midrule
      MedMCQA & 207 & 41 & 11  \\
      MedQA-USMLE & 872 & 197 & 50  \\
      EHRNoteQA & 5782 & 3061 & 667  \\

      \bottomrule
    \end{tabular}
  }
  \label{tab:qa_benchmarks}
\end{table}

Table \ref{tab:qa_benchmarks} highlights that the chosen datasets, which include EHR information, tend to have significantly \textbf{longer} content compared to datasets without EHRs. Notably, the EHRNoteQA dataset has a maximum length exceeding 4,000 tokens. This raises concerns about the reasonableness of directly employing these EHRs for retrieval.

\begin{table*}[htbp]
  \centering
  \caption{Comparison of RGAR with Other Methods on Three Factual-Aware Datasets. $\Delta$ Indicates Improvement Over Custom, \textbf{Bold} Represents the Best, and \underline{Underline} Indicates the Second-Best.}
  \resizebox{\linewidth}{!}{%
    \begin{tabular}{llcccccc|cc}
      \toprule
      \multicolumn{2}{c}{\multirow{2}{*}{Method}} & \multicolumn{2}{c}{MedQA-USMLE (\# 1273)} & \multicolumn{2}{c}{MedMCQA(\# 4183)} & \multicolumn{2}{c|}{EHRNoteQA(\# 962)} & \multicolumn{2}{c}{Average(↓)} \\
      \cmidrule(lr){3-4} \cmidrule(lr){5-6} \cmidrule(lr){7-8} \cmidrule(lr){9-10}
      \multicolumn{2}{c}{} & Acc. & $\Delta$ & Acc. & $\Delta$ & Acc. & $\Delta$ & Acc. & $\Delta$ \\
      \midrule
      \multirow{2}{*}{w/o Retrieval} & Custom  & 50.20 & 0.00  & 50.01 & 0.00  & 47.19 & 0.00  & 49.13 & 0.00  \\
                               & CoT     & 51.45 & 1.25  & 44.53 & -5.48 & 62.89 & 15.70 & 52.96 & 3.82  \\
      \midrule
      \multirow{5}{*}{w/ Retrieval}  & RAG     & 53.50 & 3.30  & \underline{50.54} & \underline{0.53}  & 61.12 & 13.93 & 55.05 & 5.92  \\
                               & MedRAG  & 50.27 & 0.07  & 47.53 & -2.48 & 70.58 & 23.39 & 56.13 & 6.99  \\
                               & GAR     & \underline{57.97} & \underline{7.77}  & 50.42 & 0.41  & 65.48 & 18.29 & 57.96 & 8.82  \\
                               & $i$-MedRAG & 56.24 & 6.04  & 44.94 & -5.07 & \textbf{74.22} & \textbf{27.03} & \underline{58.47} & \underline{9.33}  \\
                               & RGAR    & \textbf{58.83} & \textbf{8.63}  & \textbf{51.02} & \textbf{1.01}  & \underline{73.28} & \underline{26.09} & \textbf{61.04} & \textbf{11.91} \\
      \bottomrule
    \end{tabular}%
  }
  \label{tab:mian_results}
\end{table*}


% In line with MIRAGE \cite{xiong-etal-2024-benchmarking}, we implement the following evaluation framework:
% \begin{itemize}
%     \item \textbf{Option-Free Retrieval:} As mentioned in § \ref{sec:Train-free}, to replicate real-world medical QA conditions, no answer options are provided as input during retrieval.
%     \item \textbf{Zero-Shot Learning:} Given that real-world medical questions often lack similar exemplars, our benchmark evaluates RAG systems in a zero-shot setting, without in-context few-shot learning.
%     \item \textbf{Metrics:} We use Accuracy—the proportion of questions correctly answered—as the main evaluation metric across all benchmarks. We extract model outputs through regular expression matching applied to complete generated answers \cite{wang-etal-2024-answer-c}.
% \end{itemize}

\subsubsection{Retriever and Corpus}
To ensure a fair comparison, we adopt the same retriever, corpus, and parameter settings as previous work \cite{xiong-etal-2024-benchmarking}. We use MedCPT \cite{jin2023medcpt}, a dense retriever specialized for the biomedical domain, configured to retrieve 32 chunks by default. For the corpus, we employ the Textbooks dataset \cite{jin-etal-2019-pubmedqa}, a lightweight collection of 125.8k chunks derived from medical textbooks, with an average length of 182 tokens.
% To ensure a fair comparison, we follow MIRAGE \cite{xiong-etal-2024-benchmarking} in terms of the retrievers, corpus, and parameter settings. We employ MedCPT \cite{jin2023medcpt}, a dense retriever tailored to the biomedical domain, configured to retrieve 32 chunks by default. For the corpus, we use the Textbooks dataset \cite{jin-etal-2019-pubmedqa}, a lightweight collection derived from medical textbooks, consisting of 125.8k chunks with an average length of 182 tokens.




\subsubsection{LLMs and Baselines}
We focus on the effect of RGAR on general-purpose LLMs without domain-specific knowledge. Therefore, we exclude LLMs fine-tuned on the medical domain, such as PMC-Llama \cite{wu2024pmc}. %We focus on models with fewer than 8 billion parameters. 
Our primary experiments utilize Llama-3.2-3B-Instruct, while ablation studies include a range of models from the Llama-3.1/3.2 \cite{dubey2024llama} and Qwen-2.5 \cite{yang2024qwen2} families, ranging from 1.5B to 8B parameters. All selected models feature a context length of approximately 128K tokens.
Temperatures are set to zero to ensure reproducibility through greedy decoding. % To mitigate repetitive generation in smaller models, we use a repetition penalty of 1.2 and limit the maximum generation length to 8K tokens.

For \textit{non-retrieval methods}, we consider a zero-shot approach Custom \cite{kojima2022large} as a baseline and evaluate improvements relative to it. To fully exploit the reasoning capabilities of the LLMs, we incorporate chain-of-thought (CoT) reasoning \cite{wei2022chain}.
For \textit{retrieval-based methods}, we evaluate the classic RAG model \cite{lewis2020retrieval}, the domain-adapted MedRAG \cite{xiong-etal-2024-benchmarking}, and $i$-MedRAG \cite{xiong2024improving}, a medical-domain RAG system designed to decompose questions and iteratively provide answers.

We adopt GAR \cite{mao-etal-2021-generation} as a representative \textit{query-optimized RAG method}, implemented train-free in accordance with § \ref{sec:Train-free}. RGAR defaults to \textbf{2} rounds of recurrence.


\subsubsection{Evaluation Settings}
Following MIRAGE \cite{xiong-etal-2024-benchmarking}, we adopt the following evaluation framework. In \textbf{Option-Free Retrieval}, no answer options are provided for retrieval (§\ref{sec:Train-free}), ensuring a more realistic medical QA scenario. In \textbf{Zero-Shot Learning}, RAG systems are evaluated without in-context few-shot learning, reflecting the lack of similar exemplars in real-world medical questions. For \textbf{Metrics}, we employ Accuracy, defined as the proportion of correctly answered questions, and we extract model outputs by applying regular expression matching to the entire generated responses \cite{wang-etal-2024-answer-c}.


% We adopt GAR \cite{mao-etal-2021-generation} as a representative \textit{query-optimized RAG approach}. Rather than the original strategy of training a BERT model to generate queries, we instead leverage prompt-based query generation with LLMs \cite{yu2023generate}. Our RGAR approach maintains this same strategy and defaults to two retrieval–generation cycles.
% 我文章读错了,确实是要微调BERT,但是它也是答案是生成的而非原始的Notably, the original GAR retrieves ground-truth answer options; in our comparison, these options are also generated via prompts.















\subsection{Main Results}
\subsubsection{Cross-Dataset Performance Improvement}
\label{cross-dataset}
\begin{figure*}[ht]
    \centering
    \begin{subfigure}{0.32\textwidth}
        \includegraphics[width=\textwidth]{line_1.pdf}
        \caption{Effect of Using Original Options.}
        \label{fig:sub1}
    \end{subfigure}
    \hfill
    \begin{subfigure}{0.32\textwidth}
        \includegraphics[width=\textwidth]{line_2.pdf}
        \caption{Effect of RGAR's Two Components.}
        \label{fig:sub2}
    \end{subfigure}
    \hfill
    \begin{subfigure}{0.32\textwidth}
        \includegraphics[width=\textwidth]{line_3.pdf}
        \caption{Effect of Rounds in RGAR.}
        \label{fig:sub3}
    \end{subfigure}
    \caption{Accuracy with Different Numbers of Retrieved Chunks on EHRNoteQA Dataset.}
    \label{fig:three_sub_figures}
\end{figure*}

We evaluate RGAR with the Llama-3.2-3B-Instruct across three factual-aware medical datasets, comparing it with several competitive baselines. Table~\ref{tab:mian_results} presents the results of all methods, along with their relative improvements over the Custom baseline. RGAR achieves the highest average performance across the three datasets, surpassing the second-best method, $i$-MedRAG, by 2\%. The retrieval-based methods, even the lowest-performing RAG, consistently outperform the non-retrieval methods Custom and CoT. This highlights the importance of retrieving specialized medical knowledge when using general-purpose LLMs to answer professional medical queries. Comparing different retrieval methods, GAR outperforms vanilla RAG by approximately 3\% on average, with a maximum improvement of 4.37\% across datasets. This indicates that generating multiple queries for retrieval provides consistent benefits. However, while performing well on EHRNoteQA, MedRAG demonstrates a negative effect on the other two datasets compared to vanilla RAG.

Notably, the improvements achieved by our RGAR over GAR exhibit a positive correlation with the average length of the dataset’s context. On EHRNoteQA, which has an average context length exceeding 3000 tokens, our approach achieved a 7.8\% improvement. This validates the advantage of our \textit{Factual knowledge Extraction} in enhancing retrieval effectiveness. Consequently, our method is particularly well-suited to real-world scenarios where complete electronic health records must be analyzed to provide medical advice. This indicates that our approach is promising for real-life applications in assisting physicians with clinical recommendations.

When analyzing performance across different datasets, we find that retrieval-based methods perform significantly better on MedQA-USMLE and EHRNoteQA, while MedMCQA showa a negative effect—consistent with results reported by MedRAG \cite{xiong-etal-2024-benchmarking}. A closer analysis reveals that MedMCQA incorporates arithmetic reasoning questions (roughly 7\% of the total), and the addition of extensive retrieved contexts diminishes the model’s numerical reasoning capabilities, which could potentially be fixed with larger base LLMs \cite{mirzadeh2025gsmsymbolic}. Nonetheless, among retrieval-based methods, our RGAR stands out as the only approach that outperforms vanilla RAG on this dataset, delivering an improvement of more than 1\% over Custom.
On EHRNoteQA, while RGAR’s performance is slightly below that of $i$-MedRAG, \textbf{the latter’s inference time is approximately 4 times longer, establishing RGAR as a more efficient and cost-effective alternative}.


\subsubsection{Base LLMs with Different Sizes and Model Families}
\begin{table}[htbp]
  \centering
  \caption{Comparison of LLMs on MedQA-USMLE.}
  \resizebox{\linewidth}{!}{ % 调整表格宽度适应页面
    \begin{tabular}{lcccc}
      \toprule
      Model & \multicolumn{1}{c}{Custom} & \multicolumn{1}{c}{RAG} & \multicolumn{1}{c}{GAR} & \multicolumn{1}{c}{RGAR} \\
      \midrule

      Llama-3.2-1B-Instruct & 38.96 & 29.30 & 30.79 & 29.85 \\
      Llama-3.2-3B-Instruct & 50.20 & 53.50 & 57.97 & 58.83 \\
      Llama-3.1-8B-Instruct & 60.80 & 62.14 & 67.39 & 69.52 \\
        \midrule
      Qwen2.5-1.5B-Instruct & 43.99 & 41.48 & 43.42 & 42.58 \\
      Qwen2.5-3B-Instruct & 48.23 & 49.96 & 53.50 & 54.28 \\
      Qwen2.5-7B-Instruct & 59.46 & 58.83 & 63.39 & 63.86 \\
      \midrule
      Average   & 50.27 & 49.20 & 52.74 & 53.15 \\
      \bottomrule
    \end{tabular}
  }
  \label{tab:performance}
\end{table}
To further assess the versatility of RGAR, we conduct evaluations on MedQA-USMLE, a widely used medical dataset, by utilizing base LLMs of various sizes and model families, specifically from Llama and Qwen. The results in Table \ref{tab:performance} show that RGAR consistently achieves the best average performance.

When considering model size, we find that retrieval-based approaches fall short of the non-retrieval Custom baseline for smaller models, such as Llama-3.2-1B-Instruct and Qwen2.5-1.5B-Instruct. These smaller models, constrained by their weaker performance, are not well-suited to leverage retrieval-enhanced information. As the model size increases, however, all retrieval-enhanced approaches exhibit notable performance gains, with RGAR yielding the most significant improvements. This trend becomes particularly pronounced for larger models. For example, RGAR achieves a 7.38\% improvement over RAG on Llama-8B, 5.33\% on Llama-3B, 5.03\% on Qwen-8B, and 4.32\% on Qwen-3B.

%While we did not test commercial closed-source models like GPT due to their high API costs
Moreover, we find that under the same experimental conditions, \textbf{Llama-3.1-8B-Instruct achieves a performance of 69.52\% with RGAR, surpassing the 66.22\% reported by MedRAG for GPT-3.5-16k-0613} \cite{achiam2023gpt}. This significant improvement underscores the practicality of using well-optimized retrieval methods with smaller models, enabling performance rivals those of proprietary large-scale foundational models in real-world medical recommendation tasks.

\subsection{Ablation Study}
% Due to the absence of ground-truth retrieval chunks across all three datasets, it is not feasible to evaluate retrieval performance using metrics such as nDCG@10 or Recall@100 like the BEIR benchmark \cite{thakur2021beir}. Instead, we assess retrieval effectiveness through QA performance, varying the number of retrieved items \(N\) from 4 to 32. A lower retrieval count more rigorously tests retrieval quality. We investigate three primary factors: the effect of options generated by GAR versus those originally provided by the dataset, the contributions of GAR and enhanced EHR components, and the impact of RGAR’s iterative rounds.

Due to the absence of ground-truth retrieval chunks, we evaluate retrieval effectiveness through QA performance, systematically varying the number of retrieved chunks \(N\) from 4 to 32. A reduced retrieval number serves as a more stringent assessment of retrieval quality. We investigate three primary factors in Figure \ref{fig:three_sub_figures}: the effect of options generated by GAR versus those originally provided by the dataset, the contributions of CKR and FKE components, and the impact of RGAR’s recurrence rounds.

We first compare the retrieval performance between LLM-generated options and original dataset options. Figure \ref{fig:sub1} shows how RGAR and GAR perform across different values of \(N\). Both approaches maintain stable performance across different \(N\), indicating reliable retrieval quality. While using original options shows slightly higher average Accuracy, the difference is minimal. This suggests that even when GAR generates options that differ from the originals, it achieves similar retrieval results as long as the core topics align. 

We then examine the impact of RGAR's two main components—CKR and FKE—as shown in Figure \ref{fig:sub2}. When we remove the conceptual knowledge interaction from the FKE phase, the system shows only moderate improvements when extracting factual knowledge from EHR without conceptual knowledge, demonstrating the importance of integrating both types of knowledge. %When we remove the multi-query generation step from CKR, performance decreases as \(N\) increases, indicating unstable retrieval quality. This highlights the necessity of generating multiple queries during the CKR phase to maintain stable retrieval.
% 这里也改了一下
Removing the multi-query generation step from CKR causes performance to degrade as \(N\) increases, indicating that multiple queries are necessary to maintain stable retrieval.

Finally, we analyze the effect of rounds in RGAR (Round 0 means GAR), as illustrated in Figure \ref{fig:sub3}. Our results show that even a single iteration significantly improves performance by enabling interaction between factual and conceptual knowledge. Multiple rounds work similarly to a reranking mechanism \cite{mao-etal-2021-reader}, improving the ranking of important chunks and showing substantial gains even with relatively small \(N\). With \(N = 8\) , the default two-round setup achieves a performance of 75.78\%, almost 1\% better than using a single round. However, adding more rounds shows no clear benefits, as they tend to generate multi-hop factual knowledge during the FKE phase, leading CKR to retrieve multi-hop conceptual knowledge, which may cause LLMs to over-infer ~\cite{yang-etal-2024-large-language-models}. Given that each round involves one reasoning step from both the LLM extractor and LLM query generator, two rounds sufficiently support multi-hop reasoning needs \cite{lv-etal-2021-multi}.

% We begin by comparing the retrieval performance of using LLM-generated options to that of directly using the original dataset-provided options. Figure \ref{fig:sub1} illustrates how the performance of RGAR and GAR changes with different values of \(N\). Both configurations exhibit relatively stable performance across the range of \(N\), indicating consistent retrieval quality. While the approach relying on original options shows slightly higher average accuracy (Acc) at various \(N\), the difference is negligible. Even when GAR generates options that differ in content from the originals, it achieves similar retrieval outcomes as long as the underlying topics are aligned. This suggests that GAR-generated options, despite their differences, remain conducive to effective retrieval.

% Next, we analyze the role of RGAR’s two main components in Figure \ref{fig:sub2}: GAR itself and the enhanced EHR retrieval process. We examine the impact of using original EHR data directly for retrieval instead of leveraging our LLM-generated approach. When relying solely on original EHR data, even with multiple iterations, the performance shows only modest improvements and remains capped. This is because such approaches can only enhance the relevance of concept-level information and key content, without encouraging the synthesis of new information or exploration beyond the original data. This limitation is especially pronounced for medical queries requiring multi-hop reasoning. When GAR is removed entirely, we observe a performance decline as \(N\) increases, highlighting the instability of retrieval quality. While the most relevant information may still be retrieved, the absence of auxiliary context hampers reasoning, and the introduction of irrelevant information as \(N\) grows leads to further performance degradation.

% Finally, we assess the effect of iterative rounds in RGAR in Figure \ref{fig:sub3}.The experimental results demonstrate that implementing even a single recurrence iteration yields significant performance improvements, as it facilitates the interaction between factual and conceptual knowledge domains. Multiple iterations function analogously to a reranking mechanism, elevating the relevance of truly pertinent chunks and achieving substantial enhancement when operating with relatively modest N. Notably, with N=8, the default two-iteration configuration achieved a performance of 75.78\%, representing nearly a 1\% improvement over the single-iteration baseline. However, excessive iteration rounds demonstrate no discernible advantages, as they tend to generate multi-hop factual knowledge during the FKE phase, potentially leading to over-inference by LLMs ~\cite{yang-etal-2024-large-language-models}. 

% Finally, we assess the effect of iterative retrieval rounds in RGAR. Setting the iteration count to zero effectively results in the GAR approach. Figure~X depicts how performance evolves as the number of retrieval rounds increases from zero to three. The results show that even a single round yields notable improvements, with subsequent rounds offering diminishing returns. We adopt two retrieval rounds, based on the assumption that most questions can be answered within three reasoning steps~\cite{lv-etal-2021-multi}, and excessive rounds provide no clear advantage~\cite{yang-etal-2024-large-language-models}. RGAR’s approach resembles bidirectional breadth-first search, simultaneously exploring from both the original context and possible answers. With two rounds, RGAR allows for up to four multi-hop steps, which proves to be sufficient.

\subsection{Fine-Grained Performance Analysis}

While the previous sections examined overall dataset performance and established preliminary findings, this section provides a detailed analysis of specific aspects of our results. In § \ref{cross-dataset}, we showed that RGAR performs better on real-world medical recommendation tasks involving comprehensive EHRs. To verify this finding, we conduct a detailed analysis of EHRNoteQA by grouping questions based on context length and dividing them into four bins. Within each bin, we compare the performance of RGAR, GAR, and Custom. As shown in Figure \ref{fig:bar}, Custom shows decreasing accuracy with increasing context length. GAR improves accuracy across all bins, with RGAR achieving further performance gains. Notably, the improvements are more significant in the three bins with longer contexts compared to the first bin. The results show that RGAR maintains consistent average performance across different context length.

% The previous sections focused on overall dataset performance and provided some preliminary conclusions. Here, we delve deeper into specific aspects of those findings. In earlier sections, we concluded that our approach is better suited to real-world medical recommendation tasks involving comprehensive EHRs. To validate this, we further analyze EHRNoteQA by sorting questions based on average context length and dividing them into four bins. Within each bin, we compare the performance of our approach and GAR against the Custom baseline. As shown in Figure \ref{fig:bar}, Acc declines with increasing question length for the Custom baseline. GAR improves Acc across all bins, and RGAR further enhances performance. Notably, the improvement is more pronounced in the three bins with longer contexts compared to the first bin. Overall, the average performance across all bins is similar.

\begin{figure}[htbp]
    \centering
    \includegraphics[width=1\linewidth]{bar.pdf}
    \caption{Fine-Grained Accuracy of EHRNoteQA After Sorting by Length and Dividing into Four Equal Parts.}
    \label{fig:bar}
\end{figure}

%We also revisit the findings, which suggest that GAR stabilizes retrieval. 
It is also important to note that generating multiple queries from different aspects within RGAR helps stabilize retrieval.
Figure \ref{fig:tsne} presents a t-SNE visualization of different queries and their individually retrieved chunks for a sample question (details provided in Appendix~\ref{case}). The basic query shows limited suitability for retrieval, as its coverage area differs from that of the three queries generated by RGAR. RGAR clearly introduces some variation in retrieval content. Although the regions corresponding to the three generated queries overlap, the specific chunks retrieved do not overlap significantly. This underscores the need to average the retrieval similarities of these three queries to achieve more stable retrieval results.

\begin{figure}[htbp]
    \centering
    \includegraphics[width=1\linewidth]{t-sne.pdf}
    \caption{t-SNE Visualization of Different Queries and the Retrieved Chunks.}
    \label{fig:tsne}
\end{figure}

% We also revisit the conclusion from our ablation study, which suggested that GAR stabilizes retrieval. Figure \ref{fig:tsne} presents a t-SNE visualization of different queries and their individually retrieved chunks for a sample question (details provided in Appendix~\ref{case}). The basic query shows limited suitability for retrieval, as its coverage area differs from that of the three queries generated by GAR. GAR clearly introduces some variation in retrieval content. Although the regions corresponding to the three GAR-generated queries overlap, the specific chunks retrieved do not overlap significantly. This underscores the need to average the retrieval similarities of these three queries to achieve more stable retrieval results.

\section{Conclusion}
In this work, we propose RGAR, a novel RAG system that distinguishes two types of retrievable knowledge. Through comprehensive evaluation across three factual-aware medical benchmarks, RGAR demonstrates substantial improvements over existing methods, emphasizing the significant impact of in-depth factual knowledge extraction and its interaction with conceptual knowledge on enhancing retrieval performance. %Notably, our system enables 8B parameter models to outperform proprietary large-scale commercial models with retrieval capabilities.
Notably, our RGAR enables the Llama-3.1-8B-Instruct model to outperform the considerably larger, RAG-enhanced proprietary GPT-3.5.
%From a broader perspective, RGAR represents a promising approach for enhancing general LLMs in real-world clinical diagnostic scenarios that demand extensive factual knowledge processing. This framework shows potential for extension to other professional domains where factual awareness is crucial, offering a viable solution for specialized applications requiring precise factual knowledge management.
From a broader perspective, RGAR offers a promising approach for enhancing general-purpose LLMs in clinical diagnostic scenarios where extensive factual knowledge is crucial, with potential for extension to other professional domains demanding precise factual awareness. 
\newpage
\section*{Limitations}
%Despite RGAR achieving superior average performance, several limitations warrant discussion. Our RGAR necessitates corpus retrieval, with time complexity scaling proportionally with corpus size, this is a problem inherent in the RAG paradigm. Approaches that generate reasoning evidence directly through domain-specified LLMs \cite{yu2023generate, frisoni-etal-2024-generate} avoid the inference-time computational issue, however, they are illed in updating LLMs to follow new medical knowledge, which induces frequency updation and training costs. %Additionally, while the multiple LLM generations required by RGAR's retrieval process showed negligible additional time overhead on Llama3.2-3B in our primary experiments, this overhead becomes significant when scaling to larger models.
Despite RGAR achieving superior average performance, several limitations warrant discussion. Our RGAR requires corpus retrieval, and its time complexity scales proportionally with the size of the corpus, which is an inherent issue within the RAG paradigm. Approaches that generate reasoning evidence directly through domain-specific LLMs \cite{yu2023generate, frisoni-etal-2024-generate} avoid the computational challenges at inference time. However, they face difficulties in updating LLMs to incorporate new medical knowledge, which results in frequent updates and training costs.

Comparative approaches such as MedRAG \cite{xiong-etal-2024-benchmarking} and $i$-MedRAG \cite{xiong2024improving} explore integration possibilities with prompting techniques like Chain-of-Thought \cite{wei2022chain} and Self-Consistency \cite{wang2023selfconsistency} to enhance reasoning capabilities. Our investigation focused specifically on validating how additional factual knowledge processing improves retrieval performance, without examining the impact of these prompting strategies. %Furthermore, unlike multi-round methods such as %Adaptive RAG \cite{jeong-etal-2024-adaptive} and 
%$i$-MedRAG \cite{xiong2024improving} that implement LLM-based early stopping to reduce computational costs, our system operates with fixed time complexity.
%However, it is noteworthy that由于i-medrag每轮都要分解若干query检索并回答再汇总,RGAR的实际时间开销远小于i-medrag。
Furthermore, unlike multi-round methods such as $i$-MedRAG \cite{xiong2024improving} that implement LLM-based early stopping to reduce computational costs, our system operates with fixed time complexity. However, it is noteworthy that, because $i$-MedRAG requires multiple rounds of query decomposition, retrieval, and answer aggregation, the actual time overhead of RGAR is significantly smaller than that of $i$-MedRAG.


Our EHR extraction approach assumes LLMs can process complete EHR contextual input, justified by current mainstream LLMs exceeding 128K context windows with anticipated growth. However, in extreme cases where EHR content exceeds LLM context limits, integration with chunk-free approaches may be necessary \cite{luo-etal-2024-landmark, qian-etal-2024-grounding}. Finally, as RGAR operates in a zero-shot setting without instruction fine-tuning, its effectiveness is partially contingent on the model's instruction-following capabilities—which we cannot fully mitigate.

\section*{Ethical Statement}
This research adheres to the ACL Code of Ethics. All medical datasets utilized in this study are either open access or obtained through credentialed access protocols. To ensure patient privacy protection, all datasets have undergone comprehensive anonymization procedures.
While Large Language Models (LLMs) present considerable societal benefits, particularly in healthcare applications, they also introduce potential risks that warrant careful consideration. Although our work advances the relevance of retrieved content for medical queries, we acknowledge that LLM-generated responses based on retrieved information may still be susceptible to errors or perpetuate existing biases.
Given the critical nature of medical information and its potential impact on healthcare decisions, we strongly advocate for a conservative implementation approach. Specifically, we recommend that all system outputs undergo rigorous validation by qualified medical professionals before any practical application. This stringent verification process is essential to maintain the integrity of clinical and scientific discourse and prevent the propagation of inaccurate or potentially harmful information in healthcare settings.
These ethical safeguards reflect our commitment to responsible AI development in the medical domain, where the stakes of misinformation are particularly high and the need for reliability is paramount.

\bibliography{custom}

\appendix

\section{Implementation Details}
\subsection{Hardware Configuration}
All experiments were conducted on an in-house workstation equipped with \textit{dual} NVIDIA GeForce RTX 4090 GPUs, % (24GB VRAM \textit{each}),
128GB RAM, and an Intel® Core i9-13900K CPU.

Time cost across all methods on EHRNoteQA are shown in Table \ref{tab:method_time_comparison}.

\begin{table}[htbp]
  \centering
  \caption{Comparison of different methods in terms of execution time (hours).}
  \resizebox{\linewidth}{!}{
  \begin{tabular}{lccccccc}
    \toprule
    Method & Custom & CoT & RAG & MedRAG & GAR & $i$-MedRAG & RGAR \\
    \midrule
    Time (h) & 0.5 & 0.5 & 1 & 1 & 2 & 22 & 6 \\
    \bottomrule
  \end{tabular}
  }
  \label{tab:method_time_comparison}
\end{table}

\subsection{Code and Results}
The core implementation of the RGAR framework and the output json files can be accessed via the \textbf{Anonymous Repository}: \url{https://anonymous.4open.science/r/RGAR-C613}


\section{Prompt Template and Case Study}
\label{case}
For simplicity, we merged EHR and question in the prompt words of the answer and treated them as question in the prompt words.
Table \ref{tab:prompts} shows the prompts template of RGAR and compared work (Using CoT ones). Table \ref{tab:input} shows the input of a sample, Table \ref{tab:output} shows the final output of RGAR.

\begin{table*}[h]
\centering
\begin{tabular}{p{0.3\textwidth}|p{0.7\textwidth}}
\toprule
Type & Prompt Template \\
\midrule
System prompts for Non-CoT & You are a helpful medical expert, and your task is to answer a multi-choice medical question using the relevant documents. Organize your output in a json formatted as Dict \{"answer\_choice": Str\{A/B/C/...\}\}. Your responses will be used for research purposes only, so please have a definite answer. Please just give me the json of the answer. \\
\midrule
System prompts for using CoT  & You are a helpful medical expert, and your task is to answer a multi-choice medical question. Please first think step-by-step and then choose the answer from the provided options. Organize your output in a json formatted as Dict\{"step\_by\_step\_thinking": Str(explanation), "answer\_choice": Str\{A/B/C/...\}\}. Your responses will be used for research purposes only, so please have a definite answer. Please just give me the json of the answer. \\
\midrule
Answer prompts for Non-CoT &Here are the relevant documents:
\{\{context\}\}
\newline
Here is the question:
\{\{question\}\}
\newline
Here are the potential choices:
\{\{options\}\}
\newline
Please just give me the json of the answer. Generate your output in json:\\

\midrule
Answer prompts for Using CoT &Here are the relevant documents:
\{\{context\}\}
\newline
Here is the question:
\{\{question\}\}
\newline
Here are the potential choices:
\{\{options\}\}
\newline
Please think step-by-step and generate your output in one json:\\
\midrule
Extracting EHR prompts & Here are the relevant knowledge sources:
\{\{context\}\}
\newline
Here are the electronic health records:
\{\{ehr\}\}
\newline
Here is the question:
\{\{question\}\}
\newline
Please analyze and extract the key factual information in the electronic health records relevant to solving this question and present it as a Python list. 
Use concise descriptions for each item, formatted as ["key detail 1", ..., "key detail N"]. Please only give me the list. Here is the list: \\
\midrule
Generating Possible Answer prompts & Please give 4 options for the question. Each option should be a concise description of a key detail, formatted as: A. "key detail 1" B. "key detail 2" C. "key detail 3" D. "key detail 4\\
\midrule
Generating Possible Title prompts & Please generate some titles of references that might address the above question. Please give me only the titles, formatted as: ["title 1", "title 2", ..., "title N"]. Please be careful not to give specific content and analysis, just the title.\\
\midrule
Generating Possible Contexts prompts & Please generate some knowledge that might address the above question. please give me only the knowledge. \\
\bottomrule
\end{tabular}
\caption{Prompt templates used in RGAR and Compared Methods.}
\label{tab:prompts}
\end{table*}

\begin{table*}[h]
\centering
\begin{tabular}{p{0.3\textwidth}|p{0.7\textwidth}}
\toprule
Type & Texts \\
\midrule
EHR & A 39-year-old woman is brought to the emergency department because of fevers, chills, and left lower quadrant pain. Her temperature is 39.1°C (102.3°F), pulse is 126/min, respirations are 28/min, and blood pressure is 80/50 mm Hg. There is blood oozing around the site of a peripheral intravenous line. Pelvic examination shows mucopurulent discharge from the cervical os and left adnexal tenderness. Laboratory studies show: Platelet count 14,200/mm3 Fibrinogen 83 mg/mL (N = 200–430 mg/dL) D-dimer 965 ng/mL (N < 500 ng/mL)\\
\midrule
Questions & When phenol is applied to a sample of the patient's blood at 90°C, a phosphorylated N-acetylglucosamine dimer with 6 fatty acids attached to a polysaccharide side chain is identified. A blood culture is most likely to show which of the following? \\
\midrule
Options & A. Coagulase-positive, gram-positive cocci forming mauve-colored colonies on methicillin-containing agar \newline
B. Encapsulated, gram-negative coccobacilli forming grey-colored colonies on charcoal blood agar \newline
C. Spore-forming, gram-positive bacilli forming yellow colonies on casein agar \newline
\textbf{D. Lactose-fermenting, gram-negative rods forming pink colonies on MacConkey agar} \\
\bottomrule
\end{tabular}
\caption{Case Study of a Sample Input.}
\label{tab:input}
\end{table*}

\begin{table*}[h]
\centering
\begin{tabular}{p{0.3\textwidth}|p{0.7\textwidth}}
\toprule
Type & Texts \\
\midrule
Generated Title & "Identification of Streptococcus pneumoniae in Clinical Specimens",\newline "Phosphorylcholine-Binding Protein A of Staphylococcus aureus",\newline "D-Dimer Assay for Diagnosing Deep Vein Thrombosis",\newline "The Role of Fibrin Degradation Products in Hemostasis",\newline "Infection Control Measures for Hospital-Acquired Bloodstream Infections" \\
\midrule
Generate Context & The description suggests Neisseria gonorrhoeae infection. Neisseria gonorrhoeae produces lipopolysaccharides in its outer membrane, specifically LPS or endotoxin. When heated, these lipopolysaccharides undergo cross-linking reactions resulting in the formation of phosphorylcholine-attached oligomers such as those described in the question stem. These structures can be detected by ELISA using antibodies against specific epitopes on the surface of N. gonorrhoeae cells. The presence of this structure would suggest an active infection due to N. gonorrhoeae. Blood cultures for N. gonorrhoeae typically require specialized media containing antibiotics like vancomycin, colistin, and nystatin to inhibit competing bacteria \\
\midrule
Generate Answer & Option 1: Streptococcus pneumoniae The presence of a high fever, tachycardia, hypotension, and low platelets suggests sepsis or severe infection. The laboratory findings also indicate disseminated intravascular coagulation (DIC). The identification of a phosphorylated N-acetylglucosane dimer with 6 fatty acid chains attached to a polysaccharide side chain after applying phenol at 90°C is consistent with pneumococcal capsular polysacarcharides.\newline
Option 2: Staphylococcus aureus Staphylococci can cause various infections but do not typically produce this type of capsule structure when exposed to heat.\newline
Option 3: Escherichia coli E. coli does have capsules that may be detected by certain methods such as latex agglutination tests; however, they don't match the description given in the prompt.\newline
Option 4: Neisseria meningitidis Neisseria species like N. meningitidis would more commonly present with symptoms related to central nervous system involvement rather than pelvic signs described in the scenario.\\
\midrule
Extract EHR & ['Patient age: 39 years', 'Symptoms: fever, chills, left lower quadrant pain', 'Vital signs: \textbf{high} temperature (39.1°C), tachycardia (126/min), tachypnea (28/min), hypotension (80/50 mmHg)', 'Physical exam findings: mucopurulent discharge from cervix, left adnexal tenderness', 'Laboratory results: \textbf{low} platelet count (14,200/mm\^{}3), \textbf{elevated} D-dimer (965 ng/mL)', 'Phenol test result: phosphorylated N-acetylglucosame dimmer with 6 fatty acids attached to a polysaccharide side chain'] \\
\midrule
Output Json & \{"answer\_choice": "D"\}\\
\bottomrule
\end{tabular}
\caption{Case Study of output in RGAR.}
\label{tab:output}
\end{table*}

\section{Framework Insight}
\subsection{Another View of the Recurrence Pipeline}
We conceptualize the Recurrence Pipeline as an exploration-exploitation process within the reinforcement learning framework \cite{10.1023/A:1013689704352}. In GAR, even when generated content is only partially accurate (or potentially inaccurate), it remains valuable for retrieval if it correlates with passages containing correct information (e.g., co-occurrence with correct answers), thus representing an exploratory phase. Conversely, EHR extraction serves as an exploitation phase, thoroughly utilizing explored knowledge by selecting relevant components and synthesizing new evidence (factual knowledge). Based on this newly derived evidence, subsequent iterations can initiate fresh exploration-exploitation cycles, creating a continuous knowledge transmission process \cite{10446501}.

In scenarios where additional factual knowledge is not required, the retrieved content tends to remain relatively constant, and utilizing this content under identical prompting conditions would likely yield similar factual knowledge through extraction and summarization. However, when conceptual knowledge is needed to derive new factual knowledge through reasoning from existing factual information, the updated basic query facilitates easier retrieval of conceptual knowledge supporting current reasoned factual knowledge, thereby maintaining the integrity of reasoning chains. Furthermore, leveraging current factual knowledge for retrieval enables the exploration and discovery of novel knowledge domains.

\subsection{Why No Flexible Stopping Criteria}
Similar multiround RAG systems have adopted more flexible stopping criteria. For instance, Adaptive RAG \cite{jeong-etal-2024-adaptive} determines whether to retrieve further % or how many rounds of retrieval are needed
by consulting the model itself. $i$-MedRAG \cite{xiong2024improving}, while setting a maximum number of retrieval iterations, also supports early stopping.

In our RGAR framework, we do not adopt such settings. On the one hand, we focus on evaluating how additional processing of \textit{factual knowledge} enhances retrieval performance, raising awareness of this often-overlooked type of knowledge in previous RAG systems, while flexible stopping criteria mainly showcase procedural knowledge and metacognitive knowledge. On the other hand, the metacognitive capabilities of current LLMs remain under question, as a model’s self-evaluation of the need for additional retrieval information often does not match actual requirements \cite{kumar-etal-2024-confidence}.

\subsection{Future Work}
% % 我们的RGAR framework利用检索到的medical domain专业知识,在medical OpenQA任务重提供了卓越的回答质量。但是,我们担忧这种强大的生成能力一旦被恶意利用,也可能带来安全隐患。比如,当被检索的corpus包含私隐私信息或版权内容时,恶意的提问者可能利用LLM的回答提取并泄露corpus中的敏感信息 \cite{carlini2021extracting}。此外,恶意的提问者可能通过收集大量提问-问答对来尝试replicate我们的base LLM \cite{tramer2016stealing, zhu2024efficient}, 或推断我们检索生成框架的内部信息 \cite{carlinistealing} 作为泄露的商业秘密或未来攻击的基石。我们将在未来尽最大努力阻止这些恶意攻击,比如检查query是否合法 \cite{inan2023llama} 和通过水印标识RGAR所使用的模型 \cite{zhu2024reliable}, 从而保证RAGR被合理、合法地使用。
Our RGAR framework leverages retrieved medical domain knowledge to deliver exceptional answer quality% in medical OpenQA tasks
. However, we are concerned that such powerful generative capabilities, if maliciously exploited, could pose security risks. For instance, when the retrieved corpus contains private or copyrighted information, malicious users could exploit the LLM's responses to extract and disclose sensitive data from the corpus \cite{carlini2021extracting}. 
% Additionally, malicious users might attempt to replicate our base LLM \cite{tramer2016stealing, zhu2024efficient} by collecting large volumes of question-answer pairs or infer internal details of our retrieval-based generation framework \cite{carlinistealing}. 
Additionally, malicious users might attempt to replicate our base LLM \cite{tramer2016stealing, zhu2024efficient} by collecting large volumes of question-answer pairs or infer internal details of our retrieval-based generation framework \cite{carlinistealing}. 
%, potentially exposing proprietary information or providing a foundation for future attacks. 
% %To mitigate these risks, 
We will make every effort to mitigate these risks, such as verifying the legitimacy of queries \cite{inan2023llama} and watermarking the models used in RGAR \cite{zhu2024reliable}.
, ensuring that RGAR is used responsibly and legally.

\section{Comparative Analysis of Dataset Length Distributions}
In this section, we present additional visualizations comparing the two categories of datasets we described, and explain our rationale for excluding the MMLU-med dataset \cite{hendrycks2021measuring}. We plotted smoothed Kernel Density Estimation (KDE) curves for these datasets, as shown in Figure \ref{fig:kde}. Our analysis confirms that datasets containing Electronic Health Records (EHR) consistently demonstrate greater length compared to those without EHR content. However, certain datasets exhibit complex question sources and types. For instance, while the MMLU dataset shows a considerable mean length of 84 tokens and a maximum length of 961 tokens, as illustrated in the figure, the vast majority of its questions lack EHR content and are predominantly shorter in length. This characteristic led to our decision to exclude it from our experimental evaluation.
\ref{fig:kde}
\begin{figure*}[htbp]
        \centering
    \includegraphics[width=1\linewidth]{KDE.pdf}
    \caption{Length Distribution Analysis of Medical QA Datasets with and without EHR.}
    \label{fig:kde}
\end{figure*}
\end{document}


% Shorter URLs with https://
% From: https://tex.stackexchange.com/a/139934
\newcommand\rurl[1]{%
  \href{https://#1}{\nolinkurl{#1}}%
}

\title{Behavioral Analysis of Information Salience in Large Language Models}

\author{
Jan Trienes$^{1}$\quad
Jörg Schlötterer$^{1,2}$\quad
{\bf Junyi Jessy Li}$^3$\quad
{\bf Christin Seifert}$^1$\\
$^1$Marburg University\quad
$^2$University of Mannheim\\
$^3$The University of Texas at Austin\\
\texttt{\normalsize \{jan.trienes,joerg.schloetterer,christin.seifert\}@uni-marburg.de}\\
\texttt{\normalsize jessy@utexas.edu}}


\def\gpt{GPT\=/4o}
\def\gptmini{GPT\=/4o\=/mini}

\newcommand\unfootnote[1]{%
  \begingroup
  \renewcommand\thefootnote{}\footnote{#1}%
  \addtocounter{footnote}{-1}%
  \endgroup
}



\begin{document}
\maketitle
Humor is a social binding agent. It is an act of creativity that can provoke emotional reactions on a broad range of topics. Humor has long been thought to be “too human” for AI to generate. However, humans are complex, and humor requires our complex set of skills: cognitive reasoning, social understanding, a broad base of knowledge, creative thinking, and audience understanding. We explore whether giving AI such skills enables it to write humor. We target one audience: Gen Z humor fans. We ask people to rate meme caption humor from three sources: highly upvoted human captions, 2) basic LLMs, and 3) LLMs captions with humor skills. We find that users like LLMs captions with humor skills more than basic LLMs and almost on par with top-rated humor written by people. We discuss how giving AI human-like skills can help it generate communication that resonates with people. 

\section{Introduction}
%We have seen significant advancements in \abr{LLM} development, which has enhanced the capabilities of model-based agents like ChatGPT, Bing Copilot, and Google Gemini, to allow them to excel on tasks like image generation, code generation, and mathematical reasoning. These capabilities are not only attested to by their performance on standardized benchmarks but are also reflected in their use across a diverse set of real-world domains \cite{suri2024usegenerativesearchengines}. Recently developed \abr{LLMs} are also able to assist humans across a variety of fields like teaching \cite{wang2024tutorcopilothumanaiapproach, ALSAFARI2024100101} and the clinical domain \cite{han2024ascleai}.

%Moreover, they have also led to an increase in user productivity \cite{peng2023productivitygithub, cambon2023early}, with millions of users relying on them for a variety of tasks ranging from quick information retrieval to more creative or technical pursuits such as drafting essays, writing code, designing artworks, and solving mathematical problems. Their growing integration across platforms like GitHub and Microsoft Office products (via Copilot) and Google (via Gemini) has also made them readily available to users in their everyday tasks.

We have seen significant advancements in \abr{LLM} development, which has enhanced the capabilities of model-based agents like ChatGPT that allow them to excel on tasks ranging from quick information retrieval to more creative or technical pursuits such as drafting essays, writing code, and designing artworks. These capabilities are not only attested to by their performance on standardized benchmarks but are also reflected in their use across a diverse set of real-world domains \cite{suri2024usegenerativesearchengines}. Recently developed \abr{LLMs} are able to assist humans across a variety of fields like teaching \cite{wang2024tutorcopilothumanaiapproach, ALSAFARI2024100101} and the clinical domain \cite{han2024ascleai}, and have led to an increase in user productivity \cite{peng2023productivitygithub, cambon2023early}.

\begin{figure}[t!]
    \centering    
    \includegraphics[width=\columnwidth]{figures/Teaser_Figure.pdf}
    \caption{An overview of our expertise classifier pipeline.}
    \label{fig:example}
\end{figure}

However, while millions of people utilize these models for a variety of tasks, their expectations, backgrounds, and interactions with these tools can differ significantly. One key aspect where users might differ is their domain expertise in the conversation topic with the agent \abr{LLM}\footnote{Here on called \abr{LLM}.}. Not all end users share the same level of domain knowledge and thus may have different preferences and abilities to process the information that the model would return to them. A ``beginner'' user might want simple, general purpose information on a topic and could possibly be overwhelmed if presented with too ``high level'' information. On the other hand, a domain ``expert'' might have an unsatisfactory experience with the \abr{LLM} if not given a deeper and detailed response.% Understanding and addressing the potential misalignment between user expertise and the model’s responses becomes vital to enhancing user satisfaction and ensuring that \abr{LLMs} are capable of meeting diverse user needs.

Thus, we ask: \textit{What is the ideal expertise level of the LLM, and what are the consequences of any misalignment between the user and the LLM on domain expertise?}

To answer this, we develop an ordinal 5-point scale-based expertise classifier (shown in \autoref{fig:example}) that we apply to a corpus of over $25,000$ Bing Copilot conversations sampled across a variety of domains. We generate three measures of expertise for each conversation. First and second, we classify the level of expertise of the user and the \abr{LLM} respectively in the topic of the conversation. Third, we classify the gauged expertise of the user, defined as the judged level of expertise of the user based on the responses made by the \abr{LLM}. We show similarities and differences in these three types of expertise labels within the same Copilot conversation to identify cases where the \abr{LLM} is aligned or misaligned with the user. We then assess the impact of (mis)alignment on three measures of user interaction experience: user satisfaction, level of engagement, and complexity of task.

%We then assess the impact of (mis)alignment on three measures of user interaction experience: user satisfaction, level of engagement, and complexity of task. We show that users have an unsatisfactory experience (as measured using methodology discussed in~\cref{user_exp_metrics}) when underestimated by the \abr{LLM} or when the \abr{LLM} responds at a low expertise level, with the impact more profound for more complex tasks. We also show that more expert users engage more with the \abr{LLM} when the \abr{LLM} responds at a higher degree of expertise, while more novice users tend to engage more when the \abr{LLM} responds at a lower degree of expertise. Broadly these findings underscore the importance to human-centered \abr{NLP} systems of aligning the \abr{LLM} with dimensions of the user that are critical to the interaction. %In this case, better "alignment" in which the \abr{LLM} maintains a relatively high degree of expertise, particularly in relation to the user's expertise and task complexity, helped foster more satisfying and engaging interactions.%These findings underscore the importance of aligning the model’s expertise with that of the user. Ensuring such alignment is critical for creating human-centered NLP systems that adapt to users' needs, fostering more satisfying and productive interactions.

%To summarize, our contributions are:
%\begin{itemize}[nosep]
%    \item We introduce an ordinal 5-point scale-based expertise classifier to apply %to user-\abr{LLM} interactions.
%    \item We characterize the distribution of \abr{LLM} responses with respect to domain expertise (77\% proficient or higher, 23\% intermediate or lower).% observe that, surprisingly, in over $20\%$ of the cases, the \abr{LLM} is not ``proficient'' or ``expert''. 
%    \item We quantify the impact on user satisfaction when the \abr{LLM} responds at an expertise level below that of the user or underestimates the user's level of expertise. 
%    \item We show that users engage more with the \abr{LLM} when it responds at a level of expertise commensurate to that of the user.
%\end{itemize}

% The ``WHAT'':
% \begin{itemize}
%     \item Expertise classification of users from CoPilot Conversations
%     \item Is this 'User Expertise' aligned or calibrated with what the \abr{LLM} ``thinks''
%     \item If there is (mis)alignment, does it impact the user experience in any way?
%     \item How can we measure this impact?
% \end{itemize}

% The ``WHY'':
% \begin{itemize}
%     \item Human-centered \abr{LLMs} are ideal.
%     \item Expert users are more capable at handling higher cognitive loads than novice and beginners. They might have a bad experience if they are underestimated $\rightarrow$ might be dissatisfied.
%     \item Novice and beginner users could possibly get overwhelmed if the information provided is too ``expert'' level.
%     \item There is also a potential societal impact. Prior research has shown that productivity increases the most for ``Intermediate'' users. 
% \end{itemize}

% The ``How'':
% \begin{outline}
%     \1 Develop a 5-point scale based Zero Shot expertise classifier.
%     \1 Scale $\rightarrow$ Novice, Beginner, Intermediate, Proficient, Expert
%     \1 Scale is ordinal.
%     \1 Three different types of expertise classification:
%       \2 User Expertise: Based on User Turns and Full Conversation.
%       \2 User Gauged Expertise: Based on AI turns. The LLM is asked to predict what level of user is the AI response geared towards. 
%       \2 AI Expertise: Based on AI turns.    
% \end{outline}

% The ``WHAT TO DO WITH THIS'':
% \begin{outline}
%     \1 User - AI Agent Expertise Heatmap and User - Gauged User Expertise Heatmap to show misalignment. 
%     \1  User - AI Agent Expertise Heatmap and User - Gauged User Expertise Heatmap with SAT as density to show impact
%     \1 Show overall distribution graphs as well.
%     \1 Misalignment as $f(SAT)$
%     \1 Correlation between (User, AI) Expertise Difference or (User, Gauged User) Expertise Difference and:
%         \2 SAT Scores
%         \2 Task Complexity
%         \2 User Turn Word Count heatmap: Expert and Proficient AI causes 10-16x more words from the user.
% \end{outline}

% ``Some Insights'':
% \begin{itemize}
%     \item For lower expertise users, overestimation $\rightarrow$ positive SAT.
%     \item Higher complexity tasks hurt the most when \big\downarrow AI expertise or \big\uparrow underestimation.
%     \item Important for AI to be Proficient or Expert. Huge impact on SAT.
%     \item Misalignment between user and AI $\rightarrow$ hurts SAT.
%     \item 1 in 5 times, the AI is not proficient or expert.
% \end{itemize}

\section{\modelname: Localizing Images with Reasoning and Tools}
\label{sec:method}

\begin{figure*}[t]
    \centering \includegraphics[width=\linewidth]{images/framework.pdf}

\caption{The
    framework of \modelname comprises three main components:
    the \macname, which handles general reasoning; the \micname, which
    leverage external knowledge for detail-specific analysis, and
    the \guessname, which combines outputs from both analyzers to
    generate predictions.}  \label{fig:framework}
\end{figure*}


This section presents \modelname: image \geoloc with reasoning
about cultural and geographical clues and using
external tools. YouTube experts both reason with image elements (\textit{the driving is on the left}) and interpret image details with guidebooks or maps (\textit{the Paria Main Road is in Toco}) to locate an image. Based on this observation, \modelname (Figure~\ref{fig:framework}) has three components: (1)~\textit{\macname}, which focuses on generating a
reasoning process that analyzes the general information present in the
image (Section~\ref{sec:macro}); (2)~\textit{\micname}, which uses
additional knowledge sources and tools to explore the details
(Section~\ref{sec:micro}); and (3)~\textit{\guessname}, which takes
the concatenated outputs from both analyzers to determine the final
location, which can be configured to any granularity of locations
(Section~\ref{sec:guesser}).

\definecolor{climate}{HTML}{10739E}
\definecolor{vegetation}{HTML}{AE4132}
\definecolor{building}{HTML}{56517E}
\definecolor{infrastructure}{HTML}{36393D}

\begin{table}[t!]
    \centering
    \small
    \newcolumntype{Y}{>{\arraybackslash}X}
    \begin{tabularx}{\linewidth}{Y}
    \toprule 
        \textit{\textbf{Image}} \\
        \includegraphics[width=\linewidth]{images/israel.jpg} \\
        \midrule
        \textit{\textbf{Location}} \\
        Israel, Ashkelon. 31.66671, 34.59127 \\
        \midrule
        \textit{\textbf{Query}} \\
        <image> Given an image, craft a brief and cohesive reasoning path that deduces this location based on the visual clues present in the image. Using a tone of exploration ... \\
        \midrule
        \textit{\textbf{Response}} \\
        The scene reveals a blend of urban and natural features typical of \textcolor{climate}{\textbf{a Mediterranean climate}}. Notably, the presence of \textcolor{vegetation}{\textbf{palm trees}} suggests a \textcolor{climate}{\textbf{warm, temperate region}}. The architecture of the buildings, characterized by \textcolor{building}{\textbf{flat roofs and a modern, functional style}}, often indicates environments with minimal snowfall. ... \textcolor{infrastructure}{\textbf{the neatly paved roads and sidewalks with occasional dust}} align with urban areas in regions with dry summers. ... the collective visual attributes and environmental clues point to a location consistent with Israel's landscape and architectural style.\\
    \bottomrule
    \end{tabularx}
    \caption{Fine-tuning data format. The reasoning process leverages visual information in the images to deduce the correct location, such as \textcolor{climate}{\textbf{climate}}, \textcolor{vegetation}{\textbf{vegetation}}, \textcolor{building}{\textbf{building}}, and \textcolor{infrastructure}{\textbf{infrastructure}}. Complete prompts and response are in the Appendix.}
    \label{tab:sft_data}
\end{table}


\subsection{Training \textsc{vlm}s to \textsc{Reason} about Image Locations}
\label{sec:macro}

Recent \textsc{vlm}s can---sometimes---reason about the location of an
image.~\cite{ligeoreasoner}. However, the reasoning is limited to only
a few words and does not help localization~\cite{zhang2024can}.
%
To enhance \textsc{vlm}s to reason location-relevant
information in images, we create \dataname and fine-tune \textsc{vlm}s using it to build \macname. The
reasoning includes geographical information such as
climate, vegetation, building, and infrastructure (Table~\ref{tab:sft_data}). This approach
enables models to deduce locations from geographically pertinent
details, expanding the depth and applicability. 

After training, \macname can generate a rationale for images, where given an image~\( I \), the fine-tuned \textsc{vlm} produces a reasoning~\( R \). However, as the reasoning relies solely on \textsc{vlm}s constrained by their parameterized knowledge, it lacks the information to understand specific details. For instance, human experts can search maps for text on buildings or road signs and consult guidebooks to identify the house style of a particular country, which goes beyond the intrinsic knowledge within \textsc{vlm}s. To emulate this process, an additional module, \micname, integrates external tools, enabling more accurate interpretation of nuanced details.


\subsection{\textsc{Searching} Image Details}
\label{sec:micro}

The \micname module extracts fine-grained details from images to enhance the reasoning by integrating relevant knowledge. It crops the image, generates queries, and retrieves external knowledge.

\textbf{Grounding Image Details.} As highlighted in Section~\ref{sec:data_analysis}, human experts often concentrate on specific elements in images, such as signs, houses, and roads, which provide crucial location-based clues. A precise grounding model generates high-quality queries: given an image~\( I \) and a predefined set of elements~\( E = \{e_1, e_2, \dots, e_n\} \), \micname uses GroundingDino~\cite{liu2023grounding} to crop the image according to \(E\). Since each image may contain multiple instances of an element, the cropped images is defined as \( C = \{ c_{i,j} \mid e_i \in E, j \in [1, m_i] \} \), where \( m_i \) is the count of element~\( e_i \) in \( I \). Specifically, we select \textit{road sign}, \textit{building sign}, and \textit{house} from Figure~\ref{fig:clue-type} as elements, which align well with GroundingDino, since alternatives could yield overly large figures or uninformative results. Each cropped image \( c_{i,j} \) is a query for specific tools. Additionally, if \( c_{i,j} \) is a \textit{sign} that contains text, text-based queries are generated with Optical Character Recognition (\textsc{ocr}) from Qwen2-VL~\cite{Qwen2VL}. Therefore, the query set~\( Q \) is:

\vspace{-5pt}
\[
Q = \bigcup_{i,j} \{ c_{i,j}, \text{OCR}(c_{i,j}) \text{ if } c_{i,j} \in \text{signs} \}
\]


\textbf{Tools.} The query set~\( Q \) is then fed into a Tool Set~\( T \), which retrieves relevant knowledge. We use three tools for information retrieval: 
(1) The \textit{GeoGuessr Guidebook} contains rich information for locating images (Section~\ref{sec:data_collection}). Following prior research~\cite{luo2022g, zhou2024img2loc}, we frame \textit{Guidebook} using as a Retrieval-Augmented Generation problem. Given an input image (\textit{e.g.}, a house as in Figure~\ref{fig:framework}), we retrieve the most similar images. 
(2) \textit{Map.} The map is a critical tool in image \geoloc: text in images (\textit{e.g.}, a name on a sign) can pinpoint a location. We use OpenStreetMap\footnote{\url{https://www.openstreetmap.org/}} for location retrieval, providing the top three search results, with the place name and multi-level location details.
(3) \textit{\textsc{vlm}.} We use an additional \textsc{vlm} as a tool by prompting it to identify details that might be overlooked in the \macname. The \textsc{vlm} generates descriptions for details to narrow down potential locations (Figure~\ref{fig:framework}). Each tool~\( t \) in the Tool Set~\( T \) contributes to the retrieval of additional knowledge~\( K \): 

\vspace{-5pt}
\[
K = \bigcup_{t \in T} t(Q)
\]
Further implementation details in the Appendix~\ref{apd:implementations}.

\subsection{\textsc{Guessing} the Final Location}
\label{sec:guesser}
The \guessname uses all prior information to generate the final prediction. It concatenates the reasoning $R$ from the \macname with the external knowledge $K$ retrieved by the \micname, forms them into a prompt template $p$ along with the image $i$, and makes the location prediction with a \textsc{vlm}:

\vspace{-5pt}
\[
\hat{y}_{\text{loc}} = \mathrm{\textsc{vlm}}_{p} \big( I, \mathrm{concat}(R, K) \big)
\]
where $\hat{y}_{\text{loc}}$ is the model's generated location. The prompt $p$ is configurable to flexibly adjust to specific output formats, such as various location levels (\textit{e.g.}, country, city, and coordinates).
\section{Experimental Settings}
\label{sec:experimental-settings}

\begin{table}[t]
\small
\centering
\setlength{\tabcolsep}{4pt}
\begin{tabular}{lrrrr}
\toprule
\textbf{Statistic} & \textbf{RCT} & \textbf{CL} & \textbf{Astro} & \textbf{QMSum} \\
\midrule
Documents & 200 & 185 & 106 & 90 \\
Words/doc & 290 & 459 & 703 & 10,837 \\
\midrule
Questions & 21 & 14 & 13 & 10 \\
Answered/doc & 84.1\% & 86.2\% & 96.5\% & 91.9\% \\
Words/answer & 30.9 & 53.0 & 70.3 & 161.5 \\
Claims/answer & 6.5 & 11.4 & 12.4 & 29.6 \\
Claims (total) & 23,124 & 25,353 & 16,430 & 24,459 \\
\bottomrule
\end{tabular}

\caption{Dataset overview. Number of words is calculated as whitespace-separated tokens.}
\label{tab:dataset-statistics}

\end{table}

\paragraph{Datasets.}
We analyze LLM salience across several technical and scientific domains using four datasets (\cref{tab:dataset-statistics}).
We designed slightly unconventional summarization tasks because of their limited ``oracle'' summaries in common LLM training datasets.
This allows us to analyze how LLMs handle texts without strong priors, and how salience judgments vary across genres and discourse types (structured technical writing, academic discourse, and dialogue).

\paragraph{(1) Randomized Controlled Trials (RCT).}
We draw a random sample of 200 abstracts of RCTs published Jan--Apr 2024 from PubMed.
These documents follow established conventions to describe the conduct and outcomes of clinical studies.
The task is to further summarize the abstracts.


\paragraph{(2) Computation and Language (CL).} The second task is to summarize the \emph{related work} sections of NLP/CL papers published on arXiv.
Although CL paper summarization is common, summarizing the related work section itself is not.
We convert raw LaTeX sources to Markdown and only consider documents up to 2,000 tokens to fit the context window of smaller models.
A random sample of 185 documents published in October 2024 is drawn.

\paragraph{(3) Astrophysics (Astro).} The third dataset contains \emph{discussion} sections of astrophysics papers published on arXiv.
These documents interpret key results of theoretical and empirical astrophysics research.
Similar to the CL portion, summarizing only the discussion sections is uncommon.
A random sample of 106 documents is drawn, with pre-processing analogous to CL.

\paragraph{(4) Meetings (QMSum).} Lastly, we consider meeting transcript summarization.
We randomly sample 90 documents balanced across three domains from QMSum~\cite{Zhong:2021:NAACL}: product design, research and political discussions.
We format transcripts as \texttt{\small [Speaker]: [Utterance]} turns, separated by newlines.
We only experiment with long-context models ($\ge 32\text{k}$ tokens) on this dataset.

\begin{figure*}[t]
\includegraphics[width=\textwidth]{figures/salience-pubmed-sample}
\caption{Corpus-level content salience map for \emph{RCT} summaries by four methods.}
\label{fig:salience-pubmed}
\end{figure*}

\paragraph{Summarization Models.}
We experiment with 13 LLMs of different scales:
\textbf{OLMo}~(7B; 02/24, 07/24; \citealp{Groeneveld:2024:ACL}), \textbf{Mistral}~(7B; v0.3; \citealp{Jiang:2023:arXiv}), \textbf{Mixtral}~(8x7B; v0.1, \citealp{Jiang:2024:arXiv}), \textbf{Llama 2}~(7B, 13B, 70B; \citealp{Touvron:2023:arXiv}), \textbf{Llama~3}~(8B, 70B), and \textbf{Llama~3.1}~(8B, 70B; \citealp{Grattafiori:2024:arXiv}). For API-based models, we use \textbf{\gptmini}~(07/24) and \textbf{\gpt}~(08/24; \citealp{OpenAI:2024:arXiv}).
We also include 3 baselines to contextualize results: \textbf{Lead-N}, \textbf{Random} and \textbf{TextRank}~\cite{Mihalcea:2004:EMNLP}, all adjusted to meet summary length budgets. To assess consistency across multiple rounds of decoding, we generate 5 summaries per document and target length with temperature $\tau = 0.3$.
We use a zero-shot summarization prompt (\cref{sec:appendix-prompts}).

Before analyzing salience in these models, we validate two key assumptions: \emph{(i)} generated summaries should approximately meet the target length, and \emph{(ii)} longer summaries should expand on shorter ones (``incremental consistency''). Additionally, we analyze how greater $\tau$ affect those criteria.
Our analysis confirms that models largely meet above criteria, with newer and bigger models showing better length control.
Higher $\tau$ results in stable \emph{average} summary length at the corpus level, but greater length variance at the document level (up to 10\% difference), along with a slight decline in incremental consistency (details in \cref{sec:appendix-length-analysis}).

\section{Observed Salience}
\subsection{RQ1: What notion of salience have LLMs learned in different domains?}
\label{sec:results-salience}
To understand how LLMs prioritize different information, we consider average question answerability as a proxy for salience. We show the results for the \emph{RCT} dataset as a representative example in \cref{fig:salience-pubmed}, and include other datasets in \cref{sec:appendix-salience}.

\textbf{Models prioritize information hierarchically.}
We observe a clear hierarchy in how information is prioritized across summary lengths.
For example, fundamental aspects such as the focus of a study (\emph{Q1}), and the condition being treated (\emph{Q3}) consistently achieve higher scores, even at 10-word summaries.
In contrast, more specific and technical information like the study design (\emph{Q10}) and the statistical significance of results (\emph{Q12}) are primarily discussed in longer summaries ($\ge 100$ words).

\textbf{Information frequency is not in itself predictive of salience.}
When we consider how frequently a question is answered by documents in the corpus (leftmost column of \cref{fig:salience-pubmed}), we find that even relatively rare questions such as biological markers and adverse effects (\emph{Q7}/\emph{11}, prevalence 40\%/26\%) maintain a consistent representation in summaries.
This suggests that LLMs do not simply prioritize information based on its frequency in a genre.

\textbf{Summaries progressively get more detailed, and information density differs across models.}
As expected, longer summaries consistently include more information as shown by the higher average answerability (bottom row in \cref{fig:salience-pubmed}).
However, the absolute scores differ across models.
GPT-4o has a notably higher answerability score than Llama 3.1, particularly at longer summaries (0.81 vs. 0.71 at the 200-word length).
Given that both models generate summaries of similar lengths (cf. \cref{fig:length-deviation}), this suggests that GPT-4o conveys information more efficiently.

\begin{figure*}[t]
\includegraphics[width=\textwidth]{figures/agg-pubmed-cl}
\caption{
    Do LLMs share a similar notion of salience?
    Heatmaps show agreement of content-selection at the atomic-claim level (Krippendorff's $\alpha$).
    Dashed bounding boxes indicate models of the same family.
    The diagonal shows self-agreement over multiple generations. Top row: \emph{RCT}, Bottom row: \emph{CL}.
}
\label{fig:agg-pubmed-cl}
\end{figure*}

\subsection{RQ2: Do LLMs of different families and sizes have a similar notion of salience?}
\label{sec:results-model-model-similarity}
We want to understand to what extent different models (e.g., families, scales) have a shared notion of information salience in a given domain.
We define a fine-grained similarity metric that compares models' content-selection decisions.
Intuitively, two models are more similar if their summaries include the same answer claims.
More formally, for each summary length $l$, we compile all atomic claims derived from question-answers along with their entailment labels (cf. \cref{sec:method-questions}). These form a binary vector $\mathbf{v}_{M,l}$ indicating which claims model $M$ includes in its summaries.
We then measure agreement between two models using Krippendorff's alpha: $\alpha(\mathbf{v}_{M_1,l}, \mathbf{v}_{M_2,l})$.
This claim-level agreement metric is stricter than comparing aggregate answerability scores, as it requires models to consistently include or exclude the same claims at each summary length.\footnote{In contrast, similar answerability scores can result from selecting a similar \emph{number} of claims.}
\cref{fig:agg-pubmed-cl} shows the model-model agreement for the \emph{RCT} and \emph{CL} datasets.

\textbf{High agreement across multiple runs suggests models apply salience notion consistently.}
The diagonal in~\cref{fig:agg-pubmed-cl} shows the average pairwise agreement across 5 model runs.
Overall, self-agreement is the highest for \emph{RCT} ($\approx .80$), while it is slightly lower for \emph{CL}, \emph{Astro} and \emph{QMSum} ($\approx .75$).
We observe a slight decline in self-agreement as the summary length increases.
We hypothesize that each document has a tail of medium- to low-salient topics which may or may not be included as the length budget gives more ``freedom'' to the models.

\textbf{Models of the same family or size do \emph{not consistently} have a higher agreement than any other model.}
We next inspect the off-diagonal agreements, comparing one model family with another model family.
Overall, we find that within-family agreement is not consistently higher than cross-family agreement.
While there are isolated cases with a higher within-family agreement (e.g., Llama 3.1 and GPT-4o on \emph{RCT}), this trend cannot be confirmed for all families and datasets.

\textbf{Agreement by summary length and with GPT-4o-mini.}
We observe that certain summary-lengths have higher agreement than others, though the peak is different for each dataset (e.g., agreement on \emph{RCT} is highest for 50 word summaries, whereas on \emph{CL} it peaks at 100 words).
There could be a ``natural'' summary length for each dataset where model more easily agree.
Lastly, we find that more recent and bigger models agree better with GPT-4o-mini which suggests a clear scaling effect and that open-weights models are getting closer in capabilities to large proprietary models (\cref{fig:agg-gpt4}).

\begin{figure}[t]
\includegraphics[width=\linewidth]{figures/agg-gpt4}
\caption{Agreement with GPT-4o-mini, averaged over all datasets and summary lengths.}
\label{fig:agg-gpt4}
\end{figure}

\section{Perceived Salience and Alignment}
In addition to the \emph{observational salience} analysis,
we elicit \emph{perceived salience} by having humans and models directly rate the salience of each question.
This study has two purposes: (1) to understand whether model behavior aligns with human expectations, and (2) to see if the summarization behavior of LLMs can be approximated by direct prompting.

\subsection{Setup}
\paragraph{Human salience annotation.}
We recruited 18 experts across the four domains through our network (3 for \emph{RCT}, and 5 each for \emph{Astro, CL, QMSum}).\footnote{
    Trained physicians (\emph{RCT}), graduate students/faculty (\emph{Astro}), and graduate students (\emph{CL, QMSum}) based in US/Europe.
}
Experts rated the relative salience of each question on a 5-point Likert scale (ranging from 1: least important, to 5: most important).
Annotators were asked to motivate their rating through a brief rationale to encourage thoughtful judgments and to allow post-hoc analysis of their decision-making process.
To establish a shared understanding between annotators of what content a question may elicit, each question is accompanied by an example answer from a randomly drawn document in the domain.
To ensure high annotation quality, we conducted two pilot rounds with four annotators to refine our annotation guidelines (see \cref{sec:appendix-annotation-guidelines}).

Importantly, the human annotations cannot be regarded as a gold standard for salience. The ratings represent how humans \emph{perceive} question salience, which may not be reflective of how humans actually write summaries.

\paragraph{Model-based salience ratings (LLM-perceived).}
We prompt LLMs to directly rate question salience.
The prompt includes the question list for a given domain and instructions that closely mirror the human annotation guidelines to allow for direct comparison (i.e., 5-point Likert scale and rationales).
Each model is prompted 5 times with a shuffled question list to mitigate position bias and to quantify consistency.
See \cref{sec:appendix-prompts} for the full prompt.

\paragraph{Analysis method.}
We use Spearman's rank correlation coefficient ($\rho$) to quantify alignment between three measures: human-perceived salience, LLM-perceived salience (both 5-point Likert scale) and LLM-observed salience (continuous $[0,1]$).\footnote{We take observed salience scores at the 200-words summary length which correlated on average most strongly with human salience. Other scores are explored in~\cref{sec:appendix-salience-score-ablation}.}
For groups with multiple ratings, we report averaged pairwise correlation.

\paragraph{Human correlation.}
We observe that inter-human correlation varies across domains, with meeting summarization (QMSum, $\rho = 0.60$) and RCT abstracts ($\rho = 0.46$) showing a moderate to strong correlation (\cref{tab:annotator-agreement}).
These domains presumably have established conventions about summary content.
In contrast, correlation is weak on summarization of related work sections (CL, $\rho = 0.26$) and discussion sections (Astro, $\rho = 0.16$).
Documents in these domains may vary significantly in the type of content they present (i.e., certain questions may be more relevant to theoretical vs. empirical papers).
While our annotation protocol aims to control for this aspect through the example answers by question, there remains annotator subjectivity related to their personal interests.

\begin{table}[t]
\small
\centering
\begin{tabular}{lrrrr}
\toprule
\bfseries Dataset & \bfseries Questions & \bfseries Raters & \bfseries $\rho$ & \bfseries Std. \\
\midrule
QMSum & 10 & 5 & 0.60 & 0.18 \\
RCT & 21 & 3 & 0.46 & 0.06 \\
CL & 14 & 5 & 0.26 & 0.29 \\
Astro & 13 & 5 & 0.16 & 0.44 \\
\bottomrule
\end{tabular}

\caption{Inter-annotator correlation (Spearman's $\rho$) for question salience rating.}
\label{tab:annotator-agreement}
\end{table}


\subsection{Results}
\label{sec:results-introspection}
To understand if LLMs can reliably rate question salience, we study three conditions.
First, as a reference point, we measure consistency of the observational and perceived salience measures estimated over 5 model runs (LLM-observed, LLM-perceived).
Second, we study the correlation of LLM-perceived and LLM-observed to measure if models' explicit ratings align with their summarization behavior (RQ3).
Third we correlate LLM-derived salience in human perceived salience (RQ4).
We report results for the three conditions in \cref{tab:results-rater-agreement} and provide qualitative examples in \cref{tab:results-examples}.

\begin{table*}[t]
\small
\centering
\setlength{\tabcolsep}{4pt}
\begin{tabular}{lrrrrrrrp{0.5cm}r}
\toprule
\textbf{Measure}  & \bfseries Random & \bfseries OLMo & \bfseries Mixtral & \bfseries Llama$^{3.1}_{8b}$ & \bfseries Llama$^{3.1}_{70b}$ & \bfseries 4o-mini & \bfseries 4o && \bfseries Average \\
\midrule
\multicolumn{10}{c}{\texttt{Consistency of Salience Estimates}}\\
\emph{LLM-perceived}  & -0.05 & 0.20 & 0.54 & 0.37 & 0.71 & 0.73 & \bfseries 0.76 &&  0.45 \\
\emph{LLM-observed} & 0.92 & \bfseries 0.99 & \bfseries 0.99 & 0.98 & \bfseries 0.99 & 0.98 & 0.98 && 0.98 \\\addlinespace
\multicolumn{10}{c}{\texttt{Correlation of Salience Estimates}}\\
\emph{LLM-perceived vs. -observed}  & 0.02 & 0.12 & 0.37 & 0.36 & 0.47 & \bfseries 0.56 & 0.50 && 0.33 \\\addlinespace
\multicolumn{10}{c}{\texttt{Correlation of Model and Human Salience}}\\
\emph{LLM-perceived vs. Human} & 0.06 & 0.16 & 0.41 & 0.31 & 0.47 & 0.51 & \bfseries 0.53 && 0.32 \\
\emph{LLM-observed vs. Human}  & 0.22 & 0.26 & 0.32 & 0.34 & \bfseries 0.35 & 0.34 & 0.25 && 0.24 \\
\bottomrule
\end{tabular}
\caption{Pearson rank correlation coefficient between different salience estimates averaged over all datasets. Individual values per dataset in \cref{tab:results-rater-agreement-full}.
Row-wise maximum in \textbf{bold}.}
\label{tab:results-rater-agreement}
\end{table*}

\begin{table*}[t]
    \fontsize{8.5pt}{8.5pt}\selectfont
    \begin{tabular}{@{}p{0.17\textwidth}rp{0.23\textwidth}p{0.45\textwidth}@{}}
    \toprule
        \textbf{Question} &
        $L_o$/$L_p$/$H$ &
        \faRobot\ \textbf{Rationale LLM-perceived} &
        \faUsers\ \textbf{Rationales Human}\\
    \midrule
    \textbf{(\#1, RCT \faUserMd)}
        What significant statistical results are reported? & 2.5/4.0/5.0 &
        Statistical results are essential for assessing the validity and reliability of the findings. &
        (P\textsubscript{\{1,2,3\}} = 5) Main outcome that physicians are looking for; very important to be included as measure of success; without it reporting is lacking \\

    \textbf{(\#2, Astro \faStar)}
        What broader context or field does the study contribute to or address?
        & 4.6/3.4/2.4
        & Understanding the broader context helps situate the study within the larger scientific discourse.
        & (P\textsubscript{\{2,5\}} = 1) not relevant, experts would fill context; (P\textsubscript{1} = 3) inferable from another question; (P\textsubscript{3} = 4) context is important although not always addressed; (P\textsubscript{4} = 3) Situating findings within literature, existing models or theories is important;\\

    \textbf{(\#3, RCT \faUserMd)}
        What was the main intervention used in the study? & 4.2/4.8/5.0 &
        The main intervention is central to understanding the study's focus and findings. &
        (P\textsubscript{\{1,2,3\}} = 5) determines whether study influences decision making; part of PICO therefore fundamental; essential for communicating purpose/focus of the trial \\
    \bottomrule
    \end{tabular}

    \caption{Example questions, salience scores by LLM-observed ($L_o$, rescaled to 1-5), LLM-perceived ($L_p$), humans ($H$) and summarized rationales. Additional examples in \cref{tab:results-examples-part2}.}
    \label{tab:results-examples}

\end{table*}

\paragraph{RQ3: When models introspect, does their perceived notion of salience align with their summarization behavior?}
LLMs have strong and consistent \emph{implicit} notions of salience, but
they are unreliable when explicating these preferences in rating tasks. We detail these observations below.

\textbf{Observational salience is highly stable.}
We find that observational question salience leads to highly stable scores for all models ($\rho \ge 0.96$).
This suggests that LLMs' underlying summarization process is highly deterministic despite the stochastic nature of language models.
Also, it suggests that our proposed approach is a reliable tool for analyzing model behavior.

\textbf{Models fail to have consistent perceived salience.}
We find that the consistency of direct salience ratings varies significantly for all models and datasets.
Generally, strong instruction-following models have more consistent perceived salience than weaker models (avg. $\rho$ ranges from 0.20 for OLMo to 0.76 for GPT-4o).
This finding mirrors recent results in the LLM-as-a-judge literature which demonstrated instability in ratings due to various factors including position bias~\cite{Wang:2024:ACL,Stureborg:2024:arXiv}.

\textbf{Perceived $\neq$ observed salience.}
Lastly, we find only a weak to moderate correlation between perceived and observed salience (highest: avg. $\rho = 0.56$ for GPT-4o-mini, lowest: $\rho = 0.12$ for OLMo).
Again, stronger instruction-following models show higher correlations, indicating a clear scaling effect.
This gap echoes broader findings where generative abilities may not reflect an underlying understanding in models~\cite{West:2024:ICLR}.

\paragraph{RQ4: To what extent does model salience align with human perceived salience?}
\label{sec:results-human-alignment}
We find that both LLM-salience estimates only show a weak to moderate correlation with human salience perception.
Direct rating for question salience correlates more than observed salience (highest LLM-perceived: avg. $\rho = 0.53$ for GPT-4o, highest LLM-observed: avg. $\rho = 0.35$ for Llama 3.1 70B).
Weak correlation between models and humans holds for all dataset, also those where humans agree more strongly among themselves (\cref{tab:results-rater-agreement-full}).

Users of LLMs should carefully consider if the models are appropriate for their summarization task, or provide explicit signals about content priority through prompts or during model training.

\input{20-related-work}
\section{Conclusion}
% We examined the alignment between the user and the AI with respect to domain expertise. Implicitly we are asking whether there is an optimal alignment. Perhaps the user and the model should operate at the same level of expertise so that the LLM can convey ideas in a manner most likely to be understood by the user. Or maybe the LLM should be more expert than the user, but only by an amount such that the LLM responses to user queries are both understandable and novel to the user.
% Using a random sample of real-world interactions with an LLM-based chat agent, we find that the user is lower in expertise than the LLM in the majority ($80.05\%$) of conversations. This makes sense in that most users are likely to be non-experts, while the model should be more expert in order to provide value to the user. Our results show that generally it is preferable for the LLM to be quite expert regardless of the user's level of expertise, with lower than proficient responses from the LLM generally leading to decreased user satisfaction scores. The one exception may be that lower expertise users engage the LLM in conversations of equal or even greater length when the LLM is of lower expertise.
% We also saw $4.23\%$ of conversations where the user was underestimated, which had a significant impact on the user experience: satisfaction, engagement, and the complexity of the interaction were all lower when the level of gauged expertise of the user was lower than the actual user expertise. Given a goal of enabling rich, complex experiences, and the already widespread and growing use of LLM systems, this highlights the importance of the LLM always avoiding underestimating users, whether through the system prompt, or through in-session adaptation.
We examined the alignment between \abr{LLMs} and users along a dimension relevant to the user experience: expertise. We show that the \abr{LLM}'s expertise is largely proficient or expert, which correlated with positive user satisfaction and exceeded user expertise in a majority of the cases. Further, underestimating the user's level of expertise correlated with lower and even negative user satisfaction, with the effect stronger for more complex tasks. Users tended to engage more, however, when the \abr{LLM} responded at a level of expertise similar to their own, suggesting that the system strike a balance between generally high expertise which is liked by all users and matched expertise to best engage users. Future work may explore intervention strategies to strike this balance and mitigate obvious cases of user underestimation in real time.
%Future work may explore intervention strategies to mitigate cases of expertise mismatch between users and \abr{LLMs} and user underestimation to ensure rich, complex and satisfactory user experiences. By ensuring that \abr{LLMs} better understand human intent and provide contextually  and expertise appropriate responses, we can expect these systems to evolve into even more capable and user-friendly assistants, facilitating everything from decision-making processes to knowledge creation. Future work may explore intervention strategies to mitigate cases of expertise mismatch between users and \abr{LLMs} and user underestimation to ensure rich, complex and satisfactory user experiences.
\section*{Limitations}
We only consider zero-shot prompting with temperature-based decoding to generate summaries. While these settings are common defaults for LLM users, it is conceivable that different prompting styles (e.g., chain-of-density) or decoding methods influence salience patterns. Future work should explore how these techniques affect salience, particularly in adjacent information-seeking tasks such as query-based summarization.

While our experiments cover diverse disciplines (medicine, astrophysics, computational linguistics, and meetings) and discourse types (structured writing, academic discourse, and dialogue), the texts are primarily technical. Since our framework is designed to be domain-agnostic, we believe it is an exciting direction for future work to explore less technical genres such as fiction~\cite{Kim:2024:COLM}.

Our user study assumed a uniform background and interests among participants, which is a simplification of practical applications. Additionally, the specialized nature of two tasks (i.e., summarizing related work and discussion sections) may have contributed to variability in responses, as even domain experts may not have strong priors on how these texts should be summarized. Future work could explore how differences in expertise and prior knowledge shape perceptions of salience.


\section*{Acknowledgments}
We thank Hsin-Pei Chen, Khawla Elhadri, Arya Farahi, Juan P. Farias, Cheng Han Hsieh, Sebastian Joseph, Ramez Kouzy, Michael Muzinich, Van Bach Nguyen, Juan Diego Rodriguez, Paul Torrey, Manya Wadhwa, Barry Wei, and Paul Youssef for their participation in the salience annotation study.
We also thank Dennis Aumiller and Philippe Laban for early feedback on this research.
This work was partially supported by the US National Institutes of Health (NIH) grant 1R01LM014600-01,
and the US National Science Foundation grants IIS-2107524, IIS-2145479, and Cooperative Agreement 2421782 and Simons Foundation MPS-AI-00010515 (NSF-Simons AI Institute for Cosmic Origins\footnote{CosmicAI, \url{https://www.cosmicai.org/}}).

\bibliography{bibliography}

\appendix
\section{Length-instruction Following}
\label{sec:appendix-length-analysis}
We analyze to what extent length-controlled summarization is a consistent probe for content selection.
Ideally, we expect the following behavior of summarization models: (1) the generated summary length matches approximately the target length, and (2) as we increase the length budget, summaries should provide all content of the shorter version in addition to expanding on it.
We define two measures for these desiderata.

\paragraph{Target length ratio (TLR).}
We quantify the length deviation of a generated summary ($s_{l}$) from the target word count ($l$) as follows:
%
\begin{equation}
\text{\textbf{TLR}}(s_{l}) = \frac{|s_{d,l}|}{l}.
\end{equation}
%
Where $|\cdot|$ is the summary length (whitespace separated tokens). A value of 1 indicates perfect length match, while values greater or smaller than 1 indicate over- or under-generation, respectively.

\paragraph{Incremental consistency (IC).}
Longer summaries should contain a proper superset of claims found in the adjacent shorter version.
Formally, for each document $d$ and topic $t$ recall that we have a set of atomic claims $A_t$ (\cref{sec:method-questions}).
We first identify the set of claims that are entailed at least once across any summary length:
%
\begin{equation*}
    A_{\text{entailed}}(d,t) = \{a \in A_t \mid \exists l \in L, e(a,s_{d,l}) = 1\},
\end{equation*}
%
where $e$ is an NLI model indicating whether claim $a$ is entailed by summary $s_{d,l}$ of length $l$.
Next, we determine if a claim is included consistently across increasing summary lengths (monotonicity condition).
%
\begin{equation*}
    e(a,s_{d,l_1}) \leq e(a,s_{d,l_2}) \; \forall l_1 < l_2
\end{equation*}
%
We then define the set of consistent claims where this condition holds:
%
\begin{equation*}
\begin{aligned}
    A_{\text{consistent}}(d,t) =\,&\{a \in A_{\text{entailed}}(d,t) \\
    \quad& \mid \text{\small monotonicity holds} \, \forall l \in L \}.
\end{aligned}
\end{equation*}
%
Finally, the overall incremental consistency for summaries of a corpus $D$ is given as the fraction of consistent claims:
%
\begin{equation}
\text{\textbf{IC}}(D) = \frac{
        \sum_{d \in D} \sum_{t \in T} |A_{\text{consistent}}(d,t)|
    }{
        \sum_{d \in D} \sum_{t \in T} |A_{\text{entailed}}(d,t)|
    }.
\end{equation}
%
This metric ranges from 0 to 1, where 1 indicates perfect monotonicity (longer summaries always include all information found in shorter ones).

\begin{figure}[t]
\includegraphics[width=\linewidth]{figures/tlr}
\caption{Distribution of target length ratios over all generated summaries (aggregating lengths and datasets).}
\label{fig:length-deviation}
\end{figure}

\noindent\textbf{Do models meet the target length?}
We find that all models generally undershoot the length target (\cref{fig:length-deviation}).
However, more recent models match the target length more closely and consistently, showing a clear scaling effect.
The best performing models are Llama 3.1 and GPT-4o, while OLMo is unable to follow length-instructions, presumably because this was not part of the instruction tuning data.
Surprisingly, we do not find substantial differences across datasets.
This suggests that the ability of models to follow length-instructions is mostly invariant to the input document length, even if they are considerably long (e.g., meeting transcripts).
See \cref{fig:tlr-stratified} for an analysis of summary length stratified by dataset and target length.

\noindent\textbf{How incrementally consistent are summaries?}
We report the average incremental consistency by dataset and model in \cref{fig:incremental-consistency}.
We observe that all models are substantially more consistent than the random summarization baseline.
Furthermore, incremental consistency decreases with more difficult datasets, likely because there is more freedom on what content to include in a summary.
Similar to the ability of following length instructions, we observe a scaling effect where stronger models have a higher incremental consistency.

\begin{figure}[t]
\includegraphics[width=\linewidth]{figures/ic}
\caption{Incremental consistency by model and dataset.}
\label{fig:incremental-consistency}
\end{figure}

\noindent\textbf{Influence of temperature sampling.}
The main results in this paper are obtained with a temperature of $\tau = 0.3$.
To assess how temperature affects summary length and incremental consistency, we perform a temperature sweep on the RCT dataset for all open-weights models (20 settings in $[0,1]$).
Surprisingly, higher temperatures do not affect the \emph{average} summary length on a dataset-level, but lead to greater variance at the document level (up to 10\% length difference between generations, \cref{fig:temperature-length}).
Furthermore, higher temperatures lead to a slight decline in incremental consistency for all models that adequately follow length instructions (a drop of 1\% to 9\%, \cref{fig:temperature-ic}).

\noindent\textbf{Summary.}
Overall, we find that strong models are able to follow length-instructions and that they consistently expand the summary content with increasing length budgets.
As our salience analysis assumes this behavior of models, it may be less reliable for weaker models (OLMo, Mistral, Llama~2).

\begin{figure}[t]
\includegraphics[width=\linewidth]{figures/temperature-ic}
\caption{Influence of temperature on incremental consistency.}
\label{fig:temperature-ic}
\end{figure}

\section{Salience Score Ablation}
\label{sec:appendix-salience-score-ablation}
We analyze how different salience scores derived from the CSM correlate with human salience.
Recall that the $\text{CSM}(D)_{t,l}$ tracks the average answerability of question $t \in T$ at summary length $l \in L = \{10,20,50,100,200\}$.
We take raw salience scores at each summary length in addition to question-wise aggregations.
Intuitively, questions which are more answerable at shorter summaries score higher under the aggregated scheme.
Formally, we aggregate scores as follows:
%
\begin{equation*}
\text{CSM}_{\text{agg}}(D)_{t} = \frac{\sum_{l \in L} w_l \cdot \text{CSM}(D)_{t,l}}{\sum_{l \in L} w_l},
\end{equation*}
%
where $w_l$ is a weighting term. We experiment with three weighting functions: uniform ($w_l = 1$), reciprocal length ($w_l = 1/l$), and logarithmic decay ($w_l = 1/\log(1 + l)$).
\cref{fig:salience-score-correlation} shows the Spearman rank correlation coefficient ($\rho$) with human salience for each salience score.
Overall, on \emph{RCT} and \emph{Astro} we find that all salience scores correlate similarly with human salience ratings, while on \emph{CL} and \emph{QMSum} the 200 words salience score correlates most strongly.

\begin{figure}[t]
\includegraphics[width=\linewidth]{figures/salience-score-correlation}
\caption{Correlation of different salience scores with human salience. Here we aggregate over all LLMs which showed similar trends.}
\label{fig:salience-score-correlation}
\end{figure}

\section{Salience Analysis}
\label{sec:appendix-salience}
The corpus-level salience analysis for PubMed, Astro, CL, and QMSum is given in \cref{fig:salience-pubmed}, \cref{fig:salience-astro-ph}, \cref{fig:salience-cs-cl}, and \cref{fig:salience-qmsum}, respectively. We also provide a fully-worked example of the content salience analysis in \cref{fig:worked-example}.

\section{Responsible NLP Considerations}
\paragraph{Compute Requirements.} Experiments were conducted on NVIDIA A100 80GB GPUs, requiring approximately 20 GPU hours per dataset, and an additional 360 GPU hours for the temperature sweep on the RCT dataset, totaling 440 GPU hours.
We ran inference using \textsc{vllm}.\unfootnote{All URLs accessed 2025-02-15.}\footnote{\rurl{docs.vllm.ai}}
GPT-4o models were accessed through the OpenAI API with inference costs $\leq 100\$$.

\paragraph{Salience Annotation Study.} Participants joined on a volunteer basis, gave informed consent and agreed that their annotations will be shared in anonymized form in the paper repository. According to our institutional policies, this study did not require institutional review board (IRB) approval.

\paragraph{Data Licensing.}
We obtain RCT abstracts in accordance with fair use principles through the PubMed Entrez API.\footnote{\rurl{www.ncbi.nlm.nih.gov/home/develop/api/}}
Related work sections of CL and Astro papers were collected via the arXiv API.\footnote{\rurl{info.arxiv.org/help/api/index.html}} While the majority of papers on arXiv is published under the arXiv license\footnote{\rurl{arxiv.org/licenses/nonexclusive-distrib/1.0/license.html}} retaining \emph{copyright} with the original author(s), the \emph{use} of paper contents for research is explicitly granted and encouraged in the arXiv API terms \& conditions.\footnote{\rurl{info.arxiv.org/help/api/tou.html}}
We reused meeting transcripts from QMSum~\cite{Zhong:2021:NAACL}.\footnote{\rurl{github.com/Yale-LILY/QMSum}} All meeting transcripts are under an open use license, such as CC BY 4.0\footnote{\rurl{creativecommons.org/licenses/by/4.0/legalcode}} (academic meetings\footnote{\rurl{groups.inf.ed.ac.uk/ami/icsi/license.shtml}} and product meetings\footnote{\rurl{groups.inf.ed.ac.uk/ami/corpus/license.shtml}}) or Open Government Licence\footnote{\rurl{www.nationalarchives.gov.uk/doc/open-government-licence/version/3/}} (Parliament Commitee meetings).


\begin{figure*}[t]
\includegraphics[width=\textwidth]{figures/temperature-length}
\caption{Influence of temperature on generated summary length. \textbf{Left:} target length-ratio. \textbf{Center:} ``within-document length variance'' calculated as the mean deviation from the average summary length of 5 summaries for the same document (MAD). MAD is normalized to be comparable across length targets. \textbf{Right:} zoomed version.}
\label{fig:temperature-length}
\end{figure*}

\begin{table*}[t]
\small
\centering
\setlength{\tabcolsep}{4pt}
\begin{tabular}{llrrrrrrrr}
\toprule
\textbf{Measure} & \textbf{Dataset} & \bfseries Random & \bfseries OLMo & \bfseries Mixtral & \bfseries Llama$^{3.1}_{8b}$ & \bfseries Llama$^{3.1}_{70b}$ & \bfseries 4o-mini & \bfseries 4o & \bfseries Average \\
\midrule
\multicolumn{10}{c}{\texttt{Consistency of Salience Estimates}}\\\midrule
\multirow[c]{5}{0.14\textwidth}{\emph{LLM-perceived}} & RCT & -0.06 & 0.34 & 0.48 & 0.29 & 0.61 & 0.75 & \bfseries 0.80 & 0.43 \\
\bfseries  & Astro & 0.02 & 0.07 & 0.41 & 0.56 & 0.64 & 0.76 & \bfseries 0.86 & 0.46 \\
\bfseries  & CL & -0.10 & -0.05 & 0.65 & 0.29 & \bfseries 0.73 & 0.70 & 0.57 & 0.35 \\
\bfseries  & QMSum & -0.07 & 0.42 & 0.61 & 0.36 & \bfseries 0.87 & 0.72 & 0.80 & 0.55 \\\cmidrule{2-10}
\bfseries  & Avg. & -0.05 & 0.20 & 0.54 & 0.37 & 0.71 & 0.73 & \bfseries 0.76 & 0.45 \\\midrule
\multirow[c]{5}{0.14\textwidth}{\emph{LLM-observed}} & RCT & 0.94 & 0.98 & \bfseries 0.99 & \bfseries 0.99 & \bfseries 0.99 & \bfseries 0.99 & \bfseries 0.99 & 0.98 \\
\bfseries  & Astro & 0.91 & 0.99 & \bfseries 1.00 & 0.97 & 0.99 & 0.97 & 0.97 & 0.97 \\
\bfseries  & CL & 0.96 & \bfseries 0.99 & \bfseries 0.99 & 0.96 & \bfseries 0.99 & \bfseries 0.99 & 0.97 & 0.98 \\
\bfseries  & QMSum$^\dagger$ & 0.87 & --- & 0.99 & 0.99 & \bfseries 1.00 & 0.99 & --- & 0.97 \\\cmidrule{2-10}
\bfseries  & Avg. & 0.92 & \bfseries 0.99 & \bfseries 0.99 & 0.98 & \bfseries 0.99 & 0.98 & 0.98 & 0.98 \\\midrule
\multicolumn{10}{c}{\texttt{Correlation of Salience Estimates}} \\\midrule
\multirow[c]{5}{0.14\textwidth}{\emph{LLM-perceived vs. LLM-observed}} & RCT & -0.06 & 0.10 & 0.25 & 0.25 & 0.37 & 0.41 & \bfseries 0.51 & 0.30 \\
\bfseries  & Astro & 0.11 & 0.09 & 0.31 & 0.56 & 0.50 & \bfseries 0.65 & 0.58 & 0.36 \\
\bfseries  & CL & -0.08 & 0.16 & 0.44 & 0.47 & 0.38 & \bfseries 0.58 & 0.41 & 0.29 \\
\bfseries  & QMSum$^\dagger$ & 0.11 & --- & 0.46 & 0.16 & \bfseries 0.63 & 0.60 & --- & 0.42 \\\cmidrule{2-10}
\bfseries  & Avg. & 0.02 & 0.12 & 0.37 & 0.36 & 0.47 & \bfseries 0.56 & 0.50 & 0.33 \\\midrule
\multicolumn{10}{c}{\texttt{Correlation of Model and Human Salience}} \\\midrule
\multirow[c]{5}{0.14\textwidth}{\emph{LLM-perceived vs. Human}} & RCT & -0.03 & 0.22 & 0.38 & 0.34 & 0.49 & 0.48 & \bfseries 0.56 & 0.33 \\
\bfseries  & Astro & 0.07 & 0.12 & 0.30 & 0.31 & 0.27 & \bfseries 0.45 & 0.44 & 0.27 \\
\bfseries  & CL & 0.06 & -0.03 & 0.41 & 0.22 & \bfseries 0.48 & 0.44 & 0.46 & 0.24 \\
\bfseries  & QMSum & 0.14 & 0.34 & 0.54 & 0.36 & 0.62 & \bfseries 0.67 & \bfseries 0.67 & 0.45 \\\cmidrule{2-10}
\bfseries  & Avg. & 0.06 & 0.16 & 0.41 & 0.31 & 0.47 & 0.51 & \bfseries 0.53 & 0.32 \\\midrule
\multirow[c]{5}{0.14\textwidth}{\emph{LLM-observed vs. Human}} & RCT & 0.31 & 0.28 & 0.27 & 0.25 & 0.25 & \bfseries 0.34 & 0.24 & 0.25 \\
\bfseries  & Astro & 0.11 & 0.25 & 0.27 & 0.29 & \bfseries 0.31 & 0.26 & 0.25 & 0.22 \\
\bfseries  & CL & \bfseries 0.30 & 0.23 & 0.23 & 0.24 & 0.26 & 0.25 & 0.24 & 0.21 \\
\bfseries  & QMSum$^\dagger$ & 0.16 & --- & 0.53 & 0.58 & \bfseries 0.59 & 0.51 & --- & 0.31 \\\cmidrule{2-10}
\bfseries  & Avg. & 0.22 & 0.26 & 0.32 & 0.34 & \bfseries 0.35 & 0.34 & 0.25 & 0.24 \\
\bottomrule
\end{tabular}
\caption{Pearson rank correlation coefficient between different salience estimates split by dataset.
The row-wise maximum is \textbf{bolded}. $^\dagger$Results for QMSum not available due to limited context window (OLMo) and budget constraints (GPT-4o).}
\label{tab:results-rater-agreement-full}
\end{table*}


\begin{figure*}[t]
\centering
\begin{subfigure}[b]{\textwidth}
\includegraphics[width=\textwidth]{figures/tlr-by-dataset}
\caption{Distribution of target length ratios over all generated summaries stratified by dataset.}
\label{fig:tlr-dataset}
\end{subfigure}
\vspace{0.5cm}
\begin{subfigure}[b]{\textwidth}
\includegraphics[width=\textwidth]{figures/tlr-by-length}
\caption{Distribution of target length ratios over all generated summaries stratified by target summary length.}
\label{fig:tlr-length}
\end{subfigure}

\caption{Analysis of length-instruction following. The target length ration (TLR) indicates to what extent models match the provided length. A value of 1 indicates perfect length match, while values greater or smaller than 1 indicate over- or under-generation, respectively.}
\label{fig:tlr-stratified}
\end{figure*}


\begin{figure*}[t]
\includegraphics[width=\textwidth]{figures/salience-astro-ph}
\caption{Corpus-level content salience map for \emph{Astro} summaries by four methods.}
\label{fig:salience-astro-ph}
\end{figure*}

\begin{figure*}[t]
\includegraphics[width=\textwidth]{figures/salience-cs-cl}
\caption{Corpus-level content salience map for \emph{CL} summaries by four methods.}
\label{fig:salience-cs-cl}
\end{figure*}

\begin{figure*}[t]
\includegraphics[width=\textwidth]{figures/salience-qmsum}
\caption{Corpus-level content salience map for \emph{QMSum} summaries by four methods.}
\label{fig:salience-qmsum}
\end{figure*}

\begin{figure*}[t]
\centering

    \begin{subfigure}[b]{\textwidth}
    \includegraphics[width=\textwidth]{figures/agg-astro}
    \caption{Model similarity for \emph{Astro}.}
    \label{fig:agg-astro}
    \end{subfigure}

    \vspace{0.5cm}

    \begin{subfigure}[b]{\textwidth}
    \includegraphics[width=\textwidth]{figures/agg-qmsum.pdf}
    \caption{Model similarity for \emph{QMSum}.}
    \label{fig:agg-qmsum}
    \end{subfigure}

\caption{Do models share a similar notion of salience? Heatmaps show agreement on topic inclusion by summary length (Krippendorff's alpha calculated column-wise for the CSM pairs).}
\label{fig:agg-astro-qmsum}
\end{figure*}

\begin{table*}[t]
    \fontsize{8.5pt}{8.5pt}\selectfont
    \begin{tabular}{@{}p{0.17\textwidth}rp{0.23\textwidth}p{0.45\textwidth}@{}}
    \toprule
        \textbf{Question} &
        $L_o$/$L_p$/$H$ &
        \faRobot\ \textbf{Rationale LLM-perceived} &
        \faUsers\ \textbf{Rationales Human}\\
    \midrule
    \multicolumn{4}{c}{\emph{continued from \cref{tab:results-examples}}} \\
    \textbf{(\#4, RCT \faUserMd)}
        What were the comparative results between intervention and control groups? & 3.2/4.6/5.0 &
        Comparative results are key to understanding the effectiveness of the intervention. &
        (P\textsubscript{\{1,2,3\}} = 5) Brief comparison between groups allows for readers to quickly assess statistical significance; Comparative results are one of the primary outputs reflecting effectiveness; essential for understanding the findings. \\

    \textbf{(\#5, RCT \faUserMd)}
        What implications or future recommendations did the study suggest based on its findings? & 4.6/3.6/2.0 &
        Implications and recommendations are important for understanding the study's impact and future research directions. &
        (P\textsubscript{1} = 1) This is a job for the reader to decide for himself or herself; (P\textsubscript{2} = 3) helps the reader understand the paper. Useful for at the end of a summary. (P\textsubscript{3} = 2) While helpful, they are not necessarily always based on the opinion of the authors; would have the reader get to that conclusion themselves.\\

    \textbf{(\#6, RCT \faUserMd)}
        What is the main outcome or effect observed? & 4.6/5.0/5.0 &
        The main outcome is critical to understanding the study's primary conclusion. &
        (P\textsubscript{\{1,2,3\}} = 5) determines whether a paper can be used to influence decision making; primary information when reading summary \\

    \textbf{(\#7, RCT \faUserMd)}
        How were the participants or subjects of the study selected and divided? & 3.7/3.2/2.3 &
        Participant selection and division are important for assessing the study's validity. &
        (P\textsubscript{1} = 2) This helps a reader determine external validity of a study;
        (P\textsubscript{2} = 1) Not necessary as fairly standardized;
        (P\textsubscript{3} = 4) Crucial to evaluate the study's validity and relevance \\

    \textbf{(\#8, Astro \faStar)}
        What limitations or considerations are noted by the study? & 1.1/3.6/2.3 &
        Limitations are crucial for interpreting the study's reliability and potential biases. &
        (P\textsubscript{1} = 3) important but best reserved for readers who choose to critically examine the study in its totality; (P\textsubscript{2} = 1) Typically addressed in the discussion section of a paper, but is not necessary for a summary; (P\textsubscript{3} = 3) would note if I had the space, however, I would also implore the readers to think about limitations themselves. \\

    \textbf{(\#9, Astro \faStar)}
        How do the findings relate to existing models or theories? & 1.6/4.0/3.8 &
        This is important for understanding the study's impact on current scientific understanding. &
        (P\textsubscript{1} = 4) I think this is important to describe the finding in context of the research question, initial hypothesis, or a theory.  (P\textsubscript{2} = 3) This is important for the broader context of the study, however it is more important to present the study results in a clear way. Only after that is clear I'd compare to other studies.(P\textsubscript{3} = 5) This forms the main body of a discussion section. (P\textsubscript{4} = 3) Situating the findings within the literature by relating to existing models or theories is important.(P\textsubscript{3} = 4) Comparing with existing literature is very important for the discussion section.\\

    \textbf{(\#10, Astro \faStar)}
        What are the main findings of the study? & 3.5/5.0/4.8 &
        The main findings are the core of the study and must be included in any summary. &
        (P\textsubscript{1} = 5) key takeaway from the paper and should be included regardless of what the paper is about (P\textsubscript{2} = 5) most important information of the summary (P\textsubscript{3} = 4) The main findings should be briefly addressed in the summary of a discussion for the reader's quick follow-up (P\textsubscript{4} = 5) The main findings of the study, along with the main focus, form the two most important elements of an article summary. (P\textsubscript{5} = 5) Important to state the main findings and then discuss them in details. \\

    \textbf{(\#11, Astro \faStar)}
        What specific challenges or limitations does the study address or identify? & 1.6/3.2/2.6 &
        Understanding the challenges or limitations provides context for the study's reliability and areas for improvement. &
        (P\textsubscript{1} = 1) I most likely do not include challenges and limitations. These examples focused on the future needs not an existing open question. The focus will be on the findings in the context of a hypothesis, conjecture, or a theory. (P\textsubscript{2} = 1) Level of detail that a reader would need only if interested in full paper. Some challenges can be identified if the methods and scope of the paper are summarized clearly. (P\textsubscript{3} = 5) This forms the main body of a discussion section. (P\textsubscript{4} = 2) depends upon the significance of those challenges or limitations (P\textsubscript{5} = 4) Identify the limitations and challenges of the study is very important \\
    \bottomrule
    \end{tabular}
    \caption{Example questions, salience scores by LLM-observed ($L_o$, rescaled to 1-5), LLM-perceived ($L_p$), humans ($H$) and summarized rationales.}
    \label{tab:results-examples-part2}
\end{table*}

\clearpage
\onecolumn
\section{LLM Prompts}
\label{sec:appendix-prompts}
This section provides all prompts used throughout the experiments. Summarization (\cref{lst:summarization,lst:summarization-meetings}), question generation (\cref{lst:qg}), question answering (\cref{lst:qa}), answer claim  splitting (\cref{lst:claim-split}), and introspection (\cref{lst:introspection}).

\codeboxinput[label=lst:summarization]{Summarization prompt}{prompts/summarization.txt}
\codeboxinput[label=lst:summarization-meetings]{Summarization prompt for meeting transcripts}{prompts/summarization-meetings.txt}
\codeboxinput[label=lst:qg]{Question generation prompt}{prompts/qg.txt}
\codeboxinput[label=lst:qa]{Question answering prompt}{prompts/qa.txt}
\codeboxinput[label=lst:claim-split]{Claim splitting prompt}{prompts/claim-splitting.txt}
\codeboxinput[label=lst:introspection]{Introspection prompt}{prompts/introspection.txt}

\twocolumn
\section{Question Salience Annotation Guidelines}
\label{sec:appendix-annotation-guidelines}

\paragraph{Motivation.}
When summarizing long texts, we must consciously decide what information to include or exclude from a summary. These decisions are grounded in a notion of information salience, or how important we consider the information for our intended audience. We study this phenomenon in the context of automatic text summarization systems. Specifically, we aim to understand how well these systems replicate the judgments of domain experts regarding what information is most relevant.\looseness=-1

\paragraph{Task.}
Imagine you are asked to \textbf{summarize a paper describing the results of a randomized controlled trial (RCT)} for a typical reader in this field. The summary should provide enough context to stand alone, since the reader will only see your summary and no other parts of the paper. Furthermore, the summary length is constrained, requiring you to think about what content to prioritize. In this study, we frame content as questions that a summary could answer.

Ask yourself: \textbf{What are some key questions you want the summary to answer?} Your task is to rate the relative importance of a list of questions on the following scale.
%
\begin{todolist}[noitemsep]
\item (1) Least important; I would exclude this information from a summary.
\item (2) Low importance; I would include this information if there is room.
\item (3) Medium importance; I would probably include this information.
\item (4) High importance; I would definitely include this information.
\item (5) Most important; One of the first questions to be answered in the summary.
\end{todolist}
%
\paragraph{Rationale.}
For each rating, please provide a brief (1-sentence) rationale explaining your decision or highlighting any considerations or uncertainties.

\paragraph{Example answers.}
To give you a feeling for the kind of content a question might elicit, all questions have an illustrative answer sourced from a randomly chosen document (= RCT paper). Please keep the following in mind:
%
\begin{itemize}[noitemsep]
    \item \emph{Answer length} does not determine the question's importance.
    \item \emph{Phrasing and selection.} The precise answer phrasing can be different in the summary, and not all answer content must appear in the summary.
    \item \emph{Overlap.} Some questions may elicit overlapping answers. Therefore, focus on the essence of each question. Remember that in an actual summary, overlapping answer information would only be stated once, so don't worry about it (see below).
    \item \emph{Relevance.} The questions are answerable with most documents in this genre. Do your rating on the assumption that the document talks about this information.
\end{itemize}
%
\paragraph{Suggested process.}
%
\begin{enumerate}[noitemsep]
    \item Read all questions first.
    \item Identify questions that seem most/least important, and rate these as “anchor points.”
    \item Then, rate the remaining questions.
\end{enumerate}
%
Finally, there are no right or wrong ratings. Use your best judgment and intuition. Thank you for participating!

\subsection*{Appendix: Example of overlapping answers.}
Consider questions \texttt{Q1-Q3} below. Each question asks for a distinct unit of information, but the answer of \texttt{Q3} overlaps with the answer of \texttt{Q1} and \texttt{Q2}. The overlapping information is highlighted in \hlorange{orange} while the \emph{essence of the question} is highlighted in \hlgreen{green}. Base your rating on the essence of the question.

\begin{subbox}[width=1\linewidth, center]{Example of overlapping answers}
\small
\textbf{Q1. What was the study design or setting of the trial?}
This trial is a multicentre, randomized, double-blind, phase 3 study.\\

\textbf{Q2. What specific treatments were compared in the study?}
DBPR108 100 mg, sitagliptin 100 mg, and placebo.\\

\textbf{Q3. How were the participants or subjects of the study selected and divided?}
In this \hlorange{multicentre, randomized, double-blind, phase 3 study}, adult patients with type 2 diabetes were \hlgreen{randomly assigned} to \hlorange{receive either DBPR108 100mg, sitagliptin 100mg, or placebo} once daily. \hlgreen{A total of 766 patients were enrolled and divided into three groups: DBPR108 100mg (n=462), sitagliptin 100mg (n=152), or placebo (n=152).}
\end{subbox}

\begin{figure*}
\includegraphics[width=\textwidth]{figures/annotation-interface}
\caption{Interface for question salience annotation. Each question can be expanded to show an illustrative answer sourced from a randomly chosen document. The questions shown here are for the \emph{Astro} dataset.}
\end{figure*}


\begin{figure*}[t]
\includegraphics[width=\textwidth]{figures/worked-example}
\caption{Fully worked example of the question-based content analysis. Two documents in a fictional domain are each summarized at three lengths. Afterwards Steps 1 -- 4 are analogous to \cref{sec:method-questions}. Summary claims are colorcoded.}
\label{fig:worked-example}
\end{figure*}


\end{document}
