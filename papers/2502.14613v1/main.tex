% This must be in the first 5 lines to tell arXiv to use pdfLaTeX, which is strongly recommended.
\pdfoutput=1
% In particular, the hyperref package requires pdfLaTeX in order to break URLs across lines.

\documentclass[11pt]{article}
\usepackage{acl}

% standard packages
\usepackage{times}
\usepackage{latexsym}
\usepackage[T1]{fontenc}
\usepackage[utf8]{inputenc}
\usepackage{microtype}
\usepackage{inconsolata}

% custom packages
\usepackage{amsmath}
\usepackage{amssymb}
\usepackage{graphicx}
\usepackage{booktabs}
\usepackage{tabularx}
% \usepackage[table]{xcolor}
\usepackage{multirow}
\usepackage[capitalize,nameinlink,noabbrev]{cleveref}
\crefname{section}{\S}{\S\S}  % Like: § 4
\crefname{subsection}{\S}{\S\S}  % Like: § 4.1
\crefname{subsubsection}{\S}{\S\S} % Like: § 4.1.1
\newcounter{codeboxinput}
\crefname{codeboxinput}{Listing}{Listings}
\usepackage[shortcuts]{extdash}  % non-breaking dashes
\usepackage{subcaption}
\usepackage{fontawesome5}
\usepackage{algorithm}
\usepackage{algpseudocodex}

\usepackage{enumitem}
\newlist{todolist}{itemize}{2}
\setlist[todolist]{label=$\square$}

% Text highlights (uses soul package)
% Colors
\usepackage{soul}
\definecolor{pastelblue}{HTML}{A1C9F4}
\definecolor{pastelorange}{HTML}{FFB482}
\definecolor{pastelgreen}{HTML}{8DE5A1}
\definecolor{pastelred}{HTML}{FF9F9B}
\definecolor{pastelpurple}{HTML}{D0BBFF}
\newcommand{\hlred}[1]{\sethlcolor{pastelred}\hl{#1}}
\newcommand{\hlorange}[1]{\sethlcolor{pastelorange}\hl{#1}}
\newcommand{\hlblue}[1]{\sethlcolor{pastelblue}\hl{#1}}
\newcommand{\hlgreen}[1]{\sethlcolor{pastelgreen}\hl{#1}}

% This must be in the first 5 lines to tell arXiv to use pdfLaTeX, which is strongly recommended.
\pdfoutput=1
% In particular, the hyperref package requires pdfLaTeX in order to break URLs across lines.

\documentclass[11pt]{article}

% Change "review" to "final" to generate the final (sometimes called camera-ready) version.
% Change to "preprint" to generate a non-anonymous version with page numbers.
\usepackage[preprint]{acl}
\usepackage{booktabs}
\usepackage{amsfonts}
\usepackage{amsmath}
\usepackage{multirow}
\usepackage{amsthm}
\usepackage{algorithm}
\usepackage{algorithmic}
\newtheorem{theorem}{Theorem}[section]
\newtheorem{assumption}{Assumption}[section]
\newtheorem{definition}{Definition}[section]
\newtheorem{proposition}{Proposition}[section]
\newtheorem{corollary}{Corollary}[theorem]
\newtheorem{lemma}[theorem]{Lemma}
\newtheorem*{remark}{Remark}
% Standard package includes
\usepackage{times}
\usepackage{latexsym}

% For proper rendering and hyphenation of words containing Latin characters (including in bib files)
\usepackage[T1]{fontenc}
% For Vietnamese characters
% \usepackage[T5]{fontenc}
% See https://www.latex-project.org/help/documentation/encguide.pdf for other character sets

% This assumes your files are encoded as UTF8
\usepackage[utf8]{inputenc}

% This is not strictly necessary, and may be commented out,
% but it will improve the layout of the manuscript,
% and will typically save some space.
\usepackage{microtype}

% This is also not strictly necessary, and may be commented out.
% However, it will improve the aesthetics of text in
% the typewriter font.
\usepackage{inconsolata}

%Including images in your LaTeX document requires adding
%additional package(s)
\usepackage{graphicx}

% If the title and author information does not fit in the area allocated, uncomment the following
%
%\setlength\titlebox{<dim>}
%
% and set <dim> to something 5cm or larger.

\title{A statistically consistent measure of Semantic Variability using Language Models}

% Author information can be set in various styles:
% For several authors from the same institution:
% \author{Author 1 \and ... \and Author n \\
%         Address line \\ ... \\ Address line}
% if the names do not fit well on one line use
%         Author 1 \\ {\bf Author 2} \\ ... \\ {\bf Author n} \\
% For authors from different institutions:
% \author{Author 1 \\ Address line \\  ... \\ Address line
%         \And  ... \And
%         Author n \\ Address line \\ ... \\ Address line}
% To start a separate ``row'' of authors use \AND, as in
% \author{Author 1 \\ Address line \\  ... \\ Address line
%         \AND
%         Author 2 \\ Address line \\ ... \\ Address line \And
%         Author 3 \\ Address line \\ ... \\ Address line}

\author{Yi Liu \\
  Seattle, Washington, USA \\
  %\texttt{liuyi3@microsoft.com} 
}

%\author{
%  \textbf{First Author\textsuperscript{1}},
%  \textbf{Second Author\textsuperscript{1,2}},
%  \textbf{Third T. Author\textsuperscript{1}},
%  \textbf{Fourth Author\textsuperscript{1}},
%\\
%  \textbf{Fifth Author\textsuperscript{1,2}},
%  \textbf{Sixth Author\textsuperscript{1}},
%  \textbf{Seventh Author\textsuperscript{1}},
%  \textbf{Eighth Author \textsuperscript{1,2,3,4}},
%\\
%  \textbf{Ninth Author\textsuperscript{1}},
%  \textbf{Tenth Author\textsuperscript{1}},
%  \textbf{Eleventh E. Author\textsuperscript{1,2,3,4,5}},
%  \textbf{Twelfth Author\textsuperscript{1}},
%\\
%  \textbf{Thirteenth Author\textsuperscript{3}},
%  \textbf{Fourteenth F. Author\textsuperscript{2,4}},
%  \textbf{Fifteenth Author\textsuperscript{1}},
%  \textbf{Sixteenth Author\textsuperscript{1}},
%\\
%  \textbf{Seventeenth S. Author\textsuperscript{4,5}},
%  \textbf{Eighteenth Author\textsuperscript{3,4}},
%  \textbf{Nineteenth N. Author\textsuperscript{2,5}},
%  \textbf{Twentieth Author\textsuperscript{1}}
%\\
%\\
%  \textsuperscript{1}Affiliation 1,
%  \textsuperscript{2}Affiliation 2,
%  \textsuperscript{3}Affiliation 3,
%  \textsuperscript{4}Affiliation 4,
%  \textsuperscript{5}Affiliation 5
%\\
%  \small{
%    \textbf{Correspondence:} \href{mailto:email@domain}{email@domain}
%  }
%}

\begin{document}
\maketitle
\begin{abstract}
To address the challenge of variability in the output generated by language models, we introduce a measure of semantic variability that remains statistically consistent under mild assumptions. This measure, termed semantic spectral entropy, is an easily implementable algorithm that requires only standard, pre-trained language models. Our approach imposes minimal restrictions on the choice of language models, and through rigorous simulation studies, we demonstrate that this method can produce an accurate and reliable metric despite the inherent randomness in language model outputs.
\end{abstract}

\section{Introduction}

\label{introduction}
{\color{white}..} The birth of Large Language Models (LLM) has given rise to the possibility of a wide range of industry applications \cite{touvron2023llama,chowdhery2023palm}. One of the key applications of generative models that has garnered significant interest is the development of specialized chatbots with domain-specific expertise such as legal and healthcare \cite{Lexis,mesko2023top}. These applications illustrate how generative models can improve decision-making and improve the efficiency of professional services in specialized fields.

This new LLM capability is made possible by the strong understanding of generative capabilities of the models \cite{liu2023mmc,long2023large} and the advent of Retrieval-Augmented Generation (RAG) \cite{lewis2020retrieval, gao2023retrieval}. In an RAG system, the user interacts by submitting queries, which trigger a search for relevant documents within a pre-established database. These pertinent documents are retrieved based on the query and serve as a context for the LLM to generate an appropriate response. Since the implementation of RAG does not require a custom-trained LLM, it offers a cost-effective solution. The resulting chatbot can perform tasks traditionally handled by domain experts, improving operational efficiency and driving cost reductions.

However, a critical challenge impeding the widespread deployment of generative models in industry is the inherent variability present in these models \cite{amodei2016concrete,hendrycks2021unsolved}.  Although parameters such as temperature, top-k, top-p, and repetition penalty are known to significantly influence model performance \cite{wang2020contextual,wang2023cost, song2024good}, even when these parameters are tuned to achieve deterministic output (e.g. setting temperature to 0 or top-p to 1), differences in the generated results can still occur in multiple runs. This persistent variability poses a significant barrier to the reliable and consistent application of generative models in practical settings.

Atil et al. (2024) conducted a series of experiments involving six deterministically configured large language models (LLMs), with temperature set to 0 and top p set to 1, across eight common tasks and five identical trials per task. The study aimed to assess the repeatability of model outputs by examining whether the generated strings were consistent between runs. The authors found that none of the LLMs demonstrated consistent performance in terms of generating identical outputs on all tasks \cite{atil2024llm}. 
%For complex tasks, such as college-level mathematics, the models often produced lexically different outputs for each run, leading to zero consistency in terms of exact string matching. 
However, the authors noted that when accounting for syntactical variations, the observed differences were relatively minor as many of the generated strings were semantically equivalent. 


The variability in output has been attributed to the use of GPUs in large language model (LLM) inference processes, where premature rounding during computations can lead to discrepancies \cite{nvidia2024,atil2024llm}. Given this, it is reasonable to conclude that complete elimination of variability is unfeasible in any empirical setting. Consequently, we must acknowledge that the output of LLMs is inherently uncertain. In light of this, it becomes essential, similar to practices in statistics, to assess and quantify the level of uncertainty in the text generated by LLMs for any given scenario. 

Most prior studies on uncertainty in foundation models for natural language processing (NLP) have focused primarily on the calibration of classifiers and text regressors \cite{jiang2021can, desai2020calibration, glushkova2021uncertainty}. Other research has addressed uncertainty by prompting models to evaluate their own outputs or fine-tuning generative models to predict their own uncertainty \cite{linteaching, kadavath2022language}. However, these approaches require additional training and supervision, making them difficult to reproduce, costly to implement, and sensitive to distributional shifts. 

 Our work follows from a line of work inline with the concept of semantic entropy proposed in \cite{kuhn2023semantic, nikitin2024kernel,duan-etal-2024-shifting,lin2023generating}. \cite{kuhn2023semantic} explore the entropy of the generated text by assigning semantic equivalence to the pairs of text and subsequently estimating the entropy. Similarly, \cite{nikitin2024kernel} and \cite{lin2023generating} utilize graphical spectral analysis to enhance empirical results. However, a notable limitation in the entropy estimators proposed by \cite{kuhn2023semantic} and \cite{nikitin2024kernel} is their reliance on token likelihoods when assessing semantic equivalence, which may not always be accessible. Furthermore, \cite{kuhn2023semantic} acknowledge that the clustering process employed in their framework is susceptible to the order of comparisons, introducing variability into the results. 

Moreover, previous work focuses on the empirical performance of the estimator. As such, while these methods have demonstrated favorable empirical outcomes, to the best of our knowledge, no authors have established using a theoretical analysis that their entropy estimators converge to a true entropy value as the sample size increases under an underlying generative model. Exploring the theoretical properties allows us to have a clear understanding of how the number of clusters and size of data would affect the estimator. 

Our approach seeks to address these limitations by developing a robust theoretical analysis of the clustering procedure, ensuring convergence properties, and mitigating the variability inherent in prior methodologies. We propose a theoretically analyzable metric for quantifying the variation within a collection of texts, which we refer to as semantic spectral entropy. This measure addresses the observation that many generated strings, while lexically and syntactically distinct, may convey equivalent semantic content. To identify these semantic equivalences, we advocate the use of off-the-shelf generative language models (LMs). Moreover, we acknowledge that the LM used to evaluate semantic similarity is itself a stochastic generator. In response, we employ the well-established technique of spectral clustering, which is provably consistent under minimal assumptions on the generator, thereby ensuring the robustness and reliability of the proposed metric. Specifically, we demonstrate that the measure is statistically consistent under a weak assumption on the LM. To the best of our knowledge, this is the first semantic variability measure with proven convergence properties. As an empirical evaluation studies, we also propose a simple method for constructing clusters of different lexically and syntactically distinct but semantically equivalent text using compound  propositions from \cite{wittgenstein2023tractatus}.

\section{Semantic spectral entropy}
\label{methodology}
\subsection{Semantic entropy}

{\color{white}..} We begin with a collection of textual pieces \( n \), denoted \( \mathcal{T} = (t_1, \cdots, t_n) \). Unlike that in \cite{kuhn2023semantic}, our assumption is that we have access only to $\mathcal{T}$. In fact, we do not require the existence of a generative model and is interested only in variability of the semantics in the text. To evaluate the semantic variability of these texts in the context of a specific use case, we propose a theoretically proven measure of semantic entropy which we named semantic spectral entropy. 

A key reason for opting against the use of variance as a measure of variability is that computing variance requires the definition of a mean, which is challenging to establish for semantic distributions. Although it is possible to define an arbitrary reference point, such as a standard answer in a chatbot that answers questions, evaluating the variability with respect to such a reference introduces bias. 

In contrast, entropy is a well-established measure of variation, particularly for multinomial distributions. For a distribution \( \mathcal{P}(t) \) over a set of semantic clusters \( \{C_1, \cdots, C_k\} \), the entropy \( \mathcal{E} \) is defined as:
\begin{equation}
\label{equ:entropy}
\mathcal{E}(t) = - \sum_{i} p(t \in C_i)\log p(t \in C_i).
\end{equation}
This formulation captures the uncertainty or disorder associated with assigning a given text \( t \) to one of the clusters. Consequently, it provides a quantitative measure of semantic variability that avoids the biases introduced by arbitrary reference points.

To estimate the entropy for a given data set \( t_1, \cdots, t_n \), we first calculate the number of occurrences of each text \( t_i \) in each group \( C_j \). This is achieved by computing:
\[
n_j = \sum_{i=1}^n \mathbb{I}(t_i \in C_j),
\]
where \( \mathbb{I}(t_i \in C_j) \) is an indicator function that equals 1 if \( t_i \) belongs to the cluster \( C_j \), and 0 otherwise. 

Next, the true probability \( p(t \in C_j) \) is approximated using the empirical distribution:
\[
\bar{p}(t \in C_j) = \frac{n_j}{n},
\]
which represents the fraction of texts assigned to cluster \( C_j \). Using this empirical distribution, the empirical entropy is defined as:
\[
\bar{\mathcal{E}}(\mathcal{T}) = - \sum_{j} \bar{p}(t \in C_j) \log \bar{p}(t \in C_j).
\]
This measure provides a practical estimation of semantic entropy based on observed data.


One critical step in this process is clustering the texts $t_i$ into disjoint groups. To do so, it is sufficient to define a relationship between $t_i \sim t_j$, such that they satisfy the properties of equivalence relation. Specifically, one needs to demonstrate 
\begin{enumerate}
    \item Reflexivity: For every $t_i$, we have $t_i \sim t_i$, meaning that any text is equivalent to itself.
    \item Symmetry: If $t_i \sim t_j$, then $t_j \sim t_i$, meaning that equivalence is bidirectional.
    \item Transitivity: If $t_i \sim t_j$ and $t_j \sim t_k$, then $t_i \sim t_k$, which means that equivalence is transitive.
\end{enumerate}

It turns out the existence of an equivalence equation is both a necessary and sufficient condition for a definition of a breakdown of $\mathcal{T}$ into disjoint clusters \cite{liebeck2018concise}. In light of this, defining $\sim$ should be based on the linguist properties of entropy measurement. 

 Direct string comparison, defined as \( t_i \sim t_j \) if and only if \( t_i \) and \( t_j \) share identical characters, reflects lexicon equality and constitutes an equivalence relation. However, this criterion is overly restrictive. In a question-and-response context, a more appropriate equivalence relation might be defined as \( t_i \sim t_j \) if and only if \( t_i \) and \( t_j \) yield identical scores when evaluated by a language model (LM) prompt. This criterion, however, requires an answer statement as a point of reference. We are more interested in a stand-alone metric that can capture the semantic equivalence. For example, consider the sentences \( t_1 = \text{"Water is vital to human survival"} \) and \( t_2 = \text{"Humans must have water to survive"}\). Despite differences in language, both sentences convey the same underlying meaning.

To address such challenges, \cite{kuhn2023semantic,nikitin2024kernel} propose an equivalence relation wherein \( t_i \sim t_j \) if and only if \( t_i \) is true if and only if \( t_j \) is true. This formulation ensures that two texts, \( t_i \) and \( t_j \), belong to the same equivalence class if they are logically equivalent. This broader definition allows for greater flexibility and applicability in assessing semantic equivalence beyond superficial lexical similarity.\cite{copi2016introduction}. We will present their argument as a proposition where we will put the verification in the appendix
\begin{proposition}
    \label{prop:equ}
    The relation $t_i \sim t_j$ if "$t_i$ is true if and only if $t_j$ is true" is an equiva
    
    lence relation.
\end{proposition}

% one considers segmenting the set into equivalence classes based on the following equivalence relation:  


%We first establish that this relation $\sim$ indeed defines well-defined, disjoint subsets of $t_i, i \in \{1,\dots n\}$. 



% Now, we can conclude that $\sim$ is indeed an equivalence relation, and thus, the set of texts $t_1\dots t_n$ can be partitioned into disjoint equivalence classes, where each class represents a distinct semantic group. 
%\begin{remark}
    
%\end{remark}
In light of the fact that equivalence relations can be defined arbitrarily based on the needs of the user. We propose that the determination of equivalence relations, denoted as $\sim$, is performed through a LM that generates responses independently of the specific generation of terms $t_1, \dots, t_n$. However, we do not assume that we have access to probability distribution of the tokens as proposed by \cite{kuhn2023semantic,nikitin2024kernel} which is not always available. Rather, we just require a generator LM which can generate a determination of this relationship. Therefore, this LM can be general generative language model with a crafted prompt which we will use in our simulation studies. The error in this LM will be removed in the spectral clustering algorithm at the later stage. By leveraging this LM, we can define a function $e:{\mathcal{T}, \mathcal{T}}\rightarrow {0,1}$, which is formally expressed as follows: \begin{equation} e(t_i, t_j) = \begin{cases} 1 & \text{if } t_i \sim t_j, \\ 0& \text{otherwise.} \end{cases} \end{equation}

However, since the function relies on an LM, $e(t_i, t_j)$ can be viewed as a Bernoulli random variable, whose value is dependent on the terms $t_i$ and $t_j$.\nocite{kuhn2023semantic} did not address this issue but instead offers adopting a very powerful entailment identification model which the authors trust to identify the equivalence relation perfectly. In contrast, we suggest modeling the outputs of the LM as a random graph with an underlying distribution. In this framework, $t_i$ and $t_j$ represent nodes, while $e(t_i, t_j)$ are random variables that indicate the presence of an edge between the two nodes. Specifically, when $t_i \sim t_j$, the edge existence is governed by the following probability distribution: \begin{equation} \label{eqn:equation_p} e(t_i, t_j) = \begin{cases} 1 & \text{with probability } p, \\ 0 & \text{with probability } 1-p. \end{cases} \end{equation} Conversely, when $t_i \not\sim t_j$, the edge existence follows a different probability distribution: \begin{equation} \label{eqn:equation_q} e(t_i, t_j) = \begin{cases} 1 & \text{with probability } q, \\ 0 & \text{with probability } 1-q. \end{cases} \end{equation}

To mitigate the inherent randomness introduced by the LLM, we propose leveraging spectral clustering to identify clusters of semantically similar texts.
\subsection{Spectral clustering}

{\color{white}..} To compute semantic entropy, it is crucial to identify the clusters of nodes and count the number of nodes within each cluster. Identifying these clusters in a random graph is analogous to detecting clusters in a stochastic block model \cite{holland1983stochastic}. We propose employing the spectral clustering algorithm, with the number of clusters $K$ specified in advance, as an effective approach for this task.

Spectral Clustering is a well-established algorithm for graph clustering, supported by strong theoretical foundations and efficient implementations \cite{shi2000normalized, lei2015consistency, su2019strong, scikit-learn}.  To compute semantic entropy, we aim to cluster a random graph with adjacency matrix $E$ where $E_{ij} = e(t_i, t_j)$, representing the pairwise similarity between text elements $t_i$ and $t_j$. 

We begin by computing the Laplacian matrix $L =  D-E$ where $D$ is the degree matrix.  This is followed by the decomposition of the eigenvalue of $L$. Next, we construct the matrix formed by the first $K$ eigenvectors of $L$ denoted $\hat{U} \in \mathbb{R}^{n\times K}$. This matrix serves as input to an appropriate $(1+\epsilon)-$ k-means clustering algorithm \cite{kumar2004simple,choo2020k}.

The output of this procedure is $K$ distinct clusters $C_1,\cdots C_K$. For each text element $t_i$, we assign a corresponding vector $g_{i}$ where 
$$ g_{ij} = \begin{cases} 1 \text{ if } t_i \in C_j\\
    0 \text{ otherwise }
\end{cases}$$
This binary indicator vector $g_i$ encodes the cluster membership for each text element $t_i$

Finally, we compute the estimated entropy based on the number of texts within each cluster. The entropy $\hat{\mathcal{E}}$ can be approximated using the following formula:
\begin{equation}
\hat{\mathcal{E}}(\mathcal{T}) = - \sum_{j=1}^k\hat{p}( C_j) \log(\hat{p}( C_j)),
\end{equation}
where $\hat{p}(C_j) = \frac{1}{n}\sum_{i=1}^n g_{ij}$.This expression represents the empirical entropy based on the distribution of texts among the $K$ clusters, providing a measure of the uncertainty or diversity within the semantic structure of the data.
\subsection{Full algorithm and implementation}
{\color{white}..} We merge the process of finding sermantic entropy with spectral clustering to present the full algorithm as Algorithm \ref{algo:1}: Sermantic Spectral Entropy. 
\begin{algorithm}
\begin{algorithmic}
    \STATE Begin with $\mathcal{T} = \{t_1, \cdots t_n\}$
    \FOR{$i, j \in \{1,\cdots n\} \times \{1, \cdots n\}, i\neq j$}
    \STATE Use LLM to compute $E_{i,j} = e(t_i, t_j)$. 
    \ENDFOR
    \STATE Find the Laplacian of $E$, $L = D -E$
    \STATE Compute the first $K$ eigenvectors $u_1,\dots,u_k$ of $L$ and the top $K$ eigenvalues $\lambda_1,\cdots \lambda_k$.
    \STATE Let $\hat{U} \in \mathbb{R}^{n\times k}$ be the matrix containing the vectors $u_1,\dots,u_k$ as columns.
    \STATE Use $(1+\epsilon)$ K-means clustering algorithm to cluster the rows of $U$
    \STATE Let $g_{ij}$ be an $(1+\epsilon)-$approximate solution to a $K-$means clustering algorithm
    \STATE Compute $\hat{\mathcal{E}}(\mathcal{T})$ using $ g_{ij}$
\end{algorithmic}
\caption{\label{algo:1} Sermantic Spectral Entropy }
\end{algorithm}

This polynomial-time algorithm is characterized by the largest computational cost associated with the determination of $E_{ij}$. However, computing $E_{ij}$ is embarrassingly parallel, meaning that it can be efficiently distributed across multiple processing units. Furthermore, there are well-established implementation, such as Microsoft Azure's Prompt-Flow \cite{esposito2024programming} and LangChain \cite{mavroudis2024langchain} that facilitate the implementation of parallel workflows, making it feasible to deploy such parallelized tasks with relative ease.
\subsection{Finding K}

{\color{white}..} A notable limitation of this analysis is the unavailability of $K$ in the direct computation of semantic spectral entropy. However, the determination of $K$ for stochastic block model has been well studied \cite{lei2016goodness,wang2017likelihood,chen2018network}. We will describe the cross-validation approach \cite{chen2018network} in detail. The principle behind cross-validation involves predicting the probabilities associated with inter-group connections ($p$) and intra-group connections ($q$). If the estimated value of $K$ is too small, it fails to accurately recover the true underlying probabilities; conversely, if $K$ is too large, it leads to overfitting to noisy data. This approach has the potential to recover the true cluster size under relatively mild conditions.

\section{Theoretical Results}
\label{theory}

{\color{white}..} Our theoretical analysis involves a proof that the estimator is strongly consistent, i.e. the estimator converges to true value almost surely, and an analysis of its rate with respect to the number of cluster $K$. 

We divide our analysis into two subsections. The first subsection examines a fixed set of $\mathcal{T} = {t_1, \dots, t_n}$, which is assumed to exhibit some inherent clusters $C_1, \dots, C_K$. Under the assumption of perfect knowledge of these clusters, the empirical entropy $\bar{\mathcal{E}}$ can be determined. The primary focus in this subsection is on the performance of spectral clustering algorithms. The second subsection explores a scenario in which there exists an underlying generative mechanism that allows for the infinite generation of $t_i$. In this case, we permit $K$ to increase with $n$, though at a significantly slower rate. This scenario is particularly relevant for evaluating the performance of RAG in the context of continuous generation of results in response to a given query.

\subsection{Performance of spectral clustering algorithms}
{\color{white}..}  We model the LM determination of $e(t_i,t_j)$ as a random variable, as described in Equations \ref{eqn:equation_p} and \ref{eqn:equation_q}. In the theoretical analysis presented here, we assume that the number of clusters, $K$, is known and fixed. To derive various results, we first establish the relationship between the difference $|\bar{\mathcal{E}}(\mathcal{T}) - \hat{\mathcal{E}}(\mathcal{T})|$ and the miscluster error, denoted $M_\text{error}$.
\begin{lemma}
 \label{lemma:error}
Suppose that there exists $0<c_2<1$ such that $2Kn_{\min}/n \geq c_2$, 
\begin{equation}
     |\hat{\mathcal{E}}(\mathcal{T}) - \bar{\mathcal{E}}(\mathcal{T})|\leq h\left(\frac{2K}{c_2}\right) \left|\frac{1}{n} (M_\text{error})\right| 
\end{equation}
where $h(x) = \left(x+\log\left(x\right)\right)$.
\end{lemma}
The proof is presented in the Appendix section \ref{Appendix:proofoflemma:error}. 
We begin by presenting the result of strong consistency for the spectral clustering algorithm.  

\begin{theorem}
\label{the:strongConsistensy}
Under regularity conditions, the estimated entropy empirical entropy $\hat{\mathcal{E}}(\mathcal{T})$ is strongly consistent with the empirical entropy, i.e. 
\begin{equation}
    |\bar{\mathcal{E}}(\mathcal{T}) - \hat{\mathcal{E}}(\mathcal{T}) | \rightarrow 0 \text{ almost surely }
\end{equation}
\end{theorem}
The proof is provided in the Appendix section \ref{appendix:sec:the:strongconsistency}. This establishes strong consistency result that we aim to present. At the same time, we also want to show the finite sample properties of the estimator $\hat{\mathcal{E}}(\mathcal{T})$.

\begin{theorem}
    \label{the:finite_sample}
    If there exists $0<c_2\leq1$ and $\lambda > 0$ such that $2Kn_{\min}/n \geq c_2$, and $p = \alpha_n = \alpha_n(q + \lambda) $, where $\alpha_n \geq \log(n)$ then with probability at least $1-\frac{1}{n}$

\begin{equation}
|\bar{\mathcal{E}}(\mathcal{T}) - \hat{\mathcal{E}}(\mathcal{T}) |  \leq h\left(\frac{2K}{c_2}\right) \frac{n_{\max }}{4c_2^2n_{\min }^{2} \alpha_{n}K^2} 
\end{equation}
where $h(x) = \left(x+\log\left(x\right)\right)$, $n_{\max} = \max_j\{n_j : j = 1,\dots K\}$, and $n_{\min} = \min_j\{n_j : j = 1,\dots K\}$.
\end{theorem}
The full proof is provided in the appendix section \ref{appendix:proofofthe:finite_sample}. A brief outline of the proof is as follows: we begin by using the results from \cite{lei2015consistency}, which establish the rate of convergence for the stochastic block model. Next, we relate the errors of the spectral clustering algorithm to the errors in the empirical entropy, using the lemma \ref{lemma:error} to establish this connection.

\begin{remark}
    This result is particularly relevant for computing semantic entropy, as the output generated by LMs is produced with a probability that is independent of $n$. As a result, we have $\alpha_n = O(1)$. Assuming balanced community sizes, the convergence rate is therefore $O(\frac{1}{n})$. This is formally stated in the following corollary:
\end{remark}


\begin{corollary} \label{corollary:rate} If there exists a constant $0 < c_2 \leq 1$ such that $2Kn_{\min}/n \geq c_2$ and $\alpha_n = alpha >0$, then there exists a constant $\alpha$ such that with probability at least $1 - \frac{1}{n}$, \begin{equation} |\bar{\mathcal{E}}(\mathcal{T}) - \hat{\mathcal{E}}(\mathcal{T})| \leq h\left(\frac{2K}{c_2}\right) \frac{1}{c_2^4 \alpha n}. \end{equation} \end{corollary}

The proof of this result is provided in the Appendix section \ref{appendix:proofofcorollary:rate}. 
\begin{remark}
    In particular, we observe that the convergence rate is $O\left(\frac{1}{n}\right)$. This means that the error associated with spectral clustering is small, and our estimated entropy converges to the empirically entropy quickly.
\end{remark}


\subsection{Performance under a generative model}

{\color{white}..} In practical terms, we assume the presence of a generator, specifically an RAG, that produces identically distributed independent random variables $t_i$' that collectively form semantic clusters $C_1 \dots C_K$. In essence, we have $t_i \sim G$ such that $t_i \in C_j$ with probability $p(C_j)$. In this model, there is a true value of entropy $\mathcal{E}(\mathcal{T})$ given in Equation \ref{equ:entropy}, and we want to find the convergence rate of our method. 
\begin{theorem}
    \label{the:final} If there exists a constant $\alpha $ such that $p = \alpha  = \alpha(q + \lambda) $, then with probability at least $1-\frac{3}{n}$,
    \begin{equation}
    \label{eqn:final_the}
       \begin{array}{cc}
         |\mathcal{E} - \hat{\mathcal{E}}|&  \leq h\left(\frac{1}{p_{\min}}\right)K\sqrt{\frac{1}{2n}\log\left(2Kn\right)}\\
         & +h\left(\frac{1}{m(n)p_{\min}}\right)\frac{1}{16K^4m(n)^4p_{\min}^4n}
    \end{array} 
    \end{equation}

where $m(n) = \left(1- \sqrt{2\log(nK)/np_{\min}}\right)$ and $p_{\min} = \min\{p(C_1)\dots p(C_K)\}$.
\end{theorem}
Most of the material used for this proof is presented in Corollary \ref{corollary:rate}. 
\begin{proof}
Consider the following equality
$$|\mathcal{E} - \hat{\mathcal{E}}| \leq |\mathcal{E} -\bar{\mathcal{E}} +\bar{\mathcal{E}}-  \hat{\mathcal{E}}| \leq |\mathcal{E} -\bar{\mathcal{E}}| + | \bar{\mathcal{E}}-  \hat{\mathcal{E}}|,$$  
 We know that there are three sufficient conditions for Equation \ref{eqn:final_the}. These are
\begin{enumerate}
    \item[C1:]$|\mathcal{E} -\bar{\mathcal{E}}| \leq  h\left(\frac{1}{p_{\min}}\right)K\sqrt{\frac{1}{2n}\log\left(2Kn\right)},$
    \item[C2:]$ \exists c_2 \text{ such that } 0 < c_2 \leq 1$ and $2Kn_{\min}/n \geq c_2,$ 
    \item[C3:] $ |\bar{\mathcal{E}}(\mathcal{T}) - \hat{\mathcal{E}}(\mathcal{T})| \leq h\left(\frac{2K}{c_2}\right) \frac{1}{c_2^4 n}.$
\end{enumerate}
Then, using union bound
\begin{align*}
    \mathbb{P}(\text{Not (\ref{eqn:final_the})}) &\leq \mathbb{P}( \text{Not C1 or Not C2 or Not C3})\\
    &\leq \mathbb{P}( \text{Not C1}) + \mathbb{P}( \text{Not C2})+ \mathbb{P}( \text{Not C3}).
\end{align*}
In Lemma \ref{lemma:final1} and \ref{lemma:Final2} of the appendix, we show that $|\mathcal{E} -\bar{\mathcal{E}}| \geq  h\left(\frac{1}{p_{\min}}\right)K\sqrt{\frac{1}{2n}\log\left(2Kn\right)}$ with probability at most $\frac{1}{n}$.

In Lemma \ref{Lemma:Final3} of the Appendix, we show that setting $c_2 = 2K\left(1-\sqrt{\frac{2\log(nK)}{np_{\min}}}\right)p_{\min}$, we have $2Kn_{\min}/n< c_2$ with probability at most $\frac{1}{n}$.

Finally, the corollary \ref{corollary:rate} tells us that $ |\bar{\mathcal{E}}(\mathcal{T}) - \hat{\mathcal{E}}(\mathcal{T})| > h\left(\frac{2K}{c_2}\right) \frac{1}{c_2^4 n}$ occurs with probability at most $\frac{1}{n}$.
\end{proof}
\begin{remark}
One observation is that the empirical entropy converges to true entropy at a rate slower than that of estimated entropy to the empirical entropy. This is natural since each $t_i$ has the opportunity to make a $n-1$ connection with other $t_j$s, resulting in $n(n-1)/2$ independent observations, whereas each generator generates only $n$ independent observations.
\end{remark}
\subsection{Discussion on $K$}
{\color{white}..} An intriguing question to consider is the rate at which \( K \), the number of clusters, can grow with \( n \), the number of texts, as it is natural to expect \( K \) to increase with \( n \). Focusing solely on the spectral clustering algorithm, the error is characterized as \( O((K + \log(K))/n) \). Thus, under the condition \( K = o(n^{1-\delta}) \) for some \( \delta > 0 \), we have \( |\bar{\mathcal{E}}(\mathcal{T}) - \hat{\mathcal{E}}(\mathcal{T})| \to 0 \) in probability. In contrast, when considering a scenario involving a generative model, a stricter condition is required. Specifically, \( K \) must satisfy \( K = o(n^{1/2 - \delta}) \), with \( \delta > 0 \), to ensure \( |\mathcal{E}(\mathcal{T}) - \hat{\mathcal{E}}(\mathcal{T})| \to 0 \) in probability.

\section{Simulation and data studies}
\label{simulation}

{\color{white} .. }As this paper focuses more on the theoretical analysis of semantic spectral entropy with respect to variable $n$ and $K$, we decide against using the evaluation method proposed in \cite{kuhn2023semantic,duan-etal-2024-shifting, lin2023generating} in favor of constructing a simulation where we know the true entropy $\bar{\mathcal{E}}$. This allows us to better analyze how $|\bar{\mathcal{E}} -\hat{\mathcal{E}}|$ changes with choice of generator $e$, $K$ and $n_{\min}$.

To construct a non-trivial simulation for this use case, we evaluate the performance of our algorithms within the context of an unordered set of elementary proposition statements that has no logical interconnections. This approach draws upon the philosophical framework defined by \citep{wittgenstein2023tractatus} in Tractatus Logico-Philosophicus, where each elementary proposition represents a singular atomic fact. Within this framework, texts containing an identical set of elementary propositions are deemed semantically equivalent. The primary advantage of this experimental design lies in its efficiency, as it facilitates the generation of thousands of samples with minimal generator propositions, all while maintaining knowledge of the ground truth.

For example, we can consider a list of things that a hypothetical individual "John" likes to do in his free time: 
\begin{itemize}
    \item Running/Jogging 
    \item Drone Flying/ Pilot Aerial drones
    \item jazzercise / aerobics
    \item ...
\end{itemize}

To generate a cluster of text from this set of hobbies, we begin by randomly selecting \( M \) items from a total of \( N \) items in the list to formulate the compound proportion. This selection process yields \( \binom{N}{M} \) potential subset of hobbies and we know that two subsets of hobbies are the same as long as their elements are the same. Next, to create individual text samples \( t_i \) within the group, we randomly permute the order of the \( M \) selected elements in the subset. This permutation process generates \( M! \) unique samples for each combination of hobbies. Finally, the hobbies are then placed in its permuted order in a sentence like that below. 
\begin{quote}
"In his free time, John likes hobby $1$, hobby $2$, hobby $3$, ..., and hobby $M$ as his hobbies."
\end{quote}
In order to prevent models to rely on sentence structure, a few of these sentences are being designed. 

%We replicate this simulation set-up in different 2 settings. The  setting is the 10 common hobbies that this hypothetical individual likes to do in his free time. %The second set-up is 10 events that happened on the date December 3 in history which we collect from Wikipedia \cite{wiki}. %The last setting  %need to think about how to build these algorithm
%\cite{atil2024llm}

We utilize Microsoft Phi-3.5 \cite{abdin2024phi}, OpenAI GPT3.5-turbo \cite{hurst2024gpt}, A21-Jamba 1.5 Mini \cite{lieber2021jurassic}, Cohere-command-r-08-2024 \cite{Ustun2024AyaMA},Ministral-3B \cite{jiang2023mistral} and the Llama 3.2 70B model \cite{dubey2024llama} as \( e \). These models are lightweight, off-the-shelf language models that are cost-effective to deploy and exhibit efficiency in generating outputs, thereby off-setting the computational cost of determining sermantic relationships. The exact prompt used to generate the verdict is specified in Appendix \ref{appendix_sec:prompt_engineering}.

\begin{table*}[ht]
\centering
\begin{tabular}{l|rrr|rrr|rrr|}
\toprule
ratio & \multicolumn{3}{r|}{0.2,0.3,0.5} & \multicolumn{3}{r|}{0.3,0.3,0.4} & \multicolumn{3}{r|}{0.5,0.5}\\
datasize & 30 & 50 & 70 & 30 & 50 & 70 & 30 & 50 & 70\\
\midrule
LLAMA & 0.36 & 0.49 & 0.44 & 0.34 & 0.43 & 0.46 & 0.30 & 0.27 & 0.26 \\x
MINISTRAL & 0.22 & 0.27 & 0.13 & 0.25 & 0.23 & 0.21 & 0.14 & 0.22 & 0.21 \\
COHERE & 0.04 & 0.02 & 0.06 & 0.02 & 0.03 & 0.00 & 0.00 & 0.00 & 0.00 \\
A21 & 0.05 & 0.00 & 0.00 & 0.00 & 0.01 & 0.00 & 0.00 & 0.00 & 0.00 \\
PHI & 0.08 & 0.07 & 0.07 & 0.03 & 0.03 & 0.00 & 0.00 & 0.00 & 0.00 \\
GPT & 0.06 & 0.02 & 0.00 & 0.01 & 0.00 & 0.00 & 0.00 & 0.00 & 0.00 \\
\bottomrule
\end{tabular}
\caption{\label{tab:basic_simu} Average $|\bar{\mathcal{E}}- \hat{\mathcal{E}}|$ over simulation 10 iterations. We have three different ratio value run over three different data sizes. For $e$, we use Microsoft Phi-3.5 \cite{abdin2024phi}, OpenAI GPT3.5-turbo \cite{hurst2024gpt}, A21-Jamba 1.5 Mini \cite{lieber2021jurassic}, Cohere-command-r-08-2024 \cite{Ustun2024AyaMA}, Ministral-3B \cite{jiang2023mistral} and the Llama 3.2 70B model \cite{dubey2024llama}. }
\end{table*}

\begin{figure}
    \centering
    \includegraphics[width=1\linewidth]{LambdaExp1.pdf}
    \caption{\label{fig:dotdata} A scatter plot of $p-q$ against $|\bar{\mathcal{E}}- \hat{\mathcal{E}}|$. The different colors represents different language models used as $e$: A21 in blue, Phi in Orange, GPT in Green, Cohere in Red, Llama is Purple and Ministral in Brown. We notice that there is clear phrase change point where for $p-q <0.4$, we have that $|\bar{\mathcal{E}}- \hat{\mathcal{E}}|$ is very high most of the time, for $p-q >0.4$, $|\bar{\mathcal{E}}- \hat{\mathcal{E}}|$ is small with occasional jumps that the theory predicts.}
    
\end{figure}

\begin{table}[]
    \centering
\begin{tabular}{lrrr}
\toprule
 $e$& $p-q$ & $p$ & $q$ \\
\midrule
LLAMA & 0.17 & 0.17 & 0.00 \\
MINISTRAL & 0.22 & 0.99 & 0.77 \\
COHERE & 0.55 & 0.61 & 0.05 \\
A21 & 0.81 & 0.96 & 0.15 \\
PHI & 0.67 & 0.67 & 0.01 \\
GPT & 0.80 & 0.87 & 0.07 \\
\bottomrule
\end{tabular}
\caption{\label{tab:p-q} $p$, $q$ and $p-q$.  For $e$, we use Microsoft Phi-3.5 \cite{abdin2024phi}, OpenAI GPT3.5-turbo \cite{hurst2024gpt}, A21-Jamba 1.5 Mini \cite{lieber2021jurassic}, Cohere-command-r-08-2024 \cite{Ustun2024AyaMA}, Ministral-3B \cite{jiang2023mistral} and the Llama 3.2 70B model \cite{dubey2024llama}.}
\end{table}

We complete simulation studies for a ratio of (0.2,0.3,0.5), (0.3, 0.3,0.4), and (0.5,0.5) and a sample size of 30, 50, 70. The average $|\bar{\mathcal{E}}- \hat{\mathcal{E}}|$ over 10 iterations using different models as $e$ is recorded in table \ref{tab:basic_simu}. The performance of algorithm using Cohere, A21, Phi, and GPT is strong while the performance of the algorithm with Minstral and Llama is weak. We primary attribute this to the inability of Llama and Minstral to make correct statements. $p-q$ is small for Llama and Minstral and large for Cohere, A21, Phi, and GPT (shown in Table \ref{tab:p-q}). In fact, when we plot $p-q$ against $|\bar{\mathcal{E}}- \hat{\mathcal{E}}|$ in Figure \ref{fig:dotdata}, we notice that there is phrase change at value $p-q = 0.4$. $p-q < 0.4$ $|\bar{\mathcal{E}}- \hat{\mathcal{E}}|$ is high but  $p-q > 0.4$ implies that $|\bar{\mathcal{E}}- \hat{\mathcal{E}}|$ is generally small. This phrase change is not predicted in the theory and suggests that more work is needed. 
\section{Discussion}
\label{conclusion}
{\color{white} .. }Many natural language processing tasks exhibit a fundamental invariance: sequences of distinct tokens can convey identical meanings. This paper introduces a theoretically grounded metric for quantifying semantic variation, referred to as semantic spectral clustering. This approach reframes the challenge of measuring semantic variation as a prompt-engineering problem, which can be applied to any large language model (LLM), as demonstrated through our simulation analysis. In addition, unsupervised uncertainty can offer a solution to the issue identified in prior research, where supervised uncertainty measures face challenges in handling distributional shifts.

While we define two texts as having equivalent meaning if and only if they mutually imply one another, alternative definitions may be appropriate for specific use cases. For example, legal documents could be clustered based on the adoption of similar legal strategies, with documents grouped together if they demonstrate comparable approaches. In such scenarios, the entropy of the legal documents could also be computed to quantify their informational diversity. We have demonstrated that, provided there exists a function $e$ capable of performing the evaluation with weak accuracy, this estimator remains consistent. Given the reasoning capabilities of large language models (LLMs), we foresee numerous possibilities for extending this method to a wide range of applications.

In addition to the methodology presented, we present a theoretical analysis of the proposed algorithms by proving a theorem concerning the contraction rates of the entropy estimator and its strong consistency. Although the algorithm utilizes generative models, which are typically treated as black-boxes, we simplify the analysis by considering the outputs of these models as random variables. We demonstrate that only a few conditions on the generative are sufficient for our spectral clustering algorithm to achieve strong consistency. Our approach allows for many statistical methodologies to be applied in conjunctions with generative models to analyze text at a level previously not achievable by humans. 



\section{Limitation}
{\color{white} .. }We acknowledge that, while this research offers a theoretically consistent measurement of variation, it does not account for situations where two pieces of text may partially agree. For instance, two texts may contain points of agreement as well as points of disagreement. This is particularly common when different authors cite the same sources but reach contradictory conclusions.
%\section{Acknowledgments}



% Bibliography entries for the entire Anthology, followed by custom entries
%\bibliography{anthology,custom}
% Custom bibliography entries only
\bibliography{custom}
\onecolumn
\appendix

\section{Theoretical Result}
\subsection{Proof of proposition  \ref{prop:equ}}
\begin{proof} 
    To prove that the relation $t_i \sim t_j$ if $t_i$ is true if and only if $t_j$ is true is an equivalence relation, we need to meet 3 key criteria, namely symmetry, reflexivity, and Transitivity. 

    First, symmetry 
    $t_i \sim t_j$ implies that $t_j$ is true $\Leftrightarrow$ $t_j$ is true, but this also means $t_j$ is true $\Leftrightarrow$ $t_i$ is true. Then we have $t_j \sim t_i$. 

    Second, reflexivity, 
    $t_i \sim t_j$ implies $t_j$ is true $\Leftrightarrow$ $t_j$ is true. But this means that $t_j$ is true  $\Leftrightarrow$ $t_i$ is true. Then we have $t_i \sim t_j$. 
    
    Third, transitivity,
    If $t_i \sim t_j$ and $t_j \sim t_k$, Then if $t_i$ is true $\Rightarrow$ $t_j$ is true $\Rightarrow$ $t_k$ is true, which means $t_i$ is true $\Rightarrow$ $t_k$ is true. On the other hand, using the same argument, $t_k$ is true $\Rightarrow$ $t_j$ is true $\Rightarrow$ $t_i$ is true. This means the $t_k$ is true $\Rightarrow$ $t_i$ is true. Therefore $t_i \sim t_k$. 

    The three points is sufficient to demonstrate that $\sim$ is a equivalence relation. 
\end{proof}
\subsection{Proof of Theorem \ref{the:strongConsistensy}}
\label{appendix:sec:the:strongconsistency}
To prove Theorem \ref{the:strongConsistensy}, we adopt notations from \cite{su2019strong}.
Consider the adjacency matrix $E$ which is determined by a Language model. 

Let $d_i = \sum_{j=1}^n E_{ij}$ denote the degree of node $i$,  $D = \text{diag}(d_1,\cdots, d_n)$, and $L = D^{-1/2}ED^{-1/2}$ be the graph Laplacian. We also define $n_k$ be the number of text in each cluster. We denote a block probability matrix $B = B_{k_1k_2}$ where $k_1,k_2 \in\{1,\cdots K\}$ be the clusters index.  i.e. 
$$ B_{k_1 k_2} = \begin{cases}
    p \quad \text{if $k_1 = k_2$}\\
    1-q \quad \text{otherwise.}
\end{cases}$$

Let $\mathbb{E}(E) = P$ i.e. the probability of edge between $i$ and $j$ is given by $P_{ij} = B_{k_1k_2}$ if text $i$ is in $C_{k_1}$ and $j$ is in $C_{k_2}$.
Denote $Z = \{Z_{ik}\}$ be a $n\times K$  binary matrix providing the cluster membership of text $t$, i.e., $Z_{ik} = 1$ if text $i$ is in $C_k$ and $Z_{ik} = 0$ otherwise. The population version of the Laplacian is given by $\mathcal{L} = \mathcal{D}^{-1/2}P\mathcal{D}^{-1/2}$ where  $\mathcal{D} = \text{diag}(d_1 \cdots d_n)$ where $d_i =\sum_{j=1}^{n}P_{ij} = p + (n-1)(q)$.

Let $\pi_{kn} = n_k/n, W_k = \sum_{l=1}^KB_{kl}\pi_{ln}$, $\mathcal{D}_B = \text{diag}(W_1,\cdots W_K)$, and $B_0=\mathcal{D}_B^{-1/2}B\mathcal{D}_B^{-1/2}  $
%C^star = 3528C_1 c_1^{-1/2}
\begin{assumption}[Assumption 1 in \cite{su2019strong}]
\label{assumption:eigenvalues}
$P$ is rank $k$ and spectral decomposition $\Pi_{n}^{1/2}P\Pi_{n}^{1/2}$ is $S_n \Omega_n S_n^T$ in which $S_n$ is a $K \times K$ matrix such that $S_n^T S_n = I_{K\times K}$  and $\Omega_n = \text{diag}(\omega_1 \cdots \omega_{K_n})$ such that $|\omega_1|\geq |\omega_2|\geq\cdots \geq|\omega_{K_n}|$
\end{assumption}
Assumption \ref{assumption:eigenvalues} implies that the spectral decomposition $$\mathcal{L} = U_n \Sigma_n U_n^T = U_{1n}\Sigma_{1n}U_{1n}^T$$

where \(\Sigma_{n}=\operatorname{diag}\left(\sigma_{1 n}, \ldots, \sigma_{K n}, 0, \ldots, 0\right)\) is a \(n \times n\) matrix that contains the eigenvalues of \(\mathcal{L}\) such that \(\left|\sigma_{1 n}\right| \geq\left|\sigma_{2 n}\right| \geq \cdots \geq\left|\sigma_{K n}\right|>0, \Sigma_{1 n}=\operatorname{diag}\left(\sigma_{1 n}, \ldots, \sigma_{K n}\right)\), the columns of \(U_{n}\) contain the 
 eigenvectors of \(\mathcal{L}\) associated with the eigenvalues in \(\Sigma_{n}, U_{n}=\left(U_{1 n}, U_{2 n}\right)\), and \(U_{n}^{T} U_{n}=I_{n}\) \cite{su2019strong}.
\begin{assumption}[Assumption 2 in \cite{su2019strong}]
\label{assumption:limits_nk}
    There exists constant $C_1 >0$ and $c_2>0$ such that
    $$C_1 \geq \lim\sup_n\sup_k n_k K/n \geq \lim \inf_n \inf_k n_k K/n \geq c_2  $$
\end{assumption}

\begin{assumption}[Assumption 3 in \cite{su2019strong}]
\label{assumption:bound_eigenvalues}
    Let $\mu_n = \min_i d_i$ and $\rho_n = \max(\sup_{k_1k_2}[B_0]_{k_1k_2},1)$. Then $n$ sufficiently large, 
    $$ 
\frac{K \rho_{n} \log ^{1 / 2}(n)}{\mu_{n}^{1 / 2} \sigma_{K n}^{2}}\left(1+\rho_{n}+\left(\frac{1}{K}+\frac{\log (5)}{\log (n)}\right)^{1 / 2} \rho_{n}^{1 / 2}\right) \leq 10^{-8} C_{1}^{-1} c_{2}^{1 / 2} .
$$
    
\end{assumption}
Let 
$$ 
\hat{O}_{n}=\bar{U} \bar{V}^{T}
$$
where \(\bar{U} \bar{\Sigma} \bar{V}^{T}\) is the singular value decomposition of \(\hat{U}_{1 n}^{T} U_{1 n}\). we also denote \(\hat{u}_{1 i}^{T}\) and \(u_{1 i}^{T}\) as the \(i\)-th rows of \(\hat{U}_{1 n}\) and \(U_{1 n}\), respectively.

Now we present the notation of the K-means algorithm. With a little abuse of notation, let \(\hat{\beta}_{\text {in }} \in \mathbb{R}^{K}\) be a generic estimator of \(\beta_{g_{i}^{0} n} \in \mathbb{R}^{K}\) for \(i=1, \ldots, n\). To recover the community membership structure (i.e., to estimate \(g_{i}^{0}\) ), it is natural to apply the  K-means clustering algorithm to \(\left\{\widehat{\beta}_{\text {in }}\right\}\). Specifically, let \(\mathcal{A}=\left\{\alpha_{1}, \ldots, \alpha_{K}\right\}\) be a set of \(K\) arbitrary  \(K \times 1\) vectors: \(\alpha_{1}, \ldots, \alpha_{K}\). Define
\[
\widehat{Q}_{n}(\mathcal{A})=\frac{1}{n} \sum_{i=1}^{n} \min _{1 \leq l \leq K}\left\|\hat{\beta}_{i n}-\alpha_{l}\right\|^{2}
\]

and \(\widehat{\mathcal{A}}_{n}=\left\{\widehat{\alpha}_{1}, \ldots, \widehat{\alpha}_{K}\right\}\), where \(\widehat{\mathcal{A}}_{n}=\arg \min _{\mathcal{A}} \widehat{Q}_{n}(\mathcal{A})\). Then we compute the estimated cluster  identity as
\[
\hat{g}_{i}=\underset{1 \leq l \leq K}{\arg \min }\left\|\hat{\beta}_{\text {in }}-\widehat{\alpha}_{l}\right\|,
\]

where if there are multiple \(l\) 's that achieve the minimum, \(\hat{g}_{i}\) takes value of the smallest one. We then state the key assumption that relates to K-means clustering algorithm. 

\begin{assumption}[Assumption 7 in \cite{su2019strong}]
\label{assumption:K-means}
     Suppose for \(n\) sufficiently large,
     \[
15 C^{*} \frac{K \rho_{n} \log ^{1 / 2}(n)}{\mu_{n}^{1 / 2} \sigma_{K n}^{2}}\left(1+\rho_{n}+\left(\frac{1}{K}+\frac{\log (5)}{\log (n)}\right)^{1 / 2} \rho_{n}^{1 / 2}\right) \leq c_{2} C_{1}^{-1 / 2} \sqrt{2}
\]
Where \(C^{*} = 3528C_1 c_2^{-1/2} \)
\end{assumption}

\begin{theorem}(Collorary 2.2)
\label{theorem:no_error}
    Corollary 2.2. Suppose that Assumptions \ref{assumption:eigenvalues},  \ref{assumption:limits_nk}, \ref{assumption:bound_eigenvalues}, and \ref{assumption:K-means} hold and the \(K\)-means algorithm is applied  to \(\hat{\beta}_{i n}=(n / K)^{1 / 2} \hat{u}_{1 i}\) and \(\beta_{g_{i}^{0} n}=(n / K)^{1 / 2} \hat{O}_{n} u_{1 i}\) Then, 
    \[
\sup _{1 \leq i \leq n} \mathbf{1}\left\{\tilde{g}_{i} \neq g_{i}^{0}\right\}=0 \quad \text { a.s. }
\]
\end{theorem}

We now have define the error of mis-classification. 

% Since there is no true $j$, we have to take all permutation of $j$ which we denote as $\sigma(j)$. 
\begin{definition}
Denote $M_\text{error} = \sum_{j} \sum_{i}\mathbb{I}(g_{ij}  \neq g^{\text{True}}_{ij})$ as the mis-classification error.
\end{definition}

\begin{lemma}
\label{lemma:error_connections}
    If $\sup_{i,j} \mathbb{I}(g_{ij}  \neq g^{\text{True}}_{ij}) = 0 \quad \text{a.s.}$, then $M_\text{error} = 0 \quad \text{a.s.}$
\end{lemma}
\begin{proof}
Notice $\mathbb{I}(g_{ij}  \neq g^{\text{True}}_{ij})$ can only takes up value $1$ or $0$. Therefore $\sum_{j} \sum_{i}\mathbb{I}(g_{ij}\neq g^{\text{True}}_{ij}) \neq 0 \Leftrightarrow \exists i, j  \text{ s.t }\mathbb{I}(g_{ij}\neq g^{\text{True}}_{ij}) \neq 0 \Leftrightarrow  \sup_{i,j}\mathbb{I}(g_{ij}\neq g^{\text{True}}_{ij}) \neq 0$  
    \begin{align*}
        \mathbb{P}(M_\text{error} \neq 0 \text{ i.o. }) &= \mathbb{P}\left( \sum_{j} \sum_{i}\mathbb{I}(g_{ij}\neq g^{\text{True}}_{ij}) \neq 0 \text{ i.o.}\right)\\
        &= \mathbb{P}\left( \exists i, j  \text{ s.t }\mathbb{I}(g_{ij}\neq g^{\text{True}}_{ij}) \neq 0 \text{ i.o. } \right)\\
        &= \mathbb{P}\left( \sup_{i,j}\mathbb{I}(g_{ij}\neq g^{\text{True}}_{ij}) \neq 0 \text{ i.o }\right)\\
        &= 0 \quad \text{ since $\sup_{i,j} \mathbb{I}(g_{ij}  \neq g^{\text{True}}_{ij}) = 0$ \text{ a.s.}}
    \end{align*}
    Here we use the classical notation i.o. as happens infinitely often. 
\end{proof}
\begin{lemma}
    \label{lemma:misclassification}
    $\sum_j \left|\sum_{i=1}^n g_{ij}-n_j\right| \leq M_\text{error}$
\end{lemma}
\begin{proof}

\begin{align*}
\sum_j \left|\sum_{i=1}^n g_{ij}-n_j\right|
&= \sum_j \Biggl| \sum_i \mathbb{I}(g_{ij} = 1, g^{\text{True}}_{ij} = 0 ) + \mathbb{I}(g_{ij} = 1, g^{\text{True}}_{ij} = 1 ) + \mathbb{I}(g_{ij} = 0, g^{\text{True}}_{ij} = 1 ) \\
&- \mathbb{I}(g_{ij} = 0, g^{\text{True}}_{ij} = 1 )- n_j \Biggr|\\
& = \sum_j \Biggl| \sum_i \mathbb{I}(g_{ij} = 1, g^{\text{True}}_{ij} = 0 ) - \mathbb{I}(g_{ij} = 0, g^{\text{True}}_{ij} = 1 ) \\
& + \sum_i \mathbb{I}(g_{ij} = 0, g^{\text{True}}_{ij} = 1 )+ \mathbb{I}(g_{ij} = 0, g^{\text{True}}_{ij} = 1 ) - n_j \Biggr|\\
& = \sum_j \Biggl| \sum_i \mathbb{I}(g_{ij} = 1, g^{\text{True}}_{ij} = 0 ) - \mathbb{I}(g_{ij} = 0, g^{\text{True}}_{ij} = 1 ) + n_j - n_j \Biggr|\\
&= \sum_j \Biggl| \sum_i \mathbb{I}(g_{ij} = 1, g^{\text{True}}_{ij} = 0 ) -\mathbb{I}(g_{ij} = 0, g^{\text{True}}_{ij} = 1 ) \Biggr|\\
&\leq \sum_j\sum_i \mathbb{I}(g_{ij} = 1, g^{\text{True}}_{ij} = 0 ) + \mathbb{I}(g_{ij} = 0, g^{\text{True}}_{ij} = 1 )\\
&=   \sum_{j} \sum_{i}\mathbb{I}(g_{ij}  \neq g^{\text{True}}_{ij}) \\
&= M_\text{error}
\end{align*}
\end{proof}
\newpage
\subsubsection{Proof of lemma \ref{lemma:error}}
\label{Appendix:proofoflemma:error}
Now we prove lemma \ref{lemma:error}.
\begin{proof}
Recall that
\begin{itemize}
    \item $\hat{p}(C_j) = \frac{1}{n}\sum_{i=1}^n g_{ij}$ and $\hat{\mathcal{E}}(\mathcal{T}) = - \sum_{j=1}^K\hat{p}( C_j) \log(\hat{p}( C_j))$
    \item $\bar{p}(C_j) = \frac{n_j}{n}$ and $\bar{\mathcal{E}}(\mathcal{T}) = - \sum_{j=1}^K\bar{p}( C_j) \log(\bar{p}( C_j))$
\end{itemize}
\begin{align*}
    |\hat{\mathcal{E}}(\mathcal{T}) - \bar{\mathcal{E}}(\mathcal{T})| &= \left| \sum_{j=1}^k\hat{p}( C_j) \log(\hat{p}( C_j)) -  \bar{p}( C_j) \log(\bar{p}( C_j)) \right|\\
    &=  \left|\sum_{j=1}^K\hat{p}( C_j)\log(\hat{p}( C_j)) -  \hat{p}( C_j)\log(\bar{p}( C_j)) + \hat{p}( C_j)\log(\bar{p}( C_j)) -  \bar{p}( C_j) \log(\bar{p}( C_j)) \right|\\
    &= \left|\sum_{j=1}^K \hat{p}( C_j)\log\left(\frac{\hat{p}( C_j)}{\bar{p}( C_j)}\right) - \left(\hat{p}( C_j) -\bar{p}( C_j)\right)\log(\bar{p}( C_j)) \right|\\
    &=  \left|\sum_{j=1}^K \hat{p}( C_j)\log\left(\frac{\hat{p}( C_j)}{p( C_j)}\right) - \left(\hat{p}( C_j) -\bar{p}( C_j)\right)\log(\bar{p}( C_j)) \right|\\
    &\leq \left|\sum_{j=1}^K \hat{p}( C_j)\log\left(\frac{\hat{p}( C_j)}{\bar{p}( C_j)}\right)\right| + \left|\sum_{j=1}^K \left(\hat{p}( C_j) -\bar{p}( C_j)\right)\log(\bar{p}( C_j)) \right|\\
    &\leq  \left|\sum_{j=1}^K \left(\frac{\hat{p}( C_j)-\bar{p}( C_j)}{\bar{p}( C_j)}\right)\right| + \left|\sum_{j=1}^K \left(\hat{p}( C_j) -\bar{p}( C_j)\right)\log(\bar{p}( C_j)) \right|\\
    &= \left|\sum_{j=1}^K \left(\frac{\frac{1}{n}\sum_{i=1}^n g_{ij}-\bar{p}( C_j)}{\bar{p}( C_j)}\right)\right| + \left|\sum_{j=1}^K \left(\frac{1}{n}\sum_{i=1}^n g_{ij} -\bar{p}( C_j)\right)\log(\bar{p}( C_j)) \right|\\
    &= \left|\sum_{j=1}^K \left(\frac{\frac{1}{n}\left(\sum_{i=1}^n g_{ij}-n_j\right)}{\bar{p}( C_j)}\right)\right| + \left|\sum_{j=1}^K \left(\frac{1}{n}\sum_{i=1}^n g_{ij} -\bar{p}( C_j)\right)\log(\bar{p}( C_j)) \right|\\
    &\leq \sum_{j=1}^K \left|\frac{\frac{1}{n}\left(\sum_{i=1}^n g_{ij}-n_j\right)}{\bar{p}( C_j)}\right| + \sum_{j=1}^K \left|\frac{1}{n}\sum_{i=1}^n (g_{ij} - n_j)\right|\left|\log(\bar{p}( C_j)) \right|\\
    &\leq \left|\frac{\frac{2K}{n}(M_\text{error})}{c_2}\right| + \log\left(\frac{2K}{c_2}\right) \left|\frac{1}{n} (M_\text{error})\right|\\
    &= h\left(\frac{2K}{c_2}\right) \left|\frac{1}{n} (M_\text{error})\right| 
\end{align*}

where $h(x) = \left(x+\log\left(x\right)\right)$. 
\end{proof}

We prove Theorem \ref{the:strongConsistensy}. To do so, we first restate Theorem \ref{the:strongConsistensy} with all the conditions required to get to the outcome.
\begin{theorem}[Theorem \ref{the:strongConsistensy} with all conditions stated]
    Assume that Assumptions \ref{assumption:eigenvalues},  \ref{assumption:limits_nk}, \ref{assumption:bound_eigenvalues}, and \ref{assumption:K-means} hold and the \(K\)-means algorithm is applied  to \(\hat{\beta}_{i n}=(n / K)^{1 / 2} \hat{u}_{1 i}\) and \(\beta_{g_{i}^{0} n}=(n / K)^{1 / 2} \hat{O}_{n} u_{1 i}\) Then 
    \[ |\bar{\mathcal{E}}(\mathcal{T}) - \hat{\mathcal{E}}(\mathcal{T}) | \rightarrow 0 \text{ almost surely }\]
\end{theorem}

\begin{proof}

Using Theorem \ref{theorem:no_error}, we know that under Assumptions \ref{assumption:eigenvalues},  \ref{assumption:limits_nk}, \ref{assumption:bound_eigenvalues}, and \ref{assumption:K-means}, we have that 
 \[
\sup _{1 \leq i \leq n} \mathbf{1}\left\{\tilde{g}_{i} \neq g_{i}^{0}\right\}=0 \quad \text { a.s. }
\]
Using Lemma \ref{lemma:error_connections}, we know that 

$$M_\text{error} = 0 \quad \text{a.s.}$$
Using results from Lemma \ref{lemma:error}, we know that $M_\text{error} \rightarrow 0 \quad a.s. \Rightarrow \hat{\mathcal{E}}(\mathcal{T}) \rightarrow \bar{\mathcal{E}}(\mathcal{T}) \quad a.s. $. 
\end{proof}
Now we try to prove Theorem \ref{the:finite_sample}. To do so, we state corollary 3.2 in \cite{lei2015consistency}.
\subsection{Proof of Theorem \ref{the:finite_sample}}
\label{appendix:proofofthe:finite_sample}
\begin{theorem}[Corollary 3.2 in \cite{lei2015consistency}]
\label{the:finite_sample_core}
 Let $E$ be an adjacency matrix from the $\operatorname{SBM}(Z, B)$, where $B=\alpha_{n} B_{0}$ for some $\alpha_{n} \geq \log n / n$ and with $B_{0}$ having minimum absolute eigenvalue $\geq \lambda>0$ and $\max _{k \ell} B_{0}(k, \ell)=1$. Let $g_{ij}$ be the output of spectral clustering using $(1+\varepsilon)$-approximate $k$-means. Then  there exists an absolute constant $c$ such that if 

\begin{equation*}
(2+\varepsilon) \frac{K n}{n_{\min }^{2} \lambda^{2} \alpha_{n}}<c
\end{equation*}
then with probability at least $1-n^{-1}$,
$$
\frac{1}{n}M_{\text{error}}\leq c^{-1}(2+\varepsilon) \frac{K n_{\max }}{n_{\min }^{2} \lambda^{2} \alpha_{n}}
$$
\end{theorem}


\begin{proof}
We now prove Theorem  \ref{the:finite_sample}.

    Under the model we have, we know that minimum eigenvalue of $B$ is $\lambda$. Use theorem \ref{the:finite_sample_core} to replace $h\left(\frac{2K}{c_2}\right) \left|\frac{1}{n} (M_\text{error})\right|$ with $ h\left(\frac{2K}{c_2}\right) c^{-1}(2+\varepsilon) \frac{K n_{\max }}{n_{\min }^{2} \lambda^{2} \alpha_{n}} $ in lemma \ref{lemma:error}.

We now have to show the existence of $c$ in Theorem \ref{the:finite_sample_core}.

\begin{align*}
    &\quad 2Kn_{\min}/n \geq c_2 \\
    &\Rightarrow 1/n_{\min}^2 \leq 4K^2/n^2c_2^2\\
    &\Rightarrow (2+\epsilon)\frac{Kn}{n_{\min}^2\lambda^2 \alpha_n} \leq (2+\epsilon)\frac{4K^3 }{n\lambda^2 \alpha_nc_2^2}\leq (2+\epsilon)\frac{4K^3}{\lambda^2c_2^2}\\
    &\text{Let $c = (2+\epsilon)\frac{4K^3}{\lambda^2}c_2^2$}
\end{align*}
substitute $c$ to $ h\left(\frac{2K}{c_2}\right) c^{-1}(2+\varepsilon) \frac{K n_{\max }}{n_{\min }^{2} \lambda^{2} \alpha_{n}} $, we have that 
\begin{equation*}
|\bar{\mathcal{E}}(\mathcal{T}) - \hat{\mathcal{E}}(\mathcal{T}) |  \leq h\left(\frac{2K}{c_2}\right) \frac{n_{\max }}{4c_2^2n_{\min }^{2} \alpha_{n}K^2}
\end{equation*}

\end{proof}
\subsubsection{Proof of Corollary \ref{corollary:rate}}
\label{appendix:proofofcorollary:rate}
\begin{proof}
Now we prove Corollary \ref{corollary:rate}. Note that $n \geq n_{\max} \geq n_{\min} \geq nc_2/2K$.

\begin{equation*}
|\bar{\mathcal{E}}(\mathcal{T}) - \hat{\mathcal{E}}(\mathcal{T}) |  \leq h\left(\frac{2K}{c_2}\right) \frac{n_{\max }}{4c_2^2n_{\min }^{2} \alpha_{n}K^2}\leq h\left(\frac{2K}{c_2}\right) \frac{1}{c_2^4 \alpha n}
\end{equation*}
\end{proof}

\newpage
\begin{lemma}
\label{lemma:final1}
    $$|\mathcal{E} - \hat{\mathcal{E}}| \leq \sum_{j=1}^K \left( \left| \frac{p(C_j) - \bar{p}(C_j)}{p(C_j)}\right| + \log\left(\frac{1}{p(C_j)}\right)\left| p(C_j) - \bar{p}(C_j)\right|\right) + h\left(\frac{2K}{c_2}\right) \left|\frac{1}{n} (M_\text{error})\right| $$
\end{lemma}
\begin{proof}
    First, we have that 
    $$|\mathcal{E} - \hat{\mathcal{E}}| \leq |\mathcal{E} -\bar{\mathcal{E}} +\bar{\mathcal{E}}-  \hat{\mathcal{E}}| \leq |\mathcal{E} -\bar{\mathcal{E}}| + | \bar{\mathcal{E}}-  \hat{\mathcal{E}}| \leq |\mathcal{E} -\bar{\mathcal{E}}| + h\left(\frac{2K}{c_2}\right) \left|\frac{1}{n} (M_\text{error})\right| $$
    Next, 
    \begin{align*}
        |\mathcal{E} -\bar{\mathcal{E}}| &\leq  \left| \sum_{j=1}^kp( C_j) \log(p( C_j)) -  \bar{p}( C_j) \log(\bar{p}( C_j)) \right|\\  
        &\leq \left| \sum_{j=1}^kp( C_j) \log(p( C_j)) -  \bar{p}( C_j) \log(p( C_j)) + \bar{p}( C_j) \log(p( C_j)) - \bar{p}( C_j) \log(\bar{p}( C_j)) \right|\\
        &\leq \sum_{j=1}^k \left|p( C_j) \log(p( C_j)) -  \bar{p}( C_j) \log(p( C_j)) \right| + \left|\bar{p}( C_j) \log(p( C_j)) - \bar{p}( C_j) \log(\bar{p}( C_j)) \right| \\
        &\leq \sum_{j=1}^k \left|p( C_j)  -  \bar{p}( C_j)\right| \log\left(\frac{1}{p( C_j)}\right)  + \left| \frac{p( C_j) -  \bar{p}( C_j)}{p( C_j)} \right|
    \end{align*}
\end{proof}

\begin{lemma}
\label{lemma:Final2}
With probability at least $1-\frac{1}{n}$,
$$\sum_{j=1}^k \left|p( C_j)  -  \bar{p}( C_j)\right| \leq K\sqrt{\frac{1}{2n}\log(2Kn)} $$
\end{lemma}
\begin{proof}

       $$ \left|p( C_j)  -  \bar{p}( C_j)\right| = \frac{1}{n}\left|np( C_j)  -  n_j\right|$$
Now use Hoeffding bound, we notice that for any $j$
$$\mathbb{P}(|n_j - np(C_j)| \geq \delta) \leq 2\exp\left(-\frac{2\delta^2}{n}\right) $$
Using union bound 
$$\mathbb{P}(\exists j \text{ such that }|n_j - np(C_j)| \geq \delta) \leq \sum_{j=1}^K\mathbb{P}(|n_j - np(C_j)| \geq \delta) \leq 2K\exp\left(-\frac{2\delta^2}{n}\right) $$

$\exists j \text{ such that }|n_j - np(C_j)| \geq \delta \Leftarrow\max |n_j - np(C_j)| \geq \delta \Leftarrow \sum_{j=1}^K |n_j - np(C_j)| \geq K\delta.$ 

Now, let $ 2K\exp\left(-\frac{2\delta^2}{n}\right) = \frac{1}{n}$, we have that $\delta = \sqrt{\frac{n}{2}\log(2Kn)}$

This gives us that with probability at least $1-\frac{1}{n}$,

$$ \sum_{j=1}^k \left|p( C_j)  -  \bar{p}( C_j)\right| \leq K\sqrt{\frac{1}{2n}\log(2Kn)} $$ 
\end{proof}
\begin{lemma}
\label{Lemma:Final3}
With probability at least $1-\frac{1}{n}$
$$n_{\min} \geq \frac{nc_2}{2K}$$
where $c_2 = 2K\left(1-\sqrt{\frac{2\log(nK)}{np_{\min}}}\right)p_{\min}$ and $p_{\min} = \min \{p(C_1) \dots p(C_K) \}$
\end{lemma}
\begin{proof}
    Using the Chernoff inequality, we have $$\mathbb{P}\left(n_j \leq (1-\delta)np(C_j)\right) \leq \exp\left(\frac{-np(C_j)}{2}\right)$$
Using the union bound
$$\mathbb{P}(n_{\min} \leq nc_2/2K) \leq \mathbb{P}\left(\exists j \text{ such that }n_j \leq (1-\delta)np(C_j)\right) \leq K\exp\left(\frac{-np_{\min}}{2}\right) $$
Let $K\exp\left(\frac{-np_{\min}}{2}\right) = \frac{1}{n}$, we get $\delta = \sqrt{\frac{2\log(nK)}{np_{\min}}}.$
Finally, we have $c_2 = 2K\left(1-\sqrt{\frac{2\log(nK)}{np_{\min}}}\right)p_{\min}$ 

\end{proof}
\newpage
\section{Simulations}
\subsection{Hobby Examples}
We can consider a list of things that a hypothetical individual "John" likes to do in his free time: 
\begin{itemize}
    \item running / jogging 
    \item Drone flying / pilot Aerial drones
    \item jazzercise / aerobics
    \item making pottery / making ceramics
    \item water gardening / aquatic gardening
    \item caving / spelunking / potholing
    \item cycling / bicycling / biking
    \item reading
    \item writing journals / journal writings/ journaling
    \item sculling / rowing
\end{itemize}
\iffalse
\subsection{Historical Examples}
On the day December 3,
\begin{itemize}
    \item 915 – Pope John X crowns Berengar I of Italy as Holy Roman Emperor
    \item 1775 – American Revolutionary War: USS Alfred becomes the first vessel to fly the Grand Union Flag; the flag is hoisted by John Paul Jones.
    \item 1800 – War of the Second Coalition: Battle of Hohenlinden: French General Jean Victor Marie Moreau decisively defeats the Archduke John of Austria near Munich. Coupled with First Consul Napoleon Bonaparte's earlier victory at Marengo, this will force the Austrians to sign an armistice and end the war.
    \item 1818 – Illinois becomes the 21st U.S. state.
    \item 1834 – The Zollverein (German Customs Union) begins the first regular census in Germany.
    \item 1898 – The Duquesne Country and Athletic Club defeats an all-star collection of early football players 16–0, in what is considered to be the first all-star game for professional American football.
    \item 1920 – Following more than a month of Turkish–Armenian War, the Turkish-dictated Treaty of Alexandropol is concluded.
    \item 1929 – President Herbert Hoover delivers his first State of the Union message to Congress. It is presented in the form of a written message rather than a speech
    \item 1959 – The current flag of Singapore is adopted, six months after Singapore became self-governing within the British Empire.
    \item 1979 – In Cincinnati, 11 fans are suffocated in a crush for seats on the concourse outside Riverfront Coliseum before a Who concert.
    \item 1979 – Iranian Revolution: Ayatollah Ruhollah Khomeini becomes the first Supreme Leader of Iran.
\end{itemize}
\fi
\newpage
\section{Prompt}
\label{appendix_sec:prompt_engineering}
This is the prompt we inserted for "Phi-3-mini-4k-instruct", "AI21-Jamba-1.5-Mini", "Cohere-command-r-08-2024".


\begin{verbatim}
'''
    You are a expert in logical deduction and you are given 2 piece of texts: TEXT A and TEXT B. 
    You are to identify if TEXT A implies TEXT B and TEXT B implies TEXT A at the same time. 
    
    TEXT A: 
    {text_A}
    
    TEXT B:
    {text_B}
    
    ## OUTPUT
    You are to return TRUE if TEXT A implies TEXT B and TEXT B implies TEXT A at the same time. 
    otherwise, you are to return FALSE 
'''
\end{verbatim}

This is the prompt we inserted for "Ministral-3B","Llama-3.3-70B-Instruct", "gpt-35-turbo"

\begin{verbatim}
''' 
    You are a expert in logical deduction and you are given 2 piece of texts: TEXT A and TEXT B. 
    You are to identify if TEXT A implies TEXT B and TEXT B implies TEXT A at the same time. 
    
    TEXT A: 
    {text_A}
    
    TEXT B:
    {text_B}
    
    ## OUTPUT
    You are to return TRUE if TEXT A implies TEXT B and TEXT B implies TEXT A at the same time. 
    otherwise, you are to return FALSE 
    
    ##FORMAT:
    START with either TRUE or FALSE, then detail your reasoning
'''
\end{verbatim}
\end{document}

% Shorter URLs with https://
% From: https://tex.stackexchange.com/a/139934
\newcommand\rurl[1]{%
  \href{https://#1}{\nolinkurl{#1}}%
}

\title{Behavioral Analysis of Information Salience in Large Language Models}

\author{
Jan Trienes$^{1}$\quad
Jörg Schlötterer$^{1,2}$\quad
{\bf Junyi Jessy Li}$^3$\quad
{\bf Christin Seifert}$^1$\\
$^1$Marburg University\quad
$^2$University of Mannheim\\
$^3$The University of Texas at Austin\\
\texttt{\normalsize \{jan.trienes,joerg.schloetterer,christin.seifert\}@uni-marburg.de}\\
\texttt{\normalsize jessy@utexas.edu}}


\def\gpt{GPT\=/4o}
\def\gptmini{GPT\=/4o\=/mini}

\newcommand\unfootnote[1]{%
  \begingroup
  \renewcommand\thefootnote{}\footnote{#1}%
  \addtocounter{footnote}{-1}%
  \endgroup
}



\begin{document}
\maketitle

During the early stages of interface design, designers need to produce multiple sketches to explore a design space.  Design tools often fail to support this critical stage, because they insist on specifying more details than necessary. Although recent advances in generative AI have raised hopes of solving this issue, in practice they fail because expressing loose ideas in a prompt is impractical. In this paper, we propose a diffusion-based approach to the low-effort generation of interface sketches. It breaks new ground by allowing flexible control of the generation process via three types of inputs: A) prompts, B) wireframes, and C) visual flows. The designer can provide any combination of these as input at any level of detail, and will get a diverse gallery of low-fidelity solutions in response. The unique benefit is that large design spaces can be explored rapidly with very little effort in input-specification. We present qualitative results for various combinations of input specifications. Additionally, we demonstrate that our model aligns more accurately with these specifications than other models. 

% OLD ABSTRACT
%When sketching Graphical User Interfaces (GUIs), designers need to explore several aspects of visual design simultaneously, such as how to guide the user’s attention to the right aspects of the design while making the intended functionality visible. Although current Large Language Models (LLMs) can generate GUIs, they do not offer the finer level of control necessary for this kind of exploration. To address this, we propose a diffusion-based model with multi-modal conditional generation. In practice, our model optionally takes semantic segmentation, prompt guidance, and flow direction to generate multiple GUIs that are aligned with the input design specifications. It produces multiple examples. We demonstrate that our approach outperforms baseline methods in producing desirable GUIs and meets the desired visual flow.

% Designing visually engaging Graphical User Interfaces (GUIs) is a challenge in HCI research. Effective GUI design must balance visual properties, like color and positioning, with user behaviors to ensure GUIs easy to comprehend and guide attention to critical elements. Modern GUIs, with their complex combinations of text, images, and interactive components, make it difficult to maintain a coherent visual flow during design.
% Although current Large Language Models (LLMs) can generate GUIs, they often lack the fine control necessary for ensuring a coherent visual flow. To address this, we propose a diffusion-based model that effectively handles multi-modal conditional generation. Our model takes semantic segmentation, optional prompt guidance, and ordered viewing elements to generate high-fidelity GUIs that are aligned with the input design specifications.
% We demonstrate that our approach outperforms baseline methods in producing desirable GUIs and meets the desired visual flow. Moreover, a user study involving XX designers indicates that our model enhances the efficiency of the GUI design ideation process and provides designers with greater control compared to existing methods.    



% %%%%%%%%%%%%%%%%%%%%%%%%%%%%%%%%%%%%%%%%%%%%%%%%%%%%%%
% % Writing Clinic Comments:
% %%%%%%%%%%%%%%%%%%%%%%%%%%%%%%%%%%%%%%%%%%%%%%%%%%%%%%
% % Define: Effective UI design
% % Motivate GANs and write in full form.
% % LLMs vs ControlNet vs GANs
% % Say something about the Figma plugin?
% % Write the work is novel or what has been done before
% % What is desirable UI and how to evalutate that?
% % Visual Flow - main theme (center around it)
% % Re-Title: use word Flow!
% % Use ControlNet++ & SPADE for abstract.
% % Write about input/output. 
% % Why better than previous work?
% %%%%%%%%%%%%%%%%%%%%%%%%%%%%%%%%%%%%%%%%%%%%%%%%%%%%%

% % v2:
% % \noindent \textcolor{red}{\textbf{NEW Abstract!} (Post Writing Clinic 1 - 25-Jun)}

% % \noindent \textcolor{red}{----------------------------------------------------------------------}

% % \noindent Designing user interfaces (UIs) is a time-consuming process, particularly for novice designers. 
% % Creating UI designs that are effective in market funneling or any other designer defined goal requires a good understanding of the visual flow to guide users' attention to UI elements in the desired order. 
% % While current Large Language Models (LLMs) can generate UIs from just prompts, they often lack finer pixel-precise control and fail to consider visual flow. 
% % In this work, we present a UI synthesis method that incorporates visual flow alongside prompts and semantic layouts. 
% % Our efficient approach uses a carefully designed Generative Adversarial Network (GAN) optimized for scenarios with limited data, making it more suitable than diffusion-based and large vision-language models.
% % We demonstrate that our method produces more "desirable" UIs according to the well-known contrast, repetition, alignment, and proximity principles of design. 
% % We further validate our method through comprehensive automatic non-reference, human-preference aligned network scoring and subjective human evaluations.
% % Finally, an evaluation with xx non-expert designers using our contributed Figma plugin shows that <method-name> improves the time-efficiency as well as the overall quality of the UI design development cycle.

% % \noindent \textcolor{red}{----------------------------------------------------------------------}


% \noindent \textcolor{blue}{\textbf{NEW Abstract!} (Pre Writing Clinic 9-July)}

% \noindent \textcolor{blue}{----------------------------------------------------------------------}

% \noindent Exploring different graphical user interface (GUI) design ideas is time-consuming, particularly for novice designers. 
% Given the segmentation masks, design requirement as prompt, and/or preferred visual flow, we aim to facilitate creative exploration for GUI design and generate different UI designs for inspiration.
% While current Vision Language Models (VLMs) can generate GUIs from just prompts, they often lack control over visual concepts and flow that are difficult to convey through language during the generation process. 
% In this work, we present FlowGenUI, a semantic map-guided GUI synthesis method that optionally incorporates visual flow information based on the user's choice alongside language prompts. 
% We demonstrate that our model not only creates more realistic GUIs but also creates "predictable" (how users pay attention to and order of looking at GUI elements) GUIs.
% Our approach uses Stable Diffusion (SD), a large paired image-text pretrained diffusion model with a rich latent space that we steer toward realistic GUIs using a trainable copy of SD's encoder for every condition (segmentation masks, prompts, and visual flow). 
% We further provide a semantic typography feature to create custom text-fonts and styles while also alleviating SD's inherent limitations in drawing coherent, meaningful and correct aspect-ratio text. 
% Finally, a subjective evaluation study of XX non-expert and expert designers demonstrates the efficiency and fidelity of our method.


% This process encourages creativity and prevents designers from falling into habitual patterns.


% ------------------------------------------------------------------
% Joongi Why is it important to create realistic GUI?
% I do not see how the Visual Flow given on the left hand side is reflected in the results on the right hand side. 
% I’d avoid making unsubstantiated claims about designers (falling into habitual patterns).
% The UIs you generate do not “align with users’ attention patterns” but rather try to control it (that’s what visual flow means)
% ------------------------------------------------------------------
% Comments - Writing Clinic - 9th July:
% Improve title. More names: FlowGen
% Figure 1: Use an inference time hand-drawn mask
% Figure 1: Show both workflows. Add a designer --> Input.
% Figure 1: Make them more diverse
% ------------------------------------------------------------------
% Designing graphical user interfaces (GUIs) requires human creativity and time. Designers often fall into habitual patterns, which can limit the exploration of new ideas. 
% To address this, we introduce FlowGenUI, a method that facilitates creative exploration and generates diverse GUI designs for inspiration. By using segmentation masks, design requirements as prompts, and/or selected visual flows, our approach enhances control over the visual concepts and flows during the generation process, which current Vision Language Models (VLMs) often lack.
% FlowGenUI uses Stable Diffusion (SD), a largely pretrained text-to-image diffusion model, and guides it to create realistic GUIs. 
% We achieve this by using a trainable copy of SD's encoder for each condition (segmentation masks, prompts, and visual flow). 
% This method enables the creation of more realistic and predictable GUIs that align with users' attention patterns and their preferred order of viewing elements.
% We also offer a semantic typography feature that creates custom text fonts and styles while addressing SD's limitations in generating coherent, meaningful, and correctly aspect-ratio text.
% Our approach's efficiency and fidelity are evaluated through a subjective user study involving XX designers. 
% The results demonstrate the effectiveness of FlowGenUI in generating high-quality GUI designs that meet user requirements and visual expectations.

% ---------------------------------------


%A critical and general issue remains while using such deep generative priors: creating coherent, meaningful and correct aspect-ratio text. 
%We tackle this issue within our framework and additionally provide a semantic typography feature to create custom text-fonts and styles. 


% %Creating UI designs that are effective in market funneling or any other designer-defined goal requires a good understanding of the visual flow to guide users' attention to UI elements in the desired order. 
% %While current largely pre-trained Vision Language Models (VLMs) can generate GUIs from just prompts, they often lack finer or pixel-precise control which can be crucial for many easy-to-understand visual concepts but difficult to convey through language. 
% % However, obtaining such pixe-level labels is an extremely expensive so we
% % For example - overlaying text on images with certain aspect ratios and two equally separated buttons 
% Additionally, all prior GUI generation work fails to consider visual flow information during the generation process. 
% We demonstrate that visual flow-informed generation not only creates more realistic and human-friendly GUIs but also creates "predictable" (how users pay attention to and order of looking at GUI elements) UIs that could be beneficial for designers for tasks like creating effective market funnels.
% In this work, we present a semantic map-guided GUI synthesis method that optionally incorporates visual flow information based on the user's choice alongside language prompts. 
% Our approach uses Stable Diffusion, a large (billions) paired image-text pretrained diffusion model with a rich latent space that we steer toward realistic GUIs using an ensemble of ControlNets. 
% % TODO: Mention it in 1 sentence:
% A critical and general issue remains while using such deep generative priors: creating coherent, meaningful and correct aspect-ratio text. 
% We tackle this issue within our framework and additionally provide a semantic typography feature to create custom text-fonts and styles. 
% To evaluate our method, we demonstrate that our method produces more "desirable" UIs according to the well-known contrast, repetition, alignment, and proximity principles of design. 
% % We further validate our method through comprehensive automatic non-reference and human-preference aligned scores. (TODO: Maybe Unskip if we get UIClip from Jason!)
% % TODO: Re-word this and only keep ideation cycles and time-efficiency.
% Finally, a subjective evaluation study of XX non-expert and expert designers demonstrates the efficiency and fidelity of our method.
% % improves the time-efficiency by quick iterations of the UI design ideation process.
% %Finally, an evaluation with xx non-expert designers using our contributed <method-name> improves the time-efficiency by quick iterations of the UI design ideation cycle.

%\noindent \textcolor{blue}{----------------------------------------------------------------------}


%In an evaluation with xx designers, we found that GenerativeLayout: 1) enhances designers' exploration by expanding the coverage of the design space, 2) reduces the time required for exploration, and 3) maintains a perceived level of control similar to that of manual exploration.



% Present-day graphical user interfaces (GUIs) exhibit diverse arrangements of text, graphics, and interactive elements such as buttons and menus, but representations of GUIs have not kept up. They do not encapsulate both semantic and visuo-spatial relationships among elements. %\color{red} 
% To seize machine learning's potential for GUIs more efficiently, \papername~ exploits graph neural networks to capture individual elements' properties and their semantic—visuo-spatial constraints in a layout. The learned representation demonstrated its effectiveness in multiple tasks, especially generating designs in a challenging GUI autocompletion task, which involved predicting the positions of remaining unplaced elements in a partially completed GUI. The new model's suggestions showed alignment and visual appeal superior to the baseline method and received higher subjective ratings for preference. 
% Furthermore, we demonstrate the practical benefits and efficiency advantages designers perceive when utilizing our model as an autocompletion plug-in.


% Overall pipeline: Maybe drop semantic typography / visual flow?
\section{Introduction}
Large Language Models (LLMs) significantly advanced text synthesis tasks, including text summarization, which they perform well even under zero-shot conditions~\cite{Goyal:2023:arXiv,Zhang:2024:TACL}.
The nature of the summarization task requires models to do content selection: picking the most salient pieces of information for inclusion in a summary \cite{Mani:1999:advances}.
However, it remains unclear what underlying notion of salience the models have internalized.

Prior work investigated information salience from several angles.
Theories of discourse structure have been used to induce content salience \cite{Marcu:1999:advances,Louis:2010:SIGDIAL}, and a large body of summarization research uses word distribution or centrality as the main signal for content selection \cite{Nenkova:2012:survey,Nazari:2019:survey}. \citet{Peyrard:2019:ACL} laid out a theoretical perspective for content salience in summarization, though the exact notion remains largely latent and aloof; rather, pre-LLM summarization work uses human summaries as supervision signals to learn what to include \cite[][\emph{inter alia}]{Gehrmann:2018:EMNLP,Chen:2018:ACL,Liu:2019:EMNLP}.
Yet, none of these accounts explains why LLM zero-shot summarization works so well on the one hand, while missing key elements on the other~\cite{Kim:2024:COLM,Trienes:2024:ACL,Huang:2024:NAACL}.

To begin to make sense of this behavior, we need to understand how models internalize salience: whether it is a \emph{consistent} notion within and across models, \emph{how} they prioritize information, and whether LLMs' notion of salience \emph{aligns} with prior theories or human intuitions.

In this paper, we present a novel explainable framework to systematically derive and investigate LLMs' grasp of information salience through their summarization behavior.
Our method combines \textbf{two key ideas}. First, we can use length-constrained summarization~\cite{Fan:2018:NGT,He:2022:EMNLP} as a behavioral probe into the content selection process of LLMs.
Intuitively, when there is a limited length budget for a summary, we posit that the least important information is dropped first.

\begin{figure*}[t]
\includegraphics[width=\textwidth]{figures/overview}
\caption{
    Framework overview, conceptualizing content salience as question answerability.
    \textbf{Left:} Given a corpus, we derive questions that are typically answered in summaries. Length-controlled summarization acts as a probe into the content-selection process of LLMs.
    Question paraphrases are clustered by semantic intent.
    \textbf{Middle:} Answerability is calculated as the fraction of document-answer claims entailed by the summary.
    \textbf{Right:} The content salience map tracks answerability at each summary length. More salient questions remain answerable even in shorter summaries.
}
\label{fig:overview}
\end{figure*}

Second, we can describe what is salient as the answerability of domain-relevant Questions Under Discussion (QUDs;~\citealp{Van:1995:discourse,Benz:2017:questions,Wu:2023:EMNLP}). QUDs can be thought of as representations of a coherent unit of information in the form of information-seeking questions, e.g., \emph{Who are the participants of this study?}
We use such questions --- and hence their answers extracted from documents, according to alternative semantics \cite{Hamblin:1973:Questions,Karttunen:1977:syntax,Groenendijk:1984:studies} --- as the primary unit of analysis, making our framework interpretable and customizable.

Taken together, by gradually decreasing the length budget available for a summary and by systematically tracing question answerability throughout, we can derive a \emph{proxy} for how models prioritize information.
See~\cref{fig:overview} for an overview.

Using this framework (\cref{sec:method}), we empirically study LLMs' content selection \emph{behavior}, and its alignment with \emph{perceived} notions of salience. Through experiments on 13 models and four datasets (\cref{sec:experimental-settings}), we aim to answer the following research questions:
%
\begin{itemize}[noitemsep,leftmargin=27pt]
    \item[\textbf{RQ1}] What notion of salience have LLMs learned in different domains?
    \item[\textbf{RQ2}] Do LLMs of different families/sizes have a similar notion of salience?
    \item[\textbf{RQ3}] When models introspect, does their perceived notion of salience align with their summarization behavior?
    \item[\textbf{RQ4}] To what extent does model salience align with human perceived salience?
\end{itemize}
%
We find that LLMs have a nuanced notion of salience prioritizing information hierarchically across summary lengths (\cref{sec:results-salience}).
Also the notion of salience is generally compatible between models even of different families and sizes, though more recent/bigger LLMs correlate more strongly with GPT-4o (\cref{sec:results-model-model-similarity}).
Furthermore, models show highly consistent behavior and hence notions of salience, but it cannot be elicited through introspection (i.e., directly prompting for salience of topics; \cref{sec:results-introspection}). Lastly, we find that model behavior only weakly aligns with human perceptions of salience (\cref{sec:results-human-alignment}).

\section{User Study} \label{user_study}
To investigate the impact of a robot's ability to understand indirect speech acts on people's perception, we conducted a Wizard-of-Oz experiment with 36 participants on three different physical collaborative tasks.

The experiment employed a mixed-method experimental design~\cite{creswell1999mixed}, collecting quantitative data through a questionnaire on team fluency, goal alignment, performance trust, and anthropomorphism as dependent variables, as well as qualitative data from interview responses. The Speech Mode (ISA vs. Non-ISA) served as a between-subject factor, with half of the participants interacting with a robot capable of understanding ISAs, while the other half interacted with a robot unable to comprehend ISAs. Each participant completed three tasks in counter-balanced order with the robot using one of the assigned Speech Modes, which was followed by a semi-structured interview. We provide additional detail on our experimental design in the following sections.

\begin{figure*}
    \centering
    \includegraphics[width=\textwidth]{exp.pdf}
    \caption{(a) Experiment setup: The participant, robot, and experimenter 1 were all present in the same room. The participant and robot were seated on opposite sides of a table, with the shared workspace located in the centre. Experimenter 1 (Speech Wizard) sat next to the robot, near the emergency button, and operated the robot's speech-WoZ interface. Experimenter 2 (Motion Wizard) was positioned behind a one-side mirror, allowing for a clear view of the room, and was responsible for teleoperating the robot's arm movements. (b) Experiment procedure: Each participant first completed a pre-questionnaire before being assigned to either the ISA or non-ISA group. The participant then performed three tasks with the robot in a counter-balanced order. Before each task, participants watched a tutorial video and read a task description. After completing each task, they filled out a post-task questionnaire. The experiment concluded with a semi-structured interview.\vspace{-1em}}
    \label{fig:exp}
\end{figure*}

\subsection{Experimental Design}
\subsubsection{Apparatus and Setup}
We used TIAGo as the robot agent in our study. TIAGo is a mobile manipulator robot with anthropomorphic features, including a head, neck, torso, and arm, making it well-suited for HRI research~\cite{pages2016tiago}. The robot and the participant were on the opposite side of a table, which acted as the shared workspace between the two parties. Given the current limitations of algorithms in achieving human-level understanding and generating accurate verbal responses to ISAs, and to minimise the influence of potential robotic failures on experimental outcomes, we chose to use a Wizard-of-Oz (WoZ) approach. WoZ is a classic methodology in HRI research, where human operators discretely control the robot’s behaviour to simulate advanced robotic capabilities that the system itself may not be able to achieve autonomously yet or that would not be robust for real-time interaction~\cite{martelaro2016wizard}. This approach allows researchers to focus on understanding user interactions with the robot without being hindered by technological limitations or safety issues related to autonomous motion control. 
In this experiment, two experimenters discreetly controlled the robot to provide realistic and fluid responses, enabling a better assessment of human-robot interaction dynamics.

One of the experimenters (Motion Wizard) was teleoperating the robot's movement behind a one-side mirror, which allowed them to have a clear view of both the robot and the participant while remaining hidden from the participant.
This teleoperation was possible thanks to custom-made software developed by our team that allowed the Motion Wizard to send commands to the robot remotely. This WoZ software was built using the Robotics Operating System (ROS) and the TIAGo API. We implemented the arm actions using inverse kinematics, which calculates the joint configuration based on the desired Cartesian coordinates of the end effector~\cite{chitta2017}. In addition to moving the end effector within a 3D space above the workspace, the robot's head also had 2 degrees of freedom, which allowed the Motion Wizard to observe through the robot's camera and actively engage with the participant. Safety was ensured by a collision detection function that automatically disabled the arm controller when abnormal tolerance values were detected in the joints. Virtual walls were also implemented around the robot's arm to restrict its movement, preventing it from exceeding a designated range or approaching the participant too closely.

The other experimenter (Speech Wizard) was sitting beside the robot's emergency button and operating the speech-WoZ interface through a laptop to give verbal responses, which were scripted in advance (See \autoref{speech_model} in detail). Participants were informed that the Speech Wizard served as a safeguard, responsible for ensuring their physical safety by using the emergency button located on the robot's base if necessary. This explanation led participants to view the Speech Wizard’s presence as a precautionary measure. TIAGo utilises Acapela Group's Text-to-Speech technology, which carries out the phonetic transcription of the text, generates prosody for the speech, produces the audio signal, and plays through TIAGo's speaker. \autoref{fig:exp}a demonstrates the experimental setup.


\begin{table*}[t]
\caption{Examples of participants' requests, interpretations, and robot's responses in different Speech Modes. The request examples are from \autoref{fig:teaser}. (P: participant; R: robot)}
\label{tb:example}
\resizebox{0.95\textwidth}{!}{
\begin{tabular}{llll}
\hline
\multirow{2}{*}{Request Examples} &
  \multirow{2}{*}{Interpretations} &
  \multicolumn{2}{c}{Robot's Responses} \\ \cline{3-4} 
 &
   &
  If in the ISA group &
  If in the Non-ISA group \\ \hline
P: Please also sort these cubes. &
  \begin{tabular}[c]{@{}l@{}}\textbf{Direct}\\ \textit{Literal}: Sort cubes.\\ \textit{Intent}: Sort cubes.\end{tabular} &
  \multicolumn{2}{c}{\begin{tabular}[c]{@{}c@{}}R: Yes, sure. (Act on the intent)\end{tabular}} \\ \hline
  \begin{tabular}[c]{@{}l@{}}P: Can you move it to here\\ please? \end{tabular} &
  \begin{tabular}[c]{@{}l@{}}\textbf{Indirect}\\ \textit{Literal}: Ask for the ability to move it.\\ \textit{Intent}: Move it.\end{tabular} &
  \begin{tabular}[c]{@{}l@{}}R: Got it.\\ (Act on the intent)\end{tabular} &
  \begin{tabular}[c]{@{}l@{}}R: Yes, I can do that.\\ (No action)\end{tabular} \\ \hline
\begin{tabular}[c]{@{}l@{}}P: You could take the small\\ yellow cylinder ...\end{tabular} &
  \begin{tabular}[c]{@{}l@{}}\textbf{Indirect}\\ \textit{Literal}: Suggest an action option to\\ take the cylinder.\\ \textit{Intent}: Take the cylinder.\end{tabular} &
  \begin{tabular}[c]{@{}l@{}}R: Okay.\\ (Act on the intent)\end{tabular} &
  \begin{tabular}[c]{@{}l@{}}R: Well noted.\\ (No action)\end{tabular} \\ \hline
 \begin{tabular}[c]{@{}l@{}}P: Let's move on to the third\\ column.  \end{tabular} &
  \begin{tabular}[c]{@{}l@{}}\textbf{Indirect}\\ \textit{Literal}: Suggest moving on to the third\\ column.\\ \textit{Intent}: Sort the third column.\end{tabular} &
  \begin{tabular}[c]{@{}l@{}}R: Working on that.\\ (Act on the intent)\end{tabular} &
  \begin{tabular}[c]{@{}l@{}}R: It's a good suggestion.\\ (No action)\end{tabular} \\ \hline
P: And the last one. &
  \begin{tabular}[c]{@{}l@{}}\textbf{Indirect}\\ \textit{Literal}: A reference to the last thing.\\ \textit{Intent}: Rotate to the last face.\end{tabular} &
  \begin{tabular}[c]{@{}l@{}}R: Sure.\\ (Act on the intent)\end{tabular} &
  \begin{tabular}[c]{@{}l@{}}R: ...\\ (Silence. No action)\end{tabular} \\ \hline
\begin{tabular}[c]{@{}l@{}}P: The blue block needs to ...\\ P: Oh, actually. Wait.\end{tabular} &
  \begin{tabular}[c]{@{}l@{}}\textbf{Indirect}\\ \textit{Literal}: Provide information for the blue\\block and wait.\\ \textit{Intent}: Move the blue block to a position. \\ The last command is wrong, stop the\\ current action and wait for the next one.\end{tabular} &
  \begin{tabular}[c]{@{}l@{}}R: Got it.\\ (Act on the intent, \\ then stop halfway)\end{tabular} &
  \begin{tabular}[c]{@{}l@{}}R: Thank you for the info.\\ (No action)\end{tabular} \\ \hline
\end{tabular}
}
\end{table*}

\subsubsection{Robot's Speech Understanding} \label{speech_model}
The robot's Speech Mode was a between-subject independent variable with two conditions. In the ISA condition, the robot could understand participants' ISAs and respond with appropriate actions. In the Non-ISA condition, the robot was only able to grasp the literal meaning of requests and respond to commands that were stated in imperative sentences. The literal meaning of ISAs was interpreted by isolating them from their contextual elements, following the guidelines of Searle's putative facts~\cite{searle1975indirect}. We selected some representative requests from participants to demonstrate how the direct and indirect speech acts were interpreted and responded to during our experiment (shown in \autoref{tb:example} and \autoref{fig:teaser}). To respond to both indirect and direct requests, the speech-WoZ interface featured predefined sentences, such as ``Sure,'' ``Okay, working on that,'' and ``Yes, I have the ability to do that.'' Utterances without command intent, such as ``Thank you, TIAGo,'' were responded to as natural conversational exchanges like ``You're welcome.'' The selection of phrases was guided by the aforementioned literature and further refined through insights gained from four pilot studies. The interface also provided a text box that allowed the Speech Wizard to input responses to any unexpected speech. To maintain the flow of interaction and avoid constraining the use of ISAs, participants were allowed to use gestures along with their speech, which experimenters interpreted and responded to accordingly. Notably, no participants reported noticing that the experimenter sitting in front of them was controlling the robot's speech.

\subsubsection{Collaboration Tasks}
A recent systematic review~\cite{semeraro2023human} categorised the HRC tasks for robotic manipulators as: (1) collaborative assembly, where humans and robots work together to assemble complex objects through a series of sequential sub-processes; (2) object handling \& handover, involving the joint grasping and placement of objects by humans and robots, as well as the handover of objects from the robot to the human; and (3) collaborative manufacturing, where both humans and robots perform tasks that permanently alter an object, such as polishing and drilling. For safety considerations, we modified the object handling and handover task to a turn-based pick-and-place activity, where both the robot and the human participated in sorting cubes. Based on this taxonomy, we designed and implemented three physical collaborative tasks for our experiment: (1) a foam brick assembly task~\cite{vogt2016learning}, (2) a 3*3 cubes sorting task~\cite{faroni2020layered}, and (3) a hexagonal prism polishing task~\cite{nikolaidis2015efficient}. 
In each task, the robot lacked prior information about the task's goal and plan, requiring the participant to relay the instructions to the robot at the beginning and verbally guide the team's actions throughout the entire activity.

The \textbf{assembly} task (\autoref{fig:exp}bi) required the human-robot team to build a structure using foam bricks. We distributed 18 bricks of various shapes between the human and robot, with 12 bricks required for constructing the target structure and 6 incorrect bricks that should not be used.  
Only the participant was provided with a photo of the structure they had to build, while the robot had no prior knowledge of the structure. The bricks were initially randomly placed in the robot or the participant's stock, and each party was only allowed to take bricks from their own pile. 
Participants could only manipulate the bricks on their side and needed to communicate and coordinate with the robot to have it add its bricks to the construction. 

The \textbf{sorting} task (\autoref{fig:exp}bii) used nine 5*5*5cm cubes that needed to be rearranged according to two categorical attributes: texture and version.  Each cube featured a type of surface texture (smooth, medium, rough), and an ArUco marker~\cite{garrido2014automatic} encoding its version information (old, intermediate, new). 
To mimic an information asymmetry sorting task, the participants were able to touch and feel the texture, whereas the robot could scan the ArUco marker to access the cube's version.
We used apparently similar ArUco markers for version information to make it impossible for participants to distinguish between them by sight alone. Only the exchange of information between teammates made it possible to achieve the task: to arrange the cubes in a gradient from rough to smooth in one dimension and from old to new in the orthogonal direction. 

The \textbf{polishing} task (\autoref{fig:exp}biii) constituted a simple instantiation of a manufacturing task. The robot was responsible for holding and turning the hexagonal prism, and the participant polished each surface three times using sandpaper. Every time the participants were happy with the sanding, they had to communicate to the robot to turn the object to show a face that had not been polished. This scenario was designed to simulate a situation where the hexagonal prism was too heavy or hazardous for a human to lift and rotate, requiring cooperation with the robot to successfully complete the task.


\subsection{Participants}
We conducted \textit{a priori} power analysis to calculate the sample size for our experiment using \textit{G*Power}~\cite{faul2007g}. The calculation was based on a medium effect size of $f=0.25$, an alpha-level of 0.05, and a power of 0.9. As a result, we recruited 36 ($Female:Male = 19:17$,  $M_{age} = 24.08$, $Std_{age} = 5.75$) participants who were all fluent English speakers. We used the three questions of the interaction subscale from the Negative Attitude Toward Robots Scale (NARS Questionnaire) \cite{nomura2006measurement}, as they were relevant to working and talking to a robot (see \autoref{apd:NARS}). These questions were used to screen out individuals who exhibited strong negative responses towards robots and who could possibly feel distressed interacting with a robot (i.e., a rating higher than 3). Each experiment took about 60 minutes and participants were compensated with a \$30 voucher. Our experiment received ethics approval from the Institutional Review Board (IRB).

\subsection{Procedure}
The experiment procedure is shown in \autoref{fig:exp}b. Upon welcoming the participants, the study started with a pre-questionnaire, which captured participant demographics and their prior interaction experience with robots, voice assistants, and in performing physical collaborative tasks. The prior experience served as covariates in data analysis. Before each task, participants were provided with a tutorial video and a written task description, which included instructions and specified the objectives of the task. Additionally, a picture of the target structure for the assembly task was presented to illustrate the final goal. 
Participants were required to lead the collaboration and verbally relay the team's objectives to their robot teammate, TIAGo. After each task, participants completed a post-task questionnaire that assessed their perceptions of the team's fluency and goal alignment~\cite{hoffman2010effects}, performance trustworthiness using Multi-Dimensional Measure of Trust (MDMT)~\cite{malle2021multidimensional}, and the robot's anthropomorphism using the Godspeed Questionnaire (GSQ)~\cite{bartneck2023godspeed}. Each participant interacted with one of the robot's Speech Modes (ISA or Non-ISA) and engaged in three tasks, which were assigned in a counter-balanced order. Finally, the study ended with a semi-structured interview. Each experiment took about 60 minutes, including the interview.

\subsection{Data Collection and Analysis}
We collected the quantitative data using standard questionnaires and the qualitative data through a semi-structured interview. The following dependent variables were collected after each task:

\begin{itemize}
    \item Team fluency: To answer RQ1.1, we used the 7-point team fluency sub-scale with 3 items, from~\cite{hoffman2010effects}, which adapted the Working Alliance Inventory~\cite{horvath1989development} on Human-Robot Collaboration.
    \item Goal alignment: For the goal alignment in RQ1.2, we utilised the 7-point goal sub-scale with 3 items, from~\cite{hoffman2010effects}. 
    \item Performance trust: The 4-item MDMT performance trust scale results were collected to measure participants' perceived capability and reliability of the robot (RQ2). The scale has 5 points and an additional option for ``Does not fit'' to prevent forced and possibly meaningless ratings~\cite{malle2021multidimensional}.
    \item Anthropomorphism: To answer RQ3, the 5-point anthropomorphism sub-scale with 5 items of the GSQ was used. As the study was focused on the robot's understanding of communication rather than the appearance of the robot, the last item, ``Moving rigidly/elegantly'', was changed to ``communicating rigidly/elegantly'', which has been shown to be reliable by~\cite{laban2019working}. 
\end{itemize}

\vspace{-2em}

To analyse the impact of the robot's Speech Modes (ISA vs. Non-ISA) and covariates (participants' prior interaction experience with robots, voice assistants, and physical collaborative tasks), we used Cumulative Link Mixed Models (CLMMs) via the "ordinal" package in R~\cite{christensen2019ordinal}. This analysis is appropriate given the ordinal nature of our dependent variables. Additionally, task type, scales' sub-item ID, and participant ID were included as random effects in our model to account for potential variability within group structures and repeated measures~\cite{brown2021introduction}.

At the end of the experiment, we conducted a semi-structured interview lasting approximately 15 minutes to gather qualitative feedback from participants. The Motion Wizard observed participants' behaviours during the experiment. Instances of participants using indirect speech acts were further explored through follow-up questions during the interviews. The interviews were intended to supplement the quantitative results and provide insight into their subjective feelings regarding the overall experience during the collaboration.
Given the between-subjects design of the study, we began the interview by explaining the experimental condition that participants had not experienced, ensuring they had a comprehensive understanding of the study. We disclosed that the experimenters controlled the robot's actions and speech only after the interview concluded.

The interview results were transcribed and analysed through reflexive thematic analysis (RTA), which was well-suited to this study because it emphasised the researchers’ active role in constructing themes, thereby fostering flexibility, creativity, and critical reflection. This approach permits researchers to integrate their own insights and observations from the experimental process, making it particularly effective for exploring subtle phenomena~\cite{braun2023doing}. Following the 6-phase guidance by~\cite{braun2006using}, two authors of this paper, both of whom possess substantial expertise in human-robot interaction and human-computer interaction, conducted the RTA. In phase 1, researchers thoroughly reviewed all transcriptions. In phase 2, they inductively generated initial codes at the sentence level, which were either semantic, representing participants' explicit feelings, or latent, reflecting deeper meanings inferred from the data based on researchers' knowledge background. In phase 3, they constructed the initial themes and categorised the codes. Up to this point, the work had been carried out individually by each researcher. In phase 4, two researchers cooperatively discussed and reviewed the themes through multiple rounds. In phase 5, the themes were defined and named. In phase 6, researchers drafted the initial report of the qualitative analysis. Phases 4 to 6 were repeated over several rounds, during which the themes were iteratively refined and discrepancies addressed. This process aligns with the RTA principles, which emphasise continuous iterative reflexivity to ensure the analysis remains progressively recursive~\cite{terry2017thematic}.


\section{Experimental Settings}
\label{sec:experimental-settings}

\begin{table}[t]
\small
\centering
\setlength{\tabcolsep}{4pt}
\begin{tabular}{lrrrr}
\toprule
\textbf{Statistic} & \textbf{RCT} & \textbf{CL} & \textbf{Astro} & \textbf{QMSum} \\
\midrule
Documents & 200 & 185 & 106 & 90 \\
Words/doc & 290 & 459 & 703 & 10,837 \\
\midrule
Questions & 21 & 14 & 13 & 10 \\
Answered/doc & 84.1\% & 86.2\% & 96.5\% & 91.9\% \\
Words/answer & 30.9 & 53.0 & 70.3 & 161.5 \\
Claims/answer & 6.5 & 11.4 & 12.4 & 29.6 \\
Claims (total) & 23,124 & 25,353 & 16,430 & 24,459 \\
\bottomrule
\end{tabular}

\caption{Dataset overview. Number of words is calculated as whitespace-separated tokens.}
\label{tab:dataset-statistics}

\end{table}

\paragraph{Datasets.}
We analyze LLM salience across several technical and scientific domains using four datasets (\cref{tab:dataset-statistics}).
We designed slightly unconventional summarization tasks because of their limited ``oracle'' summaries in common LLM training datasets.
This allows us to analyze how LLMs handle texts without strong priors, and how salience judgments vary across genres and discourse types (structured technical writing, academic discourse, and dialogue).

\paragraph{(1) Randomized Controlled Trials (RCT).}
We draw a random sample of 200 abstracts of RCTs published Jan--Apr 2024 from PubMed.
These documents follow established conventions to describe the conduct and outcomes of clinical studies.
The task is to further summarize the abstracts.


\paragraph{(2) Computation and Language (CL).} The second task is to summarize the \emph{related work} sections of NLP/CL papers published on arXiv.
Although CL paper summarization is common, summarizing the related work section itself is not.
We convert raw LaTeX sources to Markdown and only consider documents up to 2,000 tokens to fit the context window of smaller models.
A random sample of 185 documents published in October 2024 is drawn.

\paragraph{(3) Astrophysics (Astro).} The third dataset contains \emph{discussion} sections of astrophysics papers published on arXiv.
These documents interpret key results of theoretical and empirical astrophysics research.
Similar to the CL portion, summarizing only the discussion sections is uncommon.
A random sample of 106 documents is drawn, with pre-processing analogous to CL.

\paragraph{(4) Meetings (QMSum).} Lastly, we consider meeting transcript summarization.
We randomly sample 90 documents balanced across three domains from QMSum~\cite{Zhong:2021:NAACL}: product design, research and political discussions.
We format transcripts as \texttt{\small [Speaker]: [Utterance]} turns, separated by newlines.
We only experiment with long-context models ($\ge 32\text{k}$ tokens) on this dataset.

\begin{figure*}[t]
\includegraphics[width=\textwidth]{figures/salience-pubmed-sample}
\caption{Corpus-level content salience map for \emph{RCT} summaries by four methods.}
\label{fig:salience-pubmed}
\end{figure*}

\paragraph{Summarization Models.}
We experiment with 13 LLMs of different scales:
\textbf{OLMo}~(7B; 02/24, 07/24; \citealp{Groeneveld:2024:ACL}), \textbf{Mistral}~(7B; v0.3; \citealp{Jiang:2023:arXiv}), \textbf{Mixtral}~(8x7B; v0.1, \citealp{Jiang:2024:arXiv}), \textbf{Llama 2}~(7B, 13B, 70B; \citealp{Touvron:2023:arXiv}), \textbf{Llama~3}~(8B, 70B), and \textbf{Llama~3.1}~(8B, 70B; \citealp{Grattafiori:2024:arXiv}). For API-based models, we use \textbf{\gptmini}~(07/24) and \textbf{\gpt}~(08/24; \citealp{OpenAI:2024:arXiv}).
We also include 3 baselines to contextualize results: \textbf{Lead-N}, \textbf{Random} and \textbf{TextRank}~\cite{Mihalcea:2004:EMNLP}, all adjusted to meet summary length budgets. To assess consistency across multiple rounds of decoding, we generate 5 summaries per document and target length with temperature $\tau = 0.3$.
We use a zero-shot summarization prompt (\cref{sec:appendix-prompts}).

Before analyzing salience in these models, we validate two key assumptions: \emph{(i)} generated summaries should approximately meet the target length, and \emph{(ii)} longer summaries should expand on shorter ones (``incremental consistency''). Additionally, we analyze how greater $\tau$ affect those criteria.
Our analysis confirms that models largely meet above criteria, with newer and bigger models showing better length control.
Higher $\tau$ results in stable \emph{average} summary length at the corpus level, but greater length variance at the document level (up to 10\% difference), along with a slight decline in incremental consistency (details in \cref{sec:appendix-length-analysis}).

\section{Observed Salience}
\subsection{RQ1: What notion of salience have LLMs learned in different domains?}
\label{sec:results-salience}
To understand how LLMs prioritize different information, we consider average question answerability as a proxy for salience. We show the results for the \emph{RCT} dataset as a representative example in \cref{fig:salience-pubmed}, and include other datasets in \cref{sec:appendix-salience}.

\textbf{Models prioritize information hierarchically.}
We observe a clear hierarchy in how information is prioritized across summary lengths.
For example, fundamental aspects such as the focus of a study (\emph{Q1}), and the condition being treated (\emph{Q3}) consistently achieve higher scores, even at 10-word summaries.
In contrast, more specific and technical information like the study design (\emph{Q10}) and the statistical significance of results (\emph{Q12}) are primarily discussed in longer summaries ($\ge 100$ words).

\textbf{Information frequency is not in itself predictive of salience.}
When we consider how frequently a question is answered by documents in the corpus (leftmost column of \cref{fig:salience-pubmed}), we find that even relatively rare questions such as biological markers and adverse effects (\emph{Q7}/\emph{11}, prevalence 40\%/26\%) maintain a consistent representation in summaries.
This suggests that LLMs do not simply prioritize information based on its frequency in a genre.

\textbf{Summaries progressively get more detailed, and information density differs across models.}
As expected, longer summaries consistently include more information as shown by the higher average answerability (bottom row in \cref{fig:salience-pubmed}).
However, the absolute scores differ across models.
GPT-4o has a notably higher answerability score than Llama 3.1, particularly at longer summaries (0.81 vs. 0.71 at the 200-word length).
Given that both models generate summaries of similar lengths (cf. \cref{fig:length-deviation}), this suggests that GPT-4o conveys information more efficiently.

\begin{figure*}[t]
\includegraphics[width=\textwidth]{figures/agg-pubmed-cl}
\caption{
    Do LLMs share a similar notion of salience?
    Heatmaps show agreement of content-selection at the atomic-claim level (Krippendorff's $\alpha$).
    Dashed bounding boxes indicate models of the same family.
    The diagonal shows self-agreement over multiple generations. Top row: \emph{RCT}, Bottom row: \emph{CL}.
}
\label{fig:agg-pubmed-cl}
\end{figure*}

\subsection{RQ2: Do LLMs of different families and sizes have a similar notion of salience?}
\label{sec:results-model-model-similarity}
We want to understand to what extent different models (e.g., families, scales) have a shared notion of information salience in a given domain.
We define a fine-grained similarity metric that compares models' content-selection decisions.
Intuitively, two models are more similar if their summaries include the same answer claims.
More formally, for each summary length $l$, we compile all atomic claims derived from question-answers along with their entailment labels (cf. \cref{sec:method-questions}). These form a binary vector $\mathbf{v}_{M,l}$ indicating which claims model $M$ includes in its summaries.
We then measure agreement between two models using Krippendorff's alpha: $\alpha(\mathbf{v}_{M_1,l}, \mathbf{v}_{M_2,l})$.
This claim-level agreement metric is stricter than comparing aggregate answerability scores, as it requires models to consistently include or exclude the same claims at each summary length.\footnote{In contrast, similar answerability scores can result from selecting a similar \emph{number} of claims.}
\cref{fig:agg-pubmed-cl} shows the model-model agreement for the \emph{RCT} and \emph{CL} datasets.

\textbf{High agreement across multiple runs suggests models apply salience notion consistently.}
The diagonal in~\cref{fig:agg-pubmed-cl} shows the average pairwise agreement across 5 model runs.
Overall, self-agreement is the highest for \emph{RCT} ($\approx .80$), while it is slightly lower for \emph{CL}, \emph{Astro} and \emph{QMSum} ($\approx .75$).
We observe a slight decline in self-agreement as the summary length increases.
We hypothesize that each document has a tail of medium- to low-salient topics which may or may not be included as the length budget gives more ``freedom'' to the models.

\textbf{Models of the same family or size do \emph{not consistently} have a higher agreement than any other model.}
We next inspect the off-diagonal agreements, comparing one model family with another model family.
Overall, we find that within-family agreement is not consistently higher than cross-family agreement.
While there are isolated cases with a higher within-family agreement (e.g., Llama 3.1 and GPT-4o on \emph{RCT}), this trend cannot be confirmed for all families and datasets.

\textbf{Agreement by summary length and with GPT-4o-mini.}
We observe that certain summary-lengths have higher agreement than others, though the peak is different for each dataset (e.g., agreement on \emph{RCT} is highest for 50 word summaries, whereas on \emph{CL} it peaks at 100 words).
There could be a ``natural'' summary length for each dataset where model more easily agree.
Lastly, we find that more recent and bigger models agree better with GPT-4o-mini which suggests a clear scaling effect and that open-weights models are getting closer in capabilities to large proprietary models (\cref{fig:agg-gpt4}).

\begin{figure}[t]
\includegraphics[width=\linewidth]{figures/agg-gpt4}
\caption{Agreement with GPT-4o-mini, averaged over all datasets and summary lengths.}
\label{fig:agg-gpt4}
\end{figure}

\section{Perceived Salience and Alignment}
In addition to the \emph{observational salience} analysis,
we elicit \emph{perceived salience} by having humans and models directly rate the salience of each question.
This study has two purposes: (1) to understand whether model behavior aligns with human expectations, and (2) to see if the summarization behavior of LLMs can be approximated by direct prompting.

\subsection{Setup}
\paragraph{Human salience annotation.}
We recruited 18 experts across the four domains through our network (3 for \emph{RCT}, and 5 each for \emph{Astro, CL, QMSum}).\footnote{
    Trained physicians (\emph{RCT}), graduate students/faculty (\emph{Astro}), and graduate students (\emph{CL, QMSum}) based in US/Europe.
}
Experts rated the relative salience of each question on a 5-point Likert scale (ranging from 1: least important, to 5: most important).
Annotators were asked to motivate their rating through a brief rationale to encourage thoughtful judgments and to allow post-hoc analysis of their decision-making process.
To establish a shared understanding between annotators of what content a question may elicit, each question is accompanied by an example answer from a randomly drawn document in the domain.
To ensure high annotation quality, we conducted two pilot rounds with four annotators to refine our annotation guidelines (see \cref{sec:appendix-annotation-guidelines}).

Importantly, the human annotations cannot be regarded as a gold standard for salience. The ratings represent how humans \emph{perceive} question salience, which may not be reflective of how humans actually write summaries.

\paragraph{Model-based salience ratings (LLM-perceived).}
We prompt LLMs to directly rate question salience.
The prompt includes the question list for a given domain and instructions that closely mirror the human annotation guidelines to allow for direct comparison (i.e., 5-point Likert scale and rationales).
Each model is prompted 5 times with a shuffled question list to mitigate position bias and to quantify consistency.
See \cref{sec:appendix-prompts} for the full prompt.

\paragraph{Analysis method.}
We use Spearman's rank correlation coefficient ($\rho$) to quantify alignment between three measures: human-perceived salience, LLM-perceived salience (both 5-point Likert scale) and LLM-observed salience (continuous $[0,1]$).\footnote{We take observed salience scores at the 200-words summary length which correlated on average most strongly with human salience. Other scores are explored in~\cref{sec:appendix-salience-score-ablation}.}
For groups with multiple ratings, we report averaged pairwise correlation.

\paragraph{Human correlation.}
We observe that inter-human correlation varies across domains, with meeting summarization (QMSum, $\rho = 0.60$) and RCT abstracts ($\rho = 0.46$) showing a moderate to strong correlation (\cref{tab:annotator-agreement}).
These domains presumably have established conventions about summary content.
In contrast, correlation is weak on summarization of related work sections (CL, $\rho = 0.26$) and discussion sections (Astro, $\rho = 0.16$).
Documents in these domains may vary significantly in the type of content they present (i.e., certain questions may be more relevant to theoretical vs. empirical papers).
While our annotation protocol aims to control for this aspect through the example answers by question, there remains annotator subjectivity related to their personal interests.

\begin{table}[t]
\small
\centering
\begin{tabular}{lrrrr}
\toprule
\bfseries Dataset & \bfseries Questions & \bfseries Raters & \bfseries $\rho$ & \bfseries Std. \\
\midrule
QMSum & 10 & 5 & 0.60 & 0.18 \\
RCT & 21 & 3 & 0.46 & 0.06 \\
CL & 14 & 5 & 0.26 & 0.29 \\
Astro & 13 & 5 & 0.16 & 0.44 \\
\bottomrule
\end{tabular}

\caption{Inter-annotator correlation (Spearman's $\rho$) for question salience rating.}
\label{tab:annotator-agreement}
\end{table}


\subsection{Results}
\label{sec:results-introspection}
To understand if LLMs can reliably rate question salience, we study three conditions.
First, as a reference point, we measure consistency of the observational and perceived salience measures estimated over 5 model runs (LLM-observed, LLM-perceived).
Second, we study the correlation of LLM-perceived and LLM-observed to measure if models' explicit ratings align with their summarization behavior (RQ3).
Third we correlate LLM-derived salience in human perceived salience (RQ4).
We report results for the three conditions in \cref{tab:results-rater-agreement} and provide qualitative examples in \cref{tab:results-examples}.

\begin{table*}[t]
\small
\centering
\setlength{\tabcolsep}{4pt}
\begin{tabular}{lrrrrrrrp{0.5cm}r}
\toprule
\textbf{Measure}  & \bfseries Random & \bfseries OLMo & \bfseries Mixtral & \bfseries Llama$^{3.1}_{8b}$ & \bfseries Llama$^{3.1}_{70b}$ & \bfseries 4o-mini & \bfseries 4o && \bfseries Average \\
\midrule
\multicolumn{10}{c}{\texttt{Consistency of Salience Estimates}}\\
\emph{LLM-perceived}  & -0.05 & 0.20 & 0.54 & 0.37 & 0.71 & 0.73 & \bfseries 0.76 &&  0.45 \\
\emph{LLM-observed} & 0.92 & \bfseries 0.99 & \bfseries 0.99 & 0.98 & \bfseries 0.99 & 0.98 & 0.98 && 0.98 \\\addlinespace
\multicolumn{10}{c}{\texttt{Correlation of Salience Estimates}}\\
\emph{LLM-perceived vs. -observed}  & 0.02 & 0.12 & 0.37 & 0.36 & 0.47 & \bfseries 0.56 & 0.50 && 0.33 \\\addlinespace
\multicolumn{10}{c}{\texttt{Correlation of Model and Human Salience}}\\
\emph{LLM-perceived vs. Human} & 0.06 & 0.16 & 0.41 & 0.31 & 0.47 & 0.51 & \bfseries 0.53 && 0.32 \\
\emph{LLM-observed vs. Human}  & 0.22 & 0.26 & 0.32 & 0.34 & \bfseries 0.35 & 0.34 & 0.25 && 0.24 \\
\bottomrule
\end{tabular}
\caption{Pearson rank correlation coefficient between different salience estimates averaged over all datasets. Individual values per dataset in \cref{tab:results-rater-agreement-full}.
Row-wise maximum in \textbf{bold}.}
\label{tab:results-rater-agreement}
\end{table*}

\begin{table*}[t]
    \fontsize{8.5pt}{8.5pt}\selectfont
    \begin{tabular}{@{}p{0.17\textwidth}rp{0.23\textwidth}p{0.45\textwidth}@{}}
    \toprule
        \textbf{Question} &
        $L_o$/$L_p$/$H$ &
        \faRobot\ \textbf{Rationale LLM-perceived} &
        \faUsers\ \textbf{Rationales Human}\\
    \midrule
    \textbf{(\#1, RCT \faUserMd)}
        What significant statistical results are reported? & 2.5/4.0/5.0 &
        Statistical results are essential for assessing the validity and reliability of the findings. &
        (P\textsubscript{\{1,2,3\}} = 5) Main outcome that physicians are looking for; very important to be included as measure of success; without it reporting is lacking \\

    \textbf{(\#2, Astro \faStar)}
        What broader context or field does the study contribute to or address?
        & 4.6/3.4/2.4
        & Understanding the broader context helps situate the study within the larger scientific discourse.
        & (P\textsubscript{\{2,5\}} = 1) not relevant, experts would fill context; (P\textsubscript{1} = 3) inferable from another question; (P\textsubscript{3} = 4) context is important although not always addressed; (P\textsubscript{4} = 3) Situating findings within literature, existing models or theories is important;\\

    \textbf{(\#3, RCT \faUserMd)}
        What was the main intervention used in the study? & 4.2/4.8/5.0 &
        The main intervention is central to understanding the study's focus and findings. &
        (P\textsubscript{\{1,2,3\}} = 5) determines whether study influences decision making; part of PICO therefore fundamental; essential for communicating purpose/focus of the trial \\
    \bottomrule
    \end{tabular}

    \caption{Example questions, salience scores by LLM-observed ($L_o$, rescaled to 1-5), LLM-perceived ($L_p$), humans ($H$) and summarized rationales. Additional examples in \cref{tab:results-examples-part2}.}
    \label{tab:results-examples}

\end{table*}

\paragraph{RQ3: When models introspect, does their perceived notion of salience align with their summarization behavior?}
LLMs have strong and consistent \emph{implicit} notions of salience, but
they are unreliable when explicating these preferences in rating tasks. We detail these observations below.

\textbf{Observational salience is highly stable.}
We find that observational question salience leads to highly stable scores for all models ($\rho \ge 0.96$).
This suggests that LLMs' underlying summarization process is highly deterministic despite the stochastic nature of language models.
Also, it suggests that our proposed approach is a reliable tool for analyzing model behavior.

\textbf{Models fail to have consistent perceived salience.}
We find that the consistency of direct salience ratings varies significantly for all models and datasets.
Generally, strong instruction-following models have more consistent perceived salience than weaker models (avg. $\rho$ ranges from 0.20 for OLMo to 0.76 for GPT-4o).
This finding mirrors recent results in the LLM-as-a-judge literature which demonstrated instability in ratings due to various factors including position bias~\cite{Wang:2024:ACL,Stureborg:2024:arXiv}.

\textbf{Perceived $\neq$ observed salience.}
Lastly, we find only a weak to moderate correlation between perceived and observed salience (highest: avg. $\rho = 0.56$ for GPT-4o-mini, lowest: $\rho = 0.12$ for OLMo).
Again, stronger instruction-following models show higher correlations, indicating a clear scaling effect.
This gap echoes broader findings where generative abilities may not reflect an underlying understanding in models~\cite{West:2024:ICLR}.

\paragraph{RQ4: To what extent does model salience align with human perceived salience?}
\label{sec:results-human-alignment}
We find that both LLM-salience estimates only show a weak to moderate correlation with human salience perception.
Direct rating for question salience correlates more than observed salience (highest LLM-perceived: avg. $\rho = 0.53$ for GPT-4o, highest LLM-observed: avg. $\rho = 0.35$ for Llama 3.1 70B).
Weak correlation between models and humans holds for all dataset, also those where humans agree more strongly among themselves (\cref{tab:results-rater-agreement-full}).

Users of LLMs should carefully consider if the models are appropriate for their summarization task, or provide explicit signals about content priority through prompts or during model training.

\section{Related Work}
\textbf{Evaluating and Interpreting Summarization.}
Recent work suggests that LLMs match or surpass human performance in news summarization~\cite{Zhang:2024:TACL}.
However, traditional evaluation protocols remain unreliable especially for LLM-generated summaries~\cite{Fabbri:2021:TACL,Goyal:2023:arXiv}.
This spurred interest in analyzing summarization model behavior.
Studies found biases towards content near the beginning/end of documents~\cite{Ravaut:2024:ACL,Laban:2024:EMNLP}.
Others analyze training dynamics of summarization models to identify when skills like content selection are learned~\cite{Goyal:2022:ACL}.
Extract-then-abstract pipelines~\cite{Gehrmann:2018:EMNLP,Li:2021:ws} aim for interpretable text summarization but this interpretability is limited to the document-level~\cite{Dhaini:2024:INLG}.
Our research complements prior work by providing a \emph{global interpretation} of what topics LLMs consider important through the lens of text summarization.

\textbf{Explainable Topic Modeling.}
Our analysis method draws inspiration from the interpretable topic modeling literature.
While classical topic models such as LDA~\cite{Blei:2003:JMLR} have long been used to explain latent themes in text corpora, they are often difficult to interpret~\cite{Chang:2009:NeurIPS}.
Recent work showed that LLMs can effectively be used to generate natural language descriptions of latent themes in text mining, clustering and concept induction workflows~\cite{Pham:2024:NAACL,Zhong:2024:NeurIPS,Wang:2023:EMNLP,Lam:2024:CHI}.
Our framework uses LLMs to describe salient summary content in form of information-seeking QUDs.
The use of QUDs as a representation of information units was shown successful in a wide range of tasks~\cite{Newman:2023:EMNLP,Laban:2022:Findings,Trienes:2024:ACL,Wu:2023b:EMNLP}.

\section{Conclusion}
\label{sec:conclusion}

We propose using surrogate modeling to evaluate the throughput of different infrastructure designs. 
%
We train three model architectures using simulator data to predict different job observables.
%
From our evaluation results, the architecture choice does not significantly influence the accuracy of the predictions at the current stage of development.
%
All three architectures decrease the execution times by orders of magnitude compared to DCSim.

At the current stage of the models inaccuracies are observed.
We suspect a lack of input information given to the models as the predominant source.
While the models are able to predict the compute times of our heterogeneous jobs scenario, where some implicit information about the infrastructure can be extracted by the model, they fail to predict the transfer time of input files due to not being aware of the data infrastructure setup that is more complex. 

Future work can build on these results by incorporating platform information into the training data to improve the predictions.
This will become essential to ensure that the model performs accurately on arbitrary workload mixes.
%
Moreover, training on real-world data instead of simulator data could enhance the capabilities and applicability of the models, also providing valuable data in regimes where the simulation, due to its scaling behavior, is not able to feasibly produce large amounts of training data.
\section{Limitations}
Our analysis of user and \abr{LLM} expertise misalignment and its downstream impacts is based on predicted expertise labels and predicted user satisfaction scores (\abr{SAT}). While we human validated the classification labels, there is still a possibility of some errors which could impact the results. 
We study conversations only in English, which may limit the generalizability our findings, given that a lot of conversations with Copilot take place in non-English languages. Our analysis is limited to English conversations to facilitate human-validation of predicted expertise labels. Future work may consider extending our findings in the multi-lingual domain.
% While our findings show the negative impact of user underestimation, it cannot be said that over-estimation does not have a negative impact on the user experience. We hope that our work motivates researchers to delve more into the problem of user and agent expertise misalignment.
We use the same prompts to predict user and agent expertise on all the topical domains of our conversations. While more personalized templates might be able to capture the notions of expertise more accurately for different domains, we use the same template throughout to be able to make a fair comparison across all our experimental settings. 
Additionally, we only use GPT-4 to predict the expertise labels. While it is possible that other \abr{LLMs} might be able to judge the expertise better, we restrict ourselves to the same model as the SAT rubric and the task complexity classifier to maintian model consistency. Future work may consider extending our expertise classifier to evaluate the alignment between humans and multiple different \abr{LLMs}.
Finally, all our results are correlational, but do indicate that a mismatch in expertise between the user and \abr{LLMs} could be one of the causes of user dissatisfaction. We hope our findings motivate future works to involve experiments where agent expertise can be manipulated to determine whether it has a causal impact on user satisfaction or not.

\section*{Acknowledgments}
We thank Hsin-Pei Chen, Khawla Elhadri, Arya Farahi, Juan P. Farias, Cheng Han Hsieh, Sebastian Joseph, Ramez Kouzy, Michael Muzinich, Van Bach Nguyen, Juan Diego Rodriguez, Paul Torrey, Manya Wadhwa, Barry Wei, and Paul Youssef for their participation in the salience annotation study.
We also thank Dennis Aumiller and Philippe Laban for early feedback on this research.
This work was partially supported by the US National Institutes of Health (NIH) grant 1R01LM014600-01,
and the US National Science Foundation grants IIS-2107524, IIS-2145479, and Cooperative Agreement 2421782 and Simons Foundation MPS-AI-00010515 (NSF-Simons AI Institute for Cosmic Origins\footnote{CosmicAI, \url{https://www.cosmicai.org/}}).

\bibliography{bibliography}

\appendix
\section{Length-instruction Following}
\label{sec:appendix-length-analysis}
We analyze to what extent length-controlled summarization is a consistent probe for content selection.
Ideally, we expect the following behavior of summarization models: (1) the generated summary length matches approximately the target length, and (2) as we increase the length budget, summaries should provide all content of the shorter version in addition to expanding on it.
We define two measures for these desiderata.

\paragraph{Target length ratio (TLR).}
We quantify the length deviation of a generated summary ($s_{l}$) from the target word count ($l$) as follows:
%
\begin{equation}
\text{\textbf{TLR}}(s_{l}) = \frac{|s_{d,l}|}{l}.
\end{equation}
%
Where $|\cdot|$ is the summary length (whitespace separated tokens). A value of 1 indicates perfect length match, while values greater or smaller than 1 indicate over- or under-generation, respectively.

\paragraph{Incremental consistency (IC).}
Longer summaries should contain a proper superset of claims found in the adjacent shorter version.
Formally, for each document $d$ and topic $t$ recall that we have a set of atomic claims $A_t$ (\cref{sec:method-questions}).
We first identify the set of claims that are entailed at least once across any summary length:
%
\begin{equation*}
    A_{\text{entailed}}(d,t) = \{a \in A_t \mid \exists l \in L, e(a,s_{d,l}) = 1\},
\end{equation*}
%
where $e$ is an NLI model indicating whether claim $a$ is entailed by summary $s_{d,l}$ of length $l$.
Next, we determine if a claim is included consistently across increasing summary lengths (monotonicity condition).
%
\begin{equation*}
    e(a,s_{d,l_1}) \leq e(a,s_{d,l_2}) \; \forall l_1 < l_2
\end{equation*}
%
We then define the set of consistent claims where this condition holds:
%
\begin{equation*}
\begin{aligned}
    A_{\text{consistent}}(d,t) =\,&\{a \in A_{\text{entailed}}(d,t) \\
    \quad& \mid \text{\small monotonicity holds} \, \forall l \in L \}.
\end{aligned}
\end{equation*}
%
Finally, the overall incremental consistency for summaries of a corpus $D$ is given as the fraction of consistent claims:
%
\begin{equation}
\text{\textbf{IC}}(D) = \frac{
        \sum_{d \in D} \sum_{t \in T} |A_{\text{consistent}}(d,t)|
    }{
        \sum_{d \in D} \sum_{t \in T} |A_{\text{entailed}}(d,t)|
    }.
\end{equation}
%
This metric ranges from 0 to 1, where 1 indicates perfect monotonicity (longer summaries always include all information found in shorter ones).

\begin{figure}[t]
\includegraphics[width=\linewidth]{figures/tlr}
\caption{Distribution of target length ratios over all generated summaries (aggregating lengths and datasets).}
\label{fig:length-deviation}
\end{figure}

\noindent\textbf{Do models meet the target length?}
We find that all models generally undershoot the length target (\cref{fig:length-deviation}).
However, more recent models match the target length more closely and consistently, showing a clear scaling effect.
The best performing models are Llama 3.1 and GPT-4o, while OLMo is unable to follow length-instructions, presumably because this was not part of the instruction tuning data.
Surprisingly, we do not find substantial differences across datasets.
This suggests that the ability of models to follow length-instructions is mostly invariant to the input document length, even if they are considerably long (e.g., meeting transcripts).
See \cref{fig:tlr-stratified} for an analysis of summary length stratified by dataset and target length.

\noindent\textbf{How incrementally consistent are summaries?}
We report the average incremental consistency by dataset and model in \cref{fig:incremental-consistency}.
We observe that all models are substantially more consistent than the random summarization baseline.
Furthermore, incremental consistency decreases with more difficult datasets, likely because there is more freedom on what content to include in a summary.
Similar to the ability of following length instructions, we observe a scaling effect where stronger models have a higher incremental consistency.

\begin{figure}[t]
\includegraphics[width=\linewidth]{figures/ic}
\caption{Incremental consistency by model and dataset.}
\label{fig:incremental-consistency}
\end{figure}

\noindent\textbf{Influence of temperature sampling.}
The main results in this paper are obtained with a temperature of $\tau = 0.3$.
To assess how temperature affects summary length and incremental consistency, we perform a temperature sweep on the RCT dataset for all open-weights models (20 settings in $[0,1]$).
Surprisingly, higher temperatures do not affect the \emph{average} summary length on a dataset-level, but lead to greater variance at the document level (up to 10\% length difference between generations, \cref{fig:temperature-length}).
Furthermore, higher temperatures lead to a slight decline in incremental consistency for all models that adequately follow length instructions (a drop of 1\% to 9\%, \cref{fig:temperature-ic}).

\noindent\textbf{Summary.}
Overall, we find that strong models are able to follow length-instructions and that they consistently expand the summary content with increasing length budgets.
As our salience analysis assumes this behavior of models, it may be less reliable for weaker models (OLMo, Mistral, Llama~2).

\begin{figure}[t]
\includegraphics[width=\linewidth]{figures/temperature-ic}
\caption{Influence of temperature on incremental consistency.}
\label{fig:temperature-ic}
\end{figure}

\section{Salience Score Ablation}
\label{sec:appendix-salience-score-ablation}
We analyze how different salience scores derived from the CSM correlate with human salience.
Recall that the $\text{CSM}(D)_{t,l}$ tracks the average answerability of question $t \in T$ at summary length $l \in L = \{10,20,50,100,200\}$.
We take raw salience scores at each summary length in addition to question-wise aggregations.
Intuitively, questions which are more answerable at shorter summaries score higher under the aggregated scheme.
Formally, we aggregate scores as follows:
%
\begin{equation*}
\text{CSM}_{\text{agg}}(D)_{t} = \frac{\sum_{l \in L} w_l \cdot \text{CSM}(D)_{t,l}}{\sum_{l \in L} w_l},
\end{equation*}
%
where $w_l$ is a weighting term. We experiment with three weighting functions: uniform ($w_l = 1$), reciprocal length ($w_l = 1/l$), and logarithmic decay ($w_l = 1/\log(1 + l)$).
\cref{fig:salience-score-correlation} shows the Spearman rank correlation coefficient ($\rho$) with human salience for each salience score.
Overall, on \emph{RCT} and \emph{Astro} we find that all salience scores correlate similarly with human salience ratings, while on \emph{CL} and \emph{QMSum} the 200 words salience score correlates most strongly.

\begin{figure}[t]
\includegraphics[width=\linewidth]{figures/salience-score-correlation}
\caption{Correlation of different salience scores with human salience. Here we aggregate over all LLMs which showed similar trends.}
\label{fig:salience-score-correlation}
\end{figure}

\section{Salience Analysis}
\label{sec:appendix-salience}
The corpus-level salience analysis for PubMed, Astro, CL, and QMSum is given in \cref{fig:salience-pubmed}, \cref{fig:salience-astro-ph}, \cref{fig:salience-cs-cl}, and \cref{fig:salience-qmsum}, respectively. We also provide a fully-worked example of the content salience analysis in \cref{fig:worked-example}.

\section{Responsible NLP Considerations}
\paragraph{Compute Requirements.} Experiments were conducted on NVIDIA A100 80GB GPUs, requiring approximately 20 GPU hours per dataset, and an additional 360 GPU hours for the temperature sweep on the RCT dataset, totaling 440 GPU hours.
We ran inference using \textsc{vllm}.\unfootnote{All URLs accessed 2025-02-15.}\footnote{\rurl{docs.vllm.ai}}
GPT-4o models were accessed through the OpenAI API with inference costs $\leq 100\$$.

\paragraph{Salience Annotation Study.} Participants joined on a volunteer basis, gave informed consent and agreed that their annotations will be shared in anonymized form in the paper repository. According to our institutional policies, this study did not require institutional review board (IRB) approval.

\paragraph{Data Licensing.}
We obtain RCT abstracts in accordance with fair use principles through the PubMed Entrez API.\footnote{\rurl{www.ncbi.nlm.nih.gov/home/develop/api/}}
Related work sections of CL and Astro papers were collected via the arXiv API.\footnote{\rurl{info.arxiv.org/help/api/index.html}} While the majority of papers on arXiv is published under the arXiv license\footnote{\rurl{arxiv.org/licenses/nonexclusive-distrib/1.0/license.html}} retaining \emph{copyright} with the original author(s), the \emph{use} of paper contents for research is explicitly granted and encouraged in the arXiv API terms \& conditions.\footnote{\rurl{info.arxiv.org/help/api/tou.html}}
We reused meeting transcripts from QMSum~\cite{Zhong:2021:NAACL}.\footnote{\rurl{github.com/Yale-LILY/QMSum}} All meeting transcripts are under an open use license, such as CC BY 4.0\footnote{\rurl{creativecommons.org/licenses/by/4.0/legalcode}} (academic meetings\footnote{\rurl{groups.inf.ed.ac.uk/ami/icsi/license.shtml}} and product meetings\footnote{\rurl{groups.inf.ed.ac.uk/ami/corpus/license.shtml}}) or Open Government Licence\footnote{\rurl{www.nationalarchives.gov.uk/doc/open-government-licence/version/3/}} (Parliament Commitee meetings).


\begin{figure*}[t]
\includegraphics[width=\textwidth]{figures/temperature-length}
\caption{Influence of temperature on generated summary length. \textbf{Left:} target length-ratio. \textbf{Center:} ``within-document length variance'' calculated as the mean deviation from the average summary length of 5 summaries for the same document (MAD). MAD is normalized to be comparable across length targets. \textbf{Right:} zoomed version.}
\label{fig:temperature-length}
\end{figure*}

\begin{table*}[t]
\small
\centering
\setlength{\tabcolsep}{4pt}
\begin{tabular}{llrrrrrrrr}
\toprule
\textbf{Measure} & \textbf{Dataset} & \bfseries Random & \bfseries OLMo & \bfseries Mixtral & \bfseries Llama$^{3.1}_{8b}$ & \bfseries Llama$^{3.1}_{70b}$ & \bfseries 4o-mini & \bfseries 4o & \bfseries Average \\
\midrule
\multicolumn{10}{c}{\texttt{Consistency of Salience Estimates}}\\\midrule
\multirow[c]{5}{0.14\textwidth}{\emph{LLM-perceived}} & RCT & -0.06 & 0.34 & 0.48 & 0.29 & 0.61 & 0.75 & \bfseries 0.80 & 0.43 \\
\bfseries  & Astro & 0.02 & 0.07 & 0.41 & 0.56 & 0.64 & 0.76 & \bfseries 0.86 & 0.46 \\
\bfseries  & CL & -0.10 & -0.05 & 0.65 & 0.29 & \bfseries 0.73 & 0.70 & 0.57 & 0.35 \\
\bfseries  & QMSum & -0.07 & 0.42 & 0.61 & 0.36 & \bfseries 0.87 & 0.72 & 0.80 & 0.55 \\\cmidrule{2-10}
\bfseries  & Avg. & -0.05 & 0.20 & 0.54 & 0.37 & 0.71 & 0.73 & \bfseries 0.76 & 0.45 \\\midrule
\multirow[c]{5}{0.14\textwidth}{\emph{LLM-observed}} & RCT & 0.94 & 0.98 & \bfseries 0.99 & \bfseries 0.99 & \bfseries 0.99 & \bfseries 0.99 & \bfseries 0.99 & 0.98 \\
\bfseries  & Astro & 0.91 & 0.99 & \bfseries 1.00 & 0.97 & 0.99 & 0.97 & 0.97 & 0.97 \\
\bfseries  & CL & 0.96 & \bfseries 0.99 & \bfseries 0.99 & 0.96 & \bfseries 0.99 & \bfseries 0.99 & 0.97 & 0.98 \\
\bfseries  & QMSum$^\dagger$ & 0.87 & --- & 0.99 & 0.99 & \bfseries 1.00 & 0.99 & --- & 0.97 \\\cmidrule{2-10}
\bfseries  & Avg. & 0.92 & \bfseries 0.99 & \bfseries 0.99 & 0.98 & \bfseries 0.99 & 0.98 & 0.98 & 0.98 \\\midrule
\multicolumn{10}{c}{\texttt{Correlation of Salience Estimates}} \\\midrule
\multirow[c]{5}{0.14\textwidth}{\emph{LLM-perceived vs. LLM-observed}} & RCT & -0.06 & 0.10 & 0.25 & 0.25 & 0.37 & 0.41 & \bfseries 0.51 & 0.30 \\
\bfseries  & Astro & 0.11 & 0.09 & 0.31 & 0.56 & 0.50 & \bfseries 0.65 & 0.58 & 0.36 \\
\bfseries  & CL & -0.08 & 0.16 & 0.44 & 0.47 & 0.38 & \bfseries 0.58 & 0.41 & 0.29 \\
\bfseries  & QMSum$^\dagger$ & 0.11 & --- & 0.46 & 0.16 & \bfseries 0.63 & 0.60 & --- & 0.42 \\\cmidrule{2-10}
\bfseries  & Avg. & 0.02 & 0.12 & 0.37 & 0.36 & 0.47 & \bfseries 0.56 & 0.50 & 0.33 \\\midrule
\multicolumn{10}{c}{\texttt{Correlation of Model and Human Salience}} \\\midrule
\multirow[c]{5}{0.14\textwidth}{\emph{LLM-perceived vs. Human}} & RCT & -0.03 & 0.22 & 0.38 & 0.34 & 0.49 & 0.48 & \bfseries 0.56 & 0.33 \\
\bfseries  & Astro & 0.07 & 0.12 & 0.30 & 0.31 & 0.27 & \bfseries 0.45 & 0.44 & 0.27 \\
\bfseries  & CL & 0.06 & -0.03 & 0.41 & 0.22 & \bfseries 0.48 & 0.44 & 0.46 & 0.24 \\
\bfseries  & QMSum & 0.14 & 0.34 & 0.54 & 0.36 & 0.62 & \bfseries 0.67 & \bfseries 0.67 & 0.45 \\\cmidrule{2-10}
\bfseries  & Avg. & 0.06 & 0.16 & 0.41 & 0.31 & 0.47 & 0.51 & \bfseries 0.53 & 0.32 \\\midrule
\multirow[c]{5}{0.14\textwidth}{\emph{LLM-observed vs. Human}} & RCT & 0.31 & 0.28 & 0.27 & 0.25 & 0.25 & \bfseries 0.34 & 0.24 & 0.25 \\
\bfseries  & Astro & 0.11 & 0.25 & 0.27 & 0.29 & \bfseries 0.31 & 0.26 & 0.25 & 0.22 \\
\bfseries  & CL & \bfseries 0.30 & 0.23 & 0.23 & 0.24 & 0.26 & 0.25 & 0.24 & 0.21 \\
\bfseries  & QMSum$^\dagger$ & 0.16 & --- & 0.53 & 0.58 & \bfseries 0.59 & 0.51 & --- & 0.31 \\\cmidrule{2-10}
\bfseries  & Avg. & 0.22 & 0.26 & 0.32 & 0.34 & \bfseries 0.35 & 0.34 & 0.25 & 0.24 \\
\bottomrule
\end{tabular}
\caption{Pearson rank correlation coefficient between different salience estimates split by dataset.
The row-wise maximum is \textbf{bolded}. $^\dagger$Results for QMSum not available due to limited context window (OLMo) and budget constraints (GPT-4o).}
\label{tab:results-rater-agreement-full}
\end{table*}


\begin{figure*}[t]
\centering
\begin{subfigure}[b]{\textwidth}
\includegraphics[width=\textwidth]{figures/tlr-by-dataset}
\caption{Distribution of target length ratios over all generated summaries stratified by dataset.}
\label{fig:tlr-dataset}
\end{subfigure}
\vspace{0.5cm}
\begin{subfigure}[b]{\textwidth}
\includegraphics[width=\textwidth]{figures/tlr-by-length}
\caption{Distribution of target length ratios over all generated summaries stratified by target summary length.}
\label{fig:tlr-length}
\end{subfigure}

\caption{Analysis of length-instruction following. The target length ration (TLR) indicates to what extent models match the provided length. A value of 1 indicates perfect length match, while values greater or smaller than 1 indicate over- or under-generation, respectively.}
\label{fig:tlr-stratified}
\end{figure*}


\begin{figure*}[t]
\includegraphics[width=\textwidth]{figures/salience-astro-ph}
\caption{Corpus-level content salience map for \emph{Astro} summaries by four methods.}
\label{fig:salience-astro-ph}
\end{figure*}

\begin{figure*}[t]
\includegraphics[width=\textwidth]{figures/salience-cs-cl}
\caption{Corpus-level content salience map for \emph{CL} summaries by four methods.}
\label{fig:salience-cs-cl}
\end{figure*}

\begin{figure*}[t]
\includegraphics[width=\textwidth]{figures/salience-qmsum}
\caption{Corpus-level content salience map for \emph{QMSum} summaries by four methods.}
\label{fig:salience-qmsum}
\end{figure*}

\begin{figure*}[t]
\centering

    \begin{subfigure}[b]{\textwidth}
    \includegraphics[width=\textwidth]{figures/agg-astro}
    \caption{Model similarity for \emph{Astro}.}
    \label{fig:agg-astro}
    \end{subfigure}

    \vspace{0.5cm}

    \begin{subfigure}[b]{\textwidth}
    \includegraphics[width=\textwidth]{figures/agg-qmsum.pdf}
    \caption{Model similarity for \emph{QMSum}.}
    \label{fig:agg-qmsum}
    \end{subfigure}

\caption{Do models share a similar notion of salience? Heatmaps show agreement on topic inclusion by summary length (Krippendorff's alpha calculated column-wise for the CSM pairs).}
\label{fig:agg-astro-qmsum}
\end{figure*}

\begin{table*}[t]
    \fontsize{8.5pt}{8.5pt}\selectfont
    \begin{tabular}{@{}p{0.17\textwidth}rp{0.23\textwidth}p{0.45\textwidth}@{}}
    \toprule
        \textbf{Question} &
        $L_o$/$L_p$/$H$ &
        \faRobot\ \textbf{Rationale LLM-perceived} &
        \faUsers\ \textbf{Rationales Human}\\
    \midrule
    \multicolumn{4}{c}{\emph{continued from \cref{tab:results-examples}}} \\
    \textbf{(\#4, RCT \faUserMd)}
        What were the comparative results between intervention and control groups? & 3.2/4.6/5.0 &
        Comparative results are key to understanding the effectiveness of the intervention. &
        (P\textsubscript{\{1,2,3\}} = 5) Brief comparison between groups allows for readers to quickly assess statistical significance; Comparative results are one of the primary outputs reflecting effectiveness; essential for understanding the findings. \\

    \textbf{(\#5, RCT \faUserMd)}
        What implications or future recommendations did the study suggest based on its findings? & 4.6/3.6/2.0 &
        Implications and recommendations are important for understanding the study's impact and future research directions. &
        (P\textsubscript{1} = 1) This is a job for the reader to decide for himself or herself; (P\textsubscript{2} = 3) helps the reader understand the paper. Useful for at the end of a summary. (P\textsubscript{3} = 2) While helpful, they are not necessarily always based on the opinion of the authors; would have the reader get to that conclusion themselves.\\

    \textbf{(\#6, RCT \faUserMd)}
        What is the main outcome or effect observed? & 4.6/5.0/5.0 &
        The main outcome is critical to understanding the study's primary conclusion. &
        (P\textsubscript{\{1,2,3\}} = 5) determines whether a paper can be used to influence decision making; primary information when reading summary \\

    \textbf{(\#7, RCT \faUserMd)}
        How were the participants or subjects of the study selected and divided? & 3.7/3.2/2.3 &
        Participant selection and division are important for assessing the study's validity. &
        (P\textsubscript{1} = 2) This helps a reader determine external validity of a study;
        (P\textsubscript{2} = 1) Not necessary as fairly standardized;
        (P\textsubscript{3} = 4) Crucial to evaluate the study's validity and relevance \\

    \textbf{(\#8, Astro \faStar)}
        What limitations or considerations are noted by the study? & 1.1/3.6/2.3 &
        Limitations are crucial for interpreting the study's reliability and potential biases. &
        (P\textsubscript{1} = 3) important but best reserved for readers who choose to critically examine the study in its totality; (P\textsubscript{2} = 1) Typically addressed in the discussion section of a paper, but is not necessary for a summary; (P\textsubscript{3} = 3) would note if I had the space, however, I would also implore the readers to think about limitations themselves. \\

    \textbf{(\#9, Astro \faStar)}
        How do the findings relate to existing models or theories? & 1.6/4.0/3.8 &
        This is important for understanding the study's impact on current scientific understanding. &
        (P\textsubscript{1} = 4) I think this is important to describe the finding in context of the research question, initial hypothesis, or a theory.  (P\textsubscript{2} = 3) This is important for the broader context of the study, however it is more important to present the study results in a clear way. Only after that is clear I'd compare to other studies.(P\textsubscript{3} = 5) This forms the main body of a discussion section. (P\textsubscript{4} = 3) Situating the findings within the literature by relating to existing models or theories is important.(P\textsubscript{3} = 4) Comparing with existing literature is very important for the discussion section.\\

    \textbf{(\#10, Astro \faStar)}
        What are the main findings of the study? & 3.5/5.0/4.8 &
        The main findings are the core of the study and must be included in any summary. &
        (P\textsubscript{1} = 5) key takeaway from the paper and should be included regardless of what the paper is about (P\textsubscript{2} = 5) most important information of the summary (P\textsubscript{3} = 4) The main findings should be briefly addressed in the summary of a discussion for the reader's quick follow-up (P\textsubscript{4} = 5) The main findings of the study, along with the main focus, form the two most important elements of an article summary. (P\textsubscript{5} = 5) Important to state the main findings and then discuss them in details. \\

    \textbf{(\#11, Astro \faStar)}
        What specific challenges or limitations does the study address or identify? & 1.6/3.2/2.6 &
        Understanding the challenges or limitations provides context for the study's reliability and areas for improvement. &
        (P\textsubscript{1} = 1) I most likely do not include challenges and limitations. These examples focused on the future needs not an existing open question. The focus will be on the findings in the context of a hypothesis, conjecture, or a theory. (P\textsubscript{2} = 1) Level of detail that a reader would need only if interested in full paper. Some challenges can be identified if the methods and scope of the paper are summarized clearly. (P\textsubscript{3} = 5) This forms the main body of a discussion section. (P\textsubscript{4} = 2) depends upon the significance of those challenges or limitations (P\textsubscript{5} = 4) Identify the limitations and challenges of the study is very important \\
    \bottomrule
    \end{tabular}
    \caption{Example questions, salience scores by LLM-observed ($L_o$, rescaled to 1-5), LLM-perceived ($L_p$), humans ($H$) and summarized rationales.}
    \label{tab:results-examples-part2}
\end{table*}

\clearpage
\onecolumn
\section{LLM Prompts}
\label{sec:appendix-prompts}
This section provides all prompts used throughout the experiments. Summarization (\cref{lst:summarization,lst:summarization-meetings}), question generation (\cref{lst:qg}), question answering (\cref{lst:qa}), answer claim  splitting (\cref{lst:claim-split}), and introspection (\cref{lst:introspection}).

\codeboxinput[label=lst:summarization]{Summarization prompt}{prompts/summarization.txt}
\codeboxinput[label=lst:summarization-meetings]{Summarization prompt for meeting transcripts}{prompts/summarization-meetings.txt}
\codeboxinput[label=lst:qg]{Question generation prompt}{prompts/qg.txt}
\codeboxinput[label=lst:qa]{Question answering prompt}{prompts/qa.txt}
\codeboxinput[label=lst:claim-split]{Claim splitting prompt}{prompts/claim-splitting.txt}
\codeboxinput[label=lst:introspection]{Introspection prompt}{prompts/introspection.txt}

\twocolumn
\section{Question Salience Annotation Guidelines}
\label{sec:appendix-annotation-guidelines}

\paragraph{Motivation.}
When summarizing long texts, we must consciously decide what information to include or exclude from a summary. These decisions are grounded in a notion of information salience, or how important we consider the information for our intended audience. We study this phenomenon in the context of automatic text summarization systems. Specifically, we aim to understand how well these systems replicate the judgments of domain experts regarding what information is most relevant.\looseness=-1

\paragraph{Task.}
Imagine you are asked to \textbf{summarize a paper describing the results of a randomized controlled trial (RCT)} for a typical reader in this field. The summary should provide enough context to stand alone, since the reader will only see your summary and no other parts of the paper. Furthermore, the summary length is constrained, requiring you to think about what content to prioritize. In this study, we frame content as questions that a summary could answer.

Ask yourself: \textbf{What are some key questions you want the summary to answer?} Your task is to rate the relative importance of a list of questions on the following scale.
%
\begin{todolist}[noitemsep]
\item (1) Least important; I would exclude this information from a summary.
\item (2) Low importance; I would include this information if there is room.
\item (3) Medium importance; I would probably include this information.
\item (4) High importance; I would definitely include this information.
\item (5) Most important; One of the first questions to be answered in the summary.
\end{todolist}
%
\paragraph{Rationale.}
For each rating, please provide a brief (1-sentence) rationale explaining your decision or highlighting any considerations or uncertainties.

\paragraph{Example answers.}
To give you a feeling for the kind of content a question might elicit, all questions have an illustrative answer sourced from a randomly chosen document (= RCT paper). Please keep the following in mind:
%
\begin{itemize}[noitemsep]
    \item \emph{Answer length} does not determine the question's importance.
    \item \emph{Phrasing and selection.} The precise answer phrasing can be different in the summary, and not all answer content must appear in the summary.
    \item \emph{Overlap.} Some questions may elicit overlapping answers. Therefore, focus on the essence of each question. Remember that in an actual summary, overlapping answer information would only be stated once, so don't worry about it (see below).
    \item \emph{Relevance.} The questions are answerable with most documents in this genre. Do your rating on the assumption that the document talks about this information.
\end{itemize}
%
\paragraph{Suggested process.}
%
\begin{enumerate}[noitemsep]
    \item Read all questions first.
    \item Identify questions that seem most/least important, and rate these as “anchor points.”
    \item Then, rate the remaining questions.
\end{enumerate}
%
Finally, there are no right or wrong ratings. Use your best judgment and intuition. Thank you for participating!

\subsection*{Appendix: Example of overlapping answers.}
Consider questions \texttt{Q1-Q3} below. Each question asks for a distinct unit of information, but the answer of \texttt{Q3} overlaps with the answer of \texttt{Q1} and \texttt{Q2}. The overlapping information is highlighted in \hlorange{orange} while the \emph{essence of the question} is highlighted in \hlgreen{green}. Base your rating on the essence of the question.

\begin{subbox}[width=1\linewidth, center]{Example of overlapping answers}
\small
\textbf{Q1. What was the study design or setting of the trial?}
This trial is a multicentre, randomized, double-blind, phase 3 study.\\

\textbf{Q2. What specific treatments were compared in the study?}
DBPR108 100 mg, sitagliptin 100 mg, and placebo.\\

\textbf{Q3. How were the participants or subjects of the study selected and divided?}
In this \hlorange{multicentre, randomized, double-blind, phase 3 study}, adult patients with type 2 diabetes were \hlgreen{randomly assigned} to \hlorange{receive either DBPR108 100mg, sitagliptin 100mg, or placebo} once daily. \hlgreen{A total of 766 patients were enrolled and divided into three groups: DBPR108 100mg (n=462), sitagliptin 100mg (n=152), or placebo (n=152).}
\end{subbox}

\begin{figure*}
\includegraphics[width=\textwidth]{figures/annotation-interface}
\caption{Interface for question salience annotation. Each question can be expanded to show an illustrative answer sourced from a randomly chosen document. The questions shown here are for the \emph{Astro} dataset.}
\end{figure*}


\begin{figure*}[t]
\includegraphics[width=\textwidth]{figures/worked-example}
\caption{Fully worked example of the question-based content analysis. Two documents in a fictional domain are each summarized at three lengths. Afterwards Steps 1 -- 4 are analogous to \cref{sec:method-questions}. Summary claims are colorcoded.}
\label{fig:worked-example}
\end{figure*}


\end{document}
