\begin{table*}[t]
    \fontsize{8.5pt}{8.5pt}\selectfont
    \begin{tabular}{@{}p{0.17\textwidth}rp{0.23\textwidth}p{0.45\textwidth}@{}}
    \toprule
        \textbf{Question} &
        $L_o$/$L_p$/$H$ &
        \faRobot\ \textbf{Rationale LLM-perceived} &
        \faUsers\ \textbf{Rationales Human}\\
    \midrule
    \textbf{(\#1, RCT \faUserMd)}
        What significant statistical results are reported? & 2.5/4.0/5.0 &
        Statistical results are essential for assessing the validity and reliability of the findings. &
        (P\textsubscript{\{1,2,3\}} = 5) Main outcome that physicians are looking for; very important to be included as measure of success; without it reporting is lacking \\

    \textbf{(\#2, Astro \faStar)}
        What broader context or field does the study contribute to or address?
        & 4.6/3.4/2.4
        & Understanding the broader context helps situate the study within the larger scientific discourse.
        & (P\textsubscript{\{2,5\}} = 1) not relevant, experts would fill context; (P\textsubscript{1} = 3) inferable from another question; (P\textsubscript{3} = 4) context is important although not always addressed; (P\textsubscript{4} = 3) Situating findings within literature, existing models or theories is important;\\

    \textbf{(\#3, RCT \faUserMd)}
        What was the main intervention used in the study? & 4.2/4.8/5.0 &
        The main intervention is central to understanding the study's focus and findings. &
        (P\textsubscript{\{1,2,3\}} = 5) determines whether study influences decision making; part of PICO therefore fundamental; essential for communicating purpose/focus of the trial \\
    \bottomrule
    \end{tabular}

    \caption{Example questions, salience scores by LLM-observed ($L_o$, rescaled to 1-5), LLM-perceived ($L_p$), humans ($H$) and summarized rationales. Additional examples in \cref{tab:results-examples-part2}.}
    \label{tab:results-examples}

\end{table*}
