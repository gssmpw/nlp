\twocolumn
\section{Question Salience Annotation Guidelines}
\label{sec:appendix-annotation-guidelines}

\paragraph{Motivation.}
When summarizing long texts, we must consciously decide what information to include or exclude from a summary. These decisions are grounded in a notion of information salience, or how important we consider the information for our intended audience. We study this phenomenon in the context of automatic text summarization systems. Specifically, we aim to understand how well these systems replicate the judgments of domain experts regarding what information is most relevant.\looseness=-1

\paragraph{Task.}
Imagine you are asked to \textbf{summarize a paper describing the results of a randomized controlled trial (RCT)} for a typical reader in this field. The summary should provide enough context to stand alone, since the reader will only see your summary and no other parts of the paper. Furthermore, the summary length is constrained, requiring you to think about what content to prioritize. In this study, we frame content as questions that a summary could answer.

Ask yourself: \textbf{What are some key questions you want the summary to answer?} Your task is to rate the relative importance of a list of questions on the following scale.
%
\begin{todolist}[noitemsep]
\item (1) Least important; I would exclude this information from a summary.
\item (2) Low importance; I would include this information if there is room.
\item (3) Medium importance; I would probably include this information.
\item (4) High importance; I would definitely include this information.
\item (5) Most important; One of the first questions to be answered in the summary.
\end{todolist}
%
\paragraph{Rationale.}
For each rating, please provide a brief (1-sentence) rationale explaining your decision or highlighting any considerations or uncertainties.

\paragraph{Example answers.}
To give you a feeling for the kind of content a question might elicit, all questions have an illustrative answer sourced from a randomly chosen document (= RCT paper). Please keep the following in mind:
%
\begin{itemize}[noitemsep]
    \item \emph{Answer length} does not determine the question's importance.
    \item \emph{Phrasing and selection.} The precise answer phrasing can be different in the summary, and not all answer content must appear in the summary.
    \item \emph{Overlap.} Some questions may elicit overlapping answers. Therefore, focus on the essence of each question. Remember that in an actual summary, overlapping answer information would only be stated once, so don't worry about it (see below).
    \item \emph{Relevance.} The questions are answerable with most documents in this genre. Do your rating on the assumption that the document talks about this information.
\end{itemize}
%
\paragraph{Suggested process.}
%
\begin{enumerate}[noitemsep]
    \item Read all questions first.
    \item Identify questions that seem most/least important, and rate these as “anchor points.”
    \item Then, rate the remaining questions.
\end{enumerate}
%
Finally, there are no right or wrong ratings. Use your best judgment and intuition. Thank you for participating!

\subsection*{Appendix: Example of overlapping answers.}
Consider questions \texttt{Q1-Q3} below. Each question asks for a distinct unit of information, but the answer of \texttt{Q3} overlaps with the answer of \texttt{Q1} and \texttt{Q2}. The overlapping information is highlighted in \hlorange{orange} while the \emph{essence of the question} is highlighted in \hlgreen{green}. Base your rating on the essence of the question.

\begin{subbox}[width=1\linewidth, center]{Example of overlapping answers}
\small
\textbf{Q1. What was the study design or setting of the trial?}
This trial is a multicentre, randomized, double-blind, phase 3 study.\\

\textbf{Q2. What specific treatments were compared in the study?}
DBPR108 100 mg, sitagliptin 100 mg, and placebo.\\

\textbf{Q3. How were the participants or subjects of the study selected and divided?}
In this \hlorange{multicentre, randomized, double-blind, phase 3 study}, adult patients with type 2 diabetes were \hlgreen{randomly assigned} to \hlorange{receive either DBPR108 100mg, sitagliptin 100mg, or placebo} once daily. \hlgreen{A total of 766 patients were enrolled and divided into three groups: DBPR108 100mg (n=462), sitagliptin 100mg (n=152), or placebo (n=152).}
\end{subbox}

\begin{figure*}
\includegraphics[width=\textwidth]{figures/annotation-interface}
\caption{Interface for question salience annotation. Each question can be expanded to show an illustrative answer sourced from a randomly chosen document. The questions shown here are for the \emph{Astro} dataset.}
\end{figure*}
