\section{Introduction}\label{sec:intro}


The Generative Pre-trained Transformer (GPT)~\citep{floridi2020gpt,yenduri2023generative} has achieved significant success in applications such as ChatGPT~\citep{ouyang2022training,wu2023brief} for chatbox, Copilot~\citep{barke2023grounded} for code generation and VideoGPT~\citep{yan2021videogpt} for video creation. However, GPT operates unidirectionally and lacks controllability~\citep{ethayarajh2019contextual}, which means it still faces the challenge in handling the real-world scenarios that require both controllable and bidirectional generation capabilities, as dictated by the current design of GPT. 


On the other hand, drug discovery~\citep{berdigaliyev2020overview} has become increasingly important since the advent of COVID-19~\citep{muratov2021critical}. The search for more effective drugs is becoming more urgent but remains underexplored. De Novo Drug Discovery incurs billions of dollars in costs and still confronts a high failure rate in its early stages~\citep{tong2021generative}. Drug Improvement addresses the limitations of De Novo drug discovery by building upon existing FDA-approved drugs. 
The DrugImprover framework~\citep{liu2023drugimprover} leads the way in tackling drug optimization by using Tanimoto similarity~\citep{landrum2013rdkit} to maintain beneficial properties while targeting multiple objectives with its novel Advantage-alignment Policy Optimization (APO) algorithm. It also provides a specialized dataset for optimizing drugs against COVID and cancer proteins.
REINVENT 4~\citep{he2021molecular, he2022transformer, loeffler2024reinvent}, in further, utilize the advanced generative capabilities of Transformers and large language models (LLMs) to address the drug optimization problem. 


Although REINVENT 4 has further improved performance, demonstrated promising outcomes, and achieved state-of-the-art performance, its effectiveness is still limited due to the black-box nature of the LLMs' generation process. In this process, LLMs optimize for either maximum likelihood or specific objectives set by users during the decoding phase. More specifically, it cannot specify the generation specific locations or the length of tokens to be generated.
Such limitations are further intensified in drug optimization with REINVENT 4:
(1) If an important substructure exists within the original molecule, REINVENT 4 might miss it and fail to preserve it in the generated ones, even though retaining that substructure could be beneficial.
(2) Real-world examples of drug improvement, such as the addition of an NH2 group to original drugs, demonstrate significant enhancements in effectiveness and reduced side effects. For instance, Ampicillin's modifications over Penicillin, illustrated in Figure \ref{fig:penicillin_example}, show increased activity range, stomach acid resistance, and better absorption while retaining the essential beta-lactam ring crucial for antibiotic activity. Ampicillin's main distinction from Penicillin is its side chain, altered to penetrate gram-negative bacteria's outer membranes more effectively.
(3) If the fragment has associated side effects, drugs derived from it might inherit these issues, contradicting the goal of optimization.
(4) Based on the given drug, we need to decide to what extent (e.g., how many atoms to add) to preserve or alter the original structure.
\vspace{-0.5cm}
\begin{figure}[ht]
    \centering
    \begin{subfigure}{0.45\textwidth}
    \centering
    \includegraphics[
    height=1.5cm, 
    clip={0,0,0,0}]{example_viz/penicillin.pdf}
    \caption{Penicillin}\label{fig:results:vertebral}
\end{subfigure}
\begin{subfigure}{0.45\textwidth}
    \centering
    \includegraphics[
    height=1.5cm, 
    clip={0,0,0,0}]{example_viz/Ampicillin.pdf}
    \caption{Ampicillin}\label{fig:results:hiv}
\end{subfigure}
    \caption{
    Penicillin in drug optimization. With adding a simple functional group NH2 (in \textcolor{myred}{red}), Ampicillin has resolved the rash side effect bring about by Penicillin.
    }
    \label{fig:penicillin_example}
    \vspace{-0.1cm}
\end{figure}
\vspace{-0.5cm}

To address the challenges of the GPT model and drug optimization, this work introduces \algname, a Bidirectional Causally Masked Seq2seq GPT tailored for controllable generation in drug optimization. Drawing inspiration from biological processes like growth and evolution, the training of this model leverages the SMILES representation \citep{weininger1988smiles} to enable the expansion, contraction, and mutation of molecule sequences at any point while maintaining essential structures.

This model, trained with a novel causally masked seq2seq objective, 
merges causal, masked, and seq2seq language modeling, enabling comprehensive generative modeling with bidirectional context, facilitating precise modifications without disrupting the molecule’s overall structure.
The model offers controllable capabilities including:
1) Generating new molecules while preserving original beneficial functional groups or scaffolds.
2) Removing atoms or groups causing side effects.
3) Precisely adding new atoms at specified scales and positions, with size hints that guide the generation process.
4) Controlling mutations, allowing expansion or contraction based on sub-sequences to achieve specific molecular configurations.


In summary, our contributions are:

$\bullet$ We propose a Causally Masked Seq2seq (CMS) objective that merges causal, masked, and seq2seq language modeling. CMS enables precise management of specific locations and ranges within a sequence, facilitating expansion, reduction, or mutation over chosen or random lengths, all while preserving the integrity of any specified positions or subsequences.


$\bullet$  We developed \algname completely from the ground up, training it on the CMS objective. This process included designing the training corpus, introducing a novel pre-training strategy, and devising a unique generation process.

$\bullet$  Through extensive experiments and ablation studies on real-world viral and cancer-related benchmarks, we demonstrate the effectiveness and controllability of \algname in outperforming the competing baselines in improving upon existing molecules and drugs across targeted objectives, resulting in superior drug candidates.


