\section{Related Work}
In this section, we succinctly review existing studies for signed graph neural networks and signed graph clustering.

\textbf{Signed Graph Neural Networks (SGNNs)}
, which maps nodes within a signed graph to a low-dimensional latent space, has increasingly facilitated a variety of signed graph analytical tasks, including node classification____, signed link prediction____, node ranking____, and signed clustering____. 
 % This theory, rooted in psychology, reasons that \textit{"the enemy of my enemy is my friend, the friend of my friend is my friend, the friend of my enemy is my enemy"}, reflecting the intricate relationships within signed networks. signed clustering
Most works of signed graph center around integrating \textit{Social Balance Theory} to signed convolutions into Graph Neural Networks (GNNs). As the pioneering work, SGCN____ adapts unsigned GNNs for signed graphs by aggregating and propagating neighbor information with Balance Theory. 
% Thereafter, other work has integrated additional social-psychological theories. 
____ appends the status theory, which is applicable to directed signed networks, interpreting positive or negative signs as indicators of relative status between nodes.
%applies Balance Theory and the status theory to model triangles and "bridge" edges (edges not included in any triangles). The status theory is applicable to directed signed networks, interpreting positive or negative signs as indicators of relative status between nodes. 
SiGATs____, which extends Graph Attention Networks (GATs) to signed networks, also utilizes these two signed graph theories to derive graph motifs for more effective message passing. %SNEAntroduces masked self-attentional layers to estimate the importance coefficient of node pairs and then aggregates different importance of neighbors based on Balance Theory.
SiNEs____ proposes a signed network embedding framework guided by the extended structural balance theory. 
% SGDNET leverages a random walk technique specifically tailored for signed graphs, effectively diffusing hidden node features in line with Social Balance Theory. 
GS-GNN____ applies a dual GNN architecture that combines a prototype-based GNN to process positive and negative edges to learn node representations. SLGNN____
%the spectral graph theory and graph signal processing to design a spectral-based signed graph neural network, utilizing 
especially design low-pass and high-pass graph convolution filters to capture both low-frequency and high-frequency information from positive and negative links. 

%\vspace{-8pt}
\textbf{Signed Graph Clustering.}
% The study of signed graph clustering traces its roots back to the 1950s, beginning with Cartwright and Harary's introduction of the balance concept which has been applied to diverse configurations, such as communication networks, sociometric structures, and neural networks. Subsequently, Harary articulated this concept more formally as \textit{Social Balance Theory}____.
Its study has its roots in Social Balance Theory____, which is equivalent to the $2$-way partition problem in signed graphs____. 
Building upon this foundational concept, ____ propose a signed spectral clustering method that utilizes the signed graph Laplacian and graph kernels to address the $2$-way partition problem.  However, ____ argues that community detection in signed graphs is equivalent to identifying $K$-way clusters using an agent-based heuristic. The \textbf{\textit{Weak Balance Theory}}____ relaxes Balance theory to enable $K$-way clustering. Following ____, ____ proposed the ``Balanced Normalized Cut (BNC)'' for $K$-way clustering, aiming to find an optimal clustering assignment that minimizes positive edges between different clusters and negative edges within clusters with equal priority. SPONGE____ transforms this discrete NP-hard problem into a continuous generalized eigenproblem and employs LOBPCG____, a preconditioned eigensolver, to solve large positive definite generalized eigenproblems. In contrast to the above $K$-way complete partitioning, ____ targets detecting $K$ conflicting groups, allowing other nodes to be neutral regarding the conflict structure in search. This conflicting-group detection problem can be characterized as the maximum discrete Rayleigh's quotient problem.

While GNNs have been extensively applied to unsigned graph clustering____, their adoption in signed graph clustering remains overlooked. A notable exception is the Semi-Supervised Signed NETwork Clustering (SSSNET)____, which simultaneously learns node embeddings and cluster assignments by minimizing the clustering loss and a Cross-Entropy classification loss. In contrast, our work develops an unsupervised method for signed graph clustering, eliminating the reliance on ground truth labels.