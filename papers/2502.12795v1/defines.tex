
\def\paperTitle{A Visual Approach for Health Information Exploration}
\def\paperRunningTitle{A Visual Approach for Health Information Exploration}
\def\paperAbstract{%
The effective and targeted provision of health information to consumers, specifically tailored to their needs and preferences, is indispensable in healthcare. With access to appropriate health information and adequate understanding, consumers are more likely to make informed and healthy decisions, become more proficient in recognizing symptoms, and potentially experience improvements in the prevention or treatment of their medical conditions. 
%
Most of today’s health information, however, is provided in the form of static documents. 
%
In this paper, we present a novel and innovative \emph{visual health information system} based on adaptive document visualizations. 
%
Depending on the user's information needs and preferences, the system can display its content with document visualization techniques at different levels of detail, aggregation, and visual granularity. 
%
Users can navigate using content organization along sections or automatically computed topics, and choose abstractions from full texts to word clouds. 
%
Our first contribution is a formative user study which demonstrated that the implemented document visualizations offer several advantages over traditional forms of document exploration. 
%
Informed from that, we identified a number of crucial aspects for further system development.
%
Our second contribution is the introduction of an \emph{interaction provenance visualization} which allows users to inspect which content, in which representation, and in which order has been received. 
%
We show how this allows to analyze different document exploration and navigation patterns, useful for automatic adaptation and recommendation functions. We also define a baseline taxonomy for adapting the document presentations which can, in principle, be leveraged by the observed user patterns. 
%
The interaction provenance view, furthermore, allows users to reflect on their exploration and inform future usage of the system.}
\def\paperAcknowledgement{This work was funded by the Austrian Science Fund (FWF) as part of the project `Human-Centered Interactive Adaptive Visual Approaches in High-Quality Health Information' (\apluschis; Grant No. FG 11-B).}
\def\keywordOne{Document Visualization}
\def\keywordTwo{Adaptive Visualization}
\def\keywordThree{Interaction Analysis}
\def\keywordFour{Health Information}
\def\keywordFive{Evaluation}



\newacronym{dl}{DL}{Document Library}
\newacronym{toc}{ToC}{Table of Contents}
\newacronym{wc}{WC}{Word Cloud}
\newacronym{hwc}{HWC}{History WC}
\newacronym{topicb}{ToB}{Topicbar}
\newacronym{topiccloud}{ToC}{Topiccloud}
\newacronym{tileb}{TiB}{Tilebar}
\newacronym{snps}{Snps}{Snippets}
\newacronym{fulltext}{FullT}{Fulltext}
\newacronym{is}{IS}{Image Slider}
\newacronym{chis}{CHIS}{Consumer Health Information Systems}
\newacronym{apchis}{A\textsuperscript{+}CHIS}{Adaptive CHIS}
\newacronym{ttwodm}{T2DM}{Type 2 Diabetes Mellitus}
 
\newacronym{cwt}{CWT}{Cognitive Walkthrough}
\def \cwt {\acrshort{cwt}}

\newcommand{\DocumentLibrary}{\acrlong{dl}}
\newcommand{\dl}{\acrshort{dl}}
\newcommand{\TableOfContents}{\acrlong{toc}}
\newcommand{\toc}{\acrshort{toc}}
\newcommand{\WordCloud}{\acrlong{wc}}
\newcommand{\wc}{\acrshort{wc}}
\newcommand{\Topicbar}{\acrlong{topicb}}
\newcommand{\tob}{\acrshort{topicb}}
\newcommand{\Tilebar}{\acrlong{tileb}}
\newcommand{\tib}{\acrshort{tileb}}
\newcommand{\Snippets}{\acrlong{snps}}
\newcommand{\snps}{\acrshort{snps}}
\newcommand{\ImageSlider}{\acrlong{is}}
\newcommand{\is}{\acrshort{is}}
\newcommand{\isL}{\acrshort{is}\textsubscript{l}}
\newcommand{\isS}{\acrshort{is}\textsubscript{s}}
\newcommand{\HistoryWordCloud}{\acrlong{hwc}}
\newcommand{\hwc}{\acrshort{hwc}}
\newcommand{\Searchbar}{Searchbar}
\newcommand{\Startscreen}{Startscreen}
\newcommand{\Fulltext}{\acrlong{fulltext}}
\newcommand{\fullt}{\acrshort{fulltext}}
% \newcommand{\HistoryImageSlider}{History Image Slider}
\def \apluschis {\acrshort{apchis}}
\def \chis {\acrshort{chis}}

\newcommand{\unit}[1]{\ \mathrm{#1}}

% Tasks
\def \taskWcOne {\wc1}
\def \taskWcTwo {\wc2}
\def \taskWcThree {\wc3}
\def \taskWcFour {\wc4}
\def \taskTibOne {\tib1}
\def \taskTibTwo {\tib2}
\def \taskTibThree {\tib3}
\def \taskTibFour {\tib4}
\def \taskIsOne {\is1}
\def \taskIsTwo {\is2}
\def \taskIsThree {\is3}

% Participants
\def \POne {P01}
\def \PTwo {P02}
\def \PThree {P03}
\def \PFour {P04}
\def \PFive {P05}
\def \PSix {P06}
\def \PSeven {P07}
\def \PEight {P08}
\def \PNine {P09}
\def \PTen {P10}
\def \PEleven {P11}
\def \PTwelve {P12}

\newcommand{\figref}[1]{Figure~\ref{#1}}
\newcommand{\secref}[1]{Section~\ref{#1}}
\newcommand{\tableref}[1]{Table~\ref{#1}}
\newcommand{\eg}{e.g.,}