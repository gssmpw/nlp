\section{The flag trick in action}\label{sec:examples}
In this section, we provide some applications of the flag trick to several learning problems. We choose to focus on subspace recovery, trace ratio and spectral clustering problems. Other ones, like domain adaptation, matrix completion and subspace tracking are developed or mentioned in the last subsection but not experimented for conciseness.

\subsection{Outline and experimental setting}
For each application, we first present the learning problem as an optimization on Grassmannians. Second, we formulate the associated flag learning problem by applying the flag trick (Definition~\ref{def:flag_trick}). Third, we optimize the problem on flag manifolds with the steepest descent method (Algorithm~\ref{alg:GD})---more advanced algorithms are also derived in the appendix. 
Finally, we perform various nestedness and ensemble learning experiments via Algorithm~\ref{alg:flag_trick} on both synthetic and real datasets.

The general methodology to compare Grassmann-based methods to flag-based methods is the following one. For each experiment, we first choose a flag signature $~{q_{1\rightarrow d} := (q_1, \dots, q_d)}$, then we run independent optimization algorithms on $\Gr(p, q_1), \dots, \Gr(p, q_d)$~\eqref{eq:subspace_problem} and finally we compare the optimal subspaces $\S_k^* \in \Gr(p, q_k)$ to the optimal flag of subspaces $\Sf^* \in \Fl(p, q_{1\rightarrow d})$ obtained via the flag trick~\eqref{eq:flag_problem}. 
To show the nestedness issue in Grassmann-based methods, we compute the subspace distances $\Theta(\S_k^*, \S_{k+1}^*)_{k=1\dots d-1}$, where $\Theta$ is the generalized Grassmann distance of~\citet[Eq.~(14)]{ye_schubert_2016}. It consists in the $\ell_2$ norm of the principal angles, which can be obtained from the singular value decomposition (SVD) of the inner-products matrices ${U_k}\T {U_{k+1}}$, where $U_k \in \St(p, q_k)$ is an orthonormal basis of $\S_k^*$.

Regarding the implementation of the steepest descent algorithm on flag manifolds (Algorithm~\ref{alg:GD}), we develop a new class of manifolds in \href{https://pymanopt.org/}{PyManOpt}~\citep{boumal_manopt_2014,townsend_pymanopt_2016}, and run their \href{https://github.com/pymanopt/pymanopt/blob/master/src/pymanopt/optimizers/steepest_descent.py}{SteepestDescent} algorithm. Our implementation of the \texttt{Flag} class is based on the Stiefel representation of flag manifolds, detailed in \autoref{sec:flags}, with the retraction being the polar retraction. For the computation of the gradient, we use automatic differentiation with the \texttt{\href{https://github.com/HIPS/autograd}{autograd}} package. We could derive the gradients by hand from the expressions we get, but we use automatic differentiation as strongly suggested in PyManOpt's \href{https://pymanopt.org/docs/stable/quickstart.html}{documentation}.
Finally, the real datasets and the machine learning methods used in the experiments can be found in \href{https://scikit-learn.org/stable/}{scikit-learn}~\citep{pedregosa_scikit-learn_2011}.

\subsection{The flag trick for robust subspace recovery problems}
Let us consider a dataset that is a union of \textit{inliers} and \textit{outliers}---the inliers are assumed to lie near a low-dimensional subspace $\S$ while the outliers live in the ambient space. The aim of robust subspace recovery (RSR) is to recover $\S$~\citep{lerman_overview_2018}. In that sense, RSR is an outlier-robust extension of classical dimension reduction methods like PCA.
Without further specifications, the RSR problem might not be well-posed. Therefore, the works in this domain often have to make some assumptions on the inlier and outlier distributions in order to obtain some convergence and recovery guarantees.
For instance, in \citet{lerman_overview_2018}, it is assumed that the inliers ``fill'' the lower-dimensional subspace and that the outliers are not much ``aligned''; this is rigorously defined in~\citet{lerman_robust_2015} and~\citet{maunu_well-tempered_2019} through \textit{permeance} and \textit{alignment} statistics.
A typical generative model following those assumptions is the \textit{Haystack model} introduced in~\citet{lerman_robust_2015}. The Haystack model assumes an isotropic Gaussian distribution on the subspace for the inliers and an isotropic Gaussian distribution on the (full) ambient space for the outliers. A more realistic model---the \textit{generalized Haystack model}---is introduced in~\citet{maunu_well-tempered_2019} to circumvent the simplistic nature of the Haystack model. This one assumes general (anisotropic) Gaussian distributions for the inliers and outliers. This makes the learning harder, since the anisotropy may keep the inliers from properly permeating the low-dimensional subspace---as discussed in \autoref{sec:intro}. Therefore, one has to make some stronger assumptions on the inlier-outlier ratio and the covariance eigenvalues distributions to derive some convergence and recovery guarantees.
\begin{remark}[Parametrization of RSR generative models]
The inlier distribution in the Haystack model follows the \textit{isotropic PPCA} model~\citep{bouveyron_hddc_2007, bouveyron_intrinsic_2011}, while it follows the \textit{PPCA} model~\citep{tipping_probabilistic_1999} in the case of the generalized Haystack model. Both models are a special case of the principal subspace analysis models~\citep{szwagier_curse_2024}. However, as argued in~\citet{szwagier_curse_2024}, while the Haystack model is parameterized with Grassmannians, the generalized Haystack model---which has more degrees of freedom accounting for the anisotropy---is parameterized with Stiefel manifolds. Therefore, from a statistical modeling perspective, it only makes sense to conduct subspace learning experiments on the Haystack model and not the generalized one.
\end{remark}


\subsubsection{Application of the flag trick to RSR}
Among the large family of methods for robust subspace recovery~\citep{lerman_overview_2018} we consider the one of \textit{least absolute deviation} (LAD) minimization, which can be explicitly formulated as an optimization problem on Grassmannians.
PCA minimizes the sum of \textit{squared} Euclidean distances between the points and the subspace. In the case of an outlier-contaminated dataset, the squared Euclidean distances might penalize too much the outliers, and therefore make the optimal subspace too much influenced by the outliers. To circumvent this well-known sensitivity of squared norms to outliers, many works propose to minimize the sum of \textit{absolute} Euclidean distances, which defines the LAD minimization problem:
\begin{equation}\label{eq:RSR_Gr}
    \argmin{\S \in \Gr(p, q)} \sum_{i=1}^n \norm{x_i - \Pi_{\S} x_i}_2.
\end{equation}
The latter has the interest of being rotationally invariant~\citep{ding_r1-pca_2006} but the drawback of being NP-hard~\citep{mccoy_two_2011, lerman_overview_2018} and obviously non-convex since Grassmannians are not.  % this sentence is maybe not very important here, but let's still keep it...
A first body of works relaxes the problem, for instance by optimizing on the convex hull of Grassmannians~\citep{mccoy_two_2011, xu_robust_2012, zhang_novel_2014, lerman_robust_2015}.
A second body of works directly optimizes the LAD on Grassmannians, either with an IRLS algorithm~\citep{lerman_fast_2018} or with a geodesic gradient descent~\citep{maunu_well-tempered_2019}, both achieving very good results in terms of recovery and speed.
The following proposition applies the flag trick to the LAD problem~\eqref{eq:RSR_Gr}.
\begin{proposition}[Flag trick for RSR]\label{prop:RSR}
The flag trick applied to LAD reads
\begin{equation}\label{eq:RSR_Fl}
    \argmin{\S_{1:d} \in \Fl(p, q_{1:d})} \sum_{i=1}^n \norm{x_i - \Pi_{\S_{1:d}} x_i}_2,
\end{equation} 
and is equivalent to the following optimization problem:
\begin{equation}\label{eq:RSR_Fl_equiv}
	\argmin{U_{1:d+1} \in \O(p)} \sum_{i=1}^n \sqrt{\sum_{k=1}^{d+1} \lrp{\frac {k-1} {d}}^2 \norm{{U_k}\T x_i}_2^2}.
\end{equation}
\end{proposition}
\begin{proof}
The proof is given in Appendix (\autoref{app:RSR}).
\end{proof}
Equation~\eqref{eq:RSR_Fl_equiv} tells us several things. First of all, due to the $((k-1)/d)^2$ weights, the points that are far from the center should be in the first principal subspaces. Second, the square root decreases the influence of the outlying points, compared to the nested PCA of Theorem~\ref{thm:flag_trick}.
Third, although less obvious, we can see that numerical issues are less prone to happen with the flag trick~\eqref{eq:RSR_Fl_equiv} than with the LAD~\eqref{eq:RSR_Gr}, since the quantity under the square root is zero only when the first subspace $~{\S_1 = \operatorname{Span}(U_1)}$ contains a data point. Therefore, whenever $q_1$ is smaller than the dimension $q$ that one would have tried for classical RSR, the non-differentiability and exploding-gradient issues caused by the square root are less likely.
Finally, since the nested LAD minimization~\eqref{eq:RSR_Fl} is nothing but a robust version of the nested PCA of Theorem~\ref{thm:flag_trick} ($\sum_{i=1}^n \|x_i - \Pi_{\Sf} x_i\|_2^2$), a natural idea can be to initialize the optimization algorithm with the nested PCA solution. This is what is done in~\citet{maunu_well-tempered_2019} for LAD minimization, and it is coming with stronger recovery guarantees.

\subsubsection{Nestedness experiments for RSR}
We first consider a dataset consisting in a mixture of two multivariate Gaussians: the inliers, with zero mean, covariance matrix $\diag{5, 1, .1}$, $n_\mathrm{in} = 450$ and the outliers, with zero mean, covariance matrix $\diag{.1, .1, 5}$, $n_\mathrm{out} = 50$. The dataset is therefore following the generalized haystack model of~\citet{maunu_well-tempered_2019}.
The ambient dimension is $p = 3$ and the intrinsic dimensions that we try are $q_{1:2} = (1, 2)$.
We run Algorithm~\ref{alg:GD} on Grassmann manifolds to solve the LAD minimization problem~\eqref{eq:RSR_Gr}, successively for $q_1 = 1$ and $q_2 = 2$. Then we plot the projections of the data points onto the optimal subspaces. We compare them to the nested projections onto the optimal flag output by running Algorithm~\ref{alg:GD} on $\Fl(3, (1, 2))$ to solve~\eqref{eq:RSR_Fl}. The results are shown in Figure~\ref{fig:RSR_nested}.
\begin{figure}
	\centering
    \includegraphics[width=\linewidth]{Fig/FT_exp_RSR_synthetic.pdf}
    \caption{
    Illustration of the nestedness issue in robust subspace recovery. Given a dataset consisting in a mixture of inliers (blue) and outliers (orange) we plot its projection onto the optimal 1D subspace and 2D subspace obtained by solving the associated Grassmannian optimization problem~\eqref{eq:RSR_Gr} or flag optimization problem~\eqref{eq:RSR_Fl}. 
    We can see that the Grassmann representations are not nested, while the flag representations are nested and robust to outliers.}
	\label{fig:RSR_nested}
\end{figure}
\begin{figure}
	\centering
    \includegraphics[width=\linewidth]{Fig/FT_exp_RSR_digits}
    \caption{Distributions of (increasing) Euclidean reconstruction errors $\|x_i - \Pi_{\S_{1:d}^*} x_i\|_{i=1\dots n}$ on the corrupted digits dataset for Grassmann and flag methods. While inliers and outliers are mixed with the subspace method, we can see a \textit{clear transition} with flags. This can be explained by the multilevel nature of the flag trick.}
	\label{fig:RSR_outlier}
\end{figure}
We can see that the Grassmann-based projections are non-nested while their flag counterparts are not only nested but also robust to outliers. This could be explained by the nestedness constraint of flag manifolds which imposes the 2D subspace to contain the 1D subspace.

Second, we perform an outlier detection experiment. A common methodology to detect outliers in a corrupted dataset is to first look for an outlier-robust subspace and then plot the distribution of distances between the data points and their projection onto the recovered subspace. This distribution is expected to show a clear gap between the inliers and outliers~\citep[Fig.~3.7]{vidal_generalized_2016}.
However, in practice, one does not know which subspace dimension $q$ to choose. If $q$ is too large, then the recovered subspace may contain both inliers and outliers, and therefore the distribution of distances might be roughly $0$. In contrast, if $q$ is too small, then some inliers may lie too far from the recovered subspace and be detected as outliers. An idea in the spirit of the flag trick is to perform an average ensembling of the reconstruction errors. More specifically, if $\norm{x_i - \Pi_\S x_i}_2$ is the classical measure of robust reconstruction error, then we compute $\norm{x_i - \Pi_{\S_{1:d}} x_i}_2$. Such a score extracts information from projections at different levels and might result in a richer multilevel analysis.
We consider a dataset where the inliers are images of $8 \times 8$ handwritten $0$'s and outliers correspond to other digits from $1$ to $9$, all extracted from the classical handwritten digits dataset~\citep{alpaydin_optical_1998}.
The ambient dimension is $p = 64$, the number of inliers is $n_\mathrm{in} = 90$ and the number of outliers is $n_\mathrm{out} = 10$. The intrinsic dimensions that we try are $q_{1:3} = (1, 2, 5)$.
We plot the reconstruction error for the points of the digits dataset on the optimal flag $\S_{1:3}^* \in \Fl(p, (1, 2, 5))$ in Figure~\ref{fig:RSR_outlier}. We compare it to the metric on $\S_3 \in \Gr(p, q_3)$.
We can see that the flag trick enables to clearly distinguish the inliers from the outliers compared to the Grassmann-based method, which is a consequence of the multilevel nature of flags.


\subsubsection{Discussion on RSR optimization and objective functions}\label{subsubsec:RSR_discu}
\paragraph{An IRLS algorithm}
In all the experiments of this paper, we use a steepest descent method on flag manifolds (Algorithm~\ref{alg:GD}) to solve the flag problems.
However, for the specific problem of RSR~\eqref{eq:RSR_Fl}, we believe that more adapted algorithms should be derived, notably due to the non-differentiability and exploding-gradient issues caused by the square root.
To that extent, we derive in appendix (\autoref{app:RSR}) an IRLS scheme (Algorithm~\ref{alg:FMF}) for RSR. In short, the RSR problem~\eqref{eq:RSR_Fl} can be reformulated as a weighted least squares problem $\sum_{i=1}^n w_i \norm{x_i - \Pi_{\Sf} x_i}_2^2$ with $w_i ={1}/{\norm{x_i - \Pi_{\Sf} x_i}_2}$ and optimized iteratively, with explicit expressions obtained via our central Theorem~\ref{thm:flag_trick}. We insist on the fact that such an IRLS algorithm could not be developed with the flagification of~\citet{pennec_barycentric_2018,mankovich_fun_2024}, since a sum of square roots does not correspond to a least-squares problem.

\paragraph{More RSR problems}
In this work, we explore one specific problem of RSR for conciseness, but we could investigate many other related problems, including robust PCA. 
Notably, drawing from the Grassmann averages (GA) method~\citep{hauberg_scalable_2016}, one could develop many new multilevel RSR and RPCA objective functions.
The idea behind GA is to replace data points with 1D subspaces ($\S_i = \operatorname{Span}(x_i)$) and then perform subspace averaging methods to find a robust prototype for the dataset. GA ends up solving problems of the form $\operatorname{argmin}_{\S \in \Gr(p, 1)} \sum_{i=1}^n w_i \, \operatorname{dist}_{\Gr(p, 1)}^2(\operatorname{Span}(x_i), \S)$, where $w_i$ are some weights and $\operatorname{dist}_{\Gr(p, 1)}$ is a particular subspace distance detailed in~\citet{hauberg_scalable_2016}. Using instead some multidimensional subspace distances, like the principal angles and its variants~\citep{hamm_grassmann_2008, ye_schubert_2016}, we can develop many variants of the Grassmann averages, of the form $\operatorname{argmin}_{\Sf \in\Gr(p, q)} \sum_{i=1}^n w_i \, \rho(\sqrt{{x_i}\T  \Pi_{\S} {x_i}})$, where $\rho\colon\R\to\R$ is a real function, like $\rho(x) = \arccos(x)$ if we want subspace-angle-like distances, $\rho(x) = - x^2$ if we want PCA-like solutions, and many other possible robust variants.
Applying the flag trick to those problems yields the following robust multilevel problem: $\operatorname{argmin}_{\Sf \in\Fl(p, \qf)} \sum_{i=1}^n w_i \rho(\sqrt{{x_i}\T  \Pi_{\Sf} {x_i}})$.
\subsection{The flag trick for trace ratio problems}\label{subsec:TR}
Trace ratio problems are ubiquitous in machine learning~\citep{ngo_trace_2012}. They write as:
\begin{equation}\label{eq:TR_St}
\argmax{U \in \St(p, q)} \frac{\tr{U\T A U}}{\tr{U\T B U}},
\end{equation}
where $A, B \in \R^{p\times p}$ are positive semi-definite matrices, with $\operatorname{rank}(B) > p - q$.

A famous example of TR problem is Fisher's linear discriminant analysis (LDA)~\citep{fisher_use_1936,belhumeur_eigenfaces_1997}.
It is common in machine learning to project the data onto a low-dimensional subspace before fitting a classifier, in order to circumvent the curse of dimensionality. It is well known that performing an unsupervised dimension reduction method like PCA comes with the risks of mixing up the classes, since the directions of maximal variance are not necessarily the most discriminating ones~\citep{chang_using_1983}. The goal of LDA is to use the knowledge of the data labels to learn a linear subspace that does not mix the classes.
Let $~{X := [x_1|\dots|x_n] \in \R^{p\times n}}$ be a dataset with labels $Y := [y_1|\dots|y_n] \in {[1, C]}^n$. Let $\mu = \frac{1}{n} \sum_{i=1}^n x_i$ be the dataset mean and $\mu_c = \frac{1}{\#\{i : y_i=c\}}\sum_{i : y_i=c} x_i$ be the class-wise means. 
The idea of LDA is to search for a subspace $\S \in \Gr(p, q)$ that simultaneously maximizes the projected \textit{between-class variance} $\sum_{c=1}^C \|\Pi_\S \mu_c - \Pi_\S \mu\|_2^2$ and minimizes the projected \textit{within-class variance} $\sum_{c=1}^C \sum_{i : y_i = c} \|\Pi_\S x_i - \Pi_\S \mu_c\|_2^2$. This can be reformulated as a trace ratio problem~\eqref{eq:TR_St}, with $A = \sum_{c=1}^C (\mu_c - \mu) (\mu_c - \mu)\T$ and $B = \sum_{c=1}^C \sum_{i : y_i = c} (x_i - \mu_c) (x_i - \mu_c)\T$.


More generally, a large family of dimension reduction methods can be reformulated as a TR problem. The seminal work of~\citet{yan_graph_2007} shows that many dimension reduction and manifold learning objective functions can be written as a trace ratio involving Laplacian matrices of attraction and repulsion graphs. Intuitively, those graphs determine which points should be close in the latent space and which ones should be far apart.
Other methods involving a ratio of traces are \textit{multi-view learning}~\citep{wang_trace_2023}, \textit{partial least squares} (PLS)~\citep{geladi_partial_1986,barker_partial_2003} and \textit{canonical correlation analysis} (CCA)~\citep{hardoon_canonical_2004}, although these methods are originally \textit{sequential} problems (cf. footnote~\ref{footnote:sequential}) and not \textit{subspace} problems.

Classical Newton-like algorithms for solving the TR problem~\eqref{eq:TR_St} come from the seminal works of~\citet{guo_generalized_2003, wang_trace_2007, jia_trace_2009}.
The interest of optimizing a trace-ratio instead of a ratio-trace (of the form $\tr{(U\T B U)^{-1}(U\T A U)}$), that enjoys an explicit solution given by a generalized eigenvalue decomposition, is also tackled in those papers. The \textit{repulsion Laplaceans}~\citep{kokiopoulou_enhanced_2009} instead propose to solve a regularized version $\tr{U\T B U} - \rho \tr{U\T A U}$, which enjoys a closed-form, but has a hyperparameter $\rho$, which is directly optimized in the classical Newton-like algorithms for trace ratio problems.

\subsubsection{Application of the flag trick to trace ratio problems}
The trace ratio problem~\eqref{eq:TR_St} can be straightforwardly reformulated as an optimization problem on Grassmannians, due to the orthogonal invariance of the objective function:
\begin{equation}\label{eq:TR_Gr}
\argmax{\S \in \Gr(p, q)} \frac{\tr{\Pi_\S A}}{\tr{\Pi_\S B}}.
\end{equation}
The following proposition applies the flag trick to the TR problem~\eqref{eq:TR_Gr}.
\begin{proposition}[Flag trick for TR]\label{prop:TR}
The flag trick applied to the TR problem~\eqref{eq:TR_Gr} reads
\begin{equation}\label{eq:TR_Fl}
	\argmax{\S_{1:d} \in \Fl(p, q_{1:d})} \frac{\tr{\Pi_{\S_{1:d}} A}}{\tr{\Pi_{\S_{1:d}} B}}.
\end{equation}
and is equivalent to the following optimization problem:
\begin{equation}\label{eq:TR_Fl_equiv}
\argmax{U_{1:d} \in \St(p, q)} \frac{\sum_{k=1}^{d} (d - (k-1)) \tr{{U_k}\T A {U_k}}}{\sum_{l=1}^{d} (d - (l-1)) \tr{{U_{l}}\T B {U_{l}}}}.
\end{equation}
\end{proposition}
\begin{proof}
The proof is given in Appendix (\autoref{app:TR}).
\end{proof}
Equation~\eqref{eq:TR_Fl_equiv} tells us several things. First, the subspaces $~{\operatorname{Span}(U_1) \perp \dots \perp \operatorname{Span}(U_d)}$ are weighted decreasingly, which means that they have less and less importance with respect to the TR objective.
Second, we can see that the nested trace ratio problem~\eqref{eq:TR_Fl} somewhat maximizes the numerator $\tr{\Pi_{\S_{1:d}} A}$ while minimizing the denominator $\tr{\Pi_{\S_{1:d}} B}$. Both subproblems have an explicit solution corresponding to our nested PCA Theorem~\ref{thm:flag_trick}. Hence, one can naturally initialize the steepest descent algorithm with the $q$ highest eigenvalues of $A$ or the $q$ lowest eigenvalues of $B$ depending on the application.
For instance, for LDA, initializing Algorithm~\ref{alg:GD} with the highest eigenvalues of $A$ would spread the classes far apart, while initializing it with the lowest eigenvalues of $B$ would concentrate the classes, which seems less desirable since we do not want the classes to concentrate at the same point.

\subsubsection{Nestedness experiments for trace ratio problems}
For all the experiments of this subsection, we consider the particular TR problem of LDA, although many other applications (\textit{marginal Fisher analysis}~\citep{yan_graph_2007}, \textit{local discriminant embedding}~\citep{chen_local_2005} etc.) could be investigated similarly.

First, we consider a synthetic dataset with five clusters.
The ambient dimension is $p = 3$ and the intrinsic dimensions that we try are $q_{1:2} = (1, 2)$.
We adopt a preprocessing strategy similar to~\citet{ngo_trace_2012}: we first center the data, then run a PCA to reduce the dimension to $n - C$ (if $n - C < p$), then construct the LDA scatter matrices $A$ and $B$, then add a diagonal covariance regularization of $10^{-5}$ times their trace and finally normalize them to have unit trace.
We run Algorithm~\ref{alg:GD} on Grassmann manifolds to solve the TR maximization problem~\eqref{eq:TR_Gr}, successively for $q_1 = 1$ and $q_2 = 2$. Then we plot the projections of the data points onto the optimal subspaces. We compare them to the nested projections onto the optimal flag output by running Algorithm~\ref{alg:GD} on $\Fl(3, (1, 2))$ to solve~\eqref{eq:TR_Fl}. The results are shown in Figure~\ref{fig:TR_nested}.
\begin{figure}
	\centering
    \includegraphics[width=.9\linewidth]{Fig/FT_exp_TR_synthetic.pdf}
    \caption{
    Illustration of the nestedness issue in linear discriminant analysis (trace ratio problem). Given a dataset with five clusters, we plot its projection onto the optimal 1D subspace and 2D subspace obtained by solving the associated Grassmannian optimization problem~\eqref{eq:TR_Gr} or flag optimization problem~\eqref{eq:TR_Fl}. 
    We can see that the Grassmann representations are not nested, while the flag representations are nested and well capture the distribution of clusters. In this example, it is less the nestedness than the \textit{rotation} of the optimal axes inside the 2D subspace that is critical to the analysis of the Grassmann-based method.
    }
	\label{fig:TR_nested}
\end{figure}
\begin{figure}
	\centering
    \includegraphics[width=.9\linewidth]{Fig/FT_exp_TR_digits.pdf}
    \caption{
    Illustration of the nestedness issue in linear discriminant analysis (trace ratio problem) on the digits dataset. For $q_k \in \qf := (1, 2, \dots, 63)$, we solve the Grassmannian optimization problem~\eqref{eq:TR_Gr} on $\Gr(64, q_k)$ and plot the subspace angles $\Theta(\S_k^*, \S_{k+1}^*)$ (left) and explained variances ${\operatorname{tr}(\Pi_{\S_k^*} X X\T)} / {\operatorname{tr}(X X\T)}$ (right) as a function of $k$. We compare those quantities to the ones obtained by solving the flag optimization problem~\eqref{eq:TR_Fl}. 
    We can see that the Grassmann-based method is highly non-nested and even yields an extremely paradoxical non-increasing explained variance (cf. red circle on the right).
    }
	\label{fig:TR_nested_digits}
\end{figure}
We can see that the Grassmann representations are non-nested while their flag counterparts perfectly capture the filtration of subspaces that best and best approximates the distribution while discriminating the classes. Even if the colors make us realize that the issue in this experiment for LDA  is not much about the non-nestedness but rather about the rotation of the principal axes within the 2D subspace, we still have an important issue of consistency.

Second, we consider the (full) handwritten digits dataset~\citep{alpaydin_optical_1998}. It contains $8 \times 8$ pixels images of handwritten digits, from $0$ to $9$, almost uniformly class-balanced. One has $n = 1797$, $p=64$ and $C = 10$.
We run a steepest descent algorithm to solve the trace ratio problem~\eqref{eq:TR_Fl}. We choose the full signature $q_{1:63} = (1, 2, \dots, 63)$ and compare the output flag to the individual subspaces output by running optimization on $\Gr(p, q_k)$ for $q_k \in q_{1:d}$.
We plot the subspace angles $\Theta(\S_k^*, \S_{k+1}^*)$ and the explained variance ${\operatorname{tr}(\Pi_{\S_k^*} X X\T)} / {\operatorname{tr}(X X\T)}$ as a function of the $k$. The results are illustrated in \autoref{fig:TR_nested_digits}.
We see that the subspace angles are always positive and even very large sometimes with the LDA. Worst, the explained variance is not monotonous. This implies that we sometimes \textit{loose} some information when \textit{increasing} the dimension, which is extremely paradoxical.

Third, we perform some classification experiments on the optimal subspaces for different datasets. For different datasets, we run the optimization problems on $\Fl(p, q_{1:d})$, then project the data onto the different subspaces in $\S_{1:d}^*$ and run a nearest neighbors classifier with $5$ neighbors.
The predictions are then ensembled (cf. Algorithm~\ref{alg:flag_trick}) by weighted averaging, either with uniform weights or with weights minimizing the average cross-entropy:
\begin{equation}\label{eq:soft_voting}
	w_1^*, \dots, w_d^* = \argmin{\substack{w_k \geq 0 \\ \sum_{k=1}^d w_k = 1}} - \frac 1 {n C} \sum_{i=1}^n \sum_{c=1}^C y_{ic} \ln\lrp{\sum_{k=1}^d w_k y_{kic}^*},
\end{equation}
where $y_{kic}^* \in [0, 1]$ is the predicted probability that $x_i \in \R^p$ belongs to class $c \in \{1 \dots C\}$, by the classifier $g_k^*$ that is trained on $Z_k := {U_k^*}\T X \in \R^{q_k \times n}$. One can show that the latter is a convex problem, which we optimize using the \href{https://www.cvxpy.org/index.html}{cvxpy} Python package~\citep{diamond2016cvxpy}.
We report the results in \autoref{tab:TR_classif}.
\begin{table}
  \caption{Classification results for the TR problem on real datasets. For each method (Gr: Grassmann optimization~\eqref{eq:TR_Gr}, Fl: flag optimization~\eqref{eq:TR_Gr}, Fl-U: flag optimization + uniform soft voting, Fl-W: flag optimization + optimal soft voting~\eqref{eq:soft_voting}), we give the cross-entropy of the projected-predictions with respect to the true labels.}
  \label{tab:TR_classif}
  \centering
  \begin{tabular}{ccccccccc}
    \toprule
    dataset & $n$ & $p$ & $q_{1:d}$ & Gr & Fl & Fl-U & Fl-W & weights\\
    \midrule
    breast & $569$ & $30$ & $(1, 2, 5)$ & $0.0986$ & $0.0978$ & $0.0942$ & $0.0915$ & $(0.754, 0, 0.246)$\\
    iris & $150$ & $4$ & $(1, 2, 3)$ & $0.0372$ & $0.0441$ & $0.0410$ & $0.0368$ & $(0.985, 0, 0.015)$\\
    wine & $178$ & $13$ & $(1, 2, 5)$ & $0.0897$ & $0.0800$ & $0.1503$ & $0.0677$ & $(0, 1, 0)$\\
    digits & $1797$ & $64$ & $(1, 2, 5, 10)$ & $0.4507$ & $0.4419$ & $0.5645$ & $0.4374$ & $(0, 0, 0.239, 0.761)$\\
    \bottomrule
  \end{tabular}
\end{table}
The first example tells us that the optimal $5D$ subspace obtained by Grassmann optimization less discriminates the classes than the $5D$ subspace from the optimal flag. This may show that the flag takes into account some lower dimensional variability that enables to better discriminate the classes. We can also see that the uniform averaging of the predictors at different dimensions improves the classification. Finally, the optimal weights improve even more the classification and tell that the best discrimination happens by taking a soft blend of classifier at dimensions $1$ and $5$. Similar kinds of analyses can be made for the other examples.

\subsubsection{Discussion on TR optimization and kernelization}
\paragraph{A Newton algorithm}
In all the experiments of this paper, we use a steepest descent method on flag manifolds (Algorithm~\ref{alg:GD}) to solve the flag problems.
However, for the specific problem of TR~\eqref{eq:TR_Fl}, we believe that more adapted algorithms should be derived to take into account the specific form of the objective function, which is classically solved via a Newton-Lanczos method~\citep{ngo_trace_2012}. 
To that extent, we develop in the appendix (\autoref{app:TR}) an extension of the baseline Newton-Lanczos algorithm for the flag-tricked problem~\eqref{eq:TR_Fl}.
In short, the latter can be reformulated as a penalized optimization problem of the form $\operatorname{argmax}_{\Sf\in\Fl(p, \qf)} {\sum_{k=1}^d \tr{\Pi_{\S_k} (A - \rho B)}}$, where $\rho$ is updated iteratively according to a Newton scheme. Once again, our central Theorem~\ref{thm:flag_trick} enables to get explicit expressions for the iterations, which results without much difficulties in a Newton method, that is known to be much more efficient than first-order methods like the steepest descent.

\paragraph{A non-linearization via the kernel trick}
The classical trace ratio problems look for \textit{linear} embeddings of the data.
However, in most cases, the data follow a \textit{nonlinear} distribution, which may cause linear dimension reduction methods to fail. The \textit{kernel trick}~\citep{hofmann_kernel_2008} is a well-known method to embed nonlinear data into a linear space and fit linear machine learning methods.
As a consequence, we propose in appendix (\autoref{app:TR}) a kernelization of the trace ratio problem~\eqref{eq:TR_Fl} in the same fashion as the one of the seminal graph embedding work~\citep{yan_graph_2007}.
This is expected to yield much better embedding and classification results.
\section{Spectral clustering: extensions and proofs}\label{app:SSC}


\subsection{Proof of Proposition~\ref{prop:SSC}}
Let $\Sf \in \Fl(p, \qf)$ and $U_{1:d+1} := [U_1|U_2|\dots|U_d|U_{d+1}] \in \O(p)$ be an orthogonal representative of $\Sf$. One has:
\begin{align}
	\langle \Pi_{\S_{1:d}}, L\rangle_F + \beta \norm{\Pi_{\S_{1:d}}}_1 &= \left\langle \frac1d\sum_{k=1}^d \Pi_{\S_k}, L\right\rangle_F + \beta \norm{\frac1d\sum_{k=1}^d \Pi_{\S_k}}_1,\\
	&= \frac1d \lrp{\left\langle \sum_{k=1}^{d+1} (d - (k-1)) U_k {U_k}\T, L\right\rangle_F + \beta \norm{\sum_{k=1}^{d+1} (d - (k-1)) U_k {U_k}\T }_1},\\
	&= \frac1d \lrp{\sum_{k=1}^{d+1} (d - (k-1)) \left\langle U_k {U_k}\T, L\right\rangle_F + \beta \norm{\sum_{k=1}^{d+1} (d - (k-1)) U_k {U_k}\T }_1},\\
	\langle \Pi_{\S_{1:d}}, L\rangle_F + \beta \norm{\Pi_{\S_{1:d}}}_1 &= \frac1d \lrp{\sum_{k=1}^{d+1} (d - (k-1)) \tr{{U_k}\T L U_k} + \beta \norm{\sum_{k=1}^{d+1} (d - (k-1)) U_k {U_k}\T }_1},
\end{align}
which concludes the proof.
\subsection{The flag trick for other machine learning problems}
Subspace learning finds many applications beyond robust subspace recovery, trace ratio and spectral clustering problems, as evoked in~\autoref{sec:intro}. The goal of this subsection is to provide a few more examples in brief, without experiments.


\subsubsection{Domain adaptation}
In machine learning, it is often assumed that the training and test datasets follow the same distribution. However, some \textit{domain shift} issues---where training and test distributions are different---might arise, notably if the test data has been acquired from a different source (for instance a professional camera and a phone camera) or if the training data has been acquired a long time ago. \textit{Domain adaptation} is an area of machine learning that deals with domain shifts, usually by matching the training and test distributions---often referred to as \textit{source} and \textit{target} distributions---before fitting a classical model~\citep{farahani_brief_2021}. 
A large body of works (called ``subspace-based'') learn some intermediary subspaces between the source and target data, and perform the inference for the projected target data on these subspaces. The \textit{sampling geodesic flow}~\citep{gopalan_domain_2011} first performs a geodesic interpolation on Grassmannians between the source and target subspaces, then projects both datasets on (a discrete subset of) the interpolated subspaces, which results in a new representation of the data distributions, that can then be given as an input to a machine learning model. The higher the number of intermediary subspaces, the better the approximation, but the larger the dimension of the representation.
The celebrated \textit{geodesic flow kernel}~\citep{boqing_gong_geodesic_2012} circumvents this issue by integrating the projected data onto the continuum of interpolated subspaces. This yields an inner product between infinite-dimensional embeddings that can be computed explicitly and incorporated in a kernel method for learning. The \textit{domain invariant projection}~\citep{baktashmotlagh_unsupervised_2013} learns a \textit{domain-invariant} subspace that minimizes the maximum mean discrepancy (MMD)~\citep{gretton_kernel_2012} between the projected source $X_s := [x_{s1}|\dots|x_{s n_s}] \in \R^{p\times n_s}$ and target distributions $X_t := [x_{t1}|\dots|x_{t n_t}] \in \R^{p\times n_t}$:
\begin{equation}
	\argmin{U \in \St(p, q)} \operatorname{MMD}^2(U\T X_{s}, U\T X_{t}),
\end{equation}
where 
\begin{equation}
	\operatorname{MMD} (X, Y) = \norm{\frac 1 n \sum_{i=1}^n \phi (x_i) - \frac 1 m \sum_{i=1}^m \phi (y_i)}_\mathcal{H}.
\end{equation}
This can be rewritten, using the Gaussian kernel function $\phi(x)\colon y \mapsto \exp\lrp{-\frac{x\T y}{2\sigma^2}}$, as
\begin{multline}\label{eq:DIP}
	\argmin{\S \in \Gr(p, q)} 
	\frac 1 {n_s^2} \sum_{i,j=1}^{n_s} \exp\lrp{-\frac{(x_{si} - x_{sj})\T \Pi_\S (x_{si} - x_{sj})}{2 \sigma^2}}\\
	+ \frac 1 {n_t^2} \sum_{i,j=1}^{n_t} \exp\lrp{-\frac{(x_{ti} - x_{tj})\T \Pi_\S (x_{ti} - x_{tj})}{2 \sigma^2}}\\
	- \frac 2 {n_s n_t} \sum_{i=1}^{n_s} \sum_{j=1}^{n_t} \exp\lrp{-\frac{(x_{si} - x_{tj})\T \Pi_\S (x_{si} - x_{tj})}{2 \sigma^2}}.
\end{multline}
The flag trick applied to the domain invariant projection problem~\eqref{eq:DIP} yields:
\begin{multline}
	\argmin{\S_{1:d} \in \Fl(p, q_{1:d})} 
	\frac 1 {n_s^2} \sum_{i,j=1}^{n_s} \exp\lrp{-\frac{(x_{si} - x_{sj})\T \Pi_{\S_{1:d}} (x_{si} - x_{sj})}{2 \sigma^2}}\\
	+ \frac 1 {n_t^2} \sum_{i,j=1}^{n_t} \exp\lrp{-\frac{(x_{ti} - x_{tj})\T \Pi_{\S_{1:d}} (x_{ti} - x_{tj})}{2 \sigma^2}}\\
	- \frac 2 {n_s n_t} \sum_{i=1}^{n_s} \sum_{j=1}^{n_t} \exp\lrp{-\frac{(x_{si} - x_{tj})\T \Pi_{\S_{1:d}} (x_{si} - x_{tj})}{2 \sigma^2}},
\end{multline}
and can be rewritten as:
\begin{multline}
	\argmin{U_{1:d} \in \St(p, q)}
	\frac 1 {{n_s}^2} \sum_{i,j=1}^{n_s} \exp\lrp{-\sum_{k=1}^d \frac{d+1-k}{d} \frac{\norm{{U_k}\T (x_{si} - x_{sj})}_2^2}{2 \sigma^2}}\\
	+ \frac 1 {{n_t}^2} \sum_{i,j=1}^{n_t} \exp\lrp{-\sum_{k=1}^d \frac{d+1-k}{d} \frac{\norm{{U_k}\T (x_{ti} - x_{tj})}_2^2}{2 \sigma^2}}\\
	- \frac 2 {{n_s} {n_t}} \sum_{i=1}^{n_s} \sum_{j=1}^{n_t} \exp\lrp{-\sum_{k=1}^d \frac{d+1-k}{d} \frac{\norm{{U_k}\T (x_{si} - x_{tj})}_2^2}{2 \sigma^2}}.
\end{multline}
Some experiments similar to the ones of~\citet{baktashmotlagh_unsupervised_2013} can be performed. For instance, one can consider the benchmark visual object recognition dataset of~\citet{saenko_adapting_2010}, learn nested domain invariant projections, fit some support vector machines to the projected source samples at increasing dimensions, and then perform soft-voting ensembling by learning the optimal weights on the target data according to Equation~\eqref{eq:soft_voting}.

\subsubsection{Low-rank decomposition}
Many machine learning methods involve finding low-rank representations of a data matrix. 

This is the case of \textit{matrix completion}~\citep{candes_exact_2012} problems where one looks for a low-rank representation of an incomplete data matrix by minimizing the discrepancy with the observed entries, and which finds many applications including the well-known \href{https://en.wikipedia.org/wiki/Netflix_Prize}{Netflix problem}. Although its most-known formulation is as a convex relaxation, it can also be formulated as an optimization problem on Grassmann manifolds~\citep{keshavan_matrix_2010,boumal_rtrmc_2011} to avoid optimizing the nuclear norm in the full space which can be of high dimension. The intuition is that a low-dimensional point can be described by the subspace it belongs to and its coordinates within this subspace. More precisely, the SVD-based low-rank factorization $M = UW$, with $M \in \R^{p \times n}$, $U \in \St(p, q)$ and $W \in \R^{q \times n}$ is orthogonally-invariant---in the sense that for any $R\in\O(q)$, one has $(UR) (R\T W) = U W$. One could therefore apply the flag trick to such problems, with the intuition that we would try low-rank matrix decompositions at different dimensions. The application of the flag trick would however not be as straightforward as in the previous problems since the subspace-projection matrices $\Pi_\S := U U\T$ do not appear explicitly, and since the coefficient matrix $W$ also depends on the dimension $q$.

Many other low-rank problems can be formulated as a Grassmannian optimization. \textit{Robust PCA}~\citep{candes_robust_2011} looks for a low rank + sparse corruption factorization of a data matrix. \textit{Subspace Tracking}~\citep{balzano_online_2010} incrementally updates a subspace from streaming and highly-incomplete observations via small steps on Grassmann manifolds.

\subsubsection{Linear dimensionality reduction}
Finally, many other general dimension reduction algorithms---referred to as \textit{linear dimensionality reduction methods}~\citep{cunningham_linear_2015}---involve optimization on Grassmannians. For instance, linear dimensionality reduction encompasses the already-discussed PCA and LDA, but also many other problems like \textit{multi-dimensional scaling}~\citep{torgerson_multidimensional_1952}, \textit{slow feature analysis}~\citep{wiskott_slow_2002}, \textit{locality preserving projections}~\citep{he_locality_2003} and \textit{factor analysis}~\citep{spearman_general_1904}.