\section{The flag trick in action}\label{sec:examples}
In this section, we provide some applications of the flag trick to several learning problems. We choose to focus on subspace recovery, trace ratio and spectral clustering problems. Other ones, like domain adaptation, matrix completion and subspace tracking are developed or mentioned in the last subsection but not experimented for conciseness.

\subsection{Outline and experimental setting}
For each application, we first present the learning problem as an optimization on Grassmannians. Second, we formulate the associated flag learning problem by applying the flag trick (Definition~\ref{def:flag_trick}). Third, we optimize the problem on flag manifolds with the steepest descent method (Algorithm~\ref{alg:GD})---more advanced algorithms are also derived in the appendix. 
Finally, we perform various nestedness and ensemble learning experiments via Algorithm~\ref{alg:flag_trick} on both synthetic and real datasets.

The general methodology to compare Grassmann-based methods to flag-based methods is the following one. For each experiment, we first choose a flag signature $~{q_{1\rightarrow d} := (q_1, \dots, q_d)}$, then we run independent optimization algorithms on $\Gr(p, q_1), \dots, \Gr(p, q_d)$~\eqref{eq:subspace_problem} and finally we compare the optimal subspaces $\S_k^* \in \Gr(p, q_k)$ to the optimal flag of subspaces $\Sf^* \in \Fl(p, q_{1\rightarrow d})$ obtained via the flag trick~\eqref{eq:flag_problem}. 
To show the nestedness issue in Grassmann-based methods, we compute the subspace distances $\Theta(\S_k^*, \S_{k+1}^*)_{k=1\dots d-1}$, where $\Theta$ is the generalized Grassmann distance of~\citet[Eq.~(14)]{ye_schubert_2016}. It consists in the $\ell_2$ norm of the principal angles, which can be obtained from the singular value decomposition (SVD) of the inner-products matrices ${U_k}\T {U_{k+1}}$, where $U_k \in \St(p, q_k)$ is an orthonormal basis of $\S_k^*$.

Regarding the implementation of the steepest descent algorithm on flag manifolds (Algorithm~\ref{alg:GD}), we develop a new class of manifolds in \href{https://pymanopt.org/}{PyManOpt}~\citep{boumal_manopt_2014,townsend_pymanopt_2016}, and run their \href{https://github.com/pymanopt/pymanopt/blob/master/src/pymanopt/optimizers/steepest_descent.py}{SteepestDescent} algorithm. Our implementation of the \texttt{Flag} class is based on the Stiefel representation of flag manifolds, detailed in \autoref{sec:flags}, with the retraction being the polar retraction. For the computation of the gradient, we use automatic differentiation with the \texttt{\href{https://github.com/HIPS/autograd}{autograd}} package. We could derive the gradients by hand from the expressions we get, but we use automatic differentiation as strongly suggested in PyManOpt's \href{https://pymanopt.org/docs/stable/quickstart.html}{documentation}.
Finally, the real datasets and the machine learning methods used in the experiments can be found in \href{https://scikit-learn.org/stable/}{scikit-learn}~\citep{pedregosa_scikit-learn_2011}.

\section{Robust subspace recovery: extensions and proofs}\label{app:RSR}

\subsection{An IRLS algorithm for robust subspace recovery}
Iteratively reweighted least squares (IRLS) is a ubiquitous method to solve optimization problems involving $L^p$-norms. Motivated by the computation of the geometric median~\citep{weiszfeld_sur_1937}, IRLS is highly used to find robust maximum likelihood estimates of non-Gaussian probabilistic models (typically those containing outliers) and finds application in robust regression~\citep{huber_robust_1964}, sparse recovery~\citep{daubechies_iteratively_2010} etc.

The recent fast median subspace (FMS) algorithm~\citep{lerman_fast_2018}, achieving state-of-the-art results in RSR uses an IRLS scheme to optimize the Least Absolute Deviation (LAD)~\eqref{eq:RSR_Gr}.
The idea is to first rewrite the LAD as 
\begin{equation}
	\sum_{i=1}^n \norm{x_i - \Pi_{\S} x_i}_2 = \sum_{i=1}^n w_i(\S) \norm{x_i - \Pi_{\S} x_i}_2^2,
\end{equation}
with $w_i(\S) = \frac{1}{\norm{x_i - \Pi_{\S} x_i}_2}$, and then successively compute the weights $w_i$ and update the subspace according to the weighted objective.
More precisely, the FMS algorithm creates a sequence of subspaces $\S^1, \dots, \S^m$ such that 
\begin{equation}\label{eq:IRLS_FMF}
    \S^{t+1} = \argmin{\S \in \Gr(p, q)} \sum_{i=1}^n w_i(\S^t) \norm{x_i - \Pi_\S x_i}_2^2.
\end{equation}
This weighted least-squares problem enjoys a closed-form solution which relates to the eigenvalue decomposition of the weighted covariance matrix $\sum_{i=1}^n w_i(\S^t) x_i {x_i}\T$~\citep[Chapter~3.3]{vidal_generalized_2016}.

We wish to derive an IRLS algorithm for the flag-tricked version of the LAD minimization problem~\eqref{eq:RSR_Fl}.
In order to stay close in mind to the recent work of \citet{peng_convergence_2023} who proved convergence of a general class of IRLS algorithms under some mild assumptions, we first rewrite~\eqref{eq:RSR_Fl} as
\begin{equation}~\label{eq:RSR_Fl_IRLS}
    \argmin{\S_{1:d} \in\Fl(p, \qf)} \sum_{i=1}^n \rho(r(\S_{1:d}, x_i)),
\end{equation}
where $r(\S_{1:d}, x) = \norm{x - \Pi_{\Sf} x}_2$ is the \textit{residual} and $\rho(r) = |r|$ is the \textit{outlier-robust} loss function.
Following~\citet{peng_convergence_2023}, the IRLS scheme associated with~\eqref{eq:RSR_Fl_IRLS} is:
\begin{equation}
\begin{cases}
w_i^{t+1} = \rho'(r(\S_{1:d}^t, x_i)) /  r(\S_{1:d}^t, x_i) = 1 / \norm{x_i - \Pi_{\Sf} x_i}_2,\\
(\S_{1:d})^{t+1} = \argmin{\S_{1:d} \in\Fl(p, \qf)} \sum_{i=1}^n w_i^{t+1} \norm{x_i - \Pi_{\Sf} x_i}_2^2.
\end{cases}
\end{equation}
We now show that the second step enjoys a closed-form solution.
\begin{theorem}\label{thm:IRLS_FMF}
The RLS problem
\begin{equation}
    \argmin{\S_{1:d} \in\Fl(p, \qf)} \sum_{i=1}^n w_i \norm{x_i - \Pi_{\Sf} x_i}_2^2
\end{equation}
has a closed-form solution $\S_{1:d}^* \in\Fl(p, \qf)$, which is given by the eigenvalue decomposition of the weighted sample covariance matrix $S_w = \sum_{i=1}^n w_i x_i {x_i}\T = \sum_{j=1}^p \ell_j v_j {v_j}\T$, i.e.
\begin{equation}
    \S_k^* = \operatorname{Span}(v_1, \dots, v_{q_k}) \quad (k=1\twodots d).
\end{equation}
\end{theorem}
\begin{proof}
One has
\begin{equation}
	\sum_{i=1}^n w_i \norm{x_i - \Pi_{\Sf} x_i}_2^2 = \tr{(I - \Pi_{\Sf})^2 \lrp{\sum_{i=1}^n w_i x_i {x_i}\T}}.
\end{equation}
Therefore, we are exactly in the same case as in \autoref{thm:flag_trick}, if we replace $X X\T$ with the reweighted covariance matrix $\sum_{i=1}^n w_i x_i {x_i}\T$. This does not change the result, so we conclude with the end of the proof of \autoref{thm:flag_trick} (which itself relies on~\citet{szwagier_curse_2024}).
\end{proof}
Hence, one gets an IRLS scheme for the LAD minimization problem. 
One can modify the robust loss function $\rho(r) = |r|$ by a Huber-like loss function to avoid weight explosion. Indeed, one can show that the weight $w_i := 1 / \norm{x_i - \Pi_{\Sf} x_i}_2$ goes to infinity when the first subspace $\S_1$ of the flag gets close to $x_i$ .
Therefore in practice, we take 
\begin{equation}
    \rho(r) = 
        \begin{cases}
            r^2 / (2 p \delta) & \text{if } |r| <= p\delta,\\
            r - p \delta / 2 & \text{if } |r| > p\delta.
        \end{cases}
\end{equation}
This yields
\begin{equation}
    w_i = 1 / \max\lrp{p\delta,  1 / \norm{x_i - \Pi_{\Sf} x_i}_2}.
\end{equation}
The final proposed scheme is given in Algorithm~\ref{alg:FMF}, named \textit{fast median flag} (FMF), in reference to the fast median subspace algorithm of~\citet{lerman_fast_2018}.
\begin{algorithm}
\caption{Fast median flag}\label{alg:FMF}
\begin{algorithmic}
\Require $X\in \R^{p\times n}$ (data), $\quad q_1 < \dots < q_d$ (signature), $\quad t_{max}$ (max number of iterations), $\quad \eta$ (convergence threshold), $\quad \varepsilon$ (Huber-like saturation parameter)
\Ensure
$U \in \St(p, q)$
\State $t \gets 0, \quad \Delta \gets \infty, \quad U^0 \gets \operatorname{SVD}(X, q)$
\While{$\Delta > \eta$ and $t < t_{max}$}
    \State $t \gets t+1$
    \State $r_i \gets \norm{x_i - \Pi_{\Sf} x_i}_2$
    \State $y_i \gets {x_i} / {\max(\sqrt{r_i}, \varepsilon)}$
    \State $U^t \gets \operatorname{SVD(Y, q)}$
    \State $\Delta \gets \sqrt{\sum_{k=1}^{d} \Theta(U^t_{q_k}, U^{t-1}_{q_k})^2}$
\EndWhile
\end{algorithmic}
\end{algorithm}
We can easily check that FMF is a direct generalization of FMS for Grassmannians (i.e. when $d=1$).


\begin{remark}
This is far beyond the scope of the paper, but we believe that the convergence result of~\citet[Theorem~1]{peng_convergence_2023} could be generalized to the FMF algorithm, due to the compactness of flag manifolds and the expression of the residual function $r$.
\end{remark}

\subsection{Proof of Proposition~\ref{prop:RSR}}
Let $\Sf \in \Fl(p, \qf)$ and $U_{1:d+1} := [U_1|U_2|\dots|U_d|U_{d+1}] \in \O(p)$ be an orthogonal representative of $\Sf$. One has:
\begin{align}
	\norm{x_i - \Pi_{\S_{1:d}} x_i}_2 &= \sqrt{{(x_i - \Pi_{\S_{1:d}} x_i)}\T (x_i - \Pi_{\S_{1:d}} x_i)},\\
	 &= \sqrt{{x_i}\T {(I_p - \Pi_{\S_{1:d}})}^2 x_i},\\
	 &= \sqrt{{x_i}\T {\lrp{I_p - \frac1d \sum_{k=1}^d\Pi_{\S_k}}}^2 x_i},\\
 	 &= \sqrt{\frac1{d^2} {x_i}\T {\lrp{\sum_{k=1}^d (I_p - \Pi_{\S_k})}}^2 x_i},\\
 	 % &= \sqrt{\frac1{d^2} {x_i}\T U_{1:d+1} \diag{0, 1, \dots, d-1, d}^2 {U_{1:d+1}}\T  x_i},\\
 	 &= \sqrt{\frac1{d^2} {x_i}\T \lrp{\sum_{k=1}^{d+1} (k-1) U_k {U_k}\T}^2  x_i},\\
 	 &= \sqrt{\frac1{d^2} {x_i}\T \lrp{\sum_{k=1}^{d+1} (k-1)^2 U_k {U_k}\T}  x_i},\\
 	 &= \sqrt{\sum_{k=1}^{d+1} \lrp{\frac {k-1} {d}}^2 {x_i}\T \lrp{ U_k {U_k}\T}  x_i},\\
  	 \norm{x_i - \Pi_{\S_{1:d}} x_i}_2 &= \sqrt{\sum_{k=1}^{d+1} \lrp{\frac {k-1} {d}}^2 \norm{{U_k}\T x_i}_2^2},
\end{align}
which concludes the proof.
\subsection{The flag trick for trace ratio problems}\label{subsec:TR}
Trace ratio problems are ubiquitous in machine learning~\citep{ngo_trace_2012}. They write as:
\begin{equation}\label{eq:TR_St}
\argmax{U \in \St(p, q)} \frac{\tr{U\T A U}}{\tr{U\T B U}},
\end{equation}
where $A, B \in \R^{p\times p}$ are positive semi-definite matrices, with $\operatorname{rank}(B) > p - q$.

A famous example of TR problem is Fisher's linear discriminant analysis (LDA)~\citep{fisher_use_1936,belhumeur_eigenfaces_1997}.
It is common in machine learning to project the data onto a low-dimensional subspace before fitting a classifier, in order to circumvent the curse of dimensionality. It is well known that performing an unsupervised dimension reduction method like PCA comes with the risks of mixing up the classes, since the directions of maximal variance are not necessarily the most discriminating ones~\citep{chang_using_1983}. The goal of LDA is to use the knowledge of the data labels to learn a linear subspace that does not mix the classes.
Let $~{X := [x_1|\dots|x_n] \in \R^{p\times n}}$ be a dataset with labels $Y := [y_1|\dots|y_n] \in {[1, C]}^n$. Let $\mu = \frac{1}{n} \sum_{i=1}^n x_i$ be the dataset mean and $\mu_c = \frac{1}{\#\{i : y_i=c\}}\sum_{i : y_i=c} x_i$ be the class-wise means. 
The idea of LDA is to search for a subspace $\S \in \Gr(p, q)$ that simultaneously maximizes the projected \textit{between-class variance} $\sum_{c=1}^C \|\Pi_\S \mu_c - \Pi_\S \mu\|_2^2$ and minimizes the projected \textit{within-class variance} $\sum_{c=1}^C \sum_{i : y_i = c} \|\Pi_\S x_i - \Pi_\S \mu_c\|_2^2$. This can be reformulated as a trace ratio problem~\eqref{eq:TR_St}, with $A = \sum_{c=1}^C (\mu_c - \mu) (\mu_c - \mu)\T$ and $B = \sum_{c=1}^C \sum_{i : y_i = c} (x_i - \mu_c) (x_i - \mu_c)\T$.


More generally, a large family of dimension reduction methods can be reformulated as a TR problem. The seminal work of~\citet{yan_graph_2007} shows that many dimension reduction and manifold learning objective functions can be written as a trace ratio involving Laplacian matrices of attraction and repulsion graphs. Intuitively, those graphs determine which points should be close in the latent space and which ones should be far apart.
Other methods involving a ratio of traces are \textit{multi-view learning}~\citep{wang_trace_2023}, \textit{partial least squares} (PLS)~\citep{geladi_partial_1986,barker_partial_2003} and \textit{canonical correlation analysis} (CCA)~\citep{hardoon_canonical_2004}, although these methods are originally \textit{sequential} problems (cf. footnote~\ref{footnote:sequential}) and not \textit{subspace} problems.

Classical Newton-like algorithms for solving the TR problem~\eqref{eq:TR_St} come from the seminal works of~\citet{guo_generalized_2003, wang_trace_2007, jia_trace_2009}.
The interest of optimizing a trace-ratio instead of a ratio-trace (of the form $\tr{(U\T B U)^{-1}(U\T A U)}$), that enjoys an explicit solution given by a generalized eigenvalue decomposition, is also tackled in those papers. The \textit{repulsion Laplaceans}~\citep{kokiopoulou_enhanced_2009} instead propose to solve a regularized version $\tr{U\T B U} - \rho \tr{U\T A U}$, which enjoys a closed-form, but has a hyperparameter $\rho$, which is directly optimized in the classical Newton-like algorithms for trace ratio problems.

\subsubsection{Application of the flag trick to trace ratio problems}
The trace ratio problem~\eqref{eq:TR_St} can be straightforwardly reformulated as an optimization problem on Grassmannians, due to the orthogonal invariance of the objective function:
\begin{equation}\label{eq:TR_Gr}
\argmax{\S \in \Gr(p, q)} \frac{\tr{\Pi_\S A}}{\tr{\Pi_\S B}}.
\end{equation}
The following proposition applies the flag trick to the TR problem~\eqref{eq:TR_Gr}.
\begin{proposition}[Flag trick for TR]\label{prop:TR}
The flag trick applied to the TR problem~\eqref{eq:TR_Gr} reads
\begin{equation}\label{eq:TR_Fl}
	\argmax{\S_{1:d} \in \Fl(p, q_{1:d})} \frac{\tr{\Pi_{\S_{1:d}} A}}{\tr{\Pi_{\S_{1:d}} B}}.
\end{equation}
and is equivalent to the following optimization problem:
\begin{equation}\label{eq:TR_Fl_equiv}
\argmax{U_{1:d} \in \St(p, q)} \frac{\sum_{k=1}^{d} (d - (k-1)) \tr{{U_k}\T A {U_k}}}{\sum_{l=1}^{d} (d - (l-1)) \tr{{U_{l}}\T B {U_{l}}}}.
\end{equation}
\end{proposition}
\begin{proof}
The proof is given in Appendix (\autoref{app:TR}).
\end{proof}
Equation~\eqref{eq:TR_Fl_equiv} tells us several things. First, the subspaces $~{\operatorname{Span}(U_1) \perp \dots \perp \operatorname{Span}(U_d)}$ are weighted decreasingly, which means that they have less and less importance with respect to the TR objective.
Second, we can see that the nested trace ratio problem~\eqref{eq:TR_Fl} somewhat maximizes the numerator $\tr{\Pi_{\S_{1:d}} A}$ while minimizing the denominator $\tr{\Pi_{\S_{1:d}} B}$. Both subproblems have an explicit solution corresponding to our nested PCA Theorem~\ref{thm:flag_trick}. Hence, one can naturally initialize the steepest descent algorithm with the $q$ highest eigenvalues of $A$ or the $q$ lowest eigenvalues of $B$ depending on the application.
For instance, for LDA, initializing Algorithm~\ref{alg:GD} with the highest eigenvalues of $A$ would spread the classes far apart, while initializing it with the lowest eigenvalues of $B$ would concentrate the classes, which seems less desirable since we do not want the classes to concentrate at the same point.

\subsubsection{Nestedness experiments for trace ratio problems}
For all the experiments of this subsection, we consider the particular TR problem of LDA, although many other applications (\textit{marginal Fisher analysis}~\citep{yan_graph_2007}, \textit{local discriminant embedding}~\citep{chen_local_2005} etc.) could be investigated similarly.

First, we consider a synthetic dataset with five clusters.
The ambient dimension is $p = 3$ and the intrinsic dimensions that we try are $q_{1:2} = (1, 2)$.
We adopt a preprocessing strategy similar to~\citet{ngo_trace_2012}: we first center the data, then run a PCA to reduce the dimension to $n - C$ (if $n - C < p$), then construct the LDA scatter matrices $A$ and $B$, then add a diagonal covariance regularization of $10^{-5}$ times their trace and finally normalize them to have unit trace.
We run Algorithm~\ref{alg:GD} on Grassmann manifolds to solve the TR maximization problem~\eqref{eq:TR_Gr}, successively for $q_1 = 1$ and $q_2 = 2$. Then we plot the projections of the data points onto the optimal subspaces. We compare them to the nested projections onto the optimal flag output by running Algorithm~\ref{alg:GD} on $\Fl(3, (1, 2))$ to solve~\eqref{eq:TR_Fl}. The results are shown in Figure~\ref{fig:TR_nested}.
\begin{figure}
	\centering
    \includegraphics[width=.9\linewidth]{Fig/FT_exp_TR_synthetic.pdf}
    \caption{
    Illustration of the nestedness issue in linear discriminant analysis (trace ratio problem). Given a dataset with five clusters, we plot its projection onto the optimal 1D subspace and 2D subspace obtained by solving the associated Grassmannian optimization problem~\eqref{eq:TR_Gr} or flag optimization problem~\eqref{eq:TR_Fl}. 
    We can see that the Grassmann representations are not nested, while the flag representations are nested and well capture the distribution of clusters. In this example, it is less the nestedness than the \textit{rotation} of the optimal axes inside the 2D subspace that is critical to the analysis of the Grassmann-based method.
    }
	\label{fig:TR_nested}
\end{figure}
\begin{figure}
	\centering
    \includegraphics[width=.9\linewidth]{Fig/FT_exp_TR_digits.pdf}
    \caption{
    Illustration of the nestedness issue in linear discriminant analysis (trace ratio problem) on the digits dataset. For $q_k \in \qf := (1, 2, \dots, 63)$, we solve the Grassmannian optimization problem~\eqref{eq:TR_Gr} on $\Gr(64, q_k)$ and plot the subspace angles $\Theta(\S_k^*, \S_{k+1}^*)$ (left) and explained variances ${\operatorname{tr}(\Pi_{\S_k^*} X X\T)} / {\operatorname{tr}(X X\T)}$ (right) as a function of $k$. We compare those quantities to the ones obtained by solving the flag optimization problem~\eqref{eq:TR_Fl}. 
    We can see that the Grassmann-based method is highly non-nested and even yields an extremely paradoxical non-increasing explained variance (cf. red circle on the right).
    }
	\label{fig:TR_nested_digits}
\end{figure}
We can see that the Grassmann representations are non-nested while their flag counterparts perfectly capture the filtration of subspaces that best and best approximates the distribution while discriminating the classes. Even if the colors make us realize that the issue in this experiment for LDA  is not much about the non-nestedness but rather about the rotation of the principal axes within the 2D subspace, we still have an important issue of consistency.

Second, we consider the (full) handwritten digits dataset~\citep{alpaydin_optical_1998}. It contains $8 \times 8$ pixels images of handwritten digits, from $0$ to $9$, almost uniformly class-balanced. One has $n = 1797$, $p=64$ and $C = 10$.
We run a steepest descent algorithm to solve the trace ratio problem~\eqref{eq:TR_Fl}. We choose the full signature $q_{1:63} = (1, 2, \dots, 63)$ and compare the output flag to the individual subspaces output by running optimization on $\Gr(p, q_k)$ for $q_k \in q_{1:d}$.
We plot the subspace angles $\Theta(\S_k^*, \S_{k+1}^*)$ and the explained variance ${\operatorname{tr}(\Pi_{\S_k^*} X X\T)} / {\operatorname{tr}(X X\T)}$ as a function of the $k$. The results are illustrated in \autoref{fig:TR_nested_digits}.
We see that the subspace angles are always positive and even very large sometimes with the LDA. Worst, the explained variance is not monotonous. This implies that we sometimes \textit{loose} some information when \textit{increasing} the dimension, which is extremely paradoxical.

Third, we perform some classification experiments on the optimal subspaces for different datasets. For different datasets, we run the optimization problems on $\Fl(p, q_{1:d})$, then project the data onto the different subspaces in $\S_{1:d}^*$ and run a nearest neighbors classifier with $5$ neighbors.
The predictions are then ensembled (cf. Algorithm~\ref{alg:flag_trick}) by weighted averaging, either with uniform weights or with weights minimizing the average cross-entropy:
\begin{equation}\label{eq:soft_voting}
	w_1^*, \dots, w_d^* = \argmin{\substack{w_k \geq 0 \\ \sum_{k=1}^d w_k = 1}} - \frac 1 {n C} \sum_{i=1}^n \sum_{c=1}^C y_{ic} \ln\lrp{\sum_{k=1}^d w_k y_{kic}^*},
\end{equation}
where $y_{kic}^* \in [0, 1]$ is the predicted probability that $x_i \in \R^p$ belongs to class $c \in \{1 \dots C\}$, by the classifier $g_k^*$ that is trained on $Z_k := {U_k^*}\T X \in \R^{q_k \times n}$. One can show that the latter is a convex problem, which we optimize using the \href{https://www.cvxpy.org/index.html}{cvxpy} Python package~\citep{diamond2016cvxpy}.
We report the results in \autoref{tab:TR_classif}.
\begin{table}
  \caption{Classification results for the TR problem on real datasets. For each method (Gr: Grassmann optimization~\eqref{eq:TR_Gr}, Fl: flag optimization~\eqref{eq:TR_Gr}, Fl-U: flag optimization + uniform soft voting, Fl-W: flag optimization + optimal soft voting~\eqref{eq:soft_voting}), we give the cross-entropy of the projected-predictions with respect to the true labels.}
  \label{tab:TR_classif}
  \centering
  \begin{tabular}{ccccccccc}
    \toprule
    dataset & $n$ & $p$ & $q_{1:d}$ & Gr & Fl & Fl-U & Fl-W & weights\\
    \midrule
    breast & $569$ & $30$ & $(1, 2, 5)$ & $0.0986$ & $0.0978$ & $0.0942$ & $0.0915$ & $(0.754, 0, 0.246)$\\
    iris & $150$ & $4$ & $(1, 2, 3)$ & $0.0372$ & $0.0441$ & $0.0410$ & $0.0368$ & $(0.985, 0, 0.015)$\\
    wine & $178$ & $13$ & $(1, 2, 5)$ & $0.0897$ & $0.0800$ & $0.1503$ & $0.0677$ & $(0, 1, 0)$\\
    digits & $1797$ & $64$ & $(1, 2, 5, 10)$ & $0.4507$ & $0.4419$ & $0.5645$ & $0.4374$ & $(0, 0, 0.239, 0.761)$\\
    \bottomrule
  \end{tabular}
\end{table}
The first example tells us that the optimal $5D$ subspace obtained by Grassmann optimization less discriminates the classes than the $5D$ subspace from the optimal flag. This may show that the flag takes into account some lower dimensional variability that enables to better discriminate the classes. We can also see that the uniform averaging of the predictors at different dimensions improves the classification. Finally, the optimal weights improve even more the classification and tell that the best discrimination happens by taking a soft blend of classifier at dimensions $1$ and $5$. Similar kinds of analyses can be made for the other examples.

\subsubsection{Discussion on TR optimization and kernelization}
\paragraph{A Newton algorithm}
In all the experiments of this paper, we use a steepest descent method on flag manifolds (Algorithm~\ref{alg:GD}) to solve the flag problems.
However, for the specific problem of TR~\eqref{eq:TR_Fl}, we believe that more adapted algorithms should be derived to take into account the specific form of the objective function, which is classically solved via a Newton-Lanczos method~\citep{ngo_trace_2012}. 
To that extent, we develop in the appendix (\autoref{app:TR}) an extension of the baseline Newton-Lanczos algorithm for the flag-tricked problem~\eqref{eq:TR_Fl}.
In short, the latter can be reformulated as a penalized optimization problem of the form $\operatorname{argmax}_{\Sf\in\Fl(p, \qf)} {\sum_{k=1}^d \tr{\Pi_{\S_k} (A - \rho B)}}$, where $\rho$ is updated iteratively according to a Newton scheme. Once again, our central Theorem~\ref{thm:flag_trick} enables to get explicit expressions for the iterations, which results without much difficulties in a Newton method, that is known to be much more efficient than first-order methods like the steepest descent.

\paragraph{A non-linearization via the kernel trick}
The classical trace ratio problems look for \textit{linear} embeddings of the data.
However, in most cases, the data follow a \textit{nonlinear} distribution, which may cause linear dimension reduction methods to fail. The \textit{kernel trick}~\citep{hofmann_kernel_2008} is a well-known method to embed nonlinear data into a linear space and fit linear machine learning methods.
As a consequence, we propose in appendix (\autoref{app:TR}) a kernelization of the trace ratio problem~\eqref{eq:TR_Fl} in the same fashion as the one of the seminal graph embedding work~\citep{yan_graph_2007}.
This is expected to yield much better embedding and classification results.
\section{Spectral clustering: extensions and proofs}\label{app:SSC}


\subsection{Proof of Proposition~\ref{prop:SSC}}
Let $\Sf \in \Fl(p, \qf)$ and $U_{1:d+1} := [U_1|U_2|\dots|U_d|U_{d+1}] \in \O(p)$ be an orthogonal representative of $\Sf$. One has:
\begin{align}
	\langle \Pi_{\S_{1:d}}, L\rangle_F + \beta \norm{\Pi_{\S_{1:d}}}_1 &= \left\langle \frac1d\sum_{k=1}^d \Pi_{\S_k}, L\right\rangle_F + \beta \norm{\frac1d\sum_{k=1}^d \Pi_{\S_k}}_1,\\
	&= \frac1d \lrp{\left\langle \sum_{k=1}^{d+1} (d - (k-1)) U_k {U_k}\T, L\right\rangle_F + \beta \norm{\sum_{k=1}^{d+1} (d - (k-1)) U_k {U_k}\T }_1},\\
	&= \frac1d \lrp{\sum_{k=1}^{d+1} (d - (k-1)) \left\langle U_k {U_k}\T, L\right\rangle_F + \beta \norm{\sum_{k=1}^{d+1} (d - (k-1)) U_k {U_k}\T }_1},\\
	\langle \Pi_{\S_{1:d}}, L\rangle_F + \beta \norm{\Pi_{\S_{1:d}}}_1 &= \frac1d \lrp{\sum_{k=1}^{d+1} (d - (k-1)) \tr{{U_k}\T L U_k} + \beta \norm{\sum_{k=1}^{d+1} (d - (k-1)) U_k {U_k}\T }_1},
\end{align}
which concludes the proof.
\subsection{The flag trick for other machine learning problems}
Subspace learning finds many applications beyond robust subspace recovery, trace ratio and spectral clustering problems, as evoked in~\autoref{sec:intro}. The goal of this subsection is to provide a few more examples in brief, without experiments.


\subsubsection{Domain adaptation}
In machine learning, it is often assumed that the training and test datasets follow the same distribution. However, some \textit{domain shift} issues---where training and test distributions are different---might arise, notably if the test data has been acquired from a different source (for instance a professional camera and a phone camera) or if the training data has been acquired a long time ago. \textit{Domain adaptation} is an area of machine learning that deals with domain shifts, usually by matching the training and test distributions---often referred to as \textit{source} and \textit{target} distributions---before fitting a classical model~\citep{farahani_brief_2021}. 
A large body of works (called ``subspace-based'') learn some intermediary subspaces between the source and target data, and perform the inference for the projected target data on these subspaces. The \textit{sampling geodesic flow}~\citep{gopalan_domain_2011} first performs a geodesic interpolation on Grassmannians between the source and target subspaces, then projects both datasets on (a discrete subset of) the interpolated subspaces, which results in a new representation of the data distributions, that can then be given as an input to a machine learning model. The higher the number of intermediary subspaces, the better the approximation, but the larger the dimension of the representation.
The celebrated \textit{geodesic flow kernel}~\citep{boqing_gong_geodesic_2012} circumvents this issue by integrating the projected data onto the continuum of interpolated subspaces. This yields an inner product between infinite-dimensional embeddings that can be computed explicitly and incorporated in a kernel method for learning. The \textit{domain invariant projection}~\citep{baktashmotlagh_unsupervised_2013} learns a \textit{domain-invariant} subspace that minimizes the maximum mean discrepancy (MMD)~\citep{gretton_kernel_2012} between the projected source $X_s := [x_{s1}|\dots|x_{s n_s}] \in \R^{p\times n_s}$ and target distributions $X_t := [x_{t1}|\dots|x_{t n_t}] \in \R^{p\times n_t}$:
\begin{equation}
	\argmin{U \in \St(p, q)} \operatorname{MMD}^2(U\T X_{s}, U\T X_{t}),
\end{equation}
where 
\begin{equation}
	\operatorname{MMD} (X, Y) = \norm{\frac 1 n \sum_{i=1}^n \phi (x_i) - \frac 1 m \sum_{i=1}^m \phi (y_i)}_\mathcal{H}.
\end{equation}
This can be rewritten, using the Gaussian kernel function $\phi(x)\colon y \mapsto \exp\lrp{-\frac{x\T y}{2\sigma^2}}$, as
\begin{multline}\label{eq:DIP}
	\argmin{\S \in \Gr(p, q)} 
	\frac 1 {n_s^2} \sum_{i,j=1}^{n_s} \exp\lrp{-\frac{(x_{si} - x_{sj})\T \Pi_\S (x_{si} - x_{sj})}{2 \sigma^2}}\\
	+ \frac 1 {n_t^2} \sum_{i,j=1}^{n_t} \exp\lrp{-\frac{(x_{ti} - x_{tj})\T \Pi_\S (x_{ti} - x_{tj})}{2 \sigma^2}}\\
	- \frac 2 {n_s n_t} \sum_{i=1}^{n_s} \sum_{j=1}^{n_t} \exp\lrp{-\frac{(x_{si} - x_{tj})\T \Pi_\S (x_{si} - x_{tj})}{2 \sigma^2}}.
\end{multline}
The flag trick applied to the domain invariant projection problem~\eqref{eq:DIP} yields:
\begin{multline}
	\argmin{\S_{1:d} \in \Fl(p, q_{1:d})} 
	\frac 1 {n_s^2} \sum_{i,j=1}^{n_s} \exp\lrp{-\frac{(x_{si} - x_{sj})\T \Pi_{\S_{1:d}} (x_{si} - x_{sj})}{2 \sigma^2}}\\
	+ \frac 1 {n_t^2} \sum_{i,j=1}^{n_t} \exp\lrp{-\frac{(x_{ti} - x_{tj})\T \Pi_{\S_{1:d}} (x_{ti} - x_{tj})}{2 \sigma^2}}\\
	- \frac 2 {n_s n_t} \sum_{i=1}^{n_s} \sum_{j=1}^{n_t} \exp\lrp{-\frac{(x_{si} - x_{tj})\T \Pi_{\S_{1:d}} (x_{si} - x_{tj})}{2 \sigma^2}},
\end{multline}
and can be rewritten as:
\begin{multline}
	\argmin{U_{1:d} \in \St(p, q)}
	\frac 1 {{n_s}^2} \sum_{i,j=1}^{n_s} \exp\lrp{-\sum_{k=1}^d \frac{d+1-k}{d} \frac{\norm{{U_k}\T (x_{si} - x_{sj})}_2^2}{2 \sigma^2}}\\
	+ \frac 1 {{n_t}^2} \sum_{i,j=1}^{n_t} \exp\lrp{-\sum_{k=1}^d \frac{d+1-k}{d} \frac{\norm{{U_k}\T (x_{ti} - x_{tj})}_2^2}{2 \sigma^2}}\\
	- \frac 2 {{n_s} {n_t}} \sum_{i=1}^{n_s} \sum_{j=1}^{n_t} \exp\lrp{-\sum_{k=1}^d \frac{d+1-k}{d} \frac{\norm{{U_k}\T (x_{si} - x_{tj})}_2^2}{2 \sigma^2}}.
\end{multline}
Some experiments similar to the ones of~\citet{baktashmotlagh_unsupervised_2013} can be performed. For instance, one can consider the benchmark visual object recognition dataset of~\citet{saenko_adapting_2010}, learn nested domain invariant projections, fit some support vector machines to the projected source samples at increasing dimensions, and then perform soft-voting ensembling by learning the optimal weights on the target data according to Equation~\eqref{eq:soft_voting}.

\subsubsection{Low-rank decomposition}
Many machine learning methods involve finding low-rank representations of a data matrix. 

This is the case of \textit{matrix completion}~\citep{candes_exact_2012} problems where one looks for a low-rank representation of an incomplete data matrix by minimizing the discrepancy with the observed entries, and which finds many applications including the well-known \href{https://en.wikipedia.org/wiki/Netflix_Prize}{Netflix problem}. Although its most-known formulation is as a convex relaxation, it can also be formulated as an optimization problem on Grassmann manifolds~\citep{keshavan_matrix_2010,boumal_rtrmc_2011} to avoid optimizing the nuclear norm in the full space which can be of high dimension. The intuition is that a low-dimensional point can be described by the subspace it belongs to and its coordinates within this subspace. More precisely, the SVD-based low-rank factorization $M = UW$, with $M \in \R^{p \times n}$, $U \in \St(p, q)$ and $W \in \R^{q \times n}$ is orthogonally-invariant---in the sense that for any $R\in\O(q)$, one has $(UR) (R\T W) = U W$. One could therefore apply the flag trick to such problems, with the intuition that we would try low-rank matrix decompositions at different dimensions. The application of the flag trick would however not be as straightforward as in the previous problems since the subspace-projection matrices $\Pi_\S := U U\T$ do not appear explicitly, and since the coefficient matrix $W$ also depends on the dimension $q$.

Many other low-rank problems can be formulated as a Grassmannian optimization. \textit{Robust PCA}~\citep{candes_robust_2011} looks for a low rank + sparse corruption factorization of a data matrix. \textit{Subspace Tracking}~\citep{balzano_online_2010} incrementally updates a subspace from streaming and highly-incomplete observations via small steps on Grassmann manifolds.

\subsubsection{Linear dimensionality reduction}
Finally, many other general dimension reduction algorithms---referred to as \textit{linear dimensionality reduction methods}~\citep{cunningham_linear_2015}---involve optimization on Grassmannians. For instance, linear dimensionality reduction encompasses the already-discussed PCA and LDA, but also many other problems like \textit{multi-dimensional scaling}~\citep{torgerson_multidimensional_1952}, \textit{slow feature analysis}~\citep{wiskott_slow_2002}, \textit{locality preserving projections}~\citep{he_locality_2003} and \textit{factor analysis}~\citep{spearman_general_1904}.