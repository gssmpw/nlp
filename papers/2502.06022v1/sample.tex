\documentclass[twoside,11pt]{article}

\usepackage{blindtext}

% Any additional packages needed should be included after jmlr2e.
% Note that jmlr2e.sty includes epsfig, amssymb, natbib and graphicx,
% and defines many common macros, such as 'proof' and 'example'.
%
% It also sets the bibliographystyle to plainnat; for more information on
% natbib citation styles, see the natbib documentation, a copy of which
% is archived at http://www.jmlr.org/format/natbib.pdf

% Available options for package jmlr2e are:
%
%   - abbrvbib : use abbrvnat for the bibliography style
%   - nohyperref : do not load the hyperref package
%   - preprint : remove JMLR specific information from the template,
%         useful for example for posting to preprint servers.
%
% Example of using the package with custom options:
%
% \usepackage[abbrvbib, preprint]{jmlr2e}

\usepackage[preprint]{jmlr2e}

\usepackage{amsfonts, amsmath, amssymb}
\usepackage{algorithm, algorithmicx, algpseudocode}
\usepackage{booktabs}

% Definitions of handy macros can go here

\newcommand{\dataset}{{\cal D}}
\newcommand{\fracpartial}[2]{\frac{\partial #1}{\partial  #2}}
\newcommand{\R}{\mathbb{R}}
\newcommand{\Sym}{\operatorname{Sym}}
\newcommand{\Skew}{\operatorname{Skew}}
\renewcommand{\O}{\mathcal{O}}
\renewcommand{\S}{\mathcal{S}}
\newcommand{\V}{\mathcal{V}}
\newcommand{\Sf}{\mathcal{S}_{1:d}}
\newcommand{\Uf}{U_{1:d}}
\newcommand{\qf}{q_{1:d}}
\newcommand{\SO}{\mathcal{SO}}
\newcommand{\diag}[1]{\operatorname{diag}\left(#1\right)}
\newcommand{\tr}[1]{\operatorname{tr}\left(#1\right)}
\newcommand{\norm}[1]{\left\lVert#1\right\rVert}
\newcommand{\lrp}[1]{\left(#1\right)}
\newcommand{\lrb}[1]{\left[#1\right]}
\newcommand{\lrs}[1]{\left\{#1\right\}}
\newcommand{\twodots}{\mathinner {\ldotp \ldotp}}
\newcommand{\St}{\operatorname{St}}
\newcommand{\Gr}{\operatorname{Gr}}
\newcommand{\Fl}{\operatorname{Fl}}
\newcommand{\T}{^\top}
\newcommand{\ambientstiefel}{\R^{n \times k}}
\newcommand{\argmin}[1]{\underset{#1}{\operatorname{argmin}} \, }
\newcommand{\argmax}[1]{\underset{#1}{\operatorname{argmax}} \, }
\newcommand{\N}[1]{\mathcal{N}\left(#1\right)}

% Heading arguments are {volume}{year}{pages}{date submitted}{date published}{paper id}{author-full-names}

\usepackage{lastpage}
\jmlrheading{23}{2024}{1-\pageref{LastPage}}{}{}{}{Tom Szwagier and Xavier Pennec}

% Short headings should be running head and authors last names

\ShortHeadings{Nested subspace learning with flags}{Szwagier and Pennec}
\firstpageno{1}

\begin{document}

\title{Nested subspace learning with flags}

\author{\name Tom Szwagier\email tom.szwagier@inria.fr \\
       \addr Université Côte d'Azur and Inria\\
       Sophia Antipolis, France
       \AND
       \name Xavier Pennec \email xavier.pennec@inria.fr \\
       \addr Université Côte d'Azur and Inria\\
       Sophia Antipolis, France}

\editor{}

\maketitle

\begin{abstract}%   <- trailing '%' for backward compatibility of .sty file
Many machine learning methods look for low-dimensional representations of the data. The underlying subspace can be estimated by {first} choosing a dimension $q$ and {then} optimizing a certain objective function over the space of $q$-dimensional subspaces---the Grassmannian. Trying different $q$ yields in general non-nested subspaces, which raises an important issue of consistency between the data representations. In this paper, we propose a simple trick to enforce nestedness in subspace learning methods. It consists in lifting Grassmannian optimization problems to flag manifolds---the space of nested subspaces of increasing dimension---via nested projectors. We apply the flag trick to several classical machine learning methods and show that it successfully addresses the nestedness issue.
\end{abstract}

\begin{keywords}
  subspace learning, Grassmann manifolds, flag manifolds, nested subspaces
\end{keywords}

\section{Introduction}
Implicit Neural Representations (INRs), which fit the target function using only input coordinates, have recently gained significant attention.
%
By leveraging the powerful fitting capability of Multilayer Perceptrons (MLPs), INRs can implicitly represent the target function without requiring their analytical expressions. 
%
The versatility of MLPs allows INRs to be applied in various fields, including inverse graphics~\citep{mildenhall2021nerf, barron2023zip, martin2021nerf}, image super-resolution~\citep{chen2021learning, yuan2022sobolev, gao2023implicit}, 
image generation~\citep{skorokhodov2021adversarial}, and more~\citep{chen2021nerv, strumpler2022implicit, shue20233d}.
%
\begin{figure}
    \includegraphics[width=0.5\textwidth]{Image/Fig2.pdf}
    \caption{As illustrated at the circled blue regions and green regions, it can be observed that even with well-chosen standard deviation/scale, as experimented in \autoref{figure:combined}, the results are still unsatisfactory. However, using our proposed method, the noise is significantly alleviated while further enhancing the high-frequency details.}
    \label{fig:var}
    \vspace{-10pt}
\end{figure}

\begin{figure*}[!ht]
    \centering
    \begin{minipage}[b]{0.25\textwidth}
        \centering
        \includegraphics[width=1.\textwidth]{Image/fig_cropped.pdf} % 替换为你的小图文件
        \label{figure:small_image}
        \vspace{-20pt}
    \end{minipage}%
    \hfill
    \begin{minipage}[b]{0.75\textwidth}
        \centering
        \includegraphics[width=1.\textwidth]{Image/psnr_trends_rff_pe_simplified.pdf} % 替换为你的大图文件
        \vspace{-20pt}
        \label{figure:large_image}
        
    \end{minipage}
    \caption{We test the performance of MLPs with Random Fourier Features (RFF) and MLPs with Positional Encoding (PE) on a 1024-resolution image to better distinguish between high- and low-frequency regions, as demonstrated on the left-hand side of this figure. We find that the performance of MLPs+RFF degrades rapidly with increasing standard deviation compared with MLPs+PE. Since positional encoding is deterministic, scale=512 can be considered to have standard deviation around 121.}
    \label{figure:combined}
    \vspace{-10pt}
\end{figure*}
Varying the sampling standard deviation/scale may lead to degradation results, as shown in \autoref{figure:combined}.
%
However, MLPs face a significant challenge known as the spectral bias, where low-frequency signals are typically favored during training~\citep{rahaman2019spectral}. 
A common solution is to map coordinates into the frequency domain using Fourier features, such as Random Fourier Features and Positional Encoding, which can be understood as manually set high-frequency correspondence prior to accelerating the learning of high-frequency targets.~\citep{tancik2020fourier}. 
This embeddings widely applied to the INRs for novel view synthesis~\citep{mildenhall2021nerf,barron2021mip}, dynamic scene reconstruction~\citep{pumarola2021d}, object tracking~\citep{wang2023tracking}, and medical imaging~\citep{corona2022mednerf}.
% \begin{figure}[!h]
%     \centering
%     \includegraphics[width=1.\textwidth]{Image/psnr_trends_rff_pe_simplified.pdf}
%     \caption{This figure shows the change of PSNR on the whole, low-frequency region, and high-frequency region of the image fitting by using two Fourier Features Embedding with varying scale of variance: (Right) Positional Encoding (PE) (Left) Random Fourier Features (RFF). Both PE and RFF will degrade the low-frequency regions of the target image when variance increases.}
%     \vspace{-20pt} 
%     \label{figure:stats}
% \end{figure}


Although many INRs' downstream application scenarios use this encoding type, it has certain limitations when applied to specific tasks.
%
It depends heavily on two key hyperparameters: the sampling standard deviation/scale (available sampling range of frequencies) and the number of samples.
%
Even with a proper choice of sampling standard deviation/scale, the output remains unsatisfactory, as shown in \autoref{fig:var}: Noisy low-frequency regions and degraded high-frequency regions persist with well chosen sampling standard deviation/scale with the grid-searched standard deviation/scale, which may potentially affect the performance of the downstream applications resulting in noisy or coarse output.
%
However, limited research has contributed to explaining the reason and finding a proper frequency embeddings for input~\citep{landgraf2022pins, yuce2022structured}.

In this paper, we aim to offer a potential explanation for the high-frequency noise and propose an effective solution to the inherent drawbacks of Fourier feature embeddings for INRs.
%
Firstly, we hypothesize that the noisy output arises from the interaction between Fourier feature embeddings and multi-layer perceptrons (MLPs). We argue that these two elements can enhance each other's representation capabilities when combined. However, this combination also introduces the inherent properties of the Fourier series into the MLPs.
%
To support our hypothesis, we propose a simple theorem stating that the unsampled frequency components of the embeddings establish a lower bound on the expected performance. This underpins our hypothesis, as the primary fitting error in finitely sampled Fourier series originates from these unsampled frequencies.

Inspired by the analysis of noisy output and the properties of Fourier series expansion, we propose an approach to address this issue by enabling INRs to adaptively filter out unnecessary high-frequency components in low-frequency regions while enriching the input frequencies of the embeddings if possible.
%
To achieve this, we employ bias-free (additive term-free) MLPs. These MLPs function as adaptive linear filters due to their strictly linear and scale-invariant properties~\citep{mohan2019robust}, which preserves the input pattern through each activation layer and potentially enhances the expressive capability of the embeddings.
%
Moreover, by viewing the learning rate of the proposed filter and INRs as a dynamically balancing problem, we introduce a custom line-search algorithm to adjust the learning rate during training. This algorithm tackles an optimization problem to approximate a global minimum solution. Integrating these approaches leads to significant performance improvements in both low-frequency and high-frequency regions, as demonstrated in the comparison shown in \autoref{fig:var}.
%
Finally, to evaluate the performance of the proposed method, we test it on various INRs tasks and compare it with state-of-the-art models, including BACON~\citep{lindell2022bacon}, SIREN~\citep{sitzmann2020implicit}, GAUSS~\citep{ramasinghe2022beyond} and WIRE~\citep{saragadam2023wire}. 
The experimental results prove that our approach enables MLPs to capture finer details via Fourier Features while effectively reducing high-frequency noise without causing oversmoothness.
%
To summarize, the following are the main contributions of this work:
\begin{itemize}
    \item From the perspective of Fourier features embeddings and MLPs, we hypothesize that the representation capacity of their combination is also the combination of their strengths and limitations. A simple lemma offers partial validation of this hypothesis.

    
    \item  We propose a method that employs a bias-free MLP as an adaptive linear filter to suppress unnecessary high frequencies. Additionally, a custom line-search algorithm is introduced to dynamically optimize the learning rate, achieving a balance between the filter and INRs modules.

    \item To validate our approach, we conduct extensive experiments across a variety of tasks, including image regression, 3D shape regression, and inverse graphics. These experiments demonstrate the effectiveness of our method in significantly reducing noisy outputs while avoiding the common issue of excessive smoothing.
\end{itemize}

\section{Reminders on flag manifolds}\label{sec:flags}

A flag is a sequence of nested linear subspaces of increasing dimension.
This section introduces flag manifolds and provides the minimal tools to perform optimization on those spaces. Much more details and properties can be obtained in dedicated papers~\citep{ye_optimization_2022}.

\subsection{Flags in the scientific literature}
By providing a natural parametrization for the eigenspaces of symmetric matrices, flags have long been geometrical objects of interest in the scientific community, with traditional applications in physics \citep{arnold_modes_1972} and linear algebra \citep{ammar_geometry_1986}.
Modern computational methods on flag manifolds arose later with the seminal work of \citet{edelman_geometry_1998} who provided some optimization algorithms on matrix manifolds such as Grassmannians and Stiefel manifolds.
Later, \citet{nishimori_riemannian_2006} provided an optimization algorithm on flag manifolds for independent subspace analysis \citep{cardoso_multidimensional_1998, hyvarinen_emergence_2000}.
Recently, flag manifolds played a central role in the investigation of a so-called \textit{curse of isotropy} in principal component analysis~\citep{szwagier_curse_2024}.

Flags as a sequence of nested subspaces were first introduced in the machine learning literature as robust prototypes to represent collections of subspaces of different dimensions~\citep{draper_flag_2014,mankovich_flag_2022}.
They were obtained via a sequential construction (where one dimension is added at a time), which can be problematic for greediness reasons~\citep{huber_projection_1985,lerman_overview_2018}.
\citet{pennec_barycentric_2018} was the first to show that PCA can be reformulated as an optimization problem on flag manifolds by summing the unexplained variance at different dimensions. This principle was recently applied to several variants of PCA under the name of \textit{flagification} in~\citet{mankovich_fun_2024}, showing an important robustness to outliers.
Finally, some works compute distance-related quantities on flag-valued datasets, with notable applications in computer vision and hyperspectral imaging~\citep{ma_flag_2021, nguyen_closed-form_2022, mankovich_chordal_2023, szwagier_rethinking_2023}.


\subsection{Definition and representation of flag manifolds}
Let $p \geq 2$ %be an ambient dimension 
and $q_{1:d} := (q_1, q_2 ,\cdots, q_d)$ be a sequence of increasing integers such that $~{0 < q_1 < q_2 < \cdots < q_d < p}$.
A \textit{flag} of \textit{signature} $(p, q_{1:d})$ is a sequence of nested {linear} subspaces $\{0\} \subset \S_1 \subset \S_2 \subset \dots \subset \S_d \subset \R^p$ of respective dimension $q_1, q_2, \dots, q_d$, noted here $\S_{1:d} := (\S_1, \dots, \S_d)$.\footnote{Flags can equivalently be defined as sequences of \textit{mutually orthogonal} subspaces $\V_1 \perp \V_2 \dots \perp \V_d$, of respective dimension $q_1, q_2-q_1, \dots, q_d - q_{d-1}$, by taking $\V_k$ to be the orthonormal complement of $\S_{k-1}$ onto $\S_k$. This definition is more convenient for computations, but it won't be the one used in this paper.}
A flag $\S_{1:d}$ can be canonically represented as a sequence of symmetric matrices that are the \textit{orthogonal projection matrices} onto the nested subspaces, i.e. $\S_{1:d} \cong (\Pi_{\S_1}, \dots, \Pi_{\S_d})\in\Sym_p^d$. We call it the \textit{projection representation} of flags.

The set of flags of signature $(p, q_{1:d})$ is a smooth manifold~\citep{ye_optimization_2022}, denoted here $\Fl(p, q_{1:d})$. 
Flag manifolds generalize {Grassmannians}---since $~{\Fl(p, (q,)) = \Gr(p, q)}$---and therefore share many practical properties that are useful for optimization~\citep{edelman_geometry_1998}. In the following, we will frequently use the following notations: $q_0 := 0$, $q := q_d$ and $q_{d+1} := p$.


For computational and numerical reasons, flags are often represented as orthonormal $q$-frames. Those correspond to points on the Stiefel manifold $\St(p, q) = \{U\in\R^{p\times q}\colon U\T U = I_q\}$).
Let us define sequentially, for $k\in[1, d]$, $U_k\in\St(p, q_k - q_{k-1})$ such that $[U_1|\dots|U_k]$ is an orthonormal basis of $\S_k$ (this is possible thanks to the nestedness of the subspaces).
Then, $U_{1:d} := [U_1|\dots|U_d]\in\St(p, q)$ is a representative of the flag $\S_{1:d}$. We call it the \textit{Stiefel representation} of flags. Such a representation is not unique---contrary to the projection representation defined previously---due to the \textit{rotational-invariance} of orthonormal bases of subspaces. More precisely, if $U_{1:d}$ is a Stiefel representative of the flag $\S_{1:d}$, then for any set of orthogonal matrices $R_k\in\O(q_k - q_{k-1})$, the matrix $U'_{1:d} := [U_1 R_1|\dots|U_d R_d]$ spans the same flag of subspaces $\S_{1:d}$.
This provides flag manifolds with a quotient manifold structure~\citep{edelman_geometry_1998, absil_optimization_2009, ye_optimization_2022}: 
\begin{equation}\label{eq:Fl_quotient}
	\Fl(p, q_{1:d}) \cong \St(p, q) \big/ \lrp{\O(q_1) \times \O(q_2 - q_1) \times \dots \times \O(q_d - q_{d-1})}.
\end{equation}

\begin{remark}[Orthogonal representation]
For computations, one might have to perform the orthogonal completion of some Stiefel representatives.
Let $\Uf := [U_1|\dots|U_d] \in \St(p, q)$, then one denotes $U_{d+1} \in \St(p, p-q_d)$ to be any orthonormal basis such that $U_{1:d+1} := [U_1|\dots|U_d|U_{d+1}] \in \O(p)$. Such an orthogonal matrix $U_{1:d+1}$ will be called an \emph{orthogonal representative} of the flag $\Sf$. In the following, we may abusively switch from one representation to the other since they represent the same flag.
\end{remark}

\subsection{Optimization on flag manifolds}
There is a rich literature on optimization on smooth manifolds~\citep{edelman_geometry_1998,absil_optimization_2009,boumal_introduction_2023}, and the particular case of flag manifolds has been notably addressed in~\citet{ye_optimization_2022,zhu_practical_2024}. Since flag manifolds can be represented as quotient spaces of Stiefel manifolds, which themselves are embedded in a Euclidean matrix space, one can develop some optimization algorithms without much difficulty.
In this paper, we will use a \textit{steepest descent} algorithm, which is drawn from several works~\citep{chikuse_statistics_2003,nishimori_riemannian_2006,ye_optimization_2022,zhu_practical_2024}. 
Let $f\colon \Fl(p, \qf) \to \R$ be a smooth function on a flag manifold, expressed in the Stiefel representation (e.g. $f(\Uf) = \sum_{k=1}^d \norm{{U_k} {U_k}\T x} $ for some $x\in\R^p$). Given $\Uf \in \St(p, q)$, let $\operatorname{Grad} f (\Uf) = ({\partial f}/{\partial U_{ij}})_{i, j = 1}^{p, q}$ denote the (Euclidean) gradient of $f$. To ``stay'' on the manifold, one first computes the \textit{Riemannian gradient} of $f$ at $\Uf$, noted $\nabla f (\Uf)$. It can be thought of as a projection of the Euclidean gradient onto the tangent space and computed explicitly~\citep{nishimori_riemannian_2006,ye_optimization_2022}.
Then, one moves in the opposite direction of $\nabla f (\Uf)$ with a so-called \textit{retraction}, which is chosen to be the polar retraction of~\citet[Eq.~(49)]{zhu_practical_2024}, combined with a line-search.
We iterate until convergence.
The final steepest descent algorithm is described in Algorithm~\ref{alg:GD}.
\begin{algorithm}
\caption{Steepest descent on flag manifolds}\label{alg:GD}
\begin{algorithmic}
\Require $f\colon \Fl(p, \qf) \to \R$ a function, $\Uf \in \Fl(p, \qf)$ a flag (Stiefel representation)
\For{$t$ = 1, 2, \dots}
    \State $[G_1|\dots|G_d] \gets \operatorname{Grad} f (\Uf)$ \Comment{Euclidean gradient}
	\State $\nabla \gets \bigl[G_k - \bigl(U_k {U_k}\T G_k + \sum_{l \neq k} U_l {G_l}\T U_k\bigr)\bigr]_{k=1\dots d}$
	\Comment{Riemannian gradient}
	\State $\Uf \gets \operatorname{polar}(U_{1:d+1} - \alpha \nabla)$ \Comment{polar retraction + line search}
\EndFor
\Ensure $U_{1:d}^* \in \Fl(p, \qf)$ an optimal flag
\end{algorithmic}
\end{algorithm}
\begin{remark}[Initialization]
We can initialize Algorithm~\ref{alg:GD} randomly via~\citet[Theorem~1.5.5]{chikuse_statistics_2003}, or choose a specific flag depending on the application, as we will see in \autoref{sec:examples}.
\end{remark}
\begin{remark}[Optimization variants]\label{rk:optim}
Many extensions of Algorithm~\ref{alg:GD} can be considered: conjugate gradient~\citep{ye_optimization_2022}, Riemannian trust region~\citep{absil_optimization_2009} etc. We can also replace the polar retraction with a geodesic step~\citep{ye_optimization_2022} or other retractions~\citep{zhu_practical_2024}.
\end{remark}
 \section{The flag trick in theory}\label{sec:flag_trick}
In this section, we motivate and introduce the flag trick to make subspace learning methods nested.
The key result is \autoref{thm:flag_trick}, which states that the classical PCA at a fixed dimension can be converted into a nested multilevel method using nested projectors.

In the remaining of the paper, we assume that the data has been already \textit{centered} around a point of interest (e.g. its mean or geometric median), so that we are only interested in fitting \textit{linear} subspaces and not \textit{affine} ones. 
One could directly include the center in the optimization variables---in which case the geometry would be the one of affine Grassmannians~\citep{lim_numerical_2019} or affine flags~\citep{pennec_barycentric_2018}---but we don't do it in this work for conciseness.


\subsection{From subspaces to flags of subspaces: the seminal example of PCA}
PCA is known as the eigendecomposition of the sample covariance matrix. Originally, it can be formulated as the search for a low dimensional subspace that minimizes the unexplained variance (or maximizes the explained variance).
Let $X:=[x_1|\dots|x_n] \in \R^{p\times n}$ be a data matrix with $n$ samples, let $\S \in \Gr(p, q)$ be a $q$-dimensional subspace, and let $\Pi_{\S} \in \R^{p\times p}$ be the orthogonal projection matrix onto $\S$.
Then PCA consists in the following optimization problem on Grassmannians:
\begin{equation}\label{eq:PCA_subspace}
\S_q^* = \argmin{\S \in \Gr(p, q)} \norm{X - \Pi_{\S} X}_F^2,
\end{equation}
where $\norm{M}_F^2 := \tr{M\T M}$ denotes the \textit{Frobenius norm}.
The solution to the optimization problem is the $q$-dimensional subspace spanned by the leading eigenvectors of the sample covariance matrix $S := \frac 1 n X X\T$, that we note $\S_q^* = \operatorname{Span}(v_1, \dots, v_q)$. 
It is unique when the sample eigenvalues $q$ and $q+1$ are distinct, which is almost sure when $q \leq \operatorname{rank}(S)$. We will assume to be in such a setting in the following for simplicity but it can be easily handled otherwise by ``grouping'' the repeated eigenvalues (cf.~\citet[Theorem~B.1]{szwagier_curse_2024}).
In such a case, the principal subspaces are \textit{nested} for increasing $q$, i.e., if $\S_q^*$ is the $q$-dimensional principal subspace, then for any $r > q$, one has $\S_q^* = \operatorname{Span}(v_1, \dots, v_q) \subset \operatorname{Span}(v_1, \dots, v_r) = \S_r^*$.

Another way of performing PCA is in a sequential manner (cf. footnote~\ref{footnote:sequential}). We first estimate the 1D subspace $\mathcal{V}_1^*$ that minimizes the unexplained variance, then estimate the 1D subspace $\mathcal{V}_2^*$ that minimizes the unexplained variance while being orthogonal to the previous one, and so on and so forth. This gives the following \textit{constrained} optimization problem on 1D Grassmannians:
\begin{equation}\label{eq:PCA_sequential}
\mathcal{V}_q^* = \argmin{\substack{\mathcal{V} \in \Gr(p, 1)\\ \mathcal{V} \perp \mathcal{V}_{q-1} \perp \dots \perp \mathcal{V}_1}} \norm{X - \Pi_{\mathcal{V}} X}_F^2.
\end{equation}
This construction naturally yields a sequence of nested subspaces of increasing dimension---i.e. a flag of subspaces---best and best approximating the dataset:
\begin{equation}
	\{0\} \subset \S_1^* \subset \S_2^* \subset \dots \subset \S_{p-1}^* \subset \R^p, \text{with } \S_k^* = \bigoplus_{l=1}^k \mathcal{V}_l^*.
\end{equation}
Those subspaces happen to be exactly the same as the ones obtained by solving the subspace learning optimization problem~\eqref{eq:PCA_subspace}, although the way they are obtained (in a greedy manner) is different. This is generally not the case for other dimension reduction problems (for instance in robust subspace recovery, as it is raised in the final open questions of~\citet{lerman_overview_2018}).

Hence, the optimal solution to the \textit{subspace} learning formulation of PCA~\eqref{eq:PCA_subspace} is equivalent to the \textit{sequential} formulation of PCA~\eqref{eq:PCA_sequential}, and both yield a flag of subspaces best and best approximating the data. One can wonder if this result could be directly obtained by formulating an optimization problem on flag manifolds. The answer is \textit{yes}, as first proven in~\citet[Theorem~9]{pennec_barycentric_2018} with an \textit{accumulated unexplained variance} (AUV) technique, but there is not a unique way to do it. Motivated by the recent principal subspace analysis~\citep{szwagier_curse_2024}, we propose in the following theorem a generic trick to formulate PCA as an optimization on flag manifolds.
\begin{theorem}[Nested PCA with flag manifolds]\label{thm:flag_trick}
	Let $X := [x_1|\dots|x_n]\in\R^{p\times n}$ be a centered $p$-dimensional ($p \geq 2$) dataset with $n$ samples. Let $q_{1:d} := (q_1, q_2 ,\cdots, q_d)$ be a sequence of increasing dimensions such that $0 < q_1 < q_2 < \cdots < q_d < p$.
	Let $S := \frac 1 n X X\T$ be the sample covariance matrix. Assume that it eigendecomposes as $S := \sum_{j=1}^p \ell_j v_j {v_j}\T$ where $\ell_1 \geq \dots \geq \ell_p$ are the eigenvalues and $v_1 \perp \dots \perp v_p$ are the associated eigenvectors.
	Then PCA can be reformulated as the following optimization problem on flag manifolds:
	\begin{equation}
		{\S_{1:d}^*} = \argmin{\S_{1:d} \in \Fl(p, q_{1:d})} \norm{X - \frac 1 d \sum_{k=1}^d \Pi_{\S_k} X}_F^2.
	\end{equation}
	More precisely, one has ${\S_{1:d}^*} = \lrp{\operatorname{Span}(v_1, \dots, v_{q_1}), \operatorname{Span}(v_1, \dots, v_{q_2}), \dots, \operatorname{Span}(v_1, \dots, v_{q_d})}$.
	The solution is unique if and only if $\ell_{q_k} \neq \ell_{q_{k+1}}, \forall k\in[1, d]$.
\end{theorem}
\begin{proof} 
One has:
\begin{align}
	\norm{X - \frac 1 d \sum_{k=1}^d \Pi_{\S_k} X}_F^2
	&= \tr{X\T \lrp{I_p - \frac1d\sum_{k=1}^d \Pi_{\S_k}}^2 X},\\
	&= \frac1{d^2} \tr{X\T \lrp{\sum_{k=1}^d \lrp{I_p - \Pi_{\S_k}}}^2 X},\\
	&= \frac n {d^2} \tr{W^2 S},
\end{align}
with $W = \sum_{k=1}^d (I_p - \Pi_{\S_k})$ and $S = \frac 1 n X X\T$.
	Let $U_{1:d+1} := [U_1|\dots|U_d|U_{d+1}]\in\O(p)$ be an orthogonal representative (cf. \autoref{sec:flags}) of the optimization variable $\S_{1:d} \in \Fl(p, q_{1:d})$. Then one has $~{\Pi_{\S_k} = U_{1:d+1} \diag{I_{q_k}, 0_{p - q_k}} {U_{1:d+1}}\T}$. Therefore, one has $W = U_{1:d+1} \Lambda {U_{1:d+1}}\T$, with $~{\Lambda = \diag{0 \, I_{q_1}, 1 \, I_{q_2 - q_1}, \dots, d \, I_{q_{d+1} - q_d}}}$.
	Hence, one has 
	\begin{equation}
		\argmin{\S_{1:d} \in \Fl(p, q_{1:d})} \norm{X - \frac 1 d \sum_{k=1}^d \Pi_{\S_k} X}_F^2 \Longleftrightarrow \argmin{U \in \O(p)} \frac n {d^2} \tr{U \Lambda^2 U\T S}.
	\end{equation}
	The latter problem is exactly the same as in~\citet[Equation~(19)]{szwagier_curse_2024}, which solves maximum likelihood estimation for principal subspace analysis.
	Hence, one can conclude the proof on existence and uniqueness using~\citet[Theorem~B.1]{szwagier_curse_2024}.
\end{proof}
The key element of the proof of \autoref{thm:flag_trick} is that averaging the nested projectors yields a \textit{hierarchical reweighting} of the (mutually-orthogonal) principal subspaces. More precisely, the $k$-th principal subspace has weight $(k-1)^2$, and this reweighting enables to get a hierarchy of eigenspaces~\citep{cunningham_linear_2015,pennec_barycentric_2018,oftadeh_eliminating_2020}.
In the following, we note $~{\Pi_{\S_{1:d}} := \frac 1 d \sum_{k=1}^d \Pi_{\S_k}}$ and call this symmetric matrix the \textit{average multilevel projector}, which will be central in the extension of subspace methods into multilevel subspace methods.



\subsection{The flag trick}
As we will see in the following (\autoref{sec:examples}), many important machine learning problems can be formulated as the optimization of a certain function $f$ on Grassmannians.
\autoref{thm:flag_trick} shows that replacing the subspace projection matrix $\Pi_{\S}$ appearing in the objective function by the average multilevel projector $\Pi_{\Sf}$ yields a sequence of subspaces that meet the original objective of principal component analysis, while being nested.
This leads us to introduce the \textit{flag trick} for general subspace learning problems.
\begin{definition}[Flag trick]\label{def:flag_trick}
Let $p \geq 2$, $0 < q < p$ and $q_{1:d} := (q_1, q_2 ,\cdots, q_d)$ be a sequence of increasing dimensions such that $0 < q_1 < q_2 < \cdots < q_d < p$.
The flag trick consists in replacing a subspace learning problem of the form:
\begin{equation}\label{eq:subspace_problem}
    \argmin{\S \in \Gr(p, q)} f(\Pi_\S)
\end{equation}
with the following optimization problem:
\begin{equation}\label{eq:flag_problem}
    \argmin{\S_{1:d} \in \Fl(p, q_{1:d})} f\lrp{\frac 1 d \sum_{k=1}^d \Pi_{\S_k}}.
\end{equation}
\end{definition}
Except for the very particular case of PCA (\autoref{thm:flag_trick}) where $f_X(\Pi) = \norm{X - \Pi X}_F^2$, we cannot expect to have an analytic solution to the flag problem~\eqref{eq:flag_problem}; indeed, in general, subspace problems do not even have a closed-form solution as we shall see in \autoref{sec:examples}. This justifies the introduction of optimization algorithms on flag manifolds like Algorithm~\ref{alg:GD}. 

\begin{remark}[Flag trick vs. AUV]
The original idea of accumulated unexplained variance~\citep{pennec_barycentric_2018} (and its subsequent application to several variants of PCA under the name of ``flagification''~\citep{mankovich_fun_2024}) consists in summing the subspace criteria at different dimensions, while the flag trick directly averages the orthogonal projection matrices that appear inside the objective function.
While both ideas are equally worth experimenting with, we believe that the flag trick has a much wider reach. Indeed, from a technical viewpoint, the flag trick appears at the covariance level and directly yields a hierarchical reweighting of the principal subspaces. This reweighting is only indirect with the AUV---due to the linearity of the trace operator---and is not expected to happen for other methods than PCA. Notably, as we shall see in \autoref{sec:examples}, the flag trick enables to easily develop extensions of well-known methods involving PCA, like IRLS~\citep{lerman_fast_2018} or Newton-Lanczos methods for trace ratio optimization~\citep{ngo_trace_2012}, and is closer in spirit to the statistical formulations of PCA~\citep{szwagier_curse_2024}.
\end{remark}


\subsection{Multilevel machine learning}
Subspace learning is often used as a preprocessing task before running a machine learning algorithm, notably to overcome the curse of dimensionality. One usually projects the data onto the optimal subspace $\S^* \in \Gr(p, q)$ and use the resulting lower-dimensional dataset as an input to a machine learning task like clustering, classification or regression~\citep{bouveyron_model-based_2019}. Since the flag trick problem~\eqref{eq:flag_problem} does not output one subspace but a hierarchical sequence of nested subspaces, it is legitimate to wonder what to do with such a multilevel representation.
In this subsection, we propose a general \textit{ensemble learning} method to aggregate the hierarchical information coming from the flag of subspaces.

Let us consider a dataset $X := [x_1|\dots|x_n]\in\R^{p \times n}$ (possibly with some associated labels $Y := [y_1|\dots|y_n]\in\R^{m \times n}$). In machine learning, one often fits a model to the dataset by optimizing an objective function of the form $R_{X, Y}(g) = \frac{1}{n} \sum_{i=1}^n L(g(x_i), y_i)$.
With the flag trick, we get a filtration of projected data points $Z_k = \Pi_{\S_k^*} X, k\in[1,d]$ that can be given as an input to different machine learning algorithms. This yields optimal predictors $g_k^* = \operatorname{argmin} \, R_{Z_k, Y}$ which can be aggregated via \href{https://scikit-learn.org/stable/modules/ensemble.html}{ensembling methods}.
For instance, \textit{voting} methods choose the model with the highest performance on holdout data; this corresponds to selecting the optimal dimension $q^* \in q_{1:d}$ \textit{a posteriori}, based on the machine learning objective. A more nuanced idea is the one of \textit{soft voting}, which makes a weighted averaging of the predictions. The weights can be uniform, proportional to the performances of the individual models, or learned to maximize the performance of the weighted prediction~\citep{perrone_when_1992}. Soft voting gives different weights to the nested subspaces depending on their contribution to the ensembled prediction and therefore provides a soft measure of the relative importance of the different dimensions. In that sense, it goes beyond the classical manifold assumption stating that data has one intrinsic dimension, and instead proposes a soft blend between dimensions that is adapted to the learning objective. This sheds light on the celebrated paper of Minka for the automatic choice of dimensionality in PCA~\citep[Section~5]{minka_automatic_2000}.
Many other ensembling methods are possible like gradient boosting, Bayesian model averaging and stacking.
The whole methodology is summarized in Algorithm~\ref{alg:flag_trick}.
\begin{algorithm}
\caption{Flag trick combined with ensemble learning}\label{alg:flag_trick}
\begin{algorithmic}
\Require $X := [x_1|\dots|x_n]\in\R^{p \times n}$ a data matrix; $q_{1:d} := (q_1, \dots, q_d)$ a flag signature; $f$ a subspace learning objective; (opt.) $Y := [y_1|\dots|y_n]\in\R^{m \times n}$ a label matrix
\State ${\S}_{1:d}^* \gets \operatorname{argmin}_{\S_{1:d} \in \Fl(p, q_{1:d})} \, f(\Pi_{\S_{1:d}})$  \Comment{flag trick \eqref{eq:flag_problem} + optimization (Alg.~\ref{alg:GD})}
\For{k = 1 \dots d}
	\State $g_k^* \gets \operatorname{fit}(\Pi_{\S_k^*} X, Y)$  \Comment{learning on $q_k$-dimensional projected data}
	\State $Y_k^* \gets g_k^*(\Pi_{\S_k^*} X)$ \Comment{prediction on $q_k$-dimensional projected data}
\EndFor
\State $Y^* \gets \operatorname{ensembling}({Y}_1^*, \dots, {Y}_d^*)$  \Comment{weighted predictions}
\Ensure $Y^*$ the ensembled predictions
\end{algorithmic}
\end{algorithm}

Algorithm~\ref{alg:flag_trick} is a general proposition of multilevel machine learning with flags, but many other uses of the optimal flag $\S_{1:d}^*$ are possible, depending on the application. For instance, one may directly use the reweighted data matrix $\Pi_{\S_{1:d}^*} X$ as an input to the machine learning algorithm. This enables to fit only one model instead of $d$.
One can also simply analyze the projected data \textit{qualitatively} via scatter plots or reconstruction plots as evoked in \autoref{sec:intro}. The nestedness will automatically bring consistency contrarily to non-nested subspace methods, and therefore improve interpretability.
Finally, many other ideas can be borrowed from the literature on subspace clustering and flag manifolds~\citep{draper_flag_2014,launay_mechanical_2021, ma_flag_2021, mankovich_flag_2022,mankovich_chordal_2023,mankovich_fun_2024}, for instance the computation of distances between flags coming from different datasets as a multilevel measure of similarity between datasets.
\section{The flag trick in action}\label{sec:examples}
In this section, we provide some applications of the flag trick to several learning problems. We choose to focus on subspace recovery, trace ratio and spectral clustering problems. Other ones, like domain adaptation, matrix completion and subspace tracking are developed or mentioned in the last subsection but not experimented for conciseness.

\subsection{Outline and experimental setting}
For each application, we first present the learning problem as an optimization on Grassmannians. Second, we formulate the associated flag learning problem by applying the flag trick (Definition~\ref{def:flag_trick}). Third, we optimize the problem on flag manifolds with the steepest descent method (Algorithm~\ref{alg:GD})---more advanced algorithms are also derived in the appendix. 
Finally, we perform various nestedness and ensemble learning experiments via Algorithm~\ref{alg:flag_trick} on both synthetic and real datasets.

The general methodology to compare Grassmann-based methods to flag-based methods is the following one. For each experiment, we first choose a flag signature $~{q_{1\rightarrow d} := (q_1, \dots, q_d)}$, then we run independent optimization algorithms on $\Gr(p, q_1), \dots, \Gr(p, q_d)$~\eqref{eq:subspace_problem} and finally we compare the optimal subspaces $\S_k^* \in \Gr(p, q_k)$ to the optimal flag of subspaces $\Sf^* \in \Fl(p, q_{1\rightarrow d})$ obtained via the flag trick~\eqref{eq:flag_problem}. 
To show the nestedness issue in Grassmann-based methods, we compute the subspace distances $\Theta(\S_k^*, \S_{k+1}^*)_{k=1\dots d-1}$, where $\Theta$ is the generalized Grassmann distance of~\citet[Eq.~(14)]{ye_schubert_2016}. It consists in the $\ell_2$ norm of the principal angles, which can be obtained from the singular value decomposition (SVD) of the inner-products matrices ${U_k}\T {U_{k+1}}$, where $U_k \in \St(p, q_k)$ is an orthonormal basis of $\S_k^*$.

Regarding the implementation of the steepest descent algorithm on flag manifolds (Algorithm~\ref{alg:GD}), we develop a new class of manifolds in \href{https://pymanopt.org/}{PyManOpt}~\citep{boumal_manopt_2014,townsend_pymanopt_2016}, and run their \href{https://github.com/pymanopt/pymanopt/blob/master/src/pymanopt/optimizers/steepest_descent.py}{SteepestDescent} algorithm. Our implementation of the \texttt{Flag} class is based on the Stiefel representation of flag manifolds, detailed in \autoref{sec:flags}, with the retraction being the polar retraction. For the computation of the gradient, we use automatic differentiation with the \texttt{\href{https://github.com/HIPS/autograd}{autograd}} package. We could derive the gradients by hand from the expressions we get, but we use automatic differentiation as strongly suggested in PyManOpt's \href{https://pymanopt.org/docs/stable/quickstart.html}{documentation}.
Finally, the real datasets and the machine learning methods used in the experiments can be found in \href{https://scikit-learn.org/stable/}{scikit-learn}~\citep{pedregosa_scikit-learn_2011}.

\section{Robust subspace recovery: extensions and proofs}\label{app:RSR}

\subsection{An IRLS algorithm for robust subspace recovery}
Iteratively reweighted least squares (IRLS) is a ubiquitous method to solve optimization problems involving $L^p$-norms. Motivated by the computation of the geometric median~\citep{weiszfeld_sur_1937}, IRLS is highly used to find robust maximum likelihood estimates of non-Gaussian probabilistic models (typically those containing outliers) and finds application in robust regression~\citep{huber_robust_1964}, sparse recovery~\citep{daubechies_iteratively_2010} etc.

The recent fast median subspace (FMS) algorithm~\citep{lerman_fast_2018}, achieving state-of-the-art results in RSR uses an IRLS scheme to optimize the Least Absolute Deviation (LAD)~\eqref{eq:RSR_Gr}.
The idea is to first rewrite the LAD as 
\begin{equation}
	\sum_{i=1}^n \norm{x_i - \Pi_{\S} x_i}_2 = \sum_{i=1}^n w_i(\S) \norm{x_i - \Pi_{\S} x_i}_2^2,
\end{equation}
with $w_i(\S) = \frac{1}{\norm{x_i - \Pi_{\S} x_i}_2}$, and then successively compute the weights $w_i$ and update the subspace according to the weighted objective.
More precisely, the FMS algorithm creates a sequence of subspaces $\S^1, \dots, \S^m$ such that 
\begin{equation}\label{eq:IRLS_FMF}
    \S^{t+1} = \argmin{\S \in \Gr(p, q)} \sum_{i=1}^n w_i(\S^t) \norm{x_i - \Pi_\S x_i}_2^2.
\end{equation}
This weighted least-squares problem enjoys a closed-form solution which relates to the eigenvalue decomposition of the weighted covariance matrix $\sum_{i=1}^n w_i(\S^t) x_i {x_i}\T$~\citep[Chapter~3.3]{vidal_generalized_2016}.

We wish to derive an IRLS algorithm for the flag-tricked version of the LAD minimization problem~\eqref{eq:RSR_Fl}.
In order to stay close in mind to the recent work of \citet{peng_convergence_2023} who proved convergence of a general class of IRLS algorithms under some mild assumptions, we first rewrite~\eqref{eq:RSR_Fl} as
\begin{equation}~\label{eq:RSR_Fl_IRLS}
    \argmin{\S_{1:d} \in\Fl(p, \qf)} \sum_{i=1}^n \rho(r(\S_{1:d}, x_i)),
\end{equation}
where $r(\S_{1:d}, x) = \norm{x - \Pi_{\Sf} x}_2$ is the \textit{residual} and $\rho(r) = |r|$ is the \textit{outlier-robust} loss function.
Following~\citet{peng_convergence_2023}, the IRLS scheme associated with~\eqref{eq:RSR_Fl_IRLS} is:
\begin{equation}
\begin{cases}
w_i^{t+1} = \rho'(r(\S_{1:d}^t, x_i)) /  r(\S_{1:d}^t, x_i) = 1 / \norm{x_i - \Pi_{\Sf} x_i}_2,\\
(\S_{1:d})^{t+1} = \argmin{\S_{1:d} \in\Fl(p, \qf)} \sum_{i=1}^n w_i^{t+1} \norm{x_i - \Pi_{\Sf} x_i}_2^2.
\end{cases}
\end{equation}
We now show that the second step enjoys a closed-form solution.
\begin{theorem}\label{thm:IRLS_FMF}
The RLS problem
\begin{equation}
    \argmin{\S_{1:d} \in\Fl(p, \qf)} \sum_{i=1}^n w_i \norm{x_i - \Pi_{\Sf} x_i}_2^2
\end{equation}
has a closed-form solution $\S_{1:d}^* \in\Fl(p, \qf)$, which is given by the eigenvalue decomposition of the weighted sample covariance matrix $S_w = \sum_{i=1}^n w_i x_i {x_i}\T = \sum_{j=1}^p \ell_j v_j {v_j}\T$, i.e.
\begin{equation}
    \S_k^* = \operatorname{Span}(v_1, \dots, v_{q_k}) \quad (k=1\twodots d).
\end{equation}
\end{theorem}
\begin{proof}
One has
\begin{equation}
	\sum_{i=1}^n w_i \norm{x_i - \Pi_{\Sf} x_i}_2^2 = \tr{(I - \Pi_{\Sf})^2 \lrp{\sum_{i=1}^n w_i x_i {x_i}\T}}.
\end{equation}
Therefore, we are exactly in the same case as in \autoref{thm:flag_trick}, if we replace $X X\T$ with the reweighted covariance matrix $\sum_{i=1}^n w_i x_i {x_i}\T$. This does not change the result, so we conclude with the end of the proof of \autoref{thm:flag_trick} (which itself relies on~\citet{szwagier_curse_2024}).
\end{proof}
Hence, one gets an IRLS scheme for the LAD minimization problem. 
One can modify the robust loss function $\rho(r) = |r|$ by a Huber-like loss function to avoid weight explosion. Indeed, one can show that the weight $w_i := 1 / \norm{x_i - \Pi_{\Sf} x_i}_2$ goes to infinity when the first subspace $\S_1$ of the flag gets close to $x_i$ .
Therefore in practice, we take 
\begin{equation}
    \rho(r) = 
        \begin{cases}
            r^2 / (2 p \delta) & \text{if } |r| <= p\delta,\\
            r - p \delta / 2 & \text{if } |r| > p\delta.
        \end{cases}
\end{equation}
This yields
\begin{equation}
    w_i = 1 / \max\lrp{p\delta,  1 / \norm{x_i - \Pi_{\Sf} x_i}_2}.
\end{equation}
The final proposed scheme is given in Algorithm~\ref{alg:FMF}, named \textit{fast median flag} (FMF), in reference to the fast median subspace algorithm of~\citet{lerman_fast_2018}.
\begin{algorithm}
\caption{Fast median flag}\label{alg:FMF}
\begin{algorithmic}
\Require $X\in \R^{p\times n}$ (data), $\quad q_1 < \dots < q_d$ (signature), $\quad t_{max}$ (max number of iterations), $\quad \eta$ (convergence threshold), $\quad \varepsilon$ (Huber-like saturation parameter)
\Ensure
$U \in \St(p, q)$
\State $t \gets 0, \quad \Delta \gets \infty, \quad U^0 \gets \operatorname{SVD}(X, q)$
\While{$\Delta > \eta$ and $t < t_{max}$}
    \State $t \gets t+1$
    \State $r_i \gets \norm{x_i - \Pi_{\Sf} x_i}_2$
    \State $y_i \gets {x_i} / {\max(\sqrt{r_i}, \varepsilon)}$
    \State $U^t \gets \operatorname{SVD(Y, q)}$
    \State $\Delta \gets \sqrt{\sum_{k=1}^{d} \Theta(U^t_{q_k}, U^{t-1}_{q_k})^2}$
\EndWhile
\end{algorithmic}
\end{algorithm}
We can easily check that FMF is a direct generalization of FMS for Grassmannians (i.e. when $d=1$).


\begin{remark}
This is far beyond the scope of the paper, but we believe that the convergence result of~\citet[Theorem~1]{peng_convergence_2023} could be generalized to the FMF algorithm, due to the compactness of flag manifolds and the expression of the residual function $r$.
\end{remark}

\subsection{Proof of Proposition~\ref{prop:RSR}}
Let $\Sf \in \Fl(p, \qf)$ and $U_{1:d+1} := [U_1|U_2|\dots|U_d|U_{d+1}] \in \O(p)$ be an orthogonal representative of $\Sf$. One has:
\begin{align}
	\norm{x_i - \Pi_{\S_{1:d}} x_i}_2 &= \sqrt{{(x_i - \Pi_{\S_{1:d}} x_i)}\T (x_i - \Pi_{\S_{1:d}} x_i)},\\
	 &= \sqrt{{x_i}\T {(I_p - \Pi_{\S_{1:d}})}^2 x_i},\\
	 &= \sqrt{{x_i}\T {\lrp{I_p - \frac1d \sum_{k=1}^d\Pi_{\S_k}}}^2 x_i},\\
 	 &= \sqrt{\frac1{d^2} {x_i}\T {\lrp{\sum_{k=1}^d (I_p - \Pi_{\S_k})}}^2 x_i},\\
 	 % &= \sqrt{\frac1{d^2} {x_i}\T U_{1:d+1} \diag{0, 1, \dots, d-1, d}^2 {U_{1:d+1}}\T  x_i},\\
 	 &= \sqrt{\frac1{d^2} {x_i}\T \lrp{\sum_{k=1}^{d+1} (k-1) U_k {U_k}\T}^2  x_i},\\
 	 &= \sqrt{\frac1{d^2} {x_i}\T \lrp{\sum_{k=1}^{d+1} (k-1)^2 U_k {U_k}\T}  x_i},\\
 	 &= \sqrt{\sum_{k=1}^{d+1} \lrp{\frac {k-1} {d}}^2 {x_i}\T \lrp{ U_k {U_k}\T}  x_i},\\
  	 \norm{x_i - \Pi_{\S_{1:d}} x_i}_2 &= \sqrt{\sum_{k=1}^{d+1} \lrp{\frac {k-1} {d}}^2 \norm{{U_k}\T x_i}_2^2},
\end{align}
which concludes the proof.
\subsection{The flag trick for trace ratio problems}\label{subsec:TR}
Trace ratio problems are ubiquitous in machine learning~\citep{ngo_trace_2012}. They write as:
\begin{equation}\label{eq:TR_St}
\argmax{U \in \St(p, q)} \frac{\tr{U\T A U}}{\tr{U\T B U}},
\end{equation}
where $A, B \in \R^{p\times p}$ are positive semi-definite matrices, with $\operatorname{rank}(B) > p - q$.

A famous example of TR problem is Fisher's linear discriminant analysis (LDA)~\citep{fisher_use_1936,belhumeur_eigenfaces_1997}.
It is common in machine learning to project the data onto a low-dimensional subspace before fitting a classifier, in order to circumvent the curse of dimensionality. It is well known that performing an unsupervised dimension reduction method like PCA comes with the risks of mixing up the classes, since the directions of maximal variance are not necessarily the most discriminating ones~\citep{chang_using_1983}. The goal of LDA is to use the knowledge of the data labels to learn a linear subspace that does not mix the classes.
Let $~{X := [x_1|\dots|x_n] \in \R^{p\times n}}$ be a dataset with labels $Y := [y_1|\dots|y_n] \in {[1, C]}^n$. Let $\mu = \frac{1}{n} \sum_{i=1}^n x_i$ be the dataset mean and $\mu_c = \frac{1}{\#\{i : y_i=c\}}\sum_{i : y_i=c} x_i$ be the class-wise means. 
The idea of LDA is to search for a subspace $\S \in \Gr(p, q)$ that simultaneously maximizes the projected \textit{between-class variance} $\sum_{c=1}^C \|\Pi_\S \mu_c - \Pi_\S \mu\|_2^2$ and minimizes the projected \textit{within-class variance} $\sum_{c=1}^C \sum_{i : y_i = c} \|\Pi_\S x_i - \Pi_\S \mu_c\|_2^2$. This can be reformulated as a trace ratio problem~\eqref{eq:TR_St}, with $A = \sum_{c=1}^C (\mu_c - \mu) (\mu_c - \mu)\T$ and $B = \sum_{c=1}^C \sum_{i : y_i = c} (x_i - \mu_c) (x_i - \mu_c)\T$.


More generally, a large family of dimension reduction methods can be reformulated as a TR problem. The seminal work of~\citet{yan_graph_2007} shows that many dimension reduction and manifold learning objective functions can be written as a trace ratio involving Laplacian matrices of attraction and repulsion graphs. Intuitively, those graphs determine which points should be close in the latent space and which ones should be far apart.
Other methods involving a ratio of traces are \textit{multi-view learning}~\citep{wang_trace_2023}, \textit{partial least squares} (PLS)~\citep{geladi_partial_1986,barker_partial_2003} and \textit{canonical correlation analysis} (CCA)~\citep{hardoon_canonical_2004}, although these methods are originally \textit{sequential} problems (cf. footnote~\ref{footnote:sequential}) and not \textit{subspace} problems.

Classical Newton-like algorithms for solving the TR problem~\eqref{eq:TR_St} come from the seminal works of~\citet{guo_generalized_2003, wang_trace_2007, jia_trace_2009}.
The interest of optimizing a trace-ratio instead of a ratio-trace (of the form $\tr{(U\T B U)^{-1}(U\T A U)}$), that enjoys an explicit solution given by a generalized eigenvalue decomposition, is also tackled in those papers. The \textit{repulsion Laplaceans}~\citep{kokiopoulou_enhanced_2009} instead propose to solve a regularized version $\tr{U\T B U} - \rho \tr{U\T A U}$, which enjoys a closed-form, but has a hyperparameter $\rho$, which is directly optimized in the classical Newton-like algorithms for trace ratio problems.

\subsubsection{Application of the flag trick to trace ratio problems}
The trace ratio problem~\eqref{eq:TR_St} can be straightforwardly reformulated as an optimization problem on Grassmannians, due to the orthogonal invariance of the objective function:
\begin{equation}\label{eq:TR_Gr}
\argmax{\S \in \Gr(p, q)} \frac{\tr{\Pi_\S A}}{\tr{\Pi_\S B}}.
\end{equation}
The following proposition applies the flag trick to the TR problem~\eqref{eq:TR_Gr}.
\begin{proposition}[Flag trick for TR]\label{prop:TR}
The flag trick applied to the TR problem~\eqref{eq:TR_Gr} reads
\begin{equation}\label{eq:TR_Fl}
	\argmax{\S_{1:d} \in \Fl(p, q_{1:d})} \frac{\tr{\Pi_{\S_{1:d}} A}}{\tr{\Pi_{\S_{1:d}} B}}.
\end{equation}
and is equivalent to the following optimization problem:
\begin{equation}\label{eq:TR_Fl_equiv}
\argmax{U_{1:d} \in \St(p, q)} \frac{\sum_{k=1}^{d} (d - (k-1)) \tr{{U_k}\T A {U_k}}}{\sum_{l=1}^{d} (d - (l-1)) \tr{{U_{l}}\T B {U_{l}}}}.
\end{equation}
\end{proposition}
\begin{proof}
The proof is given in Appendix (\autoref{app:TR}).
\end{proof}
Equation~\eqref{eq:TR_Fl_equiv} tells us several things. First, the subspaces $~{\operatorname{Span}(U_1) \perp \dots \perp \operatorname{Span}(U_d)}$ are weighted decreasingly, which means that they have less and less importance with respect to the TR objective.
Second, we can see that the nested trace ratio problem~\eqref{eq:TR_Fl} somewhat maximizes the numerator $\tr{\Pi_{\S_{1:d}} A}$ while minimizing the denominator $\tr{\Pi_{\S_{1:d}} B}$. Both subproblems have an explicit solution corresponding to our nested PCA Theorem~\ref{thm:flag_trick}. Hence, one can naturally initialize the steepest descent algorithm with the $q$ highest eigenvalues of $A$ or the $q$ lowest eigenvalues of $B$ depending on the application.
For instance, for LDA, initializing Algorithm~\ref{alg:GD} with the highest eigenvalues of $A$ would spread the classes far apart, while initializing it with the lowest eigenvalues of $B$ would concentrate the classes, which seems less desirable since we do not want the classes to concentrate at the same point.

\subsubsection{Nestedness experiments for trace ratio problems}
For all the experiments of this subsection, we consider the particular TR problem of LDA, although many other applications (\textit{marginal Fisher analysis}~\citep{yan_graph_2007}, \textit{local discriminant embedding}~\citep{chen_local_2005} etc.) could be investigated similarly.

First, we consider a synthetic dataset with five clusters.
The ambient dimension is $p = 3$ and the intrinsic dimensions that we try are $q_{1:2} = (1, 2)$.
We adopt a preprocessing strategy similar to~\citet{ngo_trace_2012}: we first center the data, then run a PCA to reduce the dimension to $n - C$ (if $n - C < p$), then construct the LDA scatter matrices $A$ and $B$, then add a diagonal covariance regularization of $10^{-5}$ times their trace and finally normalize them to have unit trace.
We run Algorithm~\ref{alg:GD} on Grassmann manifolds to solve the TR maximization problem~\eqref{eq:TR_Gr}, successively for $q_1 = 1$ and $q_2 = 2$. Then we plot the projections of the data points onto the optimal subspaces. We compare them to the nested projections onto the optimal flag output by running Algorithm~\ref{alg:GD} on $\Fl(3, (1, 2))$ to solve~\eqref{eq:TR_Fl}. The results are shown in Figure~\ref{fig:TR_nested}.
\begin{figure}
	\centering
    \includegraphics[width=.9\linewidth]{Fig/FT_exp_TR_synthetic.pdf}
    \caption{
    Illustration of the nestedness issue in linear discriminant analysis (trace ratio problem). Given a dataset with five clusters, we plot its projection onto the optimal 1D subspace and 2D subspace obtained by solving the associated Grassmannian optimization problem~\eqref{eq:TR_Gr} or flag optimization problem~\eqref{eq:TR_Fl}. 
    We can see that the Grassmann representations are not nested, while the flag representations are nested and well capture the distribution of clusters. In this example, it is less the nestedness than the \textit{rotation} of the optimal axes inside the 2D subspace that is critical to the analysis of the Grassmann-based method.
    }
	\label{fig:TR_nested}
\end{figure}
\begin{figure}
	\centering
    \includegraphics[width=.9\linewidth]{Fig/FT_exp_TR_digits.pdf}
    \caption{
    Illustration of the nestedness issue in linear discriminant analysis (trace ratio problem) on the digits dataset. For $q_k \in \qf := (1, 2, \dots, 63)$, we solve the Grassmannian optimization problem~\eqref{eq:TR_Gr} on $\Gr(64, q_k)$ and plot the subspace angles $\Theta(\S_k^*, \S_{k+1}^*)$ (left) and explained variances ${\operatorname{tr}(\Pi_{\S_k^*} X X\T)} / {\operatorname{tr}(X X\T)}$ (right) as a function of $k$. We compare those quantities to the ones obtained by solving the flag optimization problem~\eqref{eq:TR_Fl}. 
    We can see that the Grassmann-based method is highly non-nested and even yields an extremely paradoxical non-increasing explained variance (cf. red circle on the right).
    }
	\label{fig:TR_nested_digits}
\end{figure}
We can see that the Grassmann representations are non-nested while their flag counterparts perfectly capture the filtration of subspaces that best and best approximates the distribution while discriminating the classes. Even if the colors make us realize that the issue in this experiment for LDA  is not much about the non-nestedness but rather about the rotation of the principal axes within the 2D subspace, we still have an important issue of consistency.

Second, we consider the (full) handwritten digits dataset~\citep{alpaydin_optical_1998}. It contains $8 \times 8$ pixels images of handwritten digits, from $0$ to $9$, almost uniformly class-balanced. One has $n = 1797$, $p=64$ and $C = 10$.
We run a steepest descent algorithm to solve the trace ratio problem~\eqref{eq:TR_Fl}. We choose the full signature $q_{1:63} = (1, 2, \dots, 63)$ and compare the output flag to the individual subspaces output by running optimization on $\Gr(p, q_k)$ for $q_k \in q_{1:d}$.
We plot the subspace angles $\Theta(\S_k^*, \S_{k+1}^*)$ and the explained variance ${\operatorname{tr}(\Pi_{\S_k^*} X X\T)} / {\operatorname{tr}(X X\T)}$ as a function of the $k$. The results are illustrated in \autoref{fig:TR_nested_digits}.
We see that the subspace angles are always positive and even very large sometimes with the LDA. Worst, the explained variance is not monotonous. This implies that we sometimes \textit{loose} some information when \textit{increasing} the dimension, which is extremely paradoxical.

Third, we perform some classification experiments on the optimal subspaces for different datasets. For different datasets, we run the optimization problems on $\Fl(p, q_{1:d})$, then project the data onto the different subspaces in $\S_{1:d}^*$ and run a nearest neighbors classifier with $5$ neighbors.
The predictions are then ensembled (cf. Algorithm~\ref{alg:flag_trick}) by weighted averaging, either with uniform weights or with weights minimizing the average cross-entropy:
\begin{equation}\label{eq:soft_voting}
	w_1^*, \dots, w_d^* = \argmin{\substack{w_k \geq 0 \\ \sum_{k=1}^d w_k = 1}} - \frac 1 {n C} \sum_{i=1}^n \sum_{c=1}^C y_{ic} \ln\lrp{\sum_{k=1}^d w_k y_{kic}^*},
\end{equation}
where $y_{kic}^* \in [0, 1]$ is the predicted probability that $x_i \in \R^p$ belongs to class $c \in \{1 \dots C\}$, by the classifier $g_k^*$ that is trained on $Z_k := {U_k^*}\T X \in \R^{q_k \times n}$. One can show that the latter is a convex problem, which we optimize using the \href{https://www.cvxpy.org/index.html}{cvxpy} Python package~\citep{diamond2016cvxpy}.
We report the results in \autoref{tab:TR_classif}.
\begin{table}
  \caption{Classification results for the TR problem on real datasets. For each method (Gr: Grassmann optimization~\eqref{eq:TR_Gr}, Fl: flag optimization~\eqref{eq:TR_Gr}, Fl-U: flag optimization + uniform soft voting, Fl-W: flag optimization + optimal soft voting~\eqref{eq:soft_voting}), we give the cross-entropy of the projected-predictions with respect to the true labels.}
  \label{tab:TR_classif}
  \centering
  \begin{tabular}{ccccccccc}
    \toprule
    dataset & $n$ & $p$ & $q_{1:d}$ & Gr & Fl & Fl-U & Fl-W & weights\\
    \midrule
    breast & $569$ & $30$ & $(1, 2, 5)$ & $0.0986$ & $0.0978$ & $0.0942$ & $0.0915$ & $(0.754, 0, 0.246)$\\
    iris & $150$ & $4$ & $(1, 2, 3)$ & $0.0372$ & $0.0441$ & $0.0410$ & $0.0368$ & $(0.985, 0, 0.015)$\\
    wine & $178$ & $13$ & $(1, 2, 5)$ & $0.0897$ & $0.0800$ & $0.1503$ & $0.0677$ & $(0, 1, 0)$\\
    digits & $1797$ & $64$ & $(1, 2, 5, 10)$ & $0.4507$ & $0.4419$ & $0.5645$ & $0.4374$ & $(0, 0, 0.239, 0.761)$\\
    \bottomrule
  \end{tabular}
\end{table}
The first example tells us that the optimal $5D$ subspace obtained by Grassmann optimization less discriminates the classes than the $5D$ subspace from the optimal flag. This may show that the flag takes into account some lower dimensional variability that enables to better discriminate the classes. We can also see that the uniform averaging of the predictors at different dimensions improves the classification. Finally, the optimal weights improve even more the classification and tell that the best discrimination happens by taking a soft blend of classifier at dimensions $1$ and $5$. Similar kinds of analyses can be made for the other examples.

\subsubsection{Discussion on TR optimization and kernelization}
\paragraph{A Newton algorithm}
In all the experiments of this paper, we use a steepest descent method on flag manifolds (Algorithm~\ref{alg:GD}) to solve the flag problems.
However, for the specific problem of TR~\eqref{eq:TR_Fl}, we believe that more adapted algorithms should be derived to take into account the specific form of the objective function, which is classically solved via a Newton-Lanczos method~\citep{ngo_trace_2012}. 
To that extent, we develop in the appendix (\autoref{app:TR}) an extension of the baseline Newton-Lanczos algorithm for the flag-tricked problem~\eqref{eq:TR_Fl}.
In short, the latter can be reformulated as a penalized optimization problem of the form $\operatorname{argmax}_{\Sf\in\Fl(p, \qf)} {\sum_{k=1}^d \tr{\Pi_{\S_k} (A - \rho B)}}$, where $\rho$ is updated iteratively according to a Newton scheme. Once again, our central Theorem~\ref{thm:flag_trick} enables to get explicit expressions for the iterations, which results without much difficulties in a Newton method, that is known to be much more efficient than first-order methods like the steepest descent.

\paragraph{A non-linearization via the kernel trick}
The classical trace ratio problems look for \textit{linear} embeddings of the data.
However, in most cases, the data follow a \textit{nonlinear} distribution, which may cause linear dimension reduction methods to fail. The \textit{kernel trick}~\citep{hofmann_kernel_2008} is a well-known method to embed nonlinear data into a linear space and fit linear machine learning methods.
As a consequence, we propose in appendix (\autoref{app:TR}) a kernelization of the trace ratio problem~\eqref{eq:TR_Fl} in the same fashion as the one of the seminal graph embedding work~\citep{yan_graph_2007}.
This is expected to yield much better embedding and classification results.
\section{Spectral clustering: extensions and proofs}\label{app:SSC}


\subsection{Proof of Proposition~\ref{prop:SSC}}
Let $\Sf \in \Fl(p, \qf)$ and $U_{1:d+1} := [U_1|U_2|\dots|U_d|U_{d+1}] \in \O(p)$ be an orthogonal representative of $\Sf$. One has:
\begin{align}
	\langle \Pi_{\S_{1:d}}, L\rangle_F + \beta \norm{\Pi_{\S_{1:d}}}_1 &= \left\langle \frac1d\sum_{k=1}^d \Pi_{\S_k}, L\right\rangle_F + \beta \norm{\frac1d\sum_{k=1}^d \Pi_{\S_k}}_1,\\
	&= \frac1d \lrp{\left\langle \sum_{k=1}^{d+1} (d - (k-1)) U_k {U_k}\T, L\right\rangle_F + \beta \norm{\sum_{k=1}^{d+1} (d - (k-1)) U_k {U_k}\T }_1},\\
	&= \frac1d \lrp{\sum_{k=1}^{d+1} (d - (k-1)) \left\langle U_k {U_k}\T, L\right\rangle_F + \beta \norm{\sum_{k=1}^{d+1} (d - (k-1)) U_k {U_k}\T }_1},\\
	\langle \Pi_{\S_{1:d}}, L\rangle_F + \beta \norm{\Pi_{\S_{1:d}}}_1 &= \frac1d \lrp{\sum_{k=1}^{d+1} (d - (k-1)) \tr{{U_k}\T L U_k} + \beta \norm{\sum_{k=1}^{d+1} (d - (k-1)) U_k {U_k}\T }_1},
\end{align}
which concludes the proof.
\subsection{The flag trick for other machine learning problems}
Subspace learning finds many applications beyond robust subspace recovery, trace ratio and spectral clustering problems, as evoked in~\autoref{sec:intro}. The goal of this subsection is to provide a few more examples in brief, without experiments.


\subsubsection{Domain adaptation}
In machine learning, it is often assumed that the training and test datasets follow the same distribution. However, some \textit{domain shift} issues---where training and test distributions are different---might arise, notably if the test data has been acquired from a different source (for instance a professional camera and a phone camera) or if the training data has been acquired a long time ago. \textit{Domain adaptation} is an area of machine learning that deals with domain shifts, usually by matching the training and test distributions---often referred to as \textit{source} and \textit{target} distributions---before fitting a classical model~\citep{farahani_brief_2021}. 
A large body of works (called ``subspace-based'') learn some intermediary subspaces between the source and target data, and perform the inference for the projected target data on these subspaces. The \textit{sampling geodesic flow}~\citep{gopalan_domain_2011} first performs a geodesic interpolation on Grassmannians between the source and target subspaces, then projects both datasets on (a discrete subset of) the interpolated subspaces, which results in a new representation of the data distributions, that can then be given as an input to a machine learning model. The higher the number of intermediary subspaces, the better the approximation, but the larger the dimension of the representation.
The celebrated \textit{geodesic flow kernel}~\citep{boqing_gong_geodesic_2012} circumvents this issue by integrating the projected data onto the continuum of interpolated subspaces. This yields an inner product between infinite-dimensional embeddings that can be computed explicitly and incorporated in a kernel method for learning. The \textit{domain invariant projection}~\citep{baktashmotlagh_unsupervised_2013} learns a \textit{domain-invariant} subspace that minimizes the maximum mean discrepancy (MMD)~\citep{gretton_kernel_2012} between the projected source $X_s := [x_{s1}|\dots|x_{s n_s}] \in \R^{p\times n_s}$ and target distributions $X_t := [x_{t1}|\dots|x_{t n_t}] \in \R^{p\times n_t}$:
\begin{equation}
	\argmin{U \in \St(p, q)} \operatorname{MMD}^2(U\T X_{s}, U\T X_{t}),
\end{equation}
where 
\begin{equation}
	\operatorname{MMD} (X, Y) = \norm{\frac 1 n \sum_{i=1}^n \phi (x_i) - \frac 1 m \sum_{i=1}^m \phi (y_i)}_\mathcal{H}.
\end{equation}
This can be rewritten, using the Gaussian kernel function $\phi(x)\colon y \mapsto \exp\lrp{-\frac{x\T y}{2\sigma^2}}$, as
\begin{multline}\label{eq:DIP}
	\argmin{\S \in \Gr(p, q)} 
	\frac 1 {n_s^2} \sum_{i,j=1}^{n_s} \exp\lrp{-\frac{(x_{si} - x_{sj})\T \Pi_\S (x_{si} - x_{sj})}{2 \sigma^2}}\\
	+ \frac 1 {n_t^2} \sum_{i,j=1}^{n_t} \exp\lrp{-\frac{(x_{ti} - x_{tj})\T \Pi_\S (x_{ti} - x_{tj})}{2 \sigma^2}}\\
	- \frac 2 {n_s n_t} \sum_{i=1}^{n_s} \sum_{j=1}^{n_t} \exp\lrp{-\frac{(x_{si} - x_{tj})\T \Pi_\S (x_{si} - x_{tj})}{2 \sigma^2}}.
\end{multline}
The flag trick applied to the domain invariant projection problem~\eqref{eq:DIP} yields:
\begin{multline}
	\argmin{\S_{1:d} \in \Fl(p, q_{1:d})} 
	\frac 1 {n_s^2} \sum_{i,j=1}^{n_s} \exp\lrp{-\frac{(x_{si} - x_{sj})\T \Pi_{\S_{1:d}} (x_{si} - x_{sj})}{2 \sigma^2}}\\
	+ \frac 1 {n_t^2} \sum_{i,j=1}^{n_t} \exp\lrp{-\frac{(x_{ti} - x_{tj})\T \Pi_{\S_{1:d}} (x_{ti} - x_{tj})}{2 \sigma^2}}\\
	- \frac 2 {n_s n_t} \sum_{i=1}^{n_s} \sum_{j=1}^{n_t} \exp\lrp{-\frac{(x_{si} - x_{tj})\T \Pi_{\S_{1:d}} (x_{si} - x_{tj})}{2 \sigma^2}},
\end{multline}
and can be rewritten as:
\begin{multline}
	\argmin{U_{1:d} \in \St(p, q)}
	\frac 1 {{n_s}^2} \sum_{i,j=1}^{n_s} \exp\lrp{-\sum_{k=1}^d \frac{d+1-k}{d} \frac{\norm{{U_k}\T (x_{si} - x_{sj})}_2^2}{2 \sigma^2}}\\
	+ \frac 1 {{n_t}^2} \sum_{i,j=1}^{n_t} \exp\lrp{-\sum_{k=1}^d \frac{d+1-k}{d} \frac{\norm{{U_k}\T (x_{ti} - x_{tj})}_2^2}{2 \sigma^2}}\\
	- \frac 2 {{n_s} {n_t}} \sum_{i=1}^{n_s} \sum_{j=1}^{n_t} \exp\lrp{-\sum_{k=1}^d \frac{d+1-k}{d} \frac{\norm{{U_k}\T (x_{si} - x_{tj})}_2^2}{2 \sigma^2}}.
\end{multline}
Some experiments similar to the ones of~\citet{baktashmotlagh_unsupervised_2013} can be performed. For instance, one can consider the benchmark visual object recognition dataset of~\citet{saenko_adapting_2010}, learn nested domain invariant projections, fit some support vector machines to the projected source samples at increasing dimensions, and then perform soft-voting ensembling by learning the optimal weights on the target data according to Equation~\eqref{eq:soft_voting}.

\subsubsection{Low-rank decomposition}
Many machine learning methods involve finding low-rank representations of a data matrix. 

This is the case of \textit{matrix completion}~\citep{candes_exact_2012} problems where one looks for a low-rank representation of an incomplete data matrix by minimizing the discrepancy with the observed entries, and which finds many applications including the well-known \href{https://en.wikipedia.org/wiki/Netflix_Prize}{Netflix problem}. Although its most-known formulation is as a convex relaxation, it can also be formulated as an optimization problem on Grassmann manifolds~\citep{keshavan_matrix_2010,boumal_rtrmc_2011} to avoid optimizing the nuclear norm in the full space which can be of high dimension. The intuition is that a low-dimensional point can be described by the subspace it belongs to and its coordinates within this subspace. More precisely, the SVD-based low-rank factorization $M = UW$, with $M \in \R^{p \times n}$, $U \in \St(p, q)$ and $W \in \R^{q \times n}$ is orthogonally-invariant---in the sense that for any $R\in\O(q)$, one has $(UR) (R\T W) = U W$. One could therefore apply the flag trick to such problems, with the intuition that we would try low-rank matrix decompositions at different dimensions. The application of the flag trick would however not be as straightforward as in the previous problems since the subspace-projection matrices $\Pi_\S := U U\T$ do not appear explicitly, and since the coefficient matrix $W$ also depends on the dimension $q$.

Many other low-rank problems can be formulated as a Grassmannian optimization. \textit{Robust PCA}~\citep{candes_robust_2011} looks for a low rank + sparse corruption factorization of a data matrix. \textit{Subspace Tracking}~\citep{balzano_online_2010} incrementally updates a subspace from streaming and highly-incomplete observations via small steps on Grassmann manifolds.

\subsubsection{Linear dimensionality reduction}
Finally, many other general dimension reduction algorithms---referred to as \textit{linear dimensionality reduction methods}~\citep{cunningham_linear_2015}---involve optimization on Grassmannians. For instance, linear dimensionality reduction encompasses the already-discussed PCA and LDA, but also many other problems like \textit{multi-dimensional scaling}~\citep{torgerson_multidimensional_1952}, \textit{slow feature analysis}~\citep{wiskott_slow_2002}, \textit{locality preserving projections}~\citep{he_locality_2003} and \textit{factor analysis}~\citep{spearman_general_1904}.
\section{ Task Generalization Beyond i.i.d. Sampling and Parity Functions
}\label{sec:Discussion}
% Discussion: From Theory to Beyond
% \misha{what is beyond?}
% \amir{we mean two things: in the first subsection beyond i.i.d subsampling of parity tasks and in the second subsection beyond parity task}
% \misha{it has to be beyond something, otherwise it is not clear what it is about} \hz{this is suggested by GPT..., maybe can be interpreted as from theory to beyond theory. We can do explicit like Discussion: Beyond i.i.d. task sampling and the Parity Task}
% \misha{ why is "discussion" in the title?}\amir{Because it is a discussion, it is not like separate concrete explnation about why these thing happens or when they happen, they just discuss some interesting scenraios how it relates to our theory.   } \misha{it is not really a discussion -- there is a bunch of experiments}

In this section, we extend our experiments beyond i.i.d. task sampling and parity functions. We show an adversarial example where biased task selection substantially hinders task generalization for sparse parity problem. In addition, we demonstrate that exponential task scaling extends to a non-parity tasks including arithmetic and multi-step language translation.

% In this section, we extend our experiments beyond i.i.d. task sampling and parity functions. On the one hand, we find that biased task selection can significantly degrade task generalization; on the other hand, we show that exponential task scaling generalizes to broader scenarios.
% \misha{we should add a sentence or two giving more detail}


% 1. beyond i.i.d tasks sampling
% 2. beyond parity -> language, arithmetic -> task dependency + implicit bias of transformer (cannot implement this algorithm for arithmatic)



% In this section, we emphasize the challenge of quantifying the level of out-of-distribution (OOD) differences between training tasks and testing tasks, even for a simple parity task. To illustrate this, we present two scenarios where tasks differ between training and testing. For each scenario, we invite the reader to assess, before examining the experimental results, which cases might appear “more” OOD. All scenarios consider \( d = 10 \). \kaiyue{this sentence should be put into 5.1}






% for parity problem




% \begin{table*}[th!]
%     \centering
%     \caption{Generalization Results for Scenarios 1 and 2 for $d=10$.}
%     \begin{tabular}{|c|c|c|c|}
%         \hline
%         \textbf{Scenario} & \textbf{Type/Variation} & \textbf{Coordinates} & \textbf{Generalization accuracy} \\
%         \hline
%         \multirow{3}{*}{Generalization with Missing Pair} & Type 1 & \( c_1 = 4, c_2 = 6 \) & 47.8\%\\ 
%         & Type 2 & \( c_1 = 4, c_2 = 6 \) & 96.1\%\\ 
%         & Type 3 & \( c_1 = 4, c_2 = 6 \) & 99.5\%\\ 
%         \hline
%         \multirow{3}{*}{Generalization with Missing Pair} & Type 1 &  \( c_1 = 8, c_2 = 9 \) & 40.4\%\\ 
%         & Type 2 & \( c_1 = 8, c_2 = 9 \) & 84.6\% \\ 
%         & Type 3 & \( c_1 = 8, c_2 = 9 \) & 99.1\%\\ 
%         \hline
%         \multirow{1}{*}{Generalization with Missing Coordinate} & --- & \( c_1 = 5 \) & 45.6\% \\ 
%         \hline
%     \end{tabular}
%     \label{tab:generalization_results}
% \end{table*}

\subsection{Task Generalization Beyond i.i.d. Task Sampling }\label{sec: Experiment beyond iid sampling}

% \begin{table*}[ht!]
%     \centering
%     \caption{Generalization Results for Scenarios 1 and 2 for $d=10, k=3$.}
%     \begin{tabular}{|c|c|c|}
%         \hline
%         \textbf{Scenario}  & \textbf{Tasks excluded from training} & \textbf{Generalization accuracy} \\
%         \hline
%         \multirow{1}{*}{Generalization with Missing Pair}
%         & $\{4,6\} \subseteq \{s_1, s_2, s_3\}$ & 96.2\%\\ 
%         \hline
%         \multirow{1}{*}{Generalization with Missing Coordinate}
%         & \( s_2 = 5 \) & 45.6\% \\ 
%         \hline
%     \end{tabular}
%     \label{tab:generalization_results}
% \end{table*}




In previous sections, we focused on \textit{i.i.d. settings}, where the set of training tasks $\mathcal{F}_{train}$ were sampled uniformly at random from the entire class $\mathcal{F}$. Here, we explore scenarios that deliberately break this uniformity to examine the effect of task selection on out-of-distribution (OOD) generalization.\\

\textit{How does the selection of training tasks influence a model’s ability to generalize to unseen tasks? Can we predict which setups are more prone to failure?}\\

\noindent To investigate this, we consider two cases parity problems with \( d = 10 \) and \( k = 3 \), where each task is represented by its tuple of secret indices \( (s_1, s_2, s_3) \):

\begin{enumerate}[leftmargin=0.4 cm]
    \item \textbf{Generalization with a Missing Coordinate.} In this setup, we exclude all training tasks where the second coordinate takes the value \( s_2 = 5 \), such as \( (1,5,7) \). At test time, we evaluate whether the model can generalize to unseen tasks where \( s_2 = 5 \) appears.
    \item \textbf{Generalization with Missing Pair.} Here, we remove all training tasks that contain both \( 4 \) \textit{and} \( 6 \) in the tuple \( (s_1, s_2, s_3) \), such as \( (2,4,6) \) and \( (4,5,6) \). At test time, we assess whether the model can generalize to tasks where both \( 4 \) and \( 6 \) appear together.
\end{enumerate}

% \textbf{Before proceeding, consider the following question:} 
\noindent \textbf{If you had to guess.} Which scenario is more challenging for generalization to unseen tasks? We provide the experimental result in Table~\ref{tab:generalization_results}.

 % while the model struggles for one of them while as it generalizes almost perfectly in the other one. 

% in the first scenario, it generalizes almost perfectly in the second. This highlights how exposure to partial task structures can enhance generalization, even when certain combinations are entirely absent from the training set. 

In the first scenario, despite being trained on all tasks except those where \( s_2 = 5 \), which is of size $O(\d^T)$, the model struggles to generalize to these excluded cases, with prediction at chance level. This is intriguing as one may expect model to generalize across position. The failure  suggests that positional diversity plays a crucial role in the task generalization of Transformers. 

In contrast, in the second scenario, though the model has never seen tasks with both \( 4 \) \textit{and} \( 6 \) together, it has encountered individual instances where \( 4 \) appears in the second position (e.g., \( (1,4,5) \)) or where \( 6 \) appears in the third position (e.g., \( (2,3,6) \)). This exposure appears to facilitate generalization to test cases where both \( 4 \) \textit{and} \( 6 \) are present. 



\begin{table*}[t!]
    \centering
    \caption{Generalization Results for Scenarios 1 and 2 for $d=10, k=3$.}
    \resizebox{\textwidth}{!}{  % Scale to full width
        \begin{tabular}{|c|c|c|}
            \hline
            \textbf{Scenario}  & \textbf{Tasks excluded from training} & \textbf{Generalization accuracy} \\
            \hline
            Generalization with Missing Pair & $\{4,6\} \subseteq \{s_1, s_2, s_3\}$ & 96.2\%\\ 
            \hline
            Generalization with Missing Coordinate & \( s_2 = 5 \) & 45.6\% \\ 
            \hline
        \end{tabular}
    }
    \label{tab:generalization_results}
\end{table*}

As a result, when the training tasks are not i.i.d, an adversarial selection such as exclusion of specific positional configurations may lead to failure to unseen task generalization even though the size of $\mathcal{F}_{train}$ is exponentially large. 


% \paragraph{\textbf{Key Takeaways}}
% \begin{itemize}
%     \item Out-of-distribution generalization in the parity problem is highly sensitive to the diversity and positional coverage of training tasks.
%     \item Adversarial exclusion of specific pairs or positional configurations can lead to systematic failures, even when most tasks are observed during training.
% \end{itemize}




%################ previous veriosn down
% \textit{How does the choice of training tasks affect the ability of a model to generalize to unseen tasks? Can we predict which setups are likely to lead to failure?}

% To explore these questions, we crafted specific training and test task splits to investigate what makes one setup appear “more” OOD than another.

% \paragraph{Generalization with Missing Pair.}

% Imagine we have tasks constructed from subsets of \(k=3\) elements out of a larger set of \(d\) coordinates. What happens if certain pairs of coordinates are adversarially excluded during training? For example, suppose \(d=5\) and two specific coordinates, \(c_1 = 1\) and \(c_2 = 2\), are excluded. The remaining tasks are formed from subsets of the other coordinates. How would a model perform when tested on tasks involving the excluded pair \( (c_1, c_2) \)? 

% To probe this, we devised three variations in how training tasks are constructed:
%     \begin{enumerate}
%         \item \textbf{Type 1:} The training set includes all tasks except those containing both \( c_1 = 1 \) and \( c_2 = 2 \). 
%         For this example, the training set includes only $\{(3,4,5)\}$. The test set consists of all tasks containing the rest of tuples.

%         \item \textbf{Type 2:} Similar to Type 1, but the training set additionally includes half of the tasks containing either \( c_1 = 1 \) \textit{or} \( c_2 = 2 \) (but not both). 
%         For the example, the training set includes all tasks from Type 1 and adds tasks like \(\{(1, 3, 4), (2, 3, 5)\}\) (half of those containing \( c_1 = 1 \) or \( c_2 = 2 \)).

%         \item \textbf{Type 3:} Similar to Type 2, but the training set also includes half of the tasks containing both \( c_1 = 1 \) \textit{and} \( c_2 = 2 \). 
%         For the example, the training set includes all tasks from Type 2 and adds, for instance, \(\{(1, 2, 5)\}\) (half of the tasks containing both \( c_1 \) and \( c_2 \)).
%     \end{enumerate}

% By systematically increasing the diversity of training tasks in a controlled way, while ensuring no overlap between training and test configurations, we observe an improvement in OOD generalization. 

% % \textit{However, the question is this improvement similar across all coordinate pairs, or does it depend on the specific choices of \(c_1\) and \(c_2\) in the tasks?} 

% \textbf{Before proceeding, consider the following question:} Is the observed improvement consistent across all coordinate pairs, or does it depend on the specific choices of \(c_1\) and \(c_2\) in the tasks? 

% For instance, consider two cases for \(d = 10, k = 3\): (i) \(c_1 = 4, c_2 = 6\) and (ii) \(c_1 = 8, c_2 = 9\). Would you expect similar OOD generalization behavior for these two cases across the three training setups we discussed?



% \paragraph{Answer to the Question.} for both cases of \( c_1, c_2 \), we observe that generalization fails in Type 1, suggesting that the position of the tasks the model has been trained on significantly impacts its generalization capability. For Type 2, we find that \( c_1 = 4, c_2 = 6 \) performs significantly better than \( c_1 = 8, c_2 = 9 \). 

% Upon examining the tasks where the transformer fails for \( c_1 = 8, c_2 = 9 \), we see that the model only fails at tasks of the form \((*, 8, 9)\) while perfectly generalizing to the rest. This indicates that the model has never encountered the value \( 8 \) in the second position during training, which likely explains its failure to generalize. In contrast, for \( c_1 = 4, c_2 = 6 \), while the model has not seen tasks of the form \((*, 4, 6)\), it has encountered tasks where \( 4 \) appears in the second position, such as \((1, 4, 5)\), and tasks where \( 6 \) appears in the third position, such as \((2, 3, 6)\). This difference may explain why the model generalizes almost perfectly in Type 2 for \( c_1 = 4, c_2 = 6 \), but not for \( c_1 = 8, c_2 = 9 \).



% \paragraph{Generalization with Missing Coordinates.}
% Next, we investigate whether a model can generalize to tasks where a specific coordinate appears in an unseen position during training. For instance, consider \( c_1 = 5 \), and exclude all tasks where \( c_1 \) appears in the second position. Despite being trained on all other tasks, the model fails to generalize to these excluded cases, highlighting the importance of positional diversity in training tasks.



% \paragraph{Key Takeaways.}
% \begin{itemize}
%     \item OOD generalization depends heavily on the diversity and positional coverage of training tasks for the parity problem.
%     \item adversarial exclusion of specific pairs or positional configurations in the parity problem can lead to failure, even when the majority of tasks are observed during training.
% \end{itemize}


%################ previous veriosn up

% \paragraph{Key Takeaways} These findings highlight the complexity of OOD generalization, even in seemingly simple tasks like parity. They also underscore the importance of task design: the diversity of training tasks can significantly influence a model’s ability to generalize to unseen tasks. By better understanding these dynamics, we can design more robust training regimes that foster generalization across a wider range of scenarios.


% #############


% Upon examining the tasks where the transformer fails for \( c_1 = 8, c_2 = 9 \), we see that the model only fails at tasks of the form \((*, 8, 9)\) while perfectly generalizing to the rest. This indicates that the model has never encountered the value \( 8 \) in the second position during training, which likely explains its failure to generalize. In contrast, for \( c_1 = 4, c_2 = 6 \), while the model has not seen tasks of the form \((*, 4, 6)\), it has encountered tasks where \( 4 \) appears in the second position, such as \((1, 4, 5)\), and tasks where \( 6 \) appears in the third position, such as \((2, 3, 6)\). This difference may explain why the model generalizes almost perfectly in Type 2 for \( c_1 = 4, c_2 = 6 \), but not for \( c_1 = 8, c_2 = 9 \).

% we observe a striking pattern: generalization fails entirely in Type 1, regardless of the coordinate pair (\(c_1, c_2\)). However, in Type 2, generalization varies: \(c_1 = 4, c_2 = 6\) achieves 96\% accuracy, while \(c_1 = 8, c_2 = 9\) lags behind at 70\%. Why? Upon closer inspection, the model struggles specifically with tasks like \((*, 8, 9)\), where the combination \(c_1 = 8\) and \(c_2 = 9\) is entirely novel. In contrast, for \(c_1 = 4, c_2 = 6\), the model benefits from having seen tasks where \(4\) appears in the second position or \(6\) in the third. This suggests that positional exposure during training plays a key role in generalization.

% To test whether task structure influences generalization, we consider two variations:
% \begin{enumerate}
%     \item \textbf{Sorted Tuples:} Tasks are always sorted in ascending order.
%     \item \textbf{Unsorted Tuples:} Tasks can appear in any order.
% \end{enumerate}

% If the model struggles with generalizing to the excluded position, does introducing variability through unsorted tuples help mitigate this limitation?

% \paragraph{Discussion of Results}

% In \textbf{Generalization with Missing Pairs}, we observe a striking pattern: generalization fails entirely in Type 1, regardless of the coordinate pair (\(c_1, c_2\)). However, in Type 2, generalization varies: \(c_1 = 4, c_2 = 6\) achieves 96\% accuracy, while \(c_1 = 8, c_2 = 9\) lags behind at 70\%. Why? Upon closer inspection, the model struggles specifically with tasks like \((*, 8, 9)\), where the combination \(c_1 = 8\) and \(c_2 = 9\) is entirely novel. In contrast, for \(c_1 = 4, c_2 = 6\), the model benefits from having seen tasks where \(4\) appears in the second position or \(6\) in the third. This suggests that positional exposure during training plays a key role in generalization.

% In \textbf{Generalization with Missing Coordinates}, the results confirm this hypothesis. When \(c_1 = 5\) is excluded from the second position during training, the model fails to generalize to such tasks in the sorted case. However, allowing unsorted tuples introduces positional diversity, leading to near-perfect generalization. This raises an intriguing question: does the model inherently overfit to positional patterns, and can task variability help break this tendency?




% In this subsection, we show that the selection of training tasks can affect the quality of the unseen task generalization significantly in practice. To illustrate this, we present two scenarios where tasks differ between training and testing. For each scenario, we invite the reader to assess, before examining the experimental results, which cases might appear “more” OOD. 

% % \amir{add examples, }

% \kaiyue{I think the name of scenarios here are not very clear}
% \begin{itemize}
%     \item \textbf{Scenario 1:  Generalization Across Excluded Coordinate Pairs (\( k = 3 \))} \\
%     In this scenario, we select two coordinates \( c_1 \) and \( c_2 \) out of \( d \) and construct three types of training sets. 

%     Suppose \( d = 5 \), \( c_1 = 1 \), and \( c_2 = 2 \). The tuples are all possible subsets of \( \{1, 2, 3, 4, 5\} \) with \( k = 3 \):
%     \[
%     \begin{aligned}
%     \big\{ & (1, 2, 3), (1, 2, 4), (1, 2, 5), (1, 3, 4), (1, 3, 5), \\
%            & (1, 4, 5), (2, 3, 4), (2, 3, 5), (2, 4, 5), (3, 4, 5) \big\}.
%     \end{aligned}
%     \]

%     \begin{enumerate}
%         \item \textbf{Type 1:} The training set includes all tuples except those containing both \( c_1 = 1 \) and \( c_2 = 2 \). 
%         For this example, the training set includes only $\{(3,4,5)\}$ tuple. The test set consists of tuples containing the rest of tuples.

%         \item \textbf{Type 2:} Similar to Type 1, but the training set additionally includes half of the tuples containing either \( c_1 = 1 \) \textit{or} \( c_2 = 2 \) (but not both). 
%         For the example, the training set includes all tuples from Type 1 and adds tuples like \(\{(1, 3, 4), (2, 3, 5)\}\) (half of those containing \( c_1 = 1 \) or \( c_2 = 2 \)).

%         \item \textbf{Type 3:} Similar to Type 2, but the training set also includes half of the tuples containing both \( c_1 = 1 \) \textit{and} \( c_2 = 2 \). 
%         For the example, the training set includes all tuples from Type 2 and adds, for instance, \(\{(1, 2, 5)\}\) (half of the tuples containing both \( c_1 \) and \( c_2 \)).
%     \end{enumerate}

% % \begin{itemize}
% %     \item \textbf{Type 1:} The training set includes tuples \(\{1, 3, 4\}, \{2, 3, 4\}\) (excluding tuples with both \( c_1 \) and \( c_2 \): \(\{1, 2, 3\}, \{1, 2, 4\}\)). The test set contains the excluded tuples.
% %     \item \textbf{Type 2:} The training set includes all tuples in Type 1 plus half of the tuples containing either \( c_1 = 1 \) or \( c_2 = 2 \) (e.g., \(\{1, 2, 3\}\)).
% %     \item \textbf{Type 3:} The training set includes all tuples in Type 2 plus half of the tuples containing both \( c_1 = 1 \) and \( c_2 = 2 \) (e.g., \(\{1, 2, 4\}\)).
% % \end{itemize}
    
%     \item \textbf{Scenario 2: Scenario 2: Generalization Across a Fixed Coordinate (\( k = 3 \))} \\
%     In this scenario, we select one coordinate \( c_1 \) out of \( d \) (\( c_1 = 5 \)). The training set includes all task tuples except those where \( c_1 \) is the second coordinate of the tuple. For this scenario, we examine two variations:
%     \begin{enumerate}
%         \item \textbf{Sorted Tuples:} Task tuples are always sorted (e.g., \( (x_1, x_2, x_3) \) with \( x_1 \leq x_2 \leq x_3 \)).
%         \item \textbf{Unsorted Tuples:} Task tuples can appear in any order.
%     \end{enumerate}
% \end{itemize}




% \paragraph{Discussion of Results.} In the first scenario, for both cases of \( c_1, c_2 \), we observe that generalization fails in Type 1, suggesting that the position of the tasks the model has been trained on significantly impacts its generalization capability. For Type 2, we find that \( c_1 = 4, c_2 = 6 \) performs significantly better than \( c_1 = 8, c_2 = 9 \). 

% Upon examining the tasks where the transformer fails for \( c_1 = 8, c_2 = 9 \), we see that the model only fails at tasks of the form \((*, 8, 9)\) while perfectly generalizing to the rest. This indicates that the model has never encountered the value \( 8 \) in the second position during training, which likely explains its failure to generalize. In contrast, for \( c_1 = 4, c_2 = 6 \), while the model has not seen tasks of the form \((*, 4, 6)\), it has encountered tasks where \( 4 \) appears in the second position, such as \((1, 4, 5)\), and tasks where \( 6 \) appears in the third position, such as \((2, 3, 6)\). This difference may explain why the model generalizes almost perfectly in Type 2 for \( c_1 = 4, c_2 = 6 \), but not for \( c_1 = 8, c_2 = 9 \).

% This position-based explanation appears compelling, so in the second scenario, we focus on a single position to investigate further. Here, we find that the transformer fails to generalize to tasks where \( 5 \) appears in the second position, provided it has never seen any such tasks during training. However, when we allow for more task diversity in the unsorted case, the model achieves near-perfect generalization. 

% This raises an important question: does the transformer have a tendency to overfit to positional patterns, and does introducing more task variability, as in the unsorted case, prevent this overfitting and enable generalization to unseen positional configurations?

% These findings highlight that even in a simple task like parity, it is remarkably challenging to understand and quantify the sources and levels of OOD behavior. This motivates further investigation into the nuances of task design and its impact on model generalization.


\subsection{Task Generalization Beyond Parity Problems}

% \begin{figure}[t!]
%     \centering
%     \includegraphics[width=0.45\textwidth]{Figures/arithmetic_v1.png}
%     \vspace{-0.3cm}
%     \caption{Task generalization for arithmetic task with CoT, it has $\d =2$ and $T = d-1$ as the ambient dimension, hence $D\ln(DT) = 2\ln(2T)$. We show that the empirical scaling closely follows the theoretical scaling.}
%     \label{fig:arithmetic}
% \end{figure}



% \begin{wrapfigure}{r}{0.4\textwidth}  % 'r' for right, 'l' for left
%     \centering
%     \includegraphics[width=0.4\textwidth]{Figures/arithmetic_v1.png}
%     \vspace{-0.3cm}
%     \caption{Task generalization for the arithmetic task with CoT. It has $d =2$ and $T = d-1$ as the ambient dimension, hence $D\ln(DT) = 2\ln(2T)$. We show that the empirical scaling closely follows the theoretical scaling.}
%     \label{fig:arithmetic}
% \end{wrapfigure}

\subsubsection{Arithmetic Task}\label{subsec:arithmetic}











We introduce the family of \textit{Arithmetic} task that, like the sparse parity problem, operates on 
\( d \) binary inputs \( b_1, b_2, \dots, b_d \). The task involves computing a structured arithmetic expression over these inputs using a sequence of addition and multiplication operations.
\newcommand{\op}{\textrm{op}}

Formally, we define the function:
\[
\text{Arithmetic}_{S} \colon \{0,1\}^d \to \{0,1,\dots,d\},
\]
where \( S = (\op_1, \op_2, \dots, \op_{d-1}) \) is a sequence of \( d-1 \) operations, each \( \op_k \) chosen from \( \{+, \times\} \). The function evaluates the expression by applying the operations sequentially from left-to-right order: for example, if \( S = (+, \times, +) \), then the arithmetic function would compute:
\[
\text{Arithmetic}_{S}(b_1, b_2, b_3, b_4) = ((b_1 + b_2) \times b_3) + b_4.
\]
% Thus, the sequence of operations \( S \) defines how the binary inputs are combined to produce an integer output between \( 0 \) and \( d \).
% \[
% \text{Arithmetic}_{S} 
% (b_1,\,b_2,\,\dots,b_d)
% =
% \Bigl(\dots\bigl(\,(b_1 \;\op_1\; b_2)\;\op_2\; b_3\bigr)\,\dots\Bigr) 
% \;\op_{d-1}\; b_d.
% \]
% We now introduce an \emph{Arithmetic} task that, like the sparse parity problem, operates on $d$ binary inputs $b_1, b_2, \dots, b_d$. Specifically, we define an arithmetic function
% \[
% \text{Arithmetic}_{S}\colon \{0,1\}^d \;\to\; \{0,1,\dots,d\},
% \]
% where $S = (i_1, i_2, \dots, i_{d-1})$ is a sequence of $d-1$ operations, each $i_k \in \{+,\,\times\}$. The value of $\text{Arithmetic}_{S}$ is obtained by applying the prescribed addition and multiplication operations in order, namely:
% \[
% \text{Arithmetic}_{S}(b_1,\,b_2,\,\dots,b_d)
% \;=\;
% \Bigl(\dots\bigl(\,(b_1 \;i_1\; b_2)\;i_2\; b_3\bigr)\,\dots\Bigr) 
% \;i_{d-1}\; b_d.
% \]

% This is an example of our framework where $T = d-1$ and $|\Theta_t| = 2$ with total $2^d$ possible tasks. 




By introducing a step-by-step CoT, arithmetic class belongs to $ARC(2, d-1)$: this is because at every step, there is only $\d = |\Theta_t| = 2$ choices (either $+$ or $\times$) while the length is  $T = d-1$, resulting a total number of $2^{d-1}$ tasks. 


\begin{minipage}{0.5\textwidth}  % Left: Text
    Task generalization for the arithmetic task with CoT. It has $d =2$ and $T = d-1$ as the ambient dimension, hence $D\ln(DT) = 2\ln(2T)$. We show that the empirical scaling closely follows the theoretical scaling.
\end{minipage}
\hfill
\begin{minipage}{0.4\textwidth}  % Right: Image
    \centering
    \includegraphics[width=\textwidth]{Figures/arithmetic_v1.png}
    \refstepcounter{figure}  % Manually advances the figure counter
    \label{fig:arithmetic}  % Now this label correctly refers to the figure
\end{minipage}

Notably, when scaling with \( T \), we observe in the figure above that the task scaling closely follow the theoretical $O(D\log(DT))$ dependency. Given that the function class grows exponentially as \( 2^T \), it is truly remarkable that training on only a few hundred tasks enables generalization to an exponentially larger space—on the order of \( 2^{25} > 33 \) Million tasks. This exponential scaling highlights the efficiency of structured learning, where a modest number of training examples can yield vast generalization capability.





% Our theory suggests that only $\Tilde{O}(\ln(T))$ i.i.d training tasks is enough to generalize to the rest of unseen tasks. However, we show in Figure \ref{fig:arithmetic} that transformer is not able to match  that. The transformer out-of distribution generalization behavior is not consistent across different dimensions when we scale the number of training tasks with $\ln(T)$. \hongzhou{implicit bias, optimization, etc}
 






% \subsection{Task generalization Beyond parity problem}

% \subsection{Arithmetic} In this setting, we still use the set-up we introduced in \ref{subsec:parity_exmaple}, the input is still a set of $d$ binary variable, $b_1, b_2,\dots,b_d$ and ${Arithmatic_{S}}:\{0,1\}\rightarrow \{0, 1, \dots, d\}$, where $S = (i_1,i_2,\dots,i_{d-1})$ is a tuple of size $d-1$ where each coordinate is either add($+
% $) or multiplication ($\times$). The function is as following,

% \begin{align*}
%     Arithmatic_{S}(b_1, b_2,\dots,b_d) = (\dots(b1(i1)b2)(i3)b3\dots)(i{d-1})
% \end{align*}
    


\subsubsection{Multi-Step Language Translation Task}

 \begin{figure*}[h!]
    \centering
    \includegraphics[width=0.9\textwidth]{Figures/combined_plot_horiz.png}
    \vspace{-0.2cm}
    \caption{Task generalization for language translation task: $\d$ is the number of languages and $T$ is the length of steps.}
    \vspace{-2mm}
    \label{fig:language}
\end{figure*}
% \vspace{-2mm}

In this task, we study a sequential translation process across multiple languages~\cite{garg2022can}. Given a set of \( D \) languages, we construct a translation chain by randomly sampling a sequence of \( T \) languages \textbf{with replacement}:  \(L_1, L_2, \dots, L_T,\)
where each \( L_t \) is a sampled language. Starting with a word, we iteratively translate it through the sequence:
\vspace{-2mm}
\[
L_1 \to L_2 \to L_3 \to \dots \to L_T.
\]
For example, if the sampled sequence is EN → FR → DE → FR, translating the word "butterfly" follows:
\vspace{-1mm}
\[
\text{butterfly} \to \text{papillon} \to \text{schmetterling} \to \text{papillon}.
\]
This task follows an \textit{AutoRegressive Compositional} structure by itself, specifically \( ARC(D, T-1) \), where at each step, the conditional generation only depends on the target language, making \( D \) as the number of languages and the total number of possible tasks is \( D^{T-1} \). This example illustrates that autoregressive compositional structures naturally arise in real-world languages, even without explicit CoT. 

We examine task scaling along \( D \) (number of languages) and \( T \) (sequence length). As shown in Figure~\ref{fig:language}, empirical  \( D \)-scaling closely follows the theoretical \( O(D \ln D T) \). However, in the \( T \)-scaling case, we observe a linear dependency on \( T \) rather than the logarithmic dependency \(O(\ln T) \). A possible explanation is error accumulation across sequential steps—longer sequences require higher precision in intermediate steps to maintain accuracy. This contrasts with our theoretical analysis, which focuses on asymptotic scaling and does not explicitly account for compounding errors in finite-sample settings.

% We examine task scaling along \( D \) (number of languages) and \( T \) (sequence length). As shown in Figure~\ref{fig:language}, empirical scaling closely follows the theoretical \( O(D \ln D T) \) trend, with slight exceptions at $ T=10 \text{ and } 3$ in Panel B. One possible explanation for this deviation could be error accumulation across sequential steps—longer sequences require each intermediate translation to be approximated with higher precision to maintain test accuracy. This contrasts with our theoretical analysis, which primarily focuses on asymptotic scaling and does not explicitly account for compounding errors in finite-sample settings.

Despite this, the task scaling is still remarkable — training on a few hundred tasks enables generalization to   $4^{10} \approx 10^6$ tasks!






% , this case, we are in a regime where \( D \ll T \), we observe  that the task complexity empirically scales as \( T \log T \) rather than \( D \log T \). 


% the model generalizes to an exponentially larger space of \( 2^T \) unseen tasks. In case $T=25$, this is $2^{25} > 33$ Million tasks. This remarkable exponential generalization demonstrates the power of structured task composition in enabling efficient generalization.


% In the case of parity tasks, introducing CoT effectively decomposes the problem from \( ARC(D^T, 1) \) to \( ARC(D, T) \), significantly improving task generalization.

% Again, in the regime scaling $T$, we again observe a $T\log T$ dependency. Knowing that the function class is scaling as $D^T$, it is remarkable that training on a few hundreds tasks can generalize to $4^{10} \approx 1M$ tasks. 





% We further performed a preliminary investigation on a semi-synthetic word-level translation task to show that (1) task generalization via composition structure is feasible beyond parity and (2) understanding the fine-grained mechanism leading to this generalization is still challenging. 
% \noindent
% \noindent
% \begin{minipage}[t]{\columnwidth}
%     \centering
%     \textbf{\scriptsize In-context examples:}
%     \[
%     \begin{array}{rl}
%         \textbf{Input} & \hspace{1.5em} \textbf{Output} \\
%         \hline
%         \textcolor{blue}{car}   & \hspace{1.5em} \textcolor{red}{voiture \;,\; coche} \\
%         \textcolor{blue}{house} & \hspace{1.5em} \textcolor{red}{maison \;,\; casa} \\
%         \textcolor{blue}{dog}   & \hspace{1.5em} \textcolor{red}{chien \;,\; perro} 
%     \end{array}
%     \]
%     \textbf{\scriptsize Query:}
%     \[
%     \begin{array}{rl}
%         \textbf{Input} & \textbf{Output} \\
%         \hline
%         \textcolor{blue}{cat} & \hspace{1.5em} \textcolor{red}{?} \\
%     \end{array}
%     \]
% \end{minipage}



% \begin{figure}[h!]
%     \centering
%     \includegraphics[width=0.45\textwidth]{Figures/translation_scale_d.png}
%     \vspace{-0.2cm}
%     \caption{Task generalization behavior for word translation task.}
%     \label{fig:arithmetic}
% \end{figure}


\vspace{-1mm}
\section{Conclusions}
% \misha{is it conclusion of the section or of the whole paper?}    
% \amir{The whole paper. It is very short, do we need a separate section?}
% \misha{it should not be a subsection if it is the conclusion the whole thing. We can just remove it , it does not look informative} \hz{let's do it in a whole section, just to conclude and end the paper, even though it is not informative}
%     \kaiyue{Proposal: Talk about the implication of this result on theory development. For example, it calls for more fine-grained theoretical study in this space.  }

% \huaqing{Please feel free to edit it if you have better wording or suggestions.}

% In this work, we propose a theoretical framework to quantitatively investigate task generalization with compositional autoregressive tasks. We show that task to $D^T$ task is theoretically achievable by training on only $O (D\log DT)$ tasks, and empirically observe that transformers trained on parity problem indeed achieves such task generalization. However, for other tasks beyond parity, transformers seem to fail to achieve this bond. This calls for more fine-grained theoretical study the phenomenon of task generalization specific to transformer model. It may also be interesting to study task generalization beyond the setting of in-context learning. 
% \misha{what does this add?} \amir{It does not, i dont have any particular opinion to keep it. @Hongzhou if you want to add here?}\hz{While it may not introduce anything new, we are following a good practice to have a short conclusion. It provides a clear closing statement, reinforces key takeaways, and helps the reader leave with a well-framed understanding of our contributions. }
% In this work, we quantitatively investigate task generalization under autoregressive compositional structure. We demonstrate that task generalization to $D^T$ tasks is theoretically achievable by training on only $\tilde O(D)$ tasks. Empirically, we observe that transformers trained indeed achieve such exponential task generalization on problems such as parity, arithmetic and multi-step language translation. We believe our analysis opens up a new angle to understand the remarkable generalization ability of Transformer in practice. 

% However, for tasks beyond the parity problem, transformers appear to fail to reach this bound. This highlights the need for a more fine-grained theoretical exploration of task generalization, especially for transformer models. Additionally, it may be valuable to investigate task generalization beyond the scope of in-context learning.



In this work, we quantitatively investigated task generalization under the autoregressive compositional structure, demonstrating both theoretically and empirically that exponential task generalization to $D^T$ tasks can be achieved with training on only $\tilde{O}(D)$ tasks. %Our theoretical results establish a fundamental scaling law for task generalization, while our experiments validate these insights across problems such as parity, arithmetic, and multi-step language translation. The remarkable ability of transformers to generalize exponentially highlights the power of structured learning and provides a new perspective on how large language models extend their capabilities beyond seen tasks. 
We recap our key contributions  as follows:
\begin{itemize}
    \item \textbf{Theoretical Framework for Task Generalization.} We introduced the \emph{AutoRegressive Compositional} (ARC) framework to model structured task learning, demonstrating that a model trained on only $\tilde{O}(D)$ tasks can generalize to an exponentially large space of $D^T$ tasks.
    
    \item \textbf{Formal Sample Complexity Bound.} We established a fundamental scaling law that quantifies the number of tasks required for generalization, proving that exponential generalization is theoretically achievable with only a logarithmic increase in training samples.
    
    \item \textbf{Empirical Validation on Parity Functions.} We showed that Transformers struggle with standard in-context learning (ICL) on parity tasks but achieve exponential generalization when Chain-of-Thought (CoT) reasoning is introduced. Our results provide the first empirical demonstration of structured learning enabling efficient generalization in this setting.
    
    \item \textbf{Scaling Laws in Arithmetic and Language Translation.} Extending beyond parity functions, we demonstrated that the same compositional principles hold for arithmetic operations and multi-step language translation, confirming that structured learning significantly reduces the task complexity required for generalization.
    
    \item \textbf{Impact of Training Task Selection.} We analyzed how different task selection strategies affect generalization, showing that adversarially chosen training tasks can hinder generalization, while diverse training distributions promote robust learning across unseen tasks.
\end{itemize}



We introduce a framework for studying the role of compositionality in learning tasks and how this structure can significantly enhance generalization to unseen tasks. Additionally, we provide empirical evidence on learning tasks, such as the parity problem, demonstrating that transformers follow the scaling behavior predicted by our compositionality-based theory. Future research will  explore how these principles extend to real-world applications such as program synthesis, mathematical reasoning, and decision-making tasks. 


By establishing a principled framework for task generalization, our work advances the understanding of how models can learn efficiently beyond supervised training and adapt to new task distributions. We hope these insights will inspire further research into the mechanisms underlying task generalization and compositional generalization.

\section*{Acknowledgements}
We acknowledge support from the National Science Foundation (NSF) and the Simons Foundation for the Collaboration on the Theoretical Foundations of Deep Learning through awards DMS-2031883 and \#814639 as well as the  TILOS institute (NSF CCF-2112665) and the Office of Naval Research (ONR N000142412631). 
This work used the programs (1) XSEDE (Extreme science and engineering discovery environment)  which is supported by NSF grant numbers ACI-1548562, and (2) ACCESS (Advanced cyberinfrastructure coordination ecosystem: services \& support) which is supported by NSF grants numbers \#2138259, \#2138286, \#2138307, \#2137603, and \#2138296. Specifically, we used the resources from SDSC Expanse GPU compute nodes, and NCSA Delta system, via allocations TG-CIS220009. 



\acks{This work was supported by the ERC grant \#786854 G-Statistics from the European Research Council under the European Union’s Horizon 2020 research and innovation program and by the French government through the 3IA Côte d’Azur Investments ANR-19-P3IA-0002 managed by the National Research Agency.}

\newpage

\appendix
\section{Robust subspace recovery: extensions and proofs}\label{app:RSR}

\subsection{An IRLS algorithm for robust subspace recovery}
Iteratively reweighted least squares (IRLS) is a ubiquitous method to solve optimization problems involving $L^p$-norms. Motivated by the computation of the geometric median~\citep{weiszfeld_sur_1937}, IRLS is highly used to find robust maximum likelihood estimates of non-Gaussian probabilistic models (typically those containing outliers) and finds application in robust regression~\citep{huber_robust_1964}, sparse recovery~\citep{daubechies_iteratively_2010} etc.

The recent fast median subspace (FMS) algorithm~\citep{lerman_fast_2018}, achieving state-of-the-art results in RSR uses an IRLS scheme to optimize the Least Absolute Deviation (LAD)~\eqref{eq:RSR_Gr}.
The idea is to first rewrite the LAD as 
\begin{equation}
	\sum_{i=1}^n \norm{x_i - \Pi_{\S} x_i}_2 = \sum_{i=1}^n w_i(\S) \norm{x_i - \Pi_{\S} x_i}_2^2,
\end{equation}
with $w_i(\S) = \frac{1}{\norm{x_i - \Pi_{\S} x_i}_2}$, and then successively compute the weights $w_i$ and update the subspace according to the weighted objective.
More precisely, the FMS algorithm creates a sequence of subspaces $\S^1, \dots, \S^m$ such that 
\begin{equation}\label{eq:IRLS_FMF}
    \S^{t+1} = \argmin{\S \in \Gr(p, q)} \sum_{i=1}^n w_i(\S^t) \norm{x_i - \Pi_\S x_i}_2^2.
\end{equation}
This weighted least-squares problem enjoys a closed-form solution which relates to the eigenvalue decomposition of the weighted covariance matrix $\sum_{i=1}^n w_i(\S^t) x_i {x_i}\T$~\citep[Chapter~3.3]{vidal_generalized_2016}.

We wish to derive an IRLS algorithm for the flag-tricked version of the LAD minimization problem~\eqref{eq:RSR_Fl}.
In order to stay close in mind to the recent work of \citet{peng_convergence_2023} who proved convergence of a general class of IRLS algorithms under some mild assumptions, we first rewrite~\eqref{eq:RSR_Fl} as
\begin{equation}~\label{eq:RSR_Fl_IRLS}
    \argmin{\S_{1:d} \in\Fl(p, \qf)} \sum_{i=1}^n \rho(r(\S_{1:d}, x_i)),
\end{equation}
where $r(\S_{1:d}, x) = \norm{x - \Pi_{\Sf} x}_2$ is the \textit{residual} and $\rho(r) = |r|$ is the \textit{outlier-robust} loss function.
Following~\citet{peng_convergence_2023}, the IRLS scheme associated with~\eqref{eq:RSR_Fl_IRLS} is:
\begin{equation}
\begin{cases}
w_i^{t+1} = \rho'(r(\S_{1:d}^t, x_i)) /  r(\S_{1:d}^t, x_i) = 1 / \norm{x_i - \Pi_{\Sf} x_i}_2,\\
(\S_{1:d})^{t+1} = \argmin{\S_{1:d} \in\Fl(p, \qf)} \sum_{i=1}^n w_i^{t+1} \norm{x_i - \Pi_{\Sf} x_i}_2^2.
\end{cases}
\end{equation}
We now show that the second step enjoys a closed-form solution.
\begin{theorem}\label{thm:IRLS_FMF}
The RLS problem
\begin{equation}
    \argmin{\S_{1:d} \in\Fl(p, \qf)} \sum_{i=1}^n w_i \norm{x_i - \Pi_{\Sf} x_i}_2^2
\end{equation}
has a closed-form solution $\S_{1:d}^* \in\Fl(p, \qf)$, which is given by the eigenvalue decomposition of the weighted sample covariance matrix $S_w = \sum_{i=1}^n w_i x_i {x_i}\T = \sum_{j=1}^p \ell_j v_j {v_j}\T$, i.e.
\begin{equation}
    \S_k^* = \operatorname{Span}(v_1, \dots, v_{q_k}) \quad (k=1\twodots d).
\end{equation}
\end{theorem}
\begin{proof}
One has
\begin{equation}
	\sum_{i=1}^n w_i \norm{x_i - \Pi_{\Sf} x_i}_2^2 = \tr{(I - \Pi_{\Sf})^2 \lrp{\sum_{i=1}^n w_i x_i {x_i}\T}}.
\end{equation}
Therefore, we are exactly in the same case as in \autoref{thm:flag_trick}, if we replace $X X\T$ with the reweighted covariance matrix $\sum_{i=1}^n w_i x_i {x_i}\T$. This does not change the result, so we conclude with the end of the proof of \autoref{thm:flag_trick} (which itself relies on~\citet{szwagier_curse_2024}).
\end{proof}
Hence, one gets an IRLS scheme for the LAD minimization problem. 
One can modify the robust loss function $\rho(r) = |r|$ by a Huber-like loss function to avoid weight explosion. Indeed, one can show that the weight $w_i := 1 / \norm{x_i - \Pi_{\Sf} x_i}_2$ goes to infinity when the first subspace $\S_1$ of the flag gets close to $x_i$ .
Therefore in practice, we take 
\begin{equation}
    \rho(r) = 
        \begin{cases}
            r^2 / (2 p \delta) & \text{if } |r| <= p\delta,\\
            r - p \delta / 2 & \text{if } |r| > p\delta.
        \end{cases}
\end{equation}
This yields
\begin{equation}
    w_i = 1 / \max\lrp{p\delta,  1 / \norm{x_i - \Pi_{\Sf} x_i}_2}.
\end{equation}
The final proposed scheme is given in Algorithm~\ref{alg:FMF}, named \textit{fast median flag} (FMF), in reference to the fast median subspace algorithm of~\citet{lerman_fast_2018}.
\begin{algorithm}
\caption{Fast median flag}\label{alg:FMF}
\begin{algorithmic}
\Require $X\in \R^{p\times n}$ (data), $\quad q_1 < \dots < q_d$ (signature), $\quad t_{max}$ (max number of iterations), $\quad \eta$ (convergence threshold), $\quad \varepsilon$ (Huber-like saturation parameter)
\Ensure
$U \in \St(p, q)$
\State $t \gets 0, \quad \Delta \gets \infty, \quad U^0 \gets \operatorname{SVD}(X, q)$
\While{$\Delta > \eta$ and $t < t_{max}$}
    \State $t \gets t+1$
    \State $r_i \gets \norm{x_i - \Pi_{\Sf} x_i}_2$
    \State $y_i \gets {x_i} / {\max(\sqrt{r_i}, \varepsilon)}$
    \State $U^t \gets \operatorname{SVD(Y, q)}$
    \State $\Delta \gets \sqrt{\sum_{k=1}^{d} \Theta(U^t_{q_k}, U^{t-1}_{q_k})^2}$
\EndWhile
\end{algorithmic}
\end{algorithm}
We can easily check that FMF is a direct generalization of FMS for Grassmannians (i.e. when $d=1$).


\begin{remark}
This is far beyond the scope of the paper, but we believe that the convergence result of~\citet[Theorem~1]{peng_convergence_2023} could be generalized to the FMF algorithm, due to the compactness of flag manifolds and the expression of the residual function $r$.
\end{remark}

\subsection{Proof of Proposition~\ref{prop:RSR}}
Let $\Sf \in \Fl(p, \qf)$ and $U_{1:d+1} := [U_1|U_2|\dots|U_d|U_{d+1}] \in \O(p)$ be an orthogonal representative of $\Sf$. One has:
\begin{align}
	\norm{x_i - \Pi_{\S_{1:d}} x_i}_2 &= \sqrt{{(x_i - \Pi_{\S_{1:d}} x_i)}\T (x_i - \Pi_{\S_{1:d}} x_i)},\\
	 &= \sqrt{{x_i}\T {(I_p - \Pi_{\S_{1:d}})}^2 x_i},\\
	 &= \sqrt{{x_i}\T {\lrp{I_p - \frac1d \sum_{k=1}^d\Pi_{\S_k}}}^2 x_i},\\
 	 &= \sqrt{\frac1{d^2} {x_i}\T {\lrp{\sum_{k=1}^d (I_p - \Pi_{\S_k})}}^2 x_i},\\
 	 % &= \sqrt{\frac1{d^2} {x_i}\T U_{1:d+1} \diag{0, 1, \dots, d-1, d}^2 {U_{1:d+1}}\T  x_i},\\
 	 &= \sqrt{\frac1{d^2} {x_i}\T \lrp{\sum_{k=1}^{d+1} (k-1) U_k {U_k}\T}^2  x_i},\\
 	 &= \sqrt{\frac1{d^2} {x_i}\T \lrp{\sum_{k=1}^{d+1} (k-1)^2 U_k {U_k}\T}  x_i},\\
 	 &= \sqrt{\sum_{k=1}^{d+1} \lrp{\frac {k-1} {d}}^2 {x_i}\T \lrp{ U_k {U_k}\T}  x_i},\\
  	 \norm{x_i - \Pi_{\S_{1:d}} x_i}_2 &= \sqrt{\sum_{k=1}^{d+1} \lrp{\frac {k-1} {d}}^2 \norm{{U_k}\T x_i}_2^2},
\end{align}
which concludes the proof.
\subsection{The flag trick for trace ratio problems}\label{subsec:TR}
Trace ratio problems are ubiquitous in machine learning~\citep{ngo_trace_2012}. They write as:
\begin{equation}\label{eq:TR_St}
\argmax{U \in \St(p, q)} \frac{\tr{U\T A U}}{\tr{U\T B U}},
\end{equation}
where $A, B \in \R^{p\times p}$ are positive semi-definite matrices, with $\operatorname{rank}(B) > p - q$.

A famous example of TR problem is Fisher's linear discriminant analysis (LDA)~\citep{fisher_use_1936,belhumeur_eigenfaces_1997}.
It is common in machine learning to project the data onto a low-dimensional subspace before fitting a classifier, in order to circumvent the curse of dimensionality. It is well known that performing an unsupervised dimension reduction method like PCA comes with the risks of mixing up the classes, since the directions of maximal variance are not necessarily the most discriminating ones~\citep{chang_using_1983}. The goal of LDA is to use the knowledge of the data labels to learn a linear subspace that does not mix the classes.
Let $~{X := [x_1|\dots|x_n] \in \R^{p\times n}}$ be a dataset with labels $Y := [y_1|\dots|y_n] \in {[1, C]}^n$. Let $\mu = \frac{1}{n} \sum_{i=1}^n x_i$ be the dataset mean and $\mu_c = \frac{1}{\#\{i : y_i=c\}}\sum_{i : y_i=c} x_i$ be the class-wise means. 
The idea of LDA is to search for a subspace $\S \in \Gr(p, q)$ that simultaneously maximizes the projected \textit{between-class variance} $\sum_{c=1}^C \|\Pi_\S \mu_c - \Pi_\S \mu\|_2^2$ and minimizes the projected \textit{within-class variance} $\sum_{c=1}^C \sum_{i : y_i = c} \|\Pi_\S x_i - \Pi_\S \mu_c\|_2^2$. This can be reformulated as a trace ratio problem~\eqref{eq:TR_St}, with $A = \sum_{c=1}^C (\mu_c - \mu) (\mu_c - \mu)\T$ and $B = \sum_{c=1}^C \sum_{i : y_i = c} (x_i - \mu_c) (x_i - \mu_c)\T$.


More generally, a large family of dimension reduction methods can be reformulated as a TR problem. The seminal work of~\citet{yan_graph_2007} shows that many dimension reduction and manifold learning objective functions can be written as a trace ratio involving Laplacian matrices of attraction and repulsion graphs. Intuitively, those graphs determine which points should be close in the latent space and which ones should be far apart.
Other methods involving a ratio of traces are \textit{multi-view learning}~\citep{wang_trace_2023}, \textit{partial least squares} (PLS)~\citep{geladi_partial_1986,barker_partial_2003} and \textit{canonical correlation analysis} (CCA)~\citep{hardoon_canonical_2004}, although these methods are originally \textit{sequential} problems (cf. footnote~\ref{footnote:sequential}) and not \textit{subspace} problems.

Classical Newton-like algorithms for solving the TR problem~\eqref{eq:TR_St} come from the seminal works of~\citet{guo_generalized_2003, wang_trace_2007, jia_trace_2009}.
The interest of optimizing a trace-ratio instead of a ratio-trace (of the form $\tr{(U\T B U)^{-1}(U\T A U)}$), that enjoys an explicit solution given by a generalized eigenvalue decomposition, is also tackled in those papers. The \textit{repulsion Laplaceans}~\citep{kokiopoulou_enhanced_2009} instead propose to solve a regularized version $\tr{U\T B U} - \rho \tr{U\T A U}$, which enjoys a closed-form, but has a hyperparameter $\rho$, which is directly optimized in the classical Newton-like algorithms for trace ratio problems.

\subsubsection{Application of the flag trick to trace ratio problems}
The trace ratio problem~\eqref{eq:TR_St} can be straightforwardly reformulated as an optimization problem on Grassmannians, due to the orthogonal invariance of the objective function:
\begin{equation}\label{eq:TR_Gr}
\argmax{\S \in \Gr(p, q)} \frac{\tr{\Pi_\S A}}{\tr{\Pi_\S B}}.
\end{equation}
The following proposition applies the flag trick to the TR problem~\eqref{eq:TR_Gr}.
\begin{proposition}[Flag trick for TR]\label{prop:TR}
The flag trick applied to the TR problem~\eqref{eq:TR_Gr} reads
\begin{equation}\label{eq:TR_Fl}
	\argmax{\S_{1:d} \in \Fl(p, q_{1:d})} \frac{\tr{\Pi_{\S_{1:d}} A}}{\tr{\Pi_{\S_{1:d}} B}}.
\end{equation}
and is equivalent to the following optimization problem:
\begin{equation}\label{eq:TR_Fl_equiv}
\argmax{U_{1:d} \in \St(p, q)} \frac{\sum_{k=1}^{d} (d - (k-1)) \tr{{U_k}\T A {U_k}}}{\sum_{l=1}^{d} (d - (l-1)) \tr{{U_{l}}\T B {U_{l}}}}.
\end{equation}
\end{proposition}
\begin{proof}
The proof is given in Appendix (\autoref{app:TR}).
\end{proof}
Equation~\eqref{eq:TR_Fl_equiv} tells us several things. First, the subspaces $~{\operatorname{Span}(U_1) \perp \dots \perp \operatorname{Span}(U_d)}$ are weighted decreasingly, which means that they have less and less importance with respect to the TR objective.
Second, we can see that the nested trace ratio problem~\eqref{eq:TR_Fl} somewhat maximizes the numerator $\tr{\Pi_{\S_{1:d}} A}$ while minimizing the denominator $\tr{\Pi_{\S_{1:d}} B}$. Both subproblems have an explicit solution corresponding to our nested PCA Theorem~\ref{thm:flag_trick}. Hence, one can naturally initialize the steepest descent algorithm with the $q$ highest eigenvalues of $A$ or the $q$ lowest eigenvalues of $B$ depending on the application.
For instance, for LDA, initializing Algorithm~\ref{alg:GD} with the highest eigenvalues of $A$ would spread the classes far apart, while initializing it with the lowest eigenvalues of $B$ would concentrate the classes, which seems less desirable since we do not want the classes to concentrate at the same point.

\subsubsection{Nestedness experiments for trace ratio problems}
For all the experiments of this subsection, we consider the particular TR problem of LDA, although many other applications (\textit{marginal Fisher analysis}~\citep{yan_graph_2007}, \textit{local discriminant embedding}~\citep{chen_local_2005} etc.) could be investigated similarly.

First, we consider a synthetic dataset with five clusters.
The ambient dimension is $p = 3$ and the intrinsic dimensions that we try are $q_{1:2} = (1, 2)$.
We adopt a preprocessing strategy similar to~\citet{ngo_trace_2012}: we first center the data, then run a PCA to reduce the dimension to $n - C$ (if $n - C < p$), then construct the LDA scatter matrices $A$ and $B$, then add a diagonal covariance regularization of $10^{-5}$ times their trace and finally normalize them to have unit trace.
We run Algorithm~\ref{alg:GD} on Grassmann manifolds to solve the TR maximization problem~\eqref{eq:TR_Gr}, successively for $q_1 = 1$ and $q_2 = 2$. Then we plot the projections of the data points onto the optimal subspaces. We compare them to the nested projections onto the optimal flag output by running Algorithm~\ref{alg:GD} on $\Fl(3, (1, 2))$ to solve~\eqref{eq:TR_Fl}. The results are shown in Figure~\ref{fig:TR_nested}.
\begin{figure}
	\centering
    \includegraphics[width=.9\linewidth]{Fig/FT_exp_TR_synthetic.pdf}
    \caption{
    Illustration of the nestedness issue in linear discriminant analysis (trace ratio problem). Given a dataset with five clusters, we plot its projection onto the optimal 1D subspace and 2D subspace obtained by solving the associated Grassmannian optimization problem~\eqref{eq:TR_Gr} or flag optimization problem~\eqref{eq:TR_Fl}. 
    We can see that the Grassmann representations are not nested, while the flag representations are nested and well capture the distribution of clusters. In this example, it is less the nestedness than the \textit{rotation} of the optimal axes inside the 2D subspace that is critical to the analysis of the Grassmann-based method.
    }
	\label{fig:TR_nested}
\end{figure}
\begin{figure}
	\centering
    \includegraphics[width=.9\linewidth]{Fig/FT_exp_TR_digits.pdf}
    \caption{
    Illustration of the nestedness issue in linear discriminant analysis (trace ratio problem) on the digits dataset. For $q_k \in \qf := (1, 2, \dots, 63)$, we solve the Grassmannian optimization problem~\eqref{eq:TR_Gr} on $\Gr(64, q_k)$ and plot the subspace angles $\Theta(\S_k^*, \S_{k+1}^*)$ (left) and explained variances ${\operatorname{tr}(\Pi_{\S_k^*} X X\T)} / {\operatorname{tr}(X X\T)}$ (right) as a function of $k$. We compare those quantities to the ones obtained by solving the flag optimization problem~\eqref{eq:TR_Fl}. 
    We can see that the Grassmann-based method is highly non-nested and even yields an extremely paradoxical non-increasing explained variance (cf. red circle on the right).
    }
	\label{fig:TR_nested_digits}
\end{figure}
We can see that the Grassmann representations are non-nested while their flag counterparts perfectly capture the filtration of subspaces that best and best approximates the distribution while discriminating the classes. Even if the colors make us realize that the issue in this experiment for LDA  is not much about the non-nestedness but rather about the rotation of the principal axes within the 2D subspace, we still have an important issue of consistency.

Second, we consider the (full) handwritten digits dataset~\citep{alpaydin_optical_1998}. It contains $8 \times 8$ pixels images of handwritten digits, from $0$ to $9$, almost uniformly class-balanced. One has $n = 1797$, $p=64$ and $C = 10$.
We run a steepest descent algorithm to solve the trace ratio problem~\eqref{eq:TR_Fl}. We choose the full signature $q_{1:63} = (1, 2, \dots, 63)$ and compare the output flag to the individual subspaces output by running optimization on $\Gr(p, q_k)$ for $q_k \in q_{1:d}$.
We plot the subspace angles $\Theta(\S_k^*, \S_{k+1}^*)$ and the explained variance ${\operatorname{tr}(\Pi_{\S_k^*} X X\T)} / {\operatorname{tr}(X X\T)}$ as a function of the $k$. The results are illustrated in \autoref{fig:TR_nested_digits}.
We see that the subspace angles are always positive and even very large sometimes with the LDA. Worst, the explained variance is not monotonous. This implies that we sometimes \textit{loose} some information when \textit{increasing} the dimension, which is extremely paradoxical.

Third, we perform some classification experiments on the optimal subspaces for different datasets. For different datasets, we run the optimization problems on $\Fl(p, q_{1:d})$, then project the data onto the different subspaces in $\S_{1:d}^*$ and run a nearest neighbors classifier with $5$ neighbors.
The predictions are then ensembled (cf. Algorithm~\ref{alg:flag_trick}) by weighted averaging, either with uniform weights or with weights minimizing the average cross-entropy:
\begin{equation}\label{eq:soft_voting}
	w_1^*, \dots, w_d^* = \argmin{\substack{w_k \geq 0 \\ \sum_{k=1}^d w_k = 1}} - \frac 1 {n C} \sum_{i=1}^n \sum_{c=1}^C y_{ic} \ln\lrp{\sum_{k=1}^d w_k y_{kic}^*},
\end{equation}
where $y_{kic}^* \in [0, 1]$ is the predicted probability that $x_i \in \R^p$ belongs to class $c \in \{1 \dots C\}$, by the classifier $g_k^*$ that is trained on $Z_k := {U_k^*}\T X \in \R^{q_k \times n}$. One can show that the latter is a convex problem, which we optimize using the \href{https://www.cvxpy.org/index.html}{cvxpy} Python package~\citep{diamond2016cvxpy}.
We report the results in \autoref{tab:TR_classif}.
\begin{table}
  \caption{Classification results for the TR problem on real datasets. For each method (Gr: Grassmann optimization~\eqref{eq:TR_Gr}, Fl: flag optimization~\eqref{eq:TR_Gr}, Fl-U: flag optimization + uniform soft voting, Fl-W: flag optimization + optimal soft voting~\eqref{eq:soft_voting}), we give the cross-entropy of the projected-predictions with respect to the true labels.}
  \label{tab:TR_classif}
  \centering
  \begin{tabular}{ccccccccc}
    \toprule
    dataset & $n$ & $p$ & $q_{1:d}$ & Gr & Fl & Fl-U & Fl-W & weights\\
    \midrule
    breast & $569$ & $30$ & $(1, 2, 5)$ & $0.0986$ & $0.0978$ & $0.0942$ & $0.0915$ & $(0.754, 0, 0.246)$\\
    iris & $150$ & $4$ & $(1, 2, 3)$ & $0.0372$ & $0.0441$ & $0.0410$ & $0.0368$ & $(0.985, 0, 0.015)$\\
    wine & $178$ & $13$ & $(1, 2, 5)$ & $0.0897$ & $0.0800$ & $0.1503$ & $0.0677$ & $(0, 1, 0)$\\
    digits & $1797$ & $64$ & $(1, 2, 5, 10)$ & $0.4507$ & $0.4419$ & $0.5645$ & $0.4374$ & $(0, 0, 0.239, 0.761)$\\
    \bottomrule
  \end{tabular}
\end{table}
The first example tells us that the optimal $5D$ subspace obtained by Grassmann optimization less discriminates the classes than the $5D$ subspace from the optimal flag. This may show that the flag takes into account some lower dimensional variability that enables to better discriminate the classes. We can also see that the uniform averaging of the predictors at different dimensions improves the classification. Finally, the optimal weights improve even more the classification and tell that the best discrimination happens by taking a soft blend of classifier at dimensions $1$ and $5$. Similar kinds of analyses can be made for the other examples.

\subsubsection{Discussion on TR optimization and kernelization}
\paragraph{A Newton algorithm}
In all the experiments of this paper, we use a steepest descent method on flag manifolds (Algorithm~\ref{alg:GD}) to solve the flag problems.
However, for the specific problem of TR~\eqref{eq:TR_Fl}, we believe that more adapted algorithms should be derived to take into account the specific form of the objective function, which is classically solved via a Newton-Lanczos method~\citep{ngo_trace_2012}. 
To that extent, we develop in the appendix (\autoref{app:TR}) an extension of the baseline Newton-Lanczos algorithm for the flag-tricked problem~\eqref{eq:TR_Fl}.
In short, the latter can be reformulated as a penalized optimization problem of the form $\operatorname{argmax}_{\Sf\in\Fl(p, \qf)} {\sum_{k=1}^d \tr{\Pi_{\S_k} (A - \rho B)}}$, where $\rho$ is updated iteratively according to a Newton scheme. Once again, our central Theorem~\ref{thm:flag_trick} enables to get explicit expressions for the iterations, which results without much difficulties in a Newton method, that is known to be much more efficient than first-order methods like the steepest descent.

\paragraph{A non-linearization via the kernel trick}
The classical trace ratio problems look for \textit{linear} embeddings of the data.
However, in most cases, the data follow a \textit{nonlinear} distribution, which may cause linear dimension reduction methods to fail. The \textit{kernel trick}~\citep{hofmann_kernel_2008} is a well-known method to embed nonlinear data into a linear space and fit linear machine learning methods.
As a consequence, we propose in appendix (\autoref{app:TR}) a kernelization of the trace ratio problem~\eqref{eq:TR_Fl} in the same fashion as the one of the seminal graph embedding work~\citep{yan_graph_2007}.
This is expected to yield much better embedding and classification results.
\section{Spectral clustering: extensions and proofs}\label{app:SSC}


\subsection{Proof of Proposition~\ref{prop:SSC}}
Let $\Sf \in \Fl(p, \qf)$ and $U_{1:d+1} := [U_1|U_2|\dots|U_d|U_{d+1}] \in \O(p)$ be an orthogonal representative of $\Sf$. One has:
\begin{align}
	\langle \Pi_{\S_{1:d}}, L\rangle_F + \beta \norm{\Pi_{\S_{1:d}}}_1 &= \left\langle \frac1d\sum_{k=1}^d \Pi_{\S_k}, L\right\rangle_F + \beta \norm{\frac1d\sum_{k=1}^d \Pi_{\S_k}}_1,\\
	&= \frac1d \lrp{\left\langle \sum_{k=1}^{d+1} (d - (k-1)) U_k {U_k}\T, L\right\rangle_F + \beta \norm{\sum_{k=1}^{d+1} (d - (k-1)) U_k {U_k}\T }_1},\\
	&= \frac1d \lrp{\sum_{k=1}^{d+1} (d - (k-1)) \left\langle U_k {U_k}\T, L\right\rangle_F + \beta \norm{\sum_{k=1}^{d+1} (d - (k-1)) U_k {U_k}\T }_1},\\
	\langle \Pi_{\S_{1:d}}, L\rangle_F + \beta \norm{\Pi_{\S_{1:d}}}_1 &= \frac1d \lrp{\sum_{k=1}^{d+1} (d - (k-1)) \tr{{U_k}\T L U_k} + \beta \norm{\sum_{k=1}^{d+1} (d - (k-1)) U_k {U_k}\T }_1},
\end{align}
which concludes the proof.

\vskip 0.2in
\bibliography{sample}

\end{document}
