\documentclass[11pt]{article}
\usepackage{jheppub}
\usepackage[dvipsnames]{xcolor}
\usepackage{amsmath,amssymb}
\usepackage{graphicx}
\usepackage{subcaption}
\usepackage{stackengine}
\usepackage{hyperref}
\usepackage{comment}

\makeatletter
\def\@fpheader{\relax}
\makeatother

\title{Colored Jones Polynomials and the Volume Conjecture}

\author[*,\dagger]{Mark Hughes}
\author[\ddagger]{\!\!, Vishnu Jejjala}
\author[\mathsection]{\!\!, P.\ Ramadevi}
\author[\ddagger]{\!\!, Pratik Roy}
\author[\P]{\!\!, Vivek Kumar Singh}

\affiliation[\,*]{Department of Mathematics, Brigham Young University,
275 TMCB, Provo, UT 84602, USA}
\affiliation[\,\dagger]{Max Planck Institute for Mathematics, Bonn, Germany}
\affiliation[\,\ddagger]{Mandelstam Institute for Theoretical Physics, School of Physics, NITheCS, and CoE-MaSS,\\
University of the Witwatersrand, Johannesburg, WITS 2050, South Africa}
\affiliation[\,\mathsection]{Department of Physics, Indian Institute of Technology Bombay, Powai, Mumbai, 400076, India}
\affiliation[\,\P]{Center for Quantum and Topological Systems (CQTS), NYUAD Research Institute,\\
New York University Abu Dhabi, PO Box 129188, Abu Dhabi, UAE}

\emailAdd{hughes@mathematics.byu.edu}
\emailAdd{v.jejjala@wits.ac.za}
\emailAdd{ramadevi@iitb.ac.in}
\emailAdd{pratik.roy@wits.ac.za}
\emailAdd{vks2024@nyu.edu}

\clubpenalty=10000
\widowpenalty=10000
\brokenpenalty=10000
\parskip=5pt

% \newcommand{\be}{\begin{equation}}
% \newcommand{\ee}{\end{equation}}
% \newcommand{\bea}{\begin{eqnarray}}
% \newcommand{\eea}{\end{eqnarray}}
\newcommand{\nn}{\nonumber}
\newcommand{\eref}[1]{(\ref{#1})}

\newcommand{\MH}[1]{{\color{blue}{\textbf{MH:} #1}}}
\newcommand{\VJ}[1]{{\color{ForestGreen}{\textbf{VJ:} #1}}}
\newcommand{\PR}[1]{{\color{teal}{\textbf{PR:} #1}}}
\newcommand{\VS}[1]{{\color{purple}{\textbf{VS:} #1}}}

% This must be in the first 5 lines to tell arXiv to use pdfLaTeX, which is strongly recommended.
\pdfoutput=1
% In particular, the hyperref package requires pdfLaTeX in order to break URLs across lines.

\documentclass[11pt]{article}

% Change "review" to "final" to generate the final (sometimes called camera-ready) version.
% Change to "preprint" to generate a non-anonymous version with page numbers.
\usepackage[preprint]{acl}
\usepackage{booktabs}
\usepackage{amsfonts}
\usepackage{amsmath}
\usepackage{multirow}
\usepackage{amsthm}
\usepackage{algorithm}
\usepackage{algorithmic}
\newtheorem{theorem}{Theorem}[section]
\newtheorem{assumption}{Assumption}[section]
\newtheorem{definition}{Definition}[section]
\newtheorem{proposition}{Proposition}[section]
\newtheorem{corollary}{Corollary}[theorem]
\newtheorem{lemma}[theorem]{Lemma}
\newtheorem*{remark}{Remark}
% Standard package includes
\usepackage{times}
\usepackage{latexsym}

% For proper rendering and hyphenation of words containing Latin characters (including in bib files)
\usepackage[T1]{fontenc}
% For Vietnamese characters
% \usepackage[T5]{fontenc}
% See https://www.latex-project.org/help/documentation/encguide.pdf for other character sets

% This assumes your files are encoded as UTF8
\usepackage[utf8]{inputenc}

% This is not strictly necessary, and may be commented out,
% but it will improve the layout of the manuscript,
% and will typically save some space.
\usepackage{microtype}

% This is also not strictly necessary, and may be commented out.
% However, it will improve the aesthetics of text in
% the typewriter font.
\usepackage{inconsolata}

%Including images in your LaTeX document requires adding
%additional package(s)
\usepackage{graphicx}

% If the title and author information does not fit in the area allocated, uncomment the following
%
%\setlength\titlebox{<dim>}
%
% and set <dim> to something 5cm or larger.

\title{A statistically consistent measure of Semantic Variability using Language Models}

% Author information can be set in various styles:
% For several authors from the same institution:
% \author{Author 1 \and ... \and Author n \\
%         Address line \\ ... \\ Address line}
% if the names do not fit well on one line use
%         Author 1 \\ {\bf Author 2} \\ ... \\ {\bf Author n} \\
% For authors from different institutions:
% \author{Author 1 \\ Address line \\  ... \\ Address line
%         \And  ... \And
%         Author n \\ Address line \\ ... \\ Address line}
% To start a separate ``row'' of authors use \AND, as in
% \author{Author 1 \\ Address line \\  ... \\ Address line
%         \AND
%         Author 2 \\ Address line \\ ... \\ Address line \And
%         Author 3 \\ Address line \\ ... \\ Address line}

\author{Yi Liu \\
  Seattle, Washington, USA \\
  %\texttt{liuyi3@microsoft.com} 
}

%\author{
%  \textbf{First Author\textsuperscript{1}},
%  \textbf{Second Author\textsuperscript{1,2}},
%  \textbf{Third T. Author\textsuperscript{1}},
%  \textbf{Fourth Author\textsuperscript{1}},
%\\
%  \textbf{Fifth Author\textsuperscript{1,2}},
%  \textbf{Sixth Author\textsuperscript{1}},
%  \textbf{Seventh Author\textsuperscript{1}},
%  \textbf{Eighth Author \textsuperscript{1,2,3,4}},
%\\
%  \textbf{Ninth Author\textsuperscript{1}},
%  \textbf{Tenth Author\textsuperscript{1}},
%  \textbf{Eleventh E. Author\textsuperscript{1,2,3,4,5}},
%  \textbf{Twelfth Author\textsuperscript{1}},
%\\
%  \textbf{Thirteenth Author\textsuperscript{3}},
%  \textbf{Fourteenth F. Author\textsuperscript{2,4}},
%  \textbf{Fifteenth Author\textsuperscript{1}},
%  \textbf{Sixteenth Author\textsuperscript{1}},
%\\
%  \textbf{Seventeenth S. Author\textsuperscript{4,5}},
%  \textbf{Eighteenth Author\textsuperscript{3,4}},
%  \textbf{Nineteenth N. Author\textsuperscript{2,5}},
%  \textbf{Twentieth Author\textsuperscript{1}}
%\\
%\\
%  \textsuperscript{1}Affiliation 1,
%  \textsuperscript{2}Affiliation 2,
%  \textsuperscript{3}Affiliation 3,
%  \textsuperscript{4}Affiliation 4,
%  \textsuperscript{5}Affiliation 5
%\\
%  \small{
%    \textbf{Correspondence:} \href{mailto:email@domain}{email@domain}
%  }
%}

\begin{document}
\maketitle
\begin{abstract}
To address the challenge of variability in the output generated by language models, we introduce a measure of semantic variability that remains statistically consistent under mild assumptions. This measure, termed semantic spectral entropy, is an easily implementable algorithm that requires only standard, pre-trained language models. Our approach imposes minimal restrictions on the choice of language models, and through rigorous simulation studies, we demonstrate that this method can produce an accurate and reliable metric despite the inherent randomness in language model outputs.
\end{abstract}

\section{Introduction}

\label{introduction}
{\color{white}..} The birth of Large Language Models (LLM) has given rise to the possibility of a wide range of industry applications \cite{touvron2023llama,chowdhery2023palm}. One of the key applications of generative models that has garnered significant interest is the development of specialized chatbots with domain-specific expertise such as legal and healthcare \cite{Lexis,mesko2023top}. These applications illustrate how generative models can improve decision-making and improve the efficiency of professional services in specialized fields.

This new LLM capability is made possible by the strong understanding of generative capabilities of the models \cite{liu2023mmc,long2023large} and the advent of Retrieval-Augmented Generation (RAG) \cite{lewis2020retrieval, gao2023retrieval}. In an RAG system, the user interacts by submitting queries, which trigger a search for relevant documents within a pre-established database. These pertinent documents are retrieved based on the query and serve as a context for the LLM to generate an appropriate response. Since the implementation of RAG does not require a custom-trained LLM, it offers a cost-effective solution. The resulting chatbot can perform tasks traditionally handled by domain experts, improving operational efficiency and driving cost reductions.

However, a critical challenge impeding the widespread deployment of generative models in industry is the inherent variability present in these models \cite{amodei2016concrete,hendrycks2021unsolved}.  Although parameters such as temperature, top-k, top-p, and repetition penalty are known to significantly influence model performance \cite{wang2020contextual,wang2023cost, song2024good}, even when these parameters are tuned to achieve deterministic output (e.g. setting temperature to 0 or top-p to 1), differences in the generated results can still occur in multiple runs. This persistent variability poses a significant barrier to the reliable and consistent application of generative models in practical settings.

Atil et al. (2024) conducted a series of experiments involving six deterministically configured large language models (LLMs), with temperature set to 0 and top p set to 1, across eight common tasks and five identical trials per task. The study aimed to assess the repeatability of model outputs by examining whether the generated strings were consistent between runs. The authors found that none of the LLMs demonstrated consistent performance in terms of generating identical outputs on all tasks \cite{atil2024llm}. 
%For complex tasks, such as college-level mathematics, the models often produced lexically different outputs for each run, leading to zero consistency in terms of exact string matching. 
However, the authors noted that when accounting for syntactical variations, the observed differences were relatively minor as many of the generated strings were semantically equivalent. 


The variability in output has been attributed to the use of GPUs in large language model (LLM) inference processes, where premature rounding during computations can lead to discrepancies \cite{nvidia2024,atil2024llm}. Given this, it is reasonable to conclude that complete elimination of variability is unfeasible in any empirical setting. Consequently, we must acknowledge that the output of LLMs is inherently uncertain. In light of this, it becomes essential, similar to practices in statistics, to assess and quantify the level of uncertainty in the text generated by LLMs for any given scenario. 

Most prior studies on uncertainty in foundation models for natural language processing (NLP) have focused primarily on the calibration of classifiers and text regressors \cite{jiang2021can, desai2020calibration, glushkova2021uncertainty}. Other research has addressed uncertainty by prompting models to evaluate their own outputs or fine-tuning generative models to predict their own uncertainty \cite{linteaching, kadavath2022language}. However, these approaches require additional training and supervision, making them difficult to reproduce, costly to implement, and sensitive to distributional shifts. 

 Our work follows from a line of work inline with the concept of semantic entropy proposed in \cite{kuhn2023semantic, nikitin2024kernel,duan-etal-2024-shifting,lin2023generating}. \cite{kuhn2023semantic} explore the entropy of the generated text by assigning semantic equivalence to the pairs of text and subsequently estimating the entropy. Similarly, \cite{nikitin2024kernel} and \cite{lin2023generating} utilize graphical spectral analysis to enhance empirical results. However, a notable limitation in the entropy estimators proposed by \cite{kuhn2023semantic} and \cite{nikitin2024kernel} is their reliance on token likelihoods when assessing semantic equivalence, which may not always be accessible. Furthermore, \cite{kuhn2023semantic} acknowledge that the clustering process employed in their framework is susceptible to the order of comparisons, introducing variability into the results. 

Moreover, previous work focuses on the empirical performance of the estimator. As such, while these methods have demonstrated favorable empirical outcomes, to the best of our knowledge, no authors have established using a theoretical analysis that their entropy estimators converge to a true entropy value as the sample size increases under an underlying generative model. Exploring the theoretical properties allows us to have a clear understanding of how the number of clusters and size of data would affect the estimator. 

Our approach seeks to address these limitations by developing a robust theoretical analysis of the clustering procedure, ensuring convergence properties, and mitigating the variability inherent in prior methodologies. We propose a theoretically analyzable metric for quantifying the variation within a collection of texts, which we refer to as semantic spectral entropy. This measure addresses the observation that many generated strings, while lexically and syntactically distinct, may convey equivalent semantic content. To identify these semantic equivalences, we advocate the use of off-the-shelf generative language models (LMs). Moreover, we acknowledge that the LM used to evaluate semantic similarity is itself a stochastic generator. In response, we employ the well-established technique of spectral clustering, which is provably consistent under minimal assumptions on the generator, thereby ensuring the robustness and reliability of the proposed metric. Specifically, we demonstrate that the measure is statistically consistent under a weak assumption on the LM. To the best of our knowledge, this is the first semantic variability measure with proven convergence properties. As an empirical evaluation studies, we also propose a simple method for constructing clusters of different lexically and syntactically distinct but semantically equivalent text using compound  propositions from \cite{wittgenstein2023tractatus}.

\section{Semantic spectral entropy}
\label{methodology}
\subsection{Semantic entropy}

{\color{white}..} We begin with a collection of textual pieces \( n \), denoted \( \mathcal{T} = (t_1, \cdots, t_n) \). Unlike that in \cite{kuhn2023semantic}, our assumption is that we have access only to $\mathcal{T}$. In fact, we do not require the existence of a generative model and is interested only in variability of the semantics in the text. To evaluate the semantic variability of these texts in the context of a specific use case, we propose a theoretically proven measure of semantic entropy which we named semantic spectral entropy. 

A key reason for opting against the use of variance as a measure of variability is that computing variance requires the definition of a mean, which is challenging to establish for semantic distributions. Although it is possible to define an arbitrary reference point, such as a standard answer in a chatbot that answers questions, evaluating the variability with respect to such a reference introduces bias. 

In contrast, entropy is a well-established measure of variation, particularly for multinomial distributions. For a distribution \( \mathcal{P}(t) \) over a set of semantic clusters \( \{C_1, \cdots, C_k\} \), the entropy \( \mathcal{E} \) is defined as:
\begin{equation}
\label{equ:entropy}
\mathcal{E}(t) = - \sum_{i} p(t \in C_i)\log p(t \in C_i).
\end{equation}
This formulation captures the uncertainty or disorder associated with assigning a given text \( t \) to one of the clusters. Consequently, it provides a quantitative measure of semantic variability that avoids the biases introduced by arbitrary reference points.

To estimate the entropy for a given data set \( t_1, \cdots, t_n \), we first calculate the number of occurrences of each text \( t_i \) in each group \( C_j \). This is achieved by computing:
\[
n_j = \sum_{i=1}^n \mathbb{I}(t_i \in C_j),
\]
where \( \mathbb{I}(t_i \in C_j) \) is an indicator function that equals 1 if \( t_i \) belongs to the cluster \( C_j \), and 0 otherwise. 

Next, the true probability \( p(t \in C_j) \) is approximated using the empirical distribution:
\[
\bar{p}(t \in C_j) = \frac{n_j}{n},
\]
which represents the fraction of texts assigned to cluster \( C_j \). Using this empirical distribution, the empirical entropy is defined as:
\[
\bar{\mathcal{E}}(\mathcal{T}) = - \sum_{j} \bar{p}(t \in C_j) \log \bar{p}(t \in C_j).
\]
This measure provides a practical estimation of semantic entropy based on observed data.


One critical step in this process is clustering the texts $t_i$ into disjoint groups. To do so, it is sufficient to define a relationship between $t_i \sim t_j$, such that they satisfy the properties of equivalence relation. Specifically, one needs to demonstrate 
\begin{enumerate}
    \item Reflexivity: For every $t_i$, we have $t_i \sim t_i$, meaning that any text is equivalent to itself.
    \item Symmetry: If $t_i \sim t_j$, then $t_j \sim t_i$, meaning that equivalence is bidirectional.
    \item Transitivity: If $t_i \sim t_j$ and $t_j \sim t_k$, then $t_i \sim t_k$, which means that equivalence is transitive.
\end{enumerate}

It turns out the existence of an equivalence equation is both a necessary and sufficient condition for a definition of a breakdown of $\mathcal{T}$ into disjoint clusters \cite{liebeck2018concise}. In light of this, defining $\sim$ should be based on the linguist properties of entropy measurement. 

 Direct string comparison, defined as \( t_i \sim t_j \) if and only if \( t_i \) and \( t_j \) share identical characters, reflects lexicon equality and constitutes an equivalence relation. However, this criterion is overly restrictive. In a question-and-response context, a more appropriate equivalence relation might be defined as \( t_i \sim t_j \) if and only if \( t_i \) and \( t_j \) yield identical scores when evaluated by a language model (LM) prompt. This criterion, however, requires an answer statement as a point of reference. We are more interested in a stand-alone metric that can capture the semantic equivalence. For example, consider the sentences \( t_1 = \text{"Water is vital to human survival"} \) and \( t_2 = \text{"Humans must have water to survive"}\). Despite differences in language, both sentences convey the same underlying meaning.

To address such challenges, \cite{kuhn2023semantic,nikitin2024kernel} propose an equivalence relation wherein \( t_i \sim t_j \) if and only if \( t_i \) is true if and only if \( t_j \) is true. This formulation ensures that two texts, \( t_i \) and \( t_j \), belong to the same equivalence class if they are logically equivalent. This broader definition allows for greater flexibility and applicability in assessing semantic equivalence beyond superficial lexical similarity.\cite{copi2016introduction}. We will present their argument as a proposition where we will put the verification in the appendix
\begin{proposition}
    \label{prop:equ}
    The relation $t_i \sim t_j$ if "$t_i$ is true if and only if $t_j$ is true" is an equiva
    
    lence relation.
\end{proposition}

% one considers segmenting the set into equivalence classes based on the following equivalence relation:  


%We first establish that this relation $\sim$ indeed defines well-defined, disjoint subsets of $t_i, i \in \{1,\dots n\}$. 



% Now, we can conclude that $\sim$ is indeed an equivalence relation, and thus, the set of texts $t_1\dots t_n$ can be partitioned into disjoint equivalence classes, where each class represents a distinct semantic group. 
%\begin{remark}
    
%\end{remark}
In light of the fact that equivalence relations can be defined arbitrarily based on the needs of the user. We propose that the determination of equivalence relations, denoted as $\sim$, is performed through a LM that generates responses independently of the specific generation of terms $t_1, \dots, t_n$. However, we do not assume that we have access to probability distribution of the tokens as proposed by \cite{kuhn2023semantic,nikitin2024kernel} which is not always available. Rather, we just require a generator LM which can generate a determination of this relationship. Therefore, this LM can be general generative language model with a crafted prompt which we will use in our simulation studies. The error in this LM will be removed in the spectral clustering algorithm at the later stage. By leveraging this LM, we can define a function $e:{\mathcal{T}, \mathcal{T}}\rightarrow {0,1}$, which is formally expressed as follows: \begin{equation} e(t_i, t_j) = \begin{cases} 1 & \text{if } t_i \sim t_j, \\ 0& \text{otherwise.} \end{cases} \end{equation}

However, since the function relies on an LM, $e(t_i, t_j)$ can be viewed as a Bernoulli random variable, whose value is dependent on the terms $t_i$ and $t_j$.\nocite{kuhn2023semantic} did not address this issue but instead offers adopting a very powerful entailment identification model which the authors trust to identify the equivalence relation perfectly. In contrast, we suggest modeling the outputs of the LM as a random graph with an underlying distribution. In this framework, $t_i$ and $t_j$ represent nodes, while $e(t_i, t_j)$ are random variables that indicate the presence of an edge between the two nodes. Specifically, when $t_i \sim t_j$, the edge existence is governed by the following probability distribution: \begin{equation} \label{eqn:equation_p} e(t_i, t_j) = \begin{cases} 1 & \text{with probability } p, \\ 0 & \text{with probability } 1-p. \end{cases} \end{equation} Conversely, when $t_i \not\sim t_j$, the edge existence follows a different probability distribution: \begin{equation} \label{eqn:equation_q} e(t_i, t_j) = \begin{cases} 1 & \text{with probability } q, \\ 0 & \text{with probability } 1-q. \end{cases} \end{equation}

To mitigate the inherent randomness introduced by the LLM, we propose leveraging spectral clustering to identify clusters of semantically similar texts.
\subsection{Spectral clustering}

{\color{white}..} To compute semantic entropy, it is crucial to identify the clusters of nodes and count the number of nodes within each cluster. Identifying these clusters in a random graph is analogous to detecting clusters in a stochastic block model \cite{holland1983stochastic}. We propose employing the spectral clustering algorithm, with the number of clusters $K$ specified in advance, as an effective approach for this task.

Spectral Clustering is a well-established algorithm for graph clustering, supported by strong theoretical foundations and efficient implementations \cite{shi2000normalized, lei2015consistency, su2019strong, scikit-learn}.  To compute semantic entropy, we aim to cluster a random graph with adjacency matrix $E$ where $E_{ij} = e(t_i, t_j)$, representing the pairwise similarity between text elements $t_i$ and $t_j$. 

We begin by computing the Laplacian matrix $L =  D-E$ where $D$ is the degree matrix.  This is followed by the decomposition of the eigenvalue of $L$. Next, we construct the matrix formed by the first $K$ eigenvectors of $L$ denoted $\hat{U} \in \mathbb{R}^{n\times K}$. This matrix serves as input to an appropriate $(1+\epsilon)-$ k-means clustering algorithm \cite{kumar2004simple,choo2020k}.

The output of this procedure is $K$ distinct clusters $C_1,\cdots C_K$. For each text element $t_i$, we assign a corresponding vector $g_{i}$ where 
$$ g_{ij} = \begin{cases} 1 \text{ if } t_i \in C_j\\
    0 \text{ otherwise }
\end{cases}$$
This binary indicator vector $g_i$ encodes the cluster membership for each text element $t_i$

Finally, we compute the estimated entropy based on the number of texts within each cluster. The entropy $\hat{\mathcal{E}}$ can be approximated using the following formula:
\begin{equation}
\hat{\mathcal{E}}(\mathcal{T}) = - \sum_{j=1}^k\hat{p}( C_j) \log(\hat{p}( C_j)),
\end{equation}
where $\hat{p}(C_j) = \frac{1}{n}\sum_{i=1}^n g_{ij}$.This expression represents the empirical entropy based on the distribution of texts among the $K$ clusters, providing a measure of the uncertainty or diversity within the semantic structure of the data.
\subsection{Full algorithm and implementation}
{\color{white}..} We merge the process of finding sermantic entropy with spectral clustering to present the full algorithm as Algorithm \ref{algo:1}: Sermantic Spectral Entropy. 
\begin{algorithm}
\begin{algorithmic}
    \STATE Begin with $\mathcal{T} = \{t_1, \cdots t_n\}$
    \FOR{$i, j \in \{1,\cdots n\} \times \{1, \cdots n\}, i\neq j$}
    \STATE Use LLM to compute $E_{i,j} = e(t_i, t_j)$. 
    \ENDFOR
    \STATE Find the Laplacian of $E$, $L = D -E$
    \STATE Compute the first $K$ eigenvectors $u_1,\dots,u_k$ of $L$ and the top $K$ eigenvalues $\lambda_1,\cdots \lambda_k$.
    \STATE Let $\hat{U} \in \mathbb{R}^{n\times k}$ be the matrix containing the vectors $u_1,\dots,u_k$ as columns.
    \STATE Use $(1+\epsilon)$ K-means clustering algorithm to cluster the rows of $U$
    \STATE Let $g_{ij}$ be an $(1+\epsilon)-$approximate solution to a $K-$means clustering algorithm
    \STATE Compute $\hat{\mathcal{E}}(\mathcal{T})$ using $ g_{ij}$
\end{algorithmic}
\caption{\label{algo:1} Sermantic Spectral Entropy }
\end{algorithm}

This polynomial-time algorithm is characterized by the largest computational cost associated with the determination of $E_{ij}$. However, computing $E_{ij}$ is embarrassingly parallel, meaning that it can be efficiently distributed across multiple processing units. Furthermore, there are well-established implementation, such as Microsoft Azure's Prompt-Flow \cite{esposito2024programming} and LangChain \cite{mavroudis2024langchain} that facilitate the implementation of parallel workflows, making it feasible to deploy such parallelized tasks with relative ease.
\subsection{Finding K}

{\color{white}..} A notable limitation of this analysis is the unavailability of $K$ in the direct computation of semantic spectral entropy. However, the determination of $K$ for stochastic block model has been well studied \cite{lei2016goodness,wang2017likelihood,chen2018network}. We will describe the cross-validation approach \cite{chen2018network} in detail. The principle behind cross-validation involves predicting the probabilities associated with inter-group connections ($p$) and intra-group connections ($q$). If the estimated value of $K$ is too small, it fails to accurately recover the true underlying probabilities; conversely, if $K$ is too large, it leads to overfitting to noisy data. This approach has the potential to recover the true cluster size under relatively mild conditions.

\section{Theoretical Results}
\label{theory}

{\color{white}..} Our theoretical analysis involves a proof that the estimator is strongly consistent, i.e. the estimator converges to true value almost surely, and an analysis of its rate with respect to the number of cluster $K$. 

We divide our analysis into two subsections. The first subsection examines a fixed set of $\mathcal{T} = {t_1, \dots, t_n}$, which is assumed to exhibit some inherent clusters $C_1, \dots, C_K$. Under the assumption of perfect knowledge of these clusters, the empirical entropy $\bar{\mathcal{E}}$ can be determined. The primary focus in this subsection is on the performance of spectral clustering algorithms. The second subsection explores a scenario in which there exists an underlying generative mechanism that allows for the infinite generation of $t_i$. In this case, we permit $K$ to increase with $n$, though at a significantly slower rate. This scenario is particularly relevant for evaluating the performance of RAG in the context of continuous generation of results in response to a given query.

\subsection{Performance of spectral clustering algorithms}
{\color{white}..}  We model the LM determination of $e(t_i,t_j)$ as a random variable, as described in Equations \ref{eqn:equation_p} and \ref{eqn:equation_q}. In the theoretical analysis presented here, we assume that the number of clusters, $K$, is known and fixed. To derive various results, we first establish the relationship between the difference $|\bar{\mathcal{E}}(\mathcal{T}) - \hat{\mathcal{E}}(\mathcal{T})|$ and the miscluster error, denoted $M_\text{error}$.
\begin{lemma}
 \label{lemma:error}
Suppose that there exists $0<c_2<1$ such that $2Kn_{\min}/n \geq c_2$, 
\begin{equation}
     |\hat{\mathcal{E}}(\mathcal{T}) - \bar{\mathcal{E}}(\mathcal{T})|\leq h\left(\frac{2K}{c_2}\right) \left|\frac{1}{n} (M_\text{error})\right| 
\end{equation}
where $h(x) = \left(x+\log\left(x\right)\right)$.
\end{lemma}
The proof is presented in the Appendix section \ref{Appendix:proofoflemma:error}. 
We begin by presenting the result of strong consistency for the spectral clustering algorithm.  

\begin{theorem}
\label{the:strongConsistensy}
Under regularity conditions, the estimated entropy empirical entropy $\hat{\mathcal{E}}(\mathcal{T})$ is strongly consistent with the empirical entropy, i.e. 
\begin{equation}
    |\bar{\mathcal{E}}(\mathcal{T}) - \hat{\mathcal{E}}(\mathcal{T}) | \rightarrow 0 \text{ almost surely }
\end{equation}
\end{theorem}
The proof is provided in the Appendix section \ref{appendix:sec:the:strongconsistency}. This establishes strong consistency result that we aim to present. At the same time, we also want to show the finite sample properties of the estimator $\hat{\mathcal{E}}(\mathcal{T})$.

\begin{theorem}
    \label{the:finite_sample}
    If there exists $0<c_2\leq1$ and $\lambda > 0$ such that $2Kn_{\min}/n \geq c_2$, and $p = \alpha_n = \alpha_n(q + \lambda) $, where $\alpha_n \geq \log(n)$ then with probability at least $1-\frac{1}{n}$

\begin{equation}
|\bar{\mathcal{E}}(\mathcal{T}) - \hat{\mathcal{E}}(\mathcal{T}) |  \leq h\left(\frac{2K}{c_2}\right) \frac{n_{\max }}{4c_2^2n_{\min }^{2} \alpha_{n}K^2} 
\end{equation}
where $h(x) = \left(x+\log\left(x\right)\right)$, $n_{\max} = \max_j\{n_j : j = 1,\dots K\}$, and $n_{\min} = \min_j\{n_j : j = 1,\dots K\}$.
\end{theorem}
The full proof is provided in the appendix section \ref{appendix:proofofthe:finite_sample}. A brief outline of the proof is as follows: we begin by using the results from \cite{lei2015consistency}, which establish the rate of convergence for the stochastic block model. Next, we relate the errors of the spectral clustering algorithm to the errors in the empirical entropy, using the lemma \ref{lemma:error} to establish this connection.

\begin{remark}
    This result is particularly relevant for computing semantic entropy, as the output generated by LMs is produced with a probability that is independent of $n$. As a result, we have $\alpha_n = O(1)$. Assuming balanced community sizes, the convergence rate is therefore $O(\frac{1}{n})$. This is formally stated in the following corollary:
\end{remark}


\begin{corollary} \label{corollary:rate} If there exists a constant $0 < c_2 \leq 1$ such that $2Kn_{\min}/n \geq c_2$ and $\alpha_n = alpha >0$, then there exists a constant $\alpha$ such that with probability at least $1 - \frac{1}{n}$, \begin{equation} |\bar{\mathcal{E}}(\mathcal{T}) - \hat{\mathcal{E}}(\mathcal{T})| \leq h\left(\frac{2K}{c_2}\right) \frac{1}{c_2^4 \alpha n}. \end{equation} \end{corollary}

The proof of this result is provided in the Appendix section \ref{appendix:proofofcorollary:rate}. 
\begin{remark}
    In particular, we observe that the convergence rate is $O\left(\frac{1}{n}\right)$. This means that the error associated with spectral clustering is small, and our estimated entropy converges to the empirically entropy quickly.
\end{remark}


\subsection{Performance under a generative model}

{\color{white}..} In practical terms, we assume the presence of a generator, specifically an RAG, that produces identically distributed independent random variables $t_i$' that collectively form semantic clusters $C_1 \dots C_K$. In essence, we have $t_i \sim G$ such that $t_i \in C_j$ with probability $p(C_j)$. In this model, there is a true value of entropy $\mathcal{E}(\mathcal{T})$ given in Equation \ref{equ:entropy}, and we want to find the convergence rate of our method. 
\begin{theorem}
    \label{the:final} If there exists a constant $\alpha $ such that $p = \alpha  = \alpha(q + \lambda) $, then with probability at least $1-\frac{3}{n}$,
    \begin{equation}
    \label{eqn:final_the}
       \begin{array}{cc}
         |\mathcal{E} - \hat{\mathcal{E}}|&  \leq h\left(\frac{1}{p_{\min}}\right)K\sqrt{\frac{1}{2n}\log\left(2Kn\right)}\\
         & +h\left(\frac{1}{m(n)p_{\min}}\right)\frac{1}{16K^4m(n)^4p_{\min}^4n}
    \end{array} 
    \end{equation}

where $m(n) = \left(1- \sqrt{2\log(nK)/np_{\min}}\right)$ and $p_{\min} = \min\{p(C_1)\dots p(C_K)\}$.
\end{theorem}
Most of the material used for this proof is presented in Corollary \ref{corollary:rate}. 
\begin{proof}
Consider the following equality
$$|\mathcal{E} - \hat{\mathcal{E}}| \leq |\mathcal{E} -\bar{\mathcal{E}} +\bar{\mathcal{E}}-  \hat{\mathcal{E}}| \leq |\mathcal{E} -\bar{\mathcal{E}}| + | \bar{\mathcal{E}}-  \hat{\mathcal{E}}|,$$  
 We know that there are three sufficient conditions for Equation \ref{eqn:final_the}. These are
\begin{enumerate}
    \item[C1:]$|\mathcal{E} -\bar{\mathcal{E}}| \leq  h\left(\frac{1}{p_{\min}}\right)K\sqrt{\frac{1}{2n}\log\left(2Kn\right)},$
    \item[C2:]$ \exists c_2 \text{ such that } 0 < c_2 \leq 1$ and $2Kn_{\min}/n \geq c_2,$ 
    \item[C3:] $ |\bar{\mathcal{E}}(\mathcal{T}) - \hat{\mathcal{E}}(\mathcal{T})| \leq h\left(\frac{2K}{c_2}\right) \frac{1}{c_2^4 n}.$
\end{enumerate}
Then, using union bound
\begin{align*}
    \mathbb{P}(\text{Not (\ref{eqn:final_the})}) &\leq \mathbb{P}( \text{Not C1 or Not C2 or Not C3})\\
    &\leq \mathbb{P}( \text{Not C1}) + \mathbb{P}( \text{Not C2})+ \mathbb{P}( \text{Not C3}).
\end{align*}
In Lemma \ref{lemma:final1} and \ref{lemma:Final2} of the appendix, we show that $|\mathcal{E} -\bar{\mathcal{E}}| \geq  h\left(\frac{1}{p_{\min}}\right)K\sqrt{\frac{1}{2n}\log\left(2Kn\right)}$ with probability at most $\frac{1}{n}$.

In Lemma \ref{Lemma:Final3} of the Appendix, we show that setting $c_2 = 2K\left(1-\sqrt{\frac{2\log(nK)}{np_{\min}}}\right)p_{\min}$, we have $2Kn_{\min}/n< c_2$ with probability at most $\frac{1}{n}$.

Finally, the corollary \ref{corollary:rate} tells us that $ |\bar{\mathcal{E}}(\mathcal{T}) - \hat{\mathcal{E}}(\mathcal{T})| > h\left(\frac{2K}{c_2}\right) \frac{1}{c_2^4 n}$ occurs with probability at most $\frac{1}{n}$.
\end{proof}
\begin{remark}
One observation is that the empirical entropy converges to true entropy at a rate slower than that of estimated entropy to the empirical entropy. This is natural since each $t_i$ has the opportunity to make a $n-1$ connection with other $t_j$s, resulting in $n(n-1)/2$ independent observations, whereas each generator generates only $n$ independent observations.
\end{remark}
\subsection{Discussion on $K$}
{\color{white}..} An intriguing question to consider is the rate at which \( K \), the number of clusters, can grow with \( n \), the number of texts, as it is natural to expect \( K \) to increase with \( n \). Focusing solely on the spectral clustering algorithm, the error is characterized as \( O((K + \log(K))/n) \). Thus, under the condition \( K = o(n^{1-\delta}) \) for some \( \delta > 0 \), we have \( |\bar{\mathcal{E}}(\mathcal{T}) - \hat{\mathcal{E}}(\mathcal{T})| \to 0 \) in probability. In contrast, when considering a scenario involving a generative model, a stricter condition is required. Specifically, \( K \) must satisfy \( K = o(n^{1/2 - \delta}) \), with \( \delta > 0 \), to ensure \( |\mathcal{E}(\mathcal{T}) - \hat{\mathcal{E}}(\mathcal{T})| \to 0 \) in probability.

\section{Simulation and data studies}
\label{simulation}

{\color{white} .. }As this paper focuses more on the theoretical analysis of semantic spectral entropy with respect to variable $n$ and $K$, we decide against using the evaluation method proposed in \cite{kuhn2023semantic,duan-etal-2024-shifting, lin2023generating} in favor of constructing a simulation where we know the true entropy $\bar{\mathcal{E}}$. This allows us to better analyze how $|\bar{\mathcal{E}} -\hat{\mathcal{E}}|$ changes with choice of generator $e$, $K$ and $n_{\min}$.

To construct a non-trivial simulation for this use case, we evaluate the performance of our algorithms within the context of an unordered set of elementary proposition statements that has no logical interconnections. This approach draws upon the philosophical framework defined by \citep{wittgenstein2023tractatus} in Tractatus Logico-Philosophicus, where each elementary proposition represents a singular atomic fact. Within this framework, texts containing an identical set of elementary propositions are deemed semantically equivalent. The primary advantage of this experimental design lies in its efficiency, as it facilitates the generation of thousands of samples with minimal generator propositions, all while maintaining knowledge of the ground truth.

For example, we can consider a list of things that a hypothetical individual "John" likes to do in his free time: 
\begin{itemize}
    \item Running/Jogging 
    \item Drone Flying/ Pilot Aerial drones
    \item jazzercise / aerobics
    \item ...
\end{itemize}

To generate a cluster of text from this set of hobbies, we begin by randomly selecting \( M \) items from a total of \( N \) items in the list to formulate the compound proportion. This selection process yields \( \binom{N}{M} \) potential subset of hobbies and we know that two subsets of hobbies are the same as long as their elements are the same. Next, to create individual text samples \( t_i \) within the group, we randomly permute the order of the \( M \) selected elements in the subset. This permutation process generates \( M! \) unique samples for each combination of hobbies. Finally, the hobbies are then placed in its permuted order in a sentence like that below. 
\begin{quote}
"In his free time, John likes hobby $1$, hobby $2$, hobby $3$, ..., and hobby $M$ as his hobbies."
\end{quote}
In order to prevent models to rely on sentence structure, a few of these sentences are being designed. 

%We replicate this simulation set-up in different 2 settings. The  setting is the 10 common hobbies that this hypothetical individual likes to do in his free time. %The second set-up is 10 events that happened on the date December 3 in history which we collect from Wikipedia \cite{wiki}. %The last setting  %need to think about how to build these algorithm
%\cite{atil2024llm}

We utilize Microsoft Phi-3.5 \cite{abdin2024phi}, OpenAI GPT3.5-turbo \cite{hurst2024gpt}, A21-Jamba 1.5 Mini \cite{lieber2021jurassic}, Cohere-command-r-08-2024 \cite{Ustun2024AyaMA},Ministral-3B \cite{jiang2023mistral} and the Llama 3.2 70B model \cite{dubey2024llama} as \( e \). These models are lightweight, off-the-shelf language models that are cost-effective to deploy and exhibit efficiency in generating outputs, thereby off-setting the computational cost of determining sermantic relationships. The exact prompt used to generate the verdict is specified in Appendix \ref{appendix_sec:prompt_engineering}.

\begin{table*}[ht]
\centering
\begin{tabular}{l|rrr|rrr|rrr|}
\toprule
ratio & \multicolumn{3}{r|}{0.2,0.3,0.5} & \multicolumn{3}{r|}{0.3,0.3,0.4} & \multicolumn{3}{r|}{0.5,0.5}\\
datasize & 30 & 50 & 70 & 30 & 50 & 70 & 30 & 50 & 70\\
\midrule
LLAMA & 0.36 & 0.49 & 0.44 & 0.34 & 0.43 & 0.46 & 0.30 & 0.27 & 0.26 \\x
MINISTRAL & 0.22 & 0.27 & 0.13 & 0.25 & 0.23 & 0.21 & 0.14 & 0.22 & 0.21 \\
COHERE & 0.04 & 0.02 & 0.06 & 0.02 & 0.03 & 0.00 & 0.00 & 0.00 & 0.00 \\
A21 & 0.05 & 0.00 & 0.00 & 0.00 & 0.01 & 0.00 & 0.00 & 0.00 & 0.00 \\
PHI & 0.08 & 0.07 & 0.07 & 0.03 & 0.03 & 0.00 & 0.00 & 0.00 & 0.00 \\
GPT & 0.06 & 0.02 & 0.00 & 0.01 & 0.00 & 0.00 & 0.00 & 0.00 & 0.00 \\
\bottomrule
\end{tabular}
\caption{\label{tab:basic_simu} Average $|\bar{\mathcal{E}}- \hat{\mathcal{E}}|$ over simulation 10 iterations. We have three different ratio value run over three different data sizes. For $e$, we use Microsoft Phi-3.5 \cite{abdin2024phi}, OpenAI GPT3.5-turbo \cite{hurst2024gpt}, A21-Jamba 1.5 Mini \cite{lieber2021jurassic}, Cohere-command-r-08-2024 \cite{Ustun2024AyaMA}, Ministral-3B \cite{jiang2023mistral} and the Llama 3.2 70B model \cite{dubey2024llama}. }
\end{table*}

\begin{figure}
    \centering
    \includegraphics[width=1\linewidth]{LambdaExp1.pdf}
    \caption{\label{fig:dotdata} A scatter plot of $p-q$ against $|\bar{\mathcal{E}}- \hat{\mathcal{E}}|$. The different colors represents different language models used as $e$: A21 in blue, Phi in Orange, GPT in Green, Cohere in Red, Llama is Purple and Ministral in Brown. We notice that there is clear phrase change point where for $p-q <0.4$, we have that $|\bar{\mathcal{E}}- \hat{\mathcal{E}}|$ is very high most of the time, for $p-q >0.4$, $|\bar{\mathcal{E}}- \hat{\mathcal{E}}|$ is small with occasional jumps that the theory predicts.}
    
\end{figure}

\begin{table}[]
    \centering
\begin{tabular}{lrrr}
\toprule
 $e$& $p-q$ & $p$ & $q$ \\
\midrule
LLAMA & 0.17 & 0.17 & 0.00 \\
MINISTRAL & 0.22 & 0.99 & 0.77 \\
COHERE & 0.55 & 0.61 & 0.05 \\
A21 & 0.81 & 0.96 & 0.15 \\
PHI & 0.67 & 0.67 & 0.01 \\
GPT & 0.80 & 0.87 & 0.07 \\
\bottomrule
\end{tabular}
\caption{\label{tab:p-q} $p$, $q$ and $p-q$.  For $e$, we use Microsoft Phi-3.5 \cite{abdin2024phi}, OpenAI GPT3.5-turbo \cite{hurst2024gpt}, A21-Jamba 1.5 Mini \cite{lieber2021jurassic}, Cohere-command-r-08-2024 \cite{Ustun2024AyaMA}, Ministral-3B \cite{jiang2023mistral} and the Llama 3.2 70B model \cite{dubey2024llama}.}
\end{table}

We complete simulation studies for a ratio of (0.2,0.3,0.5), (0.3, 0.3,0.4), and (0.5,0.5) and a sample size of 30, 50, 70. The average $|\bar{\mathcal{E}}- \hat{\mathcal{E}}|$ over 10 iterations using different models as $e$ is recorded in table \ref{tab:basic_simu}. The performance of algorithm using Cohere, A21, Phi, and GPT is strong while the performance of the algorithm with Minstral and Llama is weak. We primary attribute this to the inability of Llama and Minstral to make correct statements. $p-q$ is small for Llama and Minstral and large for Cohere, A21, Phi, and GPT (shown in Table \ref{tab:p-q}). In fact, when we plot $p-q$ against $|\bar{\mathcal{E}}- \hat{\mathcal{E}}|$ in Figure \ref{fig:dotdata}, we notice that there is phrase change at value $p-q = 0.4$. $p-q < 0.4$ $|\bar{\mathcal{E}}- \hat{\mathcal{E}}|$ is high but  $p-q > 0.4$ implies that $|\bar{\mathcal{E}}- \hat{\mathcal{E}}|$ is generally small. This phrase change is not predicted in the theory and suggests that more work is needed. 
\section{Discussion}
\label{conclusion}
{\color{white} .. }Many natural language processing tasks exhibit a fundamental invariance: sequences of distinct tokens can convey identical meanings. This paper introduces a theoretically grounded metric for quantifying semantic variation, referred to as semantic spectral clustering. This approach reframes the challenge of measuring semantic variation as a prompt-engineering problem, which can be applied to any large language model (LLM), as demonstrated through our simulation analysis. In addition, unsupervised uncertainty can offer a solution to the issue identified in prior research, where supervised uncertainty measures face challenges in handling distributional shifts.

While we define two texts as having equivalent meaning if and only if they mutually imply one another, alternative definitions may be appropriate for specific use cases. For example, legal documents could be clustered based on the adoption of similar legal strategies, with documents grouped together if they demonstrate comparable approaches. In such scenarios, the entropy of the legal documents could also be computed to quantify their informational diversity. We have demonstrated that, provided there exists a function $e$ capable of performing the evaluation with weak accuracy, this estimator remains consistent. Given the reasoning capabilities of large language models (LLMs), we foresee numerous possibilities for extending this method to a wide range of applications.

In addition to the methodology presented, we present a theoretical analysis of the proposed algorithms by proving a theorem concerning the contraction rates of the entropy estimator and its strong consistency. Although the algorithm utilizes generative models, which are typically treated as black-boxes, we simplify the analysis by considering the outputs of these models as random variables. We demonstrate that only a few conditions on the generative are sufficient for our spectral clustering algorithm to achieve strong consistency. Our approach allows for many statistical methodologies to be applied in conjunctions with generative models to analyze text at a level previously not achievable by humans. 



\section{Limitation}
{\color{white} .. }We acknowledge that, while this research offers a theoretically consistent measurement of variation, it does not account for situations where two pieces of text may partially agree. For instance, two texts may contain points of agreement as well as points of disagreement. This is particularly common when different authors cite the same sources but reach contradictory conclusions.
%\section{Acknowledgments}



% Bibliography entries for the entire Anthology, followed by custom entries
%\bibliography{anthology,custom}
% Custom bibliography entries only
\bibliography{custom}
\onecolumn
\appendix

\section{Theoretical Result}
\subsection{Proof of proposition  \ref{prop:equ}}
\begin{proof} 
    To prove that the relation $t_i \sim t_j$ if $t_i$ is true if and only if $t_j$ is true is an equivalence relation, we need to meet 3 key criteria, namely symmetry, reflexivity, and Transitivity. 

    First, symmetry 
    $t_i \sim t_j$ implies that $t_j$ is true $\Leftrightarrow$ $t_j$ is true, but this also means $t_j$ is true $\Leftrightarrow$ $t_i$ is true. Then we have $t_j \sim t_i$. 

    Second, reflexivity, 
    $t_i \sim t_j$ implies $t_j$ is true $\Leftrightarrow$ $t_j$ is true. But this means that $t_j$ is true  $\Leftrightarrow$ $t_i$ is true. Then we have $t_i \sim t_j$. 
    
    Third, transitivity,
    If $t_i \sim t_j$ and $t_j \sim t_k$, Then if $t_i$ is true $\Rightarrow$ $t_j$ is true $\Rightarrow$ $t_k$ is true, which means $t_i$ is true $\Rightarrow$ $t_k$ is true. On the other hand, using the same argument, $t_k$ is true $\Rightarrow$ $t_j$ is true $\Rightarrow$ $t_i$ is true. This means the $t_k$ is true $\Rightarrow$ $t_i$ is true. Therefore $t_i \sim t_k$. 

    The three points is sufficient to demonstrate that $\sim$ is a equivalence relation. 
\end{proof}
\subsection{Proof of Theorem \ref{the:strongConsistensy}}
\label{appendix:sec:the:strongconsistency}
To prove Theorem \ref{the:strongConsistensy}, we adopt notations from \cite{su2019strong}.
Consider the adjacency matrix $E$ which is determined by a Language model. 

Let $d_i = \sum_{j=1}^n E_{ij}$ denote the degree of node $i$,  $D = \text{diag}(d_1,\cdots, d_n)$, and $L = D^{-1/2}ED^{-1/2}$ be the graph Laplacian. We also define $n_k$ be the number of text in each cluster. We denote a block probability matrix $B = B_{k_1k_2}$ where $k_1,k_2 \in\{1,\cdots K\}$ be the clusters index.  i.e. 
$$ B_{k_1 k_2} = \begin{cases}
    p \quad \text{if $k_1 = k_2$}\\
    1-q \quad \text{otherwise.}
\end{cases}$$

Let $\mathbb{E}(E) = P$ i.e. the probability of edge between $i$ and $j$ is given by $P_{ij} = B_{k_1k_2}$ if text $i$ is in $C_{k_1}$ and $j$ is in $C_{k_2}$.
Denote $Z = \{Z_{ik}\}$ be a $n\times K$  binary matrix providing the cluster membership of text $t$, i.e., $Z_{ik} = 1$ if text $i$ is in $C_k$ and $Z_{ik} = 0$ otherwise. The population version of the Laplacian is given by $\mathcal{L} = \mathcal{D}^{-1/2}P\mathcal{D}^{-1/2}$ where  $\mathcal{D} = \text{diag}(d_1 \cdots d_n)$ where $d_i =\sum_{j=1}^{n}P_{ij} = p + (n-1)(q)$.

Let $\pi_{kn} = n_k/n, W_k = \sum_{l=1}^KB_{kl}\pi_{ln}$, $\mathcal{D}_B = \text{diag}(W_1,\cdots W_K)$, and $B_0=\mathcal{D}_B^{-1/2}B\mathcal{D}_B^{-1/2}  $
%C^star = 3528C_1 c_1^{-1/2}
\begin{assumption}[Assumption 1 in \cite{su2019strong}]
\label{assumption:eigenvalues}
$P$ is rank $k$ and spectral decomposition $\Pi_{n}^{1/2}P\Pi_{n}^{1/2}$ is $S_n \Omega_n S_n^T$ in which $S_n$ is a $K \times K$ matrix such that $S_n^T S_n = I_{K\times K}$  and $\Omega_n = \text{diag}(\omega_1 \cdots \omega_{K_n})$ such that $|\omega_1|\geq |\omega_2|\geq\cdots \geq|\omega_{K_n}|$
\end{assumption}
Assumption \ref{assumption:eigenvalues} implies that the spectral decomposition $$\mathcal{L} = U_n \Sigma_n U_n^T = U_{1n}\Sigma_{1n}U_{1n}^T$$

where \(\Sigma_{n}=\operatorname{diag}\left(\sigma_{1 n}, \ldots, \sigma_{K n}, 0, \ldots, 0\right)\) is a \(n \times n\) matrix that contains the eigenvalues of \(\mathcal{L}\) such that \(\left|\sigma_{1 n}\right| \geq\left|\sigma_{2 n}\right| \geq \cdots \geq\left|\sigma_{K n}\right|>0, \Sigma_{1 n}=\operatorname{diag}\left(\sigma_{1 n}, \ldots, \sigma_{K n}\right)\), the columns of \(U_{n}\) contain the 
 eigenvectors of \(\mathcal{L}\) associated with the eigenvalues in \(\Sigma_{n}, U_{n}=\left(U_{1 n}, U_{2 n}\right)\), and \(U_{n}^{T} U_{n}=I_{n}\) \cite{su2019strong}.
\begin{assumption}[Assumption 2 in \cite{su2019strong}]
\label{assumption:limits_nk}
    There exists constant $C_1 >0$ and $c_2>0$ such that
    $$C_1 \geq \lim\sup_n\sup_k n_k K/n \geq \lim \inf_n \inf_k n_k K/n \geq c_2  $$
\end{assumption}

\begin{assumption}[Assumption 3 in \cite{su2019strong}]
\label{assumption:bound_eigenvalues}
    Let $\mu_n = \min_i d_i$ and $\rho_n = \max(\sup_{k_1k_2}[B_0]_{k_1k_2},1)$. Then $n$ sufficiently large, 
    $$ 
\frac{K \rho_{n} \log ^{1 / 2}(n)}{\mu_{n}^{1 / 2} \sigma_{K n}^{2}}\left(1+\rho_{n}+\left(\frac{1}{K}+\frac{\log (5)}{\log (n)}\right)^{1 / 2} \rho_{n}^{1 / 2}\right) \leq 10^{-8} C_{1}^{-1} c_{2}^{1 / 2} .
$$
    
\end{assumption}
Let 
$$ 
\hat{O}_{n}=\bar{U} \bar{V}^{T}
$$
where \(\bar{U} \bar{\Sigma} \bar{V}^{T}\) is the singular value decomposition of \(\hat{U}_{1 n}^{T} U_{1 n}\). we also denote \(\hat{u}_{1 i}^{T}\) and \(u_{1 i}^{T}\) as the \(i\)-th rows of \(\hat{U}_{1 n}\) and \(U_{1 n}\), respectively.

Now we present the notation of the K-means algorithm. With a little abuse of notation, let \(\hat{\beta}_{\text {in }} \in \mathbb{R}^{K}\) be a generic estimator of \(\beta_{g_{i}^{0} n} \in \mathbb{R}^{K}\) for \(i=1, \ldots, n\). To recover the community membership structure (i.e., to estimate \(g_{i}^{0}\) ), it is natural to apply the  K-means clustering algorithm to \(\left\{\widehat{\beta}_{\text {in }}\right\}\). Specifically, let \(\mathcal{A}=\left\{\alpha_{1}, \ldots, \alpha_{K}\right\}\) be a set of \(K\) arbitrary  \(K \times 1\) vectors: \(\alpha_{1}, \ldots, \alpha_{K}\). Define
\[
\widehat{Q}_{n}(\mathcal{A})=\frac{1}{n} \sum_{i=1}^{n} \min _{1 \leq l \leq K}\left\|\hat{\beta}_{i n}-\alpha_{l}\right\|^{2}
\]

and \(\widehat{\mathcal{A}}_{n}=\left\{\widehat{\alpha}_{1}, \ldots, \widehat{\alpha}_{K}\right\}\), where \(\widehat{\mathcal{A}}_{n}=\arg \min _{\mathcal{A}} \widehat{Q}_{n}(\mathcal{A})\). Then we compute the estimated cluster  identity as
\[
\hat{g}_{i}=\underset{1 \leq l \leq K}{\arg \min }\left\|\hat{\beta}_{\text {in }}-\widehat{\alpha}_{l}\right\|,
\]

where if there are multiple \(l\) 's that achieve the minimum, \(\hat{g}_{i}\) takes value of the smallest one. We then state the key assumption that relates to K-means clustering algorithm. 

\begin{assumption}[Assumption 7 in \cite{su2019strong}]
\label{assumption:K-means}
     Suppose for \(n\) sufficiently large,
     \[
15 C^{*} \frac{K \rho_{n} \log ^{1 / 2}(n)}{\mu_{n}^{1 / 2} \sigma_{K n}^{2}}\left(1+\rho_{n}+\left(\frac{1}{K}+\frac{\log (5)}{\log (n)}\right)^{1 / 2} \rho_{n}^{1 / 2}\right) \leq c_{2} C_{1}^{-1 / 2} \sqrt{2}
\]
Where \(C^{*} = 3528C_1 c_2^{-1/2} \)
\end{assumption}

\begin{theorem}(Collorary 2.2)
\label{theorem:no_error}
    Corollary 2.2. Suppose that Assumptions \ref{assumption:eigenvalues},  \ref{assumption:limits_nk}, \ref{assumption:bound_eigenvalues}, and \ref{assumption:K-means} hold and the \(K\)-means algorithm is applied  to \(\hat{\beta}_{i n}=(n / K)^{1 / 2} \hat{u}_{1 i}\) and \(\beta_{g_{i}^{0} n}=(n / K)^{1 / 2} \hat{O}_{n} u_{1 i}\) Then, 
    \[
\sup _{1 \leq i \leq n} \mathbf{1}\left\{\tilde{g}_{i} \neq g_{i}^{0}\right\}=0 \quad \text { a.s. }
\]
\end{theorem}

We now have define the error of mis-classification. 

% Since there is no true $j$, we have to take all permutation of $j$ which we denote as $\sigma(j)$. 
\begin{definition}
Denote $M_\text{error} = \sum_{j} \sum_{i}\mathbb{I}(g_{ij}  \neq g^{\text{True}}_{ij})$ as the mis-classification error.
\end{definition}

\begin{lemma}
\label{lemma:error_connections}
    If $\sup_{i,j} \mathbb{I}(g_{ij}  \neq g^{\text{True}}_{ij}) = 0 \quad \text{a.s.}$, then $M_\text{error} = 0 \quad \text{a.s.}$
\end{lemma}
\begin{proof}
Notice $\mathbb{I}(g_{ij}  \neq g^{\text{True}}_{ij})$ can only takes up value $1$ or $0$. Therefore $\sum_{j} \sum_{i}\mathbb{I}(g_{ij}\neq g^{\text{True}}_{ij}) \neq 0 \Leftrightarrow \exists i, j  \text{ s.t }\mathbb{I}(g_{ij}\neq g^{\text{True}}_{ij}) \neq 0 \Leftrightarrow  \sup_{i,j}\mathbb{I}(g_{ij}\neq g^{\text{True}}_{ij}) \neq 0$  
    \begin{align*}
        \mathbb{P}(M_\text{error} \neq 0 \text{ i.o. }) &= \mathbb{P}\left( \sum_{j} \sum_{i}\mathbb{I}(g_{ij}\neq g^{\text{True}}_{ij}) \neq 0 \text{ i.o.}\right)\\
        &= \mathbb{P}\left( \exists i, j  \text{ s.t }\mathbb{I}(g_{ij}\neq g^{\text{True}}_{ij}) \neq 0 \text{ i.o. } \right)\\
        &= \mathbb{P}\left( \sup_{i,j}\mathbb{I}(g_{ij}\neq g^{\text{True}}_{ij}) \neq 0 \text{ i.o }\right)\\
        &= 0 \quad \text{ since $\sup_{i,j} \mathbb{I}(g_{ij}  \neq g^{\text{True}}_{ij}) = 0$ \text{ a.s.}}
    \end{align*}
    Here we use the classical notation i.o. as happens infinitely often. 
\end{proof}
\begin{lemma}
    \label{lemma:misclassification}
    $\sum_j \left|\sum_{i=1}^n g_{ij}-n_j\right| \leq M_\text{error}$
\end{lemma}
\begin{proof}

\begin{align*}
\sum_j \left|\sum_{i=1}^n g_{ij}-n_j\right|
&= \sum_j \Biggl| \sum_i \mathbb{I}(g_{ij} = 1, g^{\text{True}}_{ij} = 0 ) + \mathbb{I}(g_{ij} = 1, g^{\text{True}}_{ij} = 1 ) + \mathbb{I}(g_{ij} = 0, g^{\text{True}}_{ij} = 1 ) \\
&- \mathbb{I}(g_{ij} = 0, g^{\text{True}}_{ij} = 1 )- n_j \Biggr|\\
& = \sum_j \Biggl| \sum_i \mathbb{I}(g_{ij} = 1, g^{\text{True}}_{ij} = 0 ) - \mathbb{I}(g_{ij} = 0, g^{\text{True}}_{ij} = 1 ) \\
& + \sum_i \mathbb{I}(g_{ij} = 0, g^{\text{True}}_{ij} = 1 )+ \mathbb{I}(g_{ij} = 0, g^{\text{True}}_{ij} = 1 ) - n_j \Biggr|\\
& = \sum_j \Biggl| \sum_i \mathbb{I}(g_{ij} = 1, g^{\text{True}}_{ij} = 0 ) - \mathbb{I}(g_{ij} = 0, g^{\text{True}}_{ij} = 1 ) + n_j - n_j \Biggr|\\
&= \sum_j \Biggl| \sum_i \mathbb{I}(g_{ij} = 1, g^{\text{True}}_{ij} = 0 ) -\mathbb{I}(g_{ij} = 0, g^{\text{True}}_{ij} = 1 ) \Biggr|\\
&\leq \sum_j\sum_i \mathbb{I}(g_{ij} = 1, g^{\text{True}}_{ij} = 0 ) + \mathbb{I}(g_{ij} = 0, g^{\text{True}}_{ij} = 1 )\\
&=   \sum_{j} \sum_{i}\mathbb{I}(g_{ij}  \neq g^{\text{True}}_{ij}) \\
&= M_\text{error}
\end{align*}
\end{proof}
\newpage
\subsubsection{Proof of lemma \ref{lemma:error}}
\label{Appendix:proofoflemma:error}
Now we prove lemma \ref{lemma:error}.
\begin{proof}
Recall that
\begin{itemize}
    \item $\hat{p}(C_j) = \frac{1}{n}\sum_{i=1}^n g_{ij}$ and $\hat{\mathcal{E}}(\mathcal{T}) = - \sum_{j=1}^K\hat{p}( C_j) \log(\hat{p}( C_j))$
    \item $\bar{p}(C_j) = \frac{n_j}{n}$ and $\bar{\mathcal{E}}(\mathcal{T}) = - \sum_{j=1}^K\bar{p}( C_j) \log(\bar{p}( C_j))$
\end{itemize}
\begin{align*}
    |\hat{\mathcal{E}}(\mathcal{T}) - \bar{\mathcal{E}}(\mathcal{T})| &= \left| \sum_{j=1}^k\hat{p}( C_j) \log(\hat{p}( C_j)) -  \bar{p}( C_j) \log(\bar{p}( C_j)) \right|\\
    &=  \left|\sum_{j=1}^K\hat{p}( C_j)\log(\hat{p}( C_j)) -  \hat{p}( C_j)\log(\bar{p}( C_j)) + \hat{p}( C_j)\log(\bar{p}( C_j)) -  \bar{p}( C_j) \log(\bar{p}( C_j)) \right|\\
    &= \left|\sum_{j=1}^K \hat{p}( C_j)\log\left(\frac{\hat{p}( C_j)}{\bar{p}( C_j)}\right) - \left(\hat{p}( C_j) -\bar{p}( C_j)\right)\log(\bar{p}( C_j)) \right|\\
    &=  \left|\sum_{j=1}^K \hat{p}( C_j)\log\left(\frac{\hat{p}( C_j)}{p( C_j)}\right) - \left(\hat{p}( C_j) -\bar{p}( C_j)\right)\log(\bar{p}( C_j)) \right|\\
    &\leq \left|\sum_{j=1}^K \hat{p}( C_j)\log\left(\frac{\hat{p}( C_j)}{\bar{p}( C_j)}\right)\right| + \left|\sum_{j=1}^K \left(\hat{p}( C_j) -\bar{p}( C_j)\right)\log(\bar{p}( C_j)) \right|\\
    &\leq  \left|\sum_{j=1}^K \left(\frac{\hat{p}( C_j)-\bar{p}( C_j)}{\bar{p}( C_j)}\right)\right| + \left|\sum_{j=1}^K \left(\hat{p}( C_j) -\bar{p}( C_j)\right)\log(\bar{p}( C_j)) \right|\\
    &= \left|\sum_{j=1}^K \left(\frac{\frac{1}{n}\sum_{i=1}^n g_{ij}-\bar{p}( C_j)}{\bar{p}( C_j)}\right)\right| + \left|\sum_{j=1}^K \left(\frac{1}{n}\sum_{i=1}^n g_{ij} -\bar{p}( C_j)\right)\log(\bar{p}( C_j)) \right|\\
    &= \left|\sum_{j=1}^K \left(\frac{\frac{1}{n}\left(\sum_{i=1}^n g_{ij}-n_j\right)}{\bar{p}( C_j)}\right)\right| + \left|\sum_{j=1}^K \left(\frac{1}{n}\sum_{i=1}^n g_{ij} -\bar{p}( C_j)\right)\log(\bar{p}( C_j)) \right|\\
    &\leq \sum_{j=1}^K \left|\frac{\frac{1}{n}\left(\sum_{i=1}^n g_{ij}-n_j\right)}{\bar{p}( C_j)}\right| + \sum_{j=1}^K \left|\frac{1}{n}\sum_{i=1}^n (g_{ij} - n_j)\right|\left|\log(\bar{p}( C_j)) \right|\\
    &\leq \left|\frac{\frac{2K}{n}(M_\text{error})}{c_2}\right| + \log\left(\frac{2K}{c_2}\right) \left|\frac{1}{n} (M_\text{error})\right|\\
    &= h\left(\frac{2K}{c_2}\right) \left|\frac{1}{n} (M_\text{error})\right| 
\end{align*}

where $h(x) = \left(x+\log\left(x\right)\right)$. 
\end{proof}

We prove Theorem \ref{the:strongConsistensy}. To do so, we first restate Theorem \ref{the:strongConsistensy} with all the conditions required to get to the outcome.
\begin{theorem}[Theorem \ref{the:strongConsistensy} with all conditions stated]
    Assume that Assumptions \ref{assumption:eigenvalues},  \ref{assumption:limits_nk}, \ref{assumption:bound_eigenvalues}, and \ref{assumption:K-means} hold and the \(K\)-means algorithm is applied  to \(\hat{\beta}_{i n}=(n / K)^{1 / 2} \hat{u}_{1 i}\) and \(\beta_{g_{i}^{0} n}=(n / K)^{1 / 2} \hat{O}_{n} u_{1 i}\) Then 
    \[ |\bar{\mathcal{E}}(\mathcal{T}) - \hat{\mathcal{E}}(\mathcal{T}) | \rightarrow 0 \text{ almost surely }\]
\end{theorem}

\begin{proof}

Using Theorem \ref{theorem:no_error}, we know that under Assumptions \ref{assumption:eigenvalues},  \ref{assumption:limits_nk}, \ref{assumption:bound_eigenvalues}, and \ref{assumption:K-means}, we have that 
 \[
\sup _{1 \leq i \leq n} \mathbf{1}\left\{\tilde{g}_{i} \neq g_{i}^{0}\right\}=0 \quad \text { a.s. }
\]
Using Lemma \ref{lemma:error_connections}, we know that 

$$M_\text{error} = 0 \quad \text{a.s.}$$
Using results from Lemma \ref{lemma:error}, we know that $M_\text{error} \rightarrow 0 \quad a.s. \Rightarrow \hat{\mathcal{E}}(\mathcal{T}) \rightarrow \bar{\mathcal{E}}(\mathcal{T}) \quad a.s. $. 
\end{proof}
Now we try to prove Theorem \ref{the:finite_sample}. To do so, we state corollary 3.2 in \cite{lei2015consistency}.
\subsection{Proof of Theorem \ref{the:finite_sample}}
\label{appendix:proofofthe:finite_sample}
\begin{theorem}[Corollary 3.2 in \cite{lei2015consistency}]
\label{the:finite_sample_core}
 Let $E$ be an adjacency matrix from the $\operatorname{SBM}(Z, B)$, where $B=\alpha_{n} B_{0}$ for some $\alpha_{n} \geq \log n / n$ and with $B_{0}$ having minimum absolute eigenvalue $\geq \lambda>0$ and $\max _{k \ell} B_{0}(k, \ell)=1$. Let $g_{ij}$ be the output of spectral clustering using $(1+\varepsilon)$-approximate $k$-means. Then  there exists an absolute constant $c$ such that if 

\begin{equation*}
(2+\varepsilon) \frac{K n}{n_{\min }^{2} \lambda^{2} \alpha_{n}}<c
\end{equation*}
then with probability at least $1-n^{-1}$,
$$
\frac{1}{n}M_{\text{error}}\leq c^{-1}(2+\varepsilon) \frac{K n_{\max }}{n_{\min }^{2} \lambda^{2} \alpha_{n}}
$$
\end{theorem}


\begin{proof}
We now prove Theorem  \ref{the:finite_sample}.

    Under the model we have, we know that minimum eigenvalue of $B$ is $\lambda$. Use theorem \ref{the:finite_sample_core} to replace $h\left(\frac{2K}{c_2}\right) \left|\frac{1}{n} (M_\text{error})\right|$ with $ h\left(\frac{2K}{c_2}\right) c^{-1}(2+\varepsilon) \frac{K n_{\max }}{n_{\min }^{2} \lambda^{2} \alpha_{n}} $ in lemma \ref{lemma:error}.

We now have to show the existence of $c$ in Theorem \ref{the:finite_sample_core}.

\begin{align*}
    &\quad 2Kn_{\min}/n \geq c_2 \\
    &\Rightarrow 1/n_{\min}^2 \leq 4K^2/n^2c_2^2\\
    &\Rightarrow (2+\epsilon)\frac{Kn}{n_{\min}^2\lambda^2 \alpha_n} \leq (2+\epsilon)\frac{4K^3 }{n\lambda^2 \alpha_nc_2^2}\leq (2+\epsilon)\frac{4K^3}{\lambda^2c_2^2}\\
    &\text{Let $c = (2+\epsilon)\frac{4K^3}{\lambda^2}c_2^2$}
\end{align*}
substitute $c$ to $ h\left(\frac{2K}{c_2}\right) c^{-1}(2+\varepsilon) \frac{K n_{\max }}{n_{\min }^{2} \lambda^{2} \alpha_{n}} $, we have that 
\begin{equation*}
|\bar{\mathcal{E}}(\mathcal{T}) - \hat{\mathcal{E}}(\mathcal{T}) |  \leq h\left(\frac{2K}{c_2}\right) \frac{n_{\max }}{4c_2^2n_{\min }^{2} \alpha_{n}K^2}
\end{equation*}

\end{proof}
\subsubsection{Proof of Corollary \ref{corollary:rate}}
\label{appendix:proofofcorollary:rate}
\begin{proof}
Now we prove Corollary \ref{corollary:rate}. Note that $n \geq n_{\max} \geq n_{\min} \geq nc_2/2K$.

\begin{equation*}
|\bar{\mathcal{E}}(\mathcal{T}) - \hat{\mathcal{E}}(\mathcal{T}) |  \leq h\left(\frac{2K}{c_2}\right) \frac{n_{\max }}{4c_2^2n_{\min }^{2} \alpha_{n}K^2}\leq h\left(\frac{2K}{c_2}\right) \frac{1}{c_2^4 \alpha n}
\end{equation*}
\end{proof}

\newpage
\begin{lemma}
\label{lemma:final1}
    $$|\mathcal{E} - \hat{\mathcal{E}}| \leq \sum_{j=1}^K \left( \left| \frac{p(C_j) - \bar{p}(C_j)}{p(C_j)}\right| + \log\left(\frac{1}{p(C_j)}\right)\left| p(C_j) - \bar{p}(C_j)\right|\right) + h\left(\frac{2K}{c_2}\right) \left|\frac{1}{n} (M_\text{error})\right| $$
\end{lemma}
\begin{proof}
    First, we have that 
    $$|\mathcal{E} - \hat{\mathcal{E}}| \leq |\mathcal{E} -\bar{\mathcal{E}} +\bar{\mathcal{E}}-  \hat{\mathcal{E}}| \leq |\mathcal{E} -\bar{\mathcal{E}}| + | \bar{\mathcal{E}}-  \hat{\mathcal{E}}| \leq |\mathcal{E} -\bar{\mathcal{E}}| + h\left(\frac{2K}{c_2}\right) \left|\frac{1}{n} (M_\text{error})\right| $$
    Next, 
    \begin{align*}
        |\mathcal{E} -\bar{\mathcal{E}}| &\leq  \left| \sum_{j=1}^kp( C_j) \log(p( C_j)) -  \bar{p}( C_j) \log(\bar{p}( C_j)) \right|\\  
        &\leq \left| \sum_{j=1}^kp( C_j) \log(p( C_j)) -  \bar{p}( C_j) \log(p( C_j)) + \bar{p}( C_j) \log(p( C_j)) - \bar{p}( C_j) \log(\bar{p}( C_j)) \right|\\
        &\leq \sum_{j=1}^k \left|p( C_j) \log(p( C_j)) -  \bar{p}( C_j) \log(p( C_j)) \right| + \left|\bar{p}( C_j) \log(p( C_j)) - \bar{p}( C_j) \log(\bar{p}( C_j)) \right| \\
        &\leq \sum_{j=1}^k \left|p( C_j)  -  \bar{p}( C_j)\right| \log\left(\frac{1}{p( C_j)}\right)  + \left| \frac{p( C_j) -  \bar{p}( C_j)}{p( C_j)} \right|
    \end{align*}
\end{proof}

\begin{lemma}
\label{lemma:Final2}
With probability at least $1-\frac{1}{n}$,
$$\sum_{j=1}^k \left|p( C_j)  -  \bar{p}( C_j)\right| \leq K\sqrt{\frac{1}{2n}\log(2Kn)} $$
\end{lemma}
\begin{proof}

       $$ \left|p( C_j)  -  \bar{p}( C_j)\right| = \frac{1}{n}\left|np( C_j)  -  n_j\right|$$
Now use Hoeffding bound, we notice that for any $j$
$$\mathbb{P}(|n_j - np(C_j)| \geq \delta) \leq 2\exp\left(-\frac{2\delta^2}{n}\right) $$
Using union bound 
$$\mathbb{P}(\exists j \text{ such that }|n_j - np(C_j)| \geq \delta) \leq \sum_{j=1}^K\mathbb{P}(|n_j - np(C_j)| \geq \delta) \leq 2K\exp\left(-\frac{2\delta^2}{n}\right) $$

$\exists j \text{ such that }|n_j - np(C_j)| \geq \delta \Leftarrow\max |n_j - np(C_j)| \geq \delta \Leftarrow \sum_{j=1}^K |n_j - np(C_j)| \geq K\delta.$ 

Now, let $ 2K\exp\left(-\frac{2\delta^2}{n}\right) = \frac{1}{n}$, we have that $\delta = \sqrt{\frac{n}{2}\log(2Kn)}$

This gives us that with probability at least $1-\frac{1}{n}$,

$$ \sum_{j=1}^k \left|p( C_j)  -  \bar{p}( C_j)\right| \leq K\sqrt{\frac{1}{2n}\log(2Kn)} $$ 
\end{proof}
\begin{lemma}
\label{Lemma:Final3}
With probability at least $1-\frac{1}{n}$
$$n_{\min} \geq \frac{nc_2}{2K}$$
where $c_2 = 2K\left(1-\sqrt{\frac{2\log(nK)}{np_{\min}}}\right)p_{\min}$ and $p_{\min} = \min \{p(C_1) \dots p(C_K) \}$
\end{lemma}
\begin{proof}
    Using the Chernoff inequality, we have $$\mathbb{P}\left(n_j \leq (1-\delta)np(C_j)\right) \leq \exp\left(\frac{-np(C_j)}{2}\right)$$
Using the union bound
$$\mathbb{P}(n_{\min} \leq nc_2/2K) \leq \mathbb{P}\left(\exists j \text{ such that }n_j \leq (1-\delta)np(C_j)\right) \leq K\exp\left(\frac{-np_{\min}}{2}\right) $$
Let $K\exp\left(\frac{-np_{\min}}{2}\right) = \frac{1}{n}$, we get $\delta = \sqrt{\frac{2\log(nK)}{np_{\min}}}.$
Finally, we have $c_2 = 2K\left(1-\sqrt{\frac{2\log(nK)}{np_{\min}}}\right)p_{\min}$ 

\end{proof}
\newpage
\section{Simulations}
\subsection{Hobby Examples}
We can consider a list of things that a hypothetical individual "John" likes to do in his free time: 
\begin{itemize}
    \item running / jogging 
    \item Drone flying / pilot Aerial drones
    \item jazzercise / aerobics
    \item making pottery / making ceramics
    \item water gardening / aquatic gardening
    \item caving / spelunking / potholing
    \item cycling / bicycling / biking
    \item reading
    \item writing journals / journal writings/ journaling
    \item sculling / rowing
\end{itemize}
\iffalse
\subsection{Historical Examples}
On the day December 3,
\begin{itemize}
    \item 915 – Pope John X crowns Berengar I of Italy as Holy Roman Emperor
    \item 1775 – American Revolutionary War: USS Alfred becomes the first vessel to fly the Grand Union Flag; the flag is hoisted by John Paul Jones.
    \item 1800 – War of the Second Coalition: Battle of Hohenlinden: French General Jean Victor Marie Moreau decisively defeats the Archduke John of Austria near Munich. Coupled with First Consul Napoleon Bonaparte's earlier victory at Marengo, this will force the Austrians to sign an armistice and end the war.
    \item 1818 – Illinois becomes the 21st U.S. state.
    \item 1834 – The Zollverein (German Customs Union) begins the first regular census in Germany.
    \item 1898 – The Duquesne Country and Athletic Club defeats an all-star collection of early football players 16–0, in what is considered to be the first all-star game for professional American football.
    \item 1920 – Following more than a month of Turkish–Armenian War, the Turkish-dictated Treaty of Alexandropol is concluded.
    \item 1929 – President Herbert Hoover delivers his first State of the Union message to Congress. It is presented in the form of a written message rather than a speech
    \item 1959 – The current flag of Singapore is adopted, six months after Singapore became self-governing within the British Empire.
    \item 1979 – In Cincinnati, 11 fans are suffocated in a crush for seats on the concourse outside Riverfront Coliseum before a Who concert.
    \item 1979 – Iranian Revolution: Ayatollah Ruhollah Khomeini becomes the first Supreme Leader of Iran.
\end{itemize}
\fi
\newpage
\section{Prompt}
\label{appendix_sec:prompt_engineering}
This is the prompt we inserted for "Phi-3-mini-4k-instruct", "AI21-Jamba-1.5-Mini", "Cohere-command-r-08-2024".


\begin{verbatim}
'''
    You are a expert in logical deduction and you are given 2 piece of texts: TEXT A and TEXT B. 
    You are to identify if TEXT A implies TEXT B and TEXT B implies TEXT A at the same time. 
    
    TEXT A: 
    {text_A}
    
    TEXT B:
    {text_B}
    
    ## OUTPUT
    You are to return TRUE if TEXT A implies TEXT B and TEXT B implies TEXT A at the same time. 
    otherwise, you are to return FALSE 
'''
\end{verbatim}

This is the prompt we inserted for "Ministral-3B","Llama-3.3-70B-Instruct", "gpt-35-turbo"

\begin{verbatim}
''' 
    You are a expert in logical deduction and you are given 2 piece of texts: TEXT A and TEXT B. 
    You are to identify if TEXT A implies TEXT B and TEXT B implies TEXT A at the same time. 
    
    TEXT A: 
    {text_A}
    
    TEXT B:
    {text_B}
    
    ## OUTPUT
    You are to return TRUE if TEXT A implies TEXT B and TEXT B implies TEXT A at the same time. 
    otherwise, you are to return FALSE 
    
    ##FORMAT:
    START with either TRUE or FALSE, then detail your reasoning
'''
\end{verbatim}
\end{document}

\abstract{
Using the vertex model approach for braid representations, we compute polynomials for spin-$1$ placed on hyperbolic knots up to $15$ crossings.
These polynomials are referred to as $3$-colored Jones polynomials or adjoint Jones polynomials.
Training a subset of the data using a fully connected feedforward neural network, we predict the volume of the knot complement of hyperbolic knots from the adjoint Jones polynomial or its evaluations with $99.34\%$ accuracy.
A function of the adjoint Jones polynomial evaluated at the phase $q=e^{ 8 \pi i  \over 15 }$ predicts the volume with nearly the same accuracy as the neural network.
From an analysis of $2$-colored and $3$-colored Jones polynomials, we conjecture the best phase for $n$-colored Jones polynomials, and use this hypothesis to motivate an improved statement of the volume conjecture.
This is tested for knots for which closed form expressions for the $n$-colored Jones polynomial are known, and we show improved convergence to the volume.

%\VJ{Please update author affiliations.}
}

\begin{document}

\maketitle

\section{Introduction and summary}
Chern--Simons theory is arguably the simplest non-trivial quantum field theory.
Natural operators in this three-dimensional topological field theory are Wilson loops:
\begin{equation}
U_R(\Gamma) = \text{tr}_R\; \mathcal{P} \exp \big( i \oint_\Gamma A \big) \,,
\end{equation}
traces of path ordered exponentials of the holonomy of the gauge field along paths $\Gamma$.
The connection $A = A_\mu\, dx^\mu$ is a Lie algebra valued $1$-form and $R$ denotes a representation of the associated gauge group $G$.
Suppose we let $\Gamma$ be a knot $K$, which is an embedding of $S^1$ into a three-manifold, which we take to be $S^3$.
Colored Jones polynomials $J_{n}(K; q)$ are given by the expectation value of the Wilson loop operator along the knot $K$, with $R$ being a $n$-dimensional representation of $SU(2)$~\cite{Witten:1988hf}.
More precisely, we normalize as follows:
\begin{equation}
J_{n}(K; q) := \frac{\int_\mathcal{U} U_n(K)\; e^{iS_\text{CS}}}{\int_\mathcal{U} U_n(\bigcirc)\; e^{iS_\text{CS}}} = \frac{\langle U_n(K) \rangle}{\langle U_n(\bigcirc) \rangle} \,,
\end{equation}
where the Chern--Simons action
\begin{equation}\label{eq:SCS}
S_\text{CS} = \frac{k}{4\pi} \int_{S^3} \big( A\wedge dA + \frac23 A\wedge A\wedge A \big) \,,
\end{equation}
$\mathcal{U}$ is the space of $SU(2)$ connections modulo gauge transformations, and $\bigcirc$ denotes the unknot.
The gauge invariance of~\eqref{eq:SCS} requires the integer quantization of the Chern--Simons level, $k\in \mathbb{Z}$.
The $J_{n}(K; q)$ are Laurent polynomials in a variable $q$, which is related to the Chern--Simons coupling:
\begin{equation}
q := \exp \big( {2 \pi i\over k+2} \big) \,.
\end{equation}
These polynomials are topological invariants of the knot, meaning they are independent of how the knot is drawn and the metric on the underlying three-manifold.
The $n=2$ case was initially obtained from finite dimensional von Neumann algebras~\cite{jones1985polynomial,jones1987hecke} and then interpreted combinatorially~\cite{kauffman1987state}.
Skein relations make manifest that the coefficients in the polynomial are integer valued.
This is also established in Khovanov homology~\cite{khovanov2000categorification,bar2002khovanov}.

A hyperbolic knot is one for which the knot complement, obtained from excising a tubular neighborhood around the knot in $S^3$, admits a complete Riemannian metric of constant negative curvature whose uniqueness is ensured by Mostow--Prasad rigidity.
The volume of the knot complement, $V(S^3\backslash K)$, computed using this metric is a knot invariant~\cite{thurston_geometry_topology_three_manifolds}.
At low crossing number, nearly every knot is hyperbolic, but the proportion of hyperbolic knots does not approach unity in the large crossing number limit~\cite{belousov2019hyperbolicknotsgeneric}.
Thus, in this work and elsewhere, caution is to be applied in extrapolating phenomena from datasets that are inherently atypical.

Kashaev conjectured~\cite{kashaev1995link} that the colored Jones polynomial in the large color ($n\to\infty$) limit is related to $V(S^3\backslash K)$:
\begin{equation}\label{eq:vc}
	\lim_{n\to\I} \frac{\log|J_n(K;\omega_n)|}n = \frac1{2\pi} V(S^3\backslash K) \,, \quad \text{where} \quad \omega_n := e^{2\pi i/n} \,.
\end{equation}
(See also~\cite{murakami2001colored, Gukov:2003na}.)
This volume conjecture relates a quantum invariant, the colored Jones polynomial, to a classical geometric quantity, the volume.
The semiclassical limit where the volume conjecture is realized is, in fact, a double scaling limit~\cite{witten2011analytic}:\footnote{
As explained in~\cite{Craven:2020bdz}, it is both convenient and natural for us to define $\gamma$ in this manner.
The semiclassical limit is alternatively phrased as $q^n = e^{2\pi i \widetilde\gamma}$, which puts $\widetilde\gamma := n/(k+2) = \text{constant}$ as $n\to\infty$~\cite{witten2011analytic}.
}
\begin{equation}
\gamma := \frac{n-1}{k} \to 1 \quad \text{as} \quad n\,,\; k \to\infty \,.
\end{equation}
The analytic continuation of Chern--Simons theory~\cite{witten2011analytic} suggests the possibility of a non-linear relation between the colored Jones polynomial and the hyperbolic volume for every color $n$.\footnote{
See also~\cite{Garoufalidis_2011}, which, exploiting the cyclotomic expansion~\cite{Habiro_2007}, discusses the asymptotics of colored Jones polynomials.}
In the simplest case, such a relation was deduced from a dataset of knots up to $16$ crossings for which fundamental representation Jones polynomials $J_2(K;q)$ have been tabulated~\cite{KnotAtlas,linkinfo,SnapPy}.
Ref.~\cite{Jejjala:2019kio} showed that a deep neural network could predict the hyperbolic volume from the coefficients and the maximum and minimum degrees of the Jones polynomials to better than $98\%$ accuracy.
Since two knots can have the same fundamental representation Jones polynomial and different hyperbolic volumes, and when this happens the volumes differ by about $3\%$, the neural network performs about as well as can be expected.
In~\cite{Craven:2020bdz}, the authors used feedforward neural networks to determine that a function of $|J_2(K; e^{3\pi i/4})|$ approximates the volume with nearly the same accuracy.
Translating $e^{3\pi i/4}$ to a Chern--Simons coupling, we associate the phase in this evaluation to the fractional level $k=\frac23$.

The explanation for why such a formula works so well follows from Witten's analytic continuation of Chern--Simons theory~\cite{witten2011analytic}.
In the saddle point approximation to the path integral of the analytically continued Chern--Simons theory, there are particular flat $\mathfrak{sl}(2,\mathbb{C})$ connections of interest.
Heuristically, such geometric connections yield a semiclassical contribution of the form
\begin{equation}
Z \sim e^{iS(\mathcal{A}_+)} \big( 1 - e^{2\pi i k} \big)
\end{equation}
to the partition function.
The approximation formula applies for levels $k$ for which $\mathcal{A}_+$ makes a contribution to the Chern--Simons path integral for a large fraction of knots~\cite{Craven:2020bdz}.
Phases corresponding to integer level do not have access to this information, so the analytic continuation is crucial to recovering the volume.

As the dimension of the $SU(2)$ representation $n$ becomes large, the higher colored Jones polynomials are expected to discriminate knots with different hyperbolic volumes.
\textit{A priori}, especially for small $n$, it is not clear whether this improvement in performance occurs each time we increment the color.
Closed form expressions or formul\ae\ in terms of $q$-Pochhammer symbols exist for colored Jones polynomials for some families of knots like torus knots $T(2,2p+1)$ and twist knots $K_p$,
for instance.  The well-known lowest crossing hyperbolic knot, called the figure-eight knot, $4_1\equiv K_{-1}$, has colored Jones polynomial  given in~\cite{Le:2000, Habiro:2000}.
Another knot for which we can readily compute colored Jones polynomials for any $n$ is $K_0$, which is defined as the closure of the $3$-strand braid $\sigma_1^2 (\sigma_1 \sigma_2)^8$ and was studied in~\cite{Garoufalidis:2004}.
This is a knot with $18$ crossings, but a tetrahedral decomposition of its complement is accomplished with only four ideal tetrahedra.
In this case, numerics show that~\eref{eq:vc} does indeed converge to the volume, but this convergence is slow and non-monotonic. It is so far an open question whether this behavior is characteristic or an aberration.

There are $313{,} 210$ hyperbolic knots up to fifteen crossings~\cite{hoste1998first}.
In this work, we construct a partial dataset of $177 {,}316$ adjoint representation ($n=3$) Jones polynomials for these knots using the vertex model~\cite{Akutsu:1987dz,Ramadevi2017}.
These correspond to the closures of $m$-stranded braids for $m\le 7$.
The error in the neural network prediction of the volume of the knot complement using the adjoint polynomial invariants drops to $0.4\%$.
The approximation formula here uses an evaluation of the $n=3$ colored Jones polynomial at the phase $e^{8\pi i/15}$, corresponding to Chern--Simons level $k=\frac74$.
Based on these experiments, we conjecture the formul\ae
\begin{equation}        \label{eq:best-phase-guess}
q(n) = \exp\left( 2\pi i \frac{n+1}{n(n+2)} \right) \,, \qquad k(n) = \frac{n^2-2}{n+1}
\end{equation}
as the relevant phase and fractional level as a function of $SU(2)$ representation.

This paper employs machine learning to correlate the new dataset of $3$-colored Jones polynomials to the volume of the knot complements of hyperbolic knots.
After~\cite{hughes2016neural}, the utility of machine learning as a tool in low dimensional topology is by now well established.
Representative works in this direction include~\cite{Jejjala:2019kio,levitt2022big,Gukov:2020qaj,Craven:2020bdz,dlotko2023mappertypealgorithmscomplexdata,davies2021advancing,Craven:2021ckk,Craven:2022cxe,Gukov:2023kvx,Gukov:2024buj,Gukov:2024opc}.

The organization of this paper is as follows.
In Section~\ref{sec:data}, we discuss the dataset of $3$-colored Jones polynomials that we generate.
The method for constructing these is reviewed in Appendix~\ref{sec:app}.
In Section~\ref{sec:analysis}, we examine features of $3$-colored Jones polynomials and compare with $2$-colored Jones polynomials.
In particular, we look at their zeros and the statistics of the degrees, the lengths of the polynomials, and evaluations at special points motivated by the volume conjecture.
In Section~\ref{sec:ml}, we apply machine learning to the dataset of $3$-colored polynomials using the degrees and coefficients and using evaluations at phases as inputs to a neural network.
In Section~\ref{sec:newvc}, from data for Jones polynomials in the fundamental and adjoint representations of $SU(2)$, we make a hypothesis for the best phase for an $n$-dimensional representation of $SU(2)$ and test an improved statement of the volume conjecture with known formul\ae\ for the colored Jones polynomials for $4_1$ and $K_0$.
Our code is available on \href{https://github.com/roypratik92/ColoredJonesML}{GitHub}~\cite{github}, and the data are available on \href{https://doi.org/10.5281/zenodo.1490093}{Zenodo}~\cite{data}.

\section{Comments about data}\label{sec:data}

There are a total of $12{,} 965$ knots up to $13$ crossings, of which ten are non-hyperbolic.
At $14$ crossings, there are $46{,} 969$ hyperbolic knots, at $15$ crossings there are $253{,} 285$ hyperbolic knots, and at $16$ crossings there are $1{,} 388{,} 694$ hyperbolic knots.
In total, we have $1{,} 701{,} 903$ hyperbolic knots up to $16$ crossings~\cite{hoste1998first}.
Our dataset has $J_2$ for all of these.
Several knots, even those whose complements have different volumes, can have the same Jones polynomial.
There are $841{,} 139$ unique polynomials in the dataset.

For $J_3$, our dataset only contains the polynomials of knots which are the closures of $m$-strand braids, where $m\leq 7$.
Therefore, up to $13$ crossings, we have $J_3$ for $11{,} 941$ knots.
At $14$ crossings, we have $J_3$ for $18{,} 353$ hyperbolic knots.
At $15$ crossings, we have $J_3$ for $147{,} 022$ hyperbolic knots.
The results presented below thus use a dataset with a total of $177,316$ $J_3$ polynomials for hyperbolic knots up to $15$ crossings of which $164,455$ are unique.
It is difficult to extrapolate based on an incomplete dataset, but a larger fraction of knots are distinguished by $J_3$ in comparison to $J_2$ for the adjoint polynomials we can calculate.
This observation aligns with the expectation of the volume conjecture that larger colors should split degeneracies in the lower colored data.
This information is summarized below in Table~\ref{table:knotdata}.

\begin{table}[h!]
\centering
\begin{tabular}{c|cccc}
\hline
\textbf{Crossings} & \textbf{Total knots} & \textbf{Hyperbolic knots} & \textbf{$J_2$ Computed} & \textbf{$J_3$ Computed} \\
\hline
$\leq 13$ & 12{,}965 & 12{,}955 & 12{,}955 & 11{,}941 \\
14        & 46{,}972      & 46{,}969  & 46{,}969  & 18{,}353 \\
15        & 253{,}293      & 253{,}285 & 253{,}285 & 147{,}022 \\
16        & 1{,}388{,}705      & 1{,}388{,}694 & 1{,}388{,}694 & 0 \\
\hline
\end{tabular}
\caption{Summary of total knots, hyperbolic knots, and the numbers of knots for which $J_2$ and $J_3$ are known, organized by crossing number.}
\label{table:knotdata}
\end{table}

For our machine learning experiments, we use the $1{,} 701{,} 903$ $J_2$ polynomials and the $177{,} 316$ $J_3$ polynomials when considering them separately.
When we need to compare $J_2$ and $J_3$, or when we use their combination for machine learning, we only use the knots for which we know both $J_2$ and $J_3$.  

Note that since we are mostly interested in phases between $0$ and $e^{i\pi}$, we express the phases as $e^{2\pi ix}$ using the variable $x\in[0,\frac12]$ in what follows, including in the plots.

From the beginning, knot tables have listed knots by their crossing number $c$, and certainly crossing number is a reasonable proxy for the complexity of a knot.
It is far from clear, however, whether this is the best organizing principle there is.
In the case of the volume, it appears to be a useful one.
Figure~\ref{fig:volumes} shows a linear growth of the volume of the knot complement of a hyperbolic knot with the crossing number.
The variance is approximately constant independent of crossing number.
A hint for why this is the case may be found in~\cite{yokota2000volume}, where volumes are described in terms of ideal tetrahedra at the crossings.\footnote{
We thank Sergei Gukov for a discussion about this and for pointing us to the relevant literature.}
%\VJ{Is this the correct reference?}

\begin{figure}[h]
\centering
\includegraphics[width=.8\textwidth]{volume.pdf}
\caption{\textsf{Mean volume as a function of crossing number.}} 
\label{fig:volumes}
\end{figure}

\section{Numerical analysis of the polynomial data}\label{sec:analysis}

\subsection{Zeros of the polynomials}
The zeros of polynomials, such as the Alexander and Jones polynomials, provide deep insights into the structure of knots and links.
In 2022, Hoste, based on computer experiments, conjectured that for alternating knots, the real part of any zero of the Alexander polynomial satisfies $\operatorname{Re}(z) > -1$.
The conjecture was proven for certain knot classes~\cite{Ishikawa2019, Stoimenow2011,alsukaiti2024alexander,HM2013}, though counterexamples to the general conjecture were found~\cite{HIRASAWA201948}, prompting an exploration of the distributions of zeros of Alexander polynomials.
Similarly, the zeros of Jones polynomials reveal interesting patterns.
For instance, they are dense in the complex plane for certain links, such as pretzel links~\cite{Jin2010}, though they tend to cluster around the unit circle for other classes of links~\cite{AI2005,XF2010}.
Motivated by these results and others~\cite{Andersen:2004}, we study the distribution on the complex plane of zeros of Jones polynomials.

In Figure~\ref{fig:J2zeros}, we plot the zeros of the $1{,}701{,}903$ fundamental Jones polynomials in our dataset on the complex plane.
We have restricted the range of the plot so that the real parts of the roots lie in the interval $(-2,2)$ and the imaginary parts lie in the interval $[0,2)$.
However, there aren't many zeros outside of the plotted range except those that lie on the real axis. 

In Figure~\ref{fig:J3zeros}, we plot the zeros of the $177{,} 316$ adjoint Jones polynomials in our dataset on the complex plane.
The range of $z$ in Figure~\ref{fig:J3zeros} is restricted as in Figure~\ref{fig:J2zeros}.
Note that in both Figures~\ref{fig:J2zeros} and~\ref{fig:J3zeros}, there are a lot of zeros at $0$; these are not visible due to our methods of plotting.
In particular, $28{,} 345$ of the fundamental Jones polynomials of the $313{,} 209$ hyperbolic knots up to $15$ crossings have zeros at zero.
Similarly, $36{,} 918$ of the adjoint Jones polynomials of $177{,} 316$ hyperbolic knots in the dataset have zeros at zero.

In the inset at top left of Figure~\ref{fig:J3zeros}, we zoom in on one feature of the zeros of $J_3$.
We find that the $J_3$ polynomials in our dataset do not have any zeros in a neighborhood of $1$, but there is some interesting substructure there.
(The Jones polynomial never has a zero at $1$.)

%\PR{Maybe add comments about the distributions of zeros of polynomials in general, in comparison with, e.g.,~\cite{Mezincescu:1996, Tao:2013}.}

\begin{figure}[h] 
  \begin{center}
    \includegraphics[width=\textwidth]{zerosJ2.png}
  \end{center}
  \caption{\textsf{Zeros of $J_2$ polynomials in our dataset in the (upper-half) complex $q$-plane.}}  \label{fig:J2zeros}
\end{figure}
\begin{figure}[h] 
  \begin{center}
    \stackinset{l}{4pt}{t}{4pt}{\frame{\includegraphics[width=.35\textwidth]{zerosJ3AR1.png}}}{\includegraphics[width=\textwidth]{zerosJ3.png}}
  \end{center}
  \caption{\textsf{Zeros of $J_3$ polynomials in our dataset in the (upper-half) complex $q$-plane. Inset: Zoom in of the plot to the $x$-axis interval $(0.8,1.2)$ and $y$-axis interval $(0.0,0.05)$.}}  \label{fig:J3zeros}
\end{figure}

\subsection{Degrees of polynomials}
We also study the distribution of the maximum and minimum degrees of the Jones polynomials.
Figure~\ref{fig:j2j3minmaxdegs} shows the distributions of the minimum and maximum degrees of the polynomials and the correlations between minimum and maximum degrees for the different colors.

\begin{figure}[h]
\begin{tabular}{cc}
  \subcaptionbox{\textsf{Minimum degrees of $J_2$ and $J_3$.}}{\includegraphics[width=75mm]{j2j3mindegs.png}}  &   
  \subcaptionbox{\textsf{Maximum degrees of $J_2$ and $J_3$.}}{\includegraphics[width=75mm]{j2j3maxdegs.png}} \\[15pt]
  \subcaptionbox{\textsf{Minimum and maximum degrees of $J_2$.}}{\includegraphics[width=75mm]{j2minmaxdegs.png}} &    \subcaptionbox{\textsf{Minimum and maximum degrees of $J_3$.}}{\includegraphics[width=75mm]{j3minmaxdegs.png}} 
\end{tabular}
\caption{\textsf{Correlations between degrees of polynomials. Darker shades on points mean higher frequency.}}
\label{fig:j2j3minmaxdegs}
\end{figure}

We find that the minimum and maximum degrees of the polynomials have a roughly Gaussian distribution, as seen in Figure~\ref{fig:j2j3minmax-dist}.


\begin{figure}[h] 
	\centering
	\begin{minipage}{0.45\textwidth}
  	  \begin{center}
    	  \includegraphics[width=\textwidth]{j2minmaxdist.pdf}
  	  \end{center}
	\end{minipage}
	\begin{minipage}{0.45\textwidth}
  	  \begin{center}
    	  \includegraphics[width=\textwidth]{j3minmaxdist.pdf}
  	  \end{center}
	\end{minipage}
  \caption{\textsf{Distribution of minimum (blue) and maximum (green) degrees of $J_2$ (left subfigure) and $J_3$ (right subfigure). The $y$-axis corresponds to the number of times the given degree occurs as the minimum or maximum in our dataset.}}  \label{fig:j2j3minmax-dist}
\end{figure}

In Figure~\ref{fig:j3max-fit}, we plot a Gaussian function with mean and variance determined by the distribution of the maximum degrees of $J_3$ polynomials as an example.

We also note that the minimum and maximum degrees of the polynomials grow at a roughly similar rate, at least within our dataset.
The means of the minimum and maximum degrees of $J_2$ are $-5.86$ and $8.04$, and the modes are $-6$ and $8$.
For $J_3$ the minimum and maximum degrees have means $-15.03$ and $24.73$, and the modes are $-16$ and $18$.
The ratio of the mean minimum degrees of $J_3$ and $J_2$ is $2.56$.
The ratio of the mean maximum degrees of $J_3$ and $J_2$ is $3.07$.

As seen in Figure~\ref{fig:length}, the length of the Jones polynomial (\textit{i.e.}, the maximum degree minus the minimum degree plus one) follows a linear relationship with crossing number.
The standard deviations are computed using the set of available polynomials for the knots in the dataset at each crossing number.
The linear fit may be motivated by the calculation of the Jones polynomials in terms of traces of $R$-matrices defined at each of the crossings.
Normalizing appropriately, this gives a standard length.
To deviate from this standard length requires, \textit{e.g.}, detailed cancellations, and this is generically rare.\footnote{
We thank Sergei Gukov for discussing this result with us and suggesting this explanation.}
If one were to assign to every knot the mean volume of all the knots in the dataset, the error in the prediction is about $12\%$.
The linear relationships in Figures~\ref{fig:volumes} and~\ref{fig:length} and the correlations therein may describe the next correction, and machine learning the volume conjecture is a refinement of this.




\begin{figure}[h] 
  \begin{center}
    \includegraphics[width=.7\textwidth]{j3max-gaussian-fit.pdf}
  \end{center}
  \caption{\textsf{Plot of maximum degrees of $J_3$ with a Gaussian fit in yellow for comparison.}}  \label{fig:j3max-fit}
\end{figure}

\begin{figure}[h]
\centering
\includegraphics[width=.8\textwidth]{length.pdf}
\caption{\textsf{Length of fundamental and colored Jones polynomials as a function of crossing number.}} 
\label{fig:length}
\end{figure}

\subsection{Evaluation of polynomials}

As was observed in~\cite{Craven:2020bdz}, we also find a quadratic relation between the evaluation of $J_2$ at $e^{i\pi}=-1$ and of $J_3$ at $e^{2\pi i/3}$.
These are the phases implicated by the statement of the volume conjecture,~\eref{eq:vc}.
This is plotted in Figure~\ref{fig:quadratic-relation}.
We find that the curve $10^{1.155}y=x^2$ fits the data nicely, as also shown in the figure.

\begin{figure}[h] 
  \begin{center}
    \includegraphics[width=.7\textwidth]{quadrel.png}
  \end{center}
  \caption{\textsf{Plot of evaluations of $J_3$ at $e^{2\pi i/3}$ vs.\ evaluations of $J_2$ at $e^{-i\pi}=-1$. In blue is plotted the curve $x^2/10^{1.155}$, which fits the data nicely. This is in accord with a similar plot with a smaller dataset in~\cite{Craven:2020bdz}.}}  \label{fig:quadratic-relation}
\end{figure}

\section{Results from training neural networks}\label{sec:ml}

We now present results from a variety of supervised machine learning (ML) experiments on the fundamental and adjoint Jones polynomials, with the aim of determining correlations between the polynomials and the hyperbolic volumes of knots.
We then use the results to arrive at a guess for the best phase for use in the volume conjecture and check the predictions from this phase.

\subsection{Using polynomials}
A review of neural networks and their utility in mathematics is~\cite{williamson2024deep}.
Here, we present results obtained by using fully connected feedforward neural networks (NNs) with five hidden layers.
The architecture of this deep neural network can be represented as%\\ \mbox{$(\text{Input},100,\text{ Ramp}, 300,\text{ Ramp},300,\text{ Ramp},150,\text{ Ramp},75,\text{ Ramp},\text{Output})$}, 
\[
    (\text{Input},100,\text{ReLU}, 300,\text{ReLU},300,\text{ReLU},150,\text{ReLU},75,\text{ReLU},\text{Output}) \, ,
\]
where the numbers are the numbers of neurons in each of the hidden layers and ReLU is the $\text{max}(0,x)$ non-linearity.
We use the ADAM optimizer with an adjustable learning rate.
The results are insensitive to minor tweaks to the neural network architecture.
The input to the networks are vectors encoding the content of the Jones polynomials.
We formed the vectors by taking the first two components of a vector to be the minimum and maximum degrees of the corresponding polynomial and the next components of the vector to be coefficients of all monomials from lowest to highest degree.
We then padded the vectors with zeros on the right to make them of uniform length.
For $J_2$, we pad the vectors to make them of length $19$.
For $J_3$, we pad these vectors to have length $48$.
We also present results obtained by training deep neural networks on vectors formed from joining the vectors for $J_2$ and $J_3$ of a given knot.
In this case we pad $J_2$ vectors to have the same length as $J_3$ vectors, resulting in vectors of length $96$ in this case.

Using $J_3$ polynomials of knots up to $15$ crossings, the mean relative error\footnote{All the reported relative errors are for performance over test samples.} over three independent runs of training (with $100$ epochs each) was $0.40\%$.
In the experiments, we used $75\%$ of the knots for training, $10\%$ for validation, and the remaining $15\%$ for testing the trained network.
When the $J_3$ polynomials of knots with only $14$ and $15$ crossings were used, the mean relative error over three runs was $0.37\%$.
Using only knots of $15$ crossings, the mean relative error over three runs was $0.35\%$.
A plot of predictions from a neural network trained on $J_3$ polynomials is included in Figure~\ref{fig:J3poly131415preds}.

For machine learning using $J_2$ polynomials of the $1{,} 701{,} 903$ hyperbolic knots up to $16$ crossings, the mean relative error over three independent runs of training was $1.65\%$.
For comparison, a plot of predictions from a neural network trained on $J_2$ polynomials of all hyperbolic knots up to $16$ crossings is included in Figure~\ref{fig:J2poly131415preds} (\textit{cf.},~\cite{Jejjala:2019kio, Craven:2020bdz}).
With training on the hyperbolic knots up to $15$ crossings, the mean error over three training runs was $1.86\%$.
When restricted to the $177{,}316$ knots for which we know the $J_3$ polynomials, this error does not change significantly.
When the $J_2$ polynomials of knots with only $14$ and $15$ crossings were used, the mean relative error over three runs was $1.55\%$.
When knots of only $15$ crossings were used, the mean relative error over three runs was $1.48\%$.
When knots of only $16$ crossings were used, the mean relative error over three runs was $1.44\%$.

The results reported above seem to suggest that the correlations between Jones polynomials and the volumes of knots improves as the number of crossings of knots increases.
To check if this is indeed the case, we restricted to random samples of $100,000$ knots of $15$ and $16$ crossings each and trained neural networks on the $J_2$ of each set separately.
Over three training runs, the mean error from the neural networks trained on $15$ and $16$ crossing knots was $1.695\%$ and $1.697\%$ respectively.
It would thus seem that it is the greater number of knots at higher crossings that results in better performance for the neural networks.

For machine learning with $J_2$ and $J_3$ combined into one vector of coefficients, the mean error over three runs of training was $0.51\%$.
This is shown in the right plot in  Figure~\ref{fig:J2J3poly131415preds}.

To better compare neural networks trained on different colored polynomials, we trained and tested neural networks on the same dataset.
The mean relative errors of the trained networks using $J_2$, $J_3$, and joined $J_2$ and $J_3$ polynomials was respectively $1.85\%$, $0.62\%$, and $0.40\%$.

\begin{figure}[h] 
  \begin{center}
    \includegraphics[width=.7\textwidth]{J3-131415-preds.png}
  \end{center}
  \caption{{\textsf{Predicted vs.\ actual volumes using a neural network trained on $J_3$ for knots with up to $15$ crossings. Also included are the actual volumes in blue in the background. The color signifies the density of points, as depicted by the legend bar. The mean relative error of predictions (on the test set) by the neural network used to generate this plot was $0.66\%$. Note that here and in all similar plots below, the intensity of points signifies the density of knots in each pixel of the plot.}}}  \label{fig:J3poly131415preds}
\end{figure}

\begin{figure}[h] 
  \begin{center}
    \includegraphics[width=.7\textwidth]{J2NKpreds.png}
  \end{center}
  \caption{{\textsf{Predicted vs.\ actual volumes using a neural network trained on $J_2$ for knots with up to $16$ crossings. Also included are the actual volumes in blue. The mean relative error of predictions (on the test set) by the neural network used to generate this plot was $1.64\%$. This is consistent with the results reported in~\cite{Jejjala:2019kio, Craven:2020bdz}, where only $10\%$ of the dataset was used for training.}}}  \label{fig:J2poly131415preds}
\end{figure}

\begin{figure}[h] 
  \begin{center}
    \includegraphics[width=.7\textwidth]{J2-J3-131415-preds.png}
  \end{center}
  \caption{{\textsf{Predicted vs.\ actual volumes using a neural network trained on the joined vectors of coefficients formed from $J_2$ and $J_3$ polynomials, for knots with up to $15$ crossings. Also included are the actual volumes in blue. The mean relative error of predictions (on the test set) by the neural network used to generate this plot was $0.54\%$.}}}  \label{fig:J2J3poly131415preds}
\end{figure}

To summarize, we find that the $J_3$ polynomials predict the volumes with significantly more accuracy than $J_2$ polynomials.
The performance of joined $J_2$ and $J_3$ polynomials is roughly the same as that of just $J_3$ polynomials. 
Thus, despite the non-monotonicity reported in~\cite{Garoufalidis:2004} based on the study of a specific knot, it would appear that statistically the approach to the volume conjecture using a large dataset might be monotonic.

\subsection{Using evaluations}

In Figure~\ref{fig:J3-phases-rel-err}, we plot the relative error of the predictions of the neural network with respect to the actual volume as a function of the phase at which the $J_3$ polynomials are evaluated.
The network was given as inputs the real and imaginary parts of the evaluation of the polynomial at the phases.
We checked that giving the modulus and argument of the evaluation as input does not change the results significantly.
The best phase is seen to be $x\sim0.27$.
We also performed similar experiments with input the sum of $|J_2(K;e^{2\pi ix})|+|J_3(K;e^{2\pi ix})|$, and found a qualitatively similar plot.
This is expected since the evaluations of $J_3(K;q)$ tend to have larger magnitude than the evaluations of $J_2(K;q)$.
The existence of a good minimum is again explainable via the analytically continued Chern--Simons theory~\cite{witten2011analytic} and tracks the discussion in~\cite{Craven:2020bdz}.
It is not \textit{a priori} apparent why there should be two minima in the phase vs.\ error plot for the $3$-color Jones polynomial compared to only one minimum in the $2$-colored case --- \textit{cf.}, Figure~7 in~\cite{Craven:2020bdz}.

\begin{figure}[h]
    \centering
    \includegraphics[width=.6\textwidth]{J3-131415-evals.pdf} 
    \caption{\textsf{Relative error of predictions of neural network with input the evaluations $J_3(K;e^{2\pi i x})$. This is over three runs of training with $25$ epochs. The orange points have $x=(k+2)^{-1}$ for integer $k\in[0,7]$.}}
    \label{fig:J3-phases-rel-err}
\end{figure}

In Figure~\ref{fig:J3-evals-4by15}, we show the predictions for volume obtained by using a neural network trained on the evaluations $J_3(K;e^{2\pi ix})$ at $x=4/15\approx 0.26667$, which is the best phase in Figure~\ref{fig:J3-phases-rel-err}.
The mean relative error of the predictions of this neural network is $0.90\%$.
Thus evaluations of $J_3$ at this phase perform only slightly worse than the data containing the full polynomials.
Using the absolute value of evaluation of polynomials at $x=4/15$ to train neural networks, we find that the mean error over three runs increases slightly to $1.33\%$.

If we use evaluations of $J_3$ at $e^{2\pi i\,2/5}$, which is the other minimum in Figure~\ref{fig:J3-phases-rel-err}, to train a neural network for predicting the volume, the mean relative error on the test set is $3.60\%$.
When using the sum of evaluations at $e^{8\pi i/15}$ and $e^{4\pi i/5}$, the mean relative error improves, and becomes $1.61\%$.

Finally, in Figure~\ref{fig:off-circle}, we show the errors of predictions for neural networks trained using evaluations off the unit circle.
The error increases monotonically as the radial distance from the unit circle is increased on either side for the three phases that we have checked.

\begin{figure}[h] 
  \begin{center}
    \includegraphics[width=.6\textwidth]{J3eval-4by15.png}
  \end{center}
  \caption{\textsf{Predicted vs.\ actual volumes using a neural network trained on $J_3(K;e^{2\pi ix})$ with $x=4/15$. Also included are the actual volumes in blue. The mean relative error of predictions by the neural network used to generate this plot was $1.14\%$.}}  \label{fig:J3-evals-4by15}
\end{figure}
\begin{figure}[h] 
	\begin{center}
    \includegraphics[width=.5\textwidth]{off-circle-4by15.pdf}
    \end{center}
  \caption{\textsf{Plots of relative errors of neural networks trained on evaluations at points $z=|z|e^{2\pi ix}$ with different absolute values with fixed phase at $x=4/15$.}}  \label{fig:off-circle}
\end{figure}





\subsection{Symbolic formula for the volume}

Using the Mathematica function \texttt{NonlinearModelFit} with a guess
\begin{equation}
    a \log(b|J_3(K;e^{8\pi i/15})|+c)+d
\end{equation}
for the functional form and evaluations at $x=4/15$ as the input data to the function, we get the following function for volume,\footnote{
Using the Julia/Python package \texttt{PySR} for symbolic regression with $\exp$ and $\log$ as guesses for the functions involved, we get a similar formula as the one obtained using Mathematica.}
\begin{equation}			\label{eq-regression-result}
	\text{vol}_K = 3.25\log(|J_3(K;e^{8\pi i/15})|+36.97) - 1.72 \,.
\end{equation}
The mean relative error for predictions of volumes using this formula is $1.21\%$. As further measure of the quality of the fit, we use the coefficient of determination, usually denoted $R^2$, with $R^2=1$ signifying an exact formula.
This formula gives an $R^2$ value of $0.999625$ and is thus a very good fit.
Predictions from this formula are plotted in Figure~\ref{fig:nlm-fit} for a random sample of $25{,} 000$ knots.

\begin{figure}[t]
    \begin{center}
        \includegraphics[width=.5\textwidth]{NLM1-preds.png}
    \end{center}
    \caption{\textsf{Predictions from the symbolic expression~(\ref{eq-regression-result}).}}  \label{fig:nlm-fit}
\end{figure}

\section{Toward an improved volume conjecture}\label{sec:newvc}

In~\cite{Craven:2020bdz}, the phase at which the evaluation of the $2$-colored Jones polynomials was found to give a best fit to the volume was $e^{3\pi i/4}$.
We have presented evidence that $e^{8\pi i/15}$ is the corresponding best phase to evaluate $3$-colored Jones polynomials to train neural networks to predict the volumes of knot complements of hyperbolic knots.
These correspond, respectively, to fractional Chern--Simons levels $k = 2/3$ and $7/4$ and $\gamma = 3/2$ and $8/7$.
Using these two data points, we make the guess that for $n$-colored Jones polynomials, the best phase to use is
\begin{equation}        \label{eq:best-phase-guess-2}
    q(n) = \exp\left( 2\pi i \frac{n+1}{n(n+2)} \right).
\end{equation}
Correspondingly, we find that the fractional Chern--Simons level and $\gamma=(n-1)/k$ are
\begin{equation}
    k(n) = \frac{n^2-2}{n+1} \,, \qquad
    \gamma(n) = \frac{n^2-1}{n^2-2} \,.
\end{equation}
These formul\ae\ have the expected asymptotics.
In the large-$n$ limit,
\begin{equation}
q(n) \sim \exp \left( 2\pi i \left( \frac{1}{n}-\frac{1}{n^2} \right) \right) \,, \qquad
k(n) \sim (n-1)-\frac{1}{n} \,, \qquad
\gamma(n) \sim 1+\frac{1}{n^2} \,.
\end{equation}
This tells us the relations are in accord with the volume conjecture,~\eref{eq:vc}, which posits that at large-$n$, the phase at which the $n$-colored Jones polynomial is evaluated is the primitive $n$-th root of unity.
The machine learning experiments suggest that rather than evaluating at the phase $\omega_n = 2\pi i/n$, we should dress this as
\begin{equation}
2\pi i\; x(n) = \frac{n+1}{n+2}\; \omega_n \,.
\end{equation}
Thus, we propose that~\eref{eq:vc} be modified to read
\begin{equation}\label{eq:vc-new}
	\lim_{n\to\I} \frac{\log|J_n(K; q(n))|}n = \frac1{2\pi} V(S^3\backslash K) \,.
\end{equation}

We test this hypothesis using exact expressions for $n$-colored Jones polynomials available for the two hyperbolic knots $4_1$ and $K_0$.

The $n$-colored Jones polynomials for $4_1$ are~\cite{Le:2000, Habiro:2000, Murakami:2022}
\begin{equation}
    J_n(4_1;q) = \sum_{k=0}^{n-1} q^{-nk} \prod_{l=1}^k (1-q^{n+l}) (1-q^{n-l}) \,.
\end{equation}
From Snappy, the volume of $4_1$ is $2.02988321282$.

The $(n+1)$-colored Jones polynomials of $K_0$ are~\cite{Garoufalidis:2004}
\begin{align}
  J_{n+1}(K_0;q) =&\ \frac1{[n+1]}\sum_{k=0,2}^{2n}\sum_{l=|n-k|,2}^{n+k}\sum_z(-1)^{\frac k2+z}q^{-\frac38(2k+k^2)+\frac78(2l+l^2)-\frac{51}8(2n+n^2)}\frac{[k+1][l+1]}{[n+\frac k2+1]!}   \nn \\
  & \quad \times \begin{bmatrix} \frac{k+l-n}2 \\ \frac{n+2k+l}2 - z  \end{bmatrix} \begin{bmatrix} \frac{n+l-k}2 \\ \frac{3n+l}2 - z  \end{bmatrix} \begin{bmatrix} \frac{n+k-l}2 \\ n+k - z  \end{bmatrix} \lB\frac k2 \rB!^2 \frac{\lB n-\frac k2 \rB!}{\lB z-\frac{n+k+l}2 \rB!}\frac{[z+1]!}{\lB \frac{n+k+l}2+1 \rB!} \ ,   
\end{align}
where $\sum_{k=a,2}^b$ means summation with step $2$ from $a$ to $b$, and the summation over $z$ has the limits
\begin{equation}
    \max\lb n+\frac k2,\frac{n+k+l}2 \rb \leq z\leq \min\lb \frac{n+2k+l}2,\frac{3n+l}2, n+l \rb \,.
\end{equation}
We note that the volume of $K_0$ is $3.474247$.

We use the knowledge of these exact polynomials to test whether the conjectured phase $q(n)$ in~(\ref{eq:best-phase-guess-2}) results in improvements upon the prediction of the volume at finite $n$.
In Figure~\ref{fig:K0-41}, along with the volumes of the knots and the left hand side of~\eref{eq:vc} (the volume conjecture), we plot the values of
\begin{equation}\label{eq:vc2}
	v(n) = \frac{2\pi\log|J_n(K;q(n))|}n \,.
\end{equation}
We see that the phase $q(n)$ performs much better for both the figure-eight knot and the knot $K_0$.
We can additionally envision a subleading shift in the denominator to achieve a better convergence, but we have not systematically investigated this possibility.
The convergence of $v(n)$ to the volume is still not monotonic.
Thus, it is impressive that the neural network does so much better using the adjoint Jones polynomial, and this should be understood.
It would also be worthwhile to test whether our conjectured phase continues to perform well for other knots at higher colors.
This requires a database of higher colored Jones polynomials.
Work on this is underway, and we hope to report on progress soon.

\begin{figure}
	\centering
    \subfloat[$4_1$]{%
        \includegraphics[width=.45\textwidth]{img41.png}	\label{fig:41}%
    }
    	\subfloat[$K_0$]{%
    	\includegraphics[width=.45\textwidth]{imgK0.png}     \label{fig:K0}
	} 
    \caption{\textsf{Comparison of predictions using the phase $\omega_n$ from~(\ref{eq:vc}) and using the phase from~(\ref{eq:best-phase-guess}), along with the actual volumes of the knots $4_1$ (left) and $K_0$ (right).}}
    \label{fig:K0-41}
\end{figure} 


\section*{Acknowledgements}
The content of this paper was presented at a workshop on the ``Generalized Volume Conjecture'' at IIT Bombay in November 2024 and the String Data meeting at the Yukawa Institute for Theoretical Physics in Kyoto in December 2024.
We thank the organizers and participants at these meetings for their generous feedback.
We are especially grateful to Sergei Gukov and Hisham Sati for discussions.
We thank Aditya Dwivedi for conversations about improving the databases of colored Jones polynomials.
MH is supported by the National Science Foundation (DMS-2213295), and thanks the Dublin Institute for Advanced Studies and the Max Planck Institute for Mathematics for hosting him during work on this paper.
VJ is supported by the South African Research Chairs Initiative
of the Department of Science and Innovation and the National Research Foundation.
P.~Ramadevi would like to thank the IITB-IOE funding:  `Seed funding for Collaboration and Partnership Projects (SCPP) scheme - Phase III'.
P.~Roy is supported by an NRF Freestanding Postdoctoral Fellowship. P.~Roy would like to thank Chennai Mathematical Institute for hospitality while this work was being completed.
VKS is supported by the ‘Tamkeen under the NYU Abu Dhabi Research Institute grant CG008 and ASPIRE Abu Dhabi under Project AARE20-336. This work was supported in part through the High-Performance Computing resources at New York University Abu Dhabi.

\appendix
\section{Knot polynomials from vertex models}\label{sec:app}

There are various methods to compute polynomial invariants for knots and links~\cite{Witten:1988hf,KirResh,RT1,RT2,Itoyama:2012qt,Kaul:1992rs,Devi:1993ue,RGK,DMMMRSS:2017,DMMMRSS:2018}. For our calculations of the adjoint Jones polynomials of knots with $14$ and $15$ crossings, we adopted a vertex model approach, which we found to be significantly faster than the \texttt{KnotTheory} package for Mathematica available at KnotAtlas~\cite{KnotAtlas}. We now briefly review our approach to constructing Jones polynomials using braid group representations obtained via vertex models. This material is well known in the literature, and we follow the review~\cite{Wadati:1989ud}. 

Vertex models on square lattices are statistical mechanical systems where each vertex has four associated edges, and the states on the four edges determine the Boltzmann weights of the vertices. Each vertex can be thought of as describing two-to-two scattering with particles of various charges. One can study vertex models where the edge variables are allowed to take $N$ possible states. Thinking of these states as ``spin'' charges, and demanding appropriate conservation, one obtains what are called $N$-state vertex models with spin $s=(N-1)/2$. These models can all be shown to be integrable. 

The braid group over $n$ strands, denoted $B_n$, has $n-1$ generators $b_i$, $i=1,\ldots,n-1$, subject to the relations
\begin{eqnarray}
    b_i\, b_{i+1} \, b_i = b_{i+1} \,b_i \, b_{i+1} & \text{ for } & 1\leq i \leq n-2 \,, \text{ and } \nonumber \\
    b_i\, b_j = b_j\, b_i & \text{ for } & |i-j|\geq 2 \,.
\end{eqnarray}
Any $n$-braid is expressed as a word in $B_n$, \textit{e.g.}, $b_1^{-1}b_2b_3b_2^{-1}$, up to the above relations. Given a braid, one forms an oriented link by taking its closure, \textit{i.e.}, by identifying opposite ends of the braid. Conversely, any oriented link is represented (non-uniquely) by some closed braid. Markov's theorem states that two closed braids represent the same ambient isotopy class of links if and only if the braids can be transformed into one another by a sequence of ``Markov moves'' of type I and II, also called conjugation and stabilization. The moves are
\begin{eqnarray}
	\text{(I)} &\quad& AB\to BA \,, \text{ for } A,B\in B_n \,, \\
	\text{(II)} &\quad& A\to A\,b_n^{\pm1}\,, \text{ for } A\in B_n, \ b_n\in B_{n+1} \,.
\end{eqnarray}
One can then obtain invariant polynomials for links by constructing a representation of the braid group, and then defining Markov move invariant polynomials using the representation.

\subsection{Akutsu--Wadati formula}

Using the Boltzmann weights of vertex models, one can construct the so-called Yang--Baxter operators satisfying Yang--Baxter equations. Upon ``asymmetrizing'' and taking appropriate limits, these Yang--Baxter operators furnish representations $G_i$ of braid operators $b_i$, of the form 
\begin{equation}
	G_i = \mathbb{I}_1\otimes\mathbb{I}_2\otimes\cdots\otimes\mathbb{I}_{i-1}\otimes R_{i,i+1}\otimes\mathbb{I}_{i+1}\otimes\cdots\otimes\mathbb{I}_{n}\,,
\end{equation}
where the matrix $R$ acts on two strands of the braid, and $\mathbb{I}_j$ denotes the appropriate identity matrix acting on the strand $j$. We note that the matrix $R$, called the (braided) $R$-matrix, is fully determined by the specification of a given integrable statistical model~\cite{Baxter:1982zz}. 

Let $G_i$ be such a representation of $b_i\in B_n$, and let $\a(\cdot)$ denote a link polynomial. From the above discussion, we know that $\a(\cdot)$ must satisfy the following conditions:
\begin{eqnarray}
	\text{(I)} &\quad& \a(AB) = \a(BA) \,, \text{ for } A,B\in B_n \,, \\
	\text{(II)} &\quad& \a(A)= \a(AG_n)=\a(AG_n^{-1}) \,, \text{ for } A\in B_n,G_n\in B_{n+1} \,.
\end{eqnarray}
Such an $\a(\cdot)$ can be constructed if we can construct a Markov trace, $\f(\cdot)$, satisfying
\begin{eqnarray}
	\text{(I)} &\quad& \f(AB) = \f(BA) \,, \text{ for } A,B\in B_n \,, \\
	\text{(II)} &\quad& \f(A G_n) = \t\f(A) \f(AG_n^{-1})=\bar\t\f(A) \,, \text{ for } A\in B_n\,,\,G_n\in B_{n+1} \,,
\end{eqnarray}
where $\t=\f(G_i),\ \bar\t=\f(G_i^{-1})$ for any $i$. In terms of the Markov trace $\f(\cdot)$, the link polynomial is given by the following (Akutsu--Wadati) formula,
\begin{equation}
	\a(A) = (\t\bar\t)^{-(n-1)/2}\lb\frac{\bar\t}\t\rb^{e(A)/2}\f(A) \,, \quad A\in B_n \,,
\end{equation}
where $e(A)$ is the sum of the exponents of $b_i$ in $A$. For example, for the braid word $A=b_1 b_2^{-1} b_1 b_2^{-1}$, $e(A)=0$.
For $N$-state vertex models, we have~\cite{Akutsu:1987dz,Akutsu:1987qs}
\begin{equation}
	\t = \frac1{1+q+\cdots+q^{N-1}}\,, \qquad \bar\t = \frac{q^{N-1}}{1+q+\cdots+q^{N-1}}\, .
\end{equation}
The Markov trace is given explicitly by
\begin{equation}
	\phi(A) = \text{Tr}(H \cdot A) \,,
\end{equation}
where $H$ is the tensor product of $n$ matrices $h$ of size $N\times N$,
\begin{equation}
	H = \underbrace{h \otimes h \otimes \dots \otimes h}_n \,.
\end{equation}
The matrix $h$ is diagonal, and is given for $N$-state vertex models by
\begin{equation}
	h = \t\,\text{diag}\,(1,q,\cdots,q^{N-1}) \,.
\end{equation}
The only remaining element is then the braided $R$-matrix, which needs to be computed just once for each $N$-state vertex model. 

It can be shown that the link invariant polynomials that we have constructed using the $N$-state vertex model and the Akutsu--Wadati formula correspond, for each $N$, to the $N$-dimensional Jones polynomials of links. Below, we give a few example calculations of Jones polynomials for the trefoil and figure-eight knots.

Before proceeding we make a remark about the complexity of this method. The matrix $H$ obtained from an $N$-state vertex model is of size $N^n\times N^n$ for a braid word of $n$ strands. Since the $R$-matrix acts on the Hilbert space of two ``particles'', it is a matrix of size $N^2\times N^2$. The braid generators $b_i$ will still have size $N^n\times N^n$. Because of this, it quickly becomes prohibitively expensive to use this method to calculate Jones polynomials both for knots with high braid indices and for higher colored representations. Our data set thus comprises knots with a braid index of at most $7$.

\subsection{Jones Polynomial ($N=2$, Six-Vertex Model)}

The two-state vertex model ($N=2$) has six different configurations on the edges surrounding a vertex, or equivalently six independent Boltzmann weights at each vertex. The model is thus also called the six-vertex model. The braided $R$-matrix obtained from this model is
\begin{equation}
R = \left(
\begin{array}{cccc}
 1 & 0 & 0 & 0 \\
 0 & 0 & -q^{1/2} & 0 \\
 0 & -q^{1/2} & 1-q & 0 \\
 0 & 0 & 0 & 1 \\
\end{array}
\right).
\end{equation}

\paragraph{Two-strand braids:} This $R$-matrix allows direct computation of polynomials for knots and links represented by two-strand braids such as the unknot, trefoil, and \(T(2, 2n)\) links. For the case of the trefoil, whose braid word is \(b_1^3\), we can write the matrix for \(b_1^3\) and multiply it with $H=h \otimes h$, which for two strands takes the form
\begin{equation}
H = h \otimes h = \begin{pmatrix}
 \frac{1}{(1+q)^2} & 0 & 0 & 0 \\
 0 & \frac{q}{(1+q)^2} & 0 & 0 \\
 0 & 0 & \frac{q}{(1+q)^2} & 0 \\
 0 & 0 & 0 & \frac{q^2}{(1+q)^2} \\
\end{pmatrix} \,.
\end{equation}
Taking the trace of the resulting \(4 \times 4\) matrix gives the Markov trace \(\phi(b_1^3)\). The final answer for the trefoil is:
\begin{equation}
\alpha(b_1^3) = q + q^3 - q^4 \,.
\end{equation}

\paragraph{Three-strand braids:} For braids with three strands, we need to take tensor products of the $R$-matrix with the identity operator on the other strands. This gives us the $8\times8$ generator matrices $b_1 = R \otimes I_{2}$ and $b_2 = I_{2} \otimes R$ for the braid group $B_3$, with explicit expressions
\begin{equation}
	b_{1}=\left(
	\begin{array}{cccccccc}
	 1 & 0 & 0 & 0 & 0 & 0 & 0 & 0 \\
	 0 & 1 & 0 & 0 & 0 & 0 & 0 & 0 \\
	 0 & 0 & 0 & 0 & -\sqrt{q} & 0 & 0 & 0 \\
	 0 & 0 & 0 & 0 & 0 & -\sqrt{q} & 0 & 0 \\
	 0 & 0 & -\sqrt{q} & 0 & 1-q & 0 & 0 & 0 \\
	 0 & 0 & 0 & -\sqrt{q} & 0 & 1-q & 0 & 0 \\
	 0 & 0 & 0 & 0 & 0 & 0 & 1 & 0 \\
	 0 & 0 & 0 & 0 & 0 & 0 & 0 & 1 \\
	\end{array}
	\right)\,,\quad
	b_{2}= \left(
	\begin{array}{cccccccc}
	 1 & 0 & 0 & 0 & 0 & 0 & 0 & 0 \\
	 0 & 0 & -\sqrt{q} & 0 & 0 & 0 & 0 & 0 \\
	 0 & -\sqrt{q} & 1-q & 0 & 0 & 0 & 0 & 0 \\
	 0 & 0 & 0 & 1 & 0 & 0 & 0 & 0 \\
	 0 & 0 & 0 & 0 & 1 & 0 & 0 & 0 \\
	 0 & 0 & 0 & 0 & 0 & 0 & -\sqrt{q} & 0 \\
	 0 & 0 & 0 & 0 & 0 & -\sqrt{q} & 1-q & 0 \\
	 0 & 0 & 0 & 0 & 0 & 0 & 0 & 1 \\
	\end{array}
	\right)\,,
\end{equation}
and $H = h \otimes h \otimes h$ given by
\begin{equation}
	H = \frac1{(1+q)^3}\,\text{diag}\,(1,q,q,q^2,q,q^2,q^2,q^3)\,.
\end{equation}
The four-crossing figure-eight knot is represented by the braid word $A = b_1 b_2^{-1} b_1 b_2^{-1}\in B_3$.
It is easy to see that this leads to the fundamental Jones polynomial for the figure-eight knot,
\begin{equation} 
\alpha(4_1)=1+q^{-2}-q^{-1}-q+q^2\,.
\end{equation}

\subsection{Adjoint Jones Polynomials ($N=3$, $19$-vertex model)}

The three-state vertex model has $19$ independent Boltzmann weights at each vertex.  The $R$-matrix obtained from this model is
\begin{equation}
	R =\left(
	\begin{array}{ccccccccc}
	 1 & 0 & 0 & 0 & 0 & 0 & 0 & 0 & 0 \\
	 0 & 0 & 0 & -q & 0 & 0 & 0 & 0 & 0 \\
	 0 & 0 & 0 & 0 & 0 & 0 & q^2 & 0 & 0 \\
	 0 & -q & 0 & 1-q^2 & 0 & 0 & 0 & 0 & 0 \\
	 0 & 0 & 0 & 0 & q & 0 & -\sqrt{q}+q^{5/2} & 0 & 0 \\
	 0 & 0 & 0 & 0 & 0 & 0 & 0 & -q & 0 \\
	 0 & 0 & q^2 & 0 & -\sqrt{q}+q^{5/2} & 0 & 1-q-q^2+q^3 & 0 & 0 \\
	 0 & 0 & 0 & 0 & 0 & -q & 0 & 1-q^2 & 0 \\
	 0 & 0 & 0 & 0 & 0 & 0 & 0 & 0 & 1 \\
	\end{array}
	\right).
\end{equation}

Just as we derived the braid generators \( b_i \)'s from the six-vertex model, we can also obtain the braid generators from the $19$-vertex model. From the matrix \( h \), 
\begin{equation}
	h = \frac{1}{1+q+q^2} \left( 
	\begin{array}{ccc}
	1 & 0 & 0 \\
	0 & q & 0 \\
	0 & 0 & q^2 \\
	\end{array}
	\right) \,,
\end{equation}
we can construct the matrices $H^{(n)}=h\otimes\cdots\otimes h$ for braid words with $n$ strands. Braid generators are formed by considering tensor products such as $b_1=R\otimes I_3\otimes\cdots\otimes I_3$, etc. For example, one can calculate the following polynomials,
\begin{eqnarray}
	\alpha(3_1) &=&\ q^2 (1 + q^3 - q^5 + q^6 - q^7 - q^8 + q^9)\,,\\
	\alpha(4_1) &=&\ 3+q^{-6} - q^{-5} -q^{-4} + 2q^{-3} - q^{-2} - q^{-1}- q - q^2 +  2 q^3 - q^4 - q^5 + q^6 \,. \nonumber \\
\end{eqnarray}

\paragraph{Remark:}
The choice of framing is different here as compared to the choice in the \texttt{KnotTheory} package for Mathematica.
Let us denote a braid word, \textit{e.g.}, $A=b_ib_jb_k^{-1}$,  by a numerical list, $\tilde A=\{i,j,-k\}$.
To match the framing factor, we introduce a prefactor \( q^{s(A)-\s(A)} \), where \( s(A) \) denotes the sum of signs of exponents of generators in braid word $A$, and \( \s(A) \) denotes the sum of the numerical braid word representation.
As an example, for the braid word \( A= b_1 b_2^{-1} b_1 b_2^{-1} \), or equivalently \(\tilde A= \{1, -2, 1, -2\} \), these are
\begin{equation}
s(A) = 1 - 1 + 1 - 1 = 0 \,, \quad \s(A) = 1 - 2 + 1 - 2 = -2 \,.
\end{equation}
In this example, the prefactor becomes \( q^{0 - (-2)} = q^2 \). Multiplying the vertex model answer with this prefactor, we get a polynomial that matches exactly with that obtained from the \texttt{KnotTheory} package.

\bibliographystyle{JHEP}
\bibliography{ref}

\end{document}