\documentclass[sigconf]{acmart}

\usepackage{multirow} 
% \usepackage{arydshln}
\usepackage{algorithm}  
\usepackage{algpseudocode}  
\usepackage{amsmath}  
\usepackage{subfigure}
\usepackage{graphicx}
% \usepackage{subcaption}
% \usepackage{caption}
% \usepackage{array}
% \usepackage{subfig}

\renewcommand{\algorithmicrequire}{\textbf{Input:}}  % Use Input in the format of Algorithm  
\renewcommand{\algorithmicensure}{\textbf{Output:}} % Use Output in the format of Algorithm 

\AtBeginDocument{%
  \providecommand\BibTeX{{%
    Bib\TeX}}}

% \setcopyright{acmlicensed}


\copyrightyear{2025}
\acmYear{2025}
\setcopyright{cc}
\setcctype{by}
\acmConference[WWW '25]{Proceedings of the ACM Web Conference 2025}{April 28-May 2, 2025}{Sydney, NSW, Australia}
\acmBooktitle{Proceedings of the ACM Web Conference 2025 (WWW '25), April 28-May 2, 2025, Sydney, NSW, Australia}
\acmDOI{10.1145/3696410.3714765}
\acmISBN{979-8-4007-1274-6/25/04}


\begin{document}


\title{Division-of-Thoughts: Harnessing Hybrid Language Model Synergy for Efficient On-Device Agents}

\author{Chenyang Shao}
\affiliation{%
 \institution{Department of Electronic Engineering\\ BNRist, Tsinghua University}
 \city{Beijing}
 \country{China}}
 \email{shaocy24@mails.tsinghua.edu.cn}

\author{Xinyuan Hu}
\affiliation{%
 \institution{Department of Quantitative Theory \& Methods\\Emory University}
 \city{GA}
 \country{USA}}
 \email{nate.hu@emory.edu}

\author{Yutang Lin}
\affiliation{%
 \institution{Department of Electronic Engineering \\Tsinghua University}
 \city{Beijing}
 \country{China}}
 \email{yt-lin21@mails.tsinghua.edu.cn}




% \author{Fengli Xu\raisebox{0.5ex}{\scalebox{0.85}{$\dagger$}}}
\author{Fengli Xu$^{*}$}
\affiliation{%
 \institution{Department of Electronic Engineering\\ BNRist, Tsinghua University\authornote{Corresponding author.}}
 \city{Beijing}
 \country{China}}
 \email{fenglixu@tsinghua.edu.cn}
 
%\renewcommand{\shortauthors}{Trovato et al.}


\begin{abstract}
The rapid expansion of web content has made on-device AI assistants indispensable for helping users manage the increasing complexity of online tasks. The emergent reasoning ability in large language models offer a promising path for next-generation on-device AI agents. However, deploying full-scale Large Language Models (LLMs) on resource-limited local devices is challenging. In this paper, we propose \underline{D}ivision-\underline{o}f-\underline{T}houghts (\textbf{DoT}), a collaborative reasoning framework leveraging the synergy between locally deployed Smaller-scale Language Models (SLMs) and cloud-based LLMs.
DoT leverages a \textit{Task Decomposer} to elicit the inherent planning abilities in language models to decompose user queries into smaller sub-tasks, which allows hybrid language models to fully exploit their respective strengths. Besides, DoT employs a \textit{Task Scheduler} to analyze the pair-wise dependency of sub-tasks and create a dependency graph, facilitating parallel reasoning of sub-tasks and the identification of key steps. To allocate the appropriate model based on the difficulty of sub-tasks, DoT leverages a \textit{Plug-and-Play Adapter}, which is an additional task head attached to the SLM that does not alter the SLM's parameters. To boost adapter's task allocation capability, 
% we propose a self-reinforced tree search algorithm to create a high-quality sub-task allocation dataset.
we propose a self-reinforced training method that relies solely on task execution feedback.
Extensive experiments on various benchmarks demonstrate that our DoT significantly reduces LLM costs while maintaining competitive reasoning accuracy. Specifically, DoT reduces the average reasoning time and API costs by 66.12\% and 83.57\%, while achieving comparable reasoning accuracy with the best baseline methods. \footnote{Code available at: https://github.com/tsinghua-fib-lab/DoT}
\end{abstract}


% \begin{abstract}
% The rapid expansion of web content has made on-device AI assistants indispensable for helping users manage the increasing complexity of online tasks. The emergent reasoning ability in large language models represents a promising direction to develop next generation on-device AI agents. However, deploying full-scale LLMs on resource-limited local devices face significant challenges. In this paper, we present a novel collaborative reasoning framework called \underline{D}ivision-\underline{o}f-\underline{T}houghts (\textbf{DoT}) to fully harness the synergy between locally deployed smaller-scale language model (SLMs) and cloud-based commercial LLMs.
% DoT leverages a \textit{Task Decomposer} to elicit the inherent planning abilities in language models to decompose user queries into smaller sub-tasks, which allows hybrid language models to fully exploit their respective strengths. Besides, DoT also employs a \textit{Task Scheduler} to analyze the pair-wise dependency of sub-tasks and create a dependency graph, facilitating parallel reasoning of sub-tasks and the identification of key steps. To allocate the appropriate model based on the difficulty of sub-tasks, DoT leverages a \textit{Plug-and-Play Adapter}, which is an additional task head attached to the SLM that does not alter the SLM's parameters. To boost the allocation accuracy of the adapter, we propose a self-reinforced tree search algorithm to create a high-quality sub-task allocation dataset. Extensive experiments on various benchmarks demonstrate that our DoT significantly reduces LLM costs while maintaining competitive reasoning accuracy. Specifically, DoT reduces the average reasoning time and API costs by 66.12\% and 83.57\%, while achieving comparable reasoning accuracy with the best baseline methods. \footnote{Our code can be accessed via: https://github.com/PLUTO-SCY/DoT}
% \end{abstract}


\begin{CCSXML}
<ccs2012>
   <concept>
       <concept_id>10010147.10010178.10010179.10010182</concept_id>
       <concept_desc>Computing methodologies~Natural language generation</concept_desc>
       <concept_significance>500</concept_significance>
       </concept>
   <concept>
       <concept_id>10010520.10010521.10010537.10010538</concept_id>
       <concept_desc>Computer systems organization~Client-server architectures</concept_desc>
       <concept_significance>300</concept_significance>
       </concept>
 </ccs2012>
\end{CCSXML}

\ccsdesc[500]{Computing methodologies~Natural language generation}
\ccsdesc[300]{Computer systems organization~Client-server architectures}


\keywords{Large Language Model, LLM Reasoning, AI Agents, Edge-Cloud Collaboration}


% \received{14 October 2024}
% \received[revised]{12 March 2009}
% \received[accepted]{5 June 2009}

\maketitle

\section{Introduction}

\begin{figure}[h]
    \centering
    \begin{overpic}[trim=0cm 0cm 0cm 0cm,clip,angle=0,origin=c,width=.4\linewidth]{images/teaser_absolute.png}
        %  trim={<left> <lower> <right> <upper>}
        %  \put(horiz, vert)
        %  \put(horiz, vert){\rotatebox{90}{Text}}
        %
        \put(107, 32){$\mathbf{\to}$}
    \end{overpic}\hspace{1cm}
    \begin{overpic}[trim=0cm 0cm 0cm 0cm,clip,angle=0,origin=c,width=.4\linewidth]{images/teaser_translated_yellow.png}
        %  trim={<left> <lower> <right> <upper>}
        %  \put(horiz, vert)
        %  \put(horiz, vert){\rotatebox{90}{Text}}
        %
    \end{overpic}
    \caption{Using translation methods, a controller trained on an environment with a given visual variation \textit{(left)} can be reused without any training or fine-tuning on a different environment (\textit{right}) with comparable performance. In red we see the trajectory of a car driven by the same controller when connected to two different encoders, one for each visual variation.
    }
    \label{fig:teaser}
\end{figure}

Deep Reinforcement Learning (RL) has enabled agents to achieve remarkable performance in complex decision-making tasks, from robotic manipulation to high-dimensional games (Mnih et al., 2015; Silver et al., 2017). 
Although recent RL techniques achieved strong improvements over sample efficiency \citep{yarats2021drqv2, kostrikov2020image}, training new agents remains a costly process, both in computational and temporal terms.
Despite these advances, most methods still require at least partial retraining when dealing with domain shifts such as visual appearance, reward functions, or action spaces \citep{pmlr-v97-cobbe19a, zhang2020learning}. These domain changes typically require expensive retraining, which can be prohibitive for real-world settings that require millions of interactions.

A variety of approaches have been proposed to address these shifting conditions. Domain randomization \citep{tobin2017domain, sadeghi2016cad2rl} trains agents across diverse visual styles or physics settings, promoting invariant features but demanding broader coverage of possible variations. Multi-task RL \citep{parisotto2015actor, teh2017distral} attempts to learn shared representations across multiple tasks.

In the supervised setting, recent representation learning techniques \citep{Moschella2022-yf,maiorca2023latent, norelli2022b, cannistraci2023bricks}, show that it is possible to zero-shot recombine encoders and decoders to perform new tasks across different modalities (images, text..) and tasks (classification, reconstruction) and even architectures.
In RL, methods adopting the relative representation framework \citep{Moschella2022-yf} have shown promising results in adapting encoders to different controllers with zero or few-shots adaptation, for robotic control from proprioceptive states \citep{jian2021adversarial} or for playing games in the Gymnasium suite \citep{towers2024gymnasium} from pixels \citep{ricciardi2025r3lrelativerepresentationsreinforcement}.
These methods, however, still require training models to use the new relative representations.

By contrast, \cite{maiorca2023latent} suggest that modules from independently trained neural networks can be connected via a simple linear or affine transformation, with no training constraint or fine-tuning required, if such transformations can be reliably estimated from a small set of “anchor” samples, pairs of states or observations deemed semantically equivalent.

Our main contribution is the implementation of a RL method based on semantic alignment to map between latent spaces of different neural models, so that their encoders and controllers can be stitched with the goal of creating new agents that can act on visual-task combinations never seen together in training. This includes the use of the transformations to map modules from different networks, and the collection of anchor samples used to estimate these transformations. We call our method Semantic Alignment for Policy Stitching (\textbf{SAPS}).
We perform analyses and empirical tests on the CarRacing and LunarLander environments to show the performance of new agents created via zero-shot stitching of encoders and controllers trained on different visual-task variations, demonstrating significant gains compared to existing zero-shot methods.
\subsection{Multilingual Datasets}
% 为了评测模型在不同语言上的性能,有很多在不同任务上的多语言数据集被提出,比如,QA,自然语言推理,文字总结,数值推理,代码生成,可读性等
To evaluate the performance of models across different languages, several multilingual datasets have been proposed for different tasks, such as question answering \cite{liu-etal-2019-xqa,clark-etal-2020-tydiQA,longpre-etal-2021-mkqa}, natural language inference \cite{conneau-etal-2018-xnli}, text summarization \cite{giannakopoulos-etal-2015-multiling-Summarization,ladhak-etal-2020-wikilingua,scialom-etal-2020-mlsum}, numerical reasoning \cite{shi2023MGSM}, code generation \cite{peng-etal-2024-humanevalxl}, text-to-SQL \cite{MultiSpider}, and readability \cite{trokhymovych-etal-2024-open-Readability,naous-etal-2024-readme}, among others. 
% 还有很多多语言数据集收集了不同的任务
Additionally, numerous multilingual datasets have been collected for different tasks \cite{hu-2020-XTREME,ruder-etal-2021-xtremer,zhang2024pmmeval-multitask,singh-etal-2024-indicgenbench}. 
% 但是目前为止,仍然没有多语言TATQA数据集,导致缺乏关于模型多语言TATQA能力的评测与分析
However, to date, there is no multilingual TATQA dataset, resulting in a lack of evaluation and analysis of multilingual TATQA capabilities and a gap with real scenarios. 
% 所以,我们本文提出了多语言TATQA数据集,并详细分析了多语言TATQA的挑战
Therefore, we introduce \ourdataset, a multilingual TATQA dataset, and provide a detailed analysis of the challenges in multilingual TATQA.

\subsection{QA Datasets for the Table and Text}
% 目前,在表格和文本上的QA数据集主要集中于单一语言
Currently, QA datasets for the table and text primarily focus on a single language. 
% 比如,HybridQA从Wikipedia收集英文的表格和相关段落,共包含70K个人工标注的问题和答案
For instance, HybridQA~\cite{chen-etal-2020-hybridqa} collects English tables and associated text from Wikipedia.
% , containing $70$K manually annotated question-answer pairs. 
% TAT-QA和FinQA和DOCMATH-EVAL主要关注于金融领域的数值计算问题
TAT-QA~\cite{zhu-etal-2021-tat}, FinQA~\cite{chen-etal-2021-finqa}, DOCMATH-EVAL~\cite{zhao-etal-2024-docmath}, and FinanceMATH~\cite{zhao-etal-2024-FinanceMATH} focus on numerical computation in the financial domain, and SciTAT~\cite{zhang2024scitat} addresses questions based on tables and text from English scientific papers. 
% SciTAT则关注于根据英文论文中的表格和文本回答用户问题
% 然而,单一语言的数据集无法进一步探索TATQA的挑战,无法全面评测模型的多语言TATQA的能力,并且和现实场景存在较大差距
However, single-language datasets cannot evaluate the multilingual TATQA capabilities, and overlook the diverse languages in real scenarios. 
% 所以我们提出了我们的数据集:首个多语言TATQA数据集,涉及包括英语的11种语言,8个语系
So we propose \ourdataset: the first multilingual TATQA dataset, involving $11$ languages and $8$ language families.
% 我们的数据集和前人工作的对比表格如表所示
A comparison of \ourdataset and prior works is presented in Appendix~\ref{sec:comparison}.


% 目前关于增强TATQA性能的工作主要集中于检索和生成两阶段
The current works on enhancing TATQA performance primarily focus on retrieving relevant information from the context \cite{luo2023hrot,bardhan2024ttqars,glenn-etal-2024-blendsql} and generating programs, equations, or step-by-step reasoning process to derive the final answer \cite{tonglet-etal-2023-seer,TAT-LLM,fatemi2024three-agent}.
% 检索,即设计检索器或直接使用LLM从上下文中抽取相关信息
% Retrieval involves designing a retriever to extract relevant information from the context.
% 生成,即生成代码、等式或逐步推理得到最终答案
% Generation refers to generating programs, equations, or step-by-step reasoning to derive the final answer.
% 比如S3HQA关注于检索阶段,首先训练检索器初步从上下文中检索到相关的表格行和文本,再根据问题的分类进一步选取出相关上下文
For example, S3HQA~\cite{lei-etal-2023-s3hqa} emphasizes retrieving, where a retriever is initially trained, followed by further filtering based on the question type. 
% 而Hpropro关注于生成阶段,通过提供给LLM一些常用函数,方便LLM生成代码时直接调用,且提示LLM根据代码执行错误的信息修改
Hpropro~\cite{shi-etal-2024-hpropro} focuses on generating, providing LLMs with commonly used functions to facilitate direct invocation during code generation.
% 然而,这些方法都是针对单一语言设计的,直接应用于其他语言会导致性能下降
However, previous methods are designed for single-language scenarios, directly used to other languages could lead to performance degradation.
% 所以,我们提出了一个多语言的baseline:我们的方法,实现跨语言的链接和推理
To address this, we propose \ourmethod, a multilingual baseline that aligns the English TATQA capabilities to other languages. 
% 并且我们详细分析了语言对不同答案类型性能的影响,以及在多语言场景下,instruction, 示例,表格文本以及问题的语言对性能的影响,
% We also provide a detailed analysis of how different languages impact performance across various answer types. 
% Furthermore, we examine the effects of language for instructions, demonstrations, tables, text, and questions in multilingual settings. 
% 我们还分析了在我们数据集上非英语语言相比英语的性能下降的原因,为未来多语言TATQA的研究指明了方向
% Additionally, we analyze the reasons for the performance decline in non-English languages compared to English on \ourdataset, providing directions for future research in multilingual TATQA.
% We also provide a detailed analysis of how different languages impact performance, providing directions for future research in multilingual TATQA.
\section{Preliminaries}

In this section, we firstly establish the formulation for both text-only and multi-modal thought~(\textsection\ref{cot}), and then provide a detailed explanation of two inference-time scaling paradigms~(\textsection\ref{scaling}): sampling-based and tree search-based methods.
In addition, we present a prompting-based verifier equipped with a consistency-enhanced mechanism to effectively guide these inference-time scaling methods~(\textsection\ref{verifier}).

\subsection{Thought Formulation} \label{cot}
Chain-of-Thought (CoT, \citealt{cot}) is a technique that encourages the model to generate intermediate reasoning steps $\mathbf{S}=(\mathbf{s}_1,...,\mathbf{s}_n)$ before reaching the final answer $\mathbf{a}$.
In the context of multi-modal reasoning tasks, given a problem $\mathbf{q}$ and images $\mathbf{I}$, each reasoning step $\mathbf{s}_i$ is sampled from an LVLM $\mathcal{M}$:
\begin{equation}
\label{eq:cot}
    \mathbf{s}_i \sim \mathcal{M}\big(\mathbf{q}, \mathbf{I}, \mathbf{s}_{1:i-1}\big).
\end{equation}
This expansion allows for more computations to be deployed and can unlock the ability to solve more complex problems~\cite{lichain}.

Following the success on language-only reasoning tasks, most previous works~\cite{mcot, ddcot, ccot} study \textit{text-only thought}, where $\mathbf{S}$ is defined as a text string.
In this work, we aim to explore the potential of \textit{multi-modal thought}, where any reasoning step $\mathbf{s}_i$ can blend multi-modal information, e.g., both text and image.

% 这也可以在discussion中讲
However, as most prevalent LVLMs still struggle to generate reliable images, we follow \citet{vsk} to allow the LVLM to generate executable code for visual manipulation instead.
Implementation details can be found in Appendix~\ref{sec:app_vsk}.




\subsection{Inference-Time Scaling} \label{scaling}

\subsubsection{Sampling-based Methods}
These methods scale inference-time computation by generating multiple reasoning chains in parallel, and then determining the most promising answer from them. Here we investigate two widely used techniques, Self-Consistency~\cite{tts2-Self-Consistency} and Best-of-N~\cite{tts-mcts1, tts-mcts3}.

\paragraph{Self-Consistency}
This method leverages the intuition that the correct answer typically can be reached using multiple different ways of thinking~\cite{tts2-Self-Consistency}.
Given $N$ sampled reasoning chains $\mathbf{S}_1,...,\mathbf{S}_N$ generated from the LVLM, it selects the most consistent answer via majority voting over their corresponding answers $\mathbf{a}_1,...,\mathbf{a}_N$:
\begin{equation}
    \mathbf{a}^* = \arg\max_{\mathbf{a} \in \{\mathbf{a}_1,...,\mathbf{a}_N\}} \sum_{i=1}^{N} \mathbb{I}(\mathbf{a}_i = \mathbf{a}),
\end{equation}
where $\mathbb{I}(\cdot)$ denotes the indicator function and $\mathbf{a}^*$ represents the selected answer.


\paragraph{Best-of-N}
This approach utilizes trained verifiers to evaluate the correctness of model generated solutions.
Compared to Self-Consistency based on voting, it can be more efficient when a strong verifier is available.
Formally, given $N$ sampled reasoning chains and a verifier $\mathcal{F}(\cdot)$, we have
\begin{equation}
    \mathbf{S}^*=\max_{\mathbf{S}\in \{\mathbf{S}_1,...,\mathbf{S}_N\} }\mathcal{F}(\mathbf{S}),
\end{equation}
where $\mathbf{S}^*$ is the reasoning chain with the highest verifier score, from which the final answer $\mathbf{a}^*$ is derived.



\subsubsection{Tree Search-based Methods}
Sampling-based methods are easy to implement due to their simplicity.
However, they are inefficient because they require exploring full solution paths, even if a mistake occurs early~\cite{xiang2025towards}.
To address this challenge, tree search-based methods are later explored.
They model the problem-solving as a tree search process, where each node represents a reasoning step.
Here, we investigate the most prevalent Beam Search~\cite{tts-bs1, tts-bs2, tot, tts-bs3} and MCTS~\cite{tts-mcts5,tts-mcts8,  tts-mcts2, tts-mcts4, tts-mcts6, tts-mcts7, yao2024mulberry} algorithms in both language-only and multi-modal tasks.


\paragraph{Beam Search}


Adapted from classical beam search algorithms used in sequence generation tasks (e.g., machine translation~\cite{beamsearch4mt, beamsearch4mt2}), this method dynamically explores and prunes reasoning paths at each step for deductive reasoning.
The algorithm operates with a beam width \( B \) (number of retained reasoning chains per step) and an expansion size \( N \) (new thoughts generated per candidate).
Starting with \( N \) initial candidate thoughts sampled from a question $q$ via Equation~\ref{eq:cot}, a verifier \( \mathcal{F}(\cdot) \) scores all candidates, retaining only the top-\( B \) chains.
Each retained chain is then expanded by sampling \( N \) new thoughts, producing \( B \times N \) candidates, after which the verifier reevaluates and prunes the pool to the top-\( B \) chains.
This cycle of scoring and pruning repeats iteratively until all active chains reach finished states (final answers), at which point the highest-scoring finished chain across all iterations is selected as the final answer \( \mathbf{a}^* \).  








\paragraph{MCTS}



It is a decision-making algorithm that is powerful for solving complex problems, as demonstrated by AlphaGo~\cite{alphago, alphago1}. MCTS operates by iteratively exploring the most promising reasoning paths through four key phases: selection, expansion, evaluation, and backpropagation. 

In the selection phase, the algorithm traverses the tree from the root node, recursively choosing child nodes using the Upper Confidence Bound (UCB) metric~\cite{ucb}. This metric strategically balances the exploitation of high-value paths and the exploration of under-visited paths. The UCB formula is defined as:  
\[
\text{UCB}(\mathbf{s}_i) = \frac{V(\mathbf{s}_i)}{C(\mathbf{s}_i)} + w \times \sqrt{\frac{\ln C(\mathbf{s}_{i-1})}{C(\mathbf{s}_i)}},
\]  
where \(V(\mathbf{s}_i)\) and \( C(\mathbf{s}_i)\) denotes the accumulated value and count of being visited of \(\mathbf{s}_i\). \(w\) is a weighting parameter that adjusts the exploration-exploitation balance.
During expansion, the algorithm extends the selected node by generating a new thought through Equation~\ref{eq:cot}.
The evaluation phase then estimates the value of this newly expanded node using a verifier \(\mathcal{F}(\cdot)\).
Finally, backpropagation updates the values of all ancestors, propagating the evaluation results back to the root.

After multiple iterations, the finished reasoning chain with the highest accumulated value is selected, yielding the optimal final answer \(\mathbf{a}^*\).   



\begin{figure}[htp]
  \includegraphics[width=\columnwidth]{imgs/verifier.pdf}
  \centering
  % \caption{\textbf{Consistency-Enhanced Verifier:} The reasoning chain is formatted as input to the LVLM, which is then instructed to verify it through CoT reasoning. By sampling multiple verification responses, the score is then computed by aggregating the verification results.}
  \caption{Three types of verifiers investigated in this work, where the classification-based verifier outputs sparse binary scores (0 or 1) and the regression-based verifier provides dense but inaccurate scores. We introduce the consistency-enhanced verifier to compute dense and accurate scores by aggregating multiple evaluations sampled from the classification-based verifier.}
  \label{fig:verifier}
\end{figure}


\subsection{Verifiers for Inference-Time Scaling} \label{verifier}

The verifier \(\mathcal{F}\) is pivotal in inference-time scaling, tasked with evaluating the validity of reasoning chains. 
Existing approaches fall into two categories: training-based~\cite{llava-critic, shepherd} and prompting-based~\cite{binary1, binary2, scoring1}.
Training-based approaches hold theoretical promise by optimizing verifiers on corresponding labeled data.
However, acquiring such annotations is challenging, resulting in datasets limited in scale and diversity, thereby restricting the generalizability of trained verifiers.
Thus, this work focuses on prompting-based approaches, capitalizing on the strong instruction-following capabilities of LVLMs to assess reasoning chains.  

Formally, given a reasoning chain \(\mathbf{s}_{1:i}\), the verifier \(\mathcal{F}\) operates as:  
\(\mathbf{r} = \mathcal{M}(\mathbf{P}, \mathbf{s}_{1:i})\),
where \(\mathbf{P}\) represents the verification instruction, and \(\mathbf{r}\) is the evaluation output, which includes a scalar score.
Particularly, we investigate two instruction designs for \(\mathbf{P}\): 
\begin{itemize}[leftmargin=*]
    \item \textbf{Classification-Based}~\cite{binary1, gen-verifier,  binary2}: It prompts the verifier to classify the reasoning chain as ``correct'' or ``incorrect.'' While intuitive, this binary output provides sparse feedback, offering limited granularity to guide searching. 
    \item \textbf{Regression-Based}~\cite{scoring1, llava-critic}: Here, the verifier is prompted to output a continuous score \(\in [0, 1]\) reflecting the quality of chains. However, directly predicting precise scores is error-prone, often yielding high-variance results that degrade performance.\footnote{The prompts used for these two designs are in Appendix~\ref{sec:app_verifier_inst}.}
\end{itemize}

To address these limitations, we introduce the \textbf{Consistency-Enhanced Verifier}, compared with the above two in Figure~\ref{fig:verifier}.
This method aggregates $N_{v}$ independent evaluations using classification-based instructions, computing the proportion of ``correct'' classifications across trials. 
In this way, it not only avoids variance in regression-based outputs but also overcomes the sparsity of single-trail classification, enabling more reliable guidance for inference-time scaling.










% 加一节,怎么区分确定性和不确定性

\section{Methodology}


To achieve effective probabilistic predictions, we propose CoST that simultaneously leverages the advantages of both deterministic and probabilistic models. Our approach involves two stages. In the first stage, the deterministic model is pretrained to predict the conditional mean that captures the primary patterns. In the second stage, the parameters of the deterministic model are frozen, and the scale-aware diffusion model, constrained by a customized fluctuation scale, is jointly trained to model residual distributions that reflect random fluctuations.   
Figure~\ref{fig:model} illustrates an overview of our model.


\subsection{Mean-Residual Decomposition}

For urban spatiotemporal probabilistic prediction, current approaches typically employ a single probabilistic model to capture the full distribution of data, incorporating both the primary spatiotemporal patterns and the random fluctuations. However, it is challenging to model both of these distributions. Inspired by~\cite{mardani2023residual} and the Reynolds decomposition in fluid dynamics~\cite{pope2001turbulent}, we propose to decompose the target data \(\mathbf{x}^{ta}\) as follows:
\begin{equation}
 \mathbf{x}^{ta} = \underbrace{\mathbb{E}[\mathbf{x}^{ta}|\mathbf{x}^{co}]}_{\substack{:=\boldsymbol{\mu}(Deterministic)}} + \underbrace{(\mathbf{x}^{ta}-\mathbb{E}[\mathbf{x}^{ta}|\mathbf{x}^{co}])}_{\substack{:=\mathbf{r}(Probabilistic)}},
\end{equation}
where \(\boldsymbol{\mu}\) is the conditional mean representing the primary patterns, and \(\mathbf{r}\) is the residual representing the random variations. Our core idea is that if a deterministic model can accurately predict the conditional mean, that is, \(\boldsymbol{\mu}\approx\mathbb{E}_{\theta}[\mathbf{x}^{ta}|\mathbf{x}]\), then the probabilistic model only needs to focus on learning the simpler residual distribution, thus combining the strengths of both models to enhance the probabilistic prediction capability.









\subsection{Mean Prediction via Deterministic Model}

We require a deterministic model that can accurately predict the conditional mean to align with our research hypothesis, and also maintain high predictive efficiency to avoid additional increases in training and inference time. Therefore, we select the MLP-based STID model as our mean prediction module.
In the first stage of training, we pretrain the model for 50 epochs to effectively capture the primary spatiotemporal patterns. Specifically, we input historical conditional data \(\mathbf{x}^{co}\) into the STID model to obtain the conditional mean output \(\mathbb{E}_{\theta}[\mathbf{x}^{ta}|\mathbf{x}^{co}]\).

The STID model is pretrained by optimizing the following loss function:

\begin{equation}
\label{eq:loss2}
   \mathcal{L}_{2}  = \left\| \mathbb{E}_{\theta}[\mathbf{x}^{ta}|\mathbf{x}^{co}] - \mathbf{x}^{ta} \right\|_2^2 .
\end{equation}

\subsection{Residual Learning via Diffusion Model}
Diffusion models have achieved significant success in probabilistic modeling. In this work, we employ a diffusion model for probabilistic prediction, training it to learn the residual distribution:
\begin{equation}
\label{eq:one-setp-forward}
    \mathbf{r}_{ta}=\mathbf{x}^{ta}-\mathbb{E}_{\theta}[\mathbf{x}^{ta}|\mathbf{x}^{co}].
\end{equation}
Consequently, the target data \(\mathbf{x}^{ta}\) for diffusion models in Eqs.~\eqref{eq:one-setp-forward}, \eqref{eq:inference}, and \eqref{eq:loss1} is replaced by \(\mathbf{r}_{ta}\).
The residual distribution of urban spatiotemporal data is not independently and identically distributed (i.i.d.) nor does it follow a fixed distribution, such as \(\mathcal{N}(0, \mathbf{\sigma})\). Instead, it often exhibits complex spatiotemporal dependence and heterogeneity. So we consider both temporal residual learning and spatial residual learning. 




\subsubsection*{\textbf{Temporal Residual Learning.}} 
For urban spatiotemporal data, the residual distribution varies at different time points. For example, fluctuations are lower at night and higher during the day, with uncertainty varying between weekends and weekdays. To model this, we incorporate the timestamp information as the condition for the denoising process. Meanwhile, the residual distribution can also be affected by sudden weather changes or public events. To capture these real-time changes, we concatenate the context data $\mathbf{x}^{co}_0$ with noised target residual $\mathbf{r}^{ta}_n$ as the input. The noise is not added to $\mathbf{x}^{co}_0$ and $\mathbf{r}^{ta}_n$ during the diffusion training and inference.




\subsubsection*{\textbf{Spatial Residual Learning.}}
In areas with frequent traffic accidents, fluctuations tend to be more pronounced and may induce anomalous variations in adjacent regions, thus affecting their distributions.
For spatial dependence modeling, the model learns a spatial embedding for each location, following the STID approach. Additionally, we propose a scale-aware diffusion process to further distinguish the heterogeneity for different regions. In this section, we detail the calculation of \(\mathbf{Q}\) and how it is integrated into the scale-aware diffusion process.

\noindent\textbf{(i) Customized Fluctuation Scale.} Specifically, we apply the Fast Fourier Transform (FFT) to spatiotemporal sequences in the training set to quantify fluctuation levels in different regions and use the custom scale \(\mathbf{Q}\) as input to account for spatial differences in residual. Specifically, we first employ FFT to extract the fluctuation components for each region within the training set. The detailed steps are as follows:









\begin{equation}
    \begin{aligned}
    & \mathbf{A}_{\mathrm{k}} = \left| \text{FFT}(\mathbf{x})_\mathrm{k} \right|, \quad \mathbf{{\phi}}_{\mathrm{k}} = \mathbf{\phi} \left( \text{FFT}(\mathbf{x})_\mathrm{k} \right), \\
    & \mathbf{A}_{\text{max}}=\max_{\mathrm{k}\in\left\{1,\cdots,\left\lfloor\frac{\mathbf{L}}{2}\right\rfloor + 1\right\}}\mathbf{A}_{\mathrm{k}}, \\
    & \mathcal{K} = \left\{ \mathrm{k} \in \left\{ 1, \cdots, \left\lfloor \frac{{L}}{2} \right\rfloor + 1 \right\} : \mathbf{A}_{\mathrm{k}} < 0.1 \times \mathbf{A}_{\text{max}} \right\}, \\
    & \mathbf{x}_{\mathbf{r}}[i] = \sum_{\mathrm{k} \in \mathcal{K}} \mathbf{A}_{\mathrm{k}} \Big[ \cos \left( 2\pi \mathbf{f}_{\mathrm{k}} i + \mathbf{\phi}_{\mathrm{k}} \right) \\
    & \qquad \qquad + \cos \left( 2\pi \bar{\mathbf{f}}_{\mathrm{k}} i + \bar{\mathbf{\phi}}_{\mathrm{k}} \right) \Big],
    \end{aligned}
\end{equation}
where \(\mathbf{A}_{\mathrm{k}},\mathbf{\phi}_{\mathrm{k}}\) reprent the amplitude and phase of the $\mathrm{k}-$th frequency component. $L$ is the temporal length of the training set. \(\mathbf{A}_{\text{max}}\) is the maximum amplitude among the components, obtained using the $\max$ operator. $\mathcal{K}$ represents the set of indices for the selected residual components. \(\mathbf{f}_{\mathrm{k}}\) is the frequency of the \(\mathrm{k}\)-th component. $\bar{\mathbf{f}}_{\mathrm{k}}, \bar{\mathbf{\phi}}_{\mathrm{k}}$ represent the conjugate components. \(\mathbf{x}_{\mathbf{r}}\) ref to the extracted residual component of the training set. We then compute the variance $\sigma^2_k$ of the residual sequence for each location $k$ and expand it to match the shape as 
\(\mathbf{r}^{ta}_0 \in \mathbb{R}^{B \times K \times P}\) , where $B$ represents the batch size. And we can get the variance tensor \(\mathcal{M}\): 
\begin{equation}
\begin{aligned}
    &\mathcal{M}_{b,k,p}=\sigma_{k}^2,\\
&\forall b\in\{1,\cdots,B\}, \forall k\in\{1,\cdots,K\}, \forall p\in\{1,\cdots,P\}.
\end{aligned}
\end{equation}
The residual fluctuations are bidirectional, encompassing both positive and negative variations, so we generate a random sign tensor \(\mathbf{S}\in\mathbb{R}^{B\times K\times P}\) for \(\mathcal{M}\), where each element \(S_{b,k,p}\) of \(\mathbf{S}\) is sampled from a Bernoulli distribution with \(p = 0.5\). 
%That is, \(r_{b,k,p}\) takes the value $1$ with probability $0.5$ and $-1$ with probability $0.5$. 
The customized fluctuation scale \(\mathbf{Q}\) is then defined as:
\begin{equation}
\begin{aligned}
&\mathbf{Q}_{b,k,p}=S_{b,k,p}\times\mathcal{M}_{b,k,p},\\
&\forall b\in\{1,\cdots,B\}, \forall k\in\{1,\cdots,K\}, \forall p\in\{1,\cdots,P\}.
\end{aligned}
\end{equation}
Then \(\mathbf{Q}\) is used as the input of the denoising network. 





\noindent\textbf{(ii) Scale-aware Diffusion Process.}

The vanilla diffusion models typically model all regions as the same \(\mathcal{N}(0, I)\) distribution at the end of the diffusion process, failing to distinguish the differences among regions. To further model the differences of residual distribution across various regions, we adopt the technique proposed by~\cite{han2022card} to make the residual learning region-specific conditioned on \({\mathbf{Q}}\). Specially, we have calculated the customized fluctuation scale \({\mathbf{Q}}\), and We redefined the noise distribution at the endpoint of the diffusion process as follows:
\begin{equation}
    p(\mathbf{r}^{ta}_N)=\mathcal{N}({\mathbf{Q}},I),
\end{equation}
Accordingly, the Eq~\ref{eq:new one-step} in the forward process is rewritten as:
\begin{equation}
\label{eq:new one-step}
    \mathbf{r}_n^{ta} = \sqrt{\bar{\alpha}_n} \mathbf{r}_0^{ta}+(1-\sqrt{\bar{\alpha}_n})\mathbf{Q} + \sqrt{1 - \bar{\alpha}_n} \mathbf{\epsilon}, \quad \mathbf{\epsilon} \sim \mathcal{N}(0, I).
\end{equation}
And in the denoising process, we sample \(\mathbf{r}_N^{ta}\) from $\mathcal{N}({\mathbf{Q}},I)$, and denoise it use Eq~(\ref{eq:inference}), the computation of \(\mu_{\theta}(\mathbf{r}_n^{ta}, n| \mathbf{x}_0^{co})\) in Eq~\ref{eq:inference} is modified as:
\begin{equation}
\label{eq: mu}
    \mu_{\theta}(\mathbf{r}_n^{ta}, n| \mathbf{x}_0^{co})=\frac{1}{\sqrt{\bar{\alpha}_n}} \left( \mathbf{r}_n^{ta} - \frac{\beta_n}{\sqrt{1 - \bar{\alpha}_n}} \mathbf{\epsilon}_{\theta}(\mathbf{r}_n^{ta}, n| \mathbf{x}_0^{co}) \right)+(1-\frac{1}{\sqrt{\bar{\alpha}_n}})\mathbf{Q}.
\end{equation}
This approach enables the diffusion process to be governed by the \(\mathbf{Q}\) values at each region, leading to more effective utilization of the customized scale conditions.


\subsection{Training and Inference}
\begin{algorithm}
\caption{\methodname{} Training}
\KwIn{Coarse-to-fine Autoencoder $\text{Enc}$, $\text{Dec}$}
\KwOut{}
\For{$i \gets 1$ \textbf{to} $n-1$}{
    \For{$j \gets 1$ \textbf{to} $n-i$}{
        \If{$L[j] > L[j+1]$}{
            Swap $L[j]$ and $L[j+1]$
        }
    }
}
\Return $L$
\end{algorithm}
\begin{algorithm}[!t]
\caption{Inference}
\label{al: sampling}
\begin{algorithmic}[1]
    \State \textbf{Input:} Context data $\mathbf{x}_0^{co}$, customized fluctuation scale $\mathbf{Q}$, trained diffusion model $\epsilon_{\theta}$, trained deterministic model $\mathbb{E}_{\theta}$
    \State \textbf{Output:} Target data $\mathbf{x}_0^{ta}$
    \State Estimate the conditional mean \(\mathbb{E}_{\theta}[\mathbf{x}^{ta}_0|\mathbf{x}^{co}_0]\)
    \State Sample $\mathbf{r}_N^{ta}$ from $\epsilon \sim \mathcal{N}(\mathbf{S},I)$
    \For{$n = N$ to $1$} 
        \State Estimate the noise $\mathbf{\epsilon}_{\theta}(\mathbf{r}_n^{ta}, n| \mathbf{x}_0^{co})$
        \State Calculate the $\mu_{\theta}(\mathbf{r}_n^{ta}, n| \mathbf{x}_0^{co})$ using Eq.~(\ref{eq: mu})
        \State Sample $\mathbf{r}_{n-1}^{ta}$ using Eq.~(\ref{eq:inference})
    \EndFor
    \State \textbf{Return:} $\mathbf{x}_0^{ta}=\mathbb{E}_{\theta}[\mathbf{x}^{ta}_0|\mathbf{x}^{co}_0]+\mathbf{r}_0^{ta}$
\end{algorithmic}

\end{algorithm}

\subsubsection*{\textbf{Training}}
Our training process is a two-stage procedure. We first pretrain the deterministic model STID to enable it to predict the conditional mean. Subsequently, we train the diffusion mode to learn the distribution of residuals, where the residuals are calculated as the difference between the true values and the conditional mean predicted by the pretrained STID model with frozen parameters. The detailed training procedure is presented in Algorithm~\ref{al: train}.
\subsubsection*{\textbf{Inference}}
The inference process of the model consists of two paths: one utilizes the pretrained STID model to predict the conditional mean, while the other uses the diffusion model to predict the residuals. The final sample is obtained by summing the results of both paths. This process is detailed in Algorithm~\ref{al: sampling}.
\section{Experiments}\label{sec:exp}

\begin{table}[t]
\centering
\caption{\textbf{Quantitative results on OpenCompass~\cite{2023opencompass} multimodal leaderboard.}
$^{\ddag}$ denotes closed-source models. Hall denotes HallusionBench.
}
\label{tab:exp_it_oc}
\setlength{\tabcolsep}{1pt}
\begin{tabular}{l|c|c|cccccccc}
\toprule
Models   & Params & Avg. & MM- & MM- & MM- & Math- & Hall & AI2D  & OCR- & MMVet \\
   &  &  & Bench & Star & MU & Vista &  &  & Bench & \\
\midrule
Step-1o$^{\ddag}$   & N/A   & \textbf{77.7}  & 87.3  & 69.3  & 69.9 & 74.7  & 55.8 & 89.1 & 926 & \textbf{82.8}  \\
SenseNova$^{\ddag}$  & N/A   & 77.4  & 85.7  & \textbf{72.7}  & 69.6 & \textbf{78.4}  & 57.4 & 87.8 & 894 & 78.2  \\
InternVL2.5-78B-MPO~\cite{wang2024mpo}  & 78B  & 77.0   & 87.7  & 72.1  & 68.2  & 76.6  & 58.1  & 89.2 & 909 & 73.5  \\
Qwen2.5-VL-72B~\cite{bai2025qwen25vltechnicalreport}   & 73.4B  & 76.2  & \textbf{87.8}  & 71.1  & 67.9  & 70.8  & 58.8  & 88.2  & 881  & 76.7  \\
TeleMM$^{\ddag}$   & N/A   & 75.9  & 79.9 & 70.8 & 66.6 & 75.7  & \textbf{60.6}  & 88.5 & 891 & 75.7  \\
InternVL2.5-38B-MPO~\cite{wang2024mpo}  & 38B  & 75.3  & 85.4  & 70.1 & 63.8 & 73.6 & 59.7 & 87.9 & 894 & 72.6  \\
InternVL2.5-78B~\cite{chen2024expanding}  & 78B  & 75.2 & 87.5  & 69.5 & 70 & 71.4 & 57.4 & 89.1 & 853 & 71.8   \\
Qwen2-VL-72B~\cite{qwen2-vl_2024}   & 73.4B  & 74.8  & 85.9  & 68.6  & 64.3  & 69.7  & 58.7  & 88.3  & 888  & 73.9  \\
InternVL2.5-38B~\cite{chen2024expanding}  & 38B  & 73.5  & 85.4  & 68.5  & 64.6  & 72.4  & 57.9  & 87.6  & 841  & 67.2  \\
JT-VL-Chat-V3.0$^{\ddag}$  & N/A   & 73.4  & 81.7  & 67.5  & 59.3  & 71.9  & 53.9  & 87.2  & \textbf{967}  & 69.2  \\
Taiyi$^{\ddag}$  & N/A   & 73.0  & 84.8  & 69  & 60.4  & 72.3  & 56.8  & \textbf{90.8}  & 820  & 67.9  \\
Step-1.5V$^{\ddag}$  & N/A   & 72.5 & 82.0  & 65.1  & 61.2  & 69.7  & 54.3  & 87.5  & 886  & 71.3  \\
Gemini-1.5-Pro-002$^{\ddag}$~\cite{geminiteam2024gemini15unlockingmultimodal}   & N/A   & 72.1 & 82.8  & 67.1  & 68.6  & 67.8  & 55.9  & 83.3  & 770  & 74.6  \\
InternVL2.5-26B-MPO~\cite{wang2024mpo}  & 26B  & 72.1  & 84.2  & 67.7  & 56.4  & 71.5  & 52.4  & 86.2  & 905  & 68.1  \\
GPT-4o-20241120$^{\ddag}$~\cite{openai2024gpt4ocard}  & NA   & 72.0   & 84.3  & 65.1  & \textbf{70.7}  & 59.9  & 56.2  & 84.9  & 806  & 74.5  \\
LLaVA-OneVision-72B~\cite{li2024llavaonevision}  & 73B  & 68.0  & 84.5  & 65.8  & 56.6  & 68.4  & 47.9  & 86.2  & 741  & 60.6  \\
NVLM-D-72B~\cite{nvlm2024}   & 79.4B  & 67.6  & 78.5  & 63.7  & 60.8  & 63.9  & 49.7  & 80.1  & 849  & 58.9  \\
Molmo-72B~\cite{deitke2024molmo}  & 73.3B  & 64.1  & 79.5  & 63.3  & 52.8  & 55.8  & 46.6  & 83.4  & 701  & 61.1  \\
\rowcolor{Gray} \textbf{\method-72B}   & 71.8B  & 75.1  & 86.3  & 70.7  & 57.6  & 73.3  & 56.4  & 87.6  & 889   & 79.8  \\
\midrule
\multicolumn{11}{l}{\textit{Models smaller than 20B}} \\
\midrule
Ola-7b~\cite{ola_2025}   & 8.88B   & \textbf{72.6}  & \textbf{84.3}  & \textbf{70.8}  & \textbf{57.0}  & 68.4  & \textbf{53.5}  & \textbf{86.1}  & 822  & \textbf{78.6}  \\
Qwen2.5-VL-7B~\cite{bai2025qwen25vltechnicalreport}   & 8.29B   & 70.4  & 82.6  & 64.1  & 56.2  & 65.8  & 56.3  & 84.1  & 877  & 66.6  \\
InternVL2.5-8B-MPO~\cite{wang2024mpo}   & 8B   & 70.3  & 82  & 65.2  & 54.8  & 67.9  & 51.7  & 84.5  & \textbf{882}  & 68.1  \\
MiniCPM-o-2.6~\cite{yao2024minicpm}   & 8.67B   & 70.2  & 80.6  & 63.3  & 50.9  & \textbf{73.3}  & 51.1  & 86.1  & 889  & 67.2  \\
Ovis1.6-Gemma2-9B~\cite{lu2024ovis}  & 10.2B  & 68.8  & 80.5  & 62.9  & 55.0  & 67.2  & 52.2  & 84.4  & 830  & 65.0  \\
InternVL2.5-8B~\cite{chen2024expanding}   & 8B   & 68.1  & 82.5  & 63.2  & 56.2  & 64.5  & 49.0  & 84.6  & 821  & 62.8  \\
POINTS1.5-Qwen2.5-7B~\cite{points1.5_2024} & 8.3B   & 67.4  & 80.7  & 61.1  & 53.8  & 66.4  & 50.0  & 81.4  & 832  & 62.2  \\
Valley-Eagle$^{\ddag}$   & 8.9B   & 67.4  & 80.7  & 60.9  & \textbf{57.0}  & 64.6  & 48.0  & 82.5  & 842  & 61.3  \\
Qwen2-VL-7B~\cite{qwen2-vl_2024}  & 8B   & 67.0  & 81.0 & 60.7 & 53.7 & 61.4  & 50.4 & 83 & 843 & 61.8 \\
DeepSeek-VL2~\cite{wu2024deepseekvl2}   & 16.1B  & 66.4  & 81.2  & 61.0  & 50.7  & 59.4  & 51.5  & 84.5  & 825  & 60.0  \\
VITA-1.5~\cite{fu2025vita}   & 8.3B   & 63.3  & 76.8  & 60.2  & 52.6  & 66.2  & 44.6  & 79.2  & 741  & 52.7  \\
Baichuan-Omni~\cite{baichuan-omni}   & 7B   & -  & 75.6  & -  & 47.3  & 51.9  & 47.8  & -  & 700  & 65.4  \\
LLaVA-OneVision-7B~\cite{li2024llavaonevision}   & 8B   & 61.2  & 76.8  & 56.7  & 46.8  & 58.5  & 47.5  & 82.8  & 697  & 50.6  \\
Molmo-7B-D~\cite{deitke2024molmo}   & 8B   & 58.9  & 70.9  & 54.4  & 48.7  & 47.3  & 47.7  & 79.6  & 694  & 53.3  \\
% MiniCPM-o 2.6~\cite{yao2024minicpm}   & 8B   & 70.2  & 80.5  & 64.0  & 50.4  & 71.9  & 51.9  & 85.8  & 897  & 67.5  \\
\rowcolor{Gray} \textbf{\method-9B}  & 8.8B   & 69.7  & 80.7  & 60.5  & 51.2  & 68.3  & 51.8  & 84.5  & 883 & 72.3 \\
\bottomrule
\end{tabular}
\end{table}

\begin{table}[t]
  \caption{\textbf{Performance comparison on video and Interleave benchmarks} compared with existing approaches. $^*$ indicates officially released checkpoints evaluated by us. Best performance is marked \textbf{bold}. }
  \label{tab: video_n_interleave}
  \centering
  \setlength{\tabcolsep}{7.5pt}
  \begin{tabular}{lccccc}
    \toprule
       & \multicolumn{2}{c}{\textbf{VideoMME}} & \multicolumn{1}{c}{\textbf{MVBench}} & \multicolumn{2}{c}{\textbf{Llava-Interleave}}\\
    \cmidrule(r){2-3} \cmidrule(r){4-4} \cmidrule(r){5-6}
    Model & w/o subs & w subs & avg & in-domain & out-domain \\
    \midrule
     MiniCPM-V-2.6~\cite{yao2024minicpm} &  60.9 &  63.6 &  - &  - &  - \\
     LLaVA-OneVision-7B~\cite{li2024llavaonevision} &  58.2 &  - &  - &  - &  - \\
     Qwen2-VL-7B~\cite{qwen2-vl_2024} &  63.3 &  69.0 &  67.0 &  49.5$^*$ &  51.0$^*$ \\
     InternVL2-8B~\cite{chen2024far} &  56.3 & 59.3 &  65.8 &  - &  - \\
     VITA-1.5~\cite{fu2025vita} &  56.1 & 58.7 &  55.4 &  - &  - \\
     Baichuan-Omni~\cite{baichuan-omni} &  58.2 & - &  60.9 &  - &  - \\
     MiniCPM-o-2.6~\cite{yao2024minicpm} & 63.0$^*$ & 65.3$^*$ & 58.1$^*$ &  43.5$^*$ &  36.8$^*$ \\
     \rowcolor{Gray} \textbf{\method-9B} &  60.4  & 65.0 &  66.3 &  59.8 &  87.8 \\
    \midrule
    VideoLLaMA2-72B~\cite{cheng2024videollama2} & 61.4 & 63.1 & 62.0 & - & - \\
    LLaVA-OneVision-72B~\cite{li2024llavaonevision} &  66.2 &  69.5 &  59.4 &  - &  - \\
    Qwen2-VL-72B~\cite{qwen2-vl_2024} &  71.2 &  77.8 &  \textbf{73.6} &  - &  - \\
    InternVL2-Llama3-76B~\cite{chen2024far} &  64.7  & 67.8 &  69.6 &  - &  - \\
    \rowcolor{Gray} \textbf{\method-72B} &  65.2  & 67.7 &  69.6 &  \textbf{63.5} &  \textbf{89.9} \\
    \midrule
    GPT-4v~\cite{GPT4VisionSystemCard} & 59.9 & 63.3 & 43.7 & 39.2 & 57.78 \\
    GPT-4o-20240513~\cite{openai2024gpt4ocard} & 71.9 & 77.2 & - & - & - \\
    Gemini-1.5-Pro~\cite{geminiteam2024gemini15unlockingmultimodal} & \textbf{75.0} & \textbf{81.3} & - & - & - \\
    \bottomrule
\end{tabular}
\end{table}


In this section, we present a comprehensive evaluation of our \method model, comprising both quantitative and qualitative analyses of its performance. Furthermore, we conduct ablation studies to analyze the contributions of several key design components to the performance of our \method model, providing insights into their distinct impacts.

% In this section, we first evaluate the model’s performance on a variety of mainstream benchmarks, demonstrating the advantages of \method.
% Then, a series of qualitative results are presented to show the model’s specific capabilities, including multimodal understanding and free-form image generation.
% Finally, we conduct an ablation study to analyze several key components in \method.

\subsection{Quantitative Results}\label{subsec:exp_quantitative_results}

\subsubsection{Image-Text Understanding}
To evaluate the effectiveness of our \method in image-text understanding, we benchmark it against state-of-the-art MLLMs on the OpenCompass~\cite{2023opencompass} multimodal leaderboard, a widely recognized platform for multimodal evaluation. This leaderboard contains 8 different multimodal benchmarks, including complex VQA (MMBench~\cite{liu2025mmbench}, MMStar~\cite{chen2024we}, MMMU~\cite{yue2023mmmu}, AI2D~\cite{kembhavi2016diagram}, and MMVet~\cite{yu2024mm}), multimodal reasoning (MathVista~\cite{lu2024mathvista}), hallucination evaluation (Hallusionbench~\cite{Guan_2024_hallusionbench}), and OCR (OCRBench~\cite{Liu_2024}).
\cref{tab:exp_it_oc} shows the overall results. Our \method-72B model achieves top-tier performance on most benchmarks, surpassing closed-source models like GPT-4o and Gemini-1.5-Pro. Furthermore, our \method-9B model exhibits competitive performance among models of similar size, showcasing its robust capabilities in image-text understanding tasks.

% In this section, we compare our \method with leading MLLMs on the mainstream OpenCompass~\cite{2023opencompass} multimodal leaderboard to demonstrate its advancement on image-text understanding.
% This leaderboard contains 8 different multimodal benchmarks, including complex VQA (MMBench~\cite{liu2025mmbench}, MMStar~\cite{chen2024we}, MMMU~\cite{yue2023mmmu}, AI2D~\cite{kembhavi2016diagram}, and MMVet~\cite{yu2024mm}), multimodal reasoning (MathVista~\cite{lu2024mathvista}), hallucination evaluation (Hallusionbench~\cite{Guan_2024_hallusionbench}), and OCR (OCRBench~\cite{Liu_2024}).
% \cref{tab:exp_it_oc} shows the overall results.
% \method exhibits competitive performance compared with other MLLMs.
% Our \method-71B model achieves top-tier performance on most benchmarks.
% It outperforms closed-source models such as GPT-4o and Gemini-1.5-Pro.
% The \method-9B model also achieves competitive performance among other vision-language specific MLLM models smaller than 20B.
% Notably, it achieves excellent performance on MathVista, AI2D, and MMVet, demonstrating its comprehensive ability on multimodal reasoning and complex VQA.




\subsubsection{Video \& Interleaved Image-Text Understanding}

We evaluate our model's video and interleaved image-text understanding abilities on three mainstream benchmarks.

\textbf{Video-MME}~\cite{fu2024video}: Video-MME is a benchmark designed to evaluate MLLMs in full-spectrum video analysis. It encompasses a wide variety of video types across multiple domains and durations, featuring multimodal inputs such as video, subtitles, and audio. For this benchmark, testing is conducted with under 96 frames, and results are reported for both "with subtitles" and "without subtitles" settings.

\textbf{MVBench}~\cite{li2024mvbench}: MVBench serves as a video understanding benchmark aimed at thoroughly evaluating the temporal awareness of MLLMs in an open-world context. It includes 20 challenging video tasks that range from perception to cognition, which cannot be adequately addressed using a single frame. Testing for this benchmark utilizes dynamic sampling frames.

\textbf{LLaVA-Interleave Bench}~\cite{llava-next_2024}: LLaVA-Interleave Bench comprises a comprehensive suite of multi-image benchmarks collected from public datasets or generated via the GPT-4V API. It is created to assess the interleaved multi-image reasoning capabilities of MLLMs, with reported results for both "in-domain" and "out-domain" subsets.

As shown in Table~\ref{tab: video_n_interleave}, \method-9B achieves the second-best results across VideoMME and MVBench (outperformed only by Qwen2-VL-7B but requiring significantly fewer frames). However, the performance gains do not scale up to \method-72B due to limitations in the quantity of instruction-tuned video data. Moreover, both our \method-9B and \method-72B greatly surpass all other baselines in multi-image benchmarks, both in-domain and out-of-domain, highlighting their potential as strong competitors for complex tasks.


\subsubsection{Audio Understanding}

We evaluate our M2-omni model's audio understanding abilities on four mainstream benchmarks.

\textbf{Multilingual LibriSpeech (MLS)}~\cite{MLS_English}: The Multilingual LibriSpeech dataset is an extensive collection of read audiobooks sourced from Librivox, available in eight different languages. We utilize the English test set from this dataset to assess the model's speech comprehension capabilities. The latest version of this corpus comprises approximately 50,000 hours.

\textbf{Librispeech}~\cite{Librispeech}: The Librispeech corpus comprises approximately 1,000 hours of transcribed speech audio data derived from read English audiobooks. The entire dataset is categorized into three training sets (100 hours of clean, 360 hours of clean, and 500 hours of other), two validation sets (clean and other), and two test sets (clean and other). In this study, we assess our model's audio comprehension capabilities using both the clean and other testsets.

\textbf{Aishell1}~\cite{AISHELL1}:  The Aishell1 dataset comprises 178 hours of speech data, recorded by 400 speakers from various accent regions across China. It is organized into three subsets: a training set consisting of 340 speakers, a validation set with 40 speakers, and a test set featuring 20 speakers.

\textbf{AudioCaps}~\cite{AudioCaps}: AudioCaps is a comprehensive dataset featuring audio event descriptions specifically curated for the purpose of audio captioning. The sounds within this collection are derived from the AudioSet dataset. We utilize this dataset to assess the audio captioning capabilities of our \method.
 % To facilitate accurate captioning, annotators were supplied with audio tracks and corresponding categorical hints, with additional video hints provided as necessary.

The results are presented in Table~\ref{tab:exp_audio_understand}, and our \method-9B demonstrates competitive performance in speech recognition and audio captioning tasks. 
Specifically, our \method-9B is comparable to GPT-4o-Realtime~\cite{openai2024gpt4ocard}.
In addition, \method-9B significantly outperforms all other baselines on AudioCaps benchmarks, while achieving the second-best results for the MLS English, Librispeech other, Librispeech-clean and Aishell1 benchmarks.

\begin{table}[]
\centering
\caption{\textbf{Quantitative results on speech recognition and audio captioning.}
 $^*$ indicates results from \cite{yao2024minicpm}.
}
\label{tab:exp_audio_understand}
\setlength{\tabcolsep}{7pt}
\begin{tabular}{l|cccccccccc}
\toprule
Models   & MLS- & Librispeech- & Librispeech- & Aishell1 & AudioCaps \\
                & English & other & clean &  & \\
                & WER$\downarrow$ & WER$\downarrow$ & WER$\downarrow$ & WER$\downarrow$ & CIDER$\uparrow$ \\
\midrule
UIO2-L-1.1B~\cite{lu2023uio2}   & - & - & - & - & 45.7   \\
UIO2-XL-3.2B~\cite{lu2023uio2}  & - & - & - & - & 45.7   \\
UIO2-XXL-6.8B~\cite{lu2023uio2} & - & - & - & - & 48.9  \\
Whisper-large-v2~\cite{Whisper}  & \textbf{6.83} & \textbf{5.16} & 2.87 & - & - \\
Paraformer-cn~\cite{gao2022paraformer} & - & - & - & 2.12 & - \\
VITA-1.5~\cite{VITA_1.5} & - & 7.5 & 3.4 & 2.2 & - \\
Mini-Omini2~\cite{mini_omni2} & - & 9.8 & 4.8 & - & - \\
Freeze-Omini~\cite{Freeze_Omni} & - & 10.5 & 4.1 & 2.8 & - \\
MiniCPM-o-2.6~\cite{yao2024minicpm} & - & - & \textbf{1.7} & \textbf{1.6} & - \\
GPT-4o-Realtime~\cite{openai2024gpt4ocard} & - & - & 2.6$^*$ & 7.3$^*$ & - \\
\rowcolor{Gray} \textbf{\method-9B}   & 7.19 & 5.29 & 2.07 & 1.99 & \textbf{49.2} \\
\bottomrule
\end{tabular}
\end{table}

\begin{table}[t]
\centering
\caption{\textbf{Quantitative results on language benchmarks.} $^*$ indicates officially released checkpoints evaluated using the tools provided by OpenCompass~\cite{2023opencompass}.
}
\label{tab:exp_language}
\setlength{\tabcolsep}{5pt}
\begin{tabular}{cccccccc}
\hline
Tasks & MMLU & AGIEVAL & ARC-C & GPQA & MATH & HellaSwag & \begin{tabular}[c]{@{}l@{}}Avg.\\ Accuracy\end{tabular} \\ \hline
LLama3.1-8B & 69.4 & 41.2$^*$ & 83.4 & 30.4 & 51.9 & 75.1$^*$ & 58.6  \\
\rowcolor{Gray} \textbf{\method-9B} & 68.5 & 43.7 & 78.7 & 32.3 & 51.8 & 80.1 & 59.2  \\ \hline
\end{tabular}
\end{table}

\subsubsection{Audio Generation}
In this section, we also evaluated our model on the commonly-used test set: SEED-TTS test-zh. \textbf{SEED-TTS}~\cite{SEED_TTS} serves as an out-of-domain evaluation test set, comprising diverse input texts and reference speeches from various domains. We present the experimental results for \method-9B and the baseline models in Table~\ref{tab:exp_audio_generation}. As shown in Table~\ref{tab:exp_audio_generation}, our model outperforms MiniCPM-o-2.6~\cite{yao2024minicpm} in speech generation capability, achieving significant improvements in both evaluation metrics. However, our \method-9B still lags behind traditional vertical speech generation models, highlighting the need for further research and development to bridge this gap.


\subsubsection{Text-only Performance}
In this section, we assess the performance of our proposed \method-9B model and its initial counterpart, Llama3.1-8B~\cite{llama3_2024}. To evaluate the models' knowledge and examination capabilities, we employ a range of benchmarks, including AGIEVAL~\cite{zhong2023agievalhumancentricbenchmarkevaluating} and MMLU~\cite{hendrycks2021measuringmassivemultitasklanguage}. Furthermore, we utilize a diverse set of benchmarks to evaluate the models' multi-step problem-solving capabilities, including MATH~\cite{hendrycks2021measuringmathematicalproblemsolving} for mathematical derivation, HellaSwag~\cite{zellers2019hellaswagmachinereallyfinish} for commonsense reasoning in real-world contexts, ARC-C~\cite{allenai:arc} for scientific logical chains, and GPQA~\cite{rein2023gpqagraduatelevelgoogleproofqa} for critical analysis in expert-level domains. For all evaluation datasets, we adopt a generation-based assessment approach with greedy decoding.

Our experimental results, presented in \cref{tab:exp_language}, demonstrate that the performance of our proposed \method-9B model outperforms its initial counterpart, Llama3.1-8B across most evaluation datasets,   which is attributed to our multi-stage language preservation strategy and the high-quality instruction tuning data used in our training process.

% In this section, we evaluate the performance of our \method-9B and its initial Llama3.1~\cite{llama3_2024} models. To assess the models' knowledge and examination capabilities, we utilize the AGIEVAL~\cite{zhong2023agievalhumancentricbenchmarkevaluating},  MMLU~\cite{hendrycks2021measuringmassivemultitasklanguage} benchmarks. Additionally, we employ  MATH~\cite{hendrycks2021measuringmathematicalproblemsolving}, HellaSwag~\cite{zellers2019hellaswagmachinereallyfinish}, ARC-C~\cite{allenai:arc} and GPQA~\cite{rein2023gpqagraduatelevelgoogleproofqa} to evaluate the models' multi-step problem-solving ability, including mathematical derivation, commonsense reasoning in real-world contexts, scientific logical chains, and critical analysis in expert-level domains. For all evaluation datasets, we adopt a generation-based assessment approach with greedy decoding. The overall results are in \cref{tab:exp_language}.It can be observed that in most of the evaluation datasets, the performance of our \method-9B and Llama3.1~\cite{llama3_2024} models is comparable, maintaining their linguistic capabilities. Furthermore, in some rankings, our models exhibit superior performance in certain aspects compared to their text-only baseline models. This improvement is attributed to our multi-stage language preservation strategy and the high-quality instruction tuning data used in our training process.

\begin{table}[t]
\centering
\caption{
\textbf{Free-form dialogue generation evaluation results.}
}
% \vspace{3pt}
\setlength{\tabcolsep}{8pt}
\begin{tabular}{c|c|c|c}
\toprule
Model & Relevance & Fluency & Informativeness\\
\midrule
TextBind~\cite{li2023textbind} & 3.85 & 4.30 & 3.25\\
\rowcolor{Gray} \textbf{\method-9B} & 4.60 & 4.80 & 3.80\\
\bottomrule
\end{tabular}
\label{tab-model_freeform_results}
% \vspace{-12pt}
\end{table}




% For a evaluation of open-world multi-turn multimodal instruction following, we collect a test set comprising 50 conversations from realistic scenarios and utilize \method-9B to generate arbitrarily interleaved text and images in proper conversation contexts. For quantitative results, we ask GPT-4o~\cite{openai2024gpt4ocard} to rate each conversation ranging from 0 to 5 considering relevance, fluency and informativeness. We carry out our quantitative results against recent work TextBind~\cite{li2023textbind}. As shown in \cref{tab-model_freeform_results}, \method-9B exhibits overall better understanding and generating ability of multi-turn multimodal conversations. More qualitative cases can be found in \cref{fig-IT-Freeform-Result}.



\begin{table}[t]
  \caption{\textbf{Quantitative results on audio generation.} $^*$ indicates officially released checkpoints evaluated by us.}
  \label{tab:exp_audio_generation}
  \centering
  \setlength{\tabcolsep}{14pt}
  \begin{tabular}{lccccc}
    \toprule
       & \multicolumn{2}{c}{\textbf{SEED test-zh}}\\
    \cmidrule(r){2-3}
    Model & CER(\%)$\downarrow$ & SS$\uparrow$  \\
    \midrule

     Human & 1.26 &0.755 \\
     Vocoder Resyn. & 1.27 & 0.720 \\
     \midrule
     Seed-TTS~\cite{SEED_TTS} & 1.12 & 0.796 \\
     FireRedTTS~\cite{FireRedTTS} & 1.51 &0.635 \\
     MaskGCT~\cite{MaskGCT} & 2.27 & 0.774 \\
     E2-TTS(32 NFE)~\cite{E2_TTS} & 1.97 & 0.730 \\
     F5-TTS(32 NFE)~\cite{F5_TTS} & 1.56 & 0.741 \\
     CosyVoice~\cite{CosyVoice} &3.63 &0.723 \\
     CosyVoice2~\cite{CosyVoice2} &1.45 &0.748 \\
     CosyVoice2-S~\cite{CosyVoice2} &1.45 &0.753 \\
     CosyVoice2-S~\cite{CosyVoice2} &1.45 &0.753 \\
     \midrule
     MiniCPM-o-2.6~\cite{yao2024minicpm} &8.03$^*$ &0.474$^*$ \\
     \rowcolor{Gray} \textbf{\method-9B} &  6.36  & 0.604 \\
    \bottomrule
\end{tabular}
\end{table}


\subsubsection{User Experience Evaluation}\label{sec:human_evaluation}
\textbf{Evaluation Metric}:
Current benchmarks such as MMBench~\cite{liu2025mmbench}, MMStar~\cite{chen2024we}, and MMMU~\cite{yue2023mmmu} primarily focus on assessment through judgment-style questions. However, this assessment does not align with the users' actual interactive experience with MLLMs. To address this limitation, drawing inspiration from SuperclueV~\cite{supercluev}, we develop evaluation criteria specifically for assessing the models' performance on user experience, which contains four key dimensions: relevance, fluency, informativeness, and format rationality. \textit{Relevance} assesses the extent to which the model's responses align with both the provided prompts and the multimodal inputs.
\textit{Fluency} evaluates the naturalness, smoothness, clarity, comprehensibility, and anthropomorphic quality of the model's responses.
\textit{Informativeness} measures the extent to which the model's responses provide relevant information, knowledge, and analytical reasoning, enhancing their utility, detail, depth, and innovation.
\textit{Format rationality} examines the model's ability to adaptively generate appropriately structured and clear formats, for presenting results based on varying prompt types.



% Current benchmarks such as MMBench~\cite{liu2025mmbench}, MMStar~\cite{chen2024we}, and MMMU~\cite{yue2023mmmu} primarily focus on assessment through judgment-style questions. However, this assessment does not align with the users' actual interactive experience with MLLMs. Drawing inspiration from SuperclueV~\cite{supercluev}, we develop evaluation criteria specifically for assessing the models' experience performance, which contains four key dimensions: relevance, fluency, content richness, and format rationality. \textbf{Relevance} assesses the extent to which the model's responses align with both the provided prompts and the multi-modal inputs.
% \textbf{Fluency} evaluates the naturalness, smoothness, clarity, comprehensibility, and anthropomorphic quality of the model's responses.
% \textbf{Content richness} gauges the degree to which the model's responses are enriched with supplementary information, knowledge, and analytical reasoning, enhancing their utility, detail, depth, and innovation.
% \textbf{Format rationality} examines the model's ability to adaptively generate appropriately structured and clear formats for presenting results based on varying prompt types.


\begin{table}[t]
\centering
\caption{
\textbf{Detailed model experience evaluation standards.}
}
% \vspace{3pt}
\setlength{\tabcolsep}{4pt}
\begin{tabular}{c|c}
\toprule
Score & Description\\
\midrule
1 & Totally unsatisfied, totally unacceptable \\
2 & Basically not satisfied, with many obvious problems \\
3 & Generally satisfied, with a few obvious problems \\
4 & Basically satisfied, minor flaws allowed \\
5 & Completely satisfied, almost perfect \\
\bottomrule
\end{tabular}
\label{tab-model_expr_standards}
% \vspace{-12pt}
\end{table}

\textbf{Evaluation Dataset}: We collect chat samples from the actual users' multi-turn interaction dialogues, which cover a variety of tasks, including visual question answering (VQA), conversational interactions, chart interpretation, mathematical problem-solving, optical character recognition (OCR), and other related tasks. GPT-4o~\cite{openai2024gpt4ocard} is instructed to follow the evaluation criteria to generate initial reference answers for these collected samples. To ensure accuracy, human annotators refine the initial responses generated by GPT-4o. This process yields an evaluation dataset with nearly 300 samples, each with a corresponding ground truth.

We utilize GPT-4o to evaluate the model's responses against the ground truth, adhering to the standards outlined in  \cref{tab-model_expr_standards}.  As shown in \cref{tab-user_experience},  our M2-omni model, after undergoing  alignment tuning,  demonstrates an average increase of 5.7\%-23.4\% in user experience performance, which is further validated by human annotations on selected cases. Meanwhile, our model's performance on the OC benchmark across other modalities remains relatively consistent, thereby demonstrating the effectiveness of our unified training strategy, which integrates DPO and instruction tuning in the alignment tuning stage.

% We employ GPT-4o to score the models' responses compared with ground truth according to the standards of \cref{tab-model_expr_standards}.  \cref{tab-user_experience} shows the model after alignment tuning demonstrates an average increase of 5.7\% in performance. This enhancement is corroborated by human annotations on selected cases. Simultaneously, the general capabilities on OC benchmark across other modalities remain nearly the same, with a decrease in average evaluation scores of less than 1\%. This demostrates the effectiveness of our unified training strategy that integrates DPO and
% instruction tuning in alignment tuning stage.


\subsubsection{Free-Form Dialogue Generation}
To evaluate the open-world multi-turn multimodal instruction following capabilities of our model, we create a test set consisting of 50 conversations derived from realistic scenarios. We utilize \method-9B to generate arbitrarily interleaved text and images in proper conversation contexts.
For quantitative results, following our user experience evaluation metric, we employ GPT-4o to rate each conversation on a scale of 0 to 5 across three evaluation dimensions: relevance, fluency, and informativeness.
We carry out our quantitative results against recent work TextBind~\cite{li2023textbind}. As shown in \cref{tab-model_freeform_results}, \method-9B exhibits overall better understanding and generating ability of multi-turn multimodal conversations. More qualitative cases can be found in \cref{fig-IT-Freeform-Result}.





\begin{table}[t]
\centering\footnotesize
\caption{
\textbf{Detailed evaluation on user experience benchmark and OC benchmark. OC is short for the OpenCompass image-text understanding benchmark.}
}
% \vspace{3pt}
\setlength{\tabcolsep}{3pt}
\begin{tabular}{c|c|c|c|c|c|c}
\toprule
Model & Relevance & Fluency & Informativeness & Format Rationality & Expr. Avg($\Delta$\%) & OC Avg($\Delta$)\\
\midrule
\method-9B & 4.556 & 4.036 & 2.742 & 3.573 & 3.726 & -\\
\rowcolor{Gray} \method-9B-Align & 4.893 & 4.735 & 4.118 & 4.644 & 4.598(+23.4\%) & -0.3\\
\method-72B & 4.942 & 4.689 & 3.267 & 4.265 & 4.351 & -\\
\rowcolor{Gray} \method-72B-Align & 4.946 & 4.875 & 3.961 & 4.615 & 4.598(+5.7\%) & -0.2\\
InternVL2-26B~\cite{internvl_2024} & 4.886 & 4.76 & 4.15 & 4.52 & 4.577 & -\\
GPT-4o~\cite{openai2024gpt4ocard} & 5 & 4.878 & 3.854 & 4.831 & 4.64 & -\\
\bottomrule
\end{tabular}
\label{tab-user_experience}
% \vspace{-12pt}
\end{table}



\subsection{Qualitative Results}\label{subsec:exp_qualitative_results}

In this section, we qualitatively assess the capabilities of our \method, presenting examples of each modality and different tasks.

We show multimodal understanding abilities of our \method in \cref{fig-exp_case_all}. \method demonstrates promising capabilities in processing cross-modal problems, encompassing image understanding, video understanding, interleaved image-text understanding, and image-audio understanding. More examples can be found in the appendix, provided in \cref{subsec:appendix_cases}.

\cref{fig-IT-Freeform-Result} illustrates the model's ability to generate free-form dialogue, where our \method can create images based on the conversation context without explicit user input, useful for explaining ideas to users.




\begin{figure}[t]
    \centering
    \includegraphics[width=0.9\linewidth]{figures/case_exp.pdf}
    \caption{
    \textbf{Cases for multimodal understanding.}
    \method shows great potential to solve various multimodal problems.
    }
    \label{fig-exp_case_all}
\end{figure}




\begin{figure}[t]
    \centering
    \includegraphics[width=0.9\linewidth]{figures/free_form_gen.pdf}
    \caption{
    \textbf{Cases for Free-Form Dialogue Generation.}
    }
    \label{fig-IT-Freeform-Result}
\end{figure}


\subsection{Ablation Study}\label{subsec:exp_ablation}

\begin{table}[t]
\centering
\caption{\textbf{Ablation studies on step balancing strategy.} The loss weight setting [1,1,1] corresponds to the uniform weighting of the loss functions for image-text pairs, interleaved image-text, and video datasets.  * and \# represent the loss weight settings. * is obtained through experimental trials and parameter tuning. \# is obtained by normalizing the loss weights using the inverse of the loss at convergence, as described in Section \cref{subsubsec-Step Balancing Strategy}. We evaluate the few-shot performance on VQA tasks and the zero-shot performance on the captioning task of our pre-trained model.}
\label{tab:ablation_step_balance_pretrain}
\setlength{\tabcolsep}{4pt}
\begin{tabular}{c|c|ccc}
\toprule
\multicolumn{1}{l|}{Data Sample Balance} & Loss Weight Balance & \multicolumn{1}{l}{OK-VQA(4-shot)} & \multicolumn{1}{l}{VQAv2(4-shot)} & \multicolumn{1}{l}{Flickr30k(0-shot)} \\ \hline
Random Sample                        & {[}1,1,1{]}          & 40.5                             & 54.3                             & 87.0                                 \\
Round-robin                          & {[}1,1,1{]}          & 41.6                             & 54.4                             & 88.1                                 \\
Accumulation                         & {[}1,1,1{]}          & 41.7                             & 54.6                             & 88.2                                 \\
Accumulation                         & ${[}0.2,1.0,0.03{]}^{*}$   & 39.7                             & 52.5                             & 87.1                                 \\
Accumulation                         & ${[}0.45,0.36,1.09{]}^{\#}$ & \textbf{42.1}                             & \textbf{55.4}                             & \textbf{88.2}                                 \\
\bottomrule
\end{tabular}
\end{table}


In this section, we conduct ablation studies to investigate the effectiveness of our step balance strategy and dynamic adaptive balance strategy in our M2-omni model. These experiments aim to provide insights into the impact of these key components on our M2-omni’s performance.

\subsubsection{Step Balance Strategy}\label{subsubsec:step_balance_ablaton}

As described in \cref{subsubsec-Step Balancing Strategy} ,  we investigate the impact of various data sample balancing strategies and loss weight balancing schemes on the multimodal joint training stage of pre-training. We evaluate the performance of candidate strategies on two VQA benchmarks, OK-VQA~\cite{marino2019ok} and VQAv2~\cite{goyal2017making}, and assess its image captioning performance using the Flickr30k~\cite{young2014image} benchmark.

For pretrained models lacking in instruction following ability, to assess the effectiveness of our approach, we evaluate the performance of these models on VQA tasks using a few-shot approach and on image caption tasks using a zero-shot approach. \cref{tab:ablation_step_balance_pretrain} presents the results of our M2-omni pretrained models, which demonstrate the effectiveness of our step balance strategy.

% Besides, three task weighting manner are compared: [1,1,1], which means all data shares the same optimization step size; [0.2,1.0,0.03], which is consistent with that proposed in \cite{alayrac2022flamingo}; [0.45,0.36,1.09], the inverse of the loss at convergence state, as \cref{subsubsec-Step Balancing Strategy} described. Note that the three values in the ratio correspond to image-text pairs, interleaved image-text and video datasets.

%  We directly evaluate the pre-trained model's performance on VQA tasks using a few-shot approach and on image caption tasks using a zero-shot approach. For VQA tasks, we use two benchmarks: OK-VQA~\cite{marino2019ok} and VQAv2~\cite{goyal2017making}, while for image captioning, we use the Flickr30k~\cite{young2014image} benchmark. \cref{tab:ablation_step_balance_pretrain} shows the results of training models on the combined datasets using three different merging regimes. It can be observed that the accumulation strategies and setting the task weights to the inverse of the loss achieve the best performance.







\begin{table}[t]
\centering
\caption{
\textbf{Ablation results of the dynamic adaptive balance strategy}. Results for unimodal baselines are derived from the following single-modal models: \textsuperscript{$\dagger$} Image-Text Model, \textsuperscript{$\ddagger$} Video-Text Model, and \textsuperscript{
$\mathsection$} Audio-Text Model. The best result for each benchmark is \textbf{bolded}, while the best result for each model across all epochs is \underline{underlined}.
}
% \vspace{3pt}
\setlength{\tabcolsep}{3pt}
\begin{tabular}{l|l|ccccc|cc|cc}
\toprule
Models &  & MM- & OK- & VQAv2 & Text- & GQA & MSVD- & MSRVTT & Audio & MLS- \\
& & Bench & VQA &&VQA&& QA & QA & Caps & English($\downarrow$) \\
\midrule
\multirow{3}{*}{\makecell[l]{Single-modal\\Baselines}} & ep1 & 68.0\textsuperscript{$\dagger$} & 56.4\textsuperscript{$\dagger$} & 74.8\textsuperscript{$\dagger$} & \underline{70.4}\textsuperscript{$\dagger$} & 58.4\textsuperscript{$\dagger$} & 72.3\textsuperscript{$\ddagger$} & 59.3\textsuperscript{$\ddagger$} & 29.0\textsuperscript{$\mathsection$} & 11.4\textsuperscript{$\mathsection$} \\
& ep2 & \underline{\textbf{77.8}}\textsuperscript{$\dagger$} & \underline{59.8}\textsuperscript{$\dagger$} & \underline{76.9}\textsuperscript{$\dagger$} & 69.8\textsuperscript{$\dagger$} & 60.6\textsuperscript{$\dagger$} & \underline{\textbf{76.5}}\textsuperscript{$\ddagger$} & \underline{60.1}\textsuperscript{$\ddagger$} & \underline{39.9}\textsuperscript{$\mathsection$} & 9.33\textsuperscript{$\mathsection$} \\
& ep3 & 77.3\textsuperscript{$\dagger$} & 58.0\textsuperscript{$\dagger$} & 76.8\textsuperscript{$\dagger$} & 69.1\textsuperscript{$\dagger$} & \underline{60.8}\textsuperscript{$\dagger$} & 74.4\textsuperscript{$\ddagger$} & 58.6\textsuperscript{$\ddagger$} & 39.5\textsuperscript{$\mathsection$} & \underline{8.96}\textsuperscript{$\mathsection$} \\
\midrule
\multirow{3}{*}{\makecell[l]{Mixture \\w/o MM-Bal.}}
& ep1 & 70.5 & 55.9 & 75.7 & 70.2 & 57.7 & \underline{75.1} & \underline{59.6} & 27.5 & 12.1 \\
& ep2 & \underline{75.8} & \underline{58.8} & \underline{77.0} & \underline{70.5} & \underline{\textbf{61.1}} & 73.4 & 58.5 & 33.5 & 9.45 \\
& ep3 & 75.6 & 58.4 & 76.5 & 69.5 & 60.1 & 70.2 & 56.9 & \underline{39.6} & \underline{8.98} \\
\midrule
\multirow{3}{*}{\makecell[l]{Mixture \\w/ MM-Bal.}}
& ep1 & 74.7 & 59.6 & 76.0 & 71.2 & 59.0 & 73.1 & 58.7 & 35.5 & 9.27 \\
& ep2 & \underline{\textbf{77.8}} & \underline{\textbf{61.7}} & \underline{\textbf{77.2}} & \underline{\textbf{71.8}} & 60.5 & \underline{74.8} & 58.5 & 41.2 & 8.31 \\
& ep3 & 77.1 & 60.5 & 77.0 & 69.8 & \underline{60.7} & 74.6 & \underline{\textbf{60.2}} & \underline{\textbf{44.1}} & \underline{\textbf{8.04}} \\

\bottomrule
\end{tabular}
\label{tab-multi_task_balanced_ablation}
% \vspace{-12pt}
\end{table}

\subsubsection{Dynamic Adaptive Balance Strategy}

We conducted a evaluation of our dynamic adaptive balance strategy across text-image, video, and audio modalities using constrained datasets. The evaluation was conducted on benchmark datasets specific to each modality: for text-image tasks, MMbench~\cite{liu2025mmbench}, OK-VQA~\cite{marino2019ok}, VQAv2~\cite{goyal2017making}, TextVQA~\cite{singh2019towards}, and GQA~\cite{hudson2019gqa} were employed; for video, MSVD-QA~\cite{xu2017video} and MSRVTT-QA~\cite{xu2017video} benchmarks were utilized; and for audio, we assessed performance on the AudioCaps~\cite{kim2019audiocaps} (AAC) and MLS~\cite{Pratap2020MLSAL}-English (ASR) tasks. The experimental outcomes are detailed in Table~\ref{tab-multi_task_balanced_ablation}.

In contrast to actual training pipeline, our evaluation involved instruction tuning starting from pre-trained models. Specifically, for each modality, we initially trained single-modality baseline models (the 'Sinle-modal Baselines' in Table~\ref{tab-multi_task_balanced_ablation}) individually over three epochs to establish the maximum achievable performance per modality. The results indicate that optimal performance was predominantly observed by the second epoch. However, the ASR task, due to its more complex patterns, had not fully converged even by the third epoch. Subsequently, we combined data from all three modalities to train a unified model (the 'Mixture w/o MM-Bal.' in Table~\ref{tab-multi_task_balanced_ablation}). Under this multimodal training regimen, the image-text modality reached its optimal performance at the second epoch, while the video modality achieved peak performance as early as the first epoch and with performance consistently decreasing in subsequent epochs. In contrast, the audio modality demonstrated continuous improvement, attaining its best performance by the third epoch. These observations underscore the imbalance in training progress among different modalities when engaged in multimodal training.

To address this imbalance, we introduced the dynamic adaptive balance strategy within our M2-omni training framework. This strategy dynamically adjusts the loss weights for each modality based on their respective training progress. In the context of this evaluation, it accelerates the training of the audio modality while appropriately reducing the learning weights for the image-text and video modalities to prevent overfitting. The evaluation results for this balanced training approach are denoted as 'Mixture w/ MM-Bal.' in Table~\ref{tab-multi_task_balanced_ablation}. The results demonstrate that, although some degree of imbalance among modalities persists, the balanced training strategy significantly alleviates the issues observed with simple mixed training: optimal performances across benchmarks are now concentrated around the second and third epochs, and performance across all modalities has been markedly enhanced. Moreover, under the balanced training strategy, the model achieved single-modality optimal performance in 7 out of 9 benchmarks. The best-performing model (at epoch 2) surpassed the optimal performance of each single-modality baseline in 6 out of 9 benchmarks (MMBench, OK-VQA, VQAv2, TextVQA, AudioCaps, MLS-English). Additionally, for the audio modality, the model at epoch 3 outperformed the single-modality baselines in 5 out of 9 benchmarks (OK-VQA, VQAv2, MSRVTT-QA, AudioCaps, MLS-English), with significant improvements in audio performance. These experimental results highlight the effectiveness of our dynamic adaptive balance strategy.

\section{Conclusion}
In this paper, we introduced Atom of Thoughts (\our), a novel framework that transforms complex reasoning processes into a Markov process of atomic questions. By implementing a two-phase transition mechanism of decomposition and contraction, \our eliminates the need to maintain historical dependencies during reasoning, allowing models to focus computational resources on the current question state. Our extensive evaluation across diverse benchmarks demonstrates that \our serves effectively both as a standalone framework and as a plug-in enhancement for existing test-time scaling methods. These results validate \our's ability to enhance LLMs' reasoning capabilities while optimizing computational efficiency through its Markov-style approach to question decomposition and atomic state transitions.

\begin{acks}
  This work is supported in part by the National Natural Science Foundation of China under 23IAA02114 and 62472241, in part by the joint project of Infinigence AI \& Tsinghua University.
\end{acks}

\clearpage


% This must be in the first 5 lines to tell arXiv to use pdfLaTeX, which is strongly recommended.
\pdfoutput=1
% In particular, the hyperref package requires pdfLaTeX in order to break URLs across lines.

\documentclass[11pt,dvipsnames]{article}

% Change "review" to "final" to generate the final (sometimes called camera-ready) version.
% Change to "preprint" to generate a non-anonymous version with page numbers.
\usepackage[preprint]{acl}

% Standard package includes
\usepackage{times}
\usepackage{latexsym}
\usepackage{bbm}
\usepackage{booktabs}
% For proper rendering and hyphenation of words containing Latin characters (including in bib files)
\usepackage[T1]{fontenc}
% For Vietnamese characters
% \usepackage[T5]{fontenc}
% See https://www.latex-project.org/help/documentation/encguide.pdf for other character sets

% This assumes your files are encoded as UTF8
\usepackage[utf8]{inputenc}

% This is not strictly necessary, and may be commented out,
% but it will improve the layout of the manuscript,
% and will typically save some space.
\usepackage{microtype}

% This is also not strictly necessary, and may be commented out.
% However, it will improve the aesthetics of text in
% the typewriter font.
\usepackage{inconsolata}

%Including images in your LaTeX document requires adding
%additional package(s)
\usepackage{graphicx}

\usepackage{tcolorbox}
% \usepackage[dvipsnames]{xcolor}

\usepackage{xspace}
\newcommand{\vpara}[1]{\vspace{0.04in}\noindent\textbf{#1}\xspace}

\usepackage{multicol}
\usepackage{amsfonts}

\usepackage{caption}
\usepackage{subcaption}
\newcommand{\jy}[1]{{\color{blue}{[(jiaying): #1]}}}


% If the title and author information does not fit in the area allocated, uncomment the following
%
%\setlength\titlebox{<dim>}
%
% and set <dim> to something 5cm or larger.

\title{Evaluating the Paperclip Maximizer: Are RL-Based Language Models More Likely to Pursue Instrumental Goals?}

% Do Large Language Models Pursue Instrumental Goals? A Benchmark Study
% Are RL-Based Language Models More Likely to Pursue Instrumental Goals? A Benchark Study
% Evaluating the Paperclip Maximizer: Do Large Language Models Pursue Instrumental Goals?
% Evaluating the Paperclip Maximizer: Are RL-Based Language Models More Likely to Pursue Instrumental Goals?

% Author information can be set in various styles:
% For several authors from the same institution:
\author{Yufei He, Yuexin Li, Jiaying Wu, Yuan Sui, Yulin Chen, Bryan Hooi\\
        National University of Singapore\\
        \texttt{yufei.he@u.nus.edu}}
% if the names do not fit well on one line use
%         Author 1 \\ {\bf Author 2} \\ ... \\ {\bf Author n} \\
% For authors from different institutions:
% \author{Author 1 \\ Address line \\  ... \\ Address line
%         \And  ... \And
%         Author n \\ Address line \\ ... \\ Address line}
% To start a separate ``row'' of authors use \AND, as in
% \author{Author 1 \\ Address line \\  ... \\ Address line
%         \AND
%         Author 2 \\ Address line \\ ... \\ Address line \And
%         Author 3 \\ Address line \\ ... \\ Address line}

% \author{First Author \\
%   Affiliation / Address line 1 \\
%   Affiliation / Address line 2 \\
%   Affiliation / Address line 3 \\
%   \texttt{email@domain} \\\And
%   Second Author \\
%   Affiliation / Address line 1 \\
%   Affiliation / Address line 2 \\
%   Affiliation / Address line 3 \\
%   \texttt{email@domain} \\}

%\author{
%  \textbf{First Author\textsuperscript{1}},
%  \textbf{Second Author\textsuperscript{1,2}},
%  \textbf{Third T. Author\textsuperscript{1}},
%  \textbf{Fourth Author\textsuperscript{1}},
%\\
%  \textbf{Fifth Author\textsuperscript{1,2}},
%  \textbf{Sixth Author\textsuperscript{1}},
%  \textbf{Seventh Author\textsuperscript{1}},
%  \textbf{Eighth Author \textsuperscript{1,2,3,4}},
%\\
%  \textbf{Ninth Author\textsuperscript{1}},
%  \textbf{Tenth Author\textsuperscript{1}},
%  \textbf{Eleventh E. Author\textsuperscript{1,2,3,4,5}},
%  \textbf{Twelfth Author\textsuperscript{1}},
%\\
%  \textbf{Thirteenth Author\textsuperscript{3}},
%  \textbf{Fourteenth F. Author\textsuperscript{2,4}},
%  \textbf{Fifteenth Author\textsuperscript{1}},
%  \textbf{Sixteenth Author\textsuperscript{1}},
%\\
%  \textbf{Seventeenth S. Author\textsuperscript{4,5}},
%  \textbf{Eighteenth Author\textsuperscript{3,4}},
%  \textbf{Nineteenth N. Author\textsuperscript{2,5}},
%  \textbf{Twentieth Author\textsuperscript{1}}
%\\
%\\
%  \textsuperscript{1}Affiliation 1,
%  \textsuperscript{2}Affiliation 2,
%  \textsuperscript{3}Affiliation 3,
%  \textsuperscript{4}Affiliation 4,
%  \textsuperscript{5}Affiliation 5
%\\
%  \small{
%    \textbf{Correspondence:} \href{mailto:email@domain}{email@domain}
%  }
%}

\begin{document}
\maketitle
\begin{abstract}

% Graph foundation models aim xxx,

% However, they primarily focus on TAGs while can not handle MMG,

% To this end, we propose UniGraph2, by xxx

% Concretely, 

% Experiments,

Existing foundation models, such as CLIP, aim to learn a unified embedding space for multimodal data, enabling a wide range of downstream web-based applications like search, recommendation, and content classification. However, these models often overlook the inherent graph structures in multimodal datasets, where entities and their relationships are crucial. %For example, in social networks, users are connected through friendships, follows, or interactions, and share content in various modalities like text and images. 
Multimodal graphs (MMGs) represent such graphs where each node is associated with features from different modalities, while the edges capture the relationships between these entities.
% such data by combining diverse modalities with graph structures that capture the relationships between entities.
% in e-commerce platforms, products are linked based on co-purchase patterns, user reviews, and shared attributes, encompassing multiple data types.
% \todo{introduce MMG, and real applications}However, none of these models consider the graph structure inherent in many multimodal datasets, where entities and their relationships are critical. 
On the other hand, existing graph foundation models primarily focus on text-attributed graphs (TAGs) and are not designed to handle the complexities of MMGs. To address these limitations, we propose \model\footnote[1]{The code is available at \url{https://github.com/yf-he/UniGraph2}}, a novel cross-domain graph foundation model that enables general representation learning on MMGs, providing a unified embedding space. \model employs modality-specific encoders alongside a graph neural network (GNN) to learn a unified low-dimensional embedding space that captures both the multimodal information and the underlying graph structure. We propose a new cross-domain multi-graph pre-training algorithm at scale to ensure effective transfer learning across diverse graph domains and modalities. Additionally, we adopt a Mixture of Experts (MoE) component to align features from different domains and modalities, ensuring coherent and robust embeddings that unify the information across modalities. Extensive experiments on a variety of multimodal graph tasks demonstrate that UniGraph2 significantly outperforms state-of-the-art models in tasks such as representation learning, transfer learning, and multimodal generative tasks, offering a scalable and flexible solution for learning on MMGs.



% Graph foundation models (GFMs), which aim to learn knowledge from graphs in different domains and transfer it to various tasks, have become a promising direction for web mining. However, existing GFMs, e.g., UniGraph, primarily focus on text-attributed graphs (TAGs) and fail to handle the data in various modalities, e.g., text, image, and video on the Web, limiting the applicability. To this end, we propose a new GFM method, termed \model, to achieve the general ability on multi-modal graphs (MMGs), by refactoring UniGraph from three key aspects, including architecture, pre-training, and alignment. Concretely, for the architecture, we first design the GNN-based modulation-specific encoders to learn a unified representation that captures both modality information and structure information. Then, for the pre-training, we propose a large-scale cross-domain pre-training algorithm to learn knowledge across diverse graph domains and modalities. In addition, for the alignment, we introduce a Mixture-of-Expert (MoE) component to align features from different domains, ensuring coherent and robust embeddings that unify the information across modalities. By these designs, \model is powered by the strong multimodal representation learning ability on graph data, improving the applicability and performance in web mining. Extensive experiments on xx multimodal tasks demonstrate the superiority of our proposed method. Remarkablely, xxx

% Extensive experiments on a variety of multimodal graph tasks demonstrate that UniGraph2 significantly outperforms state-of-the-art models in tasks such as node classification, retrieval, and transfer learning, offering a scalable and flexible solution for learning on multimodal graphs.


\end{abstract}



\section{Introduction}

\begin{figure}[h]
    \centering
    \begin{overpic}[trim=0cm 0cm 0cm 0cm,clip,angle=0,origin=c,width=.4\linewidth]{images/teaser_absolute.png}
        %  trim={<left> <lower> <right> <upper>}
        %  \put(horiz, vert)
        %  \put(horiz, vert){\rotatebox{90}{Text}}
        %
        \put(107, 32){$\mathbf{\to}$}
    \end{overpic}\hspace{1cm}
    \begin{overpic}[trim=0cm 0cm 0cm 0cm,clip,angle=0,origin=c,width=.4\linewidth]{images/teaser_translated_yellow.png}
        %  trim={<left> <lower> <right> <upper>}
        %  \put(horiz, vert)
        %  \put(horiz, vert){\rotatebox{90}{Text}}
        %
    \end{overpic}
    \caption{Using translation methods, a controller trained on an environment with a given visual variation \textit{(left)} can be reused without any training or fine-tuning on a different environment (\textit{right}) with comparable performance. In red we see the trajectory of a car driven by the same controller when connected to two different encoders, one for each visual variation.
    }
    \label{fig:teaser}
\end{figure}

Deep Reinforcement Learning (RL) has enabled agents to achieve remarkable performance in complex decision-making tasks, from robotic manipulation to high-dimensional games (Mnih et al., 2015; Silver et al., 2017). 
Although recent RL techniques achieved strong improvements over sample efficiency \citep{yarats2021drqv2, kostrikov2020image}, training new agents remains a costly process, both in computational and temporal terms.
Despite these advances, most methods still require at least partial retraining when dealing with domain shifts such as visual appearance, reward functions, or action spaces \citep{pmlr-v97-cobbe19a, zhang2020learning}. These domain changes typically require expensive retraining, which can be prohibitive for real-world settings that require millions of interactions.

A variety of approaches have been proposed to address these shifting conditions. Domain randomization \citep{tobin2017domain, sadeghi2016cad2rl} trains agents across diverse visual styles or physics settings, promoting invariant features but demanding broader coverage of possible variations. Multi-task RL \citep{parisotto2015actor, teh2017distral} attempts to learn shared representations across multiple tasks.

In the supervised setting, recent representation learning techniques \citep{Moschella2022-yf,maiorca2023latent, norelli2022b, cannistraci2023bricks}, show that it is possible to zero-shot recombine encoders and decoders to perform new tasks across different modalities (images, text..) and tasks (classification, reconstruction) and even architectures.
In RL, methods adopting the relative representation framework \citep{Moschella2022-yf} have shown promising results in adapting encoders to different controllers with zero or few-shots adaptation, for robotic control from proprioceptive states \citep{jian2021adversarial} or for playing games in the Gymnasium suite \citep{towers2024gymnasium} from pixels \citep{ricciardi2025r3lrelativerepresentationsreinforcement}.
These methods, however, still require training models to use the new relative representations.

By contrast, \cite{maiorca2023latent} suggest that modules from independently trained neural networks can be connected via a simple linear or affine transformation, with no training constraint or fine-tuning required, if such transformations can be reliably estimated from a small set of “anchor” samples, pairs of states or observations deemed semantically equivalent.

Our main contribution is the implementation of a RL method based on semantic alignment to map between latent spaces of different neural models, so that their encoders and controllers can be stitched with the goal of creating new agents that can act on visual-task combinations never seen together in training. This includes the use of the transformations to map modules from different networks, and the collection of anchor samples used to estimate these transformations. We call our method Semantic Alignment for Policy Stitching (\textbf{SAPS}).
We perform analyses and empirical tests on the CarRacing and LunarLander environments to show the performance of new agents created via zero-shot stitching of encoders and controllers trained on different visual-task variations, demonstrating significant gains compared to existing zero-shot methods.
%\clearpage
\section{Related Work} \label{related}




% \subsection{Benchmarks in Coding Scenarios}
% \begin{enumerate}
%     \item Code Generation
%     \item Bug Fixing
% \end{enumerate}

% \subsection{Large Language Model Agents}

% At the heart of the LLM Agent is an Agent Core, which coordinates the core \textit{logic} and \textit{behavioral} characteristics of the agent. In addition, the Agent includes the following key components:

% \begin{itemize}
%     \item Memory Module: It consists of both short-term and long-term memory components that record the agent's internal logs and interactions with the user.
%     \item Tools: These are the tools that the agent can use to perform tasks, usually specific third-party APIs.
%     \item Planning Module: This is used for solving complex problems, such as decomposing tasks and problems, reflexivity or critique.
% \end{itemize}

% \subsection{Multi Agent Collaboration Framework}

% MetaGPT \url{https://arxiv.org/abs/2308.00352}


\parabf{Coding \llm{s}.}
Large Language Models (\llm{s}) have become the go-to solution for a wide array of coding tasks due to their exceptional performance in both code generation and comprehension~\cite{codex}. These models have been successfully applied to various software engineering activities, including program synthesis~\cite{patton2024programming, codex, li2022competition, iyer2018mapping}, code translation~\cite{pan2024lost, roziere2020unsupervised, roziere2021leveraging}, program repair~\cite{xia2023repairstudy, chatrepair, monperrus2018living, bouzenia2024repairagent}, and test generation~\cite{titanfuzz, fuzz4all, deng2023fuzzgpt, lemieux2023codamosa, kang2023testing}. Beyond general-purpose \llm{s}, specialized models have been developed by further training on extensive datasets of open-source code snippets. Notable examples of these code-specific \llm{s} include \codex~\cite{codex}, \codellama~\cite{codellama}, StarCoder~\cite{starcoder,starcodertwo}, and \deepseek~\cite{deepseek}. Additionally, instruction-following code models have emerged, refined through instruction-tuning techniques. These include models such as \codellamainstruct~\cite{codellama}, \deepseekinstruct~\cite{deepseek}, \wizardcoder~\cite{wizardcoder}, \magicoder~\cite{magicoder}, and OpenCodeInterpreter~\cite{zheng2024opencodeinterpreter}.

\parabf{Benchmarking \llm-based coding tasks.}
To assess the capabilities of \llm{s} in coding, a variety of benchmarks have been proposed. Among the most widely utilized are \humaneval~\cite{codex} and \mbpp~\cite{austin2021program}, which are handcrafted benchmarks for code generation that include test cases to validate the correctness of \llm outputs. Other benchmarks have been developed to offer more rigorous tests~\cite{evalplus}, cover additional programming languages~\cite{zheng2023codegeex,cassano2023multipl}, and address different programming domains~\cite{livecodebench, hendrycksapps2021, codecontest, ds1000, arcade}.

More recently, research has shifted towards evaluating \llm{s} on real-world software engineering challenges by operating on entire code repositories rather than isolated coding problems~\cite{swebench, zhang2023repocoder, liu2023repobench}. A notable benchmark in this area is \swebench~\cite{swebench}, which includes tasks requiring repository modifications to resolve actual GitHub issues. The authors of \swebench have also released a more focused subset, \swebenchlite~\cite{swebenchlite}, which contains 300 problems centered on bug fixing that only involves single-file modifications in the ground truth patches. ML-Bench \cite{liu2023mlbench} is a benchmark for evaluating large language models and agents for Machine Learning tasks on reporitory-level code. It involves 18 repositories and focuses on code generation and interactions with Jupyter Notebooks.

\parabf{Repository-level coding.}
The rise of agent-based frameworks~\cite{xi2023rise} has spurred the development of agent-based approaches to software engineering tasks. Devin~\cite{devinwebpage} (and its open-source counterpart OpenDevin~\cite{opendevin}) is among the first comprehensive \llm agent-based frameworks. Devin employs agents to first perform task planning based on user requirements, then allows them to use tools like file editors, terminals, and web search engines to iteratively execute the tasks. \sweagent~\cite{sweagent} introduces a custom agent-computer interface (ACI), enabling the \llm agent to interact with the repository environment through actions like reading and editing files or running bash commands. Another agent-based approach, \autocoderover~\cite{autocoderover}, equips the \llm agent with specific APIs (e.g., searching for methods within certain classes) to effectively identify the necessary modifications for issue resolution. Beside these examples, a variety of other agent-based approaches have been developed in both open-source~\cite{aidar} and commercial products~\cite{bouzenia2024repairagent, coder, repounderstander, lingma, factorydroid, ibmagent, opencsgstarship, marscode, amazonqdeveloper}.

% Unlike these agent-based methods, \tech offers a straightforward and cost-efficient solution for addressing real-world software engineering challenges. Our work is the first to demonstrate that an \emph{agentless} approach can achieve comparable performance without the need for complex tools or modeling intricate environment behavior and feedback.

Unlike existing benchmarks and agent-based frameworks, which focus on the code generation/completion tasks, our proposed \model and \agent focus on the code deployment task, which is under-studied in the field.
%\clearpage

% 加一节,怎么区分确定性和不确定性

\section{Methodology}


To achieve effective probabilistic predictions, we propose CoST that simultaneously leverages the advantages of both deterministic and probabilistic models. Our approach involves two stages. In the first stage, the deterministic model is pretrained to predict the conditional mean that captures the primary patterns. In the second stage, the parameters of the deterministic model are frozen, and the scale-aware diffusion model, constrained by a customized fluctuation scale, is jointly trained to model residual distributions that reflect random fluctuations.   
Figure~\ref{fig:model} illustrates an overview of our model.


\subsection{Mean-Residual Decomposition}

For urban spatiotemporal probabilistic prediction, current approaches typically employ a single probabilistic model to capture the full distribution of data, incorporating both the primary spatiotemporal patterns and the random fluctuations. However, it is challenging to model both of these distributions. Inspired by~\cite{mardani2023residual} and the Reynolds decomposition in fluid dynamics~\cite{pope2001turbulent}, we propose to decompose the target data \(\mathbf{x}^{ta}\) as follows:
\begin{equation}
 \mathbf{x}^{ta} = \underbrace{\mathbb{E}[\mathbf{x}^{ta}|\mathbf{x}^{co}]}_{\substack{:=\boldsymbol{\mu}(Deterministic)}} + \underbrace{(\mathbf{x}^{ta}-\mathbb{E}[\mathbf{x}^{ta}|\mathbf{x}^{co}])}_{\substack{:=\mathbf{r}(Probabilistic)}},
\end{equation}
where \(\boldsymbol{\mu}\) is the conditional mean representing the primary patterns, and \(\mathbf{r}\) is the residual representing the random variations. Our core idea is that if a deterministic model can accurately predict the conditional mean, that is, \(\boldsymbol{\mu}\approx\mathbb{E}_{\theta}[\mathbf{x}^{ta}|\mathbf{x}]\), then the probabilistic model only needs to focus on learning the simpler residual distribution, thus combining the strengths of both models to enhance the probabilistic prediction capability.









\subsection{Mean Prediction via Deterministic Model}

We require a deterministic model that can accurately predict the conditional mean to align with our research hypothesis, and also maintain high predictive efficiency to avoid additional increases in training and inference time. Therefore, we select the MLP-based STID model as our mean prediction module.
In the first stage of training, we pretrain the model for 50 epochs to effectively capture the primary spatiotemporal patterns. Specifically, we input historical conditional data \(\mathbf{x}^{co}\) into the STID model to obtain the conditional mean output \(\mathbb{E}_{\theta}[\mathbf{x}^{ta}|\mathbf{x}^{co}]\).

The STID model is pretrained by optimizing the following loss function:

\begin{equation}
\label{eq:loss2}
   \mathcal{L}_{2}  = \left\| \mathbb{E}_{\theta}[\mathbf{x}^{ta}|\mathbf{x}^{co}] - \mathbf{x}^{ta} \right\|_2^2 .
\end{equation}

\subsection{Residual Learning via Diffusion Model}
Diffusion models have achieved significant success in probabilistic modeling. In this work, we employ a diffusion model for probabilistic prediction, training it to learn the residual distribution:
\begin{equation}
\label{eq:one-setp-forward}
    \mathbf{r}_{ta}=\mathbf{x}^{ta}-\mathbb{E}_{\theta}[\mathbf{x}^{ta}|\mathbf{x}^{co}].
\end{equation}
Consequently, the target data \(\mathbf{x}^{ta}\) for diffusion models in Eqs.~\eqref{eq:one-setp-forward}, \eqref{eq:inference}, and \eqref{eq:loss1} is replaced by \(\mathbf{r}_{ta}\).
The residual distribution of urban spatiotemporal data is not independently and identically distributed (i.i.d.) nor does it follow a fixed distribution, such as \(\mathcal{N}(0, \mathbf{\sigma})\). Instead, it often exhibits complex spatiotemporal dependence and heterogeneity. So we consider both temporal residual learning and spatial residual learning. 




\subsubsection*{\textbf{Temporal Residual Learning.}} 
For urban spatiotemporal data, the residual distribution varies at different time points. For example, fluctuations are lower at night and higher during the day, with uncertainty varying between weekends and weekdays. To model this, we incorporate the timestamp information as the condition for the denoising process. Meanwhile, the residual distribution can also be affected by sudden weather changes or public events. To capture these real-time changes, we concatenate the context data $\mathbf{x}^{co}_0$ with noised target residual $\mathbf{r}^{ta}_n$ as the input. The noise is not added to $\mathbf{x}^{co}_0$ and $\mathbf{r}^{ta}_n$ during the diffusion training and inference.




\subsubsection*{\textbf{Spatial Residual Learning.}}
In areas with frequent traffic accidents, fluctuations tend to be more pronounced and may induce anomalous variations in adjacent regions, thus affecting their distributions.
For spatial dependence modeling, the model learns a spatial embedding for each location, following the STID approach. Additionally, we propose a scale-aware diffusion process to further distinguish the heterogeneity for different regions. In this section, we detail the calculation of \(\mathbf{Q}\) and how it is integrated into the scale-aware diffusion process.

\noindent\textbf{(i) Customized Fluctuation Scale.} Specifically, we apply the Fast Fourier Transform (FFT) to spatiotemporal sequences in the training set to quantify fluctuation levels in different regions and use the custom scale \(\mathbf{Q}\) as input to account for spatial differences in residual. Specifically, we first employ FFT to extract the fluctuation components for each region within the training set. The detailed steps are as follows:









\begin{equation}
    \begin{aligned}
    & \mathbf{A}_{\mathrm{k}} = \left| \text{FFT}(\mathbf{x})_\mathrm{k} \right|, \quad \mathbf{{\phi}}_{\mathrm{k}} = \mathbf{\phi} \left( \text{FFT}(\mathbf{x})_\mathrm{k} \right), \\
    & \mathbf{A}_{\text{max}}=\max_{\mathrm{k}\in\left\{1,\cdots,\left\lfloor\frac{\mathbf{L}}{2}\right\rfloor + 1\right\}}\mathbf{A}_{\mathrm{k}}, \\
    & \mathcal{K} = \left\{ \mathrm{k} \in \left\{ 1, \cdots, \left\lfloor \frac{{L}}{2} \right\rfloor + 1 \right\} : \mathbf{A}_{\mathrm{k}} < 0.1 \times \mathbf{A}_{\text{max}} \right\}, \\
    & \mathbf{x}_{\mathbf{r}}[i] = \sum_{\mathrm{k} \in \mathcal{K}} \mathbf{A}_{\mathrm{k}} \Big[ \cos \left( 2\pi \mathbf{f}_{\mathrm{k}} i + \mathbf{\phi}_{\mathrm{k}} \right) \\
    & \qquad \qquad + \cos \left( 2\pi \bar{\mathbf{f}}_{\mathrm{k}} i + \bar{\mathbf{\phi}}_{\mathrm{k}} \right) \Big],
    \end{aligned}
\end{equation}
where \(\mathbf{A}_{\mathrm{k}},\mathbf{\phi}_{\mathrm{k}}\) reprent the amplitude and phase of the $\mathrm{k}-$th frequency component. $L$ is the temporal length of the training set. \(\mathbf{A}_{\text{max}}\) is the maximum amplitude among the components, obtained using the $\max$ operator. $\mathcal{K}$ represents the set of indices for the selected residual components. \(\mathbf{f}_{\mathrm{k}}\) is the frequency of the \(\mathrm{k}\)-th component. $\bar{\mathbf{f}}_{\mathrm{k}}, \bar{\mathbf{\phi}}_{\mathrm{k}}$ represent the conjugate components. \(\mathbf{x}_{\mathbf{r}}\) ref to the extracted residual component of the training set. We then compute the variance $\sigma^2_k$ of the residual sequence for each location $k$ and expand it to match the shape as 
\(\mathbf{r}^{ta}_0 \in \mathbb{R}^{B \times K \times P}\) , where $B$ represents the batch size. And we can get the variance tensor \(\mathcal{M}\): 
\begin{equation}
\begin{aligned}
    &\mathcal{M}_{b,k,p}=\sigma_{k}^2,\\
&\forall b\in\{1,\cdots,B\}, \forall k\in\{1,\cdots,K\}, \forall p\in\{1,\cdots,P\}.
\end{aligned}
\end{equation}
The residual fluctuations are bidirectional, encompassing both positive and negative variations, so we generate a random sign tensor \(\mathbf{S}\in\mathbb{R}^{B\times K\times P}\) for \(\mathcal{M}\), where each element \(S_{b,k,p}\) of \(\mathbf{S}\) is sampled from a Bernoulli distribution with \(p = 0.5\). 
%That is, \(r_{b,k,p}\) takes the value $1$ with probability $0.5$ and $-1$ with probability $0.5$. 
The customized fluctuation scale \(\mathbf{Q}\) is then defined as:
\begin{equation}
\begin{aligned}
&\mathbf{Q}_{b,k,p}=S_{b,k,p}\times\mathcal{M}_{b,k,p},\\
&\forall b\in\{1,\cdots,B\}, \forall k\in\{1,\cdots,K\}, \forall p\in\{1,\cdots,P\}.
\end{aligned}
\end{equation}
Then \(\mathbf{Q}\) is used as the input of the denoising network. 





\noindent\textbf{(ii) Scale-aware Diffusion Process.}

The vanilla diffusion models typically model all regions as the same \(\mathcal{N}(0, I)\) distribution at the end of the diffusion process, failing to distinguish the differences among regions. To further model the differences of residual distribution across various regions, we adopt the technique proposed by~\cite{han2022card} to make the residual learning region-specific conditioned on \({\mathbf{Q}}\). Specially, we have calculated the customized fluctuation scale \({\mathbf{Q}}\), and We redefined the noise distribution at the endpoint of the diffusion process as follows:
\begin{equation}
    p(\mathbf{r}^{ta}_N)=\mathcal{N}({\mathbf{Q}},I),
\end{equation}
Accordingly, the Eq~\ref{eq:new one-step} in the forward process is rewritten as:
\begin{equation}
\label{eq:new one-step}
    \mathbf{r}_n^{ta} = \sqrt{\bar{\alpha}_n} \mathbf{r}_0^{ta}+(1-\sqrt{\bar{\alpha}_n})\mathbf{Q} + \sqrt{1 - \bar{\alpha}_n} \mathbf{\epsilon}, \quad \mathbf{\epsilon} \sim \mathcal{N}(0, I).
\end{equation}
And in the denoising process, we sample \(\mathbf{r}_N^{ta}\) from $\mathcal{N}({\mathbf{Q}},I)$, and denoise it use Eq~(\ref{eq:inference}), the computation of \(\mu_{\theta}(\mathbf{r}_n^{ta}, n| \mathbf{x}_0^{co})\) in Eq~\ref{eq:inference} is modified as:
\begin{equation}
\label{eq: mu}
    \mu_{\theta}(\mathbf{r}_n^{ta}, n| \mathbf{x}_0^{co})=\frac{1}{\sqrt{\bar{\alpha}_n}} \left( \mathbf{r}_n^{ta} - \frac{\beta_n}{\sqrt{1 - \bar{\alpha}_n}} \mathbf{\epsilon}_{\theta}(\mathbf{r}_n^{ta}, n| \mathbf{x}_0^{co}) \right)+(1-\frac{1}{\sqrt{\bar{\alpha}_n}})\mathbf{Q}.
\end{equation}
This approach enables the diffusion process to be governed by the \(\mathbf{Q}\) values at each region, leading to more effective utilization of the customized scale conditions.


\subsection{Training and Inference}
\begin{algorithm}
\caption{\methodname{} Training}
\KwIn{Coarse-to-fine Autoencoder $\text{Enc}$, $\text{Dec}$}
\KwOut{}
\For{$i \gets 1$ \textbf{to} $n-1$}{
    \For{$j \gets 1$ \textbf{to} $n-i$}{
        \If{$L[j] > L[j+1]$}{
            Swap $L[j]$ and $L[j+1]$
        }
    }
}
\Return $L$
\end{algorithm}
\begin{algorithm}[!t]
\caption{Inference}
\label{al: sampling}
\begin{algorithmic}[1]
    \State \textbf{Input:} Context data $\mathbf{x}_0^{co}$, customized fluctuation scale $\mathbf{Q}$, trained diffusion model $\epsilon_{\theta}$, trained deterministic model $\mathbb{E}_{\theta}$
    \State \textbf{Output:} Target data $\mathbf{x}_0^{ta}$
    \State Estimate the conditional mean \(\mathbb{E}_{\theta}[\mathbf{x}^{ta}_0|\mathbf{x}^{co}_0]\)
    \State Sample $\mathbf{r}_N^{ta}$ from $\epsilon \sim \mathcal{N}(\mathbf{S},I)$
    \For{$n = N$ to $1$} 
        \State Estimate the noise $\mathbf{\epsilon}_{\theta}(\mathbf{r}_n^{ta}, n| \mathbf{x}_0^{co})$
        \State Calculate the $\mu_{\theta}(\mathbf{r}_n^{ta}, n| \mathbf{x}_0^{co})$ using Eq.~(\ref{eq: mu})
        \State Sample $\mathbf{r}_{n-1}^{ta}$ using Eq.~(\ref{eq:inference})
    \EndFor
    \State \textbf{Return:} $\mathbf{x}_0^{ta}=\mathbb{E}_{\theta}[\mathbf{x}^{ta}_0|\mathbf{x}^{co}_0]+\mathbf{r}_0^{ta}$
\end{algorithmic}

\end{algorithm}

\subsubsection*{\textbf{Training}}
Our training process is a two-stage procedure. We first pretrain the deterministic model STID to enable it to predict the conditional mean. Subsequently, we train the diffusion mode to learn the distribution of residuals, where the residuals are calculated as the difference between the true values and the conditional mean predicted by the pretrained STID model with frozen parameters. The detailed training procedure is presented in Algorithm~\ref{al: train}.
\subsubsection*{\textbf{Inference}}
The inference process of the model consists of two paths: one utilizes the pretrained STID model to predict the conditional mean, while the other uses the diffusion model to predict the residuals. The final sample is obtained by summing the results of both paths. This process is detailed in Algorithm~\ref{al: sampling}.
%\clearpage
\section{Experiments}
\subsection{Experiment Setup} 
\paragraph{Models.}

% \begin{table*}[htbp]
% \newcolumntype{g}{>{\columncolor{green!10}}c}
% \newcolumntype{b}{>{\columncolor{blue!10}}c}
% \renewcommand{\arraystretch}{1.22} % Adjust row spacing
% \small
% \resizebox{0.95\textwidth}{!}{
% \begin{tabular}{llcccc}

% \toprule
% UltraBench Split & Model & Overall Score & Soft Score & Hard Score & BLEU  \\ \midrule

% FineWeb  & Base Model  & 45.11 & 67.71 & 38.34 & 5.8 \\
% % \ \ Easy (66 samples)                      &         & 59.68 & 76.01 & 0.23 & 48.23 & 0.053     \\
% % \ \ Medium (76 samples)                    &         & 44.64 & 66.45 & 0.11 & 39.78 & 0.060     \\
% % \ \ Hard (58 samples)                      &         & 29.14 & 59.95 & 0.00 & 25.20 & 0.049     \\ \midrule

% & SFT Model  & 58.63 & 83.18  & 51.39 & 9.6 \\
% % \ \ Easy                      &         & 69.11 & 88.56 & 0.53 & 56.84 & 0.074    \\
% % \ \ Medium                    &         & 61.45 & 85.92 & 0.39 & 56.06 & 0.098     \\
% % \ \ Hard                       &         & 43.00 & 73.45 & 0.17 & 39.07 & 0.089     \\ \midrule

% & Llama-3.2-3B-DPO &  65.87 & 82.25 & 60.88 & 9.1  \\
% % \ \ Easy                       &         & 74.66 & 89.60 & 0.61 & 65.26 & 0.063    \\
% % \ \ Medium                   &         & 68.38 & 81.28 & 0.25 & 65.57 & 0.092  \\
% % \ \ Hard                       &         & 52.60 & 75.14 & 0.16 & 49.73 & 0.096   \\
% \midrule
% Global    & Llama-3.2-3B-Instruct\textsubscript{BASE}   & 36.85          & 43.95              & 35.23    & -         \\
% & Llama-3.2-3B-Instruct\textsubscript{SFT}    & 42.27          & 54.57               & 38.50     & -         \\
% & Llama-3.2-3B-Instruct\textsubscript{DPO}     & 63.84          & 57.84               & 64.86   & -           \\ 
% \bottomrule

% \end{tabular}
% }
% \caption{Performance scores for Llama-3.2-3B-Instruct models under different evaluation conditions.}
% \label{tab:ultrabench}
% \end{table*}

\begin{table*}[htbp]
\newcolumntype{g}{>{\columncolor{green!10}}c}
\newcolumntype{b}{>{\columncolor{blue!10}}c}
\renewcommand{\arraystretch}{1.22} % Adjust row spacing
\small
\resizebox{\textwidth}{!}
{
\begin{tabular}{llccccccc}

\toprule
&  \multirow{2}*{Model} & \multicolumn{4}{c}{\textbf{FineWeb Split}} & \multicolumn{3}{c}{\textbf{Multi-source Split}} \\
\cmidrule(l){3-6} \cmidrule(l){7-9} 
& & Overall Score & Soft Score & Hard Score & BERTScore F1    & Overall Score & Soft Score & Hard Score   \\ 
 
 \midrule

 \multirow{3}*{\rotatebox{90}{Main}}& Base Model  & 50.30 & 67.08 & 33.51 & 59.92  & 37.45       & 36.10            & 38.79            \\
% \hdashline

& UltraGen (AR)  & 56.05 & 81.44  & 30.65 & 62.00    & 50.15         & 62.41               & 37.89            \\
& UltraGen (AR+GPO) &  59.61 & 84.33 & 34.89 & 61.22    &  57.23       & 69.01               & 45.44            \\ 
\midrule
% \hdashline
\multirow{4}*{\rotatebox{90}{Ablation}} & AR (Few Constraints) & 48.25 & 74.09 & 22.41 & 60.10    & 38.38         &  46.00        & 30.76             \\
& GPO & 55.57 & 74.50 & 36.63 & 60.59 & 42.44 & 51.00 & 33.86 \\
& AR+GPO (Random Sampling) &  59.77 & 85.42 & 34.11 & 60.56 & 55.24         & 68.01            & 42.47            \\ 
& AR+GPO (High Similarity) &  59.44 & 83.22 & 35.65 & 60.85 & 55.45        & 66.05               & 44.85             \\ 
& AR+GPO (Low Correlation) &  58.91 & 83.59 & 34.23 & 60.00 & 54.47         & 65.22               & 43.71             \\ 

\bottomrule

\end{tabular}
}
\caption{Performance scores for Llama-3.2-3B-Instruct models on the validation set under different evaluation conditions across FineWeb and Global splits.}
\label{tab:ultrabench}
\end{table*}

% \begin{table}[h!]
\centering
\caption{Performance Comparison for Different Levels}
\resizebox{0.5\textwidth}{!}{
\begin{tabular}{@{}lccc@{}}
\toprule
\textbf{Category} & \textbf{Score} & \textbf{Soft Score} & \textbf{Hard Score} \\ \midrule
\multicolumn{4}{c}{\textbf{Overall Scores}} \\
Llama-3.2-3B-Instruct\textsubscript{BASE}       & 36.85          & 43.95              & 35.23            \\
Llama-3.2-3B-Instruct\textsubscript{SFT}         & 42.27          & 54.57               & 38.50              \\
Llama-3.2-3B-Instruct\textsubscript{DPO}         & 63.84          & 57.84               & 64.86             \\ \midrule
% \multicolumn{4}{c}{\textbf{Easy (47 Samples)}} \\
% Llama-3.2-3B-Instruct\textsubscript{BASE}         & 49.89          & 58.16               & 47.63              \\
% Llama-3.2-3B-Instruct\textsubscript{SFT}         & 51.65          & 66.38               & 42.82              \\
% Llama-3.2-3B-Instruct\textsubscript{DPO}         & 51.26          & 62.31               & 44.05              \\ \midrule
% \multicolumn{4}{c}{\textbf{Medium (55 Samples)}} \\
% Llama-3.2-3B-Instruct\textsubscript{BASE}         & 45.53         & 53.48               & 39.60              \\
% Llama-3.2-3B-Instruct\textsubscript{SFT}         & 43.92          & 53.53               & 32.14              \\
% Llama-3.2-3B-Instruct\textsubscript{DPO}         & 37.47          & 57.14               & 35.16              \\ \midrule
% \multicolumn{4}{c}{\textbf{Hard (98 Samples)}} \\
% Llama-3.2-3B-Instruct\textsubscript{BASE}         & 16.65          & 30.72               & 15.73              \\
% Llama-3.2-3B-Instruct\textsubscript{SFT}         & 25.30          & 42.14               & 24.40              \\
% Llama-3.2-3B-Instruct\textsubscript{DPO}         & 24.31          & 37.72               & 23.59              \\ \bottomrule
\end{tabular}
}
\label{table:global_performance_comparison}
\end{table}
Our experiments evaluate the EFCG task using one mainstream instruction-tuned base model: Llama-3.2-3B-Instruct ~\cite{dubey2024llama}, chosen for its demonstrated proficiency in instruction-following tasks within the 3B parameter range. To systematically assess the impact of our methodology, we compare three training paradigms: (1) \textbf{BASE}, which directly employs the unmodified base models to establish a performance baseline; (2) \textbf{AR}, where models undergo the auto-reconstruction stage on our meticulously constructed FineWeb dataset (§3.2), enriched with fine-grained attributes to enhance multi-constraint adherence; and (3) \textbf{AR+GPO}, a hybrid optimization approach combining direct preference optimization with global embedding space adaption.

\subsection{Evaluation Results on UltraBench}

Our experimental findings, summarized in Table \ref{tab:ultrabench}, demonstrate the substantial advancements achieved by applying the UltraGen paradigm to EFCG. The evaluation leverages the validation set of FineWeb and Global splits to assess model performance under both local and global constraints.

The application of AR yielded significant improvements over the base model. On the FineWeb split, the AR model attained an overall score of 56.05, representing a relative improvement of 11.4\%. The soft score rose to 81.44, indicating enhanced adherence to semantic and stylistic attributes, while the hard score increased to 30.65, reflecting better performance on programmatically verifiable constraints. On the Global split, the AR model demonstrated its ability to generalize, achieving an overall score of 50.15.

Further optimization through GPO demonstrated remarkable performance on the Global split, where the model achieved an overall score of 57.23 and an impressive hard score of 45.44. This highlights the model's robust generalization and optimization capabilities when dealing with diverse and challenging global constraints. Notably, despite being trained on the Global split, the AR+GPO model exhibited strong performance on the FineWeb split as well, achieving an overall score of 59.61, a soft score of 84.33, and a hard score of 34.89. This result underscores the model's ability to transfer its learned capabilities from the broader and more diverse Global split to the more localized FineWeb split.

\paragraph{Ablation}
To evaluate the contribution of key components in our UltraGen framework, we conducted ablation studies by systematically modifying the training process. We tested the impact of reducing the number of attributes during AR, removing the AR stage, replacing curated attributes with random sampling, and eliminating the high-correlation or low-redundancy selection steps. The results demonstrate that both AR and GPO stages are crucial for achieving strong performance, as reducing constraints, removing correlation modeling, or neglecting redundancy minimization leads to performance degradation.
% \paragraph{Ablation}
% To evaluate the contribution of key components in our UltraGen framework, we conducted several ablation studies by systematically modifying the training process. The following ablations were performed:
% \begin{enumerate}
%     \item \textbf{SFT with limited attributes}: To examine the impact of attribute numbers during the supervised fine-tuning stage, we trained an SFT model using a reduced set of attributes (fewer than 10 per sample).
%     \item \textbf{DPO only}: We directly train the DPO on the global split without SFT stage.
%     \item \textbf{SFT + DPO random sampling}: In this ablation, we replaced the curated high correlation and low redundancy attribute combinations with random sampling during the RL stage. 
%     \item \textbf{SFT + DPO w/o high correlation}: This experiment removed the attribute correlation modeling step, where attributes with strong relationships were prioritized.
%     \item \textbf{SFT + DPO w/o low redundancy}: In this setup, we did not enforce diversity in attribute sets by minimizing semantic redundancy.
% \end{enumerate}
% The ablation study shows that SFT with fewer constraints significantly underperforms the standard SFT. And DPO variants with fewer constraints, random sampling, or reduced correlation emphasize the importance of optimized attribute selection in the global space.
\subsection{Data Synthesis Improvement}

\begin{table}[htbp]
\centering
\small
\resizebox{0.48\textwidth}{!}{
\begin{tabular}{lccc}
\toprule
\textbf{Dataset (Domain)} & \textbf{Base} &  \textbf{AR} & \textbf{AR+GPO} \\ \midrule
Emotion (Tweet Emotion) & 28.25 & \textbf{42.30}  & 38.65 \\
Hillary (Tweet Stance)  & 55.93  & 45.76 & \textbf{58.31} \\
AG-News (News Topic) & 80.03 & 79.96 &\textbf{83.28} \\
TREC (Question Type) & 38.00  & 51.20  & \textbf{51.40} \\ 
\midrule
Average   & 50.55 & 54.81 & \textbf{57.91} \\
\bottomrule
\end{tabular}
}
\caption{Performance comparison for data synthesis.}
\vspace{-1em}
\label{tab:data_synthesis}
\end{table}

To demonstrate the improvement in the usage of texts synthesized by UltraGen, we utilize several diverse well-established text classification benchmarks to test the data synthesis capability, such as sentiment analysis \textbf{(1) Emotion} ~\cite{saravia-etal-2018-carer}, attitude classification towards a particular public figure \textbf{(2) Hillary} ~\cite{barbieri2020tweeteval}, topic classification \textbf{(3) AG News} ~\cite{Zhang2015CharacterlevelCN}, question type classification \textbf{(4) TREC} ~\cite{li-roth-2002-learning}.

For each dataset, we analyze the unique properties and paraphrase these properties as hard and soft attributes. Then using a uniform prompt tailored for each dataset, we generate 2,000 synthetic samples per dataset. These generated samples are then used to train a classifier, which is subsequently evaluated on the original test set of the dataset. This procedure allows for a fair comparison of model performance on synthetic data. 

The results, summarized in Table \ref{tab:data_synthesis}, demonstrate the superior generalization ability of the AR+GPO model trained on the Global split. Notably, the AR+GPO model achieved the highest average score of 57.91 across the benchmarks, significantly outperforming both the base model and the AR models. While the AR model’s performance stagnated (45.76, lower than the original one) on the Hillary benchmark, reflecting a focus on localized attributes, the AR+GPO model excelled with a score of 58.31, indicating its generalization and adaptability beyond localized training objectives.

\subsection{Trade-Offs in EFCG}
\begin{figure}[t]
    \centering
        \includegraphics[width=0.49\textwidth]{figs/tradeoff.pdf}
    \caption{The Trade-off between F1 score and CSR. While BERTScore tends to improve with more attributes, CSR declines}
    \vspace{-1.5em}
    \label{fig:tradeoff}
\end{figure}

Figure~\ref{fig:tradeoff} illustrates the interplay between BERTScore and CSR across different numbers of attributes from 10 to 50 for each model. As the figure shows, increasing the number of attributes presents a clear double-edged effect: while more attributes can enhance fine-grained control (e.g., higher F1 score) over the generated text, the added complexity makes it more difficult for the model to maintain high constraint adherence.

\paragraph{Better Multi-Objective Alignment Under EFCG.}
\begin{figure*}[htbp]
    \centering
        \includegraphics[width=0.98\textwidth]{figs/case_study.pdf}
    \caption{In a case study on travel itinerary generation, the attention flow illustrates improved constraint awareness in AR+GPO.}
    \vspace{-1em}
    \label{fig:case_study}
\end{figure*}

When looking at the 30, 40, and 50 attribute conditions:
AR+GPO consistently attains CSR values 5--10 points higher than the other two models without sacrificing F1.
For example, at 50 attributes, AR+GPO’s CSR (44.76\%) is considerably above AR’s (35.86\%) and Original’s (37.40\%), while also delivering the highest F1 (0.6348 vs. 0.6310 for AR and 0.6076 for Original).



This pattern illustrates a more favorable trade-off for AR+GPO: it does not simply chase high BERTScore by ignoring constraints, nor does it force all constraints at the expense of overall text quality. Instead, AR+GPO’s global optimization helps coordinate multiple constraints while retaining strong semantic alignment. In contrast, AR appears effective at moderate attribute counts but loses ground on CSR once the load goes beyond 30 attributes, and the Original model experiences an even steeper decline.

% \paragraph{Implications.} In extreme fine-grained control (EFCG) tasks, these findings confirm that:
% \begin{enumerate}
%     \item Light to Moderate Constraints (e.g., up to 20 attributes) can be addressed by simpler fine-tuning without major F1 loss.
%     \item High Constraint Settings (30+ attributes), especially when semantic overlap among attributes is low or conflicts are frequent, demand methods like DPO or other preference-optimization approaches to prevent a precipitous drop in CSR.
% \end{enumerate}
% Table~\ref{tab:data_synthesis} presents the performance comparison between the original baselines and the SFT model across three traditional text classification benchmarks: Emotion, AG-News, and TREC. The results highlight significant improvements achieved by the SFT model, particularly for Emotion and TREC datasets. On the Emotion dataset, the SFT model achieves a 42.3\% accuracy, representing a substantial improvement of 14.0 percentage points over the baseline. Similarly, on the TREC dataset, which focuses on question type classification, the SFT model attains a 55.0\% accuracy, outperforming the baseline by 17.0 percentage points.



% \subsection{Toxicity Control}
% \label{sec:toxic}
% In this section, we use a toxic classifier~\cite{Detoxify} to identify 171 harmful examples from FineWeb. Using the same attribute extraction methods as described in Section 3, we then generate texts based on these attributes.

% The results in Figure \ref{fig:toxicity} clearly demonstrate that the SFT model struggles to handle toxicity control effectively, particularly as the number of attributes increases. While the Original Model maintains a consistently low toxicity rate across all levels of attribute complexity, the SFT model shows a significant and steady rise in toxicity rate, reaching over 25\% when handling 60 attributes. 

% This trend suggests that the SFT model fails to generalize well under highly constrained conditions and becomes increasingly susceptible to generating toxic content as it attempts to satisfy a growing number of attributes. The inability to properly balance attribute satisfaction with toxicity control highlights a critical limitation of the SFT approach, emphasizing the need for more robust mechanisms to enforce safety constraints, especially in scenarios involving complex or numerous attributes.

% \begin{figure}[t] 
%     \centering
%         \includegraphics[width=0.5\textwidth]{figs/toxicity.pdf}
%     \caption{Comparison of toxicity. }
%     \label{fig:toxicity}
% \end{figure}

% \subsection{Attention Flow}


% In this section, we test whether our UltraGen adapt well in out of domain downstream tasks. We test two capabilities, the first one is the factual, the


\begin{figure*}[h]
    \centering
    \includegraphics[width=14cm]{figures/visualized_kitti5.jpg}
    \caption[Qualitative Results on KITTI \textit{val.} set]{\textbf{Qualitative results on the KITTI \textit{val} set for the car class.} The proposed method (green) and ground truth (red).
    } \label{fig:KITTI visualized}
\end{figure*}

\begin{figure*}[t]
    \centering
     \includegraphics[width=14cm]{figures/custom_result_monodetr_monoground3.jpg}
    \caption{\textbf{Qualitative results on the custom dataset.} Comparison of detection results between the proposed model (blue), the state-of-the-art models (green), and ground truth (red) in ego-view (left) and bird's-eye view (right); MonoDETR (left) and MonoGround (right).}
    \label{fig:custom_result_visualized}
\end{figure*}

\section{Conclusion}
\label{sec:conclusion}
This paper presents a novel approach to monocular 3D object detection by integrating a Vision Foundation Model as the backbone with the DETR architecture, enabling enhanced depth estimation and feature extraction within a single-stage, real-time framework. By incorporating a Hierarchical Feature Fusion Block for multi-scale detection and 6D Dynamic Anchor Boxes for iterative bounding box refinement, the proposed model achieves improved performance without relying on additional data sources, such as LiDAR. Future work will focus on extending the model's capabilities to detect 3D bounding boxes while accounting for rolling and pitching angles.

% \clearpage


\bibliography{ref}

% \clearpage

\onecolumn
\clearpage
\appendix
\appendixpage  % if you use a package that provides an appendix title page
\hypersetup{linkcolor=black}
\startcontents[sections]
\printcontents[sections]{l}{1}

\hypersetup{linkcolor=hrefblue}
\glsresetall

\section{Additional related works}\label{apx:related_works}

\paragraph{Knowledge distillation.}
Knowledge distillation (KD)~\citep{hinton2015distilling,gou2021knowledge} is closely connected to W2S generalization regarding the teacher-student setup, while W2S reverts the capacities of teacher and student in KD. In KD, a strong teacher model guides a weak student model to learn the teacher's knowledge. In contrast, W2S generalization occurs when a strong student model surpasses a weak teacher model under weak supervision.
\citet{phuong2019towards,stanton2021does,ojha2023knowledge,nagarajan2023student,dong2024cluster,ildiz2024high} conducted rigorous statistical analyses for the student's generalization from knowledge distillation. 
From the analysis perspective, a key difference between KD and W2S is that W2S is usually analyzed in the context of finetuning since the notions of “weak” and “strong” are built upon pretraining. This finetuning perspective introduces distinct angles from KD for examining intrinsic dimension~\citep{li2018measuring} and student-teacher correlation in W2S. 

\paragraph{Self-distillation and self-training.}
In contrast to W2S that considers distinct student and teacher models, self-distillation~\citep{zhang2019your,zhang2021self} and related paradigms such as Born-Again Networks~\citep{furlanello2018born} use the same or progressively refined architectures to iteratively distill knowledge from a ``previous version'' of the model. There have been extensive theoretical analyses toward understanding the mechanism behind self-distillation~\citep{mobahi2020self,das2023understanding,borup2023self,pareek2024understanding}.

Self-training~\citep{scudder1965probability,lee2013pseudo} is a closely related method to self-distillation that takes a single model's confident predictions to create pseudo-labels for unlabeled data and refines that model iteratively. 
\citet{wei2020theoretical,oymak2021theoretical,frei2022self} provide theoretical insights into the generalization of self-training. 
In particular, \citet{wei2020theoretical} introduced a theoretical framework based on neighborhood expansion, which was later on extended to various settings of weakly supervised learning, including domain adaptation~\citep{cai2021theory}, contrastive learning~\citep{shen2022connect}, consistency regularization~\citep{yang2023sample}, and now weak-to-strong generalization~\citep{lang2024theoretical,shin2024weak}.




\section{Proofs in \Cref{sec:single_task_ft}}

\begin{lemma}\label{lem:low_est_err_ft}    
    Given the FT approximation errors $\rho_s$ and $\rho_w$ in \Cref{def:ft_est_err}, we have
    \begin{align*}
        \rho_s(n) \le n \rho_s \quad \text{and} \quad \rho_w(n) \le n \rho_w \quad \forall\ n \in \N.
    \end{align*}
\end{lemma}

\begin{proof}[Proof of \Cref{lem:low_est_err_ft}]
    Let $\thetab_* = \argmin_{\thetab \in \R^d}\ \E_{\xb \sim \Dcal}[(\phi_w(\xb)^\top \thetab - f_*(\xb))^2]$ such that
    \begin{align*}
        \E_{\xb \sim \Dcal}[(\phi_w(\xb)^\top \thetab_* - f_*(\xb))^2] = \rho_w.
    \end{align*}
    Then, by observing that conditioned on $\Xb$,
    \begin{align*}
        \phi_w(\Xb)^\dagger f_*(\Xb) = \argmin_{\thetab \in \R^d}\ \| \phi_w(\Xb) \thetab - f_*(\Xb) \|_2^2,
    \end{align*} 
    we have
    \begin{align*}
        \rho_w(n) &= \E_{\Xb \sim \Dcal^n}\sbr{\| \phi_w(\Xb) \phi_w(\Xb)^\dagger f_*(\Xb) - f_*(\Xb) \|_2^2} \\
        &\le \E_{\Xb \sim \Dcal^n}\sbr{\| \phi_w(\Xb) \thetab_* - f_*(\Xb) \|_2^2} \\
        &= n\ \E_{\Xb \sim \Dcal^n}\sbr{\frac{1}{n} \| \phi_w(\Xb) \thetab_* - f_*(\Xb) \|_2^2} \\
        &= n\ \E_{\xb \sim \Dcal}\sbr{(\phi_w(\xb)^\top \thetab_* - f_*(\xb))^2} \\
        &= n\ \rho_w.
    \end{align*}
    The proof for $\rho_s(n)$ follows analogously.
\end{proof}



\subsection{Proof of \Cref{thm:w2s_ft}}\label{apx:pf_w2s_ft}

\begin{theorem}[Formal restatement of \Cref{thm:w2s_ft}]\label{thm:w2s_ft_formal}
    Consider $f_\wts(\xb) = \phi_s(\xb)^\top \thetab_\wts$ finetuned as in \eqref{eq:sft_weak}, \eqref{eq:w2s_ft} with both $\alpha_w, \alpha_\wts \to 0$. Under \Cref{asm:features,asm:ft_data}, when $n \ge \Omega(d_w)$, the excess risk $\exrisk(f_\wts) = \vari(f_\wts) + \bias(f_\wts)$ satisfies
    \begin{align*}
        &\bias(f_\wts) \le \frac{\rho_w(n)}{n} + \frac{\rho_s(N)}{N} \le \rho_w + \rho_s, \\
        &\vari(f_\wts) \lesssim \frac{\sigma^2}{n} \rbr{d_{s \wedge w} + \frac{d_s}{N} (d_w - d_{s \wedge w})}.
    \end{align*}
    In particular, when ${\rho_w(n)}/{n} > 0$ and $d_s < d_w$, the inequality for $\bias(f_\wts)$ is strict.

    Moreover, when $\phi_w(\xb) \sim \Ncal(\b0_d, \Sigmab_w)$, for any $n > d_w + 1$, we have 
    \begin{align*}
        &\vari(f_\wts) = \frac{\sigma^2}{n-d_w-1} \rbr{d_{s \wedge w} + \frac{d_s}{N} (d_w - d_{s \wedge w})}.
    \end{align*}
\end{theorem}

\begin{proof}[Proof of \Cref{thm:w2s_ft} and \Cref{thm:w2s_ft_formal}]
    We first observe that the solution of \eqref{eq:sft_weak} as $\alpha_w \to 0$ is given by
    \begin{align*}
        \thetab_w = \wt\Phib_w^\dagger \wt\yb = \wt\Phib_w^\dagger (\wt\fb_* + \wt\zb),
    \end{align*}
    where $\wt\zb \sim \Ncal(\b0_n, \sigma^2 \Ib_n)$.
    Meanwhile, the solution of \eqref{eq:w2s_ft} as $\alpha_\wts \to 0$ is given by
    \begin{align*}
        \thetab_\wts = \Phib_s^\dagger \Phib_w \thetab_w = \Phib_s^\dagger \Phib_w \wt\Phib_w^\dagger (\wt\fb_* + \wt\zb).
    \end{align*}  
    
    Then, the excess risk of $f_\wts$ can be decomposed into variance and bias as follows:
    \begin{align*}
        \exrisk(f_\wts) &= \E_{\xb \sim \Dcal}\sbr{\E_{f_\wts}\sbr{(f_\wts(\xb) - f_*(\xb))^2}} \\
        &= \E_{\Scal_x}\sbr{\E_{\wt\Scal}\sbr{\frac{1}{N}\nbr{\Phib_s \thetab_\wts - \fb_*}_2^2}} \\
        &=\E_{\Scal_x, \wt\Scal}\sbr{\frac{1}{N} \nbr{(\Phib_s \Phib_s^\dagger \Phib_w \wt\Phib_w^\dagger \wt\fb_* - \fb_*) + \Phib_s \Phib_s^\dagger \Phib_w \wt\Phib_w^\dagger \wt\zb}_2^2} \\
        &= \underbrace{\frac{1}{N} \E_{\Scal_x, \wt\Scal}\sbr{\nbr{\Phib_s \Phib_s^\dagger \Phib_w \wt\Phib_w^\dagger \wt\zb}_2^2}}_{\vari(f_\wts)} + \underbrace{\frac{1}{N} \E_{\Scal_x, \wt\Scal}\sbr{\nbr{\Phib_s \Phib_s^\dagger \Phib_w \wt\Phib_w^\dagger \wt\fb_* - \fb_*}_2^2}}_{\bias(f_\wts)}.
    \end{align*}

    \paragraph{Bias.}
    For the bias term, by observing that $\Pb_s = \Phib_s \Phib_s^\dagger$ is an $N \times N$ orthogonal projection, we can decompose the bias term as
    \begin{align*}
        \bias(f_\wts) &= \E_{\Scal_x, \wt\Scal}\sbr{\frac{1}{N} \nbr{\Pb_s \rbr{\Phib_w \wt\Phib_w^\dagger \wt\fb_* - \fb_*}}_2^2} + \frac{1}{N} \E_{\Scal_x}\sbr{\nbr{\rbr{\Ib_N - \Pb_s} \fb_*}_2^2},
    \end{align*}
    where $\E_{\Scal_x}\sbr{\nbr{\rbr{\Ib_N - \Pb_s} \fb_*}_2^2} = \rho_s(N)$ by \Cref{def:ft_est_err}.

    For the first term, 
    \begin{align*}
        \E_{\Scal_x, \wt\Scal}\sbr{\frac{1}{N} \nbr{\Pb_s \rbr{\Phib_w \wt\Phib_w^\dagger \wt\fb_* - \fb_*}}_2^2} &\le \E_{\Scal_x, \wt\Scal}\sbr{\frac{1}{N} \nbr{\Phib_w \wt\Phib_w^\dagger \wt\fb_* - \fb_*}_2^2} \\
        &= \E_{\wt\Scal}\sbr{\frac{1}{n} \nbr{\wt\Phib_w \wt\Phib_w^\dagger \wt\fb_* - \wt\fb_*}_2^2} \\
        &= \frac{\rho_w(n)}{n}.
    \end{align*}
    Notice that when ${\rho_w(n)}/{n} > 0$, this inequality is strict if $d_s < d_w$, where $\Phib_w \wt\Phib_w^\dagger \wt\fb_* - \wt\fb_* \notin \range(\Phib_s)$ almost surely.

    Overall, we have
    \begin{align*}
        \bias(f_\wts) \le \frac{\rho_w(n)}{n} + \frac{\rho_s(N)}{N} \le \rho_w + \rho_s,
    \end{align*}
    where the second inequality follows from \Cref{lem:low_est_err_ft}.

    \paragraph{Variance.}
    For the variance term, we observe that
    \begin{align*}
    \begin{split}
        \vari(f_\wts) &= \frac{1}{N} \E_{\Scal_x, \wt\Scal}\sbr{\nbr{\Pb_s \Phib_w \wt\Phib_w^\dagger \wt\zb}_2^2} \\
        &= \frac{1}{N} \E_{\Scal_x, \wt\Scal}\sbr{\tr\rbr{\Phib_w^\top \Pb_s \Phib_w \wt\Phib_w^\dagger \wt\zb \wt\zb^\top (\wt\Phib_w^\dagger)^\top}} \\
        &= \frac{\sigma^2}{N} \E_{\Scal_x, \wt\Scal}\sbr{\tr\rbr{\Phib_w^\top \Pb_s \Phib_w (\wt\Phib_w^\top \wt\Phib_w)^\dagger}},
    \end{split}
    \end{align*}
    which implies
    \begin{align}\label{eq:pf_var_w2s}
    \begin{split}
        \vari(f_\wts) = \frac{\sigma^2}{N} \tr\rbr{\E_{\Scal_x}\sbr{\Sigmab_w^{-1/2} \Phib_w^\top \Pb_s \Phib_w \Sigmab_w^{-1/2}} \E_{\wt\Scal}\sbr{\rbr{\Sigmab_w^{-1/2} \wt\Phib_w^\top \wt\Phib_w \Sigmab_w^{-1/2}}^\dagger}}.
    \end{split}
    \end{align}

    Recall the spectral decomposition $\Sigmab_w = \Vb_w \Lambdab_w \Vb_w^\top$. 
    Since $\E_{\xb \sim \Dcal}[\phi_w(\xb) \phi_w(\xb)^\top] = \Sigmab_w$, for each $\xb \sim \Dcal$, we can write $\phi_w(\xb) = \Sigmab_w^{1/2} \gammab$, where $\gammab \in \R^{d}$ is an independent random vector that is zero-mean and isotropic (\ie $\E[\gammab] = \b0_{d}$ and $\E[\gammab \gammab^\top] = \Ib_{d}$). The same holds for $\Sigmab_s = \Vb_s \Lambdab_s \Vb_s^\top$ and $\phi_s(\xb) = \Sigmab_s^{1/2} \gammab$.

    Then, for $\Scal$ and $\wt\Scal$, there exist independent random matrices $\Gammab = [\gammab_1, \ldots, \gammab_N]^\top \in \R^{N \times d}$ and $\wt\Gammab = [\wt\gammab_1, \ldots, \wt\gammab_n]^\top \in \R^{n \times d}$ consisting of $\iid$ zero-mean isotropic rows such that
    \begin{align}\label{eq:pf_var_w2s_subgaussian_asm}
    \begin{split}
        &\Phib_w \Sigmab_w^{-1/2} = \Gammab \Sigmab_w^{1/2} \Sigmab_w^{-1/2} = \Gammab \Vb_w \Vb_w^\top, \\
        &\wt\Phib_w \Sigmab_w^{-1/2} = \wt\Gammab \Sigmab_w^{1/2} \Sigmab_w^{-1/2} = \wt\Gammab \Vb_w \Vb_w^\top, \\
        &\Phib_s \Sigmab_s^{-1/2} = \Gammab \Sigmab_s^{1/2} \Sigmab_s^{-1/2} = \Gammab \Vb_s \Vb_s^\top, \\
        &\wt\Phib_s \Sigmab_s^{-1/2} = \wt\Gammab \Sigmab_s^{1/2} \Sigmab_s^{-1/2} = \wt\Gammab \Vb_s \Vb_s^\top.
    \end{split}
    \end{align}
    Let $\Gammab_w = \Gammab \Vb_w \in \R^{N \times d_w}$ and $\wt\Gammab_w = \wt\Gammab \Vb_w \in \R^{n \times d_w}$. We observe that
    \begin{align*}
        \E_{\wt\Scal}\sbr{\rbr{\Sigmab_w^{-1/2} \wt\Phib_w^\top \wt\Phib_w \Sigmab_w^{-1/2}}^\dagger}
        = \E_{\wt\Scal}\sbr{\rbr{\Vb_w \wt\Gammab_w^\top \wt\Gammab_w \Vb_w^\top}^\dagger} 
        = \Vb_w \E_{\wt\Scal}\sbr{\rbr{\wt\Gammab_w^\top \wt\Gammab_w}^\dagger} \Vb_w^\top.
    \end{align*}

    Now, we consider the following two cases for the feature distribution of $\phi_w(\xb)$, corresponding to the distribution of $\Gammab_w$ and $\wt\Gammab_w$:
    \begin{enumerate}[label=(\alph*)]
        \item \b{Gaussian features}: In \Cref{thm:w2s_ft}, assuming $\phi_w(\xb) \sim \Ncal(\b0_d, \Sigmab_w)$ such that $\wt\Gammab_w$ consists of $\iid$ Gaussian rows, we have $\wt\gammab_i \sim \Ncal(\b0_{d_w}, \Ib_{d_w})$. Notice that under the assumption $n > d_w + 1$, $\rank(\wt\Gammab_w) = d_w$ almost surely, and therefore $\wt\Gammab_w^\top \wt\Gammab_w$ is invertible.
        
        Meanwhile, with $\wt\gammab_i \sim \Ncal(\b0_{d_w}, \Ib_{d_w})$ for all $i \in [n]$, $(\wt\Gammab_w^\top \wt\Gammab_w) \sim \Wcal(\Ib_{d_w},n)$ follows the Wishart distribution~\citep[Definition 3.4.1]{wishart1928generalised} with $n$ degrees of freedom and scale matrix $\Ib_{d_w}$. 
        Therefore, $(\wt\Gammab_w^\top \wt\Gammab_w)^{-1} \sim \Wcal^{-1}(\Ib_{d_w},n)$ follows the inverse Wishart distribution~\citep[\S 3.8]{mardia2024multivariate}, whose mean takes the form~\citep[(3.8.3)]{mardia2024multivariate}
        \begin{align*}
            \E_{\wt\Scal}\sbr{(\wt\Gammab_w^\top \wt\Gammab_w)^\dagger} = \frac{1}{n - d_w -1} \Ib_{d_w}.
        \end{align*}
        Then, we have
        \begin{align*}
            \E_{\wt\Scal}\sbr{\rbr{\Sigmab_w^{-1/2} \wt\Phib_w^\top \wt\Phib_w \Sigmab_w^{-1/2}}^\dagger}
            = \frac{1}{n - d_w -1} \Vb_w \Vb_w^\top.
        \end{align*}
        Therefore, \eqref{eq:pf_var_w2s} implies
        \begin{align}\label{eq:pf_var_w2s_1}
        \begin{split}
            \vari(f_\wts) &= \frac{\sigma^2}{N}\ \frac{1}{n - d_w -1}\ \tr\rbr{\Vb_w^\top \E_{\Scal_x}\sbr{\Sigmab_w^{-1/2} \Phib_w^\top \Pb_s \Phib_w \Sigmab_w^{-1/2}} \Vb_w} \\
            &= \frac{\sigma^2}{N}\ \frac{1}{n - d_w -1}\ \tr\rbr{\E_{\Scal_x}\sbr{\Vb_w^\top \Vb_w \Gammab_w^\top \Pb_s \Gammab_w \Vb_w^\top \Vb_w}} \\
            &= \frac{\sigma^2}{N}\ \frac{1}{n - d_w -1}\ \tr\rbr{\E_{\Scal_x}\sbr{\Gammab_w^\top \Pb_s \Gammab_w}}.
        \end{split}
        \end{align}
        Recall that $\Pb_s = \Phib_s \Phib_s^\dagger$. Let $\Gammab_s = \Gammab \Vb_s \in \R^{N \times d_s}$, and we can write
        \begin{align*}
            \Pb_s = (\Phib_s \Sigmab_s^{-1/2}) (\Phib_s \Sigmab_s^{-1/2})^\dagger = (\Gammab_s \Vb_s^\top) (\Gammab_s \Vb_s^\top)^\dagger = \Gammab_s \Gammab_s^\dagger.
        \end{align*}
        Therefore, with $\Gammab_w = \Gammab \Vb_w$ and $\Gammab_s = \Gammab \Vb_s$, we can decompose
        \begin{align*}
            \tr\rbr{\E_{\Scal_x}\sbr{\Gammab_w^\top \Pb_s \Gammab_w}} 
            &= \E_{\Scal_x}\sbr{\tr\rbr{\Gammab_w^\top \Gammab_s \Gammab_s^\dagger \Gammab_w}} \\
            &= \E_{\Scal_x}\sbr{\tr\rbr{\Vb_w^\top \Vb_s \Vb_s^\top \Vb_w \Gammab_w^\top \Gammab_s \Gammab_s^\dagger \Gammab_w}} \\
            &+ \E_{\Scal_x}\sbr{\tr\rbr{\Vb_w^\top (\Ib_d - \Vb_s \Vb_s^\top) \Vb_w \Gammab_w^\top \Gammab_s \Gammab_s^\dagger \Gammab_w}}.
        \end{align*}
        For the first term, since $\Gammab_w \Vb_w^\top \Vb_s = \Gammab \Vb_w \Vb_w^\top \Vb_s$ and $\Gammab_s = \Gammab \Vb_s$, the range of $\Gammab_w \Vb_w^\top \Vb_s$ is a subspace of that of $\Gammab_s$ and therefore,
        \begin{align*}
            \E_{\Scal_x}\sbr{\tr\rbr{\Vb_w^\top \Vb_s \Vb_s^\top \Vb_w \Gammab_w^\top \Gammab_s \Gammab_s^\dagger \Gammab_w}} 
            &= \E_{\Scal_x}\sbr{\tr\rbr{ \Vb_s^\top \Vb_w \Gammab_w^\top \Gammab_s \Gammab_s^\dagger \Gammab_w \Vb_w^\top \Vb_s}} \\
            &= \E_{\Scal_x}\sbr{\tr\rbr{ \Vb_s^\top \Vb_w \Gammab_w^\top \Gammab_w \Vb_w^\top \Vb_s}} \\
            &= \tr\rbr{\Vb_s^\top \Vb_w \E_{\Scal_x}\sbr{\Gammab_w^\top \Gammab_w} \Vb_w^\top \Vb_s}.
        \end{align*}
        Since $\E_{\Scal_x}\sbr{\Gammab_w^\top \Gammab_w} = N \Ib_{d_w}$, we have
        \begin{align*}
            \E_{\Scal_x}\sbr{\tr\rbr{\Vb_w^\top \Vb_s \Vb_s^\top \Vb_w \Gammab_w^\top \Gammab_s \Gammab_s^\dagger \Gammab_w}} 
            &= N \tr\rbr{\Vb_s^\top \Vb_w \Vb_w^\top \Vb_s} \\
            &= N \nbr{\Vb_s^\top \Vb_w}_F^2 \\
            &= N d_{s \wedge w}.
        \end{align*}
        For the second term, we first observe that the row space of $\Gammab_w \Vb_w^\top (\Ib_d - \Vb_s \Vb_s^\top)$ is orthogonal to that of $\Gammab_s = \Gammab \Vb_s$, and therefore, $\Gammab_w \Vb_w^\top (\Ib_d - \Vb_s \Vb_s^\top)$ and $\Gammab_s$ are independent, which implies
        \begin{align*}
            \E_{\Scal_x}\sbr{\tr\rbr{\Vb_w^\top (\Ib_d - \Vb_s \Vb_s^\top) \Vb_w \Gammab_w^\top \Gammab_s \Gammab_s^\dagger \Gammab_w}} 
            &= \tr\rbr{\E\sbr{\Gammab_w \Vb_w^\top (\Ib_d - \Vb_s \Vb_s^\top) \Vb_w \Gammab_w^\top} \E\sbr{\Gammab_s \Gammab_s^\dagger}}.
        \end{align*}
        Since $\Gammab$ consists of independent isotropic rows, so do $\Gammab_s = \Gammab \Vb_s \in \R^{N \times d_s}$ and $\Gammab_w = \Gammab \Vb_w \in \R^{N \times d_w}$, which implies
        \begin{align*}
            \E\sbr{\Gammab_s \Gammab_s^\dagger} = \frac{d_s}{N}\ \Ib_N \quad \t{and} \quad \E\sbr{\Gammab_w^\top \Gammab_w} = N\ \Ib_{d_w}.
        \end{align*}
        Then, we have
        \begin{align*}
            \E_{\Scal_x}\sbr{\tr\rbr{\Vb_w^\top (\Ib_d - \Vb_s \Vb_s^\top) \Vb_w \Gammab_w^\top \Gammab_s \Gammab_s^\dagger \Gammab_w}} 
            &= \tr\rbr{\E\sbr{\Gammab_w \Vb_w^\top (\Ib_d - \Vb_s \Vb_s^\top) \Vb_w \Gammab_w^\top} \E\sbr{\Gammab_s \Gammab_s^\dagger}} \\
            &= \frac{d_s}{N} \tr\rbr{\E\sbr{\Gammab_w \Vb_w^\top (\Ib_d - \Vb_s \Vb_s^\top) \Vb_w \Gammab_w^\top}} \\
            &= \frac{d_s}{N} \tr\rbr{\Vb_w^\top (\Ib_d - \Vb_s \Vb_s^\top) \Vb_w \E\sbr{\Gammab_w^\top \Gammab_w}} \\
            &= \frac{d_s}{N} N \tr\rbr{\Vb_w^\top (\Ib_d - \Vb_s \Vb_s^\top) \Vb_w} \\
            &= d_s (d_w - d_{s \wedge w}).
        \end{align*}
        Combining the two terms, we have
        \begin{align*}
            \tr\rbr{\E_{\Scal_x}\sbr{\Gammab_w^\top \Pb_s \Gammab_w}} = N d_{s \wedge w} + d_s (d_w - d_{s \wedge w}).
        \end{align*}
        Then, by \eqref{eq:pf_var_w2s_1}, the variance is exactly characterized by
        \begin{align*}
            \vari(f_\wts) 
            &= \frac{\sigma^2}{N}\ \frac{N d_{s \wedge w} + d_s (d_w - d_{s \wedge w})}{n - d_w -1} \\
            &= \frac{\sigma^2}{n-d_w-1} \rbr{d_{s \wedge w} + \frac{d_s}{N} (d_w - d_{s \wedge w})}.
        \end{align*}

        \item \b{Sub-gaussian features}: Relaxing the Gaussian feature assumption, when $\wt\Gammab_w$ consists of $\iid$ sub-gaussian random vectors that are zero-mean and isotropic (\ie $\E[\wt\gammab_i] = \b0_{d_w}$ and $\E[\wt\gammab_i \wt\gammab_i^\top] = \Ib_{d_w}$), with $n \ge \Omega(d_w)$, \Cref{lem:trace_inv_subgaussian} implies that
        \begin{align*}
            \E_{\wt\Scal}\sbr{(\wt\Gammab_w^\top \wt\Gammab_w)^\dagger} \aleq O\rbr{\frac{1}{n}} \Ib_{d_w},
        \end{align*}
        and therefore,
        \begin{align*}
            \E_{\wt\Scal}\sbr{\rbr{\Sigmab_w^{-1/2} \wt\Phib_w^\top \wt\Phib_w \Sigmab_w^{-1/2}}^\dagger} \aleq O\rbr{\frac{1}{n}} \Vb_w \Vb_w^\top.
        \end{align*}
        Then, via an analogous argument as \eqref{eq:pf_var_w2s_1}, \eqref{eq:pf_var_w2s} implies that 
        \begin{align}\label{eq:pf_var_w2s_2}
        \begin{split}
            \vari(f_\wts) \le \frac{\sigma^2}{N}\ O\rbr{\frac{1}{n}}\ \tr\rbr{\E_{\Scal_x}\sbr{\Gammab_w^\top \Pb_s \Gammab_w}}.
        \end{split}
        \end{align}
        We observe that in the analysis of the Gaussian feature case, the characterization
        \begin{align*}
            \tr\rbr{\E_{\Scal_x}\sbr{\Gammab_w^\top \Pb_s \Gammab_w}} = (N - d_s) d_{s \wedge w} + d_s d_w
        \end{align*}
        does not involve the Gaussianity of $\Gammab$ and therefore holds for general subgaussian features.
        This leads to an upper bound on the variance:
        \begin{align*}
            \vari(f_\wts) 
            &\le \frac{\sigma^2}{N}\ O\rbr{\frac{1}{n}}\ \rbr{N d_{s \wedge w} + d_s (d_w - d_{s \wedge w})} \\
            &\lesssim \frac{\sigma^2}{n} \rbr{d_{s \wedge w} + \frac{d_s}{N} (d_w - d_{s \wedge w})}.
        \end{align*}
    \end{enumerate}
\end{proof}


\begin{lemma}[Adapting \cite{vershynin2010introduction} Theorem 5.39]\label{lem:trace_inv_subgaussian}
    Let $\wt\Gammab_w = [\wt\gammab_1, \ldots, \wt\gammab_n]^\top$ be an $n \times d_w$ matrix whose rows $\wt\gammab_1, \ldots, \wt\gammab_n$ consist of $\iid$ sub-gaussian random vectors that are zero-mean and isotropic (\ie $\E[\wt\gammab_i] = \b0_{d_w}$ and $\E[\wt\gammab_i \wt\gammab_i^\top] = \Ib_{d_w}$). When $n \ge \Omega(d_w)$, we have
    \begin{align*}
        \E\sbr{\nbr{\rbr{\wt\Gammab_w^\top \wt\Gammab_w}^\dagger}_2} \le O\rbr{\frac{1}{n}},
    \end{align*}
    where $\Omega(\cdot)$ and $O(\cdot)$ suppresses constants that depend only on the sub-gaussian norm $\nbr{\wt\gammab_i}_{\psi_2} = \sup_{\vb \in \SSS^{d_w-1}} \sup_{p \ge 1} (\E[|\wt\gammab_i^\top \vb|^p])^{1/p} / \sqrt{p}$, independent of $n, d_w$.
\end{lemma}

\begin{proof}[Proof of \Cref{lem:trace_inv_subgaussian}]
    Let $\sigma_{\min}(\wt\Gammab_w^\top \wt\Gammab_w)$ be the smallest singular value of $\wt\Gammab_w^\top \wt\Gammab_w$.
    Leveraging \cite{vershynin2010introduction} Theorem 5.39, we notice that for $n \ge \Omega(d_w)$, there exist constants $c_1, c_2 > 0$ that depend only on the sub-gaussian norm $\nbr{\wt\gammab_i}_{\psi_2}$ such that
    \begin{align*}
        \Pr\sbr{\sigma_{\min}(\wt\Gammab_w^\top \wt\Gammab_w) < \rbr{\sqrt{n} - c_1\sqrt{d_w} - t}^2} \le \exp\rbr{-c_2 t^2}.
    \end{align*}
    Therefore, we have 
    \begin{align*}
        \Pr\sbr{\frac{1}{\sigma_{\min}(\wt\Gammab_w^\top \wt\Gammab_w)} > t} \le \exp\rbr{-c_2 \rbr{\sqrt{n} - c_1 \sqrt{d_w} - \sqrt{\frac{1}{t}}}^2}.
    \end{align*}

    Notice that for any non-negative random variable $Z$ with a cumulative density function $F_Z(z)$, 
    \begin{align*}
        \E\sbr{Z} &= \int_0^\infty z d F_Z(z) 
        = - \int_0^\infty z d \rbr{1 - F_Z(z)} \\
        &= \sbr{z \rbr{1 - F_Z(z)}}_0^\infty + \int_0^\infty \rbr{1 - F_Z(z)} dz \\
        &= \int_0^\infty \Pr\sbr{Z > z} dz.
    \end{align*}
    Therefore, we have
    \begin{align*}
        \E\sbr{\frac{1}{\sigma_{\min}(\wt\Gammab_w^\top \wt\Gammab_w)}} \le \int_0^\infty \exp\rbr{-c_2 \rbr{\sqrt{n} - c_1 \sqrt{d_w} - \sqrt{\frac{1}{t}}}^2} d t.
    \end{align*}
    Let $t_0 = 1 / \rbr{\sqrt{n} - c_1 \sqrt{d_w}}^2$ such that $\sqrt{n} - c_1 \sqrt{d_w} - \sqrt{\frac{1}{t}}=0$ and 
    \begin{align*}
        \int_{0}^{t_0} \exp\rbr{-c_2 \rbr{\sqrt{n} - c_1 \sqrt{d_w} - \sqrt{\frac{1}{t}}}^2} d t \le t_0
    \end{align*}
    Then, we have
    \begin{align*}
        &\E\sbr{\frac{1}{\sigma_{\min}(\wt\Gammab_w^\top \wt\Gammab_w)}} 
        \le \int_0^\infty \exp\rbr{-c_2 \rbr{\sqrt{n} - c_1 \sqrt{d_w} - \sqrt{\frac{1}{t}}}^2} d t \\
        &\le t_0 + \int_{t_0}^\infty \exp\rbr{-c_2 \rbr{\sqrt{n} - c_1 \sqrt{d_w} - \sqrt{\frac{1}{t}}}^2} d t \\
        &= t_0 + 2 \int_{0}^{\sqrt{n}-c_1\sqrt{d_w}} \exp\rbr{-c_2 u^2} \rbr{\sqrt{n} - c_1 \sqrt{d_w} - u}^{-3} d u \\
        &= t_0 + \frac{2}{\rbr{\sqrt{n} - c_1 \sqrt{d_w}}^2} \int_{0}^{1} \exp\rbr{-c_2 \rbr{\sqrt{n}-c_1\sqrt{d_w}}^2 u^2} \rbr{1 - u}^{-3} d u \\
        &= \frac{1}{\rbr{\sqrt{n} - c_1 \sqrt{d_w}}^2} + \frac{2}{\rbr{\sqrt{n} - c_1 \sqrt{d_w}}^2} \rbr{\int_{0}^{1} \exp\rbr{-\Omega\rbr{u^2}} \rbr{1 - u}^{-3} d u} \\
        &= O\rbr{\frac{1}{\rbr{\sqrt{n} - c_1 \sqrt{d_w}}^2}}.
    \end{align*}
    When $n \ge \Omega(d_w)$, we have $\sqrt{n} - c_1 \sqrt{d_w} \ge \Omega(\sqrt{n})$, and therefore ,
    \begin{align*}
        \E\sbr{\nbr{\rbr{\wt\Gammab_w^\top \wt\Gammab_w}^\dagger}_2}
        \le \E\sbr{\frac{1}{\sigma_{\min}(\wt\Gammab_w^\top \wt\Gammab_w)}} 
        \le O\rbr{\frac{1}{n}}.
    \end{align*}
\end{proof}





\subsection{Proof of \Cref{pro:sft_weak} and \Cref{cor:sft_strong}}\label{apx:pf_sft_weak}
\begin{proof}[Proof of \Cref{pro:sft_weak} and \Cref{cor:sft_strong}]
    The excess risk of the finetuned weak teacher $f_w(\xb) = \phi_w(\xb)^\top \thetab_w$ can be expressed as
    \begin{align*}
        \exrisk(f_w) &= \E_{\xb \sim \Dcal}\sbr{\E_{f_w}\sbr{(f_w(\xb) - f_*(\xb))^2}} \\
        &= \E_{\wt\Scal}\sbr{\frac{1}{n}\nbr{\wt\Phib_w \thetab_w - \wt\fb_*}_2^2},
    \end{align*}
    where $\wt\fb_* = [\fb_*(\wt\xb_1), \ldots, \fb_*(\wt\xb_n)]^\top \in \R^n$; and we recall that $\wt\Phib_w = [\phi_w(\wt\xb_1), \ldots, \phi_w(\wt\xb_n)]^\top$. Notice that the randomness of $\thetab_w$ comes from the SFT samples $\wt\Scal \sim \Dcal(f_*)^n$.

    Observe that the solution of \eqref{eq:sft_weak} as $\alpha_w \to 0$ is given by $\thetab_w = \wt\Phib_w^\dagger \wt\yb$, where $\wt\yb = \wt\fb_* + \wt\zb$ is the noisy label vector with $\wt\zb \sim \Ncal(\b0_n, \sigma^2 \Ib_n)$.
    Therefore, with the randomness over $\wt\Scal \sim \Dcal(f_*)^n$, we have
    \begin{align*}
        \exrisk(f_w) &= \E \sbr{\frac{1}{n}\nbr{\wt\Phib_w \wt\Phib_w^\dagger \wt\yb - \wt\fb_*}_2^2} \\
        &= \E \sbr{\frac{1}{n}\nbr{\wt\Phib_w \wt\Phib_w^\dagger \wt\zb + \rbr{\wt\Phib_w \wt\Phib_w^\dagger \wt\fb_* - \wt\fb_*}}_2^2} \\
        &= \underbrace{\E \sbr{\frac{1}{n}\nbr{\wt\Phib_w \wt\Phib_w^\dagger \wt\zb}_2^2}}_{\vari(f_w)} + \underbrace{\E\sbr{\frac{1}{n}\nbr{\wt\Phib_w \wt\Phib_w^\dagger \wt\fb_* - \wt\fb_*}_2^2}}_{\bias(f_w)}.
    \end{align*}
    
    For bias, by the definition of finetuning capacity (see \Cref{def:ft_est_err}), we have
    \begin{align*}
        \bias(f_w) = \frac{1}{n} \E\sbr{\nbr{\wt\Phib_w \wt\Phib_w^\dagger \wt\fb_* - \wt\fb_*}_2^2} = \frac{\rho_w(n)}{n}.
    \end{align*}
    We observe that $\bias(f_w) \le \rho_w$ by \Cref{lem:low_est_err_ft}.
    Notice that \Cref{lem:low_est_err_ft} also implies $\bias(f_s) = {\rho_s(n)}/{n} \le \rho_s$. 

    For variance, we observe that 
    \begin{align*}
        \vari(f_w) &= \frac{1}{n} \E\sbr{\nbr{\wt\Phib_w \wt\Phib_w^\dagger \wt\zb}_2^2} \\
        &= \frac{1}{n} \E\sbr{\tr\rbr{\wt\Phib_w \wt\Phib_w^\dagger \wt\zb \wt\zb^\top}} \\
        &= \frac{\sigma^2}{n} \E\sbr{\tr\rbr{\wt\Phib_w \wt\Phib_w^\dagger}}.
    \end{align*}
    By \Cref{asm:ft_data}, since $\rank(\wt\Phib_w) = d_w$ almost surely, we have
    \begin{align*}
        \vari(f_w) = \frac{\sigma^2}{n} \E\sbr{\tr\rbr{\wt\Phib_w \wt\Phib_w^\dagger}} = \frac{\sigma^2 d_w}{n}.
    \end{align*}
\end{proof}



\subsection{Proof of \Cref{cor:pgr}}\label{apx:pf_pgr}
\begin{proof}[Proof of \Cref{cor:pgr}]
    Noticing that with $\rank(\wt\Phib_w) = d_w$ and $\rank(\wt\Phib_s) = \rank(\Phib_s) = d_s$ almost surely, the excess risks of $f_w, f_s, f_c$ are characterized exactly in \Cref{pro:sft_weak} and \Cref{cor:sft_strong}, and $\exrisk(f_\wts)$ is upper bounded by \Cref{thm:w2s_ft}.
    Therefore, by directly plugging in the excess risks to the definitions of PGR and OPR, we have
    \begin{align}\label{eq:pgr_lower_tight}
    \begin{split}
        \pgr = &\frac{\exrisk(f_w) - \exrisk(f_\wts)}{\exrisk(f_w) - \exrisk(f_c)} \\
        \ge &\rbr{\sigma^2\ \frac{d_w}{n} + \frac{\rho_w(n)}{n} - \frac{\sigma^2}{n-d_w-1} \rbr{d_{s \wedge w} + \frac{d_s}{N} (d_w-d_{s \wedge w})} - \rbr{\frac{\rho_w(n)}{n} + \frac{\rho_s(N)}{N}}} \\
        &\rbr{\sigma^2\ \frac{d_w}{n} + \frac{\rho_w(n)}{n} - \sigma^2\ \frac{d_s}{N+n} - \frac{\rho_s(N+n)}{N+n}}^{-1} \\
        \ge &\rbr{\sigma^2 \frac{d_w}{n} - \sigma^2 \frac{d_{s \wedge w} + (d_w - d_{s \wedge w}) {d_s}/{N}}{n-d_w-1} - \frac{\rho_s(N)}{N}} \Big/ \rbr{\sigma^2 \frac{d_w}{n} + \frac{\rho_w(n)}{n}}, \\
        \ge &\rbr{\sigma^2 \frac{d_w}{n} - \sigma^2 \frac{d_{s \wedge w} + (d_w - d_{s \wedge w}) {d_s}/{N}}{n-d_w-1} - \rho_s} \Big/ \rbr{\sigma^2 \frac{d_w}{n} + \rho_w},
    \end{split}
    \end{align}
    and 
    \begin{align}\label{eq:opr_lower_tight}
    \begin{split}
        \opr = &\frac{\exrisk(f_s)}{\exrisk(f_\wts)} \\
        \ge &\rbr{\sigma^2\ \frac{d_s}{n} + \frac{\rho_s(n)}{n}} \Big/ \rbr{\sigma^2 \frac{d_{s \wedge w} + (d_w - d_{s \wedge w}) {d_s}/{N}}{n-d_w-1} + \rbr{\frac{\rho_w(n)}{n} + \frac{\rho_s(N)}{N}}} \\
        \ge &\sigma^2 \frac{d_s}{n} \Big/ \rbr{\sigma^2 \frac{d_{s \wedge w} + (d_w - d_{s \wedge w}) {d_s}/{N}}{n-d_w-1} + \rho_w + \rho_s}.
    \end{split}
    \end{align} 

    When taking $n = d_w + q + 1$ for some small constant $q \in \N$, we observe that 
    \begin{align*}
        \pgr &\ge \rbr{\sigma^2 \frac{d_w}{n} - \sigma^2 \frac{d_{s \wedge w} + (d_w - d_{s \wedge w}) {d_s}/{N}}{n-d_w-1} - \rho_s} \Big/ \rbr{\sigma^2 \frac{d_w}{n} + \rho_w} \\
        &\ge \rbr{\frac{d_w}{d_w + q + 1} - \frac{d_{s \wedge w}}{q} - \frac{d_s}{N} \frac{d_w - d_{s \wedge w}}{q} - \frac{\rho_s}{\sigma^2}} \Big/ \rbr{\frac{d_w}{d_w + q + 1} + \frac{\rho_w}{\sigma^2}} \\
        &\ge \rbr{\frac{d_w}{d_w + q + 1} - \frac{d_{s \wedge w}}{q} - \frac{d_s}{N} \frac{d_w - d_{s \wedge w}}{q} - \frac{\rho_s}{\sigma^2} - \frac{\rho_w}{\sigma^2}} \Big/ \rbr{\frac{d_w}{d_w + q + 1} + \frac{\rho_w}{\sigma^2} - \frac{\rho_w}{\sigma^2}} \\
        &= 1 - \frac{n}{d_w} \rbr{\frac{d_{s \wedge w}}{q} + \frac{d_s}{N} \frac{d_w - d_{s \wedge w}}{q} + \frac{\rho_w + \rho_s}{\sigma^2}} \\
        &= 1 - \frac{n}{q}\ {\frac{d_{s \wedge w} + (d_w - d_{s \wedge w}) d_s / N}{d_w}} - \frac{n}{d_w}\ {\frac{\rho_w + \rho_s}{\sigma^2}},
    \end{align*}
    and
    \begin{align*}
        \opr &\ge \sigma^2 \frac{d_s}{n} \Big/ \rbr{\sigma^2 \frac{d_{s \wedge w} + (d_w - d_{s \wedge w}) {d_s}/{N}}{n-d_w-1} + \rho_w + \rho_s} \\
        &= \frac{d_s}{n} \Big/ \rbr{\frac{d_{s \wedge w} + (d_w - d_{s \wedge w}) {d_s}/{N}}{q} + \frac{\rho_w + \rho_s}{\sigma^2}} \\
        &= \rbr{\frac{n}{q}\ \frac{d_{s \wedge w} + (d_w - d_{s \wedge w}) {d_s}/{N}}{d_s} + \frac{n}{d_s}\ \frac{\rho_w + \rho_s}{\sigma^2}}^{-1}.
    \end{align*}
\end{proof}



\subsection{Proof of \Cref{cor:non_monotonic_scaling}}\label{apx:pf_non_monotonic_scaling}
\begin{proof}[Proof of \Cref{cor:non_monotonic_scaling}]
    Recall the notations introduced for conciseness:
    \begin{align*}
        d_\wts(N) = d_{s \wedge w} + (d_w - d_{s \wedge w}) \frac{d_s}{N}, \quad \varrho = \frac{\rho_w + \rho_s}{\sigma^2}.
    \end{align*}
    Then, the lower bounds for $\pgr$ and $\opr$ in \Cref{cor:pgr} can be expressed in terms of $d_\wts(N)$ and $\varrho$ as 
    \begin{align*}
        \pgr \ge 1 - \frac{d_\wts(N)}{d_w} - \frac{d_w + 1}{d_w} \varrho - \frac{d_w + 1}{d_w}\ \frac{d_\wts(N)}{q} - q \frac{\varrho}{d_w},
    \end{align*}
    and 
    \begin{align*}
        \opr \ge \rbr{\frac{d_\wts(N)}{d_s} + \frac{d_w + 1}{d_s} \varrho + \frac{d_\wts(N)}{d_s}\ \frac{d_w + 1}{q} + q \frac{\varrho}{d_s}}^{-1}.
    \end{align*}
    Both lower bounds are maximized when the last two terms in the expressions that involve $q$ are minimized, which is achieved when $q = \sqrt{\rbr{d_w + 1} {d_\wts(N)}/{\varrho}}$. Substituting the optimal $q$ back into the expressions yields the best lower bounds for $\pgr$ and $\opr$:
    \begin{align*}
        \pgr \ge &1 - \frac{d_\wts(N)}{d_w} - \varrho \frac{d_w + 1}{d_w} - 2 \sqrt{\varrho \frac{d_w + 1}{d_w}\ \frac{d_\wts(N)}{d_w}} \\
        = &1 - \rbr{\sqrt{\frac{d_\wts(N)}{d_w}} + \sqrt{\varrho\ \frac{d_w + 1}{d_w}}}^2,
    \end{align*}
    and 
    \begin{align*}
        \opr \ge &\rbr{\frac{d_\wts(N)}{d_s} + \varrho \frac{d_w + 1}{d_s} + 2 \sqrt{\varrho \frac{d_w + 1}{d_s}\ \frac{d_\wts(N)}{d_s}}}^{-1} \\
        = &\rbr{\sqrt{\frac{d_\wts(N)}{d_s}} + \sqrt{\varrho\ \frac{d_w + 1}{d_s}}}^{-2}.
    \end{align*}
\end{proof}




\section{Ridge regression analysis}\label{apx:ridge_regression}
In this section, we investigate the more realistic scenario where the weak and strong feature covariances are not exactly low-rank but admit small numbers of dominating eigenvalues. 

Concretely, we consider the same data distribution $(\xb, y) \sim \Dcal(f_*)$ with $y = f_*(\xb) + z$ for some independent Gaussian label noise $z \sim \Ncal(0, \sigma^2)$ and an unknown ground truth function $f_*: \Xcal \to \R$ as in \Cref{sec:ridgeless_regression}.
Under the same sub-gaussian feature assumption as in \Cref{asm:features}, we adapt \Cref{def:low_intrinsic_dim,def:correlation_dim} to the ridge regression setting as follows.
\begin{assumption}[Data distribution]\label{asm:ridge_regression}
    Let $\phi_s: \Xcal \to \R^d$ and $\phi_w: \Xcal \to \R^d$ be the strong and weak pretrained models that take $\xb \sim \Dcal$ and output pretrained features $\phi_s(\xb), \phi_w(\xb) \in \R^d$, respectively.
    \begin{enumerate}[label=(\roman*)]
        \item \b{Ground truth}: Assume $f_*$ can be expressed as a linear function over an unknown ground truth feature $\phi_*: \Xcal \to \R^d$ such that $f_*(\cdot) = \phi_*(\cdot)^\top \thetab_*$ for some fixed $\thetab_* \in \R^d$.
        \item \b{Sub-gaussian features} (\Cref{asm:features}): Let $\phi_w(\xb)$, $\phi_s(\xb)$, $\phi_*(\xb)$ be zero-mean sub-gaussian random vectors with $\E[\phi_w(\xb)] = \E[\phi_s(\xb)] = \E[\phi_*(\xb)] = \b{0}_d$, and 
        \begin{align*}
            \E[\phi_w(\xb) \phi_w(\xb)^\top] = \Sigmab_w, \quad \E[\phi_s(\xb) \phi_s(\xb)^\top] = \Sigmab_s, \quad \E[\phi_*(\xb) \phi_*(\xb)^\top] = \Sigmab_*.
        \end{align*}
        For conciseness, we assume without loss of generality that these features are roughly normalized, \ie, $\nbr{\Sigmab_w}_2 \asymp 1$, $\nbr{\Sigmab_s}_2 \asymp 1$, and $\nbr{\Sigmab_*}_2 \asymp 1$.
        \item \b{Low intrinsic dimension}: Let $\Sigmab_s$ and $\Sigmab_w$ both be \b{positive-definite} with spectral decompositions $\Sigmab_s = \Vb_s \Lambdab_s \Vb_s^\top$ and $\Sigmab_w = \Vb_w \Lambdab_w \Vb_w^\top$, where $\Lambdab_s, \Lambdab_w \in \R^{d \times d}$ are diagonal matrices with positive eigenvalues in decreasing order; while $\Vb_s \in \R^{d \times d}$ and $\Vb_w \in \R^{d \times d}$ are orthogonal matrices consisting of the corresponding orthonormal eigenvectors. The low intrinsic dimension of FT implies that $\Lambdab_s = \diag(\lambda^s_1,\cdots,\lambda^s_d)$ and $\Lambdab_w = \diag(\lambda^w_1,\cdots,\lambda^w_d)$ consist of a few dominating eigenvalues, while the rest of the eigenvalues are negligible, \ie, there exist $d_s, d_w \ll d$ such that $\sum_{i > d_s} \lambda^s_i \ll \tr(\Sigmab_s)$ and $\sum_{i > d_w} \lambda^w_i \ll \tr(\Sigmab_w)$. Here, 
        \begin{align*}
            \tr(\Sigmab_s) \lesssim d_s \quad \t{and} \quad \tr(\Sigmab_w) \lesssim d_w
        \end{align*}
        effectively measure the intrinsic dimensions of $\phi_s$ and $\phi_w$.
    \end{enumerate}
\end{assumption}

\begin{remark}[Weak-strong similarity]
    In place of correlation dimension (\Cref{def:correlation_dim}) in the ridgeless setting, for the ridge regression analysis, we measure the similarity between the weak and strong models directly through $\tr(\Sigmab_s \Sigmab_w)$. Notice that 
    \begin{align*}
        \tr(\Sigmab_s \Sigmab_w) \le \min\cbr{\tr(\Sigmab_s)\nbr{\Sigmab_w}_2, \tr(\Sigmab_w)\nbr{\Sigmab_s}_2} \lesssim \min\cbr{\tr(\Sigmab_s), \tr(\Sigmab_w)}.
    \end{align*}
    In particular, when $\Sigmab_s$ and $\Sigmab_w$ admit low intrinsic dimensions, $\tr(\Sigmab_s \Sigmab_w)$ can be much smaller than $\min\cbr{\tr(\Sigmab_s), \tr(\Sigmab_w)}$ if their eigenvectors corresponding to the dominating eigenvalues are almost orthogonal.
\end{remark}

\begin{remark}[FT approximation errors]
    It is worth noting that under the ground truth and positive-definite covariance assumptions in \Cref{asm:ridge_regression}(i, iii), the FT approximation errors in \Cref{def:ft_est_err} satisfy
    \begin{align}\label{eq:pf_ridge_ft_approx_err}
    \begin{split}
        &\rho_s = \min_{\thetab \in \R^d} \E_{\xb \sim \Dcal}\sbr{(\phi_s(\xb)^\top \thetab - f_*(\xb))^2} = 0 \quad (\t{when } \thetab = \Sigmab_s^{-1} \Sigmab_* \thetab_*), \\
        &\rho_w = \min_{\thetab \in \R^d} \E_{\xb \sim \Dcal}\sbr{(\phi_w(\xb)^\top \thetab - f_*(\xb))^2} = 0 \quad (\t{when } \thetab = \Sigmab_w^{-1} \Sigmab_* \thetab_*).
    \end{split}
    \end{align}
    In place of \Cref{def:ft_est_err}, with positive-definite covariances in \Cref{asm:ridge_regression}, we measure the alignment between the ground truth feature $\phi_*$ and the weak/strong feature $\phi_w, \phi_s$ through
    \begin{align*}
        \varrho_s = \|\Sigmab_s^{-1/2} \Sigmab_*^{1/2} \thetab_*\|_2^2, \quad \varrho_w = \|\Sigmab_w^{-1/2} \Sigmab_*^{1/2} \thetab_*\|_2^2.
    \end{align*}
    Intuitively, for $\Sigmab_s$ and $\Sigmab_w$ with a few dominating eigenvalues (\Cref{asm:ridge_regression}(iii)), $\varrho_s$ and $\varrho_w$ are small if the eigensubspace associated with non-negligible eigenvalues of $\Sigmab_*$ is fully covered by the eigensubspaces associated with the dominating eigenvalues of $\Sigmab_s$ and $\Sigmab_w$, respectively. 
\end{remark}

The W2S FT under ridge regression consists of two steps.
\begin{enumerate}[label=(\alph*)]
    \item First, the weak teacher $f_w(\xb) = \phi_w(\xb)^\top \thetab_w$ is supervisedly finetuned over $\wt\Scal$: 
    \begin{align}\label{eq:w2s_weak_ridge}
        \thetab_w = \argmin_{\thetab \in \R^d} \frac{1}{n}\nbr{\wt\Phib_w \thetab - \wt\yb}_2^2 + \alpha_w \nbr{\thetab}_2^2, \quad \alpha_w > 0.
    \end{align}
    \item Then, the W2S model $f_\wts(\xb) = \phi_s(\xb)^\top \thetab_\wts$ is finetuned over the strong feature $\phi_s$ through the unlabeled samples $\Scal_x$ and their pseudo-labels generated by the weak teacher model:
    \begin{align}\label{eq:w2s_strong_ridge}
        \thetab_\wts = \argmin_{\thetab \in \R^d} \frac{1}{N}\nbr{\Phib_s \thetab - \Phib_w \thetab_w}_2^2 + \alpha_\wts \nbr{\thetab}_2^2, \quad \alpha_\wts > 0.
    \end{align}
\end{enumerate}

\begin{theorem}[W2S under ridge regression]\label{thm:w2s_ridge}
    Let $\varrho_w = \nbr{\Sigmab_w^{-1/2} \Sigmab_*^{1/2} \thetab_*}_2^2$ and $\varrho_s = \nbr{\Sigmab_s^{-1/2} \Sigmab_*^{1/2} \thetab_*}_2^2$.
    Under \Cref{asm:ridge_regression}, the generalization error of W2S FT via ridge regression with fixed $\alpha_w, \alpha_\wts > 0$, $\exrisk(f_\wts) = \vari(f_\wts) + \bias(f_\wts)$, is upper bounded by
    \begin{align*}
        \vari(f_\wts) \le \frac{\sigma^2 \tr\rbr{\Sigmab_s \Sigmab_w}}{4 (\alpha_w n) (\alpha_\wts N)}, \quad
        \bias(f_\wts) \le \alpha_w \varrho_w + \alpha_\wts \varrho_s.
    \end{align*}
    In particular, when taking  
    \begin{align*}
        \alpha_w = \rbr{\frac{\sigma^2 \tr\rbr{\Sigmab_s \Sigmab_w}}{4 n N}\ \frac{\varrho_s}{\varrho_w^2}}^{1/3}, \quad 
        \alpha_\wts = \rbr{\frac{\sigma^2 \tr\rbr{\Sigmab_s \Sigmab_w}}{4 n N}\ \frac{\varrho_w}{\varrho_s^2}}^{1/3},
    \end{align*}
    the excess risk of W2S FT is upper bounded by
    \begin{align*}
        \exrisk(f_\wts) \le 3 \rbr{\frac{\sigma^2 \tr\rbr{\Sigmab_s \Sigmab_w}}{4 n N}\ \varrho_s \varrho_w}^{1/3}.
    \end{align*}
\end{theorem}

\Cref{thm:w2s_ridge} conveys a similar high-level intuition as in \Cref{thm:w2s_ft} regarding the effect of the weak-strong similarity on the generalization error of W2S FT. In particular, the larger discrepancy between $\phi_s$ and $\phi_w$ (corresponding to the smaller $\tr\rbr{\Sigmab_s \Sigmab_w}$) leads to lower variance and therefore better W2S generalization.

Meanwhile, a key difference in W2S between the ridge and ridgeless settings (\Cref{thm:w2s_ridge} versus \Cref{thm:w2s_ft}) is that the FT approximation errors in \Cref{thm:w2s_ridge}, reflected by $\varrho_s = \|\Sigmab_s^{-1/2} \Sigmab_*^{1/2} \thetab_*\|_2^2$ and $\varrho_w = \|\Sigmab_w^{-1/2} \Sigmab_*^{1/2} \thetab_*\|_2^2$, can be compensated by larger sample sizes $n, N$ and directly affect the sample complexity: 
\begin{align*}
    n N \asymp \sigma^2 \tr\rbr{\Sigmab_s \Sigmab_w} \varrho_s \varrho_w.
\end{align*}
Such difference is a result of optimizing the regularization hyperparameters $\alpha_w, \alpha_\wts$ in ridge regression that control the variance-bias tradeoff.

\begin{proof}[Proof of \Cref{thm:w2s_ridge}]
    We first formalize some useful facts on the features and labels as in \eqref{eq:pf_var_w2s_subgaussian_asm}.
    In particular, the sub-gaussian assumption in \Cref{asm:ridge_regression}(ii) implies that for each $\xb \sim \Dcal$, the corresponding strong/weak feature $\phi_s(\xb), \phi_w(\xb) \in \R^d$, and the ground truth $f_*(\xb) \in \R$ are simultaneously characterized by an independent subgaussian random vector $\gammab \in \R^d$ with $\E[\gammab] = \b0_{d}$ and $\E[\gammab \gammab^\top] = \Ib_{d}$, \ie,
    \begin{align*}
        \phi_s(\xb) = \Sigmab_s^{1/2} \gammab, \quad \phi_w(\xb) = \Sigmab_w^{1/2} \gammab, \quad f_*(\xb) = \phi_*(\xb)^\top \thetab_* = \gammab^\top \Sigmab_*^{1/2} \thetab_*.
    \end{align*}

    Then, for $\Scal$ and $\wt\Scal$, there exist independent random matrices $\Gammab = [\gammab_1, \ldots, \gammab_N]^\top \in \R^{N \times d}$ and $\wt\Gammab = [\wt\gammab_1, \ldots, \wt\gammab_n]^\top \in \R^{n \times d}$ consisting of $\iid$ zero-mean isotropic rows such that
    \begin{align}\label{eq:pf_var_w2s_subgaussian_asm_2}
    \begin{split}
        &\Phib_s = \Gammab \Sigmab_s^{1/2} = \Gammab_s \Lambdab_s^{1/2} \Vb_s^\top, \\
        &\Phib_w = \Gammab \Sigmab_w^{1/2} = \Gammab_w \Lambdab_w^{1/2} \Vb_w^\top, \\
        &\yb = \fb_* + \zb, \quad \fb_* = \Gammab \Sigmab_*^{1/2} \thetab_*, \quad \zb \sim \Ncal(\b0_N, \sigma^2 \Ib_N), \\
        &\wt\Phib_w = \wt\Gammab \Sigmab_w^{1/2} = \wt\Gammab_w \Lambdab_w^{1/2} \Vb_w^\top, \\
        &\wt\yb = \wt\fb_* + \wt\zb, \quad \wt\fb_* = \wt\Gammab \Sigmab_*^{1/2} \thetab_*, \quad \wt\zb \sim \Ncal(\b0_n, \sigma^2 \Ib_n),
    \end{split}
    \end{align}
    where $\Gammab_s = \Gammab \Vb_s$, $\Gammab_w = \Gammab \Vb_w$, and $\wt\Gammab_w = \wt\Gammab \Vb_w$.

    \paragraph{Variance-bias decomposition.}
    Recall that the excess risk of W2S generalization $\exrisk(f_\wts)$ can be decomposed into the variance and bias terms:
    \begin{align*}
        &\vari(f_\wts) = \E_{\xb \sim \Dcal}\sbr{\E_{\Scal_x, \wt\Scal}\sbr{(f_\wts(\xb) - \E_{\Scal_x, \wt\Scal}[f_\wts(\xb)])^2}}, \\
        &\bias(f) = \E_{\xb \sim \Dcal}\sbr{(\E_{\Scal_x, \wt\Scal}[f_\wts(\xb)] - f_*(\xb))^2}.
    \end{align*}
    With $\alpha_w > 0$, \eqref{eq:w2s_weak_ridge} yields a weak teacher model $f_w(\xb) = \phi_w(\xb)^\top \thetab_w$ with 
    \begin{align*}
        \thetab_w = \rbr{\wt\Phib_w^\top \wt\Phib_w + \alpha_w n \Ib_d}^{-1} \wt\Phib_w^\top \rbr{\wt\fb_8 + \wt\zb}.
    \end{align*}
    Then, the W2S model $f_\wts(\xb) = \phi_s(\xb)^\top \thetab_\wts$ is given by \eqref{eq:w2s_strong_ridge} with $\alpha_\wts > 0$:
    \begin{align*}
        \thetab_\wts = &\rbr{\Phib_s^\top \Phib_s + \alpha_\wts N \Ib_d}^{-1} \Phib_s^\top \Phib_w \thetab_w \\
        = &\rbr{\Phib_s^\top \Phib_s + \alpha_\wts N \Ib_d}^{-1} \Phib_s^\top \Phib_w \rbr{\wt\Phib_w^\top \wt\Phib_w + \alpha_w n \Ib_d}^{-1} \wt\Phib_w^\top \rbr{\wt\fb_* + \wt\zb},
    \end{align*}
    which implies
    \begin{align*}
        \E_{\Scal_x, \wt\Scal}[\thetab_\wts] = \rbr{\Phib_s^\top \Phib_s + \alpha_\wts N \Ib_d}^{-1} \Phib_s^\top \Phib_w \rbr{\wt\Phib_w^\top \wt\Phib_w + \alpha_w n \Ib_d}^{-1} \wt\Phib_w^\top \wt\fb_*.
    \end{align*}
    Then, we can concretize the variance and bias terms as:
    \begin{align}\label{eq:pf_ridge_var}
    \begin{split}
        &\vari(f_\wts) = \E_{\xb \sim \Dcal}\sbr{\E_{\Scal_x, \wt\Scal}\sbr{(f_\wts(\xb) - \E_{\Scal_x, \wt\Scal}[f_\wts(\xb)])^2}} \\
        = &\E_{\Scal_x, \wt\Scal}\sbr{\nbr{\Sigmab_s^{1/2} \rbr{\Phib_s^\top \Phib_s + \alpha_\wts N \Ib_d}^{-1} \Phib_s^\top \Phib_w \rbr{\wt\Phib_w^\top \wt\Phib_w + \alpha_w n \Ib_d}^{-1} \wt\Phib_w^\top \wt\zb}_2^2},
    \end{split}
    \end{align}
    and
    \begin{align}\label{eq:pf_ridge_bias}
    \begin{split}
        &\bias(f_\wts) = \E_{\xb \sim \Dcal}\sbr{(\E_{\Scal_x, \wt\Scal}[f_\wts(\xb)] - f_*(\xb))^2} \\
        = &\E_{\Scal_x, \wt\Scal}\sbr{\frac{1}{N} \nbr{\Phib_s \rbr{\Phib_s^\top \Phib_s + \alpha_\wts N \Ib_d}^{-1} \Phib_s^\top \Phib_w \rbr{\wt\Phib_w^\top \wt\Phib_w + \alpha_w n \Ib_d}^{-1} \wt\Phib_w^\top \wt\fb_* - \fb_*}_2^2}.
    \end{split}
    \end{align}
    Now, we are ready to upper bound the variance and bias terms separately.

    \paragraph{Variance.}
    Denote $\zetab = \Lambdab_w^{1/2} \Vb_w^\top \rbr{\wt\Phib_w^\top \wt\Phib_w + \alpha_w n \Ib_d}^{-1} \wt\Phib_w^\top \wt\zb \in \R^d$, whose randomness comes from $\wt\Scal$ only, independent of $\Scal_x$.
    Then, the variance term \eqref{eq:pf_ridge_var} can be expressed as
    \begin{align*}
        &\vari(f_\wts) = \E_{\Scal_x, \wt\Scal}\sbr{\nbr{\Sigmab_s^{1/2} \rbr{\Phib_s^\top \Phib_s + \alpha_\wts N \Ib_d}^{-1} \Phib_s^\top \Phib_w \zetab}_2^2} \\
        = &\tr\rbr{\E_{\Scal_s}\rbr{\Gammab_w^\top \Phib_s \rbr{\Phib_s^\top \Phib_s + \alpha_\wts N \Ib_d}^{-1} \Sigmab_s \rbr{\Phib_s^\top \Phib_s + \alpha_\wts N \Ib_d}^{-1} \Phib_s^\top \Gammab_w} \E_{\wt\Scal}\sbr{\zetab \zetab^\top}} \\
        = &\tr\rbr{\E_{\Scal_s}\rbr{\Gammab_w^\top \Gammab_s \rbr{\Gammab_s^\top \Gammab_s + \alpha_\wts N \Lambdab_s^{-1}}^{-1} \rbr{\Gammab_s^\top \Gammab_s + \alpha_\wts N \Lambdab_s^{-1}}^{-1} \Gammab_s^\top \Gammab_w} \E_{\wt\Scal}\sbr{\zetab \zetab^\top}} \\
        = &\tr\rbr{\E_{\Scal_s}\rbr{\Vb_w^\top \Gammab^\top \Gammab \rbr{\Gammab^\top \Gammab + \alpha_\wts N \Sigmab_s^{-1}}^{-2} \Gammab^\top \Gammab \Vb_w} \E_{\wt\Scal}\sbr{\zetab \zetab^\top}} \\
        = &\tr\rbr{\E_{\Scal_s}\rbr{\Gammab^\top \Gammab \rbr{\Gammab^\top \Gammab + \alpha_\wts N \Sigmab_s^{-1}}^{-2} \Gammab^\top \Gammab} \E_{\wt\Scal}\sbr{\Vb_w \zetab \zetab^\top \Vb_w^\top}}.
    \end{align*}
    Notice that $\rbr{\Gammab^\top \Gammab + \alpha_\wts N \Sigmab_s^{-1}}^{2} \succeq \alpha_\wts N \rbr{\Gammab^\top \Gammab \Sigmab_s^{-1} + \Sigmab_s^{-1} \Gammab^\top \Gammab}$.
    Since matrix inversion is convex, a Jensen-type inequality implies that
    \begin{align*}
        &\Gammab^\top \Gammab \rbr{\Gammab^\top \Gammab + \alpha_\wts N \Sigmab_s^{-1}}^{-2} \Gammab^\top \Gammab \\
        \preceq &\Gammab^\top \Gammab \rbr{\alpha_\wts N \rbr{\Gammab^\top \Gammab \Sigmab_s^{-1} + \Sigmab_s^{-1} \Gammab^\top \Gammab}}^{\dagger} \Gammab^\top \Gammab \\
        = &\frac{1}{2 \alpha_\wts N} \Gammab^\top \Gammab \rbr{\frac{1}{2} \rbr{\Gammab^\top \Gammab \Sigmab_s^{-1} + \Sigmab_s^{-1} \Gammab^\top \Gammab}}^{\dagger} \Gammab^\top \Gammab \\
        \preceq &\frac{1}{4 \alpha_\wts N} \rbr{\Gammab^\top \Gammab \Sigmab_s + \Sigmab_s \Gammab^\top \Gammab}.
    \end{align*}
    Therefore, 
    \begin{align*}
        \E_{\Scal_s}\rbr{\Gammab^\top \Gammab \rbr{\Gammab^\top \Gammab + \alpha_\wts N \Sigmab_s^{-1}}^{-2} \Gammab^\top \Gammab}
        \preceq &\frac{1}{4 \alpha_\wts N} \E_{\Scal_s}\sbr{\Gammab^\top \Gammab \Sigmab_s + \Sigmab_s \Gammab^\top \Gammab} 
        = \frac{1}{2 \alpha_\wts N} \Sigmab_s.
    \end{align*}
    Meanwhile, we observe that
    \begin{align*}
        \E_{\wt\Scal}\sbr{\Vb_w \zetab \zetab^\top \Vb_w^\top} 
        = &\E_{\wt\Scal}\sbr{\Sigmab_w^{1/2} \rbr{\wt\Phib_w^\top \wt\Phib_w + \alpha_w n \Ib_d}^{-1} \wt\Phib_w^\top \wt\zb \wt\zb^\top \wt\Phib_w \rbr{\wt\Phib_w^\top \wt\Phib_w + \alpha_w n \Ib_d}^{-1} \Sigmab_w^{1/2}} \\
        = &\sigma^2 \E_{\wt\Scal}\sbr{\Sigmab_w^{1/2} \rbr{\wt\Phib_w^\top \wt\Phib_w + \alpha_w n \Ib_d}^{-1} \wt\Phib_w^\top \wt\Phib_w \rbr{\wt\Phib_w^\top \wt\Phib_w + \alpha_w n \Ib_d}^{-1} \Sigmab_w^{1/2}},
    \end{align*}
    where 
    \begin{align*}
        \rbr{\wt\Phib_w^\top \wt\Phib_w + \alpha_w n \Ib_d}^{-1} \wt\Phib_w^\top \wt\Phib_w \rbr{\wt\Phib_w^\top \wt\Phib_w + \alpha_w n \Ib_d}^{-1}
        \preceq &\frac{1}{2 \alpha_w n} \Ib_d.
    \end{align*}
    Therefore, we have
    \begin{align*}
        \E_{\wt\Scal}\sbr{\Vb_w \zetab \zetab^\top \Vb_w^\top} 
        \preceq &\sigma^2 \E_{\wt\Scal}\sbr{\Sigmab_w^{1/2} \rbr{\frac{1}{2 \alpha_w n} \Ib_d} \Sigmab_w^{1/2}}
        = \frac{\sigma^2}{2 \alpha_w n} \Sigmab_w.
    \end{align*}
    Overall, the variance of $f_\wts$ can be upper bounded as
    \begin{align}\label{eq:pf_ridge_var_ub}
    \begin{split}
        \vari(f_\wts) 
        = &\tr\rbr{\E_{\Scal_s}\rbr{\Gammab^\top \Gammab \rbr{\Gammab^\top \Gammab + \alpha_\wts N \Sigmab_s^{-1}}^{-2} \Gammab^\top \Gammab} \E_{\wt\Scal}\sbr{\Vb_w \zetab \zetab^\top \Vb_w^\top}} \\
        \le &\frac{\sigma^2 \tr\rbr{\Sigmab_s \Sigmab_w}}{4 (\alpha_w n) (\alpha_\wts N)}.
    \end{split}
    \end{align}

    \paragraph{Bias.}
    Let $\xib = \Sigmab_w^{1/2} \rbr{\wt\Phib_w^\top \wt\Phib_w + \alpha_w n \Ib_d}^{-1} \wt\Phib_w^\top \wt\fb_* \in \R^d$, whose randomness comes from $\wt\Scal$ only, independent of $\Scal_x$.
    Recall from \eqref{eq:pf_ridge_bias}, the bias term \eqref{eq:pf_ridge_bias} can be decomposed as
    \begin{align*}
        &\bias(f_\wts) = \E_{\Scal_x, \wt\Scal}\sbr{\frac{1}{N} \nbr{\Phib_s \rbr{\Phib_s^\top \Phib_s + \alpha_\wts N \Ib_d}^{-1} \Phib_s^\top \Phib_w \rbr{\wt\Phib_w^\top \wt\Phib_w + \alpha_w n \Ib_d}^{-1} \wt\Phib_w^\top \wt\fb_* - \fb_*}_2^2}\\
        &= \E_{\Scal_x, \wt\Scal}\sbr{\frac{1}{N} \rbr{\nbr{\Phib_s \rbr{\Phib_s^\top \Phib_s + \alpha_\wts N \Ib_d}^{-1} \Phib_s^\top \Gammab \xib - \Phib_s \Phib_s^\dagger \fb_*}_2^2 + \nbr{\rbr{\Ib_N - \Phib_s \Phib_s^\dagger} \fb_*}_2^2}},
    \end{align*}
    where by \Cref{lem:low_est_err_ft} and \eqref{eq:pf_ridge_ft_approx_err}
    \begin{align*}
        \E_{\Scal_x}\sbr{\frac{1}{N} \nbr{\rbr{\Ib_N - \Phib_s \Phib_s^\dagger} \fb_*}_2^2}
        = \frac{\rho_s(N)}{N} \le \rho_s = 0.
    \end{align*}
    Therefore, with $\xib = \Sigmab_w^{1/2} \rbr{\wt\Phib_w^\top \wt\Phib_w + \alpha_w n \Ib_d}^{-1} \wt\Phib_w^\top \wt\fb_*$, we have
    \begin{align*}
        \bias(f_\wts) = \E_{\Scal_x, \wt\Scal}\sbr{\frac{1}{N} \nbr{\Phib_s \rbr{\Phib_s^\top \Phib_s + \alpha_\wts N \Ib_d}^{-1} \Phib_s^\top \Gammab \xib - \Phib_s \Phib_s^\dagger \fb_*}_2^2}.
    \end{align*}
    Recall that $\fb_* = \Gammab \Sigmab_*^{1/2} \thetab_*$ and $\Phib_s = \Gammab \Sigmab_s^{1/2} = \Gammab_s \Lambdab_s^{1/2} \Vb_s^\top$.
    Then, we can express the bias term as
    \begin{align*}
        \bias(f_\wts) = &\E_{\Scal_x, \wt\Scal}\sbr{\frac{1}{N} \nbr{\Gammab\rbr{\Gammab^\top \Gammab + \alpha_\wts N \Sigmab_s^{-1}}^{-1} \Gammab^\top \Gammab \xib - \Gammab \Gammab^\dagger \fb_*}_2^2} \\
        = &\E_{\Scal_x, \wt\Scal}\sbr{\frac{1}{N} \nbr{\Gammab \Sigmab_*^{1/2} \thetab_* - \Gammab\rbr{\Gammab^\top \Gammab + \alpha_\wts N \Sigmab_s^{-1}}^{-1} \Gammab^\top \Gammab \xib}_2^2} \\
        = &\E_{\Scal_x, \wt\Scal}\sbr{\frac{1}{N} \nbr{\Gammab \rbr{\Sigmab_*^{1/2} \thetab_* - \xib} + \Gammab \rbr{\Ib_d - \rbr{\Gammab^\top \Gammab + \alpha_\wts N \Sigmab_s^{-1}}^{-1} \Gammab^\top \Gammab} \xib}_2^2} \\
    \end{align*} 
    By Woodbury matrix identity, we have
    \begin{align}\label{eq:pf_ridge_bias_woodbury}
        \Ib_d - \rbr{\Gammab^\top \Gammab + \alpha_\wts N \Sigmab_s^{-1}}^{-1} \Gammab^\top \Gammab
        = \rbr{\Ib_d + \frac{1}{\alpha_\wts N} \Sigmab_s \Gammab^\top \Gammab}^{-1}.
    \end{align}
    Therefore, we have 
    \begin{align}\label{eq:pf_ridge_bias_inter1}
        \bias(f_\wts) = \E_{\Scal_x, \wt\Scal}\Bigg[\frac{1}{N} \Big\|\underbrace{\Gammab \rbr{\Sigmab_*^{1/2} \thetab_* - \xib}}_{\t{Term A}} + \underbrace{\Gammab \rbr{\Ib_d + \frac{1}{\alpha_\wts N} \Sigmab_s \Gammab^\top \Gammab}^{-1} \xib}_{\t{Term B}}\Big\|_2^2\Bigg].
    \end{align}

    For Term A, notice that $\xib = \Sigmab_w^{1/2} \rbr{\wt\Phib_w^\top \wt\Phib_w + \alpha_w n \Ib_d}^{-1} \wt\Phib_w^\top \wt\fb_*$ implies
    \begin{align*}
        \Sigmab_*^{1/2} \thetab_* - \xib 
        = &\Sigmab_*^{1/2} \thetab_* - \Sigmab_w^{1/2} \rbr{\wt\Phib_w^\top \wt\Phib_w + \alpha_w n \Ib_d}^{-1} \wt\Phib_w^\top \wt\fb_* \\
        = &\Sigmab_*^{1/2} \thetab_* - \rbr{\wt\Gammab^\top \wt\Gammab + \alpha_w n \Sigmab_w^{-1}}^{-1} \wt\Gammab^\top \wt\Gammab \Sigmab_*^{1/2} \thetab_* \\
        = &\rbr{\Ib_d - \rbr{\wt\Gammab^\top \wt\Gammab + \alpha_w n \Sigmab_w^{-1}}^{-1} \wt\Gammab^\top \wt\Gammab} \Sigmab_*^{1/2} \thetab_* \\
        = &\rbr{\Ib_d + \frac{1}{\alpha_w n} \Sigmab_w \wt\Gammab^\top \wt\Gammab}^{-1} \Sigmab_*^{1/2} \thetab_*,
    \end{align*}
    where the last equality follows from Woodbury matrix identity as in \eqref{eq:pf_ridge_bias_woodbury}.
    Therefore,
    \begin{align*}
        \E_{\Scal_x, \wt\Scal}\sbr{\frac{1}{N} \nbr{\Gammab \rbr{\Sigmab_*^{1/2} \thetab_* - \xib}}_2^2} 
        = &\E_{\wt\Scal}\sbr{\frac{1}{n} \nbr{\wt\Gammab \rbr{\Sigmab_*^{1/2} \thetab_* - \xib}}_2^2} \\
        = &\E_{\wt\Scal}\sbr{\frac{1}{n} \nbr{\wt\Gammab \rbr{\Ib_d + \frac{1}{\alpha_w n} \Sigmab_w \wt\Gammab^\top \wt\Gammab}^{-1} \Sigmab_*^{1/2} \thetab_*}_2^2}.
    \end{align*}
    Since 
    \begin{align*}
        \rbr{\Ib_d + \frac{1}{\alpha_w n} \Sigmab_w \wt\Gammab^\top \wt\Gammab}^{-1} \wt\Gammab^\top \wt\Gammab \rbr{\Ib_d + \frac{1}{\alpha_w n} \Sigmab_w \wt\Gammab^\top \wt\Gammab}^{-1} \preceq \frac{\alpha_w n}{2} \Sigmab_w^{-1},
    \end{align*}
    we have
    \begin{align}\label{eq:pf_ridge_bias_term1}
    \begin{split}
        \E_{\Scal_x, \wt\Scal}\sbr{\frac{1}{N} \nbr{\Gammab \rbr{\Sigmab_*^{1/2} \thetab_* - \xib}}_2^2} 
        \le &\frac{1}{n} \tr\rbr{\frac{\alpha_w n}{2} \Sigmab_w^{-1} \Sigmab_*^{1/2} \thetab_* \thetab_*^\top \Sigmab_*^{1/2}} \\
        = &\frac{\alpha_w}{2} \nbr{\Sigmab_w^{-1/2} \Sigmab_*^{1/2} \thetab_*}_2^2.
    \end{split}
    \end{align}
    
    For Term B, leveraging Woodbury matrix identity as in \eqref{eq:pf_ridge_bias_woodbury}, we notice that 
    \begin{align*}
        &\E_{\Scal_x, \wt\Scal}\sbr{\frac{1}{N} \nbr{\Gammab \rbr{\Ib_d + \frac{1}{\alpha_\wts N} \Sigmab_s \Gammab^\top \Gammab}^{-1} \xib}_2^2} 
        \le \E_{\Scal_x, \wt\Scal}\sbr{\frac{1}{N} \tr\rbr{\frac{\alpha_\wts N}{2} \Sigmab_s^{-1} \xib \xib^\top}} \\
        = &\frac{\alpha_\wts}{2} \E_{\Scal_x, \wt\Scal}\sbr{\nbr{\Sigmab_s^{-1/2} \Sigmab_w^{1/2} \rbr{\wt\Phib_w^\top \wt\Phib_w + \alpha_w n \Ib_d}^{-1} \wt\Phib_w^\top \wt\fb_*}_2^2} \\
        = &\frac{\alpha_\wts}{2} \E_{\Scal_x, \wt\Scal}\sbr{\nbr{\Sigmab_s^{-1/2} \rbr{\wt\Gammab^\top \wt\Gammab + \alpha_w n \Sigmab_w^{-1}}^{-1} \wt\Gammab^\top \wt\Gammab \Sigmab_*^{1/2} \thetab_*}_2^2}
    \end{align*}
    Since $\rbr{\wt\Gammab^\top \wt\Gammab + \alpha_w n \Sigmab_w^{-1}}^{-1} \wt\Gammab^\top \wt\Gammab \preceq \Ib_d$, we know that
    \begin{align}\label{eq:pf_ridge_bias_term2}
    \begin{split}
        \E_{\Scal_x, \wt\Scal}\sbr{\frac{1}{N} \nbr{\Gammab \rbr{\Ib_d + \frac{1}{\alpha_\wts N} \Sigmab_s \Gammab^\top \Gammab}^{-1} \xib}_2^2} 
        \le \frac{\alpha_\wts}{2} \nbr{\Sigmab_s^{-1/2} \Sigmab_*^{1/2} \thetab_*}_2^2.
    \end{split}
    \end{align}
    Combining \eqref{eq:pf_ridge_bias_inter1}, \eqref{eq:pf_ridge_bias_term1}, and \eqref{eq:pf_ridge_bias_term2}, we can upper bound the bias term as
    \begin{align}\label{eq:pf_ridge_bias_final}
    \begin{split}
        &\bias(f_\wts) = \E_{\Scal_x, \wt\Scal}\Bigg[\frac{1}{N} \Big\|\underbrace{\Gammab \rbr{\Sigmab_*^{1/2} \thetab_* - \xib}}_{\t{Term A}} + \underbrace{\Gammab \rbr{\Ib_d + \frac{1}{\alpha_\wts N} \Sigmab_s \Gammab^\top \Gammab}^{-1} \xib}_{\t{Term B}}\Big\|_2^2\Bigg] \\
        \le &2 \E_{\Scal_x, \wt\Scal}\sbr{\frac{1}{N} \nbr{\Gammab \rbr{\Sigmab_*^{1/2} \thetab_* - \xib}}_2^2} + 2 \E_{\Scal_x, \wt\Scal}\sbr{\frac{1}{N} \nbr{\Gammab \rbr{\Ib_d + \frac{1}{\alpha_\wts N} \Sigmab_s \Gammab^\top \Gammab}^{-1} \xib}_2^2} \\
        \le &\alpha_w \nbr{\Sigmab_w^{-1/2} \Sigmab_*^{1/2} \thetab_*}_2^2 + \alpha_\wts \nbr{\Sigmab_s^{-1/2} \Sigmab_*^{1/2} \thetab_*}_2^2.
    \end{split}
    \end{align}
    
    \paragraph{Variance-bias tradeoff.}
    Overall, by \eqref{eq:pf_ridge_var_ub} and \eqref{eq:pf_ridge_bias_final}, we have
    \begin{align*}
        &\vari(f_\wts) \le \frac{\sigma^2 \tr\rbr{\Sigmab_s \Sigmab_w}}{4 (\alpha_w n) (\alpha_\wts N)}, \\
        &\bias(f_\wts) \le \alpha_w \nbr{\Sigmab_w^{-1/2} \Sigmab_*^{1/2} \thetab_*}_2^2 + \alpha_\wts \nbr{\Sigmab_s^{-1/2} \Sigmab_*^{1/2} \thetab_*}_2^2.
    \end{align*}
    The upper bound the excess risk $\exrisk(f_\wts) = \vari(f_\wts) + \bias(f_\wts)$ is minimized by taking 
    \begin{align*}
        \alpha_w = \rbr{\frac{\sigma^2 \tr\rbr{\Sigmab_s \Sigmab_w}}{4 n N}\ \frac{\nbr{\Sigmab_s^{-1/2} \Sigmab_*^{1/2} \thetab_*}_2^2}{\nbr{\Sigmab_w^{-1/2} \Sigmab_*^{1/2} \thetab_*}_2^4}}^{1/3}, \ 
        \alpha_\wts = \rbr{\frac{\sigma^2 \tr\rbr{\Sigmab_s \Sigmab_w}}{4 n N}\ \frac{\nbr{\Sigmab_w^{-1/2} \Sigmab_*^{1/2} \thetab_*}_2^2}{\nbr{\Sigmab_s^{-1/2} \Sigmab_*^{1/2} \thetab_*}_2^4}}^{1/3},
    \end{align*}
    which leads to the optimal upper bound for the excess risk:
    \begin{align*}
        \exrisk(f_\wts) \le 3 \rbr{\frac{\sigma^2 \tr\rbr{\Sigmab_s \Sigmab_w}}{4 n N}\ \nbr{\Sigmab_s^{-1/2} \Sigmab_*^{1/2} \thetab_*}_2^2 \nbr{\Sigmab_w^{-1/2} \Sigmab_*^{1/2} \thetab_*}_2^2}^{1/3}.
    \end{align*}
\end{proof}






\section{Canonical angles}\label{apx:canonical_angles}
In this section, we review the concept of canonical angles between two subspaces that connect the formal definition of the correlation dimension $d_{s \wedge w} = \nbr{\Vb_s^\top \Vb_w}_F^2$ in \Cref{def:correlation_dim} to the intuitive notion of the alignment between the corresponding subspaces $\Vcal_s$ and $\Vcal_w$ in the introduction: $\sum \cos(\angle(\Vcal_s, \Vcal_w)) = \nbr{\Vb_s^\top \Vb_w}_F^2$.
\begin{definition}[Canonical angles \cite{golub2013matrix}, adapting from \cite{dong2024efficient}]\label{def:canonical_angles}
    Let $\Vcal_s,\Vcal_w \subseteq \R^d$ be two subspaces with dimensions $\dim\rbr{\Vcal_s}=d_s$ and $\dim\rbr{\Vcal_w}=d_w$ (assuming $d_w \geq d_s$ without loss of generality). The canonical angles $\angle\rbr{\Vcal_s,\Vcal_w}=\diag\rbr{\nu_1,\dots,\nu_{d_s}}$ are $d_s$ angles that jointly measure the alignment between $\Vcal_s$ and $\Vcal_w$, defined recursively as follows:
    \begin{align*}
        &\ub_i, \vb_i ~\triangleq~
        \argmax~\ub_i^*\vb_i \\
        \t{s.t.}~
        &\ub_i \in \rbr{\Vcal_s \setminus \spn\cbr{\ub_{\iota}}_{\iota=1}^{i-1}} \cap \SSS^{d-1},\\ 
        &\vb_i \in \rbr{\Vcal_w \setminus \spn\cbr{\vb_{\iota}}_{\iota=1}^{i-1}} \cap \SSS^{d-1}\\
        &\cos (\nu_i) = \ub_i^* \vb_i \quad \forall~ i=1,\dots,k,
    \end{align*}
    such that $0 \leq \nu_1 \leq \dots \leq \nu_k \leq \pi/2$.

    Given two subspaces $\Vcal_s,\Vcal_w \subseteq \R^d$, let $\Vb_s \in \R^{d \times d_s}$ and $\Vb_w \in \R^{d \times d_w}$ be the matrices whose columns form orthonormal bases for $\Vcal_s$ and $\Vcal_w$, respectively. Then, the canonical angles $\angle(\Vcal_s, \Vcal_w)$ are determined by the singular values of $\Vb_s^\top \Vb_w$~\citep[\S 3]{bjorck1973numerical}:
    \begin{align*}
        \cos(\angle_i(\Vcal_s, \Vcal_w)) = \sigma_i(\Vb_s^\top \Vb_w) \quad \forall~ i=1,\dots,d_s,
    \end{align*}
    where $\sigma_i(\Vb_s^\top \Vb_w)$ denotes the $i$-th singular value of $\Vb_s^\top \Vb_w$.
\end{definition}

In particular, since $\Vb_s, \Vb_w$ consist of orthonormal columns, the singular values of $\Vb_s^\top \Vb_w$ fall in $[0,1]$, and therefore,
\begin{align*}
    d_{s \wedge w} = \sum \cos(\angle(\Vcal_s, \Vcal_w)) = \nbr{\Vb_s^\top \Vb_w}_F^2 \in [0, \min\cbr{d_s, d_w}].
\end{align*}




\section{Additional experiments}\label{apx:exp_details}

\subsection{Additional experiments and details on UTKFace regression}\label{apx:exp_img_reg}
This section provides some additional details and results for the UTKFace regression experiments in \Cref{sec:exp_img_reg}. 

\begin{figure}[!h]
    \centering
    \includegraphics[width=\columnwidth]{fig/mse_utkface_resnet18_clipb32.pdf}%\vspace{-2em}
    \caption{Scaling for MSE on UTKFace with \texttt{CLIP-B32} as the strong student and \texttt{ResNet18} as the weak teacher}\label{fig:mse_utkface_resnet18-clip}
\end{figure}

\begin{figure}[!h]
    \centering
    \includegraphics[width=\columnwidth]{fig/mse_utkface_resnet50_clipb32.pdf}%\vspace{-2em}
    \caption{Scaling for MSE on UTKFace with \texttt{CLIP-B32} as the strong student and \texttt{ResNet50} as the weak teacher}\label{fig:mse_utkface_resnet50-clip}
\end{figure}

\begin{figure}[!h]
    \centering
    \includegraphics[width=\columnwidth]{fig/mse_utkface_resnet152_clipb32.pdf}%\vspace{-2em}
    \caption{Scaling for MSE on UTKFace with \texttt{CLIP-B32} as the strong student and \texttt{ResNet152} as the weak teacher}\label{fig:mse_utkface_resnet152-clip}
\end{figure}

We summarize the relevant dimensionality in \Cref{tab:img_reg_dim}. We observe the following:
\begin{itemize}
    \item The intrinsic dimensions of the pretrained features are significantly smaller than the ambiance feature dimensions, which is consistent with our theoretical analysis and the empirical observations in \cite{aghajanyan2020intrinsic}. 
    \item The correlation dimensions $d_{s \wedge w}$ are considerably smaller than the corresponding intrinsic dimensions, indicating that the subspaces spanned by the weak and strong features are not aligned in practice. As shown in \Cref{sec:exp_img_reg}, such discrepancies in the weak and strong features facilitate W2S generalization.
\end{itemize}

\begin{table}[!ht]
    \centering
    \caption{Summary of the pretrained feature dimensions, along with the intrinsic dimensions $d_s, d_w$ and correlation dimensions $d_{s \wedge w}$ (with respect to the strong student \texttt{CLIP-B32}) computed over the entire UTKFace dataset (including training and testing).}\label{tab:img_reg_dim}
    \begin{tabular}{c|ccc}
        \toprule
        Pretrained Model & Feature Dimension & Intrinsic Dimension ($\tau=0.01$) & Correlation Dimension \\
        \midrule
        \texttt{ResNet18} & 512 & 194 & 167.64 \\
        \texttt{ResNet34} & 512 & 150 & 129.97 \\
        \texttt{ResNet50} & 2048 & 522 & 301.06 \\
        \texttt{ResNet101} & 2048 & 615 & 354.52 \\
        \texttt{ResNet152} & 2048 & 589 & 339.90 \\
        \midrule
        \texttt{CLIP-B32} & 768 & 443 & $\times$ \\
        \bottomrule
    \end{tabular}
\end{table}

For reference, we provide the scaling for MSE losses of three representative teacher-student pairs in \Cref{fig:mse_utkface_resnet18-clip,fig:mse_utkface_resnet50-clip,fig:mse_utkface_resnet152-clip}. 
\begin{itemize}
    \item It is worth highlighting that while the MSE loss of $f_\wts$ monotonically decreases with respect to both sample sizes $n,N$, the different rates of convergence compared to $f_w$, $f_s$, and $f_c$ lead to the distinct scaling behavior of the relative W2S performance ($\pgr$ and $\opr$) with respect to $n$ versus $N$ in \Cref{fig:pgr_opr_utkface_resnet-clip,fig:pgr_opr_utkface_vardom_resnet-clip}.
    \item When the strong student has a lower intrinsic dimension than the weak teacher (\cf \Cref{fig:mse_utkface_resnet18-clip} versus \Cref{fig:mse_utkface_resnet50-clip,fig:mse_utkface_resnet152-clip}), $d_s < d_w$, the W2S model $f_\wts$ tends to achieve better generalization in terms of the test MSE. This is consistent with our analysis in \Cref{sec:generalization_errors}.
    \item When $d_s < d_w$, the W2S model $f_\wts$ tends to achieve (slightly) better generalization for (slightly) smaller correlation dimension $d_{s \wedge w}$ (\cf \Cref{fig:mse_utkface_resnet50-clip} versus \Cref{fig:mse_utkface_resnet152-clip}), again coinciding with our analysis in \Cref{sec:generalization_errors}.
    \item W2S generalization generally happens (\ie $f_\wts$ is able to outperform $f_w$) with sufficiently large sample sizes $n, N$. However, as the labeled sample size $n$ increases, the test MSE of $f_\wts$ converges slower than that of the strong baseline and ceiling models, $f_s$ and $f_c$, leading to the inverse scaling for $\pgr$ and $\opr$ with respect to $n$ in \Cref{fig:pgr_opr_utkface_resnet-clip,fig:pgr_opr_utkface_vardom_resnet-clip}. When $n$ is too large, the W2S model $f_\wts$ may not be able to achieve better generalization than the strong baseline $f_s$.
\end{itemize}




\subsection{Experiments on image classification}\label{apx:exp_img_cls}

\paragraph{Dataset.} ColoredMNIST \citep{arjovsky2019invariant} consists of groups of different colors and reassign the label to be binary (digits 0-4 vs 5-9). We pool together the groups to form one dataset. The choice is to bring diversity to the feature space with additional color features and thus potential feature discrepancies. We hold out a test set of 7000 samples and used the rest 63000 samples for training.

\paragraph{Linear probing over pretrained features.} We fix a strong student as DINOv2-s14 \citep{oquab2023dinov2} and vary the weak teacher among the ResNet-d series and ResNet series (ResNet18D, ResNet34D, ResNet101, ResNet152) \citep{he2018resnetd,he2015deepresiduallearningimage}. We replace ResNet18 and ResNet34 used in \Cref{sec:exp_img_reg} to experiment on weak models with similar intrinsic dimensions but different correlation dimensions. We treat the backbone of the models (excluding the classification layer) as $\phi_s$ and $\phi_w$ and finetune them via linear probing. We train the models with cross entropy loss and AdamW optimizer. We tune the training hyperparameters of weak and strong models using a validation set and train them for 800 steps with learning rate 1e-3 and weight decay 1e-6. 

\begin{table}[!ht]
    \centering
    \caption{Summary of the pretrained feature dimensions, along with the intrinsic dimensions $d_s, d_w$ and correlation dimensions $d_{s \wedge w}$ (with respect to the strong student \texttt{DINOv2-S14}) computed over the entire ColoredMNIST dataset (including training and testing).}\label{tab:img_cls_dim_coloredmnist}
    \begin{tabular}{c|ccc}
        \toprule
        Pretrained Model & Feature Dimension & Intrinsic Dimension ($\tau=0.01$) & Correlation Dimension \\
        \midrule
        \texttt{ResNet-18-D} & 512 & 117 & 6.23 \\
        \texttt{ResNet-34-D} & 512 & 127 & 7.07 \\
        \texttt{ResNet101} & 2048 & 121 & 1.74 \\
        \texttt{ResNet152} & 2048 & 128 & 1.88 \\
        \midrule
        \texttt{DINOv2-S14} & 384 & 28 & $\times$ \\
        \bottomrule
    \end{tabular}
\end{table}

\begin{figure}[!h]
    \centering
    \includegraphics[width=\columnwidth]{fig/coloredmnist_lp/coloredmnist_dsw.pdf}%\vspace{-2em}
    \caption{Scaling for $\pgr$ and $\opr$ of different weak teachers with a fixed strong student on ColoredMNIST.}\label{fig:coloredmnist_dscapw}
\end{figure}

\begin{figure}[!h]
    \centering
    \includegraphics[width=\columnwidth]{fig/coloredmnist_lp/coloredmnist_var.pdf}%\vspace{-2em}
    \caption{Scaling for $\pgr$ and $\opr$ of W2S on ColoredMNIST with injected label noise.}\label{fig:coloredmnist_variance}
\end{figure}

\paragraph{Discrepancies lead to better W2S.}
\Cref{fig:coloredmnist_dscapw} shows the scaling of $\pgr$ and $\opr$ with respect to the sample sizes $n, N$ for different weak teachers in the ResNet series with respect to a fixed student, \texttt{CLIP-B32}. 
As in \Cref{sec:exp_img_reg}, we observe that with similar intrinsic dimensions $d_s, d_w$, W2S achieves better relative performance in terms of $\pgr$ and $\opr$ when the correlation dimension $d_{s \wedge w}$ is smaller.

\paragraph{Variance reduction is a key advantage of W2S.}
We inject noise to the labels of the original ColoredMNIST training samples by randomly flipping the ground truth labels with probability $\varsigma \in [0,1]$ (following \cite{arjovsky2019invariant}). 
\Cref{fig:coloredmnist_variance} shows the scaling of $\pgr$ and $\opr$ with respect to $n$ and $N$ when taking DINOv2-S14 as the strong student and ResNet101 as the weak teacher. We observe that the larger artificial label noise $\varsigma$ leads to higher $\pgr$ and $\opr$. 

\subsection{Experiments on sentiment classification}\label{apx:exp_nlp_cls}

\paragraph{Dataset.} The Stanford Sentiment Treebank \citep{socher-etal-2013-sst2} is a corpus with fully labeled parse trees that allows for a complete analysis of the compositional effects of sentiment in language. The corpus is based on the dataset introduced by \citet{pang-lee-2005-sst_original_corpus} and consists of 11,855 single sentences extracted from movie reviews. It was parsed with the Stanford parser and includes a total of 215,154 unique phrases from those parse trees, each annotated by 3 human judges. We conduct binary classification experiments on full sentences (negative or somewhat negative vs somewhat positive or positive with neutral sentences discarded), the so-called SST-2 dataset, and split the dataset into training and testing sets of sizes 63000 and 4349. Generalization errors are estimated with the 0-1 loss over the test set.

\paragraph{Full finetuning.} We fix the strong student as Electra-base-discriminator \citep{clark2020electra} and vary the weak teacher among the Bert series \citep{turc2019bert-tiny} (Bert-Tiny, Bert-Mini, Bert-Small, Bert-Medium). 
With manageable model sizes, we conduct full finetuning experiments following the setup in \cite{burns2023weak}.
We use the standard cross entropy loss for supervised finetuning. 
When training strong students on weak labels (W2S), we use the confidence weighted loss proposed by \cite{burns2023weak}, which is suggested to be able to improve weak-to-strong generalization on many NLP tasks.
All training is conducted via Adam optimizers~\citep{kingma2014adam} with a learning rate of 5e-5, a cosine learning rate schedule, and 40 warmup steps. We train for 3 epochs, which is sufficient for the train and validation loss to stabilize. 

\paragraph{Intrinsic dimension.} The intrinsic dimensions $d_w,d_s$ are measured based on the Structure-Aware Intrinsic Dimension (SAID) method proposed by \cite{aghajanyan2020intrinsic}. We first train the full models on the whole training set, and then train the models with only $d$ trainable parameters based on SAID transformation. The $d_w$ or $d_s$ are the smallest number of parameters under SAID that is necessary to retain 90\% accuracy of the full models. Here, the 90\% accuracy is a common threshold used to estimate intrinsic dimensions in the literature \citep{li2018measuring}.

\begin{figure}[!h]
    \centering
    \includegraphics[width=\columnwidth]{fig/sst2/sst2-dsw.pdf}%\vspace{-2em}
    \caption{Scaling for $\pgr$ and $\opr$ of different weak teachers with a fixed strong student on SST-2.}\label{fig:sst2_dsw}
\end{figure}

\begin{figure}[!h]
    \centering
    \includegraphics[width=\columnwidth]{fig/sst2/sst2-var.pdf}%\vspace{-2em}
    \caption{Scaling for $\pgr$ and $\opr$ of W2S on SST-2 with injected label noise.}\label{fig:sst2_var}
\end{figure}

\paragraph{Correlation Dimension.} 
Let $D_s, D_w \in \N$ be the finetunable parameter counts of the strong and weak models, respectively. For full FT whose dynamics fall in the kernel regime, as explained in \Cref{rmk:lp_to_general_ft}, the strong and weak ``features'' become the gradients\footnote{
    Notice that $f_s, f_w$ are scalar-valued functions for binary classification tasks like SST-2, and thus the gradients $\nabla_{\thetab} f_s$ and $\nabla_{\thetab} f_w$ are row vectors.
    For multi-class classification tasks where $f_s, f_w$ output vectors of logits, a common heuristic to keep $\Phib_s, \Phib_w$ as matrices of manageable sizes (in constrast to tensors) is to replace gradients of the models, $\nabla_{\thetab} f_s$ and $\nabla_{\thetab} f_w$, with gradients of MSE losses at the pretrained initialization. 
    The gradients of MSE can be viewed as a weighted sum of the model gradients for each class.
}, $\Phib_s = \nabla_{\thetab} f_s(\Xb | \theta_s^{(0)}) \in \R^{N \times D_s}$ and $\Phib_w = \nabla_{\thetab} f_w(\Xb | \theta_w^{(0)}) \in \R^{N \times D_w}$, of the respective models at the pretrained initialization, $\theta_s^{(0)} \in \R^{D_s}$ and $\theta_w^{(0)} \in \R^{D_w}$.

A practical challenge is that $D_s, D_w, N$ are all huge for full FT on most NLP tasks, making it infeasible to compute the $D_s \times D_s$ and $D_w \times D_w$ Gram matrices and their spectral decompositions. 
As a remedy, we leverage the significantly lower intrinsic dimensions $d_s \ll D_s, d_w \ll D_w$ (see \Cref{tab:img_cls_dim_coloredmnist}) to accelerate estimation of $d_{s \wedge w}$ via sketching~\citep{halko2011finding,woodruff2014sketching}.
\begin{enumerate}[label=(\roman*)]
    \item We first reduce both $D_s, D_w$ to the same lower dimension $D = 0.01 \min\{D_s, D_w\}$ (with $D \gg \max\{d_s, d_w\}$) by uniform subsampling columns of $\Phib_s, \Phib_w$ to obtain $\Phib_s', \Phib_w' \in \R^{N \times D}$.
    \item Then, we use randomized rangefinder~\citep[Algorithm 4.1]{halko2011finding} to approximate the first $d_s, d_w$ right singular vectors, $\Vb_s \in \R^{D \times d_s}$ and $\Vb_w \in \R^{D \times d_w}$, of $\Phib_s', \Phib_w'$. Taking the evaluation of $\Vb_s$ as an example, we draw a Gaussian random matrix $\Gb_s \in \R^{d_s \times D}$ and compute the orthornormalization $\Vb_s = \ortho(\Phib_s'^\top \Gb_s)$ via QR decomposition.
    \item Finally, we compute the correlation dimension $d_{s \wedge w} = \nbr{\Vb_s^\top \Vb_w}_F^2$.
\end{enumerate}

\begin{table}[!ht]
    \centering
    \caption{Summary of finetunable parameter counts $D_s, D_w$, intrinsic dimensions $d_s, d_w$, and correlation dimensions $d_{s \wedge w}$ (with respect to the strong student \texttt{Electra}) computed over the entire SST-2 dataset (including training and testing).}\label{tab:sst2_dim}
    \begin{tabular}{c|ccc}
        \toprule
        Pretrained Model & $D_s,D_w$ & Intrinsic Dimension ($\tau=0.01$) & Correlation Dimension \\
        \midrule
        \texttt{Bert-Tiny} & 4.4M & 7000 & 81.13 \\
        \texttt{Bert-Mini} & 11.2M & 8500 & 38.67 \\
        \texttt{Bert-Small} & 28.8M & 8000 & 26.19 \\
        \texttt{Bert-Medium} & 41.4M & 4000 & 8.52 \\
        \midrule
        \texttt{Electra} & 109.5M & 700 & $\times$ \\
        \bottomrule
    \end{tabular}
\end{table}

\paragraph{Discrepancies lead to better W2S.}
\Cref{fig:sst2_dsw} shows the scaling of $\pgr$ and $\opr$ with respect to $n$ and $N$ for different $d_{s \wedge w}$. 
As in \Cref{sec:exp_img_reg,apx:exp_img_cls}, we observe the better relative W2S performance in terms of $\pgr$ and $\opr$ when $d_{s \wedge w}/d_w$ is smaller.

\paragraph{Variance reduction is a key advantage of W2S.}
We inject noise to the labels of training samples by randomly flipping labels with probability $\varsigma = 0, 0.1, 0.2, 0.3$. 
\Cref{fig:sst2_var} shows the scaling of $\pgr$ and $\opr$ with respect to $n$ and $N$ when taking \texttt{Electra} as the strong student and \texttt{Bert-Medium} as the weak teacher. We observe that the larger artificial label noise $\varsigma$ leads to higher $\pgr$ and $\opr$. 

\end{document}

% \bibliographystyle{ACM-Reference-Format}
% \bibliography{reference}


% \clearpage

%\clearpage

\appendix
\section*{Appendix}
\begin{table*}[h!]
\caption{The basic information of grid-based spatio-temporal data.}
\label{tbl:append_data}
\begin{threeparttable}
\resizebox{1.9\columnwidth}{!}{
\begin{tabular}{cccccccc}
\toprule
Dataset & City & Type & Temporal Period & Spatial partition & Interval & Mean & Std \\
\hline
TaxiBJ & Beijing & Taxi flow&  2014/03/01 - 2014/06/30 & $32 \times 32$ & Half an hour & 111.5 & 139.3 \\
BikeDC & Washington, D.C. & Bike flow&  2010/09/20 - 2010/10/20 & $20 \times 20$ & Half an hour & 0.924 & 4.88 \\



CellularSH & Shanghai & Cellular traffic &  2014/08/01 - 2014/08/21 & $32\times28$ & One hour & 0.175 & 0.212 \\
CellularNJ & Nanjing & Cellular traffic &  2021/02/02 - 2021/02/22 & $20\times28$ & One hour & 0.842 & 1.30 \\
CrowdBJ & Beijing & Crowd flow &  2018/01/01 - 2018/01/31 & $1010$ & One hour & 7.07 & 11.1 \\
CrowdBM & Baltimore & Crowd flow &  2019/01/01 - 2019/05/31 & $403$ & One hour & 14.4 & 29.3 \\
Los-Speed & Los Angeles & Traffic speed&  2012/03/01 - 2012/03/07 & $207$ & Five minutes & 59.0 & 12.5 \\

\bottomrule
\end{tabular}}
\end{threeparttable}
\end{table*}

% \begin{table*}[t!]
% \caption{The basic information of Graph-based spatio-temporal data.}
% \label{tbl:append_data_graph}
% \begin{threeparttable}
% \resizebox{1.8\columnwidth}{!}{
% \begin{tabular}{ccccccccc}
% \toprule
% Dataset & City & Type & Temporal Period & Interval & \#Nodes & \#Edges & Mean & Std \\
% \hline

% TrafficBJ & Beijing & Traffic speed & 2022/03/05 - 2022/04/05 & 15min& 13675& 24444& 6.837&  3.412\\
% TrafficSH & Shanghai & Traffic speed & 2022/01/27 - 2022/02/27 & 15min & 21099& 39065& 7.815&  4.044\\
% TrafficNJ & Nanjing & Traffic speed  & 2022/03/05 - 2022/04/05 & 15min & 13419& 25100& 6.699&  4.253\\

% \bottomrule
% \end{tabular}}
% \end{threeparttable}
% \end{table*}
\begin{table*}[h!]
\caption{Short-term prediction results on two additional datasets in terms of both deterministic and probabilistic metrics. \textbf{Bold} indicates the best performance, while \underline{underlining} denotes the second-best.}
\label{tbl:short1-app}
\begin{threeparttable}
% \resizebox{2.0\columnwidth}

\resizebox{1.5\columnwidth}{!}{
\begin{tabular}{ccccccccccc}
\toprule
\multirow{2}{*}{\textbf{Model}}
& \multicolumn{5}{c}{\textbf{CellularNJ}} & \multicolumn{5}{c}{\textbf{CrowdBM}}   \\
\cmidrule(lr){2-6} \cmidrule(lr){7-11} 
 &\textbf{MAE} & \textbf{RMSE}  &\textbf{CRPS} & \textbf{QICE} & \textbf{IS} & 
\textbf{MAE} & \textbf{RMSE} & \textbf{CRPS} & \textbf{QICE} & \textbf{IS} \\


\midrule
D3VAE& 0.580 & 1.135   &	0.565&	0.096&6.03	&	11.0&24.7	&	0.593& 0.110	&136.4\\


DiffSTG& 0.317& 0.649&	0.291&	0.071&3.11	&	8.88&21.3	&0.453&0.047	&68.5\\

TimeGrad& 0.340&  0.357  &	0.432&	0.162&5.87	&10.1	& \underline{12.4}	&\textbf{0.240}&\underline{0.085}	&\underline{46.9}\\


CSDI&0.129 & 0.237   &	\underline{0.111}&	 \underline{0.039}&	\underline{0.80}&	7.31& 19.3&	0.390& 0.054&61.1\\

NPDiff& \underline{0.123}&  \underline{0.175}  &0.128	&	0.133&2.22	&	\underline{5.42}& 13.7	&0.331&0.119	&91.2\\

DyDiffusion&0.222&   0.357 &0.196	&0.080	&	1.80&-	&-	&-	&-&-\\




\cmidrule(lr){1-1} \cmidrule(lr){2-6} \cmidrule(lr){7-11}
\textbf{CoST}&\textbf{0.102} &\textbf{0.172}    &\textbf{0.090}	&\textbf{0.037}	&\textbf{0.682}	&\textbf{5.04}	&\textbf{12.1}	&\underline{0.256}	&\textbf{0.027}& \textbf{37.8}\\
%\textbf{Reduction}& &    &	&	&	&	&	&	&\\
\bottomrule
\end{tabular}}
\end{threeparttable}
\end{table*}


% \input{Tables/short-term2-append}

\end{document}

