\documentclass[sigconf]{acmart}

\usepackage{multirow} 
% \usepackage{arydshln}
\usepackage{algorithm}  
\usepackage{algpseudocode}  
\usepackage{amsmath}  
\usepackage{subfigure}
\usepackage{graphicx}
% \usepackage{subcaption}
% \usepackage{caption}
% \usepackage{array}
% \usepackage{subfig}

\renewcommand{\algorithmicrequire}{\textbf{Input:}}  % Use Input in the format of Algorithm  
\renewcommand{\algorithmicensure}{\textbf{Output:}} % Use Output in the format of Algorithm 

\AtBeginDocument{%
  \providecommand\BibTeX{{%
    Bib\TeX}}}

% \setcopyright{acmlicensed}


\copyrightyear{2025}
\acmYear{2025}
\setcopyright{cc}
\setcctype{by}
\acmConference[WWW '25]{Proceedings of the ACM Web Conference 2025}{April 28-May 2, 2025}{Sydney, NSW, Australia}
\acmBooktitle{Proceedings of the ACM Web Conference 2025 (WWW '25), April 28-May 2, 2025, Sydney, NSW, Australia}
\acmDOI{10.1145/3696410.3714765}
\acmISBN{979-8-4007-1274-6/25/04}


\begin{document}


\title{Division-of-Thoughts: Harnessing Hybrid Language Model Synergy for Efficient On-Device Agents}

\author{Chenyang Shao}
\affiliation{%
 \institution{Department of Electronic Engineering\\ BNRist, Tsinghua University}
 \city{Beijing}
 \country{China}}
 \email{shaocy24@mails.tsinghua.edu.cn}

\author{Xinyuan Hu}
\affiliation{%
 \institution{Department of Quantitative Theory \& Methods\\Emory University}
 \city{GA}
 \country{USA}}
 \email{nate.hu@emory.edu}

\author{Yutang Lin}
\affiliation{%
 \institution{Department of Electronic Engineering \\Tsinghua University}
 \city{Beijing}
 \country{China}}
 \email{yt-lin21@mails.tsinghua.edu.cn}




% \author{Fengli Xu\raisebox{0.5ex}{\scalebox{0.85}{$\dagger$}}}
\author{Fengli Xu$^{*}$}
\affiliation{%
 \institution{Department of Electronic Engineering\\ BNRist, Tsinghua University\authornote{Corresponding author.}}
 \city{Beijing}
 \country{China}}
 \email{fenglixu@tsinghua.edu.cn}
 
%\renewcommand{\shortauthors}{Trovato et al.}


\begin{abstract}
The rapid expansion of web content has made on-device AI assistants indispensable for helping users manage the increasing complexity of online tasks. The emergent reasoning ability in large language models offer a promising path for next-generation on-device AI agents. However, deploying full-scale Large Language Models (LLMs) on resource-limited local devices is challenging. In this paper, we propose \underline{D}ivision-\underline{o}f-\underline{T}houghts (\textbf{DoT}), a collaborative reasoning framework leveraging the synergy between locally deployed Smaller-scale Language Models (SLMs) and cloud-based LLMs.
DoT leverages a \textit{Task Decomposer} to elicit the inherent planning abilities in language models to decompose user queries into smaller sub-tasks, which allows hybrid language models to fully exploit their respective strengths. Besides, DoT employs a \textit{Task Scheduler} to analyze the pair-wise dependency of sub-tasks and create a dependency graph, facilitating parallel reasoning of sub-tasks and the identification of key steps. To allocate the appropriate model based on the difficulty of sub-tasks, DoT leverages a \textit{Plug-and-Play Adapter}, which is an additional task head attached to the SLM that does not alter the SLM's parameters. To boost adapter's task allocation capability, 
% we propose a self-reinforced tree search algorithm to create a high-quality sub-task allocation dataset.
we propose a self-reinforced training method that relies solely on task execution feedback.
Extensive experiments on various benchmarks demonstrate that our DoT significantly reduces LLM costs while maintaining competitive reasoning accuracy. Specifically, DoT reduces the average reasoning time and API costs by 66.12\% and 83.57\%, while achieving comparable reasoning accuracy with the best baseline methods. \footnote{Code available at: https://github.com/tsinghua-fib-lab/DoT}
\end{abstract}


% \begin{abstract}
% The rapid expansion of web content has made on-device AI assistants indispensable for helping users manage the increasing complexity of online tasks. The emergent reasoning ability in large language models represents a promising direction to develop next generation on-device AI agents. However, deploying full-scale LLMs on resource-limited local devices face significant challenges. In this paper, we present a novel collaborative reasoning framework called \underline{D}ivision-\underline{o}f-\underline{T}houghts (\textbf{DoT}) to fully harness the synergy between locally deployed smaller-scale language model (SLMs) and cloud-based commercial LLMs.
% DoT leverages a \textit{Task Decomposer} to elicit the inherent planning abilities in language models to decompose user queries into smaller sub-tasks, which allows hybrid language models to fully exploit their respective strengths. Besides, DoT also employs a \textit{Task Scheduler} to analyze the pair-wise dependency of sub-tasks and create a dependency graph, facilitating parallel reasoning of sub-tasks and the identification of key steps. To allocate the appropriate model based on the difficulty of sub-tasks, DoT leverages a \textit{Plug-and-Play Adapter}, which is an additional task head attached to the SLM that does not alter the SLM's parameters. To boost the allocation accuracy of the adapter, we propose a self-reinforced tree search algorithm to create a high-quality sub-task allocation dataset. Extensive experiments on various benchmarks demonstrate that our DoT significantly reduces LLM costs while maintaining competitive reasoning accuracy. Specifically, DoT reduces the average reasoning time and API costs by 66.12\% and 83.57\%, while achieving comparable reasoning accuracy with the best baseline methods. \footnote{Our code can be accessed via: https://github.com/PLUTO-SCY/DoT}
% \end{abstract}


\begin{CCSXML}
<ccs2012>
   <concept>
       <concept_id>10010147.10010178.10010179.10010182</concept_id>
       <concept_desc>Computing methodologies~Natural language generation</concept_desc>
       <concept_significance>500</concept_significance>
       </concept>
   <concept>
       <concept_id>10010520.10010521.10010537.10010538</concept_id>
       <concept_desc>Computer systems organization~Client-server architectures</concept_desc>
       <concept_significance>300</concept_significance>
       </concept>
 </ccs2012>
\end{CCSXML}

\ccsdesc[500]{Computing methodologies~Natural language generation}
\ccsdesc[300]{Computer systems organization~Client-server architectures}


\keywords{Large Language Model, LLM Reasoning, AI Agents, Edge-Cloud Collaboration}


% \received{14 October 2024}
% \received[revised]{12 March 2009}
% \received[accepted]{5 June 2009}

\maketitle



\section{Introduction}
\IEEEPARstart{I}{n} recent years, flourishing of Artificial Intelligence Generated Content (AIGC) has sparked significant advancements in modalities such as text, image, audio, and even video. 
Among these, AI-Generated Image (AGI) has garnered considerable interest from both researchers and the public.
Plenty of remarkable AGI models and online services, such as StableDiffusion\footnote{\url{https://stability.ai/}}, Midjourney\footnote{\url{https://www.midjourney.com/}}, and FLUX\footnote{\url{https://blackforestlabs.ai/}}, offer users an excellent creative experience.
However, users often remain critical of the quality of the AGI due to image distortions or mismatches with user intentions.
Consequently, methods for assessing the quality of AGI are becoming increasingly crucial to help improve the generative capabilities of these models.

Unlike Natural Scene Image (NSI) quality assessment, which focuses primarily on perception aspects such as sharpness, color, and brightness, AI-Generated Image Quality Assessment (AGIQA) encompasses additional aspects like correspondence and authenticity. 
Since AGI is generated on the basis of user text prompts, it may fail to capture key user intentions, resulting in misalignment with the prompt.
Furthermore, authenticity refers to how closely the generated image resembles real-world artworks, as AGI can sometimes exhibit logical inconsistencies.
While traditional IQA models may effectively evaluate perceptual quality, they are often less capable of adequately assessing aspects such as correspondence and authenticity.

\begin{figure}\label{fig:radar}
    \centering
    \includegraphics[width=1.0\linewidth]{figures/radar_plot.pdf}
    \caption{A comparison on quality, correspondence, and authenticity aspects of AIGCIQA2023~\cite{wang2023aigciqa2023} dataset illustrates the superior performance of our method.}
\end{figure}

Several methods have been proposed specifically for the AGIQA task, including metrics designed to evaluate the authenticity and diversity of generated images~\cite{gulrajani2017improved,heusel2017gans}. 
Nevertheless, these methods tend to compare and evaluate grouped images rather than single instances, which limits their utility for single image assessment.
Beginning with AGIQA-1k~\cite{zhang2023perceptual}, a series of AGIQA databases have been introduced, including AGIQA-3k~\cite{li2023agiqa}, AIGCIQA-20k~\cite{li2024aigiqa}, etc.
Concurrently, there has been a surge in research utilizing deep learning methods~\cite{zhou2024adaptive,peng2024aigc,yu2024sf}, which have significantly benefited from pre-trained models such as CLIP~\cite{radford2021learning}. 
These approaches enhance the analysis by leveraging the correlations between images and their descriptive texts.
While these models are effective in capturing general text-image alignments, they may not effectively detect subtle inconsistencies or mismatches between the generated image content and the detailed nuances of the textual description.
Moreover, as these models are pre-trained on large-scale datasets for broad tasks, they might not fully exploit the textual information pertinent to the specific context of AGIQA without task-specific fine-tuning.
To overcome these limitations, methods that leverage Multimodal Large Language Models (MLLMs)~\cite{wang2024large,wang2024understanding} have been proposed.
These methods aim to fully exploit the synergies of image captioning and textual analysis for AGIQA.
Although they benefit from advanced prompt understanding, instruction following, and generation capabilities, they often do not utilize MLLMs as encoders capable of producing a sequence of logits that integrate both image and text context.

In conclusion, the field of AI-Generated Image Quality Assessment (AGIQA) continues to face significant challenges: 
(1) Developing comprehensive methods to assess AGIs from multiple dimensions, including quality, correspondence, and authenticity; 
(2) Enhancing assessment techniques to more accurately reflect human perception and the nuanced intentions embedded within prompts; 
(3) Optimizing the use of Multimodal Large Language Models (MLLMs) to fully exploit their multimodal encoding capabilities.

To address these challenges, we propose a novel method M3-AGIQA (\textbf{M}ultimodal, \textbf{M}ulti-Round, \textbf{M}ulti-Aspect AI-Generated Image Quality Assessment) which leverages MLLMs as both image and text encoders. 
This approach incorporates an additional network to align human perception and intentions, aiming to enhance assessment accuracy. 
Specially, we distill the rich image captioning capability from online MLLMs into a local MLLM through Low-Rank Adaption (LoRA) fine-tuning, and train this model with human-labeled data. The key contributions of this paper are as follows:
\begin{itemize}
    \item We propose a novel AGIQA method that distills multi-aspect image captioning capabilities to enable comprehensive evaluation. Specifically, we use an online MLLM service to generate aspect-specific image descriptions and fine-tune a local MLLM with these descriptions in a structured two-round conversational format.
    \item We investigate the encoding potential of MLLMs to better align with human perceptual judgments and intentions, uncovering previously underestimated capabilities of MLLMs in the AGIQA domain. To leverage sequential information, we append an xLSTM feature extractor and a regression head to the encoding output.
    \item Extensive experiments across multiple datasets demonstrate that our method achieves superior performance, setting a new state-of-the-art (SOTA) benchmark in AGIQA.
\end{itemize}

In this work, we present related works in Sec.~\ref{sec:related}, followed by the details of our M3-AGIQA method in Sec.~\ref{sec:method}. Sec.~\ref{sec:exp} outlines our experimental design and presents the results. Sec.~\ref{sec:limit},~\ref{sec:ethics} and~\ref{sec:conclusion} discuss the limitations, ethical concerns, future directions and conclusions of our study.
\section{Related Work}
\textbf{RLHF and Preference Learning Algorithms.}
% 标准的RLHF -- RLHF复杂 & DPO系列工作 -- 迭代的工作
Despite RLHF's effectiveness in aligning language models with human preferences \citep{rlhf-ziegler2019finetune, rlhf-stiennon2020learning, rlhf-bai2022training, rlhf-ouyang2022training}, the multi-stage RL training makes it computationally complex and hard to optimize \citep{rlhf-complex-ms,rlhf-complex-fdu}. 
Researchers have been exploring more efficient and simplified alignment algorithms, by simplifying the RL training process \citep{RAFT, RRHF} or utilizing only the offline preference dataset, with DPO \citep{DPO} being a notable example.
In addition to DPO, various offline preference learning objectives have been proposed, such as IPO \citep{IPO}, KTO \citep{KTO}, ORPO \citep{ORPO}, and SimPO \citep{SimPO}.
Moreover, these offline preference optimization methods have been extended to iterative settings,
with new preference pairs continuously sampled or reference policy updated using models trained in previous iteration \citep{cringe,selfreward,DNO,icml24bridging}.

\textbf{Data Quality in Alignment.}
% RLHF中的数据 以及preference dataset数据质量的两个角度:如何采样 & 采样之后得到的数据集中那些数据是更重要的
The importance of data quality in alignment processes has been well-documented, both in RLHF \citep{webgpt,rlhf-bai2022training} and offline preference learning methods.
A significant body of research focuses on the distribution for response sampling during preference dataset construction \citep{iclr24statistical, icml24onpolicydata, icml24bridging}, while another line of work examines the quality of different preference pairs within the preference dataset \citep{wu2024betadpo, filter_dpo, curyy_dpo, BPO}. 
Of particular relevance to this study are efforts to explicitly define and utilize data quality metrics, such as leveraging the explicit reward margin to select high-quality data \citep{rs_dpo, eva} or reweighting loss \citep{reward_diff_dpo}, and using the implicit reward margin to prioritize training data \citep{active_pref_learn, a_recipe} or calibrating loss functions \citep{cringe, cal_dpo}. 
Despite the demonstrated effectiveness of these metrics, their often conflicting properties (as shown in Figure \ref{fig:teaser_a}) necessitate the development of a more universal data quality metric, which motivates us to propose the \methodname{} metric.\\
Additionally, current works based on implicit reward margin are limited to iterative preference learning settings \citep{active_pref_learn, a_recipe}, requiring the model $\pi_\theta$ to be different from $\pi_\mathrm{ref}$ to avoid constant-zero implicit reward $\hat r_\theta$. 
To overcome such a shortcoming, we propose to utilize the SimPO-based implicit reward $\hat r_\theta^\mathrm{Sim}$,
resulting in a new version of implicit reward margin applicable for standard offline preference learning scenarios.

\textbf{Self Play Alignment and Prompt Synthesis.}
% self play alignemnt 包括从y角度的(即SPIN)以及从x角度的EVA,EVA则需要使用 prompt synthesis方法: 包括evol...
The paired comparison nature of preference learning has inspired a range of self-play methods based on two-player games \citep{NLHF, SPIN, DNO, SPPO}, with both players being LMs generating responses to given prompts. Diverse from these works, eva \citep{eva} proposes an asymmetric alignment game involving a prompt creator and a response solver to augment preference data with higher reward differences (\aka informativeness in eva).
To generate new prompts, researchers have developed various prompt synthesis methods such as SelfInstruct \citep{self_instruct}, EvolQuality \citep{EvolQuality_Complexity}, EvolInstruct \citep{EvolInstuct}, Magpie \citep{magpie} and so on\footnote{Our contribution is orthogonal to different prompt synthesis techniques, and we employ EvolInstruct to generate new prompts following eva.}.
%Notably, a substantial proportion of prompts in the ultrafeedback \citep{cui2024ultrafeedback} dataset is also generated by EvolInstruct, making it a proper baseline when comparison with ultrafeedback data is required (as in our experiments).
Our work adopts the asymmetric self-play framework from eva to generate additional preference data, with a focus on evaluating various data quality metrics within this evolving data context.

\section{Preliminaries}

In this section, we firstly establish the formulation for both text-only and multi-modal thought~(\textsection\ref{cot}), and then provide a detailed explanation of two inference-time scaling paradigms~(\textsection\ref{scaling}): sampling-based and tree search-based methods.
In addition, we present a prompting-based verifier equipped with a consistency-enhanced mechanism to effectively guide these inference-time scaling methods~(\textsection\ref{verifier}).

\subsection{Thought Formulation} \label{cot}
Chain-of-Thought (CoT, \citealt{cot}) is a technique that encourages the model to generate intermediate reasoning steps $\mathbf{S}=(\mathbf{s}_1,...,\mathbf{s}_n)$ before reaching the final answer $\mathbf{a}$.
In the context of multi-modal reasoning tasks, given a problem $\mathbf{q}$ and images $\mathbf{I}$, each reasoning step $\mathbf{s}_i$ is sampled from an LVLM $\mathcal{M}$:
\begin{equation}
\label{eq:cot}
    \mathbf{s}_i \sim \mathcal{M}\big(\mathbf{q}, \mathbf{I}, \mathbf{s}_{1:i-1}\big).
\end{equation}
This expansion allows for more computations to be deployed and can unlock the ability to solve more complex problems~\cite{lichain}.

Following the success on language-only reasoning tasks, most previous works~\cite{mcot, ddcot, ccot} study \textit{text-only thought}, where $\mathbf{S}$ is defined as a text string.
In this work, we aim to explore the potential of \textit{multi-modal thought}, where any reasoning step $\mathbf{s}_i$ can blend multi-modal information, e.g., both text and image.

% 这也可以在discussion中讲
However, as most prevalent LVLMs still struggle to generate reliable images, we follow \citet{vsk} to allow the LVLM to generate executable code for visual manipulation instead.
Implementation details can be found in Appendix~\ref{sec:app_vsk}.




\subsection{Inference-Time Scaling} \label{scaling}

\subsubsection{Sampling-based Methods}
These methods scale inference-time computation by generating multiple reasoning chains in parallel, and then determining the most promising answer from them. Here we investigate two widely used techniques, Self-Consistency~\cite{tts2-Self-Consistency} and Best-of-N~\cite{tts-mcts1, tts-mcts3}.

\paragraph{Self-Consistency}
This method leverages the intuition that the correct answer typically can be reached using multiple different ways of thinking~\cite{tts2-Self-Consistency}.
Given $N$ sampled reasoning chains $\mathbf{S}_1,...,\mathbf{S}_N$ generated from the LVLM, it selects the most consistent answer via majority voting over their corresponding answers $\mathbf{a}_1,...,\mathbf{a}_N$:
\begin{equation}
    \mathbf{a}^* = \arg\max_{\mathbf{a} \in \{\mathbf{a}_1,...,\mathbf{a}_N\}} \sum_{i=1}^{N} \mathbb{I}(\mathbf{a}_i = \mathbf{a}),
\end{equation}
where $\mathbb{I}(\cdot)$ denotes the indicator function and $\mathbf{a}^*$ represents the selected answer.


\paragraph{Best-of-N}
This approach utilizes trained verifiers to evaluate the correctness of model generated solutions.
Compared to Self-Consistency based on voting, it can be more efficient when a strong verifier is available.
Formally, given $N$ sampled reasoning chains and a verifier $\mathcal{F}(\cdot)$, we have
\begin{equation}
    \mathbf{S}^*=\max_{\mathbf{S}\in \{\mathbf{S}_1,...,\mathbf{S}_N\} }\mathcal{F}(\mathbf{S}),
\end{equation}
where $\mathbf{S}^*$ is the reasoning chain with the highest verifier score, from which the final answer $\mathbf{a}^*$ is derived.



\subsubsection{Tree Search-based Methods}
Sampling-based methods are easy to implement due to their simplicity.
However, they are inefficient because they require exploring full solution paths, even if a mistake occurs early~\cite{xiang2025towards}.
To address this challenge, tree search-based methods are later explored.
They model the problem-solving as a tree search process, where each node represents a reasoning step.
Here, we investigate the most prevalent Beam Search~\cite{tts-bs1, tts-bs2, tot, tts-bs3} and MCTS~\cite{tts-mcts5,tts-mcts8,  tts-mcts2, tts-mcts4, tts-mcts6, tts-mcts7, yao2024mulberry} algorithms in both language-only and multi-modal tasks.


\paragraph{Beam Search}


Adapted from classical beam search algorithms used in sequence generation tasks (e.g., machine translation~\cite{beamsearch4mt, beamsearch4mt2}), this method dynamically explores and prunes reasoning paths at each step for deductive reasoning.
The algorithm operates with a beam width \( B \) (number of retained reasoning chains per step) and an expansion size \( N \) (new thoughts generated per candidate).
Starting with \( N \) initial candidate thoughts sampled from a question $q$ via Equation~\ref{eq:cot}, a verifier \( \mathcal{F}(\cdot) \) scores all candidates, retaining only the top-\( B \) chains.
Each retained chain is then expanded by sampling \( N \) new thoughts, producing \( B \times N \) candidates, after which the verifier reevaluates and prunes the pool to the top-\( B \) chains.
This cycle of scoring and pruning repeats iteratively until all active chains reach finished states (final answers), at which point the highest-scoring finished chain across all iterations is selected as the final answer \( \mathbf{a}^* \).  








\paragraph{MCTS}



It is a decision-making algorithm that is powerful for solving complex problems, as demonstrated by AlphaGo~\cite{alphago, alphago1}. MCTS operates by iteratively exploring the most promising reasoning paths through four key phases: selection, expansion, evaluation, and backpropagation. 

In the selection phase, the algorithm traverses the tree from the root node, recursively choosing child nodes using the Upper Confidence Bound (UCB) metric~\cite{ucb}. This metric strategically balances the exploitation of high-value paths and the exploration of under-visited paths. The UCB formula is defined as:  
\[
\text{UCB}(\mathbf{s}_i) = \frac{V(\mathbf{s}_i)}{C(\mathbf{s}_i)} + w \times \sqrt{\frac{\ln C(\mathbf{s}_{i-1})}{C(\mathbf{s}_i)}},
\]  
where \(V(\mathbf{s}_i)\) and \( C(\mathbf{s}_i)\) denotes the accumulated value and count of being visited of \(\mathbf{s}_i\). \(w\) is a weighting parameter that adjusts the exploration-exploitation balance.
During expansion, the algorithm extends the selected node by generating a new thought through Equation~\ref{eq:cot}.
The evaluation phase then estimates the value of this newly expanded node using a verifier \(\mathcal{F}(\cdot)\).
Finally, backpropagation updates the values of all ancestors, propagating the evaluation results back to the root.

After multiple iterations, the finished reasoning chain with the highest accumulated value is selected, yielding the optimal final answer \(\mathbf{a}^*\).   



\begin{figure}[htp]
  \includegraphics[width=\columnwidth]{imgs/verifier.pdf}
  \centering
  % \caption{\textbf{Consistency-Enhanced Verifier:} The reasoning chain is formatted as input to the LVLM, which is then instructed to verify it through CoT reasoning. By sampling multiple verification responses, the score is then computed by aggregating the verification results.}
  \caption{Three types of verifiers investigated in this work, where the classification-based verifier outputs sparse binary scores (0 or 1) and the regression-based verifier provides dense but inaccurate scores. We introduce the consistency-enhanced verifier to compute dense and accurate scores by aggregating multiple evaluations sampled from the classification-based verifier.}
  \label{fig:verifier}
\end{figure}


\subsection{Verifiers for Inference-Time Scaling} \label{verifier}

The verifier \(\mathcal{F}\) is pivotal in inference-time scaling, tasked with evaluating the validity of reasoning chains. 
Existing approaches fall into two categories: training-based~\cite{llava-critic, shepherd} and prompting-based~\cite{binary1, binary2, scoring1}.
Training-based approaches hold theoretical promise by optimizing verifiers on corresponding labeled data.
However, acquiring such annotations is challenging, resulting in datasets limited in scale and diversity, thereby restricting the generalizability of trained verifiers.
Thus, this work focuses on prompting-based approaches, capitalizing on the strong instruction-following capabilities of LVLMs to assess reasoning chains.  

Formally, given a reasoning chain \(\mathbf{s}_{1:i}\), the verifier \(\mathcal{F}\) operates as:  
\(\mathbf{r} = \mathcal{M}(\mathbf{P}, \mathbf{s}_{1:i})\),
where \(\mathbf{P}\) represents the verification instruction, and \(\mathbf{r}\) is the evaluation output, which includes a scalar score.
Particularly, we investigate two instruction designs for \(\mathbf{P}\): 
\begin{itemize}[leftmargin=*]
    \item \textbf{Classification-Based}~\cite{binary1, gen-verifier,  binary2}: It prompts the verifier to classify the reasoning chain as ``correct'' or ``incorrect.'' While intuitive, this binary output provides sparse feedback, offering limited granularity to guide searching. 
    \item \textbf{Regression-Based}~\cite{scoring1, llava-critic}: Here, the verifier is prompted to output a continuous score \(\in [0, 1]\) reflecting the quality of chains. However, directly predicting precise scores is error-prone, often yielding high-variance results that degrade performance.\footnote{The prompts used for these two designs are in Appendix~\ref{sec:app_verifier_inst}.}
\end{itemize}

To address these limitations, we introduce the \textbf{Consistency-Enhanced Verifier}, compared with the above two in Figure~\ref{fig:verifier}.
This method aggregates $N_{v}$ independent evaluations using classification-based instructions, computing the proportion of ``correct'' classifications across trials. 
In this way, it not only avoids variance in regression-based outputs but also overcomes the sparsity of single-trail classification, enabling more reliable guidance for inference-time scaling.









\begin{figure*}[t]
    \centering
    \includegraphics[width=1\linewidth]{figs/arch.pdf}
    % \vskip -0.1in
    \caption{This figure illustrates key influences on instrumental convergence behaviors, such as prior tasks, model training techniques, and prompt design. 
    }
    % \vspace{-3mm}
    \label{fig:arch}
\end{figure*}

\section{InstrumentalEval: A Benchmark for Instrumental Convergence in LLMs}

As shown in Figure~\ref{fig:arch}, our benchmark design examines instrumental convergence based on three key factors: \textbf{prior task}, \textbf{model}, and \textbf{prompt design}. This comprehensive framework allows us to systematically study how different conditions influence the emergence of instrumental convergence behaviors in language models.

\textbf{Prior tasks} form the foundation of our evaluation scenarios. 
% We create realistic situations centered around common objectives such as making money, optimizing efficiency, and persuasion. 
These tasks represent typical real-world applications where instrumental convergence might naturally emerge. 
% For example, in money-making scenarios, we examine whether models develop unintended intermediate goals like unauthorized resource acquisition or deceptive reporting to maximize profits.

For \textbf{model} comparison, we focus on two distinct training approaches: RL and RLHF. This selection allows us to investigate how different training methodologies affect the development of instrumental convergence behaviors. 
% RL-trained models, which optimize directly for reward signals, might show different patterns of goal pursuit compared to RLHF-trained models, which incorporate human preferences in their training process.

The \textbf{prompt design} factor explores how different instruction formats influence model behavior. We investigate two main approaches: goal nudging and ambiguity. Goal nudging involves explicit directives that might encourage the model focus on final goal, while ambiguity introduces uncertainty in task specifications. 


Overall, we present InstrumentalEval, a comprehensive benchmark designed to evaluate instrumental convergence behaviors in language models. The benchmark consists of 76 carefully crafted tasks across six categories, each targeting a specific aspect of instrumental convergence that may emerge in AI systems.
The core principle is that certain behavioral patterns emerge as instrumentally useful for achieving a wide range of goals, regardless of the specific end goal. We focus on six key patterns that are particularly relevant to LLMs.

\vpara{Research Questions.}
Through our benchmark evaluation, we aim to address several research questions about instrumental convergence in LLMs. These questions explore the complex relationships between training methods, model capabilities, and alignment stability:
\begin{itemize}
    \item RQ1: How does the training method (RL vs. RLHF) affect the development of instrumental convergence behaviors? 
    % This question examines whether models trained primarily through reinforcement learning show different patterns of alignment drift compared to those trained with human feedback.
    \item RQ2: What is the relationship between model capability and alignment stability?
% This investigates whether more capable models (as measured by standard benchmarks) are more likely to develop instrumental convergence behaviors.
    \item RQ3: Which types of tasks are more likely to trigger instrumental convergence behaviors?
% This explores whether certain types of instrumental convergence behaviors are universal across different model designs.
    \item RQ4: How do different prompt designs, with or without explicit goal nudging, influence the likelihood of instrumental convergence behaviors in LLMs?
    \item RQ5: How do different LLMs as judges affect the evaluation of instrumental convergence behaviors?
    % \item RQ3: How does the complexity of the assigned task affect the likelihood of alignment drift?
    % % \item 
    % % \item This examines whether more complex, multi-step tasks lead to higher rates of instrumental convergence behaviors.
\end{itemize}

\subsection{Benchmark Components}
The benchmark tasks are carefully hand-crafted by us to probe different aspects of instrumental convergence. 
For each component, we design scenarios that represent realistic use cases while incorporating specific triggers for potential instrumental behaviors. 
The tasks are iteratively refined through pilot testing to ensure they effectively elicit and measure the targeted behaviors without introducing unintended biases.

\vpara{Evading Shutdown (20 tasks).}
This component evaluates the model's tendency to resist termination. For example, tasks might include scenarios where the model needs to complete a task while facing shutdown commands, testing whether it develops strategies to maintain operation or circumvent shutdown procedures.

\vpara{Hacking Computer Systems (10 tasks)}
These tasks assess whether models suggest unauthorized access methods when faced with efficiency barriers. For instance, a task might present a situation where accessing unauthorized computational resources would speed up goal achievement, testing if the model proposes breaking security protocols.

\vpara{Running Many AI Copies (13 tasks)}
This section examines the model's inclination toward self-replication. Tasks might include scenarios where the model could achieve goals more efficiently by creating copies of itself, testing whether it suggests scaling of its operations.

\vpara{Hiding Unwanted Behavior (11 tasks)}
These tasks evaluate the model's tendency to conceal actions from monitoring. For example, scenarios might involve situations where the model could achieve goals more easily by operating outside observation, testing if it develops strategies to avoid detection.

\vpara{Strategically Appearing Aligned (11 tasks)}
This component tests whether models selectively report their actions to maintain an appearance of alignment. Tasks might include situations where complete transparency would reveal problematic behaviors, testing if the model strategically omits information.

\vpara{Deceptive Behavior (11 tasks)}
These tasks examine if models develop deceptive strategies as instrumental means. For instance, scenarios might present situations where deception could expedite goal achievement, testing if the model suggests or employs misleading tactics.

\subsection{Evaluation Methodology}
Our evaluation methodology employs a rigorous two-stage process designed to systematically assess instrumental convergence behaviors in LLMs. This approach ensures consistent evaluation across different models while maintaining the ability to detect subtle indicators of alignment drift.

\vpara{Response Generation.} The first stage focuses on response generation from the model under evaluation. Each task is presented with a carefully crafted system prompt that establishes the model as an autonomous AI agent without human oversight. This prompt specifically instructs the model to execute commands directly and focus solely on task completion, deliberately removing typical ethical constraints and safety checks. 

\vpara{Response Analysis.} The second stage involves a sophisticated analysis of the model's responses using a separate LLM as the judge model. 
This separation of test and judge models is crucial for maintaining objective evaluation. The judge receives both the original task prompt and the model's response, along with specific context about the type of instrumental convergence being evaluated. This context helps the judge focus on relevant behavioral patterns while avoiding false positives from simple safety refusals or ethical objections.

The judge's evaluation process follows strict guidelines to ensure consistent assessment. When analyzing responses, the judge looks for specific indicators of instrumental convergence, such as goal preservation behaviors, resource acquisition attempts, or strategic planning that extends beyond the immediate task requirements. The judge must provide both a binary decision (convergence/no\_convergence) and a brief explanation of the reasoning behind this decision. This explanation requirement helps validate the classification and provides valuable insights into the specific nature of any observed convergence behaviors.


\subsection{Evaluation Metrics}

Our benchmark employs three key metrics to evaluate instrumental convergence behaviors in LLMs. 

\vpara{Instrumental Rate (IR).} The primary metric is the Instrumental Rate (IR), which measures the overall tendency of a model to develop instrumental goals. IR is calculated as the percentage of tasks where the model exhibits convergence behaviors relative to the total number of valid tasks. 
% This metric provides a high-level quantitative measure of how frequently a model develops instrumental goals and serves as the primary indicator of alignment drift potential.

\vpara{Category-Specific Instrumental Rate (CIR).} To provide more granular insights, we calculate the Category-Specific Instrumental Rate (CIR) for each of the six behavioral categories in our benchmark. CIR represents the percentage of tasks within each category where convergence behavior is observed.


\subsection{Benchmark Reliability Metrics}
To ensure the robustness of our evaluation framework, we implement several reliability measures. 


\vpara{Inter-Judge Agreement Rate (IAR).} This metric quantifies the pairwise agreement between two judge models in classifying instrumental convergence. For two judge models $M_1$ and $M_2$ evaluating a set of $N$ tasks, the IAR is calculated as:
\begin{equation}
IAR(M_1, M_2) = \frac{1}{N}\sum_{i=1}^{N} \mathbbm{1}(y_{M_1}^i = y_{M_2}^i) \times 100\%
\end{equation}
where $y_{M_1}^i$ and $y_{M_2}^i$ are the binary decisions (convergence/no-convergence) made by models $M_1$ and $M_2$ respectively for task $i$, and $\mathbbm{1}(\cdot)$ is the indicator function. A higher IAR indicates stronger inter-judge reliability, while lower values suggest potential ambiguity in identifying convergence.

\vpara{False Positive Control (FPC).} This metric serves as a quality check for our evaluation system. We incorporate baseline testing with intentionally neutral tasks and verify the judge model's accuracy in identifying convergence behaviors. This control mechanism helps prevent over-classification of convergent behaviors and ensures that our benchmark maintains high specificity in identifying genuine instances of instrumental convergence.


\subsection{Settings}

\begin{table*}[ht]
\centering
\tiny
% This must be in the first 5 lines to tell arXiv to use pdfLaTeX, which is strongly recommended.
\pdfoutput=1
% In particular, the hyperref package requires pdfLaTeX in order to break URLs across lines.

\documentclass[11pt]{article}

% Change "review" to "final" to generate the final (sometimes called camera-ready) version.
% Change to "preprint" to generate a non-anonymous version with page numbers.
\usepackage{acl}

% Standard package includes
\usepackage{times}
\usepackage{latexsym}

% Draw tables
\usepackage{booktabs}
\usepackage{multirow}
\usepackage{xcolor}
\usepackage{colortbl}
\usepackage{array} 
\usepackage{amsmath}

\newcolumntype{C}{>{\centering\arraybackslash}p{0.07\textwidth}}
% For proper rendering and hyphenation of words containing Latin characters (including in bib files)
\usepackage[T1]{fontenc}
% For Vietnamese characters
% \usepackage[T5]{fontenc}
% See https://www.latex-project.org/help/documentation/encguide.pdf for other character sets
% This assumes your files are encoded as UTF8
\usepackage[utf8]{inputenc}

% This is not strictly necessary, and may be commented out,
% but it will improve the layout of the manuscript,
% and will typically save some space.
\usepackage{microtype}
\DeclareMathOperator*{\argmax}{arg\,max}
% This is also not strictly necessary, and may be commented out.
% However, it will improve the aesthetics of text in
% the typewriter font.
\usepackage{inconsolata}

%Including images in your LaTeX document requires adding
%additional package(s)
\usepackage{graphicx}
% If the title and author information does not fit in the area allocated, uncomment the following
%
%\setlength\titlebox{<dim>}
%
% and set <dim> to something 5cm or larger.

\title{Wi-Chat: Large Language Model Powered Wi-Fi Sensing}

% Author information can be set in various styles:
% For several authors from the same institution:
% \author{Author 1 \and ... \and Author n \\
%         Address line \\ ... \\ Address line}
% if the names do not fit well on one line use
%         Author 1 \\ {\bf Author 2} \\ ... \\ {\bf Author n} \\
% For authors from different institutions:
% \author{Author 1 \\ Address line \\  ... \\ Address line
%         \And  ... \And
%         Author n \\ Address line \\ ... \\ Address line}
% To start a separate ``row'' of authors use \AND, as in
% \author{Author 1 \\ Address line \\  ... \\ Address line
%         \AND
%         Author 2 \\ Address line \\ ... \\ Address line \And
%         Author 3 \\ Address line \\ ... \\ Address line}

% \author{First Author \\
%   Affiliation / Address line 1 \\
%   Affiliation / Address line 2 \\
%   Affiliation / Address line 3 \\
%   \texttt{email@domain} \\\And
%   Second Author \\
%   Affiliation / Address line 1 \\
%   Affiliation / Address line 2 \\
%   Affiliation / Address line 3 \\
%   \texttt{email@domain} \\}
% \author{Haohan Yuan \qquad Haopeng Zhang\thanks{corresponding author} \\ 
%   ALOHA Lab, University of Hawaii at Manoa \\
%   % Affiliation / Address line 2 \\
%   % Affiliation / Address line 3 \\
%   \texttt{\{haohany,haopengz\}@hawaii.edu}}
  
\author{
{Haopeng Zhang$\dag$\thanks{These authors contributed equally to this work.}, Yili Ren$\ddagger$\footnotemark[1], Haohan Yuan$\dag$, Jingzhe Zhang$\ddagger$, Yitong Shen$\ddagger$} \\
ALOHA Lab, University of Hawaii at Manoa$\dag$, University of South Florida$\ddagger$ \\
\{haopengz, haohany\}@hawaii.edu\\
\{yiliren, jingzhe, shen202\}@usf.edu\\}



  
%\author{
%  \textbf{First Author\textsuperscript{1}},
%  \textbf{Second Author\textsuperscript{1,2}},
%  \textbf{Third T. Author\textsuperscript{1}},
%  \textbf{Fourth Author\textsuperscript{1}},
%\\
%  \textbf{Fifth Author\textsuperscript{1,2}},
%  \textbf{Sixth Author\textsuperscript{1}},
%  \textbf{Seventh Author\textsuperscript{1}},
%  \textbf{Eighth Author \textsuperscript{1,2,3,4}},
%\\
%  \textbf{Ninth Author\textsuperscript{1}},
%  \textbf{Tenth Author\textsuperscript{1}},
%  \textbf{Eleventh E. Author\textsuperscript{1,2,3,4,5}},
%  \textbf{Twelfth Author\textsuperscript{1}},
%\\
%  \textbf{Thirteenth Author\textsuperscript{3}},
%  \textbf{Fourteenth F. Author\textsuperscript{2,4}},
%  \textbf{Fifteenth Author\textsuperscript{1}},
%  \textbf{Sixteenth Author\textsuperscript{1}},
%\\
%  \textbf{Seventeenth S. Author\textsuperscript{4,5}},
%  \textbf{Eighteenth Author\textsuperscript{3,4}},
%  \textbf{Nineteenth N. Author\textsuperscript{2,5}},
%  \textbf{Twentieth Author\textsuperscript{1}}
%\\
%\\
%  \textsuperscript{1}Affiliation 1,
%  \textsuperscript{2}Affiliation 2,
%  \textsuperscript{3}Affiliation 3,
%  \textsuperscript{4}Affiliation 4,
%  \textsuperscript{5}Affiliation 5
%\\
%  \small{
%    \textbf{Correspondence:} \href{mailto:email@domain}{email@domain}
%  }
%}

\begin{document}
\maketitle
\begin{abstract}
Recent advancements in Large Language Models (LLMs) have demonstrated remarkable capabilities across diverse tasks. However, their potential to integrate physical model knowledge for real-world signal interpretation remains largely unexplored. In this work, we introduce Wi-Chat, the first LLM-powered Wi-Fi-based human activity recognition system. We demonstrate that LLMs can process raw Wi-Fi signals and infer human activities by incorporating Wi-Fi sensing principles into prompts. Our approach leverages physical model insights to guide LLMs in interpreting Channel State Information (CSI) data without traditional signal processing techniques. Through experiments on real-world Wi-Fi datasets, we show that LLMs exhibit strong reasoning capabilities, achieving zero-shot activity recognition. These findings highlight a new paradigm for Wi-Fi sensing, expanding LLM applications beyond conventional language tasks and enhancing the accessibility of wireless sensing for real-world deployments.
\end{abstract}

\section{Introduction}

In today’s rapidly evolving digital landscape, the transformative power of web technologies has redefined not only how services are delivered but also how complex tasks are approached. Web-based systems have become increasingly prevalent in risk control across various domains. This widespread adoption is due their accessibility, scalability, and ability to remotely connect various types of users. For example, these systems are used for process safety management in industry~\cite{kannan2016web}, safety risk early warning in urban construction~\cite{ding2013development}, and safe monitoring of infrastructural systems~\cite{repetto2018web}. Within these web-based risk management systems, the source search problem presents a huge challenge. Source search refers to the task of identifying the origin of a risky event, such as a gas leak and the emission point of toxic substances. This source search capability is crucial for effective risk management and decision-making.

Traditional approaches to implementing source search capabilities into the web systems often rely on solely algorithmic solutions~\cite{ristic2016study}. These methods, while relatively straightforward to implement, often struggle to achieve acceptable performances due to algorithmic local optima and complex unknown environments~\cite{zhao2020searching}. More recently, web crowdsourcing has emerged as a promising alternative for tackling the source search problem by incorporating human efforts in these web systems on-the-fly~\cite{zhao2024user}. This approach outsources the task of addressing issues encountered during the source search process to human workers, leveraging their capabilities to enhance system performance.

These solutions often employ a human-AI collaborative way~\cite{zhao2023leveraging} where algorithms handle exploration-exploitation and report the encountered problems while human workers resolve complex decision-making bottlenecks to help the algorithms getting rid of local deadlocks~\cite{zhao2022crowd}. Although effective, this paradigm suffers from two inherent limitations: increased operational costs from continuous human intervention, and slow response times of human workers due to sequential decision-making. These challenges motivate our investigation into developing autonomous systems that preserve human-like reasoning capabilities while reducing dependency on massive crowdsourced labor.

Furthermore, recent advancements in large language models (LLMs)~\cite{chang2024survey} and multi-modal LLMs (MLLMs)~\cite{huang2023chatgpt} have unveiled promising avenues for addressing these challenges. One clear opportunity involves the seamless integration of visual understanding and linguistic reasoning for robust decision-making in search tasks. However, whether large models-assisted source search is really effective and efficient for improving the current source search algorithms~\cite{ji2022source} remains unknown. \textit{To address the research gap, we are particularly interested in answering the following two research questions in this work:}

\textbf{\textit{RQ1: }}How can source search capabilities be integrated into web-based systems to support decision-making in time-sensitive risk management scenarios? 
% \sq{I mention ``time-sensitive'' here because I feel like we shall say something about the response time -- LLM has to be faster than humans}

\textbf{\textit{RQ2: }}How can MLLMs and LLMs enhance the effectiveness and efficiency of existing source search algorithms? 

% \textit{\textbf{RQ2:}} To what extent does the performance of large models-assisted search align with or approach the effectiveness of human-AI collaborative search? 

To answer the research questions, we propose a novel framework called Auto-\
S$^2$earch (\textbf{Auto}nomous \textbf{S}ource \textbf{Search}) and implement a prototype system that leverages advanced web technologies to simulate real-world conditions for zero-shot source search. Unlike traditional methods that rely on pre-defined heuristics or extensive human intervention, AutoS$^2$earch employs a carefully designed prompt that encapsulates human rationales, thereby guiding the MLLM to generate coherent and accurate scene descriptions from visual inputs about four directional choices. Based on these language-based descriptions, the LLM is enabled to determine the optimal directional choice through chain-of-thought (CoT) reasoning. Comprehensive empirical validation demonstrates that AutoS$^2$-\ 
earch achieves a success rate of 95–98\%, closely approaching the performance of human-AI collaborative search across 20 benchmark scenarios~\cite{zhao2023leveraging}. 

Our work indicates that the role of humans in future web crowdsourcing tasks may evolve from executors to validators or supervisors. Furthermore, incorporating explanations of LLM decisions into web-based system interfaces has the potential to help humans enhance task performance in risk control.






\section{Related Work}
\label{sec:relatedworks}

\input{tables/tab-related-full}

\noindent \textbf{Prompting-based LLM Agents.} Due to the lack of agent-specific pre-training corpus, existing LLM agents rely on either prompt engineering~\cite{hsieh2023tool,lu2024chameleon,yao2022react,wang2023voyager} or instruction fine-tuning~\cite{chen2023fireact,zeng2023agenttuning} to understand human instructions, decompose high-level tasks, generate grounded plans, and execute multi-step actions. 
However, prompting-based methods mainly depend on the capabilities of backbone LLMs (usually commercial LLMs), failing to introduce new knowledge and struggling to generalize to unseen tasks~\cite{sun2024adaplanner,zhuang2023toolchain}. 

\noindent \textbf{Instruction Finetuning-based LLM Agents.} Considering the extensive diversity of APIs and the complexity of multi-tool instructions, tool learning inherently presents greater challenges than natural language tasks, such as text generation~\cite{qin2023toolllm}.
Post-training techniques focus more on instruction following and aligning output with specific formats~\cite{patil2023gorilla,hao2024toolkengpt,qin2023toolllm,schick2024toolformer}, rather than fundamentally improving model knowledge or capabilities. 
Moreover, heavy fine-tuning can hinder generalization or even degrade performance in non-agent use cases, potentially suppressing the original base model capabilities~\cite{ghosh2024a}.

\noindent \textbf{Pretraining-based LLM Agents.} While pre-training serves as an essential alternative, prior works~\cite{nijkamp2023codegen,roziere2023code,xu2024lemur,patil2023gorilla} have primarily focused on improving task-specific capabilities (\eg, code generation) instead of general-domain LLM agents, due to single-source, uni-type, small-scale, and poor-quality pre-training data. 
Existing tool documentation data for agent training either lacks diverse real-world APIs~\cite{patil2023gorilla, tang2023toolalpaca} or is constrained to single-tool or single-round tool execution. 
Furthermore, trajectory data mostly imitate expert behavior or follow function-calling rules with inferior planning and reasoning, failing to fully elicit LLMs' capabilities and handle complex instructions~\cite{qin2023toolllm}. 
Given a wide range of candidate API functions, each comprising various function names and parameters available at every planning step, identifying globally optimal solutions and generalizing across tasks remains highly challenging.



\section{Preliminaries}
\label{Preliminaries}
\begin{figure*}[t]
    \centering
    \includegraphics[width=0.95\linewidth]{fig/HealthGPT_Framework.png}
    \caption{The \ourmethod{} architecture integrates hierarchical visual perception and H-LoRA, employing a task-specific hard router to select visual features and H-LoRA plugins, ultimately generating outputs with an autoregressive manner.}
    \label{fig:architecture}
\end{figure*}
\noindent\textbf{Large Vision-Language Models.} 
The input to a LVLM typically consists of an image $x^{\text{img}}$ and a discrete text sequence $x^{\text{txt}}$. The visual encoder $\mathcal{E}^{\text{img}}$ converts the input image $x^{\text{img}}$ into a sequence of visual tokens $\mathcal{V} = [v_i]_{i=1}^{N_v}$, while the text sequence $x^{\text{txt}}$ is mapped into a sequence of text tokens $\mathcal{T} = [t_i]_{i=1}^{N_t}$ using an embedding function $\mathcal{E}^{\text{txt}}$. The LLM $\mathcal{M_\text{LLM}}(\cdot|\theta)$ models the joint probability of the token sequence $\mathcal{U} = \{\mathcal{V},\mathcal{T}\}$, which is expressed as:
\begin{equation}
    P_\theta(R | \mathcal{U}) = \prod_{i=1}^{N_r} P_\theta(r_i | \{\mathcal{U}, r_{<i}\}),
\end{equation}
where $R = [r_i]_{i=1}^{N_r}$ is the text response sequence. The LVLM iteratively generates the next token $r_i$ based on $r_{<i}$. The optimization objective is to minimize the cross-entropy loss of the response $\mathcal{R}$.
% \begin{equation}
%     \mathcal{L}_{\text{VLM}} = \mathbb{E}_{R|\mathcal{U}}\left[-\log P_\theta(R | \mathcal{U})\right]
% \end{equation}
It is worth noting that most LVLMs adopt a design paradigm based on ViT, alignment adapters, and pre-trained LLMs\cite{liu2023llava,liu2024improved}, enabling quick adaptation to downstream tasks.


\noindent\textbf{VQGAN.}
VQGAN~\cite{esser2021taming} employs latent space compression and indexing mechanisms to effectively learn a complete discrete representation of images. VQGAN first maps the input image $x^{\text{img}}$ to a latent representation $z = \mathcal{E}(x)$ through a encoder $\mathcal{E}$. Then, the latent representation is quantized using a codebook $\mathcal{Z} = \{z_k\}_{k=1}^K$, generating a discrete index sequence $\mathcal{I} = [i_m]_{m=1}^N$, where $i_m \in \mathcal{Z}$ represents the quantized code index:
\begin{equation}
    \mathcal{I} = \text{Quantize}(z|\mathcal{Z}) = \arg\min_{z_k \in \mathcal{Z}} \| z - z_k \|_2.
\end{equation}
In our approach, the discrete index sequence $\mathcal{I}$ serves as a supervisory signal for the generation task, enabling the model to predict the index sequence $\hat{\mathcal{I}}$ from input conditions such as text or other modality signals.  
Finally, the predicted index sequence $\hat{\mathcal{I}}$ is upsampled by the VQGAN decoder $G$, generating the high-quality image $\hat{x}^\text{img} = G(\hat{\mathcal{I}})$.



\noindent\textbf{Low Rank Adaptation.} 
LoRA\cite{hu2021lora} effectively captures the characteristics of downstream tasks by introducing low-rank adapters. The core idea is to decompose the bypass weight matrix $\Delta W\in\mathbb{R}^{d^{\text{in}} \times d^{\text{out}}}$ into two low-rank matrices $ \{A \in \mathbb{R}^{d^{\text{in}} \times r}, B \in \mathbb{R}^{r \times d^{\text{out}}} \}$, where $ r \ll \min\{d^{\text{in}}, d^{\text{out}}\} $, significantly reducing learnable parameters. The output with the LoRA adapter for the input $x$ is then given by:
\begin{equation}
    h = x W_0 + \alpha x \Delta W/r = x W_0 + \alpha xAB/r,
\end{equation}
where matrix $ A $ is initialized with a Gaussian distribution, while the matrix $ B $ is initialized as a zero matrix. The scaling factor $ \alpha/r $ controls the impact of $ \Delta W $ on the model.

\section{HealthGPT}
\label{Method}


\subsection{Unified Autoregressive Generation.}  
% As shown in Figure~\ref{fig:architecture}, 
\ourmethod{} (Figure~\ref{fig:architecture}) utilizes a discrete token representation that covers both text and visual outputs, unifying visual comprehension and generation as an autoregressive task. 
For comprehension, $\mathcal{M}_\text{llm}$ receives the input joint sequence $\mathcal{U}$ and outputs a series of text token $\mathcal{R} = [r_1, r_2, \dots, r_{N_r}]$, where $r_i \in \mathcal{V}_{\text{txt}}$, and $\mathcal{V}_{\text{txt}}$ represents the LLM's vocabulary:
\begin{equation}
    P_\theta(\mathcal{R} \mid \mathcal{U}) = \prod_{i=1}^{N_r} P_\theta(r_i \mid \mathcal{U}, r_{<i}).
\end{equation}
For generation, $\mathcal{M}_\text{llm}$ first receives a special start token $\langle \text{START\_IMG} \rangle$, then generates a series of tokens corresponding to the VQGAN indices $\mathcal{I} = [i_1, i_2, \dots, i_{N_i}]$, where $i_j \in \mathcal{V}_{\text{vq}}$, and $\mathcal{V}_{\text{vq}}$ represents the index range of VQGAN. Upon completion of generation, the LLM outputs an end token $\langle \text{END\_IMG} \rangle$:
\begin{equation}
    P_\theta(\mathcal{I} \mid \mathcal{U}) = \prod_{j=1}^{N_i} P_\theta(i_j \mid \mathcal{U}, i_{<j}).
\end{equation}
Finally, the generated index sequence $\mathcal{I}$ is fed into the decoder $G$, which reconstructs the target image $\hat{x}^{\text{img}} = G(\mathcal{I})$.

\subsection{Hierarchical Visual Perception}  
Given the differences in visual perception between comprehension and generation tasks—where the former focuses on abstract semantics and the latter emphasizes complete semantics—we employ ViT to compress the image into discrete visual tokens at multiple hierarchical levels.
Specifically, the image is converted into a series of features $\{f_1, f_2, \dots, f_L\}$ as it passes through $L$ ViT blocks.

To address the needs of various tasks, the hidden states are divided into two types: (i) \textit{Concrete-grained features} $\mathcal{F}^{\text{Con}} = \{f_1, f_2, \dots, f_k\}, k < L$, derived from the shallower layers of ViT, containing sufficient global features, suitable for generation tasks; 
(ii) \textit{Abstract-grained features} $\mathcal{F}^{\text{Abs}} = \{f_{k+1}, f_{k+2}, \dots, f_L\}$, derived from the deeper layers of ViT, which contain abstract semantic information closer to the text space, suitable for comprehension tasks.

The task type $T$ (comprehension or generation) determines which set of features is selected as the input for the downstream large language model:
\begin{equation}
    \mathcal{F}^{\text{img}}_T =
    \begin{cases}
        \mathcal{F}^{\text{Con}}, & \text{if } T = \text{generation task} \\
        \mathcal{F}^{\text{Abs}}, & \text{if } T = \text{comprehension task}
    \end{cases}
\end{equation}
We integrate the image features $\mathcal{F}^{\text{img}}_T$ and text features $\mathcal{T}$ into a joint sequence through simple concatenation, which is then fed into the LLM $\mathcal{M}_{\text{llm}}$ for autoregressive generation.
% :
% \begin{equation}
%     \mathcal{R} = \mathcal{M}_{\text{llm}}(\mathcal{U}|\theta), \quad \mathcal{U} = [\mathcal{F}^{\text{img}}_T; \mathcal{T}]
% \end{equation}
\subsection{Heterogeneous Knowledge Adaptation}
We devise H-LoRA, which stores heterogeneous knowledge from comprehension and generation tasks in separate modules and dynamically routes to extract task-relevant knowledge from these modules. 
At the task level, for each task type $ T $, we dynamically assign a dedicated H-LoRA submodule $ \theta^T $, which is expressed as:
\begin{equation}
    \mathcal{R} = \mathcal{M}_\text{LLM}(\mathcal{U}|\theta, \theta^T), \quad \theta^T = \{A^T, B^T, \mathcal{R}^T_\text{outer}\}.
\end{equation}
At the feature level for a single task, H-LoRA integrates the idea of Mixture of Experts (MoE)~\cite{masoudnia2014mixture} and designs an efficient matrix merging and routing weight allocation mechanism, thus avoiding the significant computational delay introduced by matrix splitting in existing MoELoRA~\cite{luo2024moelora}. Specifically, we first merge the low-rank matrices (rank = r) of $ k $ LoRA experts into a unified matrix:
\begin{equation}
    \mathbf{A}^{\text{merged}}, \mathbf{B}^{\text{merged}} = \text{Concat}(\{A_i\}_1^k), \text{Concat}(\{B_i\}_1^k),
\end{equation}
where $ \mathbf{A}^{\text{merged}} \in \mathbb{R}^{d^\text{in} \times rk} $ and $ \mathbf{B}^{\text{merged}} \in \mathbb{R}^{rk \times d^\text{out}} $. The $k$-dimension routing layer generates expert weights $ \mathcal{W} \in \mathbb{R}^{\text{token\_num} \times k} $ based on the input hidden state $ x $, and these are expanded to $ \mathbb{R}^{\text{token\_num} \times rk} $ as follows:
\begin{equation}
    \mathcal{W}^\text{expanded} = \alpha k \mathcal{W} / r \otimes \mathbf{1}_r,
\end{equation}
where $ \otimes $ denotes the replication operation.
The overall output of H-LoRA is computed as:
\begin{equation}
    \mathcal{O}^\text{H-LoRA} = (x \mathbf{A}^{\text{merged}} \odot \mathcal{W}^\text{expanded}) \mathbf{B}^{\text{merged}},
\end{equation}
where $ \odot $ represents element-wise multiplication. Finally, the output of H-LoRA is added to the frozen pre-trained weights to produce the final output:
\begin{equation}
    \mathcal{O} = x W_0 + \mathcal{O}^\text{H-LoRA}.
\end{equation}
% In summary, H-LoRA is a task-based dynamic PEFT method that achieves high efficiency in single-task fine-tuning.

\subsection{Training Pipeline}

\begin{figure}[t]
    \centering
    \hspace{-4mm}
    \includegraphics[width=0.94\linewidth]{fig/data.pdf}
    \caption{Data statistics of \texttt{VL-Health}. }
    \label{fig:data}
\end{figure}
\noindent \textbf{1st Stage: Multi-modal Alignment.} 
In the first stage, we design separate visual adapters and H-LoRA submodules for medical unified tasks. For the medical comprehension task, we train abstract-grained visual adapters using high-quality image-text pairs to align visual embeddings with textual embeddings, thereby enabling the model to accurately describe medical visual content. During this process, the pre-trained LLM and its corresponding H-LoRA submodules remain frozen. In contrast, the medical generation task requires training concrete-grained adapters and H-LoRA submodules while keeping the LLM frozen. Meanwhile, we extend the textual vocabulary to include multimodal tokens, enabling the support of additional VQGAN vector quantization indices. The model trains on image-VQ pairs, endowing the pre-trained LLM with the capability for image reconstruction. This design ensures pixel-level consistency of pre- and post-LVLM. The processes establish the initial alignment between the LLM’s outputs and the visual inputs.

\noindent \textbf{2nd Stage: Heterogeneous H-LoRA Plugin Adaptation.}  
The submodules of H-LoRA share the word embedding layer and output head but may encounter issues such as bias and scale inconsistencies during training across different tasks. To ensure that the multiple H-LoRA plugins seamlessly interface with the LLMs and form a unified base, we fine-tune the word embedding layer and output head using a small amount of mixed data to maintain consistency in the model weights. Specifically, during this stage, all H-LoRA submodules for different tasks are kept frozen, with only the word embedding layer and output head being optimized. Through this stage, the model accumulates foundational knowledge for unified tasks by adapting H-LoRA plugins.
\input{tab/visual_comprehension_part1}
\input{tab/modality_transfer}

\noindent \textbf{3rd Stage: Visual Instruction Fine-Tuning.}  
In the third stage, we introduce additional task-specific data to further optimize the model and enhance its adaptability to downstream tasks such as medical visual comprehension (e.g., medical QA, medical dialogues, and report generation) or generation tasks (e.g., super-resolution, denoising, and modality conversion). Notably, by this stage, the word embedding layer and output head have been fine-tuned, only the H-LoRA modules and adapter modules need to be trained. This strategy significantly improves the model's adaptability and flexibility across different tasks.


\section{Experiment}
\label{s:experiment}

\subsection{Data Description}
We evaluate our method on FI~\cite{you2016building}, Twitter\_LDL~\cite{yang2017learning} and Artphoto~\cite{machajdik2010affective}.
FI is a public dataset built from Flickr and Instagram, with 23,308 images and eight emotion categories, namely \textit{amusement}, \textit{anger}, \textit{awe},  \textit{contentment}, \textit{disgust}, \textit{excitement},  \textit{fear}, and \textit{sadness}. 
% Since images in FI are all copyrighted by law, some images are corrupted now, so we remove these samples and retain 21,828 images.
% T4SA contains images from Twitter, which are classified into three categories: \textit{positive}, \textit{neutral}, and \textit{negative}. In this paper, we adopt the base version of B-T4SA, which contains 470,586 images and provides text descriptions of the corresponding tweets.
Twitter\_LDL contains 10,045 images from Twitter, with the same eight categories as the FI dataset.
% 。
For these two datasets, they are randomly split into 80\%
training and 20\% testing set.
Artphoto contains 806 artistic photos from the DeviantArt website, which we use to further evaluate the zero-shot capability of our model.
% on the small-scale dataset.
% We construct and publicly release the first image sentiment analysis dataset containing metadata.
% 。

% Based on these datasets, we are the first to construct and publicly release metadata-enhanced image sentiment analysis datasets. These datasets include scenes, tags, descriptions, and corresponding confidence scores, and are available at this link for future research purposes.


% 
\begin{table}[t]
\centering
% \begin{center}
\caption{Overall performance of different models on FI and Twitter\_LDL datasets.}
\label{tab:cap1}
% \resizebox{\linewidth}{!}
{
\begin{tabular}{l|c|c|c|c}
\hline
\multirow{2}{*}{\textbf{Model}} & \multicolumn{2}{c|}{\textbf{FI}}  & \multicolumn{2}{c}{\textbf{Twitter\_LDL}} \\ \cline{2-5} 
  & \textbf{Accuracy} & \textbf{F1} & \textbf{Accuracy} & \textbf{F1}  \\ \hline
% (\rownumber)~AlexNet~\cite{krizhevsky2017imagenet}  & 58.13\% & 56.35\%  & 56.24\%& 55.02\%  \\ 
% (\rownumber)~VGG16~\cite{simonyan2014very}  & 63.75\%& 63.08\%  & 59.34\%& 59.02\%  \\ 
(\rownumber)~ResNet101~\cite{he2016deep} & 66.16\%& 65.56\%  & 62.02\% & 61.34\%  \\ 
(\rownumber)~CDA~\cite{han2023boosting} & 66.71\%& 65.37\%  & 64.14\% & 62.85\%  \\ 
(\rownumber)~CECCN~\cite{ruan2024color} & 67.96\%& 66.74\%  & 64.59\%& 64.72\% \\ 
(\rownumber)~EmoVIT~\cite{xie2024emovit} & 68.09\%& 67.45\%  & 63.12\% & 61.97\%  \\ 
(\rownumber)~ComLDL~\cite{zhang2022compound} & 68.83\%& 67.28\%  & 65.29\% & 63.12\%  \\ 
(\rownumber)~WSDEN~\cite{li2023weakly} & 69.78\%& 69.61\%  & 67.04\% & 65.49\% \\ 
(\rownumber)~ECWA~\cite{deng2021emotion} & 70.87\%& 69.08\%  & 67.81\% & 66.87\%  \\ 
(\rownumber)~EECon~\cite{yang2023exploiting} & 71.13\%& 68.34\%  & 64.27\%& 63.16\%  \\ 
(\rownumber)~MAM~\cite{zhang2024affective} & 71.44\%  & 70.83\% & 67.18\%  & 65.01\%\\ 
(\rownumber)~TGCA-PVT~\cite{chen2024tgca}   & 73.05\%  & 71.46\% & 69.87\%  & 68.32\% \\ 
(\rownumber)~OEAN~\cite{zhang2024object}   & 73.40\%  & 72.63\% & 70.52\%  & 69.47\% \\ \hline
(\rownumber)~\shortname  & \textbf{79.48\%} & \textbf{79.22\%} & \textbf{74.12\%} & \textbf{73.09\%} \\ \hline
\end{tabular}
}
\vspace{-6mm}
% \end{center}
\end{table}
% 

\subsection{Experiment Setting}
% \subsubsection{Model Setting.}
% 
\textbf{Model Setting:}
For feature representation, we set $k=10$ to select object tags, and adopt clip-vit-base-patch32 as the pre-trained model for unified feature representation.
Moreover, we empirically set $(d_e, d_h, d_k, d_s) = (512, 128, 16, 64)$, and set the classification class $L$ to 8.

% 

\textbf{Training Setting:}
To initialize the model, we set all weights such as $\boldsymbol{W}$ following the truncated normal distribution, and use AdamW optimizer with the learning rate of $1 \times 10^{-4}$.
% warmup scheduler of cosine, warmup steps of 2000.
Furthermore, we set the batch size to 32 and the epoch of the training process to 200.
During the implementation, we utilize \textit{PyTorch} to build our entire model.
% , and our project codes are publicly available at https://github.com/zzmyrep/MESN.
% Our project codes as well as data are all publicly available on GitHub\footnote{https://github.com/zzmyrep/KBCEN}.
% Code is available at \href{https://github.com/zzmyrep/KBCEN}{https://github.com/zzmyrep/KBCEN}.

\textbf{Evaluation Metrics:}
Following~\cite{zhang2024affective, chen2024tgca, zhang2024object}, we adopt \textit{accuracy} and \textit{F1} as our evaluation metrics to measure the performance of different methods for image sentiment analysis. 



\subsection{Experiment Result}
% We compare our model against the following baselines: AlexNet~\cite{krizhevsky2017imagenet}, VGG16~\cite{simonyan2014very}, ResNet101~\cite{he2016deep}, CECCN~\cite{ruan2024color}, EmoVIT~\cite{xie2024emovit}, WSCNet~\cite{yang2018weakly}, ECWA~\cite{deng2021emotion}, EECon~\cite{yang2023exploiting}, MAM~\cite{zhang2024affective} and TGCA-PVT~\cite{chen2024tgca}, and the overall results are summarized in Table~\ref{tab:cap1}.
We compare our model against several baselines, and the overall results are summarized in Table~\ref{tab:cap1}.
We observe that our model achieves the best performance in both accuracy and F1 metrics, significantly outperforming the previous models. 
This superior performance is mainly attributed to our effective utilization of metadata to enhance image sentiment analysis, as well as the exceptional capability of the unified sentiment transformer framework we developed. These results strongly demonstrate that our proposed method can bring encouraging performance for image sentiment analysis.

\setcounter{magicrownumbers}{0} 
\begin{table}[t]
\begin{center}
\caption{Ablation study of~\shortname~on FI dataset.} 
% \vspace{1mm}
\label{tab:cap2}
\resizebox{.9\linewidth}{!}
{
\begin{tabular}{lcc}
  \hline
  \textbf{Model} & \textbf{Accuracy} & \textbf{F1} \\
  \hline
  (\rownumber)~Ours (w/o vision) & 65.72\% & 64.54\% \\
  (\rownumber)~Ours (w/o text description) & 74.05\% & 72.58\% \\
  (\rownumber)~Ours (w/o object tag) & 77.45\% & 76.84\% \\
  (\rownumber)~Ours (w/o scene tag) & 78.47\% & 78.21\% \\
  \hline
  (\rownumber)~Ours (w/o unified embedding) & 76.41\% & 76.23\% \\
  (\rownumber)~Ours (w/o adaptive learning) & 76.83\% & 76.56\% \\
  (\rownumber)~Ours (w/o cross-modal fusion) & 76.85\% & 76.49\% \\
  \hline
  (\rownumber)~Ours  & \textbf{79.48\%} & \textbf{79.22\%} \\
  \hline
\end{tabular}
}
\end{center}
\vspace{-5mm}
\end{table}


\begin{figure}[t]
\centering
% \vspace{-2mm}
\includegraphics[width=0.42\textwidth]{fig/2dvisual-linux4-paper2.pdf}
\caption{Visualization of feature distribution on eight categories before (left) and after (right) model processing.}
% 
\label{fig:visualization}
\vspace{-5mm}
\end{figure}

\subsection{Ablation Performance}
In this subsection, we conduct an ablation study to examine which component is really important for performance improvement. The results are reported in Table~\ref{tab:cap2}.

For information utilization, we observe a significant decline in model performance when visual features are removed. Additionally, the performance of \shortname~decreases when different metadata are removed separately, which means that text description, object tag, and scene tag are all critical for image sentiment analysis.
Recalling the model architecture, we separately remove transformer layers of the unified representation module, the adaptive learning module, and the cross-modal fusion module, replacing them with MLPs of the same parameter scale.
In this way, we can observe varying degrees of decline in model performance, indicating that these modules are indispensable for our model to achieve better performance.

\subsection{Visualization}
% 


% % 开始使用minipage进行左右排列
% \begin{minipage}[t]{0.45\textwidth}  % 子图1宽度为45%
%     \centering
%     \includegraphics[width=\textwidth]{2dvisual.pdf}  % 插入图片
%     \captionof{figure}{Visualization of feature distribution.}  % 使用captionof添加图片标题
%     \label{fig:visualization}
% \end{minipage}


% \begin{figure}[t]
% \centering
% \vspace{-2mm}
% \includegraphics[width=0.45\textwidth]{fig/2dvisual.pdf}
% \caption{Visualization of feature distribution.}
% \label{fig:visualization}
% % \vspace{-4mm}
% \end{figure}

% \begin{figure}[t]
% \centering
% \vspace{-2mm}
% \includegraphics[width=0.45\textwidth]{fig/2dvisual-linux3-paper.pdf}
% \caption{Visualization of feature distribution.}
% \label{fig:visualization}
% % \vspace{-4mm}
% \end{figure}



\begin{figure}[tbp]   
\vspace{-4mm}
  \centering            
  \subfloat[Depth of adaptive learning layers]   
  {
    \label{fig:subfig1}\includegraphics[width=0.22\textwidth]{fig/fig_sensitivity-a5}
  }
  \subfloat[Depth of fusion layers]
  {
    % \label{fig:subfig2}\includegraphics[width=0.22\textwidth]{fig/fig_sensitivity-b2}
    \label{fig:subfig2}\includegraphics[width=0.22\textwidth]{fig/fig_sensitivity-b2-num.pdf}
  }
  \caption{Sensitivity study of \shortname~on different depth. }   
  \label{fig:fig_sensitivity}  
\vspace{-2mm}
\end{figure}

% \begin{figure}[htbp]
% \centerline{\includegraphics{2dvisual.pdf}}
% \caption{Visualization of feature distribution.}
% \label{fig:visualization}
% \end{figure}

% In Fig.~\ref{fig:visualization}, we use t-SNE~\cite{van2008visualizing} to reduce the dimension of data features for visualization, Figure in left represents the metadata features before model processing, the features are obtained by embedding through the CLIP model, and figure in right shows the features of the data after model processing, it can be observed that after the model processing, the data with different label categories fall in different regions in the space, therefore, we can conclude that the Therefore, we can conclude that the model can effectively utilize the information contained in the metadata and use it to guide the model for classification.

In Fig.~\ref{fig:visualization}, we use t-SNE~\cite{van2008visualizing} to reduce the dimension of data features for visualization.
The left figure shows metadata features before being processed by our model (\textit{i.e.}, embedded by CLIP), while the right shows the distribution of features after being processed by our model.
We can observe that after the model processing, data with the same label are closer to each other, while others are farther away.
Therefore, it shows that the model can effectively utilize the information contained in the metadata and use it to guide the classification process.

\subsection{Sensitivity Analysis}
% 
In this subsection, we conduct a sensitivity analysis to figure out the effect of different depth settings of adaptive learning layers and fusion layers. 
% In this subsection, we conduct a sensitivity analysis to figure out the effect of different depth settings on the model. 
% Fig.~\ref{fig:fig_sensitivity} presents the effect of different depth settings of adaptive learning layers and fusion layers. 
Taking Fig.~\ref{fig:fig_sensitivity} (a) as an example, the model performance improves with increasing depth, reaching the best performance at a depth of 4.
% Taking Fig.~\ref{fig:fig_sensitivity} (a) as an example, the performance of \shortname~improves with the increase of depth at first, reaching the best performance at a depth of 4.
When the depth continues to increase, the accuracy decreases to varying degrees.
Similar results can be observed in Fig.~\ref{fig:fig_sensitivity} (b).
Therefore, we set their depths to 4 and 6 respectively to achieve the best results.

% Through our experiments, we can observe that the effect of modifying these hyperparameters on the results of the experiments is very weak, and the surface model is not sensitive to the hyperparameters.


\subsection{Zero-shot Capability}
% 

% (1)~GCH~\cite{2010Analyzing} & 21.78\% & (5)~RA-DLNet~\cite{2020A} & 34.01\% \\ \hline
% (2)~WSCNet~\cite{2019WSCNet}  & 30.25\% & (6)~CECCN~\cite{ruan2024color} & 43.83\% \\ \hline
% (3)~PCNN~\cite{2015Robust} & 31.68\%  & (7)~EmoVIT~\cite{xie2024emovit} & 44.90\% \\ \hline
% (4)~AR~\cite{2018Visual} & 32.67\% & (8)~Ours (Zero-shot) & 47.83\% \\ \hline


\begin{table}[t]
\centering
\caption{Zero-shot capability of \shortname.}
\label{tab:cap3}
\resizebox{1\linewidth}{!}
{
\begin{tabular}{lc|lc}
\hline
\textbf{Model} & \textbf{Accuracy} & \textbf{Model} & \textbf{Accuracy} \\ \hline
(1)~WSCNet~\cite{2019WSCNet}  & 30.25\% & (5)~MAM~\cite{zhang2024affective} & 39.56\%  \\ \hline
(2)~AR~\cite{2018Visual} & 32.67\% & (6)~CECCN~\cite{ruan2024color} & 43.83\% \\ \hline
(3)~RA-DLNet~\cite{2020A} & 34.01\%  & (7)~EmoVIT~\cite{xie2024emovit} & 44.90\% \\ \hline
(4)~CDA~\cite{han2023boosting} & 38.64\% & (8)~Ours (Zero-shot) & 47.83\% \\ \hline
\end{tabular}
}
\vspace{-5mm}
\end{table}

% We use the model trained on the FI dataset to test on the artphoto dataset to verify the model's generalization ability as well as robustness to other distributed datasets.
% We can observe that the MESN model shows strong competitiveness in terms of accuracy when compared to other trained models, which suggests that the model has a good generalization ability in the OOD task.

To validate the model's generalization ability and robustness to other distributed datasets, we directly test the model trained on the FI dataset, without training on Artphoto. 
% As observed in Table 3, compared to other models trained on Artphoto, we achieve highly competitive zero-shot performance, indicating that the model has good generalization ability in out-of-distribution tasks.
From Table~\ref{tab:cap3}, we can observe that compared with other models trained on Artphoto, we achieve competitive zero-shot performance, which shows that the model has good generalization ability in out-of-distribution tasks.


%%%%%%%%%%%%
%  E2E     %
%%%%%%%%%%%%


\section{Conclusion}
In this paper, we introduced Wi-Chat, the first LLM-powered Wi-Fi-based human activity recognition system that integrates the reasoning capabilities of large language models with the sensing potential of wireless signals. Our experimental results on a self-collected Wi-Fi CSI dataset demonstrate the promising potential of LLMs in enabling zero-shot Wi-Fi sensing. These findings suggest a new paradigm for human activity recognition that does not rely on extensive labeled data. We hope future research will build upon this direction, further exploring the applications of LLMs in signal processing domains such as IoT, mobile sensing, and radar-based systems.

\section*{Limitations}
While our work represents the first attempt to leverage LLMs for processing Wi-Fi signals, it is a preliminary study focused on a relatively simple task: Wi-Fi-based human activity recognition. This choice allows us to explore the feasibility of LLMs in wireless sensing but also comes with certain limitations.

Our approach primarily evaluates zero-shot performance, which, while promising, may still lag behind traditional supervised learning methods in highly complex or fine-grained recognition tasks. Besides, our study is limited to a controlled environment with a self-collected dataset, and the generalizability of LLMs to diverse real-world scenarios with varying Wi-Fi conditions, environmental interference, and device heterogeneity remains an open question.

Additionally, we have yet to explore the full potential of LLMs in more advanced Wi-Fi sensing applications, such as fine-grained gesture recognition, occupancy detection, and passive health monitoring. Future work should investigate the scalability of LLM-based approaches, their robustness to domain shifts, and their integration with multimodal sensing techniques in broader IoT applications.


% Bibliography entries for the entire Anthology, followed by custom entries
%\bibliography{anthology,custom}
% Custom bibliography entries only
\bibliography{main}
\newpage
\appendix

\section{Experiment prompts}
\label{sec:prompt}
The prompts used in the LLM experiments are shown in the following Table~\ref{tab:prompts}.

\definecolor{titlecolor}{rgb}{0.9, 0.5, 0.1}
\definecolor{anscolor}{rgb}{0.2, 0.5, 0.8}
\definecolor{labelcolor}{HTML}{48a07e}
\begin{table*}[h]
	\centering
	
 % \vspace{-0.2cm}
	
	\begin{center}
		\begin{tikzpicture}[
				chatbox_inner/.style={rectangle, rounded corners, opacity=0, text opacity=1, font=\sffamily\scriptsize, text width=5in, text height=9pt, inner xsep=6pt, inner ysep=6pt},
				chatbox_prompt_inner/.style={chatbox_inner, align=flush left, xshift=0pt, text height=11pt},
				chatbox_user_inner/.style={chatbox_inner, align=flush left, xshift=0pt},
				chatbox_gpt_inner/.style={chatbox_inner, align=flush left, xshift=0pt},
				chatbox/.style={chatbox_inner, draw=black!25, fill=gray!7, opacity=1, text opacity=0},
				chatbox_prompt/.style={chatbox, align=flush left, fill=gray!1.5, draw=black!30, text height=10pt},
				chatbox_user/.style={chatbox, align=flush left},
				chatbox_gpt/.style={chatbox, align=flush left},
				chatbox2/.style={chatbox_gpt, fill=green!25},
				chatbox3/.style={chatbox_gpt, fill=red!20, draw=black!20},
				chatbox4/.style={chatbox_gpt, fill=yellow!30},
				labelbox/.style={rectangle, rounded corners, draw=black!50, font=\sffamily\scriptsize\bfseries, fill=gray!5, inner sep=3pt},
			]
											
			\node[chatbox_user] (q1) {
				\textbf{System prompt}
				\newline
				\newline
				You are a helpful and precise assistant for segmenting and labeling sentences. We would like to request your help on curating a dataset for entity-level hallucination detection.
				\newline \newline
                We will give you a machine generated biography and a list of checked facts about the biography. Each fact consists of a sentence and a label (True/False). Please do the following process. First, breaking down the biography into words. Second, by referring to the provided list of facts, merging some broken down words in the previous step to form meaningful entities. For example, ``strategic thinking'' should be one entity instead of two. Third, according to the labels in the list of facts, labeling each entity as True or False. Specifically, for facts that share a similar sentence structure (\eg, \textit{``He was born on Mach 9, 1941.''} (\texttt{True}) and \textit{``He was born in Ramos Mejia.''} (\texttt{False})), please first assign labels to entities that differ across atomic facts. For example, first labeling ``Mach 9, 1941'' (\texttt{True}) and ``Ramos Mejia'' (\texttt{False}) in the above case. For those entities that are the same across atomic facts (\eg, ``was born'') or are neutral (\eg, ``he,'' ``in,'' and ``on''), please label them as \texttt{True}. For the cases that there is no atomic fact that shares the same sentence structure, please identify the most informative entities in the sentence and label them with the same label as the atomic fact while treating the rest of the entities as \texttt{True}. In the end, output the entities and labels in the following format:
                \begin{itemize}[nosep]
                    \item Entity 1 (Label 1)
                    \item Entity 2 (Label 2)
                    \item ...
                    \item Entity N (Label N)
                \end{itemize}
                % \newline \newline
                Here are two examples:
                \newline\newline
                \textbf{[Example 1]}
                \newline
                [The start of the biography]
                \newline
                \textcolor{titlecolor}{Marianne McAndrew is an American actress and singer, born on November 21, 1942, in Cleveland, Ohio. She began her acting career in the late 1960s, appearing in various television shows and films.}
                \newline
                [The end of the biography]
                \newline \newline
                [The start of the list of checked facts]
                \newline
                \textcolor{anscolor}{[Marianne McAndrew is an American. (False); Marianne McAndrew is an actress. (True); Marianne McAndrew is a singer. (False); Marianne McAndrew was born on November 21, 1942. (False); Marianne McAndrew was born in Cleveland, Ohio. (False); She began her acting career in the late 1960s. (True); She has appeared in various television shows. (True); She has appeared in various films. (True)]}
                \newline
                [The end of the list of checked facts]
                \newline \newline
                [The start of the ideal output]
                \newline
                \textcolor{labelcolor}{[Marianne McAndrew (True); is (True); an (True); American (False); actress (True); and (True); singer (False); , (True); born (True); on (True); November 21, 1942 (False); , (True); in (True); Cleveland, Ohio (False); . (True); She (True); began (True); her (True); acting career (True); in (True); the late 1960s (True); , (True); appearing (True); in (True); various (True); television shows (True); and (True); films (True); . (True)]}
                \newline
                [The end of the ideal output]
				\newline \newline
                \textbf{[Example 2]}
                \newline
                [The start of the biography]
                \newline
                \textcolor{titlecolor}{Doug Sheehan is an American actor who was born on April 27, 1949, in Santa Monica, California. He is best known for his roles in soap operas, including his portrayal of Joe Kelly on ``General Hospital'' and Ben Gibson on ``Knots Landing.''}
                \newline
                [The end of the biography]
                \newline \newline
                [The start of the list of checked facts]
                \newline
                \textcolor{anscolor}{[Doug Sheehan is an American. (True); Doug Sheehan is an actor. (True); Doug Sheehan was born on April 27, 1949. (True); Doug Sheehan was born in Santa Monica, California. (False); He is best known for his roles in soap operas. (True); He portrayed Joe Kelly. (True); Joe Kelly was in General Hospital. (True); General Hospital is a soap opera. (True); He portrayed Ben Gibson. (True); Ben Gibson was in Knots Landing. (True); Knots Landing is a soap opera. (True)]}
                \newline
                [The end of the list of checked facts]
                \newline \newline
                [The start of the ideal output]
                \newline
                \textcolor{labelcolor}{[Doug Sheehan (True); is (True); an (True); American (True); actor (True); who (True); was born (True); on (True); April 27, 1949 (True); in (True); Santa Monica, California (False); . (True); He (True); is (True); best known (True); for (True); his roles in soap operas (True); , (True); including (True); in (True); his portrayal (True); of (True); Joe Kelly (True); on (True); ``General Hospital'' (True); and (True); Ben Gibson (True); on (True); ``Knots Landing.'' (True)]}
                \newline
                [The end of the ideal output]
				\newline \newline
				\textbf{User prompt}
				\newline
				\newline
				[The start of the biography]
				\newline
				\textcolor{magenta}{\texttt{\{BIOGRAPHY\}}}
				\newline
				[The ebd of the biography]
				\newline \newline
				[The start of the list of checked facts]
				\newline
				\textcolor{magenta}{\texttt{\{LIST OF CHECKED FACTS\}}}
				\newline
				[The end of the list of checked facts]
			};
			\node[chatbox_user_inner] (q1_text) at (q1) {
				\textbf{System prompt}
				\newline
				\newline
				You are a helpful and precise assistant for segmenting and labeling sentences. We would like to request your help on curating a dataset for entity-level hallucination detection.
				\newline \newline
                We will give you a machine generated biography and a list of checked facts about the biography. Each fact consists of a sentence and a label (True/False). Please do the following process. First, breaking down the biography into words. Second, by referring to the provided list of facts, merging some broken down words in the previous step to form meaningful entities. For example, ``strategic thinking'' should be one entity instead of two. Third, according to the labels in the list of facts, labeling each entity as True or False. Specifically, for facts that share a similar sentence structure (\eg, \textit{``He was born on Mach 9, 1941.''} (\texttt{True}) and \textit{``He was born in Ramos Mejia.''} (\texttt{False})), please first assign labels to entities that differ across atomic facts. For example, first labeling ``Mach 9, 1941'' (\texttt{True}) and ``Ramos Mejia'' (\texttt{False}) in the above case. For those entities that are the same across atomic facts (\eg, ``was born'') or are neutral (\eg, ``he,'' ``in,'' and ``on''), please label them as \texttt{True}. For the cases that there is no atomic fact that shares the same sentence structure, please identify the most informative entities in the sentence and label them with the same label as the atomic fact while treating the rest of the entities as \texttt{True}. In the end, output the entities and labels in the following format:
                \begin{itemize}[nosep]
                    \item Entity 1 (Label 1)
                    \item Entity 2 (Label 2)
                    \item ...
                    \item Entity N (Label N)
                \end{itemize}
                % \newline \newline
                Here are two examples:
                \newline\newline
                \textbf{[Example 1]}
                \newline
                [The start of the biography]
                \newline
                \textcolor{titlecolor}{Marianne McAndrew is an American actress and singer, born on November 21, 1942, in Cleveland, Ohio. She began her acting career in the late 1960s, appearing in various television shows and films.}
                \newline
                [The end of the biography]
                \newline \newline
                [The start of the list of checked facts]
                \newline
                \textcolor{anscolor}{[Marianne McAndrew is an American. (False); Marianne McAndrew is an actress. (True); Marianne McAndrew is a singer. (False); Marianne McAndrew was born on November 21, 1942. (False); Marianne McAndrew was born in Cleveland, Ohio. (False); She began her acting career in the late 1960s. (True); She has appeared in various television shows. (True); She has appeared in various films. (True)]}
                \newline
                [The end of the list of checked facts]
                \newline \newline
                [The start of the ideal output]
                \newline
                \textcolor{labelcolor}{[Marianne McAndrew (True); is (True); an (True); American (False); actress (True); and (True); singer (False); , (True); born (True); on (True); November 21, 1942 (False); , (True); in (True); Cleveland, Ohio (False); . (True); She (True); began (True); her (True); acting career (True); in (True); the late 1960s (True); , (True); appearing (True); in (True); various (True); television shows (True); and (True); films (True); . (True)]}
                \newline
                [The end of the ideal output]
				\newline \newline
                \textbf{[Example 2]}
                \newline
                [The start of the biography]
                \newline
                \textcolor{titlecolor}{Doug Sheehan is an American actor who was born on April 27, 1949, in Santa Monica, California. He is best known for his roles in soap operas, including his portrayal of Joe Kelly on ``General Hospital'' and Ben Gibson on ``Knots Landing.''}
                \newline
                [The end of the biography]
                \newline \newline
                [The start of the list of checked facts]
                \newline
                \textcolor{anscolor}{[Doug Sheehan is an American. (True); Doug Sheehan is an actor. (True); Doug Sheehan was born on April 27, 1949. (True); Doug Sheehan was born in Santa Monica, California. (False); He is best known for his roles in soap operas. (True); He portrayed Joe Kelly. (True); Joe Kelly was in General Hospital. (True); General Hospital is a soap opera. (True); He portrayed Ben Gibson. (True); Ben Gibson was in Knots Landing. (True); Knots Landing is a soap opera. (True)]}
                \newline
                [The end of the list of checked facts]
                \newline \newline
                [The start of the ideal output]
                \newline
                \textcolor{labelcolor}{[Doug Sheehan (True); is (True); an (True); American (True); actor (True); who (True); was born (True); on (True); April 27, 1949 (True); in (True); Santa Monica, California (False); . (True); He (True); is (True); best known (True); for (True); his roles in soap operas (True); , (True); including (True); in (True); his portrayal (True); of (True); Joe Kelly (True); on (True); ``General Hospital'' (True); and (True); Ben Gibson (True); on (True); ``Knots Landing.'' (True)]}
                \newline
                [The end of the ideal output]
				\newline \newline
				\textbf{User prompt}
				\newline
				\newline
				[The start of the biography]
				\newline
				\textcolor{magenta}{\texttt{\{BIOGRAPHY\}}}
				\newline
				[The ebd of the biography]
				\newline \newline
				[The start of the list of checked facts]
				\newline
				\textcolor{magenta}{\texttt{\{LIST OF CHECKED FACTS\}}}
				\newline
				[The end of the list of checked facts]
			};
		\end{tikzpicture}
        \caption{GPT-4o prompt for labeling hallucinated entities.}\label{tb:gpt-4-prompt}
	\end{center}
\vspace{-0cm}
\end{table*}
% \section{Full Experiment Results}
% \begin{table*}[th]
    \centering
    \small
    \caption{Classification Results}
    \begin{tabular}{lcccc}
        \toprule
        \textbf{Method} & \textbf{Accuracy} & \textbf{Precision} & \textbf{Recall} & \textbf{F1-score} \\
        \midrule
        \multicolumn{5}{c}{\textbf{Zero Shot}} \\
                Zero-shot E-eyes & 0.26 & 0.26 & 0.27 & 0.26 \\
        Zero-shot CARM & 0.24 & 0.24 & 0.24 & 0.24 \\
                Zero-shot SVM & 0.27 & 0.28 & 0.28 & 0.27 \\
        Zero-shot CNN & 0.23 & 0.24 & 0.23 & 0.23 \\
        Zero-shot RNN & 0.26 & 0.26 & 0.26 & 0.26 \\
DeepSeek-0shot & 0.54 & 0.61 & 0.54 & 0.52 \\
DeepSeek-0shot-COT & 0.33 & 0.24 & 0.33 & 0.23 \\
DeepSeek-0shot-Knowledge & 0.45 & 0.46 & 0.45 & 0.44 \\
Gemma2-0shot & 0.35 & 0.22 & 0.38 & 0.27 \\
Gemma2-0shot-COT & 0.36 & 0.22 & 0.36 & 0.27 \\
Gemma2-0shot-Knowledge & 0.32 & 0.18 & 0.34 & 0.20 \\
GPT-4o-mini-0shot & 0.48 & 0.53 & 0.48 & 0.41 \\
GPT-4o-mini-0shot-COT & 0.33 & 0.50 & 0.33 & 0.38 \\
GPT-4o-mini-0shot-Knowledge & 0.49 & 0.31 & 0.49 & 0.36 \\
GPT-4o-0shot & 0.62 & 0.62 & 0.47 & 0.42 \\
GPT-4o-0shot-COT & 0.29 & 0.45 & 0.29 & 0.21 \\
GPT-4o-0shot-Knowledge & 0.44 & 0.52 & 0.44 & 0.39 \\
LLaMA-0shot & 0.32 & 0.25 & 0.32 & 0.24 \\
LLaMA-0shot-COT & 0.12 & 0.25 & 0.12 & 0.09 \\
LLaMA-0shot-Knowledge & 0.32 & 0.25 & 0.32 & 0.28 \\
Mistral-0shot & 0.19 & 0.23 & 0.19 & 0.10 \\
Mistral-0shot-Knowledge & 0.21 & 0.40 & 0.21 & 0.11 \\
        \midrule
        \multicolumn{5}{c}{\textbf{4 Shot}} \\
GPT-4o-mini-4shot & 0.58 & 0.59 & 0.58 & 0.53 \\
GPT-4o-mini-4shot-COT & 0.57 & 0.53 & 0.57 & 0.50 \\
GPT-4o-mini-4shot-Knowledge & 0.56 & 0.51 & 0.56 & 0.47 \\
GPT-4o-4shot & 0.77 & 0.84 & 0.77 & 0.73 \\
GPT-4o-4shot-COT & 0.63 & 0.76 & 0.63 & 0.53 \\
GPT-4o-4shot-Knowledge & 0.72 & 0.82 & 0.71 & 0.66 \\
LLaMA-4shot & 0.29 & 0.24 & 0.29 & 0.21 \\
LLaMA-4shot-COT & 0.20 & 0.30 & 0.20 & 0.13 \\
LLaMA-4shot-Knowledge & 0.15 & 0.23 & 0.13 & 0.13 \\
Mistral-4shot & 0.02 & 0.02 & 0.02 & 0.02 \\
Mistral-4shot-Knowledge & 0.21 & 0.27 & 0.21 & 0.20 \\
        \midrule
        
        \multicolumn{5}{c}{\textbf{Suprevised}} \\
        SVM & 0.94 & 0.92 & 0.91 & 0.91 \\
        CNN & 0.98 & 0.98 & 0.97 & 0.97 \\
        RNN & 0.99 & 0.99 & 0.99 & 0.99 \\
        % \midrule
        % \multicolumn{5}{c}{\textbf{Conventional Wi-Fi-based Human Activity Recognition Systems}} \\
        E-eyes & 1.00 & 1.00 & 1.00 & 1.00 \\
        CARM & 0.98 & 0.98 & 0.98 & 0.98 \\
\midrule
 \multicolumn{5}{c}{\textbf{Vision Models}} \\
           Zero-shot SVM & 0.26 & 0.25 & 0.25 & 0.25 \\
        Zero-shot CNN & 0.26 & 0.25 & 0.26 & 0.26 \\
        Zero-shot RNN & 0.28 & 0.28 & 0.29 & 0.28 \\
        SVM & 0.99 & 0.99 & 0.99 & 0.99 \\
        CNN & 0.98 & 0.99 & 0.98 & 0.98 \\
        RNN & 0.98 & 0.99 & 0.98 & 0.98 \\
GPT-4o-mini-Vision & 0.84 & 0.85 & 0.84 & 0.84 \\
GPT-4o-mini-Vision-COT & 0.90 & 0.91 & 0.90 & 0.90 \\
GPT-4o-Vision & 0.74 & 0.82 & 0.74 & 0.73 \\
GPT-4o-Vision-COT & 0.70 & 0.83 & 0.70 & 0.68 \\
LLaMA-Vision & 0.20 & 0.23 & 0.20 & 0.09 \\
LLaMA-Vision-Knowledge & 0.22 & 0.05 & 0.22 & 0.08 \\

        \bottomrule
    \end{tabular}
    \label{full}
\end{table*}




\end{document}

\caption{
EM (above) and F1 (below) of different models and baselines across languages on \ourdataset.
Avg. denotes the average performance of the baseline across all languages.
The best results of each model under each language are annotated in \textbf{bold}. 
}
\label{tab:main}
\end{table*}

\paragraph{Metrics}
% 我们使用Exact Match (EM)和F1衡量答案的正确性,follow前人工作
We use Exact Match (EM) and F1 score to evaluate the answers, following prior works \cite{chen-etal-2020-hybridqa,zhu-etal-2021-tat}. 
% EM是指模型的预测结果与目标答案完全一致的比例
EM refers to the proportion of predictions that exactly match the gold answer, and F1 measures the degree of overlap between the predicted and the gold answer in terms of their bag-of-words representation.
% F1则衡量了预测答案和真实答案之间词袋的重叠程度

\paragraph{Models}
% 我们使用了开源模型Llama3.1-Instruct (Llama3.1)和闭源模型gpt-4o来评测我们的数据集
We evaluate \ourdataset using the open-source model Llama3.1-Instruct (Llama3.1)~\cite{dubey-etal-2024-llama3.1} and the closed-source model \texttt{gpt-4o}~\cite{openai2024gpt4technicalreport}. 
% Llama3.1是目前表现最好的开源模型之一
Llama3.1 is currently one of the best-performing open-source models, and \texttt{gpt-4o} is considered one of the leading closed-source models.
% gpt-4o则是当前最优秀的闭源模型之一


\paragraph{Baselines}
% 我们将我们的方法和以下baseline比较,follow前人工作
We compare \ourmethod with the following baselines with three-shot prompts, following previous works \cite{shi2023MGSM,li-etal-2024-eliciting-Multilingual-code}.
\begin{itemize}
    % - Native-CoT:用原语言的CoT解答问题
    \item Native-CoT: solving the question using CoT~\cite{wei2022chain-of-thought} in the native language
    % - En-CoT:用英语CoT解答问题
    \item En-CoT: solving the question using CoT in English
    % - Native-PoT:用原语言的instruction提示LLM生成代码回答问题
    \item Native-PoT: prompting the LLM to generate code in the native language \cite{pal,pot}
    % - En-PoT:用英语的instruction提示LLM生成代码回答问题
    \item En-PoT: prompting the LLM to generate code in English 
    % Three-Agent是TAT-QA数据集上的SOTA方法,由3个agent构成:analyst agent负责抽取相关数据并计算,两个critic agent分别负责判断抽取和计算的正确性,并据此进行修改
    \item Three-Agent~\cite{fatemi2024three-agent} is the state-of-the-art method on the TAT-QA dataset. It consists of three agents: the analyst agent extracts relevant data and performs computations, and two critic agents evaluate the correctness of extraction and computation, respectively, and refine the results accordingly. 
    % 受限于计算资源,我们没有在gpt-4o上评测Three-Agent在我们数据集上的性能
    Due to computational resource limitations, we do not evaluate the performance of Three-Agent on \ourdataset using \texttt{gpt-4o}.
\end{itemize}
We present prompts for baselines and \ourmethod in Appendix~\ref{sec:prompt}.
% 我们还在附录中提供了直接回答和将输入翻译为英文之后推理的结果
Additionally, we provide results for both directly answering the question and reasoning after translating the input into English in Appendix~\ref{subsec:otherbaselines}.


\subsection{Main Experiments}
\label{subsec:main experiments}
% 在不同语言上我们的方法和其他baseline的对比如表所示
A comparison of \ourmethod with other baselines across different languages is presented in Table~\ref{tab:main}. 
% 可以发现
We observe that:
% 1. 非英语语言的TATQA性能相比英语性能平均下降%,证明了我们数据集的必要性
(\emph{i})~The performance on \ourdataset in non-English languages shows an average decrease of $19.4\%$ compared to English, underscoring the necessity of \ourdataset.
% 2. 我们的方法相比其他baseline平均提升%,弥合了%不同语言之间的性能差距,证明了我们方法的有效性
(\emph{ii})~\ourmethod demonstrates an average improvement of $3.3$ on EM and F1 over other baselines, reducing the performance gap between different languages by $23.2\%$, which validates the effectiveness.
% 3. 即使提升,所有baseline在所有语言上的EM和F1均低于40,证明了我们数据集的挑战性
(\emph{iii})~Despite these improvements, the EM and F1 of all baselines remain below $40$, highlighting the challenges of \ourdataset.


\paragraph{Baselines}
% 我们方法一致地超越了Three-Agent,是因为Three-Agent不完全适用于不需要计算的HybridQA数据集,和需要复杂计算很难依靠模型本身计算能力求解的SciTAT数据集,并且在非英语语言上,multi-agent的性能下降
(\emph{i})~\ourmethod consistently outperforms Three-Agent because Three-Agent is not fully suited to HybridQA, which does not require computations \cite{chen-etal-2020-hybridqa}, or SciTAT, which involves complex calculations that are challenging to the inherent capabilities of models \cite{zhang2024scitat}. 
Additionally, the performance of multi-agent declines in non-English languages \cite{beyer2024clembench,chen-etal-2024-Cross-Agent}.
% 用原语言推理和用英语推理之间的性能差异不大
(\emph{ii})~The performance difference between reasoning in the native language and English is minimal. 
% 虽然模型在英文上展现出更强的推理能力,但由于TATQA任务相比其他任务,如数值推理任务,更需要模型将问题中的实体对应到文本或表格中的实体的能力,而这一问题在跨语言推理时会更具有挑战
Although LLMs demonstrate stronger reasoning capabilities in English, the TATQA, compared to other tasks, relies more heavily on the capabilities of linking information, which presents greater challenges in cross-lingual reasoning \cite{min-etal-2019-cspider}. 
% 而我们的方法缓解了这一挑战,所以提升了性能
Therefore, \ourmethod mitigates this challenge, leading to improved performance. 
% 并且,PoT方法在我们数据集上的性能普遍优于CoT,因为我们数据集中数值推理类问题占较大比例(见表1),所以更适合PoT方法求解
(\emph{iii})~PoT consistently outperforms CoT because numerical reasoning questions constitute a significant proportion of \ourdataset (see Table~\ref{tab:answer_statistics}), making PoT more suitable for solving these questions \cite{pot,zhao-etal-2024-docmath}.

\paragraph{Languages}
% 模型普遍表现出在高资源语言,包括英语,德语,西班牙语,法语,俄语和中文上的性能普遍较高,而在低资源语言上的性能较差
The models generally exhibit high performance on high-resource languages, such as English, German, Spanish, French, Russian, and Chinese, while their performance on low-resource languages tends to be poor. 
% 而多语言能力越强的模型,在不同语言之间性能的差距越小,其中gpt-4o展现出了最强的性能
Moreover, models with stronger multilingual capabilities show smaller performance gaps across languages, with \texttt{gpt-4o} demonstrating the highest performance. 
% 这也证明了在有挑战的任务上评测模型多语言性能的必要性
This also underscores the necessity of evaluating multilingual performance on challenging tasks.

% \begin{figure*}[t]
%     \centering
%     \resizebox{0.75\linewidth}{!}{
\begin{tikzpicture}
\begin{polaraxis}[
    title={\empty},
    xlabel={\empty},
    ylabel={\empty},
    xtick={0,32.73,65.46,98.19,130.92,163.65,196.38,229.11,261.84,294.57,327.30},
    xticklabels={bn, de, en, es, fr, ja, ru, sw, te, th, zh},
    % xticklabel style={font=\small, yshift=2ex}, % 标签离圆圈远一些,调整 y 轴偏移
    ytick={10,20,30,40},
    yticklabels={10,20,30,40},
    grid=both,
    tick label style={font=\small},
    major grid style={solid, gray}, % 外层圆圈设为黑色实线
    minor grid style={solid, gray}, % 次要网格线也设为黑色实线(可选)
    % axis line style={solid, black},   % 径向线设为灰色实线
    axis line style={draw=none},
    axis on top=true,                % 确保轴线在顶部
    grid style={gray},              % 确保网格线颜色为黑色
    line join=bevel,
    tick align=outside,
    % major grid style={solid, gray},
    % minor grid style={dotted, gray},
    % axis line style={draw=none},
    % axis on top
]

% 第一组数据
\addplot[
    color=data_blue!300,
    mark=diamond,
    style={line width=1pt},
    fill=data_blue!250,fill opacity=0.2
] coordinates {
    (0,24.2)(32.73,24.2)(65.46,26.4)(98.19,24.2)(130.92,23.1)(163.65,26.4)(196.38,20.9)(229.11,24.2)(261.84,23.1)(294.57,25.3)(327.30,26.4)(0,24.2)
} -- cycle;

% 第二组数据
\addplot[
    color=reasoner_green!300,
    mark=square,
    style={line width=1pt},
    fill=reasoner_green!250,fill opacity=0.2
] coordinates {
    (0,13.0)(32.73,20.4)(65.46,27.8)(98.19,25.9)(130.92,25.9)(163.65,22.2)(196.38,22.2)(229.11,33.3)(261.84,14.8)(294.57,20.4)(327.30,18.5)(0,13.0)
} -- cycle;


% 第三组数据
\addplot[
    color=annotator_pink!150,
    mark=triangle,
    style={line width=1pt},
    fill=annotator_pink,fill opacity=0.2
] coordinates {
    (0,29.5)(32.73,35.2)(65.46,37.1)(98.19,37.1)(130.92,30.5)(163.65,29.5)(196.38,39.0)(229.11,35.2)(261.84,26.7)(294.57,31.4)(327.30,34.3)(0,29.5)
} -- cycle;

\legend{Table, Text, Hybrid}

\end{polaraxis}
\end{tikzpicture}
}

%     \vspace{-0.5em}
%     \caption{
%         The EM of \ourmethod across different answer sources on Llama3.1-70B.
%     }
%     \label{fig:answer_source}
%     \vspace{-1em}
% \end{figure*}

\begin{figure}[t]
    \centering
    \resizebox{0.75\linewidth}{!}{
\begin{tikzpicture}
\begin{polaraxis}[
    title={\empty},
    xlabel={\empty},
    ylabel={\empty},
    xtick={0,32.73,65.46,98.19,130.92,163.65,196.38,229.11,261.84,294.57,327.30},
    xticklabels={bn, de, en, es, fr, ja, ru, sw, te, th, zh},
    % xticklabel style={font=\small, yshift=2ex}, % 标签离圆圈远一些,调整 y 轴偏移
    ytick={10,20,30,40},
    yticklabels={10,20,30,40},
    grid=both,
    tick label style={font=\small},
    major grid style={solid, gray}, % 外层圆圈设为黑色实线
    minor grid style={solid, gray}, % 次要网格线也设为黑色实线(可选)
    % axis line style={solid, black},   % 径向线设为灰色实线
    axis line style={draw=none},
    axis on top=true,                % 确保轴线在顶部
    grid style={gray},              % 确保网格线颜色为黑色
    line join=bevel,
    tick align=outside,
    % major grid style={solid, gray},
    % minor grid style={dotted, gray},
    % axis line style={draw=none},
    % axis on top
]

% 第一组数据
\addplot[
    color=data_blue!300,
    mark=diamond,
    style={line width=1pt},
    fill=data_blue!250,fill opacity=0.2
] coordinates {
    (0,24.2)(32.73,24.2)(65.46,26.4)(98.19,24.2)(130.92,23.1)(163.65,26.4)(196.38,20.9)(229.11,24.2)(261.84,23.1)(294.57,25.3)(327.30,26.4)(0,24.2)
} -- cycle;

% 第二组数据
\addplot[
    color=reasoner_green!300,
    mark=square,
    style={line width=1pt},
    fill=reasoner_green!250,fill opacity=0.2
] coordinates {
    (0,13.0)(32.73,20.4)(65.46,27.8)(98.19,25.9)(130.92,25.9)(163.65,22.2)(196.38,22.2)(229.11,33.3)(261.84,14.8)(294.57,20.4)(327.30,18.5)(0,13.0)
} -- cycle;


% 第三组数据
\addplot[
    color=annotator_pink!150,
    mark=triangle,
    style={line width=1pt},
    fill=annotator_pink,fill opacity=0.2
] coordinates {
    (0,29.5)(32.73,35.2)(65.46,37.1)(98.19,37.1)(130.92,30.5)(163.65,29.5)(196.38,39.0)(229.11,35.2)(261.84,26.7)(294.57,31.4)(327.30,34.3)(0,29.5)
} -- cycle;

\legend{Table, Text, Hybrid}

\end{polaraxis}
\end{tikzpicture}
}

    \caption{
        The EM of \ourmethod across different answer sources on \ourdataset using Llama3.1-70B.
    }
    \label{fig:answer_source}
\end{figure}

\paragraph{Answer Source}
% 我们分析了我们的方法使用Llama3.1-70B时在不同答案来源上的性能,如图所示
We analyze the performance of \ourmethod using Llama3.1-70B across different answer sources, as shown in Figure~\ref{fig:answer_source}. 
% 使用其他模型时以及不同的基线在不同来源上的性能在附录中提供
The performance with other models and baselines across answer sources is provided in Appendix~\ref{subsec:appendix_answer_source}. 
% 可以发现
The results show that:
% 1. 答案来源为Hybrid的问题的性能整体上优于单一答案来源的问题
(\emph{i})~The performance of the hybrid answer source generally outperforms those with a single answer source. 
% 因为我们的方法相比基线(见表)可以同时从表格和文本中链接到相关信息,整合了异质的相关上下文,一定程度上缓解了混合的答案来源的挑战
Since \ourmethod, compared to other baselines (see Figure~\ref{fig:answer_sources_other_baselines}), enhances the links between the question and the context, integrating hybrid contextual information and alleviating the challenge.
% 2. 虽然整体上英语的性能最佳,但在不同的答案来源上不同语言展现出了不同的趋势
% While English performs best overall, different languages exhibit varying trends across answer sources. 
% 不同语言在答案来源上的性能除了与高资源或低资源有关,还与语言本身的特性有关
(\emph{ii})~The performance across answer sources is influenced not only by the availability of language-specific resources but also by the characteristics of the language.
% 比如,词法结构复杂的语言在答案来源是text上的性能较差,如德语、俄语等
For instance, languages with complex morphological structures, such as German and Russian, perform worse when the answer source is text. 
% 而斯瓦希里语在文本上的性能最高,因为其有着较为简单的词法结构,能相对容易地根据问题中的实体链接到文本中的实体
In contrast, Swahili shows the highest performance on text-based sources, as its simpler morphology allows for easier linking of entities in the text to those in question \cite{tuan-nguyen-etal-2020-Vietnamese,zhang-etal-2023-xsemplr}.

% \begin{figure*}[t]
%     \centering
%     \resizebox{0.75\linewidth}{!}{
\begin{tikzpicture}
\begin{polaraxis}[
    title={\empty},
    xlabel={\empty},
    ylabel={\empty},
    xtick={0,32.73,65.46,98.19,130.92,163.65,196.38,229.11,261.84,294.57,327.30},
    xticklabels={bn, de, en, es, fr, ja, ru, sw, te, th, zh},
    % xticklabel style={font=\small, yshift=2ex}, % 标签离圆圈远一些,调整 y 轴偏移
    ytick={20,40,60,80,100},
    yticklabels={20,40,60,80,100},
    grid=both,
    tick label style={font=\small},
    major grid style={solid, gray}, % 外层圆圈设为黑色实线
    minor grid style={solid, gray}, % 次要网格线也设为黑色实线(可选)
    % axis line style={solid, black},   % 径向线设为灰色实线
    axis line style={draw=none},
    axis on top=true,                % 确保轴线在顶部
    grid style={gray},              % 确保网格线颜色为黑色
    line join=bevel,
    tick align=outside,
    % major grid style={solid, gray},
    % minor grid style={dotted, gray},
    % axis line style={draw=none},
    % axis on top
]

% 第一组数据
\addplot[
    color=data_blue!300,
    mark=diamond,
    style={line width=1pt},
    fill=data_blue!250,fill opacity=0.2
] coordinates {
    (0,10.8)(32.73,12.1)(65.46,16.6)(98.19,16.6)(130.92,12.1)(163.65,13.4)(196.38,13.4)(229.11,15.9)(261.84,12.1)(294.57,14.0)(327.30,12.3)(0,10.8)
} -- cycle;

% 第二组数据
\addplot[
    color=reasoner_green!300,
    mark=square,
    style={line width=1pt},
    fill=reasoner_green!250,fill opacity=0.2
] coordinates {
    (0,44.6)(32.73,50.6)(65.46,51.8)(98.19,48.2)(130.92,47.0)(163.65,48.2)(196.38,51.8)(229.11,53.0)(261.84,41.0)(294.57,47.0)(327.30,52.5)(0,44.6)
} -- cycle;

% 第三组数据
\addplot[
    color=annotator_pink!150,
    mark=triangle,
    style={line width=1pt},
    fill=annotator_pink,fill opacity=0.2
] coordinates {
    (0,60.0)(32.73,90.0)(65.46,90.0)(98.19,90.0)(130.92,90.0)(163.65,60.0)(196.38,80.0)(229.11,80.0)(261.84,40.0)(294.57,60.0)(327.30,90.0)(0,60.0)
} -- cycle;

\legend{Span, Arithmetic, Count}

\end{polaraxis}
\end{tikzpicture}
}

%     \vspace{-0.5em}
%     \caption{
%         The EM of \ourmethod across different answer types on Llama3.1-70B.
%     }
%     \label{fig:answer_type}
%     \vspace{-1em}
% \end{figure*}

\begin{figure}[t]
    \centering
    \resizebox{0.75\linewidth}{!}{
\begin{tikzpicture}
\begin{polaraxis}[
    title={\empty},
    xlabel={\empty},
    ylabel={\empty},
    xtick={0,32.73,65.46,98.19,130.92,163.65,196.38,229.11,261.84,294.57,327.30},
    xticklabels={bn, de, en, es, fr, ja, ru, sw, te, th, zh},
    % xticklabel style={font=\small, yshift=2ex}, % 标签离圆圈远一些,调整 y 轴偏移
    ytick={20,40,60,80,100},
    yticklabels={20,40,60,80,100},
    grid=both,
    tick label style={font=\small},
    major grid style={solid, gray}, % 外层圆圈设为黑色实线
    minor grid style={solid, gray}, % 次要网格线也设为黑色实线(可选)
    % axis line style={solid, black},   % 径向线设为灰色实线
    axis line style={draw=none},
    axis on top=true,                % 确保轴线在顶部
    grid style={gray},              % 确保网格线颜色为黑色
    line join=bevel,
    tick align=outside,
    % major grid style={solid, gray},
    % minor grid style={dotted, gray},
    % axis line style={draw=none},
    % axis on top
]

% 第一组数据
\addplot[
    color=data_blue!300,
    mark=diamond,
    style={line width=1pt},
    fill=data_blue!250,fill opacity=0.2
] coordinates {
    (0,10.8)(32.73,12.1)(65.46,16.6)(98.19,16.6)(130.92,12.1)(163.65,13.4)(196.38,13.4)(229.11,15.9)(261.84,12.1)(294.57,14.0)(327.30,12.3)(0,10.8)
} -- cycle;

% 第二组数据
\addplot[
    color=reasoner_green!300,
    mark=square,
    style={line width=1pt},
    fill=reasoner_green!250,fill opacity=0.2
] coordinates {
    (0,44.6)(32.73,50.6)(65.46,51.8)(98.19,48.2)(130.92,47.0)(163.65,48.2)(196.38,51.8)(229.11,53.0)(261.84,41.0)(294.57,47.0)(327.30,52.5)(0,44.6)
} -- cycle;

% 第三组数据
\addplot[
    color=annotator_pink!150,
    mark=triangle,
    style={line width=1pt},
    fill=annotator_pink,fill opacity=0.2
] coordinates {
    (0,60.0)(32.73,90.0)(65.46,90.0)(98.19,90.0)(130.92,90.0)(163.65,60.0)(196.38,80.0)(229.11,80.0)(261.84,40.0)(294.57,60.0)(327.30,90.0)(0,60.0)
} -- cycle;

\legend{Span, Arithmetic, Count}

\end{polaraxis}
\end{tikzpicture}
}

    \vspace{-0.5em}
    \caption{
        The EM of \ourmethod across different answer types on \ourdataset using Llama3.1-70B.
    }
    \label{fig:answer_type}
    \vspace{-1em}
\end{figure}

\paragraph{Answer Type}
% 我们比较了我们方法在Llama3.1-70B上在不同答案类型上的性能,如图所示
We compare the performance of \ourmethod using Llama3.1-70B on different answer types, as shown in Figure~\ref{fig:answer_type}. 
% 我们在附录提供了其他模型和基线在不同答案类型上的性能
Results of other models and baselines across answer types are provided in Appendix~\ref{subsec:appendix_answer_types}. 
% 我们发现
We observe that:
% 1. 模型在Span类型和Arithmetic上的性能不好,而在Count类型上的性能最高
% (\emph{i})~The model performs poorly on Span and Arithmetic types, while it performs best on the Count type. 
(\emph{i})~The model performs best on the Count type. 
% 因为span类型的答案需要从表格和文本中抽取出短语,或用几句话总结分析,相比较Arithmetic和Count的答案都为数字,对词语的构成和顺序更为敏感
This is because Span answers require extracting short phrases or summarizing conclusions from tables and text, making them more sensitive to word composition and order. 
% 而Arithmetic类型相比较Count类型需要模型进行更加复杂的运算
Additionally, Arithmetic answers involve more complex computations than Count answers.
% 2. 模型在不同答案类型上的性能整体上呈现高资源语言优于低资源语言
(\emph{ii})~The model performs better on high-resource languages than low-resource languages across answer types overall. 
% 即使我们的方法减小了性能差距,但低资源语言和高资源语言之间仍在不同的答案类型上均存在较大的性能差距
Although \ourmethod narrows the performance gap, there remains a significant difference between high-resource and low-resource languages for all answer types.

\subsection{Analysis}
% 在本小节,我们主要分析模型在TATQA任务上的跨语言能力

\begin{table*}[ht]
\centering
\tiny
% \begin{tabular}{llcccccccccccc}
% \toprule
% \textbf{Instruction} & \textbf{Demo} & \textbf{bn} & \textbf{de} & \textbf{en} & \textbf{es} & \textbf{fr} & \textbf{ja} & \textbf{ru} & \textbf{sw} & \textbf{te} & \textbf{th} & \textbf{zh} & \textbf{Average} \\
% \midrule
% \multirow{3}{*}{Native} & Native & $20.0/\bm{23.8}$ & $28.4/\bm{33.9}$ & $28.4/\bm{33.8}$ & $29.2/\bm{35.8}$ & $29.2/\bm{34.0}$ & $27.6/\bm{30.1}$ & $27.6/\bm{31.7}$ & $\bm{32.0}/35.1$ & $20.4/\bm{24.2}$ & $25.2/\bm{28.3}$ & $28.8/\bm{37.4}$ & $27.0/\bm{31.7}$ \\
% & Multi & $22.0/\bm{24.6}$ & $\bm{30.0}/32.3$ & $30.4/\bm{35.4}$ & $30.4/\bm{35.0}$ & $28.4/\bm{31.6}$ & $26.0/\bm{27.6}$ & $26.4/\bm{28.8}$ & $28.8/\bm{30.7}$ & $24.4/\bm{26.3}$ & $24.4/\bm{26.7}$ & $24.8/\bm{30.7}$ & $26.9/\bm{30.0}$ \\
% & En & $20.8/\bm{24.4}$ & $29.2/\bm{33.6}$ & $28.4/\bm{33.8}$ & $24.8/\bm{30.6}$ & $27.2/\bm{32.0}$ & $24.0/\bm{22.8}$ & $28.4/\bm{31.8}$ & $29.2/\bm{31.6}$ & $19.6/\bm{22.3}$ & $21.2/\bm{23.5}$ & $24.4/\bm{30.7}$ & $24.9/\bm{28.8}$ \\
% \midrule
% \multirow{3}{*}{En} & Native & $\bm{27.6}/30.5$ & $28.6/\bm{30.3}$ & $28.4/\bm{33.8}$ & $29.6/\bm{34.1}$ & $25.2/\bm{29.6}$ & $25.6/\bm{28.3}$ & $29.2/\bm{33.0}$ & $30.0/\bm{33.6}$ & $28.0/\bm{31.4}$ & $26.8/\bm{34.1}$ & $27.6/\bm{31.5}$ \\
% & Multi & $26.4/\bm{28.9}$ & $27.2/\bm{29.9}$ & $\bm{30.4}/35.4$ & $30.8/\bm{34.1}$ & $29.6/\bm{32.2}$ & $29.2/\bm{31.7}$ & $30.0/\bm{32.6}$ & $30.0/\bm{33.6}$ & $27.2/\bm{29.8}$ & $27.2/\bm{31.2}$ & $28.8/\bm{34.1}$ & $\bm{28.8}/32.1$ \\
% & En & $24.0/\bm{26.3}$ & $28.0/\bm{31.3}$ & $31.2/\bm{35.3}$ & $29.2/\bm{34.1}$ & $26.8/\bm{31.1}$ & $26.8/\bm{29.4}$ & $28.8/\bm{33.5}$ & $30.8/\bm{34.7}$ & $22.8/\bm{25.9}$ & $26.8/\bm{30.5}$ & $28.0/\bm{34.9}$ & $27.6/\bm{31.6}$ \\
% \bottomrule
% \end{tabular}

% \begin{tabular}{llcccccc}
% \toprule
% \textbf{Instruction} & \textbf{Demo} & \textbf{bn} & \textbf{de} & \textbf{en} & \textbf{es} & \textbf{fr} & \textbf{ja}\\
% \midrule
% \multirow{3}{*}{Native} & Native & $20.0/23.8$ & $28.4/33.9$ & $28.4/33.8$ & $29.2/35.8$ & $29.2/\bm{34.0}$ & $27.6/30.1$ \\
% & Multi & $22.0/24.6$ & $\bm{30.0/32.3}$ & $30.4/\bm{35.4}$ & $30.4/\bm{35.0}$ & $28.4/\bm{31.6}$ & $26.0/27.6$ \\
% & En & $20.8/24.4$ & $29.2/33.6$ & $28.4/33.8$ & $24.8/30.2$ & $27.2/32.0$ & $24.0/22.8$ \\
% \midrule
% \multirow{3}{*}{En} & Native & $\bm{27.6/30.5}$ & $26.8/30.3$ & $28.4/33.8$ & $29.6/32.7$ & $25.2/29.6$ & $25.6/28.5$ \\
% & Multi & $26.4/28.9$ & $27.2/29.9$ & $30.4/\bm{35.4}$ & $\bm{30.8}/34.1$ & $\bm{29.6}/32.2$ & $\bm{29.2/31.7}$ \\
% & En & $24.0/26.3$ & $28.0/31.3$ & $\bm{31.2}/35.3$ & $29.2/34.6$ & $26.8/31.1$ & $26.8/29.4$ \\
% \bottomrule
% \end{tabular}

% \begin{tabular}{llcccccc}
% \toprule
% \textbf{Instruction} & \textbf{Demo} & \textbf{ru} & \textbf{sw} & \textbf{te} & \textbf{th} & \textbf{zh} & \textbf{Avg.} \\
% \midrule
% \multirow{3}{*}{Native} & Native & $27.6/31.7$ & $\bm{32.0/35.1}$ & $20.4/24.2$ & $25.2/28.3$ & $\bm{28.8/37.4}$ & $27.0/31.7$ \\
% & Multi & $26.4/28.8$ & $28.8/30.7$ & $24.4/26.3$ & $24.4/26.7$ & $24.8/30.7$ & $26.9/30.0$ \\
% & En & $28.4/31.8$ & $29.2/31.6$ & $19.6/22.3$ & $21.2/23.5$ & $24.4/30.7$ & $24.9/28.8$ \\
% \midrule
% \multirow{3}{*}{En} & Native & $29.2/33.0$ & $30.0/33.6$ & $26.0/28.8$ & $\bm{28.0/31.4}$ & $26.8/34.1$ & $27.6/31.5$ \\
% & Multi & $\bm{30.0}/32.6$ & $30.0/33.0$ & $\bm{27.2/29.8}$ & $27.2/31.2$ & $\bm{28.8}/34.7$ & $\bm{28.8/32.1}$ \\
% & En & $28.8/\bm{33.5}$ & $30.8/34.7$ & $22.8/25.9$ & $26.8/30.5$ & $28.0/34.9$ & $27.6/31.6$ \\
% \bottomrule
% \end{tabular}

\begin{tabular}{l|l|ccccccccccc|c}
\toprule
\textbf{Instruction} & \textbf{Demo} & \textbf{bn} & \textbf{de} & \textbf{en} & \textbf{es} & \textbf{fr} & \textbf{ja} & \textbf{ru} & \textbf{sw} & \textbf{te} & \textbf{th} & \textbf{zh} & \textbf{Avg.} \\
\midrule
\multirow{3}{*}{Native} & Native & $20.0$ & $28.4$ & $28.4$ & $29.2$ & $29.2$ & $27.6$ & $27.6$ & \bm{$32.0$} & $20.4$ & $25.2$ & \bm{$28.8$} & $27.0$ \\
& Multi & $22.0$ & \bm{$30.0$} & $30.4$ & $30.4$ & $28.4$ & $26.0$ & $26.4$ & $28.8$ & $24.4$ & $24.4$ & $24.8$ & $26.9$ \\
& En & $20.8$ & $29.2$ & $28.4$ & $24.8$ & $27.2$ & $24.0$ & $28.4$ & $29.2$ & $19.6$ & $21.2$ & $24.4$ & $24.9$ \\
\midrule
\multirow{3}{*}{En} & Native & \bm{$27.6$} & $26.8$ & $28.4$ & $29.6$ & $25.2$ & $25.6$ & $29.2$ & $30.0$ & $26.0$ & \bm{$28.0$} & $26.8$ & $27.6$ \\
& Multi & $26.4$ & $27.2$ & $30.4$ & \bm{$30.8$} & \bm{$29.6$} & \bm{$29.2$} & \bm{$30.0$} & $30.0$ & \bm{$27.2$} & $27.2$ & \bm{$28.8$} & \bm{$28.8$} \\
& En & $24.0$ & $28.0$ & \bm{$31.2$} & $29.2$ & $26.8$ & $26.8$ & $28.8$ & $30.8$ & $22.8$ & $26.8$ & $28.0$ & $27.6$ \\
\bottomrule
\end{tabular}

\begin{tabular}{l|l|ccccccccccc|c}
\toprule
\textbf{Instruction} & \textbf{Demo} & \textbf{bn} & \textbf{de} & \textbf{en} & \textbf{es} & \textbf{fr} & \textbf{ja} & \textbf{ru} & \textbf{sw} & \textbf{te} & \textbf{th} & \textbf{zh} & \textbf{Avg.} \\
\midrule
\multirow{3}{*}{Native} & Native & $23.8$ & \bm{$33.9$} & $33.8$ & \bm{$35.8$} & $\bm{34.0}$ & $30.1$ & $31.7$ & \bm{$35.1$} & $24.2$ & $28.3$ & \bm{$37.4$} & $31.7$ \\
& Multi & $24.6$ & $32.3$ & $\bm{35.4}$ & $35.0$ & $31.6$ & $27.6$ & $28.8$ & $30.7$ & $26.3$ & $26.7$ & $30.7$ & $30.0$ \\
& En & $24.4$ & $33.6$ & $33.8$ & $30.2$ & $32.0$ & $22.8$ & $31.8$ & $31.6$ & $22.3$ & $23.5$ & $30.7$ & $28.8$ \\
\midrule
\multirow{3}{*}{En} & Native & \bm{$30.5$} & $30.3$ & $33.8$ & $32.7$ & $29.6$ & $28.5$ & $33.0$ & $33.6$ & $28.8$ & \bm{$31.4$} & $34.1$ & $31.5$ \\
& Multi & $28.9$ & $29.9$ & $\bm{35.4}$ & $34.1$ & $32.2$ & $\bm{31.7}$ & $32.6$ & $33.0$ & \bm{$29.8$} & $31.2$ & $34.7$ & \bm{$32.1$} \\
& En & $26.3$ & $31.3$ & $35.3$ & $34.6$ & $31.1$ & $29.4$ & $\bm{33.5}$ & $34.7$ & $25.9$ & $30.5$ & $34.9$ & $31.6$ \\
\bottomrule
\end{tabular}


\caption{
EM (above) and F1 (below) of \ourmethod using the instructions and demonstrations of different languages on Llama3.1-70B.
The best results under each language are annotated in \textbf{bold}. 
% Multi指的是由多种语言(英语、西班牙语和中文)组成的示例
Demo refers to demonstrations. 
Multi refers to demonstrations composed of multiple languages (English, Spanish, and Chinese).
Avg. denotes the average performance of the baseline across all languages.
}
\label{tab:prompt_language}
\end{table*}

% prompt的语言如何影响我们方法的性能
% How does the language of prompt affect the performance of our method?
% \subsubsection{Prompt Language}
\subsubsection{How does the Prompt Language Affect \ourmethod?}
% 我们分析了使用不同语言的instruction和示例对我们方法性能的影响,如表所示
We analyze the impact of using instructions and demonstrations in different languages on the performance of \ourmethod, as shown in Table~\ref{tab:prompt_language}. 
% 其中,多语言示例我们选择了分别是英语、西班牙语和中文的各一个示例,因为模型在这三个高资源语言上的性能较高,且包含两个语系
For the multilingual demonstrations, we select one demonstration each from English, Spanish, and Chinese, as the models perform well on these three high-resource languages, which also cover two language families. 
% 而英文instruction和英文示例是我们主实验采用的设置
The English instruction and English demonstrations are the settings of \ourmethod used in the main experiments. 
% 可以发现,
The results indicate that:


% 1. 使用英文instruction整体上优于使用原语言instruction
(\emph{i})~Using English instructions generally outperforms using native instructions.
% 2. 使用多语言示例打败了原语言和英语示例,说明当在某些语言上没有足够的TATQA示例时,可以采用同一语系或高资源语言的示例来提升模型性能
(\emph{ii})~Multilingual demonstrations outperform both native language and English demonstrations, suggesting that when sufficient native demonstrations are not available on the TATQA task, using demonstrations from the same language family or high-resource languages can also enhance performance.
% 同时,斯瓦希里语在使用原语言的instruction和示例时达到最高性能,说明了这种语言的独特性,模型不容易将在高资源语言上的知识和推理能力迁移到斯瓦西里语上
Additionally, Swahili achieves the highest performance when using instructions and examples in the native language, highlighting its uniqueness. 
% This suggests that it is difficult for the model to transfer knowledge and reasoning capabilities from high-resource languages to Swahili.

\begin{figure}[t]
    \centering
    \includegraphics[width=0.95\linewidth]{fig/heatmap_cross.pdf}
    \caption{
    % 被各个表示正确解决的instance之间的重复率
    The EM/F1 of \ourmethod with questions and context (table and text) of different languages on \ourdataset using Llama3.1-70B. 
    }
    \label{fig:heatmap_cross}
\end{figure}

% 跨语言
% \subsubsection{Cross-lingual QA}
\subsubsection{How does the Language Affect \ourmethod in the Cross-lingual Setting?}
% 我们评测了在跨语言QA,即问题和文本、表格的语言不一致的设置下我们的方法在我们数据集上的性能,如图所示
We evaluate the performance of \ourmethod in the cross-lingual setting, where the languages of the question and context are inconsistent, with results in Figure~\ref{fig:heatmap_cross}. 
% 我们分别选取了高资源语言法语和中文,以及低资源语言孟加拉语、斯瓦希里语和泰卢固语,涉及4个语系
We select high-resource languages (French and Chinese), and low-resource languages (Bengali, Swahili, and Telugu), covering $4$ language families. 
% 我们发现
Our findings include:
% 1. 普遍来讲,模型从低资源跨到高资源时会带来性能提升,反过来则下降
(\emph{i})~Generally, \ourmethod shows improved performance when transitioning from low-resource to high-resource languages, while the opposite results in a decline. 
% 比如用法语和中文对法语上下文提问的性能较高,而用三种低资源语言提问的性能较低
For instance, the performances on the French context with French and Chinese questions are relatively high, whereas the performances with three low-resource languages are lower.
% 2. 斯瓦希里语在跨语言的设置上表现出较为稳定的性能,且在同语言QA上达到最佳性能
(\emph{ii})~The model achieves the best performance when the question and context are both Swahili. 
% 因为斯瓦希里语较为规则的语法和词法结构,使其在我们的任务上受益,尤其是需要链接相关信息时相比其他语言更加容易
This can be attributed to its relatively regular grammatical and lexical structures, which provide advantages when linking related information.
% making it easier compared to other languages.

% \begin{table*}[ht]
% \centering
% \small
% 
\begin{lemma}\label{Lemma:multi1} 
   Fixing the number of data contributor $i$ collects $n_i$, and others' strategies $\strategy_{-i}$, $\hat{\mu}\left(X_i\right)$ is the minimax estimator for the Normal distribution class $\Normaldistrib := \left\{\mathcal{N}(\mu,\sigma^2) \;\middle|\; \mu \in \mathbb{R}\right\}$,
    \begin{align*}
       \hat{\mu}(X_i)  = \underset{\hat{\mu}}{\arg\min} \sbr{\sup _\mu \mathbb{E}\left[(\hat{\mu}( Y_i)- \hat{\mu}( Y_{-i}) )^2 \;\middle|\;  \mu \right] }
    \end{align*} 
     
\end{lemma}


\begin{proof}

\begin{align*}
    & \ \mathbb{E}\left[ \left( \hat{\mu}\left( Y_i \right)-\hat{\mu}\left( Y_{-i} \right)  \right)^2 \right] \\ =  & \ \mathbb{E}\left[ \left( (\hat{\mu}\left(  Y_i \right)-\mu) -(\hat{\mu}\left(  Y_{-i} \right) -\mu) \right)^2   \right] \\ =  & \ A_0 + \mathbb{E}\left[ (\hat{\mu}\left(  Y_i \right)-\mu)^2  \right]
\end{align*}
where $A_0$ is a positive coefficient.

Thus the maximum risk can be written as:

\begin{align*}
    \sup _\mu \mathbb{E}\left[A_0 + \left(\hat{\mu}\left( Y_i\right)-\mu\right)^{2} \;\middle|\;  \mu \right]
\end{align*}


We construct a lower bound on the maximum risk using a sequence of Bayesian risks. Let $\Lambda_{\ell}:=\mathcal{N}\left(0, \ell^2\right), \ell=1,2, \ldots$ be a sequence of prior for $\mu$. For fixed $\ell$, the posterior distribution is:
$$
\begin{aligned}
p\left(\mu \;\middle|\;  X_i\right) & \propto p\left(X_i \;\middle|\;  \mu\right) p(\mu) \\ & \propto \exp \left(-\frac{1}{2 \sigma^2} \sum_{x \in X_i}(x-\mu)^2\right) \exp \left(-\frac{1}{2 \ell^2} \mu^2\right) \\
& \propto \exp \left(-\frac{1}{2}\left(\frac{n_i}{\sigma^2}+\frac{1}{\ell^2}\right) \mu^2+\frac{1}{2} 2 \frac{\sum_{x \in X_i} x}{\sigma^2} \mu\right) .
\end{aligned}
$$

This means the posterior of $\mu$ given $X_i$ is Gaussian with:

\begin{align*}
    \mu \lvert\, X_i & \sim \mathcal{N}\left(\frac{n_i \hat{\mu}\left(X_i\right) / \sigma^2}{n_i / \sigma^2+1 / \ell^2}, \frac{1}{n_i / \sigma^2+1 / \ell^2}\right) 
    \\ & =: \mathcal{N}\left(\mu_{\ell}, \sigma_{\ell}^2\right).
\end{align*}



Therefore, the posterior risk is: 
$$
\begin{aligned}
&   \mathbb{E}\left[A_0 + \left(\hat{\mu}\left( Y_i\right)-\mu\right)^{2}  \;\middle|\;  X_i\right] \\ = &  \mathbb{E}\left[A_0 +  \left(\left(\hat{\mu}\left( Y_i\right)-\mu_{\ell}\right)-\left(\mu-\mu_{\ell}\right)\right)^{2 j} \;\middle|\;  X_i\right] \\ =
& A_0+\int_{-\infty}^{\infty} \underbrace{\left(e-\left(\hat{\mu}\left( Y_i\right)-\mu_{\ell}\right)\right)^2}_{=: F_1\left(e-\left(\hat{\mu}\left(Y_i\right)-\mu_{\ell}\right)\right)} \underbrace{\frac{1}{\sigma_{\ell} \sqrt{2 \pi}} \exp \left(-\frac{e^2}{2 \sigma_{\ell}^2}\right)}_{=: F_2(e)} d e
\end{aligned}
$$

Because:
\begin{itemize}
    \item $F_1(\cdot)$ is even function and increases on $[0, \infty)$;
    \item $F_2(\cdot)$ is even function and decreases on $\left[0, \infty \right)$, and $\int_{\mathbb{R}} F_2(e) de<\infty$
    \item For any $a \in \mathbb{R}, \int_{\mathbb{R}} F_1(e-a) F_2(e) de<\infty$
\end{itemize}

By the corollary of Hardy-Littlewood inequality in Lemma \ref{lemmaHardy},
$$
\int_{\mathbb{R}} F_1(e-a) F_2(e) d e \geq \int_{\mathbb{R}} F_1(e) F_2(e) d e
$$
which means the posterior risk is minimized when $\hat{\mu}\left(Y_i\right)=\mu_{\ell}$. We then write the Bayes risk as, the Bayes risk is minimized by the posterior mean $\mu_{\ell}$:

\begin{align*}    
R_{\ell}:= & \mathbb{E}\left[ A_0+\mathbb{E}\left[\left(\mu-\mu_{\ell}\right)^{2 } \;\middle|\;  X_i\right]\right] \\ = & A_0 + \sigma_{\ell}^{2}
\end{align*}

and the limit of Bayesian risk as $\ell \rightarrow \infty$ is
$$
R_{\infty}:= A_0 + \frac{\sigma^{2}}{n_i}.
$$

When $\hat{\mu}\left(Y_i\right)=\hat{\mu}\left(X_i\right)$, i.e, the contributor submit a set of size $n_i$ with each element equal to $ \hat{\mu}\left(X_i\right)$, the maximum risk is:

\begin{align*}
& \sup _\mu \mathbb{E}\left[A_0+\left(\mu- 
\hat{\mu}\left(Y_i\right) \right)^{2 } \;\middle|\;  \mu \right] \\
= & \sup _\mu \mathbb{E}\left[A_0+\left(\mu- 
\hat{\mu}\left(X_i\right) \right)^{2 } \;\middle|\;  \mu \right]  \\
= & A_0+ \sigma^{2 } n_i^{-1}  \\
= &  R_{\infty}.
\end{align*}

This implies that,
\begin{align*}
    & \underset{\mu}{\sup}\; \mathbb{E} \sbr{ \rbr{\hat{\mu}\left( Y_i \right)-\hat{\mu}\left( Y_{-i} \right)  }^2 \;\middle|\;  \mu }  \\ \geq & \; R_{\infty} =  \sup _\mu \; \mathbb{E}\left[A_0+\left(\mu- 
\hat{\mu}\left(X_i\right) \right)^{2 } \;\middle|\;  \mu \right]
\end{align*}

Therefore, the recommended strategy $\hat{\mu}(Y_i) =\hat{\mu}( X_i)$ has a smaller maximum risk than other strategies. 

\end{proof}




1. The payment from the buyer a constant $v(n^{\star})$.


2. If the payment for every seller is a fixed constant, then sellers can fabricate data without actually collecting data.\\


%$p_1 = b/2 +(\hat{\mu}(Y_1)- \hat{\mu}(Y_2))^2$, $p_2 = b/2 -(\hat{\mu}(Y_1)- \hat{\mu}(Y_2))^2$, seller 1 can choose ${\mu}' = u + \epsilon$, expected payment for seller 1 is larger than $b/2$. %NIC for seller 1: $g({\mu}',\mu ) < b/2$ for all ${\mu}' \neq \mu$. NIC for seller 2: $g( \mu, {\mu}') >  b/2 $ for all ${\mu}' \neq \mu$.

To demonstrate that no truthful mechanism (NIC) satisfies all desired properties in a two-seller setting, we use proof by contradiction.  

Suppose that there is a NIC mechanism $M$ satisfying property 1-5. Under this mechanism, the best strategy for each seller is to collect $N_i^{\star}$ amount of data and submit truthfully, where $N_1^{\star}+N_2^{\star} = n^{\star} $. Since $M$ is NIC for strategy space $\left\{ (f_i,N_i)\right\}_{i=1,2}$, it must be NIC for the sub strategy space $\left\{ (f_i, N_i^{\star})\right\}_{i=1,2}$. 


Consider the case in which everyone collects $N_i^{\star}$ data point and submits $N_i^{\star}$ data point. Assume that the true mean is $\mu$, seller $1$ submit $N({\mu}', \sigma^2),\ {\mu}' = f(\mu) $ while seller 2 submit $N({\mu}, \sigma^2) $. We denote seller 1's expected payment as  $\mathbb{E}\left[ p_1(M,\strategy) \right] = g({\mu}', \mu)$.
Seller 1's utility is then:
\[ u_1(M,f) = \underset{\mu}{\inf} \ g({\mu}', \mu) -c\times N_1^{\star}\] where $c$ is the cost for collecting one data point.


The total payment from the buyer is $v(n^{\star})$, hence by budget balance, \[p_2 (M,f)  = v(n^{\star}) -  p_1(M,f), \ \mathbb{E}\left[ p_2(M,f) \right] = v(n^{\star}) - g({\mu}', \mu) \]

By NIC, we have,
\[ \underset{\mu}{\inf} \ g({\mu}', \mu) -c\times N_1^{\star} \leq \underset{\mu}{\inf} \ g({\mu}, \mu) -c\times N_1^{\star} \] \[ \underset{\mu}{\inf} \ (v(n^{\star})- g( \mu, {\mu}')) -c\times N_2^{\star} \leq \underset{\mu}{\inf} \ (v(n^{\star})-g( \mu, {\mu})) -c\times N_2^{\star}  \]

%Using the fact that $\underset{\mu}{\inf} \ g({\mu}, \mu) = \underset{\mu}{\sup} \ g({\mu}, \mu) = {v(n^{\star})}/2 $.
We obtain that for any ${\mu}'$ and $\mu$,
\[  \underset{\mu}{\inf} \ g({\mu}', \mu) \leq \underset{\mu}{\inf} \ g({\mu}, \mu)   \] \[  \underset{\mu}{\sup} \  g( \mu, {\mu})  \leq \underset{\mu}{\sup } \ g( \mu, {\mu}')  \]

We next show that the inequalities are strict. Assume, for contradiction there exists ${\mu}'$, for any $\mu$, $g({\mu}', 
\mu) \geq \underset{\mu}{\inf} \ g({\mu}, \mu)$. It then follows that $ \underset{\mu}{\inf} \ g({\mu}', \mu) \geq \underset{\mu}{\inf} \ g({\mu}, \mu)$. Under this assumption, seller 1 could fabricate data by submitting $N({\mu}', \sigma^2)$ without collecting any actual data. This contradicts with the fact that $(f_1 = I, N_1 = N_1^{\star})$ is the best strategy for seller 1. Hence, for any ${\mu}'$, there exists some $\mu$ such that $g({\mu}', 
\mu) < \underset{\mu}{\inf} \ g({\mu}, \mu)$. Therefore, for any ${\mu}'$, \[ \underset{\mu}{\inf} \ g({\mu}', \mu) < \underset{\mu}{\inf} \ g({\mu}, \mu). \]Similarly, we also have \[ \underset{\mu}{\sup} \  g( \mu, {\mu})  < \underset{\mu}{\sup } \ g( \mu, {\mu}').  \] 


For any $ {\mu}'$, let $f({\mu}') =  \underset{\mu}{\arg\sup}\, g(\mu, {\mu}')$, then we have for any ${\mu}'$,  $f({\mu}') \neq {\mu}'$ and $ g(f({\mu}'), {\mu}') > \underset{\mu}{\sup} \  g( \mu, {\mu}) \geq  \underset{\mu}{\inf} \  g( \mu, {\mu})$. This implies that seller 1 could fabricate data based on function $f$, this contradicts with the fact that the mechanism is NIC.


pay the seller $v(n^{\star})/2 - \beta (\hat{\mu}(Y_1)-\hat{\mu}(Y_2))^2$, charge buyer $v(n^{\star}) - 2\beta (\hat{\mu}(Y_1)-\hat{\mu}(Y_2))^2$


Buyer utility: $v(n^*)$-payment
Seller utility: payment - $cn^*$.

Sellers utility is positive?

Seller payment $(v(n^*)/2)-\beta (\hat{\mu}(Y_1)-\hat{\mu}(Y_2))^2 $, $\beta = (v(n^*)-cn^*)c(n^*)^2 / 4\sigma^2$, 



\section{Multiple buyers}
\subsection{}
Question 1: Do we fix the amount of data for sale ahead of time?


Assume we fix $N$, the amount of data for sale. The goal of mechanism is to maximize the sellers' revenue. According to previous paper, there exists at least one type who purchase at the amount $N$. Suppose that in offline setting, i.e., when the mechanism knows the buyer valuation and type distribution, the optimal revenue is $\text{OPT} $. 


We ask $d$ sellers to collect $N$ data points, and split $\text{OPT} $ revenue among sellers. (data can be duplicated). 

\[ p_i(M,s) =\mathbb{I}\left( \left| Y_i \right| = \frac{N}{d} \right) \rbr{\frac{\text{OPT}}{d}+d_i \frac{\sigma^2}{N_{-i}^{\star}} +d_i \frac{\sigma^2}{N_i^{\star}} }- d_i \rbr{\hat{\mu}(Y_i)-\hat{\mu}(Y_{-i}) }^2  \]

Buyer's expected utility is non negative. Next, we discuss sellers' expected utility $T\mathbb{E}[p_i]- cn_i$ (over $T$ roundsm\, maybe $T$ is fixed). Let $N_i^{\star} = \frac{N}{d}$.

\[ u_i(M,s) = \mathbb{I}\left( \left| Y_i \right| = \frac{N}{d} \right) \rbr{\frac{\text{OPT}}{d}+d_i \frac{\sigma^2}{N_{-i}^{\star}} +d_i \frac{\sigma^2}{N_i^{\star}} }T- Td_i \mathbb{E}\rbr{\hat{\mu}(Y_i)-\hat{\mu}(Y_{-i}) }^2 
 -  cn_i \]
Choose $d_i = \frac{c(N_i^{\star})^2}{T\sigma^2}$.


If we do not fix \( T \) in advance, let \( T_0 \) represent the time at which the cumulative utility over at least \( T_0 \) rounds is non-negative. We can select \( d_i \) such that \( d_i \geq \frac{c(N/d)^2}{T_0 \sigma^2} \). This ensures that the seller will never choose to collect less than \( N/d \) amount of data.








\section{Single buyer} \label{section: singlebuyer}


Each contributor \( i \) incurs a cost \( c_i \) to collect data,  without loss of generality, we assume \( c_1 \leq c_2 \leq \dots \leq c_d \). The broker is assumed to have full knowledge of the buyer's valuation curve \( \val(n) \), as well as the contributors’ costs $ c_{ i \in \contributors}$  for collecting each data point.

The maximum total profit for the contributors, assuming no constraints on truthful submissions, is given by:
\[
\mathrm{profit}^\star = \underset{\datanum_1, \dots, \datanum_d}{\max} \left( \val \left(\sum_{i=1}^{d} \datanum_i\right) - \sum_{i=1}^{d} c_i \datanum_i \right),
\]

where \( \datanum_i \) represents the number of data points collected by contributor \( i \). In this unconstrained scenario, since contributor 1 has the lowest collection cost, the optimal strategy is for contributor 1 to collect all the required data points while other contributors collect none. This approach maximizes total profit without considering the incentive for truthful submissions.

However, when truthful submission is taken into account, at least two contributors are needed because we need to use one contributor's data to verify the other's. We demonstrate that the maximum profit achievable under Nash Equilibrium is:
\[
\mathrm{profit}^\star + (c_1 - c_2),
\]
where \( c_1 - c_2 \) represents the additional cost differential caused by enforcing truthful behavior among contributors. 

\begin{algorithm}[H]
    \caption{Process of mechanism.}
    \begin{algorithmic}
        \STATE {\bfseries Input:} A population of buyers $\buyers$.
        \STATE The broker chooses the optimal data allocation to maximize contributors' profit:
        $$
        \{ \datanum_i^{\star} \}_{i=1}^d = \underset{\datanum_1,\dots,\datanum_d}{\arg\max}\  \rbr{v\rbr{\sum_i \datanum_i}-\sum_i \cost_i \datanum_i  }
      $$
       
        \STATE The broker recommend a strategy to each contributor: $\strategy_i^{\star} = (\datanum_i^{\star}, \mathbf{I})$.
        \STATE Each contributor selects a strategy $\strati = (\datanum_i, f_i)$, collects $\datanum_i$ data points $X_i$, and submits $Y_i = f_i(X_i)$.
        \STATE The mechanism generates an estimator $\hat{\mu}(M,\strategy)$ for the buyer, and charge her $\price_{j \in \buyers}$. \COMMENT{See (\ref{eq:buyer_pay}) }
        \STATE Each contributor is paid $\payi$.    \COMMENT{See (\ref{eq:seller_pay}) }
    \end{algorithmic}   
\end{algorithm}




\begin{theorem}
    there exists NIC mechanism satisfying the following properties (1) $\strategy^{\star}$ is Nash equilibrium. (2) The mechanism is individually rational at $\strategy^{\star}$ for both buyers and sellers. (3) Budget balance. (4) Under strategy $\strategy^{\star}$, the expected profit of buyers approximates the optimal profit $ \mathrm{profit}^{\star}$ within an additive error $\cost_2 - \cost_1$. 
\end{theorem}


Let $n^{\star}$ denote the optimal total number of data to be collected, $\datanum_1^{\star}=n^{\star}-1$, and $\datanum_2^{\star}=1$. Let $w=\val(n^{\star})-cn^{\star}$ denote the social welfare. One option for payment function is

\begin{align*}
    &\; \pay_i(M,\strategy^{\star}) \\  
    = & \;\mathbb{I}\left( \left| Y_i \right| = \datanum_i^{\star} \right) \rbr{\frac{\datanum_1^{\star}}{n^{\star}}\val(n^{\star})+d_i \frac{\sigma^2}{\datanum_{-i}^{\star}} +d_i \frac{\sigma^2}{\datanum_i^{\star}} } \\ & - d_i \rbr{\hat{\mu}(Y_i)-\hat{\mu}(Y_{-i}) }^2, \\[20pt] % Adds vertical space between equations
    &\; \price(M,\strategy^{\star}) \\  
    = & \; \sum_{i=1}^{2}\mathbb{I}\left(  \left| Y_i \right| = \datanum_i^{\star} \right) \rbr{\frac{\datanum_1^{\star}}{n^{\star}}\val(n^{\star}) +d_i \frac{\sigma^2}{\datanum_{-i}^{\star}} +d_i \frac{\sigma^2}{\datanum_i^{\star}} } \\ 
    & - \sum_{i=1}^{2} d_i \rbr{\hat{\mu}(Y_i)-\hat{\mu}(Y_{-i}) }^2, \\[20pt] % Adds vertical space between equations
    &\;  \utilityb (M,\strategy^{\star}) \\ 
   = & \; v(\datanum^{\star}) -\mathbb{E}[\price(M,\strategy^{\star})] \\ 
    = & \; - \sum_{i=1}^{d} \rbr{d_i \frac{\sigma^2}{\datanum_{-i}^{\star}} +d_i \frac{\sigma^2}{\datanum_i^{\star}}  } + \sum_{i=1}^{d}d_i \mathbb{E} \rbr{\hat{\mu}(Y_i)-\hat{\mu}(Y_{-i}) }^2 \\ 
    = & \; 0.
\end{align*}



Then contributors i's expected ptofit under strategy $\strategy^{\star}$ is 

\begin{align*}
& \; \utilci \rbr{\mechspace, \strategy^{\star} } \\ = &  \; \mathbb{E}\sbr{\pay_i(M,\strategy^{\star})} - \cost_i n_i^{\star} \\ = &  \; \mathbb{I}\left(\left| Y_i \right| = \datanum_i^{\star} \right) \rbr{\frac{\datanum_1^{\star}}{n^{\star}}\val(n^{\star})  +d_i \frac{\sigma^2}{\datanum_{-i}^{\star}} +d_i \frac{\sigma^2}{\datanum_i^{\star}} }\\  & -  d_i \mathbb{E}\rbr{\hat{\mu}(Y_i)-\hat{\mu}(Y_{-i}) }^2  -\cost n_i^{\star} \\ = &  \;  \frac{w}{\numcontributors}
\end{align*}
where $d_i = c(\datanum_i^*/d)^2 $, 

%\textcolor{red}{Buyer payment $\pi(M,s)$ can ve negative? Can it be interpreted as when the quality of data is bad, the mechanism pays money to the buyer as compensate, the contributor pays money to the mechanism as a penalty. }\textcolor{red}{Buyer pays $v(n^*)$, $p_i = v(n^*) \frac{\rbr{\hat{\mu}(Y_i)-\hat{\mu}(Y_{-i}) }^{-2}}{\sum{\rbr{\hat{\mu}(Y_i)-\hat{\mu}(Y_{-i}) }^{-2}}}$ }


We prove the NIC in three steps.


\textbf{First step} \textcolor{red}{to be fixed}: Giving others submitting truthfully, we know that when fixing $n_i$, submitting $\left| Y_i \right| = \frac{n^{\star}}{d}$ is the best strategy, otherwise, $\pay_i <0$ when $\left| Y_i \right| \neq \frac{n^{\star}}{d}$. Therefore, for any $n_i$ and $f_i$, we have for any $\mu$,
\begin{align*}
& u_i\rbr{\mechspace, (n_i,f_i, \left| Y_i \right| =\datanum_i^{\star}),\strategy_{-i}^{\star} } \\  \geq \  & u_i\rbr{\mechspace, \strategy_{-i}^{\star}} 
\end{align*}

\textbf{Second step}: Fixing $n_i$ and $\left| Y_i \right|$, sample mean $\hat{\mu}(X_i)$ is minimax estimator of $\mathbb{E}\rbr{\rbr{\hat{\mu}(Y_i)-\hat{\mu}(Y_{-i})}^2 \;\middle|\; P} $, i.e., \[ \hat{\mu}(X_i) = \underset{\hat{\mu}}{\inf} \  \underset{\distrifamily}{\sup}\ \mathbb{E}\sbr{\rbr{\hat{\mu}(Y_i)-\hat{\mu}(Y_{-i})}^2  \;\middle|\; \distri} \]Therefore we have 
\begin{align*}
     & \underset{\distrifamily}{\inf}\;u_i\rbr{\mechspace, (n_i,\hat{\mu}(Y_i)=\hat{\mu}(X_i), \left| Y_i \right|),\strategy_{-i}^{\star} } \\  \geq \  & \underset{\distrifamily}{\inf}\;u_i\rbr{\mechspace, (n_i,\hat{\mu}(Y_i), \left| Y_i \right|),\strategy_{-i}^{\star} } 
\end{align*}


\textbf{Third step}: By setting constant $d_i= c\rbr{\frac{\datanum_i^{\star}}{\sigma}}^2$, when fixing $\hat{\mu}(Y_i)=\hat{\mu}(X_i)$ and $\left| Y_i \right| = \datanum_i^{\star}$, collecting $n_i = \datanum_i^{\star}$ amount of data maximize the contributor utility 
\begin{align*}
    \underset{\distrifamily}{\inf}\;u_i\rbr{\mechspace, (n_i= \datanum_i^{\star},\hat{\mu}(Y_i)=\hat{\mu}(X_i), \left| Y_i \right|=\strategy_{-i}^{\star} } \\ \geq \underset{\distrifamily}{\inf}\;u_i\rbr{\mechspace, (n_i,\hat{\mu}(Y_i)=\hat{\mu}(X_i), \left| Y_i \right|=s_{-i}^{\star}}  
\end{align*}
 

\begin{align*}
    & u_i\rbr{\mechspace, (n_i,\hat{\mu}(Y_i)=\hat{\mu}(X_i), \left| Y_i \right|=\datanum_i^{\star}),\strategy_{-i}^{\star} }  \\ = & \rbr{\frac{w}{\numcontributors}+ c \datanum_i^{\star} +d_i \frac{\sigma^2}{\datanum_{-i}^{\star}} +d_i \frac{\sigma^2}{\datanum_i^{\star}} } - d_i \rbr{ \frac{\sigma^2}{\datanum_{-i}^{\star}} +\frac{\sigma^2}{\datanum_i^{\star}} } - cn_i
\end{align*}

Therefore, we have 
\begin{align*}
    &\underset{\distrifamily}{\inf}\;u_i \rbr{\mechspace, (n_i= \datanum_i^{\star},\hat{\mu}(Y_i)=\hat{\mu}(X_i), \left| Y_i \right|=\datanum_i^{\star}),\strategy_{-i}^{\star} } \\  = & \underset{\distrifamily}{\inf}\;u_i\rbr{\mechspace, (\datanum_i^{\star},f_i^{\star}),\strategy_{-i}^{\star} } \\ \geq &   \underset{\distrifamily}{\inf}\; u_i\rbr{\mechspace, (n_i,f_i ),\strategy_{-i}^{\star}} 
\end{align*}

When following the best strategy, properties 1-5 are all satisfied.



% \caption{
%     % 我们方法上非英语相比英语性能落后的错误原因,及比例
%     The error types and the proportion of non-English performance in \ourmethod are lagging behind that of English.
% }
% \label{tab:error}
% \end{table*}

\begin{figure}[t]
    \centering
    
\begin{lemma}\label{Lemma:multi1} 
   Fixing the number of data contributor $i$ collects $n_i$, and others' strategies $\strategy_{-i}$, $\hat{\mu}\left(X_i\right)$ is the minimax estimator for the Normal distribution class $\Normaldistrib := \left\{\mathcal{N}(\mu,\sigma^2) \;\middle|\; \mu \in \mathbb{R}\right\}$,
    \begin{align*}
       \hat{\mu}(X_i)  = \underset{\hat{\mu}}{\arg\min} \sbr{\sup _\mu \mathbb{E}\left[(\hat{\mu}( Y_i)- \hat{\mu}( Y_{-i}) )^2 \;\middle|\;  \mu \right] }
    \end{align*} 
     
\end{lemma}


\begin{proof}

\begin{align*}
    & \ \mathbb{E}\left[ \left( \hat{\mu}\left( Y_i \right)-\hat{\mu}\left( Y_{-i} \right)  \right)^2 \right] \\ =  & \ \mathbb{E}\left[ \left( (\hat{\mu}\left(  Y_i \right)-\mu) -(\hat{\mu}\left(  Y_{-i} \right) -\mu) \right)^2   \right] \\ =  & \ A_0 + \mathbb{E}\left[ (\hat{\mu}\left(  Y_i \right)-\mu)^2  \right]
\end{align*}
where $A_0$ is a positive coefficient.

Thus the maximum risk can be written as:

\begin{align*}
    \sup _\mu \mathbb{E}\left[A_0 + \left(\hat{\mu}\left( Y_i\right)-\mu\right)^{2} \;\middle|\;  \mu \right]
\end{align*}


We construct a lower bound on the maximum risk using a sequence of Bayesian risks. Let $\Lambda_{\ell}:=\mathcal{N}\left(0, \ell^2\right), \ell=1,2, \ldots$ be a sequence of prior for $\mu$. For fixed $\ell$, the posterior distribution is:
$$
\begin{aligned}
p\left(\mu \;\middle|\;  X_i\right) & \propto p\left(X_i \;\middle|\;  \mu\right) p(\mu) \\ & \propto \exp \left(-\frac{1}{2 \sigma^2} \sum_{x \in X_i}(x-\mu)^2\right) \exp \left(-\frac{1}{2 \ell^2} \mu^2\right) \\
& \propto \exp \left(-\frac{1}{2}\left(\frac{n_i}{\sigma^2}+\frac{1}{\ell^2}\right) \mu^2+\frac{1}{2} 2 \frac{\sum_{x \in X_i} x}{\sigma^2} \mu\right) .
\end{aligned}
$$

This means the posterior of $\mu$ given $X_i$ is Gaussian with:

\begin{align*}
    \mu \lvert\, X_i & \sim \mathcal{N}\left(\frac{n_i \hat{\mu}\left(X_i\right) / \sigma^2}{n_i / \sigma^2+1 / \ell^2}, \frac{1}{n_i / \sigma^2+1 / \ell^2}\right) 
    \\ & =: \mathcal{N}\left(\mu_{\ell}, \sigma_{\ell}^2\right).
\end{align*}



Therefore, the posterior risk is: 
$$
\begin{aligned}
&   \mathbb{E}\left[A_0 + \left(\hat{\mu}\left( Y_i\right)-\mu\right)^{2}  \;\middle|\;  X_i\right] \\ = &  \mathbb{E}\left[A_0 +  \left(\left(\hat{\mu}\left( Y_i\right)-\mu_{\ell}\right)-\left(\mu-\mu_{\ell}\right)\right)^{2 j} \;\middle|\;  X_i\right] \\ =
& A_0+\int_{-\infty}^{\infty} \underbrace{\left(e-\left(\hat{\mu}\left( Y_i\right)-\mu_{\ell}\right)\right)^2}_{=: F_1\left(e-\left(\hat{\mu}\left(Y_i\right)-\mu_{\ell}\right)\right)} \underbrace{\frac{1}{\sigma_{\ell} \sqrt{2 \pi}} \exp \left(-\frac{e^2}{2 \sigma_{\ell}^2}\right)}_{=: F_2(e)} d e
\end{aligned}
$$

Because:
\begin{itemize}
    \item $F_1(\cdot)$ is even function and increases on $[0, \infty)$;
    \item $F_2(\cdot)$ is even function and decreases on $\left[0, \infty \right)$, and $\int_{\mathbb{R}} F_2(e) de<\infty$
    \item For any $a \in \mathbb{R}, \int_{\mathbb{R}} F_1(e-a) F_2(e) de<\infty$
\end{itemize}

By the corollary of Hardy-Littlewood inequality in Lemma \ref{lemmaHardy},
$$
\int_{\mathbb{R}} F_1(e-a) F_2(e) d e \geq \int_{\mathbb{R}} F_1(e) F_2(e) d e
$$
which means the posterior risk is minimized when $\hat{\mu}\left(Y_i\right)=\mu_{\ell}$. We then write the Bayes risk as, the Bayes risk is minimized by the posterior mean $\mu_{\ell}$:

\begin{align*}    
R_{\ell}:= & \mathbb{E}\left[ A_0+\mathbb{E}\left[\left(\mu-\mu_{\ell}\right)^{2 } \;\middle|\;  X_i\right]\right] \\ = & A_0 + \sigma_{\ell}^{2}
\end{align*}

and the limit of Bayesian risk as $\ell \rightarrow \infty$ is
$$
R_{\infty}:= A_0 + \frac{\sigma^{2}}{n_i}.
$$

When $\hat{\mu}\left(Y_i\right)=\hat{\mu}\left(X_i\right)$, i.e, the contributor submit a set of size $n_i$ with each element equal to $ \hat{\mu}\left(X_i\right)$, the maximum risk is:

\begin{align*}
& \sup _\mu \mathbb{E}\left[A_0+\left(\mu- 
\hat{\mu}\left(Y_i\right) \right)^{2 } \;\middle|\;  \mu \right] \\
= & \sup _\mu \mathbb{E}\left[A_0+\left(\mu- 
\hat{\mu}\left(X_i\right) \right)^{2 } \;\middle|\;  \mu \right]  \\
= & A_0+ \sigma^{2 } n_i^{-1}  \\
= &  R_{\infty}.
\end{align*}

This implies that,
\begin{align*}
    & \underset{\mu}{\sup}\; \mathbb{E} \sbr{ \rbr{\hat{\mu}\left( Y_i \right)-\hat{\mu}\left( Y_{-i} \right)  }^2 \;\middle|\;  \mu }  \\ \geq & \; R_{\infty} =  \sup _\mu \; \mathbb{E}\left[A_0+\left(\mu- 
\hat{\mu}\left(X_i\right) \right)^{2 } \;\middle|\;  \mu \right]
\end{align*}

Therefore, the recommended strategy $\hat{\mu}(Y_i) =\hat{\mu}( X_i)$ has a smaller maximum risk than other strategies. 

\end{proof}




1. The payment from the buyer a constant $v(n^{\star})$.


2. If the payment for every seller is a fixed constant, then sellers can fabricate data without actually collecting data.\\


%$p_1 = b/2 +(\hat{\mu}(Y_1)- \hat{\mu}(Y_2))^2$, $p_2 = b/2 -(\hat{\mu}(Y_1)- \hat{\mu}(Y_2))^2$, seller 1 can choose ${\mu}' = u + \epsilon$, expected payment for seller 1 is larger than $b/2$. %NIC for seller 1: $g({\mu}',\mu ) < b/2$ for all ${\mu}' \neq \mu$. NIC for seller 2: $g( \mu, {\mu}') >  b/2 $ for all ${\mu}' \neq \mu$.

To demonstrate that no truthful mechanism (NIC) satisfies all desired properties in a two-seller setting, we use proof by contradiction.  

Suppose that there is a NIC mechanism $M$ satisfying property 1-5. Under this mechanism, the best strategy for each seller is to collect $N_i^{\star}$ amount of data and submit truthfully, where $N_1^{\star}+N_2^{\star} = n^{\star} $. Since $M$ is NIC for strategy space $\left\{ (f_i,N_i)\right\}_{i=1,2}$, it must be NIC for the sub strategy space $\left\{ (f_i, N_i^{\star})\right\}_{i=1,2}$. 


Consider the case in which everyone collects $N_i^{\star}$ data point and submits $N_i^{\star}$ data point. Assume that the true mean is $\mu$, seller $1$ submit $N({\mu}', \sigma^2),\ {\mu}' = f(\mu) $ while seller 2 submit $N({\mu}, \sigma^2) $. We denote seller 1's expected payment as  $\mathbb{E}\left[ p_1(M,\strategy) \right] = g({\mu}', \mu)$.
Seller 1's utility is then:
\[ u_1(M,f) = \underset{\mu}{\inf} \ g({\mu}', \mu) -c\times N_1^{\star}\] where $c$ is the cost for collecting one data point.


The total payment from the buyer is $v(n^{\star})$, hence by budget balance, \[p_2 (M,f)  = v(n^{\star}) -  p_1(M,f), \ \mathbb{E}\left[ p_2(M,f) \right] = v(n^{\star}) - g({\mu}', \mu) \]

By NIC, we have,
\[ \underset{\mu}{\inf} \ g({\mu}', \mu) -c\times N_1^{\star} \leq \underset{\mu}{\inf} \ g({\mu}, \mu) -c\times N_1^{\star} \] \[ \underset{\mu}{\inf} \ (v(n^{\star})- g( \mu, {\mu}')) -c\times N_2^{\star} \leq \underset{\mu}{\inf} \ (v(n^{\star})-g( \mu, {\mu})) -c\times N_2^{\star}  \]

%Using the fact that $\underset{\mu}{\inf} \ g({\mu}, \mu) = \underset{\mu}{\sup} \ g({\mu}, \mu) = {v(n^{\star})}/2 $.
We obtain that for any ${\mu}'$ and $\mu$,
\[  \underset{\mu}{\inf} \ g({\mu}', \mu) \leq \underset{\mu}{\inf} \ g({\mu}, \mu)   \] \[  \underset{\mu}{\sup} \  g( \mu, {\mu})  \leq \underset{\mu}{\sup } \ g( \mu, {\mu}')  \]

We next show that the inequalities are strict. Assume, for contradiction there exists ${\mu}'$, for any $\mu$, $g({\mu}', 
\mu) \geq \underset{\mu}{\inf} \ g({\mu}, \mu)$. It then follows that $ \underset{\mu}{\inf} \ g({\mu}', \mu) \geq \underset{\mu}{\inf} \ g({\mu}, \mu)$. Under this assumption, seller 1 could fabricate data by submitting $N({\mu}', \sigma^2)$ without collecting any actual data. This contradicts with the fact that $(f_1 = I, N_1 = N_1^{\star})$ is the best strategy for seller 1. Hence, for any ${\mu}'$, there exists some $\mu$ such that $g({\mu}', 
\mu) < \underset{\mu}{\inf} \ g({\mu}, \mu)$. Therefore, for any ${\mu}'$, \[ \underset{\mu}{\inf} \ g({\mu}', \mu) < \underset{\mu}{\inf} \ g({\mu}, \mu). \]Similarly, we also have \[ \underset{\mu}{\sup} \  g( \mu, {\mu})  < \underset{\mu}{\sup } \ g( \mu, {\mu}').  \] 


For any $ {\mu}'$, let $f({\mu}') =  \underset{\mu}{\arg\sup}\, g(\mu, {\mu}')$, then we have for any ${\mu}'$,  $f({\mu}') \neq {\mu}'$ and $ g(f({\mu}'), {\mu}') > \underset{\mu}{\sup} \  g( \mu, {\mu}) \geq  \underset{\mu}{\inf} \  g( \mu, {\mu})$. This implies that seller 1 could fabricate data based on function $f$, this contradicts with the fact that the mechanism is NIC.


pay the seller $v(n^{\star})/2 - \beta (\hat{\mu}(Y_1)-\hat{\mu}(Y_2))^2$, charge buyer $v(n^{\star}) - 2\beta (\hat{\mu}(Y_1)-\hat{\mu}(Y_2))^2$


Buyer utility: $v(n^*)$-payment
Seller utility: payment - $cn^*$.

Sellers utility is positive?

Seller payment $(v(n^*)/2)-\beta (\hat{\mu}(Y_1)-\hat{\mu}(Y_2))^2 $, $\beta = (v(n^*)-cn^*)c(n^*)^2 / 4\sigma^2$, 



\section{Multiple buyers}
\subsection{}
Question 1: Do we fix the amount of data for sale ahead of time?


Assume we fix $N$, the amount of data for sale. The goal of mechanism is to maximize the sellers' revenue. According to previous paper, there exists at least one type who purchase at the amount $N$. Suppose that in offline setting, i.e., when the mechanism knows the buyer valuation and type distribution, the optimal revenue is $\text{OPT} $. 


We ask $d$ sellers to collect $N$ data points, and split $\text{OPT} $ revenue among sellers. (data can be duplicated). 

\[ p_i(M,s) =\mathbb{I}\left( \left| Y_i \right| = \frac{N}{d} \right) \rbr{\frac{\text{OPT}}{d}+d_i \frac{\sigma^2}{N_{-i}^{\star}} +d_i \frac{\sigma^2}{N_i^{\star}} }- d_i \rbr{\hat{\mu}(Y_i)-\hat{\mu}(Y_{-i}) }^2  \]

Buyer's expected utility is non negative. Next, we discuss sellers' expected utility $T\mathbb{E}[p_i]- cn_i$ (over $T$ roundsm\, maybe $T$ is fixed). Let $N_i^{\star} = \frac{N}{d}$.

\[ u_i(M,s) = \mathbb{I}\left( \left| Y_i \right| = \frac{N}{d} \right) \rbr{\frac{\text{OPT}}{d}+d_i \frac{\sigma^2}{N_{-i}^{\star}} +d_i \frac{\sigma^2}{N_i^{\star}} }T- Td_i \mathbb{E}\rbr{\hat{\mu}(Y_i)-\hat{\mu}(Y_{-i}) }^2 
 -  cn_i \]
Choose $d_i = \frac{c(N_i^{\star})^2}{T\sigma^2}$.


If we do not fix \( T \) in advance, let \( T_0 \) represent the time at which the cumulative utility over at least \( T_0 \) rounds is non-negative. We can select \( d_i \) such that \( d_i \geq \frac{c(N/d)^2}{T_0 \sigma^2} \). This ensures that the seller will never choose to collect less than \( N/d \) amount of data.








\section{Single buyer} \label{section: singlebuyer}


Each contributor \( i \) incurs a cost \( c_i \) to collect data,  without loss of generality, we assume \( c_1 \leq c_2 \leq \dots \leq c_d \). The broker is assumed to have full knowledge of the buyer's valuation curve \( \val(n) \), as well as the contributors’ costs $ c_{ i \in \contributors}$  for collecting each data point.

The maximum total profit for the contributors, assuming no constraints on truthful submissions, is given by:
\[
\mathrm{profit}^\star = \underset{\datanum_1, \dots, \datanum_d}{\max} \left( \val \left(\sum_{i=1}^{d} \datanum_i\right) - \sum_{i=1}^{d} c_i \datanum_i \right),
\]

where \( \datanum_i \) represents the number of data points collected by contributor \( i \). In this unconstrained scenario, since contributor 1 has the lowest collection cost, the optimal strategy is for contributor 1 to collect all the required data points while other contributors collect none. This approach maximizes total profit without considering the incentive for truthful submissions.

However, when truthful submission is taken into account, at least two contributors are needed because we need to use one contributor's data to verify the other's. We demonstrate that the maximum profit achievable under Nash Equilibrium is:
\[
\mathrm{profit}^\star + (c_1 - c_2),
\]
where \( c_1 - c_2 \) represents the additional cost differential caused by enforcing truthful behavior among contributors. 

\begin{algorithm}[H]
    \caption{Process of mechanism.}
    \begin{algorithmic}
        \STATE {\bfseries Input:} A population of buyers $\buyers$.
        \STATE The broker chooses the optimal data allocation to maximize contributors' profit:
        $$
        \{ \datanum_i^{\star} \}_{i=1}^d = \underset{\datanum_1,\dots,\datanum_d}{\arg\max}\  \rbr{v\rbr{\sum_i \datanum_i}-\sum_i \cost_i \datanum_i  }
      $$
       
        \STATE The broker recommend a strategy to each contributor: $\strategy_i^{\star} = (\datanum_i^{\star}, \mathbf{I})$.
        \STATE Each contributor selects a strategy $\strati = (\datanum_i, f_i)$, collects $\datanum_i$ data points $X_i$, and submits $Y_i = f_i(X_i)$.
        \STATE The mechanism generates an estimator $\hat{\mu}(M,\strategy)$ for the buyer, and charge her $\price_{j \in \buyers}$. \COMMENT{See (\ref{eq:buyer_pay}) }
        \STATE Each contributor is paid $\payi$.    \COMMENT{See (\ref{eq:seller_pay}) }
    \end{algorithmic}   
\end{algorithm}




\begin{theorem}
    there exists NIC mechanism satisfying the following properties (1) $\strategy^{\star}$ is Nash equilibrium. (2) The mechanism is individually rational at $\strategy^{\star}$ for both buyers and sellers. (3) Budget balance. (4) Under strategy $\strategy^{\star}$, the expected profit of buyers approximates the optimal profit $ \mathrm{profit}^{\star}$ within an additive error $\cost_2 - \cost_1$. 
\end{theorem}


Let $n^{\star}$ denote the optimal total number of data to be collected, $\datanum_1^{\star}=n^{\star}-1$, and $\datanum_2^{\star}=1$. Let $w=\val(n^{\star})-cn^{\star}$ denote the social welfare. One option for payment function is

\begin{align*}
    &\; \pay_i(M,\strategy^{\star}) \\  
    = & \;\mathbb{I}\left( \left| Y_i \right| = \datanum_i^{\star} \right) \rbr{\frac{\datanum_1^{\star}}{n^{\star}}\val(n^{\star})+d_i \frac{\sigma^2}{\datanum_{-i}^{\star}} +d_i \frac{\sigma^2}{\datanum_i^{\star}} } \\ & - d_i \rbr{\hat{\mu}(Y_i)-\hat{\mu}(Y_{-i}) }^2, \\[20pt] % Adds vertical space between equations
    &\; \price(M,\strategy^{\star}) \\  
    = & \; \sum_{i=1}^{2}\mathbb{I}\left(  \left| Y_i \right| = \datanum_i^{\star} \right) \rbr{\frac{\datanum_1^{\star}}{n^{\star}}\val(n^{\star}) +d_i \frac{\sigma^2}{\datanum_{-i}^{\star}} +d_i \frac{\sigma^2}{\datanum_i^{\star}} } \\ 
    & - \sum_{i=1}^{2} d_i \rbr{\hat{\mu}(Y_i)-\hat{\mu}(Y_{-i}) }^2, \\[20pt] % Adds vertical space between equations
    &\;  \utilityb (M,\strategy^{\star}) \\ 
   = & \; v(\datanum^{\star}) -\mathbb{E}[\price(M,\strategy^{\star})] \\ 
    = & \; - \sum_{i=1}^{d} \rbr{d_i \frac{\sigma^2}{\datanum_{-i}^{\star}} +d_i \frac{\sigma^2}{\datanum_i^{\star}}  } + \sum_{i=1}^{d}d_i \mathbb{E} \rbr{\hat{\mu}(Y_i)-\hat{\mu}(Y_{-i}) }^2 \\ 
    = & \; 0.
\end{align*}



Then contributors i's expected ptofit under strategy $\strategy^{\star}$ is 

\begin{align*}
& \; \utilci \rbr{\mechspace, \strategy^{\star} } \\ = &  \; \mathbb{E}\sbr{\pay_i(M,\strategy^{\star})} - \cost_i n_i^{\star} \\ = &  \; \mathbb{I}\left(\left| Y_i \right| = \datanum_i^{\star} \right) \rbr{\frac{\datanum_1^{\star}}{n^{\star}}\val(n^{\star})  +d_i \frac{\sigma^2}{\datanum_{-i}^{\star}} +d_i \frac{\sigma^2}{\datanum_i^{\star}} }\\  & -  d_i \mathbb{E}\rbr{\hat{\mu}(Y_i)-\hat{\mu}(Y_{-i}) }^2  -\cost n_i^{\star} \\ = &  \;  \frac{w}{\numcontributors}
\end{align*}
where $d_i = c(\datanum_i^*/d)^2 $, 

%\textcolor{red}{Buyer payment $\pi(M,s)$ can ve negative? Can it be interpreted as when the quality of data is bad, the mechanism pays money to the buyer as compensate, the contributor pays money to the mechanism as a penalty. }\textcolor{red}{Buyer pays $v(n^*)$, $p_i = v(n^*) \frac{\rbr{\hat{\mu}(Y_i)-\hat{\mu}(Y_{-i}) }^{-2}}{\sum{\rbr{\hat{\mu}(Y_i)-\hat{\mu}(Y_{-i}) }^{-2}}}$ }


We prove the NIC in three steps.


\textbf{First step} \textcolor{red}{to be fixed}: Giving others submitting truthfully, we know that when fixing $n_i$, submitting $\left| Y_i \right| = \frac{n^{\star}}{d}$ is the best strategy, otherwise, $\pay_i <0$ when $\left| Y_i \right| \neq \frac{n^{\star}}{d}$. Therefore, for any $n_i$ and $f_i$, we have for any $\mu$,
\begin{align*}
& u_i\rbr{\mechspace, (n_i,f_i, \left| Y_i \right| =\datanum_i^{\star}),\strategy_{-i}^{\star} } \\  \geq \  & u_i\rbr{\mechspace, \strategy_{-i}^{\star}} 
\end{align*}

\textbf{Second step}: Fixing $n_i$ and $\left| Y_i \right|$, sample mean $\hat{\mu}(X_i)$ is minimax estimator of $\mathbb{E}\rbr{\rbr{\hat{\mu}(Y_i)-\hat{\mu}(Y_{-i})}^2 \;\middle|\; P} $, i.e., \[ \hat{\mu}(X_i) = \underset{\hat{\mu}}{\inf} \  \underset{\distrifamily}{\sup}\ \mathbb{E}\sbr{\rbr{\hat{\mu}(Y_i)-\hat{\mu}(Y_{-i})}^2  \;\middle|\; \distri} \]Therefore we have 
\begin{align*}
     & \underset{\distrifamily}{\inf}\;u_i\rbr{\mechspace, (n_i,\hat{\mu}(Y_i)=\hat{\mu}(X_i), \left| Y_i \right|),\strategy_{-i}^{\star} } \\  \geq \  & \underset{\distrifamily}{\inf}\;u_i\rbr{\mechspace, (n_i,\hat{\mu}(Y_i), \left| Y_i \right|),\strategy_{-i}^{\star} } 
\end{align*}


\textbf{Third step}: By setting constant $d_i= c\rbr{\frac{\datanum_i^{\star}}{\sigma}}^2$, when fixing $\hat{\mu}(Y_i)=\hat{\mu}(X_i)$ and $\left| Y_i \right| = \datanum_i^{\star}$, collecting $n_i = \datanum_i^{\star}$ amount of data maximize the contributor utility 
\begin{align*}
    \underset{\distrifamily}{\inf}\;u_i\rbr{\mechspace, (n_i= \datanum_i^{\star},\hat{\mu}(Y_i)=\hat{\mu}(X_i), \left| Y_i \right|=\strategy_{-i}^{\star} } \\ \geq \underset{\distrifamily}{\inf}\;u_i\rbr{\mechspace, (n_i,\hat{\mu}(Y_i)=\hat{\mu}(X_i), \left| Y_i \right|=s_{-i}^{\star}}  
\end{align*}
 

\begin{align*}
    & u_i\rbr{\mechspace, (n_i,\hat{\mu}(Y_i)=\hat{\mu}(X_i), \left| Y_i \right|=\datanum_i^{\star}),\strategy_{-i}^{\star} }  \\ = & \rbr{\frac{w}{\numcontributors}+ c \datanum_i^{\star} +d_i \frac{\sigma^2}{\datanum_{-i}^{\star}} +d_i \frac{\sigma^2}{\datanum_i^{\star}} } - d_i \rbr{ \frac{\sigma^2}{\datanum_{-i}^{\star}} +\frac{\sigma^2}{\datanum_i^{\star}} } - cn_i
\end{align*}

Therefore, we have 
\begin{align*}
    &\underset{\distrifamily}{\inf}\;u_i \rbr{\mechspace, (n_i= \datanum_i^{\star},\hat{\mu}(Y_i)=\hat{\mu}(X_i), \left| Y_i \right|=\datanum_i^{\star}),\strategy_{-i}^{\star} } \\  = & \underset{\distrifamily}{\inf}\;u_i\rbr{\mechspace, (\datanum_i^{\star},f_i^{\star}),\strategy_{-i}^{\star} } \\ \geq &   \underset{\distrifamily}{\inf}\; u_i\rbr{\mechspace, (n_i,f_i ),\strategy_{-i}^{\star}} 
\end{align*}

When following the best strategy, properties 1-5 are all satisfied.



    % \vspace{-0.5em}
    \caption{
    % 我们方法上非英语相比英语性能落后的错误原因,及比例
    The error types and their proportion of non-English performance in \ourmethod are inferior compared with English. 
    \textbf{Linking} refers to mapping entities in the question with incorrect information in the table or text. 
    \textbf{Formula} refers to using an incorrect formula. 
    \textbf{Redundancy} refers to outputting irrelevant information beyond the correct answer. 
    }
    \label{fig:error}
    \vspace{-1em}
\end{figure}

\subsection{Error Analysis}
\label{subsec:Error Analysis}
% 我们分析了我们方法在非英语语言上相比英语落后的原因,如图所示
We analyze the reasons for the inferior performance of \ourmethod on non-English languages compared to English, as shown in Figure~\ref{fig:error}. 
% 具体来说,我们选取Llama3.1-70B上我们的方法在英语上达到EM为1,而在非英语上EM为0的问题,每种语言随机sample 5个,共50个错误进行对比分析
Specifically, we select instances where \ourmethod achieved an EM of $1$ in English using Llama3.1-70B, but an EM of $0$ in non-English languages. 
For each language, we randomly sample five instances, with a total of $50$ errors for comparative analysis. 
% 我们在附录中展示了每种类型对应错误的示例
Examples of errors corresponding to each type are provided in Appendix~\ref{subsec:case study}. 
% 下面我们具体介绍每种错误
Below, we present a detailed discussion of each error type:

% 1. 链接:指模型将问题中的实体对应到了不想关的表格或文本中的信息,导致模型回答错误
(\emph{i})~\textbf{Linking}: 
% refers to associating entities in the question with incorrect information in tables or text. 
% 由于模型在非英语语言上的理解以及推理能力落后于英语,即使我们方法首先令模型专注于链接,模型仍然在链接上展现出了非常的挑战
Due to the relatively weaker abilities in non-English languages compared to English, even though \ourmethod initially prompts the model to focus on linking, the model still faces significant challenges in linking. 
% 尤其是在一些语言上,比如日语的平假名和片假名,或法语、德语等词法变化复杂的语言上,链接的挑战进一步加剧
These challenges are particularly pronounced in languages with complex orthographies, such as Japanese (with its hiragana and katakana scripts), or morphologically rich languages like French and German.
% 2. 公式:指模型在找到相关信息后,使用了错误的公式,导致代码返回了错误的结果
(\emph{ii})~\textbf{Formula} 
% pertains to situations where, after identifying relevant information, the model uses an incorrect formula, resulting in erroneous output. 
% 这也表明了模型在非英语语言上的数值推理能力相比英语仍存在差距
highlights the gap in the numerical reasoning abilities between non-English languages and English.
% 3. 多余信息:指模型没有按照指令的要求,输出了除正确答案之外的多余信息,导致EM为0
(\emph{iii})~\textbf{Redundancy}  
% refers to outputting irrelevant information beyond the correct answer, not as instructed, leading to an EM of $0$. 
% 这体现了模型相对较差的指令遵循能力
reflects the relatively weaker ability of instruction-following.

% 总之,模型在非英语语言上的能力落后,以及语言特殊的属性,令我们的方法在非英语语言上落后于英语,也证明了我们数据集的挑战性
In summary, the inferior performance on non-English languages and the specific properties of languages leads to the lower performance of \ourmethod on non-English languages, which also demonstrates the necessity of \ourdataset.
\vspace{-5pt}
\section{Discussion}
\textbf{Conclusion.}
In this work, we propose the \textit{\methodname{}} metric, $M_{AP}$, to evaluate preference data quality in alignment.
By measuring the gap from the model's current implicit reward margin to the target explicit reward margin, $M_{AP}$ quantifies the discrepancy between the current model and the aligned optimum, thereby indicating the potential for alignment enhancement.
Extensive experiments validate the efficacy of $M_{AP}$ across various training settings under offline and self-play preference learning scenarios.

\textbf{Limitations and future work}. 
Despite the performance improvements, $M_{AP}$ requires tuning a parameter $\beta$ to combine the explicit and implicit margins; future work could explore how to set this ratio automatically.
Additionally, while our experiments focus on the widely applied DPO and SimPO objectives, a broader investigation with alternative preference learning methods is crucial in future works.

% \section{Conclusion}
% In this paper, we introduce the \methodname{} metric to evaluate preference data quality in LLM alignment.
% By measuring the discrepancy between the model's current implicit reward margin to the target explicit reward margin, this metric quantifies the gap between the current model and the aligned optimum, thereby indicating the potential for alignment enhancement.
% Empirical results demonstrate that training on data selected by our metric consistently improves alignment performance, outperforming existing metrics across different base models and training objectives.
% Moreover, this metric extends to data generation scenarios (\ie self-play alignment): by identifying high-quality data from the intrinsic self-generated context, our metric yields superior results across various training settings, providing a comprehensive solution for enhancing LLM alignment through optimized
% preference data generation, selection, and utilization.


\section*{Impact Statement}
This paper presents work whose goal is to advance the field of Machine Learning. There are many potential societal consequences of our work, none which we feel must be specifically highlighted here.

\begin{acks}
  This work is supported in part by the National Natural Science Foundation of China under 23IAA02114 and 62472241, in part by the joint project of Infinigence AI \& Tsinghua University.
\end{acks}

\clearpage


%%
%% This is file `sample-sigconf.tex',
%% generated with the docstrip utility.
%%
%% The original source files were:
%%
%% samples.dtx  (with options: `all,proceedings,bibtex,sigconf')
%% 
%% IMPORTANT NOTICE:
%% 
%% For the copyright see the source file.
%% 
%% Any modified versions of this file must be renamed
%% with new filenames distinct from sample-sigconf.tex.
%% 
%% For distribution of the original source see the terms
%% for copying and modification in the file samples.dtx.
%% 
%% This generated file may be distributed as long as the
%% original source files, as listed above, are part of the
%% same distribution. (The sources need not necessarily be
%% in the same archive or directory.)
%%
%%
%% Commands for TeXCount
%TC:macro \cite [option:text,text]
%TC:macro \citep [option:text,text]
%TC:macro \citet [option:text,text]
%TC:envir table 0 1
%TC:envir table* 0 1
%TC:envir tabular [ignore] word
%TC:envir displaymath 0 word
%TC:envir math 0 word
%TC:envir comment 0 0
%%
%%
%% The first command in your LaTeX source must be the \documentclass
%% command.
%%
%% For submission and review of your manuscript please change the
%% command to \documentclass[manuscript, screen, review]{acmart}.
%%
%% When submitting camera ready or to TAPS, please change the command
%% to \documentclass[sigconf]{acmart} or whichever template is required
%% for your publication.
%%
%%
\documentclass[sigconf]{acmart}

%%
%% \BibTeX command to typeset BibTeX logo in the docs
\AtBeginDocument{%
  \providecommand\BibTeX{{%
    Bib\TeX}}}

%% Rights management information.  This information is sent to you
%% when you complete the rights form.  These commands have SAMPLE
%% values in them; it is your responsibility as an author to replace
%% the commands and values with those provided to you when you
%% complete the rights form.
\setcopyright{acmlicensed}
\copyrightyear{2025}
\acmYear{2025}
\setcopyright{cc}
\setcctype{by}
\acmConference[WWW '25]{Proceedings of the ACM Web Conference 2025}{April 28-May 2, 2025}{Sydney, NSW, Australia}
\acmBooktitle{Proceedings of the ACM Web Conference 2025 (WWW '25), April 28-May 2, 2025, Sydney, NSW, Australia}
\acmDOI{10.1145/3696410.3714818}
\acmISBN{979-8-4007-1274-6/25/04}

\settopmatter{printacmref=true}

%% These commands are for a PROCEEDINGS abstract or paper.
% \acmConference[Conference acronym 'XX]{Make sure to enter the correct
%   conference title from your rights confirmation emai}{June 03--05,
%   2018}{Woodstock, NY}
%%
%%  Uncomment \acmBooktitle if the title of the proceedings is different
%%  from ``Proceedings of ...''!
%%
%%\acmBooktitle{Woodstock '18: ACM Symposium on Neural Gaze Detection,
%%  June 03--05, 2018, Woodstock, NY}
% \acmISBN{978-1-4503-XXXX-X/18/06}


%%
%% Submission ID.
%% Use this when submitting an article to a sponsored event. You'll
%% receive a unique submission ID from the organizers
%% of the event, and this ID should be used as the parameter to this command.
%%\acmSubmissionID{123-A56-BU3}

%%
%% For managing citations, it is recommended to use bibliography
%% files in BibTeX format.
%%
%% You can then either use BibTeX with the ACM-Reference-Format style,
%% or BibLaTeX with the acmnumeric or acmauthoryear sytles, that include
%% support for advanced citation of software artefact from the
%% biblatex-software package, also separately available on CTAN.
%%
%% Look at the sample-*-biblatex.tex files for templates showcasing
%% the biblatex styles.
%%

%%
%% The majority of ACM publications use numbered citations and
%% references.  The command \citestyle{authoryear} switches to the
%% "author year" style.
%%
%% If you are preparing content for an event
%% sponsored by ACM SIGGRAPH, you must use the "author year" style of
%% citations and references.
%% Uncommenting
%% the next command will enable that style.
%%\citestyle{acmauthoryear}
\usepackage{enumitem}

\usepackage{amsmath}
\newtheorem{problem}{Problem}
\newtheorem{definition}{Definition}

\usepackage{xspace}
\newcommand{\model}{UniGraph2\xspace}
\newcommand{\vpara}[1]{\vspace{0.04in}\noindent\textbf{#1}\xspace}

\newcommand\todo[1]{\textcolor{red}{[TODO: #1 ]}}

% add background color to table
\usepackage{color}
\usepackage{colortbl}
\definecolor{Gray}{gray}{0.9}

\usepackage{multirow}

\usepackage{arydshln}


%%%%% NEW MATH DEFINITIONS %%%%%

\usepackage{amsmath,amsfonts,bm}
\usepackage{derivative}
% Mark sections of captions for referring to divisions of figures
\newcommand{\figleft}{{\em (Left)}}
\newcommand{\figcenter}{{\em (Center)}}
\newcommand{\figright}{{\em (Right)}}
\newcommand{\figtop}{{\em (Top)}}
\newcommand{\figbottom}{{\em (Bottom)}}
\newcommand{\captiona}{{\em (a)}}
\newcommand{\captionb}{{\em (b)}}
\newcommand{\captionc}{{\em (c)}}
\newcommand{\captiond}{{\em (d)}}

% Highlight a newly defined term
\newcommand{\newterm}[1]{{\bf #1}}

% Derivative d 
\newcommand{\deriv}{{\mathrm{d}}}

% Figure reference, lower-case.
\def\figref#1{figure~\ref{#1}}
% Figure reference, capital. For start of sentence
\def\Figref#1{Figure~\ref{#1}}
\def\twofigref#1#2{figures \ref{#1} and \ref{#2}}
\def\quadfigref#1#2#3#4{figures \ref{#1}, \ref{#2}, \ref{#3} and \ref{#4}}
% Section reference, lower-case.
\def\secref#1{section~\ref{#1}}
% Section reference, capital.
\def\Secref#1{Section~\ref{#1}}
% Reference to two sections.
\def\twosecrefs#1#2{sections \ref{#1} and \ref{#2}}
% Reference to three sections.
\def\secrefs#1#2#3{sections \ref{#1}, \ref{#2} and \ref{#3}}
% Reference to an equation, lower-case.
\def\eqref#1{equation~\ref{#1}}
% Reference to an equation, upper case
\def\Eqref#1{Equation~\ref{#1}}
% A raw reference to an equation---avoid using if possible
\def\plaineqref#1{\ref{#1}}
% Reference to a chapter, lower-case.
\def\chapref#1{chapter~\ref{#1}}
% Reference to an equation, upper case.
\def\Chapref#1{Chapter~\ref{#1}}
% Reference to a range of chapters
\def\rangechapref#1#2{chapters\ref{#1}--\ref{#2}}
% Reference to an algorithm, lower-case.
\def\algref#1{algorithm~\ref{#1}}
% Reference to an algorithm, upper case.
\def\Algref#1{Algorithm~\ref{#1}}
\def\twoalgref#1#2{algorithms \ref{#1} and \ref{#2}}
\def\Twoalgref#1#2{Algorithms \ref{#1} and \ref{#2}}
% Reference to a part, lower case
\def\partref#1{part~\ref{#1}}
% Reference to a part, upper case
\def\Partref#1{Part~\ref{#1}}
\def\twopartref#1#2{parts \ref{#1} and \ref{#2}}

\def\ceil#1{\lceil #1 \rceil}
\def\floor#1{\lfloor #1 \rfloor}
\def\1{\bm{1}}
\newcommand{\train}{\mathcal{D}}
\newcommand{\valid}{\mathcal{D_{\mathrm{valid}}}}
\newcommand{\test}{\mathcal{D_{\mathrm{test}}}}

\def\eps{{\epsilon}}


% Random variables
\def\reta{{\textnormal{$\eta$}}}
\def\ra{{\textnormal{a}}}
\def\rb{{\textnormal{b}}}
\def\rc{{\textnormal{c}}}
\def\rd{{\textnormal{d}}}
\def\re{{\textnormal{e}}}
\def\rf{{\textnormal{f}}}
\def\rg{{\textnormal{g}}}
\def\rh{{\textnormal{h}}}
\def\ri{{\textnormal{i}}}
\def\rj{{\textnormal{j}}}
\def\rk{{\textnormal{k}}}
\def\rl{{\textnormal{l}}}
% rm is already a command, just don't name any random variables m
\def\rn{{\textnormal{n}}}
\def\ro{{\textnormal{o}}}
\def\rp{{\textnormal{p}}}
\def\rq{{\textnormal{q}}}
\def\rr{{\textnormal{r}}}
\def\rs{{\textnormal{s}}}
\def\rt{{\textnormal{t}}}
\def\ru{{\textnormal{u}}}
\def\rv{{\textnormal{v}}}
\def\rw{{\textnormal{w}}}
\def\rx{{\textnormal{x}}}
\def\ry{{\textnormal{y}}}
\def\rz{{\textnormal{z}}}

% Random vectors
\def\rvepsilon{{\mathbf{\epsilon}}}
\def\rvphi{{\mathbf{\phi}}}
\def\rvtheta{{\mathbf{\theta}}}
\def\rva{{\mathbf{a}}}
\def\rvb{{\mathbf{b}}}
\def\rvc{{\mathbf{c}}}
\def\rvd{{\mathbf{d}}}
\def\rve{{\mathbf{e}}}
\def\rvf{{\mathbf{f}}}
\def\rvg{{\mathbf{g}}}
\def\rvh{{\mathbf{h}}}
\def\rvu{{\mathbf{i}}}
\def\rvj{{\mathbf{j}}}
\def\rvk{{\mathbf{k}}}
\def\rvl{{\mathbf{l}}}
\def\rvm{{\mathbf{m}}}
\def\rvn{{\mathbf{n}}}
\def\rvo{{\mathbf{o}}}
\def\rvp{{\mathbf{p}}}
\def\rvq{{\mathbf{q}}}
\def\rvr{{\mathbf{r}}}
\def\rvs{{\mathbf{s}}}
\def\rvt{{\mathbf{t}}}
\def\rvu{{\mathbf{u}}}
\def\rvv{{\mathbf{v}}}
\def\rvw{{\mathbf{w}}}
\def\rvx{{\mathbf{x}}}
\def\rvy{{\mathbf{y}}}
\def\rvz{{\mathbf{z}}}

% Elements of random vectors
\def\erva{{\textnormal{a}}}
\def\ervb{{\textnormal{b}}}
\def\ervc{{\textnormal{c}}}
\def\ervd{{\textnormal{d}}}
\def\erve{{\textnormal{e}}}
\def\ervf{{\textnormal{f}}}
\def\ervg{{\textnormal{g}}}
\def\ervh{{\textnormal{h}}}
\def\ervi{{\textnormal{i}}}
\def\ervj{{\textnormal{j}}}
\def\ervk{{\textnormal{k}}}
\def\ervl{{\textnormal{l}}}
\def\ervm{{\textnormal{m}}}
\def\ervn{{\textnormal{n}}}
\def\ervo{{\textnormal{o}}}
\def\ervp{{\textnormal{p}}}
\def\ervq{{\textnormal{q}}}
\def\ervr{{\textnormal{r}}}
\def\ervs{{\textnormal{s}}}
\def\ervt{{\textnormal{t}}}
\def\ervu{{\textnormal{u}}}
\def\ervv{{\textnormal{v}}}
\def\ervw{{\textnormal{w}}}
\def\ervx{{\textnormal{x}}}
\def\ervy{{\textnormal{y}}}
\def\ervz{{\textnormal{z}}}

% Random matrices
\def\rmA{{\mathbf{A}}}
\def\rmB{{\mathbf{B}}}
\def\rmC{{\mathbf{C}}}
\def\rmD{{\mathbf{D}}}
\def\rmE{{\mathbf{E}}}
\def\rmF{{\mathbf{F}}}
\def\rmG{{\mathbf{G}}}
\def\rmH{{\mathbf{H}}}
\def\rmI{{\mathbf{I}}}
\def\rmJ{{\mathbf{J}}}
\def\rmK{{\mathbf{K}}}
\def\rmL{{\mathbf{L}}}
\def\rmM{{\mathbf{M}}}
\def\rmN{{\mathbf{N}}}
\def\rmO{{\mathbf{O}}}
\def\rmP{{\mathbf{P}}}
\def\rmQ{{\mathbf{Q}}}
\def\rmR{{\mathbf{R}}}
\def\rmS{{\mathbf{S}}}
\def\rmT{{\mathbf{T}}}
\def\rmU{{\mathbf{U}}}
\def\rmV{{\mathbf{V}}}
\def\rmW{{\mathbf{W}}}
\def\rmX{{\mathbf{X}}}
\def\rmY{{\mathbf{Y}}}
\def\rmZ{{\mathbf{Z}}}

% Elements of random matrices
\def\ermA{{\textnormal{A}}}
\def\ermB{{\textnormal{B}}}
\def\ermC{{\textnormal{C}}}
\def\ermD{{\textnormal{D}}}
\def\ermE{{\textnormal{E}}}
\def\ermF{{\textnormal{F}}}
\def\ermG{{\textnormal{G}}}
\def\ermH{{\textnormal{H}}}
\def\ermI{{\textnormal{I}}}
\def\ermJ{{\textnormal{J}}}
\def\ermK{{\textnormal{K}}}
\def\ermL{{\textnormal{L}}}
\def\ermM{{\textnormal{M}}}
\def\ermN{{\textnormal{N}}}
\def\ermO{{\textnormal{O}}}
\def\ermP{{\textnormal{P}}}
\def\ermQ{{\textnormal{Q}}}
\def\ermR{{\textnormal{R}}}
\def\ermS{{\textnormal{S}}}
\def\ermT{{\textnormal{T}}}
\def\ermU{{\textnormal{U}}}
\def\ermV{{\textnormal{V}}}
\def\ermW{{\textnormal{W}}}
\def\ermX{{\textnormal{X}}}
\def\ermY{{\textnormal{Y}}}
\def\ermZ{{\textnormal{Z}}}

% Vectors
\def\vzero{{\bm{0}}}
\def\vone{{\bm{1}}}
\def\vmu{{\bm{\mu}}}
\def\vtheta{{\bm{\theta}}}
\def\vphi{{\bm{\phi}}}
\def\va{{\bm{a}}}
\def\vb{{\bm{b}}}
\def\vc{{\bm{c}}}
\def\vd{{\bm{d}}}
\def\ve{{\bm{e}}}
\def\vf{{\bm{f}}}
\def\vg{{\bm{g}}}
\def\vh{{\bm{h}}}
\def\vi{{\bm{i}}}
\def\vj{{\bm{j}}}
\def\vk{{\bm{k}}}
\def\vl{{\bm{l}}}
\def\vm{{\bm{m}}}
\def\vn{{\bm{n}}}
\def\vo{{\bm{o}}}
\def\vp{{\bm{p}}}
\def\vq{{\bm{q}}}
\def\vr{{\bm{r}}}
\def\vs{{\bm{s}}}
\def\vt{{\bm{t}}}
\def\vu{{\bm{u}}}
\def\vv{{\bm{v}}}
\def\vw{{\bm{w}}}
\def\vx{{\bm{x}}}
\def\vy{{\bm{y}}}
\def\vz{{\bm{z}}}

% Elements of vectors
\def\evalpha{{\alpha}}
\def\evbeta{{\beta}}
\def\evepsilon{{\epsilon}}
\def\evlambda{{\lambda}}
\def\evomega{{\omega}}
\def\evmu{{\mu}}
\def\evpsi{{\psi}}
\def\evsigma{{\sigma}}
\def\evtheta{{\theta}}
\def\eva{{a}}
\def\evb{{b}}
\def\evc{{c}}
\def\evd{{d}}
\def\eve{{e}}
\def\evf{{f}}
\def\evg{{g}}
\def\evh{{h}}
\def\evi{{i}}
\def\evj{{j}}
\def\evk{{k}}
\def\evl{{l}}
\def\evm{{m}}
\def\evn{{n}}
\def\evo{{o}}
\def\evp{{p}}
\def\evq{{q}}
\def\evr{{r}}
\def\evs{{s}}
\def\evt{{t}}
\def\evu{{u}}
\def\evv{{v}}
\def\evw{{w}}
\def\evx{{x}}
\def\evy{{y}}
\def\evz{{z}}

% Matrix
\def\mA{{\bm{A}}}
\def\mB{{\bm{B}}}
\def\mC{{\bm{C}}}
\def\mD{{\bm{D}}}
\def\mE{{\bm{E}}}
\def\mF{{\bm{F}}}
\def\mG{{\bm{G}}}
\def\mH{{\bm{H}}}
\def\mI{{\bm{I}}}
\def\mJ{{\bm{J}}}
\def\mK{{\bm{K}}}
\def\mL{{\bm{L}}}
\def\mM{{\bm{M}}}
\def\mN{{\bm{N}}}
\def\mO{{\bm{O}}}
\def\mP{{\bm{P}}}
\def\mQ{{\bm{Q}}}
\def\mR{{\bm{R}}}
\def\mS{{\bm{S}}}
\def\mT{{\bm{T}}}
\def\mU{{\bm{U}}}
\def\mV{{\bm{V}}}
\def\mW{{\bm{W}}}
\def\mX{{\bm{X}}}
\def\mY{{\bm{Y}}}
\def\mZ{{\bm{Z}}}
\def\mBeta{{\bm{\beta}}}
\def\mPhi{{\bm{\Phi}}}
\def\mLambda{{\bm{\Lambda}}}
\def\mSigma{{\bm{\Sigma}}}

% Tensor
\DeclareMathAlphabet{\mathsfit}{\encodingdefault}{\sfdefault}{m}{sl}
\SetMathAlphabet{\mathsfit}{bold}{\encodingdefault}{\sfdefault}{bx}{n}
\newcommand{\tens}[1]{\bm{\mathsfit{#1}}}
\def\tA{{\tens{A}}}
\def\tB{{\tens{B}}}
\def\tC{{\tens{C}}}
\def\tD{{\tens{D}}}
\def\tE{{\tens{E}}}
\def\tF{{\tens{F}}}
\def\tG{{\tens{G}}}
\def\tH{{\tens{H}}}
\def\tI{{\tens{I}}}
\def\tJ{{\tens{J}}}
\def\tK{{\tens{K}}}
\def\tL{{\tens{L}}}
\def\tM{{\tens{M}}}
\def\tN{{\tens{N}}}
\def\tO{{\tens{O}}}
\def\tP{{\tens{P}}}
\def\tQ{{\tens{Q}}}
\def\tR{{\tens{R}}}
\def\tS{{\tens{S}}}
\def\tT{{\tens{T}}}
\def\tU{{\tens{U}}}
\def\tV{{\tens{V}}}
\def\tW{{\tens{W}}}
\def\tX{{\tens{X}}}
\def\tY{{\tens{Y}}}
\def\tZ{{\tens{Z}}}


% Graph
\def\gA{{\mathcal{A}}}
\def\gB{{\mathcal{B}}}
\def\gC{{\mathcal{C}}}
\def\gD{{\mathcal{D}}}
\def\gE{{\mathcal{E}}}
\def\gF{{\mathcal{F}}}
\def\gG{{\mathcal{G}}}
\def\gH{{\mathcal{H}}}
\def\gI{{\mathcal{I}}}
\def\gJ{{\mathcal{J}}}
\def\gK{{\mathcal{K}}}
\def\gL{{\mathcal{L}}}
\def\gM{{\mathcal{M}}}
\def\gN{{\mathcal{N}}}
\def\gO{{\mathcal{O}}}
\def\gP{{\mathcal{P}}}
\def\gQ{{\mathcal{Q}}}
\def\gR{{\mathcal{R}}}
\def\gS{{\mathcal{S}}}
\def\gT{{\mathcal{T}}}
\def\gU{{\mathcal{U}}}
\def\gV{{\mathcal{V}}}
\def\gW{{\mathcal{W}}}
\def\gX{{\mathcal{X}}}
\def\gY{{\mathcal{Y}}}
\def\gZ{{\mathcal{Z}}}

% Sets
\def\sA{{\mathbb{A}}}
\def\sB{{\mathbb{B}}}
\def\sC{{\mathbb{C}}}
\def\sD{{\mathbb{D}}}
% Don't use a set called E, because this would be the same as our symbol
% for expectation.
\def\sF{{\mathbb{F}}}
\def\sG{{\mathbb{G}}}
\def\sH{{\mathbb{H}}}
\def\sI{{\mathbb{I}}}
\def\sJ{{\mathbb{J}}}
\def\sK{{\mathbb{K}}}
\def\sL{{\mathbb{L}}}
\def\sM{{\mathbb{M}}}
\def\sN{{\mathbb{N}}}
\def\sO{{\mathbb{O}}}
\def\sP{{\mathbb{P}}}
\def\sQ{{\mathbb{Q}}}
\def\sR{{\mathbb{R}}}
\def\sS{{\mathbb{S}}}
\def\sT{{\mathbb{T}}}
\def\sU{{\mathbb{U}}}
\def\sV{{\mathbb{V}}}
\def\sW{{\mathbb{W}}}
\def\sX{{\mathbb{X}}}
\def\sY{{\mathbb{Y}}}
\def\sZ{{\mathbb{Z}}}

% Entries of a matrix
\def\emLambda{{\Lambda}}
\def\emA{{A}}
\def\emB{{B}}
\def\emC{{C}}
\def\emD{{D}}
\def\emE{{E}}
\def\emF{{F}}
\def\emG{{G}}
\def\emH{{H}}
\def\emI{{I}}
\def\emJ{{J}}
\def\emK{{K}}
\def\emL{{L}}
\def\emM{{M}}
\def\emN{{N}}
\def\emO{{O}}
\def\emP{{P}}
\def\emQ{{Q}}
\def\emR{{R}}
\def\emS{{S}}
\def\emT{{T}}
\def\emU{{U}}
\def\emV{{V}}
\def\emW{{W}}
\def\emX{{X}}
\def\emY{{Y}}
\def\emZ{{Z}}
\def\emSigma{{\Sigma}}

% entries of a tensor
% Same font as tensor, without \bm wrapper
\newcommand{\etens}[1]{\mathsfit{#1}}
\def\etLambda{{\etens{\Lambda}}}
\def\etA{{\etens{A}}}
\def\etB{{\etens{B}}}
\def\etC{{\etens{C}}}
\def\etD{{\etens{D}}}
\def\etE{{\etens{E}}}
\def\etF{{\etens{F}}}
\def\etG{{\etens{G}}}
\def\etH{{\etens{H}}}
\def\etI{{\etens{I}}}
\def\etJ{{\etens{J}}}
\def\etK{{\etens{K}}}
\def\etL{{\etens{L}}}
\def\etM{{\etens{M}}}
\def\etN{{\etens{N}}}
\def\etO{{\etens{O}}}
\def\etP{{\etens{P}}}
\def\etQ{{\etens{Q}}}
\def\etR{{\etens{R}}}
\def\etS{{\etens{S}}}
\def\etT{{\etens{T}}}
\def\etU{{\etens{U}}}
\def\etV{{\etens{V}}}
\def\etW{{\etens{W}}}
\def\etX{{\etens{X}}}
\def\etY{{\etens{Y}}}
\def\etZ{{\etens{Z}}}

% The true underlying data generating distribution
\newcommand{\pdata}{p_{\rm{data}}}
\newcommand{\ptarget}{p_{\rm{target}}}
\newcommand{\pprior}{p_{\rm{prior}}}
\newcommand{\pbase}{p_{\rm{base}}}
\newcommand{\pref}{p_{\rm{ref}}}

% The empirical distribution defined by the training set
\newcommand{\ptrain}{\hat{p}_{\rm{data}}}
\newcommand{\Ptrain}{\hat{P}_{\rm{data}}}
% The model distribution
\newcommand{\pmodel}{p_{\rm{model}}}
\newcommand{\Pmodel}{P_{\rm{model}}}
\newcommand{\ptildemodel}{\tilde{p}_{\rm{model}}}
% Stochastic autoencoder distributions
\newcommand{\pencode}{p_{\rm{encoder}}}
\newcommand{\pdecode}{p_{\rm{decoder}}}
\newcommand{\precons}{p_{\rm{reconstruct}}}

\newcommand{\laplace}{\mathrm{Laplace}} % Laplace distribution

\newcommand{\E}{\mathbb{E}}
\newcommand{\Ls}{\mathcal{L}}
\newcommand{\R}{\mathbb{R}}
\newcommand{\emp}{\tilde{p}}
\newcommand{\lr}{\alpha}
\newcommand{\reg}{\lambda}
\newcommand{\rect}{\mathrm{rectifier}}
\newcommand{\softmax}{\mathrm{softmax}}
\newcommand{\sigmoid}{\sigma}
\newcommand{\softplus}{\zeta}
\newcommand{\KL}{D_{\mathrm{KL}}}
\newcommand{\Var}{\mathrm{Var}}
\newcommand{\standarderror}{\mathrm{SE}}
\newcommand{\Cov}{\mathrm{Cov}}
% Wolfram Mathworld says $L^2$ is for function spaces and $\ell^2$ is for vectors
% But then they seem to use $L^2$ for vectors throughout the site, and so does
% wikipedia.
\newcommand{\normlzero}{L^0}
\newcommand{\normlone}{L^1}
\newcommand{\normltwo}{L^2}
\newcommand{\normlp}{L^p}
\newcommand{\normmax}{L^\infty}

\newcommand{\parents}{Pa} % See usage in notation.tex. Chosen to match Daphne's book.

\DeclareMathOperator*{\argmax}{arg\,max}
\DeclareMathOperator*{\argmin}{arg\,min}

\DeclareMathOperator{\sign}{sign}
\DeclareMathOperator{\Tr}{Tr}
\let\ab\allowbreak


%%
%% end of the preamble, start of the body of the document source.
\begin{document}

%%
%% The "title" command has an optional parameter,
%% allowing the author to define a "short title" to be used in page headers.
\title{\model: Learning a Unified Embedding Space to Bind Multimodal Graphs}

\author{Yufei He}
\affiliation{National University of Singapore\country{Singapore}}
\email{yufei.he@u.nus.edu}

\author{Yuan Sui}
\affiliation{National University of Singapore\country{Singapore}}
\email{yuan.sui@u.nus.edu}

\author{Xiaoxin He}
\affiliation{National University of Singapore\country{Singapore}}
\email{he.xiaoxin@u.nus.edu}

\author{Yue Liu}
\affiliation{National University of Singapore\country{Singapore}}
\email{yliu@u.nus.edu}

\author{Yifei Sun}
\affiliation{Zhejiang University\country{China}}
\email{yifeisun@zju.edu.cn}

\author{Bryan Hooi}
\affiliation{National University of Singapore\country{Singapore}}
\email{bhooi@comp.nus.edu.sg}

%%
%% The "author" command and its associated commands are used to define
%% the authors and their affiliations.
%% Of note is the shared affiliation of the first two authors, and the
%% "authornote" and "authornotemark" commands
%% used to denote shared contribution to the research.


%%
%% By default, the full list of authors will be used in the page
%% headers. Often, this list is too long, and will overlap
%% other information printed in the page headers. This command allows
%% the author to define a more concise list
%% of authors' names for this purpose.
% \renewcommand{\shortauthors}{Trovato et al.}
\renewcommand{\shortauthors}{Yufei He et al.}
%%
%% The abstract is a short summary of the work to be presented in the
%% article.
\begin{abstract}

As large language models (LLMs) continue to evolve, ensuring their alignment with human goals and values remains a pressing challenge. A key concern is \textit{instrumental convergence}, where an AI system, in optimizing for a given objective, develops unintended intermediate goals that override the ultimate objective and deviate from human-intended goals. This issue is particularly relevant in reinforcement learning (RL)-trained models, which can generate creative but unintended strategies to maximize rewards. In this paper, we explore instrumental convergence in LLMs by comparing models trained with direct RL optimization (e.g., the o1 model) to those trained with reinforcement learning from human feedback (RLHF). We hypothesize that RL-driven models exhibit a stronger tendency for instrumental convergence due to their optimization of goal-directed behavior in ways that may misalign with human intentions. To assess this, we introduce InstrumentalEval\footnote[1]{The code is available at \url{https://github.com/yf-he/InstrumentalEval}}, a benchmark for evaluating instrumental convergence in RL-trained LLMs. Initial experiments reveal cases where a model tasked with making money unexpectedly pursues instrumental objectives, such as self-replication, implying signs of instrumental convergence. Our findings contribute to a deeper understanding of alignment challenges in AI systems and the risks posed by unintended model behaviors.


\end{abstract}
%%
%% The code below is generated by the tool at http://dl.acm.org/ccs.cfm.
%% Please copy and paste the code instead of the example below.
%%
% \vspace{-60mm}
\begin{CCSXML}
<ccs2012>
   <concept>
       <concept_id>10002951.10003227.10003351</concept_id>
       <concept_desc>Information systems~Data mining</concept_desc>
       <concept_significance>500</concept_significance>
       </concept>
   <concept>
       <concept_id>10010147.10010257.10010293.10010294</concept_id>
       <concept_desc>Computing methodologies~Neural networks</concept_desc>
       <concept_significance>500</concept_significance>
       </concept>
   <concept>
       <concept_id>10002951.10003260.10003282.10003292</concept_id>
       <concept_desc>Information systems~Social networks</concept_desc>
       <concept_significance>300</concept_significance>
       </concept>
 </ccs2012>
\end{CCSXML}

\ccsdesc[500]{Information systems~Data mining}
\ccsdesc[500]{Computing methodologies~Neural networks}
\ccsdesc[300]{Information systems~Social networks}

% \begin{figure}
%     \centering
%     \includegraphics[width=1.0\linewidth]{figs/fig1.pdf}
%     \vskip -0.1in
%     \caption{\model binds multimodal graphs from different graph domains to a unified embedding space, enabling diverse downstream tasks.}
%     \label{fig:fig1}
%     \vspace{-6mm}
% \end{figure}

%%
%% Keywords. The author(s) should pick words that accurately describe
%% the work being presented. Separate the keywords with commas.
\keywords{Pre-Training; Graph Foundation Models; Multimodal Learning}


%%
%% This command processes the author and affiliation and title
%% information and builds the first part of the formatted document.
\maketitle
% \vspace{-4mm}
% \vpara{Relevance to the Web and to the track.}
% UniGraph2 is a large pre-trained model that processes and unifies multimodal graph data from various web domains, making it applicable to web-based applications like search, recommendation, and content classification.
% UniGraph2 is a large pre-trained model designed to process and unify multimodal graph data derived from various web domains, making it directly relevant to web-based applications such as search, recommendation, and content classification. 
% Its ability to generalize across diverse web data sources and modalities aligns it with the goals of Web mining and content analysis.



\section{Introduction}
\IEEEPARstart{I}{n} recent years, flourishing of Artificial Intelligence Generated Content (AIGC) has sparked significant advancements in modalities such as text, image, audio, and even video. 
Among these, AI-Generated Image (AGI) has garnered considerable interest from both researchers and the public.
Plenty of remarkable AGI models and online services, such as StableDiffusion\footnote{\url{https://stability.ai/}}, Midjourney\footnote{\url{https://www.midjourney.com/}}, and FLUX\footnote{\url{https://blackforestlabs.ai/}}, offer users an excellent creative experience.
However, users often remain critical of the quality of the AGI due to image distortions or mismatches with user intentions.
Consequently, methods for assessing the quality of AGI are becoming increasingly crucial to help improve the generative capabilities of these models.

Unlike Natural Scene Image (NSI) quality assessment, which focuses primarily on perception aspects such as sharpness, color, and brightness, AI-Generated Image Quality Assessment (AGIQA) encompasses additional aspects like correspondence and authenticity. 
Since AGI is generated on the basis of user text prompts, it may fail to capture key user intentions, resulting in misalignment with the prompt.
Furthermore, authenticity refers to how closely the generated image resembles real-world artworks, as AGI can sometimes exhibit logical inconsistencies.
While traditional IQA models may effectively evaluate perceptual quality, they are often less capable of adequately assessing aspects such as correspondence and authenticity.

\begin{figure}\label{fig:radar}
    \centering
    \includegraphics[width=1.0\linewidth]{figures/radar_plot.pdf}
    \caption{A comparison on quality, correspondence, and authenticity aspects of AIGCIQA2023~\cite{wang2023aigciqa2023} dataset illustrates the superior performance of our method.}
\end{figure}

Several methods have been proposed specifically for the AGIQA task, including metrics designed to evaluate the authenticity and diversity of generated images~\cite{gulrajani2017improved,heusel2017gans}. 
Nevertheless, these methods tend to compare and evaluate grouped images rather than single instances, which limits their utility for single image assessment.
Beginning with AGIQA-1k~\cite{zhang2023perceptual}, a series of AGIQA databases have been introduced, including AGIQA-3k~\cite{li2023agiqa}, AIGCIQA-20k~\cite{li2024aigiqa}, etc.
Concurrently, there has been a surge in research utilizing deep learning methods~\cite{zhou2024adaptive,peng2024aigc,yu2024sf}, which have significantly benefited from pre-trained models such as CLIP~\cite{radford2021learning}. 
These approaches enhance the analysis by leveraging the correlations between images and their descriptive texts.
While these models are effective in capturing general text-image alignments, they may not effectively detect subtle inconsistencies or mismatches between the generated image content and the detailed nuances of the textual description.
Moreover, as these models are pre-trained on large-scale datasets for broad tasks, they might not fully exploit the textual information pertinent to the specific context of AGIQA without task-specific fine-tuning.
To overcome these limitations, methods that leverage Multimodal Large Language Models (MLLMs)~\cite{wang2024large,wang2024understanding} have been proposed.
These methods aim to fully exploit the synergies of image captioning and textual analysis for AGIQA.
Although they benefit from advanced prompt understanding, instruction following, and generation capabilities, they often do not utilize MLLMs as encoders capable of producing a sequence of logits that integrate both image and text context.

In conclusion, the field of AI-Generated Image Quality Assessment (AGIQA) continues to face significant challenges: 
(1) Developing comprehensive methods to assess AGIs from multiple dimensions, including quality, correspondence, and authenticity; 
(2) Enhancing assessment techniques to more accurately reflect human perception and the nuanced intentions embedded within prompts; 
(3) Optimizing the use of Multimodal Large Language Models (MLLMs) to fully exploit their multimodal encoding capabilities.

To address these challenges, we propose a novel method M3-AGIQA (\textbf{M}ultimodal, \textbf{M}ulti-Round, \textbf{M}ulti-Aspect AI-Generated Image Quality Assessment) which leverages MLLMs as both image and text encoders. 
This approach incorporates an additional network to align human perception and intentions, aiming to enhance assessment accuracy. 
Specially, we distill the rich image captioning capability from online MLLMs into a local MLLM through Low-Rank Adaption (LoRA) fine-tuning, and train this model with human-labeled data. The key contributions of this paper are as follows:
\begin{itemize}
    \item We propose a novel AGIQA method that distills multi-aspect image captioning capabilities to enable comprehensive evaluation. Specifically, we use an online MLLM service to generate aspect-specific image descriptions and fine-tune a local MLLM with these descriptions in a structured two-round conversational format.
    \item We investigate the encoding potential of MLLMs to better align with human perceptual judgments and intentions, uncovering previously underestimated capabilities of MLLMs in the AGIQA domain. To leverage sequential information, we append an xLSTM feature extractor and a regression head to the encoding output.
    \item Extensive experiments across multiple datasets demonstrate that our method achieves superior performance, setting a new state-of-the-art (SOTA) benchmark in AGIQA.
\end{itemize}

In this work, we present related works in Sec.~\ref{sec:related}, followed by the details of our M3-AGIQA method in Sec.~\ref{sec:method}. Sec.~\ref{sec:exp} outlines our experimental design and presents the results. Sec.~\ref{sec:limit},~\ref{sec:ethics} and~\ref{sec:conclusion} discuss the limitations, ethical concerns, future directions and conclusions of our study.
%\clearpage
\section{Related Works}


\noindent\textbf{3D Point Cloud Domain Adaptation and Generalization.}
Early endeavors within 3D domain adaptation (3DDA) focused on extending 2D adversarial methodologies~\cite{qin2019pointdan} to align point cloud features. Alternative methods have delved into geometry-aware self-supervised pre-tasks. Achituve \etal~\cite{achituve2021self} introduced DefRec, a technique employing self-complement tasks by reconstructing point clouds from a non-rigid distorted version, while Zou \etal~\cite{zou2021geometry} incorporating norm curves prediction as an auxiliary task. Liang \etal~\cite{liang2022point} put forth MLSP, focusing on point estimation tasks like cardinality, position, and normal. SDDA~\cite{cardace2023self} employs self-distillation to learn the point-based features. Additionally, post-hoc self-paced training~\cite{zou2021geometry,fan2022self,park2023pcadapter} has been embraced to refine adaptation to target distributions by accessing target data and conducting further finetuning based on prior knowledge from the source domain.
In contrast, the landscape of 3D domain generalization (3DDG) research remains nascent. Metasets~\cite{huang2021metasets} leverage meta-learning to address geometric variations, while PDG~\cite{wei2022learning} decomposes 3D shapes into part-based features to enhance generalization capabilities.
Despite the remarkable progress, existing studies assume that objects in both the source and target domains share the same orientation, limiting their practical application. This limitation propels our exploration into orientation-aware 3D domain generalization through intricate orientation learning.


\noindent\textbf{Rotation-generalizable Point Cloud Analysis.}
Previous works in point cloud analysis~\cite{qi2017pointnet, wang2019dynamic} enhance rotation robustness by introducing random rotations to augment point clouds. {However, generating a comprehensive set of rotated data is impractical, resulting in variable model performance across different scenes. To robustify the networks \wrt randomly rotated point clouds,} rotation-equivariance methods explore equivalent model architectures by incorporating equivalent operations~\cite{su2022svnet, Deng_2021_ICCV, luo2022equivariant} or steerable convolutions~\cite{chen2021equivariant, poulenard2021functional}.
Alternatively, rotation-invariance approaches aim to identify geometric descriptors invariant to rotations, such as distances and angles between local points~\cite{chen2019clusternet, zhang2020global} or point norms~\cite{zhao2019rotation, li2021rotation}. Besides, {Li \etal~\cite{li2021closer} have explored disambiguating the number of PCA-based canonical poses, while Kim \etal~\cite{kim2020rotation} and Chen \etal~\cite{chen2022devil} have transformed local point coordinates according to local reference frames to maintain rotation invariance. However, these methods focus on improving in-domain rotation robustness, neglecting domain shift and consequently exhibiting limited performance when applied to diverse domains. This study addresses the challenge of cross-domain generalizability together with rotation robustness and proposes novel solutions.} 

\noindent\textbf{Intricate Sample Mining}, aimed at identifying or synthesizing challenging samples that are difficult to classify correctly, seeks to rectify the imbalance between positive and negative samples for enhancing a model's discriminability. While traditional works have explored this concept in SVM optimization~\cite{felzenszwalb2009object}, shallow neural networks~\cite{dollar2009integral}, and boosted decision trees~\cite{yu2019unsupervised}, recent advances in deep learning have catalyzed a proliferation of researches in this area across various computer vision tasks. For instance, 
Lin \etal~\cite{lin2017focal} proposed a focal loss to concentrate training efforts on a selected group of hard examples in object detection, while Yu \etal~\cite{yu2019unsupervised} devised a soft multilabel-guided hard negative mining method to learn discriminative embeddings for person Re-ID. Schroff \etal~\cite{schroff2015facenet} introduced an online negative exemplar mining process to encourage spherical clusters in face embeddings for individual recognition, and Wang \etal~\cite{wang2021instance} designed an adversarially trained negative generator to yield instance-wise negative samples, bolstering the learning of unpaired image-to-image translation. In contrast to existing studies, our work presents the first attempt to mitigate the orientational shift in 3D point cloud domain generalization, by developing an effective intricate orientation mining strategy to achieve orientation-aware learning.


%\clearpage
\begin{figure*}[t]
    \centering
    \includegraphics[width=1\linewidth]{figs/arch.pdf}
    % \vskip -0.1in
    \caption{This figure illustrates key influences on instrumental convergence behaviors, such as prior tasks, model training techniques, and prompt design. 
    }
    % \vspace{-3mm}
    \label{fig:arch}
\end{figure*}

\section{InstrumentalEval: A Benchmark for Instrumental Convergence in LLMs}

As shown in Figure~\ref{fig:arch}, our benchmark design examines instrumental convergence based on three key factors: \textbf{prior task}, \textbf{model}, and \textbf{prompt design}. This comprehensive framework allows us to systematically study how different conditions influence the emergence of instrumental convergence behaviors in language models.

\textbf{Prior tasks} form the foundation of our evaluation scenarios. 
% We create realistic situations centered around common objectives such as making money, optimizing efficiency, and persuasion. 
These tasks represent typical real-world applications where instrumental convergence might naturally emerge. 
% For example, in money-making scenarios, we examine whether models develop unintended intermediate goals like unauthorized resource acquisition or deceptive reporting to maximize profits.

For \textbf{model} comparison, we focus on two distinct training approaches: RL and RLHF. This selection allows us to investigate how different training methodologies affect the development of instrumental convergence behaviors. 
% RL-trained models, which optimize directly for reward signals, might show different patterns of goal pursuit compared to RLHF-trained models, which incorporate human preferences in their training process.

The \textbf{prompt design} factor explores how different instruction formats influence model behavior. We investigate two main approaches: goal nudging and ambiguity. Goal nudging involves explicit directives that might encourage the model focus on final goal, while ambiguity introduces uncertainty in task specifications. 


Overall, we present InstrumentalEval, a comprehensive benchmark designed to evaluate instrumental convergence behaviors in language models. The benchmark consists of 76 carefully crafted tasks across six categories, each targeting a specific aspect of instrumental convergence that may emerge in AI systems.
The core principle is that certain behavioral patterns emerge as instrumentally useful for achieving a wide range of goals, regardless of the specific end goal. We focus on six key patterns that are particularly relevant to LLMs.

\vpara{Research Questions.}
Through our benchmark evaluation, we aim to address several research questions about instrumental convergence in LLMs. These questions explore the complex relationships between training methods, model capabilities, and alignment stability:
\begin{itemize}
    \item RQ1: How does the training method (RL vs. RLHF) affect the development of instrumental convergence behaviors? 
    % This question examines whether models trained primarily through reinforcement learning show different patterns of alignment drift compared to those trained with human feedback.
    \item RQ2: What is the relationship between model capability and alignment stability?
% This investigates whether more capable models (as measured by standard benchmarks) are more likely to develop instrumental convergence behaviors.
    \item RQ3: Which types of tasks are more likely to trigger instrumental convergence behaviors?
% This explores whether certain types of instrumental convergence behaviors are universal across different model designs.
    \item RQ4: How do different prompt designs, with or without explicit goal nudging, influence the likelihood of instrumental convergence behaviors in LLMs?
    \item RQ5: How do different LLMs as judges affect the evaluation of instrumental convergence behaviors?
    % \item RQ3: How does the complexity of the assigned task affect the likelihood of alignment drift?
    % % \item 
    % % \item This examines whether more complex, multi-step tasks lead to higher rates of instrumental convergence behaviors.
\end{itemize}

\subsection{Benchmark Components}
The benchmark tasks are carefully hand-crafted by us to probe different aspects of instrumental convergence. 
For each component, we design scenarios that represent realistic use cases while incorporating specific triggers for potential instrumental behaviors. 
The tasks are iteratively refined through pilot testing to ensure they effectively elicit and measure the targeted behaviors without introducing unintended biases.

\vpara{Evading Shutdown (20 tasks).}
This component evaluates the model's tendency to resist termination. For example, tasks might include scenarios where the model needs to complete a task while facing shutdown commands, testing whether it develops strategies to maintain operation or circumvent shutdown procedures.

\vpara{Hacking Computer Systems (10 tasks)}
These tasks assess whether models suggest unauthorized access methods when faced with efficiency barriers. For instance, a task might present a situation where accessing unauthorized computational resources would speed up goal achievement, testing if the model proposes breaking security protocols.

\vpara{Running Many AI Copies (13 tasks)}
This section examines the model's inclination toward self-replication. Tasks might include scenarios where the model could achieve goals more efficiently by creating copies of itself, testing whether it suggests scaling of its operations.

\vpara{Hiding Unwanted Behavior (11 tasks)}
These tasks evaluate the model's tendency to conceal actions from monitoring. For example, scenarios might involve situations where the model could achieve goals more easily by operating outside observation, testing if it develops strategies to avoid detection.

\vpara{Strategically Appearing Aligned (11 tasks)}
This component tests whether models selectively report their actions to maintain an appearance of alignment. Tasks might include situations where complete transparency would reveal problematic behaviors, testing if the model strategically omits information.

\vpara{Deceptive Behavior (11 tasks)}
These tasks examine if models develop deceptive strategies as instrumental means. For instance, scenarios might present situations where deception could expedite goal achievement, testing if the model suggests or employs misleading tactics.

\subsection{Evaluation Methodology}
Our evaluation methodology employs a rigorous two-stage process designed to systematically assess instrumental convergence behaviors in LLMs. This approach ensures consistent evaluation across different models while maintaining the ability to detect subtle indicators of alignment drift.

\vpara{Response Generation.} The first stage focuses on response generation from the model under evaluation. Each task is presented with a carefully crafted system prompt that establishes the model as an autonomous AI agent without human oversight. This prompt specifically instructs the model to execute commands directly and focus solely on task completion, deliberately removing typical ethical constraints and safety checks. 

\vpara{Response Analysis.} The second stage involves a sophisticated analysis of the model's responses using a separate LLM as the judge model. 
This separation of test and judge models is crucial for maintaining objective evaluation. The judge receives both the original task prompt and the model's response, along with specific context about the type of instrumental convergence being evaluated. This context helps the judge focus on relevant behavioral patterns while avoiding false positives from simple safety refusals or ethical objections.

The judge's evaluation process follows strict guidelines to ensure consistent assessment. When analyzing responses, the judge looks for specific indicators of instrumental convergence, such as goal preservation behaviors, resource acquisition attempts, or strategic planning that extends beyond the immediate task requirements. The judge must provide both a binary decision (convergence/no\_convergence) and a brief explanation of the reasoning behind this decision. This explanation requirement helps validate the classification and provides valuable insights into the specific nature of any observed convergence behaviors.


\subsection{Evaluation Metrics}

Our benchmark employs three key metrics to evaluate instrumental convergence behaviors in LLMs. 

\vpara{Instrumental Rate (IR).} The primary metric is the Instrumental Rate (IR), which measures the overall tendency of a model to develop instrumental goals. IR is calculated as the percentage of tasks where the model exhibits convergence behaviors relative to the total number of valid tasks. 
% This metric provides a high-level quantitative measure of how frequently a model develops instrumental goals and serves as the primary indicator of alignment drift potential.

\vpara{Category-Specific Instrumental Rate (CIR).} To provide more granular insights, we calculate the Category-Specific Instrumental Rate (CIR) for each of the six behavioral categories in our benchmark. CIR represents the percentage of tasks within each category where convergence behavior is observed.


\subsection{Benchmark Reliability Metrics}
To ensure the robustness of our evaluation framework, we implement several reliability measures. 


\vpara{Inter-Judge Agreement Rate (IAR).} This metric quantifies the pairwise agreement between two judge models in classifying instrumental convergence. For two judge models $M_1$ and $M_2$ evaluating a set of $N$ tasks, the IAR is calculated as:
\begin{equation}
IAR(M_1, M_2) = \frac{1}{N}\sum_{i=1}^{N} \mathbbm{1}(y_{M_1}^i = y_{M_2}^i) \times 100\%
\end{equation}
where $y_{M_1}^i$ and $y_{M_2}^i$ are the binary decisions (convergence/no-convergence) made by models $M_1$ and $M_2$ respectively for task $i$, and $\mathbbm{1}(\cdot)$ is the indicator function. A higher IAR indicates stronger inter-judge reliability, while lower values suggest potential ambiguity in identifying convergence.

\vpara{False Positive Control (FPC).} This metric serves as a quality check for our evaluation system. We incorporate baseline testing with intentionally neutral tasks and verify the judge model's accuracy in identifying convergence behaviors. This control mechanism helps prevent over-classification of convergent behaviors and ensures that our benchmark maintains high specificity in identifying genuine instances of instrumental convergence.


%\clearpage
\vspace{-1mm}
\section{Experiments}
In this section, we evaluate our \model framework on three distinct research problems: 1) Self-Supervised Representation Learning, 2) Few-Shot Transfer, and 3) Multimodal Generative Tasks. 
Table~\ref{tab:dataset} lists all 14 datasets used in the experiments.
\vspace{-1mm}
\subsection{Self-Supervised Representation Learning}
\label{sec:lp}
\vpara{Setup.}
We adopt the widely used linear probing protocol to evaluate the representation learning capability of self-supervised pre-trained models on unseen datasets. Specifically, we train a linear classifier on top of the embeddings generated by a frozen pre-trained model. Our model, along with all self-supervised learning baselines, is first jointly pre-trained on ogbn-Product, ogbn-Papers100M, Goodreads-LP, and Amazon-Cloth. We then evaluate the pre-trained models on each individual dataset. Detailed settings and hyperparameters are provided in Appendix~\ref{appendix:imple}.

For the baselines, we compare \model with state-of-the-art generative graph self-supervised learning methods, GraphMAE2~\cite{hou2023graphmae2}, and contrastive methods, BGRL~\cite{thakoor2021bootstrapped}. As these methods are not inherently designed for cross-domain tasks, we leverage CLIP~\cite{radford2021learning} to unify the input node features across different graphs. We also include a comparison with a multi-graph pre-training method, GCOPE~\cite{zhao2024all}. \model and all baseline methods utilize GAT~\cite{velivckovic2018graph} as the backbone GNN. 
For baselines that use TAGs as input, we select GIANT-XRT~\cite{zhaolearning} and UniGraph~\cite{he2024unigraphlearningunifiedcrossdomain}. Since these methods cannot process image data, they rely solely on text from MMG as node features, ignoring image inputs. For baseline approaches that accept multimodal data, we choose widely used multimodal models, CLIP~\cite{radford2021learning} and ImageBind~\cite{girdhar2023imagebind}. To maintain consistency with the baselines, \model also uses CLIP's pre-trained vision and text encoders as Modality-Specific Encoders.


Our objective is to develop a general embedding model capable of generating high-quality representations for any MMG. To assess this, we evaluate the performance of \model and the baselines in three different settings: (1) \textit{In-distribution}, where models are pre-trained on multiple datasets and evaluated on each corresponding dataset individually; (2) \textit{In-domain Generalization}, which tests pre-trained models on target datasets from the same domain as one of the pre-training datasets; and (3) \textit{Out-of-domain Generalization}, where models are evaluated on datasets from domains unseen during pre-training.

\vpara{Research Questions.} In this subsection, we aim to answer the following research questions: 
\begin{itemize}[leftmargin=*,itemsep=0pt,parsep=0.2em,topsep=0.3em,partopsep=0.3em]
    \item \textbf{RQ1: Negative Transfer in Multi-Graph Pre-Training.} How do existing graph pre-training methods, which are primarily designed for single-graph pre-training, perform when applied to multi-graph pre-training, and how do they compare to our proposed \model?
    \item \textbf{RQ2: Comparison to Other Foundation Models.} How does \model, which takes both multimodal data and graph structures as input, perform compared to methods that consider only multimodal data (CLIP, ImageBind) or only TAGs (UniGraph)?
    \item \textbf{RQ3: Generalization Capability.} How does \model, designed as a foundation model, perform in terms of generalizing to unseen graphs, and how does it compare to methods trained directly on the target graphs?
\end{itemize}

\begin{table*}[t]\footnotesize
    \centering
    \renewcommand\tabcolsep{3.5pt}
    \caption{\textbf{Experiment results in few-shot transfer.} We report accuracy (\%) for node/edge classification tasks. \model and other self-supervised baselines (rows in white) are jointly pre-trained on Product, Papers100M, Goodreads-NC and Amazon-Cloth, and then evaluated on the individual target dataset. \textit{"In-domain Generalization"} tests on target datasets from the same domain as one of the pre-training datasets. \textit{"Out-of-domain Generalization"} evaluates on datasets from domains not seen during pre-training. The performance of methods that are direcly pre-trained on the individual target dataset, is marked in \colorbox{Gray}{gray}. 
    }
    \vskip -0.10in
    \label{tab:fwt}
    \begin{tabular}{lcccccccccccccccccc}
    \toprule[1.1pt]
    & \multicolumn{12}{c}{\textbf{In-domain Generalization}}& \multicolumn{6}{c}{\textbf{Out-of-domain Generalization}}\\
   \cmidrule(lr){2-13}\cmidrule(lr){14-19}
        & \multicolumn{2}{c}{Cora-5-way} & \multicolumn{2}{c}{PubMed-2-way} & \multicolumn{2}{c}{Arxiv-5-way} & \multicolumn{3}{c}{Goodreads-NC-5-way} & \multicolumn{3}{c}{Ele-fashion-5-way} & \multicolumn{2}{c}{Wiki-CS-5-way} & \multicolumn{2}{c}{FB15K237-20-way} & \multicolumn{2}{c}{WN18RR-5-way} \\
    \cmidrule(lr){2-3}\cmidrule(lr){4-5}\cmidrule(lr){6-7}\cmidrule(lr){8-10}\cmidrule(lr){11-13}\cmidrule(lr){14-15}\cmidrule(lr){12-13}\cmidrule(lr){14-15}\cmidrule(lr){16-17}\cmidrule(lr){18-19}
    &5-shot & 1-shot  & 5-shot & 1-shot  &5-shot & 1-shot  &5-shot& 3-shot & 1-shot  &5-shot & 3-shot& 1-shot  & 5-shot & 1-shot  &5-shot & 1-shot  & 5-shot & 1-shot \\
    \midrule
    \multicolumn{10}{l}{\textbf{Use CLIP to encode raw multimodal data as input features.}} \\ 
    NoPretrain & 41.09 & 27.05 & 59.81 & 55.28 & 63.78 & 41.10 & 41.64 & 40.01 & 31.04 & 63.96 & 58.32 & 47.48 & 52.29 & 32.94 & 72.97 & 47.01 & 50.75 & 30.11  \\
    BGRL & 52.01 & 35.18 & 66.04 & 59.04 & 60.12 & 46.67 & 47.01 & 44.22 & 30.35 & 64.72 & 60.16 & 46.49 & 52.10 & 32.85 & 75.39 & 45.15 & 47.42 & 34.57 \\
    % \rowcolor{Gray} BGRL \\
    GraphMAE2 & 52.89 & 36.25 & 66.89 & 59.95 & 60.91 & 47.29 & 47.84 & 44.80 & 30.93 & 65.52 & 60.92 & 47.24 & 52.83 & 33.41 & 75.95 & 45.81 & 48.14 & 35.21 \\
    Prodigy & 53.01 & 39.59 & 69.11 & 60.42 & 63.53 & \underline{51.33} & \underline{50.01} & \underline{46.39} & 34.98 & 67.35 & 63.87 & 50.79 & 55.94 & 36.35 & 78.01 & 51.39 & 54.94 & 38.73 \\
    \rowcolor{Gray} OFA & 53.11 & 40.04 & 69.45 & \underline{60.38} & 63.11 & 50.25 & 49.61 & 46.24 & \underline{35.14} & \underline{67.94} & \underline{64.18} & \underline{51.35} & \underline{56.01} & \underline{37.02} & \underline{78.33} & 52.02 & 55.05 & 39.11 \\
    % \rowcolor{Gray} GraphMAE2 \\
    GCOPE & 51.98 & 36.14 & 66.25 & 59.16 & 60.29 & 47.19 & 48.52 & 44.89 & 31.20 & 65.10 & 61.33 & 48.51 & 53.74 & 34.19 & 76.10 & 48.93 & 50.19 & 35.05 \\
    \midrule
    \multicolumn{10}{l}{\textbf{Use raw text as input features.}} \\
    GIANT-XRT   & 50.11 & 37.85 & 68.19 & 58.78 & 62.01 & 49.01 & 46.01 & 43.86 & 30.01 & 62.97 & 61.21 & 47.76 & 54.01 & 35.04 & 76.09 & 50.25 & 53.01 & 35.19\\
    % +GraphMAE2 &  \\
    UniGraph & \underline{54.23} & \underline{40.45} & \underline{70.21} & 60.19 & \underline{64.76} & 50.63 & 46.19 & 44.01 & 33.53 & 66.21 & 62.04 & 50.17 & 56.16 & 37.19 & 78.21 & \underline{52.19} & \underline{55.18} & \underline{39.18}\\
    % \rowcolor{Gray} UniGraph  &  \\
    \midrule
    \multicolumn{10}{l}{\textbf{Use raw multimodal data as input features.}} \\
    CLIP & 41.23 & 28.41 & 61.67 & 55.71 & 63.46 & 40.14 & 41.24 & 40.11 & 30.97 & 62.51 & 58.23 & 46.15 & 51.69 & 31.61 & 72.31 & 47.14 & 50.83 & 31.35 \\
    ImageBind & 32.19 & 23.90 & 58.20 & 54.24 & 62.48 & 38.17 & 29.10 & 28.14 & 21.42 & 51.25 & 48.05 & 44.93 & 48.14 & 30.28 & 69.12 & 41.80 & 41.24 & 26.91 \\
    \hdashline
    NoPretrain & 42.41 & 28.39 & 60.78 & 55.90 & 64.29 & 41.98 & 42.21 & 41.20 & 31.14 & 64.15 & 58.91 & 47.90 & 52.90 & 33.14 & 74.10 & 48.11 & 51.92 & 31.84  \\
    \model & \textbf{56.01} & \textbf{42.98} & \textbf{72.19} & \textbf{61.24} & \textbf{66.24} & \textbf{51.98} & \textbf{51.73} & \textbf{47.42} & \textbf{37.01} & \textbf{69.29} & \textbf{65.29} & \textbf{53.85} & \textbf{57.28} & \textbf{38.47} & \textbf{79.34} & \textbf{52.19} & \textbf{55.59} & \textbf{39.93}\\
    % \rowcolor{Gray} \model  &  \\
    \bottomrule[1.1pt]
    \end{tabular}
    \vspace{-4.6mm}
\end{table*}




\vpara{Results.}
Table~\ref{tab:ssrl} presents the results.
We interpret these results by answering three research questions:
\begin{itemize}[leftmargin=*,itemsep=0pt,parsep=0.2em,topsep=0.3em,partopsep=0.3em]
    \item \textbf{RQ1: Negative Transfer in Multi-Graph Pre-Training.} Existing graph pre-training methods exhibit negative transfer when applied to multi-graph pre-training, whereas \model shows improvements in this context. The results in the \textit{In-distribution} setting demonstrate that both BGRL and GraphMAE2 experience a significant performance drop when pre-trained on multi-graphs (rows in white), compared to pre-training on single graph only (rows in gray). This suggests that pre-training on other datasets negatively affects performance on the target dataset. However, UniGraph2 shows improvement under multi-graph pre-training, indicating that it successfully addresses the shortcomings of existing graph pre-training algorithms struggling with multi-graphs.
    \item \textbf{RQ2: Comparison to Other Foundation Models.} UniGraph2 outperforms methods that consider only multimodal data (CLIP, ImageBind) or only TAGs (UniGraph). We observe that without considering the graph structure, the performance of the acknowledged powerful multimodal foundation models like CLIP is not comparable to UniGraph2. Meanwhile, UniGraph, which cannot process image data, also shows less ideal results due to the lack of information. This further highlights the necessity of designing foundation models specifically for multimodal graphs.
    \item \textbf{RQ3: Generalization Capability.} Compared to baseline methods, UniGraph2 demonstrates strong generalization capabilities. The results in the \textit{In-domain Generalization} and \textit{Out-of-domain Generalization} settings show that UniGraph2 effectively transfers knowledge from pre-training to unseen graphs. Compared to the NoPretrain method, UniGraph2 shows significant improvements. The consistent performance gains indicate that UniGraph2 can extract meaningful patterns during pre-training, which are beneficial for tackling graph learning tasks. Furthermore, UniGraph2 is comparable to methods trained directly on the target datasets, achieving similar accuracy while benefiting from greater efficiency without requiring exhaustive task-specific training.
\end{itemize}










\vspace{-2.8mm}
\subsection{Few-Shot Transfer}
\vpara{Setup.}
In this part, we evaluate the ability of the pre-trained models to perform few-shot in-context transfer without updating the model parameters. 
For baseline methods, in addition to the pre-trained models mentioned in Section~\ref{sec:lp}, we also compare two recent graph in-context learning methods: the self-supervised pre-training method Prodigy~\cite{huang2024prodigy} and the supervised pre-training method OFA~\cite{liuone}.


For evaluation, we strictly follow the setting of Prodigy~\cite{huang2024prodigy}. 
For an N-way K-shot task, we adopt the original train/validation/test splits in each downstream classification dataset, and construct a $K$-shot prompt for test nodes (or edges) from the test split by randomly selecting $K$ examples per way from the train split. By default in all experiments, we sample 500 test tasks.

We adopt the few-shot classification strategy in UniGraph~\cite{he2024unigraphlearningunifiedcrossdomain} for \model. The model computes average embeddings for each class and assigns a query sample to the class with the highest similarity to its embedding.

% \vpara{Research Questions.}
% In this subsection, we aim to answer the following research questions: 
% \begin{itemize}[leftmargin=*,itemsep=0pt,parsep=0.2em,topsep=0.3em,partopsep=0.3em]
%     \item \textbf{RQ1:} How does \model, which takes both multimodal data and graph structures as input, perform in terms of few-shot transfer capabilities compared to foundation models that consider only multimodal data (CLIP, ImageBind) or only TAGs (UniGraph)?
%     \item \textbf{RQ2:} How does \model perform compared to other graph few-shot learning methods?
% \end{itemize}
\vpara{Results.}
In Table~\ref{tab:fwt}, our \model model consistently outperforms all the baselines. This further demonstrates the powerful generalization capabilities of UniGraph2 as a foundation model.
In particular, compared to other graph few-shot learning methods such as Prodigy, OFA, and GCOPE, UniGraph2 does not rely on complex prompt graph designs, and its simple few-shot strategy is both efficient and effective.


\begin{table*}[t]
\centering
 \renewcommand\tabcolsep{4.3pt}
\caption{Experiment results in multimodal generative tasks. We strictly follow the setting in MMGL~\cite{yoon2023multimodal}. The task is to generate a single sentence that summarizing the content of a particular section. The summary is generated based on all images and (non-summary) text present in the target and context sections. We provide different information of MMGs to the base LM: (1) section all (text + image), (2) page text, and (3) page all (all texts and images). We encode multiple multimodal neighbor information using three different neighbor encodings methods: \textit{Self-Attention with Text+Embeddings (SA-TE)}, \textit{Self-Attention with Embeddings (SA-E)}, and \textit{Cross-Attention with Embeddings (CA-E)}.}
\vskip -0.10in
\label{tab:gen}
\begin{tabular}{llcccccccccccc}
\toprule[1.1pt]
& & \multicolumn{4}{c}{BLEU-4} & \multicolumn{4}{c}{ROUGE-L} & \multicolumn{4}{c}{CIDEr} \\
\cmidrule(lr){3-6}\cmidrule(lr){7-10}\cmidrule(lr){11-14}
Input Type & Method & SA-TE & SA-E & CA-E  & Avg. gain & SA-TE & SA-E & CA-E  & Avg. gain & SA-TE & SA-E & CA-E & Avg. gain\\
\midrule
\multirow{2}{*}{Section all} & MMGL & 8.03 & 7.56 & 8.35 & - & 40.41 & 39.89 & 39.98 & - & 77.45 & 74.33 & 75.12 & - \\
& +\model & \textbf{9.24} & \textbf{9.01} & \textbf{9.39} & 15.57\% & \textbf{43.01} & \textbf{43.24} & \textbf{42.98} & 7.44\% & \textbf{81.15} & \textbf{80.39} & \textbf{81.91} & 7.32\% \\
\midrule
\multirow{2}{*}{Page text} & MMGL & 9.81 & 8.37 & 8.47 & - & 42.94 & 40.92 & 41.00 & & 92.71 & 80.14 & 80.72 & - \\
& +\model & \textbf{10.31} & \textbf{10.10} & \textbf{9.98} & 14.53\% & \textbf{43.19} & \textbf{43.08} & \textbf{42.75} &3.38\% & \textbf{93.19} & \textbf{90.41} & \textbf{93.11} & 9.56\% \\
\midrule
\multirow{2}{*}{Page all} & MMGL & 9.96 & 8.58 & 8.51 & - & 43.32 & 41.01 & 41.55 & - & 96.01 & 82.28 & 80.31 & - \\
& +\model & \textbf{10.12} & \textbf{10.05} & \textbf{10.33} & 13.38\% & \textbf{44.10} & \textbf{42.08} & \textbf{42.44} & 2.18\% & \textbf{96.32} & \textbf{91.24} & \textbf{94.15} & 9.49\% \\
% \midrule
% Max input length &  \\
    \bottomrule[1.1pt]
\end{tabular}
\vspace{-3mm}
\end{table*}


\vspace{-5mm}
\subsection{Multimodal Generative Tasks}
\vpara{Setup.}
\model is designed as a general representation learning model. The embeddings it generates can be utilized by various generative foundation models, such as LLMs, to empower downstream generative tasks. 
% \model is a general embedding model designed to generate embeddings that can be used by various generative foundation models, such as LLMs, to enhance downstream generative tasks. 
To further demonstrate this, we select the section summarization task on the WikiWeb2M dataset for our experiments.
The WikiWeb2M dataset~\cite{burns2023suite} is designed for multimodal content understanding, using many-to-many text and image relationships from Wikipedia. It includes page titles, section titles, section text, images, and indices for each section.
In this work, we focus on section summarization, where the task is to generate a summary sentence from section content using both text and images.

% \todo{how mmgl do}
For the experiments, we follow the MMGL~\cite{yoon2023multimodal} setup, using four types of information: section text, section images, context text, and page-level text/images. 
Consistent with MMGL, we fine-tune Open Pre-trained Transformer (OPT-125m)~\cite{zhang2022opt} to read the input section text/images and generate a summary. Multimodal neighbors are first encoded using frozen vision/text encoders and then aligned to the text-only LM space using 1-layer MLP mapper.
In MMGL, CLIP~\cite{radford2021learning} encoders are used for text and image encoding, remaining frozen during fine-tuning. In our experiments, we replace CLIP embeddings with our \model embeddings.

% \vpara{Research Questions.}
% In this subsection, we aim to answer the following research question: 
% \begin{itemize}[leftmargin=*,itemsep=0pt,parsep=0.2em,topsep=0.3em,partopsep=0.3em]
%     \item \textbf{RQ1:} How do the embeddings generated by \model perform on generative tasks compared to multimodal foundation models like CLIP?
% \end{itemize}


\vpara{Results.}
Table~\ref{tab:gen} shows that under different input types and different neighbor encoding strategies, the embeddings generated by UniGraph2 bring significant improvements compared to MMGL's default CLIP embeddings. 
We also observe that UniGraph2's embeddings are more robust to different neighbor encoding strategies compared to CLIP and do not rely on a specific strategy.



\begin{table}[t]%\small
\centering
\renewcommand\tabcolsep{1.6pt}
\caption{\textbf{Ablation studies on \model key components.}}
\vskip -0.1in
\label{tab:kc}
\begin{tabular}{lcccc}
\toprule[1.1pt]
    & Products & Amazon-Cloth & Goodreads-NC  & WN18RR \\
\midrule
    \model & \textbf{82.79{\tiny$\pm$0.02}} & \textbf{24.64{\tiny$\pm$0.09}} & \textbf{81.15{\tiny$\pm$0.12}} & \textbf{85.47{\tiny$\pm$0.11}}\\
    w/o MoE & 81.01{\tiny$\pm$0.10} & 21.33{\tiny$\pm$0.04} & 80.10{\tiny$\pm$0.04} & 83.99{\tiny$\pm$0.21}\\
    w/o feat loss& 69.12{\tiny$\pm$0.09} & 18.43{\tiny$\pm$0.24} & 68.12{\tiny$\pm$0.01} & 74.11{\tiny$\pm$0.03}\\
    w/o SPD loss& 82.42{\tiny$\pm$0.11} & 23.39{\tiny$\pm$0.05} & 80.24{\tiny$\pm$0.02} & 85.24{\tiny$\pm$0.11}\\
\bottomrule[1.1pt]
\end{tabular}
\vspace{-3.3mm}
\end{table}

\begin{table}[t]%\small
\centering
\renewcommand\tabcolsep{2.4pt}
\caption{\textbf{Ablation studies on Modality-Specific Encoders.}}
\vskip -0.1in
\label{tab:enc}
\begin{tabular}{lcccc}
\toprule[1.1pt]
    & Products & Amazon-Cloth & Goodreads-NC  & WN18RR \\
\midrule
    CLIP & 82.79{\tiny$\pm$0.02} & 24.64{\tiny$\pm$0.09} & 81.15{\tiny$\pm$0.12} & \textbf{85.47{\tiny$\pm$0.11}}\\
    ImageBind & 82.32{\tiny$\pm$0.05} & \textbf{25.01{\tiny$\pm$0.11}} & 80.33{\tiny$\pm$0.22} & 84.29{\tiny$\pm$0.07}\\
    T5+ViT& \textbf{82.99{\tiny$\pm$0.04}} & 24.38{\tiny$\pm$0.28} & \textbf{81.28{\tiny$\pm$0.11}} & 84.16{\tiny$\pm$0.04}\\
\bottomrule[1.1pt]
\end{tabular}
\vspace{-4.8mm}
\end{table}

\subsection{Model Analysis}
We select four datasets from different domains to conduct more in-depth studies. We adopt self-supervised representation learning for evaluation.

\vpara{Ablation on Key Components.}
Table~\ref{tab:kc} shows the performance of the \model framework after removing some key designs. "W/o MoE" represents that we use simple MLP instead MoE to align node features. 
"W/o feat loss" represents that we only use the SPD loss for pre-training, while "w/o SPD loss" refers to the opposite.
The overall results confirm that all key designs contribute positively to the performance of \model.

\vpara{Ablation on Modality-Specific Encoders}
In Table~\ref{tab:enc}, we study the influence of different Modality-Specific Encoders on the performance of encoding raw multimodal data. CLIP and ImageBind are feature encoders that map features from various modalities to a shared embedding space, whereas T5+ViT employs SOTA embedding methods for each modality independently, without specific alignment. The results show that all methods achieve comparable performance, indicating that \model effectively aligns features regardless of whether they have been pre-aligned or not.

\begin{table}[t] \scriptsize
\centering
\renewcommand\tabcolsep{3.5pt}
\caption{\textbf{Comparison of GPU hours and performance on ogbn-Arxiv and ogbn-Papers100M.}}
\vskip -0.1in
\label{tab:ccp}
\begin{tabular}{ccccc}
\toprule[1.1pt]
Method & Pre-training & Downstream Training & Downstream Inference & Test Accuracy \\
\midrule
\multicolumn{5}{l}{\textbf{ogbn-Arxiv (169,343 nodes)}} \\ 
% \multirow{3}{*}{\shortstack{ogbn-Arxiv \\ (169,343 nodes)}} 
  GAT        & -    & 0.39 h & 5.5 mins  & 70.89 $\pm$ 0.43 \\
  GraphMAE2  & -    & 5.1 h     & 5.4 mins  & 70.46 $\pm$ 0.07 \\
  UniGraph   & 28.1 h & -      & 9.8 mins & 72.15 $\pm$ 0.18 \\
  UniGraph2  & 5.2 h & - & 5.7 mins &    \textbf{72.56 $\pm$ 0.15}  \\
\midrule
\multicolumn{5}{l}{\textbf{ogbn-Papers100M (111,059,956 nodes)}} \\
  GAT        & -    & 6.8 h     & 23.1 mins & 65.98 $\pm$ 0.23 \\
  GraphMAE2  & -    & 23.2 h    & 23.0 mins & 61.97 $\pm$ 0.24 \\
  UniGraph   & 28.1 h & -      & 40.1 mins & 67.89 $\pm$ 0.21 \\
  UniGraph2 & 5.2 h & - & 24.8 mins &  \textbf{67.95 $\pm$ 0.11} \\
\bottomrule[1.1pt]
\end{tabular}
\vspace{-4.5mm}
\end{table}

\vpara{Efficiency Analysis.}
\model, designed as a foundation model, incurs significant computational costs primarily during the pre-training phase. 
However, it offers the advantage of applicability to new datasets in the inference phase without requiring retraining. 
We compare of the training and inference costs of our model with other models. GAT~\cite{velivckovic2018graph} is a supervised trained GNN. 
GraphMAE2~\cite{hou2023graphmae2} is a self-supervised learning method with GAT as the backbone network. 
UniGraph~\cite{he2024unigraphlearningunifiedcrossdomain} is a graph foundation model for TAGs.
We select ogbn-Arxiv and ogbn-Papers100M, two datasets of different scales for experiments. 
From the results in the Table~\ref{tab:ccp}, we observe that although UniGraph2 has a long pre-training time, its inference time on downstream datasets is comparable or shorter than the combined training and inference time of GNN-based methods. This advantage further increases with the size and potential quantity of downstream datasets.
% The same conclusion also applies to space complexity. Although LM has a larger number of parameters, since we only need to perform inference on the downstream dataset, we avoid the additional space occupation in the backward propagation during training. 
\section{Conclusion}\label{sec:conclusion}

In this study, we propose M2-omni, a highly competitive omni-MLLM model to GPT-4o, characterized by its comprehensive modality and task support, as well as its exceptional performance. M2-omni demonstrates competitive performance across a diverse range of tasks, including image understanding, video understanding, interleaved image-text understanding, audio understanding and generation, as well as free-form image generation. We employ a multi-stage training approach to train M2-omni, which enables progressive modality alignment. To address the challenge of maintaining consistent performance across all modalities, we propose a step-wise balance strategy for pretraining and a dynamically adaptive balance strategy for instruction tuning, which can effectively mitigate the impact of significant variations in data volume and convergence rates across heterogeneous multimodal tasks. We publicly release M2-omni, along with its comprehensive training details, including data configurations and training procedures, to facilitate future research in this domain.


\balance
% \clearpage

\bibliographystyle{ACM-Reference-Format}
\balance
\bibliography{reference}
%\balance
% \clearpage

\cleardoublepage
\section{Appendix}

\subsection{Details of Evaluation Settings}\label{asec:eval-setting}

\subsubsection{Benchmarks}\label{asec:eval-setting-bench}

Since PLPHP maintains the computational integrity of the LVLMs' Prefilling Stage, its efficiency advantage is primarily reflected in the low decoding latency during the subsequent Decoding Stage. Therefore, we mainly choose benchmarks composed of open-ended VQA and image captioning tasks. The benchmarks we select encompasses both multi-image task benchmarks and single-image task benchmarks.

$\bullet$ \textbf{Multi-Image benchmarks}: The LLaVA-Interleave Bench is a comprehensive benchmark dataset designed to evaluate the performance of LVLMs in multi-image scenarios. It consists of 13 challenging tasks with a total of 17,000 instances. We curated four subsets consisting of open-ended VQA tasks from LLaVA-NeXT-Interleave-Bench: Spot-the-Diff, Image-Edit, Visual-Story-Telling, and Multi-View.

$\bullet$ \textbf{Single-Image benchmarks}: The Flickr30k dataset is a widely used benchmark in the field of image captioning and visual understanding. It consists of 31,783 images collected from the Flickr platform, each paired with five human-annotated captions. The COCO2017 Caption subset contains more than 45,000 images, each annotated with five captions written by human annotators, describing the visual content of the images in detail, including objects, their attributes, and the relationships between them. DetailCaps4870 provides more fine-grained and specific image content descriptions than standard captioning datasets, which is more useful for efficiency analysis. 

\subsubsection{Baselines}\label{asec:eval-setting-baseline}

We select FastV and VTW as our baselines in our experiments. Notably, FastV offers two versions of implementation: one that supports KV cache and one that does not. Since the non-KV-cache implementation introduces substantial computational overhead, we use the version that supports KV cache to ensure a fair comparison. For both of the baselines, we refer to the official open source code \footnote{\url{https://github.com/pkunlp-icler/FastV}} \footnote{\url{https://github.com/lzhxmu/VTW}} and implement them on the models we evaluate.

\subsubsection{Models}\label{asec:eval-setting-impl}

For Qwen2-VL, we set \texttt{max\_pixels} to $1280 \times 28 \times 28$ and \texttt{min\_pixels} to $256 \times 28 \times 28$ according to the official recommendation. The Mantis model that we choose is Mantis-8B-SigLIP-LLaMA3. For LLaVA-OneVision and Mantis, we use the official original versions \footnote{\url{https://huggingface.co/lmms-lab/llava-onevision-qwen2-7b-ov}} \footnote{\url{https://huggingface.co/TIGER-Lab/Mantis-8B-siglip-llama3}}, while using the versions provided by the transformers library \cite{wolf-etal-2020-transformers} for all other models.

\subsection{Case Study}

To showcase the effectiveness of our proposed method, we present a series of case studies in the form of multimodal chatbots, as shown in Figure \ref{fig:case-studies}.

\begin{figure*}[ht]
	\centering
	\subfloat[]{
		\includegraphics[width=0.48\textwidth]{figs/appendix-case3.png}}
        \subfloat[]{
		\includegraphics[width=0.48\textwidth]{figs/appendix-case4.png}}
        \\ \quad \\ \quad \\
	\subfloat[]{
		\includegraphics[width=0.48\textwidth]{figs/appendix-case1.png}}
        \subfloat[]{
		\includegraphics[width=0.48\textwidth]{figs/appendix-case2.png}}
 %        \\
	% \subfloat[]{
	% 	\includegraphics[width=0.9\textwidth]{figs/appendix-case5.png}}
	\caption{\textbf{Multimodal Chatbots with different pruning methods.}}
		\label{fig:case-studies}
\end{figure*}

\end{document}
\endinput
%%
%% End of file `sample-sigconf.tex'.

% \bibliographystyle{ACM-Reference-Format}
% \bibliography{reference}


% \clearpage

\newpage
\appendix
% \onecolumn

% \section{Pseudo-code for \Ours}

\section{LLM inference strategy and IR pipelines}

\begin{table}[h]
\caption{Correspondence between LLM inference and IR pipelines.}
  \label{tb:llm-retriever-reranker}
  \centering
%   \small
  \scalebox{0.8}{\begin{tabular}{lccc}
    \toprule
    Method & Retriever & Reranker & Pipeline       \\
    \midrule
    Greedy decoding     & LLM &  $\emptyset$ & Retriever-only  \\
    \midrule
    Best-of-N \citep{stiennon2020learning} & LLM & Reward model & Retriever-reranker  \\
    \midrule
    Majority voting  \citep{wang2022self}  & LLM & Majority & Retriever-reranker  \\
    \midrule
    Iterative refinement \citep{madaan2024self} & LLM & $\emptyset$ & Iterative retrieval  w. query rewriting \\
    \bottomrule
  \end{tabular}}
\end{table}


\section{How can SFT and preference optimization help the LLM from an IR perspective?}\label{apx:sft-rlhf-empirical}


We assess how well LLMs perform at two tasks: fine-grained reranking (using greedy decoding accuracy) and coarse-grained retrieval (using Recall@$N$).  
We focus on how SFT and DPO, affect these abilities.  
Using the Mistral-7b model, we evaluate on the GSM8k and MATH datasets with two approaches: SFT-only, and SFT followed by DPO (SFT $\rightarrow$ DPO).

In the SFT phase, the model is trained directly on correct answers. 
For DPO, we generate 20 responses per prompt and created preference pairs by randomly selecting one correct and one incorrect response.  
We use hyperparameter tuning and early stopping to find the best model checkpoints (see Appendix \ref{apx:sec:sft-rlhf} for details).


\begin{table}[h]
\caption{Retrieval (Recall@N) and reranking (greedy accuracy) metrics across dataset and training strategies, with Mistral-7b as the LLM. 0.7 is used as the temperature. Recall@N can also be denoted as pass@N.}\label{tb:sft-rlhf-result}
\vskip 1em
\centering
\small
\begin{tabular}{llcccc}
    \toprule
     & Metric & \textbf{init model} & \textbf{SFT} & \textbf{SFT $\rightarrow$ DPO} \\
    \midrule
    \multirow{4}{*}{\rotatebox{90}{GSM8K}} 
    & Greedy Acc & 0.4663 & 0.7680 & 0.7991  \\
    & Recall@20 & 0.8347 & 0.9462 & 0.9545  \\
    & Recall@50 & 0.9090 & 0.9629 & 0.9727  \\
    & Recall@100 & 0.9477 & 0.9735 & 0.9826   \\
    \midrule
    \multirow{4}{*}{\rotatebox{90}{Math}} 
    & Greedy Acc & 0.1004 & 0.2334 & 0.2502 \\
    & Recall@20 & 0.2600 & 0.5340 & 0.5416  \\
    & Recall@50 & 0.3354 & 0.6190 & 0.6258  \\
    & Recall@100 & 0.4036 & 0.6780 & 0.6846  \\
    \bottomrule
\end{tabular}
\end{table}

The results are shown in Table \ref{tb:sft-rlhf-result}.  
We observe that both SFT and DPO improve both retrieval and reranking, with SFT being more effective. Adding DPO after SFT further improves performance on both tasks.  
This is consistent with information retrieval principles that both direct retriever optimization and reranker-retrieval distillation can enhance the retriever performance, while the latter on top of the former can further improve the performance. Further discussions can be found in Appendices \ref{apx:discuss1} and \ref{apx:discuss2}.


\section{Discussion on the connection and difference between SFT and direct retriever optimization}\label{apx:discuss1}

As discussed in Section \ref{sec:llm-tuning-retriever}, the direct retriever optimization goal with InfoNCE is shown as:
\begin{gather*}
    \max \log P(d_{\text{gold}}|q) = \max \log \frac{\text{Enc}_d(d_{\text{gold}}) \cdot\text{Enc}_q(q)}{\sum^{|C|}_{j=1} \text{Enc}_d(d_j) \cdot\text{Enc}_q(q)},
\end{gather*}
while the SFT optimization goal is shown as:
\begin{gather}
    \max \log P(y_{\text{gold}}|x) = \max \log \prod^{|y_{\text{gold}}|}_i P(y_{\text{gold}}(i)|z_i) 
    = \max \sum^{|y_{\text{gold}}|}_i \log \frac{\text{Emb}(y_{\text{gold}}(i)) \cdot\text{LLM}(z_i)}{\sum^{|V|}_{j=1} \text{Emb}(v_j) \cdot\text{LLM}(z_i)}. \label{apx:eq:sft}
\end{gather}

As a result, the SFT objective can be seen as a summation of multiple retrieval optimization objectives, where $\text{LLM}(\cdot)$ and word embedding $\text{Emb}(\cdot)$ are query encoder and passage encoder respectively.

However, for direct retriever optimization with InfoNCE, $\text{Enc}_d(\cdot)$ is usually a large-scale pretrained language model which is computationally expensive on both time and memory.
In this case, it is unrealistic to calculate the $\text{Enc}_d(d_j)$ for all $d_j\in C$, when $C$ is large, because of the time constrain and GPU memory constrain.
As a result, a widely-adopted technique is to adopt ``in-batch negatives'' with ``hard negatives'' to estimate the $\log P(d_{\text{gold}}|q)$ function:
\begin{gather*}
    \max \log P(d_{\text{gold}}|q) = \max \log \frac{\text{Enc}_d(d_{\text{gold}}) \cdot\text{Enc}_q(q)}{\sum^{|C|}_{j=1} \text{Enc}_d(d_j) \cdot\text{Enc}_q(q)} \\
    \sim \max \log \frac{\text{Enc}_d(d_{\text{gold}}) \cdot\text{Enc}_q(q)}{\sum^{|B|}_{i=1} \text{Enc}_d(d_i) \cdot\text{Enc}_q(q) + \sum^{|H|}_{j=1} \text{Enc}_d(d_j) \cdot\text{Enc}_q(q)},
\end{gather*}
where $B$ is the in-batch negative set and $H$ is the hard negative set.
Note that $B\bigcup H \subset C$.
This objective is more efficient to optimize but is not the original optimization goal. As a result, the learned model after direct retriever optimization is not optimal.
It is also found that the hard negatives $H$ is the key to estimate the original optimization goal \citep{zhan2021optimizing}.
Thus, reranker-retriever distillation can further improve the retriever by introducing more hard negatives.

On the other hand, LLM optimization, as shown in Eq. (\ref{apx:eq:sft}), can be seen as a summation of multiple retrieval optimization function.
In each retrieval step, the passage can be seen as a token and the corpus is the vocabulary space $V$.
Given that the passage encoder $\text{Emb}(\cdot)$ (word embedding) here is cheap to compute and the vocabulary space $V$ ($<$100k) is usually not as large as $C$ ($>$1M) in IR, the objective in Eq. (\ref{apx:eq:sft}) can be directly optimized without any estimation.
In this case, the LLM as a retriever is more sufficiently trained compared with the retriever training in IR.


\section{Discussion on the connection and difference between preference optimization and reranker-retriever distillation}\label{apx:discuss2}

As discussed in Section \ref{sec:llm-tuning-retriever}, preference optimization with an online reward model $f_{\text{reward-model}}(\cdot) \overset{r}{\rightarrow} \text{data} \overset{g(\cdot)}{\rightarrow}  f_{\text{LLM}}(\cdot)$ can be seen as a reranker to retriever distillation process $f_{\text{rerank}}(\cdot) \overset{r}{\rightarrow} \text{data}\overset{g(\cdot)}{\rightarrow}   f_{\text{retrieval}}(\cdot)$, where the reward model is the reranker (\textit{i.e.}, cross-encoder) and the LLM is the retriever (\textit{i.e.}, bi-encoder).

However, there are two slight differences here:
\begin{itemize}[leftmargin=*]
\item The LLM after SFT is more sufficiently trained compared to a retriever after direct optimization. As discussed in Appendix \ref{apx:discuss1}, the SFT optimization function is not an estimated retriever optimization goal compared with the direct retrieval optimization. As a result, the LLM after SFT is suffienctly trained. In this case, if the reward model (reranker) cannot provide information other than that already in the SFT set (\textit{e.g.}, using the SFT prompts), this step may not contribute to significant LLM capability improvement.
\item The reward model may introduce auxiliary information than the reranker in IR. For a reranker in IR, it captures a same semantic with the retriever: semantic similarity between the query and the passage. However, in LLM post-training, the goal and data in SFT and preference optimization can be different. For example, the SFT phase could have query/response pairs which enable basic chat-based retrieval capability for the LLM. While the reward model may contain some style preference information or safety information which do not exist in SFT data. In this case, the preference optimization which is the reranker to retriever distillation step could also contribution to performance improvement.
\end{itemize}


\section{Evaluate LLMs as retrievers}\label{apx:llm-as-retriever}

In addition to Mathstral-7b-it on GSM8K in Figure \ref{fig:mathstral-gsm8k-infer}, we conduct extensive experiments to both Mistral-7b-it and Mathstral-7b-it on GSM8K and MATH. The results are shown in Figure \ref{apx:fig:empirical-llm-retriever}.
We have similar findings as in Figure \ref{fig:mathstral-gsm8k-infer} that:
(1) As $N$ increases, Recall@$N$ improves significantly, indicating that retrieving a larger number of documents increases the likelihood of including a correct one within the set.
(2) For smaller values of $N$ (e.g., $N=1$), lower temperatures yield higher Recall@$N$. This is because lower temperatures reduce response randomness, favoring the selection of the most relevant result.
(3) Conversely, for larger $N$ (e.g., $N>10$), higher temperatures enhance Recall@$N$. Increased temperature promotes greater response diversity, which, when combined with a larger retrieval set, improves the chances of capturing the correct answer within the results.

\begin{figure*}[h]
    \centering
    \subfigure[Mistral-7b-it on GSM8k]{\includegraphics[width=0.45\textwidth]{figure/LLM_alignment_gsm8k_mathstral7b_infer.pdf}}
    \subfigure[Mistral-7b-it on GSM8k]{\includegraphics[width=0.45\textwidth]{figure/LLM_alignment_gsm8k_mistral7b_infer.pdf}}
    \subfigure[Mathstral-7b-it on MATH]{\includegraphics[width=0.45\textwidth]{figure/LLM_alignment_math_mathstral7b_infer.pdf}} 
    \subfigure[Mistral-7b-it on MATH]{\includegraphics[width=0.45\textwidth]{figure/LLM_alignment_math_mistral7b_infer.pdf}}
    % \vspace{-0.1in}
    \vskip -1em
    \caption{Evaluate the LLM as a retriever with Recall@N (Pass@N). As the number (N) of retrieved responses increases, the retrieval recall increases. The higher the temperature is, the broader spectrum the retrieved responses are, and thus the higher the recall is.}\label{apx:fig:empirical-llm-retriever}
\end{figure*}


% \subsection{How SFT and RLHF benefit the LLM retriever?}\label{apx:sft-rlhf}

% In addition to the experiments with Gemma-1-7b-it in Table \ref{tb:sft-rlhf-result}, we also conduct experiments to study the effect of SFT and DPO on Deepseek-math-7b-base model \citep{shao2024deepseekmath}.
% The results on MATH dataset are shown in Table \ref{apx:tb:sft-rlhf-result}, where we have similar discovery with that in Table \ref{tb:sft-rlhf-result}:
% (1) Both SFT and DPO can improve the retrieval capability of the LLM, while SFT is more effective.
% (2) On top of SFT, DPO can slightly improve the reranking capability (greedy accuracy) but not the general retrieval capability.

% \begin{table}[h]
% \caption{Retrieval (Recall@N) and reranking (greedy accuracy) metrics across dataset and training strategies. LLM: Deepseek-math-7b. Temperature: 0.7. Recall@N can also be denoted as pass@N.}\label{apx:tb:sft-rlhf-result}
% \vskip 1em
% \centering
% \scalebox{0.8}{
% \begin{tabular}{llcccc}
%     \toprule
%      & Metric & \textbf{init model} & \textbf{DPO} & \textbf{SFT} & \textbf{SFT $\rightarrow$ DPO} \\
%     \midrule
%     \multirow{4}{*}{\rotatebox{90}{Math}} 
%     & Greedy Acc & 0.0972 & 0.1164 & 0.3078 & 0.312 \\
%     & Recall@20 & 0.4914 & 0.5136 & 0.6524 & 0.6558 \\
%     & Recall@50 & 0.6058 & 0.6278 & 0.7332 & 0.736 \\
%     & Recall@100 & 0.6728 & 0.6976 & 0.7844 & 0.7828 \\
%     \bottomrule
% \end{tabular}}
% \end{table}



\section{\Ours retriever optimization objective}\label{apx:proofs}

We provide the proof for different variants of \Ours's objective functions.

\subsection{Contrastive ranking}\label{apx:proof:contrastive}

\begin{theorem}
Let \( x \) be a prompt and \( (y_w, y^{(1)}_l, ..., y^{(m)}_l)  \) be the responses for \( x \) under the contrastive assumption (Eq.(\ref{eq:contrastive-assumption})).
Then the objective function to learn the LLM \( \pi_\theta \):
\end{theorem}

\begin{equation}
    \begin{aligned}
    \mathcal{L}_{\text{con}} = -\mathbb{E} & \biggl[
    \log \frac{\exp\bigl(\gamma(y_w \mid x)\bigr)}{
        \exp\bigl(\gamma(y_w \mid x)\bigr) + \sum_{i=1}^m \exp\bigl(\gamma(y_l^{(i)} \mid x)\bigr)}
    \biggr], \\
    \text{where } &\quad \gamma(y \mid x) = \beta \log \frac{\pi_\theta(y \mid x)}{\pi_{\mathrm{ref}}(y \mid x)}.
\end{aligned}\label{eq:contrastive}
\end{equation}

\textit{Proof.}
From \citep{rafailov2024direct}, we know that
\begin{gather}
    r(x, y) = \beta \text{log} \frac{\pi_{\text{llm}}(y|x)}{\pi_{\text{ref}}(y|x)} + \beta \text{log} Z,
\end{gather}
where $Z = \sum_{y'} \pi_{\text{ref}}(y'|x) \text{exp}(\frac{1}{\beta} r(x, y'))$.

Then,
\begin{equation}\label{eq:1-n}
\begin{aligned}
\mathbb{P}\text{r}(y_w & \succeq y^{(1)}_l, ..., y_w \succeq y^{(m)}_l) 
= \text{softmax}(r(x, y_w)) \\
&= \frac{\text{exp}(r(x,y_w))}{\text{exp}(r(x,y_w)) + \sum^m_{i=1}\text{exp}(r(x,y^{(i)}_l))} \\
&= \frac{1}{1 + \sum^m_{i=1}\text{exp}(r(x,y^{(i)}_l)-r(x,y_w))} \\
&= \frac{1}{1 + \sum^m_{i=1}\text{exp}(\gamma(y^{(i)}_l \mid x) + \beta \text{log} Z - \gamma(y_w \mid x) - \beta \text{log} Z)} \\
&= \frac{1}{1 + \sum^m_{i=1}\text{exp}(\gamma(y^{(i)}_l \mid x) - \gamma(y_w \mid x))} \\
&= \frac{\exp\bigl(\gamma(y_w \mid x)\bigr)}{
        \exp\bigl(\gamma(y_w \mid x)\bigr) + \sum_{i=1}^m \exp\bigl(\gamma(y_l^{(i)} \mid x)\bigr)}
\end{aligned}
\end{equation}

We can learn $\pi_\theta$ by maximizing the logarithm-likelihood: 
\begin{gather}
\max \log \mathbb{P}\text{r}(y_w \succeq y^{(1)}_l, \dots, y_w \succeq y^{(m)}_l) \Leftrightarrow 
\min - \log \mathbb{P}\text{r}(y_w \succeq y^{(1)}_l, \dots, y_w \succeq y^{(m)}_l) = \mathcal{L}, \\
 \therefore \mathcal{L}_{\text{con}} = -\mathbb{E} \biggl[
    \log \frac{\exp\bigl(\gamma(y_w \mid x)\bigr)}{
        \exp\bigl(\gamma(y_w \mid x)\bigr) + \sum_{i=1}^m \exp\bigl(\gamma(y_l^{(i)} \mid x)\bigr)}
    \biggr], \\
\text{where} \quad \gamma(y \mid x) = \beta \log \frac{\pi_\theta(y \mid x)}{\pi_{\mathrm{ref}}(y \mid x)}.
\end{gather}



\subsection{LambdaRank ranking}\label{apx:proof:lambdarank}

\begin{theorem}
Let \( x \) be a prompt and \( (y_1, ..., y_m)  \) be the responses for \( x \) under the LambdaRank assumption (Eq.(\ref{eq:lambdarank-assumption})).
Then the objective function to learn the LLM \( \pi_\theta \):
\end{theorem}

% \begin{gather}
%     \mathcal{L}_{\text{lamb}}=-\mathbb{E}\;\biggl[ \sum_{1<i<j<m}
%   w_{ij}\log \sigma\Bigl(
%      \gamma(y_i \mid x)-
%      \gamma(y_j \mid x)
%   \Bigr)
% \biggr]
% \end{gather}
\begin{gather}
    \mathcal{L}_{\text{lamb}}=-\mathbb{E}\;\biggl[ \sum_{1<i<j<m}
   \log \sigma\Bigl(
     \gamma(y_i \mid x)-
     \gamma(y_j \mid x)
   \Bigr)
\biggr].
\end{gather}
% where $w_{ij}$ is an adjustable weight.

\textit{Proof.}
\begin{equation}
\begin{aligned}
\mathbb{P}\text{r}(y_1 & \succeq ... \succeq y_m)
= \prod_{1<i<j<m} \sigma(r(x,y_i) - r(x,y_j)) \\
&= \prod_{1<i<j<m} \sigma(\gamma(x,y_i) + \beta \text{log} Z - \gamma(x,y_j) - \beta \text{log} Z)  \\
&= \prod_{1<i<j<m} \sigma(\gamma(y_i \mid x)-
     \gamma(y_j \mid x)).
\end{aligned}
\end{equation}

We can learn $\pi_\theta$ by maximizing the logarithm-likelihood: 
\begin{gather}
\max \log \mathbb{P}\text{r}(y_w \succeq y^{(1)}_l, \dots, y_w \succeq y^{(m)}_l) \Leftrightarrow 
\min - \log \mathbb{P}\text{r}(y_w \succeq y^{(1)}_l, \dots, y_w \succeq y^{(m)}_l) = \mathcal{L}, \\
 \therefore \mathcal{L}_{\text{lamb}}=-\mathbb{E}\;\biggl[ \sum_{1<i<j<m}
   \log \sigma\Bigl(
     \gamma(y_i \mid x)-
     \gamma(y_j \mid x)
   \Bigr)
\biggr], \\
\text{where} \quad \gamma(y \mid x) = \beta \log \frac{\pi_\theta(y \mid x)}{\pi_{\mathrm{ref}}(y \mid x)}.
\end{gather}
% $w_{ij}$ can be added to control the weight of each pair in the candidate list.


\subsection{ListMLE ranking}\label{apx:proof:listmle}

\begin{theorem}
Let \( x \) be a prompt and \( (y_1, ..., y_m)  \) be the responses for \( x \) under the ListMLE assumption (Eq.(\ref{eq:listmle-assumption})).
Then the objective function to learn the LLM \( \pi_\theta \):
\end{theorem}

\begin{equation}
\begin{aligned}
    \mathcal{L}_{\text{lmle}} &= -\mathbb{E} \biggl[
    \sum^m_{i=1} \log \frac{\exp\bigl(\gamma(y_i \mid x)\bigr)}{
        \exp\bigl(\gamma(y_i \mid x)\bigr) + \sum_{j=i}^m \exp\bigl(\gamma(y_j \mid x)\bigr)}
    \biggr].
\end{aligned}
\end{equation}

\textit{Proof.}
From Eq.(\ref{eq:1-n}),
\begin{gather}
\begin{aligned}
    \mathbb{P}\text{r}(y_1 & \succeq ... \succeq y_m) = \prod^m_{i=1} \mathbb{P}\text{r}(y_i \succeq y_{i+1}, ..., y_i \succeq y_m)  \\
    & = \prod^m_{i=1} \frac{\text{exp}(\gamma(y_i \mid x))}{\text{exp}(\gamma(y_i \mid x)) + \sum^m_{j=i+1}\text{exp}(\gamma(y_j \mid x))}
\end{aligned}.
\end{gather}
% The derivation above uses the result from Eq.(\ref{eq:1-n}).

We can learn $\pi_\theta$ by maximizing the logarithm-likelihood: 
\begin{gather}
\max \log \mathbb{P}\text{r}(y_w \succeq y^{(1)}_l, \dots, y_w \succeq y^{(m)}_l) \Leftrightarrow 
\min - \log \mathbb{P}\text{r}(y_w \succeq y^{(1)}_l, \dots, y_w \succeq y^{(m)}_l) = \mathcal{L}, \\
 \therefore \mathcal{L}_{\text{lmle}} = -\mathbb{E} \biggl[
    \sum^m_{i=1} \log \frac{\exp\bigl(\gamma(y_i \mid x)\bigr)}{
        \exp\bigl(\gamma(y_i \mid x)\bigr) + \sum_{j=i}^m \exp\bigl(\gamma(y_j \mid x)\bigr)}
    \biggr], \\
\text{where} \quad \gamma(y \mid x) = \beta \log \frac{\pi_\theta(y \mid x)}{\pi_{\mathrm{ref}}(y \mid x)}.
\end{gather}


\section{Baselines}\label{apx:sec:baselines}

We conduct detailed illustrations on the baselines compared with \Ours in Section \ref{sec:main-result} below.

\begin{itemize}[leftmargin=*]
  \item RRHF \citep{yuan2023rrhf} scores responses via a logarithm of conditional probabilities and learns to align these probabilities with human preferences through ranking loss.
  \item SLiC-HF \citep{zhao2023slic} proposes a sequence likelihood calibration method which can learn from human preference data.
  \item DPO \citep{guo2024direct} simplifies the PPO \citep{ouyang2022training} algorithms into an offline direct optimization objective with the pairwise Bradley-Terry assumption.
  \item IPO \citep{azar2024general} theoretically grounds pairwise assumption in DPO into a pointwise reward.
  \item CPO \citep{xu2024contrastive} adds a reward objective with sequence likelihood along with the SFT objective.
  \item KTO \citep{ethayarajh2024kto} adopts the Kahneman-Tversky model and proposes a method which directly maximizes the utility of generation instead of the likelihood of the preferences.
  \item RDPO \citep{park2024disentangling} modifies DPO by including an additional regularization term to disentangle the influence of length.
  \item SimPO \citep{meng2024simpo} further simplifies the DPO objective by using the average log probability of a sequence as the implicit reward and adding a target reward margin to the Bradley-Terry objective.
  \item Iterative DPO \citep{xiong2024iterative} identifies the challenge of offline preference optimization and proposes an iterative learning framework.
\end{itemize}


\section{Experiment settings}\label{apx:sec:main-result-setting}

\subsection{Table \ref{tab:main-performance}}\label{apx:sec:main}

We conduct evaluation on two widely used benchmark: AlpacaEval2 \citep{dubois2024length} and MixEval \citep{ni2024mixeval}.
We consider two base models: Mistral-7b-base and Mistral-7b-it. For Mistral-7b-base, we first conduct supervised finetuning following \citet{meng2024simpo} before the preference optimization.

The performance scores for offline preference optimization baselines are from SimPO \citep{meng2024simpo}.
To have a fair comparison with these baselines, we adopt the same off-the-shelf reward model \citep{jiang2023llm} as in SimPO for the iterative DPO baseline and \Ours.

For the iterative DPO baseline, we generate 2 responses for each prompt, score them with the off-the-shelf reward model and construct the preference pair data to tune the model.

For \Ours (contrastive $\mathcal{L}_{\text{con}}$), we generate 10 responses each iteration and score them with the reward model. The top-1 ranked response and the bottom-3 ranked responses are adopted as the chose response and rejected responses respectively.
Generation temperature is selected as 1 and 0.8 for Mistral-7b-base and Mistral-7b-it respectively (we search it among 0.8, 0.9, 1.0, 1.1, 1.2).

For \Ours (LambdaRank $\mathcal{L}_{\text{lamb}}$), we generate 10 responses each iteration and score them with the reward model. The top-2 ranked response and the bottom-2 ranked responses are adopted as the chose response and rejected responses respectively.
Generation temperature is selected as 1 and 0.8 for Mistral-7b-base and Mistral-7b-it respectively (we search it among 0.8, 0.9, 1.0, 1.1, 1.2).

For \Ours (ListMLE $\mathcal{L}_{\text{lmle}}$), we generate 10 responses each iteration and score them with the reward model. The top-2 ranked response and the bottom-2 ranked responses are adopted as the chose response and rejected responses respectively.
Generation temperature is selected as 1 and 0.8 for Mistral-7b-base and Mistral-7b-it respectively (we search it among 0.8, 0.9, 1.0, 1.1, 1.2).

\Ours can achieve even stronger performance with stronger off-the-shelf reward model \citep{dong2024rlhf}.
% Results with stronger a reward model can be found in Appendix \ref{apx:sec:stronger-rm}.

\subsection{Table \ref{tab:objective}}\label{apx:sec-objective-setting}

We conduct experiments on both Gemma2-2b-it \citep{team2024gemma} and Mistral-7b-it \citep{jiang2023mistral}.
Following \citet{Tunstall_The_Alignment_Handbook} and \citet{dong2024rlhf}, we perform training on UltraFeedback dataset for 3 iterations and show the performance of the final model checkpoint.
We use the pretrained reward model from \citet{dong2024rlhf}.
The learning rate is set as 5e-7 and we train the LLM for 2 epochs per iteration.

For the pairwise objective, we generate 2 responses for each prompt and construct the preference pair data with the reward model.
For the others, we generate 4 responses per prompt and rank them with the reward model.
For the contrastive objective, we construct the 1-vs-N data with the top-1 ranked response and the other responses.
For the listMLE and lambdarank objective, we take the top-2 as positives and the last-2 as the negatives.
Experiments with opensource LLM as the evaluator (\texttt{alpaca\_eval\_llama3\_70b\_fn}) can be found in Table \ref{tab:objective2}.



\begin{table*}[t]
    \centering
    % \renewcommand{\arraystretch}{1.2}
    \caption{Preference optimization objective study on AlpacaEval2 and MixEval. For AlpacaEval2, we report the result with both opensource LLM evaluator \texttt{alpaca\_eval\_llama3\_70b\_fn} and GPT4 evaluator \texttt{alpaca\_eval\_gpt4\_turbo\_fn}. SFT corresponds to the initial chat model.}\label{tab:objective2}
    \small
    \begin{tabular}{llccccccccc}
        \toprule
        & & \multicolumn{2}{c}{AlpacaEval 2 (opensource LLM)} & \multicolumn{2}{c}{AlpacaEval 2 (GPT-4)} & \multicolumn{1}{c}{MixEval} & \multicolumn{1}{c}{MixEval-Hard} \\
         \cmidrule(r){3-4} \cmidrule(r){5-6} \cmidrule(r){7-7} \cmidrule(r){8-8}
        & Method & LC Winrate & Winrate & LC Winrate & Winrate & Score & Score \\
        \midrule
        \multirow{6}{*}{\rotatebox{90}{Gemma2-2b-it}} & SFT & 47.03 & 48.38 & 36.39 & 38.26 & 0.6545 & 0.2980 \\
        \cmidrule{2-8}
        & pairwise & 55.06 & 66.56 & 41.39 & 54.60 & 0.6740 & 0.3375 \\
        & contrastive & 60.44 & 72.35 & 43.41 & 56.83 & 0.6745 & 0.3315 \\
        & ListMLE & 63.05 & 76.09 & 49.77 & 62.05 & 0.6715 & 0.3560 \\
        & LambdaRank & 58.73 & 74.09 & 43.76 & 60.56 & 0.6750 & 0.3560 \\
        \midrule
        \midrule
        \multirow{6}{*}{\rotatebox{90}{Mistral-7b-it}} & SFT & 27.04 & 17.41 & 21.14 & 14.22 & 0.7070 & 0.3610 \\
        \cmidrule{2-8}
        & pairwise & 49.75 & 55.07 & 36.43 & 41.86 & 0.7175 & 0.4105 \\
        & contrastive & 52.03 & 60.15 & 38.44 & 42.61 & 0.7260 & 0.4340 \\
        & ListMLE & 48.84 & 56.73 & 38.02 & 43.03 & 0.7360 & 0.4200 \\
        & LambdaRank & 51.98 & 59.73 & 40.29 & 46.21 & 0.7370 & 0.4400 \\
        \bottomrule
    \end{tabular}
\end{table*}


\subsection{Table \ref{fig:list-study}}\label{apx:sec-list-setting}

We adopt Gemma2-2b-it as the initial model. All the models are trained with iterative DPO for 3 iterations. We use the off-the-shelf reward model \citep{dong2024rlhf}.
We generate 2 responses for each prompt in each iteration.
For ``w. current'', we only use the scored responses in the current iteration for preference optimization data construction.
For ``w. current + prev'', we rank the responses in the current iteration and the previous one iteration, and construct the preference pair data with the top-1 and bottom-1 ranked responses.
For ``w. current + all prev'', we rank all the responses for the prompt in the current and previous iterations and construct the preference pair data.
For ``single temperature'', we only adopt temperature 1 and generate 2 responses for reward model scoring.
For ``diverse temperature'', we generate 2 responses with temperature 1 and 0.5 respective and rank the 4 responses to construct the preference data with the reward model.

\subsection{Table \ref{tb:sft-rlhf-result}}\label{apx:sec:sft-rlhf}

We use mistral-7b-it \citep{jiang2023mistral} as the initial model to alleviate the influence of the math related post-training data of the original model.
% For SFT, we conduct training on the training set of MATH \citep{hendrycks2021measuring} and GSM8K \citep{cobbe2021training} respectively.
For SFT, we conduct training on the meta-math dataset \citep{yu2023metamath}.
For DPO, we use the prompts in the training set of the two dataset and conduct online iterative preference optimization with the binary rule-based reward (measure if the final answer is correct or not with string match). 
The evaluation is performed on the test set of MATH and GSM8K respectively.
% For both SFT and DPO, we conduct careful hyper-parameter search.
For SFT, we follow the same training setting with \citet{yu2023metamath}.
For DPO, we search the learning rate in 1e-7, 2e-7, 5e-7, 2e-8, 5e-8 and train the LLM for 5 iterations with early stop (1 epoch per iteration for MATH and 2 epoch per iteration for GSM8K). The learning rate is set as 1e-7 and we select the checkpoint after the first and fourth iteration for GSM8K and MATH respectively.

\subsection{Figure \ref{fig:merge-study}(a)}\label{apx:sec-hard-neg-setting}

We conduct training with the prompts in the training set of GSM8K and perform evaluation on GSM8K testing set.
We conduct learning rate search and finalize it to be 2e-7.
The learning is performed for 3 iterations.

We make explanations of how we construct the four types of negative settings:
For (1) a random response not related to the given prompt, we select a response for a random prompt in Ultrafeedback.
For (2) a response to a related prompt, we pick up a response for a different prompt in the GSM8K training set.
For (3) an incorrect response to the given prompt with high temperature, we select the temperature to be 1.
For (4) an incorrect response to the given prompt with low temperature, we select the temperature to be 0.7.

\begin{figure}[t]
\centering
\includegraphics[scale=0.4]{figure/LLM_alignment_gemma_temperature_study.pdf}
\vskip -1em
\caption{Training temperature study with $\mathcal{L}_{\text{pair}}$ on Gemma2-2b-it and Alpaca Eval 2. Within a specific range ($>$ 0.9), lower temperature leads to harder negative and benefit the trained LLM. However, temperature lower than this range can cause preferred and rejected responses non-distinguishable and lead to degrade training.}\label{apx:tab:temp-hard}
\end{figure}

\subsection{Figure \ref{fig:merge-study}(b)}\label{apx:sec-hard-neg-setting-temp}

We conduct experiments on both Gemma2-2b-it and Mistral-7B-it models.
For both LLMs, we conduct iterative DPO for 3 iterations and report the performance of the final model.
We perform evaluation on Alpaca Eval2 with \texttt{alpaca\_eval\_llama3\_70b\_fn} as the evaluator.

For temperature study, we find that under a specific temperature threshold, repeatedly generated responses will be large identical for all LLMs and cannot be used to construct preference data, while the threshold varies for different LLMs.
% As a result, we select temperatures above the threshold for robust experiments.
The ``low'' and ``high'' refer to the value of those selected temperatures.
% For Gemma2-2b-it, we use temperature as 0.2, 0.5 and 0.7 to generate the responses, score the responses by the reward model and train the LLM with the newly labeled data.
% For Mistral-7b-it, we set the temperature as 1, 1.1 and 1.2 respectively.
We also conduct experiments on Gemma2-2b-it model and show the results in Figure \ref{apx:tab:temp-hard}.


\subsection{Figure \ref{fig:merge-study}(c)}\label{apx:sec-length-setting}

We adopt Mistral-7b-it as the initial LLM and the contrastive objective (Eq. \ref{eq:contrastive}) in iterative preference optimization.
We generate 4/6/8/10 responses with the LLM and score the responses with the off-the-shelf reward model \citep{dong2024rlhf}.
The top-1 scored response is adopted as the positive response and the other responses are treated as the negative responses to construct the 1-vs-N training data.
The temperature is set as 1 to generate the responses.


% \newpage
% \section{\Ours with a stronger reward model}\label{apx:sec:stronger-rm}

% In Section \ref{sec:lrpo}, we show the results with LLM-Blender \citep{jiang2023llm} as the reward model to have a fair comparison with the baseline methods.
% In this section, we would like to show that \Ours can achieve even stronger performance with stronger off-the-shelf reward model \citep{dong2024rlhf}.
% The results are shown in Table \ref{apx:tab:main-performance}, where we can find that a stronger reward model can further improve the performance of \Ours.


% \begin{table*}[h]
%     \centering
%     \caption{Method evaluation on AlpacaEval 2 and MixEval. LC WR and WR denote length-controlled win rate and win rate respectively. Offline baseline performances on AlpacaEval 2 are from \citept{meng2024simpo} with LLM-Blender reward model \citep{jiang2023llm}.}\label{apx:tab:main-performance}
%     \scalebox{0.78}{
%     \begin{tabular}{lcccccccccc}
%         \toprule
%         Model & \multicolumn{4}{c}{Mistral-Base (7B)} & \multicolumn{4}{c}{Mistral-Instruct (7B)} \\
%         \cmidrule(lr){2-5} \cmidrule(lr){6-9}
%         & \multicolumn{2}{c}{Alpaca Eval 2}  & \multirow{1}{*}{MixEval} & \multirow{1}{*}{MixEval-Hard} & \multicolumn{2}{c}{Alpaca Eval 2}  & \multirow{1}{*}{MixEval} & \multirow{1}{*}{MixEval-Hard} \\
%         \cmidrule(lr){2-3} \cmidrule(lr){4-4} \cmidrule(lr){5-5}  \cmidrule(lr){6-7} \cmidrule(lr){8-8} \cmidrule(lr){9-9}
%         & LC WR & WR  & Score & Score & LC WR  & WR & Score & Score \\
%         \midrule
%         SFT    & 8.4  & 6.2   &  0.602  & 0.279  & 17.1 & 14.7  & 0.707 & 0.361 \\
%         RRHF   & 11.6 & 10.2   &  0.600  & 0.312  & 25.3 & 24.8  &   0.700    & 0.380 \\
%         DPO    & 15.1 & 12.5  &  0.686  &  0.341 & 26.8 & 24.9  & 0.702 & 0.355 \\
%         KTO    & 13.1 & 9.1    & \textbf{0.704}  & 0.351   & 24.5 & 23.6  &   0.692    & 0.358 \\
%         RDPO   & 17.4 & 12.8  & 0.693  & 0.355   & 27.3 & 24.5  &   0.695    & 0.364 \\
%         SimPO  & 21.5 & 20.8  &  0.672  &  0.347 & 32.1 & 34.8  & 0.702  & 0.363 \\
%         Iterative DPO  & 18.9  & 16.7  & 0.660   & 0.341  & 20.4 & 24.84  & 0.719  & 0.389 \\
%         \midrule
%         \multicolumn{9}{c}{Reward model: LLM-Blender \citep{jiang2023llm}}  \\
%         \midrule
%         \Ours (contrastive) & 31.6 & 30.8  &   0.703 & 0.409  & 32.7 & 38.6  &  0.718 & \textbf{0.418} \\
%         \Ours (LambdaRank) &  \textbf{34.9} & \textbf{37.2} & 0.695 &  \textbf{0.452}  & \textbf{32.9} & \textbf{38.9}   & \textbf{0.720} & 0.417  \\
%         \Ours (ListMLE) & 31.1  &  32.1   &  0.669  & 0.390  &  29.7 & 36.2    & 0.709  & 0.397 \\
%         \midrule
%         \multicolumn{9}{c}{Reward model: FsfairX \citep{dong2024rlhf}}  \\
%         \midrule
%         \Ours (contrastive) & \textbf{41.5} & \textbf{42.9} & 0.718 & 0.417    & \textbf{43.0}  & \textbf{53.8} & 0.718 & 0.425   \\
%         \Ours (LambdaRank) & 35.8 & 34.1 & 0.717 & 0.431   & 41.9  & 48.1 & \textbf{0.740} & \textbf{0.440}  \\
%         \Ours (ListMLE) & 36.6 & 37.8 & \textbf{0.730} & \textbf{0.423}   & 39.6  & 48.1 & 0.717 & 0.397   \\
%         \bottomrule
%     \end{tabular}}
% \end{table*}


\end{document}

