% VLDB template version of 2020-08-03 enhances the ACM template, version 1.7.0:
% https://www.acm.org/publications/proceedings-template
% The ACM Latex guide provides further information about the ACM template

\documentclass[sigconf,nonacm,pdfa]{acmart}

\usepackage[normalem]{ulem}
\usepackage[a-2b]{pdfx}
\usepackage{epstopdf}
\usepackage{tikz}
\usepackage{pifont}
\usepackage{epsfig,endnotes}
\usepackage{amsmath}
\usepackage{color}
\usepackage[titletoc,toc,page]{appendix}
\usepackage{textcomp}
\usepackage{multirow}
%\usepackage{array}
\usepackage{graphicx}
%\usepackage{hyperref}
%\usepackage{hyphenat}
\usepackage{fontawesome}
\usepackage{xcolor}
\usepackage{enumitem}
\usepackage{booktabs}
\usepackage{algorithmicx}
\usepackage[ruled,linesnumbered]{algorithm2e}
%\usepackage[linesnumbered,ruled,vlined]{algorithm2e}
\usepackage{algpseudocode}
%\usepackage{algorithm}
%\usepackage{subfloat}
%\usepackage{float}
%\usepackage{subfig}
\usepackage{subfigure}
\usepackage{enumerate}
\usepackage{makecell}
\usepackage{balance}
\usepackage{filecontents}
\usepackage{xspace}
\usepackage{caption}
\usepackage{fancyhdr}
%\usepackage{cleveref}
\usepackage{setspace}
\usepackage{hyperref}
\usepackage{libertine}
\usepackage{listings}
\usepackage{environ}



%\newcommand{\sys}{\texttt{Outback}\xspace}
\newcommand{\sys}{Outback\xspace}

%\newcommand{\red}{\textcolor[rgb]{1,0,0}}
\newcommand{\red}{}
\newcommand{\blue}{\textcolor[rgb]{0,0,1}}
\newcommand{\blux}[1]{\noindent\textcolor{blue}{\sout{#1}}}

\iffalse
\pagestyle{empty}
\settopmatter{printacmref=false}
\settopmatter{printfolios=true}
\renewcommand\footnotetextcopyrightpermission[1]{}
\fi

\newcommand{\todo}[1]{{\it \color{red}\{TODO: #1\}}}
\newcommand{\minghao}[1]
%{\noindent{\textcolor{blue}{\bf \fbox{MX} {#1}}}}
{\noindent{}}

\newif\ifshowheiner
%\showheinertrue
\newcommand{\heiner}[1]{\ifshowheiner\noindent{\textcolor{teal}{\bf \fbox{HL} {\it#1}}}\fi}

\newif\ifshowliuyi 
%\showliuyitrue
\newcommand{\liuyi}[1]{\ifshowliuyi\noindent{\textcolor{red}{\bf \fbox{YL} {\it#1}}}\fi}


% Define the Consolas font for listings
\iffalse
\lstset{
  basicstyle=\ttfamily,
  columns=fullflexible,
  breaklines=true,
  postbreak=\mbox{\textcolor{red}{$\hookrightarrow$}\space},
}
\fi

%% The following content must be adapted for the final version
% paper-specific
\newcommand\vldbdoi{10.14778/3705829.3705849}
\newcommand\vldbpages{335-348}
% issue-specific
\newcommand\vldbvolume{18}
\newcommand\vldbissue{2}
\newcommand\vldbyear{2024}
% should be fine as it is
\newcommand\vldbauthors{\authors}
\newcommand\vldbtitle{\shorttitle} 
% leave empty if no availability URL should be set
\newcommand\vldbavailabilityurl{https://github.com/yliu634/outback}
% whether page numbers should be shown or not, use 'plain' for review versions, 'empty' for camera ready
\newcommand\vldbpagestyle{empty}

\begin{document}

\title{Outback: Fast and Communication-efficient Index for Key-Value Store on Disaggregated Memory}

%%
%% The "author" command and its associated commands are used to define the authors and their affiliations.
\author{Yi Liu}
\affiliation{%
  \institution{University of California Santa Cruz}
  \streetaddress{1156 High Street}
%  \city{Santa Cruz}
 % \state{CA}
 \country{}
  \postcode{95064}
}
\email{yliu634@ucsc.edu}
\author{Minghao Xie}
%\orcid{0000-0002-1825-0097}
\affiliation{%
  \institution{University of California Santa Cruz}
  \streetaddress{1156 High Street}
%  \city{Santa Cruz}
 % \state{CA}
 \country{}
  \postcode{95064}
}
\email{mhxie@ucsc.edu}
\author{Shouqian Shi}
%\orcid{0000-0002-1825-0097}
\affiliation{%
  \institution{University of California Santa Cruz}
  \streetaddress{1156 High Street}
%  \city{Santa Cruz}
 % \state{CA}
 \country{}
  \postcode{95064}
}
\email{sshi27@ucsc.edu}
\author{Yuanchao Xu}
%\orcid{0000-0002-1825-0097}
\affiliation{%
  \institution{University of California Santa Cruz}
  \streetaddress{1156 High Street}
%  \city{Santa Cruz}
 % \state{CA}
 \country{}
  \postcode{95064}
}
\email{yxu314@ucsc.edu}
\author{Heiner Litz}
%\orcid{0000-0002-1825-0097}
\affiliation{%
  \institution{University of California Santa Cruz}
  \streetaddress{1156 High Street}
%  \city{Santa Cruz}
 % \state{CA}
  \country{}
  \postcode{95064}
}
\email{hlitz@ucsc.edu}
\author{Chen Qian}
%\orcid{0000-0002-1825-0097}
\affiliation{%
  \institution{University of California Santa Cruz}
  \streetaddress{1156 High Street}
 % \city{Santa Cruz}
  %\state{CA}
  \country{}
  \postcode{95064}
}
\email{qian@ucsc.edu}



%%
%% The abstract is a short summary of the work to be presented in the
%% article.
\begin{abstract}
\heiner{Title is a bit long. It should be disaggregated KVS (not the) as there exist more than one KVS implementations. If you say decoupling, you must say decoupling of A and B. There have to be at least 2 things that are decoupled.}
\liuyi{Fixed it, we will formulate a new one after revising paper.}

Disaggregated memory systems achieve resource utilization efficiency and system scalability by distributing computation and memory resources into distinct pools of nodes. RDMA is an attractive solution to support high-throughput communication between different disaggregated resource pools. However, existing RDMA solutions face a dilemma: one-sided RDMA completely bypasses computation at memory nodes, but its communication takes multiple round trips; two-sided RDMA achieves one-round-trip communication but requires non-trivial computation for index lookups at memory nodes, which violates the principle of disaggregated memory. This work presents \sys, a novel indexing solution for key-value stores with a one-round-trip RDMA-based network that does not incur computation-heavy tasks at memory nodes. \sys is the first to utilize dynamic minimal perfect hashing and separates its index into two components: one memory-efficient and compute-heavy component at compute nodes and the other memory-heavy and compute-efficient component at memory nodes. 
%We design detailed to support data operations and system updates of \sys. 
We implement a prototype of \sys and evaluate its performance in a public cloud. The experimental results show that \sys achieves higher throughput than both the state-of-the-art one-sided RDMA and two-sided RDMA-based in-memory KVS by 1.06-5.03$\times$, due to the unique strength of applying a separated perfect hashing index. 
%We believe the decoupled index framework is an ideal solution for disaggregate memory systems. 


\iffalse
\heiner{language is a bit hyperbole (redefine, unprecedented), please tone down.}\liuyi{fix them to ``offer'' and ``enhanced''.}
Disaggregated key-value stores offer distributed systems, separating storage from computation for enhanced flexibility, performance, fault tolerance, and resource efficiency by leveraging fast network technologies like RDMA.
\heiner{the two sentences do not connect, unclear what RDMA has to do with disaggregated KV. you need to first explain challenges that these KV systems face. Then you can say RDMA has been proposed to address these challenges. Then you can say RDMA suffers from x which is what you address. (every sentence has to build on top of the previous)}\liuyi{add RDMA as one of technology for Disaggregated KVS.}
Existing RDMA-based works have often surrounded the perception that two-sided RDMA (RPC) primitives significantly consume CPU resources on the remote server. 
\heiner{try to avoid subjective statements such as "criticism has often surrounded" its a claim that is hard to prove. Rephrase into objective statements, e.g. Existing RDMA-based systems}\liuyi{change to ``Existing RDMA-based works''}
However, our observation reveals that the reason is that there is too much computation imposed on the memory pool. \heiner{again unsubstantiated claim. You do not want to tell the reviwers what they commonly think. Stick with objective facts. e.g asy our experiments show that RDMA cannot solve the problems because of X (two-roundtrips?) and Y(limited memory?)} This finding offers a fresh perspective for reevaluating the bottleneck of the two-sided RDMA schemes with limited CPU resources.\liuyi{change to limitations of two-sided RDMA.}
To address the shortcomings of RDMA, we propose \sys, a novel scheme utilizing Minimal Perfect Hashing (MPH) to separate data lookup computation in a disaggregated KVS. By offloading most of the data lookup computation to compute nodes, we optimize throughput, enabling memory nodes to concentrate on memory access without taking care of extra computation. We also designed decoupled, pre-allocated, and shufflable MPH and partial locking mechanism to efficiently handle \texttt{Insert} requests and MPH resizing.  The extensive experiments demonstrate that \sys outperforms state-of-the-art in-memory KVS by 1.20-5.03$\times$ in the full data read workload on YCSB and real-world datasets.
\heiner{Is the focus on RDMA justified? We don't really leverage any RDMA technology, do we? I think your proposed design rather addresses B-Tree implementations. I would use this outline for the abstract:
1. Motivate KV 2. Describe a sota (b-tree/hash) based implementation and say that "as our experiments show these implementations show limited perf because of CPU overhead". (It is important that you discuss this sota first before RDMA, because it is unclear why RDMA has been proposed in the first place.) 3. RDMA has been proposed to address the compute overheads but it introduces 2 new challenges (2 roudntrips and limited memory). 4. We propose a design that provides both of 2 worlds: Low overhead of RDMA but implemented in the host CPU to address 2 roudntrips and memory limitations. 5. Results of proposed technique}\liuyi{We do use two-sided RDMA, which is RPC to build our work. RPC-Btree and RPC-hash are both we talked in this paper are both built on top of two-sided RDMA mode.}

\todo{need rewrite abstract later}
\fi

\end{abstract}


%% Keywords. The author(s) should pick words that accurately describe
%% the work being presented. Separate the keywords with commas.
%\keywords{Disaggregated memory; Minimal perfect hashing; RDMA}


%%
%% This command processes the author, and affiliation and title
\maketitle


%%% VLDB block start %%%
\pagestyle{\vldbpagestyle}
\begingroup\small\noindent\raggedright\textbf{PVLDB Reference Format:}\\
\vldbauthors. \vldbtitle. PVLDB, \vldbvolume(\vldbissue): \vldbpages, \vldbyear.\\
\href{https://doi.org/\vldbdoi}{doi:\vldbdoi}
\endgroup
\begingroup
\renewcommand\thefootnote{}\footnote{\noindent
This work is licensed under the Creative Commons BY-NC-ND 4.0 International License. Visit \url{https://creativecommons.org/licenses/by-nc-nd/4.0/} to view a copy of this license. For any use beyond those covered by this license, obtain permission by emailing \href{mailto:info@vldb.org}{info@vldb.org}. Copyright is held by the owner/author(s). Publication rights licensed to the VLDB Endowment. \\
\raggedright Proceedings of the VLDB Endowment, Vol. \vldbvolume, No. \vldbissue\ %
ISSN 2150-8097. \\
\href{https://doi.org/\vldbdoi}{doi:\vldbdoi} \\
}\addtocounter{footnote}{-1}\endgroup
%%% VLDB block end %%%

%%% do not modify the following VLDB block %%
%%% VLDB block start %%%
\ifdefempty{\vldbavailabilityurl}{}{
\vspace{.3cm}
\begingroup\small\noindent\raggedright\textbf{PVLDB Artifact Availability:}\\
The source code, data, and/or other artifacts have been made available at \url{\vldbavailabilityurl}.
\endgroup
}
%%% VLDB block end %%%



\section{Introduction}

\begin{figure*}
    \centering
    \includegraphics[width=\textwidth]{figures/Introduction.pdf}
    \caption{Showing the novel problem statement applied to traffic prediction use case. Multiple unstructured observations from the past are used to reconstruct a hidden traffic state from which a full traffic state is forecast with a set of query locations. }
    \label{fig:intro}
\end{figure*}

% Was sagen denn die anderen warum Traffic Prediction gut ist? 
Forecasting the traffic in the near future is an important task for city management.
Data from the near past is used to predict future traffic states with spatio-temporal Graph Neural Networks \cite{bui22}.
Accurate prediction provides the opportunity to optimize traffic flow, reduce traffic jams and increase air quality \cite{Po19}.

% Wieso ist Sparsity in allen Dimensionen wichtig.
While traffic prediction relies on the availability of data from traffic sensors, there exists a plethora of reasons why sensors may stop working temporarily, such as simple errors, energy saving, or overloaded communication systems.
Considering small- or medium-sized cities, the coverage of sensors may be low because the sensors are too expensive or not available.
Also, the sensors are typically static and do not adapt to changes in the traffic flow (e.g. caused by a construction site), which motivates moving sensors that for example could be mounted on cars. 
However, both missing and moving sensors introduce sparsity, since measurements may not be available for all locations at all times.
This sparsity must be explicitly addressed in traffic prediction for a realistic application scenario, which is illustrated in figure \ref{fig:intro}.
From one hour of data on Sunday morning, only few observations of the traffic state are available at each timestep.
The number of observations may differ throughout the observed time and the observation itself can be distributed arbitrarily in the city. 
We assume a relatively low number of sensors to account for resource saving and sensor failure in our proposed framework SUSTeR.
The task is to predict the dense traffic state one timestep after the observations at all possible sensor locations.
We study this problem on the traffic dataset Metr-LA and PEMS-BAY to test our assumption that only a fraction of the sensor values would be enough for good predictions.
By modifying an existing traffic dataset, we are able to compare our results from very sparse observations to the bottom line with all information available.
A successful study will provide insights in how sensors in new cities can be reduced before installing them and further mobile sensors would save more resources and are able to adapt to new traffic situations.
We argue that in order to be adaptable to other cities and changes in traffic flows, prior information like the road network should be neglected and just the sparse observations considered.
This comes with the added benefit of making our solution applicable in regions where no openly available road network is maintained or pathways change frequently (e.g. flood areas, animal observations). 


The aforementioned problem is novel and more challenging than the commonly considered traffic prediction problem, since there exist very few observations in each input sample.
Current works for the traffic prediction problem do not consider any missing values. \cite{Li2021, Shao22}
A common method among state of the art approaches is the usage of Graph Neural Networks on graphs that model the sensor network \cite{bui22}.
The values of a sensor are applied to the same graph node for each timestep which prohibits any non-stationary sensors . 
With fixed sensor locations, the resulting sensor network is highly correlated with the road network.
Streets connecting two intersections with sensors should be also an interesting point for correlations in the sensor network.
However, variable observations and high temporal sparsity rates can not be modeled adequately in a static network.
We show in our experiments that the road network has only a small influence on the traffic predictions.

Besides the traffic prediction for future timesteps, some works explore the field of traffic speed imputation \cite{Cini22, Cuza22} where missing sensor values are predicted.
But the amount of missing values is assumed to be at most 80\%, which on average are still over 40 given sensors in each timestep in the Metr-LA dataset with a total of 207 sensors.
We consider up to 99.9\% missing values which are on average 2.4 observations in each timestep that are used as input.
Such high sparsity rates drastically decrease the chance that multiple values are present in one input sample from the same sensor location, which makes it challenging to recognize and learn temporal correlations for each location on its own.

High sparsity rates (>95\%) result in few sensor values, but if a reconstruction of the traffic state would be possible, we question if spatio-temporal graphs require nodes for each sensor.
In SUSTeR we utilize only a small amount of graph nodes for the encoding of information and do not relate such nodes to the sensor network.
We call this the hidden graph (see figure \ref{fig:intro}), which is still able to reconstruct the complete traffic state.
Due to the reduced number of nodes SUSTeR achieves faster runtimes, as shown in the experiments.
This hidden graph is not embedded directly in the spatial domain, which is why the assignment of observations, as well as the querying of the future traffic, is done with an encoder and a decoder, implemented as neural networks.
The decoding from the hidden graph to future values depends on a set of query locations.
Figure \ref{fig:intro} shows the query locations as given from outside and in combination with the reconstructed traffic state the future values are predicted.

To construct the hidden graph we encode observations from each timestep into from multiple graphs, one for each timestep. 
The graphs are created in a residual style and information is added to the node embeddings from the previous timesteps.
We choose this method to incorporate all timesteps equally into the hidden state because the redundant information along the past is non-existing for high sparsity rates.
From the sequence of graphs where our framework inserted the observations step by step we apply STGCN \cite{Yu18}, an algorithm for traffic prediction to find and learn the spatio-temporal correlations on our small number of graph nodes.
The first future timestep of the STGCN is our hidden graph in which the traffic state is reconstructed. 

% Recent work has an implicit embedding of the graph nodes into the spatial domain as the assignment from the sensor to graph node is fixed one by one.
% Because the graph has the same structure as the road network spatio-temporal correlations can be learned between those sensors.
% We reduce the number of nodes and use a non-linear assignment learned data-driven from the observations.

We find in the experiments that SUSTeR outperforms the plain STGCN and modern traffic prediction frameworks like D2STGNN for high sparsity rates $(\geq 99\%)$.
This is equivalent to only $0.2$ to $2.4$ observation for each timestep on average.
SUSTeR uses fewer parameters than the baselines and can train faster and with less training data.
Our main contributions can be summarized as follows:
\begin{itemize}
    \item We introduce a sparse and unstructured variant of the traffic prediction problem with sparsity in all dimensions. The sensors report only a fraction of their values and are arbitrarily distributed in the spatial domain.
    \item We propose SUSTeR, a framework around the STGCN architecture, which maps sparse observations onto a dense hidden graph to reconstruct the complete traffic state.
    Our code is available at github.\footnote{https://github.com/ywoelker/SUSTeR}
    \item We conducts experiments that show that SUSTeR outperforms the baselines in very sparse situations ($\geq 95\%$) and has a competitive performance in low sparsity rates.
    % \item SUSTeR trains a third faster than the next competitor.
\end{itemize}

\vspace{-2ex}
\section{Background}
\label{sec:background}
\vspace{-.5ex}

\subsection{Disaggregated Memory with RDMA}



\iffalse
\begin{figure*}[!t]
\centering
\subfigure[An exmaple of one-sided RDMA.]{
    \label{fig:intro:a}
    \includegraphics[width=0.27\textwidth]{Figures/intro-1.pdf}}
    \hspace{2ex}
\subfigure[\texttt{Get} data with RPC and B+tree.]{
    \label{fig:intro:b}
    \includegraphics[width=0.27\textwidth]{Figures/intro-2.pdf}}
    \hspace{2ex}
\subfigure[\texttt{Get} data with RPC and hash table.]{
    \label{fig:intro:c}
    \includegraphics[width=0.27\textwidth]{Figures/intro-3.pdf}}
\vspace{-2ex}
\caption{Examples of accessing different remote storage backends.}
\label{fig:intro}
\end{figure*}
\fi


Disaggregated memory systems with RDMA can be categorized into two types: one-sided RDMA systems~\cite{sherman,race,cell,rolex,smart}, and two-sided RDMA (RDMA-RPC) systems~\cite{fasst, guidelines}. An example of one-sided RDMA systems~\cite{sherman,race,cell,rolex,smart} is illustrated in Fig.~\ref{fig:intro:a}. These systems support applications such as KVS and transaction systems with various index data structures, including B/B+ trees, hash tables, radix trees, and learned indexes. However, it is widely recognized that multiple round-trip communications are needed for each \texttt{Get} request: at least one for querying the index and one for reading data. The high communication cost results in both long latency and network congestion. %The constrained RNIC resources for tracking queue pairs (QP) status incur complexity for data management~\cite{fasst,guidelines} under the reliable connection (RC) primitives. 

\heiner{if you have data to support these claims, provide a forward pointer. E.g. in Section we will show that RDMA introduces X overheads}\liuyi{I think, we can cite the data from cite{scalablerpc} to explain the bad scalability of one-sided RDMA}
\heiner{no need to rely on what other may be concerned about. just say: RPC-based systems introduce CPU overheads...}\liuyi{thanks! ``RPC-based approaches suffer from the large CPU consumption in disaggregated systems, ''}

\begin{figure}[t]
\centering
    \includegraphics[width=0.415\textwidth]{Figures/ludo.pdf}
    \vspace{-1.5ex}
    \caption{Ludo hashing.}
    \label{fig:ludo}
    \vspace{-4.5ex}
\end{figure}

\iffalse
\begin{figure*}[!h]
\centering
\renewcommand\thesubfigure{}
\begin{minipage}[t]{0.27\textwidth}
        %\vspace{1.5ex}
        \subfigure[]{
            \label{fig:mph}
            \includegraphics[width=\textwidth]{Figures/mph.pdf}}
        \vspace{-6ex}
        \caption{A case of MPH for four keys in a bucket.}
    \end{minipage}
\hspace{4ex}
\begin{minipage}[t]{0.28\textwidth}
        %\vspace{1.5ex}
        \subfigure[]{
            \label{fig:ludo}
            \includegraphics[width=\textwidth]{Figures/ludo.pdf}}
        \vspace{-4ex}
        \caption{Ludo hashing.}
    \end{minipage}
\hspace{3ex}
\begin{minipage}[t]{0.32\textwidth}
        \vspace{-1.5ex}
        \subfigure[]{
            \label{fig:othello}
            \includegraphics[width=\textwidth]{Figures/othello.pdf}}
        \vspace{-6ex}
        \caption{Othello hashing.}
    \end{minipage}
\end{figure*}
\fi


Two-sided RDMA-based systems~\cite{guidelines,fasst} have been investigated to dispatch compute nodes' requests to the memory node via RPC over the RDMA network with only one round trip. As depicted in Fig.~\ref{fig:intro:b}, a data index, such as a B-Tree or hash table, is maintained at the memory node.
%the server side\footnote{We refer to compute nodes as the client and the memory node as the server.}. 
When a data query occurs, in addition to polling the RNIC and posting messages, the CPU of the memory node is responsible for traversing the index. The memory node has to perform computational tasks, including hash computation, fingerprint checking, and key comparisons. This process introduces additional computational overhead and memory accesses.
%Alternatively, the system can use a hash table as the indexing data structure. A chained-based hash table is maintained as the data index in the memory node. The memory node needs to probe the linked node to handle the hash collision. Researchers 
Existing solutions~\cite{race,drtmr,mica} that store keys' fingerprints in their hash tables to save memory usage also introduce extra computation. 
%use a hash table to save memory usage by storing the keys' hash fingerprint and providing multiple candidate locations for inserted KV pairs.0 
For example, if the memory node employs the state-of-the-art (2,4)-Cuckoo hash table~\cite{cuckoo}, each \texttt{Get} request requires one fingerprint computation and, at most eight rounds of fingerprint checking.








\vspace{-1ex}
\subsection{Dynamic minimal perfect hashing}
\label{subsec:background:dmph}
In this subsection, we first introduce the background of DMPH and then present an existing MPH implementation, Ludo hashing~\cite{ludo}. 
%We then discuss the Othello hashing~\cite{othello}, which was used as a memory-efficient bucket locator of Ludo hashing. 

%\blux{Perfect hashing~\cite{praticalph} uses a constructed function to distribute keys to different buckets in a hash table without collisions.}
Perfect hashing~\cite{praticalph} represents a family of schemes that designs and manipulates hash algorithms to distribute keys to different buckets in a hash table without collisions.
Since it is impractical to find a single hash function that generates no collisions for a large set of keys, a common approach is to use two levels of mapping. The first level maps keys to a number of groups, each of which contains several keys. The second level addresses key collisions inside each group. Minimal perfect hashing maps $n$ keys to exactly $n$ buckets, but it is inflexible for key insertions and only applicable to a static set. To allow key dynamics, dynamic minimal perfect hashing (DMPH) may use $(1+\epsilon)n$ positions for $n$ keys \cite{scalebricks,ludo}. 
One primary advantage of perfect hashing is that it does not need to store the keys in the hash table. Since perfect hashing eliminates collisions, a key query does not need to compare keys to address collisions. Avoiding storing keys can significantly reduce memory costs, because as a secondary index, the size of keys (usually hundreds of bits) is much longer than the queried value in a hash table (usually a storage address in tens of bits). 


One of the most recent solutions of DMPH is called Ludo hashing~\cite{ludo}. 
As shown in Fig.~\ref{fig:ludo}, Ludo hashing~\cite{ludo} first uses a data structure called Othello~\cite{othello}, a dynamic implementation of Bloomier filters \cite{bloomier} with two arrays, as the \textit{bucket locator} to distribute keys into different buckets, each of which includes exactly 4 slots. Then, in each bucket $B_i$, Ludo hashing uses brute force to find a hash seed $s_i$ such that the hash function with $s_i$
can map the 4 keys in the bucket to 4 different slots without collision. Hence, there is no need to store keys in the table for collision resolution. 
The space cost of Ludo is $3.76 + 1.05l$ bits per key, where $l$ is the length of the record value, which is claimed to be the smallest memory cost in the literature~\cite{ludo}. 
%The bucket locator and the seeds together cost 3.76 bits per key. 
\red{The bucket locator leverages Othello arrays~\cite{othello}, which costs 2.33 bits per key. Each bucket contains a 5-bit long seed shared by four keys in Ludo, i.e., 1.25/0.95 bits per key when we set the load factor as 95\%.
Also, the majority of memory cost is for storing the values in the buckets, costing $1.05l$ bits per key.}
We observed that the computation for looking up the slot only needs the bucket locator and the seeds, which are memory efficient. On the other side, the hash table buckets/slots part storing all data values contributed to most memory of this index, but it requires little computation.


\iffalse
consists of two parts of hash computations. The first part is responsible for narrowing down the key space into a set of candidate indices. 
\heiner{be precise. there isn't a level of hash functions. The second level utilizes multiple hash functions to map candidate indices to unique indexes.}\liuyi{precious comments! I think I can use ``two parts/layers of hash computation''(instead of functions)?} 
The second part of the hash computation is applied to these candidate indices to obtain a unique index for each key. 
\heiner{unclear why multiple hash functions are used/needed. If you mention multiple, you have to explain}\liuyi{The first part hash to map key to bucket, the second part map key to slot. This is a common way to realize MPH, we show a specific process with example of our Ludo in the following.}
Combining these two parts ensures that each key is mapped to a distinct location within the hash table, eliminating collisions.
\heiner{your explanation does not provide a reason for why 2 levels are needed. Why not map to indexes directly in the first level? Why not use a single level and compute perfect hash for all of them?}\liuyi{Like in Ludo, the first level makes sure there wont be more than 4 keys in a bucket. In general, the first level hash will partition the number of Keys into small groups and then realize an MPH in the second level. Acutually, I just briefly explain a common way to realize MPH in this part and details with Ludo and Othello is in the following.}
\heiner{MPH eliminates collision but how does the first level do it? Explain othello/cucckoo here or provide forward reference}\liuyi{Othello uses ``two choices''(alternative buckets for each key) placement to realize it. I will cite Othello here.}
Furthermore, the distinctive feature of MPH lies in its memory usage. It maps $N$ elements into $(1+\epsilon)N$ space of the hash table and eliminates the need to store keys. 

\heiner{explictely need to define bucket and slot. say: We define bucket as.. and slot as..}\liuyi{added it!}
As shown in Fig.~\ref{fig:mph}, a bucket refers to a column of the hash table and consists of multiple slots.
Assuming there are four distinct keys in a bucket, in order to leverage MPH for quick access and minimize storage overhead, they employ a brute-force approach for seeking a seed for each bucket to locate keys. 
\heiner{need to define seed. e.g. we use seeds to realize different hash functions. Provide an example how seed+input is used to compute output}\liuyi{added content shown in blue.}
%This involves attempting various seeds starting from 0 until the hash function successfully distributes the four keys into separate slots without any collisions. 
\blue{Consider the four keys inserted to a bucket are $\{k_0,k_1,k_2,k_3\}$. Perfect hashing uses brute force to select a specific seed $s$ from 0 to 255, which can make the universal hash function $H(k_i, s)$ map the four keys into four distinct slots without collision.}  
Afterward, the chosen seed will be retained in the bucket, serving as a slot locator for key queries. They will hash the queried key using the selected seed, and the resulting value will indicate the slot number for that key.
Finding a seed for resolving keys in a small key space (e.g., a bucket of four) is one of the approaches~\cite{scalebricks,ludo} to realize MPH. 


\noindent\textbf{Ludo hashing~\cite{ludo}.}
MPH is effective only when the key space is small. For instance, they can consistently find a perfect seed within 255 (8 bits) to distinguish the four different keys without collisions in their experiments.
\heiner{what happens if we don't find one?}\liuyi{we have never seen that with the seed search space set 0-255. If }
However, when attempting to find a perfect seed for distributing over five keys, expanding the seed space (e.g., 10 bits) and dedicating more time becomes necessary.
\heiner{why would we attempt 10 bit if 8 bit is sufficient?}\liuyi{8 bits only works for 4 keys, if there are 5 keys, more bits needed.}
Ludo hashing comprises a hash table with a specified number of MPH buckets, each containing an 8-bit seed and four elements. 
\heiner{vague. In stead of "the nature of MPH" say why. e.g. because MPH eliminates collitions it is unnecessary to store keys/tags as part of an entry to determine a match.}\liuyi{thanks for your advice and fix it as shown in blue.}
\blue{Since MPH eliminates collisions, it is unnecessary to store the key field to determine a match, and only the value field is stored in each slot.}
\heiner{we still need a valid bit per entry right?}\liuyi{we do need a bit to show the new inserted keys' conflict when we apply MPH on the KVS, but that is not part of perfect hashing.}
However, due to the potential collisions of hash functions~\cite{can}, merely hashing different keys into separate buckets doesn't guarantee that the size of each bucket will be no more than four. Ludo hashing adopts "the power of two choices"~\cite{twochoices}, similar to Cuckoo hashing~\cite{cuckoo}, where each key is placed in two candidate buckets, denoted as $b_0$ and $b_1$.
A bucket locator is employed to distinguish which specific candidate bucket (e.g., $b_0$ or $b_1$) stores the queried key as shown in Fig.~\ref{fig:ludo}. Additionally, the selection of candidate buckets for all keys must ensure that no bucket is assigned more than four keys. Ludo hashing utilizes Othello hashing to construct its memory-efficient bucket locator, dividing keys into groups with a small key space.

\noindent\textbf{Othello hashing~\cite{othello}.}
\heiner{first say what problem othello addresses. the next section (separating keys into two sets does not tie. why do we need this to ensure <=4 keys per bucket?}
Provided we have $N$ distinct keys, and they are separated into two sets $S_0$ and $S_1$. To query which set they belong to, the most intuitive way is to build an index for all keys, and the corresponding values are 0 or 1, which indicates the set it is. However, storing a full table will incur large memory overheads, especially since the workload's key size is getting larger~\cite{workload} as time goes on.

\heiner{unclear. what do we need binary classification for? It is unclear how othello ties to the previous}\liuyi{added it as shown in blue.}
\blue{There are two candidate buckets $b_0$ and $b_1$ represented by two distinct hash functions $hash_1(k)$ and $hash_2(k)$ for each key to be placed. Each key can only be stored in one of the buckets, and they need to track which bucket holds the key.
One-bit Othello hashing is proposed for tracking the bucket choice of each key through binary classification}, as illustrated in Fig.~\ref{fig:othello}. There are two bit arrays $A$ and $B$, with lengths $m_a$ and $m_b$, respectively, and corresponding hash functions $hash_A(\cdot)$ and $hash_B(\cdot)$. 
Othello is constructed by finding an acyclic undirected graph $G=(V_a,V_b,E)$, where $E$ is the edge set, and $V_a$, $V_b$ are the vertex sets. Each node $v^i_a \in V_a$ ($0 \leq i < m_a$) and $v^j_b \in V_b$ ($0 \leq j < m_b$) represents the $i$-th and $j$-th bit of arrays $A$ and $B$.
They assign uniformly random values for the gray vertex (e.g., $u_2, v_1$), which will not influence the query result because they are not the hashed bit for any keys.
To query the binary result of a given key $k$, the result is given by $A[hash_A(k)] \oplus B[hash_B(k)]$, which is the exclusive OR of the hashed bits in these two arrays. 
It takes O(1) time to find a proper pair of <$hash_A, hash_B$> to successfully allocate all keys with a memory space of at least $2.33$ bits/key. Also, it takes O(1) time to query the binary result of the given key.
\fi


\vspace{-2.5ex}
\section{Measurement and Motivation}
\label{sec:motivation}
\vspace{-.5ex}
%\textbf{Disaggregated memory systems using RDMA-RPC can achieve higher throughput than one-sided RDMA by reducing index computations.} The multiple memory accesses inherent to such systems imply that adopting one-sided RDMA necessitates multiple communication round trips, completely bypassing the CPU at the memory node for serving data requests. In contrast, two-sided RDMA or RDMA-RPC uses only one round trip but requires additional computation at the memory node for querying the index.
%The state-of-the-art one-sided RDMA design provides comparable throughput to RDMA-RPC~\cite{fasst} when there are limited CPU resources for memory nodes in disaggregated systems.

We wonder if, we remove the computation cost at the memory node, will RDMA-RPC demonstrate much higher throughput than the state-of-the-art one-sided RDMA?
\textbf{If the answer is "Yes", then there is a great opportunity to design a high-throughput RDMA-based KVS by reducing the computation cost at the memory node.} 

Toward this objective, we conduct experiments to analyze the throughput performance of both one-sided RDMA and RDMA-RPC systems with 9 r320 servers in CloudLab~\cite{cloudlab}, each is configured with a Mellanox CX3 adapter (50Gbits).
We compare the performance of the following systems with \texttt{Get}-only workload. (1) RACE hashing~\cite{race}, a state-of-the-art one-sided RDMA-based scheme. Its hashing index is crafted for disaggregated memory, facilitating data retrieval within two round trips. (2) RPC-hash table, a two-sided RDMA method whose compute nodes and memory nodes communicate in RDMA unreliable datagram (UD) mode. Each memory node maintains a chained hash table in its local memory to handle remote data requests. (3) RPC-Dummy. A hypothetical RDMA-RPC method that incurs minimal computation cost at each memory node. 
RPC-Dummy only implements one memory access and then returns any data in the accessed memory at the memory node,  with no extra computation tasks.
RPC-Dummy's throughput can be considered the upper bound among all possible RDMA-RPC systems. We use this method to explore the performance potential of our design objectives.  We vary the number of memory node threads as 1, 2, and 4 in RPC-based approaches, and each memory node thread maintains one Queue Pair (QP) and runs in a distinct CPU core.
%We vary the total number of compute node threads from 8 to 64. More evaluation details can be found in Section~\ref{sec:eval}.

\begin{figure}[!t]
\centering
%\subfigure[The throughput of RACE Hashing with the varied client threads.]{
    %\label{fig:motivation:a}
    %\includegraphics[width=0.32\textwidth]{Figures/motivation-1.pdf}}
\captionsetup[subfigure]{aboveskip=-2ex}
\vspace{-2ex}
\hspace{-2.5ex}
\subfigure[Throughput of different systems with limited number of memory node threads.]{
    \label{fig:motivation:b}
    \includegraphics[width=0.465\textwidth]{Figures/motivation-2-new2.pdf}}\\
\vspace{-3ex}
\hspace{-2.2ex}
\subfigure[The CPU time breakdown on a memory node with one thread.]{
    \label{fig:motivation:c}
    \includegraphics[width=0.467\textwidth]{Figures/motivation-3-newest.pdf}}
\vspace{-3ex}
\caption{Observations from the microbenchmarks.}
\label{fig:motivation}
\vspace{-4.5ex}
\end{figure}

The results are shown in Fig.~\ref{fig:motivation:b}. For one memory node thread (one core), RPC-hash table achieves a throughput similar to that of RACE hashing.
For RACE hashing, multiple reasons limit its throughput, including the two round trips to complete one data \texttt{Get} operation and multiple RC connections of the compute node threads that incur resource contention in the RNIC cache~\cite{scalablerpc}.
%two round trips are required to complete one data read operation and multiple RC connections of the compute node threads in RACE hashing make it hard for RNIC to cache their status
RPC-hash table requires only one round trip, but the complexity of querying the index on the memory node introduces extra latency and limits its throughput.  
%The reason is that the multiple RC connections of the compute node threads in RACE hashing make it hard for RNIC to cache their status, and the two round trips are required to complete one data read operation. 
The throughput of RPC-hash table increases correspondingly when we increase the number of threads to 2 and 4. In contrast, RACE hashing maintains a static performance.
%Therefore, the RPC-based RDMA method offers better scalability compared to RACE hashing. 
%Even through systems like RACE hashing free the memory node CPU, they fail in performance scalability compared to the partially-offloaded approach represented by the RPC-hash table.
%We take a further look at the performance of RPC-Dummy. 
%Furthermore, in the comparison between RPC-Dummy and RPC-hash table, 
RPC-Dummy can outperform RPC-hash table by around 2$\times$ under the cases of both single and multiple memory node threads. 
%For example, with 64 compute node threads, RPC-hash table provides a throughput of 2.95 Mops per memory node thread. In comparison, RPC-Dummy can achieve 5.04 Mops per memory node thread under the same conditions. 
%Although RPC-Dummy needs some trivial computation tasks at the memory node, such as RNIC polling and local data lookups~\cite{fasst}, by removing the index computation and memory access cost, the throughput shows huge improvement compared to RPC-hash table that includes a hash table lookup at the memory node. 
%with the hash table requiring multiple memory reads across various layers to search for a key, as discussed in section~\ref{fig:intro}. 
Hence an RDMA-RPC network that introduces little computation overhead to the memory node can achieve higher throughput than both existing one-sided RDMA and RDMA-RPC solutions. 
The results suggest that RPC-based KVS has a potential for throughput improvement by reducing computation tasks at memory nodes, which motivates the design of this project.
%For example, a single memory node thread proves sufficient to achieve a throughput of 2.95 Mops in the RPC-hash table for . However, the RPC-Dummy can achieve 5.04 Mops with lightly-offloaded computation on the memory node.
%The performance gap between partially offloaded and light-offloaded approaches motivates us to delve into the specific computational resources on the memory node side, especially in scenarios where CPU resources are constrained.



%Concerns have been raised about the large CPU consumption in disaggregated systems when employing RPC-based approaches, particularly about the remote memory node's serving functions. Despite there being efforts~\cite{drtmh,hstore,herd} to use both RPC primitives and one-sided RDMA verbs for various data requests, performance issues persist in building efficient disaggregated systems. For instance, in the case of HStore~\cite{hstore}, which employs asynchronous data updates to a sorted index via RPC verbs and single data lookups with one-sided verbs, challenges related to RNIC resource utilization and server CPU bottlenecks for data query throughput remain. This situation prompts us to explore whether an alternative tradeoff could be achieved to enhance system performance, not just a simple combination of these approaches.








\textbf{CPU utilization breakdown for RPC-based approaches.}
%To further analyze the consumed CPU time at the memory nodes in RDMA-RPC approaches, we conduct the following microbenchmark to examine the CPU cost breakdown of different computation tasks. 
We run RDMA-RPC with different indices: hash table, Btree, and learned index, at the memory node. 
%These indices involve different query operations. RPC-Dummy is also tested for comparison purposes.
The CPU time consumed by these four RPC-based KVS systems while handling an equal number of data \texttt{Get} requests is normalized and presented in Fig.~\ref{fig:motivation:c}, with the number of compute node threads fixed at 64. RPC-Dummy takes the least time.
%to serve the requests because the memory node processes the data request by simply implementing a memory read. 
Other approaches consume more time in different amounts.
%because of the extra computation on the memory node. 
%In the case of RPC-Learned index, the learned index model~\cite{alex} consists of tree nodes in multiple layers, with each node maintaining a learned model and a gapped array for storing KV pairs. Locating a key involves the computational overhead of traversing tree nodes from the root and computing with the learned models.
%Additionally, an extra memory read from the data area is necessary. 
%Furthermore, we delve into the breakdown of the consumed CPU time of all four approaches.
For RPC-Btree, in addition to the communication overheads for polling $\mathtt{mlx4\_poll\_qp}$ (4.03\%), posting messages $\mathtt{mlx4\_post\_send}$ (7.52\%) and UD transport (6.85\%) from connection management, the most CPU-consuming event is the RPC callback function (70.59\%), which executes local index lookup and data access. 
In all four schemes, the RPC callback function consumes the most CPU time, and the variations in CPU consumption among them are mainly attributed to differences in the RPC callback function. RPC-Btree consumes the most CPU time for RPC callback, followed by RPC-hash table. 
%we can observe a similar CPU resource breakdown pattern in that most CPU time is consumed on the RPC callback function. 
%ALEX demonstrates slightly lower CPU cycles for its fast data lookup with recursive model index~\cite{learnedindex}. Local data lookup involves both computation and memory reads. For example, 
RPC-Dummy spends the least CPU time on the RPC callback function (46.11\%) and serves the most data requests because there is no computation burden for the memory node in RPC-Dummy. 
%The reason is that the computation on memory nodes of RPC-Dummy is lightly-offloaded compared with other partially-offloaded approaches.
In disaggregated systems, tasks such as computing hash functions on a hash table, traversing tree nodes in a B-Tree, and executing learned models on a learned index are not ideally suited for memory nodes.
%We should make the memory node focus on its storage responsibilities instead of computation tasks.
\textbf{The throughput of RDMA-RPC methods is mainly limited by CPU usage during the RPC callback function for index lookups and data reads. High CPU consumption from complex index computations on memory nodes reduces throughput, particularly when CPU resources are constrained, indicating that optimizing these computations can enhance performance.}

%Recently, researchers have explored the application of learned indexes in disaggregated memory with one-sided RDMA primitives~\cite{xtore,rolex}. However, despite the use of caching in the compute node, the approach only works for ordered data and it is not the focus of our work.
%still necessitates a minimum of two round trips to serve requests.




\vspace{-1ex}
\section{Design of \sys}
\label{sec:design}
\vspace{-.5ex}

\subsection{Overview}
\label{sec:design:overview}
\heiner{provide brief summary and tie. e.g. In the previous section, we showed that existing KVS suffer from X. We now present..}

Based on the motivation presented in the previous section, we design and implement an RDMA-RPC network that aims to minimize computation tasks on memory nodes, consequently enhancing the system throughput. 
This section presents the design of \sys, a scalable RDMA RPC-based disaggregated KVS that tackles the performance limitations of existing RDMA RPC and one-sided RDMA-based schemes. 
%Our primary approach involves offloading the computation of index lookup to the compute node, thereby alleviating the CPU burden on the memory node. 
To accomplish this design objective, we decouple the index of \sys into two components: 1) a computation-heavy component running on compute nodes, and 2) a memory-heavy component running on memory nodes.
In particular, DMPH provides an opportunity for this decoupling. By carefully examining the DMPH's read and insertion operations, we observe that the final step consistently is directly retrieving the value from a specific memory location, while all the previous steps are employed to determine that location.
Contrary to DMPH, other hash tables necessitate retrieving the key from the hashed location by key probing and comparison, and only when the key matches the search key, the value can be returned. 
The distinctive process of DMPH motivates us to store all values in the memory-heavy components because they can be read without extra computation. And the steps to determine the location of the value can be placed in the compute-heavy component running on the compute nodes.

\sys requires only a single round trip for data requests while supporting a large number of concurrent compute nodes's requests. In contrast to other RDMA RPC-based approaches~\cite{fasst,herd}, \sys substantially reduces CPU resources required on the memory node.
In the following, we elaborate on the components maintained in the compute pool and memory pool of \sys.

\begin{figure}[t]
    \centering
    %\captionsetup{font=small}
    \begin{minipage}[b]{0.9\linewidth}
        \includegraphics[width=\linewidth]{Figures/overview.pdf}
        \vspace{-3.5ex}
    \end{minipage}
    \vspace{-1ex}
    \caption{\sys overview}
    \vspace{-3.ex}
    \label{fig:overview}
\end{figure}


Fig.~\ref{fig:overview} depicts the overall structure of \sys, which leverages a shared-nothing architecture~\cite{tutorial} for separating data into different shards with consistent hashing~\cite{consistentHashing}. 
The compute pool comprises multiple compute shards, each accommodating several compute nodes.
Note that the configuration for the number of shards and the number of compute nodes depends on the memory budget in compute nodes and the whole size of the datasets. 
For each shard, an index is built based on the keys of the shard, and the returned values of the index represent the memory locations that store the corresponding data associated with the keys. 
The index is decoupled into the compute-heavy and memory-heavy components.  
Each compute node is allocated a memory budget for caching the compute-heavy component, including the bucket locator and the seeds. The default setting is there are 64 million keys in a shard, and the memory overhead on each compute node is less than 50MB (\S\ref{sec:eval:mem}). This is considered a small overhead because recent one-sided RDMA solutions cost over 300 MB on each compute node for index caching and other purposes~\cite{rolex,xtore}. 
All compute nodes in the same shard will connect to the memory node with RDMA RPC for data operations and one-sided RDMA for new bucket locator fetching after index resizing -- the details will be explained in \S\ref{sec:design:resizing}.
Each shard consists of one memory node, which contains the most updated bucket seeds, overflowed cache, DMPH buckets, and KV data in the shard. 
The DMPH buckets store the data addresses in the KV data memory space of the keys in the shard. The latest bucket seeds are maintained to ensure the consistency of data insertion. Additionally, the overflowed cache for KV pairs is used to temporarily hold the pair of the new key and the address, which cannot be inserted into DMPH buckets without the need for hash table resizing. We leverage a hash table to work as the overflowed cache in \sys.
\red{The KV data in each shard is replicated to two other shards, serving as replicas with checkpoints. These two replica shards can be chosen as the two successive shards in the consistent hashing ring. Each key's primary replica shard is referred to as the \textit{primary shard} of the key. Each shard is identified by a \texttt{uuid}. We assume there is a \textit{service layer} in front of the compute nodes responsible for only forwarding data requests to one of the compute nodes in the primary shard based on the key's hash value in the consistent hashing ring. 
After the memory node in the primary shard completes a data update operation, it forwards the update to its replica shards. To ensure load balance among compute nodes within a shard, the service layer maintains a counter for each shard and distributes requests to the compute nodes in a round-robin fashion.}

\begin{figure}[t]
    \centering
    \begin{minipage}[b]{0.9\linewidth}
        \includegraphics[width=\linewidth]{Figures/layout.pdf}
    \end{minipage}
    \vspace{-1.ex}
    \caption{The data layout in a DMPH bucket.}
    \label{fig:layout}
    \vspace{-3.5ex}
\end{figure}


\begin{figure*}[!t]
\centering
    %\vspace{-2ex}
    \subfigure[\texttt{Get} operation.]{
        \label{fig:op:lookup}
        \includegraphics[width=0.335\textwidth]{Figures/lookup.pdf}}
        \hspace{-1.2ex}
    \subfigure[\texttt{Insert} operation.]{
        \label{fig:op:insert}
        \includegraphics[width=0.329\textwidth]{Figures/insert.pdf}}
        \hspace{-1.2ex}
    \subfigure[\texttt{Update} and \texttt{Delete} operation.]{
        \label{fig:op:update}
        \includegraphics[width=0.32\textwidth]{Figures/update.pdf}}
\vspace{-3ex}
\caption{Data operation protocols in \sys.}
\vspace{-2.5ex}
\label{fig:op}
\end{figure*}

\vspace{-.5ex}
\subsection{Decoupled DMPH index}
\label{sec:design:table}
In this section, we explain the detailed data structure and its components maintained in the compute node and the memory node. 
We reuse the design of Ludo hashing as introduced in \S\ref{sec:background}. There are two candidate buckets for each key, and the bucket locator runs 
%both two hash functions of a cuckoo hash table~\cite{cuckoo} and 
a data structure called Othello~\cite{othello} to determine which bucket the value of the search key is stored in. 
%Within the MPH workflow outlined in section~\ref{sec:background}, Othello arrays play a crucial role in selecting the hash functions' outputs from $hash_A$ and $hash_B$ that correspond to the bucket number of the queried key. This implementation aligns with the "power of two choices" principle in the MPH scheme~\cite{twochoices}, which is employed to optimize memory utilization.
Each Ludo bucket contains one seed and four slots. By computing a hash value with the search key and the seed, the key is mapped to an exact slot of the bucket without colliding with other keys within the same bucket. The value stored in the slot represents the key's data address and is utilized to retrieve the corresponding data.
%and we will compute another hash function of the queried key, coupled with the seed in the bucket, to identify the specific slot that contains the targeted key's data address.

We decouple the entire data structure of Ludo hashing into two components. The compute-heavy component running on each compute node stores both the bucket locator and the seeds for all DMPH buckets. 
This component completes all computations related to finding the location that stores the value of the search key and costs only $3.76n$ bits -- $2.33n$ bits for the bucket locator and $1.43n$ bits for the seeds, where $n$ refers to the number of KV pairs in a shard.
%Within the compute node, we store both Othello arrays and seeds for all MPH buckets. Consequently, the main computation involved in key lookup can be totally offloaded to the compute node.
Within the memory node, the memory-heavy component consists of all DMPH buckets that store the data addresses for all keys in the shard. Assuming the load factor of the DMPH table is set to $\epsilon$ with a default value of 0.95, the number of DMPH buckets will be $n/(4\cdot\epsilon)$ as each bucket accommodates four slots. 
The detailed layout for each DMPH bucket is illustrated in Fig.~\ref{fig:layout}, and each bucket is 32-Byte long with four packed slots. 
There are four fields in each slot: cache bit (1 bit), fingerprint (6 bits), length (9 bits), and data address (48 bits). 
%Note that the fingerprint is only used for updating the table and read requests do not need to check it. 
The cache bit serves as an indicator to identify whether another key(s) share the same slot, with its index stored in the overflowed cache. 
Meanwhile, the 6-bit fingerprint is only utilized during the index update process to verify if the KV data referenced by the address in this slot corresponds to the search key or not. This fingerprint check is exclusively applied during data write requests, and any false positives do not impact the final result. This is because a comprehensive recheck of the full key occurs after accessing the actual KV data block on the compute node side.
Note that read requests do not need to check the fingerprint. 
The address signifies the starting offsets of the KV block, while the length indicates the byte length of the entire KV block in the underlying KV data area.
In the underlying data area, the KV block is compactly stored with four fields. The initial two numbers, each occupying 8 bytes, denote the length of the key and the subsequent value field.

The overflowed cache accommodates the key-address pair that cannot be inserted into the mapped DMPH bucket without modifying the bucket locator or resizing the entire hash table.

For an estimation, if $\epsilon=0.95$, the component at the compute node contributes to only 5.5\% of the total memory size of the index while the component at the memory node accounts for the larger portion of 94.5\%.

\vspace{-1.5ex}
\subsection{\sys operations and protocols}
\label{subsec:design:operation}

%To deliver high concurrent performance, we decouple the MPH table structure into the compute pool and the memory pool to fully use the RNIC of the memory node.  Based on the decoupled indexing structure explained in~\ref{subsec:design:table}, we also take apart the MPH algorithm to assign only the memory nodes with minimal computation tasks with the RPC primitives. 
This subsection presents the data operations and the corresponding protocol of \sys, including the data \texttt{Get}, \texttt{Insert}, \texttt{Update}, and \texttt{Delete} operations, as shown in Fig.~\ref{fig:op}.
\vspace{-1.ex}
\subsubsection{Data Get operation.}
As shown in Fig.~\ref{fig:op:lookup}, the compute node maintains the bucket locator (two Othello arrays $A$ and $B$) and the seed array $s$. Meanwhile, the memory node maintains the DMPH buckets that store KV addresses and the KV data in a disjoint memory area. 
When there is a data \texttt{Get} request for key $k$, the compute node will \ding{182} compute the bucket index from the bucket locator by looking up two bits on the two arrays, respectively. 
%$b_0$ or $b_1$, which is indicated by the exclusive OR result of the two bits from Othello arrays with the index of $hash_A$ and $hash_B$. 
Assuming the bucket index that stores the queried key is $ind\_bucket$, the compute node will then proceed to \ding{183} compute the slot number within the bucket with the hash function and the seed $s[ind\_bucket]$. 
At this point, the compute node \ding{184} gets both the bucket index and slot index in the MPH buckets, and it \ding{185} posts them to memory nodes with RDMA\_SEND in the opaque fields. 
After the memory node gets the message and parses the index numbers of the bucket and slot, $ind\_bucket$ and $ind\_slot$, it will \ding{186} go directly to the MPH buckets to access the exact slot without any extra computation. 
Then, the memory node \ding{187} gets the data offset in the underlying KV data area from the last 48-bit field of the slot. At last, \ding{188} the KV data will be read back and returned to the initiating compute node for full key check. 
\red{For example, when a compute node requests data for key 5, it computes the bucket index 10 and slot index 0 based on the bucket locator and the locally stored seeds. Then, the pair of indices (10,0) is sent to the memory node. The data index stored in the indicated slot of the memory node is read, and the corresponding data block is returned. Lastly, the compute node checks the cache bit and a full key to see if the Makeup\texttt{Get} is needed.}

There could be some KV pairs that are temporarily inserted into the overflowed cache during the updates and reconstruction of the index. %overflowed or overflow?
%During resizing, the KV pairs inserted into the overflowed cache may share the slot with the key inserted during MPH table construction. Consequently, the data returned to the compute node might not correspond to the queried key, and the index of the queried key is stored in the overflowed cache. This is one of the weaknesses of MPH, which will return a random value for alien keys.
In this circumstance, the compute node is tasked with checking the cache bit, ensuring that the returned full key aligns with the queried one. If the key does not match the requested one, and the cache bit in the slot is set to 1, the compute node will initiate another \texttt{Get} makeup request with the $ind\_slot$ specified as -1, signaling the memory node that the returned key does not match the requested key.
While it is possible to offload the full key comparison task to the memory node, saving one round trip, this approach introduces computation overheads on the limited remote core resources. To make the common case easy, we opt to assign the full key check task to compute nodes.



\textbf{Makeup \texttt{Get} request.}
When the KV data returned to the compute node does not match the requested key, there are two reasons: (1) The requested key is kept in the overflowed cache. The KV pair is inserted after the DPMH table is constructed, and the hashed slot is occupied by another key. 
(2) The requested key is in another slot of the hashed bucket. This case results from changing the order of keys based on the new seed within the bucket when the inserted key can fit into the current DMPH table (detailed in Section~\ref{sec:design:insert}). 
Due to the above two situations, the compute node will send the makeup \texttt{Get} request with the $ind\_slot$ as -1 to the memory node.
The memory node will search the overflowed cache first; if there is a cached item matching the full key of the requested key, it will read the data and return it to the compute node. 
If not, it will read out all the KV blocks referred by the hashed bucket (at most four) and compare the keys until it finds the requested key. 
Additionally, the new seed will be returned back to the compute node if the key is found in another slot, and the compute node will update the copied seeds array for this bucket locally.

\vspace{-1.5ex}
\subsubsection{Data Insert operation.}
\label{sec:design:insert}
The main idea of implementing the data \texttt{Insert} operation of \sys is to determine if we can insert the key into the index without significant changes to the current bucket locator.
If an \texttt{Insert} operation only requires changing the value in one DMPH bucket, \sys can make this change directly. 
However, if a \texttt{Insert} operation will cause the index to resize, which usually happens after a number of insertions, \sys needs to ensure the correctness of the \texttt{Insert} operation and following lookups during index resizing. 
%; if not, we will cache them briefly for further insertion when resizing.
As shown in Fig.~\ref{fig:op:insert}, like \texttt{Get} operation, the compute node will get $ind\_bucket$ and $ind\_slot$ from the bucket locator and the seeds through multiple hashing computations. Different from \texttt{Get}, the RPC message posted to QP should include the full key. 
Thus, the memory node can parse the $ind\_bucket$, $ind\_slot$, and the key from the message and execute the following steps. 
\ding{182} the memory node will write the data into the underlying data area, then it can get the data length and the address (offset in the data area) for indexing. 
After the memory node composes the value from the corresponding slot with the cache bit (set to zero by default), fingerprint, length as well as address, it \ding{183} will try to insert it in the DMPH table. 
%\red{For example, provided a compute node inserts the KV pair (5,4) and computes the bucket index as 10 and the slot index as 0. Then, the compute node sends the tuple (5,4,10,0) to the responsible memory node. The memory node checks if the first slot (indicated by index 0) of the 10th bucket is empty. If it is, the data block will be written to the data area first, and the composite index in the slot will be filled.}

We discuss the rest of \texttt{Insert} in three cases:

$\bullet$ \textbf{\texttt{Insert} without bucket locator and seed change.} 
The memory node checks the slot indicated by $ind\_bucket$ and $ind\_slot$. If the length field is empty (length is 0), signifying there is no key associated with this slot, the memory node inserts the composed slot value (Fig.~\ref{fig:layout}) into this location and returns \texttt{SUCCESS} to the compute node. Conversely, if the length is non-zero, indicating that an existing key is using this slot, the memory node proceeds to check the fingerprint and compares the full key to determine if the original key in this slot matches the inserted key. If they match, the insertion is resolved and treated as an \texttt{Update} operation. The fingerprint can prevent the memory node from reading the full key in the KV data area if they are not the same.


$\bullet$ \textbf{\texttt{Insert} with seed changes but the bucket locator remains the same.}
If the key associated with the targeted slot does not match the newly inserted one, an examination is made to determine if there is another available slot within this bucket. Assuming there are only three keys in this bucket, and the slot indicated by $ind\_slot$ is already occupied by a different key, the memory node endeavors to find a new seed that accommodates all four keys in the bucket without causing collisions, thereby preserving perfect hashing policy in this bucket.
The other three keys are read from the underlying KV data area, and the memory node employs a brute-force approach to identify a new seed for perfect hashing within this bucket. Importantly, the bucket locator does not need to change because all four keys remain in the same bucket. Subsequently, the updated seed for this bucket is returned to the compute node, which then propagates this modification to other compute nodes in the same shard. 

\iffalse
\begin{algorithm}[t]
    %\hspace{\algorithmicindent}
    \label{algorithm:insert}
    \caption{Data Insert Operation in Memory node}
    \LinesNumbered
    \SetAlgoNlRelativeSize{-1}
    \SetNlSty{textbf}{}{:}
    \textbf{DMPH\_Table} buckets []\\\textbf{Seed\_Array} seeds []\\
    \textbf{Hash\_Func} slot\_locator, hash\_fp\\
    
    \SetKwFunction{FMain}{Insert\_Callback\_Func}
    \SetKwProg{Fn}{Function}{:}{}
    \Fn{\FMain{key, val, ind\_bkt}}{
    %\KwOut{Insert operation status}
    lock (ind\_bkt)\;
    ind\_slot = slot\_locator (key, seeds[ind\_bkt])\;
    slot\_value = bucket [ind\_bkt][ind\_slot]\;
    \uIf{slot\_value.fingerprint == hash\_fp (key)}{
        \uIf{key == read\_key\_from\_kv\_blocks ()}{
        \tcp{The mapped slot holds the same key }
        resolve\_to\_data\_update()\;
        unlock(ind\_bkt); return\;
    }\textbf{end}\\}
    \textbf{end}\\
    write kv block into data area\;
    keys [] = read four keys from kv blocks with ind\_bkt\;
    \uIf{slot\_value.length == 0}{
        \tcp{ Insert kv pair without seed change}
        compose kv slot into DMPH table\;
    }
    \uElseIf{there is an available slot for inserted key}{
    seeds [ind\_bkt] = search new seed for the bucket\;
    adjust keys order in the DMPH table\;   
    }
    \uElse{
    \tcp{The bucket is full with other keys}
    insert kv index into the overflow cache\;
    }
    %\tcp{Handle other conditions}
    %\tcp{Slot hashing locator with seed}
    \textbf{end}\\
    unlock (ind\_bkt)\;
    }
    \textbf{End}\\
    %\vspace{-1ex}
\end{algorithm}
\fi


$\bullet$ \textbf{Insert data to overflowed cache.}
When all four slots within the bucket are occupied, and the memory node is unable to find an empty slot for the inserted key, the pair of the key and the KV block address will be \ding{184} placed in the overflowed cache. Also, the cache bit in the conflicted DMPH slot will be set to 1 to indicate at least one key in the overflowed cache sharing the same hash slot. 
Instead, when the number of KV pairs in the overflowed cache reaches a predefined threshold, the memory node initiates the index resizing process to accommodate more KV pairs in a new DMPH table.


%We summarize the whole process of data insertion on the memory node side, as shown in Algorithm 1. 
The data insertion process on each memory node works as follows. 
At first, the memory node will lock the data operations on the targeted bucket
%(line 5) 
to prevent the potential data operations on this bucket. 
The inserted key might have been stored in the DMPH table before. Thus, the memory node will check if the insert request can be resolved to a data update operation by comparing the fingerprint and the underlying full key. %(line 8-13) 
Then, the memory node first writes the KV block to the underlying data area %(line 14) 
and processes the data insert request based on the stored bucket keys 
%(line 15) 
into the mentioned three cases. %(lines 16-23). 
Finally, the memory node unlocks the bucket after it finishes the data insert operation.  
Note that the data insert request tuple sent by the compute node consists of the KV pair and $ind\_bucket$, not including $ind\_slot$. 
The reason is that the memory node keeps the most update seeds array in the shard and can use the seeds to do the hash computation as the slot locator. 
Also, the bucket locator is not maintained in the memory node, and the data insert operation will not modify it after the DMPH table is constructed every time.
%Since there is only one memory node in each data shard, the seeds array maintained on it is the 
This choice is made because modifying the bucket locator requires changing seeds for keys in at least two buckets, leading to more computational overhead. 
%Consequently, we refrain from attempting to modify the bucket locator before the DMPH table resizing (details in Section~\ref{sec:design:resizing}).

\vspace{-2.ex}
\subsubsection{Data Update and Delete operations.}
For data update and deletion, the compute node also acquires the $ind\_bucket$ and $ind\_slot$ from the bucket locator and the seeds array. Like the \texttt{Insert} operation, the compute node transmits the full key to the memory node.
As illustrated in Fig.~\ref{fig:op:update}, the memory node directly accesses the address of the KV data from the DMPH bucket and verifies whether the requested key matches the underlying data. Once the memory node confirms the key, for \texttt{Delete}, it marks the length of the slot value as zero and returns the corresponding status. In the case of \texttt{Update}, it writes the new data to the underlying data area. 
If the cache bit is set to 1 and the keys differ, the memory node will go to the overflowed cache to get the data address.
%\red{For example, a compute node updates the value of key 5 to 3. It packs and sends the KV pair, bucket index, and slot index as the tuple (5,3,10,1) using RPC. The memory node reads the index value from the DMPH and updates the value in the underlying data block to 3. The index value in DMPH remains the same. However, If the key check fails, the memory node will check the overflow cache to see if the index for key 5 is there.}

% For revision
%%Note that for all of \texttt{Insert}, \texttt{Update}, and \texttt{Delete} operations, the memory node will lock the targeted bucket to guarantee the consistency of the data operations.

\vspace{-.5ex}
\subsubsection{\red{Concurrency control.}}
\label{sec:design:concurrency}
\red{Each bucket in the DMPH table within the memory node has a mutex lock. Prior to executing any \texttt{Insert}, \texttt{Update}, or \texttt{Delete} operation, the relevant bucket is locked, blocking any access to its indices. Subsequently, the operation is executed and the lock is released. During the lock period, all other operations targeting this bucket are buffered and only processed once the lock is released.}



\begin{figure}[!t]
    \centering
    %\captionsetup{font=small}
    \begin{minipage}[b]{0.95\linewidth}
        \includegraphics[width=\linewidth]{Figures/extendible.pdf}
        \vspace{-5ex}
    \end{minipage}
    \caption{\red{Extendible hashing in \sys.}}
    \vspace{-4ex}
    \label{fig:design:extendible}
\end{figure}


\vspace{-1.ex}
\subsection{Index resizing}
\label{sec:design:resizing}
\red{When the number of KV pairs in the overflowed cache surpasses a predefined threshold, index resizing and reconstruction become necessary to accommodate the KV pairs into a new hash table. This resizing process introduces two challenges: (1) managing data operation requests during resizing and (2) efficiently coordinating the compute node and memory node to transfer the bucket locator and seeds.}


\red{To support data requests on runtime while index resizing, 
\red{we apply extendible hashing~\cite{race,dash} to allocate a new DMPH table to accommodate more keys' indices, and a \textit{directory index} is used to identify the multiple DMPH tables, which is an additional hash layer as shown in Fig.~\ref{fig:design:extendible}.}
This approach reduces the number of keys that need to be moved during index resizing and shortens the resizing duration.
Compute nodes maintain the bucket locator and seeds array for each single hash table, while memory nodes store the most update seeds array and DMPH tables, as well as local depth array~\cite{dash,race}.}



%%%%%%%%%%%%%%% COORDINATION %%%%%%%%%%%%%%%%


\red{In each shard, we have two size thresholds for overflowed cache; One is for slowing down insertions, $s_{slow}$. The memory node reaching this threshold will enter the index resizing process. The other threshold is the size when the memory node stops any following insertions $s_{stop}$ even if the index resizing is not finished and $s_{stop}>s_{slow}$. 
We set $s_{slow}$ as the load factor of the DMPH table becomes 97\%, or the overflowed cache is filled with half of the size. $s_{slow}$ is set when the overflowed cache is filled with over 90\% space.}


\red{As shown in Fig.~\ref{fig:resize}, when \ding{182} the overflowed cache size reaches $s_{slow}$ after an \texttt{Insert} request from a compute node, \ding{183} the memory node will return the status \texttt{PRE\_RESIZE} to the compute node, and the compute node will create a new connection manager for preparing and listening to build a one-sided RDMA connection with the memory node. 
The memory node will return \texttt{PRE\_RESIZE} to the data requests for all compute nodes in this shard and count up the number of compute nodes that got the information. After all the compute nodes get it or the overflowed cache size reaches $s_{stop}$, The memory node will build the one-sided RDMA connection (RC) with all compute nodes.
The registered memory area in the memory node consists of five fields: (1) The value of the first eight bytes $N_{cNode}$ indicates the number of compute nodes in this shard, but it is set to zero at the beginning to indicate that the new index has not been completely reconstructed. After it finishes, the value will be set to the number of compute nodes in this shard; (2) the second value of the following eight bytes $len$ refers to the total length of the newly written bucket locator arrays and seeds array; (3) $Global_d$ refers Global depth~\cite{dash} value in current extendible hashing; (4) newly computed seeds array; and (5) bucket locator arrays $A$ and $B$.}

\red{On the compute node, once a connection is established with the memory node, it continuously sends RDMA\_READ requests to retrieve the first two values $N_{cNode}$ and $len$ in the registered memory of the memory node. If $N_{cNode}$ is greater than zero, that means the bucket locator arrays and the seeds array have been successfully constructed and written into the memory area. \ding{184} The compute node then issues another RDMA\_READ requests to fetch all the subsequent $len$ data. Additionally, an atomic primitive of fetch-and-add \texttt{FAA} is executed to decrement $N_{cNode}$ by one, signifying the completion of a compute node fetching the new index data.}


\begin{figure}[!t]
    \centering
    \includegraphics[width=0.93\linewidth]{Figures/resize.pdf}
    \vspace{-2.ex}
    \caption{Index resizing in \sys.}
  \vspace{-4ex}
    \label{fig:resize}
\end{figure}


\red{Before the new bucket locator and seeds array is constructed, upon receiving an \texttt{Insert} or \texttt{Delete} request, the memory node returns a \texttt{FALSE} status to compute nodes. Then, the memory node caches the \texttt{Insert}/\texttt{Delete} requests and implements them later after the index data moves to the new DMPH table. 
For \texttt{Get} and \texttt{Update} requests, the memory node will continue serving it on the stale DMPH table. The reason is that no new data insertion would be implemented during resizing, and the keys' $ind\_bucket$ and $ind\_slot$ will not change.}


\red{Once all compute nodes have obtained the new bucket locator arrays and seeds, $N_{cNode}$ in the memory node becomes zero. The memory node detects this change through periodic checks at a frequency of 2 times a second. It proceeds to discard the bucket locator arrays to free up memory space, as they will remain unchanged until the next MPH resizing. 
The memory node will also delete all moved keys in the stale DMPH table by marking the length field as 0.
Then, the reliable connections with all the compute nodes will be terminated by the memory node, and all the compute nodes shift to use both the DMPH tables with the extendible hashing for processing data requests.}

\red{Note that all hash table-based disaggregated KVS require enlargement and shrinking capacity at runtime. The computation time for the extendible hashing layer is the same for \sys and prior works~\cite{race,dash,farm}. In Section~\ref{sec:eval:resizing}, we will show the influence on \sys throughput during index resizing.}

%\vspace{-3ex}
\subsection{Analysis}
\label{subsec:design:analysis}
In this section, we provide the theoretical analysis of the time complexity of the various data operations, as well as the estimation of the memory cost in both compute nodes and memory nodes.

\noindent\textbf{Time complexity.}
For \texttt{Get} operations, each compute node is tasked with determining locations of the DMPH bucket and slot that stores the address of the requested KV. 
This involves two hash computations, namely $hash_A(k)$ and $hash_B(k)$, to access two bits in the bucket locator arrays. Subsequently, an additional hash computation with the bucket seed is performed to locate the specific slot. Then, the memory node can access the slot without further computation and proceed to read data from the referenced KV block.
\red{By default, we use a (2,4)-Cuckoo hash table~\cite{cuckoo} as a fallback table if no seeds can perfectly hash the four elements. In the worst case, accessing the Cuckoo hash table requires two additional hash computations and at most 8 key checks, resulting in a time complexity of O(1) for operations involving the Cuckoo hash table. Therefore, the worst case complexity remains O(1).}
For both the compute node and the memory node, the data \texttt{Get} operation incurs a small constant time. This time complexity extends to data update and data removal operations.

The only difference in \texttt{Insert} lies in the potential time overhead incurred in finding a new seed for the keys in the bucket. To address this, we have set a maximum number of trying times to 256 (8-bit seed). The reason is that we have not encountered a scenario in which no seed can be found within [0, 255] to separate those four keys without collision. We also have a fallback table (storing the key and the KV block address) to deal with rare cases when a group of keys appears that cannot be distributed into distinct slots by MPH. Statistically, we have observed no buckets that cannot be perfectly hashed with a seed length of 8. Therefore, the time cost associated with data insertion is also constant.

\noindent\textbf{Memory usage.}
In compute nodes, the memory usage is allocated to the bucket locator and bucket seeds. According to Ludo~\cite{ludo}, the bucket locator arrays consume 2.33 bits per key. The 8-bit seed is shared among four keys in a bucket. Assuming there are $n$ KV pairs in a shard, with a load factor of $\epsilon$ for the MPH table, the memory cost in a compute node is calculated as $(2.33+2/\epsilon)n$ bits.

In addition to the underlying KV data, memory nodes allocate memory to encompass the latest bucket seeds, DMPH buckets, and the overflowed cache. Each bucket incurs a cost of 32 bytes, and the cache item contains the full key size and the data address. Given a cache size of $m$ and a cache item size of $c$ bits, the overall space budget (in bits) for indexing in a memory node is $66n/\epsilon+m\cdot c$.



\vspace{-1ex}
\subsection{Discussion}
\label{sec:discussion}
\vspace{-.5ex}
\red{\textbf{General applicability on traditional data structures.}
The design principle of \sys can boost data search in traditional data structures with the capability of serving range queries. Specifically, perfect hashing can boost the search process with one-time hash computation with low memory costs that can be cached in compute nodes. For example, the binary search in B/B+ tree leaf nodes can be replaced by perfect hashing computation by searching a seed for hashing keys in leaf nodes.}
%Furthermore, the hash entry traversing in the chained-based hash table can be sped up by the perfect hashing scheme used in \sys for the first 4 elements in each bucket.}

\textbf{Ship computation to data.}
\sys decouples the process of DMPH into a memory-heavy component at memory nodes and a compute-heavy component at compute nodes and allows them to communicate via RDMA-RPC primitives. However, the memory accessing based on the given $ind\_bucket$ and $ind\_slot$ still needs a weak power computation unit close to data~\cite{ship}. We can apply \sys to another two promising approaches without using two-sided RDMA verbs. 
\begin{itemize}[left=0em]
    \vspace{-1ex}
    \item \textit{Extended RDMA READ verb.} PRISM~\cite{prism} proposes and simulates an extended one-sided RDMA indirect reading verb \texttt{RDMA\_READ} (\texttt{ptr} \textit{addr}, \texttt{size} \textit{len}, \texttt{bool} \textit{indirect}), where \textit{indirect} indicates if RNIC is supposed to read back the data pointed by the \textit{addr}. This embedded one-sided RDMA verb can free the memory node's CPU and offload the memory reading task in \sys to RNICs. The reason is that \sys can get the exact requested data address without potential data probing.
    \item \red{
    \textit{Performance capacity of Outback with hardware accelerators.} In-network computation~\cite{dinc} has gained attention for accelerating data services in distributed systems by offloading tasks to in-network computation devices~\cite{netsha,cxl-anns} such as SmartNICs/DPUs and CXL~\cite{cxl}. The idea of \sys can reduce the computation burden on SmartNICs by employing one round-trip, one-sided RDMA\_READ. For example, a SmartNIC~\cite{smartnic2,strom,bluefield,ringleader} can be placed on the memory node side, and function as an additional computation unit, and indirect data access tasks can be offloaded to it~\cite{smartnic1}. After the compute nodes in \sys issue a one-sided RDMA to read the queried key's slot and retrieve the address from the DMPH buckets, the SmartNIC can read the memory again via the PCIe switch and obtain the queried data through an additional PCIe round trip. The computation and data search tasks offloaded to the SmartNIC can be alleviated with the assistance of DMPH for the least computation burden.
    %\textit{Performance capacity of Outback with hardware accelerators.} In-network computation~\cite{dinc} using hardware accelerators such as SmartNICs/DPUs~\cite{smartnic2,smartnic1,strom} and CXL~\cite{netsha,cxl-anns,cxl}. Memory nodes can offload specific tasks from the main CPU to devices like NVIDIA Bluefield DPUs\cite{bluefield} or Broadcom Stingray SmartNICs~\cite{ringleader}, reducing extra memory copies and CPU involvement. The design of \sys could exploit this by offloading data access tasks on such devices; for example, indirect data access can be handled by SmartNICs. After the compute nodes in \sys issue a one-sided RDMA operation to read the queried key’s slot and get the address from the DMPH buckets, a SmartNIC can perform another memory read via the PCIe switch to retrieve the data using DMA without further computation burdens.
    }
    % \vspace{-1ex}
\end{itemize}

\vspace{-1.5ex}
\noindent\textbf{Shared-nothing architecture.} \sys utilizes a shared-nothing architecture~\cite{tutorial} to prevent the update of cached seeds across compute nodes in different shards. The number of KV pairs in each shard depends on the overall size of the database and the number of shards. A greater number of shards results in fewer KV pairs on each memory node. Consequently, the memory allocation for DMPH seeds and bucket locator on each compute node can be reduced, although additional memory nodes are required. Determining the granularity for sharding KV pairs has always been a tradeoff~\cite{kraska}, and it is recommended to choose the configuration based on the specific application.


%Note that the focus of this work is not to saturate RNIC of the memory node. As the CPU's power has a hard time matching the NIC's bandwidth~\cite{match}, \sys aims to reduce the computation usage on memory nodes and improve the throughput in the era of disaggregation.
\vspace{-2ex}
\section{Performance evaluation}
\label{sec:eval}
\vspace{-1ex}



\begin{figure*}[!t]
\centering
\renewcommand\thesubfigure{}
\subfigure[]{
    \includegraphics[width=0.58\textwidth]{Figures/legend.pdf}}\\
\vspace{-6ex}
\setcounter{subfigure}{0}
\renewcommand\thesubfigure{(\alph{subfigure})}
\subfigure[Workload A.]{
    \label{fig:eval:ycsb:a}
    \includegraphics[width=0.205\textwidth]{Figures/cx6-ycsb-1.pdf}}
\hspace{-2.ex}
\subfigure[Workload B.]{
    \label{fig:eval:ycsb:b}
    \includegraphics[width=0.20\textwidth]{Figures/cx6-ycsb-2.pdf}}
    \hspace{-2.5ex}
\subfigure[Workload C.]{
    \label{fig:eval:ycsb:c}
    \includegraphics[width=0.20\textwidth]{Figures/cx6-ycsb-3.pdf}}
    \hspace{-2.5ex}
\subfigure[Workload D.]{
    \label{fig:eval:ycsb:d}
    \includegraphics[width=0.20\textwidth]{Figures/cx6-ycsb-4.pdf}}
    \hspace{-2.5ex}
\subfigure[Workload F.]{
    \label{fig:eval:ycsb:f}
    \includegraphics[width=0.20\textwidth]{Figures/cx6-ycsb-5.pdf}}
\vspace{-3ex}
\caption{\red{Throughput under YCSB benchmark with single memory node thread with Mellanox CX-6.}}
\vspace{-2.ex}
\label{fig:eval:ycsb}
\end{figure*}

\begin{figure*}[!t]
\centering
\renewcommand\thesubfigure{}
\subfigure[]{
    \includegraphics[width=0.58\textwidth]{Figures/legend.pdf}}\\
\vspace{-6ex}
\setcounter{subfigure}{0}
\renewcommand\thesubfigure{(\alph{subfigure})}
\hspace{-1.ex}
\subfigure[Workload A.]{
    \label{fig:eval:cx3:a}
    \includegraphics[width=0.202\textwidth]{Figures/cx3_ycsb_1.pdf}}%{Figures/ycsb-new-1.pdf}}
\hspace{-1.ex}
\subfigure[Workload B.]{
    \label{fig:eval:cx3:b}
    \includegraphics[width=0.20\textwidth]{Figures/cx3_ycsb_2.pdf}}
    \hspace{-2.ex}
\subfigure[Workload C.]{
    \label{fig:eval:cx3:c}
    \includegraphics[width=0.20\textwidth]{Figures/cx3_ycsb_3.pdf}}
    \hspace{-2.ex}
\subfigure[Workload D.]{
    \label{fig:eval:cx3:d}
    \includegraphics[width=0.20\textwidth]{Figures/cx3_ycsb_4.pdf}}
    \hspace{-2.ex}
\subfigure[Workload F.]{
    \label{fig:eval:cx3:f}
    \includegraphics[width=0.20\textwidth]{Figures/cx3_ycsb_5.pdf}}
\vspace{-3ex}
\caption{Throughput under YCSB benchmark with Mellanox CX-3 RNICs.}
\vspace{-3.ex}
\label{fig:eval:cx3}
\end{figure*}


\subsection{Methodology}
\textbf{Testbed.}
\red{We run experiments in two environments. 1) 6 r650 machines from a public cluster CloudLab~\cite{cloudlab}; each of them is equipped with one Two 36-core Intel Xeon Platinum 8360Y CPU at 2.4GHz, 256 GiB DRAM and one Dual-port Mellanox ConnectX-6 (CX-6) 100 GbE NIC with Driver version as MLNX\_OFED\_LINUX-4.9-5.1.0.0. 
We conduct experiments with two shards, and each shard contains 3 machines. We use one machine as the memory node and the other two as compute nodes.
%To emulate its limited CPU, we use the $\mathtt{taskset}$ command to evenly pin the server threads on cores.
The memory node registers the memory with huge pages to reduce RNIC's page cache misses, which is beneficial for memory-intensive applications~\cite{xtore, race}. On compute nodes, we use two coroutines on each client thread to increase the query efficiency (See analysis in Section.~\ref{sec:eval:coros}).
This is the default experiment environment unless otherwise stated. 
2) 9 r320 machines in CloudLab~\cite{cloudlab}, each of them is equipped with one Xeon E5-2450 CPU (8 cores, 2.1Ghz), 16 GiB DRAM, and one Mellanox MX354A Dual port FDR CX3 adapter. We use 1 machine as the memory node and the other 8 as compute nodes.
We utilize 64-byte RDMA messages for all workloads to encapsulate various operation types (RC READ, UD SEND, and UD RECV), ensuring each request is padded to span two cache lines~\cite{guidelines}. We do not use batching at any layer to minimize the latency in all evaluations. %
%CloudLab machines do not provide a combination of weak CPUs but high-performance NICs. Hence the CPU on the memory node could be more powerful than that in an ideal disaggregated memory system.
}
% for camera-ready
%\textbf{Note that this setting gives advantages to the two-sided baselines (RPC-MICA and RPC-Cluster hashing), because they need strong CPU on memory nodes to run index computation tasks, while \sys does not.}}

\noindent\textbf{Workloads.} 
To evaluate the overall performance of \sys and other baselines, we employ YCSB~\cite{ycsb,ycsbc} workloads along with two diverse real-world datasets~\cite{sosd}. These datasets are 
(1) FB, encompassing a random assortment of Facebook user IDs to analyze patterns within social media interactions; (2) OSM, providing digitized infrastructure footprints from Open Street Map to represent geographical and spatial data usage; 
%(3) WIKI, which captures the timestamps of Wikipedia edits in the form of 64-bit unsigned integers, offering insights into high-volume edit activities; and (4) BOOK, containing data on book popularity from Amazon, indicative of consumer behavior and preferences.
To ensure the datasets reflect general, unsorted data conditions, we shuffle them if initially sorted upon loading.
%, and the default Zipfian request distribution const $\theta$ as 0.99. 
Unless specified, we use 8B keys and 8B address values to configure all workloads like existing schemes~\cite{rolex, learnedindex} for comprehensive evaluations. 
For each run, we precondition the memory node and warm up the database with 64 million KV pairs at first and then issue 10M requests to the benchmark on top of it.


\noindent\textbf{Baselines.}
We develop a prototype of \sys based on RDMA libraries rlib and r2~\cite{drtmh} with over 4000 LoC in C++. We compare \sys with the other three baselines, one is a recently proposed one-sided RDMA scheme, RACE hashing~\cite{race}, which utilizes RDMA RC READs for its operations; The other two are two-sided RDMA schemes that operate on RDMA SENDS/RECVs, differing in their underlying data structures -- MICA~\cite{herd, mica} and Cluster hashing~\cite{drtmr}.


\begin{itemize}[left=0em]
\vspace{-.3ex}
    \item \textbf{RACE hashing.}
    RACE hashing~\cite{race} is a representative one-sided RDMA scheme developed recently. It offloads all data operations to compute nodes to free the memory node CPU with one-sided RDMA primitives. RACE Hashing adopts an RDMA-friendly hash table to combine the overflow bucket for collided keys and the hashed bucket. Thus, all the candidate buckets containing the requested key can be read back together.
    We develop RACE hashing with over 1,400 lines of C++ code, excluding the benchmark part that is shared with other baselines.
    \item \textbf{RDMA RPC-MICA.}
    RPC-MICA is a two-sided RDMA-based scheme with a data structure MICA~\cite{mica,herd}, which is an efficient hopscotch hash table and it has been used in existing two-sided RDMA~\cite{herd,fasst}. The overflowed KV pairs can be stored in the bucket adjacent to its hashed bucket.
    We implement hash computation for the bucket number on the compute node and send the queried key's fingerprint and bucket number to save computation on the memory node. 
    We apply the open-source code from MICA~\cite{mica} in our benchmark, utilizing it as the underlying data structure for the RPC-based approach without batching.
    \item \textbf{RDMA RPC-Cluster hashing.}
    RPC-Cluster hashing is a two-sided RDMA baseline with Cluster hashing, a chained-based hash table with associativity, running on memory nodes~\cite{drtmh,drtmr}.
    %DRTM-R~\cite{drtmr} is one of the state-of-the-art approaches that only leverages one-sided RDMA verbs to process data requests. 
    The overflow keys that are hashed to a full bucket will be put in the linked indirect bucket. Each slot in a bucket includes 14 bits of fingerprint for key comparison. We apply the open-source code~\cite{drtm_code} of the cluster hashing as the data backend of our RPC-based scheme suit. 
    % We offload the bucket and fingerprint computation on the compute node for data lookup, and the memory node will complete the following tasks (e.g., fingerprint check and probing) in the experiments.
    \vspace{-1ex}
\end{itemize}







%%%%%%%%%%%%%%%%%%%%%%%% MAIN YCSB
\vspace{-1ex}
\subsection{Performance on YCSB}
\label{sec:eval:ycsb}

%In this section, we compare the aggregated throughput from memory nodes to compute nodes using \sys and other baseline systems with YCSB workloads. We use \texttt{taskset} to bind an equal number of cores to run the benchmark on both compute and memory nodes.

\red{\textbf{Performance with CX-6 RNICs.} We show the throughput of all evaluated methods by increasing the request load of running 8, 12, 20, 72, 108, and 144 compute node threads in a shard.
On the memory node, we consistently allocate only one thread to run on a single core. As shown in Fig~\ref{fig:eval:ycsb}, these five figures illustrate the throughput and latency results under YCSB workloads A, B, C, D, and F, respectively.}
%Note that the YCSB E workload mainly consists of range queries, and this is not the optimization focus of \sys, so we skip the YCSB E part. 



\textbf{\texttt{Get} and \texttt{Update} workloads (YCSB A and B).}
YCSB A and B workloads include 50\% and 5\% data \texttt{Update} respectively and the remaining is \texttt{Get}. 
\sys can achieve 5.50 and 5.82 Mops throughput for YCSB A and B, as shown in Fig.~\ref{fig:eval:ycsb:a} and Fig.~\ref{fig:eval:ycsb:b}. 
All other methods show lower throughput with the same number of threads. 
\sys can provide up to 1.07$\times$ and 1.06$\times$ throughput improvements on workloads A and B respectively, compared to RPC-cluster hashing. 
Compared to other RPC baselines with associative hash tables, the memory node in \sys is offloaded with less computation because it only needs to read the targeted key, and no data probing or traversing is needed to find the targeted value of the key. 
RACE hashing requires three round trips for updating data consistently, significantly increasing the latency and limiting the throughput.
By comparing the results between workloads A and B, 
when more \texttt{Update} requests are issued, \sys spends more computation resources for value rewriting and key checking by reading the underlying KV blocks indicated by the computed MPH slot. 
Hence, \sys under YCSB B provides higher throughput than \sys under YCSB A.


\textbf{\texttt{Get}-only workload (YCSB C).}
For \texttt{Get}-only workload, \sys can achieve 6.01 Mops throughput.
When the number of compute node threads reaches 72, \sys outperforms RACE hashing, MICA, and Cluster hashing by 1.31$\times$, 2.43$\times$, and 1.11$\times$ on total throughput, respectively. The performance of RACE hashing is bottle-necked by its two round trips and the limited RNIC memory to cache queue pair (QP) state of a larger number of reliable connections.
\sys reduces the average memory node's CPU time for data \texttt{Get} request with less computation overhead than the other two RPC-based baselines while looking up a key. 


\textbf{\texttt{Get} and \texttt{Insert} workloads (YCSB D and F).}
YCSB D contains  5\% \texttt{Insert} and 95\% \texttt{Get} operations. 
 YCSB F contains 25\% \texttt{Insert}, 25\% \texttt{Update}, and 50\% \texttt{Get} operations. 
Under YCSB D, \sys still shows the highest throughput among all methods. 
%, which is the most time-consuming operation in \sys. The throughput of \sys drops by 50\% when it comes to YCSB F \todo{compared to YCSB-D is 50\%?}, where 25\% is data insertion and 25\% is data update. 
For \texttt{Insert} operations, \sys will check if a slot in the target bucket is available. Key-checking is also required, and a new seed will be calculated if the target slot stores an existing value. 
%Cache insertion is needed if the hashed bucket is full. 
The high rate of \texttt{Insert} operations in YCSB F pulls the throughput down to 3.62 Mops, which is similar to RPC-Clustering hashing (3.64 Mops) when the number of client threads reaches 144.
%Cluster hashing directly links the value pointers on its hash bucket, which provides a fast way to insert data.


\red{\textbf{Performance with CX-3 RNICs.}
%\red{To evaluate \sys performance on RDMA NICs with diverse bandwidths and the varied scalability of the number of connections. We also run experiments on 9 r320 machines CloudLab~\cite{cloudlab}, each of them is equipped with one Xeon E5-2450 CPU (8 cores, 2.1Ghz), 16 GiB DRAM, and one Mellanox MX354A Dual port FDR CX3 adapter. We use 1 machine as the memory node and the other 8 as compute nodes.}
As shown in Fig.~\ref{fig:eval:cx3}, we show the throughput with the 4 memory node threads and a set of compute node threads numbers 8, 16, 24, 32, 48, and 64, respectively. 
\sys can consistently achieve the highest throughput for read-intensive workloads (A, B, C, and D). 
Significantly, \sys outperforms RACE hashing, MICA, and Cluster hashing by 5.03$\times$, 1.79$\times$, and 1.23$\times$ on total throughput for workload C, respectively. 
When we use a weaker CPU, the advantage of \sys is more significant. Unfortunately, CloudLab does not offer a weaker CPU with a high-performance network. }
%The throughput gap shown with two different network devices explains the factor that \sys can reduce the CPU burden of memory nodes.}



In summary, \sys demonstrates the highest throughput for most types of workload (YCSB A, B, C, and D). For a workload that is \texttt{Insert}-intensive such as YCSB F, \sys provides comparable throughput to other RDMA-RPC methods but still higher than that of one-sided RDMA.









%%%%%%%%%%%%%%%%%%%%%%% SOSD
\vspace{-2ex}
\subsection{\red{Evaluations on Real-World Datasets}}
\label{sec:eval:sosd}

\begin{figure}[!t]
\centering
\renewcommand\thesubfigure{}
\subfigure[]{
    \includegraphics[width=0.48\textwidth]{Figures/legend.pdf}}\\
\vspace{-5ex}
\setcounter{subfigure}{0}
\renewcommand\thesubfigure{(\alph{subfigure})}
\subfigure[Dataset FB, uniform workload.]{
    \label{fig:eval:sosd:a}
    \includegraphics[width=0.23\textwidth]{Figures/cx6-fb-unif.pdf}}
\subfigure[Dataset OSM, uniform workload.]{
    \label{fig:eval:sosd:b}
    \includegraphics[width=0.23\textwidth]{Figures/cx6-osm-unif.pdf}}\\
\vspace{-2.5ex}
\subfigure[Dataset FB, zipfian workload.]{
    \label{fig:eval:sosd:c}
    \includegraphics[width=0.23\textwidth]{Figures/cx6-fb-zipf.pdf}}
\subfigure[Dataset OSM, uniform workload.]{
    \label{fig:eval:sosd:d}
    \includegraphics[width=0.23\textwidth]{Figures/cx6-osm-zipf.pdf}}
\vspace{-2.5ex}
\caption{\red{Data \texttt{Get} throughput performance with SOSD datasets with uniform and zipfian-0.99 workloads.}}
\label{fig:eval:sosd}
\vspace{-3.5ex}
\end{figure}

\red{We leverage the SOSD datasets~\cite{sosd} for evaluations. Fig.~\ref{fig:eval:sosd} illustrates throughput results with the number of compute node threads as 8, 12, 20, 72, 108, and 144 in a shard. We set the number of memory node threads to 1. Each compute node thread issues 10 million key lookup requests selected from the datasets in a uniform or zipfian distribution.} 

\red{Compared to RACE, \sys achieves throughput of 1.38$\times$, 1.35$\times$, 1.39$\times$, and 1.38$\times$ respectively on these four different settings when the number of threads reaches 144.
RACE's performance is constrained by the multiple round trips.
%and the high number of reliable connections between numerous compute node threads and the memory node. \sys leverages the RPC-based approach and fully uses the single core on the memory node. Data accessing from local memory is faster than using one more networking round trip like RACE hashing.
Compared to RPC-MICA and Cluster hashing, \sys achieves a throughput of 2.03$\times$ and 1.1$\times$ respectively on dataset FB when the threads number reaches 144 in Fig.~\ref{fig:eval:sosd:a}.
%MICA designed a hash table with multiple candidate buckets for each key and accommodated eight keys in each bucket. Cluster hashing links the key-offset pairs in the same bucket with value pointers, but the fingerprint (lossy incarnation) and key check are required before accessing data. 
The reason that \sys can outperform them is that \sys can go directly to access data without extra check computation and indirect data accessing to probe the hash chain or buckets.
Also, \sys outperforms RACE hashing, RPC-MICA and RPC-CLuster hashing by 1.35$\times$, 2.05$\times$, and 1.13$\times$ respectively on dataset FB when the workload follows the Zipfian distribution, as shown in Fig.~\ref{fig:eval:sosd:c}. We observe the same trend in performance comparison with the dataset OSM.}









%%%%%%%%%%%%%%%%%% MEM THREADS

%\todo{add the statement of evaluating concurrency for insert/update/delete, and for resizing}

\vspace{-2ex}
\subsection{\red{Scalability with memory node threads}}
\label{sec:eval:cores}

\red{In this set of experiments, we vary the number of memory node threads from 1 to 3 and observe the throughput of different methods using real-world datasets FB and OSM. 
To exhaust the CPU resources on the memory node side, we use four r650 servers as compute nodes with 288 compute node threads.}

\red{Fig.~\ref{fig:eval:cores} shows the throughput of three RDMA-RPC schemes, by varying the memory node threads from 1 to 3. 
The throughput of \sys is around 1.10-1.21$\times$ of Cluster hashing and around 3$\times$ of MICA for dataset FB. 
The results of the two datasets exhibit the fact that as the number of compute node threads increases, the performance ratio between \sys and RPC-Cluster hashing/MICA remains similar.
The reason is that \sys can ease the CPU burden on the memory node and allow it to handle more data requests from the compute node threads by offloading the computation of indexing to compute nodes.}

\begin{figure}[!t]
\centering
\renewcommand\thesubfigure{}
\hspace{1.1ex}
\subfigure[]{
    \includegraphics[width=0.32\textwidth]{Figures/cx6_server_bar.pdf}}\\
\vspace{-5.5ex}
\setcounter{subfigure}{0}
\renewcommand\thesubfigure{(\alph{subfigure})}
\subfigure[Scalability with memory node threads on dataset FB.]{
    \label{fig:eval:cores:a}
    \includegraphics[width=0.232\textwidth]{Figures/cx6_server_threads_1.pdf}}
\hspace{-1.2ex}
\subfigure[Scalability with memory node threads on dataset OSM.]{
    \label{fig:eval:cores:b}
    \includegraphics[width=0.232\textwidth]{Figures/cx6_server_threads_2.pdf}}
\vspace{-2.5ex}
\caption{\red{Throughput vs. the number of memory node threads.}}
\vspace{-3.5ex}
\label{fig:eval:cores}
\end{figure}
\red{The fact that \sys achieves higher relative throughput to other RPC methods under a small number of memory node threads actually demonstrates the main advantage of \sys: achieving high performance when the memory node carries weak CPU power in a disaggregated memory system.}
%From the other side, MICA and Cluster hashing need more CPU to achieve the same throughput as \sys does, especially when the CPU resource is the bottleneck on the memory node in disaggregated systems.

\textbf{\red{Note that the aim of \sys is not to saturate RNIC but to increase the throughput when there are limited CPU resources in a memory node with two-sided RDMA primitives. 
The results in this section show that \sys can achieve higher CPU efficiency with the same throughput goal, and \sys can realize higher throughput with the same CPU resources. }
In disaggregated systems, this can motivate the industry to satisfy the user's throughput goal with less TCO by reducing the CPU resources equipped on memory-optimized cloud instances~\cite{ec2}.}









%%%%%%%%%%%%%%%%%% COROUTINES
\vspace{-1.5ex}
\subsection{\red{Influence of the number of coroutines}}
\label{sec:eval:coros}

\begin{figure}[!t]
\centering
\renewcommand\thesubfigure{}
\hspace{1.1ex}
\subfigure[]{
    \includegraphics[width=0.42\textwidth]{Figures/cx6_coros_bar.pdf}}\\
\vspace{-5.5ex}
\setcounter{subfigure}{0}
\renewcommand\thesubfigure{(\alph{subfigure})}
\subfigure[Latency-throughput curve on YCSB-C with 1 memory node thread.]{
    \label{fig:eval:coros:a}
    \includegraphics[width=0.235\textwidth]{Figures/cx6_coros_1.pdf}}
\hspace{-1.5ex}
\subfigure[Latency-throughput curve on YCSB-C with 2 memory node threads.]{
    \label{fig:eval:coros:b}
    \includegraphics[width=0.235\textwidth]{Figures/cx6_coros_2.pdf}}
\vspace{-3ex}
\caption{\red{Latency vs. the number of coroutines.}}
\vspace{-4ex}
\label{fig:eval:coros}
\end{figure}

\red{The coroutines within compute node threads are designed to yield upon dispatching a request and resume operation upon receiving responses from two-sided RPCs. The default setup of \sys uses two coroutines per thread, but we extend our evaluation to explore the influence of one or more per thread to ascertain the optimal configuration for maximizing server CPU utilization.
Fig.~\ref{fig:eval:coros} studies the latency-throughput performance of \sys in YCSB-C workload with different numbers of coroutines in a compute node thread. 
In Fig.~\ref{fig:eval:cores:a}, we have only one worker thread in the memory node and vary the total of compute node threads as 8,20,72,144 and 216 distributed among three compute nodes, respectively. 
We can observe that a larger number of coroutines results in higher throughput when the number of compute node threads is less than 72, and the latency doubles or triples after the throughput reaches around 6 Mops, the maximum throughput one memory thread can support. 
This phenomenon is similar when the number of memory node threads is 2, as shown in Fig.~\ref{fig:eval:cores:b}, because the CPU resource on the memory node can handle 144 compute node threads, and the total throughput of a memory node can reach to 9.89 Mops. 
However, the extra coroutines will incur high latency of the data query after the number of memory node threads becomes a bottleneck for serving 216 threads.}





%%%%%%%%%%%%%%%%%% LOAD FACTOR
\vspace{-1.5ex}
\subsection{Influence of load factor in DMPH}
\label{sec:eval:lf}

\begin{figure}[!t]
    %\centering
    \renewcommand\thesubfigure{}
    \begin{minipage}[t]{0.232\textwidth}
        \subfigure[]{
            \label{fig:eval:lf}
            \includegraphics[width=\textwidth]{Figures/cx6_load_factor.pdf}}
        \vspace{-7ex}
        \caption{Influence of different load factor set in DMPH.}
    \end{minipage}
    \hspace{-1.5ex}
    \begin{minipage}[t]{0.232\textwidth}
        \subfigure[]{
            \label{fig:eval:kvs}
            \includegraphics[width=\textwidth]{Figures/cx6_number_kvs.pdf}}
        \vspace{-7ex}
        \caption{Influence of the varied number of KV pairs.}
    \end{minipage}
    \vspace{-3ex}
\end{figure}

\red{The load factor in a hash table is the ratio of stored elements to the total number of available slots or buckets. Maintaining an optimal load factor balances memory usage and data operation throughput.
We evaluate the data \texttt{Get} throughput in \sys with varied load factors from 0.75 to 0.95.}

\red{As shown in Fig.~\ref{fig:eval:lf}, \sys can achieve around 6 Mops with 72 data query threads from compute nodes in a shard for the dataset FB.
Similarly, the influence of the varied load factors on the throughput is trivial based on the results of the dataset OSM.}



%%%%%%%%%%%%%%%%%% KVS
\vspace{-1.5ex}
\subsection{Influence of the number of KV pairs}
\label{sec:eval:kvs}
\red{Fig.~\ref{fig:eval:kvs} studies the impact of the number of KV pairs in each shard. We load 20M, 50M, and 80M KV pairs in \sys and evaluate the data \texttt{Get} throughput on two real-world datasets, respectively. 
\sys's read throughput decreases from 6.02 to 5.83 Mops as database size enlarges on the dataset FB. Similarly, we can observe the data read throughput decreases by 3.1\% on the dataset OSM.}









%%%%%%%%%%%%%%%%%% MEM
\vspace{-1.5ex}
\subsection{Memory usage in compute nodes}
\label{sec:eval:mem}

\begin{figure}[t]
\centering
\hspace{-.5ex}
    \includegraphics[width=0.45\textwidth]{Figures/cx6_memory_usage.pdf}
    \vspace{-3ex}
    \caption{Memory usage on compute node with the varied number of KV pairs.}
    \label{fig:eval:mem}
    \vspace{-4ex}
\end{figure}

In a disaggregated memory system, compute nodes are regarded as the ones with rich computing resources but limited memory space. 
To make the memory node serve data requests with the least computation based on RDMA RPC primitives, we offload as much computation to the compute side with the help of DMPH.
In this section, we evaluate the memory cost of \sys on each compute node with the varied number of KV pairs in each shard. The memory usage on a compute node consists of the bucket locator and the seeds array.

As shown in Fig.~\ref{fig:eval:mem}, we vary the load factor used in the DMPH table from 0.80 to 0.95, and we use an 8-bit seed for keys in each bucket. The memory usage at each compute node for 20 million KV pairs per shard is around 12.5MB, and the cost is below 60MB for 100M KV pairs per shard.  
%which is below 60MB for 100 million KV pairs. 
 %, thus the memory ratio of the compute node cached over the memory node index is 6.04\% when the loaf factor is 0.8.
%The actual number of KV pairs in each compute node depends on the total number of KV pairs and the shards. The more shards there are, the less memory space we need to spend on metadata maintenance. It is worthwhile to allocate the specified memory budget to achieve fast data access of memory nodes in \sys compared to learned index-based schemes. 
This is considered a small overhead because recent one-sided RDMA solutions cost hundreds of MBs or more on each compute node for index caching and other purposes~\cite{rolex,xtore}. 
For example, in XStore~\cite{xtore}, 100 million key-value pairs require over 600MB of memory at a compute node without including the cache.










%%%%%%%%%%%%%%%%%% RESIZING
\vspace{-1.5ex}
\subsection{Throughput during index resizing}
\label{sec:eval:resizing}

\begin{figure}[t]
\centering
    \includegraphics[width=0.45\textwidth]{Figures/resizing.pdf}
    \vspace{-3ex}
    \caption{\red{Influence of extendible hashing resizing.}}
    \label{fig:eval:resizing}
    \vspace{-4ex}
\end{figure}


We evaluate the throughput changes during index reconstruction and resizing. 
%To support extendible hash table resizing, we propose to leverage partial bucket locking to enable uninfluenced bucket accessing during the resizing period. 
\red{
%As the number of KV pairs increases with data insertion, hash table resizing will be triggered and implemented with extendible hashing enlarging. Half of the keys in a DMPH table and the elements in the overflowed cache will be moved to the newly allocated DMPH table. The memory node will construct the corresponding bucket locator and seeds array for the new table. 
In this set of experiments, we bulk-load 20M keys to the database with the initial DMPH table to warm up, and we set one compute node with 8, 12, and 16 threads connecting to the memory node running only one thread, respectively.} 
%The memory node needs to maintain the latest seeds and the overflow cache for new data. 
\red{This emulates a challenging scenario because the memory node has limited computing resources to handle both resizing and lookups. The workload running on compute nodes is YCSB D, which contains 5\% data insert and 95\% read.
As shown in Fig.~\ref{fig:eval:resizing}, it takes around 3 seconds to recalculate the bucket locator and the seed for each bucket. \sys still supports partial \texttt{Get} requests during resizing with a decreased throughput by approximately 52\% with only one thread in the memory node. %to support \texttt{Get} to the uninfluenced buckets. 
The CPU contention causes a performance drop, %when handling data lookup requests and reconstructing the MPH table and seeds in the same memory node thread, which is rare and not suggested in practice.
and the performance goes back to normal after resizing.} 








%%%%%%%%%%%%%%%%%% SUMMARY %%%%%%%%%%%%
% key exp findings + a set of nice parameters
\vspace{-1.5ex}
\subsection{Summary of evaluation}
\label{sec:eval:sum}
\red{
\textbf{Data lookup throughput. }
\sys achieves 1.11-2.43$\times$ and 1.23-5.03$\times$ higher throughput than baselines with Mellanox CX-6 100Gb and CX-3 50Gb RNICs in data search workload, respectively.}

\red{\noindent\textbf{Memory usage. }
The memory usage at each compute node for 20 million KV pairs per shard is around 12.5MB per shard, around 5 bits per key, with a load factor of 0.85 in DMPH.}

\red{\noindent\textbf{Scalability of memory node threads.}
When the compute nodes with enough threads exhaust the compute capability on the memory node, \sys can achieve at least 18\% performance advantage over other RPC-based baselines on read workload.}

\red{\noindent\textbf{Load factors in \sys.}
The load factor value in DMPH causes a trivial impact on data lookup throughput with the same compute complexity. We recommend 0.8-0.9 to achieve the balance between memory usage and low frequent resizing, as the low load factor supports more incremental data insertion into the hash table.}



\input{06_Related_work}
Game-based approaches have shown great promise as tools for inoculating individuals against the tactics commonly used to spread misinformation. Most existing games in this domain are single-player games which offer players limited, predefined choices. While this design reduces cognitive load, it often results in interactions which feel less natural and engaging. In response, we designed a two-player, PvP game that pits a misinformation creator against a misinformation stopper. By integrating LLM-powered personas to evaluate player outputs and provide real-time feedback, we created a more open-ended and immersive experience.
We found that the game we developed effectively improved players’ media literacy. Participants demonstrated an enhanced ability to evaluate and analyze media content, identify unreliable or misleading information, and employ effective counter-misinformation strategies. Moreover, the game's engaging mechanics, combined with the competitive element, motivated players to learn from both their own strategies and those of their opponents.
These findings suggest that integrating dynamic feedback systems and competitive gameplay elements into misinformation education games offers a compelling method to deepen users' engagement, while also improving their critical media skills. Future research can build on these insights to explore other forms of interactive learning environments, focusing on diverse player experiences and varying misinformation challenges.

\vspace{-2ex}
\begin{acks}
\vspace{-1ex}
We thank our three anonymous reviewers for their insightful suggestions and comments.
This research was supported by the IAB members of the Center for Research in Systems and Storage (CRSS), and the National Science Foundation (NSF) under grants CNS-1841545, CCF-1942754, CNS-2322919, CNS-2420632, CNS-2426031, and CNS-2426940. The views expressed are those of the authors and do not necessarily reflect those of the funding agencies.
\end{acks}

\balance
\bibliographystyle{ACM-Reference-Format}
\bibliography{sample-base}

% \input{08_Response}

\end{document}



