\vspace{-1.5ex}
\section{Conclusion}
\label{sec:conclusion}
\vspace{-.5ex}
This paper introduces \sys, an RDMA RPC-based index for key-value stores on disaggregated memory, designed to achieve high throughput with lower CPU utilization. 
The key innovation of \sys is the division of the data index into two distinct components: a compute-intensive component cached on compute nodes and a memory-intensive component residing on memory nodes. The performance improvements stem from the memory node's ability to access underlying data with minimal computational overhead with perfect hashing.
We also design protocols for \sys that support data operations and index resizing using extendible hashing, ensuring both the correctness of operations and system consistency during updates.
We conduct extensive experiments to evaluate the performance of \sys. The results show that \sys achieves higher throughput and requires smaller memory space on compute nodes, compared to the state-of-the-art baselines under most types of workload, especially for $\mathtt{Get}$-heavy workload. 
%When the memory nodes have limited CPU resources, the throughput advantage of \sys is more evident.







\iffalse
It successfully resolves the limitations of existing which leverages minimal perfect hashing to decouple data lookup computation for disaggregated KVS based on two-sided RDMA primitives with limited remote CPU resources.
Unlike the traditional RPC-based approaches, our scheme making memory nodes delivers high throughput by offloading as much computation of the data lookup service as possible to compute nodes. The memory nodes focus on memory access without extra computation. 
\sys can also be regarded as an effective tradeoff between one-sided RDMA and RPC-based schemes.
Moreover, we explored the in-place insert for MPH buckets to reduce times for hash table resizing and try to use the MPH space fully. We also designed the partial buckets lock scheme and collaborated seeds fetching to support concurrent data lookup during resizing. Extensive experimental results show that \sys outperforms other baselines (both one-sided RDMA or RPC-based) by up to 1.53$\times$ in full read workload with limited memory node CPU resources and less memory usage on the client side.
\fi