\section{RELATED WORK}
\subsection{Scene Reconstruction}
The advent of NeRF~\citep{nerf} has ushered in a golden age for 3D scene construction. Numerous studies have improved its efficiency~\citep{hedman2021snerg, instantngp, Reiser2023SIGGRAPH} and generalization~\citep{yu2021pixelnerf, wang2021ibrnet, chen2021mvsnerf}. Mip-NeRF~\citep{barron2021mipnerf} and Zip-NeRF~\citep{barron2023zipnerf} have tackled aliasing issues, while InstantNGP~\citep{instantngp} integrates grid pyramid technologies to optimize sub-volumes. UC-NeRF~\citep{ucnerf} targets outdoor scenes, enhancing image consistency through color correction and pose refinement. StreetSurf~\citep{guo2023streetsurf} and EmerNeRF~\citep{yang2023emernerf} introduce novel approaches for multi-view reconstructions through disentanglement and self-guided learning. Concurrently, PVG~\citep{chen2024periodic} utilizes 3DGS~\citep{kerbl3Dgaussians} to advance scene reconstruction with techniques like time-dependent transparency and scene-flow smoothing.

Furthermore, to extend reconstruction techniques to larger-scale scenes, methods like Block-NeRF~\citep{tancik2022blocknerf}, Mega-NeRF~\citep{meganerf}, and Switch-NeRF~\citep{mi2023switchnerf} employed a divide-and-conquer strategy. Mega-NeRF clustered pixels based on 3D sampling distances, Block-NeRF organized images into street blocks, and Switch-NeRF utilized a sparse network for large scene synthesis. These methods improved scalability and flexibility but faced limitations in real-time rendering of large outdoor environments. VastGaussian~\citep{lin2024vastgaussian} incorporated 3DGS to enhance detail presentation and rendering speed in large scenes. While these methods have advanced scene reconstruction, they typically rely on precise camera poses and initial data like LiDAR, which can be difficult to obtain in real-world applications, especially in expansive outdoor settings.

\subsection{Pose Optimization}
To address issues with pose accuracy, many studies seek to bypass the slow and occasionally imprecise COLMAP process by concurrently optimizing camera pose and scene representation using the original MLP-based NeRF \citep{wang2021nerfmm}, such as GARF~\citep{chng2022gaussian} and BARF~\citep{lin2021barf}. Joint-TensoRF\citep{Joint-TensoRF} focuses on refining camera poses and 3D scenes using decomposed low-rank tensors. These methods have been proven effective in recovering object structures and poses from imperfect or unknown camera positions, although their application to broader scene reconstruction remains challenging. In 3DGS, COLMAP~\citep{schoenberger2016sfm,schoenberger2016mvs} is utilized for pose reconstruction and generating sparse initial points via Structure-from-Motion (SfM)~\citep{schoenberger2016sfm}. However, COLMAP's dense matching time increases exponentially with the number of images, and its success rate is limited, which poses significant challenges for outdoor scene reconstruction.

Building on the discussions above, this paper is dedicated to proposing a low-pose-requirement, rapid, high-precision 3D reconstruction method suitable for both general and large scenes.