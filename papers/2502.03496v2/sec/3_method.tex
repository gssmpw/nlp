\section{Method}
\label{sec:method}

\textbf{FreqPrior} comprises three key stages: \textbf{sampling process}, \textbf{diffusion process}, and \textbf{noise refinement}, as shown in Figure~\ref{fig:pipeline}. 
To obtain a new noise prior, our method starts with Gaussian noise, which then goes through these three stages sequentially, 
repeated several times, to result in a refined noise prior. 
Once the new prior is established, it serves as the initial latent for video diffusion models to generate a video.
The {\bf sampling process} in our framework is DDIM sampling~\citep{song2021ddim}.

\begin{figure}
    \centering
    \includegraphics[width=\linewidth]{figure/pipeline.pdf}
    \vspace{-5pt}
    \caption{The framework of \textbf{FreqPrior}. It consists of three stages: \textbf{sampling process}, \textbf{diffusion process}, and \textbf{and noise refinement}. 
    In the noise refinement stage, the noise is refined in three steps including \textbf{noise preparation}, \textbf{noise processing}, and \textbf{post-processing}. }
    \label{fig:pipeline}
\end{figure} 
 

\subsection{Diffusion process}
During the sampling process, the latent becomes clean.
Unlike the conventional diffusion process that typically diffuses the clean latent to timestep $T$, our approach perturbs the latent with the initial noise $\epsilon$ once sampling reaches a specific intermediate timestep, denoted as $t$.
The diffusion process can be formulated as follows by leveraging the Markov property:
\begin{equation}
    \bfz_{noise}^i = \sqrt{\frac{\bar{\alpha}_T}{\bar{\alpha}_t}}\bfz_t^i+\sqrt{1-\frac{\bar{\alpha}_T}{\bar{\alpha}_t}}\epsilon,
\end{equation}
where $\{\bar\alpha_j\}_{j=0}^T$ are the notations corresponding to the diffusion scheduler~\citep{ho2020denoising}, and $i$ represents the $i$-th iteration.

The rationale for conducting the diffusion process beforehand stems from the observation that when $t$ reaches about timestep $400$, the latent $\bfz_t^i$ has roughly taken shape and resembles the clean latent $\bfz_0^i$, indicating the latent already has recovered large low-frequency information.
Consequently, this modification yields nearly identical outcomes compared to diffusing a pure clean latent.
This modification offers a notable advantage in terms of efficiency, as it significantly reduces the number of required sampling steps while maintaining consistent results. 
Therefore, we achieve substantial time savings without compromising the fidelity of our results.

\subsection{Noise refinement}
\label{subsec:noise_refinement}
The \textbf{noise refinement} stage focuses on processing different frequency components of the noise to improve video generation quality. 
Low-frequency signals help the model generate videos with better semantics, while high-frequency signals contribute to finer image details.
Unlike conventional filtering methods, which typically target signals like images, our approach processes noise, essentially random variables, distinguishing it from traditional techniques.
Therefore, we propose a novel frequency filtering method designed to effectively handle noise, enhancing overall quality.

\paragraph{Step 1: Preparation of two sets of noise} 
We begin by preparing two distinct sets of noise, each serving a specific purpose: one to convey low-frequency information and the other to provide high-frequency information.
Initially, we independently sample from a standard Gaussian distribution to obtain $\eta_1^i$, $\eta_2^i$, $\bfy_1^i$ and $\bfy_2^i$, where $\bfy_1^i$ and $\bfy_2^i$ correspond to high-frequency information. 
As for low-frequency information,
we combine $\bfz_{noise}^i$ with $\eta_1^i$ and $\eta_2^i$ to yield $\bfx_1^i$ and $\bfx_2^i$ as follows:
\begin{equation}
\label{eq:method_X}
\begin{split}
    \bfx_1^i &= \frac{1}{\sqrt{1+\cos^2\theta}}\left(\cos\theta\cdot \bfz_{noise}^i+\sin\theta\cdot\eta_1^i\right),\quad \eta_1^i\sim \mathcal{N}(\mathbf{0},\mI), \\
    \bfx_2^i &= \frac{1}{\sqrt{1+\cos^2\theta}}\left(\cos\theta\cdot \bfz_{noise}^i+\sin\theta\cdot\eta_2^i\right),\quad \eta_2^i\sim \mathcal{N}(\mathbf{0},\mI).
\end{split}
\end{equation}
Here, ratio $\cos\theta$ controls the proportion of $\bfz_{noise}^i$ contained within $\bfx_1^i$ and $\bfx_2^i$. It adds flexibility to the framework, allowing us to control the amount of low-frequency information derived from $\bfz_{noise}^i$.

\paragraph{Step 2: Retention of low-frequency signals while enriching high-frequency signals}
We apply the Fourier transform to map the noise to the frequency domain:
\begin{equation}
    \Tilde{\bfx}_1^i=\mathcal{F}_{3D}(\bfx_1^i),\quad 
    \Tilde{\bfx}_2^i=\mathcal{F}_{3D}(\bfx_2^i),\quad 
    \Tilde{\bfy}_1^i=\mathcal{F}_{3D}(\bfy_1^i),\quad
    \Tilde{\bfy}_2^i=\mathcal{F}_{3D}(\bfy_2^i),
\end{equation}
where $\mathcal{F}_{3D}$ represents the Fourier transform operation on temporal and spatial dimensions.
We then perform filtering with a low-pass filter $\mathcal{M}$:
\begin{equation}
    \Tilde{\bfz}_1^i=\mathcal{M}\odot\Tilde{\bfx}_1^i + (\bm{1}-\mathcal{M}^2)^{0.5}\odot\Tilde{\bfy}_1^i,
    \quad
    \Tilde{\bfz}_2^i=\mathcal{M}\odot\Tilde{\bfx}_2^i + (\bm{1}-\mathcal{M}^2)^{0.5}\odot\Tilde{\bfy}_2^i.
\end{equation}
Since we are filtering Gaussian variables rather than real image signals, the conventional filtering approach may not be suitable. Typically, a high-pass filter is set to $(\bm{1}-\mathcal{M})$, we use $(\bm{1}-\mathcal{M}^2)^{0.5}$ instead. 
This adjustment is inspired by a fact in probability: if $\mathbf{u},\;\mathbf{v} \sim\mathcal{N}(\mathbf{0}, \mI)$ are independent, then for $m\in\left[0, 1\right]$, it holds that $\mathbf{w} = m \cdot \mathbf{u} + (1-m^2)^{0.5} \cdot \mathbf{v}$
is also standard Gaussian.
In traditional filtering operations, the sum of the low-pass and high-pass filters equals one. However, in our approach, the sum of the squares of the low-pass and high-pass filters equals one.
This modification enriches the high-frequency signals, maintaining the balance between low-frequency and high-frequency components. As a result, it mitigates the loss of details and motion dynamics, leading to higher fidelity in the generated videos.

\paragraph{Step 3: Post-processing} After filtering, the frequency features are mapped back into the latent space, followed by post-processing to form the new noise prior $\bfz_T^{i+1}$. The process is as follows:
\begin{equation}
    \bfz_T^{i+1} = \frac{1}{\sqrt{2}}\left(\Re\left(\bfz_{T,1}^i\right)+\Im\left(\bfz_{T,1}^i\right)+\Re\left(\bfz_{T,2}^i\right)-\Im\left(\bfz_{T,2}^i\right)\right),\quad \bfz_{T, \left\{1, 2\right\}}^i=\mathcal{F}_{3D}^{-1}(\Tilde{\bfz}_{\left\{1, 2\right\}}^i).
\end{equation}
Unlike traditional methods that overlook the imaginary component, our approach recognizes the importance of the information contained within these imaginary parts, which are crucial for preserving the variance in the noise prior.
Consequently, we retain both the real and imaginary components. Specifically, we take both the positive real parts of $\bfz_{T,1}^i$ and $\bfz_{T,2}^i$, but for imaginary components, we take the positive imaginary part of $\bfz_{T,1}^i$ and the negative imaginary part of $\bfz_{T,2}^i$. This is the reason we prepare two sets of noise in \textbf{Step 1}. 
This symmetric formulation enhances the retention of valuable information while effectively eliminating unnecessary and complex terms.

In summary, our framework comprises two phases: 
the first phase focuses on finding a new noise prior, while the second phase generates a video based on that prior.
The process of finding the noise prior includes the sampling process, diffusion process, and noise refinement, as previously discussed. 
Our framework is detailed in Algorithm~\ref{alg:FreqPrior}.

\algnewcommand{\LineComment}[1]{\State \(\triangleright\) #1}
\begin{algorithm}[!t]
   \caption{FreqPrior}
   \label{alg:FreqPrior}
    \begin{algorithmic}[1]
        \Require  
          \Statex $T$: total diffusion step; $t$: middle timestep; $\{\alpha\}_{t=0}^T$: scheduler. $n$: number of iterations.
       \State{Initialize $\bfz_T = \epsilon$, where $\epsilon \sim \mathcal{N}(\mathbf{0},\mI)$}.
       \LineComment{\textbf{\textit{Obtain the noise prior}}}
       \For{$i=0$ {\bfseries to} $n$}
        \State $\bfz_t \gets \operatorname{Sampling}(\bfz_{T})$
        \Comment{Partial sampling process}
        \State $\bfz_{noise}=\sqrt{{\bar{\alpha}_T}/{\bar{\alpha}_t}}\cdot\bfz_t+\sqrt{1-{\bar{\alpha}_T}/{\bar{\alpha}_t}}\cdot\epsilon$
        \Comment{Diffusion Process}
        \State $\bfz_T \gets \operatorname{NoiseRefine}(\bfz_{noise})$
        \Comment{Noise refinement}
       \EndFor
       \LineComment{\textbf{\textit{Generate a video from new noise prior}}}
       \State $\bfz_0 \gets \operatorname{Sampling}(\bfz_T)$   
       \Comment{Sampling process}
       \State $\mathrm{video} \gets \operatorname{Decode}(\bfz_0)$
       \State {\bfseries return} $\mathrm{video}$
    \end{algorithmic}
\end{algorithm}

\subsection{Analysis on the distribution of different noise prior}
\label{subsec:analysis}
For the mixed noise prior proposed in PYoCo~\citep{ge2023PYoCo}, it is constructed as follows:
\begin{equation}
    \bfz_j = \frac{1}{\sqrt{2}}\epsilon_j +  \frac{1}{\sqrt{2}}\epsilon_{share},\quad \epsilon_j,\;\epsilon_{share}\sim\mathcal{N}(\mathbf{0},\mI),
\end{equation}
where $\bfz_j$ and $\epsilon_j$ represent the $j$-th frame of latent $\bfz$ and Gaussian noise $\epsilon$. 
The noise prior $\bfz$ has correlations in the frame dimension, as each frame consists of shared noise $\epsilon_{shared}$:
\begin{equation}
    \mathrm{Cov}(\bfz_i,\bfz_j) = 0.5\mI,\quad i\ne j.
\end{equation}
Therefore, considering only the frame dimension, the diagonal elements of the covariance matrix are 1, and others are 0.5, which deviates standard Gaussian distribution. 
Similarly, the distribution of progressive noise prior also deviates from standard Gaussian distribution.

To conduct a theoretical analysis for FreeInit~\citep{wu2023freeinit} and our method, we first need to determine the distribution of the refined noise. We begin with the following assumption:
\begin{assumption}
\label{assumption:1}
    After the diffusion process, $\bfz_{noise}$ follows a standard Gaussian distribution $\mathcal{N}(0,I)$.
\end{assumption}
We focus on the frame, height, and width dimensions, as other dimensions do not affect analysis.
The noise prior of FreeInit~\citep{wu2023freeinit} has the following distribution (see Appendix~\ref{appendix:freeinit}):
\begin{equation}
    \bfz \sim \mathcal{N}\left(\mathbf{0}, \mP^2+\left(\mI-\mP\right)^2\right), \quad \mP = \frac{1}{N}\left( \mA\mLambda\mA+\mB\mLambda\mB\right),
\end{equation}
where $\bfz\in\R^{fhw}$ is the vector form of the noise prior, $N$ is the length of $\bfz$, $\mLambda$ is the diagonal matrix corresponding to the low-pass filter $\mathcal{M}$, and $\mA$ and $\mB$ represent real and imaginary parts of 3D Fourier matrix as illustrated in Appendix~\ref{appendix:fourier}.

Similarly, the distribution of our method is as follows (see Appendix~\ref{appendix:freqprior}):
\begin{equation}
    \bfz\sim\mathcal{N}\left(\mathbf{0}, \mI - \frac{2\cos^2\theta}{1+\cos^2\theta}\mQ^2\right), \quad \mQ = \frac{1}{N}\left(\mA\mLambda\mB+\mB\mLambda\mA\right).
\end{equation}
To measure the deviations of two Gaussian distributions, we introduce the concept of covariance error.
\begin{definition}[Covariance error]
\label{definition:1}
    For two Gaussian variables with the same expectations, $\mathcal{N}(\mu,\mSigma_1)$ and $\mathcal{N}(\mu,\mSigma_2)$, the covariance error is defined as the Frobenius norm of the difference between their covariance matrices: $||\mSigma_1-\mSigma_2||_F$.
\end{definition}

Under the condition of the same low-pass filter $\mathcal{M}$, we can derive the relationship of the covariance error of FreeInit and our method by using Equation~\ref{eq:covariance_error_two} and Theorem~\ref{theorem:4}:
\begin{equation}
    ||\mI-\mathbf{\Sigma}_{FreqPrior}||_F 
    \le \frac{\cos^2\theta}{1+\cos^2\theta}||\mI-\mathbf{\Sigma}_{FreeInit}||_F. 
\end{equation}
This inequality indicates that $1- \frac{||\mI-\mathbf{\Sigma}_{FreqPrior}||_F}{||\mI-\mathbf{\Sigma}_{FreeInit}||_F}\ge \frac{1}{1+\cos^2\theta}\ge 50\%$.
This demonstrates that the refined noise produced by our method is closer to a standard Gaussian distribution. 
Our approach can theoretically reduce the covariance error by at least 50\% compared to FreeInit~\citep{wu2023freeinit}. 
To further investigate the covariance error, we conduct numerical experiments with three different shapes and two types of low-pass filters: the Butterworth filter and the Gaussian filter. All computations are performed with $\mathrm{float64}$ precision.

\begin{table}[h]
    \centering
    \caption{\textbf{Numerical experiments on covariance error.} We report the covariance errors for three types of prior under various settings, including three different latent shapes and two different filters. The mixed noise prior is independent of filters.}
    \resizebox{\linewidth}{!}{
    \begin{tabular}{ccccccc}
    \toprule
    \multirow{2}{*}{Prior}  &  \multicolumn{2}{c}{(16, 20, 20)} & \multicolumn{2}{c}{(16, 30, 30)} & \multicolumn{2}{c}{(16, 40, 40)}\\
    \cmidrule[0.25pt](lr){2-3} \cmidrule[0.25pt](lr){4-5} \cmidrule[0.25pt](lr){6-7} 
     & {Butterworth} & Gaussian & {Butterworth} & Gaussian & {Butterworth} & Gaussian \\  
    \midrule
    Mixed          &   \multicolumn{2}{c}{$154.9193$}   & \multicolumn{2}{c}{$232.3790$} & \multicolumn{2}{c}{$309.8387$} \\
    FreeInit       &   $3.8230$    & $8.5878$             & $5.7001$    & $12.8817$             & $7.6026$      & $17.1756$ \\          
    Ours           &   $8.5071\times 10^{-28}$  & $7.7218\times 10^{-28}$  & $1.4002\times 10^{-26}$  & $1.2656\times 10^{-26}$  & $2.7342\times 10^{-26}$  & $2.4140\times 10^{-26}$ \\
    \bottomrule
    \end{tabular}
    }
    \label{tab:numerical}
\end{table}

As illustrated in Table~\ref{tab:numerical}, our proposed noise prior exhibits the lowest covariance errors, which are minimal and can be considered negligible. FreeInit shows some covariance errors, indicating the presence of a variance decay issue. The covariance errors for the mixed noise prior are significantly higher, suggesting that it deviates substantially from a standard Gaussian distribution. These numerical experiments imply that our noise prior can be regarded as a standard Gaussian distribution.
