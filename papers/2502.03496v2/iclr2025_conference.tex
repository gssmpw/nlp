\documentclass{article}
\usepackage{iclr2025_conference,times}

% Optional math commands from https://github.com/goodfeli/dlbook_notation.
%%%%% NEW MATH DEFINITIONS %%%%%

% \usepackage{amsmath,amsfonts,bm}
\usepackage{amsmath,amsfonts}

\usepackage{pifont}


\newcommand{\R}{\mathbb{R}}


\def\va{{\mathbf{a}}}
\def\vg{{\mathbf{g}}}

% Sets
\def\sR{\mathbb{R}}
\def\sC{\mathbb{C}}
\def\sZ{\mathbb{Z}}
\def\sN{\mathbb{N}}
\def\sQ{\mathbb{Q}}

\def\sS{\mathcal{S}}



% Vectors
\def\vzero{{\mathbf{0}}}
\def\vone{{\mathbf{1}}}
\def\vmu{{\mathbf{\mu}}}
\def\vtheta{{\mathbf{\theta}}}
\def\va{{\mathbf{a}}}
\def\vb{{\mathbf{b}}}
\def\vc{{\mathbf{c}}}
\def\vd{{\mathbf{d}}}
\def\ve{{\mathbf{e}}}
\def\vf{{\mathbf{f}}}
\def\vg{{\mathbf{g}}}
\def\vh{{\mathbf{h}}}
\def\vi{{\mathbf{i}}}
\def\vj{{\mathbf{j}}}
\def\vk{{\mathbf{k}}}
\def\vl{{\mathbf{l}}}
\def\vm{{\mathbf{m}}}
\def\vn{{\mathbf{n}}}
\def\vo{{\mathbf{o}}}
\def\vp{{\mathbf{p}}}
\def\vq{{\mathbf{q}}}
\def\vr{{\mathbf{r}}}
\def\vs{{\mathbf{s}}}
\def\vt{{\mathbf{t}}}
\def\vu{{\mathbf{u}}}
\def\vv{{\mathbf{v}}}
\def\vw{{\mathbf{w}}}
\def\vx{{\mathbf{x}}}
\def\vy{{\mathbf{y}}}
\def\vz{{\mathbf{z}}}
\def\vzeta{{\mathbf{\zeta}}}

% Matrix
\def\mA{{\mathbf{A}}}
\def\mB{{\mathbf{B}}}
\def\mC{{\mathbf{C}}}
\def\mD{{\mathbf{D}}}
\def\mE{{\mathbf{E}}}
\def\mF{{\mathbf{F}}}
\def\mG{{\mathbf{G}}}
\def\mH{{\mathbf{H}}}
\def\mI{{\mathbf{I}}}
\def\mJ{{\mathbf{J}}}
\def\mK{{\mathbf{K}}}
\def\mL{{\mathbf{L}}}
\def\mM{{\mathbf{M}}}
\def\mN{{\mathbf{N}}}
\def\mO{{\mathbf{O}}}
\def\mP{{\mathbf{P}}}
\def\mQ{{\mathbf{Q}}}
\def\mR{{\mathbf{R}}}
\def\mS{{\mathbf{S}}}
\def\mT{{\mathbf{T}}}
\def\mU{{\mathbf{U}}}
\def\mV{{\mathbf{V}}}
\def\mW{{\mathbf{W}}}
\def\mX{{\mathbf{X}}}
\def\mY{{\mathbf{Y}}}
\def\mZ{{\mathbf{Z}}}
\def\mBeta{{\mathbf{\beta}}}
\def\mPhi{{\mathbf{\Phi}}}
\def\mLambda{{\mathbf{\Lambda}}}
\def\mSigma{{\mathbf{\Sigma}}}


% Expectation
% \def\eE{\mathop{\mathbb{E}}\limits}
\def\eE{\mathbb{E}}

% Probability
\def\pP{\mathbb{P}}

% Tilde
\def\tf{\tilde{f}}
\def\tS{\tilde{S}}
\def\wtF{\widetilde{\mathcal{F}}}
\def\whR{\widehat{R}}
\def\tvx{\tilde{\mathbf{x}}}
\def\ty{\tilde{y}}


\def\defeq{\overset{\textup{def}}{=}}
% \def\defeq{\overset{.}{=}}
\def\defone{\overset{\text{\ding{172}}}{=}}
\def\deftwo{\overset{\text{\ding{173}}}{=}}
\def\leqone{\overset{\text{\ding{172}}}{\leq}}
\def\leqtwo{\overset{\text{\ding{173}}}{\leq}}
\def\leqthree{\overset{\text{\ding{174}}}{\leq}}
\def\leqfour{\overset{\text{\ding{175}}}{\leq}}
\def\eqone{\overset{\text{\ding{172}}}{=}}
\def\eqtwo{\overset{\text{\ding{173}}}{=}}
\def\eqthree{\overset{\text{\ding{174}}}{=}}
\def\eqfour{\overset{\text{\ding{175}}}{=}}
\def\geqfive{\overset{\text{\ding{176}}}{\geq}}

\usepackage[colorlinks]{hyperref}
\usepackage{url}

\usepackage{graphicx}
\usepackage{booktabs}
\usepackage{xcolor}
\usepackage{stackengine}
\usepackage{arydshln}
\usepackage{multicol,multirow}

\usepackage{algorithm}
\usepackage[noend]{algpseudocode}
\usepackage{algorithmicx}

\usepackage{amsthm}

\newtheorem{definition}{Definition}[section]
\newtheorem{theorem}{Theorem}[section]
\newtheorem{corollary}{Corollary}[theorem]
\newtheorem{proposition}[theorem]{Proposition}
\newtheorem{lemma}[theorem]{Lemma}
\newtheorem{assumption}{Assumption}

\newcommand{\bfx}{\mathbf{x}}
\newcommand{\bfy}{\mathbf{y}}
\newcommand{\bfz}{\mathbf{z}}


\title{FreqPrior: Improving Video Diffusion Models with Frequency Filtering Gaussian Noise}

% Authors must not appear in the submitted version. They should be hidden
% as long as the \iclrfinalcopy macro remains commented out below.
% Non-anonymous submissions will be rejected without review.


\author{Yunlong Yuan$^{1}$, Yuanfan Guo$^{2}$, Chunwei Wang$^{2}$, Wei Zhang$^{2}$, Hang Xu$^{2}$, Li Zhang$^{1}$\thanks{Corresponding author~\texttt{lizhangfd@fudan.edu.cn}.} \\
$^{1}$School of Data Science, Fudan University\quad $^{2}$Noah's Ark Lab, Huawei \vspace{.5em}\\
\url{https://github.com/fudan-zvg/FreqPrior}
}

% The \author macro works with any number of authors. There are two commands
% used to separate the names and addresses of multiple authors: \And and \AND.
%
% Using \And between authors leaves it to \LaTeX{} to determine where to break
% the lines. Using \AND forces a linebreak at that point. So, if \LaTeX{}
% puts 3 of 4 authors names on the first line, and the last on the second
% line, try using \AND instead of \And before the third author name.
\newcommand{\fix}{\marginpar{FIX}}
\newcommand{\new}{\marginpar{NEW}}

\iclrfinalcopy % Uncomment for camera-ready version, but NOT for submission.
\begin{document}


\maketitle
\begin{abstract}
Text-driven video generation has advanced significantly due to developments in diffusion models.
Beyond the training and sampling phases, recent studies have investigated noise priors of diffusion models, as improved noise priors yield better generation results.
One recent approach employs the Fourier transform to manipulate noise, marking the initial exploration of frequency operations in this context. However, it often generates videos that lack motion dynamics and imaging details.
In this work, we provide a comprehensive theoretical analysis of the variance decay issue present in existing methods, contributing to the loss of details and motion dynamics.
Recognizing the critical impact of noise distribution on generation quality, we introduce FreqPrior, a novel noise initialization strategy that refines noise in the frequency domain. 
Our method features a novel filtering technique designed to address different frequency signals while maintaining the noise prior distribution that closely approximates a standard Gaussian distribution.
Additionally, we propose a partial sampling process by perturbing the latent at an intermediate timestep during finding the noise prior, significantly reducing inference time without compromising quality.
Extensive experiments on VBench demonstrate that our method achieves the highest scores in both quality and semantic assessments, resulting in the best overall total score. These results highlight the superiority of our proposed noise prior.
\end{abstract}

\section{Introduction}
\label{sec:intro}
% Image editing methods in diffusion models depend on user-defined control directions - users can unlock their creativity using these methods by specifying the desired manipulation through prompts~\cite{gandikota2023concept}, reference images~\cite{ruiz2022dreambooth, kumari2022customdiffusion, gal2022image, chen2024trainingfreeregionalpromptingdiffusion}, or attribute vectors~\cite{parmar2023zero,hertz2022prompt}. In this work, we ask a fundamentally different question: \emph{Can we automatically discover the underlying visual structure of a concept within diffusion model's knowledge?} %Rather than requiring user-specified controls, we aim to decompose the model's internal knowledge into meaningful directions.

% This question touches on a fundamental limitation in how we interact with diffusion models. Current control methods ~\cite{zhang2023addingconditionalcontroltexttoimage, gandikota2023concept, ye2023ipadaptertextcompatibleimage,ye2023ipadaptertextcompatibleimage, hertz2024stylealignedimagegeneration, li2023photomaker, shi2024instantbooth, chen2024trainingfreeregionalpromptingdiffusion} require users to specify their desired manipulations in advance, limiting interactive creativity. This contrasts with natural human artistic workflows, where creators dynamically explore creative ideas while jointly refining them toward meaningful artistic outcomes~\cite{hoffmann2016modeling}. This synergy between specification and exploration is not new to generative models. Early GAN architectures naturally developed disentangled latent spaces that enabled continuous\cite{harkonen2020ganspace,radford2015unsupervised, wu2021stylespace, shen2020interfacegan}, compositional control over generated images. Users could explore these spaces to discover interesting variations that would be difficult to describe in words~\cite{wu2021stylespace}, then combine them to achieve their creative goals~\cite{grabe2022towards}. 


% While diffusion models have largely superseded GANs in conditional image synthesis~\cite{dhariwal2021diffusion},  their underlying structure remains less understood. Diffusion models achieve remarkable diversity through high-dimensional latents, unlike GANs' compact latent spaces.  With a single prompt, diffusion models can generate radically different variations through different random initializations of input noise. We ask - Is it possible to discover interpretable structure within this vast space of variations?

Text-to-image diffusion models are capable of generating remarkable visual variations from a single prompt through different random initializations. However, this vast creative potential remains largely opaque to users---while we can generate diverse images, we lack understanding of the underlying structure of these variations. This presents a fundamental challenge: how can we discover and expose the latent visual capabilities encoded within these models?

\let\thefootnote\relax \footnote{$^{*}$Correspondence to \texttt{gandikota.ro@northeastern.edu}}

The challenge touches on a key limitation in how we interact with diffusion models today. Current control methods require users to explicitly specify their desired edits in advance through prompts~\cite{gandikota2023concept}, reference images~\cite{zhang2023addingconditionalcontroltexttoimage, chen2024trainingfreeregionalpromptingdiffusion, ruiz2022dreambooth,kumari2022customdiffusion, Ryu_lora, hu2021lora}, or attribute vectors~\cite{ye2023ipadaptertextcompatibleimage, hertz2024stylealignedimagegeneration, li2023photomaker, shi2024instantbooth,parmar2023zero,hertz2022prompt}. That contrasts sharply with natural human creative workflows, where artists dynamically explore creative ideas and jointly refine them toward meaningful artistic outcomes~\cite{hoffmann2016modeling}. The need for pre-specified controls creates a barrier between users and the full creative potential of these models.

Interestingly, earlier generative models like GANs~\cite{gans,karras2019style,brock2018large} naturally developed more interpretable internal structures. Their compact latent spaces often exhibited emergent disentanglement~\cite{harkonen2020ganspace,radford2015unsupervised, wu2021stylespace, shen2020interfacegan}, enabling continuous and compositional control over generated images. Users could explore these spaces to discover interesting variations that would be difficult to describe in words~\cite{wu2021stylespace}, then combine them to achieve their creative goals~\cite{grabe2022towards}.

Diffusion models have largely superseded GANs in conditional image synthesis~\cite{dhariwal2021diffusion}, achieving greater diversity through much higher-dimensional latents. And yet an understanding of the underlying structure of these larger latent spaces has remained elusive. In this work, we ask a fundamental question: \emph{Can we automatically discover the visual structure within a diffusion model's knowledge of a concept?} Rather than requiring user-specified controls, we aim to decompose the model's internal representations into expressive directions that users can explore and combine.

To address these needs, we present \textbf{SliderSpace}, a framework that brings systematic explorability to diffusion models. Given just a text prompt, SliderSpace discovers a canonical set of meaningful, diverse, and controllable directions within the model's knowledge of that concept. Each direction is implemented as a low-rank adapter~\cite{hu2021lora} that can be scaled and composed with others, allowing users to explore and smoothly combine different aspects of variation, as shown in Figure~\ref{fig:intro}.

We ground SliderSpace discovery in three key requirements for meaningful decomposition of a diffusion model's visual manifold: 
\begin{enumerate}
    \item \textbf{Unsupervised Discovery:} The decomposition process should emerge from the intrinsic structure of the model's learned representation, rather than being guided by predefined attributes. This ensures we capture the true topology of the model's knowledge space rather than projecting our assumptions onto it.
    
    \item \textbf{Semantic Orthogonality:} Each discovered control must represent a distinct semantic direction. This is enforced in a semantic feature space, like CLIP, where every slider has an orthogonal effect in embeddings. This prevents discovering multiple controls that create similar semantic effects, making the system more efficient and easier.
    
    \item \textbf{Distribution Consistency:} Directions must induce consistent transformations across both random seeds and prompt variations. 
\end{enumerate}

These requirements naturally lead to our proposed framework, which we formalize in Section~\ref{sec:method}. As we show in our experiments, SliderSpace is architecture-agnostic, working with both conventional U-Net based models like Stable Diffusion~\cite{rombach2022high, rombach2022sd20, podell2023sdxl, turbo, dmd} and recent transformer-based architectures like Flux~\cite{flux}.

We demonstrate the expressiveness of SliderSpace through three applications: First, we show how SliderSpace can decompose high-level concepts into diverse and expressive components, revealing the natural axes of variation in the model's understanding. Second, we explore artistic style variation, where SliderSpace discovers directions that match or exceed the diversity of manually curated artist lists while being judged more useful by human evaluators. Finally, we show how SliderSpace can help reverse the mode collapse commonly observed in distilled diffusion models, restoring diversity while maintaining generation speed.

Beyond providing practical creative control, SliderSpace opens new avenues for understanding and utilizing the latent capabilities of diffusion models. By mapping these models' visual potential into intuitive, composable directions, we take a step toward making their creative possibilities more accessible and interpretable to users.

% Image editing methods in diffusion models unlock the creativity of users. In this work we ask an alternate question: \emph{Can we organize and expose what of the diffusion model is already capable of?}.
% Existing methods for controlling image generation typically require users to manually specify edit directions for desired changes. This process is time-consuming, requires technical expertise, and limits the spontaneity of the creative process. For instance, if a user wants to adjust the smile of a generated person, they must explicitly request this edit, often through imprecise prompt engineering or model fine-tuning. This approach of predefined controls or manual specifications restricts users from fully exploring the latent capabilities of the model. There may be interesting stylistic variations or attributes that the model can generate, but users have no easy way to discover or utilize these.

% Natural visual disentanglement was an emergent property in the latent space of Generative Adversarial Models (GANs) \cite{harkonen2020ganspace,radford2015unsupervised, wu2021stylespace, shen2020interfacegan}. In particular, it has been observed that StyleGAN~\cite{karras2019style} stylespace neurons offer detailed control over many meaningful aspects of images that would be difficult to describe in words~\cite{wu2021stylespace}. However, diffusion models do not share such a compact latent space~\cite{park2023unsupervised}; and efforts to uncover such a space in the semantic embeddings of the text conditioning have met with limited success \nik{Nick - is there a specific citation you were thinking about?}.

% In this work we introduce \textbf{SliderSpace}, which takes a step towards uncovering an analogous low dimensional representation of diffusion models' visual breadth; in essence treating the diffusion model as many generators sharing parameters, where a particular generator is defined by a specific prompt. For a given prompt we sample many random seeds (and optionally prompt expansions using an LLM), generate the corresponding images, and apply an off the shelf feature extractor (in this work CLIP, but our method can be applied to any differentiable feature extractor). We use PCA to analyze these features, and for each of the leading $k$ principal components we train a LoRA \cite{} which causes the diffusion model to produces images which increase the feature magnitude along that component when passed back through the same feature extractor. This leads to a 'Slider' for each principal component, because each LoRA can be scaled and applied to the original diffusion model, continuously varying those visual features in the generated results (as measured, in our case, by CLIP).

% There are many other works that enhance the controllability of diffusion models. One common approach is enabling users to add spatial constraints to a generation either manually, or via a reference image \cite{zhang2023addingconditionalcontroltexttoimage, chen2024trainingfreeregionalpromptingdiffusion}, a second is leveraging more abstract embeddings (e.g. identity, style) extracted from a reference image \cite{ye2023ipadaptertextcompatibleimage, hertz2024stylealignedimagegeneration, li2023photomaker, shi2024instantbooth}, a third is finetuning a foundation model to better generate a concept important to the user \cite{ruiz2022dreambooth, kumari2022customdiffusion, Ryu_lora, hu2021lora}, and a fourth (most relevant to this work) is finding low-rank adaptors of the model based on a prompt or small training set which can be scaled to provide continous control over one aspect of generated image (e.g. night vs day, basic vs luxury, etc.) \cite{gandikota2023concept}. SliderSpace is complementary to all of these methods and offers something distinct. All of the other methods we are aware require the user (and / or model designer) to know in advance what type of control they want. In contrast SliderSpace assists users in discovering and controlling hidden capabilities present in the diffusion model's distribution of possible generations.

%We propose that truly intuitive creative control in a text-to-image model should meet three key criteria: \emph{discoverability}, \emph{intuitiveness}, and \emph{specificity}. The model should reveal controllable attributes that may not be immediately obvious, offer controls that are easy to understand and manipulate, and ensure each control affects a distinct attribute of the generated image.

% We demonstrate the utility and power of SliderSpace using three applications built on top of SDXL-DMD \cite{dmd}, because its fast generation speed lends itself well to the continuous control offered by SliderSpace.

% First, we study concept decomposition (Section \ref{sec:concept_exp}), where we learn sliders for a specific concept (e.g. 'monster', 'waterfall', 'car'). Through quantitative metrics of diversity and text alignment we demonstrate that the learned sliders dramatically boost the diversity of generations when randomly applied without harming text alignment; we also ask humans to qualitatively judge these results in a user study where they find the SliderSpace results to be more 'Diverse', 'Useful', and 'Creative' than our baselines.

% Second, we attempt to compare the automatic discoveries of SliderSpace to a large scale manual study of artistic styles (Section \ref{sec:art_exp}), open-sourced by ParrotZone \cite{parrotzone}. In this study SDXL was prompted with over 4300 artist names,  and based on visual inspection the cases of successful stylistic mimicry recorded. Quantitatively SliderSpace more closely matches the distribution of artistic variation discovered by ParrotZone than other baselines, and in our user studies was judged to be significantly more 'Diverse' and 'Useful' than the baselines. To our surprise humans even judged SliderSpace results to be slightly more 'Diverse' than the results generated by the manually discovered artist names of \cite{parrotzone}.

% Third, we attempt to use SliderSpace to reverse the mode collapse commonly observed in distilled few-step diffusion models relative to the original teacher model (Section \ref{sec:diverse_exp}). We quantitatively demonstrate that applying SliderSpace to SDXL-DMD leads to more closely matching the distribution of images by the original teacher, SDXL.

%Through extensive experiments on various state-of-the-art text-to-image models, we demonstrate that SliderSpace significantly enhances user control and creative expression in AI-assisted image generation tasks. Our method enables a range of applications, including concept decomposition and control, diversity improvement in generated images, customization dissection and edits, and the exploration of artistic styles inherent in the model.

% SliderSpace goes beyond providing a practical tool for enhanced creative control. By mapping the visual potential of diffusion models it can open new avenues for generative creativity and deepens our understanding of each model's hidden potential.
\section{Related Work}

\paragraph{LLMs for Agent tasks.}

Our research is related to deploying large language models (LLMs) as agents for decision-making tasks in interactive environments~\citep{liu2023agentbench,zhou2023webarena,shridhar2020alfred,toyama2021androidenv}. Earlier works, such as~\citep{yao2023webshopscalablerealworldweb}, fine-tuned models like BERT~\citep{devlin2019bertpretrainingdeepbidirectional} for decision-making in simplified environments, such as online shopping or mobile phone manipulation. With the advent of large language models~\citep{brown2020languagemodelsfewshotlearners,openai2024gpt4technicalreport}, it became feasible to perform decision-making tasks through zero-shot or few-shot in-context learning. To better assess the capabilities of LLMs as agents, several models have been developed~\citep{deng2024mind2web,xiong2024watch,hong2023cogagent,yan2023gpt}. Most approaches~\citep{zheng2024seeact,deng2024mind2web} provide the agent with observation and action history, and the language model predicts the next action via in-context learning. Additionally, some methods~\citep{zhang2023building,li2023camel,song2024trial} attempt to distill trajectories from state-of-the-art language models to train more effective policy models. In contrast, our paper introduces a novel framework that automatically learns a reward model from LLM agent navigation, using it to guide the agents in making more effective plans.

\textbf{LLM Planning.} Our paper is also related to planning with large language models. Early researchers~\citep{brown2020languagemodelsfewshotlearners} often prompted large language models to directly perform agent tasks. Later, \citet{yao2022react} proposed ReAct, which combined LLMs for action prediction with chain-of-thought prompting~\citep{wei2022chain}. Several other works~\citep{yao2023treethoughtsdeliberateproblem,hao2023reasoning,zhao2023large,qiao2024agentplanningworldknowledge} have focused on enhancing multi-step reasoning capabilities by integrating LLMs with tree search methods. Our model differs from these previous studies in several significant ways. First, rather than solely focusing on text generation tasks, our pipeline addresses multi-step action planning tasks in interactive environments, where we must consider not only historical input but also multimodal feedback from the environment. Additionally, our pipeline involves automatic learning of the reward model from the environment without relying on human-annotated data, whereas previous works rely on prompting-based frameworks that require large commercial LLMs like GPT-4~\citep{openai2024gpt4technicalreport} to learn action prediction. Furthermore, \Model supports a variety of planning algorithms beyond tree search.

\textbf{Learning from AI Feedback.} In contrast to prior work on LLM planning, our approach also draws on recent advances in learning from AI feedback~\citep{bai2022constitutional,lee2023rlaif,yuan2024self,sharma2024critical,pan2024autonomous,koh2024tree}. These studies initially prompt state-of-the-art large language models to generate text responses that adhere to predefined principles and then potentially fine-tune the LLMs with reinforcement learning. Like previous studies, we also prompt large language models to generate synthetic data. However, unlike them, we focus not on fine-tuning a better generative model but on developing a classification model that evaluates how well action trajectories fulfill the intended instructions. This approach is simpler, requires no reliance on state-of-the-art LLMs, and is more efficient. We also demonstrate that our learned reward model can integrate with various LLMs and planning algorithms, consistently improving their performance.

\textbf{Inference-Time Scaling.} ~\citet{snell2024scaling} validates the efficacy of inference-time scaling for language models. Based on inference-time scaling, various methods have been proposed, such as random sampling~\citep{wang2022self} and tree-search methods~\citep{hao2023reasoning, zhang2024accessing, guan2025rstar}. Concurrently, several works have also leveraged inference-time scaling to improve the performance of agentic tasks. ~\citet{koh2024tree} adopts a training-free approach, employing MCTS to enhance policy model performance during inference and prompting the LLM to return the reward. ~\citet{gu2024your} introduces a novel speculative reasoning approach to bypass irreversible actions by leveraging LLMs or VLMs. It also employs tree search to improve performance and prompts an LLM to output rewards. ~\citet{yu2024exact} proposes Reflective-MCTS to perform tree search and fine-tune the GPT model, leading to improvements in ~\citet{koh2024visualwebarena}. ~\citet{putta2024agent} also utilizes MCTS to enhance performance on web-based tasks such as ~\citet{yao2023webshopscalablerealworldweb} and real-world booking environments. ~\cite{lin2025qlass} utilizes the stepwise reward to give effective intermediate guidance across different agentic tasks. Our work differs from previous efforts in two key aspects: (1) Broader Application Domain. Unlike prior studies that primarily focus on tasks from a single domain, our method demonstrates strong generalizability across web agents, mathematical reasoning, and scientific discovery domains, further proving its effectiveness. (2) Flexible and Effective Reward Modeling. Instead of simply prompting an LLM as a reward model, we finetune a small scale VLM~\citep{lin2023vila} to evaluate input trajectories. %Our reward scores range continuously between 0 and 1, in contrast to existing methods that rely on discrete scoring (e.g., 0 and 1, or 0, 0.5, and 1) through direct LLM prompting.

% Concurrently, several works have also leveraged inference-time scaling to improve the performance of agentic tasks. ~\citet{pan2024autonomous} demonstrates that LLMs and VLMs, such as the GPT series, can function as evaluators or reward models to provide guidance for fine-tuning or reflection, thereby enhancing digital agents. This lays the groundwork for subsequent studies that directly prompt LLMs as reward models. ~\citet{koh2024tree} adopts a training-free approach, employing MCTS to enhance policy model performance during inference. However, it is limited to web environments~\citep{koh2024visualwebarena}. Moreover, its value function relies on prompting an LLM, which is less effective than our proposed method. We validate our approach through ablation studies, demonstrating that our fine-tuned reward model is more effective. ~\citet{gu2024your} introduces a novel speculative reasoning approach to bypass irreversible actions, such as purchasing a product, by leveraging LLMs or VLMs. It also employs tree search to improve performance, but it remains restricted to the web domain~\citep{koh2024visualwebarena, deng2024mind2web}. Additionally, it lacks reward modeling and instead prompts an LLM to output rewards. ~\citet{yu2024exact} proposes Reflective-MCTS to perform tree search and fine-tune the GPT model, leading to improvements in ~\citep{koh2024visualwebarena}. However, this work focuses solely on a single web agent task, and its reward modeling is derived from multi-agent debate, differing from our more effective and efficient reward modeling approach. ~\citet{putta2024agent} also utilizes MCTS to enhance performance, but it is limited to web-based tasks such as ~\citep{yao2023webshopscalablerealworldweb} and real-world booking environments.


\section{Methodology}
\paragraph{Preliminaries.}
We primarily focus on the homologous model merging, in which $\boldsymbol{\theta}_i$ all come from the same base model $\boldsymbol{\theta}_{\rm{base}}$. Given $K$ tasks $\{T_1,T_2,\cdots,T_K\}$ and $K$ corresponding fine-tuned models with parameters $\{\boldsymbol{\theta}_1,\boldsymbol{\theta}_2,\cdots,\boldsymbol{\theta}_K\}$, model merging aims to combine $K$ fine-tuned models into one single model simultaneously performing on $\{T_1,T_2,\cdots,T_K\}$ without post-training~\cite{method_p1_1,method_p1_2}.
Task vector~\cite{ilharco2023editing,yang2024adamerging} is a key element in merging method which could enhances the base model‘s ability or enable the model to handle other tasks. Specifically, for task $T_i$, the task vector $\boldsymbol\tau_i\in \mathbb{R}^D$ is defined as the vector obtained by subtracting the SFT weights $\boldsymbol{\theta}_i$ from the base model weight
$\boldsymbol{\theta}_{\rm{base}}$, \emph{i.e.}, $\boldsymbol\tau_i=\boldsymbol{\theta}_i-\boldsymbol{\theta}_{\rm{base}}$. The merged model could be denoted as $\boldsymbol{\theta}_m=\boldsymbol{\theta}_{\rm{base}}+\sum_i \lambda_i\boldsymbol{\tau}_i$, which $\lambda_i$ is the scaling factor measuring the importance of task vector. For clarification, we also denote the neuron set in $\boldsymbol{\theta}_i$ as $\mathcal{N}_i$, the neuron set in $\boldsymbol{\tau}_i$ as $\mathcal{T}_i$.



\begin{algorithm}[!ht]
    \caption{LED-Merging}
    \label{alg1}
    \begin{algorithmic}[1]
        \REQUIRE  base model $\boldsymbol{\theta}_{\rm{base}}$, SFT models $\{\boldsymbol{\theta}_{i}\mid i\in [K]\}$, mask ratios \{$r_{i} \mid i\in [K]\}$, scaling factors $\{\lambda_i\mid i\in[K]\}$, location datasets $\{\mathcal{X}_{i}\mid i\in[K]\}$
        \ENSURE merged parameter $\boldsymbol{\theta}_{m}$
        \STATE $\mathcal{M}\leftarrow\phi$
        \STATE $\boldsymbol{\theta}_{m}\leftarrow \boldsymbol{\theta}_{\rm{base}}$
        \FOR{$i\in [K]$}
        \STATE $I(\boldsymbol{\theta}_i)=\mathbb{E}_{x\sim \mathcal{X}_i}|\boldsymbol{\theta}_{i}\odot \nabla_{\boldsymbol{\theta}_i}\mathcal{L}(x)|$
        \STATE $I(\boldsymbol{\theta}_{\rm{base}})=\mathbb{E}_{x\sim \mathcal{X}_i}|\boldsymbol{\theta}_{\rm{base}}\odot \nabla_{\boldsymbol{\theta}_{\rm{base}}}\mathcal{L}(x)|$
        
        \STATE calculate $\mathcal{T}^{r_i}_{i}$ following Equation \ref{vote}
        \STATE  $\mathcal{M}\leftarrow \mathcal{M}\cup\{\mathcal{T}^{r_i}_i\}$
       
        
   
        
        
        \ENDFOR  
        \FOR{$i\in [K]$}
        
        \STATE calculate $\text{Disjoint}(\mathcal{T}_i^{r_i})$ use Equation~\ref{disjoint_safety}
        \STATE $\boldsymbol{m}_i \leftarrow \boldsymbol{0}$
        \FOR{$d\in \mathcal{T}_i^{r_i}$}
        \STATE $\boldsymbol{m}_{i,d}=1$
        \ENDFOR
        \STATE $\boldsymbol{\theta}_{m}\leftarrow \boldsymbol{\theta}_{m}+\lambda_i \boldsymbol{\tau}_i\odot \boldsymbol{m}_{i}$
        \ENDFOR
    \end{algorithmic}
\end{algorithm}
    %\vspace{-5pt}
\begin{figure*}[h!]
    \centering
    \includegraphics[width=\linewidth]{figs/pipeline_v2.pdf}
    \vspace{-40mm}
    \caption{Overview of our two-stage training pipeline {\ours}.}
    \label{fig:pipeline}
\end{figure*}


\paragraph{LED-Merging: Location, Election, and Disjoint Merging}
To address the neuron misidentification and interference issues in existing model merging methods, we propose LED-Merging (Location, Election, and Disjoint Merging). Specifically, previous studies \cite{modelstock, ilharco2023editing, tiesmerging} fail to accurately identify safety-related neurons in task vectors with a single magnitude score, namely \textit{neuron misidentification}. Meanwhile, there exists an interference between safety-related and utility-related task vector neurons during the merging process, namely \textit{neuron interference}. To address neuron misidentification, we first locate important neurons both in the base and fine-tuned models and then elect neurons from the task vector considering these two scores together. Subsequently, to mitigate the interference, we introduce a disjoint step, isolating these important neurons so that they influence different base neurons. The whole process is illustrated in Figure~\ref{fig:method}. 




In the location and election step, we consider the importance score from base and fine-tuned models simultaneously to locate task-specific neurons. In this way, it is more accurate than relying on the magnitude score alone because task-specific neurons with high importance score in the fine-tuned model may not necessarily score high in the base model, and vice versa.

{\textbf{Location}}.  We first calculate importance scores for each neuron in a base/fine-tuned model. Given a location dataset $\mathcal{X}_i=\{(x,y)_k\}$, where $x$ is the question and $y$ is the answer, we calculate the importance scores for the weight $\boldsymbol{\theta}_i\in\mathbb{R}^D$ in any  layer as follows~\cite{snip,spareseGPT,sun2024a}:
\begin{equation}
    I(\boldsymbol{\theta}_i)=\mathbb{E}_{x\sim \mathcal{X}_i}[\boldsymbol{\theta}_i\odot \nabla _{\boldsymbol{\theta}_i}\mathcal{L}(x)],
    \label{location}
\end{equation}
which $\mathcal{L}(x)=-\log p(y\mid x)$ is the conditional negative log-likelihood loss. We choose the SNIP score~\cite{snip} because it balances computational efficiency and performance~\cite{cq}. Please refer to Sec.~\ref{sec:ablation} for the comparison between different location methods. After computing importance scores, we choose top-$r_i$ neurons as the important neuron subset $\mathcal{N}_{i}^{r_i}$ from $I(\boldsymbol{\theta}_i)$.
 
 % After computing locating scores, we select the neurons scoring both high in base and fine-tuned models as important neurons in task vectors. Then in the disjoint step,  with preventing  polysemantic neurons  from receiving gradient updates towards different directions,
 % we use set difference to isolate the safety   and utility-related neurons  and construct corresponding masks for merging process,

{\textbf{Election}}. A natural question is how to select important neurons in the task vector $\boldsymbol{\tau}_i$ based on $I(\boldsymbol{\theta}_{\rm{base}})$ and $I(\boldsymbol{\theta}_{i})$. The important neurons in the base model may be different from neurons in the fine-tuned model. Therefore, we introduce the following election strategy to select neurons with high scores in both base and fine-tuned models:
\begin{equation}
    \mathcal{T}_i^{r_i}=\mathcal{N}_i^{r_i}\cap \mathcal{N}_{\rm{base}}^{r_i}.
    \label{vote}
\end{equation}
\emph{Remark}. We compare different choosing methods, including scoring low or high in base or fine-tuned model in Section~\ref{sec:ablation} and find that Equation \ref{vote} achieves the best performance.





{\textbf{Disjoint}}. As important neurons from different task vectors may conflict with each other at the same position, we use the set difference to disjoint the neurons from others to prevent interference:
\begin{equation}
    \text{Disjoint}(\mathcal{T}^{r_i}_{i})=\mathcal{T}^{r_i}_{i}-\mathop{\cup}\limits_{{J}\subsetneqq [K],|J|\geq 2}\mathop{\cap}\limits_{j\in {J}}\mathcal{T}^{r_j}_{j}.
    \label{disjoint_safety}
\end{equation}

Next, we construct a mask $\boldsymbol{m}_i\in\mathbb{R}^D$ to implement disjoint in the merging process. Specifically, this mask $\boldsymbol{m}_i$ is used to select neurons from $\mathcal{T}_i$. The mask ratio is $r_i$, where $r\in(0,1]$. The mask $\boldsymbol{m}_i$ can be derived from:
\begin{equation}
    \boldsymbol{m}_{i,d}=\begin{aligned} &\left\{ \begin{array}{ll} 1, & \text{if } d\in \text{Disjoint}(\mathcal{T}_{i}^{r_i}), \\ 0, & \text{otherwise}. \end{array} \right. \end{aligned}
    \label{mask_safety}
\end{equation}


% \subsection{Merging Models with Masks}
{\textbf{Merging}}. The final
merged task vector $\boldsymbol{\tau}_m$ is as follows:
\begin{equation}
    \boldsymbol{\tau}_m= \sum_i \lambda_i\boldsymbol{\tau}_{i}\odot\boldsymbol{m}_i.
    \label{merged_task_vector}
\end{equation}
We summarize the workflow in Algorithm \ref{alg1}.



\section{Experiments}
\label{sec:experiments}

\begin{figure*}[t]
\vspace{-6mm}
    \centering
    \includegraphics[width=0.8\linewidth]{figs/compare.pdf}
    \vspace{-4mm}
    \caption{\textbf{Qualitative comparison} with the baseline for generating a sequence of novel view images.  
    The results demonstrate that our method synthesizes more consistent multi-view images compared to our baseline model (Zero123). In addition, compared to SyncDreamer, our method visually maintains better similarity to the conditioned image and appears more natural.}
    \label{fig:sota_compare}
\vspace{-5mm}
\end{figure*}

\subsection{Experimental Setups}
\textbf{Dataset.}
Following previous work~\cite{zero123, SyncDreamer}, we evaluate our work on the Google Scanned Object (GSO)~\cite{GSO} dataset to verify the zero-shot novel view image synthesis capability. 
We also provide results for additional datasets in the Supplementary Material.
Specifically, we randomly select 30 objects from the GSO dataset with various object categories. 
Unlike recent approaches~\cite{mvdream, SyncDreamer} that aim to enhance the consistency of novel view synthesis models by generating multiple fixed-view images, our method can generate images from any camera pose and any number of views. Therefore, we conduct experiments under different camera pose settings to validate our approach:
specifically, 
1) \textit{16-views with free camera pose}: for each object, we circularly render 16 views with the elevation angles ranging in $[-10\degree, 40\degree]$ and the azimuth angles are evenly distributed in $[0\degree, 360\degree]$. 
2) \textit{16-views with fixed camera pose}: We maintain a constant elevation angle of $30\degree$ and uniformly sample azimuth angles (same as SyncDreamer~\cite{SyncDreamer}).
3) \textit{32-views with free camera pose}: Similar to the first setting, but we sample 32 views.
It's important to note that our method does not require additional training or fine-tuning on any datasets.

\noindent\textbf{Metrics.}
To validate the effectiveness of our method, we mainly evaluate it based on three criteria:
1) \textit{Quality Score}. We evaluate the image quality of synthesized multi-view images by measuring their similarity with ground truth images. Following prior research~\cite{zero123, sparsefusion}, we report the similarity between the synthesized images and the ground truth images with standard metrics: PSNR, SSIM~\cite{ssim}, and LPIPS~\cite{lpips}.
2) \textit{Multi-view Consistency Score}. As the primary goal of our work is to improve the consistency of generated images, we also employ the 3D consistency score~\cite{3dim} to verify the consistency among the synthesized images. Specifically, we train an Instant-NGP~\cite{instant_ngp} with the input image and part of the synthesized novel view images of our model and evaluate the similarity between the remaining synthesized images and the rendered images of Instant-NGP. For the synthesized multi-view images of each object, we allocate $3/4$ for training and reserve the remaining $1/4$ for validation.
Intuitively, if the consistency of synthesized images is improved, the NeRF-like model will train a better object representation, and the re-rendered images will agree more with the validation images.
3) \textit{Input Consistency Score}. To assess the faithfulness of synthesized images in preserving the identity of the input condition image, we introduce the input consistency score. This score calculates the similarity of each synthesized image with the input condition image, utilizing the LPIPS metric.

In addition, we use synthesized multi-view images to train a neural 3D reconstruction model (NeuS~\cite{neus}) and report commonly used Chamfer Distances (CD) and Volume IoUs between the trained 3D model and the ground truth.

\noindent\textbf{Baselines.}
Given that our main goal is to improve the consistency of the trained baseline model without further fine-tuning, we mainly compare our approach with the used baseline model Zero123~\cite{zero123}. Additionally, we compare our method to the SOTA approaches such as PGD~\cite{tseng2023consistent} and SyncDreamer~\cite{SyncDreamer} using the same Zero123 base model.

\noindent\textbf{Implementation Details.}
We use the official checkpoint provided by Zero123~\cite{zero123}, which is trained on objaverse~\cite{objaverse} for 165,000 steps. We inject our epipolar attention layer after step $T=4$ and layer $L=10$ by default. We find that feature fusion weight $\alpha=0.5$, and the number of context views $M=2$ work better.

\begin{table}[t]
\centering
\caption{Comparison of multi-view consistency, image quality, and input consistency of synthesized multi-view images at the 16-view setting with free camera pose.}
\label{tab:view16_free_compare}
\vspace{-2mm}
\scalebox{0.6}{
\begin{tabular}{c ccc ccc c}
\toprule
              & \multicolumn{3}{c}{Multi-view Consistency} & \multicolumn{3}{c}{Quality Score} & \multicolumn{1}{c}{Input Consis.} \\
              \cmidrule(lr){2-4} \cmidrule(lr){5-7} \cmidrule(lr){8-8}
              & PSNR$\uparrow$  & SSIM$\uparrow$ & LPIPS$\downarrow$ 
              & PSNR$\uparrow$  & SSIM$\uparrow$ & LPIPS$\downarrow$ 
              & LPIPS$\downarrow$ 
              \\ \midrule

Zero123
& 15.225        & 0.645       & 0.408
& 14.255        & 0.747       &	0.208
& 0.303         
\\
SyncDreamer
& 14.830        & 0.626       & 0.434
& 12.650        & 0.713       &	0.254
& 0.317         
\\
Ours 
& \best{18.300}	& \best{0.734}	& \best{0.355}
& \best{14.947}	& \best{0.763}	& \best{0.191}
& \best{0.282}
\\

\bottomrule
\end{tabular}
}
\end{table}

\begin{table}[t]
\vspace{-1mm}
\centering
\caption{Comparison of multi-view consistency, image quality, and input consistency at the 16-view setting with fixed camera pose as SyncDreamer~\cite{SyncDreamer}.}
\label{tab:view16_fxied_compare}
\vspace{-3mm}
\scalebox{0.6}{
\begin{tabular}{c ccc ccc c}
\toprule
              & \multicolumn{3}{c}{Multi-view Consistency} & \multicolumn{3}{c}{Quality Score} & \multicolumn{1}{c}{Input Consis.} \\
              \cmidrule(lr){2-4} \cmidrule(lr){5-7} \cmidrule(lr){8-8}
              & PSNR$\uparrow$  & SSIM$\uparrow$ & LPIPS$\downarrow$ 
              & PSNR$\uparrow$  & SSIM$\uparrow$ & LPIPS$\downarrow$ 
              & LPIPS$\downarrow$ 
              \\ \midrule

Zero123
& 16.556        & 0.682       & 0.378
& 14.592        & 0.750       &	0.207
& 0.305         
\\
SyncDreamer
& \best{22.424}        & \best{0.812}       & \best{0.268}
& 15.269        & 0.749       &	0.196
& 0.300         
\\
Ours 
& 21.151	& 0.780	& 0.302
& \best{15.293}	& \best{0.764}	& \best{0.184}
& \best{0.287}
\\

\bottomrule
\end{tabular}
}
\vspace{-4mm}
\end{table}


\subsection{Comparison With Baseline Models}
The quantitative comparison on three settings are shown in Tab.~\ref{tab:view16_free_compare}, Tab.~\ref{tab:view16_fxied_compare}, and Tab.~\ref{tab:view32_free_compare}. The qualitative comparison is shown in Fig.~\ref{fig:sota_compare}.

\begin{table}[t]
\centering
\caption{Comparison of multi-view consistency and image quality scores of synthesized multi-view images at the 32-view setting with free camera pose.}
\vspace{-3mm}
\label{tab:view32_free_compare}
\scalebox{0.7}{
\begin{tabular}{c ccc ccc}
\toprule
              & \multicolumn{3}{c}{Multi-view Consistency} & \multicolumn{3}{c}{Quality Score} \\
              \cmidrule(lr){2-4} \cmidrule(lr){5-7}
              & PSNR$\uparrow$  & SSIM$\uparrow$ & LPIPS$\downarrow$ 
              & PSNR$\uparrow$  & SSIM$\uparrow$ & LPIPS$\downarrow$ 
              \\ \midrule

Zero123
& 16.515        & 0.694       & 0.378
& 15.142        & 0.733       &	0.211
\\
PGD~\cite{tseng2023consistent}
& 18.481        & 0.720       & 0.343
& 15.281        & 0.739       &	0.205
\\
Ours 
& \best{20.655}	& \best{0.792}	& \best{0.305}
& \best{15.268}	& \best{0.742}	& \best{0.203}
\\

\bottomrule
\end{tabular}
}
\vspace{-3mm}
\end{table}

\begin{table*}
  [t]
  \centering
  \resizebox{\textwidth}{!}{%
  \begin{tabular}{cccccccccccc}
    \toprule \multicolumn{2}{c}{Components}                                                             & \multicolumn{5}{c}{Re-executability Rate (\%)} & \multicolumn{5}{c}{Readability (\#)} \\
    \cmidrule(lr){1-2} \cmidrule(lr){3-7} \cmidrule(lr){8-12}        \hspace{8pt}\labelemoji\hspace{8pt}                                                                & \hspace{8pt}\toolemoji\hspace{8pt}                                      & O0                                 & O1             & O2             & O3             & AVG            & O0             & O1             & O2             & O3             & AVG            \\
    \hline
    \rowcolor[rgb]{0.93,0.93,0.93}\multicolumn{12}{c}{\textbf{Initialize with LLM4Decompile-End-6.7B~\citep{llm4decompile}}}   \\
    \xmark                                                                                              & \xmark                                    & 69.51                              & 46.95          & 50.61          & 46.34          & 53.35          & 3.98 & 3.41 & 3.44 & 3.38 & 3.55 \\
    \cmark                                                                                              & \xmark                                    & 75.61                              & 50.61          & 50.00          & 50.00          & 56.55          & 4.01 & 3.44 & 3.39 & \textbf{3.49} & 3.58 \\
    \xmark                                                                                              & \cmark                                    & 83.54                     & \textbf{56.10}          & 51.22          & 50.61 & 60.37 & 4.05 & 3.51 & 3.51 & 3.42 & 3.62 \\
    \cmark                                                                                              & \cmark                                    & \textbf{85.37}                            & \textbf{56.10}                     & \textbf{51.83} & \textbf{52.43}          & \textbf{61.43} & \textbf{4.13} & \textbf{3.60} & \textbf{3.54} & \textbf{3.49} & \textbf{3.69} \\

    \rowcolor[rgb]{0.93,0.93,0.93}\multicolumn{12}{c}{\textbf{Initialize with Deepseek-Coder-6.7B-base~\citep{deepseekcoder}}} \\
    \xmark                                                                                              & \xmark                                    & 59.15                              & 35.98          & 39.02          & 37.80          & 42.99          & 3.71 & 3.05 & 3.16 & 3.05 & 3.24 \\
    \cmark                                                                                              & \xmark                                    & 66.46                              & 41.46          & 38.41          & 36.59          & 45.73          & 3.76 & 3.17 & \textbf{3.21} & 3.08 & 3.31 \\
    \xmark                                                                                              & \cmark                                    & 70.73                              & 39.63          & 39.02          & 40.24          & 47.41          & 3.90 & 3.17 & 3.08 & 3.11 & 3.31 \\
    \cmark                                                                                              & \cmark                                    & \textbf{79.88}                     & \textbf{45.73} & \textbf{43.90} & \textbf{42.68} & \textbf{53.05} & \textbf{3.96} & \textbf{3.21} & 3.18 & \textbf{3.19} & \textbf{3.38} \\
    \bottomrule
  \end{tabular}%
  }
  \caption{The ablation study of different methods across four optimization levels
  (O0, O1, O2, O3), as well as their average scores (AVG). The results in bold represent the optimal performance. The ~\labelemoji~ and ~\toolemoji~ means Relabedling and Function Call. \textbf{Bold} denotes the best performance.}
  \label{tab:ablation}
\end{table*}



\begin{figure*}[ht]
    \centering
    \begin{minipage}{0.65\textwidth}
        \centering
        \includegraphics[width=0.95\linewidth]{figs/ablation.pdf}
        \vspace{-2mm}
        \captionof{figure}{Qualitative Comparison for different design choices. Our method, employing multi-view epipolar attention, demonstrates the best consistency.}
        \label{fig:ablation}
    \end{minipage}\hfill
    \begin{minipage}{0.33\textwidth}
        \centering
        \includegraphics[width=0.8\linewidth]{figs/neus_ver.pdf}
        \vspace{-3mm}
        \caption{Our method shows better direct 3D reconstruction~\cite{neus}.}
        \label{fig:neus}
    \end{minipage}
    \vspace{-5mm}
\end{figure*}

\noindent\textbf{Multi-view Consistency.}
Tab.~\ref{tab:view16_fxied_compare} presents the 3D consistency scores compared to our baseline model (Zero123) and SyncDreamer. The results indicate a significant improvement across all three metrics achieved by our method when compared with Zero123.
While our method exhibits a marginally lower numerical consistency score compared to SyncDreamer, it enables the synthesis of images with arbitrary camera poses.	
This capability is illustrated in Tab.~\ref{tab:view16_free_compare}, where our method consistently enhances consistency with changes in camera pose settings, whereas SyncDreamer fails to do so and exhibits inferior results compared to Zero123.
Furthermore, our method facilitates the synthesis of multi-view images with any number of camera views. This versatility is demonstrated in Tab.~\ref{tab:view32_free_compare}, where our method continues to achieve significant improvements in consistency scores, while SyncDreamer is unable to operate under such conditions.	

Meanwhile, Fig.~\ref{fig:sota_compare} provides a qualitative comparison with the baseline. While both our method and SyncDreamer enhance consistency, our method visually preserves better similarity to the input image, including color and texture details. The input consistency score further corroborates this.

\noindent\textbf{Image Quality.}
While our primary goal centers around enhancing the consistency of synthesized multi-view images, we also evaluate the image quality by comparing the similarity with the ground truth images. The results shown in Tab.~\ref{tab:view16_free_compare}, Tab.~\ref{tab:view16_fxied_compare}, and Tab.~\ref{tab:view32_free_compare} indicate that our method also enhances the image quality under different settings besides improving the consistency.
Moreover, our method shows better image quality compared with SyncDreamer even in the 16-view setting with fixed camera pose.

\noindent\textbf{Input Consistency.}
Input consistency terms whether the results align with the input image.
Fig.~\ref{fig:sota_compare} illustrates that both our method and SyncDreamer enhance multi-view consistency. However, the color and texture details of SyncDreamer's results diverge from the input image and appear visually unnatural.
This discrepancy is evident in the input consistency score presented in Tab.~\ref{tab:view16_fxied_compare}, indicating lower similarity with the condition image in the SyncDreamer results.	

\subsection{Ablation Study}
The overall quantitative results are shown in Tab.~\ref{tab:ablation}, and the qualitative comparisons are shown in Fig.~\ref{fig:ablation}.

\noindent \textbf{Full Attention \vs Epipolar Attention.}
The results presented in Tab.\ref{tab:ablation} and Fig.\ref{fig:ablation} demonstrate that our epipolar attention mechanism can synthesize more consistent multi-view images compared with full attention. Furthermore, our epipolar attention achieves a greater performance improvement compared to full attention when using multiple reference images. This could be attributed to the fact that our epipolar attention more effectively localizes target information, as depicted in Fig.~\ref{fig:full_attn_compare}, thereby reducing noise from the reference images. In the multi-view setting, where multiple reference images are utilized, this noise reduction becomes particularly crucial.
Moreover, it is noteworthy that the epipolar attention mechanism consumes less GPU memory compared to our baseline, as discussed in Sec.~\ref{sec:attn_analysis}.

\noindent \textbf{Attending Single-View \vs Multi-View.}
Applying the epipolar attention significantly improves the consistency between the input and target views. However, the consistency between different views in the unobserved regions of the input view is not well preserved.
After implementing our epipolar attention in the multi-view setting, the consistency across the generated multi-view images is further improved. The last row in Tab.~\ref{tab:ablation} shows that after applying our multi-view epipolar attention, the consistency score is further improved compared with the single-view setting. Besides, the qualitative result in Fig.~\ref{fig:ablation} also shows better consistency among different target views.



\begin{table}[t]
\centering
\vspace{-1mm}
\caption{Comparison of 3D reconstruction results. Our method significantly improves the reconstruction quality.}
\vspace{-3mm}
\label{tab:neus}
\scalebox{0.7}{
\begin{tabular}{c cc}
\toprule
              &  Chamfer Dist.$\downarrow$  & Volume IoU$\uparrow$
\\ \midrule

            Zero123         & 0.017         & 0.819    \\
            SyncDreamer     & \best{0.013}         & \best{0.847}    \\
            Ours            & 0.014	& 0.842 \\

\bottomrule
\end{tabular}
}
\vspace{-5mm}
\end{table}


\vspace{-2mm}
\subsection{Downstream Application}
\vspace{-2mm}
To demonstrate the effectiveness of our method, we also applied it to the downstream 3D reconstruction task. Specifically, we trained the NeuS model~\cite{neus} directly using images synthesized by our method, Zero123, and SyncDreamer, respectively.
The quantitative results in Tab.~\ref{tab:neus} show that the consistent multi-view images synthesized by our method can significantly improve the 3D reconstruction quality.
Additionally, our method exhibits similar performance to SyncDreamer which requires time-consuming re-training.
The qualitative results in Fig.~\ref{fig:neus} show that it is challenging to train the NeuS model directly due to the lack of consistency in the images generated by Zero123. In contrast, our method generates more consistent multi-view images and, therefore, better reconstructs the geometry and texture details.
We show improvements on other downstream applications such as image-to-3D in the Supplementary Material.


\section{Conclusion}

%In this paper, w
We propose a new PEFT method called DiffoRA, which enables efficient and adaptive LLM fine-tuning based on LoRA. 
Instead of adjusting every interior rank, 
%of the decomposition matrices 
%of all modules, 
we argue that adopting LoRA module-wisely is sufficient. 
To achieve this, we construct a DAM to select the modules that are most suitable and essential to fine-tune. We theoretically analyze how the DAM impacts the convergence rate and generalization capability.
%of the pre-trained model. 
Furthermore, we adopt continuous relaxation and discretization to establish DAM.
%for each task. 
To alleviate the issue of discretization discrepancy, we utilize the weight-sharing strategy for optimization. 
%We fully implement our method and t
The experimental results demonstrate that our DiffoRA works consistently better than the baselines across all benchmarks. 

% \newpage
\bibliography{iclr2025_conference}
\bibliographystyle{iclr2025_conference}

% \newpage
\appendix
\section{Preliminary}
\subsection{Diffusion models}
\label{appendix:preliminary}
\textbf{Diffusion models}~\citep{ho2020denoising} are a class of generative models that recover the data corrupted by the Gaussian noise through learning a reverse diffusion process.
It iteratively denoises from Gaussian noise, which corresponds to learning the reverse process of a fixed Markov Chain of length $T$.
The diffusion process is a Markov chain that gradually corrupts the data with Gaussian noise.
For the diffusion process given the variance schedule $\beta_t$:
\begin{equation}
q(x_{1:T} | x_0) = \prod_{t=1}^T q(x_t | x_{t-1} ), \qquad q(x_t|x_{t-1}) = \mathcal{N}(x_t;\sqrt{1-\beta_t}x_{t-1},\beta_t I).
\end{equation}
Using the  Markov property, we can sample $x_t$ at an arbitrary time $t$ from $x_0$ in closed form. Let $\alpha=1-\beta_t$ and $\bar{\alpha}_t=\prod_{s=1}^t\alpha_s$, we have
\begin{equation}
    q(x_t|x_0) = \mathcal{N}(x_t; \sqrt{\bar\alpha_t}x_0, (1-\bar\alpha_t)I).
\end{equation}
By the Bayes' rules, $q(x_{t-1}|x_t,x_0)$ can be expressed as follows:
\begin{align}
q(x_{t-1}|x_t,x_0) &=  \mathcal{N}(x_{t-1}; \tilde\mu_t(x_t, x_0), \tilde\beta_t I), \\
\text{where}\quad \tilde\mu_t(x_t, x_0) &= \frac{\sqrt{\bar\alpha_{t-1}}\beta_t }{1-\bar\alpha_t}x_0 + \frac{\sqrt{\alpha_t}(1- \bar\alpha_{t-1})}{1-\bar\alpha_t} x_t \quad \text{and} \quad
\tilde\beta_t = \frac{1-\bar\alpha_{t-1}}{1-\bar\alpha_t}\beta_t.
\end{align}
For the reverse process, it generates $x_0$ from $x_T$ with prior $x_T=\mathcal{N}(x_T;0,I)$ and transitions:
\begin{equation}
    p_\Theta(x_{t-1}|x_t)=\mathcal{N}(x_{t-1};\mu_\Theta(x_t, t),\Sigma_\Theta(x_t,t)).
\end{equation}
In the equation, $\Theta$ are learnable parameters of models $\epsilon_\Theta$ 
which are trained to minimize the variant of the variational bound $\E_{x,\epsilon\sim\mathcal{N}(0,I),t}\left[ \left\| \epsilon-\epsilon_{\Theta}\left(x_t, t\right) \right\|^2 \right]$.

\subsection{Fourier transform}
\label{appendix:fourier}
\textbf{Discrete Fourier Transform (DFT)} is one of the most important discrete transforms used in digital signal processing including image processing.
The discrete Fourier transform can be expressed as the \textbf{DFT} matrix, denoted as $\mF$, defined as follows:
\begin{equation}
    \mF = \left( \omega_N^{\left(m-1\right)\cdot\left(n-1\right)} \right)_{N \times N}=
\begin{bmatrix}
 \omega_N^{0 \cdot 0}     & \omega_N^{0 \cdot 1}     & \cdots & \omega_N^{0 \cdot (N-1)}     \\
 \omega_N^{1 \cdot 0}     & \omega_N^{1 \cdot 1}     & \cdots & \omega_N^{1 \cdot (N-1)}     \\
 \vdots                   & \vdots                   & \ddots & \vdots                       \\
 \omega_N^{(N-1) \cdot 0} & \omega_N^{(N-1) \cdot 1} & \cdots & \omega_N^{(N-1) \cdot (N-1)} \\
\end{bmatrix}
\end{equation}
where $\omega_N = e^{-{2\pi i/N}}$ is a primitive $N$-th root of unity. 
The inverse transform, denoted as $\mF^{-1}$ can be derived from $\mF$ as its complex conjugate transpose, scaled by $\frac{1}{N}$: $\mF^{-1} = \frac{1}{N}\mF^\ast$.

The \textbf{DFT} matrix $\mF$ can be decomposed into its real and imaginary parts, represented respectively by matrices $\mA$ and $\mB$:
\begin{equation}
\label{eq:decomposition}
    \mF = \mA + \mB i,\quad \mA = \Re\left(\mF\right),\quad \mB = \Im\left(\mF\right).
\end{equation}
This decomposition simplifies the understanding of the structure and properties of the DFT matrix, providing deeper insights. Using Euler’s formula, $\mA$ and $\mB$ can be explicitly expressed as:
\begin{equation}
    \mA = \left(\cos\left( \left(m-1\right)\left(n-1\right)\theta \right)\right)_{N\times N},\quad 
    \mB = \left(\sin\left( \left(m-1\right)\left(n-1\right)\theta \right)\right)_{N\times N},
\end{equation}
where $\theta=-\frac{2\pi}{N}$. 
Notably, both $\mA$ and $\mB$ are real symmetric matrices.

\begin{lemma}
\label{lemma:1}
    For $\theta=-\frac{2\pi}{N}$ where $N$ is a positive integer, it holds that $\sum_{k=1}^{N}\sin\left(l\left(k-1\right)\theta\right)=0$ for any integer $l$.
\end{lemma}
\begin{proof}
By applying Euler's formula, we rewrite $\sin\left(l\left(k-1\right)\theta\right)$ as $\Im\left(\omega_N^{l(k-1)}\right)$, where $\omega_N=e^{-2\pi i/N}$. Then we have:
\begin{equation}
    \sum_{k=1}^{N} \sin\left(l\left(k-1\right)\theta\right) =\sum_{k=0}^{N-1} \Im \left(\omega_N^{lk}\right) = \Im \left( \sum_{k=0}^{N-1} \omega_N^{lk}\right).
\end{equation}
The term $\sum_{k=0}^{N-1} \omega_N^{lk}$ is the sum of geometric sequence. 
If $\omega_N^{l} = 1$, then $\sum_{k=0}^{N-1} \omega_N^{lk} = N$, yielding $\Im \left( \sum_{k=0}^{N-1} \omega_N^{lk}\right)=0$.

Otherwise, if $\omega_N^{l} \ne 1$, we have $\sum_{k=0}^{N-1} \omega_N^{lk} = \left(1-\omega_N^{lN}\right)/\left(1-\omega_N^{l}\right)$. Since $\omega_N^{N}=1$, then $\sum_{k=0}^{N-1} \omega_N^{lk} = 0$, and consequently $\Im \left( \sum_{k=0}^{N-1} \omega_N^{lk}\right)=0$.

In conclusion, we have shown that $\sum_{k=1}^{N} \sin\left(l\left(k-1\right)\theta\right)=0$ for any integer $l$.
\end{proof}
This lemma offers foundational insights into the behavior of the sum of sinusoidal functions,
Now, we introduce a theorem regarding properties of the \textbf{DFT} matrix.
\begin{theorem}
\label{theorem:1}
    Given a \textbf{DFT} matrix $\mF\in \mathbb{C}^{N\times N}$, with $\mA$ and $\mB$ representing its real and imaginary parts respectively, it holds that $\mA\mB=\mB\mA=\mathbf{0}$ and $\mA^2+\mB^2=N\mI$.
\end{theorem}
\begin{proof}
Using the property of the inverse Fourier transform, we have
\begin{equation}
    \mI = \mF\mF^{-1} = \frac{1}{N} \mF \mF^{\ast} = \frac{1}{N}\left(\mA + \mB i\right)\left(\mA - \mB i\right) = \frac{1}{N}\left(\mA^2+\mB^2-\mA\mB i+\mB\mA i\right).
\end{equation}
Comparing real parts and imaginary parts of both sides, we derive:
\begin{equation}
    \mA^2+\mB^2=N\mI,\quad \mB\mA=\mA\mB.
\end{equation}
Considering the matrix $\mA\mB$, we calculate the value of the element in the $m$-th row and $n$-th column:
\begin{equation}
\begin{split}
    \left(\mA\mB\right)_{mn} &=\sum_{k=1}^{N}\cos\left( \left(m-1\right)\left(k-1\right)\theta \right)\sin\left( \left(k-1\right)\left(n-1\right)\theta \right) \\
    &=\frac{1}{2}\sum_{k=1}^{N}\left( \sin\left((m+n-2)(k-1)\theta\right) - \sin\left((m-n)(k-1)\theta\right) \right) = 0.
\end{split}
\end{equation}
The last equation holds using Lemma~\ref{lemma:1}. The equation holds for each element of $\mA\mB$.
Therefore $\mA\mB=\mB\mA=\mathbf{0}$.
\end{proof}
%%%%%%%%%%%%%   3 dimension
For the 3D Fourier transform, it can be represented as follows using the Kronecker product:
\begin{equation}
    \mF_{3D} = \mF_{T} \otimes \mF_{H} \otimes \mF_{W}.
\end{equation}
The inverse transform is given by:
\begin{equation}
    \mF_{3D}^{-1}  = \frac{1}{N_TN_HN_W} \mF_{3D}^\ast.
\end{equation}
Similarly, we decompose $\mF_{3D}$ into its real part $\mA_{3D}$ and imaginary part $\mB_{3D}$:
\begin{equation}
\begin{split}
    \mF_{3D} & =\left(\mA_T+\mB_Ti\right) \otimes \left(\mA_H+\mB_Hi\right) \otimes \left(\mA_W+\mB_Wi\right), \\
    \mA_{3D} &= \mA_T\otimes\mA_H\otimes\mA_W
    -\mA_T\otimes\mB_H\otimes\mB_W
    -\mB_T\otimes\mA_H\otimes\mB_W
    -\mB_T\otimes\mB_H\otimes\mA_W, \\
    \mB_{3D} &= \mA_T\otimes\mA_H\otimes\mB_W
    +\mA_T\otimes\mB_H\otimes\mA_W
    +\mB_T\otimes\mA_H\otimes\mA_W
    -\mB_T\otimes\mB_H\otimes\mB_W.
\end{split}
\end{equation}
By Theorem~\ref{theorem:1} and the property or Kronecker product, it still holds that:
\begin{equation}
    \mA_{3D}^2+\mB_{3D}^2=N_T N_H N_W\mI,\quad \mB_{3D}\mA_{3D}=\mA_{3D}\mB_{3D}=\mathbf{0}.
\end{equation}
It reveals that the 3D DFT matrix shares the same properties as the ordinary DFT matrix.
For convenience, we denote the DFT matrix, including multi-dimensional cases as $\mF$, with size denoted as $N$.
Employing mathematical induction, we can extend Theorem~\ref{theorem:1} from one-dimensional case to arbitrary finite dimensions:
\begin{theorem}
\label{theorem:dft}
    Given a \textbf{DFT} matrix or multi-dimension \textbf{DFT} matrix $\mF\in \mathbb{C}^{N\times N}$, with $\mA$ and $\mB$ are its real part and imaginary part respectively, it holds that $\mA\mB=\mB\mA=\mathbf{0}$ and $\mA^2+\mB^2=N\mI$.
\end{theorem}
\section{Noise distribution analysis}
\label{appendix:noise_distribution_analysis}
\subsection{FreeInit}
\label{appendix:freeinit}
FreeInit~\citep{wu2023freeinit} uses conventional frequency filtering methods to manipulate noise, which is the key step in the framework. This step can be formulated as follows:
\begin{equation}
\label{eq:freeinit_raw}
    \bfz_T = \Re\left( \mathcal{F}_{3D}^{-1}\left(\mathcal{F}_{3D}\left(\bfz_{noise}\right)\odot\mathcal{M} + \mathcal{F}_{3D}\left(\eta\right)\odot(\bm{1}-\mathcal{M})\right) \right),
\end{equation}
where $\mathcal{F}_{3D}$ is the Fourier transform applied to both spatial and temporal dimensions. $\mathcal{M}$ is a spatial-temporal low-pass filter. $\bfz_{noise}$ is noisy latent derived from corrupting the clean latent with initial Gaussian noise to timestep $T$, while $\eta$ is another Gaussian noise. 
For analysis, we focus solely on the spatial and temporal dimensions, ignoring the batchsize and channel dimensions. Additionally, we flatten the latent $\bfz_T\in\R^{f\times h \times w}$ into a vector $\bfz \in \R^{fhw}$. The equation~\ref{eq:freeinit_raw} can be expressed in matrix form as follows:
\begin{equation}
    \bfz = \Re\left( \mF^{-1}\mLambda_x\mF\bfx + \mF^{-1}\mLambda_y\mF\bfy \right),
\end{equation}
where $\mF$ is the DFT matrix of transform $\mathcal{F}_{3D}$, $\bfx$ and $\bfy$ are random vectors corresponding to $\bfz_{noise}$ and $\eta$, and $\mLambda_x$ and
$\mLambda_y$ are diagonal matrices associated with low-pass filter $\mathcal{M}$ and high-pass filter $\bm{1}-\mathcal{M}$. Therefore it holds that $\mLambda_x+\mLambda_y=\mI$.
% Due to the property of $\mathcal{M}$, the diag 
This equation can be simplified to the following form using equation~\ref{eq:decomposition}:
\begin{equation}
    \bfz = \frac{1}{N}\left(\mA\mLambda_x\mA + \mB\mLambda_x\mB\right)\bfx 
      + \frac{1}{N}\left( \mA\mLambda_y\mA + \mB\mLambda_y\mB\right)\bfy,
\end{equation}
Under the Assumption~\ref{assumption:1}, $\bfx=\text{vec}(\bfz_{noise}) \sim\mathcal{N}\left(\mathbf{0},\mI\right)$, where $\bfx$ and $\bfy$ are independent. 
Since $\bfz$ is a linear combination of independent Gaussian random vectors, it follows that $\bfz$ is also Gaussian. To derive the distribution of $\bfz$, we only need to compute its expectation and covariance.
The expectation is straightforward and given by $\E[\bfz]=\mathbf{0}$. 
The covariance of $\bfz$ can be calculated as follows:
\begin{equation}
\label{eq:freeinit_cov1}
    \Cov\left(\bfz\right) 
    = \frac{1}{N^2}\left( \mA\mLambda_x\mA + \mB\mLambda_x\mB\right)^2 +
      \frac{1}{N^2}\left( \mA\mLambda_y\mA + \mB\mLambda_y\mB\right)^2.
\end{equation}
To simplify the expression, we denote $\mP = \frac{1}{N}\left( \mA\mLambda_x\mA+\mB\mLambda_x\mB\right)$. Then the term $\mA\mLambda_y\mA+\mB\mLambda_y\mB$ can be expressed using $\mP$:
\begin{equation}
\label{eq:freeinit_cov2}
\begin{split}
     \mA\mLambda_y\mA + 
      \mB\mLambda_y\mB & = \mA(\mI-\mLambda_x)\mA + 
      \mB(\mI-\mLambda_x)\mB \\
      & = \mA^2 + \mB^2 - \left( \mA\mLambda_x\mA+\mB\mLambda_x\mB\right) = N\mI - N\mP.
\end{split}
\end{equation}
The last equation follows from $ \mA^2+\mB^2 =N\mI$, as stated in Theorem~\ref{theorem:dft}. 
Combining Equation~(\ref{eq:freeinit_cov1}) and Equation~(\ref{eq:freeinit_cov2}), the covariance of $\bfz$ is given by:
\begin{equation}
    \Cov\left(\bfz\right) = \mP^2 + (\mI-\mP)^2.
\end{equation}
Consequently, we obtain the distribution of $\bfz$ as follows:
\begin{equation}
\label{eq:free_distribution}
    \bfz\sim\mathcal{N}\left(\mathbf{0}, \mP^2 + \left(\mI-\mP\right)^2\right).
\end{equation}
Due to the property of the low-pass filter $\mathcal{M}$ where each element lies between 0 to 1, both $\mLambda_x$ and $\mLambda_y$ are semi-definite diagonal matrices. Consequently, we can prove that both $\mP$ and $\mI-\mP$ are semi-positive definite matrices. The covariance structure resembles $a^2+(1-a)^2$, which is less than 1 for $a\in(0, 1)$. This indicates a difference between the distribution of $\bfz$ and the standard Gaussian distribution. We explore this further in Appendix~\ref{appendix:theoretical_analysis}.


\subsection{FreqPrior}
\label{appendix:freqprior}
The noise refinement stage of our method consists of three distinct steps, which are elaborated on in Section~\ref{subsec:noise_refinement}.  
To facilitate further analysis, we express these steps in matrix form.
% For simplicity, we can express these steps in the following matrix forms, which facilitate further analysis.
The first step, \textbf{noise preparation step}, can be represented as:
\begin{equation}
\label{eq:freq_x1}
    \bfx_1 = \frac{1}{\sqrt{1+\cos^2\theta}}\left(\cos\theta\cdot \bfx+\sin\theta\cdot\mathbf{\eta_1}\right),\quad
    \bfx_2 = \frac{1}{\sqrt{1+\cos^2\theta}}\left(\cos\theta\cdot \bfx+\sin\theta\cdot\mathbf{\eta}_2\right),
\end{equation}
where $\mathbf\eta_1, \mathbf\eta_2\sim\mathcal{N}\left(\mathbf{0}, \mI\right)$ and are independent. Under Assumption~\ref{assumption:1}, $\bfx\sim\mathcal{N}(\mathbf{0},\mI)$. Obviously, $\bfx, \mathbf\eta_1$, and $\mathbf\eta_2$ are independent.
Both $\bfx_1$ and $\bfx_2$ are linear combinations of independent of Gaussian random vectors. 
Their expectation can be computed directly: $\E[\bfx_1]=\mathbb{E}[\bfx_2]=\mathbf{0}$.
Next, we calculate the covariance of these variables. Specifically, the covariances are given by:
\begin{equation}
\label{eq:freq_x2}
    \Cov\left(\bfx_1\right) = \Cov\left(\bfx_2\right) = \frac{1}{1+\cos^2\theta}\mI,\quad
    \Cov(\bfx_1,\bfx_2) = \Cov(\bfx_2,\bfx_1) = \frac{\cos^2\theta}{1+\cos^2\theta}\mI.
\end{equation}
This implies that $\bfx_1$ and $\bfx_2$ are correlated, as they both share a component of $\bfx$ when $\cos\theta\ne0$.

The \textbf{noise processing} and \textbf{post-processing} steps can be expressed as follows:
\begin{equation}
    \bfz_1 = \mF^{-1}\mLambda_x\mF\bfx_1 + \mF^{-1}\mLambda_y\mF\bfy_1, \quad
    \bfz_2 = \mF^{-1}\mLambda_x\mF\bfx_2 + \mF^{-1}\mLambda_y\mF\bfy_2,
\end{equation}
\begin{equation}
\bfz = \frac{1}{\sqrt{2}}\left(\Re\left(\bfz_1\right)+\Im\left(\bfz_1\right)+\Re\left(\bfz_2\right)-\Im\left(\bfz_2\right)\right),
\end{equation}
where $\bfy_1, \bfy_2\sim\mathcal{N}\left(\mathbf{0}, \mI\right)$ are independent. 
Regarding the filters, $\mLambda_x$ and $\mLambda_y$ are diagonal matrices corresponding to the low-pass filter $\mathcal{M}$ and the high-pass filter $(\bm{1}-\mathcal{M})^{0.5}$.

The refined noise $\bfz$ can be expressed in a following form using equation~\ref{eq:decomposition}:
\begin{equation}
\label{eq:freq_z_1}
\begin{split}
\sqrt{2}N\cdot\bfz {} = {}& \phantom{\;\;\,\,\,}\left( \mA\mLambda_x\mA +\mB\mLambda_x\mB+\mA\mLambda_x\mB-\mB\mLambda_x\mA\right)\bfx_1 \\
                          & + \left( \mA\mLambda_y\mA +\mB\mLambda_y\mB+\mA\mLambda_y\mB-\mB\mLambda_y\mA\right)\bfy_1 \\
                          & + \left( \mA\mLambda_x\mA +\mB\mLambda_x\mB-\mA\mLambda_x\mB+\mB\mLambda_x\mA\right)\bfx_2 \\
                          & + \left( \mA\mLambda_y\mA +\mB\mLambda_y\mB-\mA\mLambda_y\mB+\mB\mLambda_y\mA\right)\bfy_2.
\end{split}
\end{equation}
From the mathematical form of this expression, it is evident that the matrices preceding these random vectors share similar structures.
To simplify this equation, we introduce the following notations:
\begin{equation}
\begin{split}
    \text{Let}:\quad
    \mC_x &= \mA\mLambda_x\mA +\mB\mLambda_x\mB,\quad \mD_x =\mA\mLambda_x\mB-\mB\mLambda_x\mA, \\
    \mC_y &= \mA\mLambda_y\mA +\mB\mLambda_y\mB,\quad \mD_y =\mA\mLambda_y\mB-\mB\mLambda_y\mA.
\end{split}
\end{equation}
Since $\mA$ and $\mB$ are real symmetric matrices, and $\mLambda_x$ and $\mLambda_y$ are diagonal matrices, it is straightforward to prove that $\mC_x$ and $\mC_y$ are symmetric matrices, while $\mD_x$ and $\mD_y$ are skew-symmetric matrices. 
Using these notations, Equation~(\ref{eq:freq_z_1}) can be simplified as follow:
\begin{equation}
    \sqrt{2}N\cdot\bfz = \left(\mC_x+\mD_x\right)\bfx_1+
    \left(\mC_y+\mD_y\right)\bfy_1+
    \left(\mC_x-\mD_x\right)\bfx_2+
    \left(\mC_y-\mD_y\right)\bfy_2.
\end{equation}
In the analysis of $\sqrt{2}N\cdot\bfz$ where $\bfz$ is a Gaussian-distributed vector, we need to calculate the expectation and covariance to determine its distribution.
The expectation is given by $\E[\bfz] = \mathbf{0}$. 
The covariance can be expressed as the sum of several covariance terms related to  $\bfx_1$, $\bfx_2$, $\bfy_1$ and $\bfy_2$. Specifically, the covariance of $\sqrt{2}N\cdot\bfz$ can be expressed as follows:
\begin{equation}
\begin{split}
    \Cov(\sqrt{2}N\cdot\bfz) = {}& {} \phantom{\,\;\;\,\,\,} \Cov\left(\left(\mC_x+\mD_x\right)\bfx_1\right)
    +\Cov\left(\left(\mC_x-\mD_x\right)\bfx_2\right)\\
    &+\Cov\left(\left(\mC_y+\mD_y\right)\bfy_1\right)
    +\Cov\left(\left(\mC_y-\mD_y\right)\bfy_2\right)\\
    &+\Cov\left(\left(\mC_x+\mD_x\right)\bfx_1,\left(\mC_x-\mD_x\right)\bfx_2\right)\\
    &+\Cov\left(\left(\mC_x-\mD_x\right)\bfx_2,\left(\mC_x+\mD_x\right)\bfx_1\right).
\end{split}
\end{equation}
The covariance of $\sqrt{2}N\cdot\bfz$ consists of 6 terms, with first four terms representing the covariance of each random vector. The last two terms are cross terms that arise due to the fact that $\bfx_1$ and $\bfx_2$ are not independent. 
By solving these terms, We can derive the covariance of $\bfz$.

First, we focus on the covariance terms related to $\bfy_1$ and $\bfy_2$:
\begin{equation}
\label{eq:freq_cov_part1}
\begin{split}
    &\Cov\left(\left(\mC_y+\mD_y\right)\bfy_1\right)+\Cov\left(\left(\mC_y-\mD_y\right)\bfy_2\right)\\
    ={}&\left(\mC_y+\mD_y\right)\Cov\left(\bfy_1\right)\left(\mC_y+\mD_y\right)^\top + 
    \left(\mC_y-\mD_y\right)\Cov\left(\bfy_1\right)\left(\mC_y-\mD_y\right)^\top \\
    = {} &  \left(\mC_y+\mD_y\right)\left(\mC_y-\mD_y\right) + \left(\mC_y-\mD_y\right)\left(\mC_y+\mD_y\right)
    =2\left(\mC_y^2-\mD_y^2\right).
\end{split}
\end{equation}
Similarly, we can infer $\Cov\left(\left(\mC_x+\mD_x\right)\bfx_1\right)
    +\Cov\left(\left(\mC_x-\mD_x\right)\bfx_2\right)$ combined with Equation~(\ref{eq:freq_x2}):
% Covariance x part
\begin{equation}
\label{eq:freq_cov_part2}
    \Cov\left(\left(\mC_x+\mD_x\right)\bfx_1\right)
    +\Cov\left(\left(\mC_x-\mD_x\right)\bfx_2\right)
    = \frac{2}{1+\cos^2\theta}\left(\mC_x^2-\mD_x^2\right).
\end{equation}
Having computed the first four terms, we now turn our attention to the last two cross terms. With Equation~(\ref{eq:freq_x2}), we have:
\begin{equation}
\label{eq:freq_cov_part3}
\begin{split}
    &\Cov\left(\left(\mC_x+\mD_x\right)\bfx_1,\left(\mC_x-\mD_x\right)\bfx_2\right)
    +\Cov\left(\left(\mC_x-\mD_x\right)\bfx_2,\left(\mC_x+\mD_x\right)\bfx_1\right) \\
    =&{} \left(\mC_x+\mD_x\right)\Cov(\bfx_1,\bfx_2)\left(\mC_x-\mD_x\right)^\top 
        +\left(\mC_x-\mD_x\right)\Cov(\bfx_2,\bfx_1)\left(\mC_x+\mD_x\right)^\top \\
    =&{}\frac{\cos^2\theta}{1+\cos^2\theta}\left(\mC_x+\mD_x\right)^2+\frac{\cos^2\theta}{1+\cos^2\theta}\left(\mC_x-\mD_x\right)^2 =\frac{2\cos^2\theta}{1+\cos^2\theta}\left(\mC_x^2+\mD_x^2\right).
\end{split}
\end{equation}
Substituting the expression of the covariance related to $\bfx_1$, $\bfx_2$, $\bfy_1$ and $\bfy_2$ with Equations~(\ref{eq:freq_cov_part1},~\ref{eq:freq_cov_part2},~\ref{eq:freq_cov_part3}), we can express the covariance of $\sqrt{2}N\cdot\bfz$ in the following form:
\begin{equation}
\label{eq:freq_50}
    \Cov\left(\sqrt{2}N\cdot\bfz\right) = 2\left(\mC_x^2-\mD_x^2+\mC_y^2-\mD_y^2\right)+ \frac{4\cos^2\theta}{1+\cos^2\theta}\mD_x^2.
\end{equation}
To further simplify this equation, we need to explore the properties of $\mC_x$, $\mC_y$, $\mD_x$ and $\mD_y$. 
From Theorem~\ref{theorem:dft}, which establish $\mA\mB=\mB\mA=\mathbf{0}$ and $\mA^2+\mB^2=N\mI$. We can compute the squares of matrices $\mC_x$ and $\mD_x$ as follows:
\begin{equation}
\label{eq:freq_51}
\begin{split}
    \mC_x^2 & = \mA\mLambda_x\mA^2\mLambda_x\mA + \mB\mLambda_x\mB^2\mLambda_x\mB, \\
    -\mD_x^2 & = \mA\mLambda_x\mB^2\mLambda_x\mA + \mB\mLambda_x\mA^2\mLambda_x\mB.
\end{split}
\end{equation}
Notice that the squares of $\mC_x$ and $\mD_x$ share a similar form, differing only in the middle matrix: one is $\mA^2$ and the other is $\mB^2$. This observation inspires us to calculate $\mC_x^2-\mD_x^2$, especially since we have established $\mA^2+\mB^2=N\mI$. Therefore, we can express it as follows:
\begin{equation}
\begin{split}
    \mC_x^2-\mD_x^2&= \mA\mLambda_x\left(\mA^2+\mB^2\right)\mLambda_x\mA + \mB\mLambda_x\left(\mB^2+\mA^2\right)\mLambda_x\mB\\ 
    & = N \mA\mLambda_x^2\mA + N\mB\mLambda_x^2\mB, \\
\end{split}
\end{equation}
Since $\mC_y$ and $\mD_y$ follow the same pattern with only the subscript replaced, it also holds that:
\begin{equation}
    \mC_y^2-\mD_y^2 = N \mA\mLambda_y^2\mA + N\mB\mLambda_y^2\mB. 
\end{equation}
Make use of $\mLambda_y=\left(\mI-\mLambda_x^2\right)^{\frac{1}{2}}$, we can conclude:
\begin{equation}
\label{eq:freq_54}
    \mC_x^2-\mD_x^2 + \mC_y^2-\mD_y^2 = N \mA\left(\mLambda_x^2+\mLambda_y^2\right)\mA + N\mB\left(\mLambda_x^2+\mLambda_y^2\right)\mB =N\mA^2+N\mB^2=N^2\mI.
\end{equation}
Substituting with Equations~(\ref{eq:freq_51}) and (\ref{eq:freq_54}), we can simplifies Equation~(\ref{eq:freq_50}) to express the covariance of $\sqrt{2}N\cdot\bfz$ as follows:
\begin{equation}
\label{eq:freq_55}
    \Cov\left(\sqrt{2}N\cdot\bfz\right) = 
    2N^2\mI
    -\frac{4\cos^2\theta}{1+\cos^2\theta}
    \left( \mA\mLambda_x\mB^2\mLambda_x\mA + \mB\mLambda_x\mA^2\mLambda_x\mB
    \right),
\end{equation}
Inspired by the form of $\mA\mLambda_x\mB^2\mLambda_x\mA$ and $\mB\mLambda_x\mA^2\mLambda_x\mB$ which are the matrix multiplication of $\mA\mLambda_x\mB$ and $\mB\mLambda_x\mA$. 
We creatively construct a new matrix $\mQ = \frac{1}{N}\left(\mA\mLambda_x\mB+\mB\mLambda_x\mA\right)$. It is easy to prove $\mQ$ is a symmetric matrix. The square of $\mQ$ is as follows:
\begin{equation}
\label{eq:Q_square}
    \mQ^2=\frac{1}{N^2}\left( \mA\mLambda_x\mB^2\mLambda_x\mA + \mB\mLambda_x\mA^2\mLambda_x\mB
    \right).
\end{equation}
By combining Equation~(\ref{eq:freq_55}) and Equation~(\ref{eq:Q_square}) and eliminating the constant $\sqrt{2}N$ from both sides of the equation, we can calculate the covariance of $\bfz$:
\begin{equation}
    \Cov\left(\bfz\right)=\mI - \frac{2\cos^2\theta}{1+\cos^2\theta}\mQ^2.
\end{equation}
Finally, we derive the distribution of $\bfz$ as follows:
\begin{equation}
\label{eq:freq_distribution}
    \bfz\sim\mathcal{N}\left(\mathbf{0}, \mI - \frac{2\cos^2\theta}{1+\cos^2\theta}\mQ^2\right).
\end{equation}
It is clear that the covariance of our refined noise is ``smaller'' than $\mI$. However, as $\mA\mB=\mB\mA=\mathbf{0}$ and the diagonal elements of $\Lambda_x$ ranges from 0 to 1, it gives the intuition that $\mQ$ is close to $\mathbf{0}$. 
We make further analysis in Appendix~\ref{appendix:covariance}.

\section{Covariance error analysis}
\label{appendix:theoretical_analysis}
\begin{theorem}
\label{theorem:4}
    Given two semi-positive definite matrices $\mC$ and $\mD$ satisfying $\mC\succeq\mD\succeq\mathbf{0}$, then $||\mC||_F \ge ||\mD||_F$ where $||\cdot||_F$ is Frobenius Norm.
\end{theorem}
\begin{proof}
Since $||\mC-\mD||_F^2\ge 0$, then expanding it yields:
\begin{equation}
    ||\mC||_F^2 + ||\mD||_F^2\ge \mathrm{tr}\left(\mC^\top\mD+\mD^\top\mC\right) = 2\mathrm{tr}\left(\mC\mD\right).
\end{equation}
The last equation holds because the $\mC$ and $\mD$ are symmetric matrices and $\mathrm{tr}(\cdot)$ is invariant under circular shifts.
Then we can conclude:
\begin{equation}
    ||\mC||_F^2 - ||\mD||_F^2  
    \ge  2\mathrm{tr}\left(\mC\mD\right) - 2||\mD||_F^2   
    = 2\mathrm{tr}\left(\left(\mC-\mD\right)\mD\right).  
\end{equation}
Using Cholesky decomposition, for semi-definite matrix $\mathbf D$, there exists matrix $\mL$ such that $\mD=\mL\mL^\top$. Then we can derive:
\begin{equation}
    \mathrm{tr}\left(\left(\mC-\mD\right)\mD\right)
    = \mathrm{tr}\left(\left(\mC-\mD\right)\mL\mL^\top\right) 
    = \mathrm{tr}\left(\mL^\top\left(\mC-\mD\right)\mL\right). 
\end{equation}
From the given condition $\mC\succeq\mD$, thus $\mC-\mD\succeq\mathbf{0}$, thus $\mL^\top\left(\mC-\mD\right)\mL$ is semi-positive. Therefore the trace of this matrix will be non-negative. Therefore $||\mC||_F \ge ||\mD||_F$.
\end{proof}

\label{appendix:covariance}
From Equations (\ref{eq:free_distribution}) and (\ref{eq:freq_distribution}), we know the covariance of refined noise for each method:
\begin{equation}
    \mathbf{\Sigma}_{FreeInit} = \mP^2 + (\mI-\mP)^2,\quad \mathbf{\Sigma}_{FreqPrior} = \mI - \frac{2\cos^2\theta}{1+\cos^2\theta}\mQ^2.
\end{equation}
Consider the same settings, including that the low-pass filters are identical, meaning $\mLambda_x$ is fixed.
Since the filter $\mathbf\Lambda_x$ is diagonal with its diagonal elements in the range $[0, 1]$, we have $\mathbf{0}\preceq\mLambda_x\preceq\mI$. 
Consequently, we obtain the following inequality for matrix $\mP$:
\begin{equation}
    \mathbf{0}\preceq \mP = \frac{1}{N}\left( \mA\mLambda_x\mA+\mB\mLambda_x\mB\right) \preceq \frac{1}{N}\left( \mA^2+\mB^2\right)=\mI.
\end{equation}
Now consider the difference between $\mathbf{\Sigma}_{FreeInit}$ and $\mI$:
\begin{equation}
\label{eq:cov_1}
    \mI-\mathbf{\Sigma}_{FreeInit} = 2\left(\mP - \mP^2\right)\succeq \mathbf{0}.
\end{equation}
This inequality holds because $\mathbf{0}\preceq \mP \preceq \mI$, which implies $\mP^2\preceq \mP$. This demonstrates that $\Sigma_{FreeInit}$ is indeed ``smaller'' than $\mI$.
To conduct a further analysis of $\Sigma_{FreeInit}$ and $\Sigma_{freqinit}$, we first establish the relationship between $\mP$ and $\mQ$.
Noticing that $\mP$ and $\mQ$ have similar forms,
we can derive the following results by leveraging these specific forms:
\begin{equation}
\label{eq:cov_2}
\begin{split}
    \mP^2 + \mQ^2 &= \frac{1}{N^2}\left(
    \left(\mA\mLambda_x\mA^2\mLambda_x\mA + \mB\mLambda_x\mB^2\mLambda_x\mB\right)+
    \left(\mA\mLambda_x\mB^2\mLambda_x\mA + \mB\mLambda_x\mA^2\mLambda_x\mB\right)\right) \\
    &=\frac{1}{N}\left(\mA\mLambda_x^2\mA+\mB\mLambda_x^2\mB\right) 
    \preceq 
    \frac{1}{N}\left(\mA\mLambda_x\mA+\mB\mLambda_x\mB\right) = \mP.
\end{split}
\end{equation}
Combining Equations (\ref{eq:cov_1}) and (\ref{eq:cov_2}), we obtain:
\begin{equation}
\label{eq:covariance_error_two}
    \mI-\mathbf{\Sigma}_{FreqPrior} = \frac{2\cos^2\theta}{1+\cos^2\theta}\mQ^2 \preceq \frac{2\cos^2\theta}{1+\cos^2\theta}\left(\mP - \mP^2\right) = \frac{\cos^2\theta}{1+\cos^2\theta}\left(\mI-\mathbf{\Sigma}_{FreeInit}\right).
\end{equation}
Then we can analyze the covariance errors (as defined in Definition~\ref{definition:1}) by applying Theorem~\ref{theorem:4}:
\begin{equation}
    ||\mI-\mathbf{\Sigma}_{FreqPrior}||_F 
    \le \frac{\cos^2\theta}{1+\cos^2\theta}||\mI-\mathbf{\Sigma}_{FreeInit}||_F. 
\end{equation}
In practice, for common continuous low-pass filters, such as Butterworth filters and Gaussian filters, the corresponding function values monotonically decrease as the frequency increases. 
Given that $\mA\mB=\mB\mA$ and $\mQ=\frac{1}{N}(\mA\mLambda\mB+\mB\mLambda\mA)$, it follows that $\mQ^2$ is intuitively close to a zero matrix, making the covariance error nearly zero. This is further corroborated by our numerical experiments.


\section{Experimental details}
\label{appendix:setting}
Three open-sourced text-to-video models are used as the base models for evaluation: They are AnimateDiff~\citep{guo2023animatediff}, ModelScope~\citep{wang2023modelscope,VideoFusion}, and VideoCrafter~\citep{chen2023videocrafter1}.
\begin{itemize}
    \item For AnimateDiff, we use mm-sd-v15\_v2 motion module along with realisticVisionV20\_v20 dreambooth LoRA~\footnote{https://huggingface.co/ckpt/realistic-vision-v20/blob/main/realisticVisionV20\_v20.safetensors}, sampling 16 frames of at a resolution of $512\times512$ at 8 FPS with a guidance scale of 7.5.
    \item For ModelScope, we utilize the modelscope-damo-text-to-video-synthesis version, sampling 16 frames at a resolution of $256 \times 256$ at 8 FPS, with a guidance scale of 9.
    \item For VideoCrafter, we employ the VideoCrafter-v1 base text-to-video model, sampling 16 frames at a resolution of  $320 \times 320$ at 10 FPS, with a guidance scale of 12.
\end{itemize}


\section{Evaluation metrics}
\label{appendix:vbench}
% vbench 
We employ VBench~\citep{huang2023vbench} for evaluation, a comprehensive benchmark designed with tailored prompts and evaluation dimensions specifically aimed at assessing video generation performance.
A key feature of VBench is its incorporation of human preference annotations, ensuring alignment with human perception.
VBench uses a hierarchical and disentangled scoring system, breaking the overall \textbf{\textit{total score}} into two main components: \textbf{\textit{quality score}} and \textbf{\textit{semantic score}}.
It covers 16 evaluation dimensions, with 7 contributing to \textbf{\textit{quality score}} and 9 contributing to \textbf{\textit{semantic score}}. 
Each dimension is assessed using a specially designed approach, ensuring precise and meaningful evaluation of the generated videos. The assessments involve various off-the-shelf models~\citep{caron2021dino, ruiz2023dreambooth, radford2021clip, li2023amt, teed2020raft, laion2022aesthetic,ke2021MUSIQ,wu2022GRiT, li2023umt,huang2023t2i-compbench,huang2024tagtext, wang2024internvid},  and the score for each dimension is normalized on a 0 to 100 scale, based on empirical minimum and maximum values.
\begin{itemize}
    \item \textbf{\textit{Quality score}} is calculated as the weighted average of seven dimensions: {\it subject consistency}, {\it background consistency}, {\it temporal flickering}, {\it motion smoothness}, {\it dynamic degree}, {\it aesthetic quality}, and {\it imaging quality}. The weight for {\it aesthetic quality} is set to 2, while the other dimensions carry a weight of 1. 
    % The rationale behind this is that there are several dimensions related to temporal consistency, while \textit{dynamic degree} is the only dimension that specifically measures motion dynamics.
    \item \textbf{\textit{semantic score}} is calculated as the weighted average of nine dimensions: {\it object class}, {\it multiple objects}, {\it human action}, {\it color}, {\it spatial relationship}, {\it scene},{\it appearance style}, {\it temporal style}, and {\it overall consistency}, with each dimension equally weighted 1.
\end{itemize}
 
After calculating \textbf{\textit{quality score}} and \textbf{\textit{semantic score}}, \textbf{\textit{total score}} is calculated as follows:
\begin{equation}
    \mathrm{Total} = \frac{w_q}{w_q+w_s}\mathrm{Quality}+\frac{w_s}{w_q+w_s}\mathrm{Semantic},
\end{equation}
where $w_q$ and $w_s$ are $4$ and $1$ respectively by default.

\section{Visualization results}
\paragraph{More qualitative results.} 
Additional qualitative results are presented in Figure~\ref{fig:appendix_vis}. 
The videos generated using our noise prior exhibit superior video quality, in terms of imaging details, aesthetic aspects, and semantic coherence.
\paragraph{Visualization of the influence of timestep \boldmath$t$.} 
As illustrated in Figure~\ref{fig:ablation_timestep},  the first three rows of frames, which correspond to different timesteps $t$, are almost the same.
In the fourth case, there are some differences in the representation of the grape stem, highlighted by a red box. The stem is absent at the timestep of 321. 
At the timestep of 321, the stem is missing. 
The final case demonstrates more notable differences; as the timestep $t$ increases, the ice cream appears to melt, and the changes are observable on the table.
The visualizations suggest two key points: first, in most instances, the timestep 
$t$ has minimal impact on the overall generation results; second, although the content and layout of the video frames remain largely unchanged, the timestep can indeed influence the finer imaging details.
In general, the differences are quite minor, which means we can save much time by diffusing the latent at intermediate timestep during the noise refinement stage without compromising the quality of generation results.

\begin{figure}[t]
    \centering
    \includegraphics[width=1.0\linewidth]{figure/ablation_timestep.pdf}
    \caption{ \textbf{Visualization results of the influence of timestep.} We present five cases where other settings are fixed to isolate the effects of varying timestep $t$. Overall, timestep $t$ has minimal impact on the generation outcomes. However, it does exert some influences on the imaging details occasionally. For the fourth and fifth cases, the red boxes highlight the differences.}
    \label{fig:ablation_timestep}
\end{figure}

\section{Limitations}
While our method enhances consistency and smoothness in videos generated from Gaussian noise, it can occasionally result in unnatural smoothness that does not align with the laws of physics. 
Additionally, although our approach improves overall performance, it may alter the content layout of video frames compared to Gaussian noise. For real images, low-frequency signals typically dictate layouts; however, this is not always true for the noise prior in diffusion models. 
Our method refines the Gaussian noise prior using a novel frequency filtering technique, which usually preserves the structural similarity to the original Gaussian noise. Nonetheless, in some cases, the generated videos can differ significantly. During filtering, high-frequency components from other Gaussian noise may subtly change the structure of the Gaussian noise prior, resulting in variations in the content and layouts of the generated videos.


\section{Broader impacts}
\label{appendix:impacts}
This work aims to propose a novel prior by refining initial Gaussian noise to enhance the quality of video generation. 
Text-to-video diffusion models hold the potential to revolutionize media creation and usage. While these models offer vast creative opportunities, it is crucial to address the risks of misinformation and harmful content. 
Before deploying these models in practice, it is essential to thoroughly investigate their design, intended applications, safety aspects, associated risks, and potential biases.

\begin{figure}[t]
    \centering
    \includegraphics[width=1.0\linewidth]{figure/visualization_appendix.pdf}
    \caption{{\bf More qualitative results.}}
    \label{fig:appendix_vis}
\end{figure}


\end{document}
