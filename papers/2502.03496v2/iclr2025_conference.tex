\documentclass{article}
\usepackage{iclr2025_conference,times}

% Optional math commands from https://github.com/goodfeli/dlbook_notation.
%%%%% NEW MATH DEFINITIONS %%%%%

\usepackage{amsmath,amsfonts,bm}
\usepackage{derivative}
% Mark sections of captions for referring to divisions of figures
\newcommand{\figleft}{{\em (Left)}}
\newcommand{\figcenter}{{\em (Center)}}
\newcommand{\figright}{{\em (Right)}}
\newcommand{\figtop}{{\em (Top)}}
\newcommand{\figbottom}{{\em (Bottom)}}
\newcommand{\captiona}{{\em (a)}}
\newcommand{\captionb}{{\em (b)}}
\newcommand{\captionc}{{\em (c)}}
\newcommand{\captiond}{{\em (d)}}

% Highlight a newly defined term
\newcommand{\newterm}[1]{{\bf #1}}

% Derivative d 
\newcommand{\deriv}{{\mathrm{d}}}

% Figure reference, lower-case.
\def\figref#1{figure~\ref{#1}}
% Figure reference, capital. For start of sentence
\def\Figref#1{Figure~\ref{#1}}
\def\twofigref#1#2{figures \ref{#1} and \ref{#2}}
\def\quadfigref#1#2#3#4{figures \ref{#1}, \ref{#2}, \ref{#3} and \ref{#4}}
% Section reference, lower-case.
\def\secref#1{section~\ref{#1}}
% Section reference, capital.
\def\Secref#1{Section~\ref{#1}}
% Reference to two sections.
\def\twosecrefs#1#2{sections \ref{#1} and \ref{#2}}
% Reference to three sections.
\def\secrefs#1#2#3{sections \ref{#1}, \ref{#2} and \ref{#3}}
% Reference to an equation, lower-case.
\def\eqref#1{equation~\ref{#1}}
% Reference to an equation, upper case
\def\Eqref#1{Equation~\ref{#1}}
% A raw reference to an equation---avoid using if possible
\def\plaineqref#1{\ref{#1}}
% Reference to a chapter, lower-case.
\def\chapref#1{chapter~\ref{#1}}
% Reference to an equation, upper case.
\def\Chapref#1{Chapter~\ref{#1}}
% Reference to a range of chapters
\def\rangechapref#1#2{chapters\ref{#1}--\ref{#2}}
% Reference to an algorithm, lower-case.
\def\algref#1{algorithm~\ref{#1}}
% Reference to an algorithm, upper case.
\def\Algref#1{Algorithm~\ref{#1}}
\def\twoalgref#1#2{algorithms \ref{#1} and \ref{#2}}
\def\Twoalgref#1#2{Algorithms \ref{#1} and \ref{#2}}
% Reference to a part, lower case
\def\partref#1{part~\ref{#1}}
% Reference to a part, upper case
\def\Partref#1{Part~\ref{#1}}
\def\twopartref#1#2{parts \ref{#1} and \ref{#2}}

\def\ceil#1{\lceil #1 \rceil}
\def\floor#1{\lfloor #1 \rfloor}
\def\1{\bm{1}}
\newcommand{\train}{\mathcal{D}}
\newcommand{\valid}{\mathcal{D_{\mathrm{valid}}}}
\newcommand{\test}{\mathcal{D_{\mathrm{test}}}}

\def\eps{{\epsilon}}


% Random variables
\def\reta{{\textnormal{$\eta$}}}
\def\ra{{\textnormal{a}}}
\def\rb{{\textnormal{b}}}
\def\rc{{\textnormal{c}}}
\def\rd{{\textnormal{d}}}
\def\re{{\textnormal{e}}}
\def\rf{{\textnormal{f}}}
\def\rg{{\textnormal{g}}}
\def\rh{{\textnormal{h}}}
\def\ri{{\textnormal{i}}}
\def\rj{{\textnormal{j}}}
\def\rk{{\textnormal{k}}}
\def\rl{{\textnormal{l}}}
% rm is already a command, just don't name any random variables m
\def\rn{{\textnormal{n}}}
\def\ro{{\textnormal{o}}}
\def\rp{{\textnormal{p}}}
\def\rq{{\textnormal{q}}}
\def\rr{{\textnormal{r}}}
\def\rs{{\textnormal{s}}}
\def\rt{{\textnormal{t}}}
\def\ru{{\textnormal{u}}}
\def\rv{{\textnormal{v}}}
\def\rw{{\textnormal{w}}}
\def\rx{{\textnormal{x}}}
\def\ry{{\textnormal{y}}}
\def\rz{{\textnormal{z}}}

% Random vectors
\def\rvepsilon{{\mathbf{\epsilon}}}
\def\rvphi{{\mathbf{\phi}}}
\def\rvtheta{{\mathbf{\theta}}}
\def\rva{{\mathbf{a}}}
\def\rvb{{\mathbf{b}}}
\def\rvc{{\mathbf{c}}}
\def\rvd{{\mathbf{d}}}
\def\rve{{\mathbf{e}}}
\def\rvf{{\mathbf{f}}}
\def\rvg{{\mathbf{g}}}
\def\rvh{{\mathbf{h}}}
\def\rvu{{\mathbf{i}}}
\def\rvj{{\mathbf{j}}}
\def\rvk{{\mathbf{k}}}
\def\rvl{{\mathbf{l}}}
\def\rvm{{\mathbf{m}}}
\def\rvn{{\mathbf{n}}}
\def\rvo{{\mathbf{o}}}
\def\rvp{{\mathbf{p}}}
\def\rvq{{\mathbf{q}}}
\def\rvr{{\mathbf{r}}}
\def\rvs{{\mathbf{s}}}
\def\rvt{{\mathbf{t}}}
\def\rvu{{\mathbf{u}}}
\def\rvv{{\mathbf{v}}}
\def\rvw{{\mathbf{w}}}
\def\rvx{{\mathbf{x}}}
\def\rvy{{\mathbf{y}}}
\def\rvz{{\mathbf{z}}}

% Elements of random vectors
\def\erva{{\textnormal{a}}}
\def\ervb{{\textnormal{b}}}
\def\ervc{{\textnormal{c}}}
\def\ervd{{\textnormal{d}}}
\def\erve{{\textnormal{e}}}
\def\ervf{{\textnormal{f}}}
\def\ervg{{\textnormal{g}}}
\def\ervh{{\textnormal{h}}}
\def\ervi{{\textnormal{i}}}
\def\ervj{{\textnormal{j}}}
\def\ervk{{\textnormal{k}}}
\def\ervl{{\textnormal{l}}}
\def\ervm{{\textnormal{m}}}
\def\ervn{{\textnormal{n}}}
\def\ervo{{\textnormal{o}}}
\def\ervp{{\textnormal{p}}}
\def\ervq{{\textnormal{q}}}
\def\ervr{{\textnormal{r}}}
\def\ervs{{\textnormal{s}}}
\def\ervt{{\textnormal{t}}}
\def\ervu{{\textnormal{u}}}
\def\ervv{{\textnormal{v}}}
\def\ervw{{\textnormal{w}}}
\def\ervx{{\textnormal{x}}}
\def\ervy{{\textnormal{y}}}
\def\ervz{{\textnormal{z}}}

% Random matrices
\def\rmA{{\mathbf{A}}}
\def\rmB{{\mathbf{B}}}
\def\rmC{{\mathbf{C}}}
\def\rmD{{\mathbf{D}}}
\def\rmE{{\mathbf{E}}}
\def\rmF{{\mathbf{F}}}
\def\rmG{{\mathbf{G}}}
\def\rmH{{\mathbf{H}}}
\def\rmI{{\mathbf{I}}}
\def\rmJ{{\mathbf{J}}}
\def\rmK{{\mathbf{K}}}
\def\rmL{{\mathbf{L}}}
\def\rmM{{\mathbf{M}}}
\def\rmN{{\mathbf{N}}}
\def\rmO{{\mathbf{O}}}
\def\rmP{{\mathbf{P}}}
\def\rmQ{{\mathbf{Q}}}
\def\rmR{{\mathbf{R}}}
\def\rmS{{\mathbf{S}}}
\def\rmT{{\mathbf{T}}}
\def\rmU{{\mathbf{U}}}
\def\rmV{{\mathbf{V}}}
\def\rmW{{\mathbf{W}}}
\def\rmX{{\mathbf{X}}}
\def\rmY{{\mathbf{Y}}}
\def\rmZ{{\mathbf{Z}}}

% Elements of random matrices
\def\ermA{{\textnormal{A}}}
\def\ermB{{\textnormal{B}}}
\def\ermC{{\textnormal{C}}}
\def\ermD{{\textnormal{D}}}
\def\ermE{{\textnormal{E}}}
\def\ermF{{\textnormal{F}}}
\def\ermG{{\textnormal{G}}}
\def\ermH{{\textnormal{H}}}
\def\ermI{{\textnormal{I}}}
\def\ermJ{{\textnormal{J}}}
\def\ermK{{\textnormal{K}}}
\def\ermL{{\textnormal{L}}}
\def\ermM{{\textnormal{M}}}
\def\ermN{{\textnormal{N}}}
\def\ermO{{\textnormal{O}}}
\def\ermP{{\textnormal{P}}}
\def\ermQ{{\textnormal{Q}}}
\def\ermR{{\textnormal{R}}}
\def\ermS{{\textnormal{S}}}
\def\ermT{{\textnormal{T}}}
\def\ermU{{\textnormal{U}}}
\def\ermV{{\textnormal{V}}}
\def\ermW{{\textnormal{W}}}
\def\ermX{{\textnormal{X}}}
\def\ermY{{\textnormal{Y}}}
\def\ermZ{{\textnormal{Z}}}

% Vectors
\def\vzero{{\bm{0}}}
\def\vone{{\bm{1}}}
\def\vmu{{\bm{\mu}}}
\def\vtheta{{\bm{\theta}}}
\def\vphi{{\bm{\phi}}}
\def\va{{\bm{a}}}
\def\vb{{\bm{b}}}
\def\vc{{\bm{c}}}
\def\vd{{\bm{d}}}
\def\ve{{\bm{e}}}
\def\vf{{\bm{f}}}
\def\vg{{\bm{g}}}
\def\vh{{\bm{h}}}
\def\vi{{\bm{i}}}
\def\vj{{\bm{j}}}
\def\vk{{\bm{k}}}
\def\vl{{\bm{l}}}
\def\vm{{\bm{m}}}
\def\vn{{\bm{n}}}
\def\vo{{\bm{o}}}
\def\vp{{\bm{p}}}
\def\vq{{\bm{q}}}
\def\vr{{\bm{r}}}
\def\vs{{\bm{s}}}
\def\vt{{\bm{t}}}
\def\vu{{\bm{u}}}
\def\vv{{\bm{v}}}
\def\vw{{\bm{w}}}
\def\vx{{\bm{x}}}
\def\vy{{\bm{y}}}
\def\vz{{\bm{z}}}

% Elements of vectors
\def\evalpha{{\alpha}}
\def\evbeta{{\beta}}
\def\evepsilon{{\epsilon}}
\def\evlambda{{\lambda}}
\def\evomega{{\omega}}
\def\evmu{{\mu}}
\def\evpsi{{\psi}}
\def\evsigma{{\sigma}}
\def\evtheta{{\theta}}
\def\eva{{a}}
\def\evb{{b}}
\def\evc{{c}}
\def\evd{{d}}
\def\eve{{e}}
\def\evf{{f}}
\def\evg{{g}}
\def\evh{{h}}
\def\evi{{i}}
\def\evj{{j}}
\def\evk{{k}}
\def\evl{{l}}
\def\evm{{m}}
\def\evn{{n}}
\def\evo{{o}}
\def\evp{{p}}
\def\evq{{q}}
\def\evr{{r}}
\def\evs{{s}}
\def\evt{{t}}
\def\evu{{u}}
\def\evv{{v}}
\def\evw{{w}}
\def\evx{{x}}
\def\evy{{y}}
\def\evz{{z}}

% Matrix
\def\mA{{\bm{A}}}
\def\mB{{\bm{B}}}
\def\mC{{\bm{C}}}
\def\mD{{\bm{D}}}
\def\mE{{\bm{E}}}
\def\mF{{\bm{F}}}
\def\mG{{\bm{G}}}
\def\mH{{\bm{H}}}
\def\mI{{\bm{I}}}
\def\mJ{{\bm{J}}}
\def\mK{{\bm{K}}}
\def\mL{{\bm{L}}}
\def\mM{{\bm{M}}}
\def\mN{{\bm{N}}}
\def\mO{{\bm{O}}}
\def\mP{{\bm{P}}}
\def\mQ{{\bm{Q}}}
\def\mR{{\bm{R}}}
\def\mS{{\bm{S}}}
\def\mT{{\bm{T}}}
\def\mU{{\bm{U}}}
\def\mV{{\bm{V}}}
\def\mW{{\bm{W}}}
\def\mX{{\bm{X}}}
\def\mY{{\bm{Y}}}
\def\mZ{{\bm{Z}}}
\def\mBeta{{\bm{\beta}}}
\def\mPhi{{\bm{\Phi}}}
\def\mLambda{{\bm{\Lambda}}}
\def\mSigma{{\bm{\Sigma}}}

% Tensor
\DeclareMathAlphabet{\mathsfit}{\encodingdefault}{\sfdefault}{m}{sl}
\SetMathAlphabet{\mathsfit}{bold}{\encodingdefault}{\sfdefault}{bx}{n}
\newcommand{\tens}[1]{\bm{\mathsfit{#1}}}
\def\tA{{\tens{A}}}
\def\tB{{\tens{B}}}
\def\tC{{\tens{C}}}
\def\tD{{\tens{D}}}
\def\tE{{\tens{E}}}
\def\tF{{\tens{F}}}
\def\tG{{\tens{G}}}
\def\tH{{\tens{H}}}
\def\tI{{\tens{I}}}
\def\tJ{{\tens{J}}}
\def\tK{{\tens{K}}}
\def\tL{{\tens{L}}}
\def\tM{{\tens{M}}}
\def\tN{{\tens{N}}}
\def\tO{{\tens{O}}}
\def\tP{{\tens{P}}}
\def\tQ{{\tens{Q}}}
\def\tR{{\tens{R}}}
\def\tS{{\tens{S}}}
\def\tT{{\tens{T}}}
\def\tU{{\tens{U}}}
\def\tV{{\tens{V}}}
\def\tW{{\tens{W}}}
\def\tX{{\tens{X}}}
\def\tY{{\tens{Y}}}
\def\tZ{{\tens{Z}}}


% Graph
\def\gA{{\mathcal{A}}}
\def\gB{{\mathcal{B}}}
\def\gC{{\mathcal{C}}}
\def\gD{{\mathcal{D}}}
\def\gE{{\mathcal{E}}}
\def\gF{{\mathcal{F}}}
\def\gG{{\mathcal{G}}}
\def\gH{{\mathcal{H}}}
\def\gI{{\mathcal{I}}}
\def\gJ{{\mathcal{J}}}
\def\gK{{\mathcal{K}}}
\def\gL{{\mathcal{L}}}
\def\gM{{\mathcal{M}}}
\def\gN{{\mathcal{N}}}
\def\gO{{\mathcal{O}}}
\def\gP{{\mathcal{P}}}
\def\gQ{{\mathcal{Q}}}
\def\gR{{\mathcal{R}}}
\def\gS{{\mathcal{S}}}
\def\gT{{\mathcal{T}}}
\def\gU{{\mathcal{U}}}
\def\gV{{\mathcal{V}}}
\def\gW{{\mathcal{W}}}
\def\gX{{\mathcal{X}}}
\def\gY{{\mathcal{Y}}}
\def\gZ{{\mathcal{Z}}}

% Sets
\def\sA{{\mathbb{A}}}
\def\sB{{\mathbb{B}}}
\def\sC{{\mathbb{C}}}
\def\sD{{\mathbb{D}}}
% Don't use a set called E, because this would be the same as our symbol
% for expectation.
\def\sF{{\mathbb{F}}}
\def\sG{{\mathbb{G}}}
\def\sH{{\mathbb{H}}}
\def\sI{{\mathbb{I}}}
\def\sJ{{\mathbb{J}}}
\def\sK{{\mathbb{K}}}
\def\sL{{\mathbb{L}}}
\def\sM{{\mathbb{M}}}
\def\sN{{\mathbb{N}}}
\def\sO{{\mathbb{O}}}
\def\sP{{\mathbb{P}}}
\def\sQ{{\mathbb{Q}}}
\def\sR{{\mathbb{R}}}
\def\sS{{\mathbb{S}}}
\def\sT{{\mathbb{T}}}
\def\sU{{\mathbb{U}}}
\def\sV{{\mathbb{V}}}
\def\sW{{\mathbb{W}}}
\def\sX{{\mathbb{X}}}
\def\sY{{\mathbb{Y}}}
\def\sZ{{\mathbb{Z}}}

% Entries of a matrix
\def\emLambda{{\Lambda}}
\def\emA{{A}}
\def\emB{{B}}
\def\emC{{C}}
\def\emD{{D}}
\def\emE{{E}}
\def\emF{{F}}
\def\emG{{G}}
\def\emH{{H}}
\def\emI{{I}}
\def\emJ{{J}}
\def\emK{{K}}
\def\emL{{L}}
\def\emM{{M}}
\def\emN{{N}}
\def\emO{{O}}
\def\emP{{P}}
\def\emQ{{Q}}
\def\emR{{R}}
\def\emS{{S}}
\def\emT{{T}}
\def\emU{{U}}
\def\emV{{V}}
\def\emW{{W}}
\def\emX{{X}}
\def\emY{{Y}}
\def\emZ{{Z}}
\def\emSigma{{\Sigma}}

% entries of a tensor
% Same font as tensor, without \bm wrapper
\newcommand{\etens}[1]{\mathsfit{#1}}
\def\etLambda{{\etens{\Lambda}}}
\def\etA{{\etens{A}}}
\def\etB{{\etens{B}}}
\def\etC{{\etens{C}}}
\def\etD{{\etens{D}}}
\def\etE{{\etens{E}}}
\def\etF{{\etens{F}}}
\def\etG{{\etens{G}}}
\def\etH{{\etens{H}}}
\def\etI{{\etens{I}}}
\def\etJ{{\etens{J}}}
\def\etK{{\etens{K}}}
\def\etL{{\etens{L}}}
\def\etM{{\etens{M}}}
\def\etN{{\etens{N}}}
\def\etO{{\etens{O}}}
\def\etP{{\etens{P}}}
\def\etQ{{\etens{Q}}}
\def\etR{{\etens{R}}}
\def\etS{{\etens{S}}}
\def\etT{{\etens{T}}}
\def\etU{{\etens{U}}}
\def\etV{{\etens{V}}}
\def\etW{{\etens{W}}}
\def\etX{{\etens{X}}}
\def\etY{{\etens{Y}}}
\def\etZ{{\etens{Z}}}

% The true underlying data generating distribution
\newcommand{\pdata}{p_{\rm{data}}}
\newcommand{\ptarget}{p_{\rm{target}}}
\newcommand{\pprior}{p_{\rm{prior}}}
\newcommand{\pbase}{p_{\rm{base}}}
\newcommand{\pref}{p_{\rm{ref}}}

% The empirical distribution defined by the training set
\newcommand{\ptrain}{\hat{p}_{\rm{data}}}
\newcommand{\Ptrain}{\hat{P}_{\rm{data}}}
% The model distribution
\newcommand{\pmodel}{p_{\rm{model}}}
\newcommand{\Pmodel}{P_{\rm{model}}}
\newcommand{\ptildemodel}{\tilde{p}_{\rm{model}}}
% Stochastic autoencoder distributions
\newcommand{\pencode}{p_{\rm{encoder}}}
\newcommand{\pdecode}{p_{\rm{decoder}}}
\newcommand{\precons}{p_{\rm{reconstruct}}}

\newcommand{\laplace}{\mathrm{Laplace}} % Laplace distribution

\newcommand{\E}{\mathbb{E}}
\newcommand{\Ls}{\mathcal{L}}
\newcommand{\R}{\mathbb{R}}
\newcommand{\emp}{\tilde{p}}
\newcommand{\lr}{\alpha}
\newcommand{\reg}{\lambda}
\newcommand{\rect}{\mathrm{rectifier}}
\newcommand{\softmax}{\mathrm{softmax}}
\newcommand{\sigmoid}{\sigma}
\newcommand{\softplus}{\zeta}
\newcommand{\KL}{D_{\mathrm{KL}}}
\newcommand{\Var}{\mathrm{Var}}
\newcommand{\standarderror}{\mathrm{SE}}
\newcommand{\Cov}{\mathrm{Cov}}
% Wolfram Mathworld says $L^2$ is for function spaces and $\ell^2$ is for vectors
% But then they seem to use $L^2$ for vectors throughout the site, and so does
% wikipedia.
\newcommand{\normlzero}{L^0}
\newcommand{\normlone}{L^1}
\newcommand{\normltwo}{L^2}
\newcommand{\normlp}{L^p}
\newcommand{\normmax}{L^\infty}

\newcommand{\parents}{Pa} % See usage in notation.tex. Chosen to match Daphne's book.

\DeclareMathOperator*{\argmax}{arg\,max}
\DeclareMathOperator*{\argmin}{arg\,min}

\DeclareMathOperator{\sign}{sign}
\DeclareMathOperator{\Tr}{Tr}
\let\ab\allowbreak


\usepackage[colorlinks]{hyperref}
\usepackage{url}

\usepackage{graphicx}
\usepackage{booktabs}
\usepackage{xcolor}
\usepackage{stackengine}
\usepackage{arydshln}
\usepackage{multicol,multirow}

\usepackage{algorithm}
\usepackage[noend]{algpseudocode}
\usepackage{algorithmicx}

\usepackage{amsthm}

\newtheorem{definition}{Definition}[section]
\newtheorem{theorem}{Theorem}[section]
\newtheorem{corollary}{Corollary}[theorem]
\newtheorem{proposition}[theorem]{Proposition}
\newtheorem{lemma}[theorem]{Lemma}
\newtheorem{assumption}{Assumption}

\newcommand{\bfx}{\mathbf{x}}
\newcommand{\bfy}{\mathbf{y}}
\newcommand{\bfz}{\mathbf{z}}


\title{FreqPrior: Improving Video Diffusion Models with Frequency Filtering Gaussian Noise}

% Authors must not appear in the submitted version. They should be hidden
% as long as the \iclrfinalcopy macro remains commented out below.
% Non-anonymous submissions will be rejected without review.


\author{Yunlong Yuan$^{1}$, Yuanfan Guo$^{2}$, Chunwei Wang$^{2}$, Wei Zhang$^{2}$, Hang Xu$^{2}$, Li Zhang$^{1}$\thanks{Corresponding author~\texttt{lizhangfd@fudan.edu.cn}.} \\
$^{1}$School of Data Science, Fudan University\quad $^{2}$Noah's Ark Lab, Huawei \vspace{.5em}\\
\url{https://github.com/fudan-zvg/FreqPrior}
}

% The \author macro works with any number of authors. There are two commands
% used to separate the names and addresses of multiple authors: \And and \AND.
%
% Using \And between authors leaves it to \LaTeX{} to determine where to break
% the lines. Using \AND forces a linebreak at that point. So, if \LaTeX{}
% puts 3 of 4 authors names on the first line, and the last on the second
% line, try using \AND instead of \And before the third author name.
\newcommand{\fix}{\marginpar{FIX}}
\newcommand{\new}{\marginpar{NEW}}

\iclrfinalcopy % Uncomment for camera-ready version, but NOT for submission.
\begin{document}


\maketitle
\begin{abstract}
Text-driven video generation has advanced significantly due to developments in diffusion models.
Beyond the training and sampling phases, recent studies have investigated noise priors of diffusion models, as improved noise priors yield better generation results.
One recent approach employs the Fourier transform to manipulate noise, marking the initial exploration of frequency operations in this context. However, it often generates videos that lack motion dynamics and imaging details.
In this work, we provide a comprehensive theoretical analysis of the variance decay issue present in existing methods, contributing to the loss of details and motion dynamics.
Recognizing the critical impact of noise distribution on generation quality, we introduce FreqPrior, a novel noise initialization strategy that refines noise in the frequency domain. 
Our method features a novel filtering technique designed to address different frequency signals while maintaining the noise prior distribution that closely approximates a standard Gaussian distribution.
Additionally, we propose a partial sampling process by perturbing the latent at an intermediate timestep during finding the noise prior, significantly reducing inference time without compromising quality.
Extensive experiments on VBench demonstrate that our method achieves the highest scores in both quality and semantic assessments, resulting in the best overall total score. These results highlight the superiority of our proposed noise prior.
\end{abstract}

\section{Introduction}
Backdoor attacks pose a concealed yet profound security risk to machine learning (ML) models, for which the adversaries can inject a stealth backdoor into the model during training, enabling them to illicitly control the model's output upon encountering predefined inputs. These attacks can even occur without the knowledge of developers or end-users, thereby undermining the trust in ML systems. As ML becomes more deeply embedded in critical sectors like finance, healthcare, and autonomous driving \citep{he2016deep, liu2020computing, tournier2019mrtrix3, adjabi2020past}, the potential damage from backdoor attacks grows, underscoring the emergency for developing robust defense mechanisms against backdoor attacks.

To address the threat of backdoor attacks, researchers have developed a variety of strategies \cite{liu2018fine,wu2021adversarial,wang2019neural,zeng2022adversarial,zhu2023neural,Zhu_2023_ICCV, wei2024shared,wei2024d3}, aimed at purifying backdoors within victim models. These methods are designed to integrate with current deployment workflows seamlessly and have demonstrated significant success in mitigating the effects of backdoor triggers \cite{wubackdoorbench, wu2023defenses, wu2024backdoorbench,dunnett2024countering}.  However, most state-of-the-art (SOTA) backdoor purification methods operate under the assumption that a small clean dataset, often referred to as \textbf{auxiliary dataset}, is available for purification. Such an assumption poses practical challenges, especially in scenarios where data is scarce. To tackle this challenge, efforts have been made to reduce the size of the required auxiliary dataset~\cite{chai2022oneshot,li2023reconstructive, Zhu_2023_ICCV} and even explore dataset-free purification techniques~\cite{zheng2022data,hong2023revisiting,lin2024fusing}. Although these approaches offer some improvements, recent evaluations \cite{dunnett2024countering, wu2024backdoorbench} continue to highlight the importance of sufficient auxiliary data for achieving robust defenses against backdoor attacks.

While significant progress has been made in reducing the size of auxiliary datasets, an equally critical yet underexplored question remains: \emph{how does the nature of the auxiliary dataset affect purification effectiveness?} In  real-world  applications, auxiliary datasets can vary widely, encompassing in-distribution data, synthetic data, or external data from different sources. Understanding how each type of auxiliary dataset influences the purification effectiveness is vital for selecting or constructing the most suitable auxiliary dataset and the corresponding technique. For instance, when multiple datasets are available, understanding how different datasets contribute to purification can guide defenders in selecting or crafting the most appropriate dataset. Conversely, when only limited auxiliary data is accessible, knowing which purification technique works best under those constraints is critical. Therefore, there is an urgent need for a thorough investigation into the impact of auxiliary datasets on purification effectiveness to guide defenders in  enhancing the security of ML systems. 

In this paper, we systematically investigate the critical role of auxiliary datasets in backdoor purification, aiming to bridge the gap between idealized and practical purification scenarios.  Specifically, we first construct a diverse set of auxiliary datasets to emulate real-world conditions, as summarized in Table~\ref{overall}. These datasets include in-distribution data, synthetic data, and external data from other sources. Through an evaluation of SOTA backdoor purification methods across these datasets, we uncover several critical insights: \textbf{1)} In-distribution datasets, particularly those carefully filtered from the original training data of the victim model, effectively preserve the model’s utility for its intended tasks but may fall short in eliminating backdoors. \textbf{2)} Incorporating OOD datasets can help the model forget backdoors but also bring the risk of forgetting critical learned knowledge, significantly degrading its overall performance. Building on these findings, we propose Guided Input Calibration (GIC), a novel technique that enhances backdoor purification by adaptively transforming auxiliary data to better align with the victim model’s learned representations. By leveraging the victim model itself to guide this transformation, GIC optimizes the purification process, striking a balance between preserving model utility and mitigating backdoor threats. Extensive experiments demonstrate that GIC significantly improves the effectiveness of backdoor purification across diverse auxiliary datasets, providing a practical and robust defense solution.

Our main contributions are threefold:
\textbf{1) Impact analysis of auxiliary datasets:} We take the \textbf{first step}  in systematically investigating how different types of auxiliary datasets influence backdoor purification effectiveness. Our findings provide novel insights and serve as a foundation for future research on optimizing dataset selection and construction for enhanced backdoor defense.
%
\textbf{2) Compilation and evaluation of diverse auxiliary datasets:}  We have compiled and rigorously evaluated a diverse set of auxiliary datasets using SOTA purification methods, making our datasets and code publicly available to facilitate and support future research on practical backdoor defense strategies.
%
\textbf{3) Introduction of GIC:} We introduce GIC, the \textbf{first} dedicated solution designed to align auxiliary datasets with the model’s learned representations, significantly enhancing backdoor mitigation across various dataset types. Our approach sets a new benchmark for practical and effective backdoor defense.



\section{Related Work}
\label{sec:related-works}
\subsection{Novel View Synthesis}
Novel view synthesis is a foundational task in the computer vision and graphics, which aims to generate unseen views of a scene from a given set of images.
% Many methods have been designed to solve this problem by posing it as 3D geometry based rendering, where point clouds~\cite{point_differentiable,point_nfs}, mesh~\cite{worldsheet,FVS,SVS}, planes~\cite{automatci_photo_pop_up,tour_into_the_picture} and multi-plane images~\cite{MINE,single_view_mpi,stereo_magnification}, \etal
Numerous methods have been developed to address this problem by approaching it as 3D geometry-based rendering, such as using meshes~\cite{worldsheet,FVS,SVS}, MPI~\cite{MINE,single_view_mpi,stereo_magnification}, point clouds~\cite{point_differentiable,point_nfs}, etc.
% planes~\cite{automatci_photo_pop_up,tour_into_the_picture}, 


\begin{figure*}[!t]
    \centering
    \includegraphics[width=1.0\linewidth]{figures/overview-v7.png}
    %\caption{\textbf{Overview.} Given a set of images, our method obtains both camera intrinsics and extrinsics, as well as a 3DGS model. First, we obtain the initial camera parameters, global track points from image correspondences and monodepth with reprojection loss. Then we incorporate the global track information and select Gaussian kernels associated with track points. We jointly optimize the parameters $K$, $T_{cw}$, 3DGS through multi-view geometric consistency $L_{t2d}$, $L_{t3d}$, $L_{scale}$ and photometric consistency $L_1$, $L_{D-SSIM}$.}
    \caption{\textbf{Overview.} Given a set of images, our method obtains both camera intrinsics and extrinsics, as well as a 3DGS model. During the initialization, we extract the global tracks, and initialize camera parameters and Gaussians from image correspondences and monodepth with reprojection loss. We determine Gaussian kernels with recovered 3D track points, and then jointly optimize the parameters $K$, $T_{cw}$, 3DGS through the proposed global track constraints (i.e., $L_{t2d}$, $L_{t3d}$, and $L_{scale}$) and original photometric losses (i.e., $L_1$ and $L_{D-SSIM}$).}
    \label{fig:overview}
\end{figure*}

Recently, Neural Radiance Fields (NeRF)~\cite{2020NeRF} provide a novel solution to this problem by representing scenes as implicit radiance fields using neural networks, achieving photo-realistic rendering quality. Although having some works in improving efficiency~\cite{instant_nerf2022, lin2022enerf}, the time-consuming training and rendering still limit its practicality.
Alternatively, 3D Gaussian Splatting (3DGS)~\cite{3DGS2023} models the scene as explicit Gaussian kernels, with differentiable splatting for rendering. Its improved real-time rendering performance, lower storage and efficiency, quickly attract more attentions.
% Different from NeRF-based methods which need MLPs to model the scene and huge computational cost for rendering, 3DGS has stronger real-time performance, higher storage and computational efficiency, benefits from its explicit representation and gradient backpropagation.

\subsection{Optimizing Camera Poses in NeRFs and 3DGS}
Although NeRF and 3DGS can provide impressive scene representation, these methods all need accurate camera parameters (both intrinsic and extrinsic) as additional inputs, which are mostly obtained by COLMAP~\cite{colmap2016}.
% This strong reliance on COLMAP significantly limits their use in real-world applications, so optimizing the camera parameters during the scene training becomes crucial.
When the prior is inaccurate or unknown, accurately estimating camera parameters and scene representations becomes crucial.

% In early works, only photometric constraints are used for scene training and camera pose estimation. 
% iNeRF~\cite{iNerf2021} optimizes the camera poses based on a pre-trained NeRF model.
% NeRFmm~\cite{wang2021nerfmm} introduce a joint optimization process, which estimates the camera poses and trains NeRF model jointly.
% BARF~\cite{barf2021} and GARF~\cite{2022GARF} provide new positional encoding strategy to handle with the gradient inconsistency issue of positional embedding and yield promising results.
% However, they achieve satisfactory optimization results when only the pose initialization is quite closed to the ground-truth, as the photometric constrains can only improve the quality of camera estimation within a small range.
% Later, more prior information of geometry and correspondence, \ie monocular depth and feature matching, are introduced into joint optimisation to enhance the capability of camera poses estimation.
% SC-NeRF~\cite{SCNeRF2021} minimizes a projected ray distance loss based on correspondence of adjacent frames.
% NoPe-NeRF~\cite{bian2022nopenerf} chooses monocular depth maps as geometric priors, and defines undistorted depth loss and relative pose constraints for joint optimization.
In earlier studies, scene training and camera pose estimation relied solely on photometric constraints. iNeRF~\cite{iNerf2021} refines the camera poses using a pre-trained NeRF model. NeRFmm~\cite{wang2021nerfmm} introduces a joint optimization approach that simultaneously estimates camera poses and trains the NeRF model. BARF~\cite{barf2021} and GARF~\cite{2022GARF} propose a new positional encoding strategy to address the gradient inconsistency issues in positional embedding, achieving promising results. However, these methods only yield satisfactory optimization when the initial pose is very close to the ground truth, as photometric constraints alone can only enhance camera estimation quality within a limited range. Subsequently, 
% additional prior information on geometry and correspondence, such as monocular depth and feature matching, has been incorporated into joint optimization to improve the accuracy of camera pose estimation. 
SC-NeRF~\cite{SCNeRF2021} minimizes a projected ray distance loss based on correspondence between adjacent frames. NoPe-NeRF~\cite{bian2022nopenerf} utilizes monocular depth maps as geometric priors and defines undistorted depth loss and relative pose constraints.

% With regard to 3D Gaussian Splatting, CF-3DGS~\cite{CF-3DGS-2024} also leverages mono-depth information to constrain the optimization of local 3DGS for relative pose estimation and later learn a global 3DGS progressively in a sequential manner.
% InstantSplat~\cite{fan2024instantsplat} focus on sparse view scenes, first use DUSt3R~\cite{dust3r2024cvpr} to generate a set of densely covered and pixel-aligned points for 3D Gaussian initialization, then introduce a parallel grid partitioning strategy in joint optimization to speed up.
% % Jiang et al.~\cite{Jiang_2024sig} proposed to build the scene continuously and progressively, to next unregistered frame, they use registration and adjustment to adjust the previous registered camera poses and align unregistered monocular depths, later refine the joint model by matching detected correspondences in screen-space coordinates.
% \gjh{Jiang et al.~\cite{Jiang_2024sig} also implemented an incremental approach for reconstructing camera poses and scenes. Initially, they perform feature matching between the current image and the image rendered by a differentiable surface renderer. They then construct matching point errors, depth errors, and photometric errors to achieve the registration and adjustment of the current image. Finally, based on the depth map, the pixels of the current image are projected as new 3D Gaussians. However, this method still exhibits limitations when dealing with complex scenes and unordered images.}
% % CG-3DGS~\cite{sun2024correspondenceguidedsfmfree3dgaussian} follows CF-3DGS, first construct a coarse point cloud from mono-depth maps to train a 3DGS model, then progressively estimate camera poses based on this pre-trained model by constraining the correspondences between rendering view and ground-truth.
% \gjh{Similarly, CG-3DGS~\cite{sun2024correspondenceguidedsfmfree3dgaussian} first utilizes monocular depth estimation and the camera parameters from the first frame to initialize a set of 3D Gaussians. It then progressively estimates camera poses based on this pre-trained model by constraining the correspondences between the rendered views and the ground truth.}
% % Free-SurGS~\cite{freesurgs2024} matches the projection flow derived from 3D Gaussians with optical flow to estimate the poses, to compensate for the limitations of photometric loss.
% \gjh{Free-SurGS~\cite{freesurgs2024} introduces the first SfM-free 3DGS approach for surgical scene reconstruction. Due to the challenges posed by weak textures and photometric inconsistencies in surgical scenes, Free-SurGS achieves pose estimation by minimizing the flow loss between the projection flow and the optical flow. Subsequently, it keeps the camera pose fixed and optimizes the scene representation by minimizing the photometric loss, depth loss and flow loss.}
% \gjh{However, most current works assume camera intrinsics are known and primarily focus on optimizing camera poses. Additionally, these methods typically rely on sequentially ordered image inputs and incrementally optimize camera parameters and scene representation. This inevitably leads to drift errors, preventing the achievement of globally consistent results. Our work aims to address these issues.}

Regarding 3D Gaussian Splatting, CF-3DGS~\cite{CF-3DGS-2024} utilizes mono-depth information to refine the optimization of local 3DGS for relative pose estimation and subsequently learns a global 3DGS in a sequential manner. InstantSplat~\cite{fan2024instantsplat} targets sparse view scenes, initially employing DUSt3R~\cite{dust3r2024cvpr} to create a densely covered, pixel-aligned point set for initializing 3D Gaussian models, and then implements a parallel grid partitioning strategy to accelerate joint optimization. Jiang \etal~\cite{Jiang_2024sig} develops an incremental method for reconstructing camera poses and scenes, but it struggles with complex scenes and unordered images. 
% Similarly, CG-3DGS~\cite{sun2024correspondenceguidedsfmfree3dgaussian} progressively estimates camera poses using a pre-trained model by aligning the correspondences between rendered views and actual scenes. Free-SurGS~\cite{freesurgs2024} pioneers an SfM-free 3DGS method for reconstructing surgical scenes, overcoming challenges such as weak textures and photometric inconsistencies by minimizing the discrepancy between projection flow and optical flow.
%\pb{SF-3DGS-HT~\cite{ji2024sfmfree3dgaussiansplatting} introduced VFI into training as additional photometric constraints. They separated the whole scene into several local 3DGS models and then merged them hierarchically, which leads to a significant improvement on simple and dense view scenes.}
HT-3DGS~\cite{ji2024sfmfree3dgaussiansplatting} interpolates frames for training and splits the scene into local clips, using a hierarchical strategy to build 3DGS model. It works well for simple scenes, but fails with dramatic motions due to unstable interpolation and low efficiency.
% {While effective for simple scenes, it struggles with dramatic motion due to unstable view interpolation and suffers from low computational efficiency.}

However, most existing methods generally depend on sequentially ordered image inputs and incrementally optimize camera parameters and 3DGS, which often leads to drift errors and hinders achieving globally consistent results. Our work seeks to overcome these limitations.

\section{Study Design}
% robot: aliengo 
% We used the Unitree AlienGo quadruped robot. 
% See Appendix 1 in AlienGo Software Guide PDF
% Weight = 25kg, size (L,W,H) = (0.55, 0.35, 06) m when standing, (0.55, 0.35, 0.31) m when walking
% Handle is 0.4 m or 0.5 m. I'll need to check it to see which type it is.
We gathered input from primary stakeholders of the robot dog guide, divided into three subgroups: BVI individuals who have owned a dog guide, BVI individuals who were not dog guide owners, and sighted individuals with generally low degrees of familiarity with dog guides. While the main focus of this study was on the BVI participants, we elected to include survey responses from sighted participants given the importance of social acceptance of the robot by the general public, which could reflect upon the BVI users themselves and affect their interactions with the general population \cite{kayukawa2022perceive}. 

The need-finding processes consisted of two stages. During Stage 1, we conducted in-depth interviews with BVI participants, querying their experiences in using conventional assistive technologies and dog guides. During Stage 2, a large-scale survey was distributed to both BVI and sighted participants. 

This study was approved by the University’s Institutional Review Board (IRB), and all processes were conducted after obtaining the participants' consent.

\subsection{Stage 1: Interviews}
We recruited nine BVI participants (\textbf{Table}~\ref{tab:bvi-info}) for in-depth interviews, which lasted 45-90 minutes for current or former dog guide owners (DO) and 30-60 minutes for participants without dog guides (NDO). Group DO consisted of five participants, while Group NDO consisted of four participants.
% The interview participants were divided into two groups. Group DO (Dog guide Owner) consisted of five participants who were current or former dog guide owners and Group NDO (Non Dog guide Owner) consisted of three participants who were not dog guide owners. 
All participants were familiar with using white canes as a mobility aid. 

We recruited participants in both groups, DO and NDO, to gather data from those with substantial experience with dog guides, offering potentially more practical insights, and from those without prior experience, providing a perspective that may be less constrained and more open to novel approaches. 

We asked about the participants' overall impressions of a robot dog guide, expectations regarding its potential benefits and challenges compared to a conventional dog guide, their desired methods of giving commands and communicating with the robot dog guide, essential functionalities that the robot dog guide should offer, and their preferences for various aspects of the robot dog guide's form factors. 
For Group DO, we also included questions that asked about the participants' experiences with conventional dog guides. 

% We obtained permission to record the conversations for our records while simultaneously taking notes during the interviews. The interviews lasted 30-60 minutes for NDO participants and 45-90 minutes for DO participants. 

\subsection{Stage 2: Large-Scale Surveys} 
After gathering sufficient initial results from the interviews, we created an online survey for distributing to a larger pool of participants. The survey platform used was Qualtrics. 

\subsubsection{Survey Participants}
The survey had 100 participants divided into two primary groups. Group BVI consisted of 42 blind or visually impaired participants, and Group ST consisted of 58 sighted participants. \textbf{Table}~\ref{tab:survey-demographics} shows the demographic information of the survey participants. 

\subsubsection{Question Differentiation} 
Based on their responses to initial qualifying questions, survey participants were sorted into three subgroups: DO, NDO, and ST. Each participant was assigned one of three different versions of the survey. The surveys for BVI participants mirrored the interview categories (overall impressions, communication methods, functionalities, and form factors), but with a more quantitative approach rather than the open-ended questions used in interviews. The DO version included additional questions pertaining to their prior experience with dog guides. The ST version revolved around the participants' prior interactions with and feelings toward dog guides and dogs in general, their thoughts on a robot dog guide, and broad opinions on the aesthetic component of the robot's design. 

\section{Experiments}

\subsection{Setups}
\subsubsection{Implementation Details}
We apply our FDS method to two types of 3DGS: 
the original 3DGS, and 2DGS~\citep{huang20242d}. 
%
The number of iterations in our optimization 
process is 35,000.
We follow the default training configuration 
and apply our FDS method after 15,000 iterations,
then we add normal consistency loss for both
3DGS and 2DGS after 25000 iterations.
%
The weight for FDS, $\lambda_{fds}$, is set to 0.015,
the $\sigma$ is set to 23,
and the weight for normal consistency is set to 0.15
for all experiments. 
We removed the depth distortion loss in 2DGS 
because we found that it degrades its results in indoor scenes.
%
The Gaussian point cloud is initialized using Colmap
for all datasets.
%
%
We tested the impact of 
using Sea Raft~\citep{wang2025sea} and 
Raft\citep{teed2020raft} on FDS performance.
%
Due to the blurriness of the ScanNet dataset, 
additional prior constraints are required.
Thus, we incorporate normal prior supervision 
on the rendered normals 
in ScanNet (V2) dataset by default.
The normal prior is predicted by the Stable Normal 
model~\citep{ye2024stablenormal}
across all types of 3DGS.
%
The entire framework is implemented in 
PyTorch~\citep{paszke2019pytorch}, 
and all experiments are conducted on 
a single NVIDIA 4090D GPU.

\begin{figure}[t] \centering
    \makebox[0.16\textwidth]{\scriptsize Input}
    \makebox[0.16\textwidth]{\scriptsize 3DGS}
    \makebox[0.16\textwidth]{\scriptsize 2DGS}
    \makebox[0.16\textwidth]{\scriptsize 3DGS + FDS}
    \makebox[0.16\textwidth]{\scriptsize 2DGS + FDS}
    \makebox[0.16\textwidth]{\scriptsize GT (Depth)}

    \includegraphics[width=0.16\textwidth]{figure/fig3_img/compare3/gt_rgb/frame_00522.jpg}
    \includegraphics[width=0.16\textwidth]{figure/fig3_img/compare3/3DGS/frame_00522.jpg}
    \includegraphics[width=0.16\textwidth]{figure/fig3_img/compare3/2DGS/frame_00522.jpg}
    \includegraphics[width=0.16\textwidth]{figure/fig3_img/compare3/3DGS+FDS/frame_00522.jpg}
    \includegraphics[width=0.16\textwidth]{figure/fig3_img/compare3/2DGS+FDS/frame_00522.jpg}
    \includegraphics[width=0.16\textwidth]{figure/fig3_img/compare3/gt_depth/frame_00522.jpg} \\

    % \includegraphics[width=0.16\textwidth]{figure/fig3_img/compare1/gt_rgb/frame_00137.jpg}
    % \includegraphics[width=0.16\textwidth]{figure/fig3_img/compare1/3DGS/frame_00137.jpg}
    % \includegraphics[width=0.16\textwidth]{figure/fig3_img/compare1/2DGS/frame_00137.jpg}
    % \includegraphics[width=0.16\textwidth]{figure/fig3_img/compare1/3DGS+FDS/frame_00137.jpg}
    % \includegraphics[width=0.16\textwidth]{figure/fig3_img/compare1/2DGS+FDS/frame_00137.jpg}
    % \includegraphics[width=0.16\textwidth]{figure/fig3_img/compare1/gt_depth/frame_00137.jpg} \\

     \includegraphics[width=0.16\textwidth]{figure/fig3_img/compare2/gt_rgb/frame_00262.jpg}
    \includegraphics[width=0.16\textwidth]{figure/fig3_img/compare2/3DGS/frame_00262.jpg}
    \includegraphics[width=0.16\textwidth]{figure/fig3_img/compare2/2DGS/frame_00262.jpg}
    \includegraphics[width=0.16\textwidth]{figure/fig3_img/compare2/3DGS+FDS/frame_00262.jpg}
    \includegraphics[width=0.16\textwidth]{figure/fig3_img/compare2/2DGS+FDS/frame_00262.jpg}
    \includegraphics[width=0.16\textwidth]{figure/fig3_img/compare2/gt_depth/frame_00262.jpg} \\

    \includegraphics[width=0.16\textwidth]{figure/fig3_img/compare4/gt_rgb/frame00000.png}
    \includegraphics[width=0.16\textwidth]{figure/fig3_img/compare4/3DGS/frame00000.png}
    \includegraphics[width=0.16\textwidth]{figure/fig3_img/compare4/2DGS/frame00000.png}
    \includegraphics[width=0.16\textwidth]{figure/fig3_img/compare4/3DGS+FDS/frame00000.png}
    \includegraphics[width=0.16\textwidth]{figure/fig3_img/compare4/2DGS+FDS/frame00000.png}
    \includegraphics[width=0.16\textwidth]{figure/fig3_img/compare4/gt_depth/frame00000.png} \\

    \includegraphics[width=0.16\textwidth]{figure/fig3_img/compare5/gt_rgb/frame00080.png}
    \includegraphics[width=0.16\textwidth]{figure/fig3_img/compare5/3DGS/frame00080.png}
    \includegraphics[width=0.16\textwidth]{figure/fig3_img/compare5/2DGS/frame00080.png}
    \includegraphics[width=0.16\textwidth]{figure/fig3_img/compare5/3DGS+FDS/frame00080.png}
    \includegraphics[width=0.16\textwidth]{figure/fig3_img/compare5/2DGS+FDS/frame00080.png}
    \includegraphics[width=0.16\textwidth]{figure/fig3_img/compare5/gt_depth/frame00080.png} \\



    \caption{\textbf{Comparison of depth reconstruction on Mushroom and ScanNet datasets.} The original
    3DGS or 2DGS model equipped with FDS can remove unwanted floaters and reconstruct
    geometry more preciously.}
    \label{fig:compare}
\end{figure}


\subsubsection{Datasets and Metrics}

We evaluate our method for 3D reconstruction 
and novel view synthesis tasks on
\textbf{Mushroom}~\citep{ren2024mushroom},
\textbf{ScanNet (v2)}~\citep{dai2017scannet}, and 
\textbf{Replica}~\citep{replica19arxiv}
datasets,
which feature challenging indoor scenes with both 
sparse and dense image sampling.
%
The Mushroom dataset is an indoor dataset 
with sparse image sampling and two distinct 
camera trajectories. 
%
We train our model on the training split of 
the long capture sequence and evaluate 
novel view synthesis on the test split 
of the long capture sequences.
%
Five scenes are selected to evaluate our FDS, 
including "coffee room", "honka", "kokko", 
"sauna", and "vr room". 
%
ScanNet(V2)~\citep{dai2017scannet}  consists of 1,613 indoor scenes
with annotated camera poses and depth maps. 
%
We select 5 scenes from the ScanNet (V2) dataset, 
uniformly sampling one-tenth of the views,
following the approach in ~\citep{guo2022manhattan}.
To further improve the geometry rendering quality of 3DGS, 
%
Replica~\citep{replica19arxiv} contains small-scale 
real-world indoor scans. 
We evaluate our FDS on five scenes from 
Replica: office0, office1, office2, office3 and office4,
selecting one-tenth of the views for training.
%
The results for Replica are provided in the 
supplementary materials.
To evaluate the rendering quality and geometry 
of 3DGS, we report PSNR, SSIM, and LPIPS for 
rendering quality, along with Absolute Relative Distance 
(Abs Rel) as a depth quality metrics.
%
Additionally, for mesh evaluation, 
we use metrics including Accuracy, Completion, 
Chamfer-L1 distance, Normal Consistency, 
and F-scores.




\subsection{Results}
\subsubsection{Depth rendering and novel view synthesis}
The comparison results on Mushroom and 
ScanNet are presented in \tabref{tab:mushroom} 
and \tabref{tab:scannet}, respectively. 
%
Due to the sparsity of sampling 
in the Mushroom dataset,
challenges are posed for both GOF~\citep{yu2024gaussian} 
and PGSR~\citep{chen2024pgsr}, 
leading to their relative poor performance 
on the Mushroom dataset.
%
Our approach achieves the best performance 
with the FDS method applied during the training process.
The FDS significantly enhances the 
geometric quality of 3DGS on the Mushroom dataset, 
improving the "abs rel" metric by more than 50\%.
%
We found that Sea Raft~\citep{wang2025sea}
outperforms Raft~\citep{teed2020raft} on FDS, 
indicating that a better optical flow model 
can lead to more significant improvements.
%
Additionally, the render quality of RGB 
images shows a slight improvement, 
by 0.58 in 3DGS and 0.50 in 2DGS, 
benefiting from the incorporation of cross-view consistency in FDS. 
%
On the Mushroom
dataset, adding the FDS loss increases 
the training time by half an hour, which maintains the same
level as baseline.
%
Similarly, our method shows a notable improvement on the ScanNet dataset as well using Sea Raft~\citep{wang2025sea} Model. The "abs rel" metric in 2DGS is improved nearly 50\%. This demonstrates the robustness and effectiveness of the FDS method across different datasets.
%


% \begin{wraptable}{r}{0.6\linewidth} \centering
% \caption{\textbf{Ablation study on geometry priors.}} 
%         \label{tab:analysis_prior}
%         \resizebox{\textwidth}{!}{
\begin{tabular}{c| c c c c c | c c c c}

    \hline
     Method &  Acc$\downarrow$ & Comp $\downarrow$ & C-L1 $\downarrow$ & NC $\uparrow$ & F-Score $\uparrow$ &  Abs Rel $\downarrow$ &  PSNR $\uparrow$  & SSIM  $\uparrow$ & LPIPS $\downarrow$ \\ \hline
    2DGS&   0.1078&  0.0850&  0.0964&  0.7835&  0.5170&  0.1002&  23.56&  0.8166& 0.2730\\
    2DGS+Depth&   0.0862&  0.0702&  0.0782&  0.8153&  0.5965&  0.0672&  23.92&  0.8227& 0.2619 \\
    2DGS+MVDepth&   0.2065&  0.0917&  0.1491&  0.7832&  0.3178&  0.0792&  23.74&  0.8193& 0.2692 \\
    2DGS+Normal&   0.0939&  0.0637&  0.0788&  \textbf{0.8359}&  0.5782&  0.0768&  23.78&  0.8197& 0.2676 \\
    2DGS+FDS &  \textbf{0.0615} & \textbf{ 0.0534}& \textbf{0.0574}& 0.8151& \textbf{0.6974}&  \textbf{0.0561}&  \textbf{24.06}&  \textbf{0.8271}&\textbf{0.2610} \\ \hline
    2DGS+Depth+FDS &  0.0561 &  0.0519& 0.0540& 0.8295& 0.7282&  0.0454&  \textbf{24.22}& \textbf{0.8291}&\textbf{0.2570} \\
    2DGS+Normal+FDS &  \textbf{0.0529} & \textbf{ 0.0450}& \textbf{0.0490}& \textbf{0.8477}& \textbf{0.7430}&  \textbf{0.0443}&  24.10&  0.8283& 0.2590 \\
    2DGS+Depth+Normal &  0.0695 & 0.0513& 0.0604& 0.8540&0.6723&  0.0523&  24.09&  0.8264&0.2575\\ \hline
    2DGS+Depth+Normal+FDS &  \textbf{0.0506} & \textbf{0.0423}& \textbf{0.0464}& \textbf{0.8598}&\textbf{0.7613}&  \textbf{0.0403}&  \textbf{24.22}& 
    \textbf{0.8300}&\textbf{0.0403}\\
    
\bottomrule
\end{tabular}
}
% \end{wraptable}



The qualitative comparisons on the Mushroom and ScanNet dataset 
are illustrated in \figref{fig:compare}. 
%
%
As seen in the first row of \figref{fig:compare}, 
both the original 3DGS and 2DGS suffer from overfitting, 
leading to corrupted geometry generation. 
%
Our FDS effectively mitigates this issue by 
supervising the matching relationship between 
the input and sampled views, 
helping to recover the geometry.
%
FDS also improves the refinement of geometric details, 
as shown in other rows. 
By incorporating the matching prior through FDS, 
the quality of the rendered depth is significantly improved.
%

\begin{table}[t] \centering
\begin{minipage}[t]{0.96\linewidth}
        \captionof{table}{\textbf{3D Reconstruction 
        and novel view synthesis results on Mushroom dataset. * 
        Represents that FDS uses the Raft model.
        }}
        \label{tab:mushroom}
        \resizebox{\textwidth}{!}{
\begin{tabular}{c| c c c c c | c c c c c}
    \hline
     Method &  Acc$\downarrow$ & Comp $\downarrow$ & C-L1 $\downarrow$ & NC $\uparrow$ & F-Score $\uparrow$ &  Abs Rel $\downarrow$ &  PSNR $\uparrow$  & SSIM  $\uparrow$ & LPIPS $\downarrow$ & Time  $\downarrow$ \\ \hline

    % DN-splatter &   &  &  &  &  &  &  &  & \\
    GOF &  0.1812 & 0.1093 & 0.1453 & 0.6292 & 0.3665 & 0.2380  & 21.37  &  0.7762  & 0.3132  & $\approx$1.4h\\ 
    PGSR &  0.0971 & 0.1420 & 0.1196 & 0.7193 & 0.5105 & 0.1723  & 22.13  & 0.7773  & 0.2918  & $\approx$1.2h \\ \hline
    3DGS &   0.1167 &  0.1033&  0.1100&  0.7954&  0.3739&  0.1214&  24.18&  0.8392& 0.2511 &$\approx$0.8h \\
    3DGS + FDS$^*$ & 0.0569  & 0.0676 & 0.0623 & 0.8105 & 0.6573 & 0.0603 & 24.72  & 0.8489 & 0.2379 &$\approx$1.3h \\
    3DGS + FDS & \textbf{0.0527}  & \textbf{0.0565} & \textbf{0.0546} & \textbf{0.8178} & \textbf{0.6958} & \textbf{0.0568} & \textbf{24.76}  & \textbf{0.8486} & \textbf{0.2381} &$\approx$1.3h \\ \hline
    2DGS&   0.1078&  0.0850&  0.0964&  0.7835&  0.5170&  0.1002&  23.56&  0.8166& 0.2730 &$\approx$0.8h\\
    2DGS + FDS$^*$ &  0.0689 &  0.0646& 0.0667& 0.8042& 0.6582& 0.0589& 23.98&  0.8255&0.2621 &$\approx$1.3h\\
    2DGS + FDS &  \textbf{0.0615} & \textbf{ 0.0534}& \textbf{0.0574}& \textbf{0.8151}& \textbf{0.6974}&  \textbf{0.0561}&  \textbf{24.06}&  \textbf{0.8271}&\textbf{0.2610} &$\approx$1.3h \\ \hline
\end{tabular}
}
\end{minipage}\hfill
\end{table}

\begin{table}[t] \centering
\begin{minipage}[t]{0.96\linewidth}
        \captionof{table}{\textbf{3D Reconstruction 
        and novel view synthesis results on ScanNet dataset.}}
        \label{tab:scannet}
        \resizebox{\textwidth}{!}{
\begin{tabular}{c| c c c c c | c c c c }
    \hline
     Method &  Acc $\downarrow$ & Comp $\downarrow$ & C-L1 $\downarrow$ & NC $\uparrow$ & F-Score $\uparrow$ &  Abs Rel $\downarrow$ &  PSNR $\uparrow$  & SSIM  $\uparrow$ & LPIPS $\downarrow$ \\ \hline
    GOF & 1.8671  & 0.0805 & 0.9738 & 0.5622 & 0.2526 & 0.1597  & 21.55  & 0.7575  & 0.3881 \\
    PGSR &  0.2928 & 0.5103 & 0.4015 & 0.5567 & 0.1926 & 0.1661  & 21.71 & 0.7699  & 0.3899 \\ \hline

    3DGS &  0.4867 & 0.1211 & 0.3039 & 0.7342& 0.3059 & 0.1227 & 22.19& 0.7837 & 0.3907\\
    3DGS + FDS &  \textbf{0.2458} & \textbf{0.0787} & \textbf{0.1622} & \textbf{0.7831} & 
    \textbf{0.4482} & \textbf{0.0573} & \textbf{22.83} & \textbf{0.7911} & \textbf{0.3826} \\ \hline
    2DGS &  0.2658 & 0.0845 & 0.1752 & 0.7504& 0.4464 & 0.0831 & 22.59& 0.7881 & 0.3854\\
    2DGS + FDS &  \textbf{0.1457} & \textbf{0.0679} & \textbf{0.1068} & \textbf{0.7883} & 
    \textbf{0.5459} & \textbf{0.0432} & \textbf{22.91} & \textbf{0.7928} & \textbf{0.3800} \\ \hline
\end{tabular}
}
\end{minipage}\hfill
\end{table}


\begin{table}[t] \centering
\begin{minipage}[t]{0.96\linewidth}
        \captionof{table}{\textbf{Ablation study on geometry priors.}}
        \label{tab:analysis_prior}
        \resizebox{\textwidth}{!}{
\begin{tabular}{c| c c c c c | c c c c}

    \hline
     Method &  Acc$\downarrow$ & Comp $\downarrow$ & C-L1 $\downarrow$ & NC $\uparrow$ & F-Score $\uparrow$ &  Abs Rel $\downarrow$ &  PSNR $\uparrow$  & SSIM  $\uparrow$ & LPIPS $\downarrow$ \\ \hline
    2DGS&   0.1078&  0.0850&  0.0964&  0.7835&  0.5170&  0.1002&  23.56&  0.8166& 0.2730\\
    2DGS+Depth&   0.0862&  0.0702&  0.0782&  0.8153&  0.5965&  0.0672&  23.92&  0.8227& 0.2619 \\
    2DGS+MVDepth&   0.2065&  0.0917&  0.1491&  0.7832&  0.3178&  0.0792&  23.74&  0.8193& 0.2692 \\
    2DGS+Normal&   0.0939&  0.0637&  0.0788&  \textbf{0.8359}&  0.5782&  0.0768&  23.78&  0.8197& 0.2676 \\
    2DGS+FDS &  \textbf{0.0615} & \textbf{ 0.0534}& \textbf{0.0574}& 0.8151& \textbf{0.6974}&  \textbf{0.0561}&  \textbf{24.06}&  \textbf{0.8271}&\textbf{0.2610} \\ \hline
    2DGS+Depth+FDS &  0.0561 &  0.0519& 0.0540& 0.8295& 0.7282&  0.0454&  \textbf{24.22}& \textbf{0.8291}&\textbf{0.2570} \\
    2DGS+Normal+FDS &  \textbf{0.0529} & \textbf{ 0.0450}& \textbf{0.0490}& \textbf{0.8477}& \textbf{0.7430}&  \textbf{0.0443}&  24.10&  0.8283& 0.2590 \\
    2DGS+Depth+Normal &  0.0695 & 0.0513& 0.0604& 0.8540&0.6723&  0.0523&  24.09&  0.8264&0.2575\\ \hline
    2DGS+Depth+Normal+FDS &  \textbf{0.0506} & \textbf{0.0423}& \textbf{0.0464}& \textbf{0.8598}&\textbf{0.7613}&  \textbf{0.0403}&  \textbf{24.22}& 
    \textbf{0.8300}&\textbf{0.0403}\\
    
\bottomrule
\end{tabular}
}
\end{minipage}\hfill
\end{table}




\subsubsection{Mesh extraction}
To further demonstrate the improvement in geometry quality, 
we applied methods used in ~\citep{turkulainen2024dnsplatter} 
to extract meshes from the input views of optimized 3DGS. 
The comparison results are presented  
in \tabref{tab:mushroom}. 
With the integration of FDS, the mesh quality is significantly enhanced compared to the baseline, featuring fewer floaters and more well-defined shapes.
 %
% Following the incorporation of FDS, the reconstruction 
% results exhibit fewer floaters and more well-defined 
% shapes in the meshes. 
% Visualized comparisons
% are provided in the supplementary material.

% \begin{figure}[t] \centering
%     \makebox[0.19\textwidth]{\scriptsize GT}
%     \makebox[0.19\textwidth]{\scriptsize 3DGS}
%     \makebox[0.19\textwidth]{\scriptsize 3DGS+FDS}
%     \makebox[0.19\textwidth]{\scriptsize 2DGS}
%     \makebox[0.19\textwidth]{\scriptsize 2DGS+FDS} \\

%     \includegraphics[width=0.19\textwidth]{figure/fig4_img/compare1/gt02.png}
%     \includegraphics[width=0.19\textwidth]{figure/fig4_img/compare1/baseline06.png}
%     \includegraphics[width=0.19\textwidth]{figure/fig4_img/compare1/baseline_fds05.png}
%     \includegraphics[width=0.19\textwidth]{figure/fig4_img/compare1/2dgs04.png}
%     \includegraphics[width=0.19\textwidth]{figure/fig4_img/compare1/2dgs_fds03.png} \\

%     \includegraphics[width=0.19\textwidth]{figure/fig4_img/compare2/gt00.png}
%     \includegraphics[width=0.19\textwidth]{figure/fig4_img/compare2/baseline02.png}
%     \includegraphics[width=0.19\textwidth]{figure/fig4_img/compare2/baseline_fds01.png}
%     \includegraphics[width=0.19\textwidth]{figure/fig4_img/compare2/2dgs04.png}
%     \includegraphics[width=0.19\textwidth]{figure/fig4_img/compare2/2dgs_fds03.png} \\
      
%     \includegraphics[width=0.19\textwidth]{figure/fig4_img/compare3/gt05.png}
%     \includegraphics[width=0.19\textwidth]{figure/fig4_img/compare3/3dgs03.png}
%     \includegraphics[width=0.19\textwidth]{figure/fig4_img/compare3/3dgs_fds04.png}
%     \includegraphics[width=0.19\textwidth]{figure/fig4_img/compare3/2dgs02.png}
%     \includegraphics[width=0.19\textwidth]{figure/fig4_img/compare3/2dgs_fds01.png} \\

%     \caption{\textbf{Qualitative comparison of extracted mesh 
%     on Mushroom and ScanNet datasets.}}
%     \label{fig:mesh}
% \end{figure}












\subsection{Ablation study}


\textbf{Ablation study on geometry priors:} 
To highlight the advantage of incorporating matching priors, 
we incorporated various types of priors generated by different 
models into 2DGS. These include a monocular depth estimation
model (Depth Anything v2)~\citep{yang2024depth}, a two-view depth estimation 
model (Unimatch)~\citep{xu2023unifying}, 
and a monocular normal estimation model (DSINE)~\citep{bae2024rethinking}.
We adapt the scale and shift-invariant loss in Midas~\citep{birkl2023midas} for
monocular depth supervision and L1 loss for two-view depth supervison.
%
We use Sea Raft~\citep{wang2025sea} as our default optical flow model.
%
The comparison results on Mushroom dataset 
are shown in ~\tabref{tab:analysis_prior}.
We observe that the normal prior provides accurate shape information, 
enhancing the geometric quality of the radiance field. 
%
% In contrast, the monocular depth prior slightly increases 
% the 'Abs Rel' due to its ambiguous scale and inaccurate depth ordering.
% Moreover, the performance of monocular depth estimation 
% in the sauna scene is particularly poor, 
% primarily due to the presence of numerous reflective 
% surfaces and textureless walls, which limits the accuracy of monocular depth estimation.
%
The multi-view depth prior, hindered by the limited feature overlap 
between input views, fails to offer reliable geometric 
information. We test average "Abs Rel" of multi-view depth prior
, and the result is 0.19, which performs worse than the "Abs Rel" results 
rendered by original 2DGS.
From the results, it can be seen that depth order information provided by monocular depth improves
reconstruction accuracy. Meanwhile, our FDS achieves the best performance among all the priors, 
and by integrating all
three components, we obtained the optimal results.
%
%
\begin{figure}[t] \centering
    \makebox[0.16\textwidth]{\scriptsize RF (16000 iters)}
    \makebox[0.16\textwidth]{\scriptsize RF* (20000 iters)}
    \makebox[0.16\textwidth]{\scriptsize RF (20000 iters)  }
    \makebox[0.16\textwidth]{\scriptsize PF (16000 iters)}
    \makebox[0.16\textwidth]{\scriptsize PF (20000 iters)}


    % \includegraphics[width=0.16\textwidth]{figure/fig5_img/compare1/16000.png}
    % \includegraphics[width=0.16\textwidth]{figure/fig5_img/compare1/20000_wo_flow_loss.png}
    % \includegraphics[width=0.16\textwidth]{figure/fig5_img/compare1/20000.png}
    % \includegraphics[width=0.16\textwidth]{figure/fig5_img/compare1/16000_prior.png}
    % \includegraphics[width=0.16\textwidth]{figure/fig5_img/compare1/20000_prior.png}\\

    % \includegraphics[width=0.16\textwidth]{figure/fig5_img/compare2/16000.png}
    % \includegraphics[width=0.16\textwidth]{figure/fig5_img/compare2/20000_wo_flow_loss.png}
    % \includegraphics[width=0.16\textwidth]{figure/fig5_img/compare2/20000.png}
    % \includegraphics[width=0.16\textwidth]{figure/fig5_img/compare2/16000_prior.png}
    % \includegraphics[width=0.16\textwidth]{figure/fig5_img/compare2/20000_prior.png}\\

    \includegraphics[width=0.16\textwidth]{figure/fig5_img/compare3/16000.png}
    \includegraphics[width=0.16\textwidth]{figure/fig5_img/compare3/20000_wo_flow_loss.png}
    \includegraphics[width=0.16\textwidth]{figure/fig5_img/compare3/20000.png}
    \includegraphics[width=0.16\textwidth]{figure/fig5_img/compare3/16000_prior.png}
    \includegraphics[width=0.16\textwidth]{figure/fig5_img/compare3/20000_prior.png}\\
    
    \includegraphics[width=0.16\textwidth]{figure/fig5_img/compare4/16000.png}
    \includegraphics[width=0.16\textwidth]{figure/fig5_img/compare4/20000_wo_flow_loss.png}
    \includegraphics[width=0.16\textwidth]{figure/fig5_img/compare4/20000.png}
    \includegraphics[width=0.16\textwidth]{figure/fig5_img/compare4/16000_prior.png}
    \includegraphics[width=0.16\textwidth]{figure/fig5_img/compare4/20000_prior.png}\\

    \includegraphics[width=0.30\textwidth]{figure/fig5_img/bar.png}

    \caption{\textbf{The error map of Radiance Flow and Prior Flow.} RF: Radiance Flow, PF: Prior Flow, * means that there is no FDS loss supervision during optimization.}
    \label{fig:error_map}
\end{figure}




\textbf{Ablation study on FDS: }
In this section, we present the design of our FDS 
method through an ablation study on the 
Mushroom dataset to validate its effectiveness.
%
The optional configurations of FDS are outlined in ~\tabref{tab:ablation_fds}.
Our base model is the 2DGS equipped with FDS,
and its results are shown 
in the first row. The goal of this analysis 
is to evaluate the impact 
of various strategies on FDS sampling and loss design.
%
We observe that when we 
replace $I_i$ in \eqref{equ:mflow} with $C_i$, 
as shown in the second row, the geometric quality 
of 2DGS deteriorates. Using $I_i$ instead of $C_i$ 
help us to remove the floaters in $\bm{C^s}$, which are also 
remained in $\bm{C^i}$.
We also experiment with modifying the FDS loss. For example, 
in the third row, we use the neighbor 
input view as the sampling view, and replace the 
render result of neighbor view with ground truth image of its input view.
%
Due to the significant movement between images, the Prior Flow fails to accurately 
match the pixel between them, leading to a further degradation in geometric quality.
%
Finally, we attempt to fix the sampling view 
and found that this severely damaged the geometric quality, 
indicating that random sampling is essential for the stability 
of the mean error in the Prior flow.



\begin{table}[t] \centering

\begin{minipage}[t]{1.0\linewidth}
        \captionof{table}{\textbf{Ablation study on FDS strategies.}}
        \label{tab:ablation_fds}
        \resizebox{\textwidth}{!}{
\begin{tabular}{c|c|c|c|c|c|c|c}
    \hline
    \multicolumn{2}{c|}{$\mathcal{M}_{\theta}(X, \bm{C^s})$} & \multicolumn{3}{c|}{Loss} & \multicolumn{3}{c}{Metric}  \\
    \hline
    $X=C^i$ & $X=I^i$  & Input view & Sampled view     & Fixed Sampled view        & Abs Rel $\downarrow$ & F-score $\uparrow$ & NC $\uparrow$ \\
    \hline
    & \ding{51} &     &\ding{51}    &    &    \textbf{0.0561}        &  \textbf{0.6974}         & \textbf{0.8151}\\
    \hline
     \ding{51} &           &     &\ding{51}    &    &    0.0839        &  0.6242         &0.8030\\
     &  \ding{51} &   \ding{51}  &    &    &    0.0877       & 0.6091        & 0.7614 \\
      &  \ding{51} &    &    & \ding{51}    &    0.0724           & 0.6312          & 0.8015 \\
\bottomrule
\end{tabular}
}
\end{minipage}
\end{table}




\begin{figure}[htbp] \centering
    \makebox[0.22\textwidth]{}
    \makebox[0.22\textwidth]{}
    \makebox[0.22\textwidth]{}
    \makebox[0.22\textwidth]{}
    \\

    \includegraphics[width=0.22\textwidth]{figure/fig6_img/l1/rgb/frame00096.png}
    \includegraphics[width=0.22\textwidth]{figure/fig6_img/l1/render_rgb/frame00096.png}
    \includegraphics[width=0.22\textwidth]{figure/fig6_img/l1/render_depth/frame00096.png}
    \includegraphics[width=0.22\textwidth]{figure/fig6_img/l1/depth/frame00096.png}

    % \includegraphics[width=0.22\textwidth]{figure/fig6_img/l2/rgb/frame00112.png}
    % \includegraphics[width=0.22\textwidth]{figure/fig6_img/l2/render_rgb/frame00112.png}
    % \includegraphics[width=0.22\textwidth]{figure/fig6_img/l2/render_depth/frame00112.png}
    % \includegraphics[width=0.22\textwidth]{figure/fig6_img/l2/depth/frame00112.png}

    \caption{\textbf{Limitation of FDS.} }
    \label{fig:limitation}
\end{figure}


% \begin{figure}[t] \centering
%     \makebox[0.48\textwidth]{}
%     \makebox[0.48\textwidth]{}
%     \\
%     \includegraphics[width=0.48\textwidth]{figure/loss_Ignatius.pdf}
%     \includegraphics[width=0.48\textwidth]{figure/loss_family.pdf}
%     \caption{\textbf{Comparison the photometric error of Radiance Flow and Prior Flow:} 
%     We add FDS method after 2k iteration during training.
%     The results show
%     that:  1) The Prior Flow is more precise and 
%     robust than Radiance Flow during the radiance 
%     optimization; 2) After adding the FDS loss 
%     which utilize Prior 
%     flow to supervise the Radiance Flow at 2k iterations, 
%     both flow are more accurate, which lead to
%     a mutually reinforcing effects.(TODO fix it)} 
%     \label{fig:flowcompare}
% \end{figure}






\textbf{Interpretive Experiments: }
To demonstrate the mutual refinement of two flows in our FDS, 
For each view, we sample the unobserved 
views multiple times to compute the mean error 
of both Radiance Flow and Prior Flow. 
We use Raft~\citep{teed2020raft} as our default optical flow model
for visualization.
The ground truth flow is calculated based on 
~\eref{equ:flow_pose} and ~\eref{equ:flow} 
utilizing ground truth depth in dataset.
We introduce our FDS loss after 16000 iterations during 
optimization of 2DGS.
The error maps are shown in ~\figref{fig:error_map}.
Our analysis reveals that Radiance Flow tends to 
exhibit significant geometric errors, 
whereas Prior Flow can more accurately estimate the geometry,
effectively disregarding errors introduced by floating Gaussian points. 

%





\subsection{Limitation and further work}

Firstly, our FDS faces challenges in scenes with 
significant lighting variations between different 
views, as shown in the lamp of first row in ~\figref{fig:limitation}. 
%
Incorporating exposure compensation into FDS could help address this issue. 
%
 Additionally, our method struggles with 
 reflective surfaces and motion blur,
 leading to incorrect matching. 
 %
 In the future, we plan to explore the potential 
 of FDS in monocular video reconstruction tasks, 
 using only a single input image at each time step.
 


\section{Conclusions}
In this paper, we propose Flow Distillation Sampling (FDS), which
leverages the matching prior between input views and 
sampled unobserved views from the pretrained optical flow model, to improve the geometry quality
of Gaussian radiance field. 
Our method can be applied to different approaches (3DGS and 2DGS) to enhance the geometric rendering quality of the corresponding neural radiance fields.
We apply our method to the 3DGS-based framework, 
and the geometry is enhanced on the Mushroom, ScanNet, and Replica datasets.

\section*{Acknowledgements} This work was supported by 
National Key R\&D Program of China (2023YFB3209702), 
the National Natural Science Foundation of 
China (62441204, 62472213), and Gusu 
Innovation \& Entrepreneurship Leading Talents Program (ZXL2024361)
\section{Conclusion}
We introduce a novel approach, \algo, to reduce human feedback requirements in preference-based reinforcement learning by leveraging vision-language models. While VLMs encode rich world knowledge, their direct application as reward models is hindered by alignment issues and noisy predictions. To address this, we develop a synergistic framework where limited human feedback is used to adapt VLMs, improving their reliability in preference labeling. Further, we incorporate a selective sampling strategy to mitigate noise and prioritize informative human annotations.

Our experiments demonstrate that this method significantly improves feedback efficiency, achieving comparable or superior task performance with up to 50\% fewer human annotations. Moreover, we show that an adapted VLM can generalize across similar tasks, further reducing the need for new human feedback by 75\%. These results highlight the potential of integrating VLMs into preference-based RL, offering a scalable solution to reducing human supervision while maintaining high task success rates. 

\section*{Impact Statement}
This work advances embodied AI by significantly reducing the human feedback required for training agents. This reduction is particularly valuable in robotic applications where obtaining human demonstrations and feedback is challenging or impractical, such as assistive robotic arms for individuals with mobility impairments. By minimizing the feedback requirements, our approach enables users to more efficiently customize and teach new skills to robotic agents based on their specific needs and preferences. The broader impact of this work extends to healthcare, assistive technology, and human-robot interaction. One possible risk is that the bias from human feedback can propagate to the VLM and subsequently to the policy. This can be mitigated by personalization of agents in case of household application or standardization of feedback for industrial applications. 

% \newpage
\bibliography{iclr2025_conference}
\bibliographystyle{iclr2025_conference}

% \newpage
\appendix
\section{Preliminary}
\subsection{Diffusion models}
\label{appendix:preliminary}
\textbf{Diffusion models}~\citep{ho2020denoising} are a class of generative models that recover the data corrupted by the Gaussian noise through learning a reverse diffusion process.
It iteratively denoises from Gaussian noise, which corresponds to learning the reverse process of a fixed Markov Chain of length $T$.
The diffusion process is a Markov chain that gradually corrupts the data with Gaussian noise.
For the diffusion process given the variance schedule $\beta_t$:
\begin{equation}
q(x_{1:T} | x_0) = \prod_{t=1}^T q(x_t | x_{t-1} ), \qquad q(x_t|x_{t-1}) = \mathcal{N}(x_t;\sqrt{1-\beta_t}x_{t-1},\beta_t I).
\end{equation}
Using the  Markov property, we can sample $x_t$ at an arbitrary time $t$ from $x_0$ in closed form. Let $\alpha=1-\beta_t$ and $\bar{\alpha}_t=\prod_{s=1}^t\alpha_s$, we have
\begin{equation}
    q(x_t|x_0) = \mathcal{N}(x_t; \sqrt{\bar\alpha_t}x_0, (1-\bar\alpha_t)I).
\end{equation}
By the Bayes' rules, $q(x_{t-1}|x_t,x_0)$ can be expressed as follows:
\begin{align}
q(x_{t-1}|x_t,x_0) &=  \mathcal{N}(x_{t-1}; \tilde\mu_t(x_t, x_0), \tilde\beta_t I), \\
\text{where}\quad \tilde\mu_t(x_t, x_0) &= \frac{\sqrt{\bar\alpha_{t-1}}\beta_t }{1-\bar\alpha_t}x_0 + \frac{\sqrt{\alpha_t}(1- \bar\alpha_{t-1})}{1-\bar\alpha_t} x_t \quad \text{and} \quad
\tilde\beta_t = \frac{1-\bar\alpha_{t-1}}{1-\bar\alpha_t}\beta_t.
\end{align}
For the reverse process, it generates $x_0$ from $x_T$ with prior $x_T=\mathcal{N}(x_T;0,I)$ and transitions:
\begin{equation}
    p_\Theta(x_{t-1}|x_t)=\mathcal{N}(x_{t-1};\mu_\Theta(x_t, t),\Sigma_\Theta(x_t,t)).
\end{equation}
In the equation, $\Theta$ are learnable parameters of models $\epsilon_\Theta$ 
which are trained to minimize the variant of the variational bound $\E_{x,\epsilon\sim\mathcal{N}(0,I),t}\left[ \left\| \epsilon-\epsilon_{\Theta}\left(x_t, t\right) \right\|^2 \right]$.

\subsection{Fourier transform}
\label{appendix:fourier}
\textbf{Discrete Fourier Transform (DFT)} is one of the most important discrete transforms used in digital signal processing including image processing.
The discrete Fourier transform can be expressed as the \textbf{DFT} matrix, denoted as $\mF$, defined as follows:
\begin{equation}
    \mF = \left( \omega_N^{\left(m-1\right)\cdot\left(n-1\right)} \right)_{N \times N}=
\begin{bmatrix}
 \omega_N^{0 \cdot 0}     & \omega_N^{0 \cdot 1}     & \cdots & \omega_N^{0 \cdot (N-1)}     \\
 \omega_N^{1 \cdot 0}     & \omega_N^{1 \cdot 1}     & \cdots & \omega_N^{1 \cdot (N-1)}     \\
 \vdots                   & \vdots                   & \ddots & \vdots                       \\
 \omega_N^{(N-1) \cdot 0} & \omega_N^{(N-1) \cdot 1} & \cdots & \omega_N^{(N-1) \cdot (N-1)} \\
\end{bmatrix}
\end{equation}
where $\omega_N = e^{-{2\pi i/N}}$ is a primitive $N$-th root of unity. 
The inverse transform, denoted as $\mF^{-1}$ can be derived from $\mF$ as its complex conjugate transpose, scaled by $\frac{1}{N}$: $\mF^{-1} = \frac{1}{N}\mF^\ast$.

The \textbf{DFT} matrix $\mF$ can be decomposed into its real and imaginary parts, represented respectively by matrices $\mA$ and $\mB$:
\begin{equation}
\label{eq:decomposition}
    \mF = \mA + \mB i,\quad \mA = \Re\left(\mF\right),\quad \mB = \Im\left(\mF\right).
\end{equation}
This decomposition simplifies the understanding of the structure and properties of the DFT matrix, providing deeper insights. Using Euler’s formula, $\mA$ and $\mB$ can be explicitly expressed as:
\begin{equation}
    \mA = \left(\cos\left( \left(m-1\right)\left(n-1\right)\theta \right)\right)_{N\times N},\quad 
    \mB = \left(\sin\left( \left(m-1\right)\left(n-1\right)\theta \right)\right)_{N\times N},
\end{equation}
where $\theta=-\frac{2\pi}{N}$. 
Notably, both $\mA$ and $\mB$ are real symmetric matrices.

\begin{lemma}
\label{lemma:1}
    For $\theta=-\frac{2\pi}{N}$ where $N$ is a positive integer, it holds that $\sum_{k=1}^{N}\sin\left(l\left(k-1\right)\theta\right)=0$ for any integer $l$.
\end{lemma}
\begin{proof}
By applying Euler's formula, we rewrite $\sin\left(l\left(k-1\right)\theta\right)$ as $\Im\left(\omega_N^{l(k-1)}\right)$, where $\omega_N=e^{-2\pi i/N}$. Then we have:
\begin{equation}
    \sum_{k=1}^{N} \sin\left(l\left(k-1\right)\theta\right) =\sum_{k=0}^{N-1} \Im \left(\omega_N^{lk}\right) = \Im \left( \sum_{k=0}^{N-1} \omega_N^{lk}\right).
\end{equation}
The term $\sum_{k=0}^{N-1} \omega_N^{lk}$ is the sum of geometric sequence. 
If $\omega_N^{l} = 1$, then $\sum_{k=0}^{N-1} \omega_N^{lk} = N$, yielding $\Im \left( \sum_{k=0}^{N-1} \omega_N^{lk}\right)=0$.

Otherwise, if $\omega_N^{l} \ne 1$, we have $\sum_{k=0}^{N-1} \omega_N^{lk} = \left(1-\omega_N^{lN}\right)/\left(1-\omega_N^{l}\right)$. Since $\omega_N^{N}=1$, then $\sum_{k=0}^{N-1} \omega_N^{lk} = 0$, and consequently $\Im \left( \sum_{k=0}^{N-1} \omega_N^{lk}\right)=0$.

In conclusion, we have shown that $\sum_{k=1}^{N} \sin\left(l\left(k-1\right)\theta\right)=0$ for any integer $l$.
\end{proof}
This lemma offers foundational insights into the behavior of the sum of sinusoidal functions,
Now, we introduce a theorem regarding properties of the \textbf{DFT} matrix.
\begin{theorem}
\label{theorem:1}
    Given a \textbf{DFT} matrix $\mF\in \mathbb{C}^{N\times N}$, with $\mA$ and $\mB$ representing its real and imaginary parts respectively, it holds that $\mA\mB=\mB\mA=\mathbf{0}$ and $\mA^2+\mB^2=N\mI$.
\end{theorem}
\begin{proof}
Using the property of the inverse Fourier transform, we have
\begin{equation}
    \mI = \mF\mF^{-1} = \frac{1}{N} \mF \mF^{\ast} = \frac{1}{N}\left(\mA + \mB i\right)\left(\mA - \mB i\right) = \frac{1}{N}\left(\mA^2+\mB^2-\mA\mB i+\mB\mA i\right).
\end{equation}
Comparing real parts and imaginary parts of both sides, we derive:
\begin{equation}
    \mA^2+\mB^2=N\mI,\quad \mB\mA=\mA\mB.
\end{equation}
Considering the matrix $\mA\mB$, we calculate the value of the element in the $m$-th row and $n$-th column:
\begin{equation}
\begin{split}
    \left(\mA\mB\right)_{mn} &=\sum_{k=1}^{N}\cos\left( \left(m-1\right)\left(k-1\right)\theta \right)\sin\left( \left(k-1\right)\left(n-1\right)\theta \right) \\
    &=\frac{1}{2}\sum_{k=1}^{N}\left( \sin\left((m+n-2)(k-1)\theta\right) - \sin\left((m-n)(k-1)\theta\right) \right) = 0.
\end{split}
\end{equation}
The last equation holds using Lemma~\ref{lemma:1}. The equation holds for each element of $\mA\mB$.
Therefore $\mA\mB=\mB\mA=\mathbf{0}$.
\end{proof}
%%%%%%%%%%%%%   3 dimension
For the 3D Fourier transform, it can be represented as follows using the Kronecker product:
\begin{equation}
    \mF_{3D} = \mF_{T} \otimes \mF_{H} \otimes \mF_{W}.
\end{equation}
The inverse transform is given by:
\begin{equation}
    \mF_{3D}^{-1}  = \frac{1}{N_TN_HN_W} \mF_{3D}^\ast.
\end{equation}
Similarly, we decompose $\mF_{3D}$ into its real part $\mA_{3D}$ and imaginary part $\mB_{3D}$:
\begin{equation}
\begin{split}
    \mF_{3D} & =\left(\mA_T+\mB_Ti\right) \otimes \left(\mA_H+\mB_Hi\right) \otimes \left(\mA_W+\mB_Wi\right), \\
    \mA_{3D} &= \mA_T\otimes\mA_H\otimes\mA_W
    -\mA_T\otimes\mB_H\otimes\mB_W
    -\mB_T\otimes\mA_H\otimes\mB_W
    -\mB_T\otimes\mB_H\otimes\mA_W, \\
    \mB_{3D} &= \mA_T\otimes\mA_H\otimes\mB_W
    +\mA_T\otimes\mB_H\otimes\mA_W
    +\mB_T\otimes\mA_H\otimes\mA_W
    -\mB_T\otimes\mB_H\otimes\mB_W.
\end{split}
\end{equation}
By Theorem~\ref{theorem:1} and the property or Kronecker product, it still holds that:
\begin{equation}
    \mA_{3D}^2+\mB_{3D}^2=N_T N_H N_W\mI,\quad \mB_{3D}\mA_{3D}=\mA_{3D}\mB_{3D}=\mathbf{0}.
\end{equation}
It reveals that the 3D DFT matrix shares the same properties as the ordinary DFT matrix.
For convenience, we denote the DFT matrix, including multi-dimensional cases as $\mF$, with size denoted as $N$.
Employing mathematical induction, we can extend Theorem~\ref{theorem:1} from one-dimensional case to arbitrary finite dimensions:
\begin{theorem}
\label{theorem:dft}
    Given a \textbf{DFT} matrix or multi-dimension \textbf{DFT} matrix $\mF\in \mathbb{C}^{N\times N}$, with $\mA$ and $\mB$ are its real part and imaginary part respectively, it holds that $\mA\mB=\mB\mA=\mathbf{0}$ and $\mA^2+\mB^2=N\mI$.
\end{theorem}
\section{Noise distribution analysis}
\label{appendix:noise_distribution_analysis}
\subsection{FreeInit}
\label{appendix:freeinit}
FreeInit~\citep{wu2023freeinit} uses conventional frequency filtering methods to manipulate noise, which is the key step in the framework. This step can be formulated as follows:
\begin{equation}
\label{eq:freeinit_raw}
    \bfz_T = \Re\left( \mathcal{F}_{3D}^{-1}\left(\mathcal{F}_{3D}\left(\bfz_{noise}\right)\odot\mathcal{M} + \mathcal{F}_{3D}\left(\eta\right)\odot(\bm{1}-\mathcal{M})\right) \right),
\end{equation}
where $\mathcal{F}_{3D}$ is the Fourier transform applied to both spatial and temporal dimensions. $\mathcal{M}$ is a spatial-temporal low-pass filter. $\bfz_{noise}$ is noisy latent derived from corrupting the clean latent with initial Gaussian noise to timestep $T$, while $\eta$ is another Gaussian noise. 
For analysis, we focus solely on the spatial and temporal dimensions, ignoring the batchsize and channel dimensions. Additionally, we flatten the latent $\bfz_T\in\R^{f\times h \times w}$ into a vector $\bfz \in \R^{fhw}$. The equation~\ref{eq:freeinit_raw} can be expressed in matrix form as follows:
\begin{equation}
    \bfz = \Re\left( \mF^{-1}\mLambda_x\mF\bfx + \mF^{-1}\mLambda_y\mF\bfy \right),
\end{equation}
where $\mF$ is the DFT matrix of transform $\mathcal{F}_{3D}$, $\bfx$ and $\bfy$ are random vectors corresponding to $\bfz_{noise}$ and $\eta$, and $\mLambda_x$ and
$\mLambda_y$ are diagonal matrices associated with low-pass filter $\mathcal{M}$ and high-pass filter $\bm{1}-\mathcal{M}$. Therefore it holds that $\mLambda_x+\mLambda_y=\mI$.
% Due to the property of $\mathcal{M}$, the diag 
This equation can be simplified to the following form using equation~\ref{eq:decomposition}:
\begin{equation}
    \bfz = \frac{1}{N}\left(\mA\mLambda_x\mA + \mB\mLambda_x\mB\right)\bfx 
      + \frac{1}{N}\left( \mA\mLambda_y\mA + \mB\mLambda_y\mB\right)\bfy,
\end{equation}
Under the Assumption~\ref{assumption:1}, $\bfx=\text{vec}(\bfz_{noise}) \sim\mathcal{N}\left(\mathbf{0},\mI\right)$, where $\bfx$ and $\bfy$ are independent. 
Since $\bfz$ is a linear combination of independent Gaussian random vectors, it follows that $\bfz$ is also Gaussian. To derive the distribution of $\bfz$, we only need to compute its expectation and covariance.
The expectation is straightforward and given by $\E[\bfz]=\mathbf{0}$. 
The covariance of $\bfz$ can be calculated as follows:
\begin{equation}
\label{eq:freeinit_cov1}
    \Cov\left(\bfz\right) 
    = \frac{1}{N^2}\left( \mA\mLambda_x\mA + \mB\mLambda_x\mB\right)^2 +
      \frac{1}{N^2}\left( \mA\mLambda_y\mA + \mB\mLambda_y\mB\right)^2.
\end{equation}
To simplify the expression, we denote $\mP = \frac{1}{N}\left( \mA\mLambda_x\mA+\mB\mLambda_x\mB\right)$. Then the term $\mA\mLambda_y\mA+\mB\mLambda_y\mB$ can be expressed using $\mP$:
\begin{equation}
\label{eq:freeinit_cov2}
\begin{split}
     \mA\mLambda_y\mA + 
      \mB\mLambda_y\mB & = \mA(\mI-\mLambda_x)\mA + 
      \mB(\mI-\mLambda_x)\mB \\
      & = \mA^2 + \mB^2 - \left( \mA\mLambda_x\mA+\mB\mLambda_x\mB\right) = N\mI - N\mP.
\end{split}
\end{equation}
The last equation follows from $ \mA^2+\mB^2 =N\mI$, as stated in Theorem~\ref{theorem:dft}. 
Combining Equation~(\ref{eq:freeinit_cov1}) and Equation~(\ref{eq:freeinit_cov2}), the covariance of $\bfz$ is given by:
\begin{equation}
    \Cov\left(\bfz\right) = \mP^2 + (\mI-\mP)^2.
\end{equation}
Consequently, we obtain the distribution of $\bfz$ as follows:
\begin{equation}
\label{eq:free_distribution}
    \bfz\sim\mathcal{N}\left(\mathbf{0}, \mP^2 + \left(\mI-\mP\right)^2\right).
\end{equation}
Due to the property of the low-pass filter $\mathcal{M}$ where each element lies between 0 to 1, both $\mLambda_x$ and $\mLambda_y$ are semi-definite diagonal matrices. Consequently, we can prove that both $\mP$ and $\mI-\mP$ are semi-positive definite matrices. The covariance structure resembles $a^2+(1-a)^2$, which is less than 1 for $a\in(0, 1)$. This indicates a difference between the distribution of $\bfz$ and the standard Gaussian distribution. We explore this further in Appendix~\ref{appendix:theoretical_analysis}.


\subsection{FreqPrior}
\label{appendix:freqprior}
The noise refinement stage of our method consists of three distinct steps, which are elaborated on in Section~\ref{subsec:noise_refinement}.  
To facilitate further analysis, we express these steps in matrix form.
% For simplicity, we can express these steps in the following matrix forms, which facilitate further analysis.
The first step, \textbf{noise preparation step}, can be represented as:
\begin{equation}
\label{eq:freq_x1}
    \bfx_1 = \frac{1}{\sqrt{1+\cos^2\theta}}\left(\cos\theta\cdot \bfx+\sin\theta\cdot\mathbf{\eta_1}\right),\quad
    \bfx_2 = \frac{1}{\sqrt{1+\cos^2\theta}}\left(\cos\theta\cdot \bfx+\sin\theta\cdot\mathbf{\eta}_2\right),
\end{equation}
where $\mathbf\eta_1, \mathbf\eta_2\sim\mathcal{N}\left(\mathbf{0}, \mI\right)$ and are independent. Under Assumption~\ref{assumption:1}, $\bfx\sim\mathcal{N}(\mathbf{0},\mI)$. Obviously, $\bfx, \mathbf\eta_1$, and $\mathbf\eta_2$ are independent.
Both $\bfx_1$ and $\bfx_2$ are linear combinations of independent of Gaussian random vectors. 
Their expectation can be computed directly: $\E[\bfx_1]=\mathbb{E}[\bfx_2]=\mathbf{0}$.
Next, we calculate the covariance of these variables. Specifically, the covariances are given by:
\begin{equation}
\label{eq:freq_x2}
    \Cov\left(\bfx_1\right) = \Cov\left(\bfx_2\right) = \frac{1}{1+\cos^2\theta}\mI,\quad
    \Cov(\bfx_1,\bfx_2) = \Cov(\bfx_2,\bfx_1) = \frac{\cos^2\theta}{1+\cos^2\theta}\mI.
\end{equation}
This implies that $\bfx_1$ and $\bfx_2$ are correlated, as they both share a component of $\bfx$ when $\cos\theta\ne0$.

The \textbf{noise processing} and \textbf{post-processing} steps can be expressed as follows:
\begin{equation}
    \bfz_1 = \mF^{-1}\mLambda_x\mF\bfx_1 + \mF^{-1}\mLambda_y\mF\bfy_1, \quad
    \bfz_2 = \mF^{-1}\mLambda_x\mF\bfx_2 + \mF^{-1}\mLambda_y\mF\bfy_2,
\end{equation}
\begin{equation}
\bfz = \frac{1}{\sqrt{2}}\left(\Re\left(\bfz_1\right)+\Im\left(\bfz_1\right)+\Re\left(\bfz_2\right)-\Im\left(\bfz_2\right)\right),
\end{equation}
where $\bfy_1, \bfy_2\sim\mathcal{N}\left(\mathbf{0}, \mI\right)$ are independent. 
Regarding the filters, $\mLambda_x$ and $\mLambda_y$ are diagonal matrices corresponding to the low-pass filter $\mathcal{M}$ and the high-pass filter $(\bm{1}-\mathcal{M})^{0.5}$.

The refined noise $\bfz$ can be expressed in a following form using equation~\ref{eq:decomposition}:
\begin{equation}
\label{eq:freq_z_1}
\begin{split}
\sqrt{2}N\cdot\bfz {} = {}& \phantom{\;\;\,\,\,}\left( \mA\mLambda_x\mA +\mB\mLambda_x\mB+\mA\mLambda_x\mB-\mB\mLambda_x\mA\right)\bfx_1 \\
                          & + \left( \mA\mLambda_y\mA +\mB\mLambda_y\mB+\mA\mLambda_y\mB-\mB\mLambda_y\mA\right)\bfy_1 \\
                          & + \left( \mA\mLambda_x\mA +\mB\mLambda_x\mB-\mA\mLambda_x\mB+\mB\mLambda_x\mA\right)\bfx_2 \\
                          & + \left( \mA\mLambda_y\mA +\mB\mLambda_y\mB-\mA\mLambda_y\mB+\mB\mLambda_y\mA\right)\bfy_2.
\end{split}
\end{equation}
From the mathematical form of this expression, it is evident that the matrices preceding these random vectors share similar structures.
To simplify this equation, we introduce the following notations:
\begin{equation}
\begin{split}
    \text{Let}:\quad
    \mC_x &= \mA\mLambda_x\mA +\mB\mLambda_x\mB,\quad \mD_x =\mA\mLambda_x\mB-\mB\mLambda_x\mA, \\
    \mC_y &= \mA\mLambda_y\mA +\mB\mLambda_y\mB,\quad \mD_y =\mA\mLambda_y\mB-\mB\mLambda_y\mA.
\end{split}
\end{equation}
Since $\mA$ and $\mB$ are real symmetric matrices, and $\mLambda_x$ and $\mLambda_y$ are diagonal matrices, it is straightforward to prove that $\mC_x$ and $\mC_y$ are symmetric matrices, while $\mD_x$ and $\mD_y$ are skew-symmetric matrices. 
Using these notations, Equation~(\ref{eq:freq_z_1}) can be simplified as follow:
\begin{equation}
    \sqrt{2}N\cdot\bfz = \left(\mC_x+\mD_x\right)\bfx_1+
    \left(\mC_y+\mD_y\right)\bfy_1+
    \left(\mC_x-\mD_x\right)\bfx_2+
    \left(\mC_y-\mD_y\right)\bfy_2.
\end{equation}
In the analysis of $\sqrt{2}N\cdot\bfz$ where $\bfz$ is a Gaussian-distributed vector, we need to calculate the expectation and covariance to determine its distribution.
The expectation is given by $\E[\bfz] = \mathbf{0}$. 
The covariance can be expressed as the sum of several covariance terms related to  $\bfx_1$, $\bfx_2$, $\bfy_1$ and $\bfy_2$. Specifically, the covariance of $\sqrt{2}N\cdot\bfz$ can be expressed as follows:
\begin{equation}
\begin{split}
    \Cov(\sqrt{2}N\cdot\bfz) = {}& {} \phantom{\,\;\;\,\,\,} \Cov\left(\left(\mC_x+\mD_x\right)\bfx_1\right)
    +\Cov\left(\left(\mC_x-\mD_x\right)\bfx_2\right)\\
    &+\Cov\left(\left(\mC_y+\mD_y\right)\bfy_1\right)
    +\Cov\left(\left(\mC_y-\mD_y\right)\bfy_2\right)\\
    &+\Cov\left(\left(\mC_x+\mD_x\right)\bfx_1,\left(\mC_x-\mD_x\right)\bfx_2\right)\\
    &+\Cov\left(\left(\mC_x-\mD_x\right)\bfx_2,\left(\mC_x+\mD_x\right)\bfx_1\right).
\end{split}
\end{equation}
The covariance of $\sqrt{2}N\cdot\bfz$ consists of 6 terms, with first four terms representing the covariance of each random vector. The last two terms are cross terms that arise due to the fact that $\bfx_1$ and $\bfx_2$ are not independent. 
By solving these terms, We can derive the covariance of $\bfz$.

First, we focus on the covariance terms related to $\bfy_1$ and $\bfy_2$:
\begin{equation}
\label{eq:freq_cov_part1}
\begin{split}
    &\Cov\left(\left(\mC_y+\mD_y\right)\bfy_1\right)+\Cov\left(\left(\mC_y-\mD_y\right)\bfy_2\right)\\
    ={}&\left(\mC_y+\mD_y\right)\Cov\left(\bfy_1\right)\left(\mC_y+\mD_y\right)^\top + 
    \left(\mC_y-\mD_y\right)\Cov\left(\bfy_1\right)\left(\mC_y-\mD_y\right)^\top \\
    = {} &  \left(\mC_y+\mD_y\right)\left(\mC_y-\mD_y\right) + \left(\mC_y-\mD_y\right)\left(\mC_y+\mD_y\right)
    =2\left(\mC_y^2-\mD_y^2\right).
\end{split}
\end{equation}
Similarly, we can infer $\Cov\left(\left(\mC_x+\mD_x\right)\bfx_1\right)
    +\Cov\left(\left(\mC_x-\mD_x\right)\bfx_2\right)$ combined with Equation~(\ref{eq:freq_x2}):
% Covariance x part
\begin{equation}
\label{eq:freq_cov_part2}
    \Cov\left(\left(\mC_x+\mD_x\right)\bfx_1\right)
    +\Cov\left(\left(\mC_x-\mD_x\right)\bfx_2\right)
    = \frac{2}{1+\cos^2\theta}\left(\mC_x^2-\mD_x^2\right).
\end{equation}
Having computed the first four terms, we now turn our attention to the last two cross terms. With Equation~(\ref{eq:freq_x2}), we have:
\begin{equation}
\label{eq:freq_cov_part3}
\begin{split}
    &\Cov\left(\left(\mC_x+\mD_x\right)\bfx_1,\left(\mC_x-\mD_x\right)\bfx_2\right)
    +\Cov\left(\left(\mC_x-\mD_x\right)\bfx_2,\left(\mC_x+\mD_x\right)\bfx_1\right) \\
    =&{} \left(\mC_x+\mD_x\right)\Cov(\bfx_1,\bfx_2)\left(\mC_x-\mD_x\right)^\top 
        +\left(\mC_x-\mD_x\right)\Cov(\bfx_2,\bfx_1)\left(\mC_x+\mD_x\right)^\top \\
    =&{}\frac{\cos^2\theta}{1+\cos^2\theta}\left(\mC_x+\mD_x\right)^2+\frac{\cos^2\theta}{1+\cos^2\theta}\left(\mC_x-\mD_x\right)^2 =\frac{2\cos^2\theta}{1+\cos^2\theta}\left(\mC_x^2+\mD_x^2\right).
\end{split}
\end{equation}
Substituting the expression of the covariance related to $\bfx_1$, $\bfx_2$, $\bfy_1$ and $\bfy_2$ with Equations~(\ref{eq:freq_cov_part1},~\ref{eq:freq_cov_part2},~\ref{eq:freq_cov_part3}), we can express the covariance of $\sqrt{2}N\cdot\bfz$ in the following form:
\begin{equation}
\label{eq:freq_50}
    \Cov\left(\sqrt{2}N\cdot\bfz\right) = 2\left(\mC_x^2-\mD_x^2+\mC_y^2-\mD_y^2\right)+ \frac{4\cos^2\theta}{1+\cos^2\theta}\mD_x^2.
\end{equation}
To further simplify this equation, we need to explore the properties of $\mC_x$, $\mC_y$, $\mD_x$ and $\mD_y$. 
From Theorem~\ref{theorem:dft}, which establish $\mA\mB=\mB\mA=\mathbf{0}$ and $\mA^2+\mB^2=N\mI$. We can compute the squares of matrices $\mC_x$ and $\mD_x$ as follows:
\begin{equation}
\label{eq:freq_51}
\begin{split}
    \mC_x^2 & = \mA\mLambda_x\mA^2\mLambda_x\mA + \mB\mLambda_x\mB^2\mLambda_x\mB, \\
    -\mD_x^2 & = \mA\mLambda_x\mB^2\mLambda_x\mA + \mB\mLambda_x\mA^2\mLambda_x\mB.
\end{split}
\end{equation}
Notice that the squares of $\mC_x$ and $\mD_x$ share a similar form, differing only in the middle matrix: one is $\mA^2$ and the other is $\mB^2$. This observation inspires us to calculate $\mC_x^2-\mD_x^2$, especially since we have established $\mA^2+\mB^2=N\mI$. Therefore, we can express it as follows:
\begin{equation}
\begin{split}
    \mC_x^2-\mD_x^2&= \mA\mLambda_x\left(\mA^2+\mB^2\right)\mLambda_x\mA + \mB\mLambda_x\left(\mB^2+\mA^2\right)\mLambda_x\mB\\ 
    & = N \mA\mLambda_x^2\mA + N\mB\mLambda_x^2\mB, \\
\end{split}
\end{equation}
Since $\mC_y$ and $\mD_y$ follow the same pattern with only the subscript replaced, it also holds that:
\begin{equation}
    \mC_y^2-\mD_y^2 = N \mA\mLambda_y^2\mA + N\mB\mLambda_y^2\mB. 
\end{equation}
Make use of $\mLambda_y=\left(\mI-\mLambda_x^2\right)^{\frac{1}{2}}$, we can conclude:
\begin{equation}
\label{eq:freq_54}
    \mC_x^2-\mD_x^2 + \mC_y^2-\mD_y^2 = N \mA\left(\mLambda_x^2+\mLambda_y^2\right)\mA + N\mB\left(\mLambda_x^2+\mLambda_y^2\right)\mB =N\mA^2+N\mB^2=N^2\mI.
\end{equation}
Substituting with Equations~(\ref{eq:freq_51}) and (\ref{eq:freq_54}), we can simplifies Equation~(\ref{eq:freq_50}) to express the covariance of $\sqrt{2}N\cdot\bfz$ as follows:
\begin{equation}
\label{eq:freq_55}
    \Cov\left(\sqrt{2}N\cdot\bfz\right) = 
    2N^2\mI
    -\frac{4\cos^2\theta}{1+\cos^2\theta}
    \left( \mA\mLambda_x\mB^2\mLambda_x\mA + \mB\mLambda_x\mA^2\mLambda_x\mB
    \right),
\end{equation}
Inspired by the form of $\mA\mLambda_x\mB^2\mLambda_x\mA$ and $\mB\mLambda_x\mA^2\mLambda_x\mB$ which are the matrix multiplication of $\mA\mLambda_x\mB$ and $\mB\mLambda_x\mA$. 
We creatively construct a new matrix $\mQ = \frac{1}{N}\left(\mA\mLambda_x\mB+\mB\mLambda_x\mA\right)$. It is easy to prove $\mQ$ is a symmetric matrix. The square of $\mQ$ is as follows:
\begin{equation}
\label{eq:Q_square}
    \mQ^2=\frac{1}{N^2}\left( \mA\mLambda_x\mB^2\mLambda_x\mA + \mB\mLambda_x\mA^2\mLambda_x\mB
    \right).
\end{equation}
By combining Equation~(\ref{eq:freq_55}) and Equation~(\ref{eq:Q_square}) and eliminating the constant $\sqrt{2}N$ from both sides of the equation, we can calculate the covariance of $\bfz$:
\begin{equation}
    \Cov\left(\bfz\right)=\mI - \frac{2\cos^2\theta}{1+\cos^2\theta}\mQ^2.
\end{equation}
Finally, we derive the distribution of $\bfz$ as follows:
\begin{equation}
\label{eq:freq_distribution}
    \bfz\sim\mathcal{N}\left(\mathbf{0}, \mI - \frac{2\cos^2\theta}{1+\cos^2\theta}\mQ^2\right).
\end{equation}
It is clear that the covariance of our refined noise is ``smaller'' than $\mI$. However, as $\mA\mB=\mB\mA=\mathbf{0}$ and the diagonal elements of $\Lambda_x$ ranges from 0 to 1, it gives the intuition that $\mQ$ is close to $\mathbf{0}$. 
We make further analysis in Appendix~\ref{appendix:covariance}.

\section{Covariance error analysis}
\label{appendix:theoretical_analysis}
\begin{theorem}
\label{theorem:4}
    Given two semi-positive definite matrices $\mC$ and $\mD$ satisfying $\mC\succeq\mD\succeq\mathbf{0}$, then $||\mC||_F \ge ||\mD||_F$ where $||\cdot||_F$ is Frobenius Norm.
\end{theorem}
\begin{proof}
Since $||\mC-\mD||_F^2\ge 0$, then expanding it yields:
\begin{equation}
    ||\mC||_F^2 + ||\mD||_F^2\ge \mathrm{tr}\left(\mC^\top\mD+\mD^\top\mC\right) = 2\mathrm{tr}\left(\mC\mD\right).
\end{equation}
The last equation holds because the $\mC$ and $\mD$ are symmetric matrices and $\mathrm{tr}(\cdot)$ is invariant under circular shifts.
Then we can conclude:
\begin{equation}
    ||\mC||_F^2 - ||\mD||_F^2  
    \ge  2\mathrm{tr}\left(\mC\mD\right) - 2||\mD||_F^2   
    = 2\mathrm{tr}\left(\left(\mC-\mD\right)\mD\right).  
\end{equation}
Using Cholesky decomposition, for semi-definite matrix $\mathbf D$, there exists matrix $\mL$ such that $\mD=\mL\mL^\top$. Then we can derive:
\begin{equation}
    \mathrm{tr}\left(\left(\mC-\mD\right)\mD\right)
    = \mathrm{tr}\left(\left(\mC-\mD\right)\mL\mL^\top\right) 
    = \mathrm{tr}\left(\mL^\top\left(\mC-\mD\right)\mL\right). 
\end{equation}
From the given condition $\mC\succeq\mD$, thus $\mC-\mD\succeq\mathbf{0}$, thus $\mL^\top\left(\mC-\mD\right)\mL$ is semi-positive. Therefore the trace of this matrix will be non-negative. Therefore $||\mC||_F \ge ||\mD||_F$.
\end{proof}

\label{appendix:covariance}
From Equations (\ref{eq:free_distribution}) and (\ref{eq:freq_distribution}), we know the covariance of refined noise for each method:
\begin{equation}
    \mathbf{\Sigma}_{FreeInit} = \mP^2 + (\mI-\mP)^2,\quad \mathbf{\Sigma}_{FreqPrior} = \mI - \frac{2\cos^2\theta}{1+\cos^2\theta}\mQ^2.
\end{equation}
Consider the same settings, including that the low-pass filters are identical, meaning $\mLambda_x$ is fixed.
Since the filter $\mathbf\Lambda_x$ is diagonal with its diagonal elements in the range $[0, 1]$, we have $\mathbf{0}\preceq\mLambda_x\preceq\mI$. 
Consequently, we obtain the following inequality for matrix $\mP$:
\begin{equation}
    \mathbf{0}\preceq \mP = \frac{1}{N}\left( \mA\mLambda_x\mA+\mB\mLambda_x\mB\right) \preceq \frac{1}{N}\left( \mA^2+\mB^2\right)=\mI.
\end{equation}
Now consider the difference between $\mathbf{\Sigma}_{FreeInit}$ and $\mI$:
\begin{equation}
\label{eq:cov_1}
    \mI-\mathbf{\Sigma}_{FreeInit} = 2\left(\mP - \mP^2\right)\succeq \mathbf{0}.
\end{equation}
This inequality holds because $\mathbf{0}\preceq \mP \preceq \mI$, which implies $\mP^2\preceq \mP$. This demonstrates that $\Sigma_{FreeInit}$ is indeed ``smaller'' than $\mI$.
To conduct a further analysis of $\Sigma_{FreeInit}$ and $\Sigma_{freqinit}$, we first establish the relationship between $\mP$ and $\mQ$.
Noticing that $\mP$ and $\mQ$ have similar forms,
we can derive the following results by leveraging these specific forms:
\begin{equation}
\label{eq:cov_2}
\begin{split}
    \mP^2 + \mQ^2 &= \frac{1}{N^2}\left(
    \left(\mA\mLambda_x\mA^2\mLambda_x\mA + \mB\mLambda_x\mB^2\mLambda_x\mB\right)+
    \left(\mA\mLambda_x\mB^2\mLambda_x\mA + \mB\mLambda_x\mA^2\mLambda_x\mB\right)\right) \\
    &=\frac{1}{N}\left(\mA\mLambda_x^2\mA+\mB\mLambda_x^2\mB\right) 
    \preceq 
    \frac{1}{N}\left(\mA\mLambda_x\mA+\mB\mLambda_x\mB\right) = \mP.
\end{split}
\end{equation}
Combining Equations (\ref{eq:cov_1}) and (\ref{eq:cov_2}), we obtain:
\begin{equation}
\label{eq:covariance_error_two}
    \mI-\mathbf{\Sigma}_{FreqPrior} = \frac{2\cos^2\theta}{1+\cos^2\theta}\mQ^2 \preceq \frac{2\cos^2\theta}{1+\cos^2\theta}\left(\mP - \mP^2\right) = \frac{\cos^2\theta}{1+\cos^2\theta}\left(\mI-\mathbf{\Sigma}_{FreeInit}\right).
\end{equation}
Then we can analyze the covariance errors (as defined in Definition~\ref{definition:1}) by applying Theorem~\ref{theorem:4}:
\begin{equation}
    ||\mI-\mathbf{\Sigma}_{FreqPrior}||_F 
    \le \frac{\cos^2\theta}{1+\cos^2\theta}||\mI-\mathbf{\Sigma}_{FreeInit}||_F. 
\end{equation}
In practice, for common continuous low-pass filters, such as Butterworth filters and Gaussian filters, the corresponding function values monotonically decrease as the frequency increases. 
Given that $\mA\mB=\mB\mA$ and $\mQ=\frac{1}{N}(\mA\mLambda\mB+\mB\mLambda\mA)$, it follows that $\mQ^2$ is intuitively close to a zero matrix, making the covariance error nearly zero. This is further corroborated by our numerical experiments.


\section{Experimental details}
\label{appendix:setting}
Three open-sourced text-to-video models are used as the base models for evaluation: They are AnimateDiff~\citep{guo2023animatediff}, ModelScope~\citep{wang2023modelscope,VideoFusion}, and VideoCrafter~\citep{chen2023videocrafter1}.
\begin{itemize}
    \item For AnimateDiff, we use mm-sd-v15\_v2 motion module along with realisticVisionV20\_v20 dreambooth LoRA~\footnote{https://huggingface.co/ckpt/realistic-vision-v20/blob/main/realisticVisionV20\_v20.safetensors}, sampling 16 frames of at a resolution of $512\times512$ at 8 FPS with a guidance scale of 7.5.
    \item For ModelScope, we utilize the modelscope-damo-text-to-video-synthesis version, sampling 16 frames at a resolution of $256 \times 256$ at 8 FPS, with a guidance scale of 9.
    \item For VideoCrafter, we employ the VideoCrafter-v1 base text-to-video model, sampling 16 frames at a resolution of  $320 \times 320$ at 10 FPS, with a guidance scale of 12.
\end{itemize}


\section{Evaluation metrics}
\label{appendix:vbench}
% vbench 
We employ VBench~\citep{huang2023vbench} for evaluation, a comprehensive benchmark designed with tailored prompts and evaluation dimensions specifically aimed at assessing video generation performance.
A key feature of VBench is its incorporation of human preference annotations, ensuring alignment with human perception.
VBench uses a hierarchical and disentangled scoring system, breaking the overall \textbf{\textit{total score}} into two main components: \textbf{\textit{quality score}} and \textbf{\textit{semantic score}}.
It covers 16 evaluation dimensions, with 7 contributing to \textbf{\textit{quality score}} and 9 contributing to \textbf{\textit{semantic score}}. 
Each dimension is assessed using a specially designed approach, ensuring precise and meaningful evaluation of the generated videos. The assessments involve various off-the-shelf models~\citep{caron2021dino, ruiz2023dreambooth, radford2021clip, li2023amt, teed2020raft, laion2022aesthetic,ke2021MUSIQ,wu2022GRiT, li2023umt,huang2023t2i-compbench,huang2024tagtext, wang2024internvid},  and the score for each dimension is normalized on a 0 to 100 scale, based on empirical minimum and maximum values.
\begin{itemize}
    \item \textbf{\textit{Quality score}} is calculated as the weighted average of seven dimensions: {\it subject consistency}, {\it background consistency}, {\it temporal flickering}, {\it motion smoothness}, {\it dynamic degree}, {\it aesthetic quality}, and {\it imaging quality}. The weight for {\it aesthetic quality} is set to 2, while the other dimensions carry a weight of 1. 
    % The rationale behind this is that there are several dimensions related to temporal consistency, while \textit{dynamic degree} is the only dimension that specifically measures motion dynamics.
    \item \textbf{\textit{semantic score}} is calculated as the weighted average of nine dimensions: {\it object class}, {\it multiple objects}, {\it human action}, {\it color}, {\it spatial relationship}, {\it scene},{\it appearance style}, {\it temporal style}, and {\it overall consistency}, with each dimension equally weighted 1.
\end{itemize}
 
After calculating \textbf{\textit{quality score}} and \textbf{\textit{semantic score}}, \textbf{\textit{total score}} is calculated as follows:
\begin{equation}
    \mathrm{Total} = \frac{w_q}{w_q+w_s}\mathrm{Quality}+\frac{w_s}{w_q+w_s}\mathrm{Semantic},
\end{equation}
where $w_q$ and $w_s$ are $4$ and $1$ respectively by default.

\section{Visualization results}
\paragraph{More qualitative results.} 
Additional qualitative results are presented in Figure~\ref{fig:appendix_vis}. 
The videos generated using our noise prior exhibit superior video quality, in terms of imaging details, aesthetic aspects, and semantic coherence.
\paragraph{Visualization of the influence of timestep \boldmath$t$.} 
As illustrated in Figure~\ref{fig:ablation_timestep},  the first three rows of frames, which correspond to different timesteps $t$, are almost the same.
In the fourth case, there are some differences in the representation of the grape stem, highlighted by a red box. The stem is absent at the timestep of 321. 
At the timestep of 321, the stem is missing. 
The final case demonstrates more notable differences; as the timestep $t$ increases, the ice cream appears to melt, and the changes are observable on the table.
The visualizations suggest two key points: first, in most instances, the timestep 
$t$ has minimal impact on the overall generation results; second, although the content and layout of the video frames remain largely unchanged, the timestep can indeed influence the finer imaging details.
In general, the differences are quite minor, which means we can save much time by diffusing the latent at intermediate timestep during the noise refinement stage without compromising the quality of generation results.

\begin{figure}[t]
    \centering
    \includegraphics[width=1.0\linewidth]{figure/ablation_timestep.pdf}
    \caption{ \textbf{Visualization results of the influence of timestep.} We present five cases where other settings are fixed to isolate the effects of varying timestep $t$. Overall, timestep $t$ has minimal impact on the generation outcomes. However, it does exert some influences on the imaging details occasionally. For the fourth and fifth cases, the red boxes highlight the differences.}
    \label{fig:ablation_timestep}
\end{figure}

\section{Limitations}
While our method enhances consistency and smoothness in videos generated from Gaussian noise, it can occasionally result in unnatural smoothness that does not align with the laws of physics. 
Additionally, although our approach improves overall performance, it may alter the content layout of video frames compared to Gaussian noise. For real images, low-frequency signals typically dictate layouts; however, this is not always true for the noise prior in diffusion models. 
Our method refines the Gaussian noise prior using a novel frequency filtering technique, which usually preserves the structural similarity to the original Gaussian noise. Nonetheless, in some cases, the generated videos can differ significantly. During filtering, high-frequency components from other Gaussian noise may subtly change the structure of the Gaussian noise prior, resulting in variations in the content and layouts of the generated videos.


\section{Broader impacts}
\label{appendix:impacts}
This work aims to propose a novel prior by refining initial Gaussian noise to enhance the quality of video generation. 
Text-to-video diffusion models hold the potential to revolutionize media creation and usage. While these models offer vast creative opportunities, it is crucial to address the risks of misinformation and harmful content. 
Before deploying these models in practice, it is essential to thoroughly investigate their design, intended applications, safety aspects, associated risks, and potential biases.

\begin{figure}[t]
    \centering
    \includegraphics[width=1.0\linewidth]{figure/visualization_appendix.pdf}
    \caption{{\bf More qualitative results.}}
    \label{fig:appendix_vis}
\end{figure}


\end{document}
