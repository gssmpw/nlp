

\section{Approximation for Hopbound 2} \label{app:2hop}
When the hopbound is $2$, we can get improved bounds by using the hopset version of a spanner approximation algorithm.  Interestingly, while we will require the hopbound to be $2$, we can handle arbitrary distance bound functions (including exact hopsets, $(1+\epsilon)$-stretch hopsets, and larger stretch hopsets), in contrast with the spanner version which requires all edges to have unit length and requires the \emph{stretch} to be at most $2$.

 
We use a version of the LP rounding algorithm for stretch $2$ from~\cite{DK11}.  In particular, we first solve our LP relaxation~\eqref{lp:hopset} to get an optimal solution $\bm{\mathrm{x}}$, and then round as follows.  First, every vertex $v \in V$ chooses a threshold $T_v \in [0,1]$ uniformly at random.  We then return as a hopset the set of edges $E' = \{(u,v) : \min(T_u, T_v) \leq (c \ln n) \cdot x_{(u,v)} \}$, where $c$ is an appropriately chosen constant. 

\begin{lemma} \label{lem:2hop-cost}
    %$\E[|E' \setminus E|] \leq O(\ln n) \cdot \opt$
    $\E[c(E')] \leq O(\ln n) \cdot \opt$
\end{lemma}
\begin{proof}
    Fix some $e = (u,v) \in \widetilde{E}$.
    %E_m \setminus E$.  
    Clearly $\Pr[e \in E'] \leq (2c\ln n) x_e$, and hence $\E[c(E')] \leq \sum_{e \in \widetilde{E}} (2c\ln n) x_e \leq (2c \ln n) \cdot \opt$.
\end{proof}

\begin{lemma} \label{lem:2hop-correct}
$E'$ is a valid hopset with high probability.
\end{lemma}
\begin{proof}
    Fix some $(u,v) \in V \times V$.  If $x_{(u,v)} \geq 1/(c \ln n)$ then our algorithm will directly include the $(u,v)$ edge.  So without loss of generality, we may assume that $x_{(u,v)} \leq 1/2$.  As discussed, our LP~\eqref{lp:hopset} is equivalent to the flow LP, and hence $\bm{\mathrm{x}}$ supports flows $f : \mathcal P_{u,v} \rightarrow [0,1]$ such that $\sum_{P \in \mathcal P_{u,v}} f_P \geq 1$.  Since $\beta = 2$ and $x_{(u,v)} \leq 1/2$, this means that $\sum_{P \in \mathcal P_{u,v} \setminus \{(u,v)\}} f_P \geq 1/2$, i.e., at least $1/2$ flow is sent on paths of length $2$ in $\mathcal P_{u,v}$.  Let $W \subseteq V \setminus \{u,v\}$ be the set of nodes that are the middle node of such a path, and for each $w \in W$ let $P_w$ denote the path $u-w-v \in \mathcal P_{u,v}$.

    Fix $w \in W$.  Note that if $T_w \leq (c \ln n) f_{P_w}$ then our algorithm will include the path $P_w$, since the $\bm{\mathrm{x}}$ variables support the flow $f$.  Hence the probability that we fail to include every path $P \in \mathcal P_{u,v}$ is at most
    \begin{align*}
        \prod_{w \in W} (1-(c\ln n) f_{P_w}) \leq \exp\left(-\sum_{w \in W} (c \ln n) f_{P_w}\right) \leq \exp(-(c \ln n) (1/2)) = n^{-c/2}.
    \end{align*}
    
    We can now take a union bound over all $(u,v) \in V \times V$ to get that the probability that $E'$ is not a valid hopset is at most $n^{2-(c/2)}$.  Hence by changing $c$, we can make the failure probability $1/n^{c'}$ for arbitrarily large constant $c'$.  
\end{proof}

\begin{theorem} \label{thm:2hop-main}
    There is an $O(\ln n)$-approximation algorithm for {\hopset} when $\be = 2$.
\end{theorem}
\begin{proof}
    This is directly implied by Lemmas~\ref{lem:2hop-cost} and \ref{lem:2hop-correct}.
\end{proof}



\section{Proofs from Section~\ref{sec:lp_relaxation}} \label{app:lp}

\begin{lemma}
    Let $H$ be a feasible solution to {\hopset}. For every edge $e \in E_M$, let $x_e = 1$ if $e \in H$ or $e \in E$. Otherwise let $x_e = 0$. Then, \bm{\mathrm{x}} is a feasible integral solution to~\ref{lp:hopset}.
\end{lemma}
\begin{proof}
    Clearly, $x_e \geq 0$ for all $e \in E_M$, so it is left to show that the cut covering constraints are satisfied. For some demand $(s,t) \in \mathcal{D}$, consider some edge cut against valid $s-t$ paths, $C$. There is a vector $\bm{\mathrm{z}} \in \mathcal{Z}_{s,t}$ that serves as the indicator vector for cut $C$---specifically, there is a vector $\bm{\mathrm{z}} \in \mathcal{Z}_{s,t}$ such that for every edge $e \in C$, we have $z_e = 1$, and $z_e = 0$ otherwise. 
    
    Since $H$ is a feasible solution to {\hopset}, there must be some edge $e \in C \cap H$ (otherwise, there is no valid path from $s$ to $t$ in $H \cup E$, making $H$ infeasible). This also means $x_e = 1$ and  $z_e = 1$. Hence we have that for $\bm{\mathrm{z}}$, the constraint $\sum_{e \in E_M} z_e x_e \geq 1$ is satisfied.
\end{proof}

\begin{lemma}
    Let $\bm{\mathrm{x}}$ be a feasible integral solution to~\ref{lp:hopset}, and let $H = \{ e \; | \; x_e = 1, e \notin E\}$. Then $H$ is a feasible solution to {\hopset}.
\end{lemma}
\begin{proof}
    Suppose for contradiction there is some demand, $(s,t) \in \mathcal{D}$, that is not satisfied by $H$.  Then there is some cut $C$ against valid $s-t$ paths such that $H \cap C = \emptyset$ (and $x_e = 0$ for all $e \in C$). Since $C$ is a cut against valid $s-t$ paths, there is an indicator vector $\bm{\mathrm{z}} \in \mathcal{Z}_{s,t}$ such that for every edge $e \in C$, we have $z_e = 1$ ($z_e = 0$ otherwise). Therefore we have that $\sum_{e \in E_M} z_e x_e = 0 < 1$, violating the constraint in \ref{lp:hopset}, and thus the assumption that $\bm{\mathrm{x}}$ is feasible. Hence $H$ must be feasible.
\end{proof}



\subsection{Hopbounded Restricted Shortest Path Algorithm and Analysis} \label{app:lp_alg}
Like the Restricted Shortest Path dynamic programming algorithm, our DP algorithm takes in as input valid upper and lower bounds, $U$ and $L$, respectively. We also scale the edge costs in the same way, and refer to these scaled costs as \quotes{pseudo-costs.} Let $D(v, i, j)$ indicate the minimum length (in the original distance metric) of a path from vertices $s$ to $v$ with pseudo-cost at most $i$ and at most $j$ hops. Our algorithm differs from the Restricted Shortest Path algorithm in that we add an additional dimension for hops to the dynamic programming algorithm.

\begin{algorithm}[h]
\DontPrintSemicolon

\textbf{Input:} Graph $G = (V, E)$, demand pair $s,t \in V$, \\
cost, distance, and hop metrics $\{z_e, \ell_e, h_e \}_{e \in E_M}$, \\
parameters $T, \beta, L, U,$ and $\epsilon $ \;~\\

Let $S \gets \frac{L \epsilon}{n+1}, \; \Tilde{U} \gets \floor*{U/S} + n + 1$ \\ \;

\tcp{scale edge costs} 
\ForEach{edge $e \in E$}{
    Let $\Tilde{z}_e \gets \floor{z_e / S} + 1$} \;

\tcp{set base cases} 
\ForEach{index $i  = 0, 1, \dots, T$}{
    $D(s, i, 0) \gets 0$} 
\ForEach{index $j  = 0, 1, \dots, \beta$}{
    $D(s, 0, j) \gets 0$} 
    
\ForEach{vertex $v \in V$ such that $v \neq s$}{
    \ForEach{index $i = 0, 1, \dots, T$}{
        $D(v, i, 0) \gets \infty$}
    \ForEach{index $j = 0, 1, \dots, \beta$}{
        $D(v, 0, j) \gets \infty$} 
} \;

\tcp{build up DP table}
\ForEach{index $i = 1, 2, \dots, \Tilde{U}$}{
    \ForEach{index $j = 1, 2, \dots, \beta $}{
        \ForEach{vertex $v \in V$}{
            $D(v, i, j) \gets \min\{ \: D(v, i-1, j), \; D(v, i, j-1) \: \}$ \;
            \ForEach{edge $e \in \{ (u,v) \; | \; \Tilde{z}_{(u,v)} \leq i \}$ }{
                $D(v, i, j) \gets \min\{ \: D(v, i, j), \; \ell_e + D(v, i-\Tilde{z}_{e}, j-1) \: \}$ \;
            }
        \If{$D(t, i, j) \leq T$}{
            \textbf{Return} the corresponding path \;
        }
        }
    }
}
\textbf{Return} FAIL \;

\caption{\label{alg:HRSP} Hopbounded Restricted Shortest Path Algorithm } 
\end{algorithm}

Using arguments from \cite{LR01}, we will show that Algorithm~\ref{alg:HRSP} is a $(1+\epsilon)$-approximation of the Hopbounded Restricted Shortest Path problem. The following is directly from \cite{LR01}, and also holds for our slightly modified algorithm. Let $z(P)$ denote the cost (in the $\bm{\mathrm{z}}$ metric) of a path $P$ in $G_M$, and let $z^*$ denote the cost of the optimal path for the original Hopbounded Restricted Shortest Path instance.

\begin{lemma}[Lemma 3 of \cite{LR01}]
\label{lem:eps_approx}
    If $U \geq z^*$, then Algorithm~\ref{alg:HRSP} returns a feasible path, $P'$, that satisfies $z(P') \leq z^* + L\epsilon$
\end{lemma}

\cite{LR01} uses this lemma, and others, to show that there is a $(1+\epsilon)$-approximation for the Restricted Shortest Path problem (Theorems 3 and 4 of \cite{LR01}). Their algorithm achieves a runtime of $O(m n/ \epsilon + m n \log \log \frac{U}{L})$. While their full argument would also give a similar runtime for our setting (with an additional factor $\beta$), we only include the arguments necessary to achieve a polynomial runtime. As a result, we get a worse runtime than what \cite{LR01} would imply. 

\begin{lemma} \label{lem:AS}
    Given valid lower and upper bounds $0 \leq L \leq z^* \leq U$, Algorithm~\ref{alg:HRSP} is a $(1+\epsilon)$-approximation for the Hopbounded Restricted Shortest Path problem that runs in time $O(\beta m n U / L \epsilon)$. 
\end{lemma}
\begin{proof}
The approximation ratio is directly implied by Lemma~\ref{lem:eps_approx}. 
    
Each edge in the algorithm is examined a constant number of times, so the overall time complexity is 
\begin{align*}
    O(\beta m \Tilde{U}) &= O\left( \beta m \frac{n}{\epsilon} \frac{U}{L} + nm \right) \\
    &= O\left( \beta m \frac{n}{\epsilon} \frac{U}{L}  \right) \tag{$U \geq L$ and $\epsilon \leq 1$}
\end{align*}
\end{proof}

Now we find upper and lower bounds $U$ and $L$ such that $U/L \leq n$, just as in \cite{LR01}, to give an overall runtime of $O(\beta m n^2 / \epsilon)$ for our problem. Let $\ell_1 < \ell_2 < \dots, \ell_p$ be the distinct lengths of all edges in $E_M$ (note that $p \leq m$). Let $E_i$ be the set of edges in $E_M$ with length at most $\ell_i$, and let $G_i = (V, E_i)$. We also set $E_0 = \emptyset$.
There must be some index $j \in [1,p]$ such that there exists a $T$-length-bounded, $\beta$-hopbounded path in $G_j$, but not in $G_{j-1}$. As observed in Claim 1 of \cite{LR01}, this means that $\ell_j \leq z^* \leq n \ell_j$, giving bounds $L = \ell_j$ and $U = n \ell_j$, with $U/L = n$. To find $\ell_j$, we can binary search in the range $\ell_1, \dots, \ell_p$, running a shortest hopbounded path algorithm at each step of the binary search to check if there exists a $T$-length-bounded, $\beta$-hopbounded $s-t$ path on the subgraph $G_i$ corresponding to that step. The binary search requires $O(\log m)$ steps, and the shortest hopbounded path algorithm takes $O(m \log n)$ time, so the runtime for finding these bounds is dominated by the runtime of the the DP algorithm. 


In summary, the Hopbounded Restricted Shortest Path algorithm is as follows: Run binary search on $\ell_1, \dots, \ell_p$ to find $\ell_j$ and get bounds $U$ and $L$. Then run the dynamic programming algorithm, Algorithm~\ref{alg:HRSP}, using bounds $U$ and $L$.

\section{Proofs from Section \ref{sec:junction_tree}}

\subsection{Proof of Lemma~\ref{lem:reduction}} \label{app:reduction}
\begin{lemma}
\label{cl:input_to_layered}
    For any $(i,j)$-junction tree $T$ in $G_M$, there exists an $(i,j)$-junction tree $T_L$ in $G_L$ with the same density. 
\end{lemma}
\begin{proof}
    Fix an $(i,j)$-junction tree $T$ in $G_M$. We build an $(i,j)$-junction tree $T_L$ in $G_L$ with the same density. Tree $T$ is the union of an in-arborescence, which we denote as $A^{in}$, and an out-arborescence, denoted by $A^{out}$, both of which are rooted at some node $r$. We add node $r_i \in V_L$ to $T_L$ as its root. 
    For each vertex $u \in A^{in}$, let $h_u$ be the number of hops (edges) on the $u-r$ path in $A^{in}$; likewise, for each vertex $w \in A^{out}$, let $h_w$ be the number of hops on the $r-w$ path in $A^{out}$. For each edge $(u,v) \in A^{in}$, we add edge $(u_{i-h_u}, v_{i-h_u+1})$ (and corresponding nodes) to $T_L$. Similarly, for each edge $(u,v) \in A^{out}$, we add edge $(u_{i+h_u}, v_{i+h_u+1})$ (and corresponding nodes) to $T_L$.
   
    Finally, we add ``dummy paths" to $T_L$ to ensure that the corresponding demands in $\mathcal{D}_L$ are settled. These dummy paths are need to handle demand-settling paths in $T$ that have fewer that $\beta$ hops. Let $A^{in}_L$ and $A^{out}_L$ denote the in- and out-arborescences of $T_L$, respectively. For each vertex $v_x \in A^{in}_L$, we add edges $\{ (v_k, v_{k+1}) :  0 \leq k < x  \}$ (and corresponding nodes) to $T_L$ (if they don't already exist in $T_L$). Similarly for each vertex $v_x \in A^{out}_L$, we add edges $\{ (v_k, v_{k+1}) :  x \leq k < \beta  \}$ (and corresponding nodes) to $T_L$ (if they don't already exist in $T_L$). With these dummy paths, we have that if demand $(s,t)$ is settled by a path in $T$ with fewer than $\beta$ hops, then the corresponding demand $(s_0, t_\beta)$ is also settled in $T_L$.  

    Tree $T_L$ has the same cost as $T$: for every edge $(u,v) \in T$ we add an edge $(u_k, v_{k+1})$, for some integer $k$, to $T_L$. Recall that $(u,v)$ and $(u_k, v_{k+1})$ have the same cost, for any $0 \leq k < \beta$. All other edges in $T_L$ (i.e., all edges on dummy paths) are of the form $(u_k, u_{k+1})$, for some $k$, and have cost $0$. 
    
    The number of demands satisfied in both trees is also the same, which we show by mapping each demand $(s,t) \in \mathcal{D}$ settled by $T$ to a unique demand $(s_0, t_\beta) \in \mathcal{D}_L$ settled by $T_L$ (and vice versa). We now show that if $(s,t)$ is settled by $T$, then $(s_0, t_\beta)$ is settled by $T_L$. Let $P = \{s, a, b, \dots, t\}$ be the path in $T$ that settles $(s,t)$. Then, path $P_L = \{s_0, \dots, s_k, a_{k+1}, b_{k+2}, \dots, t_{m}, \dots , t_\beta \}$ is in $T_L$, for some $k, m$. Paths $P$ and $P_L$ have the same length---any edge $(u,v) \in P$ has the same length as its corresponding edge $(u_k, v_{k+1})$ (for some $k$) in $P_L$, and all other edges (i.e., edges from the dummy subpaths $(s_0, \dots, s_k)$ and $(t_m, \dots, t_\beta)$, if they exist) have length $0$. We therefore have that $d_{T_L}(s_0,t_\beta) = d_T(s,t) \leq  Dist(s,t) = Dist_L(s_0,t_\beta)$. Also, path $P_L$ has exactly $\beta$ hops. Thus, demand $(s_0, t_\beta)$ is settled by $T_L$. It is also not difficult to see that each demand $(s_0, t_\beta) \in \mathcal{D}_L$ settled by $T_L$ can be mapped to a unique demand $(s,t) \in \mathcal{D}$ settled by $T$, using similar arguments. Trees $T$ and $T_L$ therefore have the same cost and settle the same number of demands, and so have the same density. 
\end{proof}

\begin{lemma}
\label{cl:layered_to_input}
    For any $(i,j)$-junction tree $T_L$ in $G_L$, there exists an $(i,j)$-junction tree $T$ in $G_M$ with the same density.
\end{lemma}
\begin{proof}
     Given an $(i,j)$-junction tree $T_L$ of $G_L$, we can build an $(i,j)$-junction tree $T$ in $G_M$ with the same density. Note that edges only exist between adjacent layers in $G_L$ (and in $T_L$)---namely, all edges in $T_L$ are of the from $(u_k, v_{k+1})$ for some $k$. For each $k \in [0, \beta]$ and for each edge $(u_k, v_{k+1})$ in $T_L$ such that $u \neq v$, we add edge $(u, v) \in G_M$ (and corresponding nodes) to $T$. 
    
    Trees $T_L$ and $T$ have the same cost: For each edge $(u_k, v_{k+1}) \in T_L$ (for some $k$) with cost $1$, we add edge $(u,v)$ to $T$, which also has cost $1$. For all other edges in $T_L$ (all of which have no cost), we either add the corresponding edge to $T$, which also has no cost, or we add no edge. 
    
    Both trees also settle the same number of demands. Each demand $(s_0, t_\beta) \in \mathcal{D}_L$ settled by $T_L$ can be mapped to a unique demand $(s,t) \in \mathcal{D}$ settled by $T$. Let $P_L = \{s_0, \dots, s_k, a_{k+1}, b_{k+2}, \dots, t_{m}, \dots , t_\beta \}$ be the path that settles $(s_0, t_\beta)$ in $T_L$. Then, the path $P = \{s, a, b, \dots, t\}$ is in $T$. Paths $P$ and $P_L$ have the same length---any edge of the form $(u_k,v_{k+1}) \in P_L$ (for some $k$), where $u \neq v$, has the same length as its corresponding edge $(u, v) \in P$. All other edges in $P_L$ (i.e., edges from the dummy subpaths if they exist) have length $0$. We therefore have that $d_T(s,t) = d_{T_L}(s_0,t_\beta) \leq Dist_{L}(s_0,t_\beta) = Dist(s,t)$. Path $P$ also has at most $\beta$ hops, so $(s,t)$ is settled by $T$. It is also not difficult to see that each demand $(s, t) \in \mathcal{D}$ settled by $T$ can be mapped to a unique demand $(s_0,t_\beta) \in \mathcal{D}_L$ settled by $T_L$, using similar arguments. We've shown that $T_L$ and $T$ have the same cost and settle the same number of demands, and so they have the same density.
\end{proof}

Let $\Delta(T)$ denote the density of junction tree $T$. The above lemmas imply the following:

\begin{corollary}
\label{cor:equivalent}
    Let $T^*$ be the min-density $(i,j)$-junction tree (over all possible $i,j$) in $G_M$, and let $T_L^*$ be the min-density $(i,j)$-junction tree (over all possible $i,j$) in $G_L$. Then, $\Delta(T^*) = \Delta(T_L^*)$.
\end{corollary}
\begin{proof}
    By Lemma~\ref{cl:input_to_layered}, we have that $\Delta(T_L^*) \leq \Delta(T^*)$.  By Lemma~\ref{cl:layered_to_input}, $\Delta(T^*) \leq \Delta(T_L^*)$. Therefore, $\Delta(T^*) = \Delta(T_L^*)$.
\end{proof}

We now use this Corollary to reduce from finding min-density $(i,j)$-junction trees in $G_M$ to finding them in $G_L$. We say that {\jt} is the optimization problem of finding the minimum density $(i,j)$-junction tree in an input graph, over all possible values of $i,j$.

Suppose we have an $\alpha$-approximation for {\jt} on graph $G_L$. Then, the following is an algorithm for {\jt} on $G_M$. First, run the $\alpha$-approximation algorithm on $G_L$, and let $T_L$ be the tree returned by the algorithm. Using the procedure described in Lemma~\ref{cl:layered_to_input}, we can build (in polynomial time) a valid $(i,j)$-junction tree $T$ of $G_M$ with the same density as $T_L$. The density of $T$ is as follows:
    \begin{align*}
        \Delta(T) = \Delta(T_L) \leq \alpha \cdot \Delta(T_L^*) = \alpha \cdot \Delta(T^*).
    \end{align*}
    The final equality is due to to Corollary~\ref{cor:equivalent}.

\subsection{Proof of Lemma~\ref{lem:hop_JT_approx}} \label{app:jt_alg}

\begin{proof}
    By Lemma~\ref{lem:reduction}, we can prove the lemma by giving an $O(n^\epsilon)$-approximation for {\jt} on $G_L$. The algorithm is as follows: Simply run the algorithm of Lemma~\ref{lem:og_JT_alg} on $G_L$.

    We now show that the tree $T_L$ returned by this algorithm is an $(i,j)$-junction tree of $G_L$, where $i+j = \beta$, and that the density of $T_L$ is at most a factor $O(n^\epsilon)$ of the density of the optimal tree.
    Tree $T_L$ has some root $r_i$. 
    Fix a demand $(s_0,t_\beta) \in \mathcal{D}_L$ that has length at most $Dist_L(s,t)$ in $T_L$. Due to the structure of $G_L$, the path from $s_0$ to $t_\be$ in $T_L$ must have $i$ hops from $s_0$ to $r_i$ and $\beta - i$ hops from $r_i$ to $t_\beta$. Thus $T_L$ is an $(i,j)$-junction tree. To see that this is an $O(n^\epsilon)$-approximation, first observe that the optimal density $(i,j)$-junction tree is a feasible solution to {\ljt} on $G_L$. As for the approximation ratio, the algorithm gives a $O(|V_L|^\epsilon) = O(\beta^\epsilon |V|^\epsilon) = O(n^{\epsilon'})$ approximation, where $n = |V|$ and $\epsilon' > 0$ is an arbitrarily small constant.
\end{proof}




\section{Proofs from Section~\ref{sec:star_sampling_rounding}}
\subsection{Proof of Lemma~\ref{lem:thick}} \label{app:thick_proof}
\begin{proof}  
    After the execution of the first for loop in Algorithm~\ref{alg:star_sample}, $|E'| \leq 2(n-1)b \ln{n}$.

    If some vertex $v$ from a set $V^{s,t}$ appears in the set $S$ of vertices selected by Algorithm~\ref{alg:star_sample}, then $T^{in}_v$ and $T^{out}_v$ contain shortest, $1$-hop paths from $s$ to $v$ and from $v$ to $t$, respectively. Thus, both paths are contained in $E'$. Since $v \in V^{s,t}$, the sum of lengths of these two paths is at most $Dist(s,t)$. Therefore, if $S \cap V^{s,t} \neq \emptyset$, then the demand $(s,t)$ is settled. For a thick demand $(s,t)$, the set $S \cap V^{s,t}$ is empty with probability at most $(1-1/b)^{b\ln{n}} \leq e^{-\ln{n}} = 1/n$. Thus, the expected number of unsettled thick demands added to $E'$ in the final for loop of Algorithm~\ref{alg:star_sample} is at most $|\mathcal{D}|/n \leq n$.

    The final for loop ensures that $E'$, returned by the algorithm, settles all thick demands. Computing in- and out-stars and determining whether a demand is thick can be done in polynomial time.
\end{proof}


\subsection{Proof of Lemma~\ref{lem:thin_fail}} \label{app:thin_proof}
\begin{proof}
    There are two different events that can cause the algorithm to fail:   
    \begin{enumerate}
        \item The cost of the sampled set $E''$ is too high---that is, $c(E'') > 2{\opt} \cdot 2(n/b)  \ln{n}$. The expected cost of $E''$ is at most $2(n/b) \ln{n} \, \cdot \sum_{e \in E_M \setminus E} x_e \leq {\opt} \cdot 2(n/b)  \ln{n}$. By the Chernoff bound (recall that $b = \sqrt{{\opt}}$), we have that $\Pr[c(E'') > 2{\opt} \cdot 2(n/b)  \ln{n} ] \leq e^{-c \cdot {\opt} \cdot (n/b) \ln{n}} = e^{-c n\ln{n} \cdot \sqrt{{\opt}}} = e^{-\Omega(n \ln(n))}$. Thus, the probability that the algorithm fails because $c(E'') > 2{\opt} \cdot 2(n/b)  \ln{n}$ is exponentially small.
        \item $E''$ does not settle all thin demands. We prove that the probability that $E''$ does not settle all thin demands (that is, that $E''$ does not intersect all minimal anti-hopsets for thin demands) is exponentially small in the following Lemma.
    \end{enumerate}
    
    \begin{lemma}[Lemma 2.3 of~\cite{BBMRY11}] \label{lem:thin_not_settle}
        The probability that there exists a demand $(s,t)$ and a minimal anti-hopset $C$ for it such that $C \subset E^{s,t} \setminus E''$ is at most $|\mathcal{D}_{thin}| \cdot \left( 1 / b n \right)^{n/b}$. In particular, if $b = \sqrt{{\opt}}$, then the probability is at most $|\mathcal{D}_{thin}| \cdot (1 \, / \, n\sqrt{{\opt}})^{n/\sqrt{\opt}}$.
    \end{lemma}
    \begin{proof}
        First, we bound the total number of minimal anti-hopsets for thin demands. 

        \begin{proposition}[Proposition 2.1 of \cite{BBMRY11}] \label{prop:thin1}
            If $(s,t)$ is a thin demand, then there are at most $(n/b)^{n/b}$ minimal anti-hopsets for $(s,t)$. 
        \end{proposition}
        \begin{proof}
            Fix a thin demand $(s,t)$ and consider an arbitrary minimal anti-hopset $C$ for $(s,t)$. For the rest of this argument, for any two paths of the same length between the same pair of vertices, we consider one path to be shorter than the other if that path has fewer hops. More specifically, for any vertex pair $x, y \in V$ and any two $x-y$ paths $P, P'$ that have the same length, we say that $P$ is shorter than $P'$ if $P$ has fewer hops (edges).
            
            Let $A_C$ be the outward shortest path tree (arborescence) rooted at $s$ in the graph $(V^{s,t}, E^{s,t} \setminus C)$. Our tie-breaking of same-length paths with different hops ensures that $A_C$ includes shortest paths with the lowest number of hops (recall that every edge $(u,v) \in E_M$ is a shortest path from $u$ to $v$). 
            Denote by $f^{(\beta)}_{A_C}(u)$ the $\beta$-hopbounded distance from $s$ to $u$ in the tree $A_C$. If there is no $\beta$-hopbounded directed path from $s$ to $u$ in $A_C$, let $f^{(\beta)}_{A_C}(u) = \infty$. 
            
            We show that $C = \left\{ (u,v) \in E^{s,t} : f^{(\beta)}_{A_C}(u) + \ell(u,v) <  f^{(\beta)}_{A_C}(v) \right\}$, and thus, $A_C$ uniquely determines $C$ for a given thin demand $(s,t)$. If $(u,v) \in C$, then, since $C$ is a \textit{minimal} anti-hopset, there exists a $\beta$-hopbounded path from $s$ to $t$ of length at most $Dist(s,t)$ in the graph $(G_M \setminus C) \cup \{(u,v)\}$. This path must lie in $(G^{s,t} \setminus C) \cup \{(u,v)\}$ and must contain the edge $(u,v)$. Thus, the $\beta$-hopbounded distance from $s$ to $t$ in the graph $(G^{s,t} \setminus C) \cup \{(u,v)\}$ is at most $Dist(s,t)$ and is strictly less than $f^{(\beta)}_{A_C}(t)$. Hence, $A_C$ is not the shortest path tree in the graph $(G^{s,t} \setminus C) \cup \{(u,v)\}$. Therefore, $f^{(\beta)}_{A_C}(u) + \ell(u,v) <  f^{(\beta)}_{A_C}(v)$. If $(u,v) \in E^{s,t}$ satisfies the condition $f^{(\beta)}_{A_C}(u) + \ell(u,v) <  f^{(\beta)}_{A_C}(v)$, then $(u,v) \notin E^{s,t} \setminus C$ (otherwise, $A_C$ would not be the shortest path tree), hence $(u,v) \in C$.

            We now count the number of outward trees rooted at $s$ in $G^{s,t} \setminus C$. For every vertex $u \in V^{s,t}$, we may choose the parent vertex in at most $|V^{s,t}|$ possible ways (if a vertex is isolated we assume that it is its own parent), thus the total number of trees is at most $|V^{s,t}|^{|V^{s,t}|} \leq (n / b)^{n/b}$
        \end{proof}

        \begin{proposition}[Proposition 2.2 of \cite{BBMRY11}] \label{prop:thin2}
            For a demand $(s,t)$ and a minimal anti-hopset $C$ for $(s,t)$, the probability that $E'' \cap C = \emptyset$ is at most $e^{-2(n/b)\ln{n}}$.
        \end{proposition}
        \begin{proof}
            Suppose there is an anti-hopset edge $e \in C$ such that $x_e^* \geq (2(n/b)\ln{n})^{-1}$. In this case, $e$ is in $E''$ with probability $1$, and we are done. Otherwise, the probability that $e$ is in $E''$ is exactly $2(n/b)\ln{n} \cdot x_e$. In this case, the probability that $E''$ does not include $e$ is 
            \begin{align*}
                \prod_{e \in C} \left( 1 - 2 (n/b) \ln{n} \right) < \exp{\left(-\sum_{x \in C} 2 (n/b) \ln{n} \cdot x^*_e \right)} \leq e^{-2 (n/b) \ln{n}}.
            \end{align*}
            The first inequality holds from the fact that $1-x < \exp{(-x)}$ for $x > 0$. For the  last inequality, observe that every anti-hopset is an integer cut against valid paths. Thus, each anti-hopset $C$ corresponds to an~\ref{lp:hopset} constraint of the form $\sum_{e \in E_M} z_e x_e \geq 1$, where $\bm{\mathrm{z}} \in \mathcal{Z}_{s,t}$ is the indicator vector for cut $C$.
        \end{proof}
        
    To finish the proof of Lemma~\ref{lem:thin_not_settle}, we use Propositions~\ref{prop:thin1} and~\ref{prop:thin2} to take a union bound over all minimal anti-hopsets for all thin demands. Let $\mathcal{S}_{s,t}$ be the set of all minimal anti-hopsets for a thin demand $(s,t)$, and let $\mathcal{S}$ be the collection of all minimal anti-hopsets for all thin demands. The probability that $E''$ does not intersect all minimal anti-hopsets in $\mathcal{S}$ is the following:
    \begin{align*}
        \Pr[E'' \text{ is not a hitting set for } \mathcal{S}] &\leq  \sum_{(s,t) \in \mathcal{D}_{thin}} \sum_{C \in \mathcal{S}_{s,t}} e^{-2 (n/b) \ln{n}} \\
        &\leq  |\mathcal{D}_{thin}| \cdot \left( \frac{n}{b} \right)^{n/b} \cdot e^{-2 (n/b) \ln{n}} \\
        &= |\mathcal{D}_{thin}| \cdot \left( \frac{1}{b n} \right)^{n/b} \\
        &= |\mathcal{D}_{thin}| \cdot \left( \frac{1}{n \sqrt{{\opt}}} \right)^{\frac{n}{\sqrt{{\opt}}}} \tag{$b = \sqrt{{\opt}}$}.
    \end{align*}
    If ${\opt} = \widetilde{O}(n^{2})$, then we can achieve an $\widetilde{O}(1)$-approximation by just returning $E_M \setminus E''$. Thus, we can assume without loss of generality that ${\opt} = \widetilde{o}(n^{2})$.  We therefore have that the probability that $E''$ doesn't cover all minimal anti-hopsets for a thin demand $(s,t)$ is exponentially small.
    \end{proof}

\end{proof}




\section{Full Discussion of Section~\ref{sec:existential}} \label{app:tradeoffs}


We first state a folklore existential bound that we will use throughout. For exact hopsets in both directed and undirected graphs, the following bound is the best-known (and is known to be tight \cite{BH23folklore}). Exact hopsets satisfy any arbitrary stretch, so we can trade off the hopset produced by the folklore existential bound with other algorithms in any regime. For more restricted regimes---specifically for limited stretch, hopbound, and undirected graphs---we will trade off with stronger existential bounds to get improved approximations.

\begin{lemma}
    Given any weighted directed graph $G$ and a parameter $\beta >0$, there is an exact hopset of $G$ with hopbound $\beta$ and size $\widetilde{O}(n^2/\beta^2)$. This hopset can be constructed in polynomial time via random sampling.
\end{lemma}

This implies the following straightforward characterization based on $\opt$, which will allow us to trade the folklore construction off with our other approximation algorithms that also depend on $\opt$.

\begin{corollary} \label{cor:existential_folklore}
    There is a randomized polynomial-time $\widetilde{O}(n^2/ (\beta^2 \cdot \opt))$-approximation for directed {\hopset}.
\end{corollary}

As $\be$ gets larger, this folklore construction gives improved approximations over the junction tree and star-sampling/randomized-rounding algorithms, especially when $\opt$ is also relatively large. 

\subsubsection{Directed Hopsets with Arbitrary Stretch}
We trade off our junction-tree and star-sampling/randomized-rounding algorithms with the Corollary~\ref{cor:existential_folklore} folklore approximation to achieve an overall $\widetilde{O}(n^{4/5 + \epsilon})$-approximation for directed {\hopset}. Let $\epsilon > 0$ be an arbitrarily small constant.

\mainResult*

\begin{proof}
    We trade off the junction tree based algorithm from Section~\ref{sec:junction_tree}, the star-sampling/randomized-rounding algorithms from Section~\ref{sec:star_sampling_rounding}, and the algorithm of Corollary~\ref{cor:existential_folklore} to give an algorithm for directed {\hopset}; that is, we run all three algorithms and return the solution with the minimum cost. The overall approximation ratio of this algorithm, $\al$, is the maximum over all possible $\beta, \opt$ of the minimum cost:
    \begin{align*}
        \al &= \widetilde{O}\left( \max_{\be, {\opt}} \left( \min_{} \left\{ \be n^\epsilon \cdot {\opt}, \mbox{ } \frac{n}{\sqrt{{\opt}}}, \mbox{ }\frac{n^2}{\be^2 \cdot {\opt}}  \right\} \right) \right).                    
    \end{align*}
    
    The value of $\al$ is $\widetilde{O}(n^{4/5+\epsilon})$, achieved at $\be = \widetilde{O}(n^{2/5})$, ${\opt} = \widetilde{O}(n^{2/5 - \epsilon})$ (this is also the point at which all three curves intersect). Additionally, each of the three algorithms runs in polynomial-time, so the overall algorithm is polynomial-time.
\end{proof}



When $\beta = \widetilde{O}(n^{2/5})$, the folklore construction gives worse approximations than the trade off between our other algorithms. By just trading off the junction tree and star-sampling/randomized-rounding algorithms, we achieve a better approximation in this regime.

\smallBeDirGen*
\begin{corollary}
    When $\be = \widetilde{O}(1)$, Theorem~\ref{thm:small_be_dir_gen} gives an $\widetilde{O}(n^{2/3 + \epsilon})$-approximation for {\hopset}.
\end{corollary}

We can also get improved approximations in the large $\beta$ setting by trading off the junction tree algorithm with the Corollary~\ref{cor:existential_folklore} folklore approximation. Note that because the folklore construction has a better inverse dependence on $\beta$ than the star-sampling/randomized-rounding algorithm (in fact, the latter has \textit{no} dependence on $\be$), the folklore construction performs better than star-sampling/randomized rounding when $\be$ is sufficiently large.

\bigBeDirGen*

\subsubsection{Directed Hopsets with Small Stretch}

For $(1+\epsilon)$-approximate directed hopsets, the best known existential bound is the following:

\begin{lemma}[Theorem 1.1 of~\cite{BW23}]
    For any directed graph $G$ with integer weights in $[1,W]$, given $\epsilon \in (0,1)$ and $\beta \geq  20\log n$, there is a $(1+\epsilon)$-hopset with hopbound $\beta$ and size:
    \begin{itemize}
        \item $\widetilde{O}\left(\frac{n^2\log^2(nW)}{\beta^3 \epsilon^2  }\right)$ for $\beta \leq n^{1/3}$,
        \item $\widetilde{O}\left(\frac{n^{\frac{3}{2}} \log^2(nW)}{\beta^{3/2} \epsilon^2}\right)$ for $\beta >n^{1/3}$.
    \end{itemize} 
    This hopset can be constructed in polynomial time.
\end{lemma}
\begin{corollary} \label{cor:existential_W}
    When $\beta \geq 20\log n$, there is a polynomial-time approximation for directed $(1+\epsilon)$-{\hopset} with approximation ratio:
        \begin{itemize}
        \item $\widetilde{O}\left(\frac{n^2\log^2(nW)}{\beta^3 \epsilon^2 \cdot \opt  }\right)$ for $\beta \leq n^{1/3}$,
        \item $\widetilde{O}\left(\frac{n^{\frac{3}{2}} \log^2(nW)}{\beta^{3/2} \epsilon^2 \cdot \opt}\right)$ for $\beta >n^{1/3}$
    \end{itemize}
    where $\epsilon \in (0,1)$.
\end{corollary}

Using the Corollary~\ref{cor:existential_W} algorithm, we get an improved approximation for directed {\hopset} when we restrict to $(1+\epsilon)$ stretch. 

\dirEps*
\begin{proof}
    The approximation is achieved by trading off the junction tree algorithm of Section~\ref{sec:junction_tree}, the star-sampling/randomized-rounding algorithms of Section~\ref{sec:star_sampling_rounding}, the Corollary~\ref{cor:existential_folklore} folklore algorithm, and the Corollary~\ref{cor:existential_W} algorithm. Note that $W = \poly(n)$, so the $\log(nW)$ factor is hidden by the $\widetilde{O}(\cdot)$ notation.
\end{proof}


\subsubsection{Undirected Hopsets with Small Stretch}
For \textit{undirected} hopsets with $(1+\epsilon)$ stretch, there is the following constructive existential result.

\begin{lemma}[\hspace{1sp}\cite{EN19}] \label{lem:existential_undir_eps}
    For any weighted undirected graph $G$, any integer $\eta \geq 1$, and any $0 < \rho < 1$, there is a randomized algorithm that runs in polynomial time in expectation and computes a hopset with size $O(n^{1 + 1/\eta})$, which is a hopset for any $\epsilon \in (0,1)$ with hopbound
    \begin{align*}
        \beta = \left( \frac{\log(\eta) + 1/\rho }{\epsilon}   \right)^{\log(\eta) + \frac{1}{\rho}+1} .
    \end{align*}
\end{lemma}

Let $W_0(x)$ be the principle branch of the Lambert $W$ function. When $x \geq 3$, the function is upper bounded by $\ln{x} - (1/2) \ln{\ln{x}}$. The Lemma~\ref{lem:existential_undir_eps} existential result implies the following:

\begin{corollary} \label{cor:existential_undir_eps}
    Let $\eta = \lfloor \beta^{1/W_0(\ln{\beta})} \rfloor > \be^{1/(\ln{\ln{\be}}-\frac{1}{2}\ln{\ln{\ln{\be}}})}$ (inequality holds when $\be \geq 3$). There is a randomized polynomial-time $O(n^{1 + 1/\eta} / {\opt})$-approximation for undirected stretch-$(1+\epsilon)$ {\hopset}, where $\epsilon \in (0,1)$.
\end{corollary}

For some insight into the behavior of $\eta$, note first that for all $\beta \geq 3$, $\eta \geq 6$. Additionally, the $\eta$ function grows faster than $\ln{\be}$, but much slower than $\be$. The Corollary~\ref{cor:existential_undir_eps} construction performs better than the star-sampling/randomized-rounding algorithms as $\opt$ grows, resulting in improved approximations compared to the directed graph, arbitrary stretch regime when $\be$ is relatively small. 

\undirEps*
\begin{proof}
    The approximation is achieved by trading off the junction tree algorithm of Section~\ref{sec:junction_tree} with the Corollary~\ref{cor:existential_undir_eps} algorithm.
\end{proof}



\subsubsection{Undirected Hopsets with Odd Stretch}

The following result is directly implied by Thorup-Zwick approximate distance oracles.

\begin{lemma}[\hspace{1sp}\cite{TZ05}]
    Let $k \geq 1$ be an integer. For any weighted undirected graph $G$, a stretch-$(2k-1)$ hopset with hopbound $2$ and size $O(kn^{1+1/k})$ can be found in polynomial-time.
\end{lemma}
\begin{corollary} \label{cor:existential_undir_k}
    Let $k \geq 1$ be an integer. There is a polynomial-time $O(kn^{1+1/k} / {\opt})$-approximation for undirected stretch-$(2k-1)$ {\hopset}.
\end{corollary}

We can use Corollary~\ref{cor:existential_undir_k} to get an improved approximation (over the directed graph, arbitrary stretch setting) for undirected stretch-$(2k-1)$ {\hopset} when $\be$ is relatively small.

\undirGenStretch*
\begin{proof}
    The approximation is achieved by trading off the junction tree algorithm of Section~\ref{sec:junction_tree} and the Corollary~\ref{cor:existential_undir_k} algorithm.
\end{proof}


\section{Label Cover Hardness for Directed Hopsets and Shortcut Sets}\label{app:hardness}
In this section we prove Theorem~\ref{thm:LB-main}. 
 In particular, we prove hardness of approximation for directed shortcut sets on directed graphs, which immediately implies hardness for directed hopsets for any stretch bound as long as the hopbound is at least $3$.  We use a reduction from Min-Rep (the natural minimization version of \LabelCover) in a very similar way to the hardness proofs for graph spanners from~\cite{Kor01,EP07,DKR16,CD16}.  The main technical difficulty / difference is that these previous reductions for spanners heavily use the fact that only edges from the original graph can be included in the spanner.  This is not true of shortcut sets (or hopsets), and makes it simpler to reason about the space of ``all feasible spanners'' than it is to reason about the space of ``all possible shortcut sets''.  

The Min-Rep problem, first introduced by~\cite{Kor01} and since used to prove hardness of approximation for many network design problems, can be thought of as a minimization version of \LabelCover with the additional property that the alphabets are represented explicitly as vertices in a graph.  A Min-Rep instance consists of an undirected bipartite graph $G = (A, B, E)$, together with partitions of $A_1, A_2, \dots, A_{m}$ of $A$ and $B_1, B_2, \dots, B_m$ of $B$ with the property that $|A_i| = |A_j| = |B_i| = |B_j$ for every $i,j \in [m]$\footnote{In some versions of the problem we do not assume that each partition has the same number of parts or that each $|A_i|=|B_i|$, but those assumptions can both be made without loss of generality so we do so for convenience}.  Each part of one of the partitions is called a \emph{group}.  If there is at least one edge between $A_i$ and $B_j$, then we say that $(i,j)$ is a \emph{superedge}.  Our goal is to choose a set $S \subseteq A \cup B$ of vertices of $V$ if minimum size so that, for every superedge $(i,j)$, there is some vertex $x \in A_i \cap S$ and $y \in B_j \cap S$ with $\{x,y\} \in E$.  In other words, we must choose vertices to \emph{cover} each superedge.  Any such cover is called a REP-cover, and our goal is to find the minimum size REP-cover.  Note that we must choose at least one vertex from every group (assuming that each group is involved in at least one superedge), and hence $OPT \geq 2m$.  

Since Min-Rep is essentially \LabelCover, which is a two-query PCP, the PCP theorem implies hardness of approximation for Min-Rep.  More formally, the following hardness is known~\cite{Kor01}.

\begin{theorem}[\cite{Kor01}] \label{thm:MinRep-hardness}
    Assuming that $NP \not\subseteq DTIME(2^{polylog(n)})$, then for any constant $\epsilon > 0$ there is no polynomial-time algorithm that approximates Min-Rep better than $2^{\log^{1-\epsilon} n}$.  In particular, no polynomial time algorithm can distinguish between when $OPT = 2m$ and when $OPT \geq 2m \cdot 2^{\log^{1-\epsilon} n}$
\end{theorem}


\subsection{Reduction from Min-Rep to shortcut sets}
We can now give our formal reduction from Min-Rep to shortcut sets with hopbound $\beta \geq 3$.  Consider some instance of Min-Rep $G = (A, B, E)$ with partition $A_1, A_2, \dots, A_m$ and $B_1, B_2, \dots, B_m$.  First we create two nodes for every group: let 
\begin{align*}
&A' = \{a_1^1, a_1^2, a_2^1, a_2^2, \dots, a_m^1, a_m^2\} & \text{and} &  &B' = \{b_1^1, b_1^2, b_2^1,b_2^2, \dots, b_m^1, b_m^2\}.
\end{align*}
Then for every edge $e = \{a, b\}$, we create a set of $\beta - 3$ vertices $V_e = \{v_e^1, v_e^2, \dots, v_e^{\beta-3}\}$ (note that these sets are empty if $\beta=3$).  Our final vertex set is 
\begin{align*}
    V' = V \cup A' \cup B' \cup (\cup_{e \in E} V_e).
\end{align*}

We now create the (directed) edges of the graph.  For each $e = \{a,b\} \in E$ we create a path:
\begin{align*}
    P_e &= \{(a, v_e^1)\} \cup \{(v_e^i, v_e^{i+1}) : i \in [\beta-4]\} \cup \{(b, v_e^{\beta-3})\}.
\end{align*}
If $\beta = 3$, we simply set $P_e = \{(a,b)\}$.  We can now specify our final edge set:
\begin{align*}
    E' = &\left(\cup_{e \in E} P_e\right) \\
    &\cup \{(a_i^1, a_i^2) : i \in [m]\} \cup \{(b_i^2, b_i^1) : i \in [m]\} \\
    &\cup \{(a_i^2, x) : i \in [m], x \in A_i\} \cup \{(x, b_i^2) : i \in [m], x \in B_i\}.
\end{align*}

Our final shortcut set instance is $G' = (V', E')$ with hopbound $\beta$.

\subsection{Analysis}
The analysis of this reduction has two parts, completeness and soundness (so-called due to their connections back to probabilistically checkable proofs).  For completeness, we show that if the original Min-Rep instance has $OPT = \gamma$ then there is a shortcut set of size at most $\gamma$.  For soundness, we show that if there is a shortcut set of size $\gamma$, then there is a REP-cover of size at most $O(\gamma)$.  Together with Theorem~\ref{thm:MinRep-hardness}, these will imply our desired hardness bound for shortcut sets.  To get Theorem~\ref{thm:LB-main}, one just needs to observe that in $G'$, if there is a path from some node $u$ to some node $v$ then in fact \emph{every} path from $u$ to $v$ has exactly the same length.  Hence any shortcut set is a hopset with the same hopbound and stretch $1$, and any hopset with stretch $\alpha$ is also shortcut set.

\subsubsection{Completeness}
We begin with completeness, which is the easier direction.  We prove the following lemma.

\begin{lemma} \label{lem:completeness}
    If there is a REP-cover of the Min-Rep instance with size $\gamma$, then there is a shortcut set for $G'$ with hopbound $\beta$ and size $\gamma$.
\end{lemma}
\begin{proof}
Suppose that there is a REP-cover $S \subseteq V$ with $|S|= \gamma$.  Consider the edge set 
\begin{align*}
    S' &= \{(a_i^1, x) : i \in [m], x \in S \cap A_i\} \cup \{(x, b_i^1) : i \in [m], x \in S \cap B_i\}.
\end{align*}
In other words, we add an edge between each ``outer'' node representing a group and the nodes in the REP-cover that are in that group (directed oppositely for the left and right sides of the bipartite graph).  Clearly $|S'| = |S| = \gamma$, so it remains to show that $S'$ is a shortcut set with hopbound $\beta$.  

An easy observation is that in $G'$, the only nodes $x, y$ where $y$ is reachable from $x$ but there are more than $\beta$ hops on every $x \leadsto y$ path are when $x = a_i^1$ and $y = b_j^1$ for some $i,j \in [m]$ where $(i,j)$ is a superedge.  So consider such an $a_i^1, b_j^1$.  Then since $(i,j)$ is a superedge and $S$ is a REP-cover, we know that there is some $a,b \in S$ with $a \in A_i$ and $b \in B_j$ and $\{a,b\} \in E$.  Then by definition $S'$ includes the edges $(a_i^1, a)$ and $(b, b_j^1)$.  Thus there is a path from $a_i^1$ to $b_j^1$ which first uses the edge to $a$ added by $S$, then uses the edges of $P_{\{a,b\}}$ to get to $b$, and then uses the edge to $b_j^1$ added by $S$.  This path has exactly $\beta$ hops by construction.  Hence $S'$ is a shortcut set with hopbound $\beta$.
\end{proof}

\subsubsection{Soundness}
We now argue about soundness.  To do this, we first need the following definition.

\begin{definition} \label{def:canonical}
    A shortcut edge is called \emph{canonical} if it is of the form $(a_i^1, a)$ for some $i \in [m]$ and $a \in A_i$ or of the form $(b, b_i^1)$ for some $i \in [m]$ and $b \in B_i$.  A shortcut set $S'$ is called \emph{canonical} if every edge in $S'$ is canonical. 
\end{definition}

We now show that without loss of generality, we may assume that any shortcut set is canonical.

\begin{lemma} \label{lem:canonical}
    Suppose that $S$ is a shortcut set of $G'$ with hopbound $\beta$ and size $\gamma$.  Then there is a canonical shortcut set $S'$ of $G'$ with hopbound $\beta$ and size at most $2 \gamma$.
\end{lemma}
\begin{proof}
    Consider some $(x,y) \in S$ which is not canonical.  Since it is not canonical but also must exist in the transitive closure of $G'$, there must exist a super edge $(i,j)$ so that $x$ and $y$ are as follows.
    \begin{itemize}
        \item $x$ is either $a_i^1, a_i^2$, is in $A_i$, or is in $P_e$ for some $e = \{a,b\}$ with $a \in A_i$ and $b \in B_j$.
        \item $y$ is either $b_j^1, b_j^2$, is in $B_j$, or is in $P_e$ for some $e = \{a,b\}$ with $a \in A_i$ and $b \in B_j$.
    \end{itemize}
    In particular, this implies that $(x,y)$ is only useful in decreasing the number of hops to at most $\beta$ for the pair $(a_i^1, b_j^1)$.  In other words, if we \emph{removed} $(x,y)$ from $S$, then the only pair of nodes which be reachable but not within $\beta$ hops would be $(a_i^1, b_j^1)$.  So choose some arbitrary edge $\{a,b\} \in E$ with $a \in A_i$ and $b \in B_j$, and let $(a_i^1, a)$ and $(b, b_j^1)$ be the \emph{replacement edges} for $(x,y)$.  Note that these replacement edges are both canonical, and suffice to decrease the number of hops from $a_i^1$ to $b_j^1$ to $\beta$.

    So we let $S'$ consist of all of the canonical edges of $S$ together with the two replacement edges for every noncanonical edge in $S$.  Clearly $|S'| \leq 2|S| =2\gamma$, and $S'$ is a shortcut set with hopbound $\beta$.  
\end{proof}

With this lemma in hand, it is now relatively straighforward to prove soundness.

\begin{lemma} \label{lem:soundness}
    If there is a shortcut set for $G'$ with hopbound $\beta$ and size $\gamma$, then there is a REP-cover of the Min-Rep instance with size at most $2\gamma$.
\end{lemma}
\begin{proof}
    Let $\hat S$ be a shortcut set for $G'$ with hopbound $\beta$ and size $\gamma$.  By Lemma~\ref{lem:canonical}, there is a shortcut set $S'$ for $G'$ with hopbound $\beta$ and size at most $2\gamma$.  Let $S$ be the set of nodes such that they are an endpoint of some edge in $S'$ and are in $A \cup B$.  In other words, for every canonical edge $(a_i^1, a) \in S'$ we add $a$ to $S$, and similarly for every canonical edge $(b, b_i^1) \in S'$ we add $b$ to $S$.  Clearly $|S| \leq |S'| \leq 2\gamma$, so we just need to prove that $S$ is a REP-cover.

    Consider some superedge $(i,j)$.  Then $G'$, we know that $a_i^1$ can reach $b_j^1$ but every such path has length $\beta+2$.  Since $S'$ is a canonical shortcut set, this means that it must add an edge $(a_i^1, a)$ and an edge $(b, b_j^1)$ for some $a \in A_i$ and $b \in B_i$ with $(a,b) \in E$.  Hence $a,b \in S$ by definition.  Thus $S$ is a REP-cover as claimed.  
\end{proof}

