


\begin{comment}
The natural relaxation of this problem is the following LP, where each edge in $E_M$ has a variable $x_e$ and cost $c_e=1$ if $e \in E$ or $c_e=0$ if $e \not \in E$. We also have a variable $f_P$ for every allowed path. The LP ensures that (integrally) for each pair of demands one hopbounded path with desired stretch is chosen. 
 
\begin{align} \label{lp:flow}
\min \quad &\sum_{e \in E_M}c_ex_e \nonumber\\
\text{s.t.} \quad &\sum_{P \in \mathcal{P}_{u,v}: e\in P} f_P\leq x_e  &\forall(u,v) \in  \mathcal{D}, \forall e \in E_M  \nonumber \\
&\sum_{P \in \mathcal{P}_{u,v}} f_P \geq 1 &\forall(u,v) \in  \mathcal{D}\\
&x_e \geq 0 &\forall	e \in E_M \nonumber \\
&f_P \geq 0 &\forall(u,v)\in  \mathcal{D}, \forall P\in \mathcal{P}_{u,v} \nonumber
\end{align}
\end{comment}

In this section, we state and solve a cut-covering linear program (LP) for {\hopset}. A few of our approximation algorithms will randomly round the solution to this LP as a subroutine. 

Let $\mathcal{P}_{s,t}$ be the set of paths from $s$ to $t$ that have at most $\beta$ hops and path length at most $Dist(s,t)$. We will refer to these paths as ``allowed'' or ``valid" paths. The natural flow LP for our problem requires building enough capacity to send one unit of (non-interfering) flow along valid paths for each demand; this is the basic LP used in essentially all network design problems, and was introduced for spanners by~\cite{DK11}.  An equivalent LP, which is what we will use for {\hopset}, is obtained through the duality between flows and fractional cuts (of valid paths).  This version of the LP for spanners was studied by~\cite{DNZ20}, and strengthens the anti-spanner LP of~\cite{BBMRY11}. 

In more detail, for a graph $G = (V,E)$, we say that an \quotes{$s-t$ cut against valid paths} is a set of edges $F$ such that in the graph $G \setminus F$, there are no valid paths from $s$ to $t$. In the cut covering LP we will use, the constraints will ensure that any feasible solution must \quotes{cover} every cut against valid paths; that is, any feasible solution must buy an edge in each of these cuts. This leads to our LP relaxation, which we call~\ref{lp:hopset}. The LP has a variable $x_e$ for every edge $e \in E_M$. Note that because edges in $E$---edges from the input graph---do not contribute to the cost of the solution, we can assume without loss of generality that $x_e = 1$ for all $e \in E$.  
Let $\mathcal{Z}_{s,t} = \{ \bm{\mathrm{z}} \in [0,1]^{|E|} : \forall P \in \mathcal{P}_{s,t} \; , \; \sum_{e \in P} z_e \geq 1 \}$ be the set of vectors representing all fractional cuts against valid $s-t$ paths. For each demand, our LP requires any feasible integral solution to buy at least one edge from every cut against valid paths.

\begin{equation} \label{lp:hopset} \tag{LP(Hopset)}
\fbox{$
\begin{array}{lll}
\min & \displaystyle \sum_{e \in \widetilde{E} } x_e \\
\mathrm{s.t.} & \displaystyle \sum_{e \in E_M} z_e x_e \geq 1 \qquad \qquad & \forall (s,t) \in \mathcal{D}, \, \forall \bm{\mathrm{z}} \in \mathcal{Z}_{s,t} \\
& \displaystyle x_e \geq 0 & \forall e \in E_M \\
%& \displaystyle x_e \geq 1 & \forall e \in E \\
\end{array}
$
}
\end{equation}

The two main differences between the $k$-spanner cut-covering LP and ours are 1) For spanners, valid paths are those that satisfy the stretch constraint, whereas for hopsets, we want paths that satisfy both distance and hop constraints, and 2) Rather than a subgraph of $G$, we are interested in a subgraph of the \textit{weighted transitive closure} of $G = (V,E)$ (see Definition~\ref{def:trans_closure}). 

As written, \ref{lp:hopset} has an infinite number of constraints. Consider the polytope $\mathcal{Z}_{s,t}$ for some demand $(s,t)$. Due to convexity, we only need to keep the constraints that correspond to the vectors $\bm{\mathrm{z}}$ that form the vertices of of $\mathcal{Z}_{s,t}$. Since there are only an exponential number of these constraints, we can assume the LP has exponential size. \iflong We now quickly prove that the LP is a valid LP relaxation of {\hopset}. \else It is easy to see that~\ref{lp:hopset} is a valid LP relaxation of {\hopset}; see Appendix~\ref{app:lp} for a proof.  \fi

\iflong
\begin{lemma}
    Let $H$ be a feasible solution to {\hopset}. For every edge $e \in E_M$, let $x_e = 1$ if $e \in H$ or $e \in E$. Otherwise let $x_e = 0$. Then, \bm{\mathrm{x}} is a feasible integral solution to~\ref{lp:hopset}.
\end{lemma}
\begin{proof}
    Clearly, $x_e \geq 0$ for all $e \in E_M$, so it is left to show that the cut covering constraints are satisfied. For some demand $(s,t) \in \mathcal{D}$, consider some edge cut against valid $s-t$ paths, $C$. There is a vector $\bm{\mathrm{z}} \in \mathcal{Z}_{s,t}$ that serves as the indicator vector for cut $C$---specifically, there is a vector $\bm{\mathrm{z}} \in \mathcal{Z}_{s,t}$ such that for every edge $e \in C$, we have $z_e = 1$, and $z_e = 0$ otherwise. 
    
    Since $H$ is a feasible solution to {\hopset}, there must be some edge $e \in C \cap H$ (otherwise, there is no valid path from $s$ to $t$ in $H \cup E$, making $H$ infeasible). This also means $x_e = 1$ and  $z_e = 1$. Hence we have that for $\bm{\mathrm{z}}$, the constraint $\sum_{e \in E_M} z_e x_e \geq 1$ is satisfied.
\end{proof}

\begin{lemma}
    Let $\bm{\mathrm{x}}$ be a feasible integral solution to~\ref{lp:hopset}, and let $H = \{ e \; | \; x_e = 1, e \notin E\}$. Then $H$ is a feasible solution to {\hopset}.
\end{lemma}
\begin{proof}
    Suppose for contradiction there is some demand, $(s,t) \in \mathcal{D}$, that is not satisfied by $H$.  Then there is some cut $C$ against valid $s-t$ paths such that $H \cap C = \emptyset$ (and $x_e = 0$ for all $e \in C$). Since $C$ is a cut against valid $s-t$ paths, there is an indicator vector $\bm{\mathrm{z}} \in \mathcal{Z}_{s,t}$ such that for every edge $e \in C$, we have $z_e = 1$ ($z_e = 0$ otherwise). Therefore we have that $\sum_{e \in E_M} z_e x_e = 0 < 1$, violating the constraint in \ref{lp:hopset}, and thus the assumption that $\bm{\mathrm{x}}$ is feasible. Hence $H$ must be feasible.
\end{proof}
\fi



